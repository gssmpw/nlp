\documentclass[conference]{IEEEtran}
\usepackage[noadjust]{cite}
\usepackage{graphicx,color}
\usepackage[dvipsnames]{xcolor}
\usepackage{amsmath,amsbsy,amsfonts,amssymb,amsthm}
\usepackage{mathrsfs,bm}
\usepackage{mathtools}
\usepackage{flushend}
\usepackage{lipsum}

\mathtoolsset{showonlyrefs}
\ifCLASSOPTIONcompsoc
\usepackage[caption=false,font=normalsize,labelfont=sf,textfont=sf]{subfig}
\else
\usepackage[caption=false,font=footnotesize]{subfig}
\fi

\IEEEoverridecommandlockouts % To allow \thanks{} command to work

% for commenting
\newcommand{\am}{\textcolor{magenta}}

\usepackage{etoolbox}
\usepackage{tikz}
\newrobustcmd*{\squareA}[1]{\tikz{\filldraw[draw=#1,fill=#1] (0,-0)
rectangle (0.1cm,0.14cm);}}
\newrobustcmd*{\mycircle}[1]{\tikz{\filldraw[draw=#1,fill=#1] (0,-0.3) circle [radius=0.08cm];}}

\newcommand{\figr}{Fig.~}
\newcommand{\secr}{Sec.~}

\usepackage{acro}
\newacronym{rl}{RL}{Reinforcement Learning}
\newacronym{drl}{DRL}{Deep Reinforcement Learning}
\newacronym{mdp}{MDP}{Markov Decision Process}
\newacronym{ppo}{PPO}{Proximal Policy Optimization}
\newacronym{sac}{SAC}{Soft Actor-Critic}
\newacronym{epvf}{EPVF}{Explicit Policy-conditioned Value Function}
\newacronym{unf}{UNF}{Universal Neural Functional}


\begin{document}

\title{\huge On the Uncertainty of a Simple Estimator for Remote Source Monitoring over ALOHA Channels}
\author{
\IEEEauthorblockN{Andrea Munari\\
\IEEEauthorblockA{Institute of Communications and Navigation, German Aerospace Center (DLR), Wessling, Germany
}}
}
\date{}
\maketitle
\thispagestyle{empty}
\pagestyle{empty}


\begin{abstract}
    Efficient remote monitoring of distributed sources is essential for many Internet of Things (IoT) applications. This work studies the uncertainty at the receiver when tracking two-state Markov sources over a slotted random access channel without feedback, using the conditional entropy as a performance indicator, and considering the last received value as current state estimate. We provide an analytical characterization of the metric, and evaluate three access strategies: (i) maximizing throughput, (ii) transmitting only on state changes, and (iii) minimizing uncertainty through optimized access probabilities. Our results reveal that throughput optimization does not always reduce uncertainty. Moreover, while reactive policies are optimal for symmetric sources, asymmetric processes benefit from mixed strategies allowing transmissions during state persistence.
\end{abstract}


% !TeX root = main.tex 


\newcommand{\lnote}{\textcolor[rgb]{1,0,0}{Lydia: }\textcolor[rgb]{0,0,1}}
\newcommand{\todo}{\textcolor[rgb]{1,0,0.5}{To do: }\textcolor[rgb]{0.5,0,1}}


\newcommand{\state}{S}
\newcommand{\meas}{M}
\newcommand{\out}{\mathrm{out}}
\newcommand{\piv}{\mathrm{piv}}
\newcommand{\pivotal}{\mathrm{pivotal}}
\newcommand{\isnot}{\mathrm{not}}
\newcommand{\pred}{^\mathrm{predict}}
\newcommand{\act}{^\mathrm{act}}
\newcommand{\pre}{^\mathrm{pre}}
\newcommand{\post}{^\mathrm{post}}
\newcommand{\calM}{\mathcal{M}}

\newcommand{\game}{\mathbf{V}}
\newcommand{\strategyspace}{S}
\newcommand{\payoff}[1]{V^{#1}}
\newcommand{\eff}[1]{E^{#1}}
\newcommand{\p}{\vect{p}}
\newcommand{\simplex}[1]{\Delta^{#1}}

\newcommand{\recdec}[1]{\bar{D}(\hat{Y}_{#1})}





\newcommand{\sphereone}{\calS^1}
\newcommand{\samplen}{S^n}
\newcommand{\wA}{w}%{w_{\mathfrak{a}}}
\newcommand{\Awa}{A_{\wA}}
\newcommand{\Ytil}{\widetilde{Y}}
\newcommand{\Xtil}{\widetilde{X}}
\newcommand{\wst}{w_*}
\newcommand{\wls}{\widehat{w}_{\mathrm{LS}}}
\newcommand{\dec}{^\mathrm{dec}}
\newcommand{\sub}{^\mathrm{sub}}

\newcommand{\calP}{\mathcal{P}}
\newcommand{\totspace}{\calZ}
\newcommand{\clspace}{\calX}
\newcommand{\attspace}{\calA}

\newcommand{\Ftil}{\widetilde{\calF}}

\newcommand{\totx}{Z}
\newcommand{\classx}{X}
\newcommand{\attx}{A}
\newcommand{\calL}{\mathcal{L}}



\newcommand{\defeq}{\mathrel{\mathop:}=}
\newcommand{\vect}[1]{\ensuremath{\mathbf{#1}}}
\newcommand{\mat}[1]{\ensuremath{\mathbf{#1}}}
\newcommand{\dd}{\mathrm{d}}
\newcommand{\grad}{\nabla}
\newcommand{\hess}{\nabla^2}
\newcommand{\argmin}{\mathop{\rm argmin}}
\newcommand{\argmax}{\mathop{\rm argmax}}
\newcommand{\Ind}[1]{\mathbf{1}\{#1\}}

\newcommand{\norm}[1]{\left\|{#1}\right\|}
\newcommand{\fnorm}[1]{\|{#1}\|_{\text{F}}}
\newcommand{\spnorm}[2]{\left\| {#1} \right\|_{\text{S}({#2})}}
\newcommand{\sigmin}{\sigma_{\min}}
\newcommand{\tr}{\text{tr}}
\renewcommand{\det}{\text{det}}
\newcommand{\rank}{\text{rank}}
\newcommand{\logdet}{\text{logdet}}
\newcommand{\trans}{^{\top}}
\newcommand{\poly}{\text{poly}}
\newcommand{\polylog}{\text{polylog}}
\newcommand{\st}{\text{s.t.~}}
\newcommand{\proj}{\mathcal{P}}
\newcommand{\projII}{\mathcal{P}_{\parallel}}
\newcommand{\projT}{\mathcal{P}_{\perp}}
\newcommand{\projX}{\mathcal{P}_{\mathcal{X}^\star}}
\newcommand{\inner}[1]{\langle #1 \rangle}

\renewcommand{\Pr}{\mathbb{P}}
\newcommand{\Z}{\mathbb{Z}}
\newcommand{\N}{\mathbb{N}}
\newcommand{\R}{\mathbb{R}}
\newcommand{\E}{\mathbb{E}}
\newcommand{\F}{\mathcal{F}}
\newcommand{\var}{\mathrm{var}}
\newcommand{\cov}{\mathrm{cov}}


\newcommand{\calN}{\mathcal{N}}

\newcommand{\jccomment}{\textcolor[rgb]{1,0,0}{C: }\textcolor[rgb]{1,0,1}}
\newcommand{\fracpar}[2]{\frac{\partial #1}{\partial  #2}}

\newcommand{\A}{\mathcal{A}}
\newcommand{\B}{\mat{B}}
%\newcommand{\C}{\mat{C}}

\newcommand{\I}{\mat{I}}
\newcommand{\M}{\mat{M}}
\newcommand{\D}{\mat{D}}
%\newcommand{\U}{\mat{U}}
\newcommand{\V}{\mat{V}}
\newcommand{\W}{\mat{W}}
\newcommand{\X}{\mat{X}}
\newcommand{\Y}{\mat{Y}}
\newcommand{\mSigma}{\mat{\Sigma}}
\newcommand{\mLambda}{\mat{\Lambda}}
\newcommand{\e}{\vect{e}}
\newcommand{\g}{\vect{g}}
\renewcommand{\u}{\vect{u}}
\newcommand{\w}{\vect{w}}
\newcommand{\x}{\vect{x}}
\newcommand{\y}{\vect{y}}
\newcommand{\z}{\vect{z}}
\newcommand{\fI}{\mathfrak{I}}
\newcommand{\fS}{\mathfrak{S}}
\newcommand{\fE}{\mathfrak{E}}
\newcommand{\fF}{\mathfrak{F}}

\newcommand{\Risk}{\mathcal{R}}

\renewcommand{\L}{\mathcal{L}}
\renewcommand{\H}{\mathcal{H}}

\newcommand{\cn}{\kappa}
\newcommand{\nn}{\nonumber}


\newcommand{\Hess}{\nabla^2}
\newcommand{\tlO}{\tilde{O}}
\newcommand{\tlOmega}{\tilde{\Omega}}

\newcommand{\calF}{\mathcal{F}}
\newcommand{\fhat}{\widehat{f}}
\newcommand{\calS}{\mathcal{S}}

\newcommand{\calX}{\mathcal{X}}
\newcommand{\calY}{\mathcal{Y}}
\newcommand{\calD}{\mathcal{D}}
\newcommand{\calZ}{\mathcal{Z}}
\newcommand{\calA}{\mathcal{A}}
\newcommand{\fbayes}{f^B}
\newcommand{\func}{f^U}


\newcommand{\bayscore}{\text{calibrated Bayes score}}
\newcommand{\bayrisk}{\text{calibrated Bayes risk}}

\newtheorem{example}{Example}[section]
\newtheorem{exc}{Exercise}[section]
%\newtheorem{rem}{Remark}[section]

\newtheorem{theorem}{Theorem}[section]
\newtheorem{definition}{Definition}
\newtheorem{proposition}[theorem]{Proposition}
\newtheorem{corollary}[theorem]{Corollary}

\newtheorem{remark}{Remark}[section]
\newtheorem{lemma}[theorem]{Lemma}
\newtheorem{claim}[theorem]{Claim}
\newtheorem{fact}[theorem]{Fact}
\newtheorem{assumption}{Assumption}

\newcommand{\iidsim}{\overset{\mathrm{i.i.d.}}{\sim}}
\newcommand{\unifsim}{\overset{\mathrm{unif}}{\sim}}
\newcommand{\sign}{\mathrm{sign}}
\newcommand{\wbar}{\overline{w}}
\newcommand{\what}{\widehat{w}}
\newcommand{\KL}{\mathrm{KL}}
\newcommand{\Bern}{\mathrm{Bernoulli}}
\newcommand{\ihat}{\widehat{i}}
\newcommand{\Dwst}{\calD^{w_*}}
\newcommand{\fls}{\widehat{f}_{n}}


\newcommand{\brpi}{\pi^{br}}
\newcommand{\brtheta}{\theta^{br}}

% \newcommand{\M}{\mat{M}}
% \newcommand\Mmh{\mat{M}^{-1/2}}
% \newcommand{\A}{\mat{A}}
% \newcommand{\B}{\mat{B}}
% \newcommand{\C}{\mat{C}}
% \newcommand{\Et}[1][t]{\mat{E_{#1}}}
% \newcommand{\Etp}{\Et[t+1]}
% \newcommand{\Errt}[1][t]{\mat{\bigtriangleup_{#1}}}
% \newcommand\cnM{\kappa}
% \newcommand{\cn}[1]{\kappa\left(#1\right)}
% \newcommand\X{\mat{X}}
% \newcommand\fstar{f_*}
% \newcommand\Xt[1][t]{\mat{X_{#1}}}
% \newcommand\ut[1][t]{{u_{#1}}}
% \newcommand\Xtinv{\inv{\Xt}}
% \newcommand\Xtp{\mat{X_{t+1}}}
% \newcommand\Xtpinv{\inv{\left(\mat{X_{t+1}}\right)}}
% \newcommand\U{\mat{U}}
% \newcommand\UTr{\trans{\mat{U}}}
% \newcommand{\Ut}[1][t]{\mat{U_{#1}}}
% \newcommand{\Utinv}{\inv{\Ut}}
% \newcommand{\UtTr}[1][t]{\trans{\mat{U_{#1}}}}
% \newcommand\Utp{\mat{U_{t+1}}}
% \newcommand\UtpTr{\trans{\mat{U}_{t+1}}}
% \newcommand\Utptild{\mat{\widetilde{U}_{t+1}}}
% \newcommand\Us{\mat{U^*}}
% \newcommand\UsTr{\trans{\mat{U^*}}}
% \newcommand{\Sigs}{\mat{\Sigma}}
% \newcommand{\Sigsmh}{\Sigs^{-1/2}}
% \newcommand{\eye}{\mat{I}}
% \newcommand{\twonormbound}{\left(4+\DPhi{\M}{\Xt[0]}\right)\twonorm{\M}}
% \newcommand{\lamj}{\lambda_j}

% \renewcommand\u{\vect{u}}
% \newcommand\uTr{\trans{\vect{u}}}
% \renewcommand\v{\vect{v}}
% \newcommand\vTr{\trans{\vect{v}}}
% \newcommand\w{\vect{w}}
% \newcommand\wTr{\trans{\vect{w}}}
% \newcommand\wperp{\vect{w}_{\perp}}
% \newcommand\wperpTr{\trans{\vect{w}_{\perp}}}
% \newcommand\wj{\vect{w_j}}
% \newcommand\vj{\vect{v_j}}
% \newcommand\wjTr{\trans{\vect{w_j}}}
% \newcommand\vjTr{\trans{\vect{v_j}}}

% \newcommand{\DPhi}[2]{\ensuremath{D_{\Phi}\left(#1,#2\right)}}
% \newcommand\matmult{{\omega}}


\section{Introduction}



% LLM advantage
% paraphrase generation -> successive
% what pattern?
% how to escape

Paraphrases are sentences or phrases that convey the same meaning using different wording\citep{bhagat2013what}.
With the advancement of large language models (LLMs) ~\citep{llama,gpt3,gpt4}, paraphrasing has gained widespread use, leading to the emergence of numerous online paraphrasing tools powered by these models. 
These tools have become particularly popular for refining articles and posts, especially among non-native speakers.
So far, many researchers have focused on the misuse of paraphrasing, including plagiarism \citep{plagiarism} and the spread of fake news \citep{detect-gpt}.
Besides, Some research shows that paraphrased texts also suffer from issues like hallucination \citep{llm_hallu,llm_hallu2} and bias \citep{llm_bias}.
Other researchers have explored a more practical scenario known as successive paraphrasing (SP), where a text is paraphrased multiple times by different individuals for various reasons. 
\cite{can_ai} investigates how SP can evade current AI text detection methods, while \cite{ship} discusses the impact of SP on determining authorship.
Our research stands out from existing work in the field of paraphrasing. 
By analyzing the effects of successive paraphrasing, we identify risks associated with the evolution of LLMs, which we refer to as \textbf{Digital Environment Homogenization} and the underlying \textbf{AI Homogenization}.


Digital environment homogenization (DEH) has recently become a significant concern.
As AI systems generate language based on pre-existing user data and interactions, there is a risk that they could perpetuate and even amplify existing linguistic norms and biases, resulting in the homogenization of language \citep{homogenization}.
\cite{anderson2024homogenization} shows that LLMs exert a stronger homogenizing effect on creative works compared to humans. 
Additionally, \cite{deepfake} highlights the distinct distribution of LLM-generated text in contrast to human-written text. 
%As large language models (LLMs) are increasingly used, the internet is gradually becoming saturated with machine-generated texts. 
As noted by \cite{shumailov2023curse}, the corpora mixed with these LLM-generated texts can lead to model collapse during the training process.
So, digital environment homogenization could have a profound impact on the development of next-generation LLMs.
Moreover, a digitally homogenized world could reshape the thinking of future generations, potentially passing on inherent stereotypes to them \citep{kotek2023gender,shrawgi-etal-2024-uncovering}.

%\textbf{Successive paraphrasing leading to DEH, and simulation experiments}
To clearly illustrate the homogenizing effect caused by the widespread use of paraphrasing, we conduct a simplified experiment that simulates the real digital environment in which people consistently produce paraphrased content by using LLMs.
We use several widely adopted commercial LLMs to paraphrase a subset of the corpus at each step, while concurrently adding new human-generated texts.
At each step, we calculate the standard deviation of the perplexity of the corpus.
Details of the simulation experiment setup can be found in Appendix \ref{Simulation}.
The results shown in \ref{intro} indicate that as the number of steps increases, the standard deviation of perplexity decreases significantly, implicitly revealing the potential homogenization issue in the future.
To understand the underlying causes of homogenization, we designed various experiments to explore successive paraphrasing and identified two interesting characteristics: periodicity and convergence.

%\textbf{Periodicity and Convergence}
Assuming a successive paraphrasing (SP) process denoted as \( s_{1}, \dots, s_{n} \), the periodicity of SP refers to the pattern observed when \( s_{i+1} \) shows significant grammatical similarity to \( s_{i-1} \) during the paraphrasing sequence. 
This suggests that LLMs have limited expressive diversity, which significantly contributes to \textbf{DEH}.
Additionally, we conduct a deeper analysis of the generation process and identify convergence phenomena occurring during the \textbf{SP} process. 
As the step number increases, the perplexity of paraphrase \( s_{i} \) conditioned on \( s_{i-1} \) or \( s_{i+1} \) decreases inversely. 
We identify an intrinsic connection between convergence and periodicity, and we discuss how convergence gives rise to periodicity from both an intuitive and mathematical perspective.
Based on our deduction, we propose a series of tasks that exhibit periodicity, which strongly supports our reasoning.
Furthermore, we conduct the \textbf{S.P.} under the model and prompt variations settings, and it still performs the periodicity characteristic.
It also indicates the underlying homogenization in current LLMs which we refer to \textbf{AIH}.


In the end, we examine the potential effects of \textbf{DEH} and \textbf{AIH}. 
 We consider AIH an inevitable challenge in LLM theory, a view strongly supported by S.P.'s experiments.
In the current theoretical model, improving LLMs could exacerbate the issue of homogenization. 
Therefore, We propose several potential solutions to reduce these risks and discuss the path toward developing more creative LLMs.



\begin{figure}
    \centering
    \includegraphics[width=1\linewidth]{article/figs_v1/intro.png}
    \caption{The similarity confusion matrix of successive paraphrasing is depicted in the following figures. Figures (a), (b), (c), and (d) illustrate the similarity confusion matrix for English successive paraphrasing, whereas figures (e) and (f) display the matrix for Chinese.} 
    \label{intro}
\end{figure}


\section{System Model}
\label{sec:sysModel}

We study a wireless network composed of \nodes\ terminals (nodes). Time is divided in slots, and each node monitors an independent discrete-time, two-state, Markov chain taking values in $\{0,1\}$. We denote by $\Mcni$, $n\in\mathbb N$, the chain observed by terminal $i$. 
%\begin{figure}
%    \centering
%    \includegraphics[width=.8\columnwidth]{./Figures/source_MC.pdf}
%    \caption{Discrete time Markov chain \mcn\ describing the evolution of one of the monitored sources.}
%    \label{fig:mc_source}
%\end{figure}
At the start of a slot, each process transitions between its states following the one-step probabilities reported in \figr\ref{fig:markovChains}a. For convenience, we introduce the \emph{asymmetry factor} $\asymm := \qZO/\qOZ$, capturing the ratio of the average time spent in state $1$ (i.e., $1/\qOZ$) with respect to state $0$ (i.e., $1/\qZO$). When $\asymm=1$, we speak of symmetric sources. The stationary distribution of the chain follows as $\statZ = \qOZ/(\qZO+\qOZ)$ and $\statO = 1-\statZ$.

Nodes share a wireless channel, and aim at reporting the state of the monitored sources to a common receiver (sink). Specifically, at the start of a slot, each terminal independently decides whether to transmit a packet, containing the current state of the Markov chain it observes. Accordingly, a slotted ALOHA protocol is implemented. No feedback is provided by the sink, and no retransmissions are performed by nodes. Following the well-established collision channel model \cite{Abramson77:PacketBroadcasting}, we assume that a slot over which two or more packets are sent (collision) does not allow retrieval of information at the sink, whereas successful decoding takes place whenever a single terminal transmits during a slot. 

In the remainder, we will focus on access schemes in which the transmission probability of a terminal is dictated by the previous and present state of the monitored process. We denote this quantity as $\pTx_{\mc_{n-1} \mcn}$, so that the access policy is fully specified by the vector $\pTxVec = [\pTx_{00},\pTx_{01},\pTx_{10},\pTx_{11}]$. We remark that, while $\pTxVec$ is the same for all nodes, the actual contention probability of each of them depends on the current evolution of its source. In view of this, the characterization of the number of terminals accessing the channel over a slot would require to jointly track all the Markov processes. Albeit conceptually viable, this approach soon becomes cumbersome as \nodes\ grows. We thus resort to an approximation, whose tightness is discussed in Sec.~\ref*{sec:results}, and model the success probability for a transmitted packet as
\begin{align}
    \ps := (1-\avgPTx)^{\nodes-1}
    \label{eq:ps}
\end{align}
where we have introduced the ancillary quantity \mbox{$\avgPTx := \statZ [ (1{-}\qZO) \pTx_{00} {+} \qZO \pTx_{01} ] + \statO [\qOZ \pTx_{10} {+} (1{-}\qOZ) \pTx_{11}]$}.
%\begin{align}
%    \avgPTx := \statZ [\, (1-\qZO) \pTx_{00} + \qZO \pTx_{01} \,] + \statO [\,\qOZ \pTx_{10} + (1-\qOZ) \pTx_{11}\,].
%\end{align}
In other words, we consider an i.i.d. behavior for all the $\nodes-1$ terminals other than the sender of interest, assuming that each of them transmits with probability \avgPTx. This, in turn, captures the average access probability for a node in stationary conditions. 


\begin{figure}    
    \subfloat[]{
        \includegraphics[width=.25\columnwidth]{./Figures/source_MC_vertical.pdf}}
    \hspace*{.1em}
    \subfloat[]{
        \includegraphics[width=.7\columnwidth]{./Figures/absorbing_MC.pdf}}
    \caption{(a) Markov chain \Mcn\ describing a monitored source; (b) terminating Markov chain $Y_n$ used to characterize $\ent(\Mcn\given\Agen,\Estn)$. The process enters the absorbing state $\mathsf d$ when an update from the reference node is received. Conversely, it moves between $0$ and $1$, describing the corresponding current source value, so long as no update message from the reference terminal arrives.}
    \label{fig:markovChains}
    \vspace{-1em}
\end{figure}

Within our system, the sink maintains an estimate $\Estn^{(i)}$ of the state of each monitored process. To this aim, we consider a simple solution, updating the estimate every time a message containing the current state of the source is received, and retaining the previous knowledge otherwise. Formally:
\begin{align}
    \!\!\!\!\Estn^{(i)} = 
    \begin{split}
    \begin{cases}
        \hspace*{.3em} \Mcni            \!\!\!& \text{ if node $i$ delivers message at slot $n$}\\[.3em]
        \hspace*{.3em} \Est_{n-1}^{(i)} \!\!\!& \text{ otherwise}
    \end{cases}
    \end{split}
    \label{eq:dh}
\end{align} 
with $\Estn^{(i)}\in\{0,1\}$. Without loss of generality, we will focus on the behavior of a reference source, and drop superscript $i$.
%\begin{remark}
%    \emph{We note that the presented approach requires no knowledge of network cardinality, employed access policy and source statistics. The estimator is only triggered upon successful decoding of a message, and can simply be run at the application layer, without the need for any other cross-layer information exchange (e.g., presence of a collided or idle slot). As such it may be of particular relevance for a large number of practical/already deployed IoT use cases.}
%\end{remark}
 
For the setting under study, we want to characterize the uncertainty of the sink on the state of the reference process. To this aim, we note that, at time $n$, a receiver implementing the estimator in \eqref{eq:dh} only has knowledge about (i) the last received update from the node of interest, i.e., \Estn, and (ii) the time elapsed since such message was retrieved. We denote the random process describing the latter as $\Agen \in \mathbb N_0$, and observe that \Agen\ corresponds the current AoI \cite{Yates20_Survey} at the sink. Indeed, each sent packet contains up to date information on the monitored source, %implementing a generate-at-will model, 
and $\Agen$ is exactly the difference between the current time and the time stamp of the last received message. We assume that, upon reception, \Agen\ is reset to $0$.

A natural measure of the sink uncertainty at time $n$, for a given AoI-estimate pair $(\agen,\estn)$, is thus given by the entropy
\begin{align}
    \mathsf h(\agen,\estn) := \ent( \Mcn \,|\, \Agen = \agen, \Estn = \estn).
    \label{eq:cond_ent}
\end{align}
An example of the time evolution of $\condent(\agen,\estn)$ is reported in \figr\ref{fig:timeline}. %In this case, $\nodes=50$ nodes were considered, implementing a transmission policy $\pTx_{\mc_{n-1} \mcn} = 1/ \nodes$, for any $(\mc_{n-1},\mcn)$ pair, and the source parameters were set as $\qZO=0.1$, $\qOZ=0.01$. 
The metric is reset to zero each time a message from the reference node is decoded. Instead, in the absence of updates, $\condent(\agen,\estn)$ tends to the stationary entropy of the source, $\mathsf H(X) = -\statZ \log_2 \statZ -\statO \log_2 \statO$. Finally, the peak shown in the plot denotes the higher uncertainty at the sink  experienced following reception of a message notifying of a source transition to the less likely state $0$. 

Leaning on this notation, we evaluate the average performance of the system in terms of the conditional entropy
\begin{align}
    \ent(\Mcn\,|\, \Agen,\Estn) = \sum_{\substack{\agen \in \mathbb N_0 \\ \estn \in \{0,1\} } } p(\agen,\estn)\, \condent(\agen,\estn).
    \label{eq:ent_def}
\end{align}
%with $\agen \in \mathbb N_0$, and $\estn \in \{0,1\}$.

\vspace{.5em}
\textbf{Remark.} \emph{The considered estimator is only updated upon successful decoding of a message from the source of interest, and can be run at the application layer, without the need for any other cross-layer information exchange (e.g., presence of a collided or idle slot). %As such it may be of particular relevance for a large number
    %of practical/already deployed IoT use cases.  
    In the remainder, we will provide a framework to understand, as protocol designers, how the medium contention shall be tuned  to minimize the average uncertainty $\ent(\Mcn\given \Agen,\Estn)$. On the other hand, as clarified in \secr\ref{sec:analysis}, knowledge of network cardinality, access parameters, and source statistics, allows an application to base control and decisions on its current uncertainty computed via \eqref{eq:cond_ent}.}
    %On the other hand, as clarified     in Sec. III, knowledge of network cardinality, access parameters, and source statistics suffice to compute (3), allowing an 
    %application with such knowledge to base control and decisions on its current uncertainty

    \begin{figure}
        \centering
        \includegraphics[width=.9\columnwidth]{./Figures/entropyTimeline_asymmetric.pdf}
        \caption{Example of time evolution of the entropy $\condent(\agen,\estn)$. In this case, $\qZO=0.1$, $\qOZ=0.01$, $\nodes=50$, $\pTx_{\mc_{n-1}\mcn} = 1/\nodes$, $\forall \, (\mc_{n-1},\mcn)$.}
        \vspace{-1em}
        \label{fig:timeline}
    \end{figure}
\section{Analysis}
\label{sec:analysis}

%Let us start by considering the entropy
%\begin{align}
%    \condent(\agen,\estn) = -\sum_{\mcn} p(\mcn\given\agen,\estn) \log_2 p(\mcn\given\agen,\estn)
%    \label{eq:cond_entropy_formula}
%\end{align}
%whose calculation requires the conditional distribution of the  source state given the current receiver estimate and the AoI value. 
To characterize the uncertainty at the receiver, we will resort to the auxiliary terminating Markov chain $Y_n$ reported in \figr\ref{fig:markovChains}b, with state-space $\mathcal Y = \{0,1,\mathsf d\}$. The chain transitions between the two upper states so long as the sink receives no message from the reference node, with $0$ and $1$ denoting the actual current source value. In turn, the process enters the absorbing state $\mathsf d$ as soon as an update refreshing the receiver estimate is delivered. 
%\begin{figure}
%    \centering
%    \includegraphics[width=.8\columnwidth]{./Figures/absorbing_MC.pdf}
%    \caption{Terminating Markov chain useful to characterize the distributions needed in the computation of the conditional entropy $\ent(\Mcn\given\Agen,\Estn)$. The chain transitions to the absorbing state $\mathsf d$ whenever an update from the reference node is received, refreshing the estimate. Conversely, the chain moves between states $0$ and $1$, describing the corresponding current reference source value, so long as no update message from the reference terminal arrives.}
%    \label{fig:absMC}
%\end{figure}
The one-step transition matrix for the process can be written as
\begin{align}
    \mathbf P = 
    \left(
        \begin{array}{cc|c}
            q_{00}  & q_{01} & q_{0\mathsf d}\\  
            q_{10}  & q_{11} & q_{1\mathsf d}\\[.3em]
            \hline
            0       & 0      & 1
        \end{array}
    \right)
     =
    \begin{pmatrix}
        \mathbf A & \mathbf a_{\mathsf d}\\
        \mathbf 0 & 1 \\
    \end{pmatrix}
\end{align}
where $\mathbf A$ is the $2\times 2$ matrix that captures transitions between $0$ and 1, and the $2\times 1$ vector $\mathbf a_{\mathsf d}$ contains the probability of being absorbed from each of the two states. In turn, the transition probabilities can be derived for the considered system model. For instance, focusing on state $0$, the chain will move to $1$ with probability $q_{01} = \qZO (1-\pTx_{01}\ps)$. Here, the first factor captures the fact that the source has to move to state $1$, whereas the second accounts for the lack of an update delivery from the terminal over the current slot (which would lead to absorption). Similarly, the chain remains in $0$ with probability \mbox{$q_{00}=(1-\qZO)(1-\pTx_{00}\ps)$}. Finally, if a message is successfully sent, the chain moves to $\mathsf d$, regardless of the state of the source, i.e., with overall probability $q_{0\mathsf d} = (\qZO\pTx_{01}+(1-\qZO)\pTx_{00})\ps$. The transitions from $1$ are immediately derived in the same manner and are not reported for the sake of compactness. 

The chain leads to a first result, captured in the following
\begin{prop} \label{prop1}The conditional probability of the reference source being in state \mcn\ given that its current AoI is \agen\ and the last received updated contained state \estn\ is 
    \begin{align}
        p(\mcn\given\agen,\estn) = \frac{ \mathbf e_{\estn}^{\mathsf T} \mathsf A^{\agen} \,\mathbf e_{\mcn}}{\mathbf e_{\estn}^{\mathsf T} \mathsf A^{\agen} \mathbf 1_2}
        \label{eq:pmfXnGivenDeltaXnHat}
    \end{align}
    where, for any $\estn$ and \mcn\ in  $\mathcal X$, $\mathbf e_{\estn}$ and $\mathbf e_{\mcn}$ are defined as \mbox{$\mathbf e_0 = [1,0]^{\mathsf T}$}; $\mathbf e_1 = [0,1]^{\mathsf T}$.
\end{prop}
%\begin{align}
%    p(\mcn\given\agen,\estn) = \frac{[\mathsf A^{\agen}]_{\estn,\mcn}}{[\mathsf A^{\agen}]_{\estn,0} + [\mathsf A^{\agen}]_{\estn,1}}
%    \label{eq:pmfXnGivenDeltaXnHat}
%\end{align}
\begin{proof}
We observe that, for the Markov chain of \figr\ref{fig:markovChains}b, the $\ell$-step transition probability from $i$ to $j$, with $i,j \in \{0,1\}$, provides the joint distribution of the source being in state $j$ and of not having delivered an update over the last $\ell \geq 1$ slots, given that its state $\ell$ slots ago was $i$. This corresponds to having an estimate $i$, content of the last received message, and a current AoI value $\ell$. By the definition of conditional probability, the sought distribution follows as
\begin{align}
    p(\mcn\given\agen,\estn) = \frac{q_{\estn \mcn}(\agen)}{p(\agen\given\estn)}.
\end{align}
The numerator is directly given by the \estn-row, \mcn-column element of the \agen-step transition matrix $\mathbf P^\agen$ of the chain. By the structure of $\mathbf P$, it is immediate to verify that this corresponds to $\mathbf e_{\estn}^{\mathsf T} \mathsf A^{\agen} \,\mathbf e_{\mcn}$. On the other hand, %the conditional probability of having AoI value $\agen$ given a last update of value $\estn$ 
$p(\agen\given\estn)$ can be derived as the probability of the chain not to transition to state $\mathsf d$ for \agen\ steps having started in \estn. This evaluates to $\sum\nolimits_{y\in\mathcal Y\setminus\{\mathsf d\}} q_{\estn y}(\agen) = q_{\estn 0}(\agen)  + q_{\estn 1}(\agen)$, where both addends are again obtained as elements of matrix $\mathbf A^{\agen}$.
\end{proof}
%where 
%\begin{align}
%    q_{\estn\mcn}(\agen) = \mathbf e_{\estn}^{\mathsf T} \,\mathbf P^{\agen} \,\mathbf e_{\mcn}.
%\end{align}
The result allows then to evaluate the performance at the receiver given the current conditions in terms of AoI and estimate, resorting to the definition in \eqref{eq:cond_ent}.
\begin{lemma}
    The receiver uncertainty on \Mcn, given \agen\ and \estn\ can be computed for any channel access strategy $\bm \lambda$ as
        \begin{align}
        \condent(\agen,\estn) = -\sum\nolimits_{\mcn} p(\mcn\given\agen,\estn) \log_2 p(\mcn\given\agen,\estn)
    \end{align}
    where $p(\mcn\given\agen,\estn)$ is obtained through \eqref{eq:pmfXnGivenDeltaXnHat}.
\end{lemma}

Let us now turn our attention to the derivation of the conditional entropy $\ent(\Mcn\given\Agen,\Estn)$ in \eqref{eq:ent_def}, which further requires the joint distribution of the current AoI and estimate available at the receiver at a general time slot $n$, i.e., $p(\agen,\estn) = p(\agen\given\estn) p(\estn)$. In the following, we streamline the steps for its computation through Prop. \ref{prop:condXn} and \ref{prop:statEst}. In turn, these results require a preliminary characterization of the inter-refresh time. Specifically, let us denote by \Irt\ the stochastic process describing the duration between two successive message receptions from the reference source, i.e., between two estimate updates at the sink. With this definition, we have:
\begin{prop}
Conditioned on the current estimate available at the receiver, the probability distribution of the process $W$ and its expected value are given by
\begin{align}
    p(\irt\given\estn) = \mathbf e_{\estn}^{\mathsf T} \mathbf A^{\irt-1} \, \mathbf a_{\mathsf d}
    \label{eq:condPMfW}
\end{align}
\begin{align}
    \mathbb E[\Irt \given \Estn=\estn] = \mathbf e_{\estn}^{\mathsf T} (\mathbf I_2 - \mathsf A)^{-1} \, \mathbf 1_2
    \label{eq:avgW}
\end{align}
where $\mathbf e_{\estn}$, $\estn\in\mathcal X$, is defined as in Prop.\ref{prop1}.
\end{prop}
\begin{proof}
    The results follows by observing that the distribution of \Irt, conditioned on the period being characterized by an estimate value $\hat{x}$ at the receiver, corresponds to the absorption time for the auxiliary chain in \figr\ref{fig:markovChains}b when starting from $\hat{x}$. Note indeed that, counting the steps to absorption starting from state $i\in\{0,1\}$ is equivalent to assuming reception of a message at time $0$ \--- stating that the reference source is in state $i$ \---, and thus starting a period over which the sink will keep $i$ as estimate. In turn, the distribution of the absorption time can be obtained using standard methods for Markov chains \cite{Kemeny76}, leading to the discrete phase-type distribution reported in \eqref{eq:condPMfW}, and the corresponding average absorption time in \eqref{eq:avgW}.        
\end{proof}

The proposition allows to derive the statistics of the current AoI value and the stationary distribution of the estimate.
\begin{prop}\label{prop:condXn}
    Conditioned on the estimate available at the receiver, the current AoI follows the probability distribution
    \begin{align}
        p(\agen\given\estn) = \frac{1}{\mathbf e_{\estn}^{\mathsf T} (\mathbf I_2 - \mathsf A)^{-1} \, \mathbf 1_2} \cdot \sum_{w>\agen} p(w\given \estn).
        \label{eq:condPMFAge_complete}
    \end{align}
    \begin{proof}
        For a generic time instant $n$, let us indicate as $\Irt(n)$ the duration of the estimate inter-refresh period $n$ falls into. The probability that $\Irt(n)$ lasts for $w$ slots follows as
        \begin{align}
        \mathsf P \!\left( \Irt(n)=\irt \given \Estn=\estn \right) = \frac{\irt \, p(\irt\given\estn)}{\sum\nolimits_\irt \irt \, p(\irt\given\estn)}
        \label{eq:condProbW}
        \end{align}
        capturing the fraction of time spent by the system in estimate inter-refresh periods of duration $\irt$. %Focus now on $p(\agen\given\estn)$, i.e., the probability of having an AoI value of \agen\ at a generic time instant $n$, conditioned on having at that time an estimate \estn. 
        Recalling the definition of \Irt, we observe that the AoI of the reference source is $0$ at the beginning of an inter-refresh period, and grows linearly over time, reaching the maximum value of $\Irt-1$ at the start of the last slot of the interval. Therefore, the probability for the receiver to have an AoI $\Agen=\agen$ at a generic time instant $n$ falling into an inter-refresh period of duration $\Irt(n)=\irt$ is simply $1/\irt$, $\forall$ $\agen\in\{0,\dots,\irt-1\}$. Leveraging this, the conditional PMF $p(\agen\given\estn)$ can be conveniently computed as
        \begin{align}
            \begin{split}
            \!\!\!\!\!p(\agen\given\estn) &\stackrel{(a)}{=} \sum\nolimits_{\irt > \agen} \frac{1}{\irt} \cdot \mathsf P\left(\Irt(n) = \irt \given \Estn=\estn\right)\!. %\\
                                %&\stackrel{(b)}{=} \sum_{\irt > \agen} \frac{p(\irt\given\estn)}{\mathbb E[\Irt \given \Estn=\estn]}.                        
            \end{split}
            \label{eq:condPMFAge}
        \end{align}
        Within \eqref{eq:condPMFAge}, $(a)$ follows from the law of total probability, observing that the AoI can reach a value $\agen$ only over an inter-refresh period of length at least $\agen+1$ slots, and using the uniform conditional probability for the AoI that was just derived. Plugging in \eqref{eq:condProbW} and recalling \eqref{eq:avgW} leads after simple steps to the proposition statement.
        %In turn, $(b)$ plugs into the expression the results in \eqref{eq:condProbW}. The proposition statement follows leaning on \eqref{eq:avgW}.                 
    \end{proof}
\end{prop}

\begin{prop}\label{prop:statEst}
    The stationary distribution of the receiver estimate for the reference source is given by    
    \begin{align}
        p(\estn) = \frac{\mathsf c_{\estn} \cdot \mathbb E[W \given \Estn=\estn]}{\mathsf c_{0} \cdot \mathbb E[W \given \Estn=0] + \mathsf c_{1} \cdot \mathbb E[W \given \Estn=1]}.
        \label{eq:pXn}
    \end{align}
    where 
    \begin{align}
        \mathsf c_0 = \frac{(\statZ (1-\qZO) \pTx_{00} + \statO \qOZ \pTx_{10})\ps}{\avgPTx \ps}, \quad  \mathsf c_1 = 1 - \mathsf c_0.
        \label{eq:c0}
    \end{align}
\end{prop}
\begin{proof}
    We start by observing that an inter-refresh period is characterized by having an estimate $\Estn=0$ with probability $\mathsf c_0$ reported in \eqref{eq:c0}. Here, the numerator captures the probability for the source to be in $0$ and to successfully deliver an update, i.e., $\statZ(1-\qZO)\pTx_{00}\ps$, or in $1$, transition to $0$ and inform the receiver, i.e., $\statO\qOZ\pTx_{10}\ps$. In turn, the denominator is a normalizing factor that accounts for the overall probability of delivering an update, i.e., initiating a new inter-refresh period. Similarly, the probability of having an inter-refresh interval with $\Estn=1$ is simply described by the auxiliary variable $\mathsf c_1 = 1-\mathsf c_0$. Leaning on this, the stationary distribution of $\Estn$ can be expressed as the fraction of time the receiver spends with such estimate value, obtaining the expression in \eqref{eq:pXn}.
\end{proof}

To conclude, the joint PMF $p(\agen,\estn)$ can be computed using \eqref{eq:condPMFAge_complete} and \eqref{eq:pXn}, eventually providing a complete analytical characterization of the conditional entropy $H(\Mcn\given\Agen,\Estn)$.

%To characterize this, we again resort to the auxiliary Markov process in \figr\ref{fig:abs_mc}, and  
%As a starting point, we observe that the chain can be used to capture the inter-refresh time of the estimate at the sink for the source of interest. Specifically, let us denote by \Irt\ the stochastic process describing the duration between two successive receptions of a message from the reference source, i.e., between two updates of the corresponding estimate at the sink. With this definition, the distribution of \Irt, conditioned on the period being characterized by an estimate value $\hat{x}$ at the receiver, is simply the absorption time for the chain when starting from $\hat{x}$.\footnote{Note indeed that counting the steps to absorption starting from state $i\in\{0,1\}$ is equivalent to assuming that at time $0$ a message is received, stating that the reference source is in such state, and thus starting a period over which the sink has an estimate $i$.} Accordingly, $p(\irt\given\estn)$ can be obtained using standard methods for Markov chains \cite{} as the discrete phase-type distribution
%\begin{align}
%    p(\irt\given\estn) = \mathbf e_{\estn}^{\mathsf T} \mathbf A^{\irt-1} \, \mathbf a_{\mathsf d}.
%\end{align}



\begin{table*}[t]
\centering
\fontsize{11pt}{11pt}\selectfont
\begin{tabular}{lllllllllllll}
\toprule
\multicolumn{1}{c}{\textbf{task}} & \multicolumn{2}{c}{\textbf{Mir}} & \multicolumn{2}{c}{\textbf{Lai}} & \multicolumn{2}{c}{\textbf{Ziegen.}} & \multicolumn{2}{c}{\textbf{Cao}} & \multicolumn{2}{c}{\textbf{Alva-Man.}} & \multicolumn{1}{c}{\textbf{avg.}} & \textbf{\begin{tabular}[c]{@{}l@{}}avg.\\ rank\end{tabular}} \\
\multicolumn{1}{c}{\textbf{metrics}} & \multicolumn{1}{c}{\textbf{cor.}} & \multicolumn{1}{c}{\textbf{p-v.}} & \multicolumn{1}{c}{\textbf{cor.}} & \multicolumn{1}{c}{\textbf{p-v.}} & \multicolumn{1}{c}{\textbf{cor.}} & \multicolumn{1}{c}{\textbf{p-v.}} & \multicolumn{1}{c}{\textbf{cor.}} & \multicolumn{1}{c}{\textbf{p-v.}} & \multicolumn{1}{c}{\textbf{cor.}} & \multicolumn{1}{c}{\textbf{p-v.}} &  &  \\ \midrule
\textbf{S-Bleu} & 0.50 & 0.0 & 0.47 & 0.0 & 0.59 & 0.0 & 0.58 & 0.0 & 0.68 & 0.0 & 0.57 & 5.8 \\
\textbf{R-Bleu} & -- & -- & 0.27 & 0.0 & 0.30 & 0.0 & -- & -- & -- & -- & - &  \\
\textbf{S-Meteor} & 0.49 & 0.0 & 0.48 & 0.0 & 0.61 & 0.0 & 0.57 & 0.0 & 0.64 & 0.0 & 0.56 & 6.1 \\
\textbf{R-Meteor} & -- & -- & 0.34 & 0.0 & 0.26 & 0.0 & -- & -- & -- & -- & - &  \\
\textbf{S-Bertscore} & \textbf{0.53} & 0.0 & {\ul 0.80} & 0.0 & \textbf{0.70} & 0.0 & {\ul 0.66} & 0.0 & {\ul0.78} & 0.0 & \textbf{0.69} & \textbf{1.7} \\
\textbf{R-Bertscore} & -- & -- & 0.51 & 0.0 & 0.38 & 0.0 & -- & -- & -- & -- & - &  \\
\textbf{S-Bleurt} & {\ul 0.52} & 0.0 & {\ul 0.80} & 0.0 & 0.60 & 0.0 & \textbf{0.70} & 0.0 & \textbf{0.80} & 0.0 & {\ul 0.68} & {\ul 2.3} \\
\textbf{R-Bleurt} & -- & -- & 0.59 & 0.0 & -0.05 & 0.13 & -- & -- & -- & -- & - &  \\
\textbf{S-Cosine} & 0.51 & 0.0 & 0.69 & 0.0 & {\ul 0.62} & 0.0 & 0.61 & 0.0 & 0.65 & 0.0 & 0.62 & 4.4 \\
\textbf{R-Cosine} & -- & -- & 0.40 & 0.0 & 0.29 & 0.0 & -- & -- & -- & -- & - & \\ \midrule
\textbf{QuestEval} & 0.23 & 0.0 & 0.25 & 0.0 & 0.49 & 0.0 & 0.47 & 0.0 & 0.62 & 0.0 & 0.41 & 9.0 \\
\textbf{LLaMa3} & 0.36 & 0.0 & \textbf{0.84} & 0.0 & {\ul{0.62}} & 0.0 & 0.61 & 0.0 &  0.76 & 0.0 & 0.64 & 3.6 \\
\textbf{our (3b)} & 0.49 & 0.0 & 0.73 & 0.0 & 0.54 & 0.0 & 0.53 & 0.0 & 0.7 & 0.0 & 0.60 & 5.8 \\
\textbf{our (8b)} & 0.48 & 0.0 & 0.73 & 0.0 & 0.52 & 0.0 & 0.53 & 0.0 & 0.7 & 0.0 & 0.59 & 6.3 \\  \bottomrule
\end{tabular}
\caption{Pearson correlation on human evaluation on system output. `R-': reference-based. `S-': source-based.}
\label{tab:sys}
\end{table*}



\begin{table}%[]
\centering
\fontsize{11pt}{11pt}\selectfont
\begin{tabular}{llllll}
\toprule
\multicolumn{1}{c}{\textbf{task}} & \multicolumn{1}{c}{\textbf{Lai}} & \multicolumn{1}{c}{\textbf{Zei.}} & \multicolumn{1}{c}{\textbf{Scia.}} & \textbf{} & \textbf{} \\ 
\multicolumn{1}{c}{\textbf{metrics}} & \multicolumn{1}{c}{\textbf{cor.}} & \multicolumn{1}{c}{\textbf{cor.}} & \multicolumn{1}{c}{\textbf{cor.}} & \textbf{avg.} & \textbf{\begin{tabular}[c]{@{}l@{}}avg.\\ rank\end{tabular}} \\ \midrule
\textbf{S-Bleu} & 0.40 & 0.40 & 0.19* & 0.33 & 7.67 \\
\textbf{S-Meteor} & 0.41 & 0.42 & 0.16* & 0.33 & 7.33 \\
\textbf{S-BertS.} & {\ul0.58} & 0.47 & 0.31 & 0.45 & 3.67 \\
\textbf{S-Bleurt} & 0.45 & {\ul 0.54} & {\ul 0.37} & 0.45 & {\ul 3.33} \\
\textbf{S-Cosine} & 0.56 & 0.52 & 0.3 & {\ul 0.46} & {\ul 3.33} \\ \midrule
\textbf{QuestE.} & 0.27 & 0.35 & 0.06* & 0.23 & 9.00 \\
\textbf{LlaMA3} & \textbf{0.6} & \textbf{0.67} & \textbf{0.51} & \textbf{0.59} & \textbf{1.0} \\
\textbf{Our (3b)} & 0.51 & 0.49 & 0.23* & 0.39 & 4.83 \\
\textbf{Our (8b)} & 0.52 & 0.49 & 0.22* & 0.43 & 4.83 \\ \bottomrule
\end{tabular}
\caption{Pearson correlation on human ratings on reference output. *not significant; we cannot reject the null hypothesis of zero correlation}
\label{tab:ref}
\end{table}


\begin{table*}%[]
\centering
\fontsize{11pt}{11pt}\selectfont
\begin{tabular}{lllllllll}
\toprule
\textbf{task} & \multicolumn{1}{c}{\textbf{ALL}} & \multicolumn{1}{c}{\textbf{sentiment}} & \multicolumn{1}{c}{\textbf{detoxify}} & \multicolumn{1}{c}{\textbf{catchy}} & \multicolumn{1}{c}{\textbf{polite}} & \multicolumn{1}{c}{\textbf{persuasive}} & \multicolumn{1}{c}{\textbf{formal}} & \textbf{\begin{tabular}[c]{@{}l@{}}avg. \\ rank\end{tabular}} \\
\textbf{metrics} & \multicolumn{1}{c}{\textbf{cor.}} & \multicolumn{1}{c}{\textbf{cor.}} & \multicolumn{1}{c}{\textbf{cor.}} & \multicolumn{1}{c}{\textbf{cor.}} & \multicolumn{1}{c}{\textbf{cor.}} & \multicolumn{1}{c}{\textbf{cor.}} & \multicolumn{1}{c}{\textbf{cor.}} &  \\ \midrule
\textbf{S-Bleu} & -0.17 & -0.82 & -0.45 & -0.12* & -0.1* & -0.05 & -0.21 & 8.42 \\
\textbf{R-Bleu} & - & -0.5 & -0.45 &  &  &  &  &  \\
\textbf{S-Meteor} & -0.07* & -0.55 & -0.4 & -0.01* & 0.1* & -0.16 & -0.04* & 7.67 \\
\textbf{R-Meteor} & - & -0.17* & -0.39 & - & - & - & - & - \\
\textbf{S-BertScore} & 0.11 & -0.38 & -0.07* & -0.17* & 0.28 & 0.12 & 0.25 & 6.0 \\
\textbf{R-BertScore} & - & -0.02* & -0.21* & - & - & - & - & - \\
\textbf{S-Bleurt} & 0.29 & 0.05* & 0.45 & 0.06* & 0.29 & 0.23 & 0.46 & 4.2 \\
\textbf{R-Bleurt} & - &  0.21 & 0.38 & - & - & - & - & - \\
\textbf{S-Cosine} & 0.01* & -0.5 & -0.13* & -0.19* & 0.05* & -0.05* & 0.15* & 7.42 \\
\textbf{R-Cosine} & - & -0.11* & -0.16* & - & - & - & - & - \\ \midrule
\textbf{QuestEval} & 0.21 & {\ul{0.29}} & 0.23 & 0.37 & 0.19* & 0.35 & 0.14* & 4.67 \\
\textbf{LlaMA3} & \textbf{0.82} & \textbf{0.80} & \textbf{0.72} & \textbf{0.84} & \textbf{0.84} & \textbf{0.90} & \textbf{0.88} & \textbf{1.00} \\
\textbf{Our (3b)} & 0.47 & -0.11* & 0.37 & 0.61 & 0.53 & 0.54 & 0.66 & 3.5 \\
\textbf{Our (8b)} & {\ul{0.57}} & 0.09* & {\ul 0.49} & {\ul 0.72} & {\ul 0.64} & {\ul 0.62} & {\ul 0.67} & {\ul 2.17} \\ \bottomrule
\end{tabular}
\caption{Pearson correlation on human ratings on our constructed test set. 'R-': reference-based. 'S-': source-based. *not significant; we cannot reject the null hypothesis of zero correlation}
\label{tab:con}
\end{table*}

\section{Results}
We benchmark the different metrics on the different datasets using correlation to human judgement. For content preservation, we show results split on data with system output, reference output and our constructed test set: we show that the data source for evaluation leads to different conclusions on the metrics. In addition, we examine whether the metrics can rank style transfer systems similar to humans. On style strength, we likewise show correlations between human judgment and zero-shot evaluation approaches. When applicable, we summarize results by reporting the average correlation. And the average ranking of the metric per dataset (by ranking which metric obtains the highest correlation to human judgement per dataset). 

\subsection{Content preservation}
\paragraph{How do data sources affect the conclusion on best metric?}
The conclusions about the metrics' performance change radically depending on whether we use system output data, reference output, or our constructed test set. Ideally, a good metric correlates highly with humans on any data source. Ideally, for meta-evaluation, a metric should correlate consistently across all data sources, but the following shows that the correlations indicate different things, and the conclusion on the best metric should be drawn carefully.

Looking at the metrics correlations with humans on the data source with system output (Table~\ref{tab:sys}), we see a relatively high correlation for many of the metrics on many tasks. The overall best metrics are S-BertScore and S-BLEURT (avg+avg rank). We see no notable difference in our method of using the 3B or 8B model as the backbone.

Examining the average correlations based on data with reference output (Table~\ref{tab:ref}), now the zero-shoot prompting with LlaMA3 70B is the best-performing approach ($0.59$ avg). Tied for second place are source-based cosine embedding ($0.46$ avg), BLEURT ($0.45$ avg) and BertScore ($0.45$ avg). Our method follows on a 5. place: here, the 8b version (($0.43$ avg)) shows a bit stronger results than 3b ($0.39$ avg). The fact that the conclusions change, whether looking at reference or system output, confirms the observations made by \citet{scialom-etal-2021-questeval} on simplicity transfer.   

Now consider the results on our test set (Table~\ref{tab:con}): Several metrics show low or no correlation; we even see a significantly negative correlation for some metrics on ALL (BLEU) and for specific subparts of our test set for BLEU, Meteor, BertScore, Cosine. On the other end, LlaMA3 70B is again performing best, showing strong results ($0.82$ in ALL). The runner-up is now our 8B method, with a gap to the 3B version ($0.57$ vs $0.47$ in ALL). Note our method still shows zero correlation for the sentiment task. After, ranks BLEURT ($0.29$), QuestEval ($0.21$), BertScore ($0.11$), Cosine ($0.01$).  

On our test set, we find that some metrics that correlate relatively well on the other datasets, now exhibit low correlation. Hence, with our test set, we can now support the logical reasoning with data evidence: Evaluation of content preservation for style transfer needs to take the style shift into account. This conclusion could not be drawn using the existing data sources: We hypothesise that for the data with system-based output, successful output happens to be very similar to the source sentence and vice versa, and reference-based output might not contain server mistakes as they are gold references. Thus, none of the existing data sources tests the limits of the metrics.  


\paragraph{How do reference-based metrics compare to source-based ones?} Reference-based metrics show a lower correlation than the source-based counterpart for all metrics on both datasets with ratings on references (Table~\ref{tab:sys}). As discussed previously, reference-based metrics for style transfer have the drawback that many different good solutions on a rewrite might exist and not only one similar to a reference.


\paragraph{How well can the metrics rank the performance of style transfer methods?}
We compare the metrics' ability to judge the best style transfer methods w.r.t. the human annotations: Several of the data sources contain samples from different style transfer systems. In order to use metrics to assess the quality of the style transfer system, metrics should correctly find the best-performing system. Hence, we evaluate whether the metrics for content preservation provide the same system ranking as human evaluators. We take the mean of the score for every output on each system and the mean of the human annotations; we compare the systems using the Kendall's Tau correlation. 

We find only the evaluation using the dataset Mir, Lai, and Ziegen to result in significant correlations, probably because of sparsity in a number of system tests (App.~\ref{app:dataset}). Our method (8b) is the only metric providing a perfect ranking of the style transfer system on the Lai data, and Llama3 70B the only one on the Ziegen data. Results in App.~\ref{app:results}. 


\subsection{Style strength results}
%Evaluating style strengths is a challenging task. 
Llama3 70B shows better overall results than our method. However, our method scores higher than Llama3 70B on 2 out of 6 datasets, but it also exhibits zero correlation on one task (Table~\ref{tab:styleresults}).%More work i s needed on evaluating style strengths. 
 
\begin{table}%[]
\fontsize{11pt}{11pt}\selectfont
\begin{tabular}{lccc}
\toprule
\multicolumn{1}{c}{\textbf{}} & \textbf{LlaMA3} & \textbf{Our (3b)} & \textbf{Our (8b)} \\ \midrule
\textbf{Mir} & 0.46 & 0.54 & \textbf{0.57} \\
\textbf{Lai} & \textbf{0.57} & 0.18 & 0.19 \\
\textbf{Ziegen.} & 0.25 & 0.27 & \textbf{0.32} \\
\textbf{Alva-M.} & \textbf{0.59} & 0.03* & 0.02* \\
\textbf{Scialom} & \textbf{0.62} & 0.45 & 0.44 \\
\textbf{\begin{tabular}[c]{@{}l@{}}Our Test\end{tabular}} & \textbf{0.63} & 0.46 & 0.48 \\ \bottomrule
\end{tabular}
\caption{Style strength: Pearson correlation to human ratings. *not significant; we cannot reject the null hypothesis of zero corelation}
\label{tab:styleresults}
\end{table}

\subsection{Ablation}
We conduct several runs of the methods using LLMs with variations in instructions/prompts (App.~\ref{app:method}). We observe that the lower the correlation on a task, the higher the variation between the different runs. For our method, we only observe low variance between the runs.
None of the variations leads to different conclusions of the meta-evaluation. Results in App.~\ref{app:results}.

%\clearpage
%\newpage
\bibliographystyle{IEEEtran}
\bibliography{IEEEabrv,biblio_RandomAccess,biblio_AoI}

\flushend

\end{document}
