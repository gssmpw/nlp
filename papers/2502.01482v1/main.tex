\documentclass[conference]{IEEEtran}
\usepackage[noadjust]{cite}
\usepackage{graphicx,color}
\usepackage[dvipsnames]{xcolor}
\usepackage{amsmath,amsbsy,amsfonts,amssymb,amsthm}
\usepackage{mathrsfs,bm}
\usepackage{mathtools}
\usepackage{flushend}
\usepackage{lipsum}

\mathtoolsset{showonlyrefs}
\ifCLASSOPTIONcompsoc
\usepackage[caption=false,font=normalsize,labelfont=sf,textfont=sf]{subfig}
\else
\usepackage[caption=false,font=footnotesize]{subfig}
\fi

\IEEEoverridecommandlockouts % To allow \thanks{} command to work

% for commenting
\newcommand{\am}{\textcolor{magenta}}

\usepackage{etoolbox}
\usepackage{tikz}
\newrobustcmd*{\squareA}[1]{\tikz{\filldraw[draw=#1,fill=#1] (0,-0)
rectangle (0.1cm,0.14cm);}}
\newrobustcmd*{\mycircle}[1]{\tikz{\filldraw[draw=#1,fill=#1] (0,-0.3) circle [radius=0.08cm];}}

\newcommand{\figr}{Fig.~}
\newcommand{\secr}{Sec.~}

\usepackage{acro}
\begin{acronym}
\acro{gan}[GANs]{Generative Adversarial Networks}
\acro{rl}[RL]{Reinforcement Learning}
\acro{pae}[PAE]{Periodic Autoencoder}
\acro{fld}[FLD]{Fourier Latent Dynamics}
\acro{ppo}[PPO]{Proximal Policy Optimization}
\acro{fft}[FFT]{Fast Fourier Transform}
\acro{pca}[PCA]{Principal Component Analysis}
\acro{dfm}[DFM]{Deep Fourier Mimic}
\acro{dof}[DoF]{Degrees of Freedom}
\acro{mlp}[MLPs]{Multi-Layer Perceptrons}
\end{acronym}




\begin{document}

\title{\huge On the Uncertainty of a Simple Estimator for Remote Source Monitoring over ALOHA Channels}
\author{
\IEEEauthorblockN{Andrea Munari\\
\IEEEauthorblockA{Institute of Communications and Navigation, German Aerospace Center (DLR), Wessling, Germany
}}
}
\date{}
\maketitle
\thispagestyle{empty}
\pagestyle{empty}


\begin{abstract}
    Efficient remote monitoring of distributed sources is essential for many Internet of Things (IoT) applications. This work studies the uncertainty at the receiver when tracking two-state Markov sources over a slotted random access channel without feedback, using the conditional entropy as a performance indicator, and considering the last received value as current state estimate. We provide an analytical characterization of the metric, and evaluate three access strategies: (i) maximizing throughput, (ii) transmitting only on state changes, and (iii) minimizing uncertainty through optimized access probabilities. Our results reveal that throughput optimization does not always reduce uncertainty. Moreover, while reactive policies are optimal for symmetric sources, asymmetric processes benefit from mixed strategies allowing transmissions during state persistence.
\end{abstract}


\newcommand{\ours}{$\text{Q}$LASS}

Large language models (LLM) have revolutionized the landscape of natural language processing, emerging as general-purpose foundation models with remarkable abilities across multiple  domains~\cite{achiam2023gpt,touvron2023llama}.
% Beyond traditional text tasks, LLMs have demonstrated their effectiveness in various domains.
In particular, their application in biomolecular studies has recently gained significant interest, motivated by the potential to profoundly accelerate scientific innovation and drug discovery applications~\cite{zhang2024scientific,pei2024leveraging,chaves2024tx}. LLMs provide novel ways to understand and reason about molecular data, building on the wealth of available scientific literature. Additionally, their reasoning and zero-shot abilities help overcome the limitations of task-specific deep learning models, streamlining data needs and improving human-AI collaboration \cite{fang2023mol,yu2024llasmol}. 

However, given the inherent complexity and specialized nature of the field, recent works emphasize the importance of domain-specific fine-tuning to boost tasks such as molecular captioning, property prediction, or binding affinity prediction~\cite{fang2023mol,chaves2024tx,yu2024llasmol,edwards2024molcap}. Consequently, rather than employing readily available general-purpose LLMs, most efforts in drug discovery have focused on fine-tuning LLMs using biochemical annotations or instruction-tuning datasets.
%While promising, solely relying on such approaches poses critical challenges hindering the application of recent advancements in AI to drug discovery applications.

While promising, solely relying on these approaches poses significant challenges that can limit applications.
On one hand, the rapid emergence of new LLM architectures and techniques \cite{minaee2024large,zhao2023survey} complicates maintaining domain-specific models obtained through expensive fine-tuning.
More importantly, drug discovery applications often require promptly incorporating new insights as they become available, for example, as a result of new experiments or through the scientific literature---a process exacerbated by the automation of experimental workflows~\cite{tom2024self}. In addition to being impractical, regular rounds of fine-tuning to keep LLMs up-to-date with the latest scientific advances also introduce challenges such as catastrophic forgetting~\cite{luo2023empirical}.
% However, continuously incorporating new knowledge into already fine-tuned domain-specific or general-purpose LLMs is complex and impractical, also introducing challenges such as catastrophic forgetting~\cite{luo2023empirical}.
% , where the model loses previously learned information~\cite{luo2023empirical}. Therefore, we identify the need to develop methods that can readily incorporate new data in a flexible way, empowering general-purpose LLMs with such information tackling drug discovery questions. 

From this perspective, retrieval-augmented generation (RAG) methods offer a promising solution that enables dynamic adaptation of the model's knowledge without the need for expensive fine-tuning \cite{gao2023retrieval,fan2024survey}.
%RAG-based systems have found success across many domains, dynamically and adaptively incorporating information from search queries, documents, and knowledge bases, grounding LLM outputs.
However, applying this paradigm in the drug discovery domain presents important obstacles. First, retrieving relevant knowledge is difficult due to the limited domain expertise of general-purpose LLMs, combined with the vastness of the chemical space \cite{bohacek1996art} that renders exact retrieval suboptimal.
%\footnote{The chemical space is estimated to encompass up to $10^{60}$ drug-like compounds \cite{bohacek1996art}
% rendering exact retrieval impractical.
Second, biochemical data is extremely heterogeneous, spanning diverse modalities such as molecules, proteins, diseases, and complex relationships between them~\cite{wang2023scientific}. Such data can also exist across multiple sources, introducing challenges in factual integration~\cite{harris2023large}.
Finally, the available information is not necessarily relevant to the query, as it may be too general, ambiguous, or partial \cite{vamathevan2019applications}.

% In this study, we tackle these challenges introducing~\proposed, a RAG-empowered LLM multi-agent framework for molecular question-answering, with a primary focus on drug discovery. 
In this study, we tackle these challenges by introducing a \textbf{C}ollaborative framework of \textbf{L}LM \textbf{A}gents for \textbf{D}rug \textbf{D}iscovery (\proposed).
We assume a general setting where external knowledge is available as expert annotations associated with molecules or as knowledge graphs that flexibly represent diverse biochemical entities and their relationships. 
% ~\proposed leverages external knowledge for diverse drug discovery tasks.
% \proposed is powered by general-purpose LLMs and the external knowledge can be updated dynamically without LLM fine-tuning, ensuring adaptability.
\proposed is powered by general-purpose LLMs, while also integrating domain-specific LMs, when necessary, to improve molecular understanding. Notably, external knowledge can be updated dynamically without LLM fine-tuning. %, ensuring adaptability.

The multi-agent collaborative framework enables each agent to specialize in a specific data source and/or role based on their team, offering a modular solution that can improve overall information processing~\cite{chan2024chateval}. 
In particular, \proposed includes a \emph{Planning Team} to determine relevant data sources, 
a \emph{Knowledge Graph Team} to retrieve  external heterogeneous information in the knowledge graph and summarize it, also through a novel anchoring approach to retrieve related information when the query molecule is not present in the knowledge base, 
and a \emph{Molecule Understanding Team}, which analyzes the query molecule based on its structure along with summaries of external data and tools.
The flexibility of the framework enables \proposed~to address a wide range of tasks for drug discovery, including zero-shot settings, while also improving interpretability through the transparent interaction of its agents.

% The flexibility of the question-answering setting enables addressing a wide range of tasks, including zero-shot settings,
% %making it particularly effective for complex scenarios and low-data regimes,
% while also improving interpretability in the process.

Overall, we highlight the following contributions:
\begin{itemize}[leftmargin=.1in]
    \item We present~\proposed, a multi-agent framework for RAG-based question-answering in drug discovery applications. The framework leverages generalist LLMs and dynamically integrates external biochemical data from multiple sources without requiring fine-tuning.
    \item We demonstrate the flexibility of the framework by tackling diverse applications, including property-specific molecular captioning, drug-target prediction, and molecular toxicity prediction. %, and single-cell perturbation prediction tasks.
    \item We provide comprehensive experimental results showcasing the effectiveness of \proposed compared to general-purpose and domain-specific LLMs, as well as standard deep learning approaches. A further appeal of \proposed~is its flexibility and explainability, improving the interaction between scientists and AI.
    % Extensive ablation studies and case analyses highlight the framework's seamless ability to retrieve, integrate, and utilize external knowledge to address drug discovery-related questions. 
\end{itemize}

\begin{figure*}[t]
\vspace{-6pt}
    \centering
    \includegraphics[width=0.85\linewidth]{imgs/Figure1.pdf}
    \vspace{-5pt}
    \caption{
    \textbf{Overview of \proposed}.
    \textbf{(a)} General overview. Given a question about a molecule (SMILES representation), the Planning Team first evaluates the relevant data sources and models.  
    If the knowledge graph is considered relevant, a dedicated report will be created by the Knowledge Graph Team.
    The Molecule Understanding Team creates a report on the molecule based on the relevant annotation databases and tools.
    The generated reports are integrated by the Prediction Agent and used to produce the final answer. 
    \textbf{(b)} Detailed overview of the Planning Team. The Molecular Annotation Planner decides whether to use additional captioning tools to complement the information associated with the molecule. \textbf{(c)} Detailed overview of the Knowledge Graph Team. The Drug Relation and Biological Relation Agents retrieve information from the knowledge graph based on the anchor drug, the related drugs, and the set of relations between the anchor and related drugs (cfr. Section~\ref{sec:KnowledgeGraphTeam}).
    % (b) Detailed overview of the Planning Team. The Molecular Annotation Planner assesses the quality of the available evidence in the molecular annotation database and can decide to use additional captioning tools. 
    % The Knowledge Graph Planner computes the similarity between the query molecule and the most similar drug in the KG (anchor drug). Based on this information, it decides whether to use the KG by calling the Knowledge Graph Team.
    % (c) Overview of the KG retrieval process. The Drug Relation Agent computes a report based on the query drug, the anchor drug, and related drugs. The Biological Relation Agent computes a report based on the anchor drug, the related drugs, and the set of relations between the anchor and related drugs. See Section~\ref{sec:KnowledgeGraphTeam} for the definition of related drugs.
    }
    \label{fig:fig1}
    \vspace{-10pt}
\end{figure*}

\section{System Model}
\label{sec:sysModel}

We study a wireless network composed of \nodes\ terminals (nodes). Time is divided in slots, and each node monitors an independent discrete-time, two-state, Markov chain taking values in $\{0,1\}$. We denote by $\Mcni$, $n\in\mathbb N$, the chain observed by terminal $i$. 
%\begin{figure}
%    \centering
%    \includegraphics[width=.8\columnwidth]{./Figures/source_MC.pdf}
%    \caption{Discrete time Markov chain \mcn\ describing the evolution of one of the monitored sources.}
%    \label{fig:mc_source}
%\end{figure}
At the start of a slot, each process transitions between its states following the one-step probabilities reported in \figr\ref{fig:markovChains}a. For convenience, we introduce the \emph{asymmetry factor} $\asymm := \qZO/\qOZ$, capturing the ratio of the average time spent in state $1$ (i.e., $1/\qOZ$) with respect to state $0$ (i.e., $1/\qZO$). When $\asymm=1$, we speak of symmetric sources. The stationary distribution of the chain follows as $\statZ = \qOZ/(\qZO+\qOZ)$ and $\statO = 1-\statZ$.

Nodes share a wireless channel, and aim at reporting the state of the monitored sources to a common receiver (sink). Specifically, at the start of a slot, each terminal independently decides whether to transmit a packet, containing the current state of the Markov chain it observes. Accordingly, a slotted ALOHA protocol is implemented. No feedback is provided by the sink, and no retransmissions are performed by nodes. Following the well-established collision channel model \cite{Abramson77:PacketBroadcasting}, we assume that a slot over which two or more packets are sent (collision) does not allow retrieval of information at the sink, whereas successful decoding takes place whenever a single terminal transmits during a slot. 

In the remainder, we will focus on access schemes in which the transmission probability of a terminal is dictated by the previous and present state of the monitored process. We denote this quantity as $\pTx_{\mc_{n-1} \mcn}$, so that the access policy is fully specified by the vector $\pTxVec = [\pTx_{00},\pTx_{01},\pTx_{10},\pTx_{11}]$. We remark that, while $\pTxVec$ is the same for all nodes, the actual contention probability of each of them depends on the current evolution of its source. In view of this, the characterization of the number of terminals accessing the channel over a slot would require to jointly track all the Markov processes. Albeit conceptually viable, this approach soon becomes cumbersome as \nodes\ grows. We thus resort to an approximation, whose tightness is discussed in Sec.~\ref*{sec:results}, and model the success probability for a transmitted packet as
\begin{align}
    \ps := (1-\avgPTx)^{\nodes-1}
    \label{eq:ps}
\end{align}
where we have introduced the ancillary quantity \mbox{$\avgPTx := \statZ [ (1{-}\qZO) \pTx_{00} {+} \qZO \pTx_{01} ] + \statO [\qOZ \pTx_{10} {+} (1{-}\qOZ) \pTx_{11}]$}.
%\begin{align}
%    \avgPTx := \statZ [\, (1-\qZO) \pTx_{00} + \qZO \pTx_{01} \,] + \statO [\,\qOZ \pTx_{10} + (1-\qOZ) \pTx_{11}\,].
%\end{align}
In other words, we consider an i.i.d. behavior for all the $\nodes-1$ terminals other than the sender of interest, assuming that each of them transmits with probability \avgPTx. This, in turn, captures the average access probability for a node in stationary conditions. 


\begin{figure}    
    \subfloat[]{
        \includegraphics[width=.25\columnwidth]{./Figures/source_MC_vertical.pdf}}
    \hspace*{.1em}
    \subfloat[]{
        \includegraphics[width=.7\columnwidth]{./Figures/absorbing_MC.pdf}}
    \caption{(a) Markov chain \Mcn\ describing a monitored source; (b) terminating Markov chain $Y_n$ used to characterize $\ent(\Mcn\given\Agen,\Estn)$. The process enters the absorbing state $\mathsf d$ when an update from the reference node is received. Conversely, it moves between $0$ and $1$, describing the corresponding current source value, so long as no update message from the reference terminal arrives.}
    \label{fig:markovChains}
    \vspace{-1em}
\end{figure}

Within our system, the sink maintains an estimate $\Estn^{(i)}$ of the state of each monitored process. To this aim, we consider a simple solution, updating the estimate every time a message containing the current state of the source is received, and retaining the previous knowledge otherwise. Formally:
\begin{align}
    \!\!\!\!\Estn^{(i)} = 
    \begin{split}
    \begin{cases}
        \hspace*{.3em} \Mcni            \!\!\!& \text{ if node $i$ delivers message at slot $n$}\\[.3em]
        \hspace*{.3em} \Est_{n-1}^{(i)} \!\!\!& \text{ otherwise}
    \end{cases}
    \end{split}
    \label{eq:dh}
\end{align} 
with $\Estn^{(i)}\in\{0,1\}$. Without loss of generality, we will focus on the behavior of a reference source, and drop superscript $i$.
%\begin{remark}
%    \emph{We note that the presented approach requires no knowledge of network cardinality, employed access policy and source statistics. The estimator is only triggered upon successful decoding of a message, and can simply be run at the application layer, without the need for any other cross-layer information exchange (e.g., presence of a collided or idle slot). As such it may be of particular relevance for a large number of practical/already deployed IoT use cases.}
%\end{remark}
 
For the setting under study, we want to characterize the uncertainty of the sink on the state of the reference process. To this aim, we note that, at time $n$, a receiver implementing the estimator in \eqref{eq:dh} only has knowledge about (i) the last received update from the node of interest, i.e., \Estn, and (ii) the time elapsed since such message was retrieved. We denote the random process describing the latter as $\Agen \in \mathbb N_0$, and observe that \Agen\ corresponds the current AoI \cite{Yates20_Survey} at the sink. Indeed, each sent packet contains up to date information on the monitored source, %implementing a generate-at-will model, 
and $\Agen$ is exactly the difference between the current time and the time stamp of the last received message. We assume that, upon reception, \Agen\ is reset to $0$.

A natural measure of the sink uncertainty at time $n$, for a given AoI-estimate pair $(\agen,\estn)$, is thus given by the entropy
\begin{align}
    \mathsf h(\agen,\estn) := \ent( \Mcn \,|\, \Agen = \agen, \Estn = \estn).
    \label{eq:cond_ent}
\end{align}
An example of the time evolution of $\condent(\agen,\estn)$ is reported in \figr\ref{fig:timeline}. %In this case, $\nodes=50$ nodes were considered, implementing a transmission policy $\pTx_{\mc_{n-1} \mcn} = 1/ \nodes$, for any $(\mc_{n-1},\mcn)$ pair, and the source parameters were set as $\qZO=0.1$, $\qOZ=0.01$. 
The metric is reset to zero each time a message from the reference node is decoded. Instead, in the absence of updates, $\condent(\agen,\estn)$ tends to the stationary entropy of the source, $\mathsf H(X) = -\statZ \log_2 \statZ -\statO \log_2 \statO$. Finally, the peak shown in the plot denotes the higher uncertainty at the sink  experienced following reception of a message notifying of a source transition to the less likely state $0$. 

Leaning on this notation, we evaluate the average performance of the system in terms of the conditional entropy
\begin{align}
    \ent(\Mcn\,|\, \Agen,\Estn) = \sum_{\substack{\agen \in \mathbb N_0 \\ \estn \in \{0,1\} } } p(\agen,\estn)\, \condent(\agen,\estn).
    \label{eq:ent_def}
\end{align}
%with $\agen \in \mathbb N_0$, and $\estn \in \{0,1\}$.

\vspace{.5em}
\textbf{Remark.} \emph{The considered estimator is only updated upon successful decoding of a message from the source of interest, and can be run at the application layer, without the need for any other cross-layer information exchange (e.g., presence of a collided or idle slot). %As such it may be of particular relevance for a large number
    %of practical/already deployed IoT use cases.  
    In the remainder, we will provide a framework to understand, as protocol designers, how the medium contention shall be tuned  to minimize the average uncertainty $\ent(\Mcn\given \Agen,\Estn)$. On the other hand, as clarified in \secr\ref{sec:analysis}, knowledge of network cardinality, access parameters, and source statistics, allows an application to base control and decisions on its current uncertainty computed via \eqref{eq:cond_ent}.}
    %On the other hand, as clarified     in Sec. III, knowledge of network cardinality, access parameters, and source statistics suffice to compute (3), allowing an 
    %application with such knowledge to base control and decisions on its current uncertainty

    \begin{figure}
        \centering
        \includegraphics[width=.9\columnwidth]{./Figures/entropyTimeline_asymmetric.pdf}
        \caption{Example of time evolution of the entropy $\condent(\agen,\estn)$. In this case, $\qZO=0.1$, $\qOZ=0.01$, $\nodes=50$, $\pTx_{\mc_{n-1}\mcn} = 1/\nodes$, $\forall \, (\mc_{n-1},\mcn)$.}
        \vspace{-1em}
        \label{fig:timeline}
    \end{figure}
\section{Analysis}
\label{sec:analysis}

%Let us start by considering the entropy
%\begin{align}
%    \condent(\agen,\estn) = -\sum_{\mcn} p(\mcn\given\agen,\estn) \log_2 p(\mcn\given\agen,\estn)
%    \label{eq:cond_entropy_formula}
%\end{align}
%whose calculation requires the conditional distribution of the  source state given the current receiver estimate and the AoI value. 
To characterize the uncertainty at the receiver, we will resort to the auxiliary terminating Markov chain $Y_n$ reported in \figr\ref{fig:markovChains}b, with state-space $\mathcal Y = \{0,1,\mathsf d\}$. The chain transitions between the two upper states so long as the sink receives no message from the reference node, with $0$ and $1$ denoting the actual current source value. In turn, the process enters the absorbing state $\mathsf d$ as soon as an update refreshing the receiver estimate is delivered. 
%\begin{figure}
%    \centering
%    \includegraphics[width=.8\columnwidth]{./Figures/absorbing_MC.pdf}
%    \caption{Terminating Markov chain useful to characterize the distributions needed in the computation of the conditional entropy $\ent(\Mcn\given\Agen,\Estn)$. The chain transitions to the absorbing state $\mathsf d$ whenever an update from the reference node is received, refreshing the estimate. Conversely, the chain moves between states $0$ and $1$, describing the corresponding current reference source value, so long as no update message from the reference terminal arrives.}
%    \label{fig:absMC}
%\end{figure}
The one-step transition matrix for the process can be written as
\begin{align}
    \mathbf P = 
    \left(
        \begin{array}{cc|c}
            q_{00}  & q_{01} & q_{0\mathsf d}\\  
            q_{10}  & q_{11} & q_{1\mathsf d}\\[.3em]
            \hline
            0       & 0      & 1
        \end{array}
    \right)
     =
    \begin{pmatrix}
        \mathbf A & \mathbf a_{\mathsf d}\\
        \mathbf 0 & 1 \\
    \end{pmatrix}
\end{align}
where $\mathbf A$ is the $2\times 2$ matrix that captures transitions between $0$ and 1, and the $2\times 1$ vector $\mathbf a_{\mathsf d}$ contains the probability of being absorbed from each of the two states. In turn, the transition probabilities can be derived for the considered system model. For instance, focusing on state $0$, the chain will move to $1$ with probability $q_{01} = \qZO (1-\pTx_{01}\ps)$. Here, the first factor captures the fact that the source has to move to state $1$, whereas the second accounts for the lack of an update delivery from the terminal over the current slot (which would lead to absorption). Similarly, the chain remains in $0$ with probability \mbox{$q_{00}=(1-\qZO)(1-\pTx_{00}\ps)$}. Finally, if a message is successfully sent, the chain moves to $\mathsf d$, regardless of the state of the source, i.e., with overall probability $q_{0\mathsf d} = (\qZO\pTx_{01}+(1-\qZO)\pTx_{00})\ps$. The transitions from $1$ are immediately derived in the same manner and are not reported for the sake of compactness. 

The chain leads to a first result, captured in the following
\begin{prop} \label{prop1}The conditional probability of the reference source being in state \mcn\ given that its current AoI is \agen\ and the last received updated contained state \estn\ is 
    \begin{align}
        p(\mcn\given\agen,\estn) = \frac{ \mathbf e_{\estn}^{\mathsf T} \mathsf A^{\agen} \,\mathbf e_{\mcn}}{\mathbf e_{\estn}^{\mathsf T} \mathsf A^{\agen} \mathbf 1_2}
        \label{eq:pmfXnGivenDeltaXnHat}
    \end{align}
    where, for any $\estn$ and \mcn\ in  $\mathcal X$, $\mathbf e_{\estn}$ and $\mathbf e_{\mcn}$ are defined as \mbox{$\mathbf e_0 = [1,0]^{\mathsf T}$}; $\mathbf e_1 = [0,1]^{\mathsf T}$.
\end{prop}
%\begin{align}
%    p(\mcn\given\agen,\estn) = \frac{[\mathsf A^{\agen}]_{\estn,\mcn}}{[\mathsf A^{\agen}]_{\estn,0} + [\mathsf A^{\agen}]_{\estn,1}}
%    \label{eq:pmfXnGivenDeltaXnHat}
%\end{align}
\begin{proof}
We observe that, for the Markov chain of \figr\ref{fig:markovChains}b, the $\ell$-step transition probability from $i$ to $j$, with $i,j \in \{0,1\}$, provides the joint distribution of the source being in state $j$ and of not having delivered an update over the last $\ell \geq 1$ slots, given that its state $\ell$ slots ago was $i$. This corresponds to having an estimate $i$, content of the last received message, and a current AoI value $\ell$. By the definition of conditional probability, the sought distribution follows as
\begin{align}
    p(\mcn\given\agen,\estn) = \frac{q_{\estn \mcn}(\agen)}{p(\agen\given\estn)}.
\end{align}
The numerator is directly given by the \estn-row, \mcn-column element of the \agen-step transition matrix $\mathbf P^\agen$ of the chain. By the structure of $\mathbf P$, it is immediate to verify that this corresponds to $\mathbf e_{\estn}^{\mathsf T} \mathsf A^{\agen} \,\mathbf e_{\mcn}$. On the other hand, %the conditional probability of having AoI value $\agen$ given a last update of value $\estn$ 
$p(\agen\given\estn)$ can be derived as the probability of the chain not to transition to state $\mathsf d$ for \agen\ steps having started in \estn. This evaluates to $\sum\nolimits_{y\in\mathcal Y\setminus\{\mathsf d\}} q_{\estn y}(\agen) = q_{\estn 0}(\agen)  + q_{\estn 1}(\agen)$, where both addends are again obtained as elements of matrix $\mathbf A^{\agen}$.
\end{proof}
%where 
%\begin{align}
%    q_{\estn\mcn}(\agen) = \mathbf e_{\estn}^{\mathsf T} \,\mathbf P^{\agen} \,\mathbf e_{\mcn}.
%\end{align}
The result allows then to evaluate the performance at the receiver given the current conditions in terms of AoI and estimate, resorting to the definition in \eqref{eq:cond_ent}.
\begin{lemma}
    The receiver uncertainty on \Mcn, given \agen\ and \estn\ can be computed for any channel access strategy $\bm \lambda$ as
        \begin{align}
        \condent(\agen,\estn) = -\sum\nolimits_{\mcn} p(\mcn\given\agen,\estn) \log_2 p(\mcn\given\agen,\estn)
    \end{align}
    where $p(\mcn\given\agen,\estn)$ is obtained through \eqref{eq:pmfXnGivenDeltaXnHat}.
\end{lemma}

Let us now turn our attention to the derivation of the conditional entropy $\ent(\Mcn\given\Agen,\Estn)$ in \eqref{eq:ent_def}, which further requires the joint distribution of the current AoI and estimate available at the receiver at a general time slot $n$, i.e., $p(\agen,\estn) = p(\agen\given\estn) p(\estn)$. In the following, we streamline the steps for its computation through Prop. \ref{prop:condXn} and \ref{prop:statEst}. In turn, these results require a preliminary characterization of the inter-refresh time. Specifically, let us denote by \Irt\ the stochastic process describing the duration between two successive message receptions from the reference source, i.e., between two estimate updates at the sink. With this definition, we have:
\begin{prop}
Conditioned on the current estimate available at the receiver, the probability distribution of the process $W$ and its expected value are given by
\begin{align}
    p(\irt\given\estn) = \mathbf e_{\estn}^{\mathsf T} \mathbf A^{\irt-1} \, \mathbf a_{\mathsf d}
    \label{eq:condPMfW}
\end{align}
\begin{align}
    \mathbb E[\Irt \given \Estn=\estn] = \mathbf e_{\estn}^{\mathsf T} (\mathbf I_2 - \mathsf A)^{-1} \, \mathbf 1_2
    \label{eq:avgW}
\end{align}
where $\mathbf e_{\estn}$, $\estn\in\mathcal X$, is defined as in Prop.\ref{prop1}.
\end{prop}
\begin{proof}
    The results follows by observing that the distribution of \Irt, conditioned on the period being characterized by an estimate value $\hat{x}$ at the receiver, corresponds to the absorption time for the auxiliary chain in \figr\ref{fig:markovChains}b when starting from $\hat{x}$. Note indeed that, counting the steps to absorption starting from state $i\in\{0,1\}$ is equivalent to assuming reception of a message at time $0$ \--- stating that the reference source is in state $i$ \---, and thus starting a period over which the sink will keep $i$ as estimate. In turn, the distribution of the absorption time can be obtained using standard methods for Markov chains \cite{Kemeny76}, leading to the discrete phase-type distribution reported in \eqref{eq:condPMfW}, and the corresponding average absorption time in \eqref{eq:avgW}.        
\end{proof}

The proposition allows to derive the statistics of the current AoI value and the stationary distribution of the estimate.
\begin{prop}\label{prop:condXn}
    Conditioned on the estimate available at the receiver, the current AoI follows the probability distribution
    \begin{align}
        p(\agen\given\estn) = \frac{1}{\mathbf e_{\estn}^{\mathsf T} (\mathbf I_2 - \mathsf A)^{-1} \, \mathbf 1_2} \cdot \sum_{w>\agen} p(w\given \estn).
        \label{eq:condPMFAge_complete}
    \end{align}
    \begin{proof}
        For a generic time instant $n$, let us indicate as $\Irt(n)$ the duration of the estimate inter-refresh period $n$ falls into. The probability that $\Irt(n)$ lasts for $w$ slots follows as
        \begin{align}
        \mathsf P \!\left( \Irt(n)=\irt \given \Estn=\estn \right) = \frac{\irt \, p(\irt\given\estn)}{\sum\nolimits_\irt \irt \, p(\irt\given\estn)}
        \label{eq:condProbW}
        \end{align}
        capturing the fraction of time spent by the system in estimate inter-refresh periods of duration $\irt$. %Focus now on $p(\agen\given\estn)$, i.e., the probability of having an AoI value of \agen\ at a generic time instant $n$, conditioned on having at that time an estimate \estn. 
        Recalling the definition of \Irt, we observe that the AoI of the reference source is $0$ at the beginning of an inter-refresh period, and grows linearly over time, reaching the maximum value of $\Irt-1$ at the start of the last slot of the interval. Therefore, the probability for the receiver to have an AoI $\Agen=\agen$ at a generic time instant $n$ falling into an inter-refresh period of duration $\Irt(n)=\irt$ is simply $1/\irt$, $\forall$ $\agen\in\{0,\dots,\irt-1\}$. Leveraging this, the conditional PMF $p(\agen\given\estn)$ can be conveniently computed as
        \begin{align}
            \begin{split}
            \!\!\!\!\!p(\agen\given\estn) &\stackrel{(a)}{=} \sum\nolimits_{\irt > \agen} \frac{1}{\irt} \cdot \mathsf P\left(\Irt(n) = \irt \given \Estn=\estn\right)\!. %\\
                                %&\stackrel{(b)}{=} \sum_{\irt > \agen} \frac{p(\irt\given\estn)}{\mathbb E[\Irt \given \Estn=\estn]}.                        
            \end{split}
            \label{eq:condPMFAge}
        \end{align}
        Within \eqref{eq:condPMFAge}, $(a)$ follows from the law of total probability, observing that the AoI can reach a value $\agen$ only over an inter-refresh period of length at least $\agen+1$ slots, and using the uniform conditional probability for the AoI that was just derived. Plugging in \eqref{eq:condProbW} and recalling \eqref{eq:avgW} leads after simple steps to the proposition statement.
        %In turn, $(b)$ plugs into the expression the results in \eqref{eq:condProbW}. The proposition statement follows leaning on \eqref{eq:avgW}.                 
    \end{proof}
\end{prop}

\begin{prop}\label{prop:statEst}
    The stationary distribution of the receiver estimate for the reference source is given by    
    \begin{align}
        p(\estn) = \frac{\mathsf c_{\estn} \cdot \mathbb E[W \given \Estn=\estn]}{\mathsf c_{0} \cdot \mathbb E[W \given \Estn=0] + \mathsf c_{1} \cdot \mathbb E[W \given \Estn=1]}.
        \label{eq:pXn}
    \end{align}
    where 
    \begin{align}
        \mathsf c_0 = \frac{(\statZ (1-\qZO) \pTx_{00} + \statO \qOZ \pTx_{10})\ps}{\avgPTx \ps}, \quad  \mathsf c_1 = 1 - \mathsf c_0.
        \label{eq:c0}
    \end{align}
\end{prop}
\begin{proof}
    We start by observing that an inter-refresh period is characterized by having an estimate $\Estn=0$ with probability $\mathsf c_0$ reported in \eqref{eq:c0}. Here, the numerator captures the probability for the source to be in $0$ and to successfully deliver an update, i.e., $\statZ(1-\qZO)\pTx_{00}\ps$, or in $1$, transition to $0$ and inform the receiver, i.e., $\statO\qOZ\pTx_{10}\ps$. In turn, the denominator is a normalizing factor that accounts for the overall probability of delivering an update, i.e., initiating a new inter-refresh period. Similarly, the probability of having an inter-refresh interval with $\Estn=1$ is simply described by the auxiliary variable $\mathsf c_1 = 1-\mathsf c_0$. Leaning on this, the stationary distribution of $\Estn$ can be expressed as the fraction of time the receiver spends with such estimate value, obtaining the expression in \eqref{eq:pXn}.
\end{proof}

To conclude, the joint PMF $p(\agen,\estn)$ can be computed using \eqref{eq:condPMFAge_complete} and \eqref{eq:pXn}, eventually providing a complete analytical characterization of the conditional entropy $H(\Mcn\given\Agen,\Estn)$.

%To characterize this, we again resort to the auxiliary Markov process in \figr\ref{fig:abs_mc}, and  
%As a starting point, we observe that the chain can be used to capture the inter-refresh time of the estimate at the sink for the source of interest. Specifically, let us denote by \Irt\ the stochastic process describing the duration between two successive receptions of a message from the reference source, i.e., between two updates of the corresponding estimate at the sink. With this definition, the distribution of \Irt, conditioned on the period being characterized by an estimate value $\hat{x}$ at the receiver, is simply the absorption time for the chain when starting from $\hat{x}$.\footnote{Note indeed that counting the steps to absorption starting from state $i\in\{0,1\}$ is equivalent to assuming that at time $0$ a message is received, stating that the reference source is in such state, and thus starting a period over which the sink has an estimate $i$.} Accordingly, $p(\irt\given\estn)$ can be obtained using standard methods for Markov chains \cite{} as the discrete phase-type distribution
%\begin{align}
%    p(\irt\given\estn) = \mathbf e_{\estn}^{\mathsf T} \mathbf A^{\irt-1} \, \mathbf a_{\mathsf d}.
%\end{align}


\begin{table}[ht!]
\centering
\caption{\textbf{Super Resolution Performance Results.} Our proposed WGAN EEG Spatial Upsampling method significantly outperforms a baseline of Bicubic Interpolation commonly used in EEG upsampling pipelines.}
\label{tab:results}
\resizebox{0.8\linewidth}{!}{%
\begin{tabular}{@{}cccccc@{}}
\toprule
\multirow{2}{*}{\textbf{Dataset}} & \multirow{2}{*}{\textbf{Scale}} & \multicolumn{2}{c}{\textbf{Bicubic}} & \multicolumn{2}{c}{\textbf{WGAN}} \\ \cmidrule(l){3-6} 
                      &   & \textbf{MSE} & \textbf{MAE} & \textbf{MSE}    & \textbf{MAE}   \\
\toprule
\multirow{2}{*}{Val}  & 2 & 3.71E7       & 3.89E3       & \textbf{2.01E3} & \textbf{24.38} \\
                      & 4 & 7.23E7       & 6.42E3       & \textbf{8.53E3} & \textbf{63.83} \\
\midrule
\multirow{2}{*}{Test} & 2 & 3.75E7       & 3.91E3       & \textbf{2.06E3} & \textbf{24.66} \\
                      & 4 & 7.30E7       & 6.45E3       & \textbf{8.68E3} & \textbf{64.39} \\
\bottomrule
\end{tabular}%
}
\end{table}

%\clearpage
%\newpage
\bibliographystyle{IEEEtran}
\bibliography{IEEEabrv,biblio_RandomAccess,biblio_AoI}

\flushend

\end{document}
