\documentclass[conference]{IEEEtran}
\usepackage{times}

% numbers option provides compact numerical references in the text. 
\usepackage[sort&compress,numbers]{natbib}
\usepackage{multicol}
\usepackage[bookmarks=true]{hyperref}

% Define a command to switch to Monaco font
\usepackage{wrapfig}
\usepackage{url}            % simple URL typesetting
\usepackage{amsfonts}       % blackboard math symbols
\usepackage{nicefrac}       % compact symbols for 1/2, etc.
\usepackage{microtype}      % microtypography
\usepackage{xcolor}         % colors
\usepackage{amsmath,amssymb} % define this before the line numbering.
\usepackage{pifont}  % For \Xmark
\usepackage{makecell}
\usepackage{diagbox}
\usepackage{booktabs}
% % Support for easy cross-referencing
\usepackage{graphics} % for pdf, bitmapped graphics files
\usepackage{adjustbox}
\usepackage{colortbl}
\usepackage{xcolor}
% \usepackage{epsfig} % for postscript graphics files
\usepackage{empheq}
%\usepackage{mathptmx} % assumes new font selection scheme installed
% \usepackage{times} % assumes new font selection scheme installed
\usepackage{bm}
\usepackage{bbding} 
\usepackage[textwidth=1in]{todonotes}
% \usepackage{cite}
\usepackage{diagbox}
% \usepackage[linesnumbered,ruled]{algorithm2e}
% \usepackage{ulem} %to strike the words
% \usepackage{hyperref}
% \usepackage{soul}
\usepackage[small]{caption}

\newcommand{\cmark}{\ding{51}}%
\newcommand{\xmark}{\ding{55}}%
\definecolor{lightblue}{HTML}{DBE9FC}
\definecolor{lighterblue}{HTML}{f1fcfe}
\definecolor{darkblue}{HTML}{6c8ebf}
% \definecolor{themeblue}{RGB}{57, 162, 219}
% \definecolor{themegreen}{RGB}{87, 204, 153}
% \definecolor{forestgreen}{RGB}{47, 159, 87}

\usepackage[capitalize]{cleveref}
% \usepackage{todonotes}
\usepackage{float}
\usepackage{booktabs}
\usepackage{multirow}
%\usepackage{ bbold }
\usepackage{mathrsfs}
\usepackage[utf8]{inputenc}
%\usepackage{subfigure}
\usepackage{pifont}
\usepackage{threeparttable}
\usepackage[lined,boxed,commentsnumbered,ruled,vlined]{algorithm2e}
\usepackage{tcolorbox}
% \usepackage[finalizecache,cachedir=.]{minted}
\usepackage[frozencache,cachedir=.]{minted}
\SetKwComment{Comment}{$\triangleright$\ }{}

\hypersetup{
    colorlinks,
    linkcolor={red!50!black},
    citecolor={blue!50!black},
    urlcolor={blue!80!black}
}

\let\llncssubparagraph\subparagraph
%% Provide a definition to \subparagraph to keep titlesec happy
\let\subparagraph\paragraph
%% Load titlesec
\usepackage[compact]{titlesec}
%% Revert \subparagraph to the llncs definition
% \let\subparagraph\llncssubparagraph
% \titlespacing{\section}{0pt}{1.5ex}{1.2ex}
% \titlespacing{\subsection}{0pt}{1.2ex}{1.0ex}
% \titlespacing{\subsubsection}{0pt}{0.5ex}{0.1ex}
% \titlespacing{\paragraph}{5ex}{2ex}{2ex}
% \setlength{\parskip}{2pt}
\usepackage{amsthm}

\theoremstyle{definition}
\newtheorem{theorem}{Theorem}%[section]
% \newtheorem{bigTheorem}{The Big Theorem}%[section]
\newtheorem{corollary}{Corollary}%[theorem]
\newtheorem{lemma}{Lemma}%[theorem]
\newtheorem{assumption}{Assumption}%[theorem]
\newenvironment{proofsketch}{%
  \renewcommand{\proofname}{Proof Sketch}\proof}{\endproof}
\newtheorem{definition}{Definition}
\newenvironment{problemsetting}[1][htb]
  {\renewcommand{\algorithmcfname}{Problem Setting}% Update algorithm name
  \begin{algorithm}[#1]%
  }{\end{algorithm}}
% Load basic packages
\usepackage{balance}       % to better equalize the last page
\usepackage{graphics}      % for EPS, load graphicx instead 
\usepackage{hyperref}
\usepackage{color}
\usepackage{booktabs}
\usepackage{textcomp}
\usepackage{subcaption}
\usepackage{enumerate}
\usepackage{xcolor}
\usepackage{lipsum}% http://ctan.org/pkg/lipsum
\usepackage{makecell}
\usepackage{multicol}
\usepackage{multirow}
\usepackage{array}
\usepackage{verbatimbox}
\usepackage{enumitem}
\usepackage{amsmath}
\usepackage{stfloats}
\usepackage{graphicx}
\usepackage{amsthm}
\usepackage{listings}
\usepackage{caption} 
\usepackage[export]{adjustbox}
\usepackage{xspace}
\usepackage{epsfig}
\usepackage[linesnumbered]{algorithm2e}
\usepackage{algpseudocode}
\usepackage{tabularx}
\usepackage{arydshln}
\usepackage[bottom]{footmisc}
\usepackage{tcolorbox}
\usepackage{stackengine}
\usepackage{placeins}
\usepackage{graphicx}

\newcommand*{\eg}{\textit{e.g.},\xspace}
\newcommand*{\ie}{\textit{i.e.},\xspace}
\newcommand*{\vs}{\textit{vs.}\xspace}
\newcommand*{\etc}{\textit{etc.}}
\newcommand*{\st}{\textit{s.t.},\xspace}
\newcommand*{\etal}{\textit{et~al.}\xspace}
\newcommand*{\hlinespace}{\addlinespace[1ex]\hline\addlinespace[1ex]}
\newcommand*{\hdashlinespace}{\addlinespace[1ex]\hdashline\addlinespace[1ex]}
\newcommand*{\cdashlinespace}[1]{\addlinespace[1ex]\cdashline{#1}\addlinespace[1ex]}
\newcommand{\clinespace}[1]{\addlinespace[1ex]\cline{#1}\addlinespace[1ex]}

\newcolumntype{L}[1]{>{\raggedright\let\newline\\\arraybackslash\hspace{0pt}}m{#1}}
\newcolumntype{C}[1]{>{\centering\let\newline\\\arraybackslash\hspace{0pt}}m{#1}}
\newcolumntype{R}[1]{>{\raggedleft\let\newline\\\arraybackslash\hspace{0pt}}m{#1}}



\makeatletter
\def\thickhline{%
  \noalign{\ifnum0=`}\fi\hrule \@height \thickarrayrulewidth \futurelet
   \reserved@a\@xthickhline}
\def\@xthickhline{\ifx\reserved@a\thickhline
               \vskip\doublerulesep
               \vskip-\thickarrayrulewidth
             \fi
      \ifnum0=`{\fi}}
\makeatother

\makeatletter
\def\thickhlinespace{%
  \addlinespace[1ex]
  \noalign{\ifnum0=`}\fi\hrule \@height \thickarrayrulewidth \futurelet
   \reserved@a\@xthickhline
   \addlinespace[1ex]
   }
\def\@xthickhlinespace{\ifx\reserved@a\thickhline
               \vskip\doublerulesep
               \vskip-\thickarrayrulewidth
             \fi
      \ifnum0=`{\fi}}
\makeatother

\newlength{\thickarrayrulewidth}
\setlength{\thickarrayrulewidth}{3\arrayrulewidth}


\newlength\Origarrayrulewidth

% horizontal rule equivalent to \cline but with 2pt width
\newcommand{\Cline}[1]{%
 \noalign{\global\setlength\Origarrayrulewidth{\arrayrulewidth}}%
 \noalign{\global\setlength\arrayrulewidth{2pt}}\cline{#1}%
 \noalign{\global\setlength\arrayrulewidth{\Origarrayrulewidth}}%
}

% draw a vertical rule of width 2pt on both sides of a cell
\newcommand\Thickvrule[1]{%
  \multicolumn{1}{!{\vrule width 2pt}c!{\vrule width 2pt}}{#1}%
}

% draw a vertical rule of width 2pt on the left side of a cell
\newcommand\Thickvrulel[1]{%
  \multicolumn{1}{!{\vrule width 2pt}c|}{#1}%
}

% draw a vertical rule of width 2pt on the right side of a cell
\newcommand\Thickvruler[1]{%
  \multicolumn{1}{|c!{\vrule width 2pt}}{#1}%
}

\DeclareMathOperator*{\argmin}{argmin}   % Jan Hlavacek
\DeclareMathOperator*{\argmax}{argmax}   % Jan Hlavacek

\newcommand{\algrule}[1][.2pt]{\par\vskip.5\baselineskip\hrule height #1\par\vskip.5\baselineskip}

\algnewcommand{\IfThenElse}[3]{% \IfThenElse{<if>}{<then>}{<else>}
  \State \algorithmicif\ #1\ \algorithmicthen\ #2\ \algorithmicelse\ #3}
  
\newenvironment{s_itemize}{
\begin{itemize}[leftmargin=*]
  \setlength{\itemsep}{3pt}
  \setlength{\parskip}{0pt}
  \setlength{\parsep}{0pt}
}{\end{itemize}}

\newenvironment{s_enumerate}{
\begin{enumerate}[leftmargin=*]
  \setlength{\itemsep}{3pt}
  \setlength{\parskip}{0pt}
  \setlength{\parsep}{0pt}
}{\end{enumerate}}

\newcommand\acomment[1]{\textcolor{orange}{\textit{Anind: #1}}}
\newcommand\anind[1]{\textcolor{orange}{\textit{Anind: #1}}}
\newcommand\jcomment[1]{\textcolor{red}{\textit{Jen: #1}}}
\newcommand\jm[1]{\textcolor{red}{\textit{Jen: #1}}}
\newcommand\jen[1]{\textcolor{red}{\textit{Jen: #1}}}
\newcommand\tim[1]{\textcolor{magenta}{\textit{Tim: #1}}}
\newcommand\ocomment[1]{\textcolor{blue}{\textit{Orson: #1}}}
\newcommand\orson[1]{\textcolor{blue}{\textit{Orson: #1}}}
\newcommand\needinput[1]{\textcolor{red}{\textit{#1}}}


\newcommand\downred[1]{\textcolor{downredcolor}{#1}}
\newcommand\upgreen[1]{\textcolor{upgreencolor}{#1}}
\definecolor{downredcolor}{HTML}{e31a1c}
\definecolor{upgreencolor}{HTML}{33a02c}

\definecolor{DarkGreen}{HTML}{5DAC81}
% \newcommand\review[1]{\textcolor{DarkGreen}{#1}}
% \newcommand\minorreview[1]{\textcolor{DarkGreen}{#1}}
\newcommand\review[1]{\textcolor{black}{#1}}
\newcommand\minorreview[1]{\textcolor{black}{#1}}
% \newcommand\review[1]{\textcolor{black}{#1}}


\newcounter{RNum}
\renewcommand{\theRNum}{\arabic{RNum}}
\newcommand{\Remark}{\noindent\textbf{Remark}~\refstepcounter{RNum}\textbf{\theRNum}: }
\newcommand{\fref}[1]{Figure~\ref{#1}}
\newcommand{\sref}[1]{Section~\ref{#1}}
\newcommand{\tref}[1]{Table~\ref{#1}}
\newcommand{\appref}[1]{Appendix~\ref{#1}}
\newcommand{\defref}[1]{Definition~\ref{#1}}
\newcommand{\highlight}[1]{\noindent\quad\textbf{#1}:~}
\newcommand{\myparagraph}[1]{\noindent\textbf{#1}~}

\newcommand{\tablestyle}[2]{\setlength{\tabcolsep}{#1}\renewcommand{\arraystretch}{#2}\centering\footnotesize}

% https://tex.stackexchange.com/questions/170156/how-do-i-put-a-figure-before-my-abstract
\makeatletter
\apptocmd\@maketitle{{\myfigure{}\par}}{}{}
\newcommand{\removelatexerror}{\let\@latex@error\@gobble}
\makeatother
\newcommand\blfootnote[1]{%
  \begingroup
  \renewcommand\thefootnote{}\footnote{#1}%
  \addtocounter{footnote}{-1}%
  \endgroup
}

\newcommand{\td}[1]{\textcolor{blue}{[TODO]: #1\:}}
% To Be decided..
\def\model{IVNTR}

\pdfinfo{}
\setlength {\marginparwidth }{2cm}
\begin{document}

\newcommand\myfigure{%
\centering
\noindent
\includegraphics[width=\textwidth]{imgs/fig1.pdf}
\captionof{figure}{(Left) Our bilevel learning framework (\model{}) invents neural \textbf{relational} predicates from training demonstrations (with one platform), which enable the learning of a hybrid bilevel planner.
(Right) The invented predicates realize \textbf{zero-shot} compositional generalization over objects (with two platforms). 
Continuous action parameters are omitted in the figure for simplicity.}
\label{fig:teaser}
\setcounter{figure}{1}
}

% paper title
\title{Bilevel Learning for Bilevel Planning}

\author{\authorblockN{Bowen Li\authorrefmark{1}\authorrefmark{2},
Tom Silver\authorrefmark{3},
Sebastian Scherer\authorrefmark{1}, and
Alexander Gray\authorrefmark{2}}
\authorblockA{\authorrefmark{1}Carnegie Mellon University, \authorrefmark{2}Centaur AI Institute, \authorrefmark{3}Princeton University}}

\maketitle

\begin{abstract}
A robot that learns from demonstrations should not just imitate what it sees---it should understand the high-level concepts that are being demonstrated and generalize them to new tasks.
Bilevel planning is a hierarchical model-based approach where predicates (relational state abstractions) can be leveraged to achieve compositional generalization.
However, previous bilevel planning approaches depend on predicates that are either hand-engineered or restricted to very simple forms, limiting their scalability to sophisticated, high-dimensional state spaces.
To address this limitation, we present \model{}, the first bilevel planning approach capable of learning neural predicates directly from demonstrations.
Our key innovation is a neuro-symbolic bilevel learning framework that mirrors the structure of bilevel planning.
In \model{}, symbolic learning of the predicate ``effects" and neural learning of the predicate ``functions" alternate, with each providing guidance for the other.
We evaluate \model{} in six diverse robot planning domains, demonstrating its effectiveness in abstracting various continuous and high-dimensional states.
While most existing approaches struggle to generalize (with $<35\%$ success rate), our \model{} achieves an average of $77\%$ success rate on unseen tasks.
Additionally, we showcase \model{} on a mobile manipulator, where it learns to perform real-world mobile manipulation tasks and generalizes to unseen test scenarios that feature new objects, new states, and longer task horizons.
Our findings underscore the promise of learning and planning with abstractions as a path towards high-level generalization.
\blfootnote{$^\dagger$Work was partly done during internship at Centaur AI Institute. Correspondence to \{bowenli2,basti\}@andrew.cmu.edu.}
\end{abstract}

\IEEEpeerreviewmaketitle

\section{Introduction}

Imitation learning has made significant recent strides~\cite{mandlekar2023mimicgen,chi2023diffusionpolicy,zhao2023learning,wang2024equivariant,yang2024equibot}, but generalization remains an open challenge, especially when new tasks require recomposing high-level concepts that are only implicit in the training data~\cite{mao2024planning,li2024logicity}.
In \fref{fig:teaser}, a robot has seen demonstrations of stepping onto a platform to grasp an object, and other demonstrations of dumping the object into a container.
Now, faced with a new task where the container is also elevated, the robot should first move the two platforms appropriately, step onto one of them to grasp the object, and finally step on the other to dump the object.
Note that the platform arrangements must be completed before grasping the target, since the robot cannot move platforms with its hand full.
This kind of learning and reasoning requires compositional generalization (new objects); sequential generalization (new and longer action sequences); and long-horizon planning with continuous state and action spaces with sparse feedback (goals).
In sum, the robot should not just \emph{imitate} demonstrations, but also \emph{understand and leverage} the high-level concepts within the low-level states that are being demonstrated.

One promising direction to address these challenges is to learn and plan with \textit{abstractions}~\cite{li2006towards,abel2018state,konidaris2018symbols,wonglearning,curtis2022discovering,yang2024guidinglonghorizontaskmotion}.
In this work, we continue a line of recent inquiry on learning abstractions for \emph{bilevel planning}~\cite{silver2021operator,silver2022skills,silver2023predicateinvent,chitnis2021glib,kumar2024practice,kumar2023predict,liang2024visualpredicator}.
In bilevel planning, continuous low-level states are mapped into a symbolic relational state space defined by \emph{predicates} such as \texttt{Viewable(?robot,?target)} or \texttt{On(?robot,?platform)}.
Planning proceeds jointly in the symbolic high-level space and the continuous low-level space.
The key idea is that this hybrid planning can be more efficient and effective than reasoning solely in the low-level space.

The performance of bilevel planning depends substantially on the predicates used to define the abstract state space~\cite{silver2023predicateinvent}.
To avoid the need for a human engineer to manually define predicates for every new domain, recent work has considered \emph{learning predicates} from data~\cite{kulick2013active,konidaris2018skills,silver2023predicateinvent,li2023embodied,han2024interpret,liang2024visualpredicator}.
Broadly, three approaches have emerged.
The most direct one relies on human feedback (labels or guidance) during the predicate learning process~\cite{li2023embodied,migimatsu2022grounding,han2024interpret}, which is labor-intensive and does not guarantee useful abstractions for planning~\cite{silver2023predicateinvent}.
The second approach \textit{invents} predicates with surrogate objectives that are easy to optimize, \textit{e.g.}, reconstruction loss~\cite{asai2018latplan_prop,asai2019latplan_fol,asai2021latplanpddl} or bisimulation~\cite{curtis2022discovering,hansen2022bisimulation}.
While these methods simplify learning, they complicate planning due to the mismatch between the surrogate objectives and the actual planning goals~\cite{silver2023predicateinvent}.
The third approach directly invents predicates for efficient planning, making planning ``easy" but learning ``hard," as objectives like \emph{total-planning-time} and \emph{expected-planning-success} are difficult to optimize~\cite{silver2023predicateinvent}.
To address this, previous works have used program synthesis with classical grammars~\cite{silver2023predicateinvent} and foundation model-based techniques~\cite{liang2024visualpredicator,athalye2024predicate}.
However, in both cases, the predicates are invented from programmatic and pre-defined functions, which are limited in flexibility and scalability.

Our main contribution is b\textbf{I}le\textbf{V}el lear\textbf{N}ing from \textbf{TR}ansitions (\model{}), the first approach capable of learning \emph{neural} predicates that are optimized for efficient and effective bilevel planning.
Since directly incorporating the planning objective into network training is challenging, our \model{} instead constructs a candidate neural predicate pool, which is later subselected~\cite{silver2023predicateinvent}.
The key insight behind our approach is to center learning around the \emph{effects} of predicates, which provide two major benefits: 
(1) they enable the derivation of supervision labels for \textit{transition} pairs, yielding a well-structured learning objective for training the neural network; 
and (2) the inherent sparsity of predicate effects, combined with neural learning signals, facilitates efficient symbolic learning of their structure. 
To this end, \model{} presents an innovative bilevel learning framework, inspired by the structure of bilevel planning itself.
Similar to the alternation between high-level symbolic search and low-level neural sampling in bilevel planning, \model{} interleaves symbolic effect learning and neural function learning in an iterative process.
In each iteration, the symbolic learning proposes a candidate predicate effect across different actions, which provides labels for neural learning on transition pairs. 
Once the neural classifier converges, its validation loss guides the symbolic learning to propose the next candidate that could minimize the loss in the new iteration.
This iterative bilevel learning ultimately yields a compact set of neural predicates, which are then selected to optimize the planning objective~\cite{silver2023predicateinvent}.
The final set of invented predicates seamlessly integrates into operator and sampler learning frameworks~\cite{chitnis2021nsrt,silver2023predicateinvent}, ultimately forming a fully functional bilevel planner.

To evaluate the effectiveness of \model{}, we conduct extensive experiments across six diverse robot planning domains. 
These domains feature a wide range of low-level state representations, from SE(2) and SE(3) poses to high-dimensional point clouds. 
Furthermore, as shown in \fref{fig:teaser}, by leveraging relational predicates and AI planning, \model{} zero-shot generalizes to tasks with unseen entity compositions.
Finally, we deploy \model{} on a quadruped mobile manipulator (Boston Dynamics Spot) for two long-horizon mobile manipulation tasks. 
The learned predicates successfully abstract complex continuous states into representations compatible with the task planner, while also providing actionable guidance for the motion planner. 
We believe \model{} represents a pivotal step towards learning high-level abstractions from sophisticated low-level states.

\section{Problem Formulation}\label{sec:problem}

We propose a method that uses an offline demonstration dataset to learn planning \textit{abstractions} that generalize to test tasks with unseen objects and action compositions.
In this section, we describe the formal problem setting.
We follow the notation system introduced in previous work~\cite{silver2023predicateinvent}; see \appref{app:notation} for a complete notation glossary.

Planning problems are defined within a certain \textit{planning domain} $\langle \Lambda, \mathcal{C}, f, \Psi_{\mathrm{g}}, \Psi_\mathrm{sta} \rangle$ with a task distribution $\mathcal{T}$, where we can sample a \textit{planning task} $T\sim\mathcal{T}=\langle \mathcal{O}, \mathbf{x}_0, g\rangle$.

$\Lambda$ is a finite set of object \textit{types} $\lambda\in\Lambda$.
For example, the Climb-Transport domain depicted in \fref{fig:running_example} has three object types: $\Lambda=\{\mathrm{robot} (\mathtt{r}), \mathrm{platform}(\mathtt{p}), \mathrm{target}(\mathtt{t})\}$.
Each type is associated with a set of \emph{features} that characterize the state of an object of that type.\footnote{Unlike previous work~\cite{silver2023predicateinvent}, we do not assume that features are scalars; high-dimensional images and point clouds are also allowed.}
For example, $\mathrm{robot}$ has features ``BasePose'', ``HandPose'', and ``GripperOpenPercent'', among others.
A specific \textit{task} $T$ is characterized by objects $\mathcal{O}=\{\mathtt{o}_1,\mathtt{o}_2,\cdots,\mathtt{o}_N\}$, each associated with one type in $\Lambda$.
Objects are fixed within tasks but vary between tasks.
The state of a task $\mathbf{x}\in\mathcal{X}$ is defined by an assignment of feature values to all objects in the task.
For simplicity of exposition, we assume that a state with $N$ objects can be represented as a matrix $\mathbf{x} \in \mathbb{R}^{N\times K}$ for some domain-specific constant $K$; however, we show in experiments that our approach can be applied to more sophisticated object-centric state representations as well.

The action space for a domain is characterized by a set of $M$ \textit{parametrized controllers} $\mathcal{C} = \{\mathtt{C}_1, \mathtt{C}_2, \cdots, \mathtt{C}_M\}$, each of which has an object type signature $(\lambda_1, \lambda_2, \cdots, \lambda_{v})$ and a continuous parameter space $\Omega$.
For example, in \fref{fig:running_example}, \texttt{MoveToReach} has type signature $(\mathrm{robot}, \mathrm{platform})$, and continuous parameters $\Omega = \text{SE}(2)$ defining an offset 2D pose for the robot relative to the platform.
A \emph{ground action} is a controller with fully specified parameters, e.g., $\texttt{MoveToReach}(\mathtt{r}_1, \mathtt{p}_1, \omega)$ for a certain $\omega \in \Omega$.
We use underline notation to represent grounding: $\underline{\mathtt{C}}$ is a certain ground action.
A \emph{lifted action} is controller with object parameter placeholders, which are typically prefixed with ?, e.g., $\texttt{MoveToReach}(\mathrm{?r}, \mathrm{?p}, \cdot)$.
States and actions are related through
a known transition function $f(\mathbf{x}, \underline{\mathtt{C}}) \mapsto \mathbf{x}'$, which the robot can use to plan.

A \emph{predicate} $\psi$ is defined by an object type signature $(\lambda_1, \lambda_2, \ldots, \lambda_{u})$ and 
a classifier $\theta_{\psi}: \mathcal{X}\times\mathcal{O} \to \{\mathrm{True}, \mathrm{False}\}$, where $\theta_{\psi}(\mathbf{x}, (\mathtt{o}_1, \ldots, \mathtt{o}_{u}))$ evaluates the truth value of a ground predicate based on the continuous features of the input objects.
For example, the predicate \texttt{In} has type signature $(\mathrm{target}, \mathrm{target})$ and a classifier that uses the poses and shapes of two targets to determine whether one is ``in'' the other.
A \emph{ground predicate} $\underline{\psi}$ has fully specified objects.
For simplicity, we denote
$\theta_{\underline{\psi}}(\mathbf{x}) \triangleq \theta_{\psi}\left(\mathbf{x}, \left(\mathtt{o}_1, \ldots, \mathtt{o}_{u}\right)\right).$
A \emph{lifted predicate} has placeholders for objects, e.g., $\texttt{In}(\texttt{?t}, \texttt{?t})$.

Following previous work~\cite{silver2023predicateinvent}, we assume that a small set of \emph{goal predicates} $\Psi_G$ is known and used to characterize task goals.
In particular, a goal $g$ is defined by a set of ground predicates that must evaluate to True in a state for the goal to be satisfied.
For example, the goal in Figure~\ref{fig:running_example} has only one ground predicate, $\texttt{In}(\texttt{t}_1,\texttt{t}_2)$.
In this work, we make an additional assumption that any relevant \emph{static predicates} $\Psi_\mathrm{sta}$ are known.
A predicate is static if its evaluation never changes within a task (see \appref{app:domain_details} for examples).
Conversely, a predicate is \textit{dynamic} if its evaluation could change within a task; examples are provided later in \defref{def:op}.

A solution to a task is a plan $\pi=[\underline{\mathtt{C}}_1, \underline{\mathtt{C}}_2, \cdots, \underline{\mathtt{C}}_H]$, that is, a sequence of $H$ ground actions such that successive application of the transition model $\mathbf{x}_i=f\left(\mathbf{x}_{i-1}, \underline{\mathtt{C}}_i\right)$ on each $\underline{\mathtt{C}}_i \in \pi$, starting from $\mathbf{x}_0$, results in a final state $\mathbf{x}_H$ where $g$ holds.
For instance, the plan depicted in \fref{fig:teaser} right bottom finally leads to the state where $\texttt{In}(\texttt{t}_1,\texttt{t}_2)$ holds.
In sum, to generate the plan $\pi$ for a task, in each state, the robot needs to predict:
(1) the action class $\mathtt{C}\in\mathcal{C}$,
(2) the objects as the discrete parameters,
and 
(3) the continuous parameter $\omega\in\Omega$.

During training, the robot has access to an offline demonstration dataset $\mathcal{D} =\{({T}_i, \pi_i)\}_{i=1}^B$, which consists of $B$ task and solution pairs sampled from the task distribution $\mathcal{T}^\mathrm{train}$.
Note that since the transition function $f$ is known and deterministic, we can also recover the intermediate states from $\mathbf{x}_0$.
During test time, the robot is required to solve held-out tasks sampled from a different \textit{test} distribution $\mathcal{T}^\mathrm{test}$.
In practice, for the sake of evaluating generalization, test tasks typically contain new and more objects than training tasks.
For example, as shown in \fref{fig:running_example}, all training tasks only have $1$ platform, but during test, there are $2$ platforms.
The difference in object compositions could result in different lengths of plans with different action order, requiring the method to \textit{generalize} by understanding the implicit concepts present in the training demonstrations.

\section{Bilevel Planning}\label{sec:bilevel_planning}
\begin{figure}[!t]
	\centering
	\includegraphics[width=1\columnwidth]{imgs/fig2.pdf}
 \vspace{-0.4cm}
	\caption{The Climb-Transport domain is presented as a running example. We have displayed one typical training and one test task on the top. The types, actions, and provided predicates are shown at the bottom.}
 \vspace{-0.3cm}
	\label{fig:running_example}
\end{figure}

\begin{figure*}[!t]
	\centering
	\includegraphics[width=2\columnwidth]{imgs/fig3.pdf}
 \vspace{-0.1cm}
	\caption{(a) Overview of \model{} during training. Given transition pairs in the continuous space, \model{} invents neural predicates group by group, resulting in the candidate set. A subset that minimize the planning objective $J(\cdot)$ is selected from the candidates, which serves as the finial $\Psi_{\mathrm{dyn}}$. With the complete predicate set, sampler and operator learning can be achieved.
    (b) Bilevel planning with operators and samplers during test. The compositional ground predicates provides search configurations for task planner and sampling guidance for motion planner.
    }
 \vspace{-0.3cm}
	\label{fig:overview}
\end{figure*}

In this work, we propose a method for learning predicates that can be used for \emph{bilevel planning}.
We now provide a brief review of bilevel planning and refer to other references for a more in-depth discussion~\cite{chitnis2021nsrt,liang2024visualpredicator,silver2023predicateinvent,silver2022skills,li2023embodied,kumar2023predict,kumar2024practice,silver2021operator,garrett2021integrated}.

Bilevel planning uses relational abstractions to achieve sequential and compositional generalization.
The two principal abstractions are \emph{predicates} (state abstractions) and \emph{operators} (action abstractions).
Bilevel planning also uses relational \emph{samplers} to generate possible ground actions from operators.
The key idea is that planning jointly in the abstract transition system and the low-level transition system can be much more efficient than planning in the low-level transition space only.

\begin{definition}[Operator]\label{def:op}
    The \textit{operator} for a parametrized controller $\mathtt{C}$ is a tuple, $\mathtt{Op}^\mathtt{C} = \langle \mathtt{Var}, \mathtt{Pre}, \mathtt{Eff}^+, \mathtt{Eff}^- \rangle$, where
    $\mathtt{Var}$ is a tuple of object placeholders matching the type signature of $\mathtt{C}$, and
  $\mathtt{Pre},\,\mathtt{Eff}^+,\,\mathtt{Eff}^- \subseteq \Psi$, respectively \emph{preconditions}, \emph{add effects}, and \emph{delete effects}, are each a set of lifted predicates defined with variables in $\mathtt{Var}$. $\Psi$ is an oracle predicate set.
\end{definition}
For example, the operator for $\mathtt{Grasp(?r,?t)}$ could be:
\begin{align*}
& \mathtt{Pre}=\{\mathtt{HandEmpty(?r)},\mathtt{HandSees(?r,?t)}\},\\
& \mathtt{Eff}^+=\{\mathtt{Holding(?r,?t)}\},\\
& \mathtt{Eff}^-=\{\mathtt{HandEmpty(?r)},\mathtt{HandSees(?r,?t)}\}.
\end{align*}
Given a task $T = \langle \mathcal{O}, \mathbf{x}_0, g\rangle$, bilevel planning (Figure~\ref{fig:overview}b) starts by using predicates to generate an \emph{abstract state} consisting of all ground predicates with objects $\mathcal{O}$ whose classifiers evaluate to True in $\mathbf{x}_0$.
The initial ground predicates, together with the operator set and goal $g$, can then be fed into an AI planner~\cite{helmert2006fast} to generate a plan \textit{skeleton}, $\bar{\pi}$ with partially grounded actions with unspecified continuous parameters, $\underline{\bar{\mathtt{C}}}$.
To refine the actions in this skeleton $\underline{\bar{\mathtt{C}}}\in\bar{\pi}$ into fully grounded $\underline{{\mathtt{C}}}$ with the continuous parameters $\omega$, bilevel planning leverages \textit{samplers}.
\begin{definition}[Sampler]
    The \textit{sampler} $\eta^\mathtt{C}$ for a planning operator $\mathtt{Op}^\mathtt{C}$ with $v$ object placeholders is a conditional distribution $\omega \sim \eta^\mathtt{C}(\cdot \mid \mathbf{x}, (\mathtt{o}_1, \ldots, \mathtt{o}_{v}))$ that proposes continuous action parameters for $\mathtt{C}((\mathtt{o}_1, \ldots, \mathtt{o}_{v}), \cdot)$ given a state $\mathbf{x}$.
\end{definition}
Note that unlike the deterministic operators, samplers are usually stochastic and may fail in certain steps. 
Thus, bilevel planning alternates between the task planner and motion planner and uses the predicate functions $\theta_\Psi$ as ``guidance" in each step to compensate for the potential sampling failure.

Assuming we have the complete predicate set and their classifiers, previous work has studied the problem of learning \textit{operators}~\cite{chitnis2021nsrt} and \textit{samplers}~\cite{kumar2024practice,silver2022skills} from the demonstration dataset $\mathcal{D}^\mathrm{demo}$.
Since the predicates, learned operators, and samplers are \textit{relational}, they can be generally applied to held out test tasks sampled from $\mathcal{T}^\mathrm{test}$.
However, with an insufficient predicate set---for example, with only $\Psi_\mathrm{G}$ and $\Psi_\mathrm{sta}$---bilevel planning can be intractably slow~\cite{silver2023predicateinvent}.
We next introduce the \model{} framework that closes this gap by automatically inventing dynamic predicates for efficient bilevel planning.

\section{Methodology}\label{sec:ivntr}

The problem of inventing dynamic predicates $\Psi_\mathrm{dyn}$ can be decomposed into \emph{symbolic learning}---how many predicates should be invented, and with what type signatures---and \emph{classifier learning}, determining $\theta_\psi$ for each invented predicate $\psi \in \Psi_\mathrm{dyn}$.
Previous approaches~\cite{liang2024visualpredicator,silver2023predicateinvent} address these problems via a ``define-then-select" two-stage pipeline.
In the first stage, for each possible typed predicate candidate $\hat{\psi}$, its function is pre-defined via program synthesis~\cite{silver2023predicateinvent} or pre-trained foundation models~\cite{liang2024visualpredicator}.
These candidates form a large predicate pool $\hat{\Psi}_\mathrm{dyn}$.
In the second stage, to subselect predicates from the pool, each candidate predicate set $\tilde{\Psi}_\mathrm{dyn}\subseteq\hat{\Psi}_\mathrm{dyn}$ is scored with a function $J(\tilde{\Psi}_\mathrm{dyn})$ that measures both planning \emph{efficiency} and \emph{effectiveness}.
A key limitation of this ``define-then-select'' pipeline is that the classification functions within $\hat{\Psi}_\mathrm{dyn}$ are restricted to a relatively simple set.
Scaling to more general classifiers, e.g., neural networks, is nontrivial, since $J(\tilde{\Psi}_\mathrm{dyn})$ is highly non-differentiable.
To address this, we propose \model{}, a ``learn-then-select'' approach that leverages \emph{bilevel learning}.

As depicted in \fref{fig:overview} (a), given the domain types $\Lambda$, \model{} enumerates all possible typed variable compositions (with maximum input arity).
Since predicates with different typed variables take different input object features, \model{} invents them group by group independently.
For the invention of each group, \model{} draws inspiration from bilevel planning itself, where planning alternates between the symbolic level and the low level.
Similarly, \model{} interleaves \emph{symbolic learning} and \emph{neural learning}, with each providing guidance for the other.
Specifically, symbolic learning proposes \emph{effect vectors} that represent the add and delete effects for candidate predicates across all operators.
Neural learning uses these effect vectors to provide supervision for classifier learning.
The validation loss in neural learning feeds back into symbolic learning, and the process repeats.
In this section, as shown in \fref{fig:method}, we describe these steps in detail via the exemplar predicate group $\psi\in\Psi^{\mathtt{Var}}$, with the typed variables $\mathtt{Var}=\mathtt{(?r,?t)}$.

\subsection{Effect Vectors as Supervision for Neural Learning}\label{sec:neural_learning}
Suppose we had access to the symbolic components of a predicate $\psi$, but did not yet know its classifier $\theta_\psi$.
Suppose further that we had knowledge of all appearances of $\psi$ in the effect sets ($\mathtt{Eff}^+, \mathtt{Eff}^-$) for each operator $\mathtt{Op}^\mathtt{C}$.
We now describe how this knowledge---which we do not actually have, but which will be suggested by symbolic learning---can be used for supervised learning of the classifier $\theta_\psi$.

Recall that we have access to demonstrations $\mathcal{D}$, and for each demonstrated task $T$, we can recover the solution trajectory, $[\mathbf{x}_0,\underline{\mathtt{C}}_0,\mathbf{x}_1,\underline{\mathtt{C}}_1,\cdots,\underline{\mathtt{C}}_{H-1},\mathbf{x}_H]$.
If we knew the initial state ground predicates $\underline{\psi}$ in $\mathbf{x}_0$, then we could immediately recover all ground predicates for all the states in the trajectories by chaining together the operator effects.
Then, a simple binary classification framework could easily address our neural learning problem.
However, we do not have access to the initial state ground predicates---we only have access to operator effects---so we do not have direct knowledge of the abstract states in the demonstration.
Nonetheless, we can still provide supervision for neural learning by leveraging the ground predicates that are added, deleted, or stay unchanged in each \textit{transition pair}.
We provide this supervision by way of \emph{predicate effect vectors}, including the \emph{lifted effect vector} for a domain, and the \emph{ground effect vector} for a transition.

\begin{figure*}[!t]
	\centering
	\includegraphics[width=2\columnwidth]{imgs/fig4.pdf}
 \vspace{-0.1cm}
	\caption{Detailed algorithm for inventing predicate $\texttt{P2(?r,?t)}$.
    At the $t$-th iteration, the symbolic-level proposes $\Delta^\psi_t=[+1,+1,0,0]$ (right bottom), which is used to derive supervisions for the two ground transition pairs. 
    Due to the unreasonable effects, the intermediate state is labeled as both \texttt{True} and \texttt{False}, resulting in high validation loss $\mathbf{L}_t$.
    The loss then informs the symbolic learning in the $t+1$-th iteration.
    }
 \vspace{-0.4cm}
	\label{fig:method}
\end{figure*}

\begin{definition}[Lifted Effect Vector]\label{def:effect_vec}
    The \textit{lifted effect vector} for predicate $\psi$ is $\Delta^{\psi} = [\delta^{\psi}_{1}, \cdots, \delta^{\psi}_{M}] \in \{-1, 0, 1\}^M$ where:
    \[ \delta^{\psi}_{i} = \begin{cases} 
      1 & \psi \in \mathtt{Eff}^+ \text{ for } \mathtt{C}_i \\
      -1 & \psi \in \mathtt{Eff}^- \text{ for } \mathtt{C}_i \\
      0 & \text{otherwise.} 
   \end{cases}
\]
For example, in \fref{fig:method}, the effect vector $[+1,+1,0,0]$ specifies that predicate $\psi=\mathtt{P2\_1(?r,?t)}$ is the add effect for both $\mathtt{Gaze(?r,?t)}$  and $\mathtt{Grasp(?r,?t)}$\footnote{Each predicate here can appear at most once in the effect sets, but this doesn't affect the representation capability of the final predicate set.}.
\end{definition}
Lifted effect vector is a favorable symbolic representation of a lifted predicate, since its shape doesn't depend on task object compositions and can thus be learned more efficiently, as we will see.
However, to train the neural classifier on the transition pairs, we will need to derive some supervisions on \textit{ground predicates}, which is achieved by \emph{ground effect vector}.
\begin{definition}[Ground Effect Vector]\label{def:ground_effect_vec}
    Let $\mathcal{O}$ be the object set in a task $T$, $\underline{\mathtt{C}}_i$ be a ground action with objects $\mathcal{O}_{\underline{\mathtt{C}}_i}\subseteq\mathcal{O}$, then the ground effect vector $\bm{t}^{\psi, \underline{\mathtt{C}}_i} = [t_1, \cdots, t_P] \in \{-1, 0, 1\}^P$ for predicate $\psi$ grounded on $\mathcal{O}$ is defined as:
    \[ t_p
    \;=\;
    \begin{cases}
    \delta^{\psi}_i, 
    & \text{if } \mathcal{O}_{\underline{\psi}_p} \subseteq \mathcal{O}_{\underline{\mathtt{C}}_i}, \\
    0,     
    & \text{otherwise},
    \end{cases}
\]
    where $\underline{\psi}_p$ denotes the $p$-th atom with objects $\mathcal{O}_{\underline{\psi}_p}$, among the total of $P$ ground predicates. For example, in \fref{fig:method}, ground effects for $\mathtt{P2\_1(r_1,t_1)}$ will be $+1$ for both ground actions $\mathtt{Gaze(r_1,t_1)}$ and $\mathtt{Grasp(r_1,t_1)}$, while ground effects for $\mathtt{P2\_1(r_1,t_2)}$ will be $0$, as $\mathtt{(r_1,t_2)} \not\subseteq \mathtt{(r_1,t_1)}$.
\end{definition}
Importantly, this implies that the ``value" of the (potentially) non-zero entry of all ground effects from the action class $\mathtt{C}_i$ equals $\delta^{\psi}_i$, while the ``position" of the non-zero entry is decided by the object set $\mathcal{O}_{\underline{\mathtt{C}}}$ and the predicate grounding (more explanations and discussions see \appref{app:assumption} and \appref{app:same_type}).

Now, we are able to train the neural classifier $\theta_\psi$ on the transition pairs with supervisions from $\Delta^\psi$.
Specifically, consider a transition pair: $(\mathbf{x}, \underline{\mathtt{C}}_i, \mathbf{x}')$, 
we first construct the ground effect vector $\bm{t}^{\psi,\underline{\mathtt{C}}_i}$ using $\delta^\psi_i\in\Delta^\psi$.
Then, the following supervised learning objective can be established for $\theta_\psi$:
\begin{equation}
    \label{eqn:loss}
    \begin{aligned}
    \mathcal{L}(\mathbf{x}, \mathbf{x}', \theta_\psi) 
    = \mathcal{L}_\mathrm{zero} + \mathcal{L}_\mathrm{one}
    \end{aligned}
\end{equation}
\begin{equation}
    \begin{aligned}
        \hat{\mathbf{v}}, \hat{\mathbf{v}}' &= \mathrm{Ground}(\mathbf{x},\theta_\psi), \mathrm{Ground}(\mathbf{x}',\theta_\psi), 
    \end{aligned}
\end{equation}
\begin{equation}
    \label{eqn:zero-loss}
    \begin{aligned}
    \mathcal{L}_\mathrm{zero}
    = & \mathrm{Div}_\mathrm{JS}\!\Big(\hat{\mathbf{v}} \odot \mathbb{I}\big(\bm{t}^{\psi, \underline{\mathtt{C}}_i} = 0\big) \,\Big\|\, \hat{\mathbf{v}}' \odot \mathbb{I}\big(\bm{t}^{\psi, \underline{\mathtt{C}}_i}= 0\big)\Big),
    \end{aligned}
\end{equation}
\begin{equation}
    \label{eqn:non-zero-loss}
    \begin{aligned}
    \mathcal{L}_\mathrm{one}
    =  &
    \Big(\mathrm{BCE}\big([\hat{{v}}_p, \hat{{v}}'_p], [\tfrac{1 - \delta^{\psi}_{i}}{2}, \tfrac{1 + \delta^{\psi}_{i}}{2}]\Big) * \mathrm{abs}(\delta^{\psi}_{i}),
    \end{aligned}
\end{equation}
where $\hat{\mathbf{v}}, \hat{\mathbf{v}}'\in[0,1]^P$ are the predicted ground predicates by applying $\theta_\psi$ on all possible object sets from $\mathcal{O}$.
$\mathrm{Div}_\mathrm{JS}(\cdot | \cdot)$ denotes the Jensen–Shannon divergence~\cite{lin1991divergence} and Eq.\eqref{eqn:zero-loss} tries to minimize the distance between $\hat{\mathbf{v}}$ and $\hat{\mathbf{v}}'$ if the indices with zero values in $\bm{t}^{\psi, \underline{\mathtt{C}}}$.
$\hat{{v}}_p,\hat{{v}}_p'$ denotes $p$-th ground predicate, where $\mathcal{O}_{\underline{\psi}_p} \subseteq \mathcal{O}_{\underline{\mathtt{C}}_i}$. $\mathrm{BCE}(\cdot,\cdot)$ represents the Binary Cross-Entropy, which tries to directly supervise $\hat{\mathbf{v}}$ and $\hat{\mathbf{v}}'$ if $\delta^{\psi}_{i}\neq 0$.
Since we have $\Delta^\psi$ for all lifted actions, the pipeline can be applied to all ground transition pairs in $\mathcal{D}$, resulting in the global loss,
\begin{equation}
\label{eqn:global_loss}
    \mathcal{L}^{\mathcal{D}}(\theta_\psi)
    = \sum_{\mathtt{C}\in\mathcal{C}} \mathbb{E}_{(\mathbf{x},\underline{\mathtt{C}},\mathbf{x'})\sim\mathcal{D}_\mathtt{C}}\mathcal{L}(\mathbf{x}, \mathbf{x}', \theta_\psi),
\end{equation}
where $\mathcal{D}_\mathtt{C}$ denotes the distribution of the grounded transition for action $\mathtt{C}$ in the dataset $\mathcal{D}$ (See \fref{fig:method} for the examples of two transitions from different actions).
Since $\mathcal{L}$ is fully differentiable with respect to $\theta$, given a effector $\Delta^{\psi}$ and the demonstration dataset, we could leverage the general and standard deep learning framework~\cite{kingma2014adam,rumelhart1986sgd} to train a neural classifier $\theta$ that minimizes the loss $\mathcal{L}$ for any state representation.

\subsection{Neural Loss as Guidance for Symbolic Learning}

To obtain all the classifiers $\theta_{\Psi^{\mathtt{Var}}}$ for all the typed predicates $\psi\in\Psi^{\mathtt{Var}}$, the problem now becomes finding all the lifted effect vectors $\Delta^\psi\in\Delta^{\Psi^{\mathtt{Var}}}$.
As defined in \defref{def:effect_vec}, the lifted effect vectors live in the discrete world with a finite shape, which motivates us to establish some discrete optimization strategies for it.
One insight from \model{} is that, despite the large space, only a few effect vectors provide reasonable supervision, with unreasonable ones resulting in high classification error on the validation set after the training converges.

A motivating example is depicted in \fref{fig:method}, where the proposed effect vector assumes the predicate to be the add effects for both $\mathtt{Gaze}$ and $\mathtt{Grasp}$.
In the demonstration, $\mathtt{Grasp}$ closely follows $\mathtt{Gaze}$, making the intermediate state shared in the two transition pairs but with one being the next state and the other being the current state.
Then, the intermediate state will be labeled as both $\mathtt{True}$ and $\mathtt{False}$, which is unreasonable and results in high classification error.
Thus, our symbolic learning aims to efficiently find all ``reasonable" effect vectors,
\begin{definition}[Symbolic Learning Objective]
    Let the demonstrations be split into non-overlapping training and validation sets $\mathcal{D} = \mathcal{D}^\text{train} \cup \mathcal{D}^\text{val}$, the objective of our symbolic learning is to find a subset of effect vectors $\Delta^{*,\Psi^{\mathtt{Var}}} \subset \Delta^{\Psi^{\mathtt{Var}}}$, 
    \begin{equation}\label{eqn:search_obj}
        \Delta^{*,\Psi^{\mathtt{Var}}} = \left\{ \Delta^\psi \in \Delta^{\Psi^{\mathtt{Var}}} \mid \mathcal{L}^{\mathcal{D}^\text{val}}(\theta_\psi) \leq \tau \right\},
    \end{equation}
    where $\theta^\psi$ is learned from $\mathcal{D}^\text{train}$ with supervision derived from $\Delta^\psi$ using Eq.~\eqref{eqn:global_loss}, and $\mathcal{L}^{\mathcal{D}^\text{val}}$ denotes the validation loss of classifier $\theta^\psi$ calculated from Eq.~\eqref{eqn:global_loss}, and $\tau$ is a given threshold.
\end{definition}

Inspired by the fact that a predicate's effects are usually sparse among actions, we propose a tree expansion algorithm for efficiently learning $\Delta^{*,\Psi^{\mathtt{Var}}}$.
Specifically, as shown in \fref{fig:method} (with $M=4$ actions), the complete effect vector set, $\Delta^{\Psi^{\mathtt{Var}}}$, is formulated into a tree-like structure, with each node $\Delta^{\psi_n}\in\Delta^{\Psi^{\mathtt{Var}}}, n>0$ representing an effect vector.
The root $\Delta^{0}$ of the tree is an ``all-zero" effect vector, which is not associated with any potential \textit{dynamic} predicate.
The nodes in the $l-$th level represent a vector with $l$ non-zero entries.
For each non-leaf node in the $l-$th level, its ``children" in the $l+1$-th level have the same non-zero entries with one more non-zero entry.
A naive exploration in this tree is to enumerate its nodes and train neural classifiers with supervisions from each of them, which is extremely time-consuming due to the large space.
To explore the effect tree more efficiently, \model{} tries to balance the trade-off between \textit{exploration} and \textit{exploitation}~\cite{coulom2006mcts,silver2017alphago}.

Specifically, each node $\Delta^{\psi_n}$ in the tree is additionally associated with a scalar $r^n_t$, indicating its value for finding $\Delta^\psi \in\Delta^{*,\Psi^{\mathtt{Var}}}$ if we expand it at the current $t$-th iteration.
In the $t$-th iteration, we start by selecting a parent node $\mathrm{Par}(\Delta^{\psi}_t)$ with the highest current Upper Confidence bounds applied to Trees (UCT) score~\cite{silver2017alphago}.
Its child, $\Delta^{\psi}_t$ that has the current highest value $r_t$ among all the children is proposed for evaluation (index is neglected for simplicity).
The evaluation process is defined as the supervised neural learning process in \sref{sec:neural_learning}.
After the classifier $\theta^{\psi}$ converges, we collect its \textit{action-wise} validation loss $\mathbf{L}_t\in\mathbb{R}^M$ by decomposing Eq.~\eqref{eqn:global_loss} for each action $\mathtt{C}\in\mathcal{C}, |\mathcal{C}|= M$.
The loss information is then used to update the values of all the nodes in the tree, which helps us select and expand the parent node in the $t+1$-th iteration.
The tree expansion terminates if all of the existing nodes are fully expanded, or, if the max iteration has been reached.

\begin{figure*}[!t]
	\centering
	\includegraphics[width=1.8\columnwidth]{imgs/fig5.pdf}
 \vspace{-0.1cm}
	\caption{Visualization of the five domains (excluding Climb-Transport) we have studied in this work. These domains feature various state representations (including the high-dimensional point clouds in the Engrave domain) where our \model{} can be generally applied.}
 \vspace{-0.1cm}
	\label{fig:domains}
\end{figure*}

Clearly, the key to more efficient learning lies in the definition and updating strategy of the node values $r^n_t$.
As the sparsity of predicate effects among actions is indicated by the sparsity of non-zero entries in the effect vector, we try to efficiently explore the entries in the effect vectors that \textbf{should not} be zero to optimize Eq.~\eqref{eqn:search_obj}.
To achieve this, we try to leverage the guidance from the ``zero-parts" (Eq.~\eqref{eqn:zero-loss}) of $\mathbf{L}_t$.
Specifically, after the evaluation of the node $\Delta^{\psi}_t$ in the $t-$th iteration with $\mathbf{L}_t$, we use the following equations to update and compute $r^n_{t+1}$ for all the nodes in the tree,
\begin{equation}
\label{eqn:update}
    \begin{aligned}
    \mathbf{R}_{t+1} &= 
    \mathbf{R}_{t} \odot \mathbb{I}(\Delta^{\psi}_t \neq 0) 
    + \frac{(\mathbf{R}_{t} + \mathbf{L}_t)}{2} \odot \mathbb{I}(\Delta^{\psi}_t = 0). \\
        r^n_{t+1} &= \mathrm{Mean}\big(\mathbf{R}_{t+1} \odot \mathbb{I}(\Delta^{\psi_n} = 0)\big),
    \end{aligned}
\end{equation}
where $\mathbf{R}_{t}\in\mathbb{R}^M, \mathbf{R}_0=\mathbf{0}^M$ is a global value vector that stores the information from historical evaluations.
$\mathbb{I}(\Delta^{\psi}_t = 0)$ is a binary vector indicating if an entry in the evaluated node $\Delta^{\psi}_t$ equals to zero.
Here, we only update $\mathbf{R}_{t}$ with the ``zero-parts" of the loss information $\mathbf{L}_t$, which then helps update the node values.
The node values intuitively indicate how likely the loss will decrease in its children where
there are fewer zeros in the effect vectors.
From Eq.~\eqref{eqn:update}, we see that the higher loss in the zero entry indexes of $\Delta^{\psi_n}$ contributes to its higher value, encouraging the evaluation to prioritize its children, which are likely to decrease the loss.
In \appref{app:pruning}, we additionally introduce some pruning strategy for more efficient expansion.

Finally, we collect all the effect vectors $\Delta^{\psi_n}\in\Delta^{*,\Psi^{\mathtt{Var}}}$ with the associated neural classifier $\theta_{\psi_n}$ as the outcomes from the bilevel learning of typed predicates $\Psi^{\mathtt{Var}}$.
As shown in \fref{fig:overview} (a), there could exist multiple different vector-classifier pairs for the same typed predicate $\Psi^{\mathtt{Var}}$, which are treated as different predicates in the following predicate selection stage.\footnote{Following previous work~\cite{silver2023predicateinvent}, we can also add quantifiers and negations as prefix for each of the invented neural predicates.}

\subsection{Predicate Selection}
With all possible typed predicates, \model{} is able to construct the predicate pool $\hat{\Psi}_{\mathrm{dyn}}$ without any function pre-definition.
This strength has made bilevel planning applicable to complicated and high-dimensional state spaces.
Meanwhile, as the predicate functions are \textit{relational} and can be seamlessly integrated into an AI planner~\cite{helmert2006fast}, our approach can naturally achieve compositional generalization.

Yet, not all of the predicates in $\hat{\Psi}_{\mathrm{dyn}}$ are favorable for planning.
Therefore, we next try to select a subset $\tilde{\Psi}_{\mathrm{dyn}}\subset\hat{\Psi}_{\mathrm{dyn}}$ that minimizes the score function $J(\tilde{\Psi}_{\mathrm{dyn}},
\mathcal{D})$~\cite{silver2023predicateinvent}.
Specifically, with a set of candidate predicates $\tilde{\Psi} = \{\Psi_\mathrm{G},\Psi_\mathrm{sta}, \tilde{\Psi}_{\mathrm{dyn}}\}$, we start by grounding all the states $\mathbf{x}$ in the demonstration $\mathcal{D}$, which forms a ground atom dataset.
Since we already have the effect vector for each of the predicates in $\tilde{\Psi}_{\mathrm{dyn}}$, the effect sets ($\tilde{\mathtt{Eff}}^+, \tilde{\mathtt{Eff}}^-\subset \tilde{\Psi}_{\mathrm{dyn}}$) for each operator $\tilde{\mathtt{Op}}^\mathtt{C}$ can be easily obtained.
Then, the precondition set for each operator can be calculated using an intersection strategy~\cite{chitnis2021nsrt}.
The learned operators set $\tilde{\mathtt{Op}}$ is then applied to the ground atom dataset to generate plan \textit{skeletons} $\tilde{\pi}$ for each task, which are compared with the demonstration plan \textit{skeletons} $\bar{\pi}$ for objective calculation~\cite{silver2023predicateinvent}.
The objective is finally minimized by running a hill-climbing search over $\hat{\Psi}_{\mathrm{dyn}}$, resulting in the desired predicate set ${\Psi}_{\mathrm{dyn}}$.
With the complete set, we are now able to learn the planning \textit{abstractions} (operators and samplers) using standard pipelines~\cite{chitnis2021nsrt,kumar2023predict,liang2024visualpredicator} as shown in \fref{fig:overview} (a).

Note that before the predicate selection stage, all of the predicates' neural functions have been pre-trained using our \model{} framework, which has avoided the challenge of using the planning objective for learning but still achieved a powerful and adaptive model optimized for planning.

% simulated empirical
\begin{table*}[!t]
    \centering
    \setlength{\tabcolsep}{0.9mm}
    \fontsize{8}{10}\selectfont
    \begin{tabular}{cccc|ccc|ccc|ccc|ccc|ccc}
    \toprule[1.5pt]
    Domain & \multicolumn{3}{c}{Satellites~\cite{kumar2023predict}} & \multicolumn{3}{c}{Blocks~\cite{chitnis2021nsrt}} & \multicolumn{3}{c}{Measure-Mul} & \multicolumn{3}{c}{Climb-Measure} & \multicolumn{3}{c}{Climb-Transport} & \multicolumn{3}{c}{Engrave} \\
    State Space & \multicolumn{3}{c}{SE2 ($\mathbb{R}^{3\times5}$)} & \multicolumn{3}{c}{Vec3 ($\mathbb{R}^{3\times8}$)} & \multicolumn{3}{c}{SE3 ($\mathbb{R}^{7\times5}$)} & \multicolumn{3}{c}{SE3 ($\mathbb{R}^{7\times5}$)} & \multicolumn{3}{c}{SE3 ($\mathbb{R}^{7\times5}$)} & \multicolumn{3}{c}{PCD ($\mathbb{R}^{1024\times3\times6}$)} \\
    \midrule
    Test Dist & $\mathcal{T}^\mathrm{train}$    & $\mathcal{T}^\mathrm{test}$   &  $\downarrow$ & $\mathcal{T}^\mathrm{train}$    & $\mathcal{T}^\mathrm{test}$   &  $\downarrow$ & $\mathcal{T}^\mathrm{train}$    & $\mathcal{T}^\mathrm{test}$   &  $\downarrow$ & $\mathcal{T}^\mathrm{train}$    & $\mathcal{T}^\mathrm{test}$   &  $\downarrow$ & $\mathcal{T}^\mathrm{train}$    & $\mathcal{T}^\mathrm{test}$   &  $\downarrow$ & $\mathcal{T}^\mathrm{train}$    & $\mathcal{T}^\mathrm{test}$   &  $\downarrow$ \\
    Oracle & 100.0 & 100.0 & 0.00  & 100.0 & 100.0 & 0.00  & 100.0 & 100.0 & 0.00  & 90.0  & 81.6  & 0.09  & 91.2  & 82.0  & 0.10  & 100.0 & 100.0 & 0.00 \\
    \textbf{IVNTR (Ours)} & \underline{\textbf{99.2}}  & \underline{\textbf{93.2}}  & \underline{\textbf{0.06}}  & \underline{\textbf{100.0}} & \underline{\textbf{82.0}}  & \underline{\textbf{0.18}}  & \underline{\textbf{90.0}}  & \underline{\textbf{88.4}}  & \underline{\textbf{0.02}}  & \underline{\textbf{91.6}}  & \underline{\textbf{65.6}}  & \underline{\textbf{0.28}}  & \underline{\textbf{79.2}}  & \underline{\textbf{53.2}}  & \underline{\textbf{0.33}}  & \underline{\textbf{98.4}}  & \underline{\textbf{79.2}}  & \underline{\textbf{0.20}} \\
    GNN~\cite{battaglia2018gnn}   & 74.0  & 6.0   & 0.92  & 82.4  & 24.0  & 0.71  & 2.4   & 1.2   & 0.50  & 20.8  & 0.7   & 0.97  & 48.0  & 2.0   & 0.96  & 0.0   & 0.0   & 1.00 \\
    Transformer~\cite{vaswani2017tf} & 46.8  & 1.2   & 0.97  & 24.4  & 7.6   & 0.69  & 10.0  & 2.4   & 0.76  & 29.2  & 0.0   & 1.00  & 10.4  & 0.8   & 0.92  & 0.0   & 0.0   & 1.00 \\
    FOSAE~\cite{asai2019latplan_fol} & 100.0 & 34.8  & 0.65  & 2.4   & 0.4   & 0.83  & 3.6   & 1.2   & 0.67  & 21.2  & 0.0   & 1.00  & 45.6  & 0.8   & 0.98  & 0.0   & 0.0   & 1.00 \\
    Grammar~\cite{silver2023predicateinvent} & 0.0   & 0.0   & 1.00  & 0.0   & 0.0   & 1.00  & 0.0   & 0.0   & 1.00  & 0.0   & 0.0   & 1.00  & 0.0   & 0.0   & 1.00  & N/A   & N/A   & N/A \\
    Random & 0.0   & 0.0   & 1.00  & 10.6  & 1.2   & 0.89  & 0.0   & 0.0   & 1.00  & 0.0   & 0.0   & 1.00  & 0.0   & 0.0   & 1.00  & 0.0   & 0.0   & 1.00 \\
    \bottomrule[1.5pt]
    \end{tabular}%
  \caption{Empirical success rate comparison on six simulated robot planning domains.
  Among all the methods, \model{} suffers the least from the generalization challenge due to its relational structure.
  The scores are obtained from the averaged results over five random seeds.
  }
  \vspace{-0.3cm}
  \label{tab:sim_emp}%
\end{table*}%

% \subsection{Predicate Selection and Operator Learning}
% \td{Briefly introduce the objective based search.}

\section{Experiments}

\subsection{Implementation Details}
\myparagraph{System and Hardware:} All methods are evaluated on a single NVIDIA A100 GPU and an AMD EPYC 7543 32-Core CPU to ensure fairness. 
Training is conducted on the same hardware as evaluation, with domain-specific details provided in \appref{app:domain_details}. 
Real-robot experiments are performed using the Boston Dynamics Spot robot equipped with an arm.

\myparagraph{Baselines:} Since we do not assume access to the complete predicate set, most existing bilevel planning approaches~\cite{chitnis2021nsrt,kumar2023predict,kumar2024practice} are inapplicable. 
We attempted the grammar-based approach~\cite{silver2023predicateinvent}, but it failed to optimize the planning objective in most domains (see \appref{app:objective}). 
Thus, \model{} is primarily compared to relational neural policy learning methods~\cite{battaglia2018gnn,vaswani2017tf,asai2019latplan_fol}. 
Following prior works~\cite{kumar2023predict,chitnis2021nsrt,silver2023predicateinvent,silver2022skills}, baselines are trained using standard behavior cloning pipelines and evaluated with a shooting strategy; see \appref{app:baseline_details} for details.

\myparagraph{Domains:}
We evaluate the methods across six diverse robot planning domains with varying state representations, as visualized in \fref{fig:domains} and summarized in \tref{tab:sim_emp}. 
Below, we provide a high-level overview of these domains.
For more implementation details, please refer to \appref{app:domain_details}:
\begin{itemize}
    \item \textit{Satellites} comes from prior work~\cite{kumar2023predict}, which involves a group of satellites collaborating to capture sensor readings from targets. 
    States comprise SE2 poses and object attributes. 
    Training scenarios ($\mathcal{T}^\mathrm{train}$) include 2 satellites and 2 targets, while test scenarios ($\mathcal{T}^\mathrm{test}$) have 3 of each.
    \item \textit{Blocks:} Inspired by~\cite{chitnis2021nsrt}, this domain tasks a robot with manipulating 3D blocks to form goal towers. 
    Unlike vanilla Blocks World, the goals here involve ``packing" pairs of blocks into two-level towers.
    $\mathcal{T}^\mathrm{train}$ includes 4–5 blocks, while $\mathcal{T}^\mathrm{test}$ features 6–7 blocks.
    \item \textit{Measure-Mul:} In this new domain inspired by Satellites, a Spot robot calibrates a thermal camera by aligning it with a calibrator before measuring body temperatures of multiple human targets. 
    States include 6D poses of the robot and the targets. 
    Training distributions ($\mathcal{T}^\mathrm{train}$) have 2–3 humans, while test distributions ($\mathcal{T}^\mathrm{test}$) include 4.
    \item \textit{Climb-Measure} is similar to Measure-Mul but with added complexity: calibrators and human targets may be placed at high, initially unreachable locations. 
    The Spot robot must arrange platforms and climb onto them to reach targets. 
    Training ($\mathcal{T}^\mathrm{train}$) includes 0–1 platforms, while testing ($\mathcal{T}^\mathrm{test}$) requires planning with 2 platforms.
    \item \textit{Climb-Transport:} Introduced in \fref{fig:running_example}, this domain requires the Spot robot to arrange platforms to grasp a high-placed target, then transport it into a container. 
    Training setups ($\mathcal{T}^\mathrm{train}$) feature 0–1 platforms, while testing ($\mathcal{T}^\mathrm{test}$) involves 2 platforms.
    \item \textit{Engrave} features high-dimensional state spaces represented as object-centric point clouds. 
    Similar to Blocks, the goal is to ``pack" blocks. 
    However, blocks start with one irregular Gaussian surface that must be ``engraved" and ``rotated" to create a matching fit. 
    Training distributions ($\mathcal{T}^\mathrm{train}$) include 3–4 blocks, while testing ($\mathcal{T}^\mathrm{test}$) has 5–6.
\end{itemize}

% real robot empirical
\begin{table*}[!t]
    \centering
    \setlength{\tabcolsep}{1.5mm}
    \fontsize{8}{10}\selectfont
    \begin{tabular}{cccccccccc|cccccccc}
    \toprule[1.5pt]
          &       & \multicolumn{8}{c}{Climb-Measure}                             & \multicolumn{7}{c}{Climb-Transport}                       &  \\
    Planner & Seed/Task & T0    & T1    & T2    & T3    & T4    & T5    & Mean  & Avg.  & T0    & T1    & T2    & T3    & T4    & T5    & Mean  & Avg. \\
    \midrule
    \midrule
    \multirow{3}[2]{*}{Oracle (Human)} & S0    & 0.0     & 1.0     & 1.0     & 1.0     & 0.5   & 1.0     & 0.750 & \multirow{3}[2]{*}{0.833} & 0.5   & 0.0     & 1.0     & 1.0     & 1.0     & 1.0     & 0.750 & \multirow{3}[2]{*}{0.592} \\
          & S1    & 1.0     & 1.0     & 1.0     & 0.5   & 1.0     & 0.5   & 0.833 &       & 1.0    & 1.0    & 0.33  & 0.5   & 0.5   & 0.5   & 0.638 &  \\
          & S2    & 1.0    & 1.0    & 1.0    & 1.0    & 1.0    & 0.5   & 0.917 &       & 0.5   & 0.33  & 0.5   & 0     & 0.5   & 0.5   & 0.388 &  \\
    \midrule
    \multirow{3}[2]{*}{\textbf{IVNTR (Learned)}} & S0    & 1.0   & 1.0   & 0.5   & 0.0   & 1.0   & 1.0   & 0.750 & \multirow{3}[2]{*}{0.778} & 0.5   & 1.0    & 0     & 0.5   & 1.0    & 0.33  & 0.555 & \multirow{3}[2]{*}{0.546} \\
          & S1    & 1.0   & 1.0   & 1.0   & 1.0   & 1.0   & 0.0   & 0.833 &       & 0.33  & 0.5   & 1.0    & 0.33  & 0     & 0.5   & 0.443 &  \\
          & S2    & 0.5   & 1.0   & 1.0   & 1.0   & 0.0   & 1.0   & 0.750 &       & 0.5   & 0.5   & 1.0    & 1.0    & 0.5   & 0.33  & 0.638 &  \\
    \bottomrule[1.5pt]
    \end{tabular}%
  \caption{Success rate comparison of the two planners on the real robot planning tasks sampled from $\mathcal{T}^\mathrm{test}$. 
  ``Oracle" denotes bilevel planners exhaustively engineered by a human expert. 
  \model{} is learned from the demonstrations collected in the simulated environment.
  For each domain, we have tested six tasks (T0$\sim$T5) sampled from three random seeds (S0$\sim$S2).
  For each planner in each task, we run it at most three times and record the averaged task success rate.
  Our framework has achieved comparable real robot performance with the Oracle. 
  }
  \vspace{-0.3cm}
  \label{tab:real_emp}%
\end{table*}%

\begin{table}[!t]
\centering
\setlength{\tabcolsep}{1.0mm}
    \fontsize{8}{10}\selectfont
    \begin{tabular}{ccccccc}
        \toprule[1.5pt]
        Domains & \multicolumn{3}{c}{Blocks} & \multicolumn{3}{c}{Climb-Measure} \\
        Metric & $\mathcal{T}^\mathrm{train}$    & $\mathcal{T}^\mathrm{test}$   &  $J(\cdot)$ ($\times10^5$) & $\mathcal{T}^\mathrm{train}$    & $\mathcal{T}^\mathrm{test}$   & $J(\cdot)$ ($\times10^5$) \\
        \midrule
        GT-Vectors & 80.0  & 62.8  & 121.59 & 90.4  & 61.2  & 2.718 \\
        \textbf{IVNTR} & \textbf{100.0} & \textbf{82.0}  & \textbf{56.77} & \textbf{91.6}  & \textbf{65.6}  & \textbf{2.481} \\
        \bottomrule[1.5pt]
    \end{tabular}%
    \vspace{-0.1cm}
    \caption{Comparison between the predicates learning using ground-truth (GT)-vectors and using our \model{} framework. We report the task success rate and the final planning objective.}
    \vspace{-0.3cm}
  \label{tab:abla_gt_vec}%
\end{table}%

\myparagraph{Experiment Setup:}
For each domain, we manually designed an oracle bilevel planner (Oracle) to collect training demonstrations. 
We report averaged results over five random seeds for all six domains.
For each seed in Satellites, Blocks, Measure-Mul, and Engrave, we have collected $500$ demonstrations.
For Climb-Measure and Climb-Transport, $2000$ demonstrations were collected per seed.
During test, each seed in each domain includes $50$ in-domain tasks ($T\sim\mathcal{T}^\mathrm{train}$) and $50$ generalization tasks ($T\sim\mathcal{T}^\mathrm{test}$). 
We report the success rate within the same maximum planning time for all the methods.
For real-world Climb-Measure and Climb-Transport, a shared map was recorded using Spot’s default graph\_nav service for simulation and localization. 
Each domain was tested on $3$ random seeds, each with $6$ generalized tasks. 
For manipulation-based actions, we have utilized an off-the-shelf segment anything model (SAM)~\cite{lang_sam,ravi2024sam} for computing the grasping pixel.


\subsection{Empirical Results}
\myparagraph{Simulated Planning Domains:}
The empirical comparison across the six simulated domains is presented in \tref{tab:sim_emp}. 
Alongside the averaged success rates, we report the performance drop percentage during generalization.
\model{} consistently outperforms all baselines in both $\mathcal{T}^\mathrm{train}$ and $\mathcal{T}^\mathrm{test}$ across all domains. 
For complex state representations such as SE3 and PointClouds (PCD), none of the baselines achieve a success rate above $5\%$ on generalized tasks, while \model{} stably solves over $50\%$ by virtue of its relational structure.

\myparagraph{Real Robot Planning Tasks:}
All real robot tasks are sampled from $\mathcal{T}^\mathrm{test}$, making \model{} the only applicable approach. 
To benchmark performance, a human expert has exhaustively engineered oracle planners for the real robot in the two domains. 
Each approach attempts each task up to three times, with the average success rate reported in \tref{tab:real_emp}.
Despite the simulation-to-reality gap, \model{} successfully generalizes to held-out tasks, achieving results comparable to the oracle planner. 
Most real-world deployment failures stem from perception and localization errors, with examples shown in \appref{app:failures}.

\subsection{Ablation Studies}
\myparagraph{Comparison to Ground-Truth Effect Vectors:}
A notable strength of \model{} is its ability to discover non-ground-truth (GT) predicates. 
In \tref{tab:abla_gt_vec}, we replaced our tree expansion with oracle-derived GT effect vectors, where the performance on the Blocks and Climb-Measure domains are reported. 
Interestingly, \model{} minimizes the planning objective more effectively than GT vectors, resulting in its higher accuracy. 
This outcome highlights the advantage of exploring better high-level abstractions beyond human-engineered ones.

\myparagraph{Comparison to Other Search Algorithms:}
To evaluate the efficiency of our neural-informed tree expansion algorithm, we compared it with alternative search strategies: a greedy approach (Greedy) that flips the zero entry with the highest current loss, breadth-first (BFS) and depth-first (DFS) searches.
For the Engrave domain, \fref{fig:abla_search} shows the number of iterations required to find each reasonable effect vector using these methods. 
Compared to uninformed methods, our \model{} framework is at least $2\times$ more efficient.
The greedy approach exhibits high variance and is generally less reliable. 

\myparagraph{Comparison to Pure High-level Planning:}\label{sec:abla_pure_high_level}
As discussed in \sref{sec:bilevel_planning}, predicates not only enable compositional generalization through planning operators but also serve as indicators of sampler failures in low-level states. 
To assess the importance of our invented predicates for reliable low-level sampling, we disabled bilevel planning and followed the high-level plan greedily, ignoring predicate-based checks for sampler success.
As shown in \tref{tab:abla_highlevel}, this approach results in performance drops of up to $98.4\%$, underscoring the critical role of predicates as indicators of low-level state validity.

\subsection{Interpreting Invented Predicates}

\begin{table}[!t]
\centering
\setlength{\tabcolsep}{2mm}
    \fontsize{8}{10}\selectfont
    \begin{tabular}{ccccccc}
        \toprule[1.5pt]
        Domains & \multicolumn{3}{c}{Satellites} & \multicolumn{3}{c}{Measure-Mul} \\
        Sampling with $\theta^\psi$ & \checkmark    & \xmark   &  $\downarrow$ & \checkmark    & \xmark   & $\downarrow$ \\
        \midrule
        $\mathcal{T}^\mathrm{train}$ & \textbf{99.2}  & 74.0  & 0.254 & \textbf{90.0}  & 2.8   & 0.969 \\
        $\mathcal{T}^\mathrm{test}$ & \textbf{93.2}  & 16.8  & 0.820 & \textbf{88.4}  & 1.4   & 0.984 \\
        \bottomrule[1.5pt]
    \end{tabular}%
    \vspace{-0.1cm}
    \caption{Comparison between sampling with and without the invented predicate functions. Without using the functions as step-wise success indicator, the performance drops significantly.}
    \vspace{-0.3cm}
  \label{tab:abla_highlevel}%
\end{table}%

Predicates play a key role in defining the preconditions and effects of operators, specifying the order of ground actions to complete a task. 
In \tref{tab:interprete}, we display part of the precondition and effect sets for high-level actions in the Climb-Transport domain, where invented predicates capture logical relationships among actions.
For example, the $\mathtt{Drop}$ action requires $\mathtt{P2\_1}$ and $\mathtt{P2\_2}$, which are the add effects of $\mathtt{Gaze}$, $\mathtt{MTGaze}$, and $\mathtt{WalkOn}$. 
Similarly, the $\mathtt{Pick}$ action depends on $\mathtt{P1\_1}$, the delete effect of $\mathtt{Grasp}$, enforcing all $\mathtt{Pick}$ actions to precede $\mathtt{Grasp}$. 
These relational constraints over objects enable the generation of long-horizon plans that generalize to unseen object compositions. 
The complete operators are detailed in \appref{app:complete_op}.
We also provide visualizations of how predicates act as success indicators to filter out sampler failures in \appref{app:sampler_vis}.


\section{Related Works}

\begin{figure}[!t]
	\centering
	\includegraphics[width=1.01\columnwidth]{imgs/fig6.pdf}
	\caption{Comparison between \model{} with other search strategies in the Engrave domain. We report the number of iterations for each of the algorithm to find the reasonable effect vectors (different predicate could have different maximum search space $M_{\mathrm{max}}$). \model{} has demonstrated the highest efficiency in finding the desired vectors.}
    \vspace{-0.3cm}
	\label{fig:abla_search}
\end{figure}

\subsection{Learning Abstractions for Planning}
Learning abstractions is essential for reducing the complexity of long-horizon planning in high-dimensional domains.
Traditional approaches, such as hierarchical task networks (HTN)~\cite{kaelbling2011htn}, heavily rely on hand-designed abstractions.
Recent advances have shifted towards data-driven approaches that automatically discover useful abstractions from interactions~\cite{gupta2020relay,Soroush2022maple,hansen2022bisimulation,dong2019nlm,chitnis2021glib} or demonstrations~\cite{sharmadirected,kipf2019compile,chitnis2021nsrt,mao2022pdsketch}.
However, these methods struggle to generalize beyond the training environments~\cite{liang2024visualpredicator}.
Foundation models, such as large language models (LLMs) and vision-language models (VLMs), have been explored for high-level planning with minimal or no demonstrations~\cite{liu2024BLADE,fang2024keypoint,liang2024visualpredicator,silver2024generalized,wei2022cot,han2024interpret,huang2023voxposer,hu2023look,kumar2024openworld}. 
While these models leverage commonsense knowledge for efficient plan generation, two challenges remain:
(1) High-level plans from LLMs~\cite{silver2024generalized,han2024interpret,wei2022cot,huang2023voxposer} are difficult to reliably refine in the low-level space~\cite{liang2024visualpredicator}.
(2) VLM-based methods~\cite{liang2024visualpredicator,fang2024keypoint,kumar2024openworld,yang2024guidinglonghorizontaskmotion} struggle in domains where images cannot fully capture the state space.

\subsection{Task and Motion Planning}
To address these challenges, task and motion planning (TAMP) integrates high-level symbolic planning with low-level motion generation. 
Traditional TAMP methods~\cite{garrett2021integrated,garrett2020pddlstream} rely on manually designed planners~\cite{McDermott1998PDDL,karaman2011anytime}. 
These frameworks inherently support compositional generalization due to their relational structure. 
In addition, the coupling between high-level and low-level planning can handle failures at either level effectively.
However, traditional TAMP requires substantial human effort. 
Recent advances integrate learning into TAMP~\cite{chitnis2021nsrt,bougie2020skill,kumar2023predict,silver2023predicateinvent,liang2024visualpredicator,kumar2024openworld,yang2024guidinglonghorizontaskmotion}, forming bilevel planning frameworks. 
These approaches combine the strengths of TAMP with the scalability of machine learning models. 
However, most bilevel planners rely on manually engineered state abstractions (predicates), limiting their scalability and flexibility.

\begin{table}[!t]
    \centering
    \setlength{\tabcolsep}{1.2mm}
    \fontsize{8}{10}\selectfont
    \begin{tabular}{cccccc}
    \toprule[1.5pt]
     Predicates & $\mathtt{P1\_1(?r)}$ & $\mathtt{P2\_1(?r,?t)}$ & $\mathtt{P2\_2(?r,?t)}$ & $\mathtt{In(?t,?t)}$ \\
    \midrule
    \texttt{Grasp}      & $\mathtt{Pre} \mid \mathtt{Eff}^-$ & $\mathtt{Pre}$   & $\mathtt{Pre}$   &       \\
    \texttt{Gaze}       &       & $\mathtt{Pre}$   & $\mathtt{Eff}^+$   &       \\
    \texttt{MAOff}      & $\mathtt{Pre}$   &       &       &       \\
    \texttt{MAOn}       &       & $\mathtt{Pre}$   & $\mathtt{Pre}$   &       \\
    \texttt{MTGaze}     &       & $\mathtt{Eff}^+$   &       &       \\
    \texttt{WalkOn}     &       & $\mathtt{Eff}^+$   &       &       \\
    \texttt{MTPlace}    & $\mathtt{Pre}$   &       &       &       \\
    \texttt{MTReach}    &       &       &       &       \\
    \texttt{Pick}       & $\mathtt{Pre}$   &       &       &       \\
    \texttt{Drop}       &       & $\mathtt{Pre}$ & $\mathtt{Pre}$ & $\mathtt{Eff}^+$ \\
    \bottomrule[1.5pt]
    \end{tabular}
  \caption{The preconditions, add effects, and delete effects for each of the actions (the variables are neglected for simplicity) with (part of) the invented predicates in the Climb-Transport Domain. 
  $\mathtt{MA}$ is for $\mathtt{MoveAway}$ and $\mathtt{MT}$ for $\mathtt{MoveTo}$.
  The invented predicates have specified some logical constrains over the order of actions.
  }
  \vspace{-0.3cm}
  \label{tab:interprete}%
\end{table}%

\subsection{Predicate Invention for Planning}
To automate predicate generation for planning, various approaches have been proposed~\cite{liang2024visualpredicator,li2023embodied,han2024interpret,silver2023predicateinvent,asai2019latplan_fol,asai2021latplanpddl,asai2018latplan_prop,hansen2022bisimulation}. 
The most direct approaches~\cite{li2023embodied,han2024interpret} rely on domain knowledge, such as human-provided labels~\cite{li2023embodied} or LLM-based oracles~\cite{han2024interpret}. 
To \textit{invent} predicates, earlier approaches derive "easy" step-wise objectives, such as reconstruction~\cite{asai2018latplan_prop,asai2019latplan_fol,asai2021latplanpddl} or bisimulation~\cite{hansen2022bisimulation}.
Among these, LatPlan (FOSAE)~\cite{asai2019latplan_fol} learns relational neural abstractions for images by reconstructing states and identifying action spaces for planning, which is the closest work to \model{}. 
However, its implicit predicates are not optimized for efficient planning, limiting its applicability to domains with only nullary actions. 
Recent approaches\cite{silver2023predicateinvent,liang2024visualpredicator} address this by learning abstractions tailored to fast planners~\cite{helmert2006fast}. 
However, these methods struggle to \textit{learn} predicate functions due to non-differentiable objectives. 
Consequently, they often rely on pre-defined predicate candidates from program synthesis~\cite{silver2023predicateinvent} or foundation models~\cite{liang2024visualpredicator}, which constrains their applicability in more sophisticated and high-dimensional state spaces.
Our approach is motivated by the bilevel planning framework~\cite{silver2023predicateinvent} but eliminates the need for pre-defining the candidates. 
Instead, we can learn adaptive neural functions for different domains, enabling more flexible and scalable learning based planning.

\section{Limitations and Future Works}
In this work, we introduced \model{}, a bilevel learning framework that invents neural classifiers as relational planning predicates. 
These predicates enable the learning of relational bilevel planners capable of generating long-horizon plans for unseen object compositions. 
At the neural level, \model{} leverages the high-level effects of predicates across actions to provide step-wise supervisions on transition pairs. 
At the symbolic level, \model{} captures the sparsity of effects through an informed tree expansion algorithm. 
By adopting neural functions, \model{} adapts to diverse robot planning domains with continuous and high-dimensional state representations. 
Additionally, we deployed \model{} on a mobile manipulator, demonstrating its ability to achieve compositional generalization over objects and actions in long-horizon mobile manipulation tasks.

\model{} has several limitations that we leave for future work:
(1) \model{} can only invent \textit{dynamic} predicates. 
\textit{Static} predicates are still assumed as domain-level prior knowledge.
(2) Following previous works~\cite{kumar2024practice}, we have assumed the sparsity of effects; discussions about the general cases can be found in \appref{app:assumption}.
(3) Since \model{} trains different neural networks in each iteration, the learning time could be prolonged in domains with extreme complexity.
Currently, we parallelize the invention of different predicate groups on multiple GPUs for efficiency (see \appref{app:domain_details}).
(4) Neural predicates with quantifiers are less reliability due to the errors in neural classifiers.
Future work could explore the probabilistic symbolic planners~\cite{younes2004ppddl1} to handle the neural grounding noise.

\section*{Acknowledgment}
We acknowledge the support of the Air Force Research Laboratory (AFRL), DARPA, under agreement number FA8750-23-2-1015.
We also acknowledge Defence Science and Technology Agency (DSTA) under contract \#DST000EC124000205.
This work used Bridges-2 at PSC through allocation cis220039p from the Advanced Cyberinfrastructure Coordination Ecosystem: Services \& Support (ACCESS) program which is supported by NSF grants \#2138259, \#2138286, \#2138307, \#2137603, and \#213296.
We express sincere gratitude to Qinglin Feng for her valuable time in supporting our real-robot experiments and for her intelligence in motivating our Climb-Transport domain.
The authors would also like to express sincere gratitude to Jiayuan Mao (MIT), Nishanth Kumar (MIT), Prof. Katia Sycara (CMU), and Prof. Pradeep Ravikumar (CMU) for their valuable feedback, discussions, and suggestions on the early stages of this work.
Finally, the authors wish to thank our Spot robot, Spotless, for being reliable throughout our real-world experiments.

%% Use plainnat to work nicely with natbib. 

\bibliographystyle{unsrt}
\bibliography{references}

\newpage
\clearpage
\appendix
\subsection{Lloyd-Max Algorithm}
\label{subsec:Lloyd-Max}
For a given quantization bitwidth $B$ and an operand $\bm{X}$, the Lloyd-Max algorithm finds $2^B$ quantization levels $\{\hat{x}_i\}_{i=1}^{2^B}$ such that quantizing $\bm{X}$ by rounding each scalar in $\bm{X}$ to the nearest quantization level minimizes the quantization MSE. 

The algorithm starts with an initial guess of quantization levels and then iteratively computes quantization thresholds $\{\tau_i\}_{i=1}^{2^B-1}$ and updates quantization levels $\{\hat{x}_i\}_{i=1}^{2^B}$. Specifically, at iteration $n$, thresholds are set to the midpoints of the previous iteration's levels:
\begin{align*}
    \tau_i^{(n)}=\frac{\hat{x}_i^{(n-1)}+\hat{x}_{i+1}^{(n-1)}}2 \text{ for } i=1\ldots 2^B-1
\end{align*}
Subsequently, the quantization levels are re-computed as conditional means of the data regions defined by the new thresholds:
\begin{align*}
    \hat{x}_i^{(n)}=\mathbb{E}\left[ \bm{X} \big| \bm{X}\in [\tau_{i-1}^{(n)},\tau_i^{(n)}] \right] \text{ for } i=1\ldots 2^B
\end{align*}
where to satisfy boundary conditions we have $\tau_0=-\infty$ and $\tau_{2^B}=\infty$. The algorithm iterates the above steps until convergence.

Figure \ref{fig:lm_quant} compares the quantization levels of a $7$-bit floating point (E3M3) quantizer (left) to a $7$-bit Lloyd-Max quantizer (right) when quantizing a layer of weights from the GPT3-126M model at a per-tensor granularity. As shown, the Lloyd-Max quantizer achieves substantially lower quantization MSE. Further, Table \ref{tab:FP7_vs_LM7} shows the superior perplexity achieved by Lloyd-Max quantizers for bitwidths of $7$, $6$ and $5$. The difference between the quantizers is clear at 5 bits, where per-tensor FP quantization incurs a drastic and unacceptable increase in perplexity, while Lloyd-Max quantization incurs a much smaller increase. Nevertheless, we note that even the optimal Lloyd-Max quantizer incurs a notable ($\sim 1.5$) increase in perplexity due to the coarse granularity of quantization. 

\begin{figure}[h]
  \centering
  \includegraphics[width=0.7\linewidth]{sections/figures/LM7_FP7.pdf}
  \caption{\small Quantization levels and the corresponding quantization MSE of Floating Point (left) vs Lloyd-Max (right) Quantizers for a layer of weights in the GPT3-126M model.}
  \label{fig:lm_quant}
\end{figure}

\begin{table}[h]\scriptsize
\begin{center}
\caption{\label{tab:FP7_vs_LM7} \small Comparing perplexity (lower is better) achieved by floating point quantizers and Lloyd-Max quantizers on a GPT3-126M model for the Wikitext-103 dataset.}
\begin{tabular}{c|cc|c}
\hline
 \multirow{2}{*}{\textbf{Bitwidth}} & \multicolumn{2}{|c|}{\textbf{Floating-Point Quantizer}} & \textbf{Lloyd-Max Quantizer} \\
 & Best Format & Wikitext-103 Perplexity & Wikitext-103 Perplexity \\
\hline
7 & E3M3 & 18.32 & 18.27 \\
6 & E3M2 & 19.07 & 18.51 \\
5 & E4M0 & 43.89 & 19.71 \\
\hline
\end{tabular}
\end{center}
\end{table}

\subsection{Proof of Local Optimality of LO-BCQ}
\label{subsec:lobcq_opt_proof}
For a given block $\bm{b}_j$, the quantization MSE during LO-BCQ can be empirically evaluated as $\frac{1}{L_b}\lVert \bm{b}_j- \bm{\hat{b}}_j\rVert^2_2$ where $\bm{\hat{b}}_j$ is computed from equation (\ref{eq:clustered_quantization_definition}) as $C_{f(\bm{b}_j)}(\bm{b}_j)$. Further, for a given block cluster $\mathcal{B}_i$, we compute the quantization MSE as $\frac{1}{|\mathcal{B}_{i}|}\sum_{\bm{b} \in \mathcal{B}_{i}} \frac{1}{L_b}\lVert \bm{b}- C_i^{(n)}(\bm{b})\rVert^2_2$. Therefore, at the end of iteration $n$, we evaluate the overall quantization MSE $J^{(n)}$ for a given operand $\bm{X}$ composed of $N_c$ block clusters as:
\begin{align*}
    \label{eq:mse_iter_n}
    J^{(n)} = \frac{1}{N_c} \sum_{i=1}^{N_c} \frac{1}{|\mathcal{B}_{i}^{(n)}|}\sum_{\bm{v} \in \mathcal{B}_{i}^{(n)}} \frac{1}{L_b}\lVert \bm{b}- B_i^{(n)}(\bm{b})\rVert^2_2
\end{align*}

At the end of iteration $n$, the codebooks are updated from $\mathcal{C}^{(n-1)}$ to $\mathcal{C}^{(n)}$. However, the mapping of a given vector $\bm{b}_j$ to quantizers $\mathcal{C}^{(n)}$ remains as  $f^{(n)}(\bm{b}_j)$. At the next iteration, during the vector clustering step, $f^{(n+1)}(\bm{b}_j)$ finds new mapping of $\bm{b}_j$ to updated codebooks $\mathcal{C}^{(n)}$ such that the quantization MSE over the candidate codebooks is minimized. Therefore, we obtain the following result for $\bm{b}_j$:
\begin{align*}
\frac{1}{L_b}\lVert \bm{b}_j - C_{f^{(n+1)}(\bm{b}_j)}^{(n)}(\bm{b}_j)\rVert^2_2 \le \frac{1}{L_b}\lVert \bm{b}_j - C_{f^{(n)}(\bm{b}_j)}^{(n)}(\bm{b}_j)\rVert^2_2
\end{align*}

That is, quantizing $\bm{b}_j$ at the end of the block clustering step of iteration $n+1$ results in lower quantization MSE compared to quantizing at the end of iteration $n$. Since this is true for all $\bm{b} \in \bm{X}$, we assert the following:
\begin{equation}
\begin{split}
\label{eq:mse_ineq_1}
    \tilde{J}^{(n+1)} &= \frac{1}{N_c} \sum_{i=1}^{N_c} \frac{1}{|\mathcal{B}_{i}^{(n+1)}|}\sum_{\bm{b} \in \mathcal{B}_{i}^{(n+1)}} \frac{1}{L_b}\lVert \bm{b} - C_i^{(n)}(b)\rVert^2_2 \le J^{(n)}
\end{split}
\end{equation}
where $\tilde{J}^{(n+1)}$ is the the quantization MSE after the vector clustering step at iteration $n+1$.

Next, during the codebook update step (\ref{eq:quantizers_update}) at iteration $n+1$, the per-cluster codebooks $\mathcal{C}^{(n)}$ are updated to $\mathcal{C}^{(n+1)}$ by invoking the Lloyd-Max algorithm \citep{Lloyd}. We know that for any given value distribution, the Lloyd-Max algorithm minimizes the quantization MSE. Therefore, for a given vector cluster $\mathcal{B}_i$ we obtain the following result:

\begin{equation}
    \frac{1}{|\mathcal{B}_{i}^{(n+1)}|}\sum_{\bm{b} \in \mathcal{B}_{i}^{(n+1)}} \frac{1}{L_b}\lVert \bm{b}- C_i^{(n+1)}(\bm{b})\rVert^2_2 \le \frac{1}{|\mathcal{B}_{i}^{(n+1)}|}\sum_{\bm{b} \in \mathcal{B}_{i}^{(n+1)}} \frac{1}{L_b}\lVert \bm{b}- C_i^{(n)}(\bm{b})\rVert^2_2
\end{equation}

The above equation states that quantizing the given block cluster $\mathcal{B}_i$ after updating the associated codebook from $C_i^{(n)}$ to $C_i^{(n+1)}$ results in lower quantization MSE. Since this is true for all the block clusters, we derive the following result: 
\begin{equation}
\begin{split}
\label{eq:mse_ineq_2}
     J^{(n+1)} &= \frac{1}{N_c} \sum_{i=1}^{N_c} \frac{1}{|\mathcal{B}_{i}^{(n+1)}|}\sum_{\bm{b} \in \mathcal{B}_{i}^{(n+1)}} \frac{1}{L_b}\lVert \bm{b}- C_i^{(n+1)}(\bm{b})\rVert^2_2  \le \tilde{J}^{(n+1)}   
\end{split}
\end{equation}

Following (\ref{eq:mse_ineq_1}) and (\ref{eq:mse_ineq_2}), we find that the quantization MSE is non-increasing for each iteration, that is, $J^{(1)} \ge J^{(2)} \ge J^{(3)} \ge \ldots \ge J^{(M)}$ where $M$ is the maximum number of iterations. 
%Therefore, we can say that if the algorithm converges, then it must be that it has converged to a local minimum. 
\hfill $\blacksquare$


\begin{figure}
    \begin{center}
    \includegraphics[width=0.5\textwidth]{sections//figures/mse_vs_iter.pdf}
    \end{center}
    \caption{\small NMSE vs iterations during LO-BCQ compared to other block quantization proposals}
    \label{fig:nmse_vs_iter}
\end{figure}

Figure \ref{fig:nmse_vs_iter} shows the empirical convergence of LO-BCQ across several block lengths and number of codebooks. Also, the MSE achieved by LO-BCQ is compared to baselines such as MXFP and VSQ. As shown, LO-BCQ converges to a lower MSE than the baselines. Further, we achieve better convergence for larger number of codebooks ($N_c$) and for a smaller block length ($L_b$), both of which increase the bitwidth of BCQ (see Eq \ref{eq:bitwidth_bcq}).


\subsection{Additional Accuracy Results}
%Table \ref{tab:lobcq_config} lists the various LOBCQ configurations and their corresponding bitwidths.
\begin{table}
\setlength{\tabcolsep}{4.75pt}
\begin{center}
\caption{\label{tab:lobcq_config} Various LO-BCQ configurations and their bitwidths.}
\begin{tabular}{|c||c|c|c|c||c|c||c|} 
\hline
 & \multicolumn{4}{|c||}{$L_b=8$} & \multicolumn{2}{|c||}{$L_b=4$} & $L_b=2$ \\
 \hline
 \backslashbox{$L_A$\kern-1em}{\kern-1em$N_c$} & 2 & 4 & 8 & 16 & 2 & 4 & 2 \\
 \hline
 64 & 4.25 & 4.375 & 4.5 & 4.625 & 4.375 & 4.625 & 4.625\\
 \hline
 32 & 4.375 & 4.5 & 4.625& 4.75 & 4.5 & 4.75 & 4.75 \\
 \hline
 16 & 4.625 & 4.75& 4.875 & 5 & 4.75 & 5 & 5 \\
 \hline
\end{tabular}
\end{center}
\end{table}

%\subsection{Perplexity achieved by various LO-BCQ configurations on Wikitext-103 dataset}

\begin{table} \centering
\begin{tabular}{|c||c|c|c|c||c|c||c|} 
\hline
 $L_b \rightarrow$& \multicolumn{4}{c||}{8} & \multicolumn{2}{c||}{4} & 2\\
 \hline
 \backslashbox{$L_A$\kern-1em}{\kern-1em$N_c$} & 2 & 4 & 8 & 16 & 2 & 4 & 2  \\
 %$N_c \rightarrow$ & 2 & 4 & 8 & 16 & 2 & 4 & 2 \\
 \hline
 \hline
 \multicolumn{8}{c}{GPT3-1.3B (FP32 PPL = 9.98)} \\ 
 \hline
 \hline
 64 & 10.40 & 10.23 & 10.17 & 10.15 &  10.28 & 10.18 & 10.19 \\
 \hline
 32 & 10.25 & 10.20 & 10.15 & 10.12 &  10.23 & 10.17 & 10.17 \\
 \hline
 16 & 10.22 & 10.16 & 10.10 & 10.09 &  10.21 & 10.14 & 10.16 \\
 \hline
  \hline
 \multicolumn{8}{c}{GPT3-8B (FP32 PPL = 7.38)} \\ 
 \hline
 \hline
 64 & 7.61 & 7.52 & 7.48 &  7.47 &  7.55 &  7.49 & 7.50 \\
 \hline
 32 & 7.52 & 7.50 & 7.46 &  7.45 &  7.52 &  7.48 & 7.48  \\
 \hline
 16 & 7.51 & 7.48 & 7.44 &  7.44 &  7.51 &  7.49 & 7.47  \\
 \hline
\end{tabular}
\caption{\label{tab:ppl_gpt3_abalation} Wikitext-103 perplexity across GPT3-1.3B and 8B models.}
\end{table}

\begin{table} \centering
\begin{tabular}{|c||c|c|c|c||} 
\hline
 $L_b \rightarrow$& \multicolumn{4}{c||}{8}\\
 \hline
 \backslashbox{$L_A$\kern-1em}{\kern-1em$N_c$} & 2 & 4 & 8 & 16 \\
 %$N_c \rightarrow$ & 2 & 4 & 8 & 16 & 2 & 4 & 2 \\
 \hline
 \hline
 \multicolumn{5}{|c|}{Llama2-7B (FP32 PPL = 5.06)} \\ 
 \hline
 \hline
 64 & 5.31 & 5.26 & 5.19 & 5.18  \\
 \hline
 32 & 5.23 & 5.25 & 5.18 & 5.15  \\
 \hline
 16 & 5.23 & 5.19 & 5.16 & 5.14  \\
 \hline
 \multicolumn{5}{|c|}{Nemotron4-15B (FP32 PPL = 5.87)} \\ 
 \hline
 \hline
 64  & 6.3 & 6.20 & 6.13 & 6.08  \\
 \hline
 32  & 6.24 & 6.12 & 6.07 & 6.03  \\
 \hline
 16  & 6.12 & 6.14 & 6.04 & 6.02  \\
 \hline
 \multicolumn{5}{|c|}{Nemotron4-340B (FP32 PPL = 3.48)} \\ 
 \hline
 \hline
 64 & 3.67 & 3.62 & 3.60 & 3.59 \\
 \hline
 32 & 3.63 & 3.61 & 3.59 & 3.56 \\
 \hline
 16 & 3.61 & 3.58 & 3.57 & 3.55 \\
 \hline
\end{tabular}
\caption{\label{tab:ppl_llama7B_nemo15B} Wikitext-103 perplexity compared to FP32 baseline in Llama2-7B and Nemotron4-15B, 340B models}
\end{table}

%\subsection{Perplexity achieved by various LO-BCQ configurations on MMLU dataset}


\begin{table} \centering
\begin{tabular}{|c||c|c|c|c||c|c|c|c|} 
\hline
 $L_b \rightarrow$& \multicolumn{4}{c||}{8} & \multicolumn{4}{c||}{8}\\
 \hline
 \backslashbox{$L_A$\kern-1em}{\kern-1em$N_c$} & 2 & 4 & 8 & 16 & 2 & 4 & 8 & 16  \\
 %$N_c \rightarrow$ & 2 & 4 & 8 & 16 & 2 & 4 & 2 \\
 \hline
 \hline
 \multicolumn{5}{|c|}{Llama2-7B (FP32 Accuracy = 45.8\%)} & \multicolumn{4}{|c|}{Llama2-70B (FP32 Accuracy = 69.12\%)} \\ 
 \hline
 \hline
 64 & 43.9 & 43.4 & 43.9 & 44.9 & 68.07 & 68.27 & 68.17 & 68.75 \\
 \hline
 32 & 44.5 & 43.8 & 44.9 & 44.5 & 68.37 & 68.51 & 68.35 & 68.27  \\
 \hline
 16 & 43.9 & 42.7 & 44.9 & 45 & 68.12 & 68.77 & 68.31 & 68.59  \\
 \hline
 \hline
 \multicolumn{5}{|c|}{GPT3-22B (FP32 Accuracy = 38.75\%)} & \multicolumn{4}{|c|}{Nemotron4-15B (FP32 Accuracy = 64.3\%)} \\ 
 \hline
 \hline
 64 & 36.71 & 38.85 & 38.13 & 38.92 & 63.17 & 62.36 & 63.72 & 64.09 \\
 \hline
 32 & 37.95 & 38.69 & 39.45 & 38.34 & 64.05 & 62.30 & 63.8 & 64.33  \\
 \hline
 16 & 38.88 & 38.80 & 38.31 & 38.92 & 63.22 & 63.51 & 63.93 & 64.43  \\
 \hline
\end{tabular}
\caption{\label{tab:mmlu_abalation} Accuracy on MMLU dataset across GPT3-22B, Llama2-7B, 70B and Nemotron4-15B models.}
\end{table}


%\subsection{Perplexity achieved by various LO-BCQ configurations on LM evaluation harness}

\begin{table} \centering
\begin{tabular}{|c||c|c|c|c||c|c|c|c|} 
\hline
 $L_b \rightarrow$& \multicolumn{4}{c||}{8} & \multicolumn{4}{c||}{8}\\
 \hline
 \backslashbox{$L_A$\kern-1em}{\kern-1em$N_c$} & 2 & 4 & 8 & 16 & 2 & 4 & 8 & 16  \\
 %$N_c \rightarrow$ & 2 & 4 & 8 & 16 & 2 & 4 & 2 \\
 \hline
 \hline
 \multicolumn{5}{|c|}{Race (FP32 Accuracy = 37.51\%)} & \multicolumn{4}{|c|}{Boolq (FP32 Accuracy = 64.62\%)} \\ 
 \hline
 \hline
 64 & 36.94 & 37.13 & 36.27 & 37.13 & 63.73 & 62.26 & 63.49 & 63.36 \\
 \hline
 32 & 37.03 & 36.36 & 36.08 & 37.03 & 62.54 & 63.51 & 63.49 & 63.55  \\
 \hline
 16 & 37.03 & 37.03 & 36.46 & 37.03 & 61.1 & 63.79 & 63.58 & 63.33  \\
 \hline
 \hline
 \multicolumn{5}{|c|}{Winogrande (FP32 Accuracy = 58.01\%)} & \multicolumn{4}{|c|}{Piqa (FP32 Accuracy = 74.21\%)} \\ 
 \hline
 \hline
 64 & 58.17 & 57.22 & 57.85 & 58.33 & 73.01 & 73.07 & 73.07 & 72.80 \\
 \hline
 32 & 59.12 & 58.09 & 57.85 & 58.41 & 73.01 & 73.94 & 72.74 & 73.18  \\
 \hline
 16 & 57.93 & 58.88 & 57.93 & 58.56 & 73.94 & 72.80 & 73.01 & 73.94  \\
 \hline
\end{tabular}
\caption{\label{tab:mmlu_abalation} Accuracy on LM evaluation harness tasks on GPT3-1.3B model.}
\end{table}

\begin{table} \centering
\begin{tabular}{|c||c|c|c|c||c|c|c|c|} 
\hline
 $L_b \rightarrow$& \multicolumn{4}{c||}{8} & \multicolumn{4}{c||}{8}\\
 \hline
 \backslashbox{$L_A$\kern-1em}{\kern-1em$N_c$} & 2 & 4 & 8 & 16 & 2 & 4 & 8 & 16  \\
 %$N_c \rightarrow$ & 2 & 4 & 8 & 16 & 2 & 4 & 2 \\
 \hline
 \hline
 \multicolumn{5}{|c|}{Race (FP32 Accuracy = 41.34\%)} & \multicolumn{4}{|c|}{Boolq (FP32 Accuracy = 68.32\%)} \\ 
 \hline
 \hline
 64 & 40.48 & 40.10 & 39.43 & 39.90 & 69.20 & 68.41 & 69.45 & 68.56 \\
 \hline
 32 & 39.52 & 39.52 & 40.77 & 39.62 & 68.32 & 67.43 & 68.17 & 69.30  \\
 \hline
 16 & 39.81 & 39.71 & 39.90 & 40.38 & 68.10 & 66.33 & 69.51 & 69.42  \\
 \hline
 \hline
 \multicolumn{5}{|c|}{Winogrande (FP32 Accuracy = 67.88\%)} & \multicolumn{4}{|c|}{Piqa (FP32 Accuracy = 78.78\%)} \\ 
 \hline
 \hline
 64 & 66.85 & 66.61 & 67.72 & 67.88 & 77.31 & 77.42 & 77.75 & 77.64 \\
 \hline
 32 & 67.25 & 67.72 & 67.72 & 67.00 & 77.31 & 77.04 & 77.80 & 77.37  \\
 \hline
 16 & 68.11 & 68.90 & 67.88 & 67.48 & 77.37 & 78.13 & 78.13 & 77.69  \\
 \hline
\end{tabular}
\caption{\label{tab:mmlu_abalation} Accuracy on LM evaluation harness tasks on GPT3-8B model.}
\end{table}

\begin{table} \centering
\begin{tabular}{|c||c|c|c|c||c|c|c|c|} 
\hline
 $L_b \rightarrow$& \multicolumn{4}{c||}{8} & \multicolumn{4}{c||}{8}\\
 \hline
 \backslashbox{$L_A$\kern-1em}{\kern-1em$N_c$} & 2 & 4 & 8 & 16 & 2 & 4 & 8 & 16  \\
 %$N_c \rightarrow$ & 2 & 4 & 8 & 16 & 2 & 4 & 2 \\
 \hline
 \hline
 \multicolumn{5}{|c|}{Race (FP32 Accuracy = 40.67\%)} & \multicolumn{4}{|c|}{Boolq (FP32 Accuracy = 76.54\%)} \\ 
 \hline
 \hline
 64 & 40.48 & 40.10 & 39.43 & 39.90 & 75.41 & 75.11 & 77.09 & 75.66 \\
 \hline
 32 & 39.52 & 39.52 & 40.77 & 39.62 & 76.02 & 76.02 & 75.96 & 75.35  \\
 \hline
 16 & 39.81 & 39.71 & 39.90 & 40.38 & 75.05 & 73.82 & 75.72 & 76.09  \\
 \hline
 \hline
 \multicolumn{5}{|c|}{Winogrande (FP32 Accuracy = 70.64\%)} & \multicolumn{4}{|c|}{Piqa (FP32 Accuracy = 79.16\%)} \\ 
 \hline
 \hline
 64 & 69.14 & 70.17 & 70.17 & 70.56 & 78.24 & 79.00 & 78.62 & 78.73 \\
 \hline
 32 & 70.96 & 69.69 & 71.27 & 69.30 & 78.56 & 79.49 & 79.16 & 78.89  \\
 \hline
 16 & 71.03 & 69.53 & 69.69 & 70.40 & 78.13 & 79.16 & 79.00 & 79.00  \\
 \hline
\end{tabular}
\caption{\label{tab:mmlu_abalation} Accuracy on LM evaluation harness tasks on GPT3-22B model.}
\end{table}

\begin{table} \centering
\begin{tabular}{|c||c|c|c|c||c|c|c|c|} 
\hline
 $L_b \rightarrow$& \multicolumn{4}{c||}{8} & \multicolumn{4}{c||}{8}\\
 \hline
 \backslashbox{$L_A$\kern-1em}{\kern-1em$N_c$} & 2 & 4 & 8 & 16 & 2 & 4 & 8 & 16  \\
 %$N_c \rightarrow$ & 2 & 4 & 8 & 16 & 2 & 4 & 2 \\
 \hline
 \hline
 \multicolumn{5}{|c|}{Race (FP32 Accuracy = 44.4\%)} & \multicolumn{4}{|c|}{Boolq (FP32 Accuracy = 79.29\%)} \\ 
 \hline
 \hline
 64 & 42.49 & 42.51 & 42.58 & 43.45 & 77.58 & 77.37 & 77.43 & 78.1 \\
 \hline
 32 & 43.35 & 42.49 & 43.64 & 43.73 & 77.86 & 75.32 & 77.28 & 77.86  \\
 \hline
 16 & 44.21 & 44.21 & 43.64 & 42.97 & 78.65 & 77 & 76.94 & 77.98  \\
 \hline
 \hline
 \multicolumn{5}{|c|}{Winogrande (FP32 Accuracy = 69.38\%)} & \multicolumn{4}{|c|}{Piqa (FP32 Accuracy = 78.07\%)} \\ 
 \hline
 \hline
 64 & 68.9 & 68.43 & 69.77 & 68.19 & 77.09 & 76.82 & 77.09 & 77.86 \\
 \hline
 32 & 69.38 & 68.51 & 68.82 & 68.90 & 78.07 & 76.71 & 78.07 & 77.86  \\
 \hline
 16 & 69.53 & 67.09 & 69.38 & 68.90 & 77.37 & 77.8 & 77.91 & 77.69  \\
 \hline
\end{tabular}
\caption{\label{tab:mmlu_abalation} Accuracy on LM evaluation harness tasks on Llama2-7B model.}
\end{table}

\begin{table} \centering
\begin{tabular}{|c||c|c|c|c||c|c|c|c|} 
\hline
 $L_b \rightarrow$& \multicolumn{4}{c||}{8} & \multicolumn{4}{c||}{8}\\
 \hline
 \backslashbox{$L_A$\kern-1em}{\kern-1em$N_c$} & 2 & 4 & 8 & 16 & 2 & 4 & 8 & 16  \\
 %$N_c \rightarrow$ & 2 & 4 & 8 & 16 & 2 & 4 & 2 \\
 \hline
 \hline
 \multicolumn{5}{|c|}{Race (FP32 Accuracy = 48.8\%)} & \multicolumn{4}{|c|}{Boolq (FP32 Accuracy = 85.23\%)} \\ 
 \hline
 \hline
 64 & 49.00 & 49.00 & 49.28 & 48.71 & 82.82 & 84.28 & 84.03 & 84.25 \\
 \hline
 32 & 49.57 & 48.52 & 48.33 & 49.28 & 83.85 & 84.46 & 84.31 & 84.93  \\
 \hline
 16 & 49.85 & 49.09 & 49.28 & 48.99 & 85.11 & 84.46 & 84.61 & 83.94  \\
 \hline
 \hline
 \multicolumn{5}{|c|}{Winogrande (FP32 Accuracy = 79.95\%)} & \multicolumn{4}{|c|}{Piqa (FP32 Accuracy = 81.56\%)} \\ 
 \hline
 \hline
 64 & 78.77 & 78.45 & 78.37 & 79.16 & 81.45 & 80.69 & 81.45 & 81.5 \\
 \hline
 32 & 78.45 & 79.01 & 78.69 & 80.66 & 81.56 & 80.58 & 81.18 & 81.34  \\
 \hline
 16 & 79.95 & 79.56 & 79.79 & 79.72 & 81.28 & 81.66 & 81.28 & 80.96  \\
 \hline
\end{tabular}
\caption{\label{tab:mmlu_abalation} Accuracy on LM evaluation harness tasks on Llama2-70B model.}
\end{table}

%\section{MSE Studies}
%\textcolor{red}{TODO}


\subsection{Number Formats and Quantization Method}
\label{subsec:numFormats_quantMethod}
\subsubsection{Integer Format}
An $n$-bit signed integer (INT) is typically represented with a 2s-complement format \citep{yao2022zeroquant,xiao2023smoothquant,dai2021vsq}, where the most significant bit denotes the sign.

\subsubsection{Floating Point Format}
An $n$-bit signed floating point (FP) number $x$ comprises of a 1-bit sign ($x_{\mathrm{sign}}$), $B_m$-bit mantissa ($x_{\mathrm{mant}}$) and $B_e$-bit exponent ($x_{\mathrm{exp}}$) such that $B_m+B_e=n-1$. The associated constant exponent bias ($E_{\mathrm{bias}}$) is computed as $(2^{{B_e}-1}-1)$. We denote this format as $E_{B_e}M_{B_m}$.  

\subsubsection{Quantization Scheme}
\label{subsec:quant_method}
A quantization scheme dictates how a given unquantized tensor is converted to its quantized representation. We consider FP formats for the purpose of illustration. Given an unquantized tensor $\bm{X}$ and an FP format $E_{B_e}M_{B_m}$, we first, we compute the quantization scale factor $s_X$ that maps the maximum absolute value of $\bm{X}$ to the maximum quantization level of the $E_{B_e}M_{B_m}$ format as follows:
\begin{align}
\label{eq:sf}
    s_X = \frac{\mathrm{max}(|\bm{X}|)}{\mathrm{max}(E_{B_e}M_{B_m})}
\end{align}
In the above equation, $|\cdot|$ denotes the absolute value function.

Next, we scale $\bm{X}$ by $s_X$ and quantize it to $\hat{\bm{X}}$ by rounding it to the nearest quantization level of $E_{B_e}M_{B_m}$ as:

\begin{align}
\label{eq:tensor_quant}
    \hat{\bm{X}} = \text{round-to-nearest}\left(\frac{\bm{X}}{s_X}, E_{B_e}M_{B_m}\right)
\end{align}

We perform dynamic max-scaled quantization \citep{wu2020integer}, where the scale factor $s$ for activations is dynamically computed during runtime.

\subsection{Vector Scaled Quantization}
\begin{wrapfigure}{r}{0.35\linewidth}
  \centering
  \includegraphics[width=\linewidth]{sections/figures/vsquant.jpg}
  \caption{\small Vectorwise decomposition for per-vector scaled quantization (VSQ \citep{dai2021vsq}).}
  \label{fig:vsquant}
\end{wrapfigure}
During VSQ \citep{dai2021vsq}, the operand tensors are decomposed into 1D vectors in a hardware friendly manner as shown in Figure \ref{fig:vsquant}. Since the decomposed tensors are used as operands in matrix multiplications during inference, it is beneficial to perform this decomposition along the reduction dimension of the multiplication. The vectorwise quantization is performed similar to tensorwise quantization described in Equations \ref{eq:sf} and \ref{eq:tensor_quant}, where a scale factor $s_v$ is required for each vector $\bm{v}$ that maps the maximum absolute value of that vector to the maximum quantization level. While smaller vector lengths can lead to larger accuracy gains, the associated memory and computational overheads due to the per-vector scale factors increases. To alleviate these overheads, VSQ \citep{dai2021vsq} proposed a second level quantization of the per-vector scale factors to unsigned integers, while MX \citep{rouhani2023shared} quantizes them to integer powers of 2 (denoted as $2^{INT}$).

\subsubsection{MX Format}
The MX format proposed in \citep{rouhani2023microscaling} introduces the concept of sub-block shifting. For every two scalar elements of $b$-bits each, there is a shared exponent bit. The value of this exponent bit is determined through an empirical analysis that targets minimizing quantization MSE. We note that the FP format $E_{1}M_{b}$ is strictly better than MX from an accuracy perspective since it allocates a dedicated exponent bit to each scalar as opposed to sharing it across two scalars. Therefore, we conservatively bound the accuracy of a $b+2$-bit signed MX format with that of a $E_{1}M_{b}$ format in our comparisons. For instance, we use E1M2 format as a proxy for MX4.

\begin{figure}
    \centering
    \includegraphics[width=1\linewidth]{sections//figures/BlockFormats.pdf}
    \caption{\small Comparing LO-BCQ to MX format.}
    \label{fig:block_formats}
\end{figure}

Figure \ref{fig:block_formats} compares our $4$-bit LO-BCQ block format to MX \citep{rouhani2023microscaling}. As shown, both LO-BCQ and MX decompose a given operand tensor into block arrays and each block array into blocks. Similar to MX, we find that per-block quantization ($L_b < L_A$) leads to better accuracy due to increased flexibility. While MX achieves this through per-block $1$-bit micro-scales, we associate a dedicated codebook to each block through a per-block codebook selector. Further, MX quantizes the per-block array scale-factor to E8M0 format without per-tensor scaling. In contrast during LO-BCQ, we find that per-tensor scaling combined with quantization of per-block array scale-factor to E4M3 format results in superior inference accuracy across models. 


\end{document}