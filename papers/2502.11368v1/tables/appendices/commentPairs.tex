\begin{table*}[]
    \centering
    \scriptsize

\begin{tabular}{p{1cm}p{5cm}p{5cm}p{1cm}p{1.25cm}p{1cm}}
\toprule
                   Criterion &   GPT-4o-Aug &  GPT-4o-May &  BLEU &  ROUGE-L &  BERTScore \\
\midrule
  Content \& clarity of ideas &                                                    The content is generally relevant, but the clarity of ideas could be improved. Some sentences are complex and could be simplified for better understanding. For example, the sentence discussing Olson (2012) and Sanner (2008) could be broken down into shorter, more concise statements. Additionally, the review could benefit from more explicit explanations of key concepts, such as acculturation. &  The content is relevant and covers important aspects of the topic. However, some ideas are not clearly articulated. For example, the sentence "This is made worse because 'the nursing profession faces increased numbers of linguistically and culturally diverse patients who speak little or no English'" could be rephrased for clarity. Additionally, some sentences are overly complex and could be simplified for better understanding.&  0.20 &     0.30 &       0.63 \\ \midrule
          Logic of structure &                                                                                                                                                    The logical structure of the literature review could be improved. The review jumps between different topics without clear transitions, making it difficult for the reader to follow the argument. A more organized structure with clear subheadings and a logical flow of ideas would enhance readability. &                                                                                                                                                                                                                                                                                                                                                                                                         The logical structure of the literature review needs improvement. While the review covers important aspects of the psychological impact of COVID-19, the organization of the content is somewhat disjointed. Clearer subheadings and a more logical flow of ideas would help guide the reader through the review. For example, separating the discussion of different demographic groups or types of psychological impacts into distinct sections could enhance readability. &  0.17 &     0.41 &       0.70 \\ \midrule
          
   Quality of key components & The introduction provides a general overview of the topic but lacks a clear statement of the significance of the issue or specific research questions. The body of the literature review presents some relevant ideas but lacks a clear organization and focus on the pros and cons of online learning. The conclusion is missing, which is a significant omission as it should summarize key findings and highlight contributions or gaps in the literature. &  The introduction lacks a clear thesis statement and does not adequately set up the significance of the topic. The body of the review presents relevant ideas but lacks a clear structure and flow. The conclusion is weak and does not effectively summarize the key points or identify research trends or gaps. A more structured approach with clear sections and subheadings would improve readability and coherence. &  0.22 &     0.44 &       0.73 \\ \midrule
Grammar \& sentence structure &                                                                                                                                         The draft contains several grammatical errors and awkward sentence structures. For example, "The among of time is short considering the kind of assignments..." should be "The amount of time is short considering the type of assignments...". Varying sentence length and structure could also improve readability. &     The draft has several grammatical errors and issues with sentence structure. For example, "The among of time is short considering the kind of assignments" should be "The amount of time is short considering the type of assignments." Additionally, varying sentence length and structure would make the writing more engaging. Proofreading for grammatical accuracy and sentence variety is recommended. &  0.50 &     0.76 &       0.85 \\ \midrule
Grammar \& sentence structure &                                                                                                                                                                The grammatical accuracy is generally acceptable, but there are instances of awkward phrasing and run-on sentences. Varying sentence length and structure could improve readability. Additionally, ensuring subject-verb agreement and correct punctuation would enhance grammatical accuracy. & - Positive: The grammar is generally accurate.\newline- Improvement: Sentence length and variety need improvement. Some sentences are too long and complex, making them difficult to read.\newline- Example: "The health arguments against the use of cannabis include its addictive nature (Hurd et al., 2014) . It has also been directly linked to a range of adverse outcomes in physical health, which include lung cancer (Aldington et al., 2008), impaired respiratory function, cardiovascular disease, elevated systolic blood pressure, stroke (Singh et al., 2012), mental disorders (Saban et al., 2014), which include schizophrenia, especially amongst young people (Casadio et al., 2011), undesirable cognitive changes (Crean et al., 2011)." This could be broken down into shorter sentences. &  0.00 &     0.12 &       0.49 \\ 
\bottomrule
\end{tabular}

    \caption{Five random example comment pairs with their BLEU, ROUGE-L, and BERTScore scores provided. }
    \label{tab:commentPairs}
\end{table*}
