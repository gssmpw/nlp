\begin{table*}[]
    \centering
    \scriptsize
    
\begin{tabular}{p{3cm}|p{3cm}|p{4cm}|p{5cm}}
\toprule
             Comment &              Problem Extraction & Problem Classification & Correction Relevance Check \\
\midrule
The author has generally done a good job of integrating the source materials into the text, with clear summaries and explanations of the findings. However, there are some areas where the citation practices could be improved. For example, some of the in-text citations are not formatted correctly (e.g., "Wilkinson ST, et al., 2015" should be "Wilkinson et al., 2015"), and there are some inconsistencies in the reference list (e.g., some sources have DOIs, while others do not). Additionally, the author could benefit from using more precise language when describing the findings of the studies, rather than relying on general statements.  & There are some areas where the citation practices could be improved. For example, some of the in-text citations are not formatted correctly (e.g., "Wilkinson ST, et al., 2015" should be "Wilkinson et al., 2015"). & 1. Yes. The excerpt refers to a specific part of the essay by mentioning the in-text citation "Wilkinson ST, et al., 2015" and how it should be formatted. \newline 2. Yes. The excerpt offers a suggestion by indicating that the in-text citation formatting should be corrected. \newline 3. Yes. The excerpt provides a concrete correction by specifying the correct format for the citation as "Wilkinson et al., 2015." \newline\newline Final answers: Yes, Yes, Yes  &                   1. Yes, the problem pointed out in the excerpt exists in the corresponding essay. The excerpt mentions that the citation "Wilkinson ST, et al., 2015" is incorrectly formatted and should be "Wilkinson et al., 2015". In the essay, the citation appears as "Wilkinson ST, et al., 2015", which is indeed not formatted correctly according to APA style guidelines that recommend omitting the first names and middle initials of authors in in-text citations. \newline\newline

2. Yes, the problem pointed out in the excerpt is relevant to the corresponding assessment question. The assessment question asks to evaluate the essay based on the integration of source materials and citation practices. The excerpt directly addresses an issue with citation practices, which is a key aspect of the assessment question. \newline\newline

3. Yes, the correction of the problem pointed out in the excerpt is correct. The suggested correction to change "Wilkinson ST, et al., 2015" to "Wilkinson et al., 2015" aligns with APA style guidelines, which recommend using only the surname of the first author followed by "et al." for works authored by multiple people, without including initials unless necessary to distinguish between authors with the same surname. \newline\newline

Final answers: Yes, Yes, Yes \\ \noalign{\vskip 6pt} \cline{2-4} \noalign{\vskip 6pt}
 & There are some inconsistencies in the reference list (e.g., some sources have DOIs, while others do not). & 1. No - The excerpt does not refer to a specific part of the essay, as it does not point to a particular location or provide a specific example from the reference list. \newline 2. Yes - The excerpt suggests that the student should address the inconsistencies in the reference list, such as ensuring uniformity in the inclusion of DOIs. \newline 3. No - The excerpt does not provide a concrete correction, as it does not specify exactly how to fix the inconsistencies or what the corrected references should look like. \newline \newline Final answers: No, Yes, No &    \\ \noalign{\vskip 6pt} \cline{2-4} \noalign{\vskip 6pt}

 & The author could benefit from using more precise language when describing the findings of the studies, rather than relying on general statements. & 1. No - The excerpt does not refer to a specific part of the essay. It does not mention a concrete location or provide any referencable details about where the issue occurs. \newline 2. Yes - The excerpt offers a general suggestion for improvement by advising the author to use more precise language when describing the findings of the studies. \newline 3. No - The excerpt does not provide a concrete correction or specific example of how to fix the issue. \newline\newline Final answers: No, Yes, No &                \\
\bottomrule
\end{tabular}
    
    \caption{An example of implementing our framework on a feedback comment on the second assessment criteria, i.e., material integration \& citation, which is about citation practices. Our framework correctly identifies three problems identified in the comment and performs Problem Classification on each one of them, of which the first problem offers a concrete correction. The Correction Relevance Check is thus only performed on the first extracted problem.}
    \label{tab:frameworkImpExample}
\end{table*}