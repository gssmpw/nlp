\appendix

\begin{table}[]
    \centering
    \small
    \begin{tabular}{llllll}
    \toprule
    &  T1 & T2 & T3 & T4 & T5 \\ \midrule
    % Average Number of Authors & 1 & 2.4 & 1 & 2.2 & 1 \\ 
    % \midrule
    \# Essays     & 50 & 16 & 31 & 13 & 31 \\ 
    
    % \midrule
    Avg WC (w/o refs) & 845 & 1169 & 926 & 1079 & 887 \\ 
    
    % \midrule
    Avg WC (w/ refs)  & 1232 & 1583 & 1347 & 1666 & 1159 \\ \bottomrule
    \end{tabular}
    \caption{Basic statistics of the corpus. ``T'' in each column stands for ``Topic.'' ``WC'' means ``word count.''}
    \label{tab:basicStatistics}
\end{table}


\section{Corpus\label{app:corpus}}

\subsection{Basic Corpus Statistics}

Table~\ref{tab:basicStatistics} provides the basic statistics of the corpus. Note that throughout this study, we use the default word tokenizer of NLTK to compute word counts. See: \url{https://www.nltk.org/api/nltk.tokenize.html}.


\subsection{Details of the 5-Unit Tutorial Series}

Table~\ref{tab:details-of-the-5-unit-tutorial-series} presents details of the 5-unit tutorial series, including the themes, notions, activities, duration, and writing task for each unit. 

To support their writing, the authors were provided with a short, curated bibliography for each task, designed to help them focus on literature review writing while minimizing the effort required for bibliographic searches. Prior to submitting their final writing samples for expert assessments, the authors engaged in peer reviews (for topics 1, 3, and 5) or group collaboration (for topics 2 and 4). 

% Although there was no second round of feedback from human experts for each task, authors were encouraged to apply insights from previous feedback to subsequent writing tasks.

% % These authors were likely to be highly motivated L2 writers, given the fact that over 280 graduate students signed up for participation, but only 31 completed all the writing tasks, possibly due to the COVID-19 pandemic as well as the non-credit-bearing nature of the tutorial series. The participants were only compensated with a 100-Canadian-dollar git card for completing the entire tutorial series in 12 weeks, which was unlikely to be a main factor of motivation, since 20 out of 51 (39.2\%) authors quit the project even after they had submitted at least one writing samples. 


\subsection{Assessment Criteria\label{app:criteria}}

The 9 assessment criteria/questions provided to human assessors are detailed in Table~\ref{tab:assessmentCriteria}.


\begin{table*}[]
    \centering
    \scriptsize
    
    \begin{tabular}{p{2cm}|p{3.5cm}|p{3cm}|p{1.5cm}|p{3.5cm}}
    \toprule
      \textbf{Unit}  & \textbf{Key notions} & \textbf{Activities} & \textbf{Duration} & \textbf{Writing task} \\ \midrule
      1. Genre of literature review   & Components in literature review writing, material selection, citation practices & Interactive e-book, Peer-review, Discussion forum, quiz & 3 weeks & Individual writing on the social consequences of legalized cannabis \\ \midrule

      2. Structure and logic in literature review  & Types of logic structure, terms and abbreviations, Coherence, Cohesion & Interactive e-book, Discussion forum, quiz & 2 weeks & Collaborative writing on Canadian linguistic landscape \\ \midrule

      3. Sentence structures  & Sentence structures and variety, nominalization, Phrase bank and Swales’ CARS (Creating a Research Space) model & Interactive e-book, Peer-review, Discussion forum, quiz & 3 weeks & Individual writing on the pros and cons of online learning \\ \midrule

      4. Academic vocabulary  & (academic) formulaic expressions and their functions & Interactive e-book, Discussion forum, quiz & 2 weeks & Collaborative writing on lessons from the COVID- 19 pandemic \\ \midrule

      5. Grammar of reported speech   & Direct vs. indirect speech, reporting verbs and expressions, verb tenses, modal verbs & Interactive e-book, Peer-review, Discussion forum, quiz & 3 weeks & Individual writing on pacifism, peace-making, or just/justifiable war \\ \bottomrule
      
    \end{tabular}
    \caption{Details of the 5-unit online tutorial series.}
    \label{tab:details-of-the-5-unit-tutorial-series}
\end{table*}



\begin{table*}[]
    \centering
    \footnotesize
    \begin{tabular}{p{2.75cm}p{3.75cm}p{7.75cm}}
      \toprule
      Aspect & Criterion & Question \\ \midrule
         
    \multirow{3}{4em}{Selection of materials and citation practices}   & 1. Material selection & On a scale of 10 (1: Very poor, 10: Excellent), how would you evaluate the author’s selection of source materials in terms of relevance, quality, and quantity of the materials? Note: ``If the draft has a noticeable issue regarding the number or the quality of the papers reviewed, please comment on the issue." \\ \cmidrule{2-3}
         
          & 2. Material integration \newline and citation & On a scale of 10 (1: Very poor, 10: Excellent), how would you evaluate the writing for its integration of source materials (e.g., clarity of presenting information) and citation practices (e.g., use of APA or other style in both in-text citations and reference list)? \\ \midrule

     \multirow{3}{4em}{Overall structure}     & 3. Quality of key components & On a scale of 10 (1: Very poor, 10: Excellent), how would you evaluate the writing for the quality or effectiveness of each component (i.e., Introduction, Body, and Conclusions)? Note: The introduction is expected to introduce a research area, identify issue(s), and/or state the significance of the issue(s). The body of literature review should present the relevant ideas or findings of the reviewed studies and/or identify a research gap. The conclusion(s) may identify research trends or controversies and highlight the contribution of this literature review. \\ \cmidrule{2-3}
         
          & 4. Logic of structure & On a scale of 10 (1: Very poor, 10: Excellent), how would you evaluate the logical structure of this literature review? \\ \cmidrule{2-3}

          & 5. Content and clarity of ideas & On a scale of 10 (1: Very poor, 10: Excellent), how would you evaluate the content and clarity of ideas expressed in this literature review? \\ \midrule
         
     \multirow{2}{4em}{Coherence and cohesion}     & 6. Coherence & On a scale of 10 (1: Very poor, 10: Excellent), how would you evaluate the literature review for the quality of coherence (e.g., the connectivity and the naturalness of the flow of ideas in this draft)? \\ \cmidrule{2-3}

          & 7. Cohesion & On a scale of 10 (1: Very poor, 10: Excellent), how would you evaluate the literature review for the use of connectors (e.g., ‘because,’ ‘therefore,’ ‘however,’ ‘likewise’, and ‘similarly’) to link sentences in this draft? \\ \midrule

       \multirow{2}{4em}{Grammar and vocabulary}    & 8. Grammatical and sentence structure & On a scale of 10 (1: Very poor, 10: Excellent), how would you evaluate the draft for grammatical accuracy, sentence length and sentence type variety? \\ \cmidrule{2-3}
         
          & 9. Academy vocabulary & On a scale of 10 (1: Very poor, 10: Excellent), how would you evaluate the draft for vocabulary quality (e.g., use of academic expressions, the correctness of word choice, the naturalness of collocations, the complexity of vocabulary, the use of stylistically acceptable vocabulary—not too colloquial, not excessively formal or not overusing terms)? \\ \bottomrule
         
    \end{tabular}
    \caption{The 9 assessment criteria/questions, reflecting 4 general aspects of writing quality.}
    \label{tab:assessmentCriteria}
\end{table*}




\section{Feedback Comment Quality Evaluation Framework \label{app:framework}}

\subsection{Implementation}

The framework is implemented using LLMs. More concretely, we used \textsc{gpt-4o-2024-11-20} for Problem Extraction and Problem Classification, and \textsc{gpt-4-turbo-2024-04-09} for Correction Relevance Check. An example implementation of our framework can be found in Table~\ref{tab:frameworkImpExample}.

Related prompts used for implementing our framework can be found in Appendix~\ref{app:promptFramework}.


\begin{table*}[]
    \centering
    \scriptsize
    
\begin{tabular}{p{3cm}|p{3cm}|p{4cm}|p{5cm}}
\toprule
             Comment &              Problem Extraction & Problem Classification & Correction Relevance Check \\
\midrule
The author has generally done a good job of integrating the source materials into the text, with clear summaries and explanations of the findings. However, there are some areas where the citation practices could be improved. For example, some of the in-text citations are not formatted correctly (e.g., "Wilkinson ST, et al., 2015" should be "Wilkinson et al., 2015"), and there are some inconsistencies in the reference list (e.g., some sources have DOIs, while others do not). Additionally, the author could benefit from using more precise language when describing the findings of the studies, rather than relying on general statements.  & There are some areas where the citation practices could be improved. For example, some of the in-text citations are not formatted correctly (e.g., "Wilkinson ST, et al., 2015" should be "Wilkinson et al., 2015"). & 1. Yes. The excerpt refers to a specific part of the essay by mentioning the in-text citation "Wilkinson ST, et al., 2015" and how it should be formatted. \newline 2. Yes. The excerpt offers a suggestion by indicating that the in-text citation formatting should be corrected. \newline 3. Yes. The excerpt provides a concrete correction by specifying the correct format for the citation as "Wilkinson et al., 2015." \newline\newline Final answers: Yes, Yes, Yes  &                   1. Yes, the problem pointed out in the excerpt exists in the corresponding essay. The excerpt mentions that the citation "Wilkinson ST, et al., 2015" is incorrectly formatted and should be "Wilkinson et al., 2015". In the essay, the citation appears as "Wilkinson ST, et al., 2015", which is indeed not formatted correctly according to APA style guidelines that recommend omitting the first names and middle initials of authors in in-text citations. \newline\newline

2. Yes, the problem pointed out in the excerpt is relevant to the corresponding assessment question. The assessment question asks to evaluate the essay based on the integration of source materials and citation practices. The excerpt directly addresses an issue with citation practices, which is a key aspect of the assessment question. \newline\newline

3. Yes, the correction of the problem pointed out in the excerpt is correct. The suggested correction to change "Wilkinson ST, et al., 2015" to "Wilkinson et al., 2015" aligns with APA style guidelines, which recommend using only the surname of the first author followed by "et al." for works authored by multiple people, without including initials unless necessary to distinguish between authors with the same surname. \newline\newline

Final answers: Yes, Yes, Yes \\ \noalign{\vskip 6pt} \cline{2-4} \noalign{\vskip 6pt}
 & There are some inconsistencies in the reference list (e.g., some sources have DOIs, while others do not). & 1. No - The excerpt does not refer to a specific part of the essay, as it does not point to a particular location or provide a specific example from the reference list. \newline 2. Yes - The excerpt suggests that the student should address the inconsistencies in the reference list, such as ensuring uniformity in the inclusion of DOIs. \newline 3. No - The excerpt does not provide a concrete correction, as it does not specify exactly how to fix the inconsistencies or what the corrected references should look like. \newline \newline Final answers: No, Yes, No &    \\ \noalign{\vskip 6pt} \cline{2-4} \noalign{\vskip 6pt}

 & The author could benefit from using more precise language when describing the findings of the studies, rather than relying on general statements. & 1. No - The excerpt does not refer to a specific part of the essay. It does not mention a concrete location or provide any referencable details about where the issue occurs. \newline 2. Yes - The excerpt offers a general suggestion for improvement by advising the author to use more precise language when describing the findings of the studies. \newline 3. No - The excerpt does not provide a concrete correction or specific example of how to fix the issue. \newline\newline Final answers: No, Yes, No &                \\
\bottomrule
\end{tabular}
    
    \caption{An example of implementing our framework on a feedback comment on the second assessment criteria, i.e., material integration \& citation, which is about citation practices. Our framework correctly identifies three problems identified in the comment and performs Problem Classification on each one of them, of which the first problem offers a concrete correction. The Correction Relevance Check is thus only performed on the first extracted problem.}
    \label{tab:frameworkImpExample}
\end{table*}




\subsection{Annotation}

\paragraph{Guidelines} Table~\ref{tab:problemCharacterizations} provides explanations and examples of what is considered as a problem for Problem Extraction, and the three characteristics relevant to Problem Classification: whether an extracted problem (1) refers to a specific part of the essay, (2) provides a suggestion (general or specific), and (3) offers a concrete correction. 


\paragraph{Samples for Problem Extraction} We employed stratified sampling to randomly select 100 human-generated feedback comments and 108 LLM-generated feedback comments. In total, there are 208 comments for manual annotations.

For LLM-generated comments, half of them were generated under Interaction Mode 1 and the other half under Interaction Modes 2 and 3. Comments from Interaction Modes 2 and 3 were sampled together to reduce manual annotation effort, as these comments tend to be lengthy. The sampling covered the 9 assessment criteria, with 2 comments from each of the 3 LLMs used, resulting in 9 * 3 * 2 = 54 comments from Interaction Mode 1 and another 54 comments from the combined Interaction Modes 2 and 3.


\paragraph{Samples for Problem Classification} We randomly sampled 100 problems extracted from both human- and LLM-generated comments, resulting in 200 problems for annotations. 

Since the distribution of extracted problems across the nine assessment criteria are highly skewed, we ensured that there were at least 5 problems for each assessment criterion.


\paragraph{Problem Extraction} For each feedback comment, the two annotators were provided with LLM-extracted problems and asked to identify the number of correctly extracted problems (true positives), the number of incorrectly extracted problems (false positives), and the number of problems not extracted (false negatives). The number of true negatives is always set to 0, as there is no negative prediction in problem extraction.

A problem is considered correctly extracted if the LLM output contains the exact or paraphrased problem stated or implied in the feedback comment. It is acceptable if additional information relevant to the problem, such as elaborations, suggestions, clarifying questions, or quoted text from the assessed essay, is not included in the LLM-identified problems, which appears to be uncommon based on our annotations. However, if the problem and relevant additional information are extracted as separate problems, only the stated or implied problem is counted as a true positive, and the relevant information is treated as a false positive. This over-segmentation is the primary source of errors in LLM-extracted problems.


\paragraph{Problem Classification} For each extracted problem, the two annotators were asked to answer the three classification problems based on Table~\ref{tab:problemCharacterizations}. 


\begin{table*}[]
    \centering
    \footnotesize
    \begin{tabular}{p{2.75cm}p{4cm}p{7.5cm}}
      \toprule
      \textbf{Characteristic} & \textbf{Explanation} & \textbf{Examples} \\ \midrule

      If a problem is stated or implied in a comment & A problem is any writing-related issue that affects the quality of the writing, such as citation errors, logical flaws, coherence issues, grammatical mistakes, or inappropriate word choices, among others. The problem can be mentioned or implied in a given comment. & \underline{Positive Examples}
      \begin{itemize}
          \itemsep-0.25em
          % \item Decent use of connectors. I would suggest using more to reduce sentence length (implied problem: some sentences are too long)
          \item Specify what the abbreviation ``THC'' stands for. (Implied problem: ``THC'' is unspecified)
          \item There was a redundant use of ``the legalization of cannabis''.	
          % \item Very well integrated, a large number of good sources, but ...no reference list (stated problem: no reference list)
      \end{itemize} \underline{Negative Examples} \begin{itemize}
          \itemsep-0.25em
          \item Great grammatical skills, well done!
          \item Final references are well formatted. In-text references are well integrated.	
      \end{itemize} \\ \midrule

      If a problem points to a specific part of the essay & A specific part refers to a part of the essay that is easily locatable. (1) It can be a specific word, phrase, sentence, paragraph, reference etc. used in the essay. (2) It can be a concrete location, such as ``sentence 2 in paragraph 2,'' ``in paragraph 6,'' ``the first citation,'' or ``the first sentence of the paper'' and so on. (3) A less concrete location, such as ``the introduction,'' or ``the conclusion,'' is also considered a specific part if it is accompanied by some referenceable details. & \underline{Positive Examples}
      \begin{itemize}
          \itemsep-0.25em
          \item In Paragraph 2, the word ``decay'' is likely a mistake and should be replaced with ``decade''.	
          \item The sentence ``This theory still is under debate even with many authors provide a justification for that'' contains a grammatical error. The verb ``provide'' should be corrected to ``providing.''	
      \end{itemize} \underline{Negative Examples} \begin{itemize}
          \itemsep-0.25em
          \item Some of the sentences are a bit too long and fall apart a little.	
          % \item I would like to see a quotation or two.	
          \item Your paper would benefit from the use of expressions such as ``as a result'' or ``the result'' where cause and consequence are important.	
      \end{itemize} \\ \midrule

      If a problem offers some form of suggestions, general or specific & A suggestion indicates or implies ares of improvement. If the problem only contains a problem statement and it is unclear what direction the student should take to improve it, then there is no suggestion. A concrete correction is always considered a suggestion. & \underline{Positive Examples}
      \begin{itemize}
          \itemsep-0.25em
          \item Some sentences could be a bit shorter.	
          \item The use of a topic sentence for each paragraph in the main body could be improved.		
      \end{itemize} \underline{Negative Examples} \begin{itemize}
          \itemsep-0.25em
          \item The beginning of the literature review could be changed slightly.
          \item The first sentence of the paper is confusing.	
      \end{itemize} \\ \midrule

      If a problem provides a concrete correction for an identified writing issue & A concrete correction is something that can be directly applied to an essay to fix a writing problem. Corrections should not require thinking to implement, i.e. text that can be copy-pasted, or actions that can be taken following an instruction (e.g., capitalize the first letter). & \underline{Positive Examples}
      \begin{itemize}
          \itemsep-0.25em
          \item The citation ``(Toronto Star December 2016)'' should be revised to ``(Toronto Star 2016)'' to align with proper citation practices.	
          \item ``The advance of technologies'' should be corrected to ``the advancement of technologies''.		
      \end{itemize} \underline{Negative Examples} \begin{itemize}
          \itemsep-0.25em
          \item The significance of South Australian policy is unclear, as it is the first citation and the only one in the Introduction.	
          \item The conclusion is a little too short.	
      \end{itemize}  \\
      
      \bottomrule   
    \end{tabular}
    \caption{Explanations and illustrative examples of ``problems'' and their characterizations.}
    \label{tab:problemCharacterizations}
\end{table*}




\subsection{Correction Relevance Check\label{app:relevanceCheck}}

Table~\ref{tab:overallRelevanceCheckResults} demonstrates that comments generated by both humans and LLMs are overall highly relevant. However, human-generated comments tend to exhibit slightly lower relevance—either broadly or strictly—compared to those generated by LLMs. 

We conducted a small-scale error analysis to investigate the reasons behind the 8\%, 15\%, and 9\% of human-identified problems that GPT-4 incorrectly classified as not present in the essays, not adhering to the assessment criteria, and being incorrect, respectively. 

\paragraph{Problems not Present in Essays} We randomly selected 10 problems identified by GPT-4 as not present in the assessed essays. Upon reviewing each human-identified problem in the original essay, we found that 6 of these problems were indeed present, while 4 were not. Of the 4 problems that did not exist in the essays, 3 appeared to be misassigned comments (2 of these 3 were extracted from the same comment), while the remaining one seemed to be an assessor error. Among the 6 problems that GPT-4 misclassified, 4 were due to GPT-4 misunderstanding the identified problems, 1 was due to GPT-4 failing to locate a quoted word in the essay, and 1 was because GPT-4 mistakenly deemed the identified problem not to be a problem, despite its presence in the essay.

\paragraph{Problems not Adherent to the Assessment Criteria} We randomly selected 10 problems identified by GPT-4 as not adhering to the assessment criteria. Of these, 9 were related to C8 (grammar \& sentence structure), and 1 was related to C9 (academic vocabulary). Our manual validation showed that 7 of the problems were less related to grammar and sentence structure but more related to word choice or clarity of expression. The remaining 3 were misclassified by GPT-4, mostly due to its requirement that problems be explicitly related to both grammar and sentence structure in order to adhere to C8.

\paragraph{Correction being Incorrect} We randomly selected 10 problems containing corrections identified by GPT-4 as incorrect. We found that 5 of these problems involved accurate corrections, all related to grammar. There were 2 corrections proposed to be suggestions and 3 corrections that require subjective judgments to determine their correctness. 

\paragraph{Remarks} Based on this error analysis, we can attributed the discrepancy in relevance to two primary reasons: (1) human comments often include (inconsistent use of) diacritics that complicate problem extraction and characterization, and (2) human assessors may occasionally deviate from instructions, providing corrections unrelated to the assessment question. These issues are less frequent in LLM-generated comments, which benefit from their strong adherence to instructions and the ability to handle extended context windows. That said, both human- and LLM-identified problems are highly relevant.


\begin{table*}[]
    \centering
    \small

\begin{tabular}{lrrrrr}
\toprule
         Assessor &  In Essay &  In Question &  Is Correct &  Broadly Relevant &  Strictly Revelant \\
\midrule
               Human B &     87.9 &        79.4 &       85.1 &           84.4 &            72.8 \\
               Human C &     94.9 &        91.8 &       94.5 &           93.8 &            89.0 \\
               Human F &     96.3 &        86.7 &       91.4 &           90.9 &            82.3 \\ \midrule
    GPT-4o (IM 1) &    100.0 &       100.0 &      100.0 &          100.0 &           100.0 \\
Gemini-1.5 (IM 1) &     95.6 &        99.6 &       98.0 &           95.6 &            95.2 \\
   Llama-3 (IM 1) &     97.8 &        97.8 &       97.8 &           97.8 &            97.8 \\ \midrule
    GPT-4o (IM 2) &     99.6 &       100.0 &      100.0 &           99.6 &            99.6 \\
Gemini-1.5 (IM 2) &     98.3 &        98.8 &       97.5 &           97.1 &            96.6 \\
   Llama-3 (IM 2) &     94.7 &        96.2 &       96.2 &           94.4 &            92.5 \\ \midrule
    GPT-4o (IM 3) &    100.0 &        99.5 &       99.8 &           99.8 &            99.2 \\
Gemini-1.5 (IM 3) &     98.8 &        97.8 &       99.0 &           98.8 &            96.8 \\
   Llama-3 (IM 3) &     98.7 &        98.7 &       98.5 &           98.5 &            97.5 \\
\bottomrule
\end{tabular}
    
    \caption{Overall Correction Relevance Check results (\%), representing the percentage of instances each attribute is true for corrections made by an assessor. ``In Essay'': whether the problem indicated in the correction exists in the essay. ``In Question'': whether the correction relates to the assessment question. ``Is Correct'': whether the correction is correct. ``Broadly Relevant'': applicable when both ``In Essay'' and ``Is Correct'' are true. ``Strictly Revelant'': applicable when both ``Broadly Relevant'' and ``In Question'' are true.}
    \label{tab:overallRelevanceCheckResults}
\end{table*}









\section{Results\label{app:results}}


\subsection{Scores\label{app:scores}}

\paragraph{Scoring Ranges} Table~\ref{tab:descStats} summarizes the scoring ranges, in the form of means and standard deviations for each assessment criterion, as produced by three human assessors and the three LLMs under three interaction modes. 


\paragraph{Full QWK/AAR1} Table~\ref{tab:QWK_full} presents the full results for Quadratic Weighted Kappa (QWK) and Table~\ref{tab:AAR1_full} presents the full results for AAR1. 

\paragraph{Inconsistencies in Scoring by Human Assessors} First, there is an instance in the corpus, where assessor B accidentally assessed the same essay twice on separate days.\footnote{Four days apart and assessor B had no access to their earlier assessments.} While assessor B provided identical scores for 5 out of the 9 assessment criteria, discrepancies of 1 point occurred for the remaining 4 criteria, with scores alternating between (8, 7), (8, 7), (4, 5), and (7, 8). 

Second, we observe that human assessors assigned different scores to identical or similar comments, mostly within 1-point differences. For example, assessor F gave the same comment ``Decent number of citations'' three times but assigned three different scores: 6, 7, and 8. Similarly, assessor C assigned scores of 7 and 8 to the comment ``Appropriate use of connectors.'' However, when the same comment is repeated, scores tend to be very close, typically within one point. For instance, assessor A assigned a score of 8 to the comment ``Great use of academic words and formal tone'' five times, with only one instance where the score was 9.



\begin{table*}[]
    \small
    \centering

\begin{tabular}{llllllllll}
\toprule
         Assessor &             C1 &             C2 &             C3 &             C4 &             C5 &             C6 &             C7 &             C8 &             C9 \\
\midrule
             Human B & 6.7{\tiny±0.9} & 6.5{\tiny±1.2} & 7.5{\tiny±1.2} & 7.7{\tiny±1.1} & 7.7{\tiny±1.1} & 7.6{\tiny±1.1} & 7.3{\tiny±1.1} & 7.2{\tiny±1.1} & 7.5{\tiny±1.1} \\
               
              Human C & 7.8{\tiny±1.3} & 7.6{\tiny±1.3} & 7.9{\tiny±1.0} & 7.8{\tiny±1.3} & 7.8{\tiny±1.1} & 7.9{\tiny±1.1} & 8.1{\tiny±0.9} & 7.7{\tiny±1.1} & 8.2{\tiny±0.9} \\ 

              Human F & 7.0{\tiny±1.0} & 6.6{\tiny±1.0} & 6.9{\tiny±0.9} & 7.0{\tiny±0.8} & 7.1{\tiny±0.8} & 7.1{\tiny±0.8} & 7.2{\tiny±0.8} & 7.3{\tiny±0.7} & 7.0{\tiny±0.8} \\ \midrule
               
    GPT-4o (IM 1) & 7.4{\tiny±0.7} & 6.4{\tiny±0.7} & 5.7{\tiny±0.8} & 5.7{\tiny±0.9} & 6.3{\tiny±0.7} & 5.4{\tiny±0.7} & 5.5{\tiny±0.8} & 6.4{\tiny±0.9} & 6.7{\tiny±0.8} \\
    GPT-4o (IM 2) & 6.9{\tiny±0.7} & 6.0{\tiny±0.8} & 6.0{\tiny±0.8} & 5.6{\tiny±1.1} & 6.2{\tiny±0.8} & 5.4{\tiny±0.9} & 4.9{\tiny±0.8} & 6.2{\tiny±0.9} & 6.8{\tiny±0.9} \\
    GPT-4o (IM 3) & 6.9{\tiny±0.7} & 6.4{\tiny±0.7} & 6.0{\tiny±0.7} & 6.2{\tiny±0.7} & 6.4{\tiny±0.6} & 6.1{\tiny±0.7} & 6.0{\tiny±0.7} & 6.7{\tiny±0.7} & 6.8{\tiny±0.6} \\ \midrule
    
Gemini-1.5 (IM 1) & 6.3{\tiny±0.8} & 5.4{\tiny±0.7} & 5.5{\tiny±0.7} & 5.5{\tiny±1.0} & 6.0{\tiny±0.8} & 4.9{\tiny±0.8} & 4.5{\tiny±0.9} & 5.7{\tiny±0.8} & 6.1{\tiny±0.8} \\
Gemini-1.5 (IM 2) & 6.4{\tiny±0.6} & 6.3{\tiny±0.9} & 5.5{\tiny±0.7} & 5.8{\tiny±0.8} & 6.0{\tiny±0.5} & 5.4{\tiny±0.7} & 5.2{\tiny±0.8} & 6.4{\tiny±0.6} & 6.5{\tiny±0.6} \\
Gemini-1.5 (IM 3) & 6.4{\tiny±0.6} & 5.8{\tiny±0.6} & 5.5{\tiny±0.6} & 5.6{\tiny±0.5} & 5.7{\tiny±0.5} & 5.5{\tiny±0.6} & 5.4{\tiny±0.5} & 6.0{\tiny±0.6} & 6.1{\tiny±0.6} \\ \midrule

   Llama-3 (IM 1) & 7.5{\tiny±0.5} & 7.4{\tiny±0.7} & 6.4{\tiny±0.9} & 6.4{\tiny±1.2} & 7.1{\tiny±0.7} & 6.2{\tiny±0.8} & 5.2{\tiny±0.7} & 7.8{\tiny±0.5} & 7.1{\tiny±0.7} \\
   Llama-3 (IM 2) & 7.2{\tiny±0.6} & 6.8{\tiny±1.0} & 6.1{\tiny±1.1} & 6.4{\tiny±1.4} & 6.7{\tiny±1.1} & 6.2{\tiny±1.4} & 4.9{\tiny±1.4} & 7.3{\tiny±0.9} & 7.2{\tiny±0.8} \\
   Llama-3 (IM 3) & 7.2{\tiny±0.5} & 6.9{\tiny±0.5} & 6.4{\tiny±0.6} & 6.7{\tiny±0.6} & 6.8{\tiny±0.4} & 6.7{\tiny±0.6} & 5.9{\tiny±0.6} & 6.8{\tiny±0.4} & 6.8{\tiny±0.5} \\
\bottomrule
\end{tabular}

    \caption{Means and standard deviations of scores assigned by three human assessors and three LLMs prompted under three interaction modes (IM), denoted by ``IM'' in parentheses. C1: Material selection. C2: Material integration and citation; C3: Quality of key components. C4: Logic of structure. C5: Content and clarity of ideas. C6: Coherence (flow of ideas). C7: Cohesion (use of connectors). C8: Grammar and sentence structure. C9: Academic vocabulary.}
    \label{tab:descStats}
\end{table*}


\begin{table*}[]
    \footnotesize
    \centering

\begin{tabular}{lrrrrrrrrrr}
\toprule
                            Assessor &   C1 &   C2 &   C3 &    C4 &   C5 &   C6 &   C7 &   C8 &   C9 &  Overall \\
\midrule \midrule
                             Human B vs. Human F & 0.36 & 0.32 & 0.18 &  0.12 & 0.11 & 0.09 & 0.20 & 0.24 & 0.26 &     0.25 \\
                             Human B vs. Human C & 0.41 & 0.39 & 0.34 &  0.36 & 0.43 & 0.51 & 0.40 & 0.36 & 0.43 &     0.41 \\
                             Human F vs. Human C & 0.52 & 0.29 & 0.29 &  0.24 & 0.23 & 0.24 & 0.17 & 0.28 & 0.20 &     0.30 \\ \midrule \midrule
                             
                 Human B vs. GPT-4o (IM 1) & 0.23 & 0.29 & 0.08 &  0.06 & 0.11 & 0.05 & 0.06 & 0.25 & 0.10 &     0.03 \\
                 Human B vs. GPT-4o (IM 2) & 0.33 & 0.20 & 0.08 &  0.05 & 0.12 & 0.05 & 0.06 & 0.20 & 0.15 &     0.06 \\
                 Human B vs. GPT-4o (IM 3) & 0.26 & 0.30 & 0.09 &  0.07 & 0.14 & 0.07 & 0.09 & 0.28 & 0.18 &     0.10 \\ [0.15cm]
                 
             Human B vs. Gemini-1.5 (IM 1) & 0.29 & 0.15 & 0.08 &  0.07 & 0.10 & 0.04 & 0.05 & 0.12 & 0.08 &     0.07 \\
             Human B vs. Gemini-1.5 (IM 2) & 0.25 & 0.22 & 0.08 &  0.05 & 0.05 & 0.05 & 0.05 & 0.16 & 0.10 &     0.04 \\
             Human B vs. Gemini-1.5 (IM 3) & 0.25 & 0.18 & 0.06 &  0.06 & 0.06 & 0.04 & 0.05 & 0.09 & 0.03 &     0.04 \\ [0.15cm]
             
                Human B vs. Llama-3 (IM 1) & 0.10 & 0.04 & 0.07 & -0.03 & 0.03 & 0.01 & 0.03 & 0.06 & 0.08 &    -0.06 \\
                Human B vs. Llama-3 (IM 2) & 0.27 & 0.23 & 0.16 &  0.14 & 0.22 & 0.11 & 0.06 & 0.25 & 0.13 &     0.10 \\
                Human B vs. Llama-3 (IM 3) & 0.26 & 0.18 & 0.07 &  0.13 & 0.14 & 0.14 & 0.07 & 0.09 & 0.06 &     0.07 \\ \midrule
                
                 Human C vs. GPT-4o (IM 1) & 0.36 & 0.28 & 0.10 &  0.12 & 0.17 & 0.07 & 0.03 & 0.22 & 0.15 &     0.13 \\
                 Human C vs. GPT-4o (IM 2) & 0.27 & 0.21 & 0.14 &  0.07 & 0.15 & 0.05 & 0.04 & 0.20 & 0.16 &     0.11 \\
                 Human C vs. GPT-4o (IM 3) & 0.23 & 0.25 & 0.09 &  0.13 & 0.19 & 0.08 & 0.06 & 0.30 & 0.17 &     0.15 \\ [0.15cm]
                 
             Human C vs. Gemini-1.5 (IM 1) & 0.19 & 0.11 & 0.09 &  0.11 & 0.14 & 0.06 & 0.03 & 0.10 & 0.09 &     0.08 \\
             Human C vs. Gemini-1.5 (IM 2) & 0.12 & 0.21 & 0.08 &  0.05 & 0.08 & 0.06 & 0.04 & 0.15 & 0.10 &     0.08 \\
             Human C vs. Gemini-1.5 (IM 3) & 0.12 & 0.15 & 0.08 &  0.07 & 0.08 & 0.07 & 0.01 & 0.11 & 0.05 &     0.07 \\ [0.15cm]
             
                Human C vs. Llama-3 (IM 1) & 0.24 & 0.16 & 0.09 &  0.08 & 0.21 & 0.08 & 0.02 & 0.22 & 0.10 &     0.06 \\
                Human C vs. Llama-3 (IM 2) & 0.27 & 0.36 & 0.11 &  0.19 & 0.28 & 0.10 & 0.04 & 0.43 & 0.13 &     0.14 \\
                Human C vs. Llama-3 (IM 3) & 0.27 & 0.30 & 0.10 &  0.15 & 0.17 & 0.16 & 0.06 & 0.20 & 0.12 &     0.14 \\ \midrule
                
                 Human F vs. GPT-4o (IM 1) & 0.44 & 0.32 & 0.24 &  0.17 & 0.22 & 0.09 & 0.07 & 0.14 & 0.26 &     0.17 \\
                 Human F vs. GPT-4o (IM 2) & 0.51 & 0.30 & 0.36 &  0.17 & 0.25 & 0.12 & 0.06 & 0.10 & 0.27 &     0.17 \\
                 Human F vs. GPT-4o (IM 3) & 0.47 & 0.29 & 0.25 &  0.29 & 0.25 & 0.19 & 0.07 & 0.14 & 0.32 &     0.24 \\ [0.15cm] 
                 
             Human F vs. Gemini-1.5 (IM 1) & 0.37 & 0.16 & 0.22 &  0.11 & 0.18 & 0.08 & 0.03 & 0.05 & 0.13 &     0.11 \\
             Human F vs. Gemini-1.5 (IM 2) & 0.29 & 0.16 & 0.14 &  0.14 & 0.10 & 0.09 & 0.05 & 0.12 & 0.21 &     0.11 \\
             Human F vs. Gemini-1.5 (IM 3) & 0.29 & 0.12 & 0.17 &  0.14 & 0.09 & 0.09 & 0.05 & 0.03 & 0.09 &     0.10 \\ [0.15cm]
             
                Human F vs. Llama-3 (IM 1) & 0.32 & 0.07 & 0.28 &  0.24 & 0.18 & 0.19 & 0.04 & 0.10 & 0.23 &     0.13 \\
                Human F vs. Llama-3 (IM 2) & 0.50 & 0.18 & 0.23 &  0.22 & 0.21 & 0.22 & 0.05 & 0.07 & 0.26 &     0.16 \\
                Human F vs. Llama-3 (IM 3) & 0.50 & 0.21 & 0.27 &  0.35 & 0.19 & 0.25 & 0.04 & 0.05 & 0.12 &     0.18 \\ \midrule \midrule
                
    GPT-4o (IM 1) vs. Llama-3 (IM 1) & 0.59 & 0.01 & 0.35 &  0.30 & 0.28 & 0.20 & 0.36 & 0.07 & 0.41 &     0.45 \\
 GPT-4o (IM 1) vs. Gemini-1.5 (IM 1) & 0.33 & 0.38 & 0.65 &  0.64 & 0.59 & 0.58 & 0.34 & 0.51 & 0.51 &     0.60 \\
Llama-3 (IM 1) vs. Gemini-1.5 (IM 1) & 0.23 & 0.01 & 0.27 &  0.23 & 0.19 & 0.14 & 0.28 & 0.04 & 0.26 &     0.30 \\ [0.15cm]

 GPT-4o (IM 2) vs. Gemini-1.5 (IM 2) & 0.49 & 0.39 & 0.48 &  0.59 & 0.53 & 0.56 & 0.48 & 0.56 & 0.47 &     0.64 \\
    GPT-4o (IM 2) vs. Llama-3 (IM 2) & 0.62 & 0.33 & 0.57 &  0.47 & 0.57 & 0.46 & 0.52 & 0.37 & 0.52 &     0.60 \\
Gemini-1.5 (IM 2) vs. Llama-3 (IM 2) & 0.33 & 0.32 & 0.36 &  0.35 & 0.27 & 0.36 & 0.30 & 0.30 & 0.23 &     0.47 \\ [0.15cm]

    Llama-3 (IM 3) vs. GPT-4o (IM 3) & 0.56 & 0.40 & 0.48 &  0.53 & 0.36 & 0.46 & 0.55 & 0.56 & 0.58 &     0.58 \\
Llama-3 (IM 3) vs. Gemini-1.5 (IM 3) & 0.33 & 0.21 & 0.28 &  0.28 & 0.15 & 0.24 & 0.30 & 0.24 & 0.24 &     0.33 \\
 GPT-4o (IM 3) vs. Gemini-1.5 (IM 3) & 0.49 & 0.50 & 0.50 &  0.44 & 0.34 & 0.49 & 0.38 & 0.41 & 0.35 &     0.52 \\
\bottomrule
\end{tabular}
    
    \caption{Full QWK (Quadratic Weighted Kappa) results between all assessor pairs, evaluated at the level of each assessment criterion and the whole essay (``Overall''). C1: Material selection. C2: Material integration and citation; C3: Quality of key components. C4: Logic of structure. C5: Content and clarity of ideas. C6: Coherence (flow of ideas). C7: Cohesion (use of connectors). C8: Grammar and sentence structure. C9: Academic vocabulary.}
    \label{tab:QWK_full}
\end{table*}
\begin{table*}[]
    \footnotesize
    \centering

\begin{tabular}{lrrrrrrrrrr}
\toprule
                            Assessor &   C1 &   C2 &   C3 &   C4 &   C5 &   C6 &   C7 &   C8 &   C9 &  Overall \\
\midrule \midrule
                             Human B vs. Human F & 0.90 & 0.77 & 0.73 & 0.69 & 0.75 & 0.75 & 0.85 & 0.86 & 0.79 &     0.79 \\
                             Human B vs. Human C & 0.58 & 0.58 & 0.70 & 0.80 & 0.82 & 0.86 & 0.75 & 0.76 & 0.80 &     0.74 \\
                             Human F vs. Human C & 0.73 & 0.54 & 0.60 & 0.64 & 0.68 & 0.69 & 0.65 & 0.74 & 0.59 &     0.65 \\ \midrule \midrule
                             
                 Human B vs. GPT-4o (IM 1) & 0.89 & 0.84 & 0.33 & 0.27 & 0.46 & 0.25 & 0.36 & 0.74 & 0.60 &     0.53 \\
                 Human B vs. GPT-4o (IM 2) & 0.94 & 0.76 & 0.42 & 0.30 & 0.43 & 0.27 & 0.17 & 0.70 & 0.68 &     0.52 \\
                 Human B vs. GPT-4o (IM 3) & 0.94 & 0.85 & 0.37 & 0.41 & 0.50 & 0.40 & 0.54 & 0.84 & 0.74 &     0.62 \\ [0.15cm]
                 
             Human B vs. Gemini-1.5 (IM 1) & 0.88 & 0.59 & 0.27 & 0.24 & 0.33 & 0.13 & 0.12 & 0.46 & 0.42 &     0.38 \\
             Human B vs. Gemini-1.5 (IM 2) & 0.94 & 0.75 & 0.28 & 0.32 & 0.30 & 0.23 & 0.29 & 0.71 & 0.62 &     0.49 \\
             Human B vs. Gemini-1.5 (IM 3) & 0.94 & 0.72 & 0.29 & 0.25 & 0.23 & 0.24 & 0.32 & 0.60 & 0.45 &     0.45 \\ [0.15cm]
             
                Human B vs. Llama-3 (IM 1) & 0.84 & 0.76 & 0.49 & 0.54 & 0.74 & 0.46 & 0.27 & 0.79 & 0.74 &     0.63 \\
                Human B vs. Llama-3 (IM 2) & 0.88 & 0.73 & 0.38 & 0.50 & 0.65 & 0.45 & 0.25 & 0.77 & 0.71 &     0.59 \\ 
                Human B vs. Llama-3 (IM 3) & 0.89 & 0.84 & 0.50 & 0.57 & 0.70 & 0.63 & 0.46 & 0.87 & 0.72 &     0.69 \\ \midrule
                
                 Human C vs. GPT-4o (IM 1) & 0.74 & 0.54 & 0.20 & 0.22 & 0.46 & 0.10 & 0.13 & 0.45 & 0.42 &     0.36 \\
                 Human C vs. GPT-4o (IM 2) & 0.65 & 0.44 & 0.32 & 0.17 & 0.38 & 0.15 & 0.03 & 0.43 & 0.54 &     0.35 \\
                 Human C vs. GPT-4o (IM 3) & 0.62 & 0.57 & 0.27 & 0.39 & 0.48 & 0.33 & 0.24 & 0.66 & 0.57 &     0.46 \\ [0.15cm]
                 
             Human C vs. Gemini-1.5 (IM 1) & 0.42 & 0.21 & 0.17 & 0.15 & 0.30 & 0.05 & 0.03 & 0.29 & 0.23 &     0.21 \\
             Human C vs. Gemini-1.5 (IM 2) & 0.47 & 0.51 & 0.16 & 0.26 & 0.35 & 0.17 & 0.08 & 0.53 & 0.45 &     0.33 \\
             Human C vs. Gemini-1.5 (IM 3) & 0.47 & 0.32 & 0.14 & 0.18 & 0.22 & 0.15 & 0.09 & 0.38 & 0.24 &     0.24 \\ [0.15cm]
             
                Human C vs. Llama-3 (IM 1) & 0.82 & 0.73 & 0.49 & 0.47 & 0.74 & 0.46 & 0.03 & 0.86 & 0.68 &     0.59 \\
                Human C vs. Llama-3 (IM 2) & 0.71 & 0.67 & 0.24 & 0.42 & 0.57 & 0.31 & 0.13 & 0.84 & 0.72 &     0.51 \\
                Human C vs. Llama-3 (IM 3) & 0.71 & 0.75 & 0.44 & 0.57 & 0.68 & 0.57 & 0.18 & 0.68 & 0.58 &     0.57 \\ \midrule
                
                 Human F vs. GPT-4o (IM 1) & 0.94 & 0.82 & 0.61 & 0.56 & 0.80 & 0.37 & 0.41 & 0.71 & 0.87 &     0.68 \\
                 Human F vs. GPT-4o (IM 2) & 0.96 & 0.80 & 0.79 & 0.42 & 0.71 & 0.32 & 0.17 & 0.65 & 0.86 &     0.63 \\
                 Human F vs. GPT-4o (IM 3) & 0.94 & 0.85 & 0.71 & 0.83 & 0.87 & 0.70 & 0.60 & 0.86 & 0.93 &     0.81 \\ [0.15cm] 
                 
             Human F vs. Gemini-1.5 (IM 1) & 0.77 & 0.63 & 0.56 & 0.50 & 0.64 & 0.19 & 0.10 & 0.45 & 0.72 &     0.51 \\
             Human F vs. Gemini-1.5 (IM 2) & 0.80 & 0.76 & 0.54 & 0.65 & 0.62 & 0.43 & 0.27 & 0.76 & 0.89 &     0.64 \\
             Human F vs. Gemini-1.5 (IM 3) & 0.80 & 0.75 & 0.53 & 0.57 & 0.53 & 0.44 & 0.35 & 0.57 & 0.75 &     0.59 \\ [0.15cm]
             
                Human F vs. Llama-3 (IM 1) & 0.87 & 0.68 & 0.79 & 0.73 & 0.89 & 0.75 & 0.29 & 0.91 & 0.92 &     0.76 \\
                Human F vs. Llama-3 (IM 2) & 0.97 & 0.75 & 0.58 & 0.65 & 0.76 & 0.56 & 0.28 & 0.81 & 0.89 &     0.69 \\
                Human F vs. Llama-3 (IM 3) & 0.97 & 0.87 & 0.84 & 0.92 & 0.96 & 0.90 & 0.58 & 0.91 & 0.93 &     0.88 \\ \midrule \midrule
                
    GPT-4o (IM 1) vs. Llama-3 (IM 1) & 0.99 & 0.62 & 0.80 & 0.66 & 0.89 & 0.85 & 0.91 & 0.56 & 0.95 &     0.80 \\
 GPT-4o (IM 1) vs. Gemini-1.5 (IM 1) & 0.82 & 0.89 & 1.00 & 0.96 & 0.99 & 0.99 & 0.77 & 0.89 & 0.93 &     0.92 \\
Llama-3 (IM 1) vs. Gemini-1.5 (IM 1) & 0.73 & 0.42 & 0.71 & 0.63 & 0.66 & 0.59 & 0.81 & 0.22 & 0.74 &     0.61 \\ [0.15cm]

 GPT-4o (IM 2) vs. Gemini-1.5 (IM 2) & 1.00 & 0.90 & 0.95 & 0.94 & 0.99 & 0.97 & 0.93 & 0.97 & 0.96 &     0.96 \\
    GPT-4o (IM 2) vs. Llama-3 (IM 2) & 0.99 & 0.72 & 0.87 & 0.67 & 0.90 & 0.71 & 0.84 & 0.71 & 0.91 &     0.81 \\
Gemini-1.5 (IM 2) vs. Llama-3 (IM 2) & 0.97 & 0.84 & 0.72 & 0.63 & 0.78 & 0.67 & 0.76 & 0.78 & 0.85 &     0.78 \\ [0.15cm]

    Llama-3 (IM 3) vs. GPT-4o (IM 3) & 0.99 & 0.98 & 0.91 & 1.00 & 0.97 & 1.00 & 0.99 & 0.99 & 1.00 &     0.98 \\
Llama-3 (IM 3) vs. Gemini-1.5 (IM 3) & 0.97 & 0.85 & 0.81 & 0.91 & 0.85 & 0.81 & 0.99 & 0.94 & 0.97 &     0.90 \\
 GPT-4o (IM 3) vs. Gemini-1.5 (IM 3) & 1.00 & 1.00 & 0.95 & 0.99 & 0.99 & 1.00 & 0.99 & 0.96 & 0.96 &     0.98 \\
\bottomrule
\end{tabular}
    
    \caption{Full AAR1 (adjacent agreement rate with $k=1$) results between all assessor pairs, evaluated at the level of each assessment criterion and the whole essay (``Overall''). C1: Material selection. C2: Material integration and citation; C3: Quality of key components. C4: Logic of structure. C5: Content and clarity of ideas. C6: Coherence (flow of ideas). C7: Cohesion (use of connectors). C8: Grammar and sentence structure. C9: Academic vocabulary.}
    \label{tab:AAR1_full}
\end{table*}


\subsection{Comments\label{app:comments}}
 
\begin{table*}[]
    \centering
    \small
    % \tiny

% \begin{tabular}{lllllllllll}
% \toprule
%      Attr &          Assessor &            C1 &             C2 &             C3 &             C4 &             C5 &             C6 &              C7 &             C8 &             C9 \\
% \midrule

%  CR &                 B &           5.6 &           28.2 &            2.8 &            0.7 &            2.1 &            3.5 &            13.4 &           91.5 &           66.2 \\
%   &                 C &         100 &          100 &          100 &           98.9 &          100 &           98.9 &           100 &          100 &          100 \\
%   &                 F &          99.1 &           97.2 &           96.2 &           80.2 &           87.7 &           83.0 &            89.6 &           90.6 &           82.1 \\
  
% PR &                 B &            75 &            100 &            100 &            100 &            100 &             80 &              84 &            100 &             95 \\
%  &                 C &            19 &             72 &             54 &             50 &             81 &             40 &              29 &             79 &             80 \\
% &                 F &            47 &             84 &             82 &             44 &             61 &             50 &              49 &             64 &             82 \\ 

%    &                 B &  26{\tiny±23} &   43{\tiny±32} &   59{\tiny±33} &    45{\tiny±0} &   50{\tiny±53} &   34{\tiny±26} &    46{\tiny±24} &  147{\tiny±83} &   97{\tiny±85} \\
%    &                 C &  17{\tiny±23} & 104{\tiny±122} &   39{\tiny±38} &   56{\tiny±77} & 112{\tiny±102} &   38{\tiny±69} &    26{\tiny±39} &  103{\tiny±88} &   65{\tiny±94} \\
%    &                 F &  26{\tiny±40} &   77{\tiny±75} &   51{\tiny±30} &   30{\tiny±39} &   51{\tiny±57} &   31{\tiny±31} &    27{\tiny±33} &   52{\tiny±67} &   79{\tiny±92} \\ \midrule

%    &     {\tiny GPT-4o (IM 1)} &  79{\tiny±10} &   82{\tiny±13} &    72{\tiny±8} &    59{\tiny±7} &   61{\tiny±10} &    53{\tiny±7} &     55{\tiny±9} &   59{\tiny±12} &   67{\tiny±10} \\
%    &     {\tiny GPT-4o (IM 2)} & 291{\tiny±44} &  353{\tiny±40} &  332{\tiny±30} &  333{\tiny±37} &  374{\tiny±39} &  362{\tiny±42} &   347{\tiny±36} &  370{\tiny±45} &  357{\tiny±42} \\
%    &     {\tiny GPT-4o (IM 3)} & 295{\tiny±40} &  437{\tiny±40} &  372{\tiny±51} &  380{\tiny±40} &  444{\tiny±37} &  402{\tiny±39} &   358{\tiny±60} &  422{\tiny±50} &  321{\tiny±40} \\
   
%    & {\tiny Gemini-1.5 (IM 1)} &  98{\tiny±22} &  126{\tiny±29} &  120{\tiny±33} &   82{\tiny±19} &   91{\tiny±27} &   84{\tiny±22} &    85{\tiny±24} &   90{\tiny±26} &   99{\tiny±49} \\
%    & {\tiny Gemini-1.5 (IM 2)} & 378{\tiny±88} & 446{\tiny±111} & 512{\tiny±106} & 399{\tiny±103} & 425{\tiny±121} & 397{\tiny±109} & 867{\tiny±2032} & 468{\tiny±148} & 400{\tiny±107} \\
%    & {\tiny Gemini-1.5 (IM 3)} & 374{\tiny±86} & 654{\tiny±107} & 689{\tiny±110} &  592{\tiny±94} & 655{\tiny±102} & 559{\tiny±125} &  473{\tiny±109} & 642{\tiny±131} & 505{\tiny±343} \\
   
%    &    {\tiny Llama-3 (IM 1)} &  90{\tiny±13} &   91{\tiny±18} &   87{\tiny±14} &   60{\tiny±10} &   64{\tiny±15} &   54{\tiny±12} &    52{\tiny±15} &   56{\tiny±13} &   62{\tiny±13} \\
%    &    {\tiny Llama-3 (IM 2)} & 331{\tiny±35} &  368{\tiny±41} & 438{\tiny±111} & 466{\tiny±197} &  357{\tiny±68} & 351{\tiny±107} &   317{\tiny±88} &  345{\tiny±82} &  355{\tiny±97} \\
%    &    {\tiny Llama-3 (IM 3)} & 333{\tiny±35} &  425{\tiny±53} &  409{\tiny±49} &  378{\tiny±51} &  481{\tiny±56} &  389{\tiny±51} &   366{\tiny±58} &  445{\tiny±62} &  362{\tiny±44} \\
% \bottomrule
% \end{tabular}



\begin{tabular}{lllllllllll}
\toprule
Attr &          Assessor &             C1 &             C2 &              C3 &             C4 &             C5 &             C6 &              C7 &             C8 &             C9 \\
\midrule

  CR &               Human  B &            5.6 &           28.2 &             2.8 &            0.7 &            2.1 &            3.5 &            13.4 &           91.5 &           66.2 \\
   &                 Human C &          100 &          100 &           100 &           98.9 &          100 &           98.9 &           100 &          100 &          100 \\
   &                Human F &           99.1 &           97.2 &            96.2 &           80.2 &           87.7 &           83.0 &            89.6 &           90.6 &           82.1 \\ 
   &                 All LLMs &           100 &           100 &            100 &           100 &           100 &           100 &            100 &           100 &           100 \\ \midrule \midrule

    AL &                Human B &   26{\tiny±23} &   43{\tiny±32} &    59{\tiny±33} &    45{\tiny±0} &   50{\tiny±53} &   34{\tiny±26} &    46{\tiny±24} &  147{\tiny±83} &   97{\tiny±85} \\
   &                Human C &   17{\tiny±23} & 104{\tiny±122} &    39{\tiny±38} &   56{\tiny±77} & 112{\tiny±102} &   38{\tiny±69} &    26{\tiny±39} &  103{\tiny±88} &   65{\tiny±94} \\
   &                Human F &   26{\tiny±40} &   77{\tiny±75} &    51{\tiny±30} &   30{\tiny±39} &   51{\tiny±57} &   31{\tiny±31} &    27{\tiny±33} &   52{\tiny±67} &   79{\tiny±92} \\ [0.15cm]

  &     GPT-4o (IM 1) &   79{\tiny±10} &   82{\tiny±13} &     72{\tiny±8} &    59{\tiny±7} &   61{\tiny±10} &    53{\tiny±7} &     55{\tiny±9} &   59{\tiny±12} &   67{\tiny±10} \\
   & Gemini-1.5 (IM 1) &   98{\tiny±22} &  126{\tiny±29} &   120{\tiny±33} &   82{\tiny±19} &   91{\tiny±27} &   84{\tiny±22} &    85{\tiny±24} &   90{\tiny±26} &   99{\tiny±49} \\
   &    Llama-3 (IM 1) &   90{\tiny±13} &   91{\tiny±18} &    87{\tiny±14} &   60{\tiny±10} &   64{\tiny±15} &   54{\tiny±12} &    52{\tiny±15} &   56{\tiny±13} &   62{\tiny±13} \\ [0.15cm]
   

   &     GPT-4o (IM 2) &  291{\tiny±44} &  353{\tiny±40} &   332{\tiny±30} &  333{\tiny±37} &  374{\tiny±39} &  362{\tiny±42} &   347{\tiny±36} &  370{\tiny±45} &  357{\tiny±42} \\
   & Gemini-1.5 (IM 2) &  378{\tiny±88} & 446{\tiny±111} &  512{\tiny±106} & 399{\tiny±103} & 425{\tiny±121} & 397{\tiny±109} & 867{\tiny±2032} & 468{\tiny±148} & 400{\tiny±107} \\
   &    Llama-3 (IM 2) &  331{\tiny±35} &  368{\tiny±41} &  438{\tiny±111} & 466{\tiny±197} &  357{\tiny±68} & 351{\tiny±107} &   317{\tiny±88} &  345{\tiny±82} &  355{\tiny±97} \\ [0.15cm]
   

   &     GPT-4o (IM 3) &  295{\tiny±40} &  437{\tiny±40} &   372{\tiny±51} &  380{\tiny±40} &  444{\tiny±37} &  402{\tiny±39} &   358{\tiny±60} &  422{\tiny±50} &  321{\tiny±40} \\
   & Gemini-1.5 (IM 3) &  374{\tiny±86} & 654{\tiny±107} &  689{\tiny±110} &  592{\tiny±94} & 655{\tiny±102} & 559{\tiny±125} &  473{\tiny±109} & 642{\tiny±131} & 505{\tiny±343} \\ 
   &    Llama-3 (IM 3) &  333{\tiny±35} &  425{\tiny±53} &   409{\tiny±49} &  378{\tiny±51} &  481{\tiny±56} &  389{\tiny±51} &   366{\tiny±58} &  445{\tiny±62} &  362{\tiny±44} \\ \midrule \midrule

  PR &                Human B &             75 &            100 &             100 &            100 &            100 &             80 &              84 &            100 &             95 \\
   &                Human C &             19 &             72 &              54 &             50 &             81 &             40 &              29 &             79 &             80 \\
   &                Human F &             47 &             84 &              82 &             44 &             61 &             50 &              49 &             64 &             82 \\ 
   &                 All LLMs &           100 &           100 &            100 &           100 &           100 &           100 &            100 &           100 &           100 \\ \midrule \midrule
   
  AP &                Human B & 1.1{\tiny±1.0} & 2.1{\tiny±1.4} &  2.0{\tiny±1.2} &   1.0{\tiny±0} & 1.3{\tiny±0.6} & 1.0{\tiny±0.7} &  1.2{\tiny±0.9} & 5.3{\tiny±4.0} & 3.5{\tiny±3.0} \\
   &                Human C & 0.2{\tiny±0.6} & 2.1{\tiny±2.3} &  0.9{\tiny±1.1} & 1.1{\tiny±1.3} & 2.1{\tiny±1.8} & 0.9{\tiny±1.6} &  0.4{\tiny±0.6} & 2.5{\tiny±2.2} & 1.9{\tiny±1.8} \\
   &                Human F & 0.7{\tiny±1.0} & 2.4{\tiny±2.0} &  1.4{\tiny±1.0} & 0.8{\tiny±1.0} & 1.2{\tiny±1.4} & 0.8{\tiny±1.0} &  0.7{\tiny±0.9} & 1.4{\tiny±1.8} & 2.3{\tiny±2.2} \\ [0.15cm]
   
   &     GPT-4o (IM 1) & 1.8{\tiny±0.7} & 2.3{\tiny±0.8} &  3.4{\tiny±0.6} & 2.3{\tiny±0.8} & 2.0{\tiny±0.9} & 1.8{\tiny±0.7} &  1.3{\tiny±0.6} & 1.9{\tiny±0.7} & 2.2{\tiny±0.8} \\
   & Gemini-1.5 (IM 1) & 2.1{\tiny±0.8} & 2.6{\tiny±0.9} &  3.3{\tiny±1.0} & 1.9{\tiny±0.7} & 2.1{\tiny±0.8} & 2.5{\tiny±0.8} &  2.2{\tiny±0.7} & 2.4{\tiny±0.8} & 2.6{\tiny±1.5} \\
   &    Llama-3 (IM 1) & 2.2{\tiny±0.5} & 2.4{\tiny±0.6} &  3.1{\tiny±0.9} & 2.0{\tiny±0.7} & 2.3{\tiny±0.8} & 2.1{\tiny±0.6} &  1.5{\tiny±0.7} & 2.0{\tiny±0.7} & 2.3{\tiny±0.5} \\ [0.15cm]

   &     GPT-4o (IM 2) & 3.8{\tiny±0.8} & 4.8{\tiny±1.0} &  5.8{\tiny±1.5} & 4.6{\tiny±1.1} & 5.1{\tiny±0.9} & 5.5{\tiny±1.1} &  5.7{\tiny±1.2} & 5.0{\tiny±0.9} & 4.9{\tiny±1.1} \\
   & Gemini-1.5 (IM 2) & 5.0{\tiny±2.2} & 5.7{\tiny±2.5} &  8.2{\tiny±3.2} & 5.7{\tiny±2.6} & 6.1{\tiny±2.8} & 5.9{\tiny±2.7} &  5.7{\tiny±2.1} & 5.0{\tiny±2.2} & 5.4{\tiny±2.3} \\
   &    Llama-3 (IM 2) & 5.0{\tiny±1.7} & 5.7{\tiny±2.2} &  8.4{\tiny±3.0} & 8.1{\tiny±3.8} & 6.7{\tiny±2.7} & 6.9{\tiny±2.9} &  6.2{\tiny±1.9} & 6.1{\tiny±2.2} & 6.6{\tiny±2.4} \\ [0.15cm]
   
   &     GPT-4o (IM 3) & 3.9{\tiny±0.7} & 6.5{\tiny±1.7} &  8.5{\tiny±2.2} & 5.6{\tiny±1.1} & 7.7{\tiny±1.4} & 5.8{\tiny±1.0} &  5.1{\tiny±1.2} & 6.7{\tiny±1.8} & 5.2{\tiny±1.3} \\
   & Gemini-1.5 (IM 3) & 4.9{\tiny±2.2} & 9.0{\tiny±2.9} & 10.7{\tiny±3.4} & 9.3{\tiny±2.6} & 9.2{\tiny±2.8} & 8.2{\tiny±2.9} &  6.3{\tiny±1.6} & 9.1{\tiny±3.2} & 7.3{\tiny±3.4} \\
   &    Llama-3 (IM 3) & 5.1{\tiny±1.6} & 6.7{\tiny±2.2} &  8.9{\tiny±2.1} & 6.1{\tiny±2.1} & 7.6{\tiny±1.9} & 5.5{\tiny±1.8} &  5.4{\tiny±1.7} & 6.5{\tiny±2.0} & 5.9{\tiny±2.0} \\
\bottomrule
\end{tabular}

    \caption{General statistics of feedback comments generated by human and LLM assessors. CR (\%): comment rate, i.e., the percentage of time a comment is provided. AL: \textit{average length} (measured in tokens) of the provided comments (excluding cases where comments are not given), along with their respective standard deviations. PR (\%): problem rate, i.e., the percentage of time a problem is mentioned or implied in the \textit{provided comments}. AP: \textit{average number of problems} identified in the provided comments, along with their respective standard deviations. ``All LLMs'' means all three LLMs across the three interaction modes. C1: Material selection. C2: Material integration and citation; C3: Quality of key components. C4: Logic of structure. C5: Content and clarity of ideas. C6: Coherence (flow of ideas). C7: Cohesion (use of connectors). C8: Grammar and sentence structure. C9: Academic vocabulary.}
    \label{tab:generalStatsFeedbackComments}
\end{table*}




Table~\ref{tab:generalStatsFeedbackComments} presents the general statistics of feedback comments generated by human assessors and LLMs under the three interaction modes.



\subsection{Score-Comment Interaction}

Fig.~\ref{fig:score-comment-interaction} provides the full results of the correlations measured between scores and the token counts of or the numbers of identified problems in the related comments.


\begin{figure*}
    \centering
    \includegraphics[width=1\linewidth]{figures/score-comment-interaction.png}
    \caption{Heatmaps showing score-comment correlations between scores and the length of the related comments (left) and between scores and the number of problems identified in the related comments (right). Darker blue shades indicate a stronger negative correlation and darker orange shades a stronger positive correlation, with gray-ish colors indicating negligible correlations. To ensure meaningful analysis, correlations are calculated only when at least 10 score-comment pairs are available. C1: Material selection. C2: Material integration and citation; C3: Quality of key components. C4: Logic of structure. C5: Content and clarity of ideas. C6: Coherence (flow of ideas). C7: Cohesion (use of connectors). C8: Grammar and sentence structure. C9: Academic vocabulary.}
    \label{fig:score-comment-interaction}
\end{figure*}


\section{Further Analyses\label{app:furtherAnalyses}}

Table~\ref{tab:commentPairs} provides five random example comment pairs sampled from GPT-4o-Aug and GPT-4o-May prompted under default prompt setting specified in Section~\ref{sec:prompting}. We find that when BERTScore is low (the last row), the comment pair is less similar compared to other pairs. While other two metrics (BLEU and ROUGE-L) are highly correlated with BERTScore (BLUE: 0.78, ROUGE-L: 0.88, Pearson), they consistently yield lower values than BERTScore. This indicates that these two lexical overlap-based metrics may be less effective at measuring comment reliability compared to the semantic similarity captured by BERTScore.


\begin{table*}[]
    \centering
    \scriptsize

\begin{tabular}{p{1cm}p{5cm}p{5cm}p{1cm}p{1.25cm}p{1cm}}
\toprule
                   Criterion &   GPT-4o-Aug &  GPT-4o-May &  BLEU &  ROUGE-L &  BERTScore \\
\midrule
  Content \& clarity of ideas &                                                    The content is generally relevant, but the clarity of ideas could be improved. Some sentences are complex and could be simplified for better understanding. For example, the sentence discussing Olson (2012) and Sanner (2008) could be broken down into shorter, more concise statements. Additionally, the review could benefit from more explicit explanations of key concepts, such as acculturation. &  The content is relevant and covers important aspects of the topic. However, some ideas are not clearly articulated. For example, the sentence "This is made worse because 'the nursing profession faces increased numbers of linguistically and culturally diverse patients who speak little or no English'" could be rephrased for clarity. Additionally, some sentences are overly complex and could be simplified for better understanding.&  0.20 &     0.30 &       0.63 \\ \midrule
          Logic of structure &                                                                                                                                                    The logical structure of the literature review could be improved. The review jumps between different topics without clear transitions, making it difficult for the reader to follow the argument. A more organized structure with clear subheadings and a logical flow of ideas would enhance readability. &                                                                                                                                                                                                                                                                                                                                                                                                         The logical structure of the literature review needs improvement. While the review covers important aspects of the psychological impact of COVID-19, the organization of the content is somewhat disjointed. Clearer subheadings and a more logical flow of ideas would help guide the reader through the review. For example, separating the discussion of different demographic groups or types of psychological impacts into distinct sections could enhance readability. &  0.17 &     0.41 &       0.70 \\ \midrule
          
   Quality of key components & The introduction provides a general overview of the topic but lacks a clear statement of the significance of the issue or specific research questions. The body of the literature review presents some relevant ideas but lacks a clear organization and focus on the pros and cons of online learning. The conclusion is missing, which is a significant omission as it should summarize key findings and highlight contributions or gaps in the literature. &  The introduction lacks a clear thesis statement and does not adequately set up the significance of the topic. The body of the review presents relevant ideas but lacks a clear structure and flow. The conclusion is weak and does not effectively summarize the key points or identify research trends or gaps. A more structured approach with clear sections and subheadings would improve readability and coherence. &  0.22 &     0.44 &       0.73 \\ \midrule
Grammar \& sentence structure &                                                                                                                                         The draft contains several grammatical errors and awkward sentence structures. For example, "The among of time is short considering the kind of assignments..." should be "The amount of time is short considering the type of assignments...". Varying sentence length and structure could also improve readability. &     The draft has several grammatical errors and issues with sentence structure. For example, "The among of time is short considering the kind of assignments" should be "The amount of time is short considering the type of assignments." Additionally, varying sentence length and structure would make the writing more engaging. Proofreading for grammatical accuracy and sentence variety is recommended. &  0.50 &     0.76 &       0.85 \\ \midrule
Grammar \& sentence structure &                                                                                                                                                                The grammatical accuracy is generally acceptable, but there are instances of awkward phrasing and run-on sentences. Varying sentence length and structure could improve readability. Additionally, ensuring subject-verb agreement and correct punctuation would enhance grammatical accuracy. & - Positive: The grammar is generally accurate.\newline- Improvement: Sentence length and variety need improvement. Some sentences are too long and complex, making them difficult to read.\newline- Example: "The health arguments against the use of cannabis include its addictive nature (Hurd et al., 2014) . It has also been directly linked to a range of adverse outcomes in physical health, which include lung cancer (Aldington et al., 2008), impaired respiratory function, cardiovascular disease, elevated systolic blood pressure, stroke (Singh et al., 2012), mental disorders (Saban et al., 2014), which include schizophrenia, especially amongst young people (Casadio et al., 2011), undesirable cognitive changes (Crean et al., 2011)." This could be broken down into shorter sentences. &  0.00 &     0.12 &       0.49 \\ 
\bottomrule
\end{tabular}

    \caption{Five random example comment pairs with their BLEU, ROUGE-L, and BERTScore scores provided. }
    \label{tab:commentPairs}
\end{table*}





\section{Prompts\label{app:prompts}}

Note that, any word followed by a dollar sign ``\$'' is a placeholder for all prompt templates included in this section. For example, ``\$comment'' is a placeholder for a comment. 


\subsection{Prompts for the Feedback Comment Quality Evaluation Framework Pipeline\label{app:promptFramework}}


The full prompt templates for the three steps in the pipeline of the feedback comment quality evaluation framework are given below. Among these three prompts, the prompt for Problem Extraction contains three in-context exemplars, whereas the prompts for the other two steps are zero-shot prompts. 

\subsubsection{Prompt for Problem Extraction}

\begin{quote}

\footnotesize

You will be given a feedback comment written for a student's essay. Your task is to identify and extract all the writing-related problems mentioned or implied in the comment, along with any explanations, suggestions, corrections, questions, quotations, or other relevant information provided in the comment for each extracted problem. \newline

A writing-related problem is any issue that affects the quality of the writing, such as citation errors,
logical flaws, coherence issues, grammatical mistakes, or inappropriate word choices, among others. \newline

\#\#\# Extraction Instructions\newline

- Each extracted problem must be clear and can be understood without the need to refer to the original comment. \newline

- Each extracted problem must faithfully reflect the provided comment by including any relevant information. Relevant information includes a further explanation or an elaboration of the problem, a suggestion for improvement, a concrete correction, a clarifying question, an excerpt (possibly without quotation marks) from the student's essay, or any other relevant information that helps to understand the problem. \newline

- Whenever possible, extract each problem and the relevant information as they are written in the comment. \newline

\#\#\# Output Instructions\newline

- Output each extracted problem along with their relevant information line by line headed by ``-''.
- Output ``None'' if no writing-related problems are mentioned or implied in the comment.\newline

\#\#\# Examples\newline

Example 1 input:\newline

The content is generally informative and relevant, but the clarity of ideas could be improved. Some sentences are overly complex and could be simplified for better understanding. For instance, the sentence ``Gandhi's Satyagraha as an adequate substitute for violent methods of conducting social conflict in an early and thorough philosophical examination of Gandhi's attitude to violence in extreme group conflict'' is difficult to parse and could be rephrased for clarity.\newline

Example 1 output: \newline

- The clarity of ideas could be improved. Some sentences are overly complex and could be simplified for better understanding. For instance, the sentence ``Gandhi's Satyagraha as an adequate substitute for violent methods of conducting social conflict in an early and thorough philosophical examination of Gandhi's attitude to violence in extreme group conflict'' is difficult to parse and could be rephrased for clarity.\newline

Example 2 input:\newline

The content and clarity of ideas are generally good, but there are some areas where the author could provide more depth or analysis. For example, the author could have explored the potential reasons why students in India may be more vulnerable to substance abuse, or discussed the implications of legalization for public health policy. To improve, the author could revisit the body of the literature review and provide more nuanced analysis of the findings.\newline

Example 2 output:\newline

- There are some areas where the author could provide more depth or analysis. For example, the author could have explored the potential reasons why students in India may be more vulnerable to substance abuse, or discussed the implications of legalization for public health policy. To improve, the author could revisit the body of the literature review and provide more nuanced analysis of the findings.\newline

Example 3 input:\newline

The author has generally done a good job of integrating source materials and presenting information clearly. However, there are some instances where the connections between ideas could be more explicitly stated, and the citation practices could be more consistent (e.g., some sources are cited with author names, while others are cited with only the year).\newline

Example 3 output:\newline

- There are some instances where the connections between ideas could be more explicitly stated.\newline
- The citation practices could be more consistent (e.g., some sources are cited with author names, while others are cited with only the year).\newline

\#\#\# Input\newline

\$comment\newline

\#\#\# Output

\end{quote}


\subsubsection{Prompt for Problem Classification}


\begin{quote}
\footnotesize

You will be given an excerpt of a feedback comment written for a student's essay. Your task is to answer the following questions: \newline

1. Does the excerpt refer to a specific part of the essay? A specific part refers to a part of the essay that can be easily located by the student.
For example, it can be a specific word, phrase, sentence, paragraph, reference etc. used in the essay. It can be a concrete location, such as ``sentence 2 in paragraph 2,'' ``in paragraph 6,'' ``the first citation,'' or ``the first sentence of the paper'' and so on. A less concrete location, such as ``the introduction,'' or ``the conclusion,'' is also considered a specific part if it is accompanied by some referenceable details, such as ``The significance of South Australian policy is unclear, as it is the first citation and the only one in the Introduction.'' Note that the excerpt may only contain a quoted text from the essay, in which case, the quoted text is considered a specific part.\newline

2. Does the excerpt offer some form of suggestions, general or specific, for the student to improve the essay? If the excerpt only describes a problem and it is unclear what the student should do to fix it, then there is no suggestion. If the excerpt provides a concrete correction, it is considered a suggestion.\newline

3. Does the excerpt provide a concrete correction for the student to apply? Note that when the excerpt only contains a quoted text from the essay and there are some notes indicating a correction (e.g., adding/removing a punctuation, correcting a spelling), this is considered a correction.\newline

Answer each question with ``Yes'' or ``No'' based on the content of the excerpt and briefly justify your answer. After answering all the questions, 
produce your final answers in a newline separated by commas.

Excerpt: \$excerpt

\end{quote}


\subsubsection{Prompt for Correction Relevancy Check}


\begin{quote}

\footnotesize

You will be given an excerpt of a feedback comment written for a student's essay according to an assessment question. Your task is to answer the following questions: \newline

1. Does the problem pointed out in the excerpt exist in the corresponding essay? If the excerpt uses a quoted text to point out a problem, check if the quoted text is present in the essay. Please note that the quoted text may not be an exact match either due to misspellings, capitalization errors etc., or because the quoted already contains the correction in place. \newline

2. Is the problem pointed out in the excerpt relevant to the corresponding assessment question? Check if the excerpt is broadly related to any aspect of the assessment question. \newline

3. Is the correction of the problem pointed out in the excerpt correct? If the problem does exist in the essay, check if the correction fixes the problem or presents a plausible solution or improvement. \newline

Here is the essay: \newline

\$essay \newline

Here is the assessment question:\newline

\$question \newline

Here is the excerpt:\newline

\$excerpt\newline

Answer each question with ``Yes'' or ``No'' utilizing all the information provided and briefly justify your answer. After answering all the questions, produce your final answers in a newline separated by commas.
    
\end{quote}




\subsection{Prompts for the Main Experiments\label{app:mainPrompts}}


Our prompts consist of three parts: (1) a system prompt part that provides general background information and specifies the writing topic and some general assessment guidance; (2) a writing part that includes an entire literature review (with references); (3) an assessment instruction part, where one or multiple assessment questions (see Table~\ref{tab:assessmentCriteria}) are asked in various manners according to the interaction modes.  

We keep the system prompt fixed across the three interaction modes. For the main experiments, the system prompt is as follows:

\begin{quote}
\footnotesize

You are an expert academic writing instructor specializing in graduate-level work, with particular experience supporting students who speak English as an additional language. You have been asked to evaluate a literature review submitted by a graduate student on the following topic: \$Topic. The review was written in 2021, so references after this year are not expected.

When assessing the student's writing, please strictly follow the instruction provided to you and make sure your score/feedback is carefully considered and constructive. Please provide your comments and/or suggestions with as much detail and specificity as possible. Please provide specific examples of sentences, paragraphs or sections that you think could use improvement. If you write comments, please start them with something positive. Please proceed with things that could be improved, would make things clearer for the reader, would make the text flow better, etc.

\end{quote}


For the writing part, we explicitly mark the beginning and end of the writing for clarity: 

\begin{quote}
\footnotesize

\#\#\#\#\#\#\#\#\#\# Writing starts \#\#\#\#\#\#\#\#\#\#

\$writing

\#\#\#\#\#\#\#\#\#\# Writing ends \#\#\#\#\#\#\#\#\#\#
\end{quote}

The specifics of how the assessment instruction part is constructed are detailed below. 



\subsubsection{Interaction Mode 1}

In Interaction Mode 1, all assessment questions (see Table\ref{tab:assessmentCriteria}) are asked at once:

\begin{quote}
\footnotesize

Q1: \{Assessment question 1\}

Q2: \{Assessment question 2\}

...

Q9: \{Assessment question 9\}

\end{quote}


After these assessment questions is an answer instruction: 

\begin{quote}
\footnotesize

For each of the 9 questions above, provide your comments or suggestions if any, followed by your score out of 10. Please indicate which question you are providing feedback for by starting your response with `A1:', `A2:', etc. Each response should use the following format:

Score: ...

Comments or suggestions: ...

\end{quote}


Note that we use ``if any'' to denote the optionality of the comments and suggestions. We tried putting ``(Optional)'' after ``Comments or suggestions,'' but that does not make a difference. 



\subsubsection{Interaction Mode 2}

In Interaction Mode 2, the assessment questions are presented sequentially and one at a time. Below is the basic structure:

\begin{quote}
\footnotesize

Q$_i$: \{The $i$th assessment question.\}

\{Answer instruction\}

A$_i$:
    
\end{quote}


The answer instruction resembles the one used in the Interaction Mode 1. 

\begin{quote}
\footnotesize

Provide your score out of 10, followed by comments or suggestions if any. Your response should use the following format:

Score: ...

Comments or suggestions: ...
    
\end{quote}


Note that, we append LLM's response to the $i$th assessment question to the original prompt to form a new prompt, to which the next assessment question is added. This way, the writing is only provided once (at the beginning), but the LLM will have access to previous assessment questions as well as its answers to those questions.


\subsubsection{Interaction Mode 3}

In Interaction Mode 3, each assessment question is asked independently, so there are 9 separate prompts for each essay. 

The structure for the assessment part of the prompt is similar to that in Interaction Mode 2, but without indexation and prefix ``Q/A'':

\begin{quote}
\footnotesize

\{An assessment question.\}

\{Answer instruction\}
    
\end{quote}

The answer instruction works exactly the same as in Interaction Mode 2. 


\subsection{Prompts for the Follow-Up Experiments\label{app:followUpPrompts}}

\subsubsection{System Prompt Simplification}

Below is a simplified system prompt removing the helpful information from the default system prompt used in Section~\ref{sec:experiments}.

\begin{quote}
    \footnotesize
You are an expert academic writing instructor for graduate students. You have been asked to evaluate a literature review submitted by a student below. The writing is broadly related to the following topic: \$Topic. 

When assessing the student's writing, please strictly follow the instruction provided to you and make sure your score/feedback is carefully considered and constructive.
\end{quote}


\subsection{Prompts for Assessing Specificity and Helpfulness}


\begin{quote}
    \footnotesize

You will be given a feedback comment written for a student’s essay according to an assessment question. Your task is to rate the feedback comment on (1) specificity and (2) helpfulness, using a scale from 1 to 10, where 1 is the lowest and 10 is the highest. Conclude your response with the final ratings in this format: "Specificity: X, Helpfulness: X" (where X is a score from 1 to 10).

Here is the essay:

\$essay

Here is the assessment question:

\$question

Here is the feedback comment:

\$feedback

Please rate the specificity and helpfulness of the feedback comment.

\end{quote}