Our analysis assumes the Nash solver runtime is represented by~$T$, which may or may not depend upon the size of the game tree.
Our algorithm combines the solver of choice with dynamic programming to solve for the SPE, working on each subgame (minus the information sets of the subgame already included in the partial SPE) for time $T$. If a solution is found for a subgame $G_{\theta_1}$, and the algorithm moves on to compute a solution for the subgame~$G_{\theta_2}$, where $G_{\theta_1} \subset G_{\theta_2}$, $G_{\theta_1}$ is not traversed again when finding the next solution.
Thus, if we assume that each part of the tree is traversed at most $T$ times, the runtime of the algorithm is also $O(T) \cdot O(\abs{H})$.
Alternatively, if we consider that in the worst-case, the number of actions in an information set is $A$ and the total number of information sets in $G$ is $\abs{\mathcal{I}}$, the runtime of the algorithm is $O \left(T A \cdot \abs{\mathcal{I}} \right)$, which is much tighter than that of GBI.

Our method for finding SPE applies to extensive-form games of imperfect information, as we demonstrate in the following lemma.

\begin{lemma}
\textsc{ComputeSPE} can find the SPE of any game $G$ of imperfect information.
\end{lemma}
\begin{proof}
There are two cases that arise when $G$ has imperfect information. In the first case, $G$ has no subgames besides itself. The subroutine \textsc{GetInitialSPE} within \textsc{ComputeSPE} is called on $G$ itself, as the height of the root $h_0$ in $\Psi$ must be 1. Since \textsc{GetInitialSPE} solves a given subgame using the black-box Nash solver, \textsc{ComputeSPE} returns the resulting NE, which is therefore the SPE.

In the second case, $G$ contains nontrivial subgames. \textsc{ComputeSPE} begins by first solving each of the subgames at height $k=1$ in the tree with the black-box Nash solver. The NE returned for each of these games must by definition be the SPE for each of these games. Consider this the base case for proof by induction. Then, the solution $\left.\bm{\sigma}^{SPE}\right|_{G_{\theta}}$ for any subgame $G_{\theta}$ at height $k$ is fixed, and the solver is applied to the subgame $G_{\theta'}$ at height $k + 1$ that contains it so as to find the optimal strategy $\left.\bm{\sigma}^{k+1}\right|_{G_{\theta'}}$ for that larger subgame (without overwriting the solutions for subgames at lower heights). $\left.\bm{\sigma}\right|_{G_{\theta'}}$ will consist of $\left.\bm{\sigma}^{SPE}\right|_{G_{\theta}}$ and the optimal joint strategy profile for the information sets that comprise the rest of $G_{\theta'}$ found via the Nash solver. This optimal profile is the SPE for that particular subgame. Since this continues for all subgames leading up to $h_0$, it follows by induction that the solution is ultimately the union of all SPE for the subgames of $G$, which by definition is the SPE. 
\end{proof}