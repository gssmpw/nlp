\section{RELATED WORK}
Image aesthetic and composition evaluation is an open research subject with ongoing applications and a history of interest in the field. Older methods that use saliency and rule of thirds based techniques \cite{oldphotocomp} used limited datasets and were implemented using algorithms. Other methods in related areas aim to recompose or improve images \cite{oldqualityassess}. While these methods have been shown to work, they have limited generalization and often make assumptions about inital photo composition. Later the AVA dataset \cite{AVA} was collected and labeled: it provided a large dataset to the task and is now the most common benchmark for the photo aesthetic analysis task.

With the introduction of a large dataset, it opened the gateway for use of neural networks.  Methods and papers like  DMA-Net \cite{DMA-Net} utilize convolutional neural networks to estimate aesthetic rankings of the AVA dataset. Further work lead to papers like NIMA \cite{NIMA} where improvements to the loss functions were made to better represent diversity in aesthetic rankings, as well as increasing accuracy and correlation scores. 

State of the art methods like MP\_adam \cite{MPADAM} and A-Lamp \cite{A-LAMP} make use of patches, attention mechanisms, and significant hardware to focus in on specific aesthetic features attacking accuracy metrics. Pool-3FC \cite{POOL3FC}  pools features in a comparable way to also squeeze the most amount of information out of the source data.