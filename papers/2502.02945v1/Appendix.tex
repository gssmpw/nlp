\appendix


\begin{table*}[hbt!]
\centering
\small
\begin{tabular}{| >{\centering\arraybackslash}m{0.1\textwidth} | >{\raggedright\arraybackslash}m{0.8\textwidth} |}
\hline
\rowcolor{lightgray} % 设置第一行背景颜色
\textbf{Type} & \multicolumn{1}{c|}{\textbf{Template}} \\
\hline
1 & The student has previously, in chronological order, answered question with ID=74 [WrapQEmb]
involving concept ID=6 [WrapCEmb] correctly, ...,
question with ID=42 [WrapQEmb] involving concept ID=5 [WrapCEmb] incorrectly.
Please predict whether the student will answer the next question with ID=44 [NextWrapQEmb]
involving concept ID=5 [NextWrapCEmb] correctly. 
Response with ‘Yes’ or ‘No’. \\
\hline
2 & The student has previously, in chronological order, answered question with ID=3117 [QidEmb]
correctly, question with ID=2964 [QidEmb] correctly, 
question with ID=5627 [QidEmb] incorrectly, ...,
question with ID=5532 [QidEmb] correctly.
Please predict whether the student will answer the next question with ID=5707 [NextQidEmb] correctly.
Response with ‘Yes’ or ‘No’. \\
\hline
3 & The student has previously, in chronological order, answered question involving concept ID=15 [CidEmb]
correctly, ...,
question involving concept ID=30 [NextCidEmb] correctly.
Please predict whether the student will answer the next question involving concept ID=30 correctly.
Response with ‘Yes’ or ‘No’. \\
\hline
\end{tabular}
\caption{The prompt templates for LLM-KT(Ours)}
\label{table:llm-kt-templates}
\end{table*}

\begin{table*}[hbt!]
\centering
\small
\begin{tabular}{| >{\centering\arraybackslash}m{0.1\textwidth} | >{\raggedright\arraybackslash}m{0.8\textwidth} |}
\hline
\rowcolor{lightgray} % 设置第一行背景颜色
\textbf{Type} & \multicolumn{1}{c|}{\textbf{Template}} \\
\hline
1 & The student has previously, in chronological order, answered question with ID=74
involving concept ID=6 correctly, question with ID=80 involving concept ID=6 correctly,
...,
question with ID=42 involving concept ID=5 incorrectly.
Please predict whether the student will answer the next question with ID=44
involving concept ID=5 correctly.
Response with ‘Yes’ or ‘No’. \\
\hline
2 & The student has previously, in chronological order, answered question with ID=3117
correctly, question with ID=2964 correctly,
...,
question with ID=5532 correctly.
Please predict whether the student will answer the next question with ID=5707 correctly.
Response with ‘Yes’ or ‘No’. \\
\hline
3 & The student has previously, in chronological order, answered question involving concept ID=15
correctly, question involving concept ID=15 correctly,
...,
question involving concept ID=30 correctly.
Please predict whether the student will answer the next question involving concept ID=30 correctly.
Response with ‘Yes’ or ‘No’. \\
\hline
\end{tabular}
\caption{The prompt templates for LLM-FT$_\mathrm{ID}$}
\label{table:llm-kt-id-templates}
\end{table*}

\begin{table*}[hbt!]
\centering
\small
\begin{tabular}{| >{\centering\arraybackslash}m{0.1\textwidth} | >{\raggedright\arraybackslash}m{0.8\textwidth} |}
\hline
\rowcolor{lightgray} % 设置第一行背景颜色
\textbf{Type} & \multicolumn{1}{c|}{\textbf{Template}} \\
\hline
1 & The student has previously, in chronological order, answered question with ID=[qid$_{74}$] involving concept ID=[cid$_{6}$] correctly, question with ID=[qid$_{80}$] involving concept ID=[cid$_{6}$] correctly, ..., question with ID=[qid$_{42}$] involving concept ID=[cid$_{5}$] incorrectly. Please predict whether the student will answer the next question with ID=[qid$_{44}$] involving concept ID=[cid$_{5}$] correctly. Response with `Yes' or `No'. \\
\hline
2 & The student has previously, in chronological order, answered question with ID=[qid$_{3117}$] correctly, question with ID=[qid$_{2964}$] correctly, question with ID=[qid$_{5627}$] incorrectly, ..., question with ID=[qid$_{5532}$] correctly. Please predict whether the student will answer the next question with ID=[qid$_{5707}$] correctly. Response with `Yes' or `No'. \\
\hline
3 & The student has previously, in chronological order, answered question involving concept ID=[cid$_{15}$] correctly, question involving concept ID=[cid$_{15}$] correctly, question involving concept ID=[cid$_{30}$] incorrectly, ..., question involving concept ID=[cid$_{30}$] correctly. Please predict whether the student will answer the next question involving concept ID=[cid$_{30}$] correctly. Response with `Yes' or `No'. \\
\hline
\end{tabular}
\caption{The prompt templates for LLM-FT$_\mathrm{TokenID}$}
\label{table:llm-kt-tokenid-templates}
\end{table*}

\begin{table*}[hbt!]
\centering
\small
\begin{tabular}{| >{\centering\arraybackslash}m{0.1\textwidth} | >{\raggedright\arraybackslash}m{0.8\textwidth} |}
\hline
\rowcolor{lightgray} % 设置第一行背景颜色
\textbf{Type} & \multicolumn{1}{c|}{\textbf{Template}} \\
\hline
4 & In this task, we aim to determine whether the student can answer the question correctly based on the student's history record of academic exercises. \newline
    The student's history record of academic exercises is given as follows: \newline
    1) How would this calculation be written? Pic$_{290-0}$ \newline
    % A:8+(2÷5)=2; \\ B:(8+2)÷5=2; \\ C:8+2÷5=2; \\ D:(8+2÷5)=2 \\
    A:8+(2÷5)=2 B:(8+2)÷5=2 C:8+2÷5=2 D:(8+2÷5)=2 \newline
    Related knowledge concepts: Basic Arithmetic \newline
    The student answered this question correctly \newline
    2) Which symbol belongs in the box? Pic$_{749-0}$ \newline
    % A:$>$; \\ B:$<$; \\ C:$=$; \\ D:$\ge$ \\
    A:$>$  B:$<$  C:$=$ D:$\ge$ \newline
    Related knowledge concepts: Basic Arithmetic \newline
    The student answered this question correctly \newline
    % ..., \newline
    3) What is the output of this Function Machine? Pic$_{836-0}$ \newline
    % A:10p; \\ B:7p; \\ C:5(p+2); \\ D:5p+2 \\
    A:10p B:7p  C:5(p+2)  D:5p+2 \newline
    Related knowledge concepts: Writing Expressions \newline
    The student answered this question incorrectly \newline
    The target question is given as follows: \newline
    Tom and Katie are arguing about the result of this Function Machine: Pic$_{856-0}$. Tom says the output is: 3n-12. Katie says the output is:3(n-4). Who is correct? \newline
    % A:Only Tom; \\ B:Only Katie; \\ C:Both Tom and Katie; \\ D:Neither is correct \\
    A:Only Tom B:Only Katie C:Both Tom and Katie  D:Neither is correct \newline
    Related knowledge concepts: Writing Expressions \newline
    Please predict whether the student would answer the target question correctly. Response with `Yes' or `No'. \\
\hline
5 & The student has previously, in chronological order, answered question involving concept ``Basic Arithmetic" correctly, question involving concept ``Basic Arithmetic" correctly, ..., question involving concept ``Basic Arithmetic" incorrectly, question involving concept ``Basic Arithmetic" correctly, ..., question involving concept ``Ordering Negative Numbers" incorrectly, question involving concept ``Ordering Negative Numbers" correctly. Please predict whether the student will answer the next question involving concept ``Ordering Negative Numbers" correctly. Response with `Yes' or `No'. \\
\hline
\end{tabular}
\caption{The prompt templates for LLM-FT$_\mathrm{Text}$}
\label{table:llm-kt-text-templates}
\end{table*}



\section{Prompt Templates}

In this section, we provide detailed descriptions of the prompt templates used for different datasets in our study. These templates are designed to handle various types of data and adapt to the specific requirements of each dataset. 

We introduce five distinct prompt templates:
\begin{itemize}
    \item \textbf{Type 1} (Combined Question and Concept ID Template): Used for datasets with both QIDs and CIDs, applicable to Assist2009 and Nips2020.
    \item \textbf{Type 2} (Question ID-Only Template): Used exclusively for datasets with only QIDs, such as Junyi.
    \item \textbf{Type 3} (Concept ID-Only Template): Used exclusively for datasets with only CIDs, like Assist2015.
    \item \textbf{Type 4} (Contextual Question Template): Used for datasets with text associated with questions, applicable only to Nips2020.
    \item \textbf{Type 5} (Contextual Concept Template): Used for datasets with concept text, like Assist2009.
\end{itemize}


\section{Terminology Explanation}

% \begin{description}
\begin{itemize}
    \item \textbf{QID} (Question ID): The unique identifier for each question, used to track and model the sequence of a student’s answers. 
    \item \textbf{CID} (Concept ID): The unique identifier for the knowledge concept tied to each question.
    \item \textbf{WrapQEmb} (Wrapped Question Embedding): The embedding formed by combining the QID and the question’s text, leveraging both identity and semantic content.
    \item \textbf{WrapCEmb} (Wrapped Concept Embedding): Similar to ‘WrapQEmb’, but combines the CID with the concept’s text.
    \item \textbf{QidEmb} (Question ID Embedding): An embedding of the QID, used without the question’s text in templates focused on identity.
    \item \textbf{CidEmb} (Concept ID Embedding): An embedding of the CID, used without the concept’s text in simpler templates.
    \item \textbf{NextWrapQEmb} (Next Wrapped Question Embedding): The fused embedding for the next QID, combining its ID and text similar to ‘WrapQEmb’.
    \item \textbf{NextQidEmb} (Next Question ID Embedding): The next QID’s embedding, used without the question’s text.
    \item \textbf{NextWrapCEmb} (Next Wrapped Concept Embedding): The fused embedding for the next CID, combining its ID and text.
    \item \textbf{NextCidEmb} (Next Concept ID Embedding): The next CID’s embedding, used without the concept’s text.
% \end{description}
\end{itemize}

