\section{Related work}
Several results concerning the objective of minimizing the total weighted completion time on parallel machines serve as key cornerstones in the field of scheduling theory. As mentioned above, one of the seminal results in the field is the $O(n \log n)$ time algorithm for the single machine $1||\sum w_jC_j$~\cite{Smith1956}. Another classical variant of the problem that can be solved in $O(n \log n)$ time is $P|| \sum C_j$ where jobs have unit-weight and the scheduling is done on an arbitrary number of identical parallel machines~\cite{ConwayMM1967} ($P$ replaces $Pm$ in the first field of the three field notation to indicate that the number of machines is not fixed). Nevertheless, other special cases are much harder to solve. For example, when $p_j=w_j$ for $j\in\{1,\ldots,n\}$ the corresponding $Pm||\sum p_j C_j$ is NP-hard for any fixed $m \geq 2$~\cite{BrunoCoffman}. Moreover, when $m$ is arbitrary, the corresponding $P||\sum w_j C_j$ becomes strongly NP-hard (see problem SS13 in Garey and Johnson~\cite{GareyJohnson}). The fact that $Pm||\sum w_j C_j$ is only weakly NP-hard is due to Lawler and Moore's pseudo-polynomial time algorithm mentioned above~\cite{LawlerMoore}. Jansen and Kahler~\cite{Jansenee} showed that $P2||\sum w_j C_j$ cannot be solved in $\tilde{O}(P^{1-\varepsilon})$ time for any $\varepsilon > 0$ assuming SETH.%, using the fact that the same bound was already known for the \textsc{Partition} problem~\cite{AbboudBHS17}.

A few variants of $Pm||\sum w_jC_j$ have been studied from the perspective of parameterized complexity. Mnich and Wiese~\cite{MnichW15} studied a variant where rejection is allowed, and the accepted jobs are scheduled on a single machine. They presented FPT and W[1]-hardness results regarding various parameters of the problem.
%of $P||\sum w_j C_j$ where rejections are allowed, and presented FPT and W[1]-hardness results regarding various parameters of the problem. 
Knop and Kouteck{\'{y}}~\cite{DBLP:journals/scheduling/KnopK18} showed that $P||\sum w_j C_j$ is fixed parametrized tractable (FPT) for parameter $m+k$, where $k$ is the number of distinct efficiencies in the input set of jobs. This is done by formulating the problem as an $n$ fold integer programming, and using Hemmecke \emph{et al}~\cite{DBLP:journals/mp/HemmeckeOR13} algorithm to solve the later problem. Complimenting this result, Knop and Kouteck{\'{y}} also showed that the problem is W[1]-hard when only $m$ is taken as a parameter.% (when $k$ is taken as a parameter the problem is para-NP-hard, as the case of $k=1$ includes \textsc{Partition} as a special case). %The fact that $Pm||\sum w_j C_j$ is NP-hard only in the ordinary sense is due to the fact that the problem admits a pseudo-polynomial $O(P^{m-1}\cdot n)$ time algorithm~\cite{LawlerMoore}. Jansen and Kahler~\cite{Jansenee} showed that $P2||\sum w_j C_j$ cannot be solved in $\tilde{O}(P^{1-\varepsilon})$ time for any $\varepsilon > 0$ assuming SETH, using the fact that the same bound was already known for the \textsc{Partition} problem~\cite{AbboudBHS17}.


%Complimenting these two results, Knop and Kouteck{\'{y}}~\cite{DBLP:journals/scheduling/KnopK18} proved that the $P||\sum w_j C_j$ is W[1]-hard with respect to $m$. They also showed that the problem is fixed parametrized tractable (FPT) if we combine $m$ with the number of different efficiencies as parameters. This is done by formulating the problem as an $n$ fold integer programming, and using Hemmecke \emph{et al}~\cite{DBLP:journals/mp/HemmeckeOR13} algorithm to solve the later problem. The fact that $Pm||\sum w_j C_j$ is NP-hard only in the ordinary sense is due to the fact that the problem admits a pseudo-polynomial $O(P^{m-1}\cdot n)$ time algorithm~\cite{LawlerMoore}. Jansen and Kahler~\cite{Jansenee} showed that $P2||\sum w_j C_j$ cannot be solved in $\tilde{O}(P^{1-\varepsilon})$ time for any $\varepsilon > 0$ assuming SETH, using the fact that the same bound was already known for the \textsc{Partition} problem~\cite{AbboudBHS17}.


%Nevertheless, when the number of machines is arbitrary, the corresponding $P||\sum w_j C_j$ is strongly NP-hard (see problem SS13 in Garey and Johnson~\cite{GareyJohnson}).

%Jansen and Kahler~\cite{Jansenee} showed that $P2||\sum w_j C_j$ cannot be solved in $\tilde{O}(P^{1-\varepsilon})$ time for any $\varepsilon > 0$ assuming SETH, using the fact that the same bound was already known for the \textsc{Partition} problem~\cite{AbboudBHS17}. To date, these are the two best bounds for the time complexity required to solve $Pm||\sum w_j C_j$ problems in pseudo-polynomial time.

%The $P|| \sum C_j$ problem, a variant of $Pm|| \sum w_jC_j$ where all jobs have uniform weights and an arbitrary number of identical machines are available, is known to be solvable in polynomial time~\cite{ConwayMM1967}. 

Despite limited progress in exact algorithms for minimizing total weighted completion times on parallel machines, substantial advances have been made in approximation methods. Sahni~\cite{SahniFPTAS} used the $P2||\sum w_jC_j$ problem as a key example to demonstrate the construction of fully polynomial-time approximation schemes (FPTAS) for scheduling problems. He presented a~$(1+\varepsilon)$-approximation algorithm for the problem, which runs in $O(n^2/\varepsilon)$ time. A classical result regarding the $P||\sum w_j C_j$ problem (the variant where the number of machines is arbitrary) shows that applying the WSPT rule greedily leads to a $(1+\sqrt{2})/2$-approximation~\cite{KK1986}. Skutella and Woeginger~\cite{DBLP:conf/stoc/SkutellaW99} improved this result by providing a polynomial time approximation scheme (PTAS) for the $P||\sum w_j C_j$ problem. A PTAS for the more general case where
machines are unrelated and jobs have arbitrary release dates appears in~\cite{DBLP:conf/focs/AfratiBCKKKMQSSS99}. Over the years, various approximation algorithms have been developed for several other strongly NP-hard generalizations of $1||\sum w_j C_j$ focusing on problems with arbitrary release dates and/or with precedence constraints (see for instance~\cite{DBLP:conf/stoc/BansalSS16,DBLP:journals/mp/CorreaM22,DBLP:conf/focs/ImL16,DBLP:conf/sosa/JagerW24,Jaeger:18:Approximating-total-weighted,DBLP:conf/focs/Li17,SittersYang:17:A-2+epsilon-approximation}). 

%In contrast for the large advance in the design of approximation algorithms and schemes over the years, it is quite surprising that the design of exact (pseudo polynomial time) algorithms was quite neglected. We aim to move one step forward to close this gap in the literature by being the first to present a pseudo polynomial time algorithm for $Pm||\sum w_j C_j$, alternative to the classical one by Lawler and Moore~\cite{LawlerMoore}. We do hope this will attract other researchers into this interesting and unexplored research field.




%variants of the problem. A main focus was on problems with arbitrary release dates. The single machine counterpart, i.e., $1|r_j|\sum w_jC_j$, is already strongly NP-hard~\cite{LenstraRinnooy-Kan:77:Complexity-of-machine}. Phillips \emph{et al.}~\cite{DBLP:conf/wads/PhillipsSW95} provided an elegant 2-approximation algorithm for this problem constructed from two phases. In the first phase we obtain an optimal preemptive solution for the problem using the shortest remaining processing time (SRPT) rule. %According to this rule, we schedule at any time the available job that has the least remaining processing time left. 
%In the second phase, we order the jobs based on an increasing order of completion times in the optimal preemptive schedule, and then schedule the jobs non-preemptively based on this ordering. A simple proof then shows that the completion time of any of the jobs increases by at most 2 which yields a 2-approximation algorithm. Other 2-approximation algorithms have been developed over the years (see, e.g.,~\cite{Goemans:97:Improved-approximation}). The 2-approximation barrier was broken by~\cite{ChekuriMotwani:01:Approximation-techniques} which provided a $\frac{e}{e-1}$-approximation algorithm ($\frac{e}{e-1}\approx 1.58$) for the $1|r_j|\sum w_jC_j$ problem. Following this set of papers, various approximation algorithms have been presented to several other extensions of the problem, including to cases where 


%($i$) there are precedence constraints among jobs (see, e.g.,~\cite{Skutella:06:List-scheduling,SittersYang:17:A-2+epsilon-approximation}); 


%($ii$) the processing is done on identical machines working in parallel (see, e.g.,~\cite{SchulzSkutella:97:Scheduling-LPs-bear,Jaeger:18:Approximating-total-weighted}) and ($iii$) the processing is done on unrelated machines working in parallel (see, e.g.,~\cite{SchulzSkutella:97:Scheduling-LPs-bear,AfratiBampis:99:Approximation-schemes}).
 

 

%Many things missing:
%\begin{itemize}
%\item FPT and $n$-fold,
%\item classical results concerning precedence constraints
%\item Garg, Kumar, Minimizing Average Flow-time : Upper and Lower Bounds
%\item Bansal, Aravind Srinivasan, and Ola Svensson. Lift-and-round to improve weighted
%completion time on unrelated machines.
%\item Clifford Stein et al., Approximation Techniques for Average Completion Time Scheduling
%\item Probably many more
%\end{itemize}