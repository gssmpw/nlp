% This must be in the first 5 lines to tell arXiv to use pdfLaTeX, which is strongly recommended.
\pdfoutput=1
% In particular, the hyperref package requires pdfLaTeX in order to break URLs across lines.

\documentclass[11pt]{article}

% Change "review" to "final" to generate the final (sometimes called camera-ready) version.
% Change to "preprint" to generate a non-anonymous version with page numbers.
\usepackage[preprint]{acl}
\usepackage{amsmath}
% Standard package includes
\usepackage{times}
\usepackage{latexsym}
\usepackage{makecell}
\usepackage{multirow}
\usepackage{colortbl}

\usepackage{caption}
\usepackage{subcaption}
% For proper rendering and hyphenation of words containing Latin characters (including in bib files)
\usepackage[T1]{fontenc}
% For Vietnamese characters
% \usepackage[T5]{fontenc}
% See https://www.latex-project.org/help/documentation/encguide.pdf for other character sets

% This assumes your files are encoded as UTF8
\usepackage[utf8]{inputenc}

% This is not strictly necessary, and may be commented out,
% but it will improve the layout of the manuscript,
% and will typically save some space.
\usepackage{microtype}

% This is also not strictly necessary, and may be commented out.
% However, it will improve the aesthetics of text in
% the typewriter font.
\usepackage{inconsolata}

%Including images in your LaTeX document requires adding
%additional package(s)
\usepackage{graphicx}
% If the title and author information does not fit in the area allocated, uncomment the following
%
%\setlength\titlebox{<dim>}
%
% and set <dim> to something 5cm or larger.

\title{FCoT-VL:Advancing Text-oriented Large Vision-Language Models with Efficient Visual Token Compression}

% Author information can be set in various styles:
% For several authors from the same institution:
% \author{Author 1 \and ... \and Author n \\
%         Address line \\ ... \\ Address line}
% if the names do not fit well on one line use
%         Author 1 \\ {\bf Author 2} \\ ... \\ {\bf Author n} \\
% For authors from different institutions:
% \author{Author 1 \\ Address line \\  ... \\ Address line
%         \And  ... \And
%         Author n \\ Address line \\ ... \\ Address line}
% To start a separate ``row'' of authors use \AND, as in
% \author{Author 1 \\ Address line \\  ... \\ Address line
%         \AND
%         Author 2 \\ Address line \\ ... \\ Address line \And
%         Author 3 \\ Address line \\ ... \\ Address line}

\author{
  \textbf{Jianjian Li \textsuperscript{1}},
  \textbf{Junquan Fan\textsuperscript{1}},
  \textbf{Feng Tang\textsuperscript{2}}\thanks{corresponding author},
  \textbf{Gang Huang\textsuperscript{2}}\footnotemark[1],
  \textbf{Shitao Zhu\textsuperscript{2}},
  \textbf{Songlin Liu\textsuperscript{2}} \\
   \textbf{Nian Xie\textsuperscript{2}},
  \textbf{Wulong Liu\textsuperscript{2}},
  \textbf{Yong Liao\textsuperscript{1}\footnotemark[1]}
\\
  \textsuperscript{1}University of Science and Technology of China.\\
  \textsuperscript{2}Huawei Noah’s Ark Lab.\\
\\
  % \small{
  %   \textbf{Correspondence:} \href{hanjiayi@inspur.com}{hanjiayi@inspur.com}
  % }
%}
}

%\author{
%  \textbf{First Author\textsuperscript{1}},
%  \textbf{Second Author\textsuperscript{1,2}},
%  \textbf{Third T. Author\textsuperscript{1}},
%  \textbf{Fourth Author\textsuperscript{1}},
%\\
%  \textbf{Fifth Author\textsuperscript{1,2}},
%  \textbf{Sixth Author\textsuperscript{1}},
%  \textbf{Seventh Author\textsuperscript{1}},
%  \textbf{Eighth Author \textsuperscript{1,2,3,4}},
%\\
%  \textbf{Ninth Author\textsuperscript{1}},
%  \textbf{Tenth Author\textsuperscript{1}},
%  \textbf{Eleventh E. Author\textsuperscript{1,2,3,4,5}},
%  \textbf{Twelfth Author\textsuperscript{1}},
%\\
%  \textbf{Thirteenth Author\textsuperscript{3}},
%  \textbf{Fourteenth F. Author\textsuperscript{2,4}},
%  \textbf{Fifteenth Author\textsuperscript{1}},
%  \textbf{Sixteenth Author\textsuperscript{1}},
%\\
%  \textbf{Seventeenth S. Author\textsuperscript{4,5}},
%  \textbf{Eighteenth Author\textsuperscript{3,4}},
%  \textbf{Nineteenth N. Author\textsuperscript{2,5}},
%  \textbf{Twentieth Author\textsuperscript{1}}
%\\
%\\
%  \textsuperscript{1}Affiliation 1,
%  \textsuperscript{2}Affiliation 2,
%  \textsuperscript{3}Affiliation 3,
%  \textsuperscript{4}Affiliation 4,
%  \textsuperscript{5}Affiliation 5
%\\
%  \small{
%    \textbf{Correspondence:} \href{mailto:email@domain}{email@domain}
%  }
%}

\begin{document}
\maketitle
\begin{abstract}
Testing Autonomous Driving Systems (ADS) is crucial for ensuring their safety, reliability, and performance. Despite numerous testing methods available that can generate diverse and challenging scenarios to uncover potential vulnerabilities, these methods often treat ADS as a black-box, primarily focusing on identifying system failures like collisions or near-misses without pinpointing the specific modules responsible for these failures. Understanding the root causes of failures is essential for effective debugging and subsequent system repair. We observed that existing methods also fall short in generating diverse failures that adequately test the distinct modules of an ADS, such as perception, prediction, planning and control.

To bridge this gap, we introduce \tool, the first root-cause-aware testing method for ADS. Unlike previous approaches, \tool not only generates scenarios leading to collisions but also showing which specific module triggered the failure. This method targets specific modules, creating test scenarios that highlight the weaknesses of these given modules. Specifically, our approach involves designing module-specific oracles to ascertain module failures and employs a module-directed testing strategy that includes module-specific feedback, adaptive seed selection, and mutation. This strategy guides the generation of tests that effectively provoke module-specific failures. We evaluated \tool across four critical ADS modules and four testing scenarios. The results demonstrate that our method can effectively and efficiently generate scenarios where errors in targeted modules are responsible for ADS failures. It generates 216.7 expected scenarios in total, while the best-performing baseline detects only 79.0 scenarios. 
Our approach represents a significant innovation in ADS testing by focusing on identifying and rectifying module-specific errors within the system, moving beyond conventional black-box failure detection.
\end{abstract}

\keywords{Module-Specific Failure, Autonomous Driving System, Testing}

\section{Introduction}\label{sec:introduction}
% -- Outline
% ---- LLMs are popular
% ---- There're many stakeholders in the training and inference loop
% ---- Adversaries in the training loop are a problem -- malpractice, poisoning
% ---- Also, showing compliance
% ---- Need a framework to prove the integrity of the pipeline
% ---- Enter Atlas

% ---- LLMs are popular
In recent years, machine learning (ML) models, have become increasingly popular.
The pervasive use of large language models (LLMs), in particular, and multi-stakeholder
involvement in model creation and deployment exacerbate security and privacy risks.
These considerations are emphasized by the global nature and the complexity of
large-scale ML deployments with different lifecycle stages:
%gathering and sanitizing the data from different sources,
%training and inferencing across many data centers,
%compliance with local laws or corporate policies.

% ---- There're many stakeholders in the training and inference loop
%Additionally, different stages of the ML development pipeline come with their own stakeholders:
\begin{enumerate}[label=\arabic*)]
    \item Collection and sanitation of a \emph{training} dataset from several public and proprietary sources.
    %\item Solicitation and facilitation of training.
    \item Provisioning of the training environment (hardware and software).
    \item Execution of training across many data centers.
    \item Construction of a \emph{testing} dataset from several sources, and the evaluation.
    \item Deployment and use of the model for inference that is compliant with local laws or corporate policies.
    %\item Use of the model in compliance with local laws or corporate policies.
\end{enumerate}

% ---- Adversaries in the training loop are a problem -- malpractice, poisoning
Each of these stages is vulnerable to malicious or dishonest parties.
For example, data can be poisoned~\cite{biggio2012poisoning,carlini2024poisoning} during collection or training.
Service providers executing outsourced training can shorten or omit critical steps to reduce their cost.
Model providers can serve smaller models in SaaS, or even distribute malicious ones.

% ---- Also, showing compliance
On the other hand, responsible model builders and other stakeholders may be incentivised or required to provide security and trust guarantees.
They may want to prove low bias in their training data, offer easily verifiable performance claims, or guarantee end-to-end integrity of the model creation in high risk domains.

% ---- Need a framework to prove the integrity of the pipeline
To address these challenges, it is necessary to guarantee the integrity of the entire ML lifecycle --
beginning with the data, through the training, and finally, the evaluation and deployment.
Was the data modified?
Did the hardware and software environment adhere to the specification?
Did the contractor follow the specified training procedure?
Can I trust the evaluation?
How can I guarantee that I am interacting with the intended model?
These are example questions that showcase the breadth of the involved challenges that must be tackled to provide end-to-end security.

% --- Enter Atlas
In this work, we introduce \atlas, a framework for enhancing the security and transparency of the lifecycle of ML models.
\atlas establishes the baseline of fundamental components and capabilities needed for comprehensive provenance tracking
at each stage of the ML lifecycle.
Subsequently, \atlas defines the core integrity requirements for verifiable ML lifecycle transparency.
We provide a reference implementation that instantiates \atlas using hardware-based security mechanisms -- with trusted execution environment (TEE),
including attestations.% , and comprehensive metadata-based provenance tracking.
%Our implementation satisfies all \atlas requirements.

We claim the following contributions:
\begin{enumerate}[label=\arabic*.]\label{sec:introduction:contributions}
    \item We introduce \atlas, a framework designed for end-to-end ML lifecycle transparency.
    \item We instantiate \atlas using TEEs and metadata-based provenance tracking.
    \item We evaluate our \atlas prototype through two case studies:
        \begin{enumerate*}[label=\arabic*)]
            \item fine-tuning of a BERT model~\cite{lin2023metabert, lin2023metabertimpl};
            \item fine-tuning of a bge-reranker model~\cite{chen2023bge}
        \end{enumerate*}.
\end{enumerate}

%\msm{revise: Integrate this motivation into intro}
%Organizations frequently leverage pre-trained models, outsource training processes, and integrate components from multiple sources,
%making it difficult to verify the authenticity and trustworthiness of their ML systems. This complexity is further compounded
%by the potential for malicious modifications at various stages of the model lifecycle, from data preparation through deployment.
%The involvement of various third parties in ML model development and deployment
%creates critical challenges in ensuring supply chain integrity.
%
%While Software Bills of Materials (SBOMs) and AI Bills of Materials (AI BOMs) provide basic inventory tracking for model components,
%they fall short in addressing the dynamic nature of ML pipelines. These approaches typically offer point-in-time snapshots but
%fail to capture the complex transformations, fine-tuning operations, and runtime modifications that characterize modern ML workflows.
%Additionally, they lack cryptographic guarantees about the integrity of recorded information and cannot effectively track the provenance
% of model weights and training data.
%
% These approaches demonstrate the growing importance of ML supply chain security.
% However, they are typically applied in an ad-hoc fashion, highlighting the need
% for a more integrated approach that combines comprehensive lineage tracking,
% strong cryptographic properties, and practical integration capabilities with existing ML development and deployment pipelines.
%
%A comprehensive solution requires not just documentation of components, but verifiable evidence of their origins,
%transformations, and integrity throughout the entire model lifecycle. This need has driven interest in more robust
%provenance tracking mechanisms that can:
%
%\begin{itemize}
%\item Provide cryptographic proof of model lineage
%\item Track and verify all pipeline transformations
%\item Maintain tamper-evident records of training processes
%\item Ensure integrity of model artifacts across organizational boundaries
%\end{itemize}
%
%Several existing tools and frameworks
%commonly focusing on different components of the model lifecycle and provenance tracking.
%While these solutions offer valuable capabilities, they often address only specific parts of the end-to-end ML
%supply chain rather than providing comprehensive coverage.
%\msm{end-revise}
%
%\todo{add discussion of EU-CRA AI Act requirements for model documentation and FDA guidelines for AI/ML in healthcare}

%The remainder of this paper is organized as follows:
%in Section~\ref{sec:background-related} we provide an overview of the necessary background, and the related work;
%Section~\ref{sec:problem} presents the challenge of providing integrity in the ML pipeline, the threat model, and the system assumptions;
%in Section~\ref{sec:framework} we present \atlas -- our framework for providing ML integrity;
%Section~\ref{sec:implementation} covers implementation details;
%in Section~\ref{sec:eval}, we show that \atlas is effective across three dimensions: training overhead $<8\%$, the verification time increases linearly with the size of the model, and it is compatible with PyTorch and Tensorflow;
%in Section~\ref{sec:casestudies} we present the case studies;
%in Section~\ref{sec:discussion} we discuss additional considerations for \atlas,
%and Section~\ref{sec:conclusion} concludes the paper and provides directions for future work.


\section{Related Works}
\subsection{Vision Large Language Models}


%%丰富类内容

%% 应该包含 LLaVA , Gemini,internvl2,qwenvl2,deepseek-v
In recent years, open-source  VLLMs have made significant advancements, driven by contributions from both academia and industry. 
Earlier models, such as BLIP-2 \cite{li2023blip}, MiniGPT\cite{zhu2023minigpt} and LLaVA\cite{liu2024visual,liu2024llava}, have proven to be effective for vision-language tasks via bridging off-the-shelf ViTs and LLMs. 
However, early VLLMs struggle with processing images containing
fine-grained details, especially for OCR-like tasks such as charts\cite{masry2022chartqa}, documents\cite{mathew2021docvqa},
and infographics\cite{mathew2022infographicvqa}. To this end, InternVL series propose
an adaptive cropping method to convert vanilla images as several fixed image patches. For example, InternLM-XComposer2-4KHD\cite{dong2024internlm} increases 336 pixels of CLIP to 4K resolution and gets strong document understanding ability. InternVL2
obtains promising results on text-oriented benchmarks via scaling up image resolution and ViT model parameters. Moreover, QwenVL2 \cite{wang2024qwen2vl} proposes a native dynamic processing of images at varying resolutions. This image processing setting generates more visual tokens and suppress adaptive cropping VLLMs. However,
high-resolution processing pipelines bring substantial computational overhead in both training and inference stages, hindering real-world deployment.



% Following works scale up image resolution to enhance the capabilities of fine-grained perception\cite{chen2024internvl,chen2024far,hong2024cogvlm2,glm2024chatglm,liu2024llavanext,wang2024qwen2vl}.
% InternVL families\cite{chen2024internvl,chen2024far} introduce a dynamic resolution cropping scheme to generate dynamic visual tokens, enhancing the capture of fine-grained details at the cost of higher visual tokens. However,
% this brings substantial computational overhead in both training and inference stages, hindering real-world deployment.



Beyond high-resolution tricks, many works reveal that high-quality datas are more important for advancing document understanding. Recent studies\cite{hu2024minicpm,li2024baichuan,li2025eagle} highlight the critical role of data quality in VLLMs. For instance, InternVL-2.5\cite{chen2024expanding} enhanced performance of previous version through collecting more diverse dataset and data processing pipelines.

In this paper, we also explore how to obtain high-quality post-training datas to match frontier open-source VLLMs. Specifically, Our FCoT-VL outperforms the base model InternVL2
on many benchmarks like ChartQA\cite{masry2022chartqa} and MathVista\cite{lu2024mathvista}, despite reducing visual tokens by 50$\%$.



\begin{figure*}[t]
  \centering
  \includegraphics[width=\textwidth]{overview.pdf}
  \caption{Overall Structure of FCoT-VL. FCoT-VL is a self-distillation architecture in which only the Student-Projector and Compress-Module are learned, while all the other modules remain frozen. The student and teacher models share the same ViT encoder and the LLM decoder.}
  \label{fig:Overvieww}
\end{figure*}
\subsection{Visual Compression Schemes}
% Introduce Compression Scheme Evolution from Vit to Vllm
% Since the appearance of Vision Transformer (VIT), many schemes on compressing the number of tokens extracted by VITs are proposed by different scholars to reduce visual infomation redundancy. 
Visual compression, a key focus in high-resolution VLLMs, aims to efficiently reduce the use of vision tokens, minimizing computational and memory overheads. The inherent redundancy of visual data, compared to dense textual data, underscores the importance of compression.

Solutions to visual compression can be broadly categorized into two main approaches: training-free and training-based ones. Training-free methods dynamically select more important vision tokens via various strategies during decoding stage. For instance, SparseVLM\cite{zhang2024sparsevlm} and VisionZip \cite{yang2024visionzip} prioritize tokens based on attention scores. ToMe\cite{bolya2022token} and LLaVA-PruMerge\cite{shang2024llava} cluster tokens using cosine similarity. However, the training-free paradigms suffer from significant performance drops in text-orientated benchmarks. 
In contrast, training-based methods focus on optimizing the visual adaptor by incorporating external modules for token reducing. For instance, LLaMA-VID\cite{li2024llama} enhances visual information extraction through Q-Former\cite{li2023blip} with context tokens. Similarly, models like C-Abstractor\cite{cha2024honeybee} and LDP\cite{chu2024mobilevlm} serve as promising alternatives for visual token compressing.

Training from scratch necessitates extensive alignment datasets and substantial computational resources, often consuming thousands of GPU days. In this work, we present an efficient training token-compressing framework that achieves comparable performance while significantly reducing both data and computational requirements.



% Early works focus on projectors in VLLMs , like QwenVL (\citeauthor{bai2023qwen}), applying the Qformer-like structure which utilize learnable queries to extract dense information  by cross-attention layers. Recently, Tokenpacker (\citeauthor{li2024tokenpacker}) use multi-level features as queries and values to improve the performance. Some other works, like C-abstractor (\citeauthor{cha2024honeybee}) and LDP (\citeauthor{chu2024mobilevlm}), prove that convolution layers can generate compressed tokens for VLLMs.

% Some recent works leverage merging tokens to compress visual information based on similarity. For instance, ToMe (\citeauthor{bolya2022token}) divides visual tokens into two sets, connects the most similar tokens based on cosine similarity between sets and averaging features before concatenating the sets; LLaVA-PruMerge (\citeauthor{shang2024llava}) dynamically selects the most important visual tokens and merges similar ones using a clustering approach. Besides, some works pay more attention to the attention score of models. SparseVLM (\cite{zhang2024sparsevlm}) selects the relevant text rater based on the attention score between vision and text, and then selects the more important visual token based on the attention score between the rater and vision. VisionZip (\cite{yang2024visionzip}) focuses on selecting the most informative token by the attention score and discarding the less informative tokens to reduce the number of overall tokens. But most of current works are based on LLaVa-1.5 which is not high-resolution schema. Few works, like TextHawk2 (\cite{yu2024texthawk2}), did some research on high-resolution schema.

% Except in the vit stage, some works propose token pruning methods happening in the LLM stage. FastV (\citeauthor{chen2025image}) ranks tokens by the average attention value with all other tokens, and prunes out the last R tokens. Additionally, DocKylin (\citeauthor{zhang2024dockylin}) also provide a method in the data processing stage called Adaptive Pixel Slimming (APS) that reduce the resolution of images.
 
% \subsection{Model merge}
% In recent years, model merging has emerged as a prominent research area, focusing on combining multiple task-specific models into a single, versatile model with broad capabilities (\citeauthor{wortsman2022model,yan2024corrective}). Unlike multi-task learning (\citeauthor{crawshaw2020multi}), which also seeks to develop a single model capable of handling multiple tasks, model merging prioritizes the integration of model parameters without the need for access to the original training data (\citeauthor{matena2111merging}). One well-known approach to model merging is Average Merging (\citeauthor{wortsman2022model}), where the merged model is constructed by averaging the parameters of the individual models. Another method, Task Arithmetic (\citeauthor{ilharco2022editing}), incorporates a pre-defined scaling factor to adjust the influence of different models. Fisher Merging (\citeauthor{matena2111merging}) applies weighted averaging of parameters, with the weights determined by the Fisher information matrix (\citeauthor{matena2111merging}). In contrast, RegMean (\citeauthor{jin2022dataless}) solves the merging problem by optimizing a linear regression framework with closed-form solutions. Lastly, TIES-Merging (\citeauthor{yadav2023resolving}) addresses task conflicts identified (\citeauthor{ilharco2022editing}) by trimming parameters with low magnitudes, resolving sign inconsistencies, and merging parameters with consistent signs in a disjoint manner.

    

\section{Method}
\label{sec:method}

\subsection{Weaknesses of Previous Conditioning Methods}

The most popular form of latent image conditioning typically converts conditioning signals to images, before processing them with typical image processing models. While this approach is powerful, it exhibits limitations in handling complex image synthesis tasks, particularly when incorporating heterogeneous or sparse input conditions. Some approaches, such as \textit{LayoutDiffusion} \cite{zheng_layoutdiffusion_2024}, tackle this with custom attention modules that attend to bounding boxes with learned positional embeddings. However, these approaches neglect to include multiple modalities and the relationships between them, which overlooks nuanced interactions between conditioning signals i.e. disambiguating spatial ordering between overlapping boxes. 

% For example, interactions between conditions which may not explicitly exist in the discrete spatial image domain.

% These approaches force diverse modalities, like mixed spatial and categorical information directly into a unified image space, which overlooks nuanced interactions between conditioning signals. For example, interactions between conditions which may not explicitly exist in the discrete spatial image domain.

Previous conditional diffusion research that utilise graph data opt for complex multi-stage training procedures such as masked contrastive pre-training using graph triplets \cite{yang_diffusion-based_2022}. This is not only time-consuming, but also fails to exploit potential benefits of training an end-to-end system that integrates graph data directly into image processing. 
% Furthermore, other work has shown that the repeated conditioning diffusion models (i.e. time or text conditioning) is superior to simply providing   

We tackle these problems by representing images and their conditioning signals as a single graph, which is processed by a bespoke GNN architecture. This allows repeated interactions between conditioning signals and the image throughout the synthesis process, enabling more flexible and dynamic representations that account for both the current image features and interactions between conditioning signals. By maintaining separate pathways for distinct input types, our approach supports heterogeneous and sparse conditioning, leading to better generalisation, finer control, and more precise manipulation of generated images. This simple yet powerful method can be easily integrated into a wide range of existing vision models.

\begin{figure}
    \centering    \includegraphics[width=1\linewidth]{icml2023/hig_fig2.pdf}
\vspace{-20pt}
    \caption{(\textbf{a}) Overview of the proposed architecture. The HIG is encoded into a latent representation through a MP-GNN which is then used as a condition $c_f$ in a ControlNet. (\textbf{b}) Details of the MP-GNN module. Note: HMP is shorthand for heterogenous magnitude preserving operations applied across all nodes.}
    \label{fig:architecture}
\end{figure}

\subsection{Heterogeneous Image Graphs}

To improve on previous approaches we develop a new approach to condition images via the HIG representation. In this manner, we fully exploit variable-length and heterogeneous conditions to aid in image synthesis.

\textbf{Image Graphs.} When faced with the challenge of conditioning images with graphs we first convert images into representations amenable for graph processing. We reshape image features into image nodes pixel-wise in line with other works \cite{liu_cnn-enhanced_2021, han_vision_2022}. In practice, these nodes represent more than a single pixel, for example a latent image patch. This can be due to performing latent image diffusion \cite{rombach_high-resolution_2022, podell_sdxl_2023} where images are first pre-compressed to latent images, or due to prior processing by the image processing model. In contrast to other works \cite{tian_image_nodate, han_vision_2022, tarasiewicz_graph_2021}, we decide to leave image nodes unconnected; this loosely decouples image conditioning from processing. Image nodes are conditioned and later converted back into an image representation, allowing existing architectures to handle processing. Connecting image nodes in a locally dense fashion gains little benefit over highly optimised $3 \times 3$ convolutional operations. Formally, image nodes exist in a discrete space \( f : \mathbb{Z}^2 \to \mathbb{R}^C \). For an image of size \(M \times N\), we define \( f(i, j) \) where \( i, j \in \mathbb{Z} \) and \( 0 \leq i < M \), \( 0 \leq j < N \).

\textbf{Conditioning Graphs}. Conditioning graphs consist of nodes and edges, where each node has features defined as $ g : \mathcal{V} \to \mathbb{R}^F$, where $\mathcal{V}$ represents the set of nodes and $\mathbb{R}^F$ the feature space. Nodes may have spatial ties to the image domain, which we materialise via edges linking image and conditioning nodes. We use conditioning nodes to indicate semantics within the scene, for instance, a node may represent an object (e.g., a \textit{person}). Whereas we utilise different edge types to represent both spatial, abstract relationships and additional semantics. For instance, an edge between two object nodes may encode interactions or attributes (e.g., a person \textit{wearing} a {\textit{yellow}} hat). The graph structure reflects real-world data: often sparse and heterogeneous. We therefore construct graphs on a per task-basis to best leverage the available data and its dependencies.
Formally, each edge \( e \in \mathcal{E} \) connects two nodes \( (v_i, v_j) \in \mathcal{V} \times \mathcal{V} \) and represents a relationship between them. Edges represent any dependency, allowing for abstract relationships to be included.

% To continue the example, if spatial information for both the \textit{person} and the \textit{hat} is available, the graph would contain a node for each object and an edge connecting them, with the edge encoding the relationship \textit{wearing}. 


% \textbf{Conditioning Graphs.} In contrast, conditioning graphs are represented by sets of nodes and edges, with each node having associated features defined by a function $( g : \mathcal{V} \to \mathbb{R}^F$, where $\mathcal{V}$ represents the set of nodes and $\mathbb{R}^F$ the feature space. Although nodes \textit{may} have explicit spatial ties to the discrete image domain, we materialise these through edges between image and conditioning nodes. However, these relationships may be the product of spatial properties of conditioning nodes. As such, subsets of $\mathbb{R^F}$ may represent spatial coordinates \( (x, y) \in \mathbb{R}^2 \) that satisfy \( 0 \leq x < M \) and \( 0 \leq y < N \). Conditioning nodes are not restricted to pixel grid positions, nor the number of spatial dimensions e.g. nodes may represent 3D properties of the real world. Nodes and edges may represent properties independent of spatial dimensions. For example, nodes in the graph can represent concrete objects in the image (e.g., a \textit{person}), while edges between them may represent abstract interactions or attributes (e.g., a person \textit{wearing} a {\textit{yellow}} hat). The graph structure may be sparse, and heterogeneous (multiple types of nodes and edges). Conditioning graphs are constructed on a per-task basis to optimally leverage available data and its dependencies. Formally, each edge \( e \in \mathcal{E} \) connects two nodes \( (v_i, v_j) \in \mathcal{V} \times \mathcal{V} \) and represents a relationship between them. To continue the example, if spatial information for both the \textit{person} and the \textit{hat} is available, the graph would contain a node for each object and an edge connecting them, with the edge encoding the relationship \textit{wearing}. Edges can represent any dependency, allowing for abstract relationships to be included in the graph.

\textbf{Connecting Image and Conditioning Nodes.} With image and conditioning nodes defined, we are close to the complete HIG representation. To enable conditioning between the image and conditioning graphs, we must construct edges between the two. These connections are determined on a per-task basis, depending on the available data, with explicit choices described in Section 4. However, when spatial information is available i.e. segmentation masks or bounding boxes, it enables direct connections between the image graph and the conditioning graph. Specifically, edges are created between image nodes relevant to spatial conditionings (i.e. pixels within the bounding box) and conditioning nodes representing the corresponding semantic class (i.e. class label). This linkage facilitates information flow across the graphs, integrating pixel-level details with higher-level semantic representations. 

% Additionally, the flexibility of heterogeneous GNNs allows for connections from the image back to the graph with different sets of learned weights. This approach enables the image to influence the graph structure while leveraging the rich semantic details present in the image—such as color or object sub-class—throughout much of the diffusion training scheme, while still respecting the different types of information carried by the node types.

\subsection{Model Architecture}

To be compatible with the EDM2 U-Net architecture \footnote{\href{https://github.com/NVlabs/edm2}{https://github.com/NVlabs/edm2}}, we propose the addition of a magnitude-preserving \textit{Heterogenous Image Graph Neural Network} (HIGnn) as the conditioning network to be used in a ControlNet strategy.

\textbf{HIGnn.} The general architecture of the HIG conditioning block requires two primary capabilities: representation switching and HIG processing. To handle switching between image features and image nodes on the HIG we consider the update function $\mathcal{U}_{\text{i}\rightarrow\text{g}}$. This update functions reshapes image features $\mathbf{x_i} \in \mathbb{R}^{N \times C \times H \times W}$ into image nodes pixel wise $\mathbf{x_g} \in \mathbb{R}^{N\cdot H \cdot W \times C}$ and applies an optional projection to ensure correct dimensionality. For the current set of image pixels $\mathbf{x_i}$, we retrieve HIG image nodes $\mathbf{x_g}$ by
\begin{equation}
\mathbf{x_g} = \mathcal{U}_{\text{i}\rightarrow\text{g}}(\mathbf{x_i}) = \hat{W}R(\mathbf{x_i}),  
 \label{eq:HIG_update}
\end{equation}
where $R$ reshapes the image, and $\hat{W}$ is a learned projection with forced magnitude preservation from \cite{karras_analyzing_2024}. Refer to Appendix \ref{appendix:edm2_preliminaries} for greater detail into the mathematical preliminaries of \cite{karras_analyzing_2024}. We consider the reverse operation of converting from graph nodes to an image $\mathcal{U}_{\text{g}\rightarrow\text{i}}$ in a similiar fashion. 

Once we have the HIG updated with current image nodes we can process it with a GNN. We identify several areas where magnitudes can grow and address them each in turn. In practice many varieties of heterogenous message passing GNN could be used, we create our own magnitude preserving graph convolutional operator similiar to Hamilton et al. \cite{hamilton_inductive_2018} for its simplicity and stability. The basic approach propagates information through two branches, a pseudo `skip-connection' applied to the current node, and a learned pooling operation of the local neighbourhood, and we add the ability to include edge information in the neighbourhood pooling. If edge attributes $\mathbf{a}_i$ are present we integrate them via magnitude preserving concatenation to the pooling branch. Formally, the HIGConv operator applied per meta-path to get updated node embeddings $\mathbf{x}_i'$ is defined as:
% \begin{equation}
%     \mathbf{x}_i' = \psi\left(\hat{W}^{\Phi}_1 \mathbf{x}_i +^\text{mp} \hat{W}^{\Phi}_2 \cdot \frac{1}{\sqrt{|\mathcal{N}^{\Phi}|}} \sum_{j \in \mathcal{N}^{\Phi}(i)} [\mathbf{x}_j \|^\text{mp} \mathbf{a}_j] \right),
%     \label{eq:hignn_operator}
% \end{equation}
\begin{equation}
    \mathbf{x}_g' = \psi\left(\hat{W}^{\Phi}_1 \mathbf{x}_g 
    \underset{0 \text{ if } |\mathcal{N}^{\Phi}(i)| = 0}{\underbrace{+^\text{mp} \hat{W}^{\Phi}_2 \cdot \frac{1}{\sqrt{|\mathcal{N}^{\Phi}(i)|}} \sum_{j \in \mathcal{N}^{\Phi}(i)} [\mathbf{x}_j \|^\text{mp} \mathbf{a}_j]}}\right)    \label{eq:hignn_operator}
\end{equation}

% \[
%     \mathbf{x}_i' = \psi\left(\hat{W}^{\Phi}_1 \mathbf{x}_i +^\text{mp} 
%     \underset{+ 0 \text{ if } |\mathcal{N}^{\Phi}(i)| = 0}{\underbrace{\hat{W}^{\Phi}_2 \cdot \frac{1}{\sqrt{|\mathcal{N}^{\Phi}(i)|}} \sum_{j \in \mathcal{N}^{\Phi}(i)} [\mathbf{x}_j \|^\text{mp} \mathbf{a}_j]}}\right).
% \]

where we choose $\psi$ to be magnitude preserving SiLU operator, and $+^\text{mp}$ the magnitude preserving sum (See Appendix \ref{appendix:edm2_preliminaries}), and both meta-path weights $\hat{W}^{\Phi}_1$ and $\hat{W}^{\Phi}_2$ have forced magnitude. $\mathcal{N}$ indicates the local node neighbourhood and is defined by the connectivity of graph. In order to achieve magnitude preservation we first assume all neighbourhood features to be of unit length, we then summate them scale them by the square root of the neighbourhood size ($\sqrt{|\mathcal{N}^{\Phi}|}$), see Appendix \ref{appendix:sum_random} for details. It is important to address unconnected or `zero-degree' nodes, in this case we ignore the right hand side of the equation, and only take the residual path. Note that simply setting the  neighbourhood to zero unintentionally changes the feature magnitudes when mp-sum is applied, since it assumes both vectors to be of unit length. Finally to combine information across meta-paths, we use the same method and sum across paths before normalising by the inverse square root of the number of incoming meta-paths ($|\Phi_i| = |\{\Phi_k \mid x_i \in \Phi_k\}|$)

% To formulate a heterogeneous GNN with learned projections per meta-path ($\mathbf{\Phi} = \{\Phi_1 ... \Phi_n\}$), we must preserve magnitudes when combining meta-paths.

\begin{equation}
\Tilde{\mathbf{x}}_g = \frac{1}{\sqrt{|\Phi_g|}} \sum_{\Phi \in \Phi_g} \mathbf{x}'_g,
\label{eq:meta_path}
\end{equation}

We verify that this approach is guaranteed to maintain magnitudes under certain conditions of the underlying graph data. In particular, for graph-data of sufficient size this approach holds for graphs which do not have identical features attached to the same node since this breaks the independence assumption. 

% An interesting interpretation of this formulation with respect to image synthesis is to observe how different receptive fields change. The typical convolutional operator used in U-Net models define a local image receptive field $\mathcal{R}$, self-attention  defines a global image receptive field $\mathcal{A}$, and the HIGnn defines receptive fields over meta-path relationships $\mathcal{N}^{\Phi}$ for both the image and conditioning variables. We postulate this to an advantage over other conditioning methods as it allows instant communication between different conditioning signals and parts of the image whilst remaining computationally tractable.

\textbf{EDM2 ControlNet Integration.} To integrate conditioning into a generative model, we adopt a strategy similar to ControlNet \cite{zhang_adding_2023}, i.e. a frozen EDM2 pre-trained model, with a trainable copy the encoder integrated with the conditioning HIGnn. Refer to Figure \ref{fig:architecture} for an overview of our proposed architecture, we employ 4 HIG blocks for our base model. The EDM2 checkpoints are only available for class-conditional generation of the 1000 ImageNet classes, yet we find them easy to adapt to our natural image datasets.  To facilitate this we unfreeze the embedding network. To integrate features we adopt $1\times1$ convolutions with a learnable zero-gain in a similar fashion to the original ControlNet, but we note that traditional summation may damage feature magnitudes. We find that naively integrating is harmful to training. Instead, we apply magnitude preserving summation, which, in contrast to the original ControlNet paper, directly alters the primary network features. This yields poor generative quality at step 0, but proves to be quick to train and to be best in practice.

In the trainable encoder we integrate our proposed HIGnn after the initial convolution block. We opt to keep the dimension of the GNN matched to that of the generative model. Finally, to generate samples we opt for the non-stochastic EDM2 sampler, and use the recent advancements in auto-guidance \cite{karras_guiding_2024}, we use our control model as the primary network, and use the unconditional XS ImageNet checkpoint released with EDM2 as the guidance network \cite{karras_analyzing_2024, karras_guiding_2024}. 

% We do not use EMA

\begin{figure}[t]
    %
    \begin{center}
    \centerline{\includegraphics[width=0.5\columnwidth]{experiments/recht_loss.pdf}}
    %
    \caption{Matrix factorization via a 2-layer linear network}
    \label{recht}
\end{center}
\end{figure}

\section{Experiments}
\label{sec:experiments}
The prior works of FOOF, LocoProp, PRONG have shown to compare competitively
with other sophisticated optimizers such as K-FAC \citep{martens2015}.
Given their discussed equivalence with LNB, we focus on experimentally confirming
the contributions of this work: feature whitening via preconditioning the
gradient vector, and the applicability and effectiveness on
the realistic networks of ViT \citep{vit} and UNet \citep{unet}.

Due to its de facto status, we benchmark convergence with Adam in both
iteration and wall time.
In all experiments, the default EMA values were used ($\beta_1=0.9$, $\beta_2=0.999$)
for Adam while the learning rate was grid searched around $1e^{-4}$
All timings were recorded from a NVIDIA L4 GPU. Due to the deterministic nature of performing
2 conjugate gradient steps for LNB, the variance in reported times is negligible.


\subsection{Matrix factorization}
We start with the pathological example from \citet{recht}. This is a matrix
factorization problem formulated as a two-layer linear network:
$\sum_{i=1}^{n} \Vert W_1 W_2 x_i - y_i\Vert^2$,
where $y_i=Ax_i$ for a poorly conditioned matrix, $\kappa(A)=10^5$. Due to the conditioning
and the columns being correlated, it is known that gradient descent converges slowly for this
problem, whereas GN converges quickly. Because the LNB preconditioner is decorrelating
the feature space, we would also expect fast convergence.

We use the same initialization as in the notebook, but in order to magnify the differences,
we increase the dimensions by a factor of $10$: $n=10^4$,
$W_2 \in \sR^{60 \times 60}$, $W_1 \in \sR^{100 \times 60}$. Learning rates were tuned via
grid search to find fast and stable convergence for each method. The results are plotted in Figure \ref{recht}
and reproduce the prior reported slow convergence of Adam and demonstrate fast convergence with LNB.

\begin{figure}[t]
    %
    \begin{center}
    %
    \subfigure[Original pixels ($x$)]{\includegraphics[width=0.49\columnwidth]{experiments/mnist_acc.pdf}}
    %
    \subfigure[Inverted pixels ($1-x$)]{\includegraphics[width=0.49\columnwidth]{experiments/inv_mnist_acc.pdf}}
    %
    \vskip -0.1in
    \caption{MNIST test accuracy evolution trained on the original data (a) vs. inverted pixels (b).
    The step-size is parenthesized.}
    \label{fig:mnist}
    \end{center}
\end{figure}


\subsection{MLP}
We reproduce the MLP result in \citet{grub2010} that compares boosting and gradient
descent for a 2-layer MLP on MNIST using 800-node layers, $\tanh$ activation,
Glorot Normal initialization and a batch size of 1,000.

In Fig \ref{fig:mnist}-a., we plot the test accuracies w.r.t. epoch with the best two learning rates for Adam.
We first note that Adam and LNB obtain better than the prior reported accuracy of $98.3\%$.
Second, LNB achieved this performance with a fixed (ridge) regularizer $\lambda$, whereas prior
work heavily tuned this.
One explanation for the difference is that LNB is taking an adaptive step size according to the metric
and this was not derived before.
Third, there is little observed difference between boosting and gradient descent on this dataset.
This can be explained due to that most of the binary pixels in MNIST are zero, so the feature space
of the vectorized images is low rank, i.e., decorrelating provides little benefit.

\begin{figure}[t]
    \begin{center}
    \subfigure[\texttt{train} loss]{\includegraphics[width=0.49\columnwidth]{experiments/vit_loss.pdf}}
    \subfigure[\texttt{test} accuracy]{\includegraphics[width=0.49\columnwidth]{experiments/vit_acc.pdf}}
    %
    \caption{ViT performances on CIFAR10.}
    \label{fig:vit}
    \end{center}
\end{figure}

However, whitening does include a centering step. If we were to shift the feature space, we would expect
to get same performance. In Fig \ref{fig:mnist}-b, we plot the same models when trained and tested on pixels
$1-x$, where $x$ are the original binary pixel values used in Fig \ref{fig:mnist}-a. We observe that LNB gets very similar performance
while Adam (and other methods not invariant to affine reparameterizations) degrade. Although we could (and should)
simply normalize the features before training, this example illustrates a case of how feature scaling can
greatly affect convergence.

\subsection{Vision Transformer}
We train a vision transformer \cite{vit} using the notebook from Equinox \cite{eqx}
on CIFAR10.
The only modification we make is to not learn the affine terms in the LayerNorm
in order to speed up experimentation and we observed no performance benefit with it.
The train and test fold performances are show in \Figref{fig:vit}, where we observe
faster convergence and better generalization with LNB. Excluding JIT compilation time,
the duration per epoch for LNB and Adam is 1.26 min and 0.85 min, respectively. 

\begin{figure}[t]
    \begin{center}
    \subfigure[\texttt{train} loss]{\includegraphics[width=0.49\columnwidth]{experiments/voc_loss.pdf}}
    \subfigure[\texttt{val} accuracy]{\includegraphics[width=0.49\columnwidth]{experiments/voc_acc.pdf}}
    %
    \caption{UNet performances on VOC Segmentation.}
    \label{fig:unet}
    \end{center}
\end{figure}

\subsection{UNet}
We train a UNet \cite{unet} on the 2012 VOC Segmentation Challenge dataset \cite{voc}. The images are
pixelwise normalized into the range $[0,1]$ using ImageNet mean and variance R,G,B pixel values
and then zero-padded to $500 \times 500$ size and then downsized to $384 \times 384$.
No data augmentation is performed. We plot the results in \Figref{fig:unet} and
remark that LNB converges very quickly, using the same learning rate as with ViT, and
avoids overfitting. While both optimizers converge to comparable performance on the
\texttt{val} split, the rapid progress by LNB suggests it would be able to leverage
more data effectively.
However, excluding JIT compilation time,
the duration per epoch for LNB and Adam is 2.92 min and 1.27 min, respectively, and
this highlights the trade-off between convergence w.r.t. iterations vs. wall time.
While LNB converged marginally faster in wall time and is significantly
easier to implement to prior equivalent work, it is future work to better understand
in what deep networks does the whitening behavior lead to better generalization as
demonstrated in the other three experiments.


\section{Conclusion}
We have presented Digital Twin Buildings, a framework for extracting the 3D mesh of a building, for connecting the building to Google Maps Platform APIs, and for Multi-Agent Large Language Models data analytics. We demonstrate this by extracting visual description keywords and captions of the building from multi-view multi-scale images of the building. The framework can also be used to process different data modalities sourced from Google Cloud Services. This approach enables richer semantic understanding, seamless integration with geospatial data, and enhanced interaction with real-world structures, paving the way for advanced applications in urban analytics, navigation, and virtual environments.


% \section{Engines}

% To produce a PDF file, pdf\LaTeX{} is strongly recommended (over original \LaTeX{} plus dvips+ps2pdf or dvipdf). Xe\LaTeX{} also produces PDF files, and is especially suitable for text in non-Latin scripts.

% \section{Preamble}

% The first line of the file must be
% \begin{quote}
% \begin{verbatim}
% \documentclass[11pt]{article}
% \end{verbatim}
% \end{quote}

% To load the style file in the review version:
% \begin{quote}
% \begin{verbatim}
% \usepackage[review]{acl}
% \end{verbatim}
% \end{quote}
% For the final version, omit the \verb|review| option:
% \begin{quote}
% \begin{verbatim}
% \usepackage{acl}
% \end{verbatim}
% \end{quote}

% To use Times Roman, put the following in the preamble:
% \begin{quote}
% \begin{verbatim}
% \usepackage{times}
% \end{verbatim}
% \end{quote}
% (Alternatives like txfonts or newtx are also acceptable.)

% Please see the \LaTeX{} source of this document for comments on other packages that may be useful.

% Set the title and author using \verb|\title| and \verb|\author|. Within the author list, format multiple authors using \verb|\and| and \verb|\And| and \verb|\AND|; please see the \LaTeX{} source for examples.

% By default, the box containing the title and author names is set to the minimum of 5 cm. If you need more space, include the following in the preamble:
% \begin{quote}
% \begin{verbatim}
% \setlength\titlebox{<dim>}
% \end{verbatim}
% \end{quote}
% where \verb|<dim>| is replaced with a length. Do not set this length smaller than 5 cm.

% \section{Document Body}

% \subsection{Footnotes}

% Footnotes are inserted with the \verb|\footnote| command.\footnote{This is a footnote.}

% \subsection{Tables and figures}

% See Table~\ref{tab:accents} for an example of a table and its caption.
% \textbf{Do not override the default caption sizes.}

% \begin{table}
%   \centering
%   \begin{tabular}{lc}
%     \hline
%     \textbf{Command} & \textbf{Output} \\
%     \hline
%     \verb|{\"a}|     & {\"a}           \\
%     \verb|{\^e}|     & {\^e}           \\
%     \verb|{\`i}|     & {\`i}           \\
%     \verb|{\.I}|     & {\.I}           \\
%     \verb|{\o}|      & {\o}            \\
%     \verb|{\'u}|     & {\'u}           \\
%     \verb|{\aa}|     & {\aa}           \\\hline
%   \end{tabular}
  % \begin{tabular}{lc}
  %   \hline
  %   \textbf{Command} & \textbf{Output} \\
  %   \hline
  %   \verb|{\c c}|    & {\c c}          \\
  %   \verb|{\u g}|    & {\u g}          \\
  %   \verb|{\l}|      & {\l}            \\
  %   \verb|{\~n}|     & {\~n}           \\
  %   \verb|{\H o}|    & {\H o}          \\
  %   \verb|{\v r}|    & {\v r}          \\
  %   \verb|{\ss}|     & {\ss}           \\
  %   \hline
  % \end{tabular}
%   \caption{Example commands for accented characters, to be used in, \emph{e.g.}, Bib\TeX{} entries.}
%   \label{tab:accents}
% \end{table}

% As much as possible, fonts in figures should conform
% to the document fonts. See Figure~\ref{fig:experiments} for an example of a figure and its caption.

% Using the \verb|graphicx| package graphics files can be included within figure
% environment at an appropriate point within the text.
% The \verb|graphicx| package supports various optional arguments to control the
% appearance of the figure.
% You must include it explicitly in the \LaTeX{} preamble (after the
% \verb|\documentclass| declaration and before \verb|\begin{document}|) using
% \verb|\usepackage{graphicx}|.





% \subsection{Hyperlinks}

% Users of older versions of \LaTeX{} may encounter the following error during compilation:
% \begin{quote}
% \verb|\pdfendlink| ended up in different nesting level than \verb|\pdfstartlink|.
% \end{quote}
% This happens when pdf\LaTeX{} is used and a citation splits across a page boundary. The best way to fix this is to upgrade \LaTeX{} to 2018-12-01 or later.

% \subsection{Citations}

% \begin{table*}
%   \centering
%   \begin{tabular}{lll}
%     \hline
%     \textbf{Output}           & \textbf{natbib command} & \textbf{ACL only command} \\
%     \hline
%     \citep{Gusfield:97}       & \verb|\citep|           &                           \\
%     \citealp{Gusfield:97}     & \verb|\citealp|         &                           \\
%     \citet{Gusfield:97}       & \verb|\citet|           &                           \\
%     \citeyearpar{Gusfield:97} & \verb|\citeyearpar|     &                           \\
%     \citeposs{Gusfield:97}    &                         & \verb|\citeposs|          \\
%     \hline
%   \end{tabular}
%   \caption{\label{citation-guide}
%     Citation commands supported by the style file.
%     The style is based on the natbib package and supports all natbib citation commands.
%     It also supports commands defined in previous ACL style files for compatibility.
%   }
% \end{table*}

% Table~\ref{citation-guide} shows the syntax supported by the style files.
% We encourage you to use the natbib styles.
% You can use the command \verb|\citet| (cite in text) to get ``author (year)'' citations, like this citation to a paper by \citet{Gusfield:97}.
% You can use the command \verb|\citep| (cite in parentheses) to get ``(author, year)'' citations \citep{Gusfield:97}.
% You can use the command \verb|\citealp| (alternative cite without parentheses) to get ``author, year'' citations, which is useful for using citations within parentheses (e.g. \citealp{Gusfield:97}).

% A possessive citation can be made with the command \verb|\citeposs|.
% This is not a standard natbib command, so it is generally not compatible
% with other style files.

% \subsection{References}

% \nocite{Ando2005,andrew2007scalable,rasooli-tetrault-2015}

% The \LaTeX{} and Bib\TeX{} style files provided roughly follow the American Psychological Association format.
% If your own bib file is named \texttt{custom.bib}, then placing the following before any appendices in your \LaTeX{} file will generate the references section for you:
% \begin{quote}
% \begin{verbatim}
% \bibliography{custom}
% \end{verbatim}
% \end{quote}

% You can obtain the complete ACL Anthology as a Bib\TeX{} file from \url{https://aclweb.org/anthology/anthology.bib.gz}.
% To include both the Anthology and your own .bib file, use the following instead of the above.
% \begin{quote}
% \begin{verbatim}
% \bibliography{anthology,custom}
% \end{verbatim}
% \end{quote}

% Please see Section~\ref{sec:bibtex} for information on preparing Bib\TeX{} files.

% \subsection{Equations}

% An example equation is shown below:
% \begin{equation}
%   \label{eq:example}
%   A = \pi r^2
% \end{equation}

% Labels for equation numbers, sections, subsections, figures and tables
% are all defined with the \verb|\label{label}| command and cross references
% to them are made with the \verb|\ref{label}| command.

% This an example cross-reference to Equation~\ref{eq:example}.

% \subsection{Appendices}

% Use \verb|\appendix| before any appendix section to switch the section numbering over to letters. See Appendix~\ref{sec:appendix} for an example.

% \section{Bib\TeX{} Files}
% \label{sec:bibtex}

% Unicode cannot be used in Bib\TeX{} entries, and some ways of typing special characters can disrupt Bib\TeX's alphabetization. The recommended way of typing special characters is shown in Table~\ref{tab:accents}.

% Please ensure that Bib\TeX{} records contain DOIs or URLs when possible, and for all the ACL materials that you reference.
% Use the \verb|doi| field for DOIs and the \verb|url| field for URLs.
% If a Bib\TeX{} entry has a URL or DOI field, the paper title in the references section will appear as a hyperlink to the paper, using the hyperref \LaTeX{} package.

% \section*{Acknowledgments}

% This document has been adapted
% by Steven Bethard, Ryan Cotterell and Rui Yan
% from the instructions for earlier ACL and NAACL proceedings, including those for
% ACL 2019 by Douwe Kiela and Ivan Vuli\'{c},
% NAACL 2019 by Stephanie Lukin and Alla Roskovskaya,
% ACL 2018 by Shay Cohen, Kevin Gimpel, and Wei Lu,
% NAACL 2018 by Margaret Mitchell and Stephanie Lukin,
% Bib\TeX{} suggestions for (NA)ACL 2017/2018 from Jason Eisner,
% ACL 2017 by Dan Gildea and Min-Yen Kan,
% NAACL 2017 by Margaret Mitchell,
% ACL 2012 by Maggie Li and Michael White,
% ACL 2010 by Jing-Shin Chang and Philipp Koehn,
% ACL 2008 by Johanna D. Moore, Simone Teufel, James Allan, and Sadaoki Furui,
% ACL 2005 by Hwee Tou Ng and Kemal Oflazer,
% ACL 2002 by Eugene Charniak and Dekang Lin,
% and earlier ACL and EACL formats written by several people, including
% John Chen, Henry S. Thompson and Donald Walker.
% Additional elements were taken from the formatting instructions of the \emph{International Joint Conference on Artificial Intelligence} and the \emph{Conference on Computer Vision and Pattern Recognition}.

% Bibliography entries for the entire Anthology, followed by custom entries
% \bibliography{anthology,custom}
% Custom bibliography entries only
\bibliography{custom}

\appendix

\section{Appendix}
\label{sec:appendix}


% \begin{table*}[t]\scriptsize
% \renewcommand{\arraystretch}{1.2}

%     % \begin{subtable*}{0.47\textwidth}
%     %     \centering
%     %     \setlength{\tabcolsep}{0.9mm} 
%     %     \hline
%     %     \begin{tabular}{ll}
%     %          task  & dataset \\
%     %          \hline
%     %                                       & big\_font,IIT-CDIP,chart2dict,ChartSFT \\
%     %                                       & DocumentText,hand\_write \\
%     %                                       & html\_zh\_lay,html\_zh\_text \\
%     %                                       & receipt\_EATEN,receipt\_by\_ei \\
%     %                                       & SceneText,table\_markdown,table2md(en \& zh) \\
%     %          \multirow{-6}{*}{Others}      & double\_column\_render(en \& zh) \\
    
    
%     %                                       & Laion-EN (en)~\cite{schuhmann2022laion5b}, Laion-ZH (zh)~\cite{schuhmann2022laion5b}\\
%     %                                       & COYO (zh)~\cite{byeon2022coyo}, \\ 
%     %          \multirow{-3}{*}{Captioning} & GRIT (zh)~\cite{peng2023kosmos2}, COCO (en)~\cite{chen2015cococaption}, TextCaps (en)~\cite{sidorov2020textcaps}   \\
    
    
             
%     %          \rowcolor{gray!15}
%     %                                       & DocStruct4M-bbox,docstruct4M\_check\_wo\_nat,html\_zh\_bbox,synthdog\_bbox(en \& zh) \\
%     %          \rowcolor{gray!15}
%     %                                       & Objects365 (en\&zh)~\cite{shao2019objects365}, GRIT (en\&zh)~\cite{peng2023kosmos2},      \\
%     %          \rowcolor{gray!15}
%     %          \multirow{-3}{*}{Detection}  & All-Seeing (en\&zh)~\cite{wang2023allseeing} \\
    
             
%     %                                       & ANYWORD \\    
%     %                                       & Wukong-OCR (zh)~\cite{gu2022wukong}, LaionCOCO-OCR (en)~\cite{schuhmann2022laioncoco},                      \\
%     %          \multirow{-3}{*}{OCR (large)}& Common Crawl PDF \\
%     %          \rowcolor{gray!15}
%     %                                       & stvqa\_ocr,\\
%     %          \rowcolor{gray!15}
%     %                                       & MMC-Inst , LSVT (zh)~\cite{sun2019lsvt}, ST-VQA (en)~\cite{biten2019stvqa}        \\
%     %          \rowcolor{gray!15}
%     %                                       & RCTW-17 (zh)~\cite{shi2017rctw17}, ReCTs (zh)~\cite{zhang2019rects}, ArT (en\&zh)~\cite{chng2019art},       \\
%     %          \rowcolor{gray!15}
%     %                                       & SynthDoG (en\&zh)~\cite{kim2022synthdog}, COCO-Text (en)~\cite{veit2016cocotext},                           \\
%     %          \rowcolor{gray!15}
%     %                                       & ChartQA-OCR, CTW-OCR, DocVQA-OCR    \\
%     %          \rowcolor{gray!15}
%     %           \multirow{-6}{*}{OCR (small)}& TextOCR, PlotQA-OCR, InfoVQA-OCR        \\
%     %     \hline
%     %     \end{tabular}
%     %     \caption{Datasets used in the pre-training stage. 
%     % }
%     % \label{tab:pretraining}
%     % \end{subtable*}
%     \begin{subtable*}{}
%         \setlength\tabcolsep{6.4pt}
%         \begin{tabular}{l|l}
%     task & dataset \\
%     \hline
%     Captioning                    & TextCaps (en)~\cite{sidorov2020textcaps}, ShareGPT4V (en\&zh)~\cite{chen2023sharegpt4v}                        \\
%     \rowcolor{gray!15}
%                                   & VQAv2 (en)~\cite{goyal2017vqav2}, GQA (en)~\cite{hudson2019gqa}, OKVQA (en)~\cite{marino2019okvqa},            \\
%     \rowcolor{gray!15}
%     \multirow{-2}{*}{General QA}  & VSR (en)~\cite{liu2023vsr}, VisualDialog (en)~\cite{das2017visualdialog}                                       \\
%     \multirow{-1}{*}{Science}     & AI2D (en)~\cite{kembhavi2016ai2d}, ScienceQA (en)~\cite{lu2022scienceqa}, TQA (en)~\cite{kembhavi2017tqa}      \\
%     \rowcolor{gray!15}
%                                   & ChartQA (en)~\cite{masry2022chartqa}, MMC-Inst (en)~\cite{liu2023mmcinst}, DVQA (en)~\cite{kafle2018dvqa},     \\
%     \rowcolor{gray!15}
%     \multirow{-2}{*}{Chart}       & PlotQA (en)~\cite{methani2020plotqa}, LRV-Instruction (en)~\cite{liu2023lrv-instruction}                       \\
    
%                                   & GeoQA+ (en)~\cite{cao2022geoqa_plus}, TabMWP (en)~\cite{lu2022tablemwp}, MathQA (en)~\cite{yu2023mathqa},      \\
%     \multirow{-2}{*}{Mathematics} & CLEVR-Math/Super (en)~\cite{lindstrom2022clevrmath, li2023superclevr}, Geometry3K (en)~\cite{lu2021geometry3k} \\
%     \rowcolor{gray!15}
%                                   & KVQA (en)~\cite{shah2019kvqa}, A-OKVQA (en)~\cite{schwenk2022aokvqa}, ViQuAE (en)~\cite{lerner2022viquae},     \\
%     \rowcolor{gray!15}
%     \multirow{-2}{*}{Knowledge}   & Wikipedia (en\&zh)~\cite{he2023wanjuan}                                                                        \\
%                                   & OCRVQA (en)~\cite{mishra2019ocrvqa}, InfoVQA (en)~\cite{mathew2022infographicvqa}, TextVQA (en)~\cite{singh2019textvqa}, \\ 
%                                   & ArT (en\&zh)~\cite{chng2019art}, COCO-Text (en)~\cite{veit2016cocotext}, CTW (zh)~\cite{yuan2019ctw},          \\
%                                   & LSVT (zh)~\cite{sun2019lsvt}, RCTW-17 (zh)~\cite{shi2017rctw17}, ReCTs (zh)~\cite{zhang2019rects},             \\
%     \multirow{-4}{*}{OCR}         & SynthDoG (en\&zh)~\cite{kim2022synthdog}, ST-VQA (en)~\cite{biten2019stvqa}                                    \\
%     \rowcolor{gray!15}
%     Document                      & DocVQA (en)~\cite{clark2017docqa}, Common Crawl PDF (en\&zh)                                                   \\
%     Grounding                     & RefCOCO/+/g (en)~\cite{yu2016refcoco,mao2016refcocog}, Visual Genome (en)~\cite{krishna2017vg}                            \\
%     \rowcolor{gray!15}
%                                   & LLaVA-150K (en\&zh)~\cite{liu2023llava}, LVIS-Instruct4V (en)~\cite{wang2023lvisinstruct4v},                   \\
%     \rowcolor{gray!15}
%                                   & ALLaVA (en\&zh)~\cite{chen2024allava}, Laion-GPT4V (en)~\cite{laion_gpt4v_dataset},                            \\
%     \rowcolor{gray!15}
%     \multirow{-3}{*}{Conversation}& TextOCR-GPT4V (en)~\cite{textocr_gpt4v_dataset},  SVIT (en\&zh)~\cite{zhao2023svit}                            \\
%                                   & OpenHermes2.5 (en)~\cite{OpenHermes2_5}, Alpaca-GPT4 (en)~\cite{taori2023alpaca},                              \\
%     \multirow{-2}{*}{Text-only}   & ShareGPT (en\&zh)~\cite{zheng2023vicuna}, COIG-CQIA (zh)~\cite{bai2024coig}                                    \\
%         \end{tabular}
%         \centering
%             \caption{Datasets used in the fine-tuning stage.
%         }
%     \label{tab:finetuning}
%     \end{subtable*}
% \caption{\textbf{Summary of datasets used in InternVL 1.5.} 
% To construct large-scale OCR datasets, we utilized PaddleOCR \cite{li2022paddleocr} to perform OCR in Chinese on images from Wukong \cite{gu2022wukong} and in English on images from LAION-COCO \cite{schuhmann2022laioncoco}.
% }
% \label{tab:dataset}
% \end{table*}
% \begin{subtable}{\columnwidth}
%     \centering
%     \begin{tabular}{ccc}
%         \hline
%         \makecell{compress ratio} & Train Data  & Time Cost\\
%         \hline
%         50\% & 1.5M &  \\
%         75\% & 1.5M &  \\
%         \hline
%     \end{tabular}
%     \caption{Time cost of differnt SFT Settings. The time consumption is greatly reduced.}
% \end{subtable}
\subsection{Training settings}
Our FCoT-VL was trained in two distinct stages: re-alignment and post-trian. 
As shown in Table~\ref{tab:training_settings}, we present the training details of FCoT-VL in different stages. The details are as follows:

For both stages, we train models with 64 ascend 910 NPUs with the packed batch size is set to 512.
In the re-alignment pre-training, we employ a 2 million image-text pairs to learn the projector and compress module. This allows the VLLMs to re-align the compressed visual token with the language token space. Specifically, we craft the optimization tasks of recognizing text in document images and converting charts and tables into pythondict/markdown format. We set the training epoch as 1, which requires approximately 48 hours using 64 NPUs for 2B scale. In the subsequent instruction tuning phase, we make all parameters of FCoT-VL learnable and keep most of the settings unchanged, except context length, training data and training epochs. 
% \begin{table}[htbp]
%     \renewcommand{\arraystretch}{1.2}
%     \centering
%     \resizebox{\columnwidth}{!}{
%         \begin{tabular}{lcc|cc}
%             \hline
%             \textbf{Settings} & \multicolumn{2}{c|}{\textbf{InternVL2-2B}} & \multicolumn{2}{c}{\textbf{InternVL2-8B}} \\
%             \hline
%             & Realignment & SFT & Realignment & SFT \\
%             Trainable & Projector and Compress Module & Full Parameters & Projector and Compress Module & Full Parameters \\
%             Packed Batch Size & 512 & 512 & 512 & 512 \\
%             Learning Rate & $1e^{-5}$ & $1e^{-5}$ & $1e^{-5}$ & $1e^{-5}$ \\

%             Context Length & 4096 & 5120 & 4096 & 5120 \\
%             Image Tile Threshold & 12 & 12 & 12 & 12 \\
%             ViT Drop Path & 0.1 & 0.1& 0.1 & 0.1 \\
%             Weight Decay & 0.01 & 0.01 & 0.01 & 0.01 \\
%             Training Epochs & 1 & 3 & 1 & 3 \\
%             \hline
%             Dataset & Pre-train & Fine-tune & Pre-train & Fine-tune \\
%             Training Examples  & $\sim2M$ & $\sim4.5M$ & $\sim2M$ & $\sim4.5M$ \\
%             \hline
%         \end{tabular}
%     }
%     \caption{Training settings for InternVL2-2B and InternVL2-8B.}
%     \label{tab:training_settings}
% \end{table}

  \begin{table}[htbp]
    \renewcommand{\arraystretch}{1.2}
    \centering
    \resizebox{\columnwidth}{!}{
        \begin{tabular}{l|cc}
            \hline
            \textbf{Settings} & \textbf{Re-alignment} & \textbf{Post-train} \\
            \hline
            \rowcolor{gray!15}
            Trainable & \makecell{Projector,\\ Compress Module} & Full Parameters \\
            Packed Batch Size & 512 & 512  \\
            \rowcolor{gray!15}
            Learning Rate & $1e^{-5}$ & $1e^{-5}$ \\
            Context Length & 4096 & 5120 \\
            \rowcolor{gray!15}
            Image Tile Threshold & 12 & 12 \\
            ViT Drop Path & 0.1 & 0.1 \\
            \rowcolor{gray!15}
            Weight Decay & 0.01 & 0.01  \\
            Training Epochs & 1 & 3 \\
            \hline
            \rowcolor{gray!15}
            Dataset & Pre-train & Fine-tune \\
            Training Examples  & $\sim2M$ & $\sim4.5M$ \\
            \hline
        \end{tabular}
    }
    \caption{Detailed Training settings for InternVL2-2B and InternVL2-8B.}
    \label{tab:training_settings}
\end{table}

% \clearpage


\subsection{Model Capabilities and Qualitative Examples}
In this section, we present some practical examples of our FCoT-VL.
\begin{figure*}[htbp]
    \centering
    \includegraphics[width=\textwidth]{q1.png}  % 图片路径
    \caption{The model excels in understanding scheduling-related queries. Image source:\cite{mathew2021docvqa}}  % 标题
    \label{ex1}  % 标签
\end{figure*}

\begin{figure*}[htbp]
    \centering
    \includegraphics[width=\textwidth]{q2.png}  % 图片路径
    \caption{The model demonstrates excellence in recognizing handwritten text in emails. Image source:\cite{mathew2021docvqa}}  % 标题
    \label{ex2}  % 标签
\end{figure*}

\begin{figure*}[htbp]
    \centering
    \includegraphics[width=\textwidth]{q3.png}  % 图片路径
    \caption{The model demonstrates excellence in recognizing printed text and images in books. Image source:\cite{mathew2021docvqa}}  % 标题
    \label{ex3}  % 标签
\end{figure*}


\begin{figure*}[htbp]
    \centering
    \includegraphics[width=\textwidth]{q4.png}  % 图片路径
    \caption{The model displays an adeptness in understanding line charts. Image source:\cite{mathew2021docvqa}}  % 标题
    \label{ex4}  % 标签
\end{figure*}


\begin{figure*}[htbp]
    \centering
    \includegraphics[width=\textwidth]{q5.png}  % 图片路径
    \caption{The model displays an adeptness in understanding images of natural animals. Image source:\cite{lu2022scienceqa}}  % 标题
    \label{ex5}  % 标签
\end{figure*}





\begin{figure*}[htbp]
    \centering
    \includegraphics[width=\textwidth]{q8.png}  % 图片路径
    \caption{The model displays an adeptness in understanding bar charts. Image source:\cite{masry2022chartqa}}  % 标题
    \label{ex8}  % 标签
\end{figure*}


\begin{figure*}[htbp]
    \centering
    \includegraphics[width=\textwidth]{q9.png}  % 图片路径
    \caption{The model displays an adeptness in understanding curve charts. Image source:\cite{masry2022chartqa}}  % 标题
    \label{ex9}  % 标签
\end{figure*}

\begin{figure*}[htbp]
    \centering
    \includegraphics[width=\textwidth]{q10.png}  % 图片路径
    \caption{The model displays an adeptness in recognizing handwritten Chinese characters.}  % 标题
    \label{ex10}  % 标签
\end{figure*}

\begin{figure*}[htbp]
    \centering
    \includegraphics[width=\textwidth]{q11.png}  % 图片路径
    \caption{The model displays an adeptness in understanding Chinese flight ticket information.Image source:\cite{wang2024qwen2vl}}  % 标题
    \label{ex11}  % 标签
\end{figure*}


\begin{figure*}[htbp]
    \centering
    \includegraphics[width=\textwidth]{q12.png}  % 图片路径
    \caption{The model displays an adeptness in calculating information from Chinese bar charts. }  % 标题
    \label{ex12}  % 标签
\end{figure*}

\begin{figure*}[htbp]
    \centering
    \includegraphics[width=0.25\textwidth]{q6.png}  % 调整图片宽度
    \caption{The model displays an adeptness in understanding posters with dense information. Image source:\cite{mathew2022infographicvqa}}  % 标题
    \label{ex6}  % 标签
\end{figure*}


\begin{figure*}[htbp]
    \centering
    \includegraphics[width=0.23\textwidth]{q7.png}  % 图片路径
    \caption{The model displays an adeptness in understanding posters with intertwined text and images. Image source:\cite{mathew2022infographicvqa}}  % 标题
    \label{ex7}  % 标签
\end{figure*}

\end{document}
