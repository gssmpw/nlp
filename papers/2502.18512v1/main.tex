% This must be in the first 5 lines to tell arXiv to use pdfLaTeX, which is strongly recommended.
\pdfoutput=1
% In particular, the hyperref package requires pdfLaTeX in order to break URLs across lines.

\documentclass[11pt]{article}

% Change "review" to "final" to generate the final (sometimes called camera-ready) version.
% Change to "preprint" to generate a non-anonymous version with page numbers.
\usepackage[preprint]{acl}
\usepackage{amsmath}
% Standard package includes
\usepackage{times}
\usepackage{latexsym}
\usepackage{makecell}
\usepackage{multirow}
\usepackage{colortbl}

\usepackage{caption}
\usepackage{subcaption}
% For proper rendering and hyphenation of words containing Latin characters (including in bib files)
\usepackage[T1]{fontenc}
% For Vietnamese characters
% \usepackage[T5]{fontenc}
% See https://www.latex-project.org/help/documentation/encguide.pdf for other character sets

% This assumes your files are encoded as UTF8
\usepackage[utf8]{inputenc}

% This is not strictly necessary, and may be commented out,
% but it will improve the layout of the manuscript,
% and will typically save some space.
\usepackage{microtype}

% This is also not strictly necessary, and may be commented out.
% However, it will improve the aesthetics of text in
% the typewriter font.
\usepackage{inconsolata}

%Including images in your LaTeX document requires adding
%additional package(s)
\usepackage{graphicx}
% If the title and author information does not fit in the area allocated, uncomment the following
%
%\setlength\titlebox{<dim>}
%
% and set <dim> to something 5cm or larger.

\title{FCoT-VL:Advancing Text-oriented Large Vision-Language Models with Efficient Visual Token Compression}

% Author information can be set in various styles:
% For several authors from the same institution:
% \author{Author 1 \and ... \and Author n \\
%         Address line \\ ... \\ Address line}
% if the names do not fit well on one line use
%         Author 1 \\ {\bf Author 2} \\ ... \\ {\bf Author n} \\
% For authors from different institutions:
% \author{Author 1 \\ Address line \\  ... \\ Address line
%         \And  ... \And
%         Author n \\ Address line \\ ... \\ Address line}
% To start a separate ``row'' of authors use \AND, as in
% \author{Author 1 \\ Address line \\  ... \\ Address line
%         \AND
%         Author 2 \\ Address line \\ ... \\ Address line \And
%         Author 3 \\ Address line \\ ... \\ Address line}

\author{
  \textbf{Jianjian Li \textsuperscript{1}},
  \textbf{Junquan Fan\textsuperscript{1}},
  \textbf{Feng Tang\textsuperscript{2}}\thanks{corresponding author},
  \textbf{Gang Huang\textsuperscript{2}}\footnotemark[1],
  \textbf{Shitao Zhu\textsuperscript{2}},
  \textbf{Songlin Liu\textsuperscript{2}} \\
   \textbf{Nian Xie\textsuperscript{2}},
  \textbf{Wulong Liu\textsuperscript{2}},
  \textbf{Yong Liao\textsuperscript{1}\footnotemark[1]}
\\
  \textsuperscript{1}University of Science and Technology of China.\\
  \textsuperscript{2}Huawei Noah’s Ark Lab.\\
\\
  % \small{
  %   \textbf{Correspondence:} \href{hanjiayi@inspur.com}{hanjiayi@inspur.com}
  % }
%}
}

%\author{
%  \textbf{First Author\textsuperscript{1}},
%  \textbf{Second Author\textsuperscript{1,2}},
%  \textbf{Third T. Author\textsuperscript{1}},
%  \textbf{Fourth Author\textsuperscript{1}},
%\\
%  \textbf{Fifth Author\textsuperscript{1,2}},
%  \textbf{Sixth Author\textsuperscript{1}},
%  \textbf{Seventh Author\textsuperscript{1}},
%  \textbf{Eighth Author \textsuperscript{1,2,3,4}},
%\\
%  \textbf{Ninth Author\textsuperscript{1}},
%  \textbf{Tenth Author\textsuperscript{1}},
%  \textbf{Eleventh E. Author\textsuperscript{1,2,3,4,5}},
%  \textbf{Twelfth Author\textsuperscript{1}},
%\\
%  \textbf{Thirteenth Author\textsuperscript{3}},
%  \textbf{Fourteenth F. Author\textsuperscript{2,4}},
%  \textbf{Fifteenth Author\textsuperscript{1}},
%  \textbf{Sixteenth Author\textsuperscript{1}},
%\\
%  \textbf{Seventeenth S. Author\textsuperscript{4,5}},
%  \textbf{Eighteenth Author\textsuperscript{3,4}},
%  \textbf{Nineteenth N. Author\textsuperscript{2,5}},
%  \textbf{Twentieth Author\textsuperscript{1}}
%\\
%\\
%  \textsuperscript{1}Affiliation 1,
%  \textsuperscript{2}Affiliation 2,
%  \textsuperscript{3}Affiliation 3,
%  \textsuperscript{4}Affiliation 4,
%  \textsuperscript{5}Affiliation 5
%\\
%  \small{
%    \textbf{Correspondence:} \href{mailto:email@domain}{email@domain}
%  }
%}

\begin{document}
\maketitle
\begin{abstract}
We present an image blending pipeline, \textit{IBURD}, that creates realistic synthetic images to assist in the training of deep detectors for use on underwater autonomous vehicles (AUVs) for marine debris detection tasks. 
Specifically, IBURD generates both images of underwater debris and their pixel-level annotations, using source images of debris objects, their annotations, and target background images of marine environments. 
With Poisson editing and style transfer techniques, IBURD is even able to robustly blend transparent objects into arbitrary backgrounds and automatically adjust the style of blended images using the blurriness metric of target background images. 
These generated images of marine debris in actual underwater backgrounds address the data scarcity and data variety problems faced by deep-learned vision algorithms in challenging underwater conditions, and can enable the use of AUVs for environmental cleanup missions. 
Both quantitative and robotic evaluations of IBURD demonstrate the efficacy of the proposed approach for robotic detection of marine debris. 
\end{abstract}




\section{Introduction}

In today’s rapidly evolving digital landscape, the transformative power of web technologies has redefined not only how services are delivered but also how complex tasks are approached. Web-based systems have become increasingly prevalent in risk control across various domains. This widespread adoption is due their accessibility, scalability, and ability to remotely connect various types of users. For example, these systems are used for process safety management in industry~\cite{kannan2016web}, safety risk early warning in urban construction~\cite{ding2013development}, and safe monitoring of infrastructural systems~\cite{repetto2018web}. Within these web-based risk management systems, the source search problem presents a huge challenge. Source search refers to the task of identifying the origin of a risky event, such as a gas leak and the emission point of toxic substances. This source search capability is crucial for effective risk management and decision-making.

Traditional approaches to implementing source search capabilities into the web systems often rely on solely algorithmic solutions~\cite{ristic2016study}. These methods, while relatively straightforward to implement, often struggle to achieve acceptable performances due to algorithmic local optima and complex unknown environments~\cite{zhao2020searching}. More recently, web crowdsourcing has emerged as a promising alternative for tackling the source search problem by incorporating human efforts in these web systems on-the-fly~\cite{zhao2024user}. This approach outsources the task of addressing issues encountered during the source search process to human workers, leveraging their capabilities to enhance system performance.

These solutions often employ a human-AI collaborative way~\cite{zhao2023leveraging} where algorithms handle exploration-exploitation and report the encountered problems while human workers resolve complex decision-making bottlenecks to help the algorithms getting rid of local deadlocks~\cite{zhao2022crowd}. Although effective, this paradigm suffers from two inherent limitations: increased operational costs from continuous human intervention, and slow response times of human workers due to sequential decision-making. These challenges motivate our investigation into developing autonomous systems that preserve human-like reasoning capabilities while reducing dependency on massive crowdsourced labor.

Furthermore, recent advancements in large language models (LLMs)~\cite{chang2024survey} and multi-modal LLMs (MLLMs)~\cite{huang2023chatgpt} have unveiled promising avenues for addressing these challenges. One clear opportunity involves the seamless integration of visual understanding and linguistic reasoning for robust decision-making in search tasks. However, whether large models-assisted source search is really effective and efficient for improving the current source search algorithms~\cite{ji2022source} remains unknown. \textit{To address the research gap, we are particularly interested in answering the following two research questions in this work:}

\textbf{\textit{RQ1: }}How can source search capabilities be integrated into web-based systems to support decision-making in time-sensitive risk management scenarios? 
% \sq{I mention ``time-sensitive'' here because I feel like we shall say something about the response time -- LLM has to be faster than humans}

\textbf{\textit{RQ2: }}How can MLLMs and LLMs enhance the effectiveness and efficiency of existing source search algorithms? 

% \textit{\textbf{RQ2:}} To what extent does the performance of large models-assisted search align with or approach the effectiveness of human-AI collaborative search? 

To answer the research questions, we propose a novel framework called Auto-\
S$^2$earch (\textbf{Auto}nomous \textbf{S}ource \textbf{Search}) and implement a prototype system that leverages advanced web technologies to simulate real-world conditions for zero-shot source search. Unlike traditional methods that rely on pre-defined heuristics or extensive human intervention, AutoS$^2$earch employs a carefully designed prompt that encapsulates human rationales, thereby guiding the MLLM to generate coherent and accurate scene descriptions from visual inputs about four directional choices. Based on these language-based descriptions, the LLM is enabled to determine the optimal directional choice through chain-of-thought (CoT) reasoning. Comprehensive empirical validation demonstrates that AutoS$^2$-\ 
earch achieves a success rate of 95–98\%, closely approaching the performance of human-AI collaborative search across 20 benchmark scenarios~\cite{zhao2023leveraging}. 

Our work indicates that the role of humans in future web crowdsourcing tasks may evolve from executors to validators or supervisors. Furthermore, incorporating explanations of LLM decisions into web-based system interfaces has the potential to help humans enhance task performance in risk control.







\section{Related Works}
\subsection{Vision Large Language Models}


%%丰富类内容

%% 应该包含 LLaVA , Gemini,internvl2,qwenvl2,deepseek-v
In recent years, open-source  VLLMs have made significant advancements, driven by contributions from both academia and industry. 
Earlier models, such as BLIP-2 \cite{li2023blip}, MiniGPT\cite{zhu2023minigpt} and LLaVA\cite{liu2024visual,liu2024llava}, have proven to be effective for vision-language tasks via bridging off-the-shelf ViTs and LLMs. 
However, early VLLMs struggle with processing images containing
fine-grained details, especially for OCR-like tasks such as charts\cite{masry2022chartqa}, documents\cite{mathew2021docvqa},
and infographics\cite{mathew2022infographicvqa}. To this end, InternVL series propose
an adaptive cropping method to convert vanilla images as several fixed image patches. For example, InternLM-XComposer2-4KHD\cite{dong2024internlm} increases 336 pixels of CLIP to 4K resolution and gets strong document understanding ability. InternVL2
obtains promising results on text-oriented benchmarks via scaling up image resolution and ViT model parameters. Moreover, QwenVL2 \cite{wang2024qwen2vl} proposes a native dynamic processing of images at varying resolutions. This image processing setting generates more visual tokens and suppress adaptive cropping VLLMs. However,
high-resolution processing pipelines bring substantial computational overhead in both training and inference stages, hindering real-world deployment.



% Following works scale up image resolution to enhance the capabilities of fine-grained perception\cite{chen2024internvl,chen2024far,hong2024cogvlm2,glm2024chatglm,liu2024llavanext,wang2024qwen2vl}.
% InternVL families\cite{chen2024internvl,chen2024far} introduce a dynamic resolution cropping scheme to generate dynamic visual tokens, enhancing the capture of fine-grained details at the cost of higher visual tokens. However,
% this brings substantial computational overhead in both training and inference stages, hindering real-world deployment.



Beyond high-resolution tricks, many works reveal that high-quality datas are more important for advancing document understanding. Recent studies\cite{hu2024minicpm,li2024baichuan,li2025eagle} highlight the critical role of data quality in VLLMs. For instance, InternVL-2.5\cite{chen2024expanding} enhanced performance of previous version through collecting more diverse dataset and data processing pipelines.

In this paper, we also explore how to obtain high-quality post-training datas to match frontier open-source VLLMs. Specifically, Our FCoT-VL outperforms the base model InternVL2
on many benchmarks like ChartQA\cite{masry2022chartqa} and MathVista\cite{lu2024mathvista}, despite reducing visual tokens by 50$\%$.



\begin{figure*}[t]
  \centering
  \includegraphics[width=\textwidth]{overview.pdf}
  \caption{Overall Structure of FCoT-VL. FCoT-VL is a self-distillation architecture in which only the Student-Projector and Compress-Module are learned, while all the other modules remain frozen. The student and teacher models share the same ViT encoder and the LLM decoder.}
  \label{fig:Overvieww}
\end{figure*}
\subsection{Visual Compression Schemes}
% Introduce Compression Scheme Evolution from Vit to Vllm
% Since the appearance of Vision Transformer (VIT), many schemes on compressing the number of tokens extracted by VITs are proposed by different scholars to reduce visual infomation redundancy. 
Visual compression, a key focus in high-resolution VLLMs, aims to efficiently reduce the use of vision tokens, minimizing computational and memory overheads. The inherent redundancy of visual data, compared to dense textual data, underscores the importance of compression.

Solutions to visual compression can be broadly categorized into two main approaches: training-free and training-based ones. Training-free methods dynamically select more important vision tokens via various strategies during decoding stage. For instance, SparseVLM\cite{zhang2024sparsevlm} and VisionZip \cite{yang2024visionzip} prioritize tokens based on attention scores. ToMe\cite{bolya2022token} and LLaVA-PruMerge\cite{shang2024llava} cluster tokens using cosine similarity. However, the training-free paradigms suffer from significant performance drops in text-orientated benchmarks. 
In contrast, training-based methods focus on optimizing the visual adaptor by incorporating external modules for token reducing. For instance, LLaMA-VID\cite{li2024llama} enhances visual information extraction through Q-Former\cite{li2023blip} with context tokens. Similarly, models like C-Abstractor\cite{cha2024honeybee} and LDP\cite{chu2024mobilevlm} serve as promising alternatives for visual token compressing.

Training from scratch necessitates extensive alignment datasets and substantial computational resources, often consuming thousands of GPU days. In this work, we present an efficient training token-compressing framework that achieves comparable performance while significantly reducing both data and computational requirements.



% Early works focus on projectors in VLLMs , like QwenVL (\citeauthor{bai2023qwen}), applying the Qformer-like structure which utilize learnable queries to extract dense information  by cross-attention layers. Recently, Tokenpacker (\citeauthor{li2024tokenpacker}) use multi-level features as queries and values to improve the performance. Some other works, like C-abstractor (\citeauthor{cha2024honeybee}) and LDP (\citeauthor{chu2024mobilevlm}), prove that convolution layers can generate compressed tokens for VLLMs.

% Some recent works leverage merging tokens to compress visual information based on similarity. For instance, ToMe (\citeauthor{bolya2022token}) divides visual tokens into two sets, connects the most similar tokens based on cosine similarity between sets and averaging features before concatenating the sets; LLaVA-PruMerge (\citeauthor{shang2024llava}) dynamically selects the most important visual tokens and merges similar ones using a clustering approach. Besides, some works pay more attention to the attention score of models. SparseVLM (\cite{zhang2024sparsevlm}) selects the relevant text rater based on the attention score between vision and text, and then selects the more important visual token based on the attention score between the rater and vision. VisionZip (\cite{yang2024visionzip}) focuses on selecting the most informative token by the attention score and discarding the less informative tokens to reduce the number of overall tokens. But most of current works are based on LLaVa-1.5 which is not high-resolution schema. Few works, like TextHawk2 (\cite{yu2024texthawk2}), did some research on high-resolution schema.

% Except in the vit stage, some works propose token pruning methods happening in the LLM stage. FastV (\citeauthor{chen2025image}) ranks tokens by the average attention value with all other tokens, and prunes out the last R tokens. Additionally, DocKylin (\citeauthor{zhang2024dockylin}) also provide a method in the data processing stage called Adaptive Pixel Slimming (APS) that reduce the resolution of images.
 
% \subsection{Model merge}
% In recent years, model merging has emerged as a prominent research area, focusing on combining multiple task-specific models into a single, versatile model with broad capabilities (\citeauthor{wortsman2022model,yan2024corrective}). Unlike multi-task learning (\citeauthor{crawshaw2020multi}), which also seeks to develop a single model capable of handling multiple tasks, model merging prioritizes the integration of model parameters without the need for access to the original training data (\citeauthor{matena2111merging}). One well-known approach to model merging is Average Merging (\citeauthor{wortsman2022model}), where the merged model is constructed by averaging the parameters of the individual models. Another method, Task Arithmetic (\citeauthor{ilharco2022editing}), incorporates a pre-defined scaling factor to adjust the influence of different models. Fisher Merging (\citeauthor{matena2111merging}) applies weighted averaging of parameters, with the weights determined by the Fisher information matrix (\citeauthor{matena2111merging}). In contrast, RegMean (\citeauthor{jin2022dataless}) solves the merging problem by optimizing a linear regression framework with closed-form solutions. Lastly, TIES-Merging (\citeauthor{yadav2023resolving}) addresses task conflicts identified (\citeauthor{ilharco2022editing}) by trimming parameters with low magnitudes, resolving sign inconsistencies, and merging parameters with consistent signs in a disjoint manner.

    

\section{Preliminaries}
\label{Preliminaries}
\begin{figure*}[t]
    \centering
    \includegraphics[width=0.95\linewidth]{fig/HealthGPT_Framework.png}
    \caption{The \ourmethod{} architecture integrates hierarchical visual perception and H-LoRA, employing a task-specific hard router to select visual features and H-LoRA plugins, ultimately generating outputs with an autoregressive manner.}
    \label{fig:architecture}
\end{figure*}
\noindent\textbf{Large Vision-Language Models.} 
The input to a LVLM typically consists of an image $x^{\text{img}}$ and a discrete text sequence $x^{\text{txt}}$. The visual encoder $\mathcal{E}^{\text{img}}$ converts the input image $x^{\text{img}}$ into a sequence of visual tokens $\mathcal{V} = [v_i]_{i=1}^{N_v}$, while the text sequence $x^{\text{txt}}$ is mapped into a sequence of text tokens $\mathcal{T} = [t_i]_{i=1}^{N_t}$ using an embedding function $\mathcal{E}^{\text{txt}}$. The LLM $\mathcal{M_\text{LLM}}(\cdot|\theta)$ models the joint probability of the token sequence $\mathcal{U} = \{\mathcal{V},\mathcal{T}\}$, which is expressed as:
\begin{equation}
    P_\theta(R | \mathcal{U}) = \prod_{i=1}^{N_r} P_\theta(r_i | \{\mathcal{U}, r_{<i}\}),
\end{equation}
where $R = [r_i]_{i=1}^{N_r}$ is the text response sequence. The LVLM iteratively generates the next token $r_i$ based on $r_{<i}$. The optimization objective is to minimize the cross-entropy loss of the response $\mathcal{R}$.
% \begin{equation}
%     \mathcal{L}_{\text{VLM}} = \mathbb{E}_{R|\mathcal{U}}\left[-\log P_\theta(R | \mathcal{U})\right]
% \end{equation}
It is worth noting that most LVLMs adopt a design paradigm based on ViT, alignment adapters, and pre-trained LLMs\cite{liu2023llava,liu2024improved}, enabling quick adaptation to downstream tasks.


\noindent\textbf{VQGAN.}
VQGAN~\cite{esser2021taming} employs latent space compression and indexing mechanisms to effectively learn a complete discrete representation of images. VQGAN first maps the input image $x^{\text{img}}$ to a latent representation $z = \mathcal{E}(x)$ through a encoder $\mathcal{E}$. Then, the latent representation is quantized using a codebook $\mathcal{Z} = \{z_k\}_{k=1}^K$, generating a discrete index sequence $\mathcal{I} = [i_m]_{m=1}^N$, where $i_m \in \mathcal{Z}$ represents the quantized code index:
\begin{equation}
    \mathcal{I} = \text{Quantize}(z|\mathcal{Z}) = \arg\min_{z_k \in \mathcal{Z}} \| z - z_k \|_2.
\end{equation}
In our approach, the discrete index sequence $\mathcal{I}$ serves as a supervisory signal for the generation task, enabling the model to predict the index sequence $\hat{\mathcal{I}}$ from input conditions such as text or other modality signals.  
Finally, the predicted index sequence $\hat{\mathcal{I}}$ is upsampled by the VQGAN decoder $G$, generating the high-quality image $\hat{x}^\text{img} = G(\hat{\mathcal{I}})$.



\noindent\textbf{Low Rank Adaptation.} 
LoRA\cite{hu2021lora} effectively captures the characteristics of downstream tasks by introducing low-rank adapters. The core idea is to decompose the bypass weight matrix $\Delta W\in\mathbb{R}^{d^{\text{in}} \times d^{\text{out}}}$ into two low-rank matrices $ \{A \in \mathbb{R}^{d^{\text{in}} \times r}, B \in \mathbb{R}^{r \times d^{\text{out}}} \}$, where $ r \ll \min\{d^{\text{in}}, d^{\text{out}}\} $, significantly reducing learnable parameters. The output with the LoRA adapter for the input $x$ is then given by:
\begin{equation}
    h = x W_0 + \alpha x \Delta W/r = x W_0 + \alpha xAB/r,
\end{equation}
where matrix $ A $ is initialized with a Gaussian distribution, while the matrix $ B $ is initialized as a zero matrix. The scaling factor $ \alpha/r $ controls the impact of $ \Delta W $ on the model.

\section{HealthGPT}
\label{Method}


\subsection{Unified Autoregressive Generation.}  
% As shown in Figure~\ref{fig:architecture}, 
\ourmethod{} (Figure~\ref{fig:architecture}) utilizes a discrete token representation that covers both text and visual outputs, unifying visual comprehension and generation as an autoregressive task. 
For comprehension, $\mathcal{M}_\text{llm}$ receives the input joint sequence $\mathcal{U}$ and outputs a series of text token $\mathcal{R} = [r_1, r_2, \dots, r_{N_r}]$, where $r_i \in \mathcal{V}_{\text{txt}}$, and $\mathcal{V}_{\text{txt}}$ represents the LLM's vocabulary:
\begin{equation}
    P_\theta(\mathcal{R} \mid \mathcal{U}) = \prod_{i=1}^{N_r} P_\theta(r_i \mid \mathcal{U}, r_{<i}).
\end{equation}
For generation, $\mathcal{M}_\text{llm}$ first receives a special start token $\langle \text{START\_IMG} \rangle$, then generates a series of tokens corresponding to the VQGAN indices $\mathcal{I} = [i_1, i_2, \dots, i_{N_i}]$, where $i_j \in \mathcal{V}_{\text{vq}}$, and $\mathcal{V}_{\text{vq}}$ represents the index range of VQGAN. Upon completion of generation, the LLM outputs an end token $\langle \text{END\_IMG} \rangle$:
\begin{equation}
    P_\theta(\mathcal{I} \mid \mathcal{U}) = \prod_{j=1}^{N_i} P_\theta(i_j \mid \mathcal{U}, i_{<j}).
\end{equation}
Finally, the generated index sequence $\mathcal{I}$ is fed into the decoder $G$, which reconstructs the target image $\hat{x}^{\text{img}} = G(\mathcal{I})$.

\subsection{Hierarchical Visual Perception}  
Given the differences in visual perception between comprehension and generation tasks—where the former focuses on abstract semantics and the latter emphasizes complete semantics—we employ ViT to compress the image into discrete visual tokens at multiple hierarchical levels.
Specifically, the image is converted into a series of features $\{f_1, f_2, \dots, f_L\}$ as it passes through $L$ ViT blocks.

To address the needs of various tasks, the hidden states are divided into two types: (i) \textit{Concrete-grained features} $\mathcal{F}^{\text{Con}} = \{f_1, f_2, \dots, f_k\}, k < L$, derived from the shallower layers of ViT, containing sufficient global features, suitable for generation tasks; 
(ii) \textit{Abstract-grained features} $\mathcal{F}^{\text{Abs}} = \{f_{k+1}, f_{k+2}, \dots, f_L\}$, derived from the deeper layers of ViT, which contain abstract semantic information closer to the text space, suitable for comprehension tasks.

The task type $T$ (comprehension or generation) determines which set of features is selected as the input for the downstream large language model:
\begin{equation}
    \mathcal{F}^{\text{img}}_T =
    \begin{cases}
        \mathcal{F}^{\text{Con}}, & \text{if } T = \text{generation task} \\
        \mathcal{F}^{\text{Abs}}, & \text{if } T = \text{comprehension task}
    \end{cases}
\end{equation}
We integrate the image features $\mathcal{F}^{\text{img}}_T$ and text features $\mathcal{T}$ into a joint sequence through simple concatenation, which is then fed into the LLM $\mathcal{M}_{\text{llm}}$ for autoregressive generation.
% :
% \begin{equation}
%     \mathcal{R} = \mathcal{M}_{\text{llm}}(\mathcal{U}|\theta), \quad \mathcal{U} = [\mathcal{F}^{\text{img}}_T; \mathcal{T}]
% \end{equation}
\subsection{Heterogeneous Knowledge Adaptation}
We devise H-LoRA, which stores heterogeneous knowledge from comprehension and generation tasks in separate modules and dynamically routes to extract task-relevant knowledge from these modules. 
At the task level, for each task type $ T $, we dynamically assign a dedicated H-LoRA submodule $ \theta^T $, which is expressed as:
\begin{equation}
    \mathcal{R} = \mathcal{M}_\text{LLM}(\mathcal{U}|\theta, \theta^T), \quad \theta^T = \{A^T, B^T, \mathcal{R}^T_\text{outer}\}.
\end{equation}
At the feature level for a single task, H-LoRA integrates the idea of Mixture of Experts (MoE)~\cite{masoudnia2014mixture} and designs an efficient matrix merging and routing weight allocation mechanism, thus avoiding the significant computational delay introduced by matrix splitting in existing MoELoRA~\cite{luo2024moelora}. Specifically, we first merge the low-rank matrices (rank = r) of $ k $ LoRA experts into a unified matrix:
\begin{equation}
    \mathbf{A}^{\text{merged}}, \mathbf{B}^{\text{merged}} = \text{Concat}(\{A_i\}_1^k), \text{Concat}(\{B_i\}_1^k),
\end{equation}
where $ \mathbf{A}^{\text{merged}} \in \mathbb{R}^{d^\text{in} \times rk} $ and $ \mathbf{B}^{\text{merged}} \in \mathbb{R}^{rk \times d^\text{out}} $. The $k$-dimension routing layer generates expert weights $ \mathcal{W} \in \mathbb{R}^{\text{token\_num} \times k} $ based on the input hidden state $ x $, and these are expanded to $ \mathbb{R}^{\text{token\_num} \times rk} $ as follows:
\begin{equation}
    \mathcal{W}^\text{expanded} = \alpha k \mathcal{W} / r \otimes \mathbf{1}_r,
\end{equation}
where $ \otimes $ denotes the replication operation.
The overall output of H-LoRA is computed as:
\begin{equation}
    \mathcal{O}^\text{H-LoRA} = (x \mathbf{A}^{\text{merged}} \odot \mathcal{W}^\text{expanded}) \mathbf{B}^{\text{merged}},
\end{equation}
where $ \odot $ represents element-wise multiplication. Finally, the output of H-LoRA is added to the frozen pre-trained weights to produce the final output:
\begin{equation}
    \mathcal{O} = x W_0 + \mathcal{O}^\text{H-LoRA}.
\end{equation}
% In summary, H-LoRA is a task-based dynamic PEFT method that achieves high efficiency in single-task fine-tuning.

\subsection{Training Pipeline}

\begin{figure}[t]
    \centering
    \hspace{-4mm}
    \includegraphics[width=0.94\linewidth]{fig/data.pdf}
    \caption{Data statistics of \texttt{VL-Health}. }
    \label{fig:data}
\end{figure}
\noindent \textbf{1st Stage: Multi-modal Alignment.} 
In the first stage, we design separate visual adapters and H-LoRA submodules for medical unified tasks. For the medical comprehension task, we train abstract-grained visual adapters using high-quality image-text pairs to align visual embeddings with textual embeddings, thereby enabling the model to accurately describe medical visual content. During this process, the pre-trained LLM and its corresponding H-LoRA submodules remain frozen. In contrast, the medical generation task requires training concrete-grained adapters and H-LoRA submodules while keeping the LLM frozen. Meanwhile, we extend the textual vocabulary to include multimodal tokens, enabling the support of additional VQGAN vector quantization indices. The model trains on image-VQ pairs, endowing the pre-trained LLM with the capability for image reconstruction. This design ensures pixel-level consistency of pre- and post-LVLM. The processes establish the initial alignment between the LLM’s outputs and the visual inputs.

\noindent \textbf{2nd Stage: Heterogeneous H-LoRA Plugin Adaptation.}  
The submodules of H-LoRA share the word embedding layer and output head but may encounter issues such as bias and scale inconsistencies during training across different tasks. To ensure that the multiple H-LoRA plugins seamlessly interface with the LLMs and form a unified base, we fine-tune the word embedding layer and output head using a small amount of mixed data to maintain consistency in the model weights. Specifically, during this stage, all H-LoRA submodules for different tasks are kept frozen, with only the word embedding layer and output head being optimized. Through this stage, the model accumulates foundational knowledge for unified tasks by adapting H-LoRA plugins.

\begin{table*}[!t]
\centering
\caption{Comparison of \ourmethod{} with other LVLMs and unified multi-modal models on medical visual comprehension tasks. \textbf{Bold} and \underline{underlined} text indicates the best performance and second-best performance, respectively.}
\resizebox{\textwidth}{!}{
\begin{tabular}{c|lcc|cccccccc|c}
\toprule
\rowcolor[HTML]{E9F3FE} &  &  &  & \multicolumn{2}{c}{\textbf{VQA-RAD \textuparrow}} & \multicolumn{2}{c}{\textbf{SLAKE \textuparrow}} & \multicolumn{2}{c}{\textbf{PathVQA \textuparrow}} &  &  &  \\ 
\cline{5-10}
\rowcolor[HTML]{E9F3FE}\multirow{-2}{*}{\textbf{Type}} & \multirow{-2}{*}{\textbf{Model}} & \multirow{-2}{*}{\textbf{\# Params}} & \multirow{-2}{*}{\makecell{\textbf{Medical} \\ \textbf{LVLM}}} & \textbf{close} & \textbf{all} & \textbf{close} & \textbf{all} & \textbf{close} & \textbf{all} & \multirow{-2}{*}{\makecell{\textbf{MMMU} \\ \textbf{-Med}}\textuparrow} & \multirow{-2}{*}{\textbf{OMVQA}\textuparrow} & \multirow{-2}{*}{\textbf{Avg. \textuparrow}} \\ 
\midrule \midrule
\multirow{9}{*}{\textbf{Comp. Only}} 
& Med-Flamingo & 8.3B & \Large \ding{51} & 58.6 & 43.0 & 47.0 & 25.5 & 61.9 & 31.3 & 28.7 & 34.9 & 41.4 \\
& LLaVA-Med & 7B & \Large \ding{51} & 60.2 & 48.1 & 58.4 & 44.8 & 62.3 & 35.7 & 30.0 & 41.3 & 47.6 \\
& HuatuoGPT-Vision & 7B & \Large \ding{51} & 66.9 & 53.0 & 59.8 & 49.1 & 52.9 & 32.0 & 42.0 & 50.0 & 50.7 \\
& BLIP-2 & 6.7B & \Large \ding{55} & 43.4 & 36.8 & 41.6 & 35.3 & 48.5 & 28.8 & 27.3 & 26.9 & 36.1 \\
& LLaVA-v1.5 & 7B & \Large \ding{55} & 51.8 & 42.8 & 37.1 & 37.7 & 53.5 & 31.4 & 32.7 & 44.7 & 41.5 \\
& InstructBLIP & 7B & \Large \ding{55} & 61.0 & 44.8 & 66.8 & 43.3 & 56.0 & 32.3 & 25.3 & 29.0 & 44.8 \\
& Yi-VL & 6B & \Large \ding{55} & 52.6 & 42.1 & 52.4 & 38.4 & 54.9 & 30.9 & 38.0 & 50.2 & 44.9 \\
& InternVL2 & 8B & \Large \ding{55} & 64.9 & 49.0 & 66.6 & 50.1 & 60.0 & 31.9 & \underline{43.3} & 54.5 & 52.5\\
& Llama-3.2 & 11B & \Large \ding{55} & 68.9 & 45.5 & 72.4 & 52.1 & 62.8 & 33.6 & 39.3 & 63.2 & 54.7 \\
\midrule
\multirow{5}{*}{\textbf{Comp. \& Gen.}} 
& Show-o & 1.3B & \Large \ding{55} & 50.6 & 33.9 & 31.5 & 17.9 & 52.9 & 28.2 & 22.7 & 45.7 & 42.6 \\
& Unified-IO 2 & 7B & \Large \ding{55} & 46.2 & 32.6 & 35.9 & 21.9 & 52.5 & 27.0 & 25.3 & 33.0 & 33.8 \\
& Janus & 1.3B & \Large \ding{55} & 70.9 & 52.8 & 34.7 & 26.9 & 51.9 & 27.9 & 30.0 & 26.8 & 33.5 \\
& \cellcolor[HTML]{DAE0FB}HealthGPT-M3 & \cellcolor[HTML]{DAE0FB}3.8B & \cellcolor[HTML]{DAE0FB}\Large \ding{51} & \cellcolor[HTML]{DAE0FB}\underline{73.7} & \cellcolor[HTML]{DAE0FB}\underline{55.9} & \cellcolor[HTML]{DAE0FB}\underline{74.6} & \cellcolor[HTML]{DAE0FB}\underline{56.4} & \cellcolor[HTML]{DAE0FB}\underline{78.7} & \cellcolor[HTML]{DAE0FB}\underline{39.7} & \cellcolor[HTML]{DAE0FB}\underline{43.3} & \cellcolor[HTML]{DAE0FB}\underline{68.5} & \cellcolor[HTML]{DAE0FB}\underline{61.3} \\
& \cellcolor[HTML]{DAE0FB}HealthGPT-L14 & \cellcolor[HTML]{DAE0FB}14B & \cellcolor[HTML]{DAE0FB}\Large \ding{51} & \cellcolor[HTML]{DAE0FB}\textbf{77.7} & \cellcolor[HTML]{DAE0FB}\textbf{58.3} & \cellcolor[HTML]{DAE0FB}\textbf{76.4} & \cellcolor[HTML]{DAE0FB}\textbf{64.5} & \cellcolor[HTML]{DAE0FB}\textbf{85.9} & \cellcolor[HTML]{DAE0FB}\textbf{44.4} & \cellcolor[HTML]{DAE0FB}\textbf{49.2} & \cellcolor[HTML]{DAE0FB}\textbf{74.4} & \cellcolor[HTML]{DAE0FB}\textbf{66.4} \\
\bottomrule
\end{tabular}
}
\label{tab:results}
\end{table*}
\begin{table*}[ht]
    \centering
    \caption{The experimental results for the four modality conversion tasks.}
    \resizebox{\textwidth}{!}{
    \begin{tabular}{l|ccc|ccc|ccc|ccc}
        \toprule
        \rowcolor[HTML]{E9F3FE} & \multicolumn{3}{c}{\textbf{CT to MRI (Brain)}} & \multicolumn{3}{c}{\textbf{CT to MRI (Pelvis)}} & \multicolumn{3}{c}{\textbf{MRI to CT (Brain)}} & \multicolumn{3}{c}{\textbf{MRI to CT (Pelvis)}} \\
        \cline{2-13}
        \rowcolor[HTML]{E9F3FE}\multirow{-2}{*}{\textbf{Model}}& \textbf{SSIM $\uparrow$} & \textbf{PSNR $\uparrow$} & \textbf{MSE $\downarrow$} & \textbf{SSIM $\uparrow$} & \textbf{PSNR $\uparrow$} & \textbf{MSE $\downarrow$} & \textbf{SSIM $\uparrow$} & \textbf{PSNR $\uparrow$} & \textbf{MSE $\downarrow$} & \textbf{SSIM $\uparrow$} & \textbf{PSNR $\uparrow$} & \textbf{MSE $\downarrow$} \\
        \midrule \midrule
        pix2pix & 71.09 & 32.65 & 36.85 & 59.17 & 31.02 & 51.91 & 78.79 & 33.85 & 28.33 & 72.31 & 32.98 & 36.19 \\
        CycleGAN & 54.76 & 32.23 & 40.56 & 54.54 & 30.77 & 55.00 & 63.75 & 31.02 & 52.78 & 50.54 & 29.89 & 67.78 \\
        BBDM & {71.69} & {32.91} & {34.44} & 57.37 & 31.37 & 48.06 & \textbf{86.40} & 34.12 & 26.61 & {79.26} & 33.15 & 33.60 \\
        Vmanba & 69.54 & 32.67 & 36.42 & {63.01} & {31.47} & {46.99} & 79.63 & 34.12 & 26.49 & 77.45 & 33.53 & 31.85 \\
        DiffMa & 71.47 & 32.74 & 35.77 & 62.56 & 31.43 & 47.38 & 79.00 & {34.13} & {26.45} & 78.53 & {33.68} & {30.51} \\
        \rowcolor[HTML]{DAE0FB}HealthGPT-M3 & \underline{79.38} & \underline{33.03} & \underline{33.48} & \underline{71.81} & \underline{31.83} & \underline{43.45} & {85.06} & \textbf{34.40} & \textbf{25.49} & \underline{84.23} & \textbf{34.29} & \textbf{27.99} \\
        \rowcolor[HTML]{DAE0FB}HealthGPT-L14 & \textbf{79.73} & \textbf{33.10} & \textbf{32.96} & \textbf{71.92} & \textbf{31.87} & \textbf{43.09} & \underline{85.31} & \underline{34.29} & \underline{26.20} & \textbf{84.96} & \underline{34.14} & \underline{28.13} \\
        \bottomrule
    \end{tabular}
    }
    \label{tab:conversion}
\end{table*}

\noindent \textbf{3rd Stage: Visual Instruction Fine-Tuning.}  
In the third stage, we introduce additional task-specific data to further optimize the model and enhance its adaptability to downstream tasks such as medical visual comprehension (e.g., medical QA, medical dialogues, and report generation) or generation tasks (e.g., super-resolution, denoising, and modality conversion). Notably, by this stage, the word embedding layer and output head have been fine-tuned, only the H-LoRA modules and adapter modules need to be trained. This strategy significantly improves the model's adaptability and flexibility across different tasks.



\section{Experiments}

We conduct comprehensive experiments across multiple datasets and model architectures to validate our method's ability to decouple explanation robustness from classification robustness. Our evaluation addresses three key research questions:
\begin{itemize}
    \item \textbf{RQ1:} Does \ours have better quantify uncertainties?
    \item \textbf{RQ2:} How do different ensemble methods and information from both dimensions help?
    \item \textbf{RQ3:} Is\ours robust to different settings? 
\end{itemize}


\begin{table}[H]
\centering
\resizebox{!}{0.11\textwidth}{
\begin{tabular}{@{}lc@{}}
\toprule
\textbf{Measure} & \textbf{Details} \\ 
\midrule
$U_{\textit{Eigv}}(Dis)$ & \multicolumn{1}{c}{Spectral eigenvalue on the disagreement.} \\ 
$U_{\textit{Ecc}}(Dis)$ & \multicolumn{1}{c}{Average distance in responses' disagreement.} \\ 
$U_{\textit{Degree}}(Dis)$ & \multicolumn{1}{c}{Degree of disagreement similarity Matrix.} \\ 
$U_{\textit{Eigv}}(Agre)$ & \multicolumn{1}{c}{Spectral eigenvalue on the agreement.} \\ 
$U_{\textit{Ecc}}(Agre)$ & \multicolumn{1}{c}{Average distance in responses' agreement.} \\ 
$U_{\textit{Degree}}(Agre)$ & \multicolumn{1}{c}{Degree Matrix of agreement Matrix.} \\ 
$p(true)$ & \multicolumn{1}{c}{Entropy of knowledge dimension responses} \\ 
$D-UE$ & \multicolumn{1}{c}{eigenvalue from Laplacian of a directional graph} \\ 

\bottomrule
\end{tabular}}
\vspace{-1mm}
\caption{The baseline methods and explanations.}
\vspace{-5mm}
\label{tab:baslines}
\end{table}

\subsection{Experimental Setup}
\label{sec:setup}
\subsubsection{Datasets} As mentioned in \cref{sec:background}, following prior works~\cite{lin2022teaching}, we focus on open-form question-answering 
(QA) tasks in this paper. We adopt 4 different classic QA datasets. Coqa~\cite{reddy2019coqa} is a conversational question-answering dataset that contains dialogues with free-form answers grounded in diverse passages, which is the easiest dataset among all datasets. HotpotQA~\cite{yang2018hotpotqa} is a multi-hop QA dataset that demands reasoning over multiple Wikipedia paragraphs to derive correct answers. NQ-Open~\cite{kwiatkowski2019natural} consists of real-world queries from Google Search, requiring models to retrieve and answer questions without explicit context, which is the hardest dataset. 
\subsubsection{Models to Evaluate} We evaluate \ours on Llama family~\cite{touvron2023llama}, which is the one of the most popular LLMs. In detail, we use Llama-2-13b and Llama-2-7B to demonstrate the effectiveness of \ours with different model sizes and use Llama-3.1-8B~\cite{dubey2024llama} to that \ours could also work on the different version of Llama. To further demonstrate the generalization ability for other architectures,  we also use Phi4~\cite{abdin2024phi} and Deepseek-R1-distill-7B~\cite{guo2025deepseek} in our paper.


%\textcolor{red}{Not sure whether there is a section labeled as "sec eva metric" refered by Sec. }

\subsubsection{Evaluation Metrics} Effective uncertainty measures should accurately represent the reliability of LLM responses, with higher uncertainty more likely leading to incorrect generations and vice versa~\cite{lin2023generating,kuhn2023semantic}. Following prior works~\cite{lin2023generating,da2024llm}, we mainly use UQ values to predict whether an answer is correct or not. Following prior works~\cite{lin2023generating,da2024llm}, we will use Area Under Receiver Operating Characteristic (AUROC) and Area Under Accuracy Rejection Curve (AUARC) as evaluation metrics, where a higher AUROC or AUARC demonstrates better uncertainty measures. To compute AUROC and AUARC, the accuracy of each original response is required. Following previous works~\cite{da2024llm,lin2023generating}, we use another LLM to provide correctness from 0-100 to each response. If the correctness is greater than 70, we label the response as correct. In this paper, we use Qwen-34B~\cite{bai2023qwen} to evaluate the correctness.

\subsubsection{Knowledge Extracted Models} In this paper, we mainly use llama-2-13b~\cite{touvron2023llama} as the auxiliary models to extract the knowledge dimension of responses. To demonstrate the robustness of \ours with different knowledge-extracted models, we also contain the results for different LLMs as knowledge-extracted models.

\subsubsection{Baselines} In this paper, we compared \ours with baselines that use semantic dimension response and knowledge dimension response. For semantic dimension, we mainly compared with methods that come from \citet{lin2023generating}. In detail, we incorporate six distinct methods from \citet{lin2023generating}, which differ based on the operations applied after computing similarity and whether they utilize agreement (entailment) probabilities or disagreement (contradiction) logits to construct the similarity matrix. For knowledge dimension, we use D-UE~\cite{da2024llm} and $p(true)$~\cite{kadavath2022language} as the baselines. Note that we use $p(true)$ on the knowledge dimension of response. We show the detailed explanations of all baselines in \cref{tab:baslines}

\begin{figure*}[t]
\centering
\begin{minipage}[t]{0.32\linewidth}
  \centering
  \includegraphics[width=\linewidth,trim=0 0 0 1cm, clip]{images/ablation.pdf}
  \captionof{figure}{Ablation studies that show that \ours fully utilizes all the information from both dimensions.}
  \label{fig:ablation}
\end{minipage}\hfill
\begin{minipage}[t]{0.32\linewidth}
  \centering
  \includegraphics[width=\linewidth]{images/knowledge_extract.pdf}
  \captionof{figure}{Performance for different knowledge extract models on CoQA and NQ\_Open with llama3.1.}
  \label{fig:knowledge_extract}
\end{minipage}\hfill
\begin{minipage}[t]{0.32\linewidth}
  \centering
  \includegraphics[width=\linewidth]{images/Jacc.pdf}
  \captionof{figure}{Performance that uses Jaccard similarity on CoQA and NQ\_Open with llama3.1.}
  \label{fig:jacc}
\end{minipage}
\end{figure*}

\subsection{Does \ours have better quantify uncertainties? (RQ1)}
\label{sec:main_result}
In this section, we explore whether \ours has better uncertainties compared with state-of-the-art uncertainty quantification methods. In \cref{tab:main_results}, we compare \ours with 8 baselines across three different datasets and five different models as introduced in \cref{sec:setup} In detail, we have the following observations:

\noindent $\bullet$ Compared with all baseline methods, \ours achieves the best performance overall. Especially when we consider AUROC. For AUARC, \ours achieves the best performance for NQ\_Open while \ours also achieves the comparable performance for CoQA in most scenarios. These results demonstrate that \ours has better quantify uncertainties overall. \\
\noindent $\bullet$ Among all datasets, \ours achieves the highest performance improvement on NQ\_Open, which is the most difficult dataset among all datasets and may lose to baselines for an easier dataset like CoQA. This indicates \ours could perform even better when the task is harder, where uncertainty quantification is more important. \\
\noindent $\bullet$ Two different ensemble methods show very similar results. Min strategy performs better than the sum strategy under $61.51\%$ situations, indicating that difficult datasets might also have more complex structures that single one tensor decomposition might oversight some information while using min structure could reduce such oversight by considering the best cases. However, both ensemble methods show a better performance than all baselines, which proves the effectiveness of tensor decomposition. \\

From these results, we get a conclusion that overall, \ours have better uncertainties.


\subsection{How do different ensemble methods and information from both dimensions help? (RQ2)}
\label{sec:ablation}
In this section, we use more experiments to prove the necessity of using information from both semantic and knowledge dimensions as well as using tensor decomposition. In detail, we consider the following methods: 1) \ours with only semantic responses, 2) \ours with only knowledge responses and 3) Concatenating similarity matrices from semantic and knowledge dimensions into a 2D matrix and applying SVD, 4) only using one tensor decomposition. In \cref{fig:ablation}, we show the comparison between \ours and other methods.  The results show that \ours consistently outperforms its variants and SVD method that repeated information will dominate the features, showing the effectiveness of our framework.




\subsection{Is \ours robust to different settings? (RQ3)}
\subsubsection{Different Knowledge Extracted Models} Knowledge extracted models influence the claim extraction in \ours as stated in \cref{subsubsec:knowledge}. Therefore, in this section, We test the robustness of \ours on various knowledge extracted models. unlike using llama2-13b in \cref{sec:main_result} and \cref{sec:ablation}, we conduct experiments on CoQA and NQ\_open using llama2-7b and llama3.1 as the knowledge extracted models, We show the results in \cref{fig:knowledge_extract}. From the figure, we can see that using Phi4 could even achieve a better result, indicating \ours has more potential with the development of LLMs. 

\subsubsection{Different Accuracy Thresholds} Different accuracy thresholds lead to different accuracy and influence the evaluation of uncertainties. In the previous experiments, we all set the accuracy threshold to 70 as mentioned in \cref{sec:setup}.  To test the robustness of \ours under different accuracy thresholds, we choose an extra dataset TriviaQA~\cite{joshi2017triviaqa}, which is considered the easiest dataset, and NQ\_Open, which is the most challenging dataset in our paper to conduct experiments. We show the results with accuracy thresholds of 70 and 90 in \cref{tab:Accuracy_threshold}. From the results, we can see that increasing the accuracy threshold decreases the performance of all baselines while the performance of \ours could even increase for datasets with different difficulties, showing the robustness of \ours in different settings. 

\subsubsection{Different Similarity Metrics} Finally, different similarity metrics lead to different similarity matrices. Therefore, to test whether \ours also has a good performance for different similarities, we use Jaccard similarity instead of using an NLI model in this section and the results are presented in \cref{fig:jacc}. The results show that using Jaccard similarity will boost the performance for a simple dataset like CoQA but hurt the performance for a difficult dataset like NQ\_Open. This is because the answer to a simple question might not have a deeper semantic meaning that requires NLI models. However, \ours can still outperform baseline methods that also use Jaccard similarity, showing the robustness of \ours.





\begin{table}[h]
    \centering
    \resizebox{0.5\textwidth}{!}{
    \begin{tabular}{lcccc}
        \toprule
        \multirow{2}{*}{Methods} & \multicolumn{2}{c}{Accuracy Threshold: 0.7} & \multicolumn{2}{c}{Accuracy Threshold: 0.9} \\
        \cmidrule(lr){2-3} \cmidrule(lr){4-5}
        & AUROC & AUARC & AUROC & AUARC \\
        \midrule
        \multicolumn{5}{c}{\textbf{Dataset: TriviaQA} [Easy]} \\
        \midrule
        Eigv(Dis) & 0.8261 & 0.8094 & 0.8100& 0.7604\\
        Ecc(Dis) & 0.8063& 0.7940&0.7892 & 0.7415\\
        Degree(Dis) &0.8399 & 0.8163&0.8259 & 0.7694\\
        Eigv(Agre) &0.8436 &0.8116 &0.8351 & 0.7721 \\
        Ecc(Agre) & \textbf{0.8510}&0.8189 & 0.8374&0.7721 \\
        Degree(Agre) &0.8396 &\textbf{0.8397} &0.8384 & 0.7739\\
        \ours-Sum &0.8428 &0.8144 & 0.8438&0.7749 \\
        \ours-Min &0.8431 &0.8149 & \textbf{0.8440} & \textbf{0.7754}\\
        \midrule
        \multicolumn{5}{c}{\textbf{Dataset: NQ\_Open} [Hard]} \\
        \midrule
        Eigv(Dis) & 0.6162 & 0.7300 &0.5636 &0.6017 \\
        Ecc(Dis) & 0.6210& 0.7330& 0.5658&0.5941 \\
        Degree(Dis) &0.6130 & 0.7168&0.5662 &0.6033 \\
        Eigv(Agre) &0.6258 &0.7276 & 0.6146& 0.6290 \\
        Ecc(Agre) & 0.6273&0.7311 &0.6239 &0.6344\\
        Degree(Agre) &0.6286 &0.7355 & 0.6221&0.6299 \\
        \ours-Sum &\textbf{0.6334} &\textbf{0.7410} &\textbf{0.6351} &\textbf{0.6430} \\
        \ours-Min &0.6332 &0.7409 & 0.6350 &0.6429 \\
        \bottomrule
    \end{tabular}
    }
    \caption{Comparison of different methods across different accuracy thresholds on TrivialQA and NQ\_Open with llama2-13B. The results show that our methods outperform baselines after increasing the accuracy threshold, indicating that our methods have an advantage on more difficult datasets.}
    \vspace{-7mm}
    \label{tab:Accuracy_threshold}
\end{table}



\section{Conclusion}
In this paper, we propose ChineseEcomQA, a scalable question-answering benchmark designed to rigorously assess LLMs on fundamental e-commerce concepts. ChineseEcomQA is characterized by three core features: Focus on Fundamental Concept, E-Commerce Generalizability, and Domain-Specific Expertise, which collectively enable systematic evaluation of LLMs' e-commerce knowledge. Leveraging ChineseEcomQA, we conduct extensive evaluations on mainstream LLMs, yielding critical insights into their capabilities and limitations. Our findings not only highlight performance disparities across models but also delineate actionable directions for advancing LLM applications in the e-commerce domain.

% \section{Engines}

% To produce a PDF file, pdf\LaTeX{} is strongly recommended (over original \LaTeX{} plus dvips+ps2pdf or dvipdf). Xe\LaTeX{} also produces PDF files, and is especially suitable for text in non-Latin scripts.

% \section{Preamble}

% The first line of the file must be
% \begin{quote}
% \begin{verbatim}
% \documentclass[11pt]{article}
% \end{verbatim}
% \end{quote}

% To load the style file in the review version:
% \begin{quote}
% \begin{verbatim}
% \usepackage[review]{acl}
% \end{verbatim}
% \end{quote}
% For the final version, omit the \verb|review| option:
% \begin{quote}
% \begin{verbatim}
% \usepackage{acl}
% \end{verbatim}
% \end{quote}

% To use Times Roman, put the following in the preamble:
% \begin{quote}
% \begin{verbatim}
% \usepackage{times}
% \end{verbatim}
% \end{quote}
% (Alternatives like txfonts or newtx are also acceptable.)

% Please see the \LaTeX{} source of this document for comments on other packages that may be useful.

% Set the title and author using \verb|\title| and \verb|\author|. Within the author list, format multiple authors using \verb|\and| and \verb|\And| and \verb|\AND|; please see the \LaTeX{} source for examples.

% By default, the box containing the title and author names is set to the minimum of 5 cm. If you need more space, include the following in the preamble:
% \begin{quote}
% \begin{verbatim}
% \setlength\titlebox{<dim>}
% \end{verbatim}
% \end{quote}
% where \verb|<dim>| is replaced with a length. Do not set this length smaller than 5 cm.

% \section{Document Body}

% \subsection{Footnotes}

% Footnotes are inserted with the \verb|\footnote| command.\footnote{This is a footnote.}

% \subsection{Tables and figures}

% See Table~\ref{tab:accents} for an example of a table and its caption.
% \textbf{Do not override the default caption sizes.}

% \begin{table}
%   \centering
%   \begin{tabular}{lc}
%     \hline
%     \textbf{Command} & \textbf{Output} \\
%     \hline
%     \verb|{\"a}|     & {\"a}           \\
%     \verb|{\^e}|     & {\^e}           \\
%     \verb|{\`i}|     & {\`i}           \\
%     \verb|{\.I}|     & {\.I}           \\
%     \verb|{\o}|      & {\o}            \\
%     \verb|{\'u}|     & {\'u}           \\
%     \verb|{\aa}|     & {\aa}           \\\hline
%   \end{tabular}
  % \begin{tabular}{lc}
  %   \hline
  %   \textbf{Command} & \textbf{Output} \\
  %   \hline
  %   \verb|{\c c}|    & {\c c}          \\
  %   \verb|{\u g}|    & {\u g}          \\
  %   \verb|{\l}|      & {\l}            \\
  %   \verb|{\~n}|     & {\~n}           \\
  %   \verb|{\H o}|    & {\H o}          \\
  %   \verb|{\v r}|    & {\v r}          \\
  %   \verb|{\ss}|     & {\ss}           \\
  %   \hline
  % \end{tabular}
%   \caption{Example commands for accented characters, to be used in, \emph{e.g.}, Bib\TeX{} entries.}
%   \label{tab:accents}
% \end{table}

% As much as possible, fonts in figures should conform
% to the document fonts. See Figure~\ref{fig:experiments} for an example of a figure and its caption.

% Using the \verb|graphicx| package graphics files can be included within figure
% environment at an appropriate point within the text.
% The \verb|graphicx| package supports various optional arguments to control the
% appearance of the figure.
% You must include it explicitly in the \LaTeX{} preamble (after the
% \verb|\documentclass| declaration and before \verb|\begin{document}|) using
% \verb|\usepackage{graphicx}|.





% \subsection{Hyperlinks}

% Users of older versions of \LaTeX{} may encounter the following error during compilation:
% \begin{quote}
% \verb|\pdfendlink| ended up in different nesting level than \verb|\pdfstartlink|.
% \end{quote}
% This happens when pdf\LaTeX{} is used and a citation splits across a page boundary. The best way to fix this is to upgrade \LaTeX{} to 2018-12-01 or later.

% \subsection{Citations}

% \begin{table*}
%   \centering
%   \begin{tabular}{lll}
%     \hline
%     \textbf{Output}           & \textbf{natbib command} & \textbf{ACL only command} \\
%     \hline
%     \citep{Gusfield:97}       & \verb|\citep|           &                           \\
%     \citealp{Gusfield:97}     & \verb|\citealp|         &                           \\
%     \citet{Gusfield:97}       & \verb|\citet|           &                           \\
%     \citeyearpar{Gusfield:97} & \verb|\citeyearpar|     &                           \\
%     \citeposs{Gusfield:97}    &                         & \verb|\citeposs|          \\
%     \hline
%   \end{tabular}
%   \caption{\label{citation-guide}
%     Citation commands supported by the style file.
%     The style is based on the natbib package and supports all natbib citation commands.
%     It also supports commands defined in previous ACL style files for compatibility.
%   }
% \end{table*}

% Table~\ref{citation-guide} shows the syntax supported by the style files.
% We encourage you to use the natbib styles.
% You can use the command \verb|\citet| (cite in text) to get ``author (year)'' citations, like this citation to a paper by \citet{Gusfield:97}.
% You can use the command \verb|\citep| (cite in parentheses) to get ``(author, year)'' citations \citep{Gusfield:97}.
% You can use the command \verb|\citealp| (alternative cite without parentheses) to get ``author, year'' citations, which is useful for using citations within parentheses (e.g. \citealp{Gusfield:97}).

% A possessive citation can be made with the command \verb|\citeposs|.
% This is not a standard natbib command, so it is generally not compatible
% with other style files.

% \subsection{References}

% \nocite{Ando2005,andrew2007scalable,rasooli-tetrault-2015}

% The \LaTeX{} and Bib\TeX{} style files provided roughly follow the American Psychological Association format.
% If your own bib file is named \texttt{custom.bib}, then placing the following before any appendices in your \LaTeX{} file will generate the references section for you:
% \begin{quote}
% \begin{verbatim}
% \bibliography{custom}
% \end{verbatim}
% \end{quote}

% You can obtain the complete ACL Anthology as a Bib\TeX{} file from \url{https://aclweb.org/anthology/anthology.bib.gz}.
% To include both the Anthology and your own .bib file, use the following instead of the above.
% \begin{quote}
% \begin{verbatim}
% \bibliography{anthology,custom}
% \end{verbatim}
% \end{quote}

% Please see Section~\ref{sec:bibtex} for information on preparing Bib\TeX{} files.

% \subsection{Equations}

% An example equation is shown below:
% \begin{equation}
%   \label{eq:example}
%   A = \pi r^2
% \end{equation}

% Labels for equation numbers, sections, subsections, figures and tables
% are all defined with the \verb|\label{label}| command and cross references
% to them are made with the \verb|\ref{label}| command.

% This an example cross-reference to Equation~\ref{eq:example}.

% \subsection{Appendices}

% Use \verb|\appendix| before any appendix section to switch the section numbering over to letters. See Appendix~\ref{sec:appendix} for an example.

% \section{Bib\TeX{} Files}
% \label{sec:bibtex}

% Unicode cannot be used in Bib\TeX{} entries, and some ways of typing special characters can disrupt Bib\TeX's alphabetization. The recommended way of typing special characters is shown in Table~\ref{tab:accents}.

% Please ensure that Bib\TeX{} records contain DOIs or URLs when possible, and for all the ACL materials that you reference.
% Use the \verb|doi| field for DOIs and the \verb|url| field for URLs.
% If a Bib\TeX{} entry has a URL or DOI field, the paper title in the references section will appear as a hyperlink to the paper, using the hyperref \LaTeX{} package.

% \section*{Acknowledgments}

% This document has been adapted
% by Steven Bethard, Ryan Cotterell and Rui Yan
% from the instructions for earlier ACL and NAACL proceedings, including those for
% ACL 2019 by Douwe Kiela and Ivan Vuli\'{c},
% NAACL 2019 by Stephanie Lukin and Alla Roskovskaya,
% ACL 2018 by Shay Cohen, Kevin Gimpel, and Wei Lu,
% NAACL 2018 by Margaret Mitchell and Stephanie Lukin,
% Bib\TeX{} suggestions for (NA)ACL 2017/2018 from Jason Eisner,
% ACL 2017 by Dan Gildea and Min-Yen Kan,
% NAACL 2017 by Margaret Mitchell,
% ACL 2012 by Maggie Li and Michael White,
% ACL 2010 by Jing-Shin Chang and Philipp Koehn,
% ACL 2008 by Johanna D. Moore, Simone Teufel, James Allan, and Sadaoki Furui,
% ACL 2005 by Hwee Tou Ng and Kemal Oflazer,
% ACL 2002 by Eugene Charniak and Dekang Lin,
% and earlier ACL and EACL formats written by several people, including
% John Chen, Henry S. Thompson and Donald Walker.
% Additional elements were taken from the formatting instructions of the \emph{International Joint Conference on Artificial Intelligence} and the \emph{Conference on Computer Vision and Pattern Recognition}.

% Bibliography entries for the entire Anthology, followed by custom entries
% \bibliography{anthology,custom}
% Custom bibliography entries only
\bibliography{custom}

\appendix

\section{Appendix}
\label{sec:appendix}


% \begin{table*}[t]\scriptsize
% \renewcommand{\arraystretch}{1.2}

%     % \begin{subtable*}{0.47\textwidth}
%     %     \centering
%     %     \setlength{\tabcolsep}{0.9mm} 
%     %     \hline
%     %     \begin{tabular}{ll}
%     %          task  & dataset \\
%     %          \hline
%     %                                       & big\_font,IIT-CDIP,chart2dict,ChartSFT \\
%     %                                       & DocumentText,hand\_write \\
%     %                                       & html\_zh\_lay,html\_zh\_text \\
%     %                                       & receipt\_EATEN,receipt\_by\_ei \\
%     %                                       & SceneText,table\_markdown,table2md(en \& zh) \\
%     %          \multirow{-6}{*}{Others}      & double\_column\_render(en \& zh) \\
    
    
%     %                                       & Laion-EN (en)~\cite{schuhmann2022laion5b}, Laion-ZH (zh)~\cite{schuhmann2022laion5b}\\
%     %                                       & COYO (zh)~\cite{byeon2022coyo}, \\ 
%     %          \multirow{-3}{*}{Captioning} & GRIT (zh)~\cite{peng2023kosmos2}, COCO (en)~\cite{chen2015cococaption}, TextCaps (en)~\cite{sidorov2020textcaps}   \\
    
    
             
%     %          \rowcolor{gray!15}
%     %                                       & DocStruct4M-bbox,docstruct4M\_check\_wo\_nat,html\_zh\_bbox,synthdog\_bbox(en \& zh) \\
%     %          \rowcolor{gray!15}
%     %                                       & Objects365 (en\&zh)~\cite{shao2019objects365}, GRIT (en\&zh)~\cite{peng2023kosmos2},      \\
%     %          \rowcolor{gray!15}
%     %          \multirow{-3}{*}{Detection}  & All-Seeing (en\&zh)~\cite{wang2023allseeing} \\
    
             
%     %                                       & ANYWORD \\    
%     %                                       & Wukong-OCR (zh)~\cite{gu2022wukong}, LaionCOCO-OCR (en)~\cite{schuhmann2022laioncoco},                      \\
%     %          \multirow{-3}{*}{OCR (large)}& Common Crawl PDF \\
%     %          \rowcolor{gray!15}
%     %                                       & stvqa\_ocr,\\
%     %          \rowcolor{gray!15}
%     %                                       & MMC-Inst , LSVT (zh)~\cite{sun2019lsvt}, ST-VQA (en)~\cite{biten2019stvqa}        \\
%     %          \rowcolor{gray!15}
%     %                                       & RCTW-17 (zh)~\cite{shi2017rctw17}, ReCTs (zh)~\cite{zhang2019rects}, ArT (en\&zh)~\cite{chng2019art},       \\
%     %          \rowcolor{gray!15}
%     %                                       & SynthDoG (en\&zh)~\cite{kim2022synthdog}, COCO-Text (en)~\cite{veit2016cocotext},                           \\
%     %          \rowcolor{gray!15}
%     %                                       & ChartQA-OCR, CTW-OCR, DocVQA-OCR    \\
%     %          \rowcolor{gray!15}
%     %           \multirow{-6}{*}{OCR (small)}& TextOCR, PlotQA-OCR, InfoVQA-OCR        \\
%     %     \hline
%     %     \end{tabular}
%     %     \caption{Datasets used in the pre-training stage. 
%     % }
%     % \label{tab:pretraining}
%     % \end{subtable*}
%     \begin{subtable*}{}
%         \setlength\tabcolsep{6.4pt}
%         \begin{tabular}{l|l}
%     task & dataset \\
%     \hline
%     Captioning                    & TextCaps (en)~\cite{sidorov2020textcaps}, ShareGPT4V (en\&zh)~\cite{chen2023sharegpt4v}                        \\
%     \rowcolor{gray!15}
%                                   & VQAv2 (en)~\cite{goyal2017vqav2}, GQA (en)~\cite{hudson2019gqa}, OKVQA (en)~\cite{marino2019okvqa},            \\
%     \rowcolor{gray!15}
%     \multirow{-2}{*}{General QA}  & VSR (en)~\cite{liu2023vsr}, VisualDialog (en)~\cite{das2017visualdialog}                                       \\
%     \multirow{-1}{*}{Science}     & AI2D (en)~\cite{kembhavi2016ai2d}, ScienceQA (en)~\cite{lu2022scienceqa}, TQA (en)~\cite{kembhavi2017tqa}      \\
%     \rowcolor{gray!15}
%                                   & ChartQA (en)~\cite{masry2022chartqa}, MMC-Inst (en)~\cite{liu2023mmcinst}, DVQA (en)~\cite{kafle2018dvqa},     \\
%     \rowcolor{gray!15}
%     \multirow{-2}{*}{Chart}       & PlotQA (en)~\cite{methani2020plotqa}, LRV-Instruction (en)~\cite{liu2023lrv-instruction}                       \\
    
%                                   & GeoQA+ (en)~\cite{cao2022geoqa_plus}, TabMWP (en)~\cite{lu2022tablemwp}, MathQA (en)~\cite{yu2023mathqa},      \\
%     \multirow{-2}{*}{Mathematics} & CLEVR-Math/Super (en)~\cite{lindstrom2022clevrmath, li2023superclevr}, Geometry3K (en)~\cite{lu2021geometry3k} \\
%     \rowcolor{gray!15}
%                                   & KVQA (en)~\cite{shah2019kvqa}, A-OKVQA (en)~\cite{schwenk2022aokvqa}, ViQuAE (en)~\cite{lerner2022viquae},     \\
%     \rowcolor{gray!15}
%     \multirow{-2}{*}{Knowledge}   & Wikipedia (en\&zh)~\cite{he2023wanjuan}                                                                        \\
%                                   & OCRVQA (en)~\cite{mishra2019ocrvqa}, InfoVQA (en)~\cite{mathew2022infographicvqa}, TextVQA (en)~\cite{singh2019textvqa}, \\ 
%                                   & ArT (en\&zh)~\cite{chng2019art}, COCO-Text (en)~\cite{veit2016cocotext}, CTW (zh)~\cite{yuan2019ctw},          \\
%                                   & LSVT (zh)~\cite{sun2019lsvt}, RCTW-17 (zh)~\cite{shi2017rctw17}, ReCTs (zh)~\cite{zhang2019rects},             \\
%     \multirow{-4}{*}{OCR}         & SynthDoG (en\&zh)~\cite{kim2022synthdog}, ST-VQA (en)~\cite{biten2019stvqa}                                    \\
%     \rowcolor{gray!15}
%     Document                      & DocVQA (en)~\cite{clark2017docqa}, Common Crawl PDF (en\&zh)                                                   \\
%     Grounding                     & RefCOCO/+/g (en)~\cite{yu2016refcoco,mao2016refcocog}, Visual Genome (en)~\cite{krishna2017vg}                            \\
%     \rowcolor{gray!15}
%                                   & LLaVA-150K (en\&zh)~\cite{liu2023llava}, LVIS-Instruct4V (en)~\cite{wang2023lvisinstruct4v},                   \\
%     \rowcolor{gray!15}
%                                   & ALLaVA (en\&zh)~\cite{chen2024allava}, Laion-GPT4V (en)~\cite{laion_gpt4v_dataset},                            \\
%     \rowcolor{gray!15}
%     \multirow{-3}{*}{Conversation}& TextOCR-GPT4V (en)~\cite{textocr_gpt4v_dataset},  SVIT (en\&zh)~\cite{zhao2023svit}                            \\
%                                   & OpenHermes2.5 (en)~\cite{OpenHermes2_5}, Alpaca-GPT4 (en)~\cite{taori2023alpaca},                              \\
%     \multirow{-2}{*}{Text-only}   & ShareGPT (en\&zh)~\cite{zheng2023vicuna}, COIG-CQIA (zh)~\cite{bai2024coig}                                    \\
%         \end{tabular}
%         \centering
%             \caption{Datasets used in the fine-tuning stage.
%         }
%     \label{tab:finetuning}
%     \end{subtable*}
% \caption{\textbf{Summary of datasets used in InternVL 1.5.} 
% To construct large-scale OCR datasets, we utilized PaddleOCR \cite{li2022paddleocr} to perform OCR in Chinese on images from Wukong \cite{gu2022wukong} and in English on images from LAION-COCO \cite{schuhmann2022laioncoco}.
% }
% \label{tab:dataset}
% \end{table*}
% \begin{subtable}{\columnwidth}
%     \centering
%     \begin{tabular}{ccc}
%         \hline
%         \makecell{compress ratio} & Train Data  & Time Cost\\
%         \hline
%         50\% & 1.5M &  \\
%         75\% & 1.5M &  \\
%         \hline
%     \end{tabular}
%     \caption{Time cost of differnt SFT Settings. The time consumption is greatly reduced.}
% \end{subtable}
\subsection{Training settings}
Our FCoT-VL was trained in two distinct stages: re-alignment and post-trian. 
As shown in Table~\ref{tab:training_settings}, we present the training details of FCoT-VL in different stages. The details are as follows:

For both stages, we train models with 64 ascend 910 NPUs with the packed batch size is set to 512.
In the re-alignment pre-training, we employ a 2 million image-text pairs to learn the projector and compress module. This allows the VLLMs to re-align the compressed visual token with the language token space. Specifically, we craft the optimization tasks of recognizing text in document images and converting charts and tables into pythondict/markdown format. We set the training epoch as 1, which requires approximately 48 hours using 64 NPUs for 2B scale. In the subsequent instruction tuning phase, we make all parameters of FCoT-VL learnable and keep most of the settings unchanged, except context length, training data and training epochs. 
% \begin{table}[htbp]
%     \renewcommand{\arraystretch}{1.2}
%     \centering
%     \resizebox{\columnwidth}{!}{
%         \begin{tabular}{lcc|cc}
%             \hline
%             \textbf{Settings} & \multicolumn{2}{c|}{\textbf{InternVL2-2B}} & \multicolumn{2}{c}{\textbf{InternVL2-8B}} \\
%             \hline
%             & Realignment & SFT & Realignment & SFT \\
%             Trainable & Projector and Compress Module & Full Parameters & Projector and Compress Module & Full Parameters \\
%             Packed Batch Size & 512 & 512 & 512 & 512 \\
%             Learning Rate & $1e^{-5}$ & $1e^{-5}$ & $1e^{-5}$ & $1e^{-5}$ \\

%             Context Length & 4096 & 5120 & 4096 & 5120 \\
%             Image Tile Threshold & 12 & 12 & 12 & 12 \\
%             ViT Drop Path & 0.1 & 0.1& 0.1 & 0.1 \\
%             Weight Decay & 0.01 & 0.01 & 0.01 & 0.01 \\
%             Training Epochs & 1 & 3 & 1 & 3 \\
%             \hline
%             Dataset & Pre-train & Fine-tune & Pre-train & Fine-tune \\
%             Training Examples  & $\sim2M$ & $\sim4.5M$ & $\sim2M$ & $\sim4.5M$ \\
%             \hline
%         \end{tabular}
%     }
%     \caption{Training settings for InternVL2-2B and InternVL2-8B.}
%     \label{tab:training_settings}
% \end{table}

  \begin{table}[htbp]
    \renewcommand{\arraystretch}{1.2}
    \centering
    \resizebox{\columnwidth}{!}{
        \begin{tabular}{l|cc}
            \hline
            \textbf{Settings} & \textbf{Re-alignment} & \textbf{Post-train} \\
            \hline
            \rowcolor{gray!15}
            Trainable & \makecell{Projector,\\ Compress Module} & Full Parameters \\
            Packed Batch Size & 512 & 512  \\
            \rowcolor{gray!15}
            Learning Rate & $1e^{-5}$ & $1e^{-5}$ \\
            Context Length & 4096 & 5120 \\
            \rowcolor{gray!15}
            Image Tile Threshold & 12 & 12 \\
            ViT Drop Path & 0.1 & 0.1 \\
            \rowcolor{gray!15}
            Weight Decay & 0.01 & 0.01  \\
            Training Epochs & 1 & 3 \\
            \hline
            \rowcolor{gray!15}
            Dataset & Pre-train & Fine-tune \\
            Training Examples  & $\sim2M$ & $\sim4.5M$ \\
            \hline
        \end{tabular}
    }
    \caption{Detailed Training settings for InternVL2-2B and InternVL2-8B.}
    \label{tab:training_settings}
\end{table}

% \clearpage


\subsection{Model Capabilities and Qualitative Examples}
In this section, we present some practical examples of our FCoT-VL.
\begin{figure*}[htbp]
    \centering
    \includegraphics[width=\textwidth]{q1.png}  % 图片路径
    \caption{The model excels in understanding scheduling-related queries. Image source:\cite{mathew2021docvqa}}  % 标题
    \label{ex1}  % 标签
\end{figure*}

\begin{figure*}[htbp]
    \centering
    \includegraphics[width=\textwidth]{q2.png}  % 图片路径
    \caption{The model demonstrates excellence in recognizing handwritten text in emails. Image source:\cite{mathew2021docvqa}}  % 标题
    \label{ex2}  % 标签
\end{figure*}

\begin{figure*}[htbp]
    \centering
    \includegraphics[width=\textwidth]{q3.png}  % 图片路径
    \caption{The model demonstrates excellence in recognizing printed text and images in books. Image source:\cite{mathew2021docvqa}}  % 标题
    \label{ex3}  % 标签
\end{figure*}


\begin{figure*}[htbp]
    \centering
    \includegraphics[width=\textwidth]{q4.png}  % 图片路径
    \caption{The model displays an adeptness in understanding line charts. Image source:\cite{mathew2021docvqa}}  % 标题
    \label{ex4}  % 标签
\end{figure*}


\begin{figure*}[htbp]
    \centering
    \includegraphics[width=\textwidth]{q5.png}  % 图片路径
    \caption{The model displays an adeptness in understanding images of natural animals. Image source:\cite{lu2022scienceqa}}  % 标题
    \label{ex5}  % 标签
\end{figure*}





\begin{figure*}[htbp]
    \centering
    \includegraphics[width=\textwidth]{q8.png}  % 图片路径
    \caption{The model displays an adeptness in understanding bar charts. Image source:\cite{masry2022chartqa}}  % 标题
    \label{ex8}  % 标签
\end{figure*}


\begin{figure*}[htbp]
    \centering
    \includegraphics[width=\textwidth]{q9.png}  % 图片路径
    \caption{The model displays an adeptness in understanding curve charts. Image source:\cite{masry2022chartqa}}  % 标题
    \label{ex9}  % 标签
\end{figure*}

\begin{figure*}[htbp]
    \centering
    \includegraphics[width=\textwidth]{q10.png}  % 图片路径
    \caption{The model displays an adeptness in recognizing handwritten Chinese characters.}  % 标题
    \label{ex10}  % 标签
\end{figure*}

\begin{figure*}[htbp]
    \centering
    \includegraphics[width=\textwidth]{q11.png}  % 图片路径
    \caption{The model displays an adeptness in understanding Chinese flight ticket information.Image source:\cite{wang2024qwen2vl}}  % 标题
    \label{ex11}  % 标签
\end{figure*}


\begin{figure*}[htbp]
    \centering
    \includegraphics[width=\textwidth]{q12.png}  % 图片路径
    \caption{The model displays an adeptness in calculating information from Chinese bar charts. }  % 标题
    \label{ex12}  % 标签
\end{figure*}

\begin{figure*}[htbp]
    \centering
    \includegraphics[width=0.25\textwidth]{q6.png}  % 调整图片宽度
    \caption{The model displays an adeptness in understanding posters with dense information. Image source:\cite{mathew2022infographicvqa}}  % 标题
    \label{ex6}  % 标签
\end{figure*}


\begin{figure*}[htbp]
    \centering
    \includegraphics[width=0.23\textwidth]{q7.png}  % 图片路径
    \caption{The model displays an adeptness in understanding posters with intertwined text and images. Image source:\cite{mathew2022infographicvqa}}  % 标题
    \label{ex7}  % 标签
\end{figure*}

\end{document}
