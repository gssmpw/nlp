% \citet{c:23} outlined that music generation model often defaults to Western tonal and rhythmic structures when tasked with generating non-Western music such as Indian or Middle Eastern genres. As a result, a generated piece intended to mimic an \textit{Hindustani Classical \footnote{Hindustani Classical music is a traditional system of music that emphasizes melodic development based on ragas (melodic frameworks) and talas (rhythmic cycles).}} may sound like a Western pop melody played on a sitar. Similarly, SunoAI, when attempting to generate  \textit{Makamat}\footnote{Makam, in traditional Arabic music, is a melodic mode system defining pitches, patterns, and improvisation, central to Arabian art music, with 72 heptatonic scales} music of the Middle East, may round off the microtones to the nearest Western equivalent, resulting in a piece that lacks the distinctive sound of Arabic music.

% Importantly, culture manifests differently across languages, influencing how models interpret and generate content. A model prompted in one language may produce notably different outputs compared to prompts in another language~\cite{}, reflecting the intricate relationship between language and cultural context. This phenomenon extends to various domains, including music, where cultural nuances are deeply embedded in linguistic expressions and artistic traditions.

% The field of music generation has witnessed significant advancements with the advent of large pre-trained models~\cite{}. However, these models often exhibit a bias towards Western music~\cite{}, leaving the rich diversity of global musical traditions underrepresented. This paper introduces a novel approach to address this imbalance by integrating cultural music knowledge through parameter-efficient training with adapters at the decoding side.

% Our study begins with a comprehensive analysis of existing music datasets, revealing a stark disparity in the representation of non-Western music. Particularly noteworthy is the scarcity of non-Western music data, with merely 5.7\% of the total hours available online. This finding highlights the urgent need for more inclusive and diverse musical datasets to foster truly global music generation by AI models.


% Despite
% their sophistication, much less attention has been given to fair and equitable representation of the musical styles from around the world in AI-generated music, which is essential in promoting
% global musical diversity~\cite{brookings2023}\textcolor{red}{what is referenced paper about?}.

% Generating music is a challenging task considering that it requires leveraging the full frequency spectrum of the signal~\cite{}. This means sampling the signal at a higher rate: 44.1kHZ or 48kHz versus 16kHz for speech, creating complex structures composed from different instruments. \citet{} identified that human listeners are highly sensitive to disharmony, thus music generation has to be flawless. The ability to control the generation process has been widely used in recent years, \citet{} applied...., \citet{} discovered diffusion based approach, ....

% Recent research has highlighted the importance of cultural exploration in LLMs, emphasizing that models reflecting particular cultural inclinations are not inherently problematic or stereotyping. Instead, as noted by \citet{10.1093/pnasnexus/pgae346}, such models can signify alignment with the views of specific populations, highlighting cultural significance. This perspective underscores the value of preserving and representing diverse cultural expressions. We concur with the terminology introduced in \cite{}, which refers to this concept as Cultural Trends instead of Cultural Bias, and will therefore use the term "Cultural Trends" throughout the study.

% The use of Cultural Trends emphasizes that a model reflecting a particular cultural inclination does not inherently imply danger or stereotyping. Instead, it signifies alignment with the views
% of a specific population, highlighting cultural significance."