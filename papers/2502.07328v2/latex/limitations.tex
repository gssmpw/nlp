Our work relies on adapter-based techniques for cross-cultural adaptation but there is a need to explore additional architectural configurations to further optimize low-resource fine-tuning such as LoRA~\cite{DBLP:journals/corr/abs-2106-09685} or Compacter~\cite{Davison2021CompacterEL} approaches. 

Additionally, our approach only focused on a few genres, and future work should aim to incorporate a broader range of musical styles. Our investigation involves only Hindustani classical and Turkish Makam traditions, leaving other genres from the Dunya dataset unexplored. This narrow focus stems not from a lack of curiosity, but from our limited cultural expertise - a constraint we acknowledge upfront.

We also trained separate models for Hindustani Classical and Turkish Makam music; combining these into a single model could offer greater generalization across genres. 

Another limitation lies in the evaluation process. Human evaluations were conducted on a limited number of samples with a duration of 10 seconds, and more genre-specific assessments are necessary. We also believe that computing objective metrics for underrepresented genres may obscure the full picture because the backbone models used to compute these metrics may not have been trained on various underrepresented genres, resulting in an erroneous portrayal of genres.