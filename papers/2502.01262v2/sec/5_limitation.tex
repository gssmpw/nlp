\section{Limitations}
\label{sec5:limitation}

The proposed method demonstrates superior transferability compared to existing methods. However, a drawback of the proposed method is the presence of the user-defined parameter $\tau$ and loss balance parameter $\lambda_{t}$. While the ablation study illustrates performance variations according to different $\tau$ values, there would be better $\tau$ which leads higher attack performance. Additionally, we observe that attack performance varies depending on how the two loss terms, $L_{ex}$ and $L_{in}$, are adjusted through the $\lambda_{t}$ value. Ideally, the $\tau$ and $\lambda_{t}$ values should be determined automatically by taking into account the characteristics of the input image, the source model, and feature distributions. In future research, we plan to investigate techniques for automatically selecting optimal $\tau$ and $\lambda_{t}$ values.

\begin{figure}
\centering
\includegraphics[width=0.95\linewidth]{fig5.lamda.pdf}
\caption{mIoU performance across different loss terms. (S) and (T) indicate the source and target models, respectively.}
\label{fig:fig5}
\end{figure}