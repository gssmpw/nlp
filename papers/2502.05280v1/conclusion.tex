This paper has proposed the \emph{smart contract} model
for
cross-chain protocols in which  parties, some honest and some not, interact through trusted smart contracts residing on multiple, independent ledgers.
As a case study, it  presented a simple two-way  cross-chain swap protocol, 
solving a fundamental task in this setting.
Many other tasks have been considered in real systems that would be interesting to formalize
in the smart contract model, such  as auctions, loans and options.
  
The smart contract model differs from classical
models of distribute computing,
refactoring the roles of active participants (processes, parties)
and of passive communication (objects, smart contracts).
%SR: not accurate, many papers have considered Byzantine failures
%In the classical distributed models,
%active participants are usually prone to crashes but not to Byzantine failures.
In classical models,
Byzantine processes interact with Byzantine processes,
while in the smart contract model,
Byzantine parties interact with honest contracts.
The  model requires a nuanced, game theoretic notion of correctness distinct from that of classical models.

The model assumes smart contracts cannot communicate with one another,
because of the lack of a practical way
to ensure that replaying  such cross-contract communication would always
and everywhere produce exactly the same results.
Nevertheless,
there are emerging real-world mechanisms that do support limited
forms of cross-chain communication.
An \emph{Oracle}~\cite{Chainlink} is a mechanism that allows a contract to read data
from an external source (for example, the current dollar/euro exchange rate).
Token bridges~\cite{axelar,Wormhole,layerzero} provide a way to effectively
transfer assets from one ledger to another by ``freezing'' an asset
at the source ledger and creating a matching ``wrapped'' asset at the target ledger.
So-called \emph{layer two} solutions~\cite{rollups,KalodnerGCWF2018,plasma}
allow computations to be moved from one ledger to another.
We leave it to future work to expand the model presented here to encompass
these increasingly important technologies.

The model's computational power can vary depending on access to
cryptographic primitives.
For example,
cryptographic hashes enable atomic cross-chain swaps~\cite{bitcoinwiki,bip199,decred,Herlihy2018,tiersnolan,barterdex,ZakharyAE2019,Catalyst} and payment networks~\cite{DeckerW2015,bolt,HeilmanLG2019,raiden,PoonD2016}.
Public key infrastructure supports atomic broadcast~\cite{Herlihy2018,HerlihyLS2021}.
To fully capture both the distributed and cryptographic aspects of these applications
it might be necessary to extend the model in a way similar to Canetti et al.~\cite{CanettiCKLP07}.
Moreover,
it might be intriguing to extend the model to encompass probabilistic behavior,
perhaps using extensions to I/O automata similar to those of Civit and Potop~\cite{CivitP22b}.    
