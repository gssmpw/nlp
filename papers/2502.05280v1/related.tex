Various papers have described formal models of various types of ledgers.
Our focus differs in two ways. 
First, we are interested in the behavior of networks of smart contracts,
and view the ledger itself as a ``black box'' platform on top of which the contracts
are implemented.  
In this model, 
smart contracts exist on their own.
The contracts form a network of state machines,
where each contract exports a set of \emph{functions} (methods) called by parties,
operating on contract stae.
Each function's code is visible to all parties, and each party trusts the code will be executed correctly.
It is immaterial at this level of abstraction
if the smart contracts are implemented by a blockchain,
by a tamper-proof replicated database,
or some other future technology.
Second, our focus is on \emph{cross-chain transactions}, meaning that each smart contract runs on a different ledger.

Perhaps the first formal model of a permissionless blockchain system was proposed by Garay et al.~\cite{GarayKL2015,GarayKL2015-j2024} for Bitcoin. 
Later papers
aimed at modeling at a higher level of abstraction.
Anceaume et al.~\cite{AnceaumeLPT17} were the first to make a connection between the Bitcoin ledger and classical distributed shared objects. They introduced the notion of  a \emph{ Distributed Ledger Register} (DLR) where the value of the register has a tree topology instead of a single value as in the classical theory of distributed registers. 
The vertices of the tree are blocks of transactions cryptographically linked. 
%The DLR properties were crafted to fit the behavior of permissionless blockchains such as Bitcoin and Ethereum. 
Later, Anta et al.~\cite{AntaKGN18} introduced the {\em Distributed Ledger Object} (DLO) formalism that  defines the ledger object as an ordered sequence of records, abstracting away from registers.  This work has been extended~\cite{AntaGN19} to Multi-Distributed Ledger Objects (MDLO), the result of aggregating multiple Distributed Ledger Objects - DLO (a DLO is a formalization of the blockchain) and that supports append and get operations of records (e.g., transactions)  from multiple clients concurrently.
To model the behavior of distributed ledgers at runtime, Anceaume et al.\cite{AnceaumePLPP19} introduced the {\em Blockchain Abstract Data Type} abstraction, which provides a lower-level abstraction of Distributed Ledgers, suitable for both permissioned and permissionless systems.
Rajsbaum and Raynal~\cite{RR19cacm} provide a discussion of how ledgers are the next logical step in a long history of implementations of a sequential specification on a concurrent system. 
%
Other lines of research include Zappal{\`{a} et al.~\cite{ZappalaBPS21} who propose a game theoretical framework to formally characterize the robustness of blockchains systems in terms of resilience to rational deviations and immunity to Byzantine behaviors. The framework is sufficiently general to characterize the robustness of various blockchain protocols (e.g.  Bitcoin, Tendermint, Lightning Network, a side-chain protocols and a cross-chain swap protocols).  
Our model goes beyond  other models in the literature limited to simple payments~\cite{FreyGRT2021,GuerraouiKMPS2018,SliwinskiW2019} or swaps \cite{BelottiMPS20}.

Other research investigates formal models for blockchain-based contracts (overview in Bartoletti \emph{et al}~\cite{BBACZ21}).
In contrast to our work, they
address formal verification in the context of specific chains, such as Bitcoin or Ethereum ,
while our goal is a model reasonably independent of any particular system.

As a precursor to cross-chain coordination,
in the classic \emph{fair exchange} problem,
Alice and Bob wish to exchange digital assets
in such a way that either the exchange takes place,
or each party keeps its token.
Fair exchange has been widely studied.
It encompasses other important problems, such as contract signing,
and has been also studied for more than two participants.
It is well-known that this problem is unsolvable without a trusted third party~\cite{Pagnia1999OnTI}
because of its relation to consensus.
Fair exchange protocols remain relevant because smart contracts can be viewed
as trusted third parties.
See Tas et al.~\cite{ErtemSeresZMKBN2024} for references and applications to data storage.

The swap protocol of \secref{example} is synchronous.
Lys et al.~\cite{LysMP21} describe an alternative protocol
in a partially synchronous setting.