We illustrate an application of the model by analyzing a simple
\emph{two-party cross-chain swap} task,
along with a protocol~\cite{tiersnolan}.

\subsection{The Task}
Imagine that Alice is willing to exchange her asset $a$
for Bob's asset $b$, and vice-versa.
They employ two contracts:
$C_A$ controls ownership of $a$ on one ledger,
and $C_B$ controls ownership of $b$ on another ledger.
Alice and Bob have access to a \emph{cryptographic hash function}
$H(\cdot)$ where it is infeasible to reconstruct $s$ given $H(s)$.

Formally, the swap task is given by $(\cI_P,\cI_C,\cO_C, U)$, where
\begin{itemize}
\item
  $\cI_P$ consists of a single input party vector $I_P$
  with an entry for Alice, $I_P[A] = (B,a,b)$,
  and an entry for Bob, $I_P[B] = (A,b,a)$,
  indicating that each party is willing to swap its asset for the other's.

\item
  $\cI_C$ consists of a single input contract state vector $I_C$, with
  $C_A$'s entry $I_C[C_A] = [a \hookrightarrow A]$, (meaning $A$ owns $a$), and
  $C_B$'s entry $I_C[C_B] = [b \hookrightarrow B]$, (meaning $B$ owns $b$).

\item
  $\cO_C$ consists of four possible output contract state vectors:
  \begin{equation*}
    O_C = \begin{cases}
    [a \hookrightarrow A, b \hookrightarrow A]&\text{$A$ owns $a,b$}\\
    [a \hookrightarrow B, b \hookrightarrow A]&\text{$A$ owns $b$, $B$ owns $a$}\\
    [a \hookrightarrow A, b \hookrightarrow B]&\text{$A$ owns $a$, $B$ owns $b$}\\
    [a \hookrightarrow B, b \hookrightarrow B]&\text{$B$ owns $a,b$}.
  \end{cases}
  \end{equation*}
 
\item
Alice's utility function is:
\begin{equation}
  \eqnlabel{aliceUtility}
  U(I_P,I_C,O_C)[A] =
  \begin{cases}
    2 &\text{if } O_C = [a \hookrightarrow A, b \hookrightarrow A]\\
    1 &\text{if } O_C = [a \hookrightarrow B, b \hookrightarrow A]\\
    0 &\text{if } O_C = [a \hookrightarrow A, b \hookrightarrow B]\\
    -1 &\text{if } O_C = [a \hookrightarrow B, b \hookrightarrow B].
  \end{cases}
\end{equation}
Alice strongly prefers the outcome where she acquires both assets,
but also prefers the outcome where the swap takes place.
She accepts the outcome where assets do not change hands,
and she considers herself worse off
if Bob acquires her asset while retaining his own.
Bob's utility function is symmetric.
\end{itemize}

\subsection{The Protocol}

The protocol proceeds in rounds:
\begin{enumerate}
\item\label{aliceEscrows}
  Alice creates secret $s$ and hashkey $h=H(s)$.
  She instructs $C_A$ to assume temporary ownership of $a$,
  to transfer $a$ to Bob if he produces $s$ within two rounds,
  and to refund $a$ to Alice if Bob fails to meet that deadline.

\item\label{bobEscrows}
  Bob verifies that Alice has escrowed $a$ at $C_A$,
  If so, he learns $h$,
  and he instructs $C_B$ to assume temporary ownership of $b$,
  to transfer $b$ to Alice if she produces a matching $s$ within one round,
  and to refund $b$ to Bob if Alice fails to meet that deadline.
  If Bob sees that Alice has not escrowed $a$, he exits the protocol.

\item\label{aliceClaims}
  Alice verifies that Bob has escrowed $b$ at $C_B$.
  If so, she sends $s$ to $C_B$, making $s$ public,
  and causing $C_B$ to transfer $b$ to her.
  If Alice observs that Bob has not escrowed, she exits the protocol.
  If, b the end of this round, $C_B$ did not receive $s$ from Alice,
  then $C_B$ refunds $b$ to Bob.

\item\label{bobClaims}
  If Alice took possession of $b$, then Bob learns $s$ from $C_B$.
  He sends $s$ to $C_A$,
  causing $C_A$ to transfer $a$ to him, completing the swap.
  If, by the end of this round, $C_A$ did not receive $s$ from Bob,
  then $C_A$ refunds $a$ to Alice.
\end{enumerate}

Formally, the protocol is given by $(\cI_P,\cI_C,\cO_C,\Xi)$, where:
\begin{itemize}
\item
  $\cI_P, \cI_C$ and $\cO_C$ are defined as in the swap task above.
\item
  Since the task has only a single input party vector
  and a single input contract state vector,
  we can restrict the execution function
  $\Xi: \cI_P \times \cI_C \times 2^\cP \to \cO_C$
  to its compliance set argument alone
  $\Xi: 2^\cP \to \cO_C$.
  \begin{itemize}
  \item
    If both parties are compliant, the swap takes place:
    \begin{equation}
      \eqnlabel{bothComply}
      \Xi(\set{A,B}) = [a \hookrightarrow B, b \hookrightarrow A].
    \end{equation}
  \item
    Alice can deviate in several ways:
    \begin{equation*}
      \eqnlabel{bobComplies}
      \Xi(\set{B}) = \begin{cases}
        [a \hookrightarrow A, b \hookrightarrow B]
          &\text{Alice fails to escrow $a$ at Step \ref{aliceEscrows}}\\
        [a \hookrightarrow A, b \hookrightarrow B]
          &\text{Alice fails to claim $b$ at Step \ref{aliceClaims}}\\
        [a \hookrightarrow B, b \hookrightarrow B]
          &\text{Alice reveals $s$ to Bob after she escrows $a$, without claiming $b$}
      \end{cases}
    \end{equation*}

  \item
    Bob can deviate in several ways:
    \begin{equation*}
      \eqnlabel{aliceComplies}
      \Xi(\set{A}) = \begin{cases}
        [a \hookrightarrow A, b \hookrightarrow B]
          &\text{Bob fails to escrow $b$ at Step \ref{bobEscrows}}\\
        [a \hookrightarrow A, b \hookrightarrow A]
          &\text{Bob fails to claim $a$ at Step \ref{bobClaims}}
      \end{cases}
    \end{equation*}
  \end{itemize}

\item
  If both parties deviate, $\Xi(\emptyset)$ contains all four outcomes.
\end{itemize}

\subsection{The Proof}

\begin{lemma}[Liveness]
  \lemmalabel{liveness}
  If both parties are compliant,
  then for $O_C \in \Xi(\cP)$, 
  and $P \in \cP$,
  $U(I_P,I_C,O_C)[P] > 0$.
\end{lemma}
\begin{proof}
  Recall that $\Xi(\cP) = [a \hookrightarrow B, b \hookrightarrow A])$,
  which has utility 1 for both Alice and Bob (\eqnref{aliceUtility}).
\end{proof}

\begin{lemma}[Safety]
  \lemmalabel{safety}
  No compliant party ends up with negative utility.
\end{lemma}
\begin{proof} 
By case analysis:
\begin{itemize}
\item
  If both parties are compliant,
  both have positive utility by \lemmaref{liveness}.

\item
  If Alice alone is compliant,
  \begin{equation*}
    \Xi(\set{A}) = \set{
      [a \hookrightarrow A, b \hookrightarrow B],
      [a \hookrightarrow A, b \hookrightarrow A]
    }
  \end{equation*}
  Each of these outcomes has non-negative utility for Alice (\eqnref{aliceUtility}).

\item
  If Bob alone is compliant,
  \begin{equation*}
    \Xi(\set{B}) = \set{
      [a \hookrightarrow A, b \hookrightarrow B],
      [a \hookrightarrow B, b \hookrightarrow B]}
  \end{equation*}
  Each of these outcomes has non-negative utility for Bob.
\end{itemize}    
\end{proof}

\begin{lemma}[Coalition Nash Equilibrium]
  \lemmalabel{equilibrium}
  No coalition of parties can increase its collective utility
  by deviating from the protocol while the others comply.
\end{lemma}
\begin{proof}
Since there are only two parties, of which one is compliant,
coalitions are singletons.
  \begin{itemize}
  \item
    Suppose the coalition is Alice.
    If both comply (\eqnref{bothComply}),
    Alice's utility is 1 (\eqnref{aliceUtility}).
    If Alice deviates while Bob complies (\eqnref{bobComplies}),
    Alice's utility is either 0 or -1 (\eqnref{aliceUtility}).

  \item
    Suppose the coalition is Bob.
    If both comply (\eqnref{bothComply}), Bob's utility is 1.
    If Bob deviates while Alice complies (\eqnref{aliceComplies}),
    Bob's utility is either 0 or -1.
  \end{itemize}
\end{proof}

