Many of the problems that arise in the context of blockchains and decentralized
finance can be seen as variations on classical problems of distributed computing.
The \emph{smart contract model} proposed here is intended to capture both the
similarities and the differences between classical and blockchain-based models of distributed computing.
The focus is on \emph{cross-chain} protocols
in which a collection of parties,
some honest and some perhaps not,
interact through trusted smart contracts residing on multiple, independent ledgers.

While cross-chain protocols are capable of general computations,
they are primarily used to track ownership of assets
such as cryptocurrencies or other valuable data.
For this reason, the smart contract model differs in some essential ways from
familiar models of distributed and concurrent computing.
Because parties are potentially Byzantine,
tasks to be solved are formulated using elementary game-theoretic notions,
taking into account the utility to each party of each possible outcome.
As in the classical model,
the parties provide task inputs and agree on a desired sequence of
proposed asset transfers.
Unlike the classical model,
the contracts, not the parties, determine task outputs in the form of executed asset transfers,
since they alone have the power to control ownership.