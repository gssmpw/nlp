\documentclass[conference]{IEEEtran}
\usepackage{wrapfig}

% If you use the following package, be sure to comment out \usepackate{xcolor}

\usepackage{multirow,mathtools } \usepackage{algorithm,algpseudocode}

% Very useful for various special symbols
\usepackage{pifont}
% For \rowcolor
\usepackage{color, colortbl}

\usepackage{blindtext}
\usepackage{lipsum}

\usepackage{multirow}
\usepackage{graphicx}
\usepackage{listings}

\usepackage{bbm}
\usepackage{dsfont}

\usepackage[most]{tcolorbox}

\newcommand{\JH}[1]{\textcolor{blue}{JH: #1}}
\newcommand{\SL}[1]{\textcolor{purple}{SL: #1}}
\newcommand{\JC}[1]{\textcolor{cyan}{JC: #1}}
\newcommand{\SM}[1]{\textcolor{red}{SM: #1}}
\newcommand{\CF}[1]{\textcolor{orange}{CF: #1}}
\newcommand{\RZ}[1]{\textcolor{red}{RZ: #1}}
\newcommand{\LL}[1]{\textcolor{red}{LL: #1}}
\newcommand{\YY}[1]{\textcolor{red}{YY: #1}}


\makeatletter
\newcommand*\titleheader[1]{\gdef\@titleheader{#1}}
\AtBeginDocument{%                                                               
  \let\st@red@title\@title
  \def\@title{%                                                                 
    \bgroup\normalfont\normalsize\centering\@titleheader\par\egroup
    \vskip1ex\st@red@title}
}
\makeatother

\title{\vspace{-8pt}Compact Yet Highly Accurate Printed Classifiers Using Sequential Support Vector Machine Circuits\vspace{-1ex}}
\author{\IEEEauthorblockN{
Ilias~Sertaridis, %\IEEEauthorrefmark{1},
Spyridon~Besias, %\IEEEauthorrefmark{1},
Florentia~Afentaki,
Konstantinos~Balaskas
and~Georgios~Zervakis
%\thanks{\IEEEauthorrefmark{1}Equal contribution, and the order was decided by a coin flip.}
}
\IEEEauthorblockA{
Department of Computer Engineering \& Informatics, University of Patras, Patras, Greece
}
\IEEEauthorblockA{
\{st1072480, st1072524, afentaki, kompalas, zervakis\}@ceid.upatras.gr
\vspace{-3ex}
}}
\titleheader{\vspace{-20pt}To appear at the 2025 IEEE International Symposium on Circuits and Systems (ISCAS), May 25-28 2025, London, UK.}

\begin{document}




\maketitle

\begin{abstract}
Printed Electronics (PE) technology has emerged as a promising alternative to silicon-based computing.
It offers attractive properties such as on-demand ultra-low-cost fabrication, mechanical flexibility, and conformality.
However, PE are governed by large feature sizes, prohibiting the realization of complex printed Machine Learning (ML) classifiers.
Leveraging PE's ultra-low non-recurring engineering and fabrication costs, designers can fully customize hardware to a specific ML model and dataset, significantly reducing circuit complexity.
Despite significant advancements, state-of-the-art solutions achieve area efficiency at the expense of considerable accuracy loss.
Our work mitigates this by designing area- and power-efficient printed ML classifiers with little to no accuracy degradation.
Specifically, we introduce the first sequential Support Vector Machine (SVM) classifiers, exploiting the hardware efficiency of bespoke control and storage units and a single Multiply-Accumulate compute engine.
Our SVMs yield on average 6x lower area and 4.6\% higher accuracy compared to the printed state of the art.
\end{abstract}

\begin{IEEEkeywords}
Machine Learning, Support Vector Machines, Printed Electronics
\end{IEEEkeywords}

\bstctlcite{IEEEexample:BSTcontrol} % More than 6 authors --> use et. al

\documentclass[../main.tex]{subfiles}
\graphicspath{{../images/}}
\makeatletter
\def\input@path{{../images/}}
\makeatother
\begin{document}
\section{Introduction}
\begin{figure}
\centering
\begin{tikzpicture}
\node[inner sep=0pt] (ws) at (0, 0) {
\includegraphics[height=.4\textwidth, trim={10cm 0 10cm 0},clip]{world_space.png}};
\node[inner sep=0pt] (cs) at (6,0) {\includegraphics[height=.4\textwidth, trim={10cm 1cm 10cm 4cm},clip]{conf_space.png}};
\end{tikzpicture}
\vspace{-5pt}
\label{fig:pbrm_intro}
\caption{\textbf{Left}: Shows world space obstacles as grey spheres. Robots start and goal configuration is colored red and green, respectively. Configurations along the computed path are colored transparent blue. \textbf{Right:} Mapped world space scenario to configuration space. Obstacle region is the grey mesh. Red spheres are collision-free regions computed by the neural SCDF. The optimized shortest path in the convex corridor is the blue curve.}
\vspace{-25pt}
\end{figure}
Motion planning is the problem of finding a collision-free trajectory that connects a given start and goal configuration. The planning takes place in the configuration space of the robot. For single body robots, like mobile robots or drones, the configuration space and the world space are usually the same. This simplifies the planning, since explicit obstacle representations are available which enables geometrical tools like separating hyperplanes, smallest distance to obstacles etc., to be used when designing motion planning algorithms. For multi-body robots like manipulators, the situation is completely different. The world space obstacles are usually mapped to non-convex regions, and to make the problem even harder, the mapping is usually not known. Forming explicit representations of the obstacle region in the configuration space is usually too expensive or intractable. Despite all of this, sampling based planners are used with great success, which mainly is due to their use of implicit representations of the obstacle region. The basic idea is to construct a graph in the configuration space that covers and connects the collision-free region. From this graph, a path can be extracted that connects a given start and goal configuration. The approach is computationally expensive, since the graph is constructed with the smallest geometrical building block available, points, which represents a collision-check. Furthermore, the extracted paths from the graph are non-smooth and jagged due to the stochastic nature of the approach. This adds an additional post-processing step to the process, where the paths are shortcutted and smoothened, before the path can be used for tracking. Clearly a lot of time is invested to form this graph and produce smooth paths. Thus, if the obstacles start to move, then all of this work is done in no use, since all points that make up this graph need to be re-verified, which is simply too time consuming to be done in real time.
\\\\
In this work, we want to address the existing drawbacks of the sampling based planners. Our main contribution is an improved motion planner where each vertex in the graph covers a collision-free region in the form of a sphere instead of a point and where the edges are formed with neighboring intersecting spheres. This representation has the advantage of instead of returning piecewise linear paths, returning a sequence of overlapping spheres, i.e. a convex corridor, that connects a given start and goal configuration, illustrated in Figure \ref{fig:pbrm_intro}. This convex corridor allows us to use convex optimization to produce smooth trajectories, instead of computationally expensive post-processing methods. The representation further allows us to estimate the coverage of the collision-free space, which gives us awareness and feedback in the offline roadmap construction phase. Finally, our representation is simple to adapt to moving obstacles, simply requery for the new radii and recheck for intersections. 
\\\\
The spherical collision-free regions are formed using a signed distance function (SDF), which is a function that returns the smallest distance from an arbitrary point to the boundary of an obstacle. As the name implies, the distance is signed, thus if the point is inside the obstacle it is negative otherwise positive. If the distance is positive, a sphere with radius equal to the distance is guaranteed to cover a collision-free region. Using an SDF in motion planning is not new, but what is novel about our approach is that we express the distance in the configuration space instead of the world space and by doing so allows us to form these convex collision-free regions. We refer to the resulting SDF as a signed configuration distance function (SCDF). Computing an SCDF analytically is non-trivial, our approach is therefore to parameterize the SCDF with a deep neural network and learn the mapping by supervised learning. Our resulting neural SCDF can compute distances for different parameter values of obstacle shapes and we also show how multiple distances can be combined, thus making our approach flexible.
\section{Related work}
Motion planning algorithms can roughly be divided into three families, grid-based, sampling based and optimization based methods. Grid-based methods (GBM) discretize the planning space from which a graph is then compiled. A standard search method is A$^\star$ \citep{a_star}, which is classified as an \textit{informed} search method, since it employs a heuristic function to speed up the search. A$^\star$ guarantees to return an optimal path at the level of discretization used. GBMs usually discretize the planning space by a regular lattice and this limits the GBMs to problems with low dimensionality due to the curse of dimensionality. Thus, GBMs are usually limited to single-body robots where the degrees of freedom (DOF) are low. To overcome the inherent scaling problem with the GBMs, stochastic methods are usually used for multi-body robots. These methods are termed as sampling-based methods (SBM) and core members within this family are the rapidly-exploring random trees (RRT) \citep{rrt} and the probabilistic roadmap (PRM) \citep{prm}. RRT grows a tree from the start configuration and explores the collision-free region in a rapid way until it is able to connect to the goal region. RRT is usually improved by bi-directional planning \citep{rrt_connect}, i.e. an additional tree is grown from the goal configuration and the trees are tested for connection after any tree has been expanded. RRT is a single-query method, thus it searches for a path from scratch each time it is queried. Contrary to this, PRM is a multi-query method, which solves for multiple queries without starting from scratch. PRM does this by creating a roadmap (graph) that covers the collision-free space as an offline step. The graph is then used to solve for multiple queries. PRMs are used in cases where the environment does not change since the extra offline step is too computationally costly and needs to be re-done if the environment is changed. In our work, we address this inherent issue by using a different roadmap representation. Our vertices in the graph cover a collision-free region in the form of spheres and we form the edges by checking for intersecting spheres. If something in the environment changes, we recompute the spheres radii and recheck the intersections, without relying on collision detection. We use a trained neural network to compute the sphere radius, therefore querying for the radius can be done fast, hence our representation enables the PRM for dynamic environments.
\\\\
In the recent decades, optimization based methods (OBM) \citep{chomp, schulman, itomp, stomp} have been introduced as an alternative to SBM for multi-body robots. Like the SBM, the OBMs scale well to higher dimensional problems and produce smoother motion. It is common to use a SDF in the optimization since it is a smooth function, thus enabling gradient-based methods. However, the standard way of expressing the SDF is in world space. The distance therefore needs to be mapped to the configuration space by the forward kinematics. This mapping makes the optimization problem a non-linear program (NLP), which is computationally expensive to solve. Recently, a different approach has been proposed. In \cite{mp_gcs} motion planning is formulated as a convex optimization problem by using the graph of convex sets framework \citep{gcs}. The underlying idea is to decompose the collision-free space into intersecting convex sets from which a convex optimization problem is formulated. In cases where an explicit representation of the obstacles in the configuration space exists, like for single-body robots, creating collision-free convex regions can be done fast \citep{iris}. For multi-body robots, this is non-trivial. Existing work does this successfully \citep{iris_nlp, iris_c} by an optimization based approach, but the methods are still too time consuming to be used in the presence of moving obstacles. Our approach is instead to use deep learning to learn an SDF expressed in the configuration space. With this, we can query for shortest distances to the collision boundary, which allows us to expand spherical regions which are collision-free. Our approach is fast and therefore enables our suggested roadmap planner to be used in dynamic environments.
\\\\
Recent research has focused on learning collision detection \citep{fk_kernel_distance, diffco, graphdistnet} by predicting the signed distance between the robot links and the surrounding obstacles in the world space. The learned SDF is used in trajectory optimization but since the distance is expressed in the world space, the problem becomes an NLP and therefore takes a long time to solve. We take a novel approach and suggest to instead express the signed distance in the configuration space. This allows us to improve the PRM at the same time as it enables convex optimization for trajectory optimization, which runs faster and is more reliable than NLP solvers. In \cite{cspf} a learned signed distance function in the configuration space is proposed similar to our approach. However, their approach is restricted to point cloud representations, while we propose to represent the obstacles as parameterized geometric shapes, e.g. spheres. Furthermore, we also show how to use our learned SCDF to improve an existing roadmap planner.
\section{Problem formulation}
A robot is located in the world space, $\W \subset \R^3 $. The unique location of the robot is given by its configuration $\q \in \C$, where $\C$ is the configuration space. The set of points covered by the robots bodies at a certain configuration is expressed as $\B(\q) \subset \W$. The robot is surrounded by $\NrObst$ obstacles $\O = \bigcup_{i=1}^{\NrObst} \O_i$, where  $\O_i \subset \W$. The representation of the obstacle in the configuration space is the set $\C\O_i = \{\q \in \C \: |\: \B(\q) \cap \O_i \neq \emptyset \}$. The obstacle space is formed as $\Co = \bigcup_{i=1}^{\NrObst} \C \O_i$. The complement is referred to as the free space, $\Cf = \C \setminus \Co$. The path planning problem is a tuple, ($\Cf$, $\qStart$, $\qGoal$), where we want to connect a query pair, consisting of a start, $\qStart$, and goal configuration, $\qGoal$, with a geometric path, $\q(s): [0, 1] \mapsto \Cf$, such that $\q(0)=\qStart$ and $\q(1)=\qGoal$, or report correctly when such a path does not exist.
\end{document}


\section{Related Work} \label{sec:related}

% \textbf{Adversarial Attack}
\textbf{Attacks on SLAM.} 
%With the rise of machine learning, 
The robustness of computer vision systems is being actively investigated. With the emergence of adversarial images in the digital domain by adding optimized noise directly to images~\cite{szegedy2013intriguing,carlini2017towards}, researchers find that such attacks also exist physically in the real world \cite{eykholt2018robust,song2018physical,zhao2019seeing}. To fill the gap between attacks in the digital and physical worlds, recent studies have demonstrated that attacks on real-world computer vision systems are practical \cite{eykholt2018robust,li2019adversarial,man2020ghostimage,sharif2016accessorize,zhao2019seeing,zhou2018invisible}. However, attacks on traditional computer vision methods such as SLAM are relatively less explored. \cite{yoshida2022adversarial} proposes an attack against the scan matching algorithm in LiDAR-based SLAM, while most SLAMs in AR/VR devices rely on different sensors like RGB/depth cameras and IMUs. \cite{ikram2022perceptual} and \cite{chen2024adversary} mislead visual SLAM by poisoning the images with special patterns, and \cite{wang2021can} causes the camera to fail using infrared light. In our work, we demonstrate attacks on Visual-Inertial SLAM (VI-SLAM) by perturbing the IMU readings, rather than cameras, and showing its impact on XR user experience. 

\textbf{Acoustic Injection Attacks.} Among various physical attacks, acoustic injection attacks are attractive due to their low cost. Son~\etal~\cite{son2015rocking} were the first to introduce acoustic attacks on MEMS gyroscopes, demonstrating how these attacks could lead to sensor denial-of-service and result in drone crashes. WALNUT~\cite{trippel2017walnut} expanded on this by developing output biasing and control attacks that enable precise manipulation of MEMS accelerometer outputs using modulated sound waves. Wang et al.~\cite{wang2017sonic} demonstrated a sonic gun, showcasing the vulnerability of various smart devices (\eg drones and self-balancing vehicles) to acoustic attacks. Tu et al. \cite{tu2018injected} designed side-swing and switching attacks to alter the outputs of MEMS gyroscopes and accelerometers. Furthermore, Ji et al. \cite{ji2021poltergeist} fool the object detectors by applying acoustic attack to the image stabilizers commonly used in modern cameras. However, none of the existing works study the relationship between the acoustic injections and SLAM outputs on recent XR devices. 

% \zijian{Do we need one session about security in AR/VR?}
% \yicheng{TODO}
%\jiasi{cite the AIVR paper (UMass Amherst?) paper is we have not already. They add IMU perturbation but w/o SLAM, iirc} \yicheng{Cited}

\textbf{XR Security and Privacy.} 
%Security and privacy concerns in XR systems have gained significant attention. 
For single-user XR systems, researchers have demonstrated various side-channel attacks to extract sensitive information (\eg keystrokes) through video feeds~\cite{ling2019know}, head movements~\cite{nair2023unique, slocum2023going}, architectural hints~\cite{zhang2023its,shang2020arspy}, power usage~\cite{li2024dangers}, and EM side-channel leakages~\cite{al2021vr}. In multi-user XR systems, Su et al.~\cite{su2024remote} use avatar motion data to infer keystrokes in shared VR environments. Slocum et al.~\cite{slocum2024doesn} reveal vulnerabilities in the shared state frameworks of multi-user AR. Similarly, Lebeck et al.~\cite{lebeck2017securing} highlight risks like deceptive virtual objects and emphasize access control for managing shared physical and virtual spaces. Ruth et al.~\cite{ruth2019secure} further propose a secure multi-user AR framework focusing on content sharing and permissions.
Chandio et al.~\cite{chandio2024stealthy} %introduced a multi-modal spatiotemporal attack that 
simultaneously manipulated visual and inertial sensors to disrupt XR pose estimation. However, their study evaluated the attack using offline datasets and assumed the attacker's capability to manipulate IMU data streams through acoustic means, without real experiments. Ours is the first to demonstrate acoustic injection attacks on recent XR devices, like the Hololens 2, in the real world.
 


\section{Printed Bespoke Sequential SVMs}
\label{sec:svm}
\begin{figure}[!t]
    \centering
    \includegraphics[width=\columnwidth]{figures/sequential_svm-idx.pdf} \vspace{-4ex}
    \caption{\blue{Overview of our proposed sequential \gls{svm} circuits.
    They consist of three main components memory, SVM compute engine and control unit.}}
    \label{fig:architecture}\vspace{-3ex}
\end{figure}

\gls{svm} is a supervised learning algorithm able to classify data by determining the optimal hyperplane for separating classes in a high-dimensional feature space.
\glspl{svm} are effective in handling small, high-dimensional datasets, offering robustness against overfitting.
In brief, \glspl{svm} compute a number of support vectors
%, one for each pair of classes,
and based on the obtained results, determine the class with the most wins.
In printed hardware, linear kernels are typically preferred within support vectors, to maximize area efficiency.
Existing implementations~\cite{Mubarik:MICRO:2020:printedml, Armeniakos:TCAD2023:cross} design fully parallel bespoke circuits, where model parameters are hardwired into the circuit implementation, eliminating the need for costly sequential elements in \gls{pe}, but requiring a hardware multiplier for each \gls{svm} weight.
In contrast, our work focuses on designing sequential \glspl{svm} to minimize the required arithmetic units, making it crucial, nevertheless, to optimize the cost of memory and registers.
Notably, in CMOS technologies, a D Flip-flop's area-equivalent is $4$ NAND gates, whereas in EGFET, this number rises to $6$--a $50$\% increase.

An abstract overview of our proposed sequential SVM circuit is shown in~\cref{fig:architecture}.
It consists of three main components, i.e., parameter storage, control, and compute, all working in sync to fold the entire \gls{svm} prediction over a single \gls{mac} unit. 
Each iteration involves fetching the respective support vector from memory (as determined by the control unit) and computing it in the support vector engine (a multi-cycle operation over a single \gls{mac}).
The result is fed back to the control unit, which either advances the computation (intermediate step) or selects the winning class (final step).
Importantly, our design choices for implementing our sequential \gls{svm} circuits target area efficiency.
Nevertheless, since power consumption in EGFET circuits is mostly static and internal (short-circuit), minimizing area also effectively reduces power.

\subsection{SVM Training} \label{sec:train}
Two established strategies for SVM multi-class classification are OvO and OvA (One-vs-All)~\cite{bishop2006pattern:ovo_ova}.
In OvO, each binary classifier distinguishes between two classes, requiring $\frac{n(n-1)}{2}$ classifiers for $n$ classes.
OvA uses $n$ classifiers, each separating one class from the rest.
OvO requires storing more support vectors but works with smaller subsets of training data, avoiding accuracy loss from imbalanced datasets.
Prioritizing high accuracy and relying on the inherent area efficiency of our sequential implementation, we choose OvO.
For the examined datasets (Section~\ref{sec:eval}), using quantized support vectors, OvO delivers $8.7$\% higher accuracy, on average, compared to OvA.


%Training is performed using scikit-learn's LinearSVC class, with randomized parameter optimization, to train a classifier for each possible class pair.
%Inputs are normalized to $[0,1]$, and training/testing is done with a random $80$\%/$20$\% split.
Training is performed using scikit-learn's LinearSVC class to train a classifier (support vector) for each possible class pair.
Inputs are truncated to $4$-bit fixed-point, and post-training, weights and biases are quantized with min-max linear scaling to the lowest precision that results in negligible accuracy loss (within $0.5$\%).
Additionally, SVM inference is profiled to determine the minimum precision needed for the support vector engine’s accumulator.
The Verilog description for each trained SVM is automatically generated using code templates.
The extracted precisions and SVM parameters are stored in a configuration file.
The support vector engine is the same, with only precision adjustments, while a unique control unit is generated for each model.

\subsection{Support Vector Storage}\label{sec:memory}
Sequential architectures must store model parameters in memory (weights and biases in our \glspl{svm}).
A printed \gls{rom} is a prominent choice for this.
A compact \gls{rom} was proposed and fabricated in~\cite{Bleier:ISCA:2020:printedml}, outperforming other state-of-the-art designs.
It uses a crossbar architecture, where cross-points are shorted by printing conductive material (such as PEDOT:PSS) to represent bit values.
By varying the geometry of the conductive material, multi-bit values can be stored in a single printed dot.
An analog-to-digital converter (ADC) is needed to read stored values.
Considering the cost of ADCs and ROM cells in~\cite{Bleier:ISCA:2020:printedml}, we deduce that storing \mbox{$2$-bit} values is optimal for reading $2$-bit to $8$-bit words as for our model parameters.
This assumes the use of one to four \mbox{$2$-bit} ADCs to maintain reasonable performance.
Since our design uses a single-MAC \gls{svm} engine, we need to process only a single model parameter (weight or bias) per cycle.
One support vector is stored per crossbar row.
Based on the selected support vector (from the control unit), the respective row is activated. 
Then, a set of columns, depending on the precision of the model parameters, are activated each cycle.
A counter (see Section~\ref{sec:engine}) that selects the appropriate parameter is used as the column index.
Nevertheless, such an architecture is highly vulnerable to printing variations, which can alter stored parameters and lead to significant accuracy degradation~\cite{Zhao:DATE2023}. 

As a more robust alternative, we propose the use of bespoke MUXs.
The MUX input data are hardwired to the model parameters, with the row and column indexes described above serving as the input select signals.
While this approach offers improved robustness, it is less dense than ROM storage.
For example, for the Pendigits dataset (see Section\ref{sec:eval}) that features $45$ support vectors with $18$ $8$-bit parameters each, the area, power, and latency of the ROM are $2.4$cm$^2$, $1.9$mW, and $15$ms, respectively.
The corresponding values for MUX-based storage are $5.0$cm$^2$, $5.6$mW, and $19$ms, respectively.
Although less hardware-efficient, the overhead of the MUX-based storage is not prohibitive, especially considering that this example represents the worst-case difference among all datasets examined.
Therefore, we choose bespoke MUX-based storage for our SVMs to prioritize accuracy and robustness.



% \noindent\textbf{MAC-Based Computational Engine:}
\subsection{Single MAC Support Vector Engine} \label{sec:engine}
Our control unit makes decisions by evaluating one support vector at a time.
Hence, we design a support vector engine to compute the corresponding output:
\begin{equation}
    y = \begin{cases}
        1 & \text{if}\, \sum_{i=1}^{m} w_i x_i + b \geq 0 \\
        0 & \text{otherwise,}
    \end{cases}
\label{eq:compute_mac}
\end{equation}
where $w_i$ and $b$ are the support vector's weights and bias, and $x_i$ are the input activations.
Striving for area efficiency, we fold the computation of \eqref{eq:compute_mac} over a single MAC unit.

As shown in \cref{fig:architecture}, a register stores the accumulation result, whose size is minimized based on partial sum profiling during SVM training.
A second small register is used to store a counter, which serves as the column index for the memory and for synchronization.
In the first cycle, the bias is fetched, initializing the accumulation register.
In each subsequent cycle, a weight is read from memory, multiplied by the respective input, and the product is accumulated.
Once all model parameters are processed, the engine outputs a ready signal, providing the inverted sign of the accumulation.

\begin{figure}[!t]
    \centering
    % \includegraphics[width=\columnwidth]{example-image-b}
    \includegraphics[width=.9\columnwidth]{figures/sequential_svm_example-cropped.pdf}
    \caption{Example execution of our sequential \glspl{svm}. $4$ classes and $6$ input features/weights are considered. Support vectors corresponding to selected nodes are shown on the right, along with the classification output of the support vector engine. The output class is predicted in $3\times 7=21$ cycles. Fixed point values are represented by integers.
    }
    \label{fig:example}\vspace{-4ex}
\end{figure}


% \noindent\textbf{Control Unit:}
\subsection{Control Unit} \label{sec:control_unit}
OvO computes a support vector for each pair of classes, with the class attaining the highest score being selected as the SVM output.
In~\cite{Armeniakos:DATE2022:axml,Armeniakos:TCAD2023:cross}, all support vectors are computed in parallel, and a voter circuit determines the winning class.
While this approach requires a voter, a sequential implementation also requires a log$_2\frac{n(n-1)}{2}$ bit counter and additional registers to track the winning class and its score.
To minimize register use, we implement OvO as a decision-directed acyclic graph (DDAG)~\cite{ovo}--specifically, a binary decision tree--for our control \gls{fsm}, requiring only a small register of log$_2\frac{n(n-1)}{2}$ bits to store the FSM state.
%A software implementation of OvO as a DDAG is detailed in~\cite{ovo}, where each state represents the current winning class.
Each DDAG node (FSM state) represents the current winning class, and is linked to a support vector that compares this class against a new class, yet to be considered.
FSM selects the corresponding support vector by its row index in the memory.
%According to its row index, the support vector is fetched from memory.
After $m+1$ cycles (where $m$ is the number of input features), the support vector is computed in our engine, and the winner of this comparison is determined.
The newly identified winning class becomes the current winner, prompting the DDAG to transition to the right if the new class prevails, or to the left otherwise.

The hardware implementation of the FSM is straightforward.
It uses a hardcoded row index per state and only requires a MUX to select between two predefined next states based on the support vector engine's output.
\cref{fig:example} presents a detailed example of our sequential execution and FSM flow.
The FSM has $\frac{n(n-1)}{2}$ states (the number of OvO support vectors) but requires only $n-1$ support vector evaluations, resulting in a total of $(n-1) \times (m+1)$ cycles for each classification.


% The binary decision tree is implemented via a lightweight \gls{fsm} for our control unit.
% Each node of the tree corresponds to a binary classifier, distinguishing between two selected classes.
% The binary decision is produced after $m+1$ cycles within our computational engine, as described in \cref{eq:compute_mac}.
% Given the predicted binary class at a given node, the decision tree progresses left or right, traversing through a single path and multiple support vectors, until it reaches a leaf node, where the final class decision is made.
% We observe that our approach does not necessitate a costly voting circuit at its output.
% Overall, for $n$-class classification, only $n$ iterations are required to compute the output class.
% By leveraging the lightweight binary classifiers and their small associated kernels from the OvO approach, this hierarchical structure minimizes both area and power consumption, which are key requirements in printed devices.




% An example of the algorithmic flow of our bespoke sequential \gls{svm} implementation and its \gls{fsm}-based control unit can be seen in \cref{fig:example}.
% Assuming $5$-class classification and $6$ input features (weights) per input (support) vector, \cref{fig:example} presents a step-by-step walkthrough of each stage of our tree-based \gls{svm}.
% The root node fetches the $S_{1,5}$ classifier from our \gls{mux}-based memory to distinguish between classes $1$ and $5$.
% The computational engine produces a positive sum, indicating that class $5$ prevails, signaling a move on the right towards lower levels of the tree.
% A path is formed as each binary classification produces a dominant class, to finally reach at the leaf nodes.
% Without the need for any voting mechanism, class $4$ is predicted after a total of $n\times m=30$ cycles.
% \cref{fig:example} showcases the simplicity and effectiveness of our sequential \gls{svm} implementation in correctly classifiying input data.

\section{Evaluation}
We provide three sets of insights into this section, organised as \textit{findings (F*)}. We quantitatively study the effect of the adversarial and counterfactual perturbations on the performance of informal reasoners and autoformalisation methods. Then, we dive deeper into method variants. Finally, 
we analyse the nature of formalisation errors made by the models.

\subsection{Robustness Analysis}
\paragraph{\textbf{\emph{F1: Noise perturbations have a stronger effect on formalisation methods than informal \ac{LLM} reasoners.}}}
Table~\ref{tab:distraction_k4_formalisation} shows that, on average, the accuracy of both direct and \ac{CoT} informal reasoning remains between $73\%$ and $74\%$ in the face of added noise. While the autoformalisation method performs similarly to informal reasoners on the original dataset, its performance decreases between $4\%$ and $11\%$. The accuracy drops especially with logical (L) and tautological (T) distractions, whose logical language formats trick the \ac{LLM} into formalizing the noisy clauses. On the other hand, the linguistically complex and more natural sentences of encyclopedic distractions show a minor effect, suggesting that \acp{LLM} successfully avoids formalizing the more complicated sentences.

\paragraph{\textbf{\emph{F2: All \ac{LLM}-based reasoning methods suffer a drop for counterfactual perturbations.}}} % influence .}}}
Table~\ref{tab:distraction_k4_formalisation} shows that counterfactual statements cause a significant decrease in performance for both the informal reasoners and autoformalisation methods of between $12\%$ and $13\%$ on average. 
Moreover, this observation also holds for all tested models, i.e., none are robust towards counterfactual perturbations across every evaluated dimension. Even the strongest model, GPT 4o-mini, yields a performance of 63-68\%, which is relatively close to the random performance of 50\%. The high impact of counterfactual statements (the single ``not'' inserted) could be due to the inability of \acp{LLM} to overwrite prior knowledge with explicitly stated information or memorization of the answers. We study the error sources further in §\ref{subsec:errors}.  

\noindent \paragraph{\textbf{\emph{F3: Introducing multiple noise sentences has an effect only for logical distractions.}}}
We show the impact of introducing between one and four sentences for the two top-performing autoformalisation models in Figure~\ref{fig:length_distraction}. The figure shows similar trends with and without counterfactual perturbations.
As additional logical distractions are introduced, the model performance consistently decreases. Tautological (T) distractions lead to a decline in accuracy with a single disruptive sentence, yet adding more noise does not worsen the outcome. 
The tautological corpus introduces truth constants for all sentences as a persistent unseen logical construct. Given that this leads only to a decrease for a single occurrence, we can assume that a model can consistently handle the same unseen logical construct. In contrast, the logical corpus increases the chance of adding text, requiring new, previously unseen reasoning constructs for each added sentence. The impact of encyclopedic noise remains negligible, generalising F1 to $k$ sentences. Similarly, counterfactual perturbations remain much more effective for all settings, generalising F2.

\begin{table}[!t]
\small
\setlength{\modelspacing}{2pt}
\setlength{\tabcolsep}{1.7pt} % Default value: 6pt
\setlength{\belowrulesep}{4pt}
\begin{threeparttable}
    \centering
    \begin{tabular}{cc l r rrr @{\quad} rrrr}
\toprule
\multirow{2}{*}{} & \multirow{2}{*}{} & Reasoning & \multirow{2}{*}{O} & \multicolumn{3}{c}{Distraction} & \multicolumn{4}{c}{Counterfactual} \\
 & & Format & & E& L & T & $\text{O}_C$ & $\text{E}_C$& $\text{L}_C$ & $\text{T}_C$\\
\midrule
\multirow{6}{*}{\rotatebox{90}{Gemma-2}} & \multirow{3}{*}{\rotatebox{90}{9b}}
   & Informal (direct) & \textbf{0.78} & \textbf{0.80} & \textbf{0.79} & \textbf{0.77} & 0.58 & 0.52 & 0.50 & 0.59 \\
 & & Informal (CoT) & 0.72 & 0.78 & 0.73 & 0.76 & 0.61 & \textbf{0.57} & \textbf{0.60} & \textbf{0.66} \\
 & & Formal (FOL) & 0.62 & 0.58 & 0.52 & 0.53 & \textbf{0.63} & 0.52 & 0.46 & 0.46 \\[\modelspacing]
\cmidrule{2-11}
 & \multirow{3}{*}{\rotatebox{90}{27b}} 
   & Informal (direct) & 0.71 & 0.69 & \textbf{0.66} & \textbf{0.68} & 0.59 & 0.51 & 0.54 & 0.59 \\
 & & Informal (CoT) & 0.66 & 0.65 & 0.64 & 0.63 & 0.62 & 0.58 & \textbf{0.62} & \textbf{0.64} \\
 & & Formal (FOL) & \textbf{0.74} & \textbf{0.74} & 0.61 & 0.61 & \underline{\textbf{0.72}} & \underline{\textbf{0.67}} & 0.58 & 0.51 \\[\modelspacing]
\midrule
\multirow{6}{*}{\rotatebox{90}{Mistral}} & \multirow{3}{*}{\rotatebox{90}{7B}} 
   & Informal (direct) & 0.77 & \textbf{0.77} & 0.75 & \textbf{0.79} & \textbf{0.63} & \textbf{0.54} & \textbf{0.54} & \textbf{0.66} \\
 & & Informal (CoT) & \textbf{0.79} & 0.75 & \textbf{0.77} & 0.78 & 0.55 & 0.52 & \textbf{0.54} & 0.58 \\
 & & Formal (FOL) & 0.62 & 0.58 & 0.54 & 0.57 & 0.50 & \textbf{0.54} & 0.51 & 0.52 \\[\modelspacing]
\cmidrule{2-11}
 & \multirow{3}{*}{\rotatebox{90}{Small}} 
   & Informal (direct) & \textbf{0.77} & \textbf{0.76} & \textbf{0.76} & \textbf{0.75} & 0.61 & 0.51 & 0.56 & 0.59 \\
 & & Informal (CoT) & 0.72 & 0.72 & 0.72 & 0.71 & \textbf{0.62} & \textbf{0.59} & \textbf{0.62} & \textbf{0.68} \\
 & & Formal (FOL) & 0.68 & 0.59 & 0.53 & 0.64 & 0.54 & 0.55 & 0.49 & 0.51 \\[\modelspacing]
\midrule
\multirow{6}{*}{\rotatebox{90}{Llama-3.1}} & \multirow{3}{*}{\rotatebox{90}{8B}} 
   & Informal (direct) & 0.63 & 0.61 & 0.64 & 0.66 & 0.61 & \textbf{0.62} & 0.59 & 0.61 \\
 & & Informal (CoT) & 0.73 & \textbf{0.73} & \textbf{0.71} & \textbf{0.72} & \textbf{0.62} & 0.59 & \textbf{0.61} & \textbf{0.65} \\
 & & Formal (FOL) & \textbf{0.77} & 0.71 & 0.63 & 0.52 & 0.60 & 0.58 & 0.55 & 0.52 \\[\modelspacing]
\cmidrule{2-11}
 & \multirow{3}{*}{\rotatebox{90}{70B}} 
   & Informal (direct) & 0.77 & 0.74 & 0.74 & 0.73 & 0.62 & 0.53 & 0.56 & 0.64 \\
 & & Informal (CoT) & \textbf{0.78} & \textbf{0.75} & \textbf{0.76} & \textbf{0.76} & 0.64 & 0.61 & \textbf{0.66} & \underline{\textbf{0.73}} \\
 & & Formal (FOL) & 0.74 & 0.73 & 0.71 & 0.71 & \textbf{0.66} & \textbf{0.62} & 0.59 & 0.57 \\[\modelspacing]
 \midrule
\multirow{3}{*}{\rotatebox{90}{GPT}} & \multirow{3}{*}{\rotatebox{90}{4o-mini}} 
   & Informal (direct) & 0.78 & 0.77 & 0.79 & 0.79 & 0.64 & 0.61 & 0.61 & 0.63 \\
 & & Informal (CoT) & 0.80 & 0.80 & \underline{\textbf{0.81}} & \underline{\textbf{0.82}} & \textbf{0.68} & \textbf{0.63} & \underline{\textbf{0.68}} & \textbf{0.64} \\
 & & Formal (FOL) & \underline{\textbf{0.84}} & \underline{\textbf{0.82}} & 0.73 & 0.79 & 0.63 & 0.62 & 0.57 & 0.54 \\[\modelspacing]
 \midrule
\multicolumn{2}{c}{\multirow{3}{*}{\textbf{Avg}}} 
 & Informal (direct) & 0.74 & 0.73 & 0.73 & 0.73 & 0.61 & 0.55 & 0.56 & 0.62 \\
 & & Informal (CoT) & 0.74 & 0.74 & 0.73 & 0.74 & 0.62 & 0.58 & 0.62 & 0.65 \\
  & & Formal (FOL) & 0.72 & 0.68 &	0.61 & 0.62 & 0.61 & 0.59 & 0.54 & 0.52 \\
\bottomrule
\end{tabular}
\caption{Accuracies of informal and autoformalisation-based deductive reasoners. The best overall model per dataset is underlined; the best model version is marked in bold.}
\label{tab:distraction_k4_formalisation}
\end{threeparttable}
\end{table} 

\begin{figure}[!t]
    \centering
    \scriptsize
    \begin{tikzpicture}
        \begin{axis}[name=gpt,
            title={GPT-4o-mini},
            width=0.6\linewidth,
            height=0.6\linewidth,
            xlabel={\# Noise sentences},
            ylabel={Accuracy},
            xmin=-0.1, xmax=4.1,
            ymin=0.5, ymax=0.9,
            xtick={1,2,4},
            ytick={0.55, 0.6, 0.65, 0.75, 0.8, 0.85},
            title style={yshift=-0.6em},
            legend style={at={(1,-0.15)},
	           anchor=north,legend columns=-1},
            x label style={at={(axis description cs:1,-0.05)},anchor=north},
            y label style={at={(axis description cs:-0.15,0.5)},anchor=south},
            ymajorgrids=true,
            grid style=dashed,
        ]
            \addplot[color=blue, mark=square,]
                coordinates {
                (0,0.848076939582825)(1,0.823076903820038)(2,0.826923072338104)(4,0.821153819561005)
                };
            \addplot[color=red, mark=triangle,]
                coordinates {
                (0,0.848076939582825)(1,0.817307710647583)(2,0.801923096179962)(4,0.759615361690521)
                };
            \addplot[color=green, mark=diamond,] 
                coordinates {
                (0,0.848076939582825)(1,0.767307698726654)(2,0.769230782985687)(4,0.803846180438995)
                };
            \addplot[color=blue, mark=square*] 
                coordinates {
                (0,0.627777755260468)(1,0.622222244739533)(2,0.600000023841858)(4,0.633333325386047)
                };
            \addplot[color=red, mark=triangle*,] 
                coordinates {
                (0,0.627777755260468)(1,0.611111104488373)(2,0.611111104488373)(4,0.594444453716278)
                };
            \addplot[color=green, mark=diamond*,] 
                coordinates {
                (0,0.627777755260468)(1,0.572222232818604)(2,0.538888871669769)(4,0.555555582046509)
                };
                \legend{E,L,T,$\text{E}_C$, $\text{L}_C$ , $\text{T}_C$}
        \end{axis}

        \begin{axis}[name=llama, at={($(gpt.east)+(0.1cm,0)$)},anchor=west,
            title={Llama 3.1 70b},
            width=0.6\linewidth,
            height=0.6\linewidth,
            xmin=-0.1,, xmax=4.1,
            ymin=0.5, ymax=0.9,
            xtick={1,2,4},
            ytick={0.55, 0.6, 0.65, 0.75, 0.8, 0.85},
            title style={yshift=-0.6em},
            yticklabel=\empty,
            ymajorgrids=true,
            grid style=dashed,
        ]
            \addplot[color=blue, mark=square,]
                coordinates {
                (0,0.838461518287659)(1,0.817307710647583)(2,0.805769205093384)(4,0.817307710647583)
                };
            \addplot[color=red, mark=triangle,]
                coordinates {
                (0,0.838461518287659)(1,0.819230794906616)(2,0.803846180438995)(4,0.771153867244721)
                };
            \addplot[color=green, mark=diamond,]
                coordinates {
                (0,0.838461518287659)(1,0.803846180438995)(2,0.807692289352417)(4,0.805769205093384)
                };
            \addplot[color=blue, mark=square*]
                coordinates {
                (0,0.627777755260468)(1,0.622222244739533)(2,0.577777802944183)(4,0.594444453716278)
                };
            \addplot[color=red, mark=triangle*,]
                coordinates {
                (0,0.627777755260468)(1,0.583333313465118)(2,0.561111092567444)(4,0.577777802944183)
                };
            \addplot[color=green, mark=diamond*,]
                coordinates {
                (0,0.627777755260468)(1,0.627777755260468)(2,0.566666662693024)(4,0.577777802944183)
                };
        \end{axis}
    \end{tikzpicture}
    \caption{Influence of the number of noisy sentences for FOL.}
    \label{fig:length_distraction}
\end{figure}



\subsection{Impact of Method Design}
\paragraph{\textbf{\emph{F4: \ac{CoT} prompting is most impactful when both noise and counterfactual perturbations are applied.}}}
The accuracies for the individual \acp{LLM} in Table~\ref{tab:distraction_k4_formalisation} show that the impact of \ac{CoT} is negligible for noise-only datasets (first four columns). Meanwhile, the benefit from \ac{CoT} is most pronounced in the datasets that combine noise and counterfactual perturbations.
The better-performing informal prompting strategy for a model remains stable for all types of distractions. Still, the decline in performance due to counterfactuals leads to a less consistent preference for a specific prompting style.

\paragraph{\textbf{\emph{F5: The best-performing grammar differs per model and is unstable across data versions.}}}

The evaluation of different logical forms for formal \ac{LLM}-based reasoning in Table~\ref{tab:distraction_k4_logical_form} shows the preference of some models for specific syntactic formats.
Llama 3.1 70B has a considerable improvement of $12\%$ with TPTP syntax on the original set, while Llama 3.1 8B benefits from the R-FOL syntax. However, all grammars show a declining accuracy trend and increased syntax errors for noise perturbations, where the best grammar loses its advantage over the rest. 
When comparing the grammars on the counterfactual partitions, we observe that TPTP is consistently more robust than the standard first-order logic grammar. Here, GPT 4o-mini shows a reduction from $O$ to $O_C$ of $20\%$ for FOL and only $12\%$ for the TPTP grammar. Since this does not correlate with fewer syntax errors, the formalisation in TPTP prevents semantical errors for counterfactual premises. 
A positive reading of these results, especially the minor differences between FOL and R-FOL, is that autoformalisation \acp{LLM} can adapt to the grammar syntax prescribed in the prompt without further loss in performance.

\begin{table}[!t]
\small
\setlength{\modelspacing}{2pt}
\setlength{\tabcolsep}{1.7pt} % Default value: 6pt
\setlength{\belowrulesep}{4pt}
\begin{threeparttable}
    \centering
    \begin{tabular}{cc l r rrr @{\quad} rrrr}
\toprule
\multirow{2}{*}{} & \multirow{2}{*}{} & Grammar & \multirow{2}{*}{O} & \multicolumn{3}{c}{Distraction} & \multicolumn{4}{c}{Counterfactual} \\
 & & Syntax & & E& L & T & $\text{O}_C$ & $\text{E}_C$& $\text{L}_C$ & $\text{T}_C$\\
\midrule
\multirow{6}{*}{\rotatebox{90}{Llama-3.1}} & \multirow{3}{*}{\rotatebox{90}{8B}} 
   & FOL & 0.77 & \textbf{0.71} & 0.61 & \textbf{0.53} & 0.58 & \textbf{0.55} & 0.52 & \textbf{0.56} \\
 & & R-FOL & \textbf{0.78} & 0.69 & \textbf{0.62} & \textbf{0.53} & 0.58 & \textbf{0.55} & \textbf{0.54} & 0.52 \\
 & & TPTP & 0.73 & 0.67 & 0.55 & 0.51 & \textbf{0.68} & 0.54 & 0.46 & 0.51 \\[\modelspacing]
\cmidrule{2-11}
 & \multirow{3}{*}{\rotatebox{90}{70B}} 
   & FOL & 0.76 & 0.73 & 0.71 & \textbf{0.72} & 0.67 & 0.57 & 0.63 & 0.56 \\
 & & R-FOL & 0.76 & 0.73 & 0.67 & 0.71 & 0.64 & 0.57 & 0.53 & 0.64 \\
 & & TPTP & \underline{\textbf{0.88}} & \underline{\textbf{0.84}} & \underline{\textbf{0.81}} & \textbf{0.72} & \underline{\textbf{0.81}} & \underline{\textbf{0.68}} & \underline{\textbf{0.67}} & \underline{\textbf{0.68}} \\[\modelspacing]
\midrule
\multirow{3}{*}{\rotatebox{90}{GPT}} & \multirow{3}{*}{\rotatebox{90}{4o-mini}} 
   & FOL & \textbf{0.84} & \textbf{0.82} & \textbf{0.72} & \underline{\textbf{0.78}} & 0.64 & \textbf{0.63} & \textbf{0.61} & 0.51 \\
 & & R-FOL & \textbf{0.84} & 0.77 & 0.70 & \underline{\textbf{0.78}} & \textbf{0.72} & 0.56 & 0.54 & \textbf{0.63} \\
 & & TPTP & 0.83 & \textbf{0.82} & 0.71 & 0.71 & 0.69 & \textbf{0.63} & 0.57 & 0.57 \\
\bottomrule
\end{tabular}
\caption{Accuracies of different formalisation grammars for autoformalisation.}
\label{tab:distraction_k4_logical_form}
\end{threeparttable}
\end{table} 

\paragraph{\textbf{\emph{F6: Feedback does not help \acp{LLM} self-correct to mitigate robustness issues.}}}
\autoref{tab:distraction_k4_feedback} shows the results with different error recovery mechanisms. The results indicate that no feedback strategy emerges as a winner in the different datasets. 
All feedback variants reduce syntax errors for noise perturbations, but given the lack of a consistent increase in accuracy, the corrected formalisations are most likely to contain semantic errors still. 
The type of feedback message only has a minor influence on correcting syntax errors, whereas Llama 3.1 70b and GPT 4o-mini correct slightly more syntax errors with specific error messages. This finding aligns with \cite{huang2023large}, who also found that \acp{LLM} cannot consistently self-correct their reasoning after receiving relevant feedback.

\begin{table}[!ht]
\small
\setlength{\modelspacing}{2pt}
\setlength{\tabcolsep}{1.7pt} % Default value: 6pt
\setlength{\belowrulesep}{4pt}
\begin{threeparttable}
    \centering
    \begin{tabular}{cc l r rrr @{\quad} rrrr}
\toprule
\multirow{2}{*}{} & \multirow{2}{*}{} & \multirow{2}{*}{Feedback} & \multirow{2}{*}{O} & \multicolumn{3}{c}{Distraction} & \multicolumn{4}{c}{Counterfactual} \\
 & & & & E& L & T & $\text{O}_C$ & $\text{E}_C$& $\text{L}_C$ & $\text{T}_C$\\
\midrule
\multirow{8}{*}{\rotatebox{90}{Llama-3.1}} & \multirow{4}{*}{\rotatebox{90}{8B}} 
   & No recovery & 0.77 & \textbf{0.72} & 0.62 & 0.53 & 0.59 & 0.58 & 0.56 & \textbf{0.56} \\
 & & Error type & \textbf{0.79} & 0.71 & 0.63 & \textbf{0.56} & \textbf{0.66} & 0.54 & 0.52 & 0.51 \\
 & & Error message & 0.78 & 0.71 & \textbf{0.67} & 0.55 & 0.59 & 0.53 & \underline{\textbf{0.64}} & 0.49 \\
 & & Warning & 0.74 & 0.66 & 0.58 & 0.55 & 0.55 & \textbf{0.60} & 0.49 & 0.49 \\[\modelspacing]
\cmidrule{2-11}
 & \multirow{4}{*}{\rotatebox{90}{70B}} 
   & No recovery & \textbf{0.77} & \textbf{0.72} & \textbf{0.73} & 0.71 & \textbf{0.64} & 0.59 & \textbf{0.61} & 0.56 \\
 & & Error type & 0.72 & 0.70 & 0.72 & \textbf{0.73} & 0.62 & 0.56 & 0.60 & 0.58 \\
 & & Error message & 0.71 & 0.70 & \textbf{0.73} & 0.71 & \textbf{0.64} & 0.59 & 0.54 & \underline{\textbf{0.64}} \\
 & & Warning & 0.69 & \textbf{0.72} & 0.72 & 0.72 & 0.62 & \underline{\textbf{0.65}} & \textbf{0.61} & 0.63 \\[\modelspacing]
\midrule
\multirow{4}{*}{\rotatebox{90}{GPT}} & \multirow{4}{*}{\rotatebox{90}{4o-mini}} 
   & No recovery & \underline{\textbf{0.84}} & \underline{\textbf{0.82}} & 0.73 & 0.79 & 0.64 & \textbf{0.62} & 0.56 & \textbf{0.56} \\
 & & Error type & 0.83 & 0.79 & 0.74 & 0.76 & 0.67 & 0.57 & 0.56 & \textbf{0.56} \\
 & & Error message & \underline{\textbf{0.84}} & 0.78 & \underline{\textbf{0.77}} & \underline{\textbf{0.80}} & 0.62 & 0.59 & 0.56 & \textbf{0.56} \\
 & & Warning & \underline{\textbf{0.84}} & 0.75 & 0.73 & 0.76 & \underline{\textbf{0.70}} & 0.61 & \textbf{0.61} & 0.55 \\
 \bottomrule
\end{tabular}
\caption{Accuracies of error recovery strategies.}
\label{tab:distraction_k4_feedback}
\end{threeparttable}
\end{table} 

\subsection{Error Analysis}
\label{subsec:errors}
\paragraph{\textbf{\emph{F7: Autoformalisation increases syntax errors for noise perturbations.}}}
The low performance for noise perturbations correlates with more syntax errors for all models and distraction categories (cf. execution rates in Table~\ref{tab:appendix_k4_formalisation_exec}). The three worst-performing models (both Mistral models, Gemma-2 9b) generate, at best, for $37\%$  and, at worst, for only $4\%$ of the samples, a valid logical form.
Gemma-2 9b and Llama3.1 8b produce more syntax errors than the larger counterparts, suggesting that larger models are more robust towards noise perturbations. 
The accuracy of syntactically valid samples is higher than the informal reasoning methods for most distractions (Table~\ref{tab:appendix_k4_formalisation_vacc}), motivating informal reasoning as a backup strategy for formal reasoning. The error message feedback reveals two common syntax errors: 1) errors by models with an initial low execution rate exhibit issues with the template structure, including using incorrect keywords or adding conversational phrases;
2) perturbation-related errors, the most common of which is using undefined truth constants as part of tautological distractions. 

\paragraph{\textbf{\emph{F8: Autoformalisation increases semantic errors for counterfactuals.}}}
Unlike the introduced noise, counterfactual perturbations do not lead to more syntax errors. The execution rate in Table~\ref{tab:appendix_k4_formalisation_exec} is stable or improves for counterfactuals. However, we see a drop in accuracy for the counterfactual column $\text{O}_C$ in Table~\ref{tab:distraction_k4_formalisation} and can conclude that the number of logical forms with semantic errors has to increase. This suggests that the introduced negation is not correctly formalised. Looking at the warnings generated by the feedback mechanism, for GPT 4o-mini, $161$ warning messages are generated on the unperturbed data. $54$ of these were fixed with a single iteration. Not considering predicates and individuals as part of the context is the most frequent warning across all models. 
\section*{Conclusion}
This paper aims to enhance our understanding of the computational complexity of computing various Shapley value variants. We found that for various ML models --- including decision trees, regression tree ensembles, weighted automata, and linear regression --- both local and global interventional and baseline SHAP can be computed in polynomial time under HMM modeled distributions. This extends popular algorithms, such as TreeSHAP, beyond their empirical distributional scope. We also establish strict complexity gaps between the various SHAP variants (baseline, interventional, and conditional) and prove the intractability of computing SHAP for tree ensembles and neural networks in simplified scenarios. Overall, we present SHAP as a versatile framework whose complexity depends on four key factors: \begin{inparaenum}[(i)] \item model type, \item SHAP variant, \item distribution modeling approach, \item and local vs. global explanations\end{inparaenum}. We believe this perspective provides deeper insight into the computational complexity of SHAP, paving the way for future work.




%We believe that our framework provides a more intricate understanding of SHAP computation complexity across different models, distributions, and variants, paving the way for further research.

Our work opens promising directions for future research. First, expanding our computational analysis to other SHAP-related metrics, such as asymmetric SHAP~\citep{frye20} and SAGE~\citep{covert2020understanding}, would be valuable. Additionally, we aim to explore more expressive distribution classes and relaxed assumptions beyond those in Section \ref{sec:tractable} while maintaining tractable SHAP computation. Finally, when exact computation is intractable (Section \ref{sec:intractable}), investigating the approximability of SHAP metrics through approximation and parameterized complexity theory~\citep{downey2012parameterized} is an important direction.

%Our work opens several promising avenues for future research on the computational properties of explainable AI methods, with a particular focus on SHAP. First, it would be interesting to broaden the computational analysis conducted in this work to include other popular SHAP-related metrics in the literature, such as asymmetric SHAP \cite{frye20} and SAGE \cite{covert2020understanding}. Also, in the future, we aim to explore more expressive distribution classes and relaxed distributional assumptions—extending beyond those examined in Section \ref{sec:tractable} —that still yield tractable SHAP computation. Finally, when exact computation proves intractable (Section \ref{sec:intractable}), it is worthwhile to theoretically investigate the question of the approximability of computing the SHAP metrics across various configurations, through the lens of approximation and parametrized complexity theory \cite{arora2009computational}.

%This paper aims to deepen our understanding of the computational complexity involved in obtaining different Shapley value variants. We found that for a variety of ML models, including decision trees, tree ensembles for regression, weighted automata, and linear regression models — computing both local and global interventional and baseline SHAP can be done in polynomial time when distributions are modeled by HMMs. This extends the distributional scope of popular algorithms like TreeSHAP, which is limited to empirical distributions. Additionally, we demonstrate a strict complexity gap between SHAP variants, showing that interventional and baseline SHAP can be strictly easier to compute than conditional SHAP. Despite these positive results, we uncovered intractability for various SHAP variants in neural networks and tree ensembles. Finally, we provided generalized complexity relations across SHAP variants. We believe that our framework offers a deeper understanding of the complexity involved in computing SHAP across various variants, models, distributions, as well as in both local and global computations, laying the groundwork for future research.

\blue{\section*{Acknowledgment}
This work is funded by the H.F.R.I call “Basic research Financing (Horizontal support of all Sciences)” under the National Recovery and Resilience Plan “Greece 2.0” (H.F.R.I. Project Number: 17048).}

% Generated by IEEEtran.bst, version: 1.14 (2015/08/26)
\begin{thebibliography}{10}
\providecommand{\url}[1]{#1}
\csname url@samestyle\endcsname
\providecommand{\newblock}{\relax}
\providecommand{\bibinfo}[2]{#2}
\providecommand{\BIBentrySTDinterwordspacing}{\spaceskip=0pt\relax}
\providecommand{\BIBentryALTinterwordstretchfactor}{4}
\providecommand{\BIBentryALTinterwordspacing}{\spaceskip=\fontdimen2\font plus
\BIBentryALTinterwordstretchfactor\fontdimen3\font minus \fontdimen4\font\relax}
\providecommand{\BIBforeignlanguage}[2]{{%
\expandafter\ifx\csname l@#1\endcsname\relax
\typeout{** WARNING: IEEEtran.bst: No hyphenation pattern has been}%
\typeout{** loaded for the language `#1'. Using the pattern for}%
\typeout{** the default language instead.}%
\else
\language=\csname l@#1\endcsname
\fi
#2}}
\providecommand{\BIBdecl}{\relax}
\BIBdecl

\bibitem{Bleier:ISCA:2020:printedml}
N.~Bleier \emph{et~al.}, ``Printed microprocessors,'' in \emph{Annu. Int. Symp. Computer Architecture (ISCA)}, jun 2020, pp. 213--226.

\bibitem{smartpackaging2022}
\BIBentryALTinterwordspacing
X.~Luo, ``Application of inkjet-printing technology in developing indicators/sensors for intelligent packaging systems,'' \emph{Current Opinion in Food Science}, vol.~46, p. 100868, 2022. [Online]. Available: \url{https://www.sciencedirect.com/science/article/pii/S2214799322000704}
\BIBentrySTDinterwordspacing

\bibitem{disposable:JSNB:2023}
\BIBentryALTinterwordspacing
A.~Beniwal, P.~Ganguly, A.~K. Aliyana, G.~Khandelwal, and R.~Dahiya, ``Screen-printed graphene-carbon ink based disposable humidity sensor with wireless communication,'' \emph{Sensors and Actuators B: Chemical}, vol. 374, p. 132731, 2023. [Online]. Available: \url{https://www.sciencedirect.com/science/article/pii/S0925400522013740}
\BIBentrySTDinterwordspacing

\bibitem{salivary:Talanta:2020}
\BIBentryALTinterwordspacing
R.~K. Mishra \emph{et~al.}, ``Simultaneous detection of salivary $\delta$9-tetrahydrocannabinol and alcohol using a wearable electrochemical ring sensor,'' \emph{Talanta}, vol. 211, p. 120757, 2020. [Online]. Available: \url{https://www.sciencedirect.com/science/article/pii/S0039914020300485}
\BIBentrySTDinterwordspacing

\bibitem{bodytemperature:sna:2020}
\BIBentryALTinterwordspacing
J.-W. Lee \emph{et~al.}, ``High sensitivity flexible paper temperature sensor and body-attachable patch for thermometers,'' \emph{Sensors and Actuators A: Physical}, 2020. [Online]. Available: \url{https://www.sciencedirect.com/science/article/pii/S0924424720305495}
\BIBentrySTDinterwordspacing

\bibitem{pressuresensor:research:2022}
Y.~Li \emph{et~al.}, ``The soft-strain effect enabled high-performance flexible pressure sensor and its application in monitoring pulse waves,'' \emph{Research}, 2022.

\bibitem{wearable:adma:2022}
\BIBentryALTinterwordspacing
J.~Gao \emph{et~al.}, ``Ultra-robust and extensible fibrous mechanical sensors for wearable smart healthcare,'' \emph{Advanced Materials}, vol.~34, no.~20, p. 2107511, 2022. [Online]. Available: \url{https://onlinelibrary.wiley.com/doi/abs/10.1002/adma.202107511}
\BIBentrySTDinterwordspacing

\bibitem{Wearable:acssensors:2019}
S.~Agarwala \emph{et~al.}, ``Wearable bandage-based strain sensor for home healthcare: Combining 3d aerosol jet printing and laser sintering,'' \emph{ACS Sensors}, vol.~4, no.~1, pp. 218--226, 2019.

\bibitem{healthcare:Nanoscale:2024}
\BIBentryALTinterwordspacing
K.~Zhou, R.~Ding, X.~Ma, and Y.~Lin, ``Printable and flexible integrated sensing systems for wireless healthcare,'' \emph{Nanoscale}, vol.~16, pp. 7264--7286, 2024. [Online]. Available: \url{http://dx.doi.org/10.1039/D3NR06099C}
\BIBentrySTDinterwordspacing

\bibitem{cui2016printed}
Z.~Cui, \emph{Printed electronics: materials, technologies and applications}.\hskip 1em plus 0.5em minus 0.4em\relax John Wiley \& Sons, 2016.

\bibitem{Mubarik:MICRO:2020:printedml}
M.~H. Mubarik \emph{et~al.}, ``Printed machine learning classifiers,'' in \emph{Annu. Int. Symp. Microarchitecture (MICRO)}, 2020, pp. 73--87.

\bibitem{Henkel:ICCAD2022:expedition}
J.~Henkel \emph{et~al.}, ``Approximate computing and the efficient machine learning expedition,'' in \emph{International Conference On Computer Aided Design (ICCAD)}, 2022, pp. 1--9.

\bibitem{Armeniakos:DATE2022:axml}
G.~Armeniakos, G.~Zervakis, D.~Soudris, M.~B. Tahoori, and J.~Henkel, ``Cross-layer approximation for printed machine learning circuits,'' in \emph{Design Automation and Test in Europe (DATE)}, 2022, pp. 190--195.

\bibitem{Armeniakos:TCAD2023:cross}
G.~Armeniakos, G.~Zervakis, D.~Soudris, M.~B. Tahoori, and J.~Henkel, ``Model-to-circuit cross-approximation for printed machine learning classifiers,'' \emph{IEEE Transactions on Computer-Aided Design of Integrated Circuits and Systems}, pp. 1--1, 2023.

\bibitem{Armeniakos:TC2023:codesign}
G.~Armeniakos, G.~Zervakis, D.~Soudris, M.~B. Tahoori, and J.~Henkel, ``Co-design of approximate multilayer perceptron for ultra-resource constrained printed circuits,'' \emph{IEEE Transactions on Computers}, pp. 1--8, 2023.

\bibitem{Afentaki:ICCAD23:hollistic}
F.~Afentaki \emph{et~al.}, ``Bespoke approximation of multiplication-accumulation and activation targeting printed multilayer perceptrons,'' in \emph{2023 IEEE/ACM International Conference on Computer Aided Design (ICCAD)}, 2023, pp. 1--9.

\bibitem{Afentaki:DATE2024:embedding}
F.~Afentaki, M.~Hefenbrock, G.~Zervakis, and M.~B. Tahoori, ``Embedding hardware approximations in discrete genetic-based training for printed mlps,'' in \emph{2024 Design, Automation \& Test in Europe Conference \& Exhibition (DATE)}.\hskip 1em plus 0.5em minus 0.4em\relax IEEE, 2024, pp. 1--6.

\bibitem{Mrazek:ICCAD2024}
V.~Mrazek \emph{et~al.}, ``Evolutionary approximation of ternary neurons for on-sensor printed neural networks,'' in \emph{International Conference On Computer Aided Design (ICCAD)}, 2024.

\bibitem{ovo}
J.~Platt, N.~Cristianini, and J.~Shawe-Taylor, ``Large margin dags for multiclass classification,'' \emph{Advances in neural information processing systems}, vol.~12, 1999.

\bibitem{Marques:Materials:2019}
C.~Marques \emph{et~al.}, ``{Progress Report on “From Printed Electrolyte-Gated Metal-Oxide Devices to Circuits”},'' \emph{Advanced Materials}, vol.~31, 2019.

\bibitem{bishop2006pattern:ovo_ova}
C.~M. Bishop and N.~M. Nasrabadi, \emph{Pattern recognition and machine learning}.\hskip 1em plus 0.5em minus 0.4em\relax Springer, 2006, vol.~4.

\bibitem{Zhao:DATE2023}
H.~Zhao \emph{et~al.}, ``Highly-bespoke robust printed neuromorphic circuits,'' in \emph{Design, Automation \& Test in Europe Conference \& Exhibition (DATE)}, 2023.

\bibitem{Dua:2019:uci_datasets}
\BIBentryALTinterwordspacing
D.~Dua and C.~Graff, ``{UCI} machine learning repository,'' 2017. [Online]. Available: \url{http://archive.ics.uci.edu/ml}
\BIBentrySTDinterwordspacing

\bibitem{PrintedBatteries2018}
S.~Lanceros‐Méndez and C.~M. Costa, \emph{Printed Batteries: Materials, Technologies and Applications}.\hskip 1em plus 0.5em minus 0.4em\relax Wiley, 2018.

\end{thebibliography}



\end{document}
