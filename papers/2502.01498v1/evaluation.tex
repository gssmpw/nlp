
\section{Results and Analysis} \label{sec:eval}

In this section, we evaluate our printed sequential SVM classifiers in terms of accuracy and hardware overhead.
The evaluation is based on five datasets (see Table~\ref{tab:svmfp}) from the UCI ML repository~\cite{Dua:2019:uci_datasets}, selected because they involve sensor inputs relevant to printed applications~\cite{Mubarik:MICRO:2020:printedml} and are frequently used by the state of the art.
We use Synopsys Design Compiler and PrimeTime for hardware evaluation, targeting frequency in the Hz range, typical for printed applications~\cite{Bleier:ISCA:2020:printedml}.
The fully parallel SVM classifiers--designed as described in~\cite{Mubarik:MICRO:2020:printedml}--serve as our baselines.
Additionally, we compare against~\cite{Armeniakos:TCAD2023:cross,Armeniakos:TC2023:codesign}, which, among all existing approximate printed digital \gls{ml} classifiers, achieve relatively high accuracy across all datasets.
To train our SVMs we use randomized parameter optimization, inputs are normalized to $[0,1]$, and training/testing is done with a \blue{randomly distributed $80$\%/$20$\% split}.
All accuracies hereafter are reported on the test dataset.
\blue{Table~\ref{tab:svmfp} presents the details of our trained SVMs}, reporting also the FP32 software accuracy.

Table~\ref{tab:compsoa} presents the hardware evaluation of our sequential SVMs \blue{in comparison to state-of-the-art designs}.
As shown, \blue{our SVMs exhibit negligible to no accuracy loss compared to the FP32 model.}
The average area of our designs is $3.7$cm$^2$, with a maximum of $7.8$cm$^2$, making them realistic and suitable for most printed applications~\cite{Armeniakos:TCAD2023:cross}. 
Similarly, the average power consumption is $4.2$mW, with a maximum of $8.7$mW, enabling all our circuits to be powered by existing printed batteries, e.g., Zinergy $15$mW or BlueSpark $6$mW.
% Table~\ref{tab:compsoa} demonstrates the feasibility and practicality of our SVMs while maintaining high accuracy.
Table~\ref{tab:compsoa} demonstrates the feasibility and practicality of our highly accurate SVMs.
Even when considering the power required for ADCs to process sensor data (e.g., $0.33$mW for the required $4$-bit ADC~\cite{Bleier:ISCA:2020:printedml}), our SVMs can still operate within printed battery constraints.
Since our circuits process one input feature per cycle, a single ADC can be shared across many input sensors.
In contrast, ADC costs might become bottlenecks in parallel designs like~\cite{Mubarik:MICRO:2020:printedml, Armeniakos:TCAD2023:cross, Armeniakos:TC2023:codesign}, \blue{where inputs are processed simultaneously}.
% which require all inputs simultaneously, might face bottlenecks due to ADC costs in their parallel designs. 
We also evaluated our SVMs \blue{using a printed crossbar ROM for storage}, which would reduce the area for Cardio, Dermatology, Pendigits, RedWine, and WhiteWine by $10$\%, $23$\%, $28$\%, $8$\%, and $3$\%, respectively.
However, to fully exploit these (limited) gains, a variation-aware training approach, e.g., similar to~\cite{Zhao:DATE2023}, would be necessary, which does not consistently yield reliable results across all datasets.

\begin{table}[t!]
\setlength\tabcolsep{3pt} 
\caption{SVM model analysis for the examined datasets.}
\label{tab:svmfp}
\footnotesize
\centering
\renewcommand{\arraystretch}{1}
%\begin{adjustbox}{width=\textwidth} % Automatically resize table to fit page width

\begin{tabular}{l| S[table-format=2.1]S[table-format=2.0]S[table-format=2.0]S[table-format=2.0]}

\hline
\textbf{Dataset} & \textbf{FP32 Acc (\%)} & \textbf{\#Inputs} & \textbf{\#Classes} & \textbf{\#Support Vectors}\\ \hline
Cardio & 93.4 & 21 & 3  & 3 \\
Dermatology & 98.6 & 33 & 6  & 15 \\ 
Pendigits & 97.7 & 17 & 10 & 45 \\
RedWine   & 66.2 & 11 & 6  & 15 \\
WhiteWine & 56.6 & 11  & 7 & 21 \\ \hline
\end{tabular}\vspace{-5ex}

\end{table}


Compared to fully parallel exact SVMs~\cite{Mubarik:MICRO:2020:printedml}, as shown in Table~\ref{tab:compsoa}, our sequential SVMs achieve $10$x lower area and $30$x lower power, on average.
For some datasets, our SVMs achieve higher accuracy due to our use of the more flexible LinearSVC class (compared to the SVC used in~\cite{Mubarik:MICRO:2020:printedml}), which provides better scalability and a wider range of penalty and loss functions.
Against the approximate SVMs~\cite{Armeniakos:TCAD2023:cross}, our circuits exhibit, on average, $6$x and $12$x lower area and power, respectively, while attaining $4.8$\% higher accuracy.
Compared to approximate MLPs~\cite{Armeniakos:TC2023:codesign}, our SVMs yield $2$x less area and $6$x less power on average, with $6$\% higher accuracy.
For RedWine, \cite{Armeniakos:TC2023:codesign} demonstrates a lower area, but both circuits are small-enough, whereas our design achieves $10\%$ higher accuracy.
Compared to more approximate digital printed neural networks~\cite{Mrazek:ICCAD2024},~\cite{Afentaki:ICCAD23:hollistic} (not included in Table~\ref{tab:compsoa}), our SVMs feature higher area but achieve, on average, $7.4$\% and $7.9$\% higher accuracy than~\cite{Mrazek:ICCAD2024} and~\cite{Afentaki:ICCAD23:hollistic}, respectively.
Interestingly, for the Pendigits dataset (the most complex examined), our SVMs offer $6$x lower area, $5$x lower power, and $4$\% higher accuracy than~\cite{Mrazek:ICCAD2024}, and $3$x lower area and power, and $8$\% higher accuracy compared to~\cite{Afentaki:ICCAD23:hollistic}.
Overall, as dataset complexity increases, the advantages of our sequential approach become even more pronounced, with higher gains over the state-of-the-art.
% This highlights the critical impact of our approach, making it necessary for achieving high accuracy in more complex datasets.
Furthermore, it's important to note that, except for RedWine and WhiteWine~\cite{Armeniakos:TC2023:codesign}, all state-of-the-art classifiers in Table~\ref{tab:compsoa} consume over $30$mW, which does not align with printed batteries~\cite{Mubarik:MICRO:2020:printedml}, and would be impractical for printed applications.
Concluding, within the strict physical constraints of printed applications in area and power availability, our SVMs constitute the most accurate printed classifiers that successfully meet these requirements.

\begin{table}[t!]
\setlength\tabcolsep{4pt} 
\caption{Hardware evaluation and comparison with state of the art.}
\label{tab:compsoa}
\footnotesize
\centering
\renewcommand{\arraystretch}{1}
%\begin{adjustbox}{width=\textwidth} % Automatically resize table to fit page width

\begin{tabular}{ll|S[table-format=2.1]|S[table-format=3.1]S[table-format=3.1]S[table-format=2]}

\hline
\textbf{Dataset} & \textbf{Technique} & {\thead{\textbf{Accuracy}\\ (\%)}} & {\thead{\textbf{Area} \\ (cm$^{2}$)}} & {\thead{\textbf{Power}\\ (mW)}} & {\thead{\textbf{Freq.}\\ (Hz)}}\\ \hline
Cardio & Parl SVM~\cite{Mubarik:MICRO:2020:printedml} & 90.0 & 15.1 & 57.4 & 13 \\
Cardio & Ax Parl SVM~\cite{Armeniakos:TCAD2023:cross} & 89.0 & 17.0 & 48.9 & 13 \\
Cardio & Ax Parl MLP~\cite{Armeniakos:TC2023:codesign} & 87.0 & 6.1 & 20.8 & 5 \\
Cardio & \oursTable{Ours Seq. SVM} & 93.1 & 3.1 & 3.6 & 24 \\
\hline
Dermatology & Parl SVM~\cite{Mubarik:MICRO:2020:printedml} & 97.2 & 60.4 & 182.9 & 8 \\
Dermatology & \oursTable{Ours Seq. SVM} & 98.6 & 4.9 & 5.3 & 18 \\
\hline
Pendigits & Parl SVM~\cite{Mubarik:MICRO:2020:printedml} & 97.8 & 123.8 & 364.4 & 4 \\
Pendigits & Ax Parl SVM~\cite{Armeniakos:TCAD2023:cross} & 97.0 & 97.0 & 183.7 & 4 \\
Pendigits & Ax Parl MLP~\cite{Armeniakos:TC2023:codesign} & 93.0 & 32.7 & 99.2 & 4 \\
Pendigits & \oursTable{Ours Seq. SVM} & 97.6 & 7.8 & 8.7 & 15 \\
\hline
RedWine & Parl SVM~\cite{Mubarik:MICRO:2020:printedml} & 57.0 & 23.5 & 92.8 & 15 \\
RedWine & Ax Parl SVM~\cite{Armeniakos:TCAD2023:cross} & 56.0 & 11.7 & 21.3 & 15 \\
RedWine & Ax Parl MLP~\cite{Armeniakos:TC2023:codesign} & 56.0 & 1.1 & 3.9 & 5 \\
RedWine & \oursTable{Ours Seq. SVM} & 66.2 & 3.1 & 3.5 & 19 \\
\hline
WhiteWine & Parl SVM~\cite{Mubarik:MICRO:2020:printedml} & 53.0 & 28.3 & 112.4 & 17 \\
WhiteWine & Ax Parl SVM~\cite{Armeniakos:TCAD2023:cross} & 52.0 & 11.0 & 34.7 & 17 \\
WhiteWine & Ax Parl MLP~\cite{Armeniakos:TC2023:codesign} & 53.0 & 6.5 & 21.3 & 5 \\
WhiteWine & \oursTable{Ours Seq. SVM} & 56.2 & 3.4 & 3.9 & 20 \\
\hline
\end{tabular}\vspace{-4ex}

\end{table}


As shown in Table~\ref{tab:compsoa}, folding the SVM execution allows for higher clock frequencies.
However, this frequency gain does not fully offset the increased latency and energy demands due to the additional cycles required.
Nevertheless, this is unlikely to be an issue for most applications, as performance is not the primary focus in low-throughput printed applications~\cite{Henkel:ICCAD2022:expedition}.
Similarly, printed batteries are customizable in capacity, shape, and voltage~\cite{PrintedBatteries2018}, so energy concerns might be secondary compared to power availability.