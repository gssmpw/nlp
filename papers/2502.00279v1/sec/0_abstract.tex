
\begin{abstract}
% One of the major challenges in semi-supervised learning is label shift, where the unlabeled class distribution is different from the labeled class distribution and is also unknown. This creates difficulty for the standard pseudo-labeling approach because the classifiers do not transfer well from labeled to unlabeled data. As a result it is typically assumed that the unlabeled data class distribution is either known apriori, or estimated on-the-fly using the pseudo-labels themselves. Based on the intuition that learning the average of class conditionals is easier than learning them point-wise, we show that by using better methods, the unlabeled class distribution can be estimated accurately, which then results in better pseudo-labels and eventually improves point-wise classification. We show this improvement in the recently proposed Simpro.

A major challenge in Semi-Supervised Learning (SSL) is the limited information available about the class distribution in the unlabeled data.
In many real-world applications this arises from the prevalence of long-tailed distributions, where the standard pseudo-label approach to SSL is biased towards the labeled class distribution and thus performs poorly on unlabeled data.
Existing methods typically assume that the unlabeled class distribution is either known a priori, which is unrealistic in most situations, or estimate it on-the-fly using the pseudo-labels themselves. 
We propose to explicitly estimate the unlabeled class distribution, which is a finite-dimensional parameter, \emph{as an initial step}, using a doubly robust estimator with a strong theoretical guarantee; this estimate can then be integrated into existing methods to pseudo-label the unlabeled data during training more accurately.
Experimental results demonstrate that incorporating our techniques into common pseudo-labeling approaches improves their performance. 

\end{abstract}