\section{Related Work}
AI and more specifically machine learning (ML) has been used to solve certain tasks in WDS. These include demand prediction \cite{WU2023104545}, leakage detection \cite{Fan2021, valerie_icpram24} and sensor fault detection \cite{valerie_sensor}. Moreover, there exist some approaches for the tasks like optimal sensor placement \cite{Candelieri_2022, su15042981}. In the real world, pressure readings from a few sensors are available. Hence, the task of estimating the state of the WDS at every node using these sparse readings is important for leakage and sensor fault detection. This task has attracted considerable attention recently as demonstrated by \cite{su15042981, ashraf2023spatial, hajgato2021pressure, Truong_2024, xing2022stateestimation}.

Graph neural networks (GNNs) are a promising DL method to deal with network structures like WDS. There exist a lot of different GNN architectures such as spectral graph convolutional neural networks (GCNs) \cite{Bruna2014SpectralNA,kipf2017semi,defferrard2016convolutional,henaff2015deep,levie2018cayleynets,li2018adaptive}, spatial GCNs \cite{hamilton2017inductive,monti2017geometric,gao2018large,niepert2016learning,xu2018powerful,velickovic2018graph}, and recursive graph and tree models  \cite{scarselli2009, diss}. Similar to \cite{Ashraf_Strotherm_Hermes_Hammer_2024}, our proposed architecture employs custom message-passing GNNs that fall under the umbrella of spatial GCNs. 

The task of hydraulic state estimation in WDS involves estimating the pressure head at every node and water flow through every link given the demands at every consumer node and the pressure heads at the reservoirs. EPANET is the prevalent hydraulic simulator that solves a system of equations to solve this task \cite{rossman2020epanet}. A faster alternative method using edge diffusion was recently proposed by \cite{KERIMOV2025122980} with promising results. However, it is unclear how well the method generalizes to changes in input features. A few DL surrogate models for hydraulic state estimation already exist. Reference \cite{xing2022stateestimation} combines the inputs required by EPANET with sparse sensor measurements to solve this task. However, the efficacy of their method is indeterminate given their limited empirical evaluation. Using the same input data as \cite{xing2022stateestimation}, \cite{KERIMOV2024121933} proposed surrogate metamodels using edge-based GNNs. Although they demonstrate good results, their approach is limited to only single-reservoir WDSs. Reference \cite{Ashraf_Strotherm_Hermes_Hammer_2024} are the first to solve the task of state estimation in WDS using a DL model showing promising results. However, their approach suffers from limited generalizability to unseen input features (demands, pipe diameters, etc.). Moreover, since they only experimented with rather small WDSs, it is unclear how well their model scales to larger and more realistic WDSs. 

WDSs are characterized by different features (pressures, demands, flows, pipe attributes, etc.) having values of different magnitudes that are tied together through physical laws. Other domains in critical infrastructure like electricity grids also exhibit similar characteristics \cite{ghamizi2024}. Hence, traditional data normalization methods used in DL cannot be directly applied without violating the physical relationships. This constitutes an open problem that is addressed in this paper.