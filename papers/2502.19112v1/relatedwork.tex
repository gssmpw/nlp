\section{Literature Review}
\subsection{Emerging Roles in CSCL}
The Zone of Proximal Development (ZPD) theory supports the idea that individuals will not be able to achieve certain goals without support from others (Vygotsky, 1978) and is considered one of the theoretical foundations for understanding the benefits of collaborative learning. Dillenbourg (1999) argued that collaborative learning stimulates learning mechanisms through certain forms of interactions among group members, instead of being a learning mechanism or method by itself. There is no guarantee that learning at the individual level will occur in CSCL, regardless of the completion of the group tasks (Dillenbourg, 1999). Successful collaborative learning not only requires intersubjective meaning-making and co-construction of knowledge at the group level but also requires active contribution from each member at the individual level (Järvelä \& Järvenoja, 2011; Peterson \& Roseth, 2016). Based on the constructivist theory (Piaget, 1976), the occurrence of learning at the individual level requires individuals to actively engage in the process and contribute to the group collaboration. Understanding students’ emerging roles based on their individual contributions is thus considered critical in CSCL. This further raises a key question of how to analyze students’ emerging roles based on their individual contributions using trace data in collaborative learning processes.\\

Compared to scripted roles with assigned tasks and duties known prior to beginning a CSCL project, emerging roles are developed spontaneously during the self-organizing collaborative learning processes (Strijbos \& Weinberger, 2010). Based on the conceptual framework proposed by Strijbos and De Laat (2010), students’ emerging roles can be seen at three levels, including the micro-level roles that focus on process- or product-oriented tasks, the meso-level roles that reflect the patterns of participatory behaviour, and the macro-level roles that relate to the stances of individuals towards collaborative learning. Within the three levels, product-oriented roles are related to the accomplishment of specific tasks, while the process-oriented roles are related to the task management activities for facilitating the collaboration processes (Strijbos \& De Laat, 2010).\\

Previous studies have mostly focused on analyzing students’ process-oriented emerging roles by analyzing their social interaction patterns in large-group collaborative learning (e.g., online discussions) using network analysis, constructing learner profiles based on structural properties and classifying them into different categories of emerging roles using predefined thresholds or data-driven clustering methods (Aviv et al., 2003; Saqr \& López-Pernas, 2022; Kim \& Ketenci, 2019; Saqr et al., 2018; Marcos-García et al., 2015; Turkkila \& Lommi, 2020; Medina et al., 2016). Several studies have analyzed students’ emerging roles in small group CSCL based on the communication content generated by students in the learning processes using various methods, including qualitative coding, computational linguistic techniques, text mining methods, and epistemic network analysis (Dowell et al., 2019, 2020; Ferreira et al., 2022; Xie et al., 2018; Swiecki, 2021; Swiecki \& Shaffer, 2020). Strijbos and De Laat (2010) proposed a conceptual framework describing eight types of participative stances along the dimensions of group size, orientation, and effort. Within small group collaborations, they proposed four types of roles — over-rider, ghost, free rider, and captain, along the dimension of orientation and efforts — with a qualitative coding approach to analyze student transcriptions of their group communications (Strijbos \& De Laat, 2010). However, the qualitative coding involved in analyzing student communication content is costly and challenging for large-scale datasets. Additionally, the heuristic text mining approaches used in previous studies (e.g., LDA topic modelling) require subjective decisions, which can lead to difficulty gauging and evaluating the accuracy and interpretability of the analytical results (Talib et al., 2016).

\subsection{Bipartite Network Analysis}
Networks are the graphical representations of real-world systems through a collection of nodes joined together in pairs by edges representing a wide variety of relationships. Network analysis offers a powerful analytical tool to study relational data as well as to analyze structural characteristics at individual- and system-levels. Social network analysis (SNA), one of the commonly used methods in social science studies, focuses on analyzing social systems through modelling the social interactions among social actors, such as in friendship networks. SNA enables us to quantify individual and group status based on the structural characteristics of social connections (Wasserman \& Faust, 1994; Newman, 2018). It has been considered an effective approach to analyze social dynamics and interactions among group members in computer-supported collaborative learning (CSCL; Chen \& Poquet, 2022; Saqr et al., 2022; Dado \& Bodemer, 2017; Gašević et al., 2019; Oshima et al., 2012; Reffay \& Chanier, 2003). The applications of SNA in CSCL are centred around analyzing communication-based ties for understanding the information flow among participants as well as identifying active or peripheral actors in online discussion forums and wiki-based collaborative writing (Dado \& Bodemer, 2017; de Laat et al., 2007; Rabbany et al., 2014; Saqr \& López-Pernas, 2022). Kaliisa et al. (2022) highlighted the need for innovative network approaches for social learning analytics.\\

Bipartite network analysis (BNA) models the connections between two types of node sets (i.e., two-mode network), in contrast to traditional social network analysis, which focuses on the relationships among nodes of a single type (Newman, 2018; Wasserman \& Faust, 1994). Bipartite networks model the heterogenous relationships between different types of entities. The heterogenous nature of bipartite networks makes them a powerful and flexible tool for modelling the complex relationships within various social and physical systems (Valejo et al., 2021). In CSCL, a one-mode network can be used to capture the social interactions among group members, with nodes in the network representing students. A bipartite network in CSCL can be used to measure the associations between different types of nodes, such as student–subtask relationships.\\

Currently, there is little usage of BNA in learning analytics and CSCL. Matsuzaw et al. (2011) developed an analytical platform for analyzing word-discourse relationships using bipartite network analysis in order to understand the knowledge-building process among group members in CSCL. Feng et al. (2024) constructed bipartite projections from a heterogenous tripartite network consisting of multimodal process data in small-group collaborative learning to identify the significant behavioural engagement strategies of individual students based on the associations between student communication and spatial movements. This previous research shows that BNA offers an effective tool to model and analyze the structural regularities within heterogenous relationships (e.g., students, artefacts, theory-informed constructs) involved in collaborative learning.\\

Networks are the graphical representations of real-world systems through a collection of nodes joined together in pairs by edges representing a wide variety of relationships. Network analysis offers a powerful analytical tool to study relational data as well as to analyze structural characteristics at individual- and system-levels. Social network analysis (SNA), one of the commonly used methods in social science studies, focuses on analyzing social systems through modelling the social interactions among social actors, such as in friendship networks. SNA enables us to quantify individual and group status based on the structural characteristics of social connections (Wasserman \& Faust, 1994; Newman, 2018). It has been considered an effective approach to analyze social dynamics and interactions among group members in computer-supported collaborative learning (CSCL; Chen \& Poquet, 2022; Saqr et al., 2022; Dado \& Bodemer, 2017; Gašević et al., 2019; Oshima et al., 2012; Reffay \& Chanier, 2003). The applications of SNA in CSCL are centred around analyzing communication-based ties for understanding the information flow among participants as well as identifying active or peripheral actors in online discussion forums and wiki-based collaborative writing (Dado \& Bodemer, 2017; de Laat et al., 2007; Rabbany et al., 2014; Saqr \& López-Pernas, 2022; Chen \& Huang, 2019). Kaliisa et al. (2022) highlighted the need for innovative network approaches for social learning analytics.\\

Bipartite network analysis (BNA) models the connections between two types of node sets (i.e., two-mode network), in contrast to traditional social network analysis, which focuses on the relationships among nodes of a single type (Newman, 2018; Wasserman \& Faust, 1994). Bipartite networks model the heterogenous relationships between different types of entities. The heterogenous nature of bipartite networks makes them a powerful and flexible tool for modelling the complex relationships within various social and physical systems (Valejo et al., 2021). In CSCL, a one-mode network can be used to capture the social interactions among group members, with nodes in the network representing students. A bipartite network in CSCL can be used to measure the associations between different types of nodes, such as student–subtask relationships.\\

Currently, there is little usage of BNA in learning analytics and CSCL. Matsuzaw et al. (2011) developed an analytical platform for analyzing word-discourse relationships using bipartite network analysis in order to understand the knowledge-building process among group members in CSCL. Feng et al. (2024) constructed bipartite projections from a heterogenous tripartite network consisting of multimodal process data in small-group collaborative learning to identify the significant behavioural engagement strategies of individual students based on the associations between student communication and spatial movements. This previous research shows that BNA offers an effective tool to model and analyze the structural regularities within heterogenous relationships (e.g., students, artefacts, theory-informed constructs) involved in collaborative learning.


\subsection{Research Gaps}

\noindent In general, three aspects are not fully explored in the previous studies analyzing students’ emerging roles in CSCL:
\begin{enumerate}
    \item Novel representations for analyzing students’ emerging roles in small group CSCL
    \item Quantitative analysis of the quantity and heterogeneity of individual contributions
    \item Effects of scripted roles on emerging roles
\end{enumerate}

This study introduces a bipartite network approach to gauge students’ emerging roles based on the two dimensions of individual contributions — quantity and heterogeneity — using the interactions between students and the subtasks involved in their small group CSCL processes. Student–subtask engagement data offers a valuable opportunity to acquire an effective understanding of the emerging product-oriented roles in a learning process. Compared to social interaction data, student–subtask engagement data provides direct evidence of cognitive engagement in the process of production. This study utilized a bipartite network approach to capture the student–subtask interactions and further analyze the emerging roles of students based on the quantity and heterogeneity of individual contributions. The importance of the variety or breadth of subtasks engaged in by a student in the learning process is underexamined. Weinberger and Fischer (2006) introduced a multidimensional conceptual framework for understanding argumentative knowledge construction in CSCL, which considers participation from the perspectives of individual participation quantity and group-level heterogeneity. The analysis of individual participation was based on the word count within discourses; the group-level heterogeneity was determined by the standard deviation of individuals’ contributed words within a group. In the current study, we examine both the quantity and heterogeneity at the individual level. In this context, the heterogeneity of individual contribution reflects to what extent a student engages in a variety of subtasks in collaborative learning, rather than the equality of participation among individuals within a group. The heterogeneity of individual contribution examined in this study therefore provides an important lens to understand the breadth of possible cognitive learning opportunities a student encounters during a collaboration (Kawakubo et al., 2022). Lastly, previous studies have mainly focused on investigating the value of scripted roles on students’ cognitive presence and knowledge construction (Gašević et al., 2015; Wise \& Chiu, 2011; De Wever et al., 2010; Rolim et al., 2019), which we have expanded to focus on the impact of scripted roles on emerging roles. This study contributes towards bridging these three research gaps by introducing a bipartite network method for uncovering students’ emerging roles based on an analysis of individual contributions, as well as exploring the effect of assigned leadership roles on their emergent individual contributions.