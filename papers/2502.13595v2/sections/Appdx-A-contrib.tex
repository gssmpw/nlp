\section{Contributions}
\label{sec:contributions}

We list the contributions of every author in \autoref{tab:contributions}. The possible types of contributions and their associated points are:
\begin{itemize}
\item \textbf{New dataset:} A new dataset includes creating a new implementation (subclass) of a task using a new dataset. 2 points were awarded for implementing the task and 4 points for each new language introduced by the task. 
\item \textbf{New task:} An implementation of a new task category such as multi-label classification or instruction retrieval. 2 points were given for a new task, as well as points following adding a new dataset.
\item \textbf{Annotations:} Many existing datasets were not yet annotated with proper metadata. To encourage high-quality annotations we awarded 1 point for each full dataset annotation.
\item \textbf{Fixes:} These included bug fixes, usability fixes, speed improvements and more. For these, we typically awarded 2-10 points depending on the size of the contribution.
\item \textbf{Running Models:} This includes both running and implementing models for MMTEB. We typically awarded 1 point per model run on a full set of relevant tasks. Relevant tasks for a specific model are limited to those pertinent to its language. For instance, a Russian model does not need to be run on French tasks.
\item \textbf{Review PR:} A large part of ensuring good dataset quality comes from the dataset review. We award 2 points for a review. If a PR had multiple reviewers, 2 points were awarded to each. Often reviewers finalized dataset additions, helped with data formatting, and resolving bugs. In many cases, adding 2 points for review was considered either too low (a perfect PR with little to no corrections) or too high (lengthy discussion examining dataset quality, debugging implementations and more), however on average we believe it was appropriate.
\item \textbf{Writing:} At this point many of the authors writing the paper already qualified for co-authorship and thus had reasonable experience with the MMTEB point system. Thus, it was generally possible to discuss a reasonable amount of points based on the efforts made in earlier stages.
\item \textbf{Coordination:} Included Coordination of contributors and initial ideation were given points at the end of the project based on relative effort. These points were given, similar to paper writing, based on relative effort.
\end{itemize}

A total of 10 points had to be obtained to be invited as a co-author. To see each contribution mapped to specific PRs, see \url{https://github.com/embeddings-benchmark/mteb/tree/main/docs/mmteb/points}, where the name of JSON files corresponds to the PR id.


{
\tiny
\begin{longtable}{lccccccccc}
\caption[]{Contributions by GitHub users. See \autoref{tab:authors} for the mapping between authors and GitHub handles} \\
\label{tab:contributions} \\
\toprule
 & Total & Bug fixes & Review PR & New dataset & Dataset annotations & Paper writing & Coordination & New task & Running Models \\
GitHub &  &  &  &  &  &  &  &  &  \\
\midrule
\endfirsthead
\caption[]{(Continued) Contributions by GitHub users. See \autoref{tab:authors} for the mapping between authors and GitHub handles} \\
\toprule
Github & Total & Bug  & Review & New & Dataset & Paper & Coordination & New & Running \\
Handle &  & fixes & PR & dataset & annotations & writing &  & task & Models \\
\midrule
\endhead
\midrule
\multicolumn{10}{r}{Continued on next page} \\
\midrule
\endfoot
\bottomrule
\endlastfoot
KennethEnevoldsen & 597 & 87 & 326 & 68 & 35 & 0 & 81 & 0 & 0 \\
isaac-chung & 433 & 50 & 194 & 120 & 1 & 12 & 54 & 2 & 0 \\
imenelydiaker & 358 & 24 & 144 & 120 & 0 & 0 & 70 & 0 & 0 \\
awinml & 302 & 0 & 2 & 300 & 0 & 0 & 0 & 0 & 0 \\
x-tabdeveloping & 239 & 10 & 32 & 144 & 0 & 0 & 41 & 12 & 0 \\
davidstap & 176 & 0 & 0 & 176 & 0 & 0 & 0 & 0 & 0 \\
jaygala24 & 149 & 0 & 0 & 149 & 0 & 0 & 0 & 0 & 0 \\
wissam-sib & 144 & 4 & 6 & 134 & 0 & 0 & 0 & 0 & 0 \\
Muennighoff & 142 & 0 & 48 & 0 & 0 & 0 & 70 & 0 & 24 \\
orionw & 125 & 20 & 20 & 0 & 0 & 0 & 75 & 10 & 0 \\
dokato & 112 & 12 & 6 & 94 & 0 & 0 & 0 & 0 & 0 \\
gentaiscool & 110 & 0 & 0 & 110 & 0 & 0 & 0 & 0 & 0 \\
jupyterjazz & 108 & 0 & 0 & 108 & 0 & 0 & 0 & 0 & 0 \\
SaitejaUtpala & 102 & 0 & 0 & 102 & 0 & 0 & 0 & 0 & 0 \\
vaibhavad & 93 & 8 & 4 & 6 & 0 & 0 & 75 & 0 & 0 \\
MathieuCiancone & 88 & 0 & 0 & 88 & 0 & 0 & 0 & 0 & 0 \\
schmarion & 88 & 0 & 0 & 88 & 0 & 0 & 0 & 0 & 0 \\
GabrielSequeira & 88 & 0 & 0 & 88 & 0 & 0 & 0 & 0 & 0 \\
digantamisra98 & 71 & 0 & 0 & 71 & 0 & 0 & 0 & 0 & 0 \\
shreeya-dhakal & 62 & 0 & 8 & 54 & 0 & 0 & 0 & 0 & 0 \\
Rysias & 58 & 0 & 0 & 58 & 0 & 0 & 0 & 0 & 0 \\
Samoed & 51 & 22 & 2 & 18 & 0 & 0 & 0 & 0 & 9 \\
gowitheflow-1998 & 50 & 0 & 0 & 50 & 0 & 0 & 0 & 0 & 0 \\
sivareddyg & 50 & 0 & 0 & 0 & 0 & 0 & 50 & 0 & 0 \\
asparius & 48 & 0 & 14 & 34 & 0 & 0 & 0 & 0 & 0 \\
Akash190104 & 46 & 0 & 0 & 46 & 0 & 0 & 0 & 0 & 0 \\
MartinBernstorff & 43 & 13 & 8 & 2 & 0 & 0 & 20 & 0 & 0 \\
staoxiao & 40 & 0 & 0 & 40 & 0 & 0 & 0 & 0 & 0 \\
akshita-sukhlecha & 40 & 4 & 0 & 36 & 0 & 0 & 0 & 0 & 0 \\
rafalposwiata & 36 & 0 & 0 & 36 & 0 & 0 & 0 & 0 & 0 \\
bp-high & 36 & 0 & 0 & 36 & 0 & 0 & 0 & 0 & 0 \\
KranthiGV & 34 & 0 & 14 & 20 & 0 & 0 & 0 & 0 & 0 \\
bjoernpl & 28 & 0 & 0 & 28 & 0 & 0 & 0 & 0 & 0 \\
rasdani & 28 & 0 & 0 & 28 & 0 & 0 & 0 & 0 & 0 \\
loicmagne & 28 & 28 & 0 & 0 & 0 & 0 & 0 & 0 & 0 \\
jphme & 28 & 0 & 0 & 28 & 0 & 0 & 0 & 0 & 0 \\
ShawonAshraf & 28 & 0 & 0 & 28 & 0 & 0 & 0 & 0 & 0 \\
violenil & 26 & 0 & 0 & 26 & 0 & 0 & 0 & 0 & 0 \\
mariyahendriksen & 24 & 0 & 0 & 0 & 0 & 24 & 0 & 0 & 0 \\
dwzhu-pku & 24 & 0 & 0 & 24 & 0 & 0 & 0 & 0 & 0 \\
hgissbkh & 23 & 13 & 2 & 0 & 0 & 3 & 0 & 5 & 0 \\
jankounchained & 22 & 8 & 0 & 14 & 0 & 0 & 0 & 0 & 0 \\
taeminlee & 22 & 0 & 0 & 22 & 0 & 0 & 0 & 0 & 0 \\
tomaarsen & 22 & 0 & 2 & 0 & 0 & 0 & 20 & 0 & 0 \\
kwojtasi & 22 & 0 & 0 & 22 & 0 & 0 & 0 & 0 & 0 \\
mrshu & 21 & 0 & 4 & 16 & 1 & 0 & 0 & 0 & 0 \\
crystina-z & 21 & 0 & 0 & 21 & 0 & 0 & 0 & 0 & 0 \\
ManuelFay & 20 & 13 & 0 & 2 & 0 & 0 & 0 & 5 & 0 \\
AlexeyVatolin & 20 & 20 & 0 & 0 & 0 & 0 & 0 & 0 & 0 \\
Andrian0s & 20 & 2 & 4 & 14 & 0 & 0 & 0 & 0 & 0 \\
rbroc & 20 & 0 & 0 & 20 & 0 & 0 & 0 & 0 & 0 \\
john-b-yang & 20 & 0 & 0 & 0 & 0 & 20 & 0 & 0 & 0 \\
mmhamdy & 20 & 0 & 0 & 20 & 0 & 0 & 0 & 0 & 0 \\
manandey & 18 & 0 & 0 & 18 & 0 & 0 & 0 & 0 & 0 \\
thakur-nandan & 18 & 0 & 0 & 18 & 0 & 0 & 0 & 0 & 0 \\
PranjalChitale & 16 & 0 & 0 & 16 & 0 & 0 & 0 & 0 & 0 \\
Sakshamrzt & 16 & 0 & 4 & 12 & 0 & 0 & 0 & 0 & 0 \\
sted97 & 16 & 0 & 0 & 16 & 0 & 0 & 0 & 0 & 0 \\
dipam7 & 16 & 0 & 2 & 14 & 0 & 0 & 0 & 0 & 0 \\
artemsnegirev & 14 & 0 & 0 & 12 & 2 & 0 & 0 & 0 & 0 \\
taidnguyen & 14 & 0 & 0 & 14 & 0 & 0 & 0 & 0 & 0 \\
jordiclive & 12 & 10 & 0 & 2 & 0 & 0 & 0 & 0 & 0 \\
guenthermi & 12 & 0 & 0 & 12 & 0 & 0 & 0 & 0 & 0 \\
slvnwhrl & 12 & 0 & 0 & 12 & 0 & 0 & 0 & 0 & 0 \\
Art3mis07 & 12 & 0 & 0 & 12 & 0 & 0 & 0 & 0 & 0 \\
xhluca & 12 & 4 & 2 & 6 & 0 & 0 & 0 & 0 & 0 \\
anpalmak2003 & 12 & 0 & 0 & 9 & 3 & 0 & 0 & 0 & 0 \\
ab1992ao & 11 & 0 & 0 & 8 & 3 & 0 & 0 & 0 & 0 \\
MariyaTikhonova & 11 & 0 & 0 & 7 & 4 & 0 & 0 & 0 & 0 \\
henilp105 & 11 & 2 & 0 & 0 & 9 & 0 & 0 & 0 & 0 \\
simon-clematide & 10 & 0 & 0 & 10 & 0 & 0 & 0 & 0 & 0 \\
jimmy-lin & 10 & 0 & 0 & 0 & 0 & 0 & 10 & 0 & 0 \\
sarahooker & 10 & 0 & 0 & 0 & 0 & 10 & 0 & 0 & 0 \\
swj0419 & 10 & 0 & 0 & 10 & 0 & 0 & 0 & 0 & 0 \\
xiamengzhou & 10 & 0 & 0 & 10 & 0 & 0 & 0 & 0 & 0 \\
ABorghini & 10 & 0 & 0 & 10 & 0 & 0 & 0 & 0 & 0 \\
xu3kev & 10 & 0 & 0 & 10 & 0 & 0 & 0 & 0 & 0 \\
malteos & 10 & 0 & 0 & 10 & 0 & 0 & 0 & 0 & 0 \\
ljvmiranda921 & 10 & 0 & 0 & 10 & 0 & 0 & 0 & 0 & 0 \\
howard-yen & 10 & 0 & 0 & 10 & 0 & 0 & 0 & 0 & 0 \\
hongjin-su & 10 & 0 & 0 & 10 & 0 & 0 & 0 & 0 & 0 \\
guangyusong & 10 & 0 & 0 & 10 & 0 & 0 & 0 & 0 & 0 \\
Alenush & 10 & 0 & 0 & 6 & 4 & 0 & 0 & 0 & 0 \\
cassanof & 10 & 1 & 0 & 8 & 0 & 0 & 0 & 0 & 1 \\
HLasse & 10 & 5 & 0 & 0 & 5 & 0 & 0 & 0 & 0 \\
ZhengLiu101 & 10 & 0 & 0 & 10 & 0 & 0 & 0 & 0 & 0 \\
Ruqyai & 10 & 0 & 8 & 2 & 0 & 0 & 0 & 0 & 0 \\
izhx & 6 & 0 & 0 & 6 & 0 & 0 & 0 & 0 & 0 \\
marcobellagente93 & 6 & 0 & 0 & 6 & 0 & 0 & 0 & 0 & 0 \\
monikernemo & 2 & 0 & 0 & 2 & 0 & 0 & 0 & 0 & 0 \\
NouamaneTazi & 2 & 0 & 2 & 0 & 0 & 0 & 0 & 0 & 0 \\
MexicanLemonade & 2 & 0 & 0 & 2 & 0 & 0 & 0 & 0 & 0 \\
bakrianoo & 2 & 0 & 0 & 2 & 0 & 0 & 0 & 0 & 0 \\
PhilipMay & 2 & 0 & 2 & 0 & 0 & 0 & 0 & 0 & 0 \\
achibb & 2 & 0 & 0 & 2 & 0 & 0 & 0 & 0 & 0 \\
antoniolanza1996 & 2 & 2 & 0 & 0 & 0 & 0 & 0 & 0 & 0 \\
cslizc & 2 & 0 & 0 & 2 & 0 & 0 & 0 & 0 & 0 \\
hanhainebula & 2 & 0 & 0 & 2 & 0 & 0 & 0 & 0 & 0 \\
\end{longtable}

}



% \begin{table*}
% \centering
% {\scriptsize
% \begin{tabular}{llll}
% \toprule
% GitHub & First name & Last name & Affiliations \\
% \midrule
% KennethEnevoldsen & Kenneth & Enevoldsen & Aarhus University, Denmark \\
% x-tabdeveloping & Márton & Kardos & Aarhus University, Denmark \\
% imenelydiaker & Imene & Kerboua & Esker, Lyon, France, INSA Lyon, LIRIS, Lyon, France \\
% wissam-sib & Wissam & Siblini & N/A \\
% GabrielSequeira & Gabriel & Sequeira & N/A \\
% schmarion & Marion & Schaeffer & Wikit, Lyon, France \\
% MathieuCiancone & Mathieu & Ciancone & Wikit, Lyon, France \\
% MartinBernstorff & Martin & Bernstorff & Aarhus University, Denmark \\
% staoxiao & Shitao & Xiao & Beijing Academy of Artificial Intelligence \\
% ZhengLiu101 & Zheng & Liu & Beijing Academy of Artificial Intelligence \\
% achibb & Aaron & Chibb & N/A \\
% cassanof & Federico & Cassano & Northeastern University, Boston, USA \\
% taidnguyen & Nguyen & Tai & University of Pennsylvania \\
% xu3kev & Wen-Ding & Li & Cornell University \\
% Rysias & Jonathan & Rystrøm & University of Oxford, UK \\
% taeminlee & Taemin & Lee & Korea University Human-Inspired AI Research \\
% izhx & Xin & Zhang & Harbin Institute of Technology, Shenzhen \\
% orionw & Orion & Weller & Johns Hopkins University \\
% slvnwhrl & Silvan & Wehrli & Robert Koch Institute, Berlin, Germany \\
% manandey & Manan & Dey & Salesforce, India \\
% isaac-chung & Isaac & Chung & N/A \\
% asparius & Ömer & Çağatan & Koç University,Turkey \\
% rafalposwiata & Rafał & Poświata & National Information Processing Institute, Warsaw, Poland \\
% rbroc & Roberta & Rocca & Aarhus University, Denmark \\
% awinml & Ashwin & Mathur & N/A \\
% guangyusong & Guangyu & Song & Tano Labs \\
% davidstap & David & Stap & University of Amsterdam. \\
% HLasse & Lasse & Hansen & Aarhus University, Denmark \\
% jaygala24 & Jay & Gala & MBZUAI \\
% digantamisra98 & Diganta & Misra & Mila - Quebec AI Institute \\
% PranjalChitale & Pranjal & Chitale & Indian Institute of Technology Madras \\
% Akash190104 & Akash & Kundu & Heritage Institute of Technology, Kolkata, Apart Research \\
% dwzhu-pku & Dawei & Zhu & Peking University \\
% ljvmiranda921 & Lester James & Miranda & Allen Institute for AI \\
% Sakshamrzt & Saksham & Thakur & N/A \\
% Andrian0s & Andrianos & Michail & University of Zurich \\
% simon-clematide & Simon & Clematide & University of Zurich \\
% SaitejaUtpala & Saiteja & Utpala & Microsoft Research \\
% mmhamdy & Mohammed & Hamdy & Cohere For AI Community \\
% jupyterjazz & Saba & Sturua & Jina AI \\
% Ruqyai & Ruqiya & Bin Safi & NaN \\
% KranthiGV & Kranthi Kiran & GV & New York University \\
% shreeya-dhakal & Shreeya & Dhakal & Individual Contributor \\
% dipam7 & Dipam & Vasani & Individual Contributor \\
% Art3mis07 & Gayatri & K & R. V. College of Engineering, Bengaluru \\
% jankounchained & Jan & Kostkan & Aarhus University, Denmark \\
% bp-high & Bhavish & Pahwa & Microsoft Research \\
% rasdani & Daniel & Auras & ellamind, Germany \\
% ShawonAshraf & Shawon & Ashraf & ellamind, Germany \\
% bjoernpl & Björn & Plüster & ellamind, Germany \\
% jphme & Jan Philipp & Harries & ellamind, Germany \\
% malteos & Malte & Ostendorff & Occiglot \\
% ManuelFay & Manuel & Faysse & CentraleSupélec, Illuin Technology \\
% hgissbkh & Hippolyte & Gisserot-Boukhlef & CentraleSupélec, Artefact Research Center \\
% sted97 & Simone & Tedeschi & Sapienza University of Rome \\
% gentaiscool & Genta Indra & Winata & N/A \\
% henilp105 & Henil & Panchal & Nirma University \\
% ABorghini & Alessia & Borghini & Sapienza University of Rome \\
% jordiclive & Jordan & Clive & Imperial College London \\
% gowitheflow-1998 & Chenghao & Xiao & Durham University \\
% mariyahendriksen & Mariya & Hendriksen & University of Amsterdam \\
% dokato & Dominik & Krzemiński & Cohere For AI Community \\
% Samoed & Roman & Solomatin & ITMO \\
% Alenush & Alena & Fenogenova & SaluteDevices, Russia \\
% ab1992ao & Aleksandr & Abramov & SaluteDevices, Russia \\
% artemsnegirev & Artem & Snegirev & SaluteDevices, Russia \\
% anpalmak2003 & Anna & Maksimova & SaluteDevices, Russia \\
% MariyaTikhonova & Maria & Tikhonova & SaluteDevices, HSE University, Russia \\
% vaibhavad & Vaibhav & Adlakha & McGill University, Mila - Quebec AI Institute, ServiceNow Research \\
% sivareddyg & Siva & Reddy & McGill University, Mila - Quebec AI Institute, ServiceNow Research, Facebook CIFAR AI Chair \\
% guenthermi & Michael & Günther & Jina AI \\
% violenil & Isabelle & Mohr & Jina AI \\
% akshita-sukhlecha & Akshita & Sukhlecha & N/A \\
% Muennighoff & Niklas & Muennighoff & Stanford University, Contextual AI \\
% AlexeyVatolin & Aleksei & Vatolin & FRC CSC RAS \\
% xhluca & Xing Han & Lù & McGill University, Mila - Quebec AI Institute \\
% crystina-z & Xinyu & Zhang & University of Waterloo \\
% tomaarsen & Tom & Aarsen & Hugging Face \\
% crystina-z & Xinyu & Zhang & University of Waterloo \\
% mrshu & Marek & Suppa & Comenius University in Bratislava, Cisco Systems \\
% swj0419 & Weijia & Shi & University of Washington \\
% xiamengzhou & Mengzhou & Xia & Princeton University \\
% john-b-yang & John & Yang & Stanford University \\
% \bottomrule
% \end{tabular}
% }
% \caption{Author overview, along with their affiliations and GitHub handles.}
% \label{tab:authors}
% \end{table*}



\begin{table*}
\centering
{\tiny
\begin{tabular}{llll}
\toprule
GitHub & First name & Last name & Affiliations \\
\midrule
KennethEnevoldsen & Kenneth & Enevoldsen & Aarhus University \\
x-tabdeveloping & Márton & Kardos & Aarhus University \\
imenelydiaker & Imene & Kerboua & INSA Lyon, LIRIS \\
wissam-sib & Wissam & Siblini & Individual Contributor \\
GabrielSequeira & Gabriel & Sequeira & Individual Contributor \\
schmarion & Marion & Schaeffer & Wikit \\
MathieuCiancone & Mathieu & Ciancone & Wikit \\
MartinBernstorff & Martin & Bernstorff & Aarhus University \\
staoxiao & Shitao & Xiao & Beijing Academy of Artificial Intelligence \\
ZhengLiu101 & Zheng & Liu & Beijing Academy of Artificial Intelligence \\
achibb & Aaron & Chibb & Individual Contributor \\
cassanof & Federico & Cassano & Northeastern University and Cursor AI \\
taidnguyen & Nguyen & Tai & University of Pennsylvania \\
xu3kev & Wen-Ding & Li & Cornell University \\
Rysias & Jonathan & Rystrøm & University of Oxford \\
taeminlee & Taemin & Lee & Korea University Human-Inspired AI Research \\
izhx & Xin & Zhang & Harbin Institute of Technology \\
orionw & Orion & Weller & Johns Hopkins University \\
slvnwhrl & Silvan & Wehrli & Robert Koch Institute \\
manandey & Manan & Dey & Salesforce \\
isaac-chung & Isaac & Chung & Individual Contributor \\
asparius & Ömer & Çağatan & Koç University,Turkey \\
rafalposwiata & Rafał & Poświata & National Information Processing Institute \\
rbroc & Roberta & Rocca & Aarhus University \\
awinml & Ashwin & Mathur & Individual Contributor \\
guangyusong & Guangyu & Song & Tano Labs \\
davidstap & David & Stap & University of Amsterdam \\
HLasse & Lasse & Hansen & Aarhus University \\
jaygala24 & Jay & Gala & MBZUAI \\
digantamisra98 & Diganta & Misra & Max Planck Institute for Intelligent Systems and ELLIS Institute Tübingen \\
PranjalChitale & Pranjal & Chitale & Indian Institute of Technology \\
Akash190104 & Akash & Kundu & Heritage Institute of Technology and Apart Research \\
dwzhu-pku & Dawei & Zhu & Peking University \\
ljvmiranda921 & Lester James & Miranda & Allen Institute for AI \\
Andrian0s & Andrianos & Michail & University of Zurich \\
simon-clematide & Simon & Clematide & University of Zurich \\
SaitejaUtpala & Saiteja & Utpala & Microsoft Research \\
mmhamdy & Mohammed & Hamdy & Cohere For AI Community \\
jupyterjazz & Saba & Sturua & Jina AI \\
Ruqyai & Ruqiya & Bin Safi & NaN \\
KranthiGV & Kranthi Kiran & GV & New York University \\
shreeya-dhakal & Shreeya & Dhakal & Individual Contributor \\
dipam7 & Dipam & Vasani & Individual Contributor \\
Art3mis07 & Gayatri & K & R. V. College of Engineering \\
jankounchained & Jan & Kostkan & Aarhus University \\
bp-high & Bhavish & Pahwa & Microsoft Research \\
rasdani & Daniel & Auras & ellamind, Germany \\
ShawonAshraf & Shawon & Ashraf & ellamind, Germany \\
bjoernpl & Björn & Plüster & ellamind, Germany \\
jphme & Jan Philipp & Harries & ellamind, Germany \\
malteos & Malte & Ostendorff & Occiglot \\
ManuelFay & Manuel & Faysse & CentraleSupélec and Illuin Technology \\
hgissbkh & Hippolyte & Gisserot-Boukhlef & CentraleSupélec and Artefact Research Center \\
sted97 & Simone & Tedeschi & Sapienza University of Rome \\
gentaiscool & Genta Indra & Winata & Individual Contributor \\
henilp105 & Henil & Panchal & Nirma University \\
ABorghini & Alessia & Borghini & Sapienza University of Rome \\
jordiclive & Jordan & Clive & Imperial College London \\
gowitheflow-1998 & Chenghao & Xiao & Durham University \\
mariyahendriksen & Mariya & Hendriksen & University of Amsterdam \\
dokato & Dominik & Krzemiński & Cohere For AI Community \\
Samoed & Roman & Solomatin & AI Talent Hub and ITMO University \\
Alenush & Alena & Fenogenova & SaluteDevices \\
ab1992ao & Aleksandr & Abramov & SaluteDevices \\
artemsnegirev & Artem & Snegirev & SaluteDevices \\
anpalmak2003 & Anna & Maksimova & SaluteDevices \\
MariyaTikhonova & Maria & Tikhonova & SaluteDevices and HSE University \\
vaibhavad & Vaibhav & Adlakha & Mila, McGill University and ServiceNow Research \\
sivareddyg & Siva & Reddy & Mila, McGill University and ServiceNow Research \\
guenthermi & Michael & Günther & Jina AI \\
violenil & Isabelle & Mohr & Jina AI \\
akshita-sukhlecha & Akshita & Sukhlecha & Individual Contributor \\
Muennighoff & Niklas & Muennighoff & Stanford University and Contextual AI \\
AlexeyVatolin & Aleksei & Vatolin & FRC CSC RAS \\
xhluca & Xing Han & Lù & Mila, McGill University \\
crystina-z & Xinyu & Zhang & University of Waterloo \\
tomaarsen & Tom & Aarsen & Hugging Face \\
mrshu & Marek & Suppa & Comenius University Bratislava and Cisco Systems \\
swj0419 & Weijia & Shi & University of Washington \\
xiamengzhou & Mengzhou & Xia & Princeton University \\
john-b-yang & John & Yang & Stanford University \\
thakur-nandan & Nandan & Thakur & University of Waterloo \\
loicmagne & Loic & Magne & Individual Contributor \\
sarahooker & Sara & Hooker & Cohere For AI \\
kwojtasi & Konrad & Wojtasik & Wrocław University of Science and Technology \\
jimmy-lin & Jimmy & Lin & University of Waterloo \\
hongjin-su & Hongjin & Su & University of Hong Kong \\
howard-yen & Howard & Yen & Princeton University \\

\bottomrule
\end{tabular}
}
\caption{Author overview, along with their affiliations and GitHub handles.}
\label{tab:authors}
\end{table*}
