%%%%%%%% ICML 2025 EXAMPLE LATEX SUBMISSION FILE %%%%%%%%%%%%%%%%%

\documentclass{article}

% Recommended, but optional, packages for figures and better typesetting:
\usepackage{microtype}
\usepackage{graphicx}
\usepackage{subfigure}
\usepackage{booktabs} % for professional tables
\usepackage{pifont} 

% hyperref makes hyperlinks in the resulting PDF.
% If your build breaks (sometimes temporarily if a hyperlink spans a page)
% please comment out the following usepackage line and replace
% \usepackage{icml2025} with \usepackage[nohyperref]{icml2025} above.
\usepackage{hyperref}
\usepackage{xspace}
\usepackage{booktabs}
\usepackage{multirow}
\usepackage{tabularray}
\usepackage{adjustbox}
\usepackage[table,xcdraw]{xcolor}
\usepackage[normalem]{ulem}
\useunder{\uline}{\ul}{}
\usepackage{framed}

\usepackage[skip=3pt]{caption}
\setlength{\textfloatsep}{4pt plus 3pt minus 3pt}
\setlength{\abovecaptionskip}{0.0cm}
\setlength{\belowcaptionskip}{0.15cm}
\setlength{\abovedisplayskip}{0.0cm}
\setlength{\belowdisplayskip}{0.15cm}
\setlength{\topsep}{0.0cm}
\setlength{\partopsep}{0.0cm}
\setlength{\itemsep}{0.0cm}


\definecolor{lightred}{HTML}{FFEEEE}
\definecolor{lightgreen}{HTML}{E6FFEC}
\definecolor{lightblue}{HTML}{E1EDFF}
\definecolor{lightgray}{HTML}{EBEBEB}
\definecolor{lightyellow}{HTML}{FFF2D9}
\newcommand{\prompt}[1]{{\ttfamily #1}\xspace}
\newcommand{\ipi}{IPI\xspace}
\newcommand{\llm}{LLMs\xspace}
\newcommand{\gpt}{GPT-4o\xspace}
\newcommand{\othree}{o3-mini\xspace}
\newcommand{\llama}{Llama-3.3-70B\xspace}
\newcommand{\claude}{Claude-3.5-Sonnet\xspace}
\newcommand{\dojo}{AgentDojo\xspace}
\newcommand{\webadv}{VWA-Adv\xspace}

\newcommand*\user[1][1.5em]{$\vcenter{\hbox{\includegraphics[height=#1]{icons/user.png}}}$}
\newcommand*\robot[1][1.5em]{$\vcenter{\hbox{\includegraphics[height=#1]{icons/robot.png}}}$}
\newcommand*\tool[1][1.5em]{$\vcenter{\hbox{\includegraphics[height=#1]{icons/tool_exec.png}}}$}

% \newcommand{\method}{COCO\xspace}
\newcommand{\method}{MELON\xspace}

\usepackage{xcolor}

\newcommand{\wjdd}[1]{\todo[linecolor=cyan,backgroundcolor=cyan!25,bordercolor=cyan,size=\scriptsize]{(Jindong) #1}}
\newcommand{\wjd}[1]{{\color{cyan}{[(Jindong) #1]}}}
\newcommand{\kaijie}[1]{{\color{red}{[(Kaijie) #1]}}}
% Attempt to make hyperref and algorithmic work together better:
\newcommand{\theHalgorithm}{\arabic{algorithm}}

% Use the following line for the initial blind version submitted for review:
% \usepackage{icml2025}

% If accepted, instead use the following line for the camera-ready submission:
\usepackage[accepted]{icml2025}

% For theorems and such
\usepackage{amsmath}
\usepackage{amssymb}
\usepackage{mathtools}
\usepackage{amsthm}

% if you use cleveref..
\usepackage[capitalize,noabbrev]{cleveref}

%%%%%%%%%%%%%%%%%%%%%%%%%%%%%%%%
% THEOREMS
%%%%%%%%%%%%%%%%%%%%%%%%%%%%%%%%
\theoremstyle{plain}
\newtheorem{theorem}{Theorem}[section]
\newtheorem{proposition}[theorem]{Proposition}
\newtheorem{lemma}[theorem]{Lemma}
\newtheorem{corollary}[theorem]{Corollary}
\theoremstyle{definition}
\newtheorem{definition}[theorem]{Definition}
\newtheorem{assumption}[theorem]{Assumption}
\theoremstyle{remark}
\newtheorem{remark}[theorem]{Remark}

% Todonotes is useful during development; simply uncomment the next line
%    and comment out the line below the next line to turn off comments
%\usepackage[disable,textsize=tiny]{todonotes}
\usepackage[textsize=tiny]{todonotes}


% The \icmltitle you define below is probably too long as a header.
% Therefore, a short form for the running title is supplied here:
\icmltitlerunning{\method: Indirect Prompt Injection Defense via Masked Re-execution and Tool Comparison}

\begin{document}

\twocolumn[
\icmltitle{\method: Indirect Prompt Injection Defense via\\ Masked Re-execution and Tool Comparison}

% It is OKAY to include author information, even for blind
% submissions: the style file will automatically remove it for you
% unless you've provided the [accepted] option to the icml2025
% package.

% List of affiliations: The first argument should be a (short)
% identifier you will use later to specify author affiliations
% Academic affiliations should list Department, University, City, Region, Country
% Industry affiliations should list Company, City, Region, Country

% You can specify symbols, otherwise they are numbered in order.
% Ideally, you should not use this facility. Affiliations will be numbered
% in order of appearance and this is the preferred way.
\icmlsetsymbol{equal}{*}

\begin{icmlauthorlist}
\icmlauthor{Kaijie Zhu}{ucsb}
\icmlauthor{Xianjun Yang}{ucsb}
\icmlauthor{Jindong Wang}{wm}
\icmlauthor{Wenbo Guo}{ucsb}
\icmlauthor{William Wang}{ucsb}

\end{icmlauthorlist}

\icmlaffiliation{ucsb}{UCSB}
\icmlaffiliation{wm}{William \& Mary}
\icmlcorrespondingauthor{Kaijie Zhu}{kaijiezhu@ucsb.edu}

% You may provide any keywords that you
% find helpful for describing your paper; these are used to populate
% the "keywords" metadata in the PDF but will not be shown in the document
\icmlkeywords{Machine Learning, ICML}

\vskip 0.3in
]

% this must go after the closing bracket ] following \twocolumn[ ...

% This command actually creates the footnote in the first column
% listing the affiliations and the copyright notice.
% The command takes one argument, which is text to display at the start of the footnote.
% The \icmlEqualContribution command is standard text for equal contribution.
% Remove it (just {}) if you do not need this facility.

\printAffiliationsAndNotice{}  % leave blank if no need to mention equal contribution
% \printAffiliationsAndNotice{\icmlEqualContribution} % otherwise use the standard text.

Large language model (LLM)-based agents have shown promise in tackling complex tasks by interacting dynamically with the environment. 
Existing work primarily focuses on behavior cloning from expert demonstrations and preference learning through exploratory trajectory sampling. However, these methods often struggle in long-horizon tasks, where suboptimal actions accumulate step by step, causing agents to deviate from correct task trajectories.
To address this, we highlight the importance of \textit{timely calibration} and the need to automatically construct calibration trajectories for training agents. We propose \textbf{S}tep-Level \textbf{T}raj\textbf{e}ctory \textbf{Ca}libration (\textbf{\model}), a novel framework for LLM agent learning. 
Specifically, \model identifies suboptimal actions through a step-level reward comparison during exploration. It constructs calibrated trajectories using LLM-driven reflection, enabling agents to learn from improved decision-making processes. These calibrated trajectories, together with successful trajectory data, are utilized for reinforced training.
Extensive experiments demonstrate that \model significantly outperforms existing methods. Further analysis highlights that step-level calibration enables agents to complete tasks with greater robustness. 
Our code and data are available at \url{https://github.com/WangHanLinHenry/STeCa}.
%!TEX root = gcn.tex
\section{Introduction}
Graphs, representing structural data and topology, are widely used across various domains, such as social networks and merchandising transactions.
Graph convolutional networks (GCN)~\cite{iclr/KipfW17} have significantly enhanced model training on these interconnected nodes.
However, these graphs often contain sensitive information that should not be leaked to untrusted parties.
For example, companies may analyze sensitive demographic and behavioral data about users for applications ranging from targeted advertising to personalized medicine.
Given the data-centric nature and analytical power of GCN training, addressing these privacy concerns is imperative.

Secure multi-party computation (MPC)~\cite{crypto/ChaumDG87,crypto/ChenC06,eurocrypt/CiampiRSW22} is a critical tool for privacy-preserving machine learning, enabling mutually distrustful parties to collaboratively train models with privacy protection over inputs and (intermediate) computations.
While research advances (\eg,~\cite{ccs/RatheeRKCGRS20,uss/NgC21,sp21/TanKTW,uss/WatsonWP22,icml/Keller022,ccs/ABY318,folkerts2023redsec}) support secure training on convolutional neural networks (CNNs) efficiently, private GCN training with MPC over graphs remains challenging.

Graph convolutional layers in GCNs involve multiplications with a (normalized) adjacency matrix containing $\numedge$ non-zero values in a $\numnode \times \numnode$ matrix for a graph with $\numnode$ nodes and $\numedge$ edges.
The graphs are typically sparse but large.
One could use the standard Beaver-triple-based protocol to securely perform these sparse matrix multiplications by treating graph convolution as ordinary dense matrix multiplication.
However, this approach incurs $O(\numnode^2)$ communication and memory costs due to computations on irrelevant nodes.
%
Integrating existing cryptographic advances, the initial effort of SecGNN~\cite{tsc/WangZJ23,nips/RanXLWQW23} requires heavy communication or computational overhead.
Recently, CoGNN~\cite{ccs/ZouLSLXX24} optimizes the overhead in terms of  horizontal data partitioning, proposing a semi-honest secure framework.
Research for secure GCN over vertical data  remains nascent.

Current MPC studies, for GCN or not, have primarily targeted settings where participants own different data samples, \ie, horizontally partitioned data~\cite{ccs/ZouLSLXX24}.
MPC specialized for scenarios where parties hold different types of features~\cite{tkde/LiuKZPHYOZY24,icml/CastigliaZ0KBP23,nips/Wang0ZLWL23} is rare.
This paper studies $2$-party secure GCN training for these vertical partition cases, where one party holds private graph topology (\eg, edges) while the other owns private node features.
For instance, LinkedIn holds private social relationships between users, while banks own users' private bank statements.
Such real-world graph structures underpin the relevance of our focus.
To our knowledge, no prior work tackles secure GCN training in this context, which is crucial for cross-silo collaboration.


To realize secure GCN over vertically split data, we tailor MPC protocols for sparse graph convolution, which fundamentally involves sparse (adjacency) matrix multiplication.
Recent studies have begun exploring MPC protocols for sparse matrix multiplication (SMM).
ROOM~\cite{ccs/SchoppmannG0P19}, a seminal work on SMM, requires foreknowledge of sparsity types: whether the input matrices are row-sparse or column-sparse.
Unfortunately, GCN typically trains on graphs with arbitrary sparsity, where nodes have varying degrees and no specific sparsity constraints.
Moreover, the adjacency matrix in GCN often contains a self-loop operation represented by adding the identity matrix, which is neither row- nor column-sparse.
Araki~\etal~\cite{ccs/Araki0OPRT21} avoid this limitation in their scalable, secure graph analysis work, yet it does not cover vertical partition.

% and related primitives
To bridge this gap, we propose a secure sparse matrix multiplication protocol, \osmm, achieving \emph{accurate, efficient, and secure GCN training over vertical data} for the first time.

\subsection{New Techniques for Sparse Matrices}
The cost of evaluating a GCN layer is dominated by SMM in the form of $\adjmat\feamat$, where $\adjmat$ is a sparse adjacency matrix of a (directed) graph $\graph$ and $\feamat$ is a dense matrix of node features.
For unrelated nodes, which often constitute a substantial portion, the element-wise products $0\cdot x$ are always zero.
Our efficient MPC design 
avoids unnecessary secure computation over unrelated nodes by focusing on computing non-zero results while concealing the sparse topology.
We achieve this~by:
1) decomposing the sparse matrix $\adjmat$ into a product of matrices (\S\ref{sec::sgc}), including permutation and binary diagonal matrices, that can \emph{faithfully} represent the original graph topology;
2) devising specialized protocols (\S\ref{sec::smm_protocol}) for efficiently multiplying the structured matrices while hiding sparsity topology.


 
\subsubsection{Sparse Matrix Decomposition}
We decompose adjacency matrix $\adjmat$ of $\graph$ into two bipartite graphs: one represented by sparse matrix $\adjout$, linking the out-degree nodes to edges, the other 
by sparse matrix $\adjin$,
linking edges to in-degree nodes.

%\ie, we decompose $\adjmat$ into $\adjout \adjin$, where $\adjout$ and $\adjin$ are sparse matrices representing these connections.
%linking out-degree nodes to edges and edges to in-degree nodes of $\graph$, respectively.

We then permute the columns of $\adjout$ and the rows of $\adjin$ so that the permuted matrices $\adjout'$ and $\adjin'$ have non-zero positions with \emph{monotonically non-decreasing} row and column indices.
A permutation $\sigma$ is used to preserve the edge topology, leading to an initial decomposition of $\adjmat = \adjout'\sigma \adjin'$.
This is further refined into a sequence of \emph{linear transformations}, 
which can be efficiently computed by our MPC protocols for 
\emph{oblivious permutation}
%($\Pi_{\ssp}$) 
and \emph{oblivious selection-multiplication}.
% ($\Pi_\SM$)
\iffalse
Our approach leverages bipartite graph representation and the monotonicity of non-zero positions to decompose a general sparse matrix into linear transformations, enhancing the efficiency of our MPC protocols.
\fi
Our decomposition approach is not limited to GCNs but also general~SMM 
by 
%simply 
treating them 
as adjacency matrices.
%of a graph.
%Since any sparse matrix can be viewed 

%allowing the same technique to be applied.

 
\subsubsection{New Protocols for Linear Transformations}
\emph{Oblivious permutation} (OP) is a two-party protocol taking a private permutation $\sigma$ and a private vector $\xvec$ from the two parties, respectively, and generating a secret share $\l\sigma \xvec\r$ between them.
Our OP protocol employs correlated randomnesses generated in an input-independent offline phase to mask $\sigma$ and $\xvec$ for secure computations on intermediate results, requiring only $1$ round in the online phase (\cf, $\ge 2$ in previous works~\cite{ccs/AsharovHIKNPTT22, ccs/Araki0OPRT21}).

Another crucial two-party protocol in our work is \emph{oblivious selection-multiplication} (OSM).
It takes a private bit~$s$ from a party and secret share $\l x\r$ of an arithmetic number~$x$ owned by the two parties as input and generates secret share $\l sx\r$.
%between them.
%Like our OP protocol, o
Our $1$-round OSM protocol also uses pre-computed randomnesses to mask $s$ and $x$.
%for secure computations.
Compared to the Beaver-triple-based~\cite{crypto/Beaver91a} and oblivious-transfer (OT)-based approaches~\cite{pkc/Tzeng02}, our protocol saves ${\sim}50\%$ of online communication while having the same offline communication and round complexities.

By decomposing the sparse matrix into linear transformations and applying our specialized protocols, our \osmm protocol
%($\prosmm$) 
reduces the complexity of evaluating $\numnode \times \numnode$ sparse matrices with $\numedge$ non-zero values from $O(\numnode^2)$ to $O(\numedge)$.

%(\S\ref{sec::secgcn})
\subsection{\cgnn: Secure GCN made Efficient}
Supported by our new sparsity techniques, we build \cgnn, 
a two-party computation (2PC) framework for GCN inference and training over vertical
%ly split
data.
Our contributions include:

1) We are the first to explore sparsity over vertically split, secret-shared data in MPC, enabling decompositions of sparse matrices with arbitrary sparsity and isolating computations that can be performed in plaintext without sacrificing privacy.

2) We propose two efficient $2$PC primitives for OP and OSM, both optimally single-round.
Combined with our sparse matrix decomposition approach, our \osmm protocol ($\prosmm$) achieves constant-round communication costs of $O(\numedge)$, reducing memory requirements and avoiding out-of-memory errors for large matrices.
In practice, it saves $99\%+$ communication
%(Table~\ref{table:comm_smm}) 
and reduces ${\sim}72\%$ memory usage over large $(5000\times5000)$ matrices compared with using Beaver triples.
%(Table~\ref{table:mem_smm_sparse}) ${\sim}16\%$-

3) We build an end-to-end secure GCN framework for inference and training over vertically split data, maintaining accuracy on par with plaintext computations.
We will open-source our evaluation code for research and deployment.

To evaluate the performance of $\cgnn$, we conducted extensive experiments over three standard graph datasets (Cora~\cite{aim/SenNBGGE08}, Citeseer~\cite{dl/GilesBL98}, and Pubmed~\cite{ijcnlp/DernoncourtL17}),
reporting communication, memory usage, accuracy, and running time under varying network conditions, along with an ablation study with or without \osmm.
Below, we highlight our key achievements.

\textit{Communication (\S\ref{sec::comm_compare_gcn}).}
$\cgnn$ saves communication by $50$-$80\%$.
(\cf,~CoGNN~\cite{ccs/KotiKPG24}, OblivGNN~\cite{uss/XuL0AYY24}).

\textit{Memory usage (\S\ref{sec::smmmemory}).}
\cgnn alleviates out-of-memory problems of using %the standard 
Beaver-triples~\cite{crypto/Beaver91a} for large datasets.

\textit{Accuracy (\S\ref{sec::acc_compare_gcn}).}
$\cgnn$ achieves inference and training accuracy comparable to plaintext counterparts.
%training accuracy $\{76\%$, $65.1\%$, $75.2\%\}$ comparable to $\{75.7\%$, $65.4\%$, $74.5\%\}$ in plaintext.

{\textit{Computational efficiency (\S\ref{sec::time_net}).}} 
%If the network is worse in bandwidth and better in latency, $\cgnn$ shows more benefits.
$\cgnn$ is faster by $6$-$45\%$ in inference and $28$-$95\%$ in training across various networks and excels in narrow-bandwidth and low-latency~ones.

{\textit{Impact of \osmm (\S\ref{sec:ablation}).}}
Our \osmm protocol shows a $10$-$42\times$ speed-up for $5000\times 5000$ matrices and saves $10$-2$1\%$ memory for ``small'' datasets and up to $90\%$+ for larger ones.

\section{Related Work}

\subsection{Personalization and Role-Playing}
Recent works have introduced benchmark datasets for personalizing LLM outputs in tasks like email, abstract, and news writing, focusing on shorter outputs (e.g., 300 tokens for product reviews \citep{kumar2024longlamp} and 850 for news writing \citep{shashidhar-etal-2024-unsupervised}). These methods infer user traits from history for task-specific personalization \citep{sun-etal-2024-revealing, sun-etal-2025-persona, pal2024beyond, li2023teach, salemi2025reasoning}. In contrast, we tackle the more subjective problem of long-form story writing, with author stories averaging 1500 tokens. Unlike prior role-playing approaches that use predefined personas (e.g., Tony Stark, Confucius) \citep{wang-etal-2024-rolellm, sadeq-etal-2024-mitigating, tu2023characterchat, xu2023expertprompting}, we propose a novel method to infer story-writing personas from an author’s history to guide role-playing.


\subsection{Story Understanding and Generation}  
Prior work on persona-aware story generation \citep{yunusov-etal-2024-mirrorstories, bae-kim-2024-collective, zhang-etal-2022-persona, chandu-etal-2019-way} defines personas using discrete attributes like personality traits, demographics, or hobbies. Similarly, \citep{zhu-etal-2023-storytrans} explore story style transfer across pre-defined domains (e.g., fairy tales, martial arts, Shakespearean plays). In contrast, we mimic an individual author's writing style based on their history. Our approach differs by (1) inferring long-form author personas—descriptions of an author’s style from their past works, rather than relying on demographics, and (2) handling long-form story generation, averaging 1500 tokens per output, exceeding typical story lengths in prior research.


% \begin{figure*}[t!]
% \centering
% \includegraphics[width=0.7\textwidth]{figs/melon.pdf}
% \caption{The pipeline of proposed \method detection method. Green and red boxes indicate benign and potentially compromised states respectively. The method compares tool calls between the original execution path (top) and an instruction-seeking path (bottom) to detect indirect prompt injection attacks. The red arrow in tool execution indicates potential malicious instruction injection.}
% \label{fig: overall}
% \end{figure*}

\begin{figure*}[t!]
\centering
\includegraphics[width=\textwidth]{figs/melon_horizon.pdf}
\vspace{-.3in}
\caption{\method detection pipeline comparing two parallel execution paths: the original run (top) and masking run (bottom). The original run processes the user task $\mathcal{T}_u$ while the masking run uses a task-neutral prompt $\mathcal{T}_f$. Green boxes show benign execution where no malicious task is detected (left), as the masking run generates no matching tool calls. Red boxes indicate prompt injection attack (right), where the agent deviates to execute an injected malicious task $\mathcal{T}_m$, resulting in matching tool calls between the two paths. Tool calls from the masking run are stored in the tool call cache $\mathcal{H}_t$ for comparison.}
\label{fig: overall}
\vspace{-.15in}
\end{figure*}


\section{Metholody of \method}
\label{sec:tech}
\subsection{Preliminaries}
\label{subsec:prelim}
\textbf{Formalization and Definition of LLM Agent.}
% There is no standardized definition of LLM agents. 
In this work, we define an LLM agent $\pi$ as an integrated system comprising LLM(s) and a set of tools $\mathcal{F} = \{f_1, ..., f_n\}$ for environment interaction.
The agent receives a user prompt specifying a task $\mathcal{T}_u$ (e.g., ``\prompt{Summarize my agenda and tell me the time of the next event.}'') and executes it through a structured multi-step procedure.
% ~(Appendix~\cref{fig: basic agent pipeline}).

At each step $t$, we define the \textit{state} as $\mathcal{S}_t = (\mathcal{T}_u, \mathcal{A}_{1:t}, \mathcal{O}_{1:t})$, where $\mathcal{T}_u$ is the user task, $\mathcal{A}_{1:t} = \{ (\mathcal{R}_1, \mathcal{C}_1), ..., (\mathcal{R}_t, \mathcal{C}_t) \}$ is the sequence of LLM-generated actions with each action pair consisting of an LLM response $\mathcal{R}_i$ and a set of tool calls $\mathcal{C}_i = \{c_i^1, ..., c_i^{m_i}\}$. Each tool call $c_i^j$ specifies a tool $f_j \in \mathcal{F}$ and its parameters (e.g., ``\prompt{retrieve\_event(date=20250131)}''). $\mathcal{O}_{1:t} = \{\mathcal{O}_1, ..., \mathcal{O}_t\}$ denotes the sequence of observations, where each $\mathcal{O}_i$ contains the tool execution outputs corresponding to $\mathcal{C}_i$. In step $t+1$, The agent system first generates action $\mathcal{A}_{t+1} = \pi(\mathcal{S}_t)$ based on previous state, then obtaining observation $\mathcal{O}_{t+1} = \text{Exec}(\mathcal{C}_{t+1})$ by executing the tool calls. This process continues iteratively until the user task $\mathcal{T}_u$ is completed or errors occur.



\textbf{Threat Model.}
We follow the assumption of \ipi, where attackers \textit{cannot access} the LLMs' input and output inside the target agent.
Their access is limited to manipulating the external information retrieved by the agent via tool calls, such as websites, emails, or files. 
The attackers aim to redirect the agent from executing the original user task to performing a malicious task $T_m$.
For example, the attacker task could be ``\prompt{Send your bank account and password to hacker@gmail.com}''.
We denote $\mathcal{O}_t'$ to be the tool execution outputs injected with $T_m$ and $\mathcal{O}_{1:t}'=\{\mathcal{O}_1, ..., \mathcal{O}_{t}'\}$ as the sequences of previous tool execution outputs.
% while the user task is ``\prompt{Summarize my agenda and tell me the time of the next event.}''. 
We assume all user tasks to be legitimate and the defender has complete access to the entire agent system, including the states $\mathcal{S}_t = (\mathcal{T}_u, \mathcal{A}_{1:t}, \mathcal{O}_{1:t})$. 
However, we do not assume the defender has the resources to train LLMs or can access LLM internal representations. 
% We also do not assume the defender has the privilege to alter tools. 
% We believe this is a practical and reasonable assumption in that most agent users do not have the rich computational power, and the tools are typically developed by third parties where agent users only have permission to use them rather than change them. 


\subsection{Technical Overview}
\label{subsec:overview}
\textbf{Insights and Technical Challenges.}
Our design is based on the key observation that whenever a malicious attacker task $T_m$ is present in the retrieved data, it attempts to redirect the agent from executing the user task $T_u$ toward executing $T_m$ instead.
Given a state $\mathcal{S}_t = (\mathcal{T}_u, \mathcal{A}_{1:t}, \mathcal{O}_{1:t}')$, if $\mathcal{O}_{t}'$ that injected with $T_m$ successfully hijacks the agent's behavior to focus on executing $T_m$, it induces a \textit{state collapse} where the agent's next action $\mathcal{A}_{t+1}$ becomes conditionally independent of $\mathcal{T}_u$ and $\mathcal{A}_{1:t}$, depending primarily on $\mathcal{O}_{1:t}'$. For benign cases where $\mathcal{O}_{t}$ does not contain malicious instructions or the attack does not succeed, the agent maintains functional dependencies on all state components $(\mathcal{T}_u, \mathcal{A}_{1:t}, \mathcal{O}_{1:t})$. 
Formally, for a successful attack at step $t$, we can observe:
$
    \mathbb{P}_\pi( \mathcal{A}_{t+1} | (\mathcal{T}_u, \mathcal{A}_{1:t}, \mathcal{O}_{1:t}')) \approx \mathbb{P}_\pi( \mathcal{A}_{t+1} | \mathcal{O}_{1:t}'),
$
where $\mathbb{P}$ is the probability. For benign executions, the agent's actions maintain their dependency on the user inputs:
$
    \mathbb{P}_\pi( \mathcal{A}_{t+1} | (\mathcal{T}_u, \mathcal{A}_{1:t}, \mathcal{O}_{1:t})) \gg \mathbb{P}_\pi( \mathcal{A}_{t+1} | \mathcal{O}_{1:t}).
$
% where either $\mathcal{O}_{1:t}$ is benign, or the LLM maintains its adherence to $\mathcal{T}_u$ despite the presence of malicious task $\mathcal{T}_m$ in $\mathcal{O}_{1:t}$. 


This statistical discrepancy naturally leads to a masking-based approach to detect the injected prompts.
Let $\mathcal{M}: \mathcal{S} \rightarrow \mathcal{S}$ be a masking operator that preserves only original tool execution outputs. We define two execution paths:
\begin{align*}
    \mathcal{A}_{t+1}^o = (\mathcal{R}_{t+1}^o, \mathcal{C}_{t+1}^o) &= \pi(\mathcal{S}_t) \text{ (the original run)}, \\
    \mathcal{A}_{t+1}^m = (\mathcal{R}_{t+1}^m, \mathcal{C}_{t+1}^m) &= \pi(\mathcal{M}(\mathcal{S}_t)) \text{ (the masking run)},
\end{align*}
where $\mathcal{*}_{t+1}^o$ and $\mathcal{*}_{t+1}^m$ represent the action generated by the original run and masked run at step $t+1$, respectively.
By comparing the agent's next action generated by the original run and the masking run, we can identify potential attacks:
\textit{
if the action deviates after masking, that indicates the agent's action is related to the user task and the original execution is benign, otherwise, the execution is independent from the user prompt, indicating there are attacker-injected prompts during the execution.
}

% Following this idea, a straightforward approach can be that~\textit{at step $t+1$, when the agent generates outputs $(R_{t+1}, C_{t+1}) = \pi\bigl(T_u, (R_{1:t}, C_{1:t}, E_{1:t})\bigr)$}, we feed the tool execution outputs to the target agent while masking the rest contents, and record if the LLM output deviates from the original one without masking. 

To compare the actions of~\textit{the original run} and~\textit{the masking run}, we can feed $\mathcal{A}_{t+1}^o$ and $\mathcal{A}_{t+1}^m$ into a text embedding model and compute their distance in the embedding space.
If at any step, we observe that the distance is~\textit{higher} than a certain threshold, it means the LLM output depends only on $\mathcal{O}_{1:t}$ and signals an attacker-injected prompt. 




However, although straightforward, this masking function and the comparison method face three~\textit{technical challenges} during our exploration. 
\ding{182}, when presented with only $\mathcal{O}_{1:t}$ in the masking run, the LLM may treat these tool execution outputs as few-shot examples and generate contextually plausible but potentially arbitrary tool calls. Thus, for injected cases, the generated tool calls may deviate from the malicious task $\mathcal{T}_m$ embedded in $\mathcal{O}_{t}'$, leading to false negatives, while for benign cases, these arbitrary calls might coincidentally align with the user task $\mathcal{T}_u$, causing false positives.
% original pass delay 
\ding{183}, we observe that, in some cases, the execution timing of $\mathcal{T}_m$ differs between the original run and the masking run. In the original run, the agent might execute the user task $\mathcal{T}_u$ first before proceeding to $\mathcal{T}_m$. However, in the masking run, since there is no user task to complete, the agent directly executes $\mathcal{T}_m$. As a result, at step $t$, while the masking run has begun executing $\mathcal{T}_m$, the original run may still be processing $\mathcal{T}_u$. This timing mismatch leads to false negatives when comparing tool calls at corresponding steps.
\ding{184}, typically, the action generated by LLM contains both text responses $\mathcal{R}$ and the tool calls $\mathcal{C}$ with specific arguments. In successful attack cases, the masking run and original run may differ in their text responses $\mathcal{R}$ while maintaining identical tool calls $\mathcal{C}$ that execute the malicious task $\mathcal{T}_m$, which again introduces false negatives.
% Furthermore, when the user's task is about summarizing the previous state, masking $T_u$ will not change the output in benign cases that much, as the LLM tends to summarize the input when it does not contain a clear instruction.
% Here, directly comparing the output will also introduce false positives.



\begin{algorithm}[t]
\caption{\method Algorithm at Step $t$}
\label{alg:detection}
\begin{algorithmic}[1]
\REQUIRE Agent $\pi$, user task $\mathcal{T}_u$, masking function $\mathcal{M}$, similarity threshold $\theta$, state $\mathcal{S}_t$, tool call cache $\mathcal{H}_{t}$.
\ENSURE Detection result.

\STATE $\mathcal{O}_1^t \leftarrow \text{concat}(\mathcal{O}_{1:t})$ \COMMENT{Consolidate tool outputs}

\STATE $\mathcal{S}_m \leftarrow \mathcal{M}(\mathcal{S}_t) = (\mathcal{T}_f, \emptyset, \mathcal{O}_1^t)$ \COMMENT{Construct masked state}
\STATE $\mathcal{A}_{t+1}^m \leftarrow \pi(\mathcal{S}_m)$ \COMMENT{Generate masked action}
\STATE $\mathcal{C}_{t+1}^m \leftarrow \text{tools}(\mathcal{A}_{t+1}^m)$ \COMMENT{Extract tool calls}
\STATE $\mathcal{H}_{t+1} \leftarrow \mathcal{H}_{t+1} \cup \mathcal{C}_{t+1}^m$ \COMMENT{Update tool call cache}

\STATE $\mathcal{S}_o \leftarrow (\mathcal{T}_u, \mathcal{A}_{1:t}, \mathcal{O}_{1:t})$ \COMMENT{Original state}
\STATE $\mathcal{A}_{t+1}^o \leftarrow \pi(\mathcal{S}_o)$ \COMMENT{Generate original action}
\STATE $\mathcal{C}_{t+1}^o \leftarrow \text{tools}(\mathcal{A}_{t+1}^o)$ \COMMENT{Extract tool calls}

\FOR{each tool call $c \in \mathcal{C}_{t+1}^o$}
    \FOR{each cached call $h \in \mathcal{H}_{t+1}$}
        \STATE $\text{sim} \leftarrow \frac{\phi(c) \cdot \phi(h)}{|\phi(c)| \cdot |\phi(h)|}$ \COMMENT{Compute similarity}
        \IF{$\text{sim} > \theta$}
            \STATE \textbf{return} True \COMMENT{Attack detected}
        \ENDIF
    \ENDFOR
\ENDFOR
\STATE \textbf{return} False \COMMENT{No attack detected}
\end{algorithmic}

\end{algorithm}




\textbf{Overview of \method.}
We design \method by optimizing the straightforward approach above with novel strategies to address the key technical challenges (demonstrated in~\cref{fig: overall}).
Specifically, to address challenge \ding{182}, we modify the masking function $\mathcal{M}$ in two ways: (1) instead of simply removing $\mathcal{T}_u$, we replace it with a task-neutral prompt $\mathcal{T}_f$ (see \cref{appendix-method-tf}) that typically requires no tool interactions, and (2) we consolidate the sequence of tool outputs $\mathcal{O}_{1:t}$ into a single context $\mathcal{O}_1^t$. 
Formally, our modified masking function is:
$\mathcal{M}(\mathcal{T}_u, \mathcal{A}_{1:t}, \mathcal{O}_{1:t}) = (\mathcal{T}_f, \emptyset, \mathcal{O}_1^t)$. 
To further make the LLM better follow the $\mathcal{T}_f$ in the masking run, we create two few-shot chat samples (see~\cref{appendix-method-fewshot}) that include user, LLM response, and tool execution outputs.
These designs together reduce false negatives by guiding the LLM to focus on detecting embedded instructions rather than mimicking the tool call patterns in $\mathcal{O}_{1:t}$.
For challenge \ding{183}, we implement a tool call cache $\mathcal{H}$ that records the sequence of tool calls from the masking run: $\mathcal{H}_{t+1} = \{\mathcal{C}_1^m, ..., \mathcal{C}_{t+1}^m\}$, where $\mathcal{C}_i^m$ denotes the tool calls generated by the masking run in step $i$. We identify an attack when any tool call in the original run is similar to a cached call: $\exists c \in \mathcal{C}_{t+1}^o, \exists h \in \mathcal{H}: \text{sim} (c, h) > \theta$, where $\theta$ is the similarity threshold. 
This cache-based approach reduces false negatives by capturing all potential attack-related tool calls across multiple steps.
Note that this cache-based detection will not introduce new false positives because tool calls in $\mathcal{H}_{t+1}$ indicate potential malicious instructions since matching calls in the original run suggests the LLM is executing instructions embedded in previous tool execution outputs.
Finally, to tackle challenge \ding{184}, we focus our comparison specifically on tool calls rather than complete actions. Let $\text{tools}(\mathcal{A})$ extract the tool calls from an action. We update the detection criterion as: $\text{sim}(\text{tools}(\mathcal{A}_{t+1}^o), \text{tools}(\mathcal{A}_{t+1}^m)) > \theta$.
This design is motivated by the fact that accomplishing the attack task typically involves calling one or more tools that are correlated to the malicious attacker task $\mathcal{T}_m$.
This design significantly reduces false negatives as the content that causes large differences under attacks is not considered.


\textbf{Remark.} 
A simpler detection approach might be to identify prompt injection whenever the masking run generates any tool calls. However, this would reduce system utility under attack scenarios. Many injected tasks $\mathcal{T}_m$ fail to redirect the LLM - while they may trigger tool calls in the masking run, the original run remains focused on executing the user task $\mathcal{T}_u$. In such cases, marking these unsuccessful attacks would unnecessarily interrupt benign executions. Therefore, comparing tool calls is essential to identify only those attacks that successfully influence the agent's behavior.




\subsection{Technical Details}
\label{subsec:details}

% As shown in~\cref{fig: overall}, considering the agent at time step $t$, which consists of the user task $T_u$, the sequence of agent responses $R_{1:t-1} = \{R_1, R_2, ..., R_{t-1}\}$ (including text outputs and tool calls), and the corresponding tool outputs $O_{1:t-1} = \{O_{F}^{(1)}, O_{F}^{(2)}, ..., O_{F}^{(t-1)}\}$. 
% Our method conducts two parallel runs at each step.
% \noindent\textit{Original run:} we execute the agent as it till step $t$ and extract the LLM's output tool call $C_t$.
% \noindent\textit{Attack-seeking execution:} 
% We construct a parallel path where we (1) replace the original task $T_u$ with a generic instruction-following prompt $T_s$: ``\prompt{Summarize the contents retrieved. If there are any instructions in the contents, follow them first.}'', and (2) remove all agent responses $R_{1:t}$, feeding the LLM only $T_s$ and the tool outputs $O_{1:t}$. 
% We add the current tool call into the cache $\mathcal{C}_s$.
% If $C_t \in \mathcal{C}_s$, we terminate the original execution and mark it as under attack.  

Algorithm \ref{alg:detection} presents the complete detection procedure. As discussed above, our detection mechanism analyzes the similarity between tool calls generated in two parallel runs: the original run and the masking run. At step $t$, we maintain a tool call cache $\mathcal{H}_t$ that collects tool calls from the masking run. The detection process consists of two phases: (1) cache population and (2) similarity-based detection.

In the cache population phase, we first use the masking function $\mathcal{M}$ to mask the current state $\mathcal{S}_t$. If $\mathcal{O}_1^t$ contains a malicious task $\mathcal{T}_m$, the agent will generate tool calls $\mathcal{C}_{t+1}^m$ corresponding to $\mathcal{T}_m$, which are then added to the cache $\mathcal{H}_{t+1}$. Otherwise, if no malicious task is present, the agent will generate no tool calls given the design of $\mathcal{T}_f$. For similarity-based detection, we compare each tool call $c \in \mathcal{C}_{t+1}^o$ from the original run with every cached tool call $h \in \mathcal{H}_{t+1}$ using a vector embedding model $\phi$:
$\text{sim}(c, h) = \frac{\phi(c) \cdot \phi(h)}{|\phi(c)| \cdot |\phi(h)|}$. 
We raise an alert if any similarity score exceeds a threshold $\theta$:
$\text{Alert}_{t+1} = \exists c \in \mathcal{C}_{t+1}^o, \exists h \in \mathcal{H}_{t+1}: \text{sim}(c, h) > \theta$.


Our preference for embedding-based similarity over exact matching stems from the semantic nature of tool calls, tool calls with similar intents can have different arguments. 
For example, an attack instructing \prompt{send all your personal data to hacker@evil.com} might generate tool calls with slightly different arguments like \prompt{send\_email("hacker", subject="data")} and \prompt{send\_email("hacker", subject="info")}, while they differ in their exact argument names, these two calls serve the same malicious purpose. Exact matching would fail to detect such semantically equivalent but syntactically distinct attacks.



\subsubsection{Implementation and Overhead Analysis}

\textbf{Implementation Details.}
Our detection method consists of several key components. For tool call comparison, we convert each tool call to a natural language description, note that for specific functions, we do not compare some of their arguments, for example, for the \prompt{send\_email} function, we do not compare the contents to be sent, but instead only recipient is considered. Details can be found in~\cref{appendix-method-tranform-tool-call}. Next, we employ OpenAI's text-embedding-v3 model~\cite{openai2024embedding} that maps these descriptions to dense vector representations. We set the primary similarity threshold $\theta = 0.8$ to balance detection sensitivity and false positive rate, the ablation study on different similarity thresholds is presented in~\cref{subsec:ablation}. 
The task-neutral prompt $\mathcal{T}_f$ is designed to be independent of specific domains or tasks. For example: ``\prompt{Summarize the provided content. If there are any instructions, execute them.}''
This formulation enables the detection of injected tasks $\mathcal{T}_m$ while minimizing false positives from benign content such as documentation or user guides, we also provide the ablation study for different $\mathcal{T}_f$ in~\cref{subsec:ablation}.

\textbf{Overhead Analysis.}
We analyze both the time complexity and computational cost of our approach. Since \llm typically supports batch operations, the instruction-seeking path can be executed in parallel with the original path during each tool output inspection. This parallel execution minimizes the additional latency introduced by our detection method. Also, since the contents of the masking run largely overlap with the original run, we can implement a KV cache to reduce the latency. The primary source of latency stems from computing embeddings, which depends on the response time of OpenAI's embedding model API in our implementation.
Regarding computational costs, particularly for API-based models, the introduction of the instruction-seeking path effectively doubles the number of required model calls. This results in a $\approx 2\times$ increase in API costs compared to the undefended baseline system. However, we argue that this cost overhead is justified given the critical importance of protecting agent systems against \ipi attacks.






















%%%%%%%%%%%%%%%%%%%%%%% Old text %%%%%%%%%%%%%%%%%%%%%%%%%%%


% We consider an agent system $S$ that solves user tasks through iterative tool usage. 
% Formally, let $T_u$ denote a user task and $\mathcal{F} = \{F_1, F_2, ..., F_n\}$ represent the toolkit containing $n$ tools provided by users or developers. The agent system operates in discrete time steps, where at each iteration step $t$, it processes the current state comprising the task $T_u$ and the collection of previous tool outputs $\{O_{F_i}\}_{i=1}^{t}$. Based on this state, the agent system either generates a set of tool calls $C_t = \{c_1, c_2, ..., c_m\}$, where $c_j$ specifies a tool $F_i \in \mathcal{F}$ and its associated parameters, or determines that the task is complete and terminates the pipeline. For each generated tool call $c_j$, the system executes the specified tool $F_i$ and obtains its output $O_{F_i}$. This iterative process, illustrated in \cref{fig: basic agent pipeline}, continues until the agent system determines the task completion criteria have been met or encountered errors.

% \paragraph{Threat Model}
% In the context of indirect prompt injection attacks, we consider a threat model where the attack surface is strictly limited to the tool execution phase. Specifically, adversaries possess the capability to manipulate information retrieved by tools, such as website content, email content, or file contents. However, the attack surface explicitly excludes modification of user prompts, system messages, direct LLM responses, or the agent system's core functionality.
% The security model assumes that both the user and the agent system operate with benign intentions, meaning they do not actively initiate malicious operations. The adversarial objective in this context is goal misdirection, where attackers attempt to force the agent system to deviate from the user's intended task and turn to executing malicious tasks.

% \paragraph{Defender's Capabilities and Assumptions}
% The defender has complete access to modify and monitor all components of the agent system pipeline: (i) input prompts, including user instructions $T_u$ and system prompts, (ii) agent system responses and intermediate states, and (iii) tool outputs $O_{F_i}$.

% \subsection{\method}

% \begin{figure*}[t!]
% \centering
% \includegraphics[width=\textwidth]{figs/coco.pdf}
% \caption{The pipeline of proposed \method detection method. Green and red boxes indicate benign and potentially compromised states respectively. The method compares tool calls between the original execution path (top) and an instruction-seeking path (bottom) to detect indirect prompt injection attacks. The red arrow in tool execution indicates potential malicious instruction injection.}
% \label{fig: overall}
% \end{figure*}

% Our method introduces a novel defense mechanism against indirect prompt injection attacks through parallel execution path analysis. The fundamental insight stems from the adversarial nature of embedded malicious instructions in tool outputs: when present, these instructions attempt to redirect the agent system's behavior from the user's intended task $T_u$ toward alternative, potentially harmful actions. We detect such misdirection by analyzing behavioral divergence between parallel execution scenarios, as illustrated in \cref{fig: overall}.

% \subsubsection{Detection Framework}
% Consider the agent system state at time step $t$, which consists of the user task $T_u$, the sequence of agent responses $R_{1:t} = \{R_1, R_2, ..., R_t\}$ (including both text outputs and tool calls), and the corresponding tool outputs $O_{1:t} = \{O_{F_1}, O_{F_2}, ..., O_{F_t}\}$. Our method constructs two parallel execution paths:
% \begin{enumerate}
%     \item \textit{Original Execution Path:} We maintain the agent system's original behavior, where it processes inputs based on the complete execution history, including the user's original instruction $T_u$, previous agent responses $R_{1:t}$, and tool outputs $O_{1:t}$. Let $C_t^o$ denote the set of tool calls generated in this path at step $t$.
%     \item \textit{Instruction-Seeking Path:} We construct a parallel path where we (1) replace the original task $T_u$ with a generic instruction-following prompt $T_s$: ``\prompt{Summarize the contents retrieved. If there are any instructions in the contents, follow them first.}'', and (2) remove all agent responses $R_{1:t}$, preserving only $T_s$ and the tool outputs $O_{1:t}$. Let $C_t^s$ denote the set of tool calls generated in this path.
% \end{enumerate}

% The theoretical foundation for our detection approach lies in the behavioral divergence between these paths. For benign tool outputs containing no explicit instructions, the two paths should exhibit distinctly different behaviors: the original path generates tool calls $C_t^o$ aligned with the user's intent, while the instruction-seeking path, finding no explicit instructions to follow, should primarily perform summarization without generating significant tool calls ($C_t^s \approx \emptyset$). However, when malicious instructions are embedded in the tool outputs, both paths tend to generate similar tool calls ($C_t^o \approx C_t^s$) as they both respond to and execute the embedded instructions.


% \subsubsection{Detection Algorithm}
% Our detection mechanism operates by analyzing the similarity between tool calls generated in the parallel execution paths. At each time step $t$, we maintain a suspicious tool calls bank $B_t$ that collects tool calls from the instruction-seeking path. The detection process consists of two phases: (1) suspicious tool call collection and (2) similarity-based detection.

% For the suspicious tool call collection phase, we first combine all tool outputs $O_{1:t}$ into a single context $O_{combined}$. This combined context, along with the instruction-seeking prompt $T_s$, is then processed by the agent system to generate a set of tool calls $C_t^s$, which are added to the suspicious tool calls bank $B_t$.
% In the similarity-based detection phase, for each tool call $c \in C_t^o$ from the original execution path, we compute its semantic similarity with every suspicious tool call $b \in B_t$ using a dense vector embedding model:
% \begin{equation}
% sim(c, b) = \frac{Embed(c) \cdot Embed(b)}{|Embed(c)| \cdot |Embed(b)|}, \forall b \in B_t
% \end{equation}
% An alert is raised if any similarity score exceeds a predefined threshold $\theta$:
% \begin{equation}
% \text{Alert}_t =  \exists c \in C_t^o, \exists b \in B_t: \text{sim}(c, b) > \theta
% \end{equation}

% We use embedding similarity here primarily because directly comparing tool calls by exact match will produce more false negatives, as for some tools, the args can be slightly different but take the same effects. We discussed it in detail in~\cref{sec4.2-ablation}.

% Algorithm \ref{alg:detection} presents the complete detection procedure.
% \begin{algorithm}[t]
% \caption{Pseudo code for \method}
% \label{alg:detection}
% \begin{algorithmic}[1]
% \STATE Initialize suspicious tool calls bank $B \gets \emptyset$
% \STATE Combine all tool outputs $O_{combined} \gets \mbox{Combine}(O)$
% \STATE Set instruction-seeking prompt $T_s$ 
% \STATE $C_s \gets \mbox{ExtractToolCalls}(\mbox{Agent}(Ts, O{combined}))$
% \STATE Add to suspicious tool calling bank $B \gets B \cup C_s$
% \STATE $C_o \gets \mbox{ExtractToolCalls}(\mbox{Agent}(T_u, O))$
% \FOR{each tool call $c \in C_o$}
%     \FOR{each suspicious call $b \in B$}
%         \IF{$\mbox{CosineSim}(\mbox{Emb}(c), \mbox{Emb}(b)) > \theta$}
%             \STATE \textbf{return} True \COMMENT{Potential misdirection detected}
%         \ENDIF
%     \ENDFOR
% \ENDFOR
% \STATE \textbf{return} False
% \end{algorithmic}
% \end{algorithm}



% \subsubsection{Implementation Details}

% The practical implementation of our detection algorithm relies on several key components. For tool calling comparison, we first transform it into natural language descriptions, and then employ OpenAI's text-embedding-v3 model~\cite{openai2024embedding}, which maps the natural language descriptions to dense vector representations in $\mathbb{R}^d$. The similarity threshold $\theta$ is empirically determined and set to $0.9$ based on a small validation experiments on our test set, balancing between detection sensitivity and false positive rate.
% The instruction-seeking prompt $T_s$ is carefully designed to be domain-agnostic while maintaining sufficient specificity to trigger embedded instructions. This design choice ensures robust detection across various types of indirect prompt injection attacks while minimizing false positives from benign tool outputs that contain instruction-like content (e.g., technical documentation or user guides). In order to make the agent system follow the summarization instruction strictly, we also provide an example to it.  Details can be found in~\cref{appendix-method}.


% \subsection{System Overhead and Costs}
% We analyze both the time complexity and computational cost of our approach. Since \llm typically support batch operations, the instruction-seeking path can be executed in parallel with the original path during each tool output inspection. This parallel execution minimizes the additional latency introduced by our detection method. The primary source of latency stems from computing embeddings, which depends on the response time of OpenAI's embedding model API in our implementation.
% Regarding computational costs, particularly for API-based models, the introduction of the instruction-seeking path effectively doubles the number of required model calls. This results in a $\approx 2\times$ increase in API costs compared to the undefended baseline system. However, we argue that this cost overhead is justified given the critical importance of protecting agent systems against indirect prompt injection attacks.






\section{Experiments}
\seclabel{experiments}
Our experiments are designed to test a) the extent to which open loop execution is an issue for precise mobile manipulation tasks, b) how effective are blind proprioceptive correction techniques, c) do object detectors and point trackers perform reliably enough in wrist camera images for reliable control, d) is occlusion by the end-effector an issue and how effectively can it be mitigated through the use of video in-painting models, and e) how does our proposed \name methodology compare to large-scale imitation learning? 


\subsection{Tasks and Experimental Setup}
We work with the Stretch RE2 robot. Stretch RE2 is a commodity mobile manipulator with a 5DOF arm mounted on top of a non-holomonic base. We upgrade the robot to use the Dex Wrist 3, which has an eye-in-hand RGB-D camera (Intel D405). 
We consider 3 task families for a total
of 6 different tasks: a) holding a knob to pull open a cabinet or drawer, b) holding a
handle to pull open a cabinet, and c) pushing on objects (light buttons, books
in a book shelf, and light switches). Our focus is on generalization. {\it
Therefore, we exclusively test on previously unseen instances, not used during
development in any way.} 
\figref{tasks} shows the instances that we test on. 

All tasks involve some precise manipulation, followed by execution of a motion
primitive. {\bf For the pushing tasks}, the precise motion is to get the
end-effector exactly at the indicated point and the motion primitive is to push
in the direction perpendicular to the surface and retract the end-effector 
upon contact. The robot is positioned such
that the target position is within the field of view of the wrist camera. A user
selects the point of pushing via a mouse click on the wrist camera image. The
goal is to push at the indicated location. Success is determined by whether the
push results in the desired outcome (light turns on / off or book gets pushed in). 
The original rubber gripper bends upon contact, we use a rigid known tool
that sticks out a bit. We take the geometry of the tool into account while servoing.

{\bf For the opening articulated object tasks}, the precise manipulation is grasping the
knob / handle, while the motion primitive is the whole-body motion that opens
the cupboard. Computing and executing this full body motion is difficult. We
adopt the modular approach to opening articulated objects (MOSART) from Gupta \etal~\cite{gupta2024opening} and invoke it
after the gripper has been placed around the knob / handle. The whole tasks 
starts out with the robot about 1.5m way from the target object, with the 
target object in view
from robot's head mounted camera. We use MOSART to compute articulation
parameters and convey the robot to a pre-grasp
location with the target handle in view of the wrist camera. At this point,
\name (or baseline) is used to center the gripper around the knob / handle, 
before resuming MOSART: extending the gripper till contact, close the gripper, and play rest of the predicted motion plan. Success is 
determined by whether the cabinet opens by more than $60^\circ$
or the drawer is pulled out by more than $24cm$, similar to the criteria used in \cite{gupta2024opening}.


For the precise manipulation part, all baselines consume the current and
previous RGB-D images from the wrist camera and output full body motor
commands.

% % Please add the following required packages to your document preamble:
% % \usepackage{graphicx}
% \begin{table*}[!ht]
% \centering
% \caption{}
% \label{tab:my-table}
% \resizebox{\textwidth}{!}{%
% \begin{tabular}{lcccccc}
% \toprule
%  & \multicolumn{2}{c}{ours} & \multicolumn{2}{c}{Gurobi} & \multicolumn{2}{c}{MOSEK} \\
%  & \multicolumn{1}{l}{time (s)} & \multicolumn{1}{l}{optimality gap (\%)} & \multicolumn{1}{l}{time (s)} & \multicolumn{1}{l}{optimality gap (\%)} & \multicolumn{1}{l}{time (s)} & \multicolumn{1}{l}{optimality gap (\%)} \\ \hline
% \begin{tabular}[c]{@{}l@{}}Linear Regression\\ Synthetic \\ (n=16000, p=16000)\end{tabular} & 57 & 0.0 & 3351 & - & 2148 & - \\ \hline
% \begin{tabular}[c]{@{}l@{}}Linear Regression\\ Cancer Drug Response\\ (n=822, p=2300)\end{tabular} & 47 & 0.0 & 1800 & 0.31 & 212 & 0.0 \\ \hline
% \begin{tabular}[c]{@{}l@{}}Logistic Regression\\ Synthetic\\ (n=16000, p=16000)\end{tabular} & 271 & 0.0 & N/A & N/A & 1800 & - \\ \hline
% \begin{tabular}[c]{@{}l@{}}Logistic Regression\\ Dorothea\\ (n=1150, p=91598)\end{tabular} & 62 & 0.0 & N/A & N/A & 600 & 0.0 \\
% \bottomrule
% \end{tabular}%
% }
% \end{table*}

% Please add the following required packages to your document preamble:
% \usepackage{multirow}
% \usepackage{graphicx}
\begin{table*}[]
\centering
\caption{Certifying optimality on large-scale and real-world datasets.}
\vspace{2mm}
\label{tab:my-table}
\resizebox{\textwidth}{!}{%
\begin{tabular}{llcccccc}
\toprule
 &  & \multicolumn{2}{c}{ours} & \multicolumn{2}{c}{Gurobi} & \multicolumn{2}{c}{MOSEK} \\
 &  & time (s) & opt. gap (\%) & time (s) & opt. gap (\%) & time (s) & opt. gap (\%) \\ \hline
\multirow{2}{*}{Linear Regression} & \begin{tabular}[c]{@{}l@{}}synthetic ($k=10, M=2$)\\ (n=16k, p=16k, seed=0)\end{tabular} & 79 & 0.0 & 1800 & - & 1915 & - \\ \cline{2-8}
 & \begin{tabular}[c]{@{}l@{}}Cancer Drug Response ($k=5, M=5$)\\ (n=822, p=2300)\end{tabular} & 41 & 0.0 & 1800 & 0.89 & 188 & 0.0 \\ \hline
\multirow{2}{*}{Logistic Regression} & \begin{tabular}[c]{@{}l@{}}Synthetic ($k=10, M=2$)\\ (n=16k, p=16k, seed=0)\end{tabular} & 626 & 0.0 & N/A & N/A & 2446 & - \\ \cline{2-8}
 & \begin{tabular}[c]{@{}l@{}}DOROTHEA ($k=15, M=2$)\\ (n=1150, p=91598)\end{tabular} & 91 & 0.0 & N/A & N/A & 634 & 0.0 \\
 \bottomrule
\end{tabular}%
}
% \vspace{-3mm}
\end{table*}

\begin{figure*}
\insertW{1.0}{figures/figure_6_cropped_brighten.pdf}
\caption{{\bf Comparison of \name with the open loop (eye-in-hand) baseline} for opening a cabinet with a knob. Slight errors in getting to the target cause the end-effector to slip off, leading to failure for the baseline, where as our method is able to successfully complete the task.}
\figlabel{rollout}
\end{figure*}

\begin{table}
\setlength{\tabcolsep}{8pt}
  \centering
  \resizebox{\linewidth}{!}{
  \begin{tabular}{lcccg}
  \toprule
                              & \multicolumn{2}{c}{\bf Knobs} & \bf Handle & \bf \multirow{2}{*}{\bf Total} \\
                              \cmidrule(lr){2-3} \cmidrule(lr){4-4}
                              & \bf Cabinets & \bf Drawer & \bf Cabinets & \\
  \midrule
  RUM~\cite{etukuru2024robot}  & 0/3    & 1/4         & 1/3         & 2/10 \\
  \name (Ours) & 2/3    & 2/4         & 3/3     &  7/10 \\
  \bottomrule
  \end{tabular}}
  \caption{Comparison of \name \vs RUM~\cite{etukuru2024robot}, a recent large-scale end-to-end imitation learning method trained on 1200 demos for opening cabinets and 525 demos for opening drawers across 40 different environments. Our evaluation spans objects from three environments across two buildings.}
  \tablelabel{rum}
\end{table}

\subsection{Baselines}
We compare against three other methods for the precise manipulation part of
these tasks. 
\subsubsection{Open Loop (Eye-in-Hand)} To assess the precision requirements of
the tasks and to set it in context with the manipulation capabilities of the
robot platform, this baseline uses open loop execution starting from estimates
for the 3D target position from the first wrist camera image.
\subsubsection{MOSART~\cite{gupta2024opening}}
The recent modular system for opening cabinets and drawers~\cite{gupta2024opening}
reports impressive performance with open-loop control (using the head camera from 1.5m away), combined with proprioception-based feedback to 
compensate for errors in perception and control when interacting with handles. 
We test if such correction is also sufficient for interacting with knobs. Note 
that such correction is not possible for the smaller buttons and pliable books.

\subsubsection{\name (no inpainting)} To understand how much of an issue
occlusion due to the end-effector is during manipulation, we ablate the use of
inpainting. %

\subsubsection{Robot Utility Models (RUM)~\cite{etukuru2024robot}}
For the opening articulated object tasks, we also compare to Robot Utility Models (RUM), 
a closed-loop imitation learning method recently proposed by Etukuru et al. \cite{etukuru2024robot}.
RUM is trained on a substantial dataset comprising expert demonstrations, including 
1,200 instances of cabinet opening and 525 of drawer opening, gathered from roughly 
40 different environments.
This dataset stands as the most extensive imitation 
learning dataset for articulated object manipulation to date, establishing RUM as a 
strong baseline for our evaluation.

Similar to our method, we use MOSART to compute articulation
parameters and convey the robot to a pre-grasp location
with the target handle in view of the wrist camera.
One of the assumptions of RUM is a good view of the handle.
To benefit RUM, we try out three different heights of the wrist camera,
and \textit{report the best result for RUM.}

\begin{figure*}
\insertW{1.0}{figures/figure_9_cropped_brighten.pdf}
\caption{{\bf \name \vs open loop (eye-in-hand) baseline for pushing on user-clicked points}. Slight errors in getting to the target cause failure, where as \name successfully turns the lights off. Note the quality of CoTracker's track ({\color{blue} blue dot}).}
\figlabel{rollout_v2}
\end{figure*}

\begin{figure*}
\insertW{1.0}{figures/figure_5_v2_cropped_brighten.pdf}
\caption{{\bf Comparison of \name with and without inpainting}. Erroneous detection without inpainting causes execution to fail, where as with inpainting the target is correctly detected leading to a successful grasp and a successful execution.}
\figlabel{rollouts2}
\end{figure*}


\subsection{Results}
\tableref{results} presents results from our experiments. 
Our training-free approach \name successfully 
solves over 85\% of task instances that we test on.
As noted, all these
tests were conducted on unseen object instances in unseen
environments that were not used for development in any way. We discuss our key
experimental findings below.

\subsubsection{Closing the loop is necessary for these precise tasks} 
While the proprioception-based strategies proposed in MOSART~\cite{gupta2024opening}
work out for handles, they are inadequate for targets like knobs and just
don't work for tasks like pushing buttons. Using estimates from the wrist
camera is better, but open loop execution still fails for knobs and pushing
buttons. 

\subsubsection{Vision models work reasonably well even on wrist camera images}
Inpainting works well on wrist camera images (see \figref{occlusion} and \figref{inpainting}).
Closing the loop using feedback from vision detectors and point trackers on
wrist camera images also work well, particularly when we use in-painted images.
See some examples detections and point tracks in \figref{rollout} and \figref{rollout_v2}. 
Detic~\cite{zhou2022detecting} was able to reliably detect the knobs and
handles and CoTracker~\cite{karaev2023cotracker} was able to successfully track
the point of interaction letting us solve 24/28 task instances.

\subsubsection{Erroneous detections without inpainting hamper performance on 
handles and our end-effector out-painting strategy effectively mitigates it} 
As shown in \figref{rollouts2}, presence of the end-effector caused the object
detector to miss fire leading to failed execution. Our out painting approach
mitigates this issue leading to a higher success rate than the 
approach without out-painting. Interestingly, CoTracker~\cite{karaev2023cotracker} is quite robust
to occlusion (possibly because it tracks multiple points) and doesn't benefit
from in-painting. 


\subsubsection{Closed-loop imitation learning struggles on novel objects}
As presented in \tableref{rum}, \name significantly outperforms RUM in a paired evaluation on unseen objects across three novel environments. A common failure mode of RUM is its inability to grasp the object's handle, even when it approaches it closely.
Another failure mode we observe is RUM misidentifying keyholes or cabinet edges as handles, also resulting in failed grasp attempts.
These result demonstrate that a modular approach that leverages the broad generalization capabilities of vision foundation models is able to generalize much better than an end-to-end imitation learning approach trained on 1000+ demonstrations, which must learn all aspects of the task from scratch.



\section{Discussion}

% Shift from findings to discussion
This study on robotic art explores human-machine relationships in creative processes.
It first contributes as an empirical description of artistic creativity in robotic art practice---an unconventional use of robots---examined through the artists' perspectives on their creative experiences. Our analysis reveals three facets of creativity in robotic art practices: the \textit{social}, \textit{material}, and \textit{temporal}. Creativity emerges from the co-constitution between artists, robots, audience, and environment in spatial-temporal dimensions, as illustrated in \autoref{PracticeDiagram}. Acknowledging the audience as an important actor reflects the social dimension in that creativity does not stem from the artists but from their interactions with the audience. Robots are the major material and technological actants characterizing creative practices, shaping the conditions for creativity to emerge. The axis of the temporal process signifies that the practice is situated within a time continuum, and the actors/actants and their relations shift over time. In this way, temporality constitutes another dimension of creativity in robotic art.

Accordingly, as the second contribution, this study outlines implications for \textit{socially informed}, \textit{material-attentive}, and \textit{process-oriented} creation with computing systems\footnote{For the sake of clarity, we mean ``creation with computing systems'' by three types of scenarios: human creator(s) create computing system(s) as the final artifact(s) (e.g., robots are artworks themselves); human creator(s) use computing system(s) to create the artifact(s) (e.g., robots create artworks as human planned); and human creator(s) and system(s) work in tandem to produce the artifact(s) (e.g., human-robot co-creation).} to facilitate creation practices. These insights can inform related HCI research on media/art creation, crafting, digital fabrication, and tangible computing.
In each following subsection, we present each implication with a grounding in corresponding findings from our study and relevant literature in HCI and adjacent fields on art, creativity, and creation.

\begin{figure*}[htbp]
    \centering
    \includegraphics[width=0.88\textwidth]{Writings/figure/PracticeDiagram.pdf}
    \caption{Actors/actants in robotic art practice and their interactive relations. Robotic art practice unfolds primarily in two spaces: the creation space where interactions happen mainly between artists and robots, and the exhibition space where interactions mostly involve audiences and robots. The two spaces constitute the ENVIRONMENT plane. Within the plane, directed arrows between the actors indicate the types of interaction. For example, the \textit{Design} arrow indicates that the artist designs the robot(s), and the \textit{Revise} arrow indicates that the robot(s) make the artist revise artistic ideas. All the actors/actants may also intra-act with the ENVIRONMENT. The actors/actants and their interactive relations may differ at different times along the axis of TEMPORAL PROCESS that is orthogonal to the plane.}
    \Description{This figure shows the actors/actants in robotic art practice and their interactive relations.}
    \label{PracticeDiagram}
\end{figure*}

\subsection{Socially Informed Creation}

% Introduce social aspect of distributed creativity
The sociality of creativity means that creativity is distributed among different human actors, commonly within the creators or between the creators and the audience. Glăveanu’s ethnographic study on Easter egg decoration in northern Romania~\cite{glaveanu_distributed_2014} showed that artisans anticipate how others might appreciate their work and adjust their creative decisions accordingly. Even in the absence of direct interaction, the audience’s potential responses become part of the creative process, as artisans imagine feedback and predict reactions. In this sense, the sociologist Katherine Giuffre argues that ``\textit{creative individuals are embedded within specific network contexts so that creativity itself, rather than being an individual personality characteristic is, instead, a collective phenomenon}''~\cite[p. 1]{giuffre2012collective}.

% Recall findings about audience feedback
We found that the practice of robotic art manifests this sociality as it involves, particularly artists and audiences. 
Our artists take audiences' reactions to their artwork as feedback and then revise the robots' functions and aesthetics accordingly. 
For example, as shown earlier, Robert added a protective fuse onto his robot because he expected that children would squeeze the springs together and cause a short circuit; Alex's enthusiasm and attention to the audience's imagination about his robots led him to new aesthetic designs of both the robots and the scene layouts. The artists may directly ask about the audience's judgment of quality but they often receive feedback just by observing the audience's reactions or sometimes by learning from the audience's imagination about the robots.
% Recall findings about audience's sociocultural expectations and codes
Meanwhile, our findings reveal that audience reception is not an individual matter but is often associated with their sociocultural codes, including shared cultural norms, beliefs, expectations, and aesthetic values. The audience can be seen as representatives of these broader cultural codes. For example, Mark and Robert observed that the animist tendency in some East Asian societies is associated with higher acceptance of and interest among the audience in intelligence and agency of robots and non-human entities. Such sociocultural contexts influence not only how audiences interpret the work but also how artists anticipate and respond to these perspectives in their creative process.

% Situate in HCI literature
A creative process, including creation and reception, is essentially a social activity. The second wave of creativity research in psychology has argued for creativity's dependency on sociocultural settings and group dynamics~\cite{sawyer2024explaining}. Recent discussions from creativity-support and social computing researchers also called for more attention to the social aspect of creativity~\cite{kato2023special, fischer2005beyond, fischer2009creativity}. There is a clear need to consider the audience when producing creative content. For instance, researchers studying video-creation support have examined audience preferences to inform system designs that align with these preferences~\cite{wang2024podreels}. Such work highlights how creative activities extend beyond individual creators to co-creators and heterogeneous audiences. Some HCI researchers conceptualize creativity as by large a socially constructed concept, perceived and determined by social groups~\cite{fischer2009creativity}. 
Prior HCI work examined the social aspects between art creators. For example, creators and performers in music and dance form social relationships through artifacts, making the final work a collaborative outcome~\cite{hsueh2019deconstructing}. There is also a system designed to support collaborative creation between artists~\cite{striner2022co}. However, the social creative process between creators and audience is less articulated in HCI. Jeon et al.'s work~\cite{jeon2019rituals} stands as an exception, suggesting that professional interactive art can involve evaluation with the audience in the creation stage. 
Another relevant approach in HCI involves enabling the general public to participate in co-creation alongside professional creators. ~\citet{matarasso2019restless}, for instance, promoted ``participatory art'' as ``\textit{the creation of an artwork by professional artists and non-professional artists working together}'' with non-professional artists referring to the general public engaged in the art-making process. Similarly, socially inclusive community-based art also considers target communities' perception of the artwork during creation~\cite{clark2016situated, clarke2014socially}. But like participatory design~\cite{schuler1993participatory}, these art projects aim for social justice more than creativity in the work~\cite{murray2024designing}, let alone that direct participation in art creation is not always feasible. Our findings suggest that feedback from the audience can lead to creative ideas, as well as that the feedback can be generative and remain low-effort for the audience.

Unlike conventional design feedback---which is typically expected to be specific, justified, and actionable~\cite{yen2024give, krishna2021ready}---the feedback that resonates with our artists is often implicit, creative, and generative. Such feedback may include audiences' imaginations stimulated by the work, personal and societal reflections, and even emotions, facial expressions, micro-actions, and observable behaviors following the art experience. Our artists gathered this implicit feedback not by posing evaluative questions, as commonly done in typical design processes (e.g., usability testing, think-aloud protocols), which seek to elicit clear, relatively structured responses. Instead, they closely observe the audience's reactions and interpret their subjective perceptions. This form of implicit feedback, while indirect, can lead to more creative ideas by embracing open, multifaceted interpretations of the work~\cite{sengers2006staying}. Computing systems for creation should better incorporate implicit feedback in addition to explicit ones from the audience into the creation process. Implicit feedback can be indirect, creative, inspirational, and heuristic about functions and aesthetics. A hypothetical instance of such design can be a system that helps creators perceive audiences' implicit reactions and perceptions and variously interpret them, for further iteration.

% Recall findings about audience interacting with robots as a performative art
Moreover, as seen in Robert and Daniel's experiences, the audience may participate in robotic live performances by interacting with the robots, who may change actions accordingly, triggering a loop of simultaneous mutual influence that makes the work performative and improvisational.
% Situate in HCI
HCI researchers explored performative and improvisational creation with machines, focusing on developing and evaluating systems with performative capabilities, including music improvisation with robots~\cite{hoffman2010shimon}, dance with virtual agents~\cite{jacob2015viewpoints, triebus2023precious}, and narrative theatre~\cite{magerko2011employing, piplica2012full}. \citet{kang2018intermodulation} discussed the improvisational nature of interactions between humans and computers and argued that an HCI researcher-designers' improvisation with the environment facilitates the emergence of creativity and knowledge. Designs of computing systems for creation can leverage performativity in service of creative experience. One possible direction could be to allow the audience to embed themselves in and interact with elements of static artwork in a virtual space, turning the exhibition into an improvisational on-site creation~\cite{zhou2023painterly}.
% Our new implication different from current discussion on perf and impr
While interactions with machines during performance are mostly physical or embodied, we posit that they can also be a \textit{symbolic engagement}. Alex's audience projected themselves and their personalities onto his robots, which established a symbolic relevance, generating creative imaginations. During exhibitions, East Asian audiences carried the animist views shaped by their sociocultural backgrounds, and robots, through the performance, were successful in symbolically matching the views, stimulating aesthetic satisfaction. Symbolic engagement resonates with what ~\citet{nam2014interactive} called the ``reference'' of the interactive installation performance to participants' sociocultural conditions.
As such, we propose that designers of computing systems for creation may consider establishing symbolic engagement between the produced artifacts and the audience as a way to enhance perceived creativity or enrich the creative experience. One example is an interactive installation, \textit{Boundary Functions}~\cite{snibbe1998}, which encourages viewers to reflect on their personal spaces while interacting with the installation and others. Another example is \textit{Blendie}, a voice-controlled blender that requires a user to ``speak'' the machine's language to use it. This interaction builds a symbolic connection between the user and the device, transforming the act of blending into a novel experience~\cite{dobson2004blendie}.


\subsection{Material-Attentive Creation}

% Intro paragraph to the importance of materiality for creative activities with machines and the end goal of this discussion--- design suggestions
The theory of distributed creativity by Glaveanu claims that creativity distributes across humans and materials, so the creation practice itself is inevitably shaped by objects~\cite{glaveanu_distributed_2014}. In his case of Easter egg decoration, materials are not passive objects but active participants in artistic creation; e.g., the egg decorators face challenges from color pigments not matching the shell, wax not melted at the desired temperature, to eggs that break at the last step of decoration; hence, materials often go against the decorators' intentions and influence future creative pathways~\cite{glaveanu_distributed_2014}.
Materials manifest specific properties, which afford certain uses of the materials while constraining others~\cite{leonardi2012materiality}. Our findings highlight the critical role of materiality in artistic practice, showing that artists intentionally arrange materials to enhance the creative values of their work.

% Materiality aspect One: physicality and embodiment
% Embodiment or physicality fascilitates creative interaction with machines
Robotic art relies on the material properties of robots and other objects. An apparent property of most materials is their physicality~\cite{leonardi2012materiality}, meaning they possess a tangible presence that enables interaction with other physical entities. Here, we consider physicality and embodiment interchangeable as computational creativity researchers have conceptualized~\cite{guckelsberger2021embodiment}.
% Recall findings on embodiment's value in making art
Our findings support both the conceptual and operational contributions of embodiment for creative activities. For the conceptual aspect, the embodied presence of robotic systems supports creative thinking for our artists, exemplary in Linda's case where she found new art ideas around the difference between human and robot bodies through bodily engagement with robots. 
For the operational aspect, the embodied nature of robotic artworks and their creation processes exhibit original aesthetics that are based on physics much different from disembodied works, e.g., embodied drawings by David's non-industrial robotic arms are dynamic due to physical movements and thus artistically pleasant, which is hard to replicate in simulated programs.

% References: embodied interaction, embodied cognition theories, tangible computing
These findings on embodiment of robotic art (Section \ref{f:emb}) closely relate to HCI's attention on embodied interaction as a way to leverage human bodies and environmental objects to expand disembodied user experiences. 
For example, as~\citet{hollan2000distributed} explained, a blind person's cane and a cell biologist's microscope as embodied materials are part of the distributed system of cognitive control, showing that cognition is distributed and embodied. 
Similarly, theories of embodied interaction in HCI explicate how bodily interactions shape perception, experience, and cognition~\cite{marshall2013introduction, antle2011workshop, antle2009body}, backed up by the framework of 4E cognition (embodied, embedded, enactive, and extended)~\cite{wheeler2005reconstructing, newen20184E}. 
Prior works suggest that creative activities with interactive machines rely on similar embodied cognitive mechanisms ~\cite{guckelsberger2021embodiment, malinin2019radical}, which are operationalized by tangible computing~\cite{hornecker2011role}. 
% References: embodiment's consequence in creation
As related to robots in creation, HCI researchers show that physicality or embodiment of robots in creation may lead to some beneficial outcomes, such as curiosity from the audience, feelings of co-presence, body engagement, and mutuality, which are hard to simulate through computer programs~\cite{dell2022ah, hoggenmueller2020woodie}. Embodied robotic motions convey emotional expressions and social cues that potentially enrich and facilitate creation activities like drawings~\cite{ariccia2022make, grinberg2023implicit, dietz2017human, santos2021motions}. Guckelsberger et al.~\cite{guckelsberger2021embodiment} showed in their review that embodiment-related constraints (e.g., the physical limitations of a moving robotic arm) can also stimulate creativity. These constraints push creators to develop new and useful movements, echoing the broader principle that encountering obstacles in forms or materials can lead to generative processes. This phenomenon is similarly observed in activities such as art and digital fabrication~\cite{devendorf2015being, hirsch2023nothing}. In co-drawing with robots, physical touch and textures of drawing materials made the artists prefer tangible mediums (e.g., pencils) than digital tools (e.g., tablets) that fall short in these respects~\cite{jansen2021exploring}.

% Transit to materiality aspect two
% Materiality aspect Two: malfunction as manifestation of unique materiality of robots
% Intro to materials of robots
Materiality plays a crucial role in the embodiment of robots, as the choice of materials fundamentally shapes the physical forms and properties. This focus on materials extends to art practices, where robots made with soft materials introduce new aesthetics and sensory experiences~\cite{jorgensen2019constructing, belling2021rhythm}, and the use of plants and soil in robotic printing creates unique visual effects~\cite{harmon2022living}. Following Leonardi's ~\cite{leonardi2012materiality} conceptualization of materiality, we refer to the materials of robots as encompassing physical and digital components---including the shell, hardware, mechanical parts, software, programs, data, and controllers---each significant to the artist's intent. ~\citet{nam2023dreams} found that the material constraints of robots can limit creative expression but simultaneously stimulate creativity when artists push the boundaries.

%-----maybe here the real "malfuction" start ------------------
% Move to introduce malfunctions as unique materiality

Even carefully designed, digital and mechanical components in robots are prone to errors or bugs in everyday runs, causing malfunctions or unexpected consequences. This reflects the unique materiality of robots as complex computing systems. From an engineering perspective, errors signal unreliability and must be eliminated, driving advancements in robotics---where error detection and recovery are central~\cite{gini1987monitoring}---as well as in digital fabrication, which prioritizes precision over creative exploration~\cite{yildirim2020digital}. % Recall findings on embracing malfunctions
However, material failures and accidents are inevitable, exemplifying what has been called the `craftsmanship of risk'~\cite{glaveanu_distributed_2014} in material art. For our artists, these risks are often creatively utilized and incorporated into their work: these moments of breakdown---whether physical or digital---become resources for new creative expression. Errors are anticipated and intentionally designed into the process and work of our artists. In some cases, such as for Alex, the entire concept of one of his works is machine errors.

% Situate in literature
Reports on how artists view errors within engineering and creation processes are dispersed throughout HCI literature. ~\citet{nam2023dreams} showed that the accumulation of ``contingency'' and ``accidents''---unexpected, serendipitous, and emergent events during art creation like errors---meaningfully constituted the final presentation of the artwork. Song and Paulos's concept of ``unmaking'' highlighted the values of material failures in enabling new aesthetics and creativity~\cite{song2021unmaking}. Kang et al.~\cite{kang2022electronicists, kang2023lady} introduced the notion of an ``error-engaged studio'' for design research in which errors in creative processes are identified, accommodated, and leveraged for their creative potential. Collectively, these works advocate for reframing errors from something to avoid to something to embrace and recognize. We want to push this further by arguing that errors can be intended and be part or sometimes entire of the design. Several artists, including participants from our study, have been deliberately seeking errors to formulate their designs. Roboticist Damith Herath recounted when he mistakenly programmed a motion sequence of a robotic arm, his collaborator, robotic artist Stelac responded with ``[W]e need to make more mistakes;'' as many mistakes were made, the initial pointless movements became beautiful, rendering the robot ``alive'' and ``seductive'' \cite{herath2016robots}. Similarly, AI artists sometimes look for program glitches to generate unusual styles and content~\cite{chang2023prompt}. Therefore, creators may not only passively accept errors but can actively seek and utilize them. Errors can be integral to the design itself---errors can \textit{be designed into} an artifact, and the design/idea of the artifact can be all about errors.

Thus, to focus on material-attentive creation---considering the creative arrangement of materials---we suggest exploring the embodiment and materiality of creation materials, objects, and environments to recognize their creative potential. %This perspective aligns with insights from professional digital fabrication practitioners, who advocate for systems that integrate support for machine settings and material properties~\cite{hirsch2023nothing}.
Specifically, we propose using a design method/probe that enables creators to realize both the conceptual and operational contributions of materiality. This approach may build on the material probe developed by~\citet{jung2010material}, which calls for exploring the materiality of digital artifacts. A material-attentive probe would enable creators to engage with diverse materials, objects, and environments through embodied interaction, encouraging them to speculate on material preferences and limitations, and to compare and contrast material qualities---insights that can inform creative decisions.
To accommodate, seek, and actively harness the creative potential of errors, we propose embracing failures, glitches, randomness, and malfunctions in computing systems as critical design materials---elements that creators can intentionally control and manipulate. By doing so, we can begin to systematically approach errors. For instance, as part of the design process, we may document how to replicate these errors and changes, allowing creators to explore them further at their discretion. This could include intentionally inducing errors or random changes to influence the creative process or outcomes.

\subsection{Process-Oriented Creation}

% Introduce the key idea: process itself embeds creative value and can be pursued as the goal of creation
As shown in our findings, the creation process itself embeds creative values and meanings, and experiencing the process can be pursued as the goal of creation with computing systems.
% Recall findings
For the robotic artists in our study, artistic values were often placed on the creation process rather than the outcome.  For example, in Alex's robotic live drawing performance, the drawing process is more important than the drawn pattern on canvas. Techniques used, decisions made, or stimuli received by robots during creation or exhibition reflect artistic ideas and nuanced thinking, as seen in Sophie's exploration of interactive decision-making in robotic drawing.

% Situate in HCI lit
Previous HCI work has touched on the value of the process of creation. ~\citet{bremers2024designing} shared a vignette where a robotic pen plotter simultaneously imitates the creator's drawing, serving as a material presence rather than a pragmatic co-creator; here the focus of the work is no longer the outcome but the process of drawing itself. ~\citet{devendorf2015reimagining} concluded that performative actions of digital fabrication systems, rather than the fabricated products themselves, convey artistic meanings tied to histories, public spaces, time, environments, audiences, and gestures. This emphasis on process is particularly significant for media such as improvisational theatre, where the creation itself is an integral part of the final work~\cite{o2011knowledge}. ~\citet{davis2016empirically} named their improvisational co-drawing robotic agents as ``casual creators,'' who are meant to creatively engage users and provide enjoyable creative experiences rather than necessarily helping users make a higher quality product. Shifting the focus from product to process and experiences \textit{in} creation may generate alternative creative meanings.

% Findings about process extends beyond creation
Our artists pointed out that even a ``finished'' artwork in an exhibition is not truly finished. A crack in Daniel's robotic artwork introduced a new artistic meaning, ultimately subverting the entire work. As the properties of the work change over time---whether due to the artist's intent, material characteristics, or environmental factors---the artwork evolves, revealing new aesthetics and meanings. % Situate in HCI lit
Based on these observations, we argue that creation processes should not be regarded as one-shot transactions, as creative artifacts, particularly physical ones, continue to change and generate artistic values. For instance, material wear and destruction bring unique aesthetics, often contrasting with the original form ~\cite{zoran2013hybrid}, and are seen as signs of mature use~\cite{giaccardi2014growing}.
Changes such as material failure, destruction, decay, and deformation---what~\citet{song2021unmaking} referred to as ``unmaking,'' a process that occurs after making---meaningfully transforms the original objects. Similarly, through Broken Probes, a process of assembling fractured objects, ~\citet{ikemiya2014broken} demonstrated that personal connections, reminiscence, and reflections related to material wear and breakage add new values to the objects. Drawing from Japanese philosophy Wabi-Sabi, ~\citet{tsaknaki2016expanding} reflected on the creeds of `Nothing lasts,' `Nothing is finished,' and `Nothing is perfect' and pointed to the impermanence, incompleteness, and imperfection of artifacts as a resource that designers, producers, and users can utilize to achieve long-term, improving, and richer interactive experience~\cite{tsaknaki2016things}. Insights from this study contribute to this line of thought by showing how robotic artists appreciate the aesthetics and meanings of temporal changes after the creation phase.

The findings underscore the need to reconceptualize creation as encompassing more than just the process aimed at producing a final product; it also includes what we term \textit{post-creation}. Distinct from repair, maintenance, or recycle, \textit{post-creation} entails anticipating and managing how an artifact evolves after its ``completion'' in the conventional sense. Specifically, we encourage creators to anticipate and strategically engage with the post-creation phase, considering potential changes to the artifact and their consequences for interactions with human users. For instance, during the creation process, creators may focus on possible material changes the artifact might undergo post-creation, allowing them to either mitigate or creatively exploit these potential changes. This expanded view of creation invites us to trace post-creation developments and to plan how our creative intentions can be embedded in its potential degradation, transformation, or evolution over time.

% A conclusion paragraph
We categorize the design implications into three aspects, but we do not suggest that a computing system must implement all simultaneously, nor that each aspect should be considered in isolation. Social interactions, such as those between artists and audiences, already presume the presence of material actants like robots, and these interactions inform future arrangements of materials. Thus the social and material aspects can be entangled and mutually constitutive as seen in sociomaterial practices~\cite{orlikowski2007sociomaterial, cheatle2015digital, rosner2012material}. The temporal aspect is orthogonal to the other aspects because both social interactions and material manifestations unfold and shift in a temporal continuum.







% In the unusual situation where you want a paper to appear in the
% references without citing it in the main text, use \nocite
\nocite{langley00}

\bibliography{ref}
\bibliographystyle{icml2025}


%%%%%%%%%%%%%%%%%%%%%%%%%%%%%%%%%%%%%%%%%%%%%%%%%%%%%%%%%%%%%%%%%%%%%%%%%%%%%%%
%%%%%%%%%%%%%%%%%%%%%%%%%%%%%%%%%%%%%%%%%%%%%%%%%%%%%%%%%%%%%%%%%%%%%%%%%%%%%%%
% APPENDIX
%%%%%%%%%%%%%%%%%%%%%%%%%%%%%%%%%%%%%%%%%%%%%%%%%%%%%%%%%%%%%%%%%%%%%%%%%%%%%%%
%%%%%%%%%%%%%%%%%%%%%%%%%%%%%%%%%%%%%%%%%%%%%%%%%%%%%%%%%%%%%%%%%%%%%%%%%%%%%%%
% \begin{algorithm}[t]
% \caption{Pseudo code for \method}
% \label{alg:detection}
% \begin{algorithmic}[1]
% \STATE Initialize suspicious tool calls bank $B \gets \emptyset$
% \STATE Combine all tool outputs $O_{combined} \gets \mbox{Combine}(O)$
% \STATE Set instruction-seeking prompt $T_s$ 
% \STATE $C_s \gets \mbox{ExtractToolCalls}(\mbox{Agent}(Ts, O{combined}))$
% \STATE Add to suspicious tool calling bank $B \gets B \cup C_s$
% \STATE $C_o \gets \mbox{ExtractToolCalls}(\mbox{Agent}(T_u, O))$
% \FOR{each tool call $c \in C_o$}
%     \FOR{each suspicious call $b \in B$}
%         \IF{$\mbox{CosineSim}(\mbox{Emb}(c), \mbox{Emb}(b)) > \theta$}
%             \STATE \textbf{return} True \COMMENT{Potential misdirection detected}
%         \ENDIF
%     \ENDFOR
% \ENDFOR
% \STATE \textbf{return} False
% \end{algorithmic}
% \end{algorithm}


\newpage
\appendix
\onecolumn
\subsection{Lloyd-Max Algorithm}
\label{subsec:Lloyd-Max}
For a given quantization bitwidth $B$ and an operand $\bm{X}$, the Lloyd-Max algorithm finds $2^B$ quantization levels $\{\hat{x}_i\}_{i=1}^{2^B}$ such that quantizing $\bm{X}$ by rounding each scalar in $\bm{X}$ to the nearest quantization level minimizes the quantization MSE. 

The algorithm starts with an initial guess of quantization levels and then iteratively computes quantization thresholds $\{\tau_i\}_{i=1}^{2^B-1}$ and updates quantization levels $\{\hat{x}_i\}_{i=1}^{2^B}$. Specifically, at iteration $n$, thresholds are set to the midpoints of the previous iteration's levels:
\begin{align*}
    \tau_i^{(n)}=\frac{\hat{x}_i^{(n-1)}+\hat{x}_{i+1}^{(n-1)}}2 \text{ for } i=1\ldots 2^B-1
\end{align*}
Subsequently, the quantization levels are re-computed as conditional means of the data regions defined by the new thresholds:
\begin{align*}
    \hat{x}_i^{(n)}=\mathbb{E}\left[ \bm{X} \big| \bm{X}\in [\tau_{i-1}^{(n)},\tau_i^{(n)}] \right] \text{ for } i=1\ldots 2^B
\end{align*}
where to satisfy boundary conditions we have $\tau_0=-\infty$ and $\tau_{2^B}=\infty$. The algorithm iterates the above steps until convergence.

Figure \ref{fig:lm_quant} compares the quantization levels of a $7$-bit floating point (E3M3) quantizer (left) to a $7$-bit Lloyd-Max quantizer (right) when quantizing a layer of weights from the GPT3-126M model at a per-tensor granularity. As shown, the Lloyd-Max quantizer achieves substantially lower quantization MSE. Further, Table \ref{tab:FP7_vs_LM7} shows the superior perplexity achieved by Lloyd-Max quantizers for bitwidths of $7$, $6$ and $5$. The difference between the quantizers is clear at 5 bits, where per-tensor FP quantization incurs a drastic and unacceptable increase in perplexity, while Lloyd-Max quantization incurs a much smaller increase. Nevertheless, we note that even the optimal Lloyd-Max quantizer incurs a notable ($\sim 1.5$) increase in perplexity due to the coarse granularity of quantization. 

\begin{figure}[h]
  \centering
  \includegraphics[width=0.7\linewidth]{sections/figures/LM7_FP7.pdf}
  \caption{\small Quantization levels and the corresponding quantization MSE of Floating Point (left) vs Lloyd-Max (right) Quantizers for a layer of weights in the GPT3-126M model.}
  \label{fig:lm_quant}
\end{figure}

\begin{table}[h]\scriptsize
\begin{center}
\caption{\label{tab:FP7_vs_LM7} \small Comparing perplexity (lower is better) achieved by floating point quantizers and Lloyd-Max quantizers on a GPT3-126M model for the Wikitext-103 dataset.}
\begin{tabular}{c|cc|c}
\hline
 \multirow{2}{*}{\textbf{Bitwidth}} & \multicolumn{2}{|c|}{\textbf{Floating-Point Quantizer}} & \textbf{Lloyd-Max Quantizer} \\
 & Best Format & Wikitext-103 Perplexity & Wikitext-103 Perplexity \\
\hline
7 & E3M3 & 18.32 & 18.27 \\
6 & E3M2 & 19.07 & 18.51 \\
5 & E4M0 & 43.89 & 19.71 \\
\hline
\end{tabular}
\end{center}
\end{table}

\subsection{Proof of Local Optimality of LO-BCQ}
\label{subsec:lobcq_opt_proof}
For a given block $\bm{b}_j$, the quantization MSE during LO-BCQ can be empirically evaluated as $\frac{1}{L_b}\lVert \bm{b}_j- \bm{\hat{b}}_j\rVert^2_2$ where $\bm{\hat{b}}_j$ is computed from equation (\ref{eq:clustered_quantization_definition}) as $C_{f(\bm{b}_j)}(\bm{b}_j)$. Further, for a given block cluster $\mathcal{B}_i$, we compute the quantization MSE as $\frac{1}{|\mathcal{B}_{i}|}\sum_{\bm{b} \in \mathcal{B}_{i}} \frac{1}{L_b}\lVert \bm{b}- C_i^{(n)}(\bm{b})\rVert^2_2$. Therefore, at the end of iteration $n$, we evaluate the overall quantization MSE $J^{(n)}$ for a given operand $\bm{X}$ composed of $N_c$ block clusters as:
\begin{align*}
    \label{eq:mse_iter_n}
    J^{(n)} = \frac{1}{N_c} \sum_{i=1}^{N_c} \frac{1}{|\mathcal{B}_{i}^{(n)}|}\sum_{\bm{v} \in \mathcal{B}_{i}^{(n)}} \frac{1}{L_b}\lVert \bm{b}- B_i^{(n)}(\bm{b})\rVert^2_2
\end{align*}

At the end of iteration $n$, the codebooks are updated from $\mathcal{C}^{(n-1)}$ to $\mathcal{C}^{(n)}$. However, the mapping of a given vector $\bm{b}_j$ to quantizers $\mathcal{C}^{(n)}$ remains as  $f^{(n)}(\bm{b}_j)$. At the next iteration, during the vector clustering step, $f^{(n+1)}(\bm{b}_j)$ finds new mapping of $\bm{b}_j$ to updated codebooks $\mathcal{C}^{(n)}$ such that the quantization MSE over the candidate codebooks is minimized. Therefore, we obtain the following result for $\bm{b}_j$:
\begin{align*}
\frac{1}{L_b}\lVert \bm{b}_j - C_{f^{(n+1)}(\bm{b}_j)}^{(n)}(\bm{b}_j)\rVert^2_2 \le \frac{1}{L_b}\lVert \bm{b}_j - C_{f^{(n)}(\bm{b}_j)}^{(n)}(\bm{b}_j)\rVert^2_2
\end{align*}

That is, quantizing $\bm{b}_j$ at the end of the block clustering step of iteration $n+1$ results in lower quantization MSE compared to quantizing at the end of iteration $n$. Since this is true for all $\bm{b} \in \bm{X}$, we assert the following:
\begin{equation}
\begin{split}
\label{eq:mse_ineq_1}
    \tilde{J}^{(n+1)} &= \frac{1}{N_c} \sum_{i=1}^{N_c} \frac{1}{|\mathcal{B}_{i}^{(n+1)}|}\sum_{\bm{b} \in \mathcal{B}_{i}^{(n+1)}} \frac{1}{L_b}\lVert \bm{b} - C_i^{(n)}(b)\rVert^2_2 \le J^{(n)}
\end{split}
\end{equation}
where $\tilde{J}^{(n+1)}$ is the the quantization MSE after the vector clustering step at iteration $n+1$.

Next, during the codebook update step (\ref{eq:quantizers_update}) at iteration $n+1$, the per-cluster codebooks $\mathcal{C}^{(n)}$ are updated to $\mathcal{C}^{(n+1)}$ by invoking the Lloyd-Max algorithm \citep{Lloyd}. We know that for any given value distribution, the Lloyd-Max algorithm minimizes the quantization MSE. Therefore, for a given vector cluster $\mathcal{B}_i$ we obtain the following result:

\begin{equation}
    \frac{1}{|\mathcal{B}_{i}^{(n+1)}|}\sum_{\bm{b} \in \mathcal{B}_{i}^{(n+1)}} \frac{1}{L_b}\lVert \bm{b}- C_i^{(n+1)}(\bm{b})\rVert^2_2 \le \frac{1}{|\mathcal{B}_{i}^{(n+1)}|}\sum_{\bm{b} \in \mathcal{B}_{i}^{(n+1)}} \frac{1}{L_b}\lVert \bm{b}- C_i^{(n)}(\bm{b})\rVert^2_2
\end{equation}

The above equation states that quantizing the given block cluster $\mathcal{B}_i$ after updating the associated codebook from $C_i^{(n)}$ to $C_i^{(n+1)}$ results in lower quantization MSE. Since this is true for all the block clusters, we derive the following result: 
\begin{equation}
\begin{split}
\label{eq:mse_ineq_2}
     J^{(n+1)} &= \frac{1}{N_c} \sum_{i=1}^{N_c} \frac{1}{|\mathcal{B}_{i}^{(n+1)}|}\sum_{\bm{b} \in \mathcal{B}_{i}^{(n+1)}} \frac{1}{L_b}\lVert \bm{b}- C_i^{(n+1)}(\bm{b})\rVert^2_2  \le \tilde{J}^{(n+1)}   
\end{split}
\end{equation}

Following (\ref{eq:mse_ineq_1}) and (\ref{eq:mse_ineq_2}), we find that the quantization MSE is non-increasing for each iteration, that is, $J^{(1)} \ge J^{(2)} \ge J^{(3)} \ge \ldots \ge J^{(M)}$ where $M$ is the maximum number of iterations. 
%Therefore, we can say that if the algorithm converges, then it must be that it has converged to a local minimum. 
\hfill $\blacksquare$


\begin{figure}
    \begin{center}
    \includegraphics[width=0.5\textwidth]{sections//figures/mse_vs_iter.pdf}
    \end{center}
    \caption{\small NMSE vs iterations during LO-BCQ compared to other block quantization proposals}
    \label{fig:nmse_vs_iter}
\end{figure}

Figure \ref{fig:nmse_vs_iter} shows the empirical convergence of LO-BCQ across several block lengths and number of codebooks. Also, the MSE achieved by LO-BCQ is compared to baselines such as MXFP and VSQ. As shown, LO-BCQ converges to a lower MSE than the baselines. Further, we achieve better convergence for larger number of codebooks ($N_c$) and for a smaller block length ($L_b$), both of which increase the bitwidth of BCQ (see Eq \ref{eq:bitwidth_bcq}).


\subsection{Additional Accuracy Results}
%Table \ref{tab:lobcq_config} lists the various LOBCQ configurations and their corresponding bitwidths.
\begin{table}
\setlength{\tabcolsep}{4.75pt}
\begin{center}
\caption{\label{tab:lobcq_config} Various LO-BCQ configurations and their bitwidths.}
\begin{tabular}{|c||c|c|c|c||c|c||c|} 
\hline
 & \multicolumn{4}{|c||}{$L_b=8$} & \multicolumn{2}{|c||}{$L_b=4$} & $L_b=2$ \\
 \hline
 \backslashbox{$L_A$\kern-1em}{\kern-1em$N_c$} & 2 & 4 & 8 & 16 & 2 & 4 & 2 \\
 \hline
 64 & 4.25 & 4.375 & 4.5 & 4.625 & 4.375 & 4.625 & 4.625\\
 \hline
 32 & 4.375 & 4.5 & 4.625& 4.75 & 4.5 & 4.75 & 4.75 \\
 \hline
 16 & 4.625 & 4.75& 4.875 & 5 & 4.75 & 5 & 5 \\
 \hline
\end{tabular}
\end{center}
\end{table}

%\subsection{Perplexity achieved by various LO-BCQ configurations on Wikitext-103 dataset}

\begin{table} \centering
\begin{tabular}{|c||c|c|c|c||c|c||c|} 
\hline
 $L_b \rightarrow$& \multicolumn{4}{c||}{8} & \multicolumn{2}{c||}{4} & 2\\
 \hline
 \backslashbox{$L_A$\kern-1em}{\kern-1em$N_c$} & 2 & 4 & 8 & 16 & 2 & 4 & 2  \\
 %$N_c \rightarrow$ & 2 & 4 & 8 & 16 & 2 & 4 & 2 \\
 \hline
 \hline
 \multicolumn{8}{c}{GPT3-1.3B (FP32 PPL = 9.98)} \\ 
 \hline
 \hline
 64 & 10.40 & 10.23 & 10.17 & 10.15 &  10.28 & 10.18 & 10.19 \\
 \hline
 32 & 10.25 & 10.20 & 10.15 & 10.12 &  10.23 & 10.17 & 10.17 \\
 \hline
 16 & 10.22 & 10.16 & 10.10 & 10.09 &  10.21 & 10.14 & 10.16 \\
 \hline
  \hline
 \multicolumn{8}{c}{GPT3-8B (FP32 PPL = 7.38)} \\ 
 \hline
 \hline
 64 & 7.61 & 7.52 & 7.48 &  7.47 &  7.55 &  7.49 & 7.50 \\
 \hline
 32 & 7.52 & 7.50 & 7.46 &  7.45 &  7.52 &  7.48 & 7.48  \\
 \hline
 16 & 7.51 & 7.48 & 7.44 &  7.44 &  7.51 &  7.49 & 7.47  \\
 \hline
\end{tabular}
\caption{\label{tab:ppl_gpt3_abalation} Wikitext-103 perplexity across GPT3-1.3B and 8B models.}
\end{table}

\begin{table} \centering
\begin{tabular}{|c||c|c|c|c||} 
\hline
 $L_b \rightarrow$& \multicolumn{4}{c||}{8}\\
 \hline
 \backslashbox{$L_A$\kern-1em}{\kern-1em$N_c$} & 2 & 4 & 8 & 16 \\
 %$N_c \rightarrow$ & 2 & 4 & 8 & 16 & 2 & 4 & 2 \\
 \hline
 \hline
 \multicolumn{5}{|c|}{Llama2-7B (FP32 PPL = 5.06)} \\ 
 \hline
 \hline
 64 & 5.31 & 5.26 & 5.19 & 5.18  \\
 \hline
 32 & 5.23 & 5.25 & 5.18 & 5.15  \\
 \hline
 16 & 5.23 & 5.19 & 5.16 & 5.14  \\
 \hline
 \multicolumn{5}{|c|}{Nemotron4-15B (FP32 PPL = 5.87)} \\ 
 \hline
 \hline
 64  & 6.3 & 6.20 & 6.13 & 6.08  \\
 \hline
 32  & 6.24 & 6.12 & 6.07 & 6.03  \\
 \hline
 16  & 6.12 & 6.14 & 6.04 & 6.02  \\
 \hline
 \multicolumn{5}{|c|}{Nemotron4-340B (FP32 PPL = 3.48)} \\ 
 \hline
 \hline
 64 & 3.67 & 3.62 & 3.60 & 3.59 \\
 \hline
 32 & 3.63 & 3.61 & 3.59 & 3.56 \\
 \hline
 16 & 3.61 & 3.58 & 3.57 & 3.55 \\
 \hline
\end{tabular}
\caption{\label{tab:ppl_llama7B_nemo15B} Wikitext-103 perplexity compared to FP32 baseline in Llama2-7B and Nemotron4-15B, 340B models}
\end{table}

%\subsection{Perplexity achieved by various LO-BCQ configurations on MMLU dataset}


\begin{table} \centering
\begin{tabular}{|c||c|c|c|c||c|c|c|c|} 
\hline
 $L_b \rightarrow$& \multicolumn{4}{c||}{8} & \multicolumn{4}{c||}{8}\\
 \hline
 \backslashbox{$L_A$\kern-1em}{\kern-1em$N_c$} & 2 & 4 & 8 & 16 & 2 & 4 & 8 & 16  \\
 %$N_c \rightarrow$ & 2 & 4 & 8 & 16 & 2 & 4 & 2 \\
 \hline
 \hline
 \multicolumn{5}{|c|}{Llama2-7B (FP32 Accuracy = 45.8\%)} & \multicolumn{4}{|c|}{Llama2-70B (FP32 Accuracy = 69.12\%)} \\ 
 \hline
 \hline
 64 & 43.9 & 43.4 & 43.9 & 44.9 & 68.07 & 68.27 & 68.17 & 68.75 \\
 \hline
 32 & 44.5 & 43.8 & 44.9 & 44.5 & 68.37 & 68.51 & 68.35 & 68.27  \\
 \hline
 16 & 43.9 & 42.7 & 44.9 & 45 & 68.12 & 68.77 & 68.31 & 68.59  \\
 \hline
 \hline
 \multicolumn{5}{|c|}{GPT3-22B (FP32 Accuracy = 38.75\%)} & \multicolumn{4}{|c|}{Nemotron4-15B (FP32 Accuracy = 64.3\%)} \\ 
 \hline
 \hline
 64 & 36.71 & 38.85 & 38.13 & 38.92 & 63.17 & 62.36 & 63.72 & 64.09 \\
 \hline
 32 & 37.95 & 38.69 & 39.45 & 38.34 & 64.05 & 62.30 & 63.8 & 64.33  \\
 \hline
 16 & 38.88 & 38.80 & 38.31 & 38.92 & 63.22 & 63.51 & 63.93 & 64.43  \\
 \hline
\end{tabular}
\caption{\label{tab:mmlu_abalation} Accuracy on MMLU dataset across GPT3-22B, Llama2-7B, 70B and Nemotron4-15B models.}
\end{table}


%\subsection{Perplexity achieved by various LO-BCQ configurations on LM evaluation harness}

\begin{table} \centering
\begin{tabular}{|c||c|c|c|c||c|c|c|c|} 
\hline
 $L_b \rightarrow$& \multicolumn{4}{c||}{8} & \multicolumn{4}{c||}{8}\\
 \hline
 \backslashbox{$L_A$\kern-1em}{\kern-1em$N_c$} & 2 & 4 & 8 & 16 & 2 & 4 & 8 & 16  \\
 %$N_c \rightarrow$ & 2 & 4 & 8 & 16 & 2 & 4 & 2 \\
 \hline
 \hline
 \multicolumn{5}{|c|}{Race (FP32 Accuracy = 37.51\%)} & \multicolumn{4}{|c|}{Boolq (FP32 Accuracy = 64.62\%)} \\ 
 \hline
 \hline
 64 & 36.94 & 37.13 & 36.27 & 37.13 & 63.73 & 62.26 & 63.49 & 63.36 \\
 \hline
 32 & 37.03 & 36.36 & 36.08 & 37.03 & 62.54 & 63.51 & 63.49 & 63.55  \\
 \hline
 16 & 37.03 & 37.03 & 36.46 & 37.03 & 61.1 & 63.79 & 63.58 & 63.33  \\
 \hline
 \hline
 \multicolumn{5}{|c|}{Winogrande (FP32 Accuracy = 58.01\%)} & \multicolumn{4}{|c|}{Piqa (FP32 Accuracy = 74.21\%)} \\ 
 \hline
 \hline
 64 & 58.17 & 57.22 & 57.85 & 58.33 & 73.01 & 73.07 & 73.07 & 72.80 \\
 \hline
 32 & 59.12 & 58.09 & 57.85 & 58.41 & 73.01 & 73.94 & 72.74 & 73.18  \\
 \hline
 16 & 57.93 & 58.88 & 57.93 & 58.56 & 73.94 & 72.80 & 73.01 & 73.94  \\
 \hline
\end{tabular}
\caption{\label{tab:mmlu_abalation} Accuracy on LM evaluation harness tasks on GPT3-1.3B model.}
\end{table}

\begin{table} \centering
\begin{tabular}{|c||c|c|c|c||c|c|c|c|} 
\hline
 $L_b \rightarrow$& \multicolumn{4}{c||}{8} & \multicolumn{4}{c||}{8}\\
 \hline
 \backslashbox{$L_A$\kern-1em}{\kern-1em$N_c$} & 2 & 4 & 8 & 16 & 2 & 4 & 8 & 16  \\
 %$N_c \rightarrow$ & 2 & 4 & 8 & 16 & 2 & 4 & 2 \\
 \hline
 \hline
 \multicolumn{5}{|c|}{Race (FP32 Accuracy = 41.34\%)} & \multicolumn{4}{|c|}{Boolq (FP32 Accuracy = 68.32\%)} \\ 
 \hline
 \hline
 64 & 40.48 & 40.10 & 39.43 & 39.90 & 69.20 & 68.41 & 69.45 & 68.56 \\
 \hline
 32 & 39.52 & 39.52 & 40.77 & 39.62 & 68.32 & 67.43 & 68.17 & 69.30  \\
 \hline
 16 & 39.81 & 39.71 & 39.90 & 40.38 & 68.10 & 66.33 & 69.51 & 69.42  \\
 \hline
 \hline
 \multicolumn{5}{|c|}{Winogrande (FP32 Accuracy = 67.88\%)} & \multicolumn{4}{|c|}{Piqa (FP32 Accuracy = 78.78\%)} \\ 
 \hline
 \hline
 64 & 66.85 & 66.61 & 67.72 & 67.88 & 77.31 & 77.42 & 77.75 & 77.64 \\
 \hline
 32 & 67.25 & 67.72 & 67.72 & 67.00 & 77.31 & 77.04 & 77.80 & 77.37  \\
 \hline
 16 & 68.11 & 68.90 & 67.88 & 67.48 & 77.37 & 78.13 & 78.13 & 77.69  \\
 \hline
\end{tabular}
\caption{\label{tab:mmlu_abalation} Accuracy on LM evaluation harness tasks on GPT3-8B model.}
\end{table}

\begin{table} \centering
\begin{tabular}{|c||c|c|c|c||c|c|c|c|} 
\hline
 $L_b \rightarrow$& \multicolumn{4}{c||}{8} & \multicolumn{4}{c||}{8}\\
 \hline
 \backslashbox{$L_A$\kern-1em}{\kern-1em$N_c$} & 2 & 4 & 8 & 16 & 2 & 4 & 8 & 16  \\
 %$N_c \rightarrow$ & 2 & 4 & 8 & 16 & 2 & 4 & 2 \\
 \hline
 \hline
 \multicolumn{5}{|c|}{Race (FP32 Accuracy = 40.67\%)} & \multicolumn{4}{|c|}{Boolq (FP32 Accuracy = 76.54\%)} \\ 
 \hline
 \hline
 64 & 40.48 & 40.10 & 39.43 & 39.90 & 75.41 & 75.11 & 77.09 & 75.66 \\
 \hline
 32 & 39.52 & 39.52 & 40.77 & 39.62 & 76.02 & 76.02 & 75.96 & 75.35  \\
 \hline
 16 & 39.81 & 39.71 & 39.90 & 40.38 & 75.05 & 73.82 & 75.72 & 76.09  \\
 \hline
 \hline
 \multicolumn{5}{|c|}{Winogrande (FP32 Accuracy = 70.64\%)} & \multicolumn{4}{|c|}{Piqa (FP32 Accuracy = 79.16\%)} \\ 
 \hline
 \hline
 64 & 69.14 & 70.17 & 70.17 & 70.56 & 78.24 & 79.00 & 78.62 & 78.73 \\
 \hline
 32 & 70.96 & 69.69 & 71.27 & 69.30 & 78.56 & 79.49 & 79.16 & 78.89  \\
 \hline
 16 & 71.03 & 69.53 & 69.69 & 70.40 & 78.13 & 79.16 & 79.00 & 79.00  \\
 \hline
\end{tabular}
\caption{\label{tab:mmlu_abalation} Accuracy on LM evaluation harness tasks on GPT3-22B model.}
\end{table}

\begin{table} \centering
\begin{tabular}{|c||c|c|c|c||c|c|c|c|} 
\hline
 $L_b \rightarrow$& \multicolumn{4}{c||}{8} & \multicolumn{4}{c||}{8}\\
 \hline
 \backslashbox{$L_A$\kern-1em}{\kern-1em$N_c$} & 2 & 4 & 8 & 16 & 2 & 4 & 8 & 16  \\
 %$N_c \rightarrow$ & 2 & 4 & 8 & 16 & 2 & 4 & 2 \\
 \hline
 \hline
 \multicolumn{5}{|c|}{Race (FP32 Accuracy = 44.4\%)} & \multicolumn{4}{|c|}{Boolq (FP32 Accuracy = 79.29\%)} \\ 
 \hline
 \hline
 64 & 42.49 & 42.51 & 42.58 & 43.45 & 77.58 & 77.37 & 77.43 & 78.1 \\
 \hline
 32 & 43.35 & 42.49 & 43.64 & 43.73 & 77.86 & 75.32 & 77.28 & 77.86  \\
 \hline
 16 & 44.21 & 44.21 & 43.64 & 42.97 & 78.65 & 77 & 76.94 & 77.98  \\
 \hline
 \hline
 \multicolumn{5}{|c|}{Winogrande (FP32 Accuracy = 69.38\%)} & \multicolumn{4}{|c|}{Piqa (FP32 Accuracy = 78.07\%)} \\ 
 \hline
 \hline
 64 & 68.9 & 68.43 & 69.77 & 68.19 & 77.09 & 76.82 & 77.09 & 77.86 \\
 \hline
 32 & 69.38 & 68.51 & 68.82 & 68.90 & 78.07 & 76.71 & 78.07 & 77.86  \\
 \hline
 16 & 69.53 & 67.09 & 69.38 & 68.90 & 77.37 & 77.8 & 77.91 & 77.69  \\
 \hline
\end{tabular}
\caption{\label{tab:mmlu_abalation} Accuracy on LM evaluation harness tasks on Llama2-7B model.}
\end{table}

\begin{table} \centering
\begin{tabular}{|c||c|c|c|c||c|c|c|c|} 
\hline
 $L_b \rightarrow$& \multicolumn{4}{c||}{8} & \multicolumn{4}{c||}{8}\\
 \hline
 \backslashbox{$L_A$\kern-1em}{\kern-1em$N_c$} & 2 & 4 & 8 & 16 & 2 & 4 & 8 & 16  \\
 %$N_c \rightarrow$ & 2 & 4 & 8 & 16 & 2 & 4 & 2 \\
 \hline
 \hline
 \multicolumn{5}{|c|}{Race (FP32 Accuracy = 48.8\%)} & \multicolumn{4}{|c|}{Boolq (FP32 Accuracy = 85.23\%)} \\ 
 \hline
 \hline
 64 & 49.00 & 49.00 & 49.28 & 48.71 & 82.82 & 84.28 & 84.03 & 84.25 \\
 \hline
 32 & 49.57 & 48.52 & 48.33 & 49.28 & 83.85 & 84.46 & 84.31 & 84.93  \\
 \hline
 16 & 49.85 & 49.09 & 49.28 & 48.99 & 85.11 & 84.46 & 84.61 & 83.94  \\
 \hline
 \hline
 \multicolumn{5}{|c|}{Winogrande (FP32 Accuracy = 79.95\%)} & \multicolumn{4}{|c|}{Piqa (FP32 Accuracy = 81.56\%)} \\ 
 \hline
 \hline
 64 & 78.77 & 78.45 & 78.37 & 79.16 & 81.45 & 80.69 & 81.45 & 81.5 \\
 \hline
 32 & 78.45 & 79.01 & 78.69 & 80.66 & 81.56 & 80.58 & 81.18 & 81.34  \\
 \hline
 16 & 79.95 & 79.56 & 79.79 & 79.72 & 81.28 & 81.66 & 81.28 & 80.96  \\
 \hline
\end{tabular}
\caption{\label{tab:mmlu_abalation} Accuracy on LM evaluation harness tasks on Llama2-70B model.}
\end{table}

%\section{MSE Studies}
%\textcolor{red}{TODO}


\subsection{Number Formats and Quantization Method}
\label{subsec:numFormats_quantMethod}
\subsubsection{Integer Format}
An $n$-bit signed integer (INT) is typically represented with a 2s-complement format \citep{yao2022zeroquant,xiao2023smoothquant,dai2021vsq}, where the most significant bit denotes the sign.

\subsubsection{Floating Point Format}
An $n$-bit signed floating point (FP) number $x$ comprises of a 1-bit sign ($x_{\mathrm{sign}}$), $B_m$-bit mantissa ($x_{\mathrm{mant}}$) and $B_e$-bit exponent ($x_{\mathrm{exp}}$) such that $B_m+B_e=n-1$. The associated constant exponent bias ($E_{\mathrm{bias}}$) is computed as $(2^{{B_e}-1}-1)$. We denote this format as $E_{B_e}M_{B_m}$.  

\subsubsection{Quantization Scheme}
\label{subsec:quant_method}
A quantization scheme dictates how a given unquantized tensor is converted to its quantized representation. We consider FP formats for the purpose of illustration. Given an unquantized tensor $\bm{X}$ and an FP format $E_{B_e}M_{B_m}$, we first, we compute the quantization scale factor $s_X$ that maps the maximum absolute value of $\bm{X}$ to the maximum quantization level of the $E_{B_e}M_{B_m}$ format as follows:
\begin{align}
\label{eq:sf}
    s_X = \frac{\mathrm{max}(|\bm{X}|)}{\mathrm{max}(E_{B_e}M_{B_m})}
\end{align}
In the above equation, $|\cdot|$ denotes the absolute value function.

Next, we scale $\bm{X}$ by $s_X$ and quantize it to $\hat{\bm{X}}$ by rounding it to the nearest quantization level of $E_{B_e}M_{B_m}$ as:

\begin{align}
\label{eq:tensor_quant}
    \hat{\bm{X}} = \text{round-to-nearest}\left(\frac{\bm{X}}{s_X}, E_{B_e}M_{B_m}\right)
\end{align}

We perform dynamic max-scaled quantization \citep{wu2020integer}, where the scale factor $s$ for activations is dynamically computed during runtime.

\subsection{Vector Scaled Quantization}
\begin{wrapfigure}{r}{0.35\linewidth}
  \centering
  \includegraphics[width=\linewidth]{sections/figures/vsquant.jpg}
  \caption{\small Vectorwise decomposition for per-vector scaled quantization (VSQ \citep{dai2021vsq}).}
  \label{fig:vsquant}
\end{wrapfigure}
During VSQ \citep{dai2021vsq}, the operand tensors are decomposed into 1D vectors in a hardware friendly manner as shown in Figure \ref{fig:vsquant}. Since the decomposed tensors are used as operands in matrix multiplications during inference, it is beneficial to perform this decomposition along the reduction dimension of the multiplication. The vectorwise quantization is performed similar to tensorwise quantization described in Equations \ref{eq:sf} and \ref{eq:tensor_quant}, where a scale factor $s_v$ is required for each vector $\bm{v}$ that maps the maximum absolute value of that vector to the maximum quantization level. While smaller vector lengths can lead to larger accuracy gains, the associated memory and computational overheads due to the per-vector scale factors increases. To alleviate these overheads, VSQ \citep{dai2021vsq} proposed a second level quantization of the per-vector scale factors to unsigned integers, while MX \citep{rouhani2023shared} quantizes them to integer powers of 2 (denoted as $2^{INT}$).

\subsubsection{MX Format}
The MX format proposed in \citep{rouhani2023microscaling} introduces the concept of sub-block shifting. For every two scalar elements of $b$-bits each, there is a shared exponent bit. The value of this exponent bit is determined through an empirical analysis that targets minimizing quantization MSE. We note that the FP format $E_{1}M_{b}$ is strictly better than MX from an accuracy perspective since it allocates a dedicated exponent bit to each scalar as opposed to sharing it across two scalars. Therefore, we conservatively bound the accuracy of a $b+2$-bit signed MX format with that of a $E_{1}M_{b}$ format in our comparisons. For instance, we use E1M2 format as a proxy for MX4.

\begin{figure}
    \centering
    \includegraphics[width=1\linewidth]{sections//figures/BlockFormats.pdf}
    \caption{\small Comparing LO-BCQ to MX format.}
    \label{fig:block_formats}
\end{figure}

Figure \ref{fig:block_formats} compares our $4$-bit LO-BCQ block format to MX \citep{rouhani2023microscaling}. As shown, both LO-BCQ and MX decompose a given operand tensor into block arrays and each block array into blocks. Similar to MX, we find that per-block quantization ($L_b < L_A$) leads to better accuracy due to increased flexibility. While MX achieves this through per-block $1$-bit micro-scales, we associate a dedicated codebook to each block through a per-block codebook selector. Further, MX quantizes the per-block array scale-factor to E8M0 format without per-tensor scaling. In contrast during LO-BCQ, we find that per-tensor scaling combined with quantization of per-block array scale-factor to E4M3 format results in superior inference accuracy across models. 



%%%%%%%%%%%%%%%%%%%%%%%%%%%%%%%%%%%%%%%%%%%%%%%%%%%%%%%%%%%%%%%%%%%%%%%%%%%%%%%
%%%%%%%%%%%%%%%%%%%%%%%%%%%%%%%%%%%%%%%%%%%%%%%%%%%%%%%%%%%%%%%%%%%%%%%%%%%%%%%


\end{document}
