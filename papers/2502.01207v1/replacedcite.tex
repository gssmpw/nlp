\section{Related Work, Background, and Motivation}
\label{sec_background}
\subsection{Related Work}
Recently, driven by breakthroughs in autonomous systems, various research teams and projects were investigating, transferring, and applying control, planning, and perception methods to reach the autonomous operation of surface vessels in different tasks and application scenarios. Those applications can be separated into autonomous \textit{water-taxis},  \mbox{autonomous} \textit{ferries}, and autonomous \textit{research and container vessels}.  \mbox{\textbf{Autonomous water taxis}} are considered e.g. by the \texttt{RoBoat} project in Amsterdam ____ and the \texttt{SeaBubble} project on the Seine River ____ to achieve an economic ____ and sustainable ____ operation. 
Regarding  \mbox{\textbf{autonomous ferries}}, the \texttt{Zeabuz} project in Stockholm ____, the German \texttt{CAPTN} project with \texttt{MS WaveLab} ____, the Norwegian autonomous ferry prototypes \texttt{milliAmpere} ____ and \texttt{milliAmpere2} ____, the \texttt{GreenHopper} from DTU in Aalborg ____, and the \texttt{Akoon} project with the river ferry \texttt{Horst} ____ are examples for recent investigations and prototypes considering the future autonomous mobility on waterways ____ to achieve efficient operation ____. 
Recent examples of projects considering autonomous  \mbox{\textbf{research and container vessels}} are the Chinese \texttt{Jin Dou Yun O Hao} project ____, the German research vessel \texttt{DENEB} ____, and the Japanese experimental vessel \texttt{Shinpo} ____. Autonomous container vessels ____ are considered in the European \texttt{AUTOSHIP} project ____ or by the \texttt{Yara Birkeland} project ____. 
Note that water taxi applications on which we focus in this paper are typically characterized by being based on small vessels transporting passengers on short individually chosen routes.  


\subsection{Background}
The \textit{HTWG Konstanz - University of Applied Sciences} is located on the shores of Lake Constance, a water-based transportation hub. Lake Constance has a long history of important trade and transportation routes, connecting cities in Germany, Switzerland, and Austria ____. % The \textit{Seerhein}  is a river connecting different parts of the lake that underlines the importance of water transportation by separating the urban districts. 
The design and development of the research vessel \textit{Solgenia} is part of the research focus on autonomous water-based transportation. The different environmental conditions in the Constance region visualized in Figure~\ref{fig_current} present unique challenges and opportunities for autonomous ship navigation. 
In addition, there is a strong economic motivation to explore ASVs in this region. Lake Constance hosts a thriving economy that relies on transporting commuters and tourists by water, making it an ideal environment for implementing cost-effective and sustainable transportation solutions.  In the future, ASVs can further strengthen the already robust water economy around Lake Constance and be part of the way to a more efficient operation.

\subsection{Motivation for ASVs}
ASVs present numerous advantages for transporting people and executing tasks, offering enormous potential across a wide range of domains ____. Some of the most important domains are visualized in Figure~\ref{fig_domains} and described in the following list:
\begin{enumerate}
    \item \textbf{Enhanced Safety:}
    Driver assistance systems included in ASVs can significantly reduce the risk of accidents by human error. With 59.1~\% from 2014 to 2022 these are a major reason for such accidents ____. Equipped with advanced sensors and algorithms, these vessels can achieve a reliable perception, which is especially beneficial in challenging weather and current conditions.
    \item \textbf{Increased Efficiency}
    ASVs can optimize their maritime routes to minimize travel time and/or energy consumption. This is the basis for efficient operation, predictable schedules, and fewer delays ____. The reduced need for the number of people in the crew lowers operational costs and enables an operation even in a shortage of skilled workers.
    \item \textbf{Economic Advantages:}
     The lower operating costs associated with ASVs, including reduced crew requirements and fuel efficiency, can result in significant cost savings and lower ticket prices for passengers or cargo ____. These savings, combined with the ability to offer more frequent services, can increase profitability.
    \item \textbf{Improved Accessibility:}
    ASVs have the potential to independently serve remote and undersupplied regions in the future and improve connectivity for remote communities ____. Their flexibility also enables the development of on-demand services, similar to ride-sharing on the water, which further improves accessibility.
    \item \textbf{Technological Advancement:}
     ASVs are driving innovation in marine technology and other research areas ____. Autonomous ships can collect large amounts of data about the marine environment that can be used for research and environmental protection. 
    \item \textbf{Emergency Response and Rescue:}
   ASVs are suitable for emergency responses and search or rescue operations,  as they can be rapidly deployed without risking human lives ____. Their potential ability to locate and reach people in distress, combined with their potential precision and efficiency, could significantly reduce response times in maritime emergencies.
\end{enumerate}
Summarizing, ASVs have the potential to restructure transportation on the waterways as they offer significant advantages in the described terms. The impact of ASVs on maritime transport is likely to increase as technology advances, reshaping the future of passenger transportation by water.
\begin{figure}[]
	\centering 
	\includegraphics[width=0.48\textwidth, angle=0]{Figures/figure_4.JPG}	
	\caption{Most important advantages and domains for the usage of ASVs.} 
	\label{fig_domains}%
\end{figure}
  \begin{figure}[b!]
	\centering 
	\includegraphics[width=0.48\textwidth, angle=0]{Figures/figure_5.JPG}	
	\caption{CAD model of the vessel's hull including the hardware components.} 
	\label{fig_hull}% Inventor Professional 2024 
\end{figure}