\section{Experiments}
\label{sec:experiment}

\subsection{Experimental Setup}\label{sec:exp_set}
\noindent \textbf{Implementation Details.} 
Our proposed model is fine-tuned on VITON-HD~\cite{choi2021viton}. As with other works~\cite{xu2024ootdiffusion,choi2024improving,velioglu2024tryoffdiff}, we divide it into a training dataset and a testing dataset. Then, we use IDM~\cite{choi2024improving} to prepare the custom datasets for person-to-person task and manually filter out a subset for training. We adopt the FLUX-Fill-dev~\cite{flux} as our foundation model and fine-tuning it on both garment-to-person and person-to-person datasets. In inference stage, the model samples 30 steps to get the final fitting outputs.

\subsection{Qualitative and Quantitative Comparison}\label{sec:exp_comp}
We compare our model with garment-to-person methods OOTD~\cite{xu2024ootdiffusion}, IDM~\cite{choi2024improving}, and CatVTON-FLUX~\cite{catvton-flux}. To adapt these methods for person-to-person tasks, we employ segmentation~\cite{ravi2024sam} and try-off~\cite{velioglu2024tryoffdiff} to extract garment from the reference person. We initially utilize unpaired testing datasets and assess the fidelity of the generated fitting image distributions with three key metrics: FID~\cite{heusel2017gans}, CLIP-FID~\cite{kynkaanniemi2022role} and KID~\cite{binkowski2018demystifying} metrics. In order to more fully evaluate our model, we process the testing dataset using the data preparation method outlined in~\cref{sec:data_preparation} and extract paired datasets such as $\left(P_{mn}, P_{nm}, P_{mm}\right)$ and $\left(P_{nm}, P_{mn}, P_{nn}\right)$. On this dataset, we evaluate the aforementioned metrics and additionally compute SSIM~\cite{wang2004image}, LPIPS~\cite{zhang2018unreasonable} and DISTS~\cite{ding2020image} to evaluate the reconstruction quality between the generated fitting image and corresponding ground truth.

\begin{figure*}[ht]
    \centering
    \includegraphics[width=0.95\linewidth]{figs/fig4_method.png}
    \caption{Qualitative comparison. The first two columns show the inputs to different models. In the person-to-person task, the three garment-to-person methods rely on segmentation and try-off techniques to obtain the garment on the reference person. In contrast, our method directly generates the outputs based on the reference person.}
    \label{fig:fig4_method}
\end{figure*}
\noindent \textbf{Qualitative Comparison.}
As illustrated in~\cref{fig:fig4_method}, our method achieves superior fidelity in person-to-person task. While other methods can adapt to person-to-person task using segmentation or try-off techniques, they often introduce significant artifacts. Despite not requiring a separate input of the person pose, our method effectively preserves the original pose with high accuracy.


\begin{table*}[htbp]
\centering
\begin{tabular}{l|cccccc|ccc}
\toprule
\multirow{2}{*}{Model} & \multicolumn{6}{c|}{Paired Person2Person}            & \multicolumn{3}{c}{Unpaired Person2Person} \\ \cmidrule(){2-10} 
                       & SSIM$\uparrow$    & LPIPS$\downarrow$  & DISTS$\downarrow$  & FID$\downarrow$     & CLIP-FID$\downarrow$ & KID*$\downarrow$    & FID$\downarrow$             & CLIP-FID$\downarrow$       & KID*$\downarrow$          \\ \midrule
Seg+OOTD             & 0.8404 & 0.1445 & 0.1081 & 12.4351  & 3.3757 & 3.5754      & 13.3704   & 3.9595        & 4.3530       \\
Seg+IDM              & \underline{0.8727} & 0.1170 & 0.0957 & 11.0887  & 2.6419 & 3.6665      & 10.8623  & \underline{2.6477}       & 3.0886       \\
Seg+CatVTON     & 0.8715 & 0.1150 & \underline{0.0897} & \underline{9.7622}  & 2.9928 & 2.5167      & 10.6096  & 3.0508       & 2.8575       \\ \midrule
TROF+OOTD            & 0.8409 & 0.1368 & 0.1047 & 11.1590  & 3.0541 & \underline{2.1543}     & 11.7932   & 3.5123       & 2.5396       \\
TROF+IDM             & \textbf{0.8761} & \underline{0.1139} & 0.0950 & 10.5302  & \underline{2.5589} & 2.3982      & 11.2508  & 2.7594       & 2.5920      \\
TROF+CatVTON    & 0.8723 & 0.1158 & 0.0923 & 9.8190  & 2.6181 & \textbf{1.9341}       & \underline{10.5839}  & 2.7688        & \textbf{2.3509}       \\  \midrule
Ours                 & 0.8688 & \textbf{0.1122} & \textbf{0.0870} & \textbf{9.3223} & \textbf{2.1333}   & 2.1581  & \textbf{10.3465}         & \textbf{2.2885}        & \underline{2.4658}      \\ \bottomrule
\end{tabular}
\caption{Quantitative comparison with other methods on person-to-person task. The KID metric is multiplied by the factor 1e3 to ensure a similar order of magnitude to the other metrics.}
\label{tab:quantitative_person}
\end{table*}









\begin{table*}[htbp]
\centering
\begin{tabular}{l|cccccc|ccc}
\toprule
\multirow{2}{*}{Model} & \multicolumn{6}{c|}{Paired Garment2Person}            & \multicolumn{3}{c}{Unpaired Garment2Person} \\ \cmidrule(){2-10} 
                       & SSIM$\uparrow$    & LPIPS$\downarrow$  & DISTS$\downarrow$  & FID$\downarrow$     & CLIP-FID$\downarrow$ & KID*$\downarrow$    & FID$\downarrow$             & CLIP-FID$\downarrow$       & KID*$\downarrow$          \\ \midrule
OOTD             & 0.8556 & 0.1118 & 0.0849 & 6.8680  & 2.2030 & \textbf{1.4632}       & 9.8221 & 2.8306 & \textbf{1.6700}      \\
IDM              & \textbf{0.8789} & \textbf{0.0940} & 0.0806 & 6.6752  & 2.1008 & 1.7398   & \underline{9.6548} & \underline{2.4607} & 1.8081       \\
CatVTON     & \underline{0.8774} & \underline{0.0975} & \textbf{0.0776} & \textbf{6.3788}  & 2.2642 & 1.6641      & 9.7696 & 2.7375 & 2.0727      \\ 
Ours                 & 0.8761 & 0.0986 & \underline{0.0790} & \underline{6.4206} & \textbf{1.8431}   & \underline{1.5260}  & \textbf{9.5728} & \textbf{2.2566} & \underline{1.7624}      \\ \bottomrule
\end{tabular}
\caption{Quantitative comparison with other methods on person-to-person task. The KID metric is multiplied by the factor 1e3 to ensure a similar order of magnitude to the other metrics.}
\label{tab:quantitative_garment}
\end{table*}
\noindent \textbf{Quantitative Comparison.}
Quantitative results demonstrate that our method excels in both person-to-person task, as evidenced in~\cref{tab:quantitative_person}, and garment-to-person task, as shown in~\cref{tab:quantitative_garment}, outperforming existing methods across multiple metrics. Additionally, quantitative results indicate that the try-off method is more effective than the segmentation method in facilitating the realization of person-to-person tasks.