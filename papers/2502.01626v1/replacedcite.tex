\section{Related Work}
\label{sec:related}

\subsection{Virtual Try-On}
Current virtual try-on methods can be categorized into garment-to-person____ and person-to-person____. The methods corresponding to the former are mostly trained on VTON-HD____ and DressCode____, which contain high-resolution paired data of standard garments and person images. These methods mask the areas of garment to be generated, indicating where the new garment should be placed, but this leads to the loss of foreground information. In contrast, ____ proposes a mask-free method that does not explicitly define the areas to be fitted, helping to preserve the original image details. However, these methods still rely on garments as references. To achieve person-to-person try-on, ____ suggests using an additional model to restore the standard garment from the person image. Several works____ attempt to address person image-based try-on using unpaired datasets due to limitations in the datasets, but the fitting results remain suboptimal. In our work, we use a state-of-the-art VTON model____ to create a person-to-person dataset and design a mask-free framework based on FLUX-Fill-dev____ capable of handling both garment-to-person and person-to-person tasks.