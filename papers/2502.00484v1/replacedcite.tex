\section{Related Work}
Since the early work of Bogolmonaia et al. ____, and later addressed in ____, many articles have been written on allocating a divisible budget to a given set of projects. The problem has many applications: how much time (resp., space) 
is given to various topics in a given conference (resp., textbook),  which voting power is given to the parties composing a parliament, how much money is allocated to different charities by a company, etc.     


Budget divisions are constructed on the basis of preferences expressed by some agents (a.k.a. voters). Depending on the context, the format used to express preferences and the way they are aggregated vary. Also, the decision about every given project can be binary (fund it entirely or not at all), or it can be a real %amount 
within a given range (e.g., allocating a percentage of the total budget).     
Let us review some previous related works that fall into the collective budget framework.   

%There are many articles dealing with budgets that need to satisfy a group of people. 


In the \emph{portioning} problem ____, we are given a perfectly divisible budget of one unit to be spent on a set of $m$ projects, and some information about the preferences of $n$ voters. A solution is a vector ${\bf x} \in [0,1]^m$ satisfying $\sum_{j=1}^m x_j=1$. An \emph{aggregation mechanism} (mechanism in short) is a function whose input and output are the preferences of the voters and a solution, respectively.



In the seminal article of Freeman et al. ____ on portioning, every agent $i \in [n]$ declares a score vector $\textbf{s}^i=(s^i_1, \ldots,s^i_m)$ such that $\sum_{j=1}^m s^i_j=1$. The cost of agent $i$ for the solution ${\bf x}$  is the L1-distance between $\textbf{x}$ and $\textbf{s}^i$, i.e.,  $dist_1(\textbf{x},\textbf{s}^i):=\sum_{j \in [m]} |s^i_j - x_j|$. The authors are interested in \emph{strategyproof} mechanisms for which no voter  should be able to change the output in her favor by misreporting her score vector. They give a class of strategyproof mechanisms named \emph{moving phantom} which leverages Moulin's method ____. Within this class, they identify a specific mechanism, called \emph{independent markets}, which satisfies an extra fairness constraint named \emph{proportionality}.         


In ____, Elkind et al. study the independent markets mechanism and two families of mechanisms from an axiomatic viewpoint. The first family is said to be coordinate-wise, in the sense that it produces a solution $\textbf{x}$ such that $x_j$ is a function of the multiset $\{s^i_j \mid i \in [n]\}$ for each $j \in [m]$. The second family defines $\textbf{x}$ as an optimum of a social welfare function. The desirable axioms under consideration include strategyproofness, proportionality, monotonicity and Pareto optimality, to name a few.  

Caragiannis et al. ____ also consider the portioning problem. The focus is on strategyproof mechanisms whose L$_1$-distance between their output and the mean vector $(\sum_{i\in [n]}s^i_j/n)_{j \in [m]}$ should be upper bounded by some quantity $\alpha$. They give strategyproof mechanisms such that $\alpha \leq 1/2$ when $m=2$, and $\alpha \leq 2/3+\epsilon$ when $m=3$. Going further, Freeman and Schmidt{-}Kraepelin ____ proposed to also consider the L$_\infty$-distance, as a measure of fairness between the projects ____.


In the article by Michorzewski et al. ____ on portioning, 
voters have separately declared the subset of projects that they approve. The utility of some voter $i$ for the solution ${\bf x}$ is $u_i({\bf x})=\sum_{j\in A_i} x_j$ where $A_i \subseteq [m]$ is the set of projects approved by voter $i$. The authors consider the worst-case deterioration of the utilitarian social welfare of various (probabilistic) voting rules (e.g., Nash social welfare maximization) and the fairness axioms (e.g., individual fair share) that they guarantee. 



In the work of Airiau et al. ____ on portioning, every agent  has ranked the projects by order of preference. Each ranking is converted into scores for the projects (using, for example, plurality, veto, or Borda), and the scores are used for evaluating budget divisions. Then, a solution maximizing the social welfare is sought, considering  various notions of social welfare: utilitarian, egalitarian, or Nash. This process gives rise to various {\em positional social decision schemes}. The authors have analyzed and evaluated the %computational 
complexity of those schemes, together with the fairness axioms that they possibly satisfy. % ____.        

In ____, Goyal et al. study the distortion of randomized mechanisms with low sample-complexity 
for the portioning problem where the social cost of an output $\textbf{x}$ is $\sum_{i=1}^n dist_1(\textbf{x},\textbf{s}^i)$. Their main result is a mechanism which uses 3 votes, the distortion of which is between $1.38$ and $1.66$.    



Brill et al. ____ consider elections in which voters express approval votes over parties and a given number of seats must be distributed among the parties. This model is related to the previous budget division problems (the budget is the total number of seats, and the parties are the projects), with the specificity that parties get an integral number of seats. 





In the model introduced and studied by Wagner and Meir ____, every agent $i$ expresses an amount $t_i$ for modification of an initial budget $B_0$, namely agent $i$ prescribes a budget of $B_0+n t_i$ where $t_i >-B_0/n$. On top of that, every agent declares an allocation of the resulting budget on $m$ given projects. Wagner and Meir show that one can construct a VCG-like mechanism in this setting, preventing any agent's incentive to report false declarations. Another incentive-compatible mechanism has been proposed in a slightly different setting where there is no single exogenous budget, but each agent $i$ has her own budget $B_i$ ____. The problem is to determine the agents’ contributions, and %properly 
distribute them on $m$ given projects while taking into account the agents utilities. 



As mentioned previously, participatory budgeting is a %well-studied 
paradigm in which citizens can propose projects of public interest (e.g., refurbishing a school) and then vote to determine which of these projects are actually financed with public funds ____. Projects are indivisible, meaning they are either fully funded or not funded at all. Similarly, in the model of Cardi et al., there is a common budget to be used for funding the ``private'' projects submitted by some agents  ____. A valid solution is a subset of projects whose total cost does not exceed the budget, and each agent's utility for a given solution is equal to the cost of the accepted projects submitted by her. In this setting, various notions of fairness and efficiency, together with their possible combination, have been addressed. 




The previously mentioned works share important similarities with the model studied in this article. However, the specificity of our model is mainly based on the way of satisfying the agents, making our results not directly comparable with those of the literature. Indeed, previous works assume that every agent $i$ has a cardinal (dis)-utility with respect to a solution. In the present article, the agents have two possible statuses, depending on whether they are satisfied or not by the outcome. In addition, in contrast with previous works, we assume that each agent is equally interested in \emph{all} projects, and demands represent additional resources needed to make the status of existing projects acceptable (according to the agent's opinion). If an agent is satisfied with the current state of a project, then no need to devote extra resources from her point of view. On the contrary, if an agent is not satisfied by the state of a project, then some efforts are necessary and she would allocate new resources to it in order to make it acceptable. Thus, the objective of the present work is to reach global satisfaction of an agent after the extra resources from the budget are allocated.