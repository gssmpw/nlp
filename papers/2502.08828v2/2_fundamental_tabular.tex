\vspace{-0.2cm}
\section{Data-Centric AI: Tabular Learning}

Tabular data is one of the most common data formats in the real world, widely used in fields such as bioinformatics, healthcare, and marketing. It has a well-defined structure, where rows represent samples and columns represent features, allowing for precise representation of numerical and categorical information and their relationships. Additionally, tabular data is highly interpretable. However, its practical application faces several challenges, including many redundant features, complex feature interactions, the difficulty of automated modeling, and privacy concerns due to data sensitivity. These issues limit model efficiency and performance, hindering the widespread adoption of AI for tabular data.

To address these challenges, Data-Centric AI focuses on improving data quality and representation to enhance model performance, with feature selection and feature generation as its core tasks. Feature selection eliminates redundant and irrelevant information, retaining only the most valuable features for modeling, while feature generation constructs a more discriminative feature space, enabling models to learn complex patterns more effectively. By optimizing these aspects, Data-Centric AI reconstructs a refined and information-rich data representation, improving the practicality and reliability of tabular data in AI applications.

\textbf{Feature selection}  aims to identify and retain the most informative features while discarding redundant or irrelevant ones, thereby improving model performance, efficiency, and interoperability.

\textbf{Feature generation} aims to rebuild a new feature space from an original feature set (e.g. $[f_1, f_2] \rightarrow [\frac{f_1}{f_2}, f_1-f_2, \frac{f_1+f_2}{f_1}]$), where $f$ represents a feature (a.k.a, a column) of a tabular dataset. It can advance the power (structural, interaction, and expression levels) of data to make data AI-ready.


Since tabular data lacks inherent spatial or sequential structure, feature extraction is typically explicit, relying heavily on manual feature engineering techniques such as interaction term construction and aggregation. In contrast, unstructured data (e.g., images and audio) can leverage deep learning to automatically extract implicit features, whereas tabular data still faces challenges in automating feature construction.
Reinforcement learning (RL) and generative AI introduce new possibilities for automating and optimizing feature engineering in tabular data. RL enables intelligent search and optimization mechanisms for automated feature selection and transformation, reducing reliance on manual engineering. Meanwhile, generative AI can create new features or enhance existing ones, improving the expressiveness of tabular data.
Tabular data-centric AI aims to transform raw datasets into optimized representations that enhance interpretability, computational efficiency, and predictive accuracy. This approach bridges the gap between traditional manual feature engineering and automated techniques, offering a more efficient and intelligent solution for tabular data analysis.


