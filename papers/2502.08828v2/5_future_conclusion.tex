\section{Challenges and Future Research Directions}

As tabular data-centric AI continues to evolve, researchers face numerous challenges while also encountering new opportunities for innovation. This chapter explores six key issues: automation in feature engineering, interpretability, privacy preservation, computational efficiency, scalability, and the integration of large language models (LLMs) and multimodal learning. For each aspect, we analyze the existing challenges and outline potential research directions to advance the field.

\noindent\textbf{Automation in Feature Engineering: Balancing Adaptability and Efficiency.}
Automating feature engineering has significantly improved data processing efficiency by reducing manual intervention. However, real-world applications often demand domain-specific customizations that automated methods struggle to accommodate. Furthermore, current approaches face scalability limitations when dealing with high-dimensional, large-scale datasets, leading to excessive computational costs. Future Research Directions:
1) Human-in-the-loop hybrid systems: Combining automated techniques with expert knowledge can create adaptable systems that maintain efficiency while incorporating domain-specific insights.
2) Resource-efficient strategies: Optimizing computational frameworks for automated feature engineering will enable their deployment in environments with limited processing power, increasing accessibility.

\noindent\textbf{Enhancing Interpretability: Making Feature Transformations Transparent.}
Feature engineering plays a crucial role in model performance, yet its processes are often opaque. In high-stakes domains such as healthcare, and finance~\cite{liu2025calorie,leng2023nlrp3}, a lack of transparency can undermine trust, making it essential to understand the rationale behind feature transformations. Future Research Directions:
1) Traceable transformation workflows: Developing transparent feature engineering pipelines will allow users to track each transformation step and its relationship with model predictions.
2) Quantifying feature impact: Establishing methods to measure the influence of feature transformations on performance helps identify the most critical steps, fostering confidence in AI-driven decisions.

\noindent\textbf{Privacy-Preserving Feature Engineering.}
With growing concerns about data privacy, federated learning has emerged as a promising approach for distributed feature engineering. However, challenges remain in handling heterogeneous data distributions, reducing communication overhead, and ensuring semantic consistency across data sources. Future Research Directions:
1) Cross-source feature alignment: Developing robust semantic mapping techniques will ensure that data from different origins can be processed consistently without compromising privacy.
2) Reducing communication costs: Optimizing distributed computing strategies will enhance the efficiency of federated learning by minimizing data exchange while maintaining performance.


\noindent\textbf{Computational Efficiency and Scalability: Optimizing Large-Scale Feature Engineering}
As datasets become increasingly complex, the computational demands of feature engineering grow, making many existing approaches impractical for real-time applications. Iterative feature generation and optimization further exacerbate these challenges, leading to significant computational bottlenecks. Future Research Directions:
1) Lightweight feature engineering algorithms: Developing approximation-based or low-rank decomposition methods can reduce computational burdens without sacrificing effectiveness.
2) Leveraging parallel computing and hardware acceleration: Distributed processing frameworks and GPU/TPU-based optimizations can improve the scalability of feature engineering, enabling real-time deployment.

\noindent\textbf{Large Language Models and Multimodal Feature Engineering: Expanding the Scope of Representation Learning}
The rapid advancement of LLMs and multimodal learning offers new possibilities for feature engineering by integrating structured tabular data with text, images, and other modalities. However, encoding tabular data effectively within LLM architectures remains a challenge, as does preserving the distinct characteristics of each modality during integration. Future Research Directions:
1) Developing specialized tabular embeddings: Designing encoding techniques tailored to tabular data will improve its compatibility with LLM-based frameworks.
2) Cross-modal feature alignment: Ensuring that multimodal data representations preserve semantic relationships is essential for achieving meaningful feature integration.
3) Efficient fine-tuning and knowledge transfer: Optimizing model adaptation techniques will enhance the practicality of LLM-based feature generation for tabular tasks.

\noindent\textbf{Future Outlook: Balancing Performance and Interpretability}
As feature engineering techniques become increasingly sophisticated, striking a balance between performance and interpretability is a pressing challenge. Complex transformation pipelines can obscure relationships between raw inputs and model outputs, making it difficult to ensure transparency, particularly in critical domains where accountability is essential. Future Research Directions:
1) Developing traceable feature generation frameworks: Logging and visualization tools can provide researchers with clear insights into how feature transformations affect model behavior.
2) User-friendly interpretability tools: Designing intuitive interfaces will enable non-technical stakeholders to understand feature engineering processes, facilitating broader AI adoption in real-world applications.

\vspace{-0.2cm}
\section{Conclusion}
Tabular data-centric AI is evolving with RL-based optimization and generative modeling playing a key role in feature engineering. This study examines the challenges and future directions in this domain, focusing on automation, interpretability, and scalability. By addressing these issues, we highlight both the limitations of existing methods and the transformative potential of AI-driven approaches.
Recent advances, particularly RL frameworks for adaptive feature selection and deep generative models for data augmentation, are redefining tabular data processing. These methods enhance autonomy and efficiency, bridging the gap between raw data and high-performance models. The integration of large language models and multimodal learning further expands the capabilities of tabular AI, enabling it to handle more diverse data sources.
Future research should refine RL-based optimization for stable training and efficient rewards while advancing generative techniques for high-quality, domain-adaptive features. As these challenges are addressed, tabular AI will continue to push automated machine learning forward, improving scalability and interpretability. This study aims to inspire further innovation, equipping researchers with new tools to maximize AI-driven feature engineering.