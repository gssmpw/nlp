% \section{Tabular Feature Engineering in Data-Centric AI: Methods Comparison, Best Practices, and Strategic Selection}
% \label{sec:comparison}

% \begin{table*}[h]
% \centering
% \caption{Comparison of RL-based and Generative-based Methods}
% \label{tab:rl_vs_gen}
% \begin{tabular}{c|c|c} 
% \hline
% \textbf{Aspect}    & \textbf{RL-based Methods}                                                                                & \textbf{Generative-based Methods}                                                                                \\ 
% \hline\hline
% Performance        & \begin{tabular}[c]{@{}c@{}}Suited for sequential \\ decision-making, \\ can adapt in real time\end{tabular}            & \begin{tabular}[c]{@{}c@{}}Efficient parallel exploration \\ of large feature spaces, \\ excels at complex data transformations\end{tabular}  \\ 
% \hline
% Interpretability   & \begin{tabular}[c]{@{}c@{}}Moderate; often needs \\ external explainability tools\end{tabular}           & \begin{tabular}[c]{@{}c@{}}Variable; latent-space mappings \\ can be opaque, but structured \\ generation may retain clarity\end{tabular}      \\
% \hline
% Adaptability       & \begin{tabular}[c]{@{}c@{}}Dynamically adjusts feature \\ selection strategies \\ in streaming/online data\end{tabular} & \begin{tabular}[c]{@{}c@{}}Well-suited for static/\\semi-static data; large-scale \\ batch feature generation\end{tabular}                     \\
% \hline
% Automation         & \begin{tabular}[c]{@{}c@{}}Requires reward design, \\ but automatically learns \\ iterative feature selection\end{tabular}     & \begin{tabular}[c]{@{}c@{}}Highly automated search \\ in embedding/latent spaces, \\ yet may need domain constraints\end{tabular}               \\
% \hline
% \end{tabular}
% \end{table*}

\section{Strengths and Limitations of RL-based vs. Generative Methods}

RL-based feature engineering leverages reinforcement learning to automatically optimize feature space, improving data quality and model adaptability, while generative AI synthesizes high-quality data or augments key features to mitigate data scarcity. Both approaches enhance the core capabilities of data-centric AI, making data quality optimization a key driver of AI performance. We summarize the key differences between these two approaches in terms of performance, interpretability, adaptability, and automation.

\noindent
\textbf{Performance.}
\begin{itemize}
    \item \emph{RL-based Methods:} RL-based methods optimize feature transformations by aligning them with the final task objective, making them effective for dynamically changing data. They excel in high-dimensional datasets by capturing nonlinear relationships and adapting features automatically. However, RL has high computational costs, slow convergence, and relies on well-designed reward functions. Poor reward design can lead to suboptimal solutions, limiting effectiveness.
    \item \emph{Generative-based Methods:} Generative-based methods reformulate feature engineering as a continuous optimization problem, allowing smoother transformations and more efficient search in high-dimensional spaces. This makes them more stable than RL in some cases but also dependent on high-quality training data. However, generative models may introduce biases, and their training remains computationally expensive, especially for complex datasets.
\end{itemize}

\noindent
\textbf{Interpretability.}
\begin{itemize}
    \item \emph{RL-based Methods:} RL-based methods offer better interpretability because they rely on discrete decision paths, making feature transformations traceable through policy networks or Q-values. However, as feature space complexity increases, deep RL models become harder to interpret, and their effectiveness depends on task-specific reward functions, which may not always provide clear reasoning for feature selection.
    \item \emph{Generative-based Methods:} Generative-based methods optimize features in a continuous space, allowing smoother transformations and potential insights from latent variable representations. However, their interpretability is limited due to the black-box nature of deep generative models, making it difficult to directly correlate generated features with input data patterns. High-dimensional probability modeling further complicates feature importance analysis.
\end{itemize}

\noindent
\textbf{Adaptability and Automation.}
\begin{itemize}
    \item \emph{RL-based Methods:} RL-based methods exhibit strong adaptability by dynamically adjusting feature selection and transformation strategies to match different data distributions. They are well-suited for rapidly changing environments, such as time-series or streaming data. However, their automation is constrained by the need for well-designed reward functions, as poor reward design can lead to suboptimal strategies. Additionally, RL relies on discrete search, which, while capable of exploring complex feature combinations, is often slow and less efficient for large-scale feature engineering.
    \item \emph{Generative-based Methods:} Generative-based methods optimize features in a continuous space, making them highly adaptable to various data patterns, especially in unsupervised or semi-supervised scenarios. Their automation level is higher than RL since generative models can directly learn latent data structures and synthesize high-quality features without predefined search strategies. However, their adaptability depends on training data quality—if data distributions shift significantly, the generated features may become invalid or biased. Moreover, training generative models can be complex, often requiring careful tuning of architectures and hyperparameters to perform well across different tasks.
\end{itemize}
\vspace{-0.4cm}
\section{Practical Strategies for Effective Feature Engineering}
Feature engineering plays a critical role in optimizing machine learning performance, especially when employing RL-based or generative-based approaches. While these methods offer automation and adaptability, their effectiveness depends on proper implementation. This section provides key strategies for designing robust feature engineering pipelines, balancing model complexity, domain expertise, explainability, scalability, and ethical considerations.

\noindent
\textbf{1) Start Simple Before Adopting Advanced Models.}
A common mistake in feature engineering is jumping directly to complex RL policies or high-capacity generative models. Instead, it is more effective to begin with straightforward methods and gradually increase complexity:
\begin{itemize}
    \item \emph{RL Starter Policy:} An initial policy can rely on simple reward signals, such as validation accuracy, before incorporating more sophisticated constraints like feature cost or domain-specific penalties. This stepwise refinement ensures that early-stage learning remains stable while progressively aligning with real-world objectives.
    \item \emph{Generative Bootstrapping:} it is beneficial to first apply basic transformations, such as linear autoencoders, to understand fundamental data patterns. More advanced architectures like GANs or Transformer-based generative models should only be introduced once the baseline has demonstrated meaningful feature improvements.
\end{itemize}

\noindent
\textbf{2) Incorporate Domain Knowledge for Smarter Decision-Making.}
While automation reduces manual effort, human expertise remains crucial for guiding feature engineering:
\begin{itemize}
    \item In RL, domain experts can shape the agent’s reward function and define an action space that prioritizes meaningful features. This prevents RL from selecting redundant or irrelevant transformations, improving both efficiency and interpretability.
    \item For generative-based approaches, experts can constrain latent variables or enforce certain transformation rules, ensuring that generated features align with practical needs. This is particularly important in fields like healthcare and finance, where feature validity must comply with industry regulations.
\end{itemize}

\noindent
\textbf{3) Ensure Continuous Validation and Interpretability.}
Feature engineering should be an iterative process, with ongoing evaluation of its impact on downstream tasks:
\begin{itemize}
    \item Newly derived features should be tested for their contribution to model accuracy, robustness, and generalization. Monitoring feature utilization can help identify redundant or misleading transformations.
    \item Interpretability tools such as local surrogate models (e.g., LIME~\cite{KDD@LIME}) or attribution techniques help explain how RL-based policies select features or how generative models manipulate latent spaces. This is particularly crucial in high-risk applications where transparency is required.
\end{itemize}

\noindent
\textbf{4) Address Scalability and Computational Constraints.}
Large-scale datasets introduce computational challenges, making it necessary to optimize resource allocation:
\begin{itemize}
    \item For RL-based methods, high-dimensional action spaces can lead to slow exploration and increased training costs. Techniques like hierarchical RL or offline RL can mitigate this issue by reducing search complexity and improving efficiency.
    \item For generative models, training on vast datasets with high-dimensional features can be computationally expensive. Instead of processing entire datasets, a more practical approach is to train on smaller subsets or compressed representations before scaling to full-sized data.
\end{itemize}

\noindent
\textbf{5) Maintain Ethical Standards and Regulatory Compliance.}
Automated feature engineering must align with privacy and ethical guidelines when dealing with sensitive data:
\begin{itemize}
    \item Privacy-preserving strategies should be applied when using generative models for sensitive datasets. Techniques such as differential privacy~\cite{baiprivacy} or federated learning~\cite{kamatchi2025securing} can prevent the unintentional leakage of confidential information.
    \item Regulatory compliance requires transparency—RL-based feature selection may produce transient or dynamic features that are difficult to track. Maintaining clear logs of selected transformations is essential for auditing, particularly in regulated industries.
\end{itemize}
\vspace{-0.2cm}
\section{Choosing RL-Based vs. Generative AI}
Selecting the right feature engineering approach depends on various factors, including the nature of the data, computational constraints, and the specific optimization objectives. RL-based methods and generative models offer distinct advantages, making them suitable for different scenarios. This section outlines key considerations for choosing between these two approaches and explores how they can be effectively combined.

\noindent
\textbf{RL-based Feature Engineering for Dynamic and Sequential Optimization.}
Reinforcement learning is particularly well-suited for scenarios where feature importance evolves over time or where decisions must be made sequentially. When working with streaming data or environments that require continuous adaptation, RL agents can iteratively refine feature transformations based on real-time feedback. This makes RL effective in applications such as online learning systems, reinforcement-based recommendation models, and adaptive data preprocessing pipelines. Additionally, RL-based methods excel in optimization tasks where incremental improvements are necessary, ensuring that feature selection aligns with long-term performance gains.

\noindent
\textbf{Generative AI for Large-Scale Feature Exploration and Static Datasets.}
Generative models are more appropriate when dealing with high-dimensional datasets where exploring numerous feature transformations simultaneously is beneficial. Unlike RL, which searches in a discrete space, generative models optimize features in a continuous space, allowing for smoother transformations and better scalability. These methods are particularly useful in batch-processing environments, where they can generate diverse feature representations that are later filtered and refined. Use cases include data augmentation, synthetic feature generation, and unsupervised discovery of latent patterns, making them effective for improving model performance when labeled data is limited.

\noindent
\textbf{Hybrid Approaches: Leveraging the Strengths of Both Methods.}
In some cases, combining RL-based and generative approaches can lead to superior performance. One strategy involves using generative models to create a large pool of candidate features, which an RL agent then evaluates and selects dynamically. Conversely, RL can optimize feature selection in streaming applications, while a generative model periodically retrains in an offline setting to propose new feature transformations. This hybrid strategy balances adaptability with computational efficiency, ensuring both short-term flexibility and long-term feature diversity.

\noindent
\textbf{Key Considerations for Choosing the Right Approach.}
The decision to use RL-based or generative feature engineering should be guided by several practical factors:
\begin{itemize}
    \item \emph{Nature of the Data:} If data is continuously changing, RL is the better choice. If the dataset is static or semi-static, generative methods may be more effective.
    \item \emph{Feature Space Complexity:} RL is useful when feature selection requires careful, iterative refinement. Generative AI is more efficient for searching through a vast feature space.
    \item \emph{Computational Resources:} RL training can be expensive due to the need for extensive exploration. Generative models, while also computationally demanding, can often be trained in a more controlled offline manner.
    \item \emph{Interpretability Requirements:} RL’s decision-making process is more transparent since it follows a structured selection path, whereas generative models may introduce complex transformations that are harder to explain.
\end{itemize}
