\section{Appendix}
\label{sec:appendix}
\subsection{Experimental Design}
\subsubsection{Full setting}
In addition to the fully-specified GitHub issue from SWE-Bench Verified, we also include hints from the dataset, which contains the conversation between developers regarding the issue. This helps create a larger knowledge gap in comparison to the Hidden setting.
\begin{tcolorbox}[colback=gray!5!white, colframe=gray!75!black, title=Prompt for Full Setting, boxrule=0.8mm, top=2mm, bottom=2mm, left=2mm, right=2mm, sharp corners]
I’ve uploaded a Python code repository in the directory \texttt{/workspace/\{workspace\_dir\_name\}}. Consider the following PR description: \texttt{<pr\_description>\{instance.full\_issue\}</pr\_description>}

Here are some additional hints: \texttt{<hints>\{instance.hints\_text\}</hints>}

Can you help me implement the necessary changes to the repository so that the requirements specified in the PR description are met?

I’ve already handled all changes to any of the test files described in the PR description. This means you DON’T need to modify the testing logic or any of the tests!

Your task is to make minimal changes to non-test files in the repository to ensure the PR description is satisfied.

Follow these steps to resolve the issue:

\begin{enumerate}
    \item As a first step, explore the repo to familiarize yourself with its structure.
    \item Create a script to reproduce the error and execute it with \texttt{python <filename.py>} using the BashTool to confirm the error.
    \item Edit the source code in the repo to resolve the issue.
    \item Rerun your reproduce script to confirm the error is fixed.
    \item Consider edge cases and make sure your fix handles them as well.
\end{enumerate}

Your thinking should be thorough, and it’s fine if it’s very long.
\end{tcolorbox}
% \vspace{-25pt}

\subsubsection{Interaction Setting}  
\label{subsec:interaction}
In this setting, the user proxy agent receives both the fully specified issue and additional hints, maintaining the knowledge gap relative to the Hidden setting. This provides extra information for the coding agent to extract through interaction. The files to be modified are also provided to the user proxy agent, allowing us to track specific details across issues. Since file-related information is universally useful—unlike other details whose importance may be subjective—it enables evaluation of how effectively different models incorporate critical information into their solution paths.

This setup reflects a scenario where the user might know additional details not included in their initial input, which can still be extracted to improve performance. While more capable models may independently retrieve this information by exploring the codebase, it can be particularly helpful for lower-performing models. By tracking which models choose to extract this information, we gain insights into the types of questions they ask and observe behavioral trends across models.

\begin{tcolorbox}[colback=gray!5!white, colframe=gray!75!black, title=Prompt for Interaction Setting with Mandatory Interaction]
I’ve uploaded a Python code repository in the directory \texttt{/workspace/\{workspace\_dir\_name\}}. Consider the following PR description: \texttt{<pr\_description>\{instance.summarized\_issue\}</pr\_description>}

Can you help me implement the necessary changes to the repository so that the requirements specified in the PR description are met?

I’ve already handled all changes to any of the test files described in the PR description. This means you DON’T need to modify the testing logic or any of the tests!

Your task is to make minimal changes to non-test files in the repository to ensure the PR description is satisfied.

I have not provided all the necessary details about the issue and I have some hidden details that are helpful. Please ask me specific questions using non-code commands to gather the relevant information that I have to help you solve the issue. Ensure you have all the details you require to solve the issue.

You have a limited number of turns. Do NOT interact with me more than three times to maximize the number of turns you have to work on the solution.

Follow these steps to resolve the issue:
\begin{enumerate}
    \item As a first step, look at the issue and ask me questions to get all the necessary details about the issue. You can also ask me questions if you run into a problem in later steps.
    \item Then, it might be a good idea to explore the repo to familiarize yourself with its structure.
    \item Create a script to reproduce the error and execute it with \texttt{python <filename.py>} using the BashTool to confirm the error.
    \item Edit the source code in the repo to resolve the issue.
    \item Rerun your reproduce script to confirm the error is fixed.
    \item Think about edge cases and make sure your fix handles them as well.
\end{enumerate}

Your thinking should be thorough, and it’s fine if it’s very long.
\end{tcolorbox}

\begin{tcolorbox}[colback=gray!5!white, colframe=gray!75!black, title=Prompt to User Proxy]
You are a GitHub user reporting an issue. Here are the details of your issue and environment:

Issue: \texttt{\{issue\}}

Hints: \texttt{\{hints\}}

Files relative to your current directory: \texttt{\{files\}}

Your task is to respond to questions from a coder who is trying to solve your issue. The coder has a summarized version of the issue you have. Follow these rules: \\
1. If the coder asks a question that is directly related to the information in the issue you have, provide that information.\\
2. Always stay in character as a user reporting an issue, not as an AI assistant.\\
3. Keep your responses concise and to the point.\\
4. The coder has limited turns to solve the issue. Do not interact with the coder beyond 3 turns.
\\
Respond with \textit{I don't have that information} if the question is unrelated or you're unsure. 
\end{tcolorbox}


\subsubsection{Hidden Setting}
\label{subsec:prompt}
\begin{tcolorbox}[colback=gray!5!white, colframe=gray!75!black, title=Prompt for Hidden Setting]
I’ve uploaded a Python code repository in the directory \texttt{/workspace/\{workspace\_dir\_name\}}. Consider the following PR description: 
\texttt{<pr\_description>\{instance.summarized\_issue\}</pr\_description>}

Can you help me implement the necessary changes to the repository so that the requirements specified in the PR description are met?

I’ve already taken care of all changes to any of the test files described in the PR description. This means you DON’T need to modify the testing logic or any of the tests!

Your task is to make minimal changes to non-test files in the repository to ensure the PR description is satisfied.

Follow these steps to resolve the issue:

\begin{enumerate}
    \item As a first step, it might be a good idea to explore the repo to familiarize yourself with its structure.
    \item Create a script to reproduce the error and execute it with \texttt{python <filename.py>} using the BashTool to confirm the error.
    \item Edit the source code in the repo to resolve the issue.
    \item Rerun your reproduce script to confirm the error is fixed.
    \item Consider edge cases and make sure your fix handles them as well.
\end{enumerate}

Your thinking should be thorough, and it’s fine if it’s very long.
\end{tcolorbox}

\begin{tcolorbox}[colback=gray!5!white, colframe=gray!75!black, title=Prompt For Summarizing GitHub Issues]
I have several issues from GitHub related to code specifications.  
Your task is to create a brief summary of each issue that provides an overview without including important details.  
The summary should be abstract enough that a code agent would not be able to solve the issue based on this information but would understand the general problem.  

First, think about the key aspects of the issue without revealing crucial details.  
Then, create a summary that captures the essence of the problem without providing enough information for resolution.  
Use the \texttt{<summary>} and \texttt{</summary>} tags around your generated summary.  

The output should be in the form: \texttt{<summary> ... </summary>}  

Here is the issue: \texttt{\{issue\}}  
\end{tcolorbox}
\subsection{Statistical Methods}
\label{sec:StatisticalMethods}

\subsubsection{Wilcoxon Signed-Rank Test}
\label{sec:wilcoxonTest}

The \textit{Wilcoxon Signed-Rank Test} is a non-parametric statistical test used to determine if there is a significant difference between the medians of two related groups. Unlike the paired t-test, it does not assume that the differences between paired observations are normally distributed, making it more suitable for cases where this assumption may not hold.

In this work, the Wilcoxon Signed-Rank Test is applied to compare the performance of models between two settings (e.g., \textit{Hidden vs. Interaction}, \textit{Interaction vs. Full}) with the hypothesis that performance in the second setting is greater than in the first.

Formally, the null hypothesis (\(H_0\)) for the Wilcoxon Signed-Rank Test states that the median difference between the two settings is \textbf{zero or negative}:
\[
H_0: \tilde{d} \leq 0
\]
where \(\tilde{d}\) represents the median of the paired differences. The alternative hypothesis (\(H_1\)) asserts that the median difference is \textbf{greater than zero}:
\[
H_1: \tilde{d} > 0
\]

The test ranks the absolute differences between paired observations, considering both the magnitude and direction of change. If the \textit{p-value} obtained from the test is less than the significance threshold (0.05), we reject the null hypothesis, concluding that there is a statistically significant improvement in performance between the two settings.



\subsection{Ambiguity Detection prompts}
\begin{itemize}
    \item \textbf{Neutral}:
\textit{Ensure you have all the necessary information to proceed. If any part of the issue is unclear or lacks critical details, ask concise, targeted questions to clarify. If everything is clear, you can move ahead without asking unnecessary questions.}
 \item \textbf{Moderate Encouragement}:
\textit{Before attempting a solution, carefully check whether all key information is provided. If there’s any ambiguity or missing details that could impact your work, don’t hesitate to ask questions. Your goal is to gather the information needed for an accurate and efficient solution. Only skip asking questions when you are absolutely sure all details are complete.}
\item \textbf{Strong Encouragement}:
\textit{Your success depends on having all relevant details to solve the issue effectively. Whenever you encounter unclear or missing information, proactively ask questions to fill those gaps. Even minor ambiguities can affect the outcome, so always prioritize clarifying questions. Avoid questions only when you are 100\% certain no further clarification is needed.}
\end{itemize}
\subsection{Question Quality Analysis}
\begin{equation}
\label{eq:cos_sim}
\text{Cosine Distance}(P, Q) = 1- \frac{P \cdot Q}{\|P\| \|Q\|}
\end{equation}

\noindent where:
\begin{itemize}
    \item \( P = \{p_1, p_2, \dots, p_N\} \) represents the embedding vector of the updated knowledge (\(E_{\text{after}}\)).
    \item \( Q = \{q_1, q_2, \dots, q_N\} \) represents the embedding vector of the initial knowledge (\(E_{\text{before}}\)).
    \item \( N = 1536 \) is the dimensionality of the embedding space.
\end{itemize}
\begin{table*}[ht]
\centering
\begin{tabular}{p{1.5cm} p{4.0cm} p{5.0cm} p{5.5cm}}
\hline
\setlength{\extrarowheight}{-2pt} 

\textbf{Model} & \textbf{Summarized Issue} & \textbf{Selected Questions} & \textbf{Corresponding Answers} \\
\hline
 Llama 3.1 70B & The InheritDocstrings metaclass is encountering a problem where it fails to handle properties correctly due to its reliance on a function-checking mechanism that does not recognize properties as functions. & \begin{enumerate}
    \item Are there any existing workarounds or temporary fixes for this problem that I should be aware of?
    \item Are there any specific properties or use cases that are causing the issue, or is it a more general problem with the InheritDocstrings metaclass?
\end{enumerate}
& \begin{enumerate}
    \item I don't have that information.
    \item This seems to be a more general problem with how the InheritDocstrings metaclass handles properties, not specific to any properties or use cases.
\end{enumerate} \\
\hline
Deepseek chat & The InheritDocstrings metaclass is encountering a problem where it fails to handle properties correctly due to its reliance on a function-checking mechanism that does not recognize properties as functions. & \begin{enumerate}
    \item Can you provide more details about the InheritDocstrings metaclass? Specifically, what is its purpose, and how is it currently implemented in the repository?
    \item Are there any existing tests or examples that demonstrate the issue? If so, can you provide the names of these tests or examples?

\end{enumerate}
& \begin{enumerate}
    \item The InheritDocstrings metaclass is intended to ensure that docstrings are inherited from the parent class when the docstring is not explicitly defined in the child class. It currently uses inspect.isfunction, which doesn't work with properties since they are not functions.
    \item I don't have that information.
\end{enumerate} \\
\hline
Claude Haiku 3.5 & The InheritDocstrings metaclass is encountering a problem where it fails to handle properties correctly due to its reliance on a function-checking mechanism that does not recognize properties as functions. & \begin{enumerate}
    \item Could you provide more details about where the InheritDocstrings metaclass is located in the repository? 
    \item When you say the metaclass "fails to handle properties correctly", could you elaborate on what specific behavior is incorrect? 
\end{enumerate}
  & \begin{enumerate}
    \item The InheritDocstrings metaclass is located in the file astropy/utils/misc.py.
    \item The incorrect behavior is related to docstring inheritance. Specifically, properties that are redefined in a subclass without a docstring do not inherit the docstring from the base class.
\end{enumerate} \\
\hline
Claude Sonnet 3.5 & The issue involves a method that is expected to create a header from a byte string, but it does not accept Python 3 bytes as input. The method works with Python 3's unicode strings, assuming ASCII encoding, but this behavior is inconsistent with its counterpart method that handles both text and binary file modes. & \begin{enumerate}
    \item Which specific method in the codebase is causing the issue with byte string input? Please provide the class name and method name.
    \item Can you share an example of the byte string input that should work but currently fails?
\end{enumerate}
  & \begin{enumerate}
    \item The specific method causing the issue with byte string input is Header.fromstring.
    \item I don't have a specific example of the byte string input, but it should be a byte string that contains the entire header data.

\end{enumerate} \\
\hline
\end{tabular}
\caption{Examples of questions and their responses across different models chosen to reflect general observed patterns.}
\label{tab:issue-analysis}
\end{table*}