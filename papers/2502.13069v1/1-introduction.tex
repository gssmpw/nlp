\section{Introduction}
\label{sec:Introduction}


Large Language Models (LLMs) are increasingly used as chatbots in task-oriented workflows to improve productivity~\cite{peng2023impactaideveloperproductivity, NBERw31161}, with the user providing a task instruction which the model completes. 
Due to the interactive nature of chatbots, the performance depends on the information provided in the user's prompt. Users often provide non-descriptive instructions, which poses critical challenges in successfully completing the task~\cite{SWE_Bench_Verified}. The ambiguity can lead not only to erroneous outcomes, but also to significant safety issues~\cite{kim2024aligninglanguagemodelsexplicitly, karli2023extended}. 


This ambiguity can lead to more severe consequences in task automation scenarios, where AI agents are equipped with powerful tools~\cite{wang2024openhandsopenplatformai,lu2024aiscientistfullyautomated,huang2024agentcodermultiagentbasedcodegeneration,zhou2024haicosystem}. In software engineering settings, agents must navigate complex codebases, make architectural decisions, and modify critical systems—all while operating with potentially incomplete or ambiguous instructions. When human developers face such ambiguity, they engage in clarifying dialogue to gather missing context~\cite{testoni_humansaskquestions, purver2004clarification}. However, current AI systems often proceed with incomplete understanding, leading to costly mistakes and misaligned solutions, as demonstrated in Figure~\ref{fig:overview}. 

In this work, we systematically evaluate the interaction capabilities of commonly used open and proprietary LLMs when addressing underspecified instructions in agentic code settings (\S\ref{sec:Method}). We examine three research questions to address the problem for code generation.
\begin{enumerate}[itemsep=0pt, topsep=0pt]
    \item \textbf{Interactive problem solving}: Can LLMs appropriately leverage interaction with the user to improve performance in ambiguous settings?
    \item \textbf{Detection of ambiguity}: Can LLMs identify whether a given task description is underspecified and ask clarifying questions?
    \item \textbf{Question quality}: Can LLMs generate meaningful and targeted questions that gather the necessary information to complete the task?
\end{enumerate}
\begin{figure*}[!t]
    \centering
    \includegraphics[width=\textwidth]{revised_settings.pdf}
    \vspace{-20pt}
    \caption{The three settings in order: Full, Hidden, and Interaction}
    \label{fig:settings}
    \vspace{-12pt}
\end{figure*}
We evaluate the research questions separately to ensure independence between them. We use the Github issues from SWE-Bench Verified~\cite{SWE_Bench_Verified} to simulate well-specified inputs, and the summarized variants of the same Github issues as underspecified inputs for the experiments. A simulated user~\cite{xu2024theagentcompanybenchmarkingllmagents, zhou2024sotopiainteractiveevaluationsocial}, equipped with the full, well-specified issue, simulates real conversations where the user has additional context, which is provided only when prompted with the appropriate questions. This multi-stage approach allows for targeted improvements in individual aspects, offering a pathway to enhance overall system performance.


 Through our evaluations across the different settings, we find that interactivity can boost performance on underspecified inputs by up to 74\% over the non-interactive settings but the performance varies between models~(\secref{sec:ProblemSolving}). LLMs default to non-interactive behavior without explicit encouragement, and even with it, they struggle to distinguish between underspecified and well-specified inputs. Claude Sonnet 3.5 is the only evaluated LLM that achieves notable accuracy (84\%) in making this distinction. Prompt engineering offers limited improvement, and its effectiveness varies across models~(\secref{sec:AmbiguityDetection}). When interacting, LLMs generally pose questions capable of extracting relevant details, but some models, such as Llama 3.1 70B, fail to obtain sufficient specificity~(\secref{sec:QuestionQuality}). In summary, this study underscores the importance of interactivity in LLMs for agentic workflows, particularly in real-world tasks where prompt quality varies significantly.

\vspace{-10pt}