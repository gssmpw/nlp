\section{Related Work}
\label{sec:related}

\textbf{Score Matching ANMs\ \ } A recent line of work also use signatures in the score function for causal discovery in ANMs. ____ use properties of Gaussian noise to derive conditions on the score that are satisfied by non-cause variables. ____ show that even under arbitrary distributions the score is a deterministic function of the noise variables, and fit the model using nonparametric regression to evaluate this condition. These methods are restricted to the ANM case, but are focused on sink node (non-cause) identification in multivariate SCMs, while our focus is on more general model classes in the bivariate setting. Note the continuity equation \eqref{eq:cond:score:continuity} simplifies for ANMs, we show how to recover the identities used in this line of work in \cref{sec:appendix:scorebasedanm}.

\textbf{Cocycles in Causal Modeling\ \ }
Connections between causal models and flows of dynamical systems have been previously established by ____ in a more general setting, using a class of maps called cocycles. A two-parameter flow is an example of a cocycle. ____ focused on distribution-robust inference in a known causal graph with multiple cause variables, rather than bivariate discovery. Furthermore, they do not analyze the cocycle as a dynamical system nor make any connection to the score function.  

\textbf{Continuous Normalizing Flows\ \ } A popular class of velocity-based probabilistic model are  continuous normalizing flows (CNF) ____. There, instead of conditioning, time is an auxiliary variable, and the generative model is for a marginal density $\varphi_{0,1}(y_0)$, where $y_0$ is drawn from an arbitrary base distribution. Our learning objective \eqref{eqn:loss_est}, which targets conditional distributions instead, is similar to recently proposed simulation-free objectives, which attempt to target the velocity directly in CNFs, e.g., Flow Matching ____. Normalizing flows have been used for causal inference ____ with a known causal graph. The causal auto-regressive flow model ____ when restricted to the bivariate case represents a flexible LSNM, and was used for likelihood-based causal discovery.


\begin{figure}
    \centering
    \includegraphics[width=1\linewidth]{score_estim_fig.pdf}
    \vspace{-20pt}
    \caption{We test our velocity parametrization on a well-specified synthetic data generated from an LSNM with Gaussian noise. Top left: the ground truth velocity and causal curves. Top right: the estimated velocity when given the analytically computed ground truth scores. Bottom: estimated scores. Note that the curves from estimated scores tend to ignore variation in the tails (in the $y$ direction). On the other hand, estimating the velocity under the true score is very effective at recovering the underlying SCM.}
    \label{fig:score}
    \vspace{-5pt}
\end{figure}