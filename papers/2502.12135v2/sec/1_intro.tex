% \vspace{-10pt}
\section{Introduction}
\label{sec:intro}

\begin{figure*}
    \centering
\includegraphics[width=\textwidth]
    % \includegraphics[scale=0.5]
    {fig/teaser.pdf}
    % \vspace{-25pt}
    \caption{\textbf{Given a 3D model, \ours{} can automatically generate the skeleton and skinning weights, making the model articulation-ready without further manual refinement.} The input meshes are generated by Rodin Gen-1 \cite{zhang2024clay} and Tripo 2.0 \cite{tripo3d}. The meshes and skeletons are rendered using Maya Software Renderer \cite{AutodeskMaya2024}.}
    \label{teaser}
    % \vspace{-15pt}
  \end{figure*}

The rapid advancement of 3D content creation has led to an increasing demand for articulation-ready 3D models, especially in gaming, VR/AR, and robotics simulation. Converting static 3D models into articulation-ready versions traditionally requires professional artists to manually place skeletons, define joint hierarchies and specify skinning weights, which is both time-consuming and demands significant expertise, making it a major bottleneck in modern content creation pipelines.


To address these issues, various automatic approaches for skeleton extraction have been proposed, which can be categorized into template-based \cite{baran2007automatic, li2021learning} and template-free methods \cite{xu2020rignet, xu2019predicting, huang2013l1, au2008skeleton}. Template-based methods, like Pinocchio \cite{baran2007automatic}, fit predefined skeletal templates to input shapes. While they achieve satisfactory results for specific categories like human characters, they struggle to generalize to objects with varying structural patterns. Moreover, these methods mostly rely on distance metrics between joints and vertices for skinning weight prediction, which often fail on shapes with complex topology. Many template-free methods \cite{huang2013l1, au2008skeleton, cao2010point, lin2021point2skeleton, tagliasacchi2012mean} extract curve skeletons from meshes or point clouds using shape medial axis or the centerline of shapes, but often produce densely packed joints that are unsuitable for animation.
Recent deep learning methods like RigNet \cite{xu2020rignet} have shown promise in predicting skeletons and skinning weights directly from input shapes. However, they rely heavily on carefully crafted features and make strong assumptions about shape orientation, limiting their ability to handle diverse object categories. These limitations stem from two fundamental challenges: the lack of a large-scale, diverse dataset for training generalizable models, and the inherent difficulty in designing an effective framework capable of handling complex mesh topologies, accommodating varying skeleton structures, and ensuring the coherent generation of both accurate skeletons and skinning weights.

To overcome these challenges, we first introduce Articulation-XL, a large-scale dataset containing over 33k 3D models with high-quality articulation annotations carefully curated from Objaverse-XL \cite{deitke2023objaverse, deitke2024objaverse}. Built upon this benchmark, we propose MagicArticulate, a novel framework that addresses both skeleton generation and skinning weight prediction. Specifically, we reformulate skeleton generation as an auto-regressive sequence modeling task, enabling our model to naturally handle varying numbers of bones or joints within skeletons across different 3D models. For skinning weight prediction, we develop a functional diffusion framework that learns to generate smoothly transitioning skinning weights over mesh surfaces by incorporating volumetric geodesic distance priors between vertices and joints, effectively handling complex mesh topologies that challenge traditional geometric-based methods. These designs demonstrate superior scalability on large-scale datasets and generalize well across diverse object categories, without requiring assumptions about shape orientation or topology.

Extensive experiments on our Articulation-XL and \res{} \cite{ModelsResource2019} collected by Xu et al. \cite{xu2019predicting, xu2020rignet}
demonstrate the effectiveness of MagicArticulate in both skeleton generation and skinning weight prediction. The proposed methods also generalize well to 3D models from various sources, including artist-created assets, and models generated by AI techniques. With the generated skeleton and skinning weights, our method automatically creates ready-to-animate assets that support natural pose manipulation without manual refinement (\Cref{teaser}), particularly beneficial for large-scale animation content creation. 

Our key contributions include: (1) The first large-scale articulation benchmark containing over 33k models with high-quality articulation annotations; (2) A novel two-stage framework that effectively handles both skeleton generation and skinning weight prediction; (3) State-of-the-art performance and demonstrated practicality in real-world animation pipelines.
