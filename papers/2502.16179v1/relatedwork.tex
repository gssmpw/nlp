\section{Related Work}
LPWANs technologies have captured tremendous research interest in investigating the performance of different technologies under interference from the LEO satellite scenario. \cite{temim2022low,wang2022performance} concluded the feasibility of deployment with LEO satellite and performance of LoRa, SigFox, and NB-IoT. All of them can reach a $1000$ km communication range under specific settings, satisfying LEO satellite's needs. Specifically, LoRa is based on a license-free band with bandwidth such as $125$ kHz or $250$ kHz and LoRa data rate can range from as low as 0.3 kbit/s to around 50 kbit/s per channel, depending on the configuration of the spreading factor and bandwidth \cite{LoRaAlliance2020}.

As for the immunity of LoRa toward the Doppler effect, some recent research studies analyzed its sensitivity in mobility scenarios, such as vehicle-to-everything (V2X) scenario \cite{torres2021experimental, li2018lora}
and satellite-to-Earth communication scenarios \cite{doroshkin2018laboratory,doroshkin2019experimental,zadorozhny2022first,cao2021influence,colombo2022low,ameloot2021characterizing,ullah2023understanding,lapapan2021lora,ben2022new,kang2022regression}.  
\cite{torres2021experimental, li2018lora} showed that LoRa is sensitive to the Doppler effect due to the velocity and concluded that lower $SF$ has less sensitivity to the effect due to a higher reception threshold based on the results of packet delivery rate.
\cite{doroshkin2018laboratory,doroshkin2019experimental} proved that LoRa holds immunity to the Doppler effect for orbit with altitudes higher than $550$km. \cite{zadorozhny2022first} proposed that under $SF \leq 11$ and bandwidth $B> 31.25$ kHz, LoRa has high immunity towards the Doppler effect, based on the results obtained from NORBY Cubesat operating at $560$km.
\cite{cao2021influence} concluded that when $SF$ is greater than $9$, within the range of the maximum Doppler-Rate, the packet error rate will reach $100$\% as the Doppler rate increases.  
Among them, some works were carried out with TLE dataset from the satellites \cite{zadorozhny2022first}, while other researchers set up some similar scenarios or computer environments to simulate the satellite-to-Earth communication \cite{doroshkin2018laboratory,doroshkin2019experimental,cao2021influence}.

Other research papers discussed and distinguished the impact of high Doppler shift or Doppler rate. \cite{colombo2022low} proposed a low-cost SDR method for evaluating LoRa satellite communication. 
\cite{ameloot2021characterizing} evaluates the impact of Doppler shift on LoRa in body-centric applications and analyzes the performance of symbol detection levels such as bit error rate (BER) or symbol error rate (SER). 
\cite{ullah2023understanding} analyzed the impact of the high Doppler shift or Doppler rate based on packet losses of LoRa in a Direct-to-Satellite (DtS) connectivity scenario and offered suggestions for the selection of suitable parameters to mitigate the Doppler effect based on a parameter analysis. They further demonstrated that changes in $SF$, payload length, and LoRaWAN low data rate optimization (LDRO) impact the vulnerability to the high Doppler rate more than that to the high Doppler shift.

However, the aforementioned related works demonstrated various research gaps. 
\cite{doroshkin2019experimental} lacked a thorough analysis of LoRa in terms of different parameters such as $SF$, transmission start time, etc. \cite{zadorozhny2022first} only focused on a specific setting of orbit height, which cannot be generalized. The indexes used for evaluating the performance of LoRa in the presence of Doppler effect are limited to symbol or bit error rate \cite{cao2021influence}, packet error rate \cite{colombo2022low,doroshkin2019experimental,zadorozhny2022first,ullah2023understanding}, and received signal strength indication (RSSI) \cite{lapapan2021lora}. There is still no detailed analysis focusing on the orthogonality or cross-correlation of LoRa waveforms, which is significant for assessing the immunity to interference. \cite{benkhelifa2022orthogonal,croce2017impact} had analyzed the orthogonality or cross-correlation of LoRa and proved LoRa to be nonorthogonal, but that was based on the scenario without the Doppler effect. In addition, to be general, the relative locations between the ground device and the satellite should be determined by many factors on a 3-dimensional plane. But \cite{cao2021influence,colombo2022low,ullah2023understanding} modeled the Doppler effect on a 2-dimensional plane, assuming that the satellite can always pass directly above the ground device. 
\cite{colombo2022low,ameloot2021characterizing,ullah2023understanding} provided questionable conclusions on the Doppler shift effect on LoRa performance. 
\cite{colombo2022low} showed that LoRa performance on packet error ratio is immune to the high Doppler shift, but not to the high Doppler rate. \cite{ameloot2021characterizing} confirmed that BER or SER are sensitive to the high Doppler shift, which can be eliminated if the Doppler shift does not exceed $10$\% of the bandwidth. 
\cite{ullah2023understanding} showed that the impact of high Doppler shift or Doppler rate on packet delivery ratio depends on various parameters ($SF$, payload length, orbit height, etc.).

Therefore, considering the limitations of related works mentioned above, this paper will systematically analyze the immunity of LoRa modulation under the Doppler effect in a general 3-dimensional satellite-to-Earth communication scenario given various parameters ($SF$, bandwidth, transmission start time, etc.), and analyze the impact of both high Doppler shift and high Doppler rate respectively on the orthogonality criteria and the characteristics of BER performance.