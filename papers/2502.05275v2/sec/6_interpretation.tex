\section{Failure Interpretation} \label{sec:interpretation}

\begin{figure}
    \centering
    \begin{subfigure}[b]{\linewidth}
         \centering
         \includegraphics[width=\linewidth]{fig/ship-land.pdf}
         \caption{Failure caused by spurious correlation.}
         \label{viz:ship-land}
         \vspace{0.2cm}
     \end{subfigure}
    
     \begin{subfigure}[b]{\linewidth}
         \centering
         \includegraphics[width=\linewidth]{fig/ship-sky.pdf}
         \caption{Failure caused by cross-category resemblance.}
         \label{viz:ship-sky}
     \end{subfigure}
    \caption{Failure interpretation with human-level concepts. We show the confidence scores of the top $3$ categories (left histograms) and similarity scores of the top $10$ concepts (right histograms) from \textit{CIFAR-10}. Standard methods might output overconfident misclassifications due to: (a) \textit{spurious correlation} and (b) \textit{cross-category resemblance}. Concept-level signals not only achieves better failure detection capability in such scenarios but also enables further interpretation of \textit{why} the model fails. ``auto" is short for ``automobile."}
    \label{fig:visual}
\end{figure}

ORCA not only achieves superior failure detection but also enables failure interpretation with human-level concepts.
We discuss two scenarios that cause the model to output overconfident values on misclassified samples: \textit{spurious correlation} and \textit{cross-category resemblance} (Fig.~\ref{fig:visual}). 

In the former scenario (Fig.~\ref{viz:ship-land}), the presence of a road (a spurious feature) leads the model to misclassify the ship as a land vehicle, automobile or truck. 
We demonstrate that a standard model struggles to identify such failures, resulting in a high confidence score for automobile. 
In contrast, ORCA leverages human-level concepts, offering more nuanced signals for a refined assessment of the model's confidence. 
For instance, strong responses from concepts like ``road vehicle" and ``four wheels" for automobile, and ``cargo area" and ``trailer" for truck, contribute to a significantly lower confidence. 
Furthermore, we can easily interpret \textit{why} the model makes such a prediction through concepts.

In the latter scenario (Fig.~\ref{viz:ship-sky}), the ship (sailboat) bears a resemblance to an airplane from a distance. The similarity between the sky and water also creates an illusion of the object being airborne. The top-$K$ concepts from our method exhibit strong responses to concepts associated with airplanes and birds. Analyzing this information allows us to confidently deduce that the model misclassifies the image as an airplane due to the sky-like background and the object's resemblance to an airplane.


\begin{links}
\link{Code}{https://github.com/Nyquixt/ORCA}
\end{links}
