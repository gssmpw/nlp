\section{Discussion and Future Work} \label{sec:disc}

This paper introduces the Flexible Bivariate Beta Mixture Model (FBBMM), a novel probabilistic clustering model leveraging the flexibility of the bivariate beta distribution. Experimental results show that FBBMM outperforms popular clustering algorithms such as $k$-means, MeanShift, DBSCAN, Gaussian Mixture Models, and MBMM, particularly on nonconvex clusters. Its ability to handle a wide range of cluster shapes and correlations makes it highly effective.

FBBMM offers several advantages. Its use of the beta distribution allows for flexible cluster shapes, capturing complex structures more accurately than traditional models. It supports soft clustering, assigning probabilities to data points for belonging to clusters, which is versatile for overlapping clusters. Additionally, FBBMM is generative, capable of producing new data resembling the original dataset, useful for tasks like data augmentation and simulation.

However, FBBMM has limitations, including higher computational complexity due to iterative parameter estimation. Future work could focus on improving efficiency through parallelization or better optimization strategies, extending FBBMM to multivariate data, and enhancing robustness to noise and outliers. Applying FBBMM in diverse domains such as bioinformatics and image analysis could further validate its versatility and impact.

