\section{Related Work} \label{sec:rel-work}

Clustering algorithms can be categorized into four types: centroid-based, density-based, hierarchical, and distribution-based methods. Each has its strengths and limitations, as discussed below, followed by a comparison with our proposed Flexible Bivariate Beta Mixture Model (FBBMM).

Centroid-based methods like $k$-means~\cite{macqueen1967some} are computationally efficient but assume convex clusters, making them unsuitable for nonconvex data. Density-based methods like DBSCAN~\cite{ester1996density} identify clusters of arbitrary shapes and are robust to noise but depend heavily on hyperparameter tuning. Hierarchical methods, such as agglomerative clustering~\cite{day1984efficient}, build a tree-like structure and do not require pre-specifying cluster numbers but are computationally expensive and struggle with large datasets. Distribution-based models like Gaussian Mixture Models (GMM)~\cite{reynolds2009gaussian} handle soft clustering but are limited to elliptical cluster shapes. MBMM~\cite{hsu2024multivariate} addresses this by assuming multivariate beta distributions, allowing nonconvex clusters but restricting correlations to be positive.

FBBMM overcomes these limitations by employing the flexible bivariate beta distribution, enabling it to model both convex and nonconvex clusters and handle positive and negative correlations. It supports soft clustering and is generative, capable of producing new data points for tasks like data augmentation. Although FBBMM handles bivariate data, this limitation can be mitigated using dimension reduction techniques such as PCA or autoencoders.

\begin{table*}[tb]
\centering
\caption{Comparison of Clustering Algorithms}
\label{tab:method-cmp}
\resizebox{\columnwidth}{!}{
\begin{tabular}{@{}cccccc@{}}
\toprule
\textbf{Algorithm} & \textbf{Type} & \textbf{Shape} & \textbf{Assignment} & \textbf{Noise Robustness} & \textbf{Generative} \\ \midrule
$k$-means & Centroid-based & Convex & Hard & Low & No \\
DBSCAN & Density-based & Arbitrary & Hard & High & No \\
Agglomerative & Hierarchical & Arbitrary & Hard & Medium & No \\
GMM & Distribution-based & Convex & Soft & Low & Yes \\
MBMM & Distribution-based & Flexible & Soft & Medium & Yes \\
FBBMM & Distribution-based & Flexible & Soft & Medium & Yes \\ \bottomrule
\end{tabular}
}
\end{table*}

As shown in Table~\ref{tab:method-cmp}, FBBMM's flexibility in cluster shapes and ability to handle positive and negative correlations make it a more versatile and effective clustering method compared to traditional approaches.