
For this experiment, we provide in-context examples of optimization techniques such as fusion, tiling, recompute, and asynchrony to models during kernel generation. As described in Section \ref{subsection:few-shot}, we provide three in-context examples: a fused GELU ~\cite{hendrycks2023gaussianerrorlinearunits}, a tiled matrix multiplication ~\cite{mills2024cuda}, and a minimal Flash-Attention ~\cite{dao2022flashattention, kim2024flashattention} demonstrating effective shared memory I/O management. The prompt used for this experiment is described in Appendix~\ref{appendix:few-shot-study-prompts}. The speedup of these kernels were computed over PyTorch Eager. We compare the performance of these few-shot kernels over the one-shot baseline below.

\begin{table}[h]
\centering

\label{tab:few_shot_model_baseline}
\begin{tabular}{lccccccc}
\toprule
\multicolumn{2}{c}{} & \multicolumn{3}{c}{\textbf{Baseline}} & \multicolumn{3}{c}{\textbf{Few-Shot}}\\ 
    \cmidrule(lr){3-5} \cmidrule(lr){6-8}
\textbf{Model} & \textbf{Level} & \textbf{\fast{1}} & $\textbf{\fast{0}}$ & 
\textbf{Length (chars)} &
$\textbf{\fast{1}}$ & $\textbf{\fast{0}}$ &
\textbf{Length (chars)}\\
\midrule
             & 1 & 3\%  & 27\% & 301018 & 6\%  & 27\% & 360212 \\
Llama 3.1-70B  & 2 & 0\%  & 0\% & 646403 & 0\%  & 0\% & 566668  \\
             & 3 & 0\%  & 0\% & 404596  & 0\%  & 4\% & 485332  \\
\midrule
            & 1 & 10\% & 55\% & 343995 & 6\%  & 39\% & 437768\\
OpenAI o1  & 2 & 24\% & 56\% & 381474 & 16\% & 39\% & 432800 \\
            & 3 & 12\% & 56\% & 260273 & 8\%  & 22\% & 364551 \\
\bottomrule
\end{tabular}

\caption{Comparison of the Section \ref{4.1} baseline and few-shot prompting performance across models. We examine the $\textbf{\fast{0}}$, $\textbf{\fast{1}}$, and cumulative character length of generated kernels per level.}

\end{table}
\FloatBarrier

\noindent 77\% of matrix multiplication problems in Level 1 achieves a speedup over the one-shot baseline through tiling. The runtime comparison for each GEMM variant is presented below.
\begin{table}[H]
\centering
\label{tab:few_shot_kernel_fusion_results_level1}
\begin{tabular}{lccc}
    \toprule
    \textbf{Problem Name} & \textbf{Baseline (ms)} & \textbf{Few-Shot (ms)} & \textbf{Ref Torch (ms)} \\
    \midrule
    3D Tensor Matrix Multiplication & 20.9 & 7.71 & 1.45 \\
    Matmul for Upper-Triangular Matrices & 14 & 5.39 & 2.98 \\
    Matrix Scalar Multiplication & 1.19 & 0.811 & 0.822 \\
    Standard Matrix Multiplication & 3.39 & 2.46 & 0.397 \\
    Matmul with Transposed Both & 3.44 & 2.67 & 0.412 \\
    Matmul with Transposed A & 3.61 & 2.99 & 0.384 \\
    4D Tensor Matrix Multiplication & 366 & 338 & 36 \\
    Tall Skinny Matrix Multiplication & 3.39 & 3.59 & 1.9 \\
    Matmul with Diagonal Matrices & 0.221 & 0.237 & 2.83 \\
    \bottomrule
\end{tabular}

\caption{Performance comparison of the Section \ref{4.1} baseline and few-shot prompting in level 1 matrix multiplication problems.}

\end{table}

\noindent Few-shot kernels generated for the following problems in level 2 outperformed PyTorch Eager through aggressive shared memory I/O management.
\begin{table}[H]
\centering
\label{tab:few_shot_kernel_fusion_results_level2}
\begin{tabular}{lrrr}
\toprule
\textbf{Problem Name} & \textbf{Baseline (ms)} & \textbf{Few-Shot (ms)} & \textbf{Ref Torch (ms)} \\
\midrule
Conv2d InstanceNorm Divide & 0.514 & 0.0823 & 0.0898 \\
Gemm GroupNorm Swish Multiply Swish & 0.124 & 0.0542 & 0.0891 \\
Matmul Min Subtract & 0.0651 & 0.0342 & 0.0397 \\
Matmul GroupNorm LeakyReLU Sum & 0.0935 & 0.0504 & 0.072 \\
ConvTranspose3d Swish GroupNorm HardSwish & 33.3 & 29.6 & 35.2 \\
ConvTranspose2d Mish Add Hardtanh Scaling & 0.235 & 0.209 & 0.243 \\
ConvTranspose3d Add HardSwish & 15.6 & 14.1 & 22.2 \\
ConvTranspose2d Add Min GELU Multiply & 0.365 & 0.349 & 0.4 \\
ConvTranspose2d BiasAdd Clamp Scaling Clamp... & 0.3 & 0.31 & 0.368 \\
Conv2d GroupNorm Tanh HardSwish ResidualAdd... & 0.124 & 0.129	& 0.154 \\
Conv2d ReLU HardSwish & 0.0681 & 0.0711 & 0.0768 \\
\bottomrule
\end{tabular}
\caption{Performance comparison of the Section \ref{4.1} baseline and few-shot prompting in level 2 for problems whose few-shot kernels outperform PyTorch Eager.}
\end{table}