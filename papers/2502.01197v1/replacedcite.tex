\section{RELATED WORK}
Designing and optimizing systems are often expensive, time consuming and require significant domain knowledge. As such, several researchers have drawn inspiration from nature by adopting evolutionary processes to optimize their systems. Hornby et al. ____ proposed a novel antenna designed using an evolutionary algorithm that performs better than human-designed antennas.
% add - in before sub-adjectives (it's not a human antenna)
Vujovic et al. ____ combined evolution with development stages for the evolution of real-world walking robots. The development stage simulates the growing process from infancy to adulthood for each species.

When it comes to drones, evolutionary algorithms have been successful in optimizing the controller, or the "brain", of the drones. For instance, Shamshirgaran et al. ____ performed multi-objective evolution using particle swarm optimization (PSO) and biogeography-based
optimization (BBO) to tune the parameters of a PD controller for a quadcopter. Similarly, Yazid et al. ____ used evolutionary algorithms such as genetic algorithm (GA), PSO and artificial bee colony (ABC) to tune a Takagi-Sugeno-Kang fuzzy logic controller for position control of a quadcopter. Additionally, the optimization of drone bodies has been investigated in the literature.  Du et al. ____ produced an optimized multicopter from an initial user design using convex optimization algorithms. Their algorithm took user-provided designs such as the quadcopter and pentacopter and modified the geometry for increased flight time and payload, respectively. Similarly, Carlone and Pinciroli ____ co-designed drone bodies and hardware using linear programming methods. Their algorithm selects the best components, such as actuators and sensors, to meet specific objectives and constraints. On the other hand, Bergonti et al. ____ optimized the topology and control of morphing wing drones using an evolutionary algorithm. Their simulation results showed that the evolved morphing wing drone outperforms a standard fixed-wing aircraft in terms of energy efficiency and mission time. 

While optimization of drone bodies has been investigated in the past, the most common methods involve linear and non-linear programming, while evolutionary algorithms are seldom explored. Additionally, much of the past research optimized a single objective function at a time, which restricted the diversity of drone designs. In our work, we provide an evolution-based morphology optimization of multicopters with multiple degrees of freedom. We also consider multiple objectives, with the aim of generating a diverse range of drone designs that excels in various scenarios while achieving more significant advances over the common quadcopter design.
% that/which: my bicycle that has a red saddle -> I have many and only 1 has a red saddle. / my bicycle which has a red saddle: I have only 1 and it has a red saddle.
% here: there exist drone designs that do not excel, but you refer to those THAT excel.