With the meta-statistical framework, statistical \textit{inference} becomes synonymous with \textit{inference} in machine learning, and what is hard for statistical inference reveals itself as hard to learn for our models. Our experiments reveal interesting difficulties in statistical inference. Predicting normality was the \textit{easiest} task for meta-statistical models, requiring only 50K parameters for strong generalization. Estimating mutual information, as expected, demanded significantly larger models (~1M parameters). Surprisingly, predicting the standard deviation was particularly difficult: while small models (~<10K parameters) could easily approximate sample standard deviation (descriptive), nearly 1M parameters were needed to predict the standard deviation (inferential) better than \texttt{np.std}. Training a model to predict only the corrective term also required nearly 1M parameters and yielded an estimator equivalent to directly estimating the true standard deviation, suggesting that finite-sample errors is the main driver of difficulty in this problem. This raises intriguing questions about what makes meta-statistical models work and fail: do these models implicitly perform Bayesian inference with input-dependent priors? 

Statistical inference is fundamentally constrained by irreducible errors arising from finite sample sizes, imposing inherent limits on any estimator's performance \cite{casella2024statistical}. Consequently, no meta-statistical estimators can surpass these fundamental limits. However, the approach provides a flexible framework for finding estimators with a desired bias-variance trade-off by modifying the loss function. Moreover, the approach allows for the incorporation of complex prior information through the choice of meta-distribution, effectively guiding the estimator’s behavior in a principled manner.

\xhdr{Limitations and Future Work}
While meta-statistical learning brings the advantages of machine learning to statistical inference, it also imports its challenges. A key question is the choice of meta-distribution during training—what constitutes a good meta-distribution to sample from? Additionally, the evaluation of estimators becomes more difficult; a model trained on a narrow meta-distribution might generalize poorly outside its training regime.

Interpretability is another challenge. The precise algorithm computed at inference to perform the statistical inference becomes unknown and difficult to interpret \cite{molnar2022,electronics8080832,teney2022predicting}. Also, like LLMs, meta-statistical models could exhibit unexpected failure cases and lack strict guarantees of validity. However, they also offer a promising testbed for mechanistic interpretability \cite{olah2020zoom} research: they process structured numerical inputs without tokenization, operate in a single forward pass, and construct mathematical representations rather than linguistically ambiguous ones.

Failure cases also merit further study. Models struggled when input scales exceeded training ranges and we found one case of poor generalization to one unseen distribution family (log-normal) in the standard deviation estimation task (documented in \Appref{app:failure}). In the normality test setting, we believe that standardizing the datasets would make training harder but encourage better generalization OoMD by preventing the meta-statistical estimators from picking up on spurious associations between the meta-distribution and the labels. Overall, like LLMs, these models would benefit from larger and more diverse training data. Future directions include learned row embeddings to accommodate varying input row dimensions and magnitudes, as well as scaling laws to guide the training of larger models with optimized data mixes. One limitation of this work is the focus on one-dimensional datasets to explore inference tasks in a controlled setting, but real-world inference involves high-dimensional data, where traditional estimators often struggle. Scaling to higher dimensions is a key next step, and we anticipate meta-statistical models to generalize well. In general, this work focuses on demonstrating promises of the meta-statistical perspective encouraging further efforts in crafting and evaluating learned statistical estimators with methods inspired by natural language processing.