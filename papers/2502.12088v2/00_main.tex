\documentclass{article} % For LaTeX2e
% \usepackage{iclr2025_conference,times}

\usepackage[preprint]{icml2025}
% \usepackage[accepted]{icml2025}

% Optional math commands from https://github.com/goodfeli/dlbook_notation.
%%%%% NEW MATH DEFINITIONS %%%%%

\usepackage{amsmath,amsfonts,bm}
\usepackage{derivative}
% Mark sections of captions for referring to divisions of figures
\newcommand{\figleft}{{\em (Left)}}
\newcommand{\figcenter}{{\em (Center)}}
\newcommand{\figright}{{\em (Right)}}
\newcommand{\figtop}{{\em (Top)}}
\newcommand{\figbottom}{{\em (Bottom)}}
\newcommand{\captiona}{{\em (a)}}
\newcommand{\captionb}{{\em (b)}}
\newcommand{\captionc}{{\em (c)}}
\newcommand{\captiond}{{\em (d)}}

% Highlight a newly defined term
\newcommand{\newterm}[1]{{\bf #1}}

% Derivative d 
\newcommand{\deriv}{{\mathrm{d}}}

% Figure reference, lower-case.
\def\figref#1{figure~\ref{#1}}
% Figure reference, capital. For start of sentence
\def\Figref#1{Figure~\ref{#1}}
\def\twofigref#1#2{figures \ref{#1} and \ref{#2}}
\def\quadfigref#1#2#3#4{figures \ref{#1}, \ref{#2}, \ref{#3} and \ref{#4}}
% Section reference, lower-case.
\def\secref#1{section~\ref{#1}}
% Section reference, capital.
\def\Secref#1{Section~\ref{#1}}
% Reference to two sections.
\def\twosecrefs#1#2{sections \ref{#1} and \ref{#2}}
% Reference to three sections.
\def\secrefs#1#2#3{sections \ref{#1}, \ref{#2} and \ref{#3}}
% Reference to an equation, lower-case.
\def\eqref#1{equation~\ref{#1}}
% Reference to an equation, upper case
\def\Eqref#1{Equation~\ref{#1}}
% A raw reference to an equation---avoid using if possible
\def\plaineqref#1{\ref{#1}}
% Reference to a chapter, lower-case.
\def\chapref#1{chapter~\ref{#1}}
% Reference to an equation, upper case.
\def\Chapref#1{Chapter~\ref{#1}}
% Reference to a range of chapters
\def\rangechapref#1#2{chapters\ref{#1}--\ref{#2}}
% Reference to an algorithm, lower-case.
\def\algref#1{algorithm~\ref{#1}}
% Reference to an algorithm, upper case.
\def\Algref#1{Algorithm~\ref{#1}}
\def\twoalgref#1#2{algorithms \ref{#1} and \ref{#2}}
\def\Twoalgref#1#2{Algorithms \ref{#1} and \ref{#2}}
% Reference to a part, lower case
\def\partref#1{part~\ref{#1}}
% Reference to a part, upper case
\def\Partref#1{Part~\ref{#1}}
\def\twopartref#1#2{parts \ref{#1} and \ref{#2}}

\def\ceil#1{\lceil #1 \rceil}
\def\floor#1{\lfloor #1 \rfloor}
\def\1{\bm{1}}
\newcommand{\train}{\mathcal{D}}
\newcommand{\valid}{\mathcal{D_{\mathrm{valid}}}}
\newcommand{\test}{\mathcal{D_{\mathrm{test}}}}

\def\eps{{\epsilon}}


% Random variables
\def\reta{{\textnormal{$\eta$}}}
\def\ra{{\textnormal{a}}}
\def\rb{{\textnormal{b}}}
\def\rc{{\textnormal{c}}}
\def\rd{{\textnormal{d}}}
\def\re{{\textnormal{e}}}
\def\rf{{\textnormal{f}}}
\def\rg{{\textnormal{g}}}
\def\rh{{\textnormal{h}}}
\def\ri{{\textnormal{i}}}
\def\rj{{\textnormal{j}}}
\def\rk{{\textnormal{k}}}
\def\rl{{\textnormal{l}}}
% rm is already a command, just don't name any random variables m
\def\rn{{\textnormal{n}}}
\def\ro{{\textnormal{o}}}
\def\rp{{\textnormal{p}}}
\def\rq{{\textnormal{q}}}
\def\rr{{\textnormal{r}}}
\def\rs{{\textnormal{s}}}
\def\rt{{\textnormal{t}}}
\def\ru{{\textnormal{u}}}
\def\rv{{\textnormal{v}}}
\def\rw{{\textnormal{w}}}
\def\rx{{\textnormal{x}}}
\def\ry{{\textnormal{y}}}
\def\rz{{\textnormal{z}}}

% Random vectors
\def\rvepsilon{{\mathbf{\epsilon}}}
\def\rvphi{{\mathbf{\phi}}}
\def\rvtheta{{\mathbf{\theta}}}
\def\rva{{\mathbf{a}}}
\def\rvb{{\mathbf{b}}}
\def\rvc{{\mathbf{c}}}
\def\rvd{{\mathbf{d}}}
\def\rve{{\mathbf{e}}}
\def\rvf{{\mathbf{f}}}
\def\rvg{{\mathbf{g}}}
\def\rvh{{\mathbf{h}}}
\def\rvu{{\mathbf{i}}}
\def\rvj{{\mathbf{j}}}
\def\rvk{{\mathbf{k}}}
\def\rvl{{\mathbf{l}}}
\def\rvm{{\mathbf{m}}}
\def\rvn{{\mathbf{n}}}
\def\rvo{{\mathbf{o}}}
\def\rvp{{\mathbf{p}}}
\def\rvq{{\mathbf{q}}}
\def\rvr{{\mathbf{r}}}
\def\rvs{{\mathbf{s}}}
\def\rvt{{\mathbf{t}}}
\def\rvu{{\mathbf{u}}}
\def\rvv{{\mathbf{v}}}
\def\rvw{{\mathbf{w}}}
\def\rvx{{\mathbf{x}}}
\def\rvy{{\mathbf{y}}}
\def\rvz{{\mathbf{z}}}

% Elements of random vectors
\def\erva{{\textnormal{a}}}
\def\ervb{{\textnormal{b}}}
\def\ervc{{\textnormal{c}}}
\def\ervd{{\textnormal{d}}}
\def\erve{{\textnormal{e}}}
\def\ervf{{\textnormal{f}}}
\def\ervg{{\textnormal{g}}}
\def\ervh{{\textnormal{h}}}
\def\ervi{{\textnormal{i}}}
\def\ervj{{\textnormal{j}}}
\def\ervk{{\textnormal{k}}}
\def\ervl{{\textnormal{l}}}
\def\ervm{{\textnormal{m}}}
\def\ervn{{\textnormal{n}}}
\def\ervo{{\textnormal{o}}}
\def\ervp{{\textnormal{p}}}
\def\ervq{{\textnormal{q}}}
\def\ervr{{\textnormal{r}}}
\def\ervs{{\textnormal{s}}}
\def\ervt{{\textnormal{t}}}
\def\ervu{{\textnormal{u}}}
\def\ervv{{\textnormal{v}}}
\def\ervw{{\textnormal{w}}}
\def\ervx{{\textnormal{x}}}
\def\ervy{{\textnormal{y}}}
\def\ervz{{\textnormal{z}}}

% Random matrices
\def\rmA{{\mathbf{A}}}
\def\rmB{{\mathbf{B}}}
\def\rmC{{\mathbf{C}}}
\def\rmD{{\mathbf{D}}}
\def\rmE{{\mathbf{E}}}
\def\rmF{{\mathbf{F}}}
\def\rmG{{\mathbf{G}}}
\def\rmH{{\mathbf{H}}}
\def\rmI{{\mathbf{I}}}
\def\rmJ{{\mathbf{J}}}
\def\rmK{{\mathbf{K}}}
\def\rmL{{\mathbf{L}}}
\def\rmM{{\mathbf{M}}}
\def\rmN{{\mathbf{N}}}
\def\rmO{{\mathbf{O}}}
\def\rmP{{\mathbf{P}}}
\def\rmQ{{\mathbf{Q}}}
\def\rmR{{\mathbf{R}}}
\def\rmS{{\mathbf{S}}}
\def\rmT{{\mathbf{T}}}
\def\rmU{{\mathbf{U}}}
\def\rmV{{\mathbf{V}}}
\def\rmW{{\mathbf{W}}}
\def\rmX{{\mathbf{X}}}
\def\rmY{{\mathbf{Y}}}
\def\rmZ{{\mathbf{Z}}}

% Elements of random matrices
\def\ermA{{\textnormal{A}}}
\def\ermB{{\textnormal{B}}}
\def\ermC{{\textnormal{C}}}
\def\ermD{{\textnormal{D}}}
\def\ermE{{\textnormal{E}}}
\def\ermF{{\textnormal{F}}}
\def\ermG{{\textnormal{G}}}
\def\ermH{{\textnormal{H}}}
\def\ermI{{\textnormal{I}}}
\def\ermJ{{\textnormal{J}}}
\def\ermK{{\textnormal{K}}}
\def\ermL{{\textnormal{L}}}
\def\ermM{{\textnormal{M}}}
\def\ermN{{\textnormal{N}}}
\def\ermO{{\textnormal{O}}}
\def\ermP{{\textnormal{P}}}
\def\ermQ{{\textnormal{Q}}}
\def\ermR{{\textnormal{R}}}
\def\ermS{{\textnormal{S}}}
\def\ermT{{\textnormal{T}}}
\def\ermU{{\textnormal{U}}}
\def\ermV{{\textnormal{V}}}
\def\ermW{{\textnormal{W}}}
\def\ermX{{\textnormal{X}}}
\def\ermY{{\textnormal{Y}}}
\def\ermZ{{\textnormal{Z}}}

% Vectors
\def\vzero{{\bm{0}}}
\def\vone{{\bm{1}}}
\def\vmu{{\bm{\mu}}}
\def\vtheta{{\bm{\theta}}}
\def\vphi{{\bm{\phi}}}
\def\va{{\bm{a}}}
\def\vb{{\bm{b}}}
\def\vc{{\bm{c}}}
\def\vd{{\bm{d}}}
\def\ve{{\bm{e}}}
\def\vf{{\bm{f}}}
\def\vg{{\bm{g}}}
\def\vh{{\bm{h}}}
\def\vi{{\bm{i}}}
\def\vj{{\bm{j}}}
\def\vk{{\bm{k}}}
\def\vl{{\bm{l}}}
\def\vm{{\bm{m}}}
\def\vn{{\bm{n}}}
\def\vo{{\bm{o}}}
\def\vp{{\bm{p}}}
\def\vq{{\bm{q}}}
\def\vr{{\bm{r}}}
\def\vs{{\bm{s}}}
\def\vt{{\bm{t}}}
\def\vu{{\bm{u}}}
\def\vv{{\bm{v}}}
\def\vw{{\bm{w}}}
\def\vx{{\bm{x}}}
\def\vy{{\bm{y}}}
\def\vz{{\bm{z}}}

% Elements of vectors
\def\evalpha{{\alpha}}
\def\evbeta{{\beta}}
\def\evepsilon{{\epsilon}}
\def\evlambda{{\lambda}}
\def\evomega{{\omega}}
\def\evmu{{\mu}}
\def\evpsi{{\psi}}
\def\evsigma{{\sigma}}
\def\evtheta{{\theta}}
\def\eva{{a}}
\def\evb{{b}}
\def\evc{{c}}
\def\evd{{d}}
\def\eve{{e}}
\def\evf{{f}}
\def\evg{{g}}
\def\evh{{h}}
\def\evi{{i}}
\def\evj{{j}}
\def\evk{{k}}
\def\evl{{l}}
\def\evm{{m}}
\def\evn{{n}}
\def\evo{{o}}
\def\evp{{p}}
\def\evq{{q}}
\def\evr{{r}}
\def\evs{{s}}
\def\evt{{t}}
\def\evu{{u}}
\def\evv{{v}}
\def\evw{{w}}
\def\evx{{x}}
\def\evy{{y}}
\def\evz{{z}}

% Matrix
\def\mA{{\bm{A}}}
\def\mB{{\bm{B}}}
\def\mC{{\bm{C}}}
\def\mD{{\bm{D}}}
\def\mE{{\bm{E}}}
\def\mF{{\bm{F}}}
\def\mG{{\bm{G}}}
\def\mH{{\bm{H}}}
\def\mI{{\bm{I}}}
\def\mJ{{\bm{J}}}
\def\mK{{\bm{K}}}
\def\mL{{\bm{L}}}
\def\mM{{\bm{M}}}
\def\mN{{\bm{N}}}
\def\mO{{\bm{O}}}
\def\mP{{\bm{P}}}
\def\mQ{{\bm{Q}}}
\def\mR{{\bm{R}}}
\def\mS{{\bm{S}}}
\def\mT{{\bm{T}}}
\def\mU{{\bm{U}}}
\def\mV{{\bm{V}}}
\def\mW{{\bm{W}}}
\def\mX{{\bm{X}}}
\def\mY{{\bm{Y}}}
\def\mZ{{\bm{Z}}}
\def\mBeta{{\bm{\beta}}}
\def\mPhi{{\bm{\Phi}}}
\def\mLambda{{\bm{\Lambda}}}
\def\mSigma{{\bm{\Sigma}}}

% Tensor
\DeclareMathAlphabet{\mathsfit}{\encodingdefault}{\sfdefault}{m}{sl}
\SetMathAlphabet{\mathsfit}{bold}{\encodingdefault}{\sfdefault}{bx}{n}
\newcommand{\tens}[1]{\bm{\mathsfit{#1}}}
\def\tA{{\tens{A}}}
\def\tB{{\tens{B}}}
\def\tC{{\tens{C}}}
\def\tD{{\tens{D}}}
\def\tE{{\tens{E}}}
\def\tF{{\tens{F}}}
\def\tG{{\tens{G}}}
\def\tH{{\tens{H}}}
\def\tI{{\tens{I}}}
\def\tJ{{\tens{J}}}
\def\tK{{\tens{K}}}
\def\tL{{\tens{L}}}
\def\tM{{\tens{M}}}
\def\tN{{\tens{N}}}
\def\tO{{\tens{O}}}
\def\tP{{\tens{P}}}
\def\tQ{{\tens{Q}}}
\def\tR{{\tens{R}}}
\def\tS{{\tens{S}}}
\def\tT{{\tens{T}}}
\def\tU{{\tens{U}}}
\def\tV{{\tens{V}}}
\def\tW{{\tens{W}}}
\def\tX{{\tens{X}}}
\def\tY{{\tens{Y}}}
\def\tZ{{\tens{Z}}}


% Graph
\def\gA{{\mathcal{A}}}
\def\gB{{\mathcal{B}}}
\def\gC{{\mathcal{C}}}
\def\gD{{\mathcal{D}}}
\def\gE{{\mathcal{E}}}
\def\gF{{\mathcal{F}}}
\def\gG{{\mathcal{G}}}
\def\gH{{\mathcal{H}}}
\def\gI{{\mathcal{I}}}
\def\gJ{{\mathcal{J}}}
\def\gK{{\mathcal{K}}}
\def\gL{{\mathcal{L}}}
\def\gM{{\mathcal{M}}}
\def\gN{{\mathcal{N}}}
\def\gO{{\mathcal{O}}}
\def\gP{{\mathcal{P}}}
\def\gQ{{\mathcal{Q}}}
\def\gR{{\mathcal{R}}}
\def\gS{{\mathcal{S}}}
\def\gT{{\mathcal{T}}}
\def\gU{{\mathcal{U}}}
\def\gV{{\mathcal{V}}}
\def\gW{{\mathcal{W}}}
\def\gX{{\mathcal{X}}}
\def\gY{{\mathcal{Y}}}
\def\gZ{{\mathcal{Z}}}

% Sets
\def\sA{{\mathbb{A}}}
\def\sB{{\mathbb{B}}}
\def\sC{{\mathbb{C}}}
\def\sD{{\mathbb{D}}}
% Don't use a set called E, because this would be the same as our symbol
% for expectation.
\def\sF{{\mathbb{F}}}
\def\sG{{\mathbb{G}}}
\def\sH{{\mathbb{H}}}
\def\sI{{\mathbb{I}}}
\def\sJ{{\mathbb{J}}}
\def\sK{{\mathbb{K}}}
\def\sL{{\mathbb{L}}}
\def\sM{{\mathbb{M}}}
\def\sN{{\mathbb{N}}}
\def\sO{{\mathbb{O}}}
\def\sP{{\mathbb{P}}}
\def\sQ{{\mathbb{Q}}}
\def\sR{{\mathbb{R}}}
\def\sS{{\mathbb{S}}}
\def\sT{{\mathbb{T}}}
\def\sU{{\mathbb{U}}}
\def\sV{{\mathbb{V}}}
\def\sW{{\mathbb{W}}}
\def\sX{{\mathbb{X}}}
\def\sY{{\mathbb{Y}}}
\def\sZ{{\mathbb{Z}}}

% Entries of a matrix
\def\emLambda{{\Lambda}}
\def\emA{{A}}
\def\emB{{B}}
\def\emC{{C}}
\def\emD{{D}}
\def\emE{{E}}
\def\emF{{F}}
\def\emG{{G}}
\def\emH{{H}}
\def\emI{{I}}
\def\emJ{{J}}
\def\emK{{K}}
\def\emL{{L}}
\def\emM{{M}}
\def\emN{{N}}
\def\emO{{O}}
\def\emP{{P}}
\def\emQ{{Q}}
\def\emR{{R}}
\def\emS{{S}}
\def\emT{{T}}
\def\emU{{U}}
\def\emV{{V}}
\def\emW{{W}}
\def\emX{{X}}
\def\emY{{Y}}
\def\emZ{{Z}}
\def\emSigma{{\Sigma}}

% entries of a tensor
% Same font as tensor, without \bm wrapper
\newcommand{\etens}[1]{\mathsfit{#1}}
\def\etLambda{{\etens{\Lambda}}}
\def\etA{{\etens{A}}}
\def\etB{{\etens{B}}}
\def\etC{{\etens{C}}}
\def\etD{{\etens{D}}}
\def\etE{{\etens{E}}}
\def\etF{{\etens{F}}}
\def\etG{{\etens{G}}}
\def\etH{{\etens{H}}}
\def\etI{{\etens{I}}}
\def\etJ{{\etens{J}}}
\def\etK{{\etens{K}}}
\def\etL{{\etens{L}}}
\def\etM{{\etens{M}}}
\def\etN{{\etens{N}}}
\def\etO{{\etens{O}}}
\def\etP{{\etens{P}}}
\def\etQ{{\etens{Q}}}
\def\etR{{\etens{R}}}
\def\etS{{\etens{S}}}
\def\etT{{\etens{T}}}
\def\etU{{\etens{U}}}
\def\etV{{\etens{V}}}
\def\etW{{\etens{W}}}
\def\etX{{\etens{X}}}
\def\etY{{\etens{Y}}}
\def\etZ{{\etens{Z}}}

% The true underlying data generating distribution
\newcommand{\pdata}{p_{\rm{data}}}
\newcommand{\ptarget}{p_{\rm{target}}}
\newcommand{\pprior}{p_{\rm{prior}}}
\newcommand{\pbase}{p_{\rm{base}}}
\newcommand{\pref}{p_{\rm{ref}}}

% The empirical distribution defined by the training set
\newcommand{\ptrain}{\hat{p}_{\rm{data}}}
\newcommand{\Ptrain}{\hat{P}_{\rm{data}}}
% The model distribution
\newcommand{\pmodel}{p_{\rm{model}}}
\newcommand{\Pmodel}{P_{\rm{model}}}
\newcommand{\ptildemodel}{\tilde{p}_{\rm{model}}}
% Stochastic autoencoder distributions
\newcommand{\pencode}{p_{\rm{encoder}}}
\newcommand{\pdecode}{p_{\rm{decoder}}}
\newcommand{\precons}{p_{\rm{reconstruct}}}

\newcommand{\laplace}{\mathrm{Laplace}} % Laplace distribution

\newcommand{\E}{\mathbb{E}}
\newcommand{\Ls}{\mathcal{L}}
\newcommand{\R}{\mathbb{R}}
\newcommand{\emp}{\tilde{p}}
\newcommand{\lr}{\alpha}
\newcommand{\reg}{\lambda}
\newcommand{\rect}{\mathrm{rectifier}}
\newcommand{\softmax}{\mathrm{softmax}}
\newcommand{\sigmoid}{\sigma}
\newcommand{\softplus}{\zeta}
\newcommand{\KL}{D_{\mathrm{KL}}}
\newcommand{\Var}{\mathrm{Var}}
\newcommand{\standarderror}{\mathrm{SE}}
\newcommand{\Cov}{\mathrm{Cov}}
% Wolfram Mathworld says $L^2$ is for function spaces and $\ell^2$ is for vectors
% But then they seem to use $L^2$ for vectors throughout the site, and so does
% wikipedia.
\newcommand{\normlzero}{L^0}
\newcommand{\normlone}{L^1}
\newcommand{\normltwo}{L^2}
\newcommand{\normlp}{L^p}
\newcommand{\normmax}{L^\infty}

\newcommand{\parents}{Pa} % See usage in notation.tex. Chosen to match Daphne's book.

\DeclareMathOperator*{\argmax}{arg\,max}
\DeclareMathOperator*{\argmin}{arg\,min}

\DeclareMathOperator{\sign}{sign}
\DeclareMathOperator{\Tr}{Tr}
\let\ab\allowbreak

%%%%%%%%%%%%%%%%%%%%%%%%%%%%%%%%%%%%%%%%%%%%%%%%%%%%%%%%%%%%%%%%%%%%%%%%%%%%%%%
% Introduction
%%%%%%%%%%%%%%%%%%%%%%%%%%%%%%%%%%%%%%%%%%%%%%%%%%%%%%%%%%%%%%%%%%%%%%%%%%%%%%%

% Include this file in all LaTeX papers that you write at dlab by adding a line
% "\input{dlab_macros}" right after the "\documentclass" command.

%%%%%%%%%%%%%%%%%%%%%%%%%%%%%%%%%%%%%%%%%%%%%%%%%%%%%%%%%%%%%%%%%%%%%%%%%%%%%%%
% Some standard packages
%%%%%%%%%%%%%%%%%%%%%%%%%%%%%%%%%%%%%%%%%%%%%%%%%%%%%%%%%%%%%%%%%%%%%%%%%%%%%%%

\usepackage[utf8]{inputenc}
\usepackage[T1]{fontenc}
\usepackage{hyphenat}
\usepackage{xspace}
\usepackage{amsmath}
\usepackage{amsfonts}
% \usepackage{hyperref}
\usepackage{url}
\usepackage{booktabs}
\usepackage{multirow}
%\usepackage{subfig}
\usepackage{makecell}
\usepackage{caption}
\usepackage{minibox}
\usepackage{bbm}
\usepackage{graphicx}
\usepackage{balance}
\usepackage{mathtools}
\usepackage{color}
\usepackage{marvosym}
\usepackage{ifthen}
\usepackage{textcomp}
\usepackage{enumitem}
\usepackage{verbatim}
\usepackage{algorithm}
% \usepackage{algorithmic}
\usepackage{numprint}
\usepackage{balance}

\usepackage{amsthm}
\theoremstyle{plain}
\newtheorem{theorem}{Theorem}
\newtheorem{definition}[theorem]{Definition}
\newtheorem{lemma}[theorem]{Lemma}
\newtheorem{proposition}[theorem]{Proposition}
\newtheorem{example}[theorem]{Example}

%%%%%%%%%%%%%%%%%%%%%%%%%%%%%%%%%%%%%%%%%%%%%%%%%%%%%%%%%%%%%%%%%%%%%%%%%%%%%%%
% How to include TODOs and notes
%%%%%%%%%%%%%%%%%%%%%%%%%%%%%%%%%%%%%%%%%%%%%%%%%%%%%%%%%%%%%%%%%%%%%%%%%%%%%%%

% Adapted from the widely circulating chato-notes.sty -- thanks, ChaTo!

\newcommand{\chatoDisplayMode}[1]{#1}

% If you quickly want to hide all notes, e.g., to check how long your paper
% would be without them, add the following line to your preamble or uncomment
% it here.
% \renewcommand{\chatoDisplayMode}[1]{}

% Usage:
% \todo[Your name]{What needs to be done}
% \note[Your name]{A note to include in a box}
% \inote{An inline note}
% \citemissing{} (if you want to indicate a missing citation)

\definecolor{MyRed}{rgb}{0.6,0.0,0.0} 
\definecolor{MyBlack}{rgb}{0.1,0.1,0.1} 
\newcommand{\inred}[1]{{\color{MyRed}\sf\textbf{\textsc{#1}}}}
\newcommand{\frameit}[2]{
  \begin{center}
  {\color{MyRed}
  \framebox[.9\columnwidth][l]{
    \begin{minipage}{.85\columnwidth}
    \inred{#1}: {\sf\color{MyBlack}#2}
    \end{minipage}
  }\\
  }
  \end{center}
}

\newcommand{\note}[2][]{\chatoDisplayMode{\def\@tmpsig{#1}\frameit{{\Pointinghand} Note}{#2\ifx \@tmpsig \@empty \else \mbox{ --\em #1}\fi}}}
\newcommand{\todo}[2][]{\chatoDisplayMode{\def\@tmpsig{#1}\frameit{{\Writinghand} To-do}{#2\ifx \@tmpsig \@empty \else \mbox{ --\em #1}\fi}}}
\newcommand{\inote}[1]{\chatoDisplayMode{\inred{$\lceil${{\Pointinghand} }} {\sf #1} \inred{$\rfloor$}}}
\newcommand{\citemissing}[0]{\chatoDisplayMode{\inred{[citation]}}}

%%%%%%%%%%%%%%%%%%%%%%%%%%%%%%%%%%%%%%%%%%%%%%%%%%%%%%%%%%%%%%%%%%%%%%%%%%%%%%%
% How to make your edits conspicuous
%%%%%%%%%%%%%%%%%%%%%%%%%%%%%%%%%%%%%%%%%%%%%%%%%%%%%%%%%%%%%%%%%%%%%%%%%%%%%%%

% In the final stages of editing, it is often useful to mark edits in color, so
% everyone can easily see what was changed. To do so, define a command that has
% the same name as you and use your favorite color.
\newcommand{\bob}[1]{\textcolor{red}{#1}}
\newcommand{\yourname}[1]{\textcolor{blue}{#1}}

%%%%%%%%%%%%%%%%%%%%%%%%%%%%%%%%%%%%%%%%%%%%%%%%%%%%%%%%%%%%%%%%%%%%%%%%%%%%%%%
% Latin abbreviations
%%%%%%%%%%%%%%%%%%%%%%%%%%%%%%%%%%%%%%%%%%%%%%%%%%%%%%%%%%%%%%%%%%%%%%%%%%%%%%%

% Don't use plain text for Latin abbreviations such as "e.g.", "i.e.", etc.
% Use these macros instead. Advantage: you can consistently change their style,
% e.g., if you want to typeset them in italics at some point.

% Latin abbreviations in normal font.
\newcommand{\abbrevStyle}[1]{#1}
% Latin abbreviations in italics.
% \newcommand{\abbrevStyle}[1]{\textit{#1}}

\newcommand{\ie}{\abbrevStyle{i.e.}\xspace}
\newcommand{\eg}{\abbrevStyle{e.g.}\xspace}
\newcommand{\cf}{\abbrevStyle{cf.}\xspace}
\newcommand{\etal}{\abbrevStyle{et al.}\xspace}
% \newcommand{\vs}{\abbrevStyle{vs.}\xspace}
\newcommand{\etc}{\abbrevStyle{etc.}\xspace}
\newcommand{\viz}{\abbrevStyle{viz.}\xspace}

%%%%%%%%%%%%%%%%%%%%%%%%%%%%%%%%%%%%%%%%%%%%%%%%%%%%%%%%%%%%%%%%%%%%%%%%%%%%%%%
% Referring to sections, figures, tables, etc.
%%%%%%%%%%%%%%%%%%%%%%%%%%%%%%%%%%%%%%%%%%%%%%%%%%%%%%%%%%%%%%%%%%%%%%%%%%%%%%%

% To refer to sections, figures, tables, etc., use the following macros.
% Don't type "Section~1", "Fig.~1", etc., manually. This way, you can easily
% and consistently switch between styles, e.g., if you want to use "Sec."
% instead of "Section" at some point.

% \newcommand{\Secref}[1]{Sec.~\ref{#1}}
\newcommand{\Eqnref}[1]{Eq.~\ref{#1}}
\newcommand{\Dashsecref}[2]{Sec.~\ref{#1}--\ref{#2}}
\newcommand{\Dblsecref}[2]{Sec.~\ref{#1} and \ref{#2}}
\newcommand{\Tabref}[1]{Table~\ref{#1}}
% \newcommand{\Figref}[1]{Fig.~\ref{#1}}
\newcommand{\Dashfigref}[2]{Fig.~\ref{#1}--\ref{#2}}
\newcommand{\Appref}[1]{Appendix~\ref{#1}}
\newcommand{\Thmref}[1]{Thm.~\ref{#1}}
\newcommand{\Lemmaref}[1]{Lemma~\ref{#1}}
\newcommand{\Defref}[1]{Def.~\ref{#1}}

%%%%%%%%%%%%%%%%%%%%%%%%%%%%%%%%%%%%%%%%%%%%%%%%%%%%%%%%%%%%%%%%%%%%%%%%%%%%%%%
% Paragraph headings
%%%%%%%%%%%%%%%%%%%%%%%%%%%%%%%%%%%%%%%%%%%%%%%%%%%%%%%%%%%%%%%%%%%%%%%%%%%%%%%

% Academic text is often much more legible if you give important paragraphs a
% concise name that describes what the paragraph is about. Use the \xhdr
% command for this.
\newcommand{\xhdr}[1]{\vspace{1.7mm}\noindent{{\bf #1.}}}

% Same as \xhdr, but without a period after the heading. Use this version if
% the heading is directly integrated into the first sentence of the paragraph;
% e.g., "\xhdrNoPeriod{Results} are shown in \Figref{fig}."
\newcommand{\xhdrNoPeriod}[1]{\vspace{1.7mm}\noindent{{\bf #1}}}

%%%%%%%%%%%%%%%%%%%%%%%%%%%%%%%%%%%%%%%%%%%%%%%%%%%%%%%%%%%%%%%%%%%%%%%%%%%%%%%
% More compact lists
%%%%%%%%%%%%%%%%%%%%%%%%%%%%%%%%%%%%%%%%%%%%%%%%%%%%%%%%%%%%%%%%%%%%%%%%%%%%%%%

% In some styles, list items are widely spaced. To condense them and save some
% space, you may use this command.
\newcommand{\denselist}{ \itemsep -2pt\topsep-10pt\partopsep-10pt }

% Same, but with slightly different spacing.
\newcommand{\denselistRefs}{ \itemsep -2pt\topsep-5pt\partopsep-7pt }

%%%%%%%%%%%%%%%%%%%%%%%%%%%%%%%%%%%%%%%%%%%%%%%%%%%%%%%%%%%%%%%%%%%%%%%%%%%%%%%
% Miscellaneous useful macros
%%%%%%%%%%%%%%%%%%%%%%%%%%%%%%%%%%%%%%%%%%%%%%%%%%%%%%%%%%%%%%%%%%%%%%%%%%%%%%%

% Some bibliography styles make it hard to typeset references like
% "Einstein et al. (1905)". This command provides a convenient way to do so.
\newcommand{\textcite}[1]{\citeauthor{#1} \shortcite{#1}}

% When you frequently refer to Wikipedia articles, Wikidata entities, etc., it
% may be useful to typeset those in a particular font. Use the \cpt (for
% "concept") command for this purpose.
\newcommand{\cpt}[1]{\textsc{\MakeLowercase{#1}}}

% To exclude a large portion of text from the PDF, wrap it in \hide.
\newcommand{\hide}[1]{}

% Wrap matrix variables in \mtx. Don't make them bold etc. manually. By using
% a macro, you can consistently change the rendering style at any point.
\newcommand{\mtx}[1]{\mathbf{#1}}

% Equal sign for definition that is centered properly.
\newcommand{\defeq}{\vcentcolon=}

% Transpose of a matrix, e.g., $A\trans{}$.
\newcommand{\trans}{^\top}

% \argmin and \argmax.
% \DeclareMathOperator*{\argmax}{arg\,max}
% \DeclareMathOperator*{\argmin}{arg\,min}

%%%%%%%%%%%%%%%%%%%%%%%%%%%%%%%%%%%%%%%%%%%%%%%%%%%%%%%%%%%%%%%%%%%%%%%%%%%%%%%
% Hyphenation
%%%%%%%%%%%%%%%%%%%%%%%%%%%%%%%%%%%%%%%%%%%%%%%%%%%%%%%%%%%%%%%%%%%%%%%%%%%%%%%

% Some words are ill-hyphenated by default. Here you can define the correct
% hyphenation once, and it is then used consistently.

\hyphenation{
Wi-ki-pe-dia
Wi-ki-me-dia
Wi-ki-da-ta
De-ter-mine
Page-Rank
web-page
web-pages
da-ta-set
Grish-man
}

%%%%%%%%%%%%%%%%%%%%%%%%%%%%%%%%%%%%%%%%%%%%%%%%%%%%%%%%%%%%%%%%%%%%%%%%%%%%%%%
% Avoid widows!
%%%%%%%%%%%%%%%%%%%%%%%%%%%%%%%%%%%%%%%%%%%%%%%%%%%%%%%%%%%%%%%%%%%%%%%%%%%%%%%

% The term "widow" refers to the first line of a paragraph if it is the last
% line on a page, or to the last line of a paragraph if it is the first line on
% a page. Widows are considered a cardinal typesetting sin, so avoid them at
% all cost, via the following commands.

\widowpenalty=10000
\clubpenalty=10000

%%%%%%%%%%%%%%%%%%%%%%%%%%%%%%%%%%%%%%%%%%%%%%%%%%%%%%%%%%%%%%%%%%%%%%%%%%%%%%%
% Enable section numbering in the AAAI style (e.g., used by ICWSM)
%%%%%%%%%%%%%%%%%%%%%%%%%%%%%%%%%%%%%%%%%%%%%%%%%%%%%%%%%%%%%%%%%%%%%%%%%%%%%%%

% In the AAAI style, this enables section numbering.
\setcounter{secnumdepth}{2}

%%%%%%%%%%%%%%%%%%%%%%%%%%%%%%%%%%%%%%%%%%%%%%%%%%%%%%%%%%%%%%%%%%%%%%%%%%%%%%%
% Listing authors in a space-economic way in the ACM style
%%%%%%%%%%%%%%%%%%%%%%%%%%%%%%%%%%%%%%%%%%%%%%%%%%%%%%%%%%%%%%%%%%%%%%%%%%%%%%%

% By default, using "\documentclass[sigconf]{acmart}" will list authors in rows
% of 2, which can take up a lot of space. To get more authors in one row, use
% something like this:
% \author{
%   \authorbox{Author 1}{Affiliation 1}{Email 1}
%   \authorbox{Author 2}{Affiliation 2}{Email 2}
%   ...
% }

% If you use \authorbox, you will also have to suppress the standard reference
% block, by pasting the following row somewhere before "\begin{document}" ...
% \settopmatter{printacmref=false, printfolios=false}

% ... and add the reference block manually after abstract and \maketitle (also
% if you use asterisks and daggers after author names without necessarily using
% \authorbox), like this:
% {\fontsize{8pt}{8pt} \selectfont
% \textbf{ACM Reference Format:}\\
% Roland Aydin, Lars Klein, Arnaud Miribel, and Robert West.
% 2020.
% Broccoli: Sprinkling Lightweight Vocabulary Learning into Everyday Information Diets.
% In \textit{Proceedings of The Web Conference 2020 (WWW '20), April 20--24, 2020, Taipei, Taiwan.}
% ACM, New York, NY, USA, 11 pages. \url{https://doi.org/10.1145/3366423.3380209}}

\newcommand{\affilSize}{9pt}
\newcommand{\authorbox}[3]{
  \minibox[c]{
    #1\\
    {\fontsize{\affilSize}{\affilSize}\selectfont{}#2}\\
    {\fontsize{\affilSize}{\affilSize}\selectfont{}#3}
  }
}

% If you add an asterisk or dagger after an author name, add the corresponding
% footnote using the \blfootnote{} command.
\newcommand\blfootnote[1]{%
  \begingroup
  \renewcommand\thefootnote{}\footnote{#1}%
  \addtocounter{footnote}{-1}%
  \endgroup
}

%%%%%%%%%%%%%%%%%%%%%%%%%%%%%%%%%%%%%%%%%%%%%%%%%%%%%%%%%%%%%%%%%%%%%%%%%%%%%%%
% Some tricks to make papers that use the Times font nicer.
%%%%%%%%%%%%%%%%%%%%%%%%%%%%%%%%%%%%%%%%%%%%%%%%%%%%%%%%%%%%%%%%%%%%%%%%%%%%%%%

\makeatletter
\newcommand{\iffont}[2]{\ifthenelse{\equal{\f@family}{#1}}{#2}{}}
\makeatother

% If the paper font is Times (e.g., in the AAAI style) ...
% \iffont{ptm}{
  % ... we also want to typeset math in Times ...
  \usepackage{mathptmx}

  % ... and use a nicer Greek font, since the default is ugly.
  \DeclareSymbolFont{greek}{OML}{cmm}{m}{n}
  \DeclareMathSymbol{\alpha}{\mathalpha}{greek}{"0B}
  \DeclareMathSymbol{\beta}{\mathalpha}{greek}{"0C}
  \DeclareMathSymbol{\gamma}{\mathalpha}{greek}{"0D}
  \DeclareMathSymbol{\delta}{\mathalpha}{greek}{"0E}
  \DeclareMathSymbol{\epsilon}{\mathalpha}{greek}{"0F}
  \DeclareMathSymbol{\zeta}{\mathalpha}{greek}{"10}
  \DeclareMathSymbol{\eta}{\mathalpha}{greek}{"11}
  \DeclareMathSymbol{\theta}{\mathalpha}{greek}{"12}
  \DeclareMathSymbol{\iota}{\mathalpha}{greek}{"13}
  \DeclareMathSymbol{\kappa}{\mathalpha}{greek}{"14}
  \DeclareMathSymbol{\lambda}{\mathalpha}{greek}{"15}
  \DeclareMathSymbol{\mu}{\mathalpha}{greek}{"16}
  \DeclareMathSymbol{\nu}{\mathalpha}{greek}{"17}
  \DeclareMathSymbol{\xi}{\mathalpha}{greek}{"18}
  \DeclareMathSymbol{\pi}{\mathalpha}{greek}{"19}
  \DeclareMathSymbol{\rho}{\mathalpha}{greek}{"1A}
  \DeclareMathSymbol{\sigma}{\mathalpha}{greek}{"1B}
  \DeclareMathSymbol{\tau}{\mathalpha}{greek}{"1C}
  \DeclareMathSymbol{\upsilon}{\mathalpha}{greek}{"1D}
  \DeclareMathSymbol{\phi}{\mathalpha}{greek}{"1E}
  \DeclareMathSymbol{\chi}{\mathalpha}{greek}{"1F}
  \DeclareMathSymbol{\psi}{\mathalpha}{greek}{"20}
  \DeclareMathSymbol{\omega}{\mathalpha}{greek}{"21}
  \DeclareMathSymbol{\varepsilon}{\mathalpha}{greek}{"22}
  \DeclareMathSymbol{\vartheta}{\mathalpha}{greek}{"23}
  \DeclareMathSymbol{\varpi}{\mathalpha}{greek}{"24}
  \DeclareMathSymbol{\varrho}{\mathalpha}{greek}{"25}
  \DeclareMathSymbol{\varsigma}{\mathalpha}{greek}{"26}
  \DeclareMathSymbol{\varphi}{\mathalpha}{greek}{"27}
  \DeclareSymbolFont{otone}{OT1}{cmr}{m}{n}
  \DeclareMathSymbol{\Gamma}{\mathalpha}{otone}{0}
  \DeclareMathSymbol{\Delta}{\mathalpha}{otone}{1}
  \DeclareMathSymbol{\Theta}{\mathalpha}{otone}{2}
  \DeclareMathSymbol{\Lambda}{\mathalpha}{otone}{3}
  \DeclareMathSymbol{\Xi}{\mathalpha}{otone}{4}
  \DeclareMathSymbol{\Pi}{\mathalpha}{otone}{5}
  \DeclareMathSymbol{\Sigma}{\mathalpha}{otone}{6}
  \DeclareMathSymbol{\Upsilon}{\mathalpha}{otone}{7}
  \DeclareMathSymbol{\Phi}{\mathalpha}{otone}{8}
  \DeclareMathSymbol{\Psi}{\mathalpha}{otone}{9}
  \DeclareMathSymbol{\Omega}{\mathalpha}{otone}{10}
  \DeclareSymbolFont{syms}{OML}{cmm}{m}{it}
  \DeclareMathSymbol{\partial}{\mathord}{syms}{"40}
  \DeclareMathAlphabet{\mathbold}{OML}{cmm}{b}{it}
  \DeclareSymbolFont{largesymbols}{OMX}{cmex}{m}{n}

  % If you want to use a less curly \mathcal font than the default one used
  % with Times, uncomment this line.
  %\DeclareMathAlphabet{\mathcal}{OMS}{cmsy}{m}{n}
% }


\usepackage{hyperref}
\usepackage{url}
\usepackage{graphicx}
\usepackage{tikz-cd}
\usepackage{amsthm}
\usepackage{subcaption}
% \newtheorem{theorem}{Theorem}
% \newtheorem{definition}[theorem]{Definition}
% \newtheorem*{notation}{Notation}
\usepackage{wrapfig}
\usepackage{bbm}
\usepackage{booktabs}


% \newcommand{\todo}[1]{\textcolor{red}{\textbf{TODO: #1}}}

% \definecolor{myblue}{RGB}{0, 102, 204}

\icmltitlerunning{Meta-Statistical Learning}

\begin{document}

\twocolumn[
\icmltitle{Meta-Statistical Learning: Supervised Learning of Statistical Inference}



\begin{icmlauthorlist}
\icmlauthor{Maxime Peyrard}{xxx}
\icmlauthor{Kyunghyun Cho}{yyy,zzz}
\end{icmlauthorlist}

\icmlaffiliation{xxx}{Université Grenoble Alpes, CNRS, Grenoble INP, LIG}
\icmlaffiliation{yyy}{New York University}
\icmlaffiliation{zzz}{Genentech}

\icmlcorrespondingauthor{Maxime Peyrard}{maxime.peyrard@univ-grenoble-alpes.fr}

% You may provide any keywords that you
% find helpful for describing your paper; these are used to populate
% the "keywords" metadata in the PDF but will not be shown in the document
\icmlkeywords{Machine Learning, ICML}

\vskip 0.3in
]


\printAffiliationsAndNotice{\icmlEqualContribution} % otherwise use the standard text.

\begin{abstract}
This work demonstrates that the tools and principles driving the success of large language models (LLMs) can be repurposed to tackle distribution-level tasks, where the goal is to predict properties of the data-generating distribution rather than labels for individual datapoints. These tasks encompass statistical inference problems such as parameter estimation, hypothesis testing, or mutual information estimation. 
Framing these tasks within traditional machine learning pipelines is challenging, as supervision is typically tied to individual datapoint.
% 
We propose \textit{meta-statistical learning}, a framework inspired by multi-instance learning that reformulates statistical inference tasks as supervised learning problems. In this approach, entire datasets are treated as single inputs to neural networks, which predict distribution-level parameters. Transformer-based architectures, without positional encoding, provide a natural fit due to their permutation-invariance properties.
% 
By training on large-scale synthetic datasets, meta-statistical models can leverage the scalability and optimization infrastructure of Transformer-based LLMs. We demonstrate the framework’s versatility with applications in hypothesis testing and mutual information estimation, showing strong performance, particularly for small datasets where traditional neural methods struggle. 
\end{abstract}


\section{Introduction}
\label{sec:introduction}
\section{Introduction}
\label{section:introduction}

% redirection is unique and important in VR
Virtual Reality (VR) systems enable users to embody virtual avatars by mirroring their physical movements and aligning their perspective with virtual avatars' in real time. 
As the head-mounted displays (HMDs) block direct visual access to the physical world, users primarily rely on visual feedback from the virtual environment and integrate it with proprioceptive cues to control the avatar’s movements and interact within the VR space.
Since human perception is heavily influenced by visual input~\cite{gibson1933adaptation}, 
VR systems have the unique capability to control users' perception of the virtual environment and avatars by manipulating the visual information presented to them.
Leveraging this, various redirection techniques have been proposed to enable novel VR interactions, 
such as redirecting users' walking paths~\cite{razzaque2005redirected, suma2012impossible, steinicke2009estimation},
modifying reaching movements~\cite{gonzalez2022model, azmandian2016haptic, cheng2017sparse, feick2021visuo},
and conveying haptic information through visual feedback to create pseudo-haptic effects~\cite{samad2019pseudo, dominjon2005influence, lecuyer2009simulating}.
Such redirection techniques enable these interactions by manipulating the alignment between users' physical movements and their virtual avatar's actions.

% % what is hand/arm redirection, motivation of study arm-offset
% \change{\yj{i don't understand the purpose of this paragraph}
% These illusion-based techniques provide users with unique experiences in virtual environments that differ from the physical world yet maintain an immersive experience. 
% A key example is hand redirection, which shifts the virtual hand’s position away from the real hand as the user moves to enhance ergonomics during interaction~\cite{feuchtner2018ownershift, wentzel2020improving} and improve interaction performance~\cite{montano2017erg, poupyrev1996go}. 
% To increase the realism of virtual movements and strengthen the user’s sense of embodiment, hand redirection techniques often incorporate a complete virtual arm or full body alongside the redirected virtual hand, using inverse kinematics~\cite{hartfill2021analysis, ponton2024stretch} or adjustments to the virtual arm's movement as well~\cite{li2022modeling, feick2024impact}.
% }

% noticeability, motivation of predicting a probability, not a classification
However, these redirection techniques are most effective when the manipulation remains undetected~\cite{gonzalez2017model, li2022modeling}. 
If the redirection becomes too large, the user may not mitigate the conflict between the visual sensory input (redirected virtual movement) and their proprioception (actual physical movement), potentially leading to a loss of embodiment with the virtual avatar and making it difficult for the user to accurately control virtual movements to complete interaction tasks~\cite{li2022modeling, wentzel2020improving, feuchtner2018ownershift}. 
While proprioception is not absolute, users only have a general sense of their physical movements and the likelihood that they notice the redirection is probabilistic. 
This probability of detecting the redirection is referred to as \textbf{noticeability}~\cite{li2022modeling, zenner2024beyond, zenner2023detectability} and is typically estimated based on the frequency with which users detect the manipulation across multiple trials.

% version B
% Prior research has explored factors influencing the noticeability of redirected motion, including the redirection's magnitude~\cite{wentzel2020improving, poupyrev1996go}, direction~\cite{li2022modeling, feuchtner2018ownershift}, and the visual characteristics of the virtual avatar~\cite{ogawa2020effect, feick2024impact}.
% While these factors focus on the avatars, the surrounding virtual environment can also influence the users' behavior and in turn affect the noticeability of redirection.
% One such prominent external influence is through the visual channel - the users' visual attention is constantly distracted by complex visual effects and events in practical VR scenarios.
% Although some prior studies have explored how to leverage user blindness caused by visual distractions to redirect users' virtual hand~\cite{zenner2023detectability}, there remains a gap in understanding how to quantify the noticeability of redirection under visual distractions.

% visual stimuli and gaze behavior
Prior research has explored factors influencing the noticeability of redirected motion, including the redirection's magnitude~\cite{wentzel2020improving, poupyrev1996go}, direction~\cite{li2022modeling, feuchtner2018ownershift}, and the visual characteristics of the virtual avatar~\cite{ogawa2020effect, feick2024impact}.
While these factors focus on the avatars, the surrounding virtual environment can also influence the users' behavior and in turn affect the noticeability of redirection.
This, however, remains underexplored.
One such prominent external influence is through the visual channel - the users' visual attention is constantly distracted by complex visual effects and events in practical VR scenarios.
We thus want to investigate how \textbf{visual stimuli in the virtual environment} affect the noticeability of redirection.
With this, we hope to complement existing works that focus on avatars by incorporating environmental visual influences to enable more accurate control over the noticeability of redirected motions in practical VR scenarios.
% However, in realistic VR applications, the virtual environment often contains complex visual effects beyond the virtual avatar itself. 
% We argue that these visual effects can \textbf{distract users’ visual attention and thus affect the noticeability of redirection offsets}, while current research has yet taken into account.
% For instance, in a VR boxing scenario, a user’s visual attention is likely focused on their opponent rather than on their virtual body, leading to a lower noticeability of redirection offsets on their virtual movements. 
% Conversely, when reaching for an object in the center of their field of view, the user’s attention is more concentrated on the virtual hand’s movement and position to ensure successful interaction, resulting in a higher noticeability of offsets.

Since each visual event is a complex choreography of many underlying factors (type of visual effect, location, duration, etc.), it is extremely difficult to quantify or parameterize visual stimuli.
Furthermore, individuals respond differently to even the same visual events.
Prior neuroscience studies revealed that factors like age, gender, and personality can influence how quickly someone reacts to visual events~\cite{gillon2024responses, gale1997human}. 
Therefore, aiming to model visual stimuli in a way that is generalizable and applicable to different stimuli and users, we propose to use users' \textbf{gaze behavior} as an indicator of how they respond to visual stimuli.
In this paper, we used various gaze behaviors, including gaze location, saccades~\cite{krejtz2018eye}, fixations~\cite{perkhofer2019using}, and the Index of Pupil Activity (IPA)~\cite{duchowski2018index}.
These behaviors indicate both where users are looking and their cognitive activity, as looking at something does not necessarily mean they are attending to it.
Our goal is to investigate how these gaze behaviors stimulated by various visual stimuli relate to the noticeability of redirection.
With this, we contribute a model that allows designers and content creators to adjust the redirection in real-time responding to dynamic visual events in VR.

To achieve this, we conducted user studies to collect users' noticeability of redirection under various visual stimuli.
To simulate realistic VR scenarios, we adopted a dual-task design in which the participants performed redirected movements while monitoring the visual stimuli.
Specifically, participants' primary task was to report if they noticed an offset between the avatar's movement and their own, while their secondary task was to monitor and report the visual stimuli.
As realistic virtual environments often contain complex visual effects, we started with simple and controlled visual stimulus to manage the influencing factors.

% first user study, confirmation study
% collect data under no visual stimuli, different basic visual stimuli
We first conducted a confirmation study (N=16) to test whether applying visual stimuli (opacity-based) actually affects their noticeability of redirection. 
The results showed that participants were significantly less likely to detect the redirection when visual stimuli was presented $(F_{(1,15)}=5.90,~p=0.03)$.
Furthermore, by analyzing the collected gaze data, results revealed a correlation between the proposed gaze behaviors and the noticeability results $(r=-0.43)$, confirming that the gaze behaviors could be leveraged to compute the noticeability.

% data collection study
We then conducted a data collection study to obtain more accurate noticeability results through repeated measurements to better model the relationship between visual stimuli-triggered gaze behaviors and noticeability of redirection.
With the collected data, we analyzed various numerical features from the gaze behaviors to identify the most effective ones. 
We tested combinations of these features to determine the most effective one for predicting noticeability under visual stimuli.
Using the selected features, our regression model achieved a mean squared error (MSE) of 0.011 through leave-one-user-out cross-validation. 
Furthermore, we developed both a binary and a three-class classification model to categorize noticeability, which achieved an accuracy of 91.74\% and 85.62\%, respectively.

% evaluation study
To evaluate the generalizability of the regression model, we conducted an evaluation study (N=24) to test whether the model could accurately predict noticeability with new visual stimuli (color- and scale-based animations).
Specifically, we evaluated whether the model's predictions aligned with participants' responses under these unseen stimuli.
The results showed that our model accurately estimated the noticeability, achieving mean squared errors (MSE) of 0.014 and 0.012 for the color- and scale-based visual stimili, respectively, compared to participants' responses.
Since the tested visual stimuli data were not included in the training, the results suggested that the extracted gaze behavior features capture a generalizable pattern and can effectively indicate the corresponding impact on the noticeability of redirection.

% application
Based on our model, we implemented an adaptive redirection technique and demonstrated it through two applications: adaptive VR action game and opportunistic rendering.
We conducted a proof-of-concept user study (N=8) to compare our adaptive redirection technique with a static redirection, evaluating the usability and benefits of our adaptive redirection technique.
The results indicated that participants experienced less physical demand and stronger sense of embodiment and agency when using the adaptive redirection technique. 
These results demonstrated the effectiveness and usability of our model.

In summary, we make the following contributions.
% 
\begin{itemize}
    \item 
    We propose to use users' gaze behavior as a medium to quantify how visual stimuli influences the noticebility of redirection. 
    Through two user studies, we confirm that visual stimuli significantly influences noticeability and identify key gaze behavior features that are closely related to this impact.
    \item 
    We build a regression model that takes the user's gaze behavioral data as input, then computes the noticeability of redirection.
    Through an evaluation study, we verify that our model can estimate the noticeability with new participants under unseen visual stimuli.
    These findings suggest that the extracted gaze behavior features effectively capture the influence of visual stimuli on noticeability and can generalize across different users and visual stimuli.
    \item 
    We develop an adaptive redirection technique based on our regression model and implement two applications with it.
    With a proof-of-concept study, we demonstrate the effectiveness and potential usability of our regression model on real-world use cases.

\end{itemize}

% \delete{
% Virtual Reality (VR) allows the user to embody a virtual avatar by mirroring their physical movements through the avatar.
% As the user's visual access to the physical world is blocked in tasks involving motion control, they heavily rely on the visual representation of the avatar's motions to guide their proprioception.
% Similar to real-world experiences, the user is able to resolve conflicts between different sensory inputs (e.g., vision and motor control) through multisensory integration, which is essential for mitigating the sensory noise that commonly arises.
% However, it also enables unique manipulations in VR, as the system can intentionally modify the avatar's movements in relation to the user's motions to achieve specific functional outcomes,
% for example, 
% % the manipulations on the avatar's movements can 
% enabling novel interaction techniques of redirected walking~\cite{razzaque2005redirected}, redirected reaching~\cite{gonzalez2022model}, and pseudo haptics~\cite{samad2019pseudo}.
% With small adjustments to the avatar's movements, the user can maintain their sense of embodiment, due to their ability to resolve the perceptual differences.
% % However, a large mismatch between the user and avatar's movements can result in the user losing their sense of embodiment, due to an inability to resolve the perceptual differences.
% }

% \delete{
% However, multisensory integration can break when the manipulation is so intense that the user is aware of the existence of the motion offset and no longer maintains the sense of embodiment.
% Prior research studied the intensity threshold of the offset applied on the avatar's hand, beyond which the embodiment will break~\cite{li2022modeling}. 
% Studies also investigated the user's sensitivity to the offsets over time~\cite{kohm2022sensitivity}.
% Based on the findings, we argue that one crucial factor that affects to what extent the user notices the offset (i.e., \textit{noticeability}) that remains under-explored is whether the user directs their visual attention towards or away from the virtual avatar.
% Related work (e.g., Mise-unseen~\cite{marwecki2019mise}) has showcased applications where adjustments in the environment can be made in an unnoticeable manner when they happen in the area out of the user's visual field.
% We hypothesize that directing the user's visual attention away from the avatar's body, while still partially keeping the avatar within the user's field-of-view, can reduce the noticeability of the offset.
% Therefore, we conduct two user studies and implement a regression model to systematically investigate this effect.
% }

% \delete{
% In the first user study (N = 16), we test whether drawing the user's visual attention away from their body impacts the possibility of them noticing an offset that we apply to their arm motion in VR.
% We adopt a dual-task design to enable the alteration of the user's visual attention and a yes/no paradigm to measure the noticeability of motion offset. 
% The primary task for the user is to perform an arm motion and report when they perceive an offset between the avatar's virtual arm and their real arm.
% In the secondary task, we randomly render a visual animation of a ball turning from transparent to red and becoming transparent again and ask them to monitor and report when it appears.
% We control the strength of the visual stimuli by changing the duration and location of the animation.
% % By changing the time duration and location of the visual animation, we control the strengths of attraction to the users.
% As a result, we found significant differences in the noticeability of the offsets $(F_{(1,15)}=5.90,~p=0.03)$ between conditions with and without visual stimuli.
% Based on further analysis, we also identified the behavioral patterns of the user's gaze (including pupil dilation, fixations, and saccades) to be correlated with the noticeability results $(r=-0.43)$ and they may potentially serve as indicators of noticeability.
% }

% \delete{
% To further investigate how visual attention influences the noticeability, we conduct a data collection study (N = 12) and build a regression model based on the data.
% The regression model is able to calculate the noticeability of the offset applied on the user's arm under various visual stimuli based on their gaze behaviors.
% Our leave-one-out cross-validation results show that the proposed method was able to achieve a mean-squared error (MSE) of 0.012 in the probability regression task.
% }

% \delete{
% To verify the feasibility and extendability of the regression model, we conduct an evaluation study where we test new visual animations based on adjustments on scale and color and invite 24 new participants to attend the study.
% Results show that the proposed method can accurately estimate the noticeability with an MSE of 0.014 and 0.012 in the conditions of the color- and scale-based visual effects.
% Since these animations were not included in the dataset that the regression model was built on, the study demonstrates that the gaze behavioral features we extracted from the data capture a generalizable pattern of the user's visual attention and can indicate the corresponding impact on the noticeability of the offset.
% }

% \delete{
% Finally, we demonstrate applications that can benefit from the noticeability prediction model, including adaptive motion offsets and opportunistic rendering, considering the user's visual attention. 
% We conclude with discussions of our work's limitations and future research directions.
% }

% \delete{
% In summary, we make the following contributions.
% }
% % 
% \begin{itemize}
%     \item 
%     \delete{
%     We quantify the effects of the user's visual attention directed away by stimuli on their noticeability of an offset applied to the avatar's arm motion with respect to the user's physical arm. 
%     Through two user studies, we identified gaze behavioral features that are indicative of the changes in noticeability.
%     }
%     \item 
%     \delete{We build a regression model that takes the user's gaze behavioral data and the offset applied to the arm motion as input, then computes the probability of the user noticing the offset.
%     Through an evaluation study, we verified that the model needs no information about the source attracting the user's visual attention and can be generalizable in different scenarios.
%     }
%     \item 
%     \delete{We demonstrate two applications that potentially benefit from the regression model, including adaptive motion offsets and opportunistic rendering.
%     }

% \end{itemize}

\begin{comment}
However, users will lose the sense of embodiment to the virtual avatars if they notice the offset between the virtual and physical movements.
To address this, researchers have been exploring the noticing threshold of offsets with various magnitudes and proposing various redirection techniques that maintain the sense of embodiment~\cite{}.

However, when users embody virtual avatars to explore virtual environments, they encounter various visual effects and content that can attract their attention~\cite{}.
During this, the user may notice an offset when he observes the virtual movement carefully while ignoring it when the virtual contents attract his attention from the movements.
Therefore, static offset thresholds are not appropriate in dynamic scenarios.

Past research has proposed dynamic mapping techniques that adapted to users' state, such as hand moving speed~\cite{frees2007prism} or ergonomically comfortable poses~\cite{montano2017erg}, but not considering the influence of virtual content.
More specifically, PRISM~\cite{frees2007prism} proposed adjusting the C/D ratio with a non-linear mapping according to users' hand moving speed, but it might not be optimal for various virtual scenarios.
While Erg-O~\cite{montano2017erg} redirected users' virtual hands according to the virtual target's relative position to reduce physical fatigue, neglecting the change of virtual environments. 

Therefore, how to design redirection techniques in various scenarios with different visual attractions remains unknown.
To address this, we investigate how visual attention affects the noticing probability of movement offsets.
Based on our experiments, we implement a computational model that automatically computes the noticing probability of offsets under certain visual attractions.
VR application designers and developers can easily leverage our model to design redirection techniques maintaining the sense of embodiment adapt to the user's visual attention.
We implement a dynamic redirection technique with our model and demonstrate that it effectively reduces the target reaching time without reducing the sense of embodiment compared to static redirection techniques.

% Need to be refined
This paper offers the following contributions.
\begin{itemize}
    \item We investigate how visual attractions affect the noticing probability of redirection offsets.
    \item We construct a computational model to predict the noticing probability of an offset with a given visual background.
    \item We implement a dynamic redirection technique adapting to the visual background. We evaluate the technique and develop three applications to demonstrate the benefits. 
\end{itemize}



First, we conducted a controlled experiment to understand how users perceived the movement offset while subjected to various distractions.
Since hand redirection is one of the most frequently used redirections in VR interactions, we focused on the dynamic arm movements and manually added angular offsets to the' elbow joint~\cite{li2022modeling, gonzalez2022model, zenner2019estimating}. 
We employed flashing spheres in the user's field of view as distractions to attract users' visual attention.
Participants were instructed to report the appearing location of the spheres while simultaneously performing the arm movements and reporting if they perceived an offset during the movement. 
(\zhipeng{Add the results of data collection. Analyze the influence of the distance between the gaze map and the offset.}
We measured the visual attraction's magnitude with the gaze distribution on it.
Results showed that stronger distractions made it harder for users to notice the offset.)
\zhipeng{Need to rewrite. Not sure to use gaze distribution or a metric obtained from the visual content.}
Secondly, we constructed a computational model to predict the noticing probability of offsets with given visual content.
We analyzed the data from the user studies to measure the influence of visual attractions on the noticing probability of offsets.
We built a statistical model to predict the offset's noticing probability with a given visual content.
Based on the model, we implement a dynamic redirection technique to adjust the redirection offset adapted to the user's current field of view.
We evaluated the technique in a target selection task compared to no hand redirection and static hand redirection.
\zhipeng{Add the results of the evaluation.}
Results showed that the dynamic hand redirection technique significantly reduced the target selection time with similar accuracy and a comparable sense of embodiment.
Finally, we implemented three applications to demonstrate the potential benefits of the visual attention adapted dynamic redirection technique.
\end{comment}

% This one modifies arm length, not redirection
% \citeauthor{mcintosh2020iteratively} proposed an adaptation method to iteratively change the virtual avatar arm's length based on the primary tasks' performance~\cite{mcintosh2020iteratively}.



% \zhipeng{TO ADD: what is redirection}
% Redirection enables novel interactions in Virtual Reality, including redirected walking, haptic redirection, and pseudo haptics by introducing an offset to users' movement.
% \zhipeng{TO ADD: extend this sentence}
% The price of this is that users' immersiveness and embodiment in VR can be compromised when they notice the offset and perceive the virtual movement not as theirs~\cite{}.
% \zhipeng{TO ADD: extend this sentence, elaborate how the virtual environment attracts users' attention}
% Meanwhile, the visual content in the virtual environment is abundant and consistently captures users' attention, making it harder to notice the offset~\cite{}.
% While previous studies explored the noticing threshold of the offsets and optimized the redirection techniques to maintain the sense of embodiment~\cite{}, the influence of visual content on the probability of perceiving offsets remains unknown.  
% Therefore, we propose to investigate how users perceive the redirection offset when they are facing various visual attractions.


% We conducted a user study to understand how users notice the shift with visual attractions.
% We used a color-changing ball to attract the user's attention while instructing users to perform different poses with their arms and observe it meanwhile.
% \zhipeng{(Which one should be the primary task? Observe the ball should be the primary one, but if the primary task is too simple, users might allocate more attention on the secondary task and this makes the secondary task primary.)}
% \zhipeng{(We need a good and reasonable dual-task design in which users care about both their pose and the visual content, at least in the evaluation study. And we need to be able to control the visual content's magnitude and saliency maybe?)}
% We controlled the shift magnitude and direction, the user's pose, the ball's size, and the color range.
% We set the ball's color-changing interval as the independent factor.
% We collect the user's response to each shift and the color-changing times.
% Based on the collected data, we constructed a statistical model to describe the influence of visual attraction on the noticing probability.
% \zhipeng{(Are we actually controlling the attention allocation? How do we measure the attracting effect? We need uniform metrics, otherwise it is also hard for others to use our knowledge.)}
% \zhipeng{(Try to use eye gaze? The eye gaze distribution in the last five seconds to decide the attention allocation? Basically constructing a model with eye gaze distribution and noticing probability. But the user's head is moving, so the eye gaze distribution is not aligned well with the current view.)}

% \zhipeng{Saliency and EMD}
% \zhipeng{Gaze is more than just a point: Rethinking visual attention
% analysis using peripheral vision-based gaze mapping}

% Evaluation study(ideal case): based on the visual content, adjusting the redirection magnitude dynamically.

% \zhipeng{(The risk is our model's effect is trivial.)}

% Applications:
% Playing Lego while watching demo videos, we can accelerate the reaching process of bricks, and forbid the redirection during the manipulation.

% Beat saber again: but not make a lot of sense? Difficult game has complicated visual effects, while allows larger shift, but do not need large shift with high difficulty




\section{Meta-Statistical Learning}
\label{sec:meta_ml}
%\begin{figure*}[ht]
    \centering
    \includegraphics[width=\textwidth, trim=79 280 93 123, clip]{figures/framework_img.pdf}
    \caption{The pipeline of the \ENDow{} framework 
    %where each component is specified in a given configuration. 
    which yields a downstream task score and a WER score of the transcript set input to the task. The pipeline is executed for several severeties of noising and types of cleaning techniques. %Acoustic noising is applied at $k$ intensities, providing $k+1$ audio versions (including the non-noised version), eventually producing $k+2$ transcript versions (including the source transcript). Applying transcript cleaning reveals the effect of \textit{types} of noise. 
    Resulting scores are plotted on a graph for the analyses, as in, e.g., \autoref{fig_cleaning_graphs}.}
    %The pipeline is executed on $k+1$ intensities of acoustic noising (including the non-noised version), producing $k+2$ scores for the downstream task (including execution on the source transcripts). This process eventually describes the effect of the \textit{intensity} of transcript noise on the downstream task. The process is repeated for $m$ cleaning techniques ($m+1$ when including no cleaning), to analyze the benefit of a cleaning approach and the effect of the \textit{types} of transcript noise.}
    \label{fig_framework}
\end{figure*}

\section{A Framework for Measuring \ENDow{}}
\label{sec_framework}

%Inspired by the vast research conducted on investigating the effect of transcription noise on downstream NLU tasks, our goal is to formulate a framework that systematically analyzes SLU pipelines. 
With the purpose of systematically analyzing SLU pipelines,
our framework's objective is to describe the behavior of downstream tasks as a function of the noise score (e.g., WER, which we use throughout the paper, but any transcription noise metric can be applied) and the type of noise in transcripts. 
%, and what cleaning methods effectively boost the outcomes.

The \textbf{input} to the framework is an SLU dataset $D = (T, O)$, where $T$ is a set of reference transcripts and $O$ are the respective expected outcomes. For example, a set of meetings and their respective summaries, for the task of meeting summarization.
%Note that some datasets also provide audio files along with the transcripts, denoted $A$. 

The framework consists of a pipeline (illustrated in \autoref{fig_framework}) which includes a text-to-speech (TTS) model to generate audio files for $T$; the acoustic noising method and intensity to apply on the audio; an ASR system for audio transcription; the transcript cleaning technique; the downstream task model; and the evaluation metrics for the task. The components in the pipeline are flexibly set according to the use-case being analyzed.

The framework \textbf{output}s a report on the behavior of the SLU pipeline at the different noise levels and with the cleaning techniques assessed (\S{\ref{sec_framework_analysis}}).

% \paragraph{Framework input.}
% The framework receives two inputs:

% \noindent
% 1. An SLU dataset $D = (T, O)$ that contains reference transcripts ($T$) and the respective expected outcome of each transcript instance ($O$). For example, a set of meetings and their summaries. Note that some datasets also provide audio files along with the transcripts, denoted $A$.

% \noindent
% 2. A configuration specifying the components of the SLU pipeline (\S{\ref{sec_framework_noise}} and \S{\ref{sec_framework_task}}). The pipeline includes: A text-to-speech (TTS) model to generate audio files for transcripts; the noising method to apply on the audio, and number of noising intensities to apply; an ASR system for audio transcription; the transcript cleaning techniques; the downstream task model; and the evaluation metrics for the task.

% \paragraph{Framework output.}
% The framework outputs a 
% %quantitative
% report on the behavior of the SLU pipeline at different noise levels and cleaning techniques (\S{\ref{sec_framework_analysis}}).


\subsection{Preparing Transcripts with Varying Noise}
\label{sec_framework_noise}

\paragraph{Creating initial audio files.}
Audio files are first created for the input transcripts, in case the SLU dataset lacks them (or when using a non-SLU dataset), or to begin the analysis with clean audio\footnote{That is, with clear speech, without background noise or overlapping speakers.} for greater control over the subsequent noising process.
%Audio files are first prepared in several scenarios: (1) when the input SLU dataset provides transcripts without source audio files; (2) when a non-SLU dataset is used to enrich data availability; (3) to start the analysis with clean audio files\footnote{That is, with clear speech, no background environmental sounds, and without overlapping speakers.} for greater control over the subsequent noising process.
%Many SLU datasets provide transcripts without source audio files. Analogously, non-SLU datasets can be leveraged for our setting to enrich analyses.
%In some SLU datasets, the source audio files are not available.
%, and only the transcripts are on hand.
%Also, starting the analysis with clean audio files\footnote{That is, with clear speech, no background environmental sounds, and without overlapping speakers.} can provide greater control over the noising process.
%Alternatively, one may wish to initiate the \ENDow{} analysis with clean audio files\footnote{That is, with clear speech, no background sounds, and without multiple speakers at a time.} in order to have more control over the noising process.
%In such scenarios,
The TTS system is executed on each input (transcript) in dataset $D$, resulting in the corresponding set of audio files $A$.
%transcript in $T$ of input dataset $D$. The result of this step is the set of audio files $A$.
%\footnote{Technical details for the pipeline are in Appendix \ref{sec_appendix_implementation}.}

\paragraph{Adding noise to audio files.}
Given the audio files $A$, each is acoustically impaired at $k$ levels to increase transcription difficulty, preferably under realistic acoustic conditions.
%Given the set of audio files $A$, each audio file is to be acoustically impaired at $k$ levels, with the purpose of making the speech difficult to transcribe at different severities, preferably based on realistic acoustic settings. 
To that end, reverberation (i.e., sound reflection, like echoing) is applied, and background sounds are added with increasing intensity (signal-to-noise ratio) \citep{wang2018speechsep}.
%, according to the input configuration.
This stage yields a collection of audio sets $\{A_i\}_{i=1}^k$ (and we define $A_0 = A$), where the severity of impairment increases as $i$ increases.

\paragraph{Transcribing audio files.}
The ASR model is then executed on the audio files in sets $\{A_i\}_{i=0}^k$, resulting in respective transcripts $\{\widehat{T_i}\}_{i=0}^k$. Overall, there are $k+2$ sets of transcripts for dataset $D$: the $k+1$ ASR-generated sets and reference set $T$. It is expected that as $i$ increases, $\widehat{T_i}$ will have a higher WER score (more errors) with respect to $T$.
%For each $A_i$, the configured ASR model is run on each of $A_i$'s audio files to generate the respective transcript. This stage results in a collection of ASR-generated transcripts $\{\widehat{T_i}\}_{i=0}^k$. Overall, there are $k+2$ sets of transcripts for dataset $D$: the $k+1$ ASR-generated sets, and the reference set $T$. It is expected that as $i$ increases, $\widehat{T_i}$ will have a higher WER score (more errors) with respect to $T$.

\paragraph{Cleaning transcripts.}
Each non-reference transcript (in all sets $\widehat{T_i}$) is partially repaired using one of $m$ cleaning techniques. This culminates in sets $\{\{\widehat{T_{i_j}}\}_{j=1}^m\}_{i=0}^k$, and $\widehat{T_{i_0}} = \widehat{T_i}$ (when no cleaning is performed on $\widehat{T_i}$), encompassing $(k+1)*(m+1)$ different levels and types of transcript noise.
%Each non-reference transcript (those in sets $\widehat{T_i}$) is partially repaired using $m$ different cleaning techniques, as configured. In all, there are $(k+1)*(m+1)$ non-reference transcript sets with different levels and types of noise. I.e., the prepared sets are $\{\{\widehat{T_{i_j}}\}_{j=0}^m\}_{i=0}^k$, where $\widehat{T_{i_0}} = \widehat{T_i}$ (when no cleaning is performed on $\widehat{T_i}$).


\subsection{Executing the Downstream Task}
\label{sec_framework_task}
Next, the task model is executed on each of the transcripts in the prepared transcript sets, producing the respective predicted outputs $\{\{\widehat{O_{i_j}}\}_{j=0}^m\}_{i=0}^k$, and $\widehat{O}$ for the reference transcripts $T$. The predicted outputs in each set $\widehat{O_{i_j}}$ and $\widehat{O}$ are then evaluated against the respective expected outcomes in $O$. 
Finally, this process culminates with the overall score of each dataset variant $\{\{s_{i_j}\}_{j=0}^m\}_{i=0}^k$ and $s$.\footnote{To clarify, $s$ is the score obtained on reference transcripts $T$, portraying a standard execution of the SLU task on input dataset $D$. Score $s_{i_j}$ is for one of the noisy dataset variants.}
%(with respective margins of error).

In addition, the WER score is computed for each transcript set $\widehat{T_{i_j}}$ with respect to references $T$. Accordingly, this produces WER scores $\{\{w_{i_j}\}_{j=0}^m\}_{i=0}^k$ (see Appendix \ref{sec_appendix_implementation_wer} for details). Notice that $T$'s WER is $0$. With the task scores and respective WER scores, we can now assess and compare the performance of the dataset variants.


\subsection{Analyzing the Results}
\label{sec_framework_analysis}

Each of the WER and task score-pairs $(w_{i_j}, s_{i_j})$ is a data point that can be plotted on a graph.
The curve $l_j = [(0, s)] \cdot [(w_{i_j}, s_{i_j})]_{i=0}^k$ describes the behavior of a task model as noise increases in the transcripts (as $i$ increases), when applying cleaning technique $j$ (or when no cleaning is enforced, at $j=0$).
%See an illustration of the graph in \autoref{fig_framework}.
These curves form a basis for analyzing the configured SLU pipeline, as explained next.
(See \autoref{fig_framework_graph} in the Appendix for visualization.)
%The line $[(0, s)]; [(w_{i_0}, s_{i_0})]_{i=0}^k$ (when $j=0$) describes the behavior of the SLU pipeline as noise increases in the transcripts, and without any enforced cleaning. For each cleaning technique $j$, the line $[(0, s)]; [(w_{i_j}, s_{i_j})]_{i=0}^k$ describes the corresponding behavior when cleaning is enforced. For example... \todo{forward reference to an example plot}.

\paragraph{Model performance vs. noise level.}
%A curve $l_j$ describes the behavior of the NLU task model as transcripts bear more errors.
%As noise accumulates the results on the downstream task exacerbate.
As transcript noise accumulates, NLU task model performance is expected to degrade.
One question to ask is: \textit{how much transcript noise can the task model tolerate before its performance is jeopardized?} To that end, we define the \textbf{\textit{noise-toleration point}} (NTP) as follows. For curve $l_j$, described by function\footnote{Note that the curve is not continuous since it is made up of several discrete segments. See Appendix \ref{sec_appendix_implementation_ntp} for details on how the noise-toleration point is computed.} $f_j$, and the respective upper and lower bound functions $f_j^{\text{upper}}$ and $f_j^{\text{lower}}$ (based on the margins-of-error), we define $l_j$'s noise-toleration point, $w^t_j$, as the WER score when $f_j^{\text{lower}}(0) = f_j^{\text{upper}}(w^t_j)$, i.e., 
the lowest WER at which the task score becomes statistically significantly lower than when transcripts have no noise, indicating a notable drop in task-model performance due to noise.
%the minimum WER score where the task score is statistically significantly smaller than the task score when there is no noise in the transcripts, marking a significant drop in task performance.
%See the illustration in \autoref{fig_framework}.
%Nevertheless, a task model has a \textit{noise-toleration point}. This is the extent of noise until which the task-scores are relatively stable, i.e., the task model can tolerate that amount of errors in the transcripts. This point also marks the onset of a more substantial downward trend of task-scores. We define the noise-toleration point as the noise level until which the task-score does not fluctuate more than a specified threshold. \todo{Formalize with an equation?} For example, \todo{refer to a pplot in the results and say what the point is...}

Another question to ask about the SLU pipeline is: \textit{how do different models behave comparatively, with respect to noise level?} The general behavior is approximated with the \textbf{\textit{area-under-the-curve}} (AUC), which can be compared between curves to judge which model is generally more tolerant to noise. Furthermore, by focusing on a certain region in the graph, the localized behavior is comparable. For example, in \autoref{fig_noclean_graphs}a, the GPT model is the better model at lower WER levels, but drops to the bottom rank at high WER levels.\footnote{The reliability of the analyses increases with the number of points constructing a curve (increasing $k$) and with a broader coverage of the WER score range (between 0 and 1).}
%more levels of noise that enrich a model's curve (larger value of $k$ and a broader range of word-error rates).

%The behavior of a curve changes between tasks, models and noise types, and the examination described yields a practical understanding of what to expect from a tested SLU solution.


\paragraph{Comparing cleaning techniques.}
Applying a cleaning technique on transcripts decreases the noise, and consequently shifts the plots leftward. Cleaning a transcript also essentially means that the \textit{type} of noise changes, and therefore the task model reacts differently to the errors in the transcripts, potentially altering the behavior of the curves altogether.
The point $(w_{i_j}, s_{i_j})$ with respect to point $(w_{i_0}, s_{i_0})$ portrays how much ``effort'' is required (the decrease in WER: $w_{i_0} - w_{i_j}$) in order to change the task score from $s_{i_0}$ to $s_{i_j}$. The effect of each cleaning method $j$ varies, and therefore all $l_j$s are compared with respect to $l_0$ (e.g., see \autoref{fig_cleaning_graphs}). Ultimately, an effective cleaning technique should increase the task scores with minimum effort.

Formally, let $\Delta w_{i_j} = w_{i_0} - w_{i_j}$ be the change in WER for noising level $i$ and cleaning method $j$, and $\delta s_{i_j} = (s_{i_0} - s_{i_j}) / s$ be the respective relative\footnote{The change in task-score is normalized by the score at WER=0 to get the relative change. The change in WER is already on a 0-to-1 scale, and is not further normalized.} change in the task-score. The pointwise effectiveness score of cleaning technique $j$ at noise-level $i$ is measured as $e_{i_j} = \delta s_{i_j} / \sqrt{\Delta w_{i_j} + \epsilon}$.\footnote{We applied a square root transformation on the \textit{effort} ($\Delta w_{i_j}$) to reduce the impact of the larger changes at noisier levels, and to increase the weight of the change in task score ($\delta s_{i_j}$). $\epsilon$ is a minuscule value to prevent division by zero.} Finally, we measure the \textit{\textbf{cleaning-effectiveness score}} (CES) of cleaning method $j$ with the average: $\frac{1}{k+1} \sum_{i=0}^k e_{i_j}$. The higher the score, the better the overall improvement in the downstream task with a lower effort of cleaning. A score of 0 means that the cleaning procedure had no effect on the task-model's results, and a negative score means that there was a deterioration of task results, on average. 

The CES metric captures the two objectives of a cleaning technique: heightened task results for lesser effort.
%\todo{try out the acceleration of the changes -- get the derivative of the regression line formed by the delta\_y / delta\_x line -- the larger the acceleration, the better.}
The metric suggests how comparably effective a cleaning method is for the data and task-model in question. As such, it compares the effects of different \textit{types} of noise in the transcripts, as we exemplify in our experiments in Section \ref{sec_results}.




%The more levels of impairment, the more precise the \ENDow{} assessment will be

\section{Experimental Setup}
\label{sec:experimental_setup}
\section{Experimental Setup}

We evaluate state-of-the-art IARs: VAR-\textit{d}\{16, 20, 24, 30\} (\textit{d} = model depth), RAR-\{B, L, XL, XXL\}, MAR-\{B, L, H\}, trained for class-conditioned generation. The IARs' sizes cover a broad spectrum between 208M for MAR-B, and 2.1B parameters for \varbig. We use IARs shared by the authors of their respective papers in their repositories, 
with details in~\cref{app:model_details}. As these models were trained on ImageNet-1k~\citep{deng2009imagenet} dataset, we use it to perform our privacy attacks. 
For MIA and DI, we take 10000 samples from the training set as members and also 10000 samples from the validation set as non-members. To perform data extraction attack, we use all images from the training data. Additionally, we leverage the known validation set to check for false positives.




\section{Experiments on Descriptive Tasks}
\label{sec:experiments_desc}
In descriptive tasks, the label \( y \) of a dataset \( \mathcal{D} \) is the output of an algorithm \( A \) applied to \( \mathcal{D} \), i.e., \( y = A(\mathcal{D}) \). Simple tasks like median or correlation serve as unit testing of meta-statistical models. However, for more computationally intensive algorithms, such as optimal transport, meta-statistical models could serve as fast approximations. For datasets \( \mathcal{D} \in \mathbb{R}^{n \times m} \), we consider four descriptive tasks: the \textbf{per-column median} label \( y \in \mathbb{R}^m \) consists of the medians of each column. The \textbf{Pearson correlation} coefficient \( y \in \mathbb{R} \) is computed between the two columns. The \textbf{win rate} (Bradley-Terry) is the fraction of rows where the value in the first column exceeds that in the second: \( y = \frac{1}{n} \sum_{i=1}^n \mathbb{I}(\mathcal{D}_{i,1} > \mathcal{D}_{i,2})\), where \( \mathbb{I}(\cdot) \) is the indicator function. Finally, the 1D \textbf{optimal transport} (OT) label \( y \in \mathbb{R} \) is the optimal transport cost between the empirical distributions of the two columns.


\xhdr{Meta-Dataset Generation}
To construct the meta-dataset, we sample datasets \( D \) from predefined probability distributions as described in \Secref{sec:experimental_setup}. Once a dataset is sampled we simply compute the target label \( y \) by applying the target algorithm. We experiment with various numbers of columns $m$. By having $k > 1$, we produce $k$ computation in parallel with one forward pass (independently of the batch dimension). We observe no significant difference when varying $k$ and fix $k=2$ in the experiments. The meta-dataset contains $30K$ training meta datapoints per task, with dataset sizes sampled from $n \in [5, 300]$. Details about meta-datasets and which distribution families are in- or out-of-meta-distribution are provided in \Appref{app:desc_details}.


\xhdr{Meta-Statistical Models}  
After optimizing hyperparameters and architecture choices (e.g., pooling mechanisms and head-to-dimensionality ratio) on a small validation set of 1K meta datapoints, we compare four meta-statistical model variants: LSTM, Vanilla Transformer (VT), and two ST2 variants with 16 or 32 inducing points. ST2(16) is the fastest model for both training and inference. In \Appref{app:eff}, we show that VT scales quadratically, while LSTM and ST2 scale linearly, with better slopes for ST2. Additionally, ST2(16) achieves a 12x faster training time per batch normalized by parameters compared to VT, meaning an ST2(16) model with 12 times more parameters can be trained in the same time as VT. However, for consistency in reporting, we compare models with approximately the same number of parameters (\( \sim 10K \) in this section).


\begin{figure}[t] 
    \centering
    \begin{subfigure}[t]{0.49\textwidth}
        \centering
        \includegraphics[width=\textwidth]{images/per_column_median.pdf} 
        \caption{\texttt{Median} prediction}
        \label{fig:subfig_b}
    \end{subfigure}
    \vspace{0.5cm} 
    \begin{subfigure}[t]{0.49\textwidth}
        \centering
        \includegraphics[width=\textwidth]{images/correlation.pdf}
        \caption{\texttt{Correlation} prediction}
        \label{fig:subfig_c}
    \end{subfigure}
    \hfill
    \vspace{-0.5cm}
    \caption{\textbf{Generalization across dataset lengths and meta-distributions.} The left panel shows MSE as a function of dataset length for in-meta-distribution datasets, while the right panel displays the same for out-of-meta-distribution datasets. The vertical red line marks the largest dataset seen during training ($n = 300$). LSTM is excluded due to its errors being an order of magnitude higher. Additional tasks can be found in \Appref{app:gen_plots}.}
    \label{fig:generalization}
\end{figure}

\xhdr{In-meta-distribution performance}  
\Tabref{tab:performance_comparison} shows the MSE of the four meta-statistical models on a test set sampled from the same meta-distribution as the training data. All models approximate the descriptive tasks well, but the LSTM-based model, lacking permutation invariance, performs worse than attention-based models. 
Notably, ST2, despite being much faster than VT, narrowly outperforms it. Given its strength and efficiency, ST2(16) is our main model in the rest of the paper, with VT considered as an alternative baseline.

\xhdr{Generalization Performance}  
We evaluate meta-statistical models' generalization capabilities on two aspects:  (i) \textbf{Out-of-Meta-Distribution (OoMD)}: Datasets from unseen distributions. (ii) \textbf{Length Generalization}: Datasets with lengths outside the training range. \Figref{fig:generalization} shows strong length generalization, where models maintain their performance for larger datasets than seen during training, both IMD and OoMD. 
They are also robust to OoMD datasets despite a small performance degradation. Manual inspection reveals that the degradation mainly comes from cases where the magnitude of the input values exceeds the range seen during training. This is discussed further in \Secref{sec:discussion}.
Additional results and generalization plots are provided in \Appref{app:desc_details}.


\begin{table}[t]
\centering
\resizebox{0.95\columnwidth}{!}{
\begin{tabular}{@{}l|c|c|c|c@{}}
\toprule
& \textbf{Median} & \textbf{Corr} & \textbf{WinRate (BT)} & \textbf{OT (1D)} \\ 
\midrule
\midrule
LSTM
& $2.9e^{-1}$ \scriptsize{$\pm 0.8$}
& $5.9e^{-2}$ \scriptsize{$\pm 1.5$}
& $4.4e^{-2}$ \scriptsize{$\pm 0.9$}
& $8.5e^{-2}$ \scriptsize{$\pm 2.9$}\\
VT
& $\mathbf{6.0e^{-2}}$ \scriptsize{$\pm 1.9$}
& $\mathbf{9.2e^{-3}}$ \scriptsize{$\pm 4.6$}
& $7.1e^{-3}$ \scriptsize{$\pm 1.5$}
& $\mathbf{5.5e^{-2}}$ \scriptsize{$\pm 1.4$}\\
ST2(16)
& $\mathbf{4.2e^{-2}}$ \scriptsize{$\pm 1.7$}
& $\mathbf{7.5e^{-3}}$ \scriptsize{$\pm 2.8$}
& $\mathbf{2.9e^{-3}}$ \scriptsize{$\pm 1.2$}
& $\mathbf{4.5e^{-2}}$ \scriptsize{$\pm 1.9$}\\
ST2(32)
& $\mathbf{4.4e^{-2}}$ \scriptsize{$\pm 0.9$}
& $\mathbf{9.1e^{-3}}$ \scriptsize{$\pm 5.1$}
& $\mathbf{1.6e^{-2}}$ \scriptsize{$\pm 0.5$}
& $\mathbf{3.0e^{-2}}$ \scriptsize{$\pm 1.5$}\\
\bottomrule

\end{tabular}
}
\caption{Performance comparison meta-statistical models across tasks, measured by Mean Squared Error with respect to correct output on the test set. \textbf{Bold} indicates no significant difference with the best model.}
\label{tab:performance_comparison}
\end{table}


\section{Experiments on Inferential Tasks}
\label{sec:experiments_inf}
In inferential tasks, the label \( y \) represents a property \( g \) of the underlying distribution \( P_X \) from which a dataset \( \mathcal{D} \) is sampled: \( y = g(P_X) \). We illustrate the meta-statistical framework with three such tasks: standard deviation estimation, normality testing, and mutual information estimation. Details on meta-dataset creation and models are in \Appref{app:std}. For all tasks in this section, the dataset sizes during training are sampled from $n \in [5, 150]$, depicted by vertical red lines in the plots. 

\subsection{Standard Deviation Estimation}  
The standard deviation (\( \sigma = \sqrt{\mathbb{E}[(X - \mathbb{E}[X])^2]}\)) quantifies the spread of a distribution \( P_X \). Unlike the mean or variance, estimating \( \sigma \) is non-trivial due to the square root's non-linearity \cite{gurland1971simple,gupta1952estimation}. In fact, no universal unbiased estimator exists across all distributions \cite{gurland1971simple,fenstad1980robust}. We use this task to show meta-statistical learning in action.  


\xhdr{Meta-Dataset} To create the meta-dataset, we follow the procedure outlined in \Secref{sec:experimental_setup}, keeping different distribution families for in- and out-of-meta-distribution. We use 100K meta datapoints for training.

\begin{figure}
    \centering
    \includegraphics[width=0.85\linewidth]{images/std_line_plot.pdf}
    \caption{MSE of $\sigma$ estimators as a function of dataset sizes, for dataset sampled \textbf{out-of-meta-distribution}.}
    \label{fig:std_line_plot}
\end{figure}


\xhdr{Meta-Statistical Model}  
We train two ST2-based models: ST2$_{\text{std}}$, which predicts the standard deviation \( \sigma \), and ST2$_{\text{fsd}}$, which estimates the finite sample correction to apply to the sample standard deviation, defined as \( y = \sigma - \text{np.std}(X) \). This allows constructing a corrected estimator by adjusting \texttt{np.std} with ST2$_{\text{fsd}}$'s predictions. Both models share the same architecture: 16 inducing points, five hidden layers (128 dimensions), and 12 attention heads per layer, totaling around 950K parameters.  


\xhdr{Results} 
We compare the ST2-based estimator to the sample standard deviation (\texttt{np.std} with Bessel’s correction) across dataset lengths for out-of-meta-distribution scenarios in \Figref{fig:std_line_plot}. ST2$_{\text{std}}$ achieves strong MSE performance, converging to a low error in high-sample sizes. Also, the learned correction from ST2$_{\text{fsd}}$ further reduces the bias of \texttt{np.std}, effectively capturing finite sample errors. Notably, ST2$_{\text{fsd}}$ also lowers variance of \texttt{np.std}, suggesting the correction is data-dependent rather than a fixed offset. 
Full tables of bias, variance, and MSE across distributions and dataset lengths are provided in \Appref{app:std}, confirming these observations.


\subsection{Normality Testing}
The task is now to determine whether a dataset \( \mathcal{D} \sim P_X \) originates from a normal distribution, formulated as a binary classification task: \( y = 1 \) if \( P_X \) is normal, \( y = 0 \) otherwise. Normality testing is crucial in hypothesis testing, model selection, and preprocessing \cite{shapiro1965analysis, razali2011power}, particularly before applying t-tests, linear regression, or ANOVA with small samples \cite{altman1990practical, das2016brief, doi:10.4078/jrd.2019.26.1.5}. However, standard tests struggle in low-sample settings \cite{razali2011power}. We propose to train meta-statistical models for normality classification, aiming for robust generalization in such regimes. Details on meta-dataset creation and model properties are in \Appref{app:norm_test}.

\begin{figure}[t]
    \centering
    \includegraphics[width=0.85\linewidth]{images/norm_test_oomd.pdf}
    \caption{Accuracy of normality classifiers as a function of dataset sizes. The non-normal distributions are sampled \textbf{out-of-meta distribution} for meta-statistical models.}
    \label{fig:norm_t_oomd}
    \vspace{-0.5cm}
\end{figure}


\xhdr{Meta-dataset creation}  
We construct a balanced meta-dataset of normally and not normally distributed datasets following the process described in \Secref{sec:experimental_setup}. For non-normality, we choose an alternative distribution from a predefined set detailed in \Appref{app:norm_test}. We use 40K meta datapoints for training. 

\xhdr{Estimators}  
We transform traditional normality tests into binary classifiers by thresholding their \( p \)-values, optimizing the threshold on the training meta-dataset for maximum classification accuracy. We consider four widely used tests: the \textit{Shapiro-Wilk test} \cite{shapiro1965analysis}, known to be effective for small samples \cite{razali2011power}; the \textit{D'Agostino-Pearson test} \cite{d1973tests}, which combines skewness and kurtosis; the \textit{Kolmogorov-Smirnov test} \cite{massey1951kolmogorov}, a non-parametric test based on cumulative distribution differences; and the \textit{Jarque-Bera test} \cite{jarque1987test}, which assesses skewness and kurtosis deviations from theoretical expectations.

We then train two meta-statistical models: one based on VT and another on ST2 with 16 inducing points. Both use four layers, a hidden dimensionality of 32, and 12 attention heads. The classification head is a single-layer MLP with 32 neurons, totaling approximately 50K parameters per model.

\begin{table}[h]
\centering
\begin{tabular}{l|cccc}
\toprule
 & Accuracy $\uparrow$ & AuROC $\uparrow$ & Brier $\downarrow$ & BT $\uparrow$ \\
\midrule
KS & $0.88$ {\scriptsize± $0.01$} & $0.93$ {\scriptsize± $0.01$} & $0.09$ {\scriptsize± $0.01$} & $0.12$ {\scriptsize± $0.02$} \\
SW & $0.89$ {\scriptsize± $0.01$} & $0.95$ {\scriptsize± $0.01$} & $0.18$ {\scriptsize± $0.01$} & $0.16$ {\scriptsize± $0.03$} \\
JB & $0.88$ {\scriptsize± $0.01$} & $0.93$ {\scriptsize± $0.01$} & $0.16$ {\scriptsize± $0.01$} & $0.13$ {\scriptsize± $0.02$} \\
AP & $0.90$ {\scriptsize± $0.01$} & $0.95$ {\scriptsize± $0.01$} & $0.18$ {\scriptsize± $0.01$} & $0.17$ {\scriptsize± $0.03$} \\
ST2 & $\mathbf{0.92}$ {\scriptsize± $0.01$} & $\mathbf{0.97}$ {\scriptsize± $0.01$} & $\mathbf{0.06}$ {\scriptsize± $0.01$} & $\mathbf{0.25}$ {\scriptsize± $0.04$} \\
VT & $0.91$ {\scriptsize± $0.01$} & $\mathbf{0.97}$ {\scriptsize± $0.01$} & $\mathbf{0.07}$ {\scriptsize± $0.01$} & $0.17$ {\scriptsize± $0.03$} \\
\bottomrule
\end{tabular}
\caption{Normality test classifiers with datasets drawn from Gaussian or Uniform distributions of sizes $n \in [10, 300]$. AuROC refers to the area under the ROC curve, Brier loss is the calibration error, and BT measures the relative strengths of classifiers in a paired evaluation.}
\label{tab:normality_test}
\end{table}


\xhdr{Results}  
\Figref{fig:norm_t_oomd} summarizes the accuracy of the proposed meta-statistical models, in settings where negative labels correspond to datasets sampled from distribution families unseen during training. Consistent with prior comparisons of normality tests, Shapiro-Wilk and D'Agostino-Pearson perform best among the baselines \cite{razali2011power}. Across all dataset sizes, meta-statistical models consistently and largely outperform baselines, with particularly strong gains in small-sample settings (\( n < 100 \)), making them highly relevant for biomedical applications \cite{doi:10.4078/jrd.2019.26.1.5}. Meta-statistical models achieve near-perfect accuracy (\( > 0.98 \)) as \( n \) increases demonstrating their consistency. Overall, this task seems relatively easy for meta-statistical models, which generalize smoothly out-of-meta distribution. However, note that the training of meta-statistical models could be harder if the input datasets are standardized during training (see \Secref{sec:discussion}). 

While classification lacks a direct bias-variance formulation, we analyze false positive and false negative rates as well as precision and recall in \Appref{app:prec_rec}, showing more balanced error profiles for meta-statistical estimators. In \Tabref{tab:normality_test}, we present key metrics for evaluating classifier performance: the Area Under the Receiver Operating Characteristic Curve (AuROC), the Brier Score, and the Bradley-Terry (BT) scores from a paired evaluation.
The AuROC measures a classifier's ability to discriminate between positive and negative classes across different decision thresholds. A higher AuROC indicates better separability. Unlike accuracy, AuROC provides a threshold-independent measure of performance. The Brier loss \cite{brier1950verification} quantifies the calibration of a model’s predicted probabilities. Lower values indicate better calibration. The Bradley-Terry (BT) score \cite{bradley1952rank,NIPS2004_825f9cd5} ranks models based on pairwise comparisons, assessing how often one classifier outperforms another across test instances \cite{peyrard-etal-2021-better,colombo-etal-2023-glass}. The accuracy scores are lower than those of \Figref{fig:norm_t_oomd} because the uniform is among the hardest out-of-meta-distribution to recognize as non-Gaussian. Still, across metrics, the meta-statistical estimators perform strongly. In particular, we find it interesting that they are particularly well-calibrated.


\subsection{Mutual Information Estimation}
Mutual information (MI) quantifies the dependency between two random variables \(X\) and \(Y\) and is defined as:
\[
\mathrm{MI}(X; Y) = \int \int P_{X,Y}(x, y) \log \frac{P_{X,Y}(x, y)}{P_X(x) P_Y(y)} \, dx \, dy.
\]
Here, \(P_X\) and \(P_Y\) denote the marginal distributions of \(X\) and \(Y\), respectively.

MI possesses key properties such as invariance to homeomorphisms and adherence to the \textit{Data Processing Inequality}, making it fundamental in machine learning and related fields \cite{inv_bottleneck, pmlr-v80-belghazi18a, repr_learning, tishby2000informationbottleneckmethod}. However, MI estimation remains challenging, particularly for small sample sizes and non-Gaussian distributions \cite{song2020understandinglimitationsvariationalmutual, pmlr-v108-mcallester20a, NEURIPS2023_36b80eae}. 

We adopt a meta-statistical approach, training models to predict \( y = \mathrm{MI}(X; Y) \) between two dataset columns. Focusing on low-sample, non-Gaussian settings, but we restrict experiments to the one-dimensional case for simplicity. Details on meta-dataset creation, models, and extra results are provided in \Appref{app:mi_details}.


\xhdr{Meta-dataset Creation}
We construct a meta-dataset inspired by the benchmark methodology in \cite{NEURIPS2023_36b80eae}, where distributions with ground-truth MI are generated in two steps: (i) by sampling a distribution with known MI, (ii) optionally applying MI-preserving transformations. This process creates complex distributions and datasets with known MI. For generating meta-dataset in this way, we again follow the process described in \Secref{sec:experimental_setup} using different base-distribution and MI-preserving transformations between in-meta-distribution and out-of-meta-distribution. We use 50K meta datapoints for training.


\xhdr{Estimators}
We compare our approach with the best-performing 1D estimators from \cite{NEURIPS2023_36b80eae}, including Kraskov-St\"ogbauer-Grassberger (KSG) \cite{PhysRevE.69.066138}, Canonical Correlation Analysis (CCA) \cite{pml2Book}, and three neural estimators: MINE \cite{pmlr-v80-belghazi18a}, InfoNCE \cite{repr_learning}, and NWJE \cite{NIPS2007_72da7fd6, NIPS2016_cedebb6e, pmlr-v97-poole19a}. We train two meta-statistical models: one based on Vanilla Transformer (VT) and the other on Set Transformer 2 (ST2). Both models consist of five layers, with a hidden dimensionality of 256 and 12 attention heads. The regression head is a single hidden-layer MLP with 128 neurons, resulting in models with approximately 1M parameters.


\xhdr{Estimation Performance}
The mean squared error (MSE) results for both in- and out-of-meta-distribution testing are shown in \Tabref{tab:mi_test}. Meta-statistical models outperform baseline estimators across all sample sizes, with significant advantages in low-sample scenarios. Baseline models, particularly neural ones, struggle with small sample sizes, while only KSG and CCA begin to match meta-statistical models for sample sizes greater than 100 in the out-of-meta-distribution regime.

\xhdr{Bias and Variance of MI Estimators}
We examine the bias and variance of MI estimators by resampling datasets from fixed distributions and measuring the variance and bias of the estimates. In \Figref{fig:mi_bias_var}, we visualize the bias and variance for a challenging distribution identified by previous works \cite{NEURIPS2023_36b80eae} (additive noise). Even at a sample size of $n = 100$, meta-statistical models show clear improvements in both bias (estimates centered around 0) and variance. A more detailed analysis of bias and variance is available in \Appref{app:mi_details} (\Tabref{tab:mi_bias_variance}). Compared to baseline estimators, meta-statistical models demonstrate significantly lower bias, close to zero, and lower or comparable variance. These results are promising, suggesting that further scaling could create even more robust meta-statistical MI estimators. Currently, the ST2 model can be trained in less than an hour on a single GPU, with inference orders of magnitude faster than existing neural baselines.


\begin{figure}
    \centering
    \includegraphics[width=0.9\linewidth]{images/additive-noise.pdf}
    \caption{We estimate statistics for MI estimators over 150 resampled datasets of size \( n = 100 \) from a fixed distribution (additive noise \cite{NEURIPS2023_36b80eae}). Each dot represents the difference between an estimate and the true mutual information (MI).}
    \label{fig:mi_bias_var}
\end{figure}

\begin{table}[t]
\centering
\resizebox{0.95\columnwidth}{!}{
\begin{tabular}{@{}l|cc|cc@{}}
\toprule
& \multicolumn{2}{c|}{IMD} & \multicolumn{2}{c}{OoMD} \\ 
% \cmidrule(lr){2-4} \cmidrule(lr){5-7}
$n \in $ & $[10, 100]$ & $[100, 200]$ & $[10, 100]$ & $[100, 200]$ \\ 
\midrule
\midrule
CCA
& $7.4e^{-2}$ \scriptsize{$\pm 9.3$}
& $1.4e^{-2}$ \scriptsize{$\pm 1.1$}
% ----
& $1.3e^{-1}$ \scriptsize{$\pm 1.2$}
& $4.9e^{-2}$ \scriptsize{$\pm 3.3$}\\
KSG
& $2.9e^{-2}$ \scriptsize{$\pm 1.5$}
& $7.8e^{-3}$ \scriptsize{$\pm 3.5$}
% ----
& $\mathbf{1.2e^{-2}}$ \scriptsize{$\pm 0.4$}
& $\mathbf{7.2e^{-3}}$ \scriptsize{$\pm 2.3$}\\
MINE
& $2.5e^{0}$ \scriptsize{$\pm 2.4$}
& $2.8e^{-2}$ \scriptsize{$\pm 1.2$}
% ---
& $5.4e^{0}$ \scriptsize{$\pm 7.7$}
& $1.6e^{-1}$ \scriptsize{$\pm 2.1$}\\
NWJE
& -
& $7.3e^{-2}$ \scriptsize{$\pm 4.7$}
% --
& $6.6e^{0}$ \scriptsize{$\pm 7.5$}
& $6.3e^{-2}$ \scriptsize{$\pm 5.4$}\\
InfoNCE
& $1.5e^{1}$ \scriptsize{$\pm 2.2$}
& $1.9e^{-2}$ \scriptsize{$\pm 0.7$}
% ---
& $2.3e^{1}$ \scriptsize{$\pm 3.4$}
& $3.4e^{-1}$ \scriptsize{$\pm 4.5$}\\
\midrule
VT
& $\mathbf{4.6e^{-3}}$ \scriptsize{$\pm 2.4$}
& $\mathbf{2.5e^{-3}}$ \scriptsize{$\pm 1.3$}
% ---
& $\mathbf{1.5e^{-2}}$ \scriptsize{$\pm 0.8$}
& $\mathbf{7.7e^{-3}}$ \scriptsize{$\pm 3.2$}\\
ST2(16)
& $\mathbf{6.2e^{-3}}$ \scriptsize{$\pm 3.0$}
& $\mathbf{2.4e^{-3}}$ \scriptsize{$\pm 1.1$}
% ---
& $\mathbf{1.3e^{-2}}$ \scriptsize{$\pm 0.7$}
& $\mathbf{8.5e^{-3}}$ \scriptsize{$\pm 3.1$}\\
\bottomrule
\end{tabular}
}
\caption{MSE loss of mutual information estimators both in- and out-of-meta-distribution. \textbf{Bold} indicates no significant difference with the best estimator.}
\label{tab:mi_test}
\end{table}

\section{Discussion}
\label{sec:discussion}
\section{Discussion and Conclusion}
\label{sec:discussion}


\textbf{Conclusion.} In this paper, we propose LRM to utilize diffusion models for step-level reward modeling, based on the insights that diffusion models possess text-image alignment abilities and can perceive noisy latent images across different timesteps. To facilitate the training of LRM, the MPCF strategy is introduced to address the inconsistent preference issue in LRM's training data. We further propose LPO, a method that employs LRM for step-level preference optimization, operating entirely within the latent space. LPO not only significantly reduces training time but also delivers remarkable performance improvements across various evaluation dimensions, highlighting the effectiveness of employing the diffusion model itself to guide its preference optimization. We hope our findings can open new avenues for research in preference optimization for diffusion models and contribute to advancing the field of visual generation.

\textbf{Limitations and Future Work.} (1) The experiments in this work are conducted on UNet-based models and the DDPM scheduling method. Further research is needed to adapt these findings to larger DiT-based models \cite{sd3} and flow matching methods \cite{flow_match}. (2) The Pick-a-Pic dataset mainly contains images generated by SD1.5 and SDXL, which generally exhibit low image quality. Introducing higher-quality images is expected to enhance the generalization of the LRM. (3) As a step-level reward model, the LRM can be easily applied to reward fine-tuning methods \cite{alignprop, draft}, avoiding lengthy inference chain backpropagation and significantly accelerating the training speed. (4) The LRM can also extend the best-of-N approach to a step-level version, enabling exploration and selection at each step of image generation, thereby achieving inference-time optimization similar to GPT-o1 \cite{gpt_o1}.


% \section*{Acknowledgments}

\section*{Impact Statements}
This paper presents work whose goal is to advance the field of Machine Learning. Meta-statistical learning aims to enhance inference in low-sample settings, benefiting applied fields of Science like medicine and economics by improving estimator reliability. Learned estimators may inherit biases from the data they are trained on, potentially leading to misleading conclusions if not carefully validated. Further, as with any data-driven methodology, interpretability remains a challenge; understanding why a model makes a particular statistical inference is crucial for scientific rigor. 

% This paper presents work whose goal is to advance the field of Machine Learning. There are many potential societal consequences of our work, none which we feel must be specifically highlighted here.


% \clearpage



% \section*{Glossary of Important Terms}
% \label{sec:glossary}
% \input{__glossary}


% Entries for the entire Anthology, followed by custom entries
% \clearpage
\bibliography{anthology,main}
\bibliographystyle{acl_natbib}

% 
\appendix
\clearpage

\newpage
\appendix
\appendices

\section{Robot Setups}\label{app:robot_setup}

\subsection{Simulation Robot Setups}
To ensure fairness, we utilize the same Franka Panda arm for evaluations in both the LIBERO~\cite{LIBERO23} and our Open6DOR V2 benchmarks. For SIMPLER~\cite{simplerenv24}, we use the Google Robot exclusively to conduct the baseline experiments, adhering to all configurations outlined in SIMPLER, as presented in Table~\ref{tab:simpler_env}. 

\subsection{Real World Robot Setups}
As for manipulation tasks, in \cref{fig:robots}, we perform 6-DoF rearrangement tasks using the Franka Panda equipped with a gripper and the UR robot arm with a LeapHand, while articulated object manipulation is conducted using the Flexiv arm equipped with a suction tool. All the robot arms mount a Realsense D415 camera to its end for image capturing.
\begin{figure}[h!]
\centering
\includegraphics[width=0.96\linewidth]{figs/src/robots.pdf}
\vspace{-5pt}
\captionof{figure}{\textbf{The robots used in our real-world experiments.}}
\vspace{-5pt}
\label{fig:robots}
\end{figure}


In \cref{fig:franka_setup}, we present the workspace and robotic arm for real-world 6-DoF rearrangement. Unlike Rekep~\cite{ReKep24}, CoPa~\cite{CoPa24} et al., we utilize only a single RealSense D415 camera. This setup significantly reduces the additional overhead associated with environmental setup and multi-camera calibration, and it is more readily reproducible.
\begin{figure}[h!]
\centering
\includegraphics[width=1.0\linewidth]{figs/src/franka_setup.pdf}
\captionof{figure}{\textbf{6-DoF rearrangement robot setup.}}
\vspace{-10pt}
\label{fig:franka_setup}
\end{figure}


As for navigation tasks, we provide a visualization of our robotic dog in~\cref{fig:dog_setup}. Following Uni-Navid~\cite{uninavid24}, our robotic dog is Unitree GO2 and we mount a RealSense D455 camera on the head of the robotic dog. Here, we only use the RGB frames with a resolution of $640\times480$ in the setting of  $90^\circ$ HFOV. We also mount a portable Wi-Fi at the back of the robot dog, which is used to communicate with the remote server (send captured images and receive commands). Unitree GO2 is integrated with a LiDAR-L1, which is only used for local motion planning. 
\begin{figure}
\begin{center}
  \includegraphics[width=0.7\linewidth]{figs/src/robotdog.pdf}
\end{center}
   \caption{\textbf{Navigation robot setup.} We use Unitree GO2 as our embodiment, and we mount RealSense D455, a portable Wi-Fi and a LiDAR-L1. Note that, our model only takes RGB frames as input. The portable Wi-Fi is used for communication with the remote server and the Lidar is used for the local controller API of Unitree Dog.}
   \vspace{-10pt}
\label{fig:dog_setup}
\end{figure}


\section{Additional Experiments}\label{app:add_exp}

\subsection{Articulated Objects Manipulation Evaluation}
We further integrate \ours~with articulated object manipulation, as illustrated in \cref{tab:manip}, and evaluate its practicality in robotic manipulation tasks using the PartNet-Mobility Dataset within the SAPIEN~\cite{SAPIEN20} simulator. Our experimental setup follows ManipLLM~\cite{ManipLLM24}, employing the same evaluation metrics. Specifically, we directly utilize the segmentation centers provided by SAM as contact points, leverage PointSO to generate contact directions, and use VLM to determine subsequent motion directions. The results demonstrate significant improvements over the baseline. Notably, our model achieves this performance without dividing the data into training and testing sets, operating instead in a fully zero-shot across most tasks. This underscores the robustness and generalization of our approach.
% !TEX root = ../../top.tex
% !TEX spellcheck = en-US

% Temporary modification of the table column separation width
\setlength\mytabcolsep{\tabcolsep}
\setlength\tabcolsep{2pt}


\renewcommand{\manipimgcar}[1]{\includegraphics[width=0.13\linewidth]{#1}}
\renewcommand{\manipimgmix}[1]{\includegraphics[width=0.07\linewidth]{#1}}
\renewcommand{\manipimg}[1]{\includegraphics[width=0.08\linewidth]{#1}}


\begin{figure*}[t]
	\centering
	\small
%	\vspace{-3mm}
	\begin{tabular}{c|c|c}
		% Cars 
		\begin{tabular}{c|c}
			\manipimgcar{fig/manip/car_olivier_lat_489_277/init} & \manipimgcar{fig/manip/car_olivier_lat_34_308/init} \\
			\manipimgcar{fig/manip/car_olivier_lat_489_277/final} & \manipimgcar{fig/manip/car_olivier_lat_34_308/final} \\
		\end{tabular}
		&
		% Mixers 
		\begin{tabular}{cc|cc}
			\manipimgmix{fig/manip/mixer_lat_1424_171/init} & \manipimgmix{fig/manip/mixer_lat_1424_171/final} & \manipimgmix{fig/manip/mixer_lat_957_629/init} & \manipimgmix{fig/manip/mixer_lat_957_629/final} \\
		\end{tabular}
		&
		% Chairs
		\begin{tabular}{cc|cc}
			\manipimg{fig/manip/chair_sepreg_lat_550_531/init} & \manipimg{fig/manip/chair_sepreg_lat_550_531/final} & \manipimg{fig/manip/chair_sepreg_lat_745_952/init} & \manipimg{fig/manip/chair_sepreg_lat_745_952/final} \\
		\end{tabular} \\
		\midrule
		% Cars 2
		\begin{tabular}{c|c}
		\manipimgcar{fig/manip/car_olivier_pose_290/init} & \manipimgcar{fig/manip/car_olivier_pose_691/init} \\
		\manipimgcar{fig/manip/car_olivier_pose_290/final} & \manipimgcar{fig/manip/car_olivier_pose_691/final} \\
		\end{tabular}
		&
		% Mixers 2
		\begin{tabular}{cc|cc}
		\manipimgmix{fig/manip/mixer_pose_872/init} & \manipimgmix{fig/manip/mixer_pose_872/final} & \manipimgmix{fig/manip/mixer_pose_183/init} & \manipimgmix{fig/manip/mixer_pose_183/final} \\
		\end{tabular}
		&
		% Chairs 2
		\begin{tabular}{cc|cc}
		\manipimg{fig/manip/chair_sepreg_pose_422/init} & \manipimg{fig/manip/chair_sepreg_pose_422/final} & \manipimg{fig/manip/chair_sepreg_pose_859/init} & \manipimg{fig/manip/chair_sepreg_pose_859/final} \\
		\end{tabular} \\
	\end{tabular}
	\caption{\textbf{Additional shape manipulation.} We manipulate four shapes per dataset: (\textit{top}) by changing the latent of specific parts (car body, mixer helix, and all chair parts) and (\textit{bottom}) by editing part poses (car wheels, mixer width, chair width and height). In all cases, the parts adapts to the modifications and to each other, maintaining a coherent whole.}
	\label{fig:supp-manip}
\end{figure*}


% Restore table column separation width
\setlength{\tabcolsep}{\mytabcolsep}


\subsection{Spatial VQA on EmbSpatial-Bench~\cite{embspatial24} \& SpatialBot-Bench~\cite{SpatialBot24}}
To further demonstrate \sofar's spatial reasoning capabilities, we conducted Spatial VQA tests within the EmbSpatial-Bench~\cite{embspatial24} and SpatialBot-Bench~\cite{SpatialBot24}. As shown in \cref{tab:embspatial,tab:spatialbot}, \sofar~significantly outperformed all baselines, achieving more than a 20\% performance improvement in EmbSpatial-Bench.
\begin{table}[h!]
\centering
\setlength{\tabcolsep}{8.5pt}
\caption{Zero-shot performance of LVLMs in EmbSpatial-Bench~\cite{embspatial24}. \textbf{Bold} indicates the best results.}
\resizebox{0.97\linewidth}{!}{
\begin{tabular}{lcc}
\toprule
Model & Generation & Likelihood \\
\midrule
BLIP-2~\cite{BLIP223} & 37.99 & 35.71 \\
InstructBLIP~\cite{InstructBLIP23} & 38.85 & 33.41 \\
MiniGPT4~\cite{MiniGPT4_23} & 23.54 & 31.70 \\
LLaVA-1.6~\cite{LLaVA23} & 35.19 & 38.84 \\
\midrule
GPT-4V~\cite{GPT4Vision23} & 36.07 & - \\
\rowcolor{linecolor1}Qwen-VL-Max~\cite{qwenvl23} & 49.11 & - \\
\rowcolor{linecolor2}\textbf{\ours} & \textbf{70.88} & - \\
\bottomrule
\end{tabular}
}
\label{tab:embspatial}
\end{table}
\begin{table}[h!]
\centering
\setlength{\tabcolsep}{3pt}
\caption{Zero-shot performance of LVLMs in SpatialBot-Bench~\cite{SpatialBot24}.
SpatialBot-3B: SpatialBot-Phi2-3B-RGB, SpatialBot-8B: SpatialBot-Llama3-8B-RGB.
}
\resizebox{1.00\linewidth}{!}{
\begin{tabular}{lcccccc}
\toprule
Model & Pos & Exist & Count & Reach & Size & Avg \\
\midrule
ChatGPT-4o~\cite{GPT4o24} & 70.6 & 85.0 & 84.5 & 51.7 & \textbf{43.3} & 67.0\\
SpatialBot-3B~\cite{SpatialBot24} & 64.7 & 80.0 & 88.0 & \textbf{61.7} & 28.3 & 64.5\\
SpatialBot-8B~\cite{SpatialBot24} & 55.9 & 80.0 & \textbf{91.2} & 40.0 & 20.0 & 57.4\\
\midrule
\rowcolor{linecolor2}\textbf{\ours} & \textbf{76.5} & \textbf{87.5} & 80.0 & 57.5 & 40.0 & \textbf{68.3}\\
\bottomrule
\end{tabular}
}
\label{tab:spatialbot}
\end{table}


\subsection{Close-Loop Execution Experiment}\label{app:close_loop}
We demonstrate the closed-loop replan capabilities of \sofar~within Simpler-Env~\cite{simplerenv24} in \cref{fig:close_loop}. The instruction for both tasks is ``pick the coke can'' In \cref{fig:close_loop} (a), the model initially misidentified the coke can as a Fanta can. After correction by the VLM, the model re-identified and located the correct object. In \cref{fig:close_loop} (b), the model accidentally knocks over the Coke can during motion due to erroneous motion planning. Subsequently, the model re-plans and successfully achieves the grasp.

\subsection{Long Horizon Object Manipulation Experiment}\label{app:long_horizon}
\cref{fig:long_horizon} illustrates the execution performance of our model on long-horizon tasks. Through the VLM~\cite{GPT4o24,gemini23}, complex instructions such as ``making breakfast'' and ``cleaning up the desktop'' can be decomposed into sub-tasks. In the second example, we deliberately chose complex and uncommon objects as assets, such as ``Aladdin's lamp'' and ``puppets'', but \sofar~is able to successfully complete all tasks.
\begin{figure}[h!]
\centering
\includegraphics[width=1.0\linewidth]{figs/src/close_loop_execution.pdf}
\vspace{-15pt}
\captionof{figure}{\textbf{Close-loop execution of our \sofar.}}
\label{fig:close_loop}
\end{figure}


\subsection{In the Wild Evaluation of Semantic Orientation}
We provide a qualitative demonstration of the accuracy of PointSO under in-the-wild conditions, as shown in \cref{fig:in_the_wild}, where the predicted Semantic Orientation is marked in the images. We obtained single-sided point clouds by segmenting objects using Florence-2~\cite{florence2} and SAM~\cite{SAM23} and fed them into PointSO. It can be observed that our model achieves good performance across different views, objects, and instructions, which proves the effectiveness and generalization of PointSO.
\begin{figure*}[h!]
\centering
\includegraphics[width=1.0\linewidth]{figs/src/long_horizon.pdf}
\captionof{figure}{\textbf{Long-horizon object manipulation experiment of our \sofar.}}
\label{fig:long_horizon}
\end{figure*}

\begin{figure}[h!]
\centering
\includegraphics[width=1.0\linewidth]{figs/src/in_the_wild.pdf}
\vspace{-15pt}
\captionof{figure}{\textbf{In the wild evaluation of PointSO.}}
\label{fig:in_the_wild}
\end{figure}

\begin{figure}[t!]
\centering
\includegraphics[width=1.0\linewidth]{figs/src/cross_view.pdf}
\captionof{figure}{\textbf{Cross view generalization of our \sofar.}}
\label{fig:cross_view}
\end{figure}


\subsection{Cross-View Generalization}
\sofar~gets point clouds in the world coordinate system using an RGB-D camera to obtain grasping poses, and it is not limited to a fixed camera perspective. In addition, PointSO generates partial point clouds from different perspectives through random camera views to serve as data augmentation for training data, which also generalizes to camera perspectives in the real world. \cref{fig:cross_view} illustrates \sofar's generalization capability for 6-DoF object manipulation across different camera poses. It can be observed that whether it's a front view, side view, or ego view, \sofar~can successfully execute the ``upright the bottle'' instruction.

\subsection{Failure Case Distribution Analysis}
Based on the failure cases from real-world experiments, we conducted a quantitative analysis of the failure case distribution for \sofar, with the results shown in \cref{fig:failure_case}. It can be observed that 31\% of the failures originated from grasping issues, including objects being too small, inability to generate reasonable grasping poses, and instability after grasping leading to sliding or dropping. Next, 23\% were due to incorrect or inaccurate Semantic Orientation prediction. For tasks such as upright or upside - down, highly precise angle estimation (<5°) is required for smooth execution. Object analysis and detection accounted for approximately 20\% of the errors. The instability of open-vocabulary detection modules like Florence2~\cite{florence2} and Grounding DINO~\cite{groundingdino23} often led to incorrect detection of out-of-distribution objects or object parts. In addition, since our Motion Planning did not take into account the working space range of the robotic arm and potential collisions of the manipulated object, occasional deadlocks and collisions occurred during motion. Finally, there were issues with the Task Planning of the VLM~\cite{GPT4o24,gemini23}. For some complex Orientations, the VLM occasionally failed to infer the required angles and directions to complete the task. Employing a more powerful, thought-enabled VLM~\cite{gpt_o1} might alleviate such errors.

\begin{figure*}[t]
\centering
\includegraphics[width=\linewidth]{CameraReady/Figures/failure_case_image.pdf}
\caption{Failure cases of our approach.}
\label{fig:failure_case}
\end{figure*}

\subsection{Ablation Study}\label{app:ablation}
\subsubsection{Scaling Law}
The scaling capability of models and data is one of the most critical attributes today and a core feature of foundation models~\cite{FoundationModel21}. We investigate the performance of PointSO across different data scales, as illustrated in \cref{tab:scaling_law}. 
We obtain the subset for OrienText300K from Objaverse-LVIS, which consists of approximately 46,000 3D objects with category annotations. The selection was based on the seven criteria mentioned in the main text. Objects meeting all seven criteria formed the strict subset, comprising around 15k objects. When including objects without textures and those of lower quality, the total increases to approximately 26k objects.
It can be seen that the increase in data volume is the most significant factor driving the performance improvement of PointSO. It can be anticipated that with further data expansion, such as Objaverse-XL~\cite{ObjaverseXL23}, PointSO will achieve better performance.
\begin{table}[t!]
\setlength{\tabcolsep}{7pt}
\caption{\textbf{Data scaling property} of semantic orientation with different training data scales evaluated on OrienText300K test split. All experiments are conducted with PointSO-Base.
}
\label{tab:scaling_law}
\centering
\resizebox{1.0\linewidth}{!}{
\begin{tabular}{lccccc}
\toprule[0.95pt]
    Data Scale & \texttt{45°} & \texttt{30°} & \texttt{15°} & \texttt{5°} & Average \\ 
    \midrule[0.6pt]
    5\% & 57.03 & 46.09 & 39.84 & 27.34 & 42.58 \\
    10\% & 61.72 & 53.13 & 43.75 & 30.47 & 47.27 \\
    50\% & 76.56 & 72.66 & 66.41 & 56.25 & 67.97 \\
    \rowcolor{linecolor2}100\% & \textbf{79.69} & \textbf{77.34} & \textbf{70.31} & \textbf{62.50} & \textbf{72.46} \\
    \bottomrule[0.95pt]
\end{tabular}
}
\end{table}

\subsubsection{Cross-Modal Fusion Choices}\label{app:fusion}
We further conduct an ablation study on the multi-modal fusion methods in PointSO, testing commonly used feature fusion techniques such as cross-attention, multiplication, addition, and concatenation, as shown in \cref{tab:fusion}. The results indicate that simple addition achieves the best performance. This may be attributed to the fact that instructions in the semantic domain are typically composed of short phrases or sentences, and the text CLS token already encodes sufficiently high-level semantic information.
% \usepackage{multirow}
% \usepackage{booktabs}


\begin{table}[h]
	\centering
	\caption{Model performance under two coefficient fusion methods.}
	\begin{tabular}{c|ccc} 
		\toprule
		\multirow{2}{*}{Fusion mode} & \multicolumn{3}{c}{R40}                           \\
		& Easy           & Moderate       & Hard            \\ 
		\hline
		straight                     & 90.92          & 82.84          & 80.29           \\
		our                          & \textbf{91.96} & \textbf{83.31} & \textbf{80.59}  \\
		\bottomrule
	\end{tabular}
\label{tabel6}
\end{table}

\begin{table*}[h!]
\setlength{\tabcolsep}{3pt}
\caption{\textbf{Ablation study of open vocabulary detection modules} on Open6DOR~\cite{Open6DOR24} perception tasks.
}
\label{tab:detection_ab}
\centering
\resizebox{0.98\linewidth}{!}{
\begin{tabular}{lccccccccccc}
\toprule[0.95pt]
\multirow{2}{*}[-0.5ex]{Method} & \multicolumn{3}{c}{\textbf{Position Track}} & \multicolumn{4}{c}{\textbf{Rotation Track}} & \multicolumn{3}{c}{\textbf{6-DoF Track}} & \multirow{2}{*}[-0.5ex]{Time Cost (s)}\\
\cmidrule(lr){2-4} \cmidrule(lr){5-8} \cmidrule(lr){9-11}
& Level 0 & Level 1 & Overall & Level 0 & Level 1 & Level 2 & Overall & Position & Rotation & Overall \\ 
\midrule[0.6pt]
YOLO-World~\cite{yoloworld24} & 59.0 & 37.7 & 53.3 & 48.3 & 36.1 & 62.0 & 44.9 & 53.4 & 44.6 & 27.8 & \textbf{7.4s}\\
\rowcolor{linecolor1}Grounding DINO~\cite{groundingdino23} & 92.2 & 71.5 & 86.7 & 64.7 & 41.1 & 69.8 & 55.5 & 87.2 & 51.6 & 44.6 & 9.2s\\
\rowcolor{linecolor2}Florence-2~\cite{florence2} & \textbf{96.0} & \textbf{81.5} & \textbf{93.0} & \textbf{68.6} & \textbf{42.2} & \textbf{70.1} & \textbf{57.0} & \textbf{92.7} & \textbf{52.7} & \textbf{48.7} & \textbf{8.5s}\\
\bottomrule[0.95pt]
\end{tabular}
}
\end{table*}

\subsubsection{Open Vocabulary Object Detection Module}
\sofar~utilize a detection foundation model to localize the interacted objects or parts, then generate masks with SAM~\cite{SAM23}. Although not the SOTA performance on the COCO benchmark, Florence-2~\cite{florence2} exhibits remarkable generalization in in-the-wild detection tasks, even in simulator scenarios. \cref{tab:detection_ab} illustrates the performance of various detection modules in Open6DOR~\cite{Open6DOR24} Perception, where Florence-2 achieves the best results and outperforms Grounding DINO~\cite{groundingdino23} and YOLO-World~\cite{yoloworld24}.

\vspace{3pt}
\section{Additional Implementation Details}\label{app:implementation_details}

\subsection{Detail Real World Experiment Results}\label{app:detail_realworld}
To fully demonstrate the generalization of \sofar~rather than cherry-picking, we carefully design 60 different real-world experimental tasks, covering more than 100 different and diverse objects. Similar to the Open6DOR~\cite{Open6DOR24} benchmark in the simulator, we divide these 60 tasks into three parts: position-track, orientation-track, and the most challenging comprehensive \& 6-DoF-track. Each track is further divided into simple and hard levels. The position-simple track includes tasks related to front \& back \& left \& right spatial relationships, while the position-hard track includes tasks related to between, center, and customized. The orientation-simple track includes tasks related to the orientation of object parts, and the orientation-hard track includes tasks related to whether the object is upright or flipped (with very strict requirements for angles in both upright and flipped cases). Comprehensive tasks involve complex instruction understanding and long-horizon tasks; 6-DoF tasks simultaneously include requirements for both object position and orientation instructions. In \cref{tab:detailed_realworld}, we present the complete task instructions, as well as the performance metrics of \sofar~and the baseline. Due to the large number of tasks, we performed each task three times. It can be seen that \sofar~achieves the best performance in all tracks, especially in the orientation-track and comprehensive \& 6-DoF-track. We also show all the objects used in the real-world experiments in \cref{fig:real_obj}, covering a wide range of commonly and uncommonly used objects in daily life.

\vspace{-10pt}
\begin{table*}[t!]
\setlength{\tabcolsep}{6pt}
\caption{\textbf{Detailed zero-shot real-world 6-DoF rearrangement results}.}
\label{tab:detailed_realworld}
\centering
\resizebox{0.96\linewidth}{!}{
\begin{tabular}{lcccc}
\toprule[0.95pt]
    Task & CoPa~\cite{CoPa24} & ReKep-Auto~\cite{ReKep24} & \sofar-LLaVA~(Ours) & \sofar~(Ours) \\ 
    \midrule[0.6pt]
    \multicolumn{5}{c}{\textit{Positional Object Manipulation}}\\
    \midrule
    Move the soccer ball to the right of the bread. & 2/3 & 3/3 & 3/3 & \textbf{3/3} \\
    Place the doll to the right of the lemon. & 3/3 & 3/3 & 3/3 & \textbf{3/3} \\
    Put the pliers on the right side of the soccer ball. & 1/3 & 1/3 & 3/3 & \textbf{2/3} \\
    Move the pen to the right of the doll. & 3/3 & 2/3 & 3/3 & \textbf{3/3} \\
    Place the carrot on the left of the croissant. & 2/3 & 3/3 & 2/3 & \textbf{2/3} \\
    Move the avocado to the left of the baseball. & 3/3 & 2/3 & 2/3 & \textbf{3/3} \\
    Pick the pepper and place it to the left of the charger. & 1/3 & 2/3 & 2/3 & \textbf{2/3} \\
    Place the baseball on the left side of the mug. & 3/3 & 2/3 & 2/3 & \textbf{3/3} \\
    Arrange the flower in front of the potato. & 2/3 & 3/3 & 2/3 & \textbf{3/3} \\
    Put the volleyball in front of the knife. & 3/3 & 3/3 & 3/3 & \textbf{3/3} \\
    Place the ice cream cone in front of the potato. & 2/3 & 3/3 & 2/3 & \textbf{3/3} \\
    Move the bitter melon to the front of the forklift. & 2/3 & 1/3 & 2/3 & \textbf{2/3} \\
    Place the orange at the back of the stapler. & 3/3 & 2/3 & 3/3 & \textbf{3/3} \\
    Move the panda toy to the back of the shampoo bottle. & 2/3 & 3/3 & 3/3 & \textbf{2/3} \\
    pick the pumpkin and place it behind the pomegranate. & \textbf{3/3} & 2/3 & 1/3 & 2/3 \\
    Place the basketball at the back of the board wipe. & 2/3 & 2/3 & 3/3 & \textbf{2/3} \\
    Put the apple inside the box. & 3/3 & 2/3 & 3/3 & \textbf{3/3} \\
    Place the waffles on the center of the plate. & 3/3 & 2/3 & 3/3 & \textbf{3/3} \\
    Move the hamburger into the bowl.& 2/3 & 2/3 & 2/3 & \textbf{3/3} \\
    Pick the puppet and put it into the basket. & 1/3 & 2/3 & 2/3 & \textbf{2/3} \\
    Drop the grape into the box. & 2/3 & 3/3 & 3/3 & \textbf{2/3} \\
    Put the doll between the lemon and the USB. & 2/3 & 2/3 & 2/3 & \textbf{3/3} \\
    Set the duck toy in the center of the cart, bowl, and camera. & 2/3 & 1/3 & 2/3 & \textbf{2/3} \\
    Place the strawberry between the Coke bottle and the glue. & 2/3 & 2/3 & 3/3 & \textbf{3/3} \\
    Put the pen behind the basketball and in front of the vase. & 2/3 & 1/3 & 2/3 & \textbf{2/3} \\
    Total success rate& 74.7\% & 72.0\% & 81.3\% & \textbf{85.3\%} \\
    \midrule
    \multicolumn{5}{c}{\textit{Orientational Object Manipulation}}\\
    \midrule
    Turn the yellow head of the toy car to the right. & 2/3 & 2/3 & 1/3 & \textbf{2/3} \\
    Adjust the knife handle so it points to the right. & 2/3 & 1/3 & 2/3 & \textbf{2/3} \\
    Rotate the cap of the bottle towards the right. & 2/3 & 2/3 & 2/3 & \textbf{2/3}\\
    Rotate the tip of the screwdriver to face the right. & 0/3 & 0/3 & 1/3 & \textbf{1/3}\\
    Rotate the stem of the apple to the right. & 0/3 & 1/3 & 1/3 & \textbf{2/3}\\
    Turn the front of the toy car to the left. & 0/3 & 0/3 & 2/3 & \textbf{2/3} \\
    Rotate the cap of the bottle towards the left. & 2/3 & 1/3 & 1/3 & \textbf{2/3}\\
    Adjust the pear's stem to the right. & 1/3 & 1/3 & 1/3 & \textbf{1/3}\\
    Turn the mug handle to the right. & 1/3 & 1/3 & 2/3 & \textbf{2/3}\\
    Rotate the handle of the mug to towards right.& 2/3 & 1/3 &\textbf{2/3} & 1/3\\
    Rotate the box so the text side faces forward. & 0/3 & 1/3 & 0/3 & \textbf{1/3}\\
    Adjust the USB port to point forward. & 0/3 & 0/3 & 1/3 & \textbf{1/3}\\
    Set the bottle upright. & 0/3 & 1/3 & 0/3 & \textbf{1/3}\\
    Place the coffee cup in an upright position. & 1/3 & 1/3 & 2/3 & \textbf{2/3}\\
    Upright the statue of liberty& 0/3 & 0/3 & \textbf{1/3} & 0/3\\
    Stand the doll upright. & 0/3 & 1/3 & 0/3 & \textbf{1/3}\\
    Right the Coke can. & 0/3 & 0/3 & 1/3 & \textbf{1/3}\\
    Flip the bottle upside down. & 0/3 & 0/3 & 0/3 & \textbf{1/3}\\
    Turn the coffee cup upside down. & 0/3 & 0/3 & 1/3 & \textbf{1/3}\\
    Invert the shampoo bottle upside down. & 0/3 & 0/3 & 0/3 & \textbf{0/3}\\
    Total success rate& 21.7\% & 23.3\% & 35.0\% & \textbf{43.3\%} \\
    \midrule
    \multicolumn{5}{c}{\textit{Comprehensive 6-DoF Object Manipulation}}\\
    \midrule
    Pull out a tissue.& 3/3 & 3/3 & 2/3 & \textbf{3/3}\\
    Place the right bottle into the
    box and arrange it in a 3×3 pattern. & 0/3 & 0/3 & 0/3 & \textbf{1/3}\\
    Take the tallest box and position it on the right side. & 1/3 & 1/3 & 3/3 & \textbf{3/3}\\
    Grasp the error bottle and put it on the right side. & 1/3 & 2/3 & 1/3 & \textbf{2/3} \\
    Take out the green test tube and place it between the two bottles. & 2/3 & 2/3 & 3/3 & \textbf{3/3}\\
    Pack the objects on the table into the box one by one. & 1/3 & 1/3 & 0/3 & \textbf{1/3}\\
    Rotate the loopy doll to face the yellow dragon doll & 0/3 & 1/3 & 1/3 & \textbf{1/3}\\
    Right the fallen wine glass and arrange it neatly in a row. & 0/3 & 0/3 & 0/3 & \textbf{0/3}\\
    Grasp the handle of the knife and cut the bread.& 0/3 & 0/3 & 0/3 & \textbf{1/3}\\
    Pick the baseball into the cart and turn the cart to facing right. & 0/3 & 0/3 & 1/3 & \textbf{2/3}\\
    Place the mug on the left of the ball and the handle turn right. & 0/3 & 0/3 & 1/3 & \textbf{1/3}\\
    Aim the camera at the toy truck. & 1/3 & 0/3 & 1/3 & \textbf{1/3}\\
    Rotate the flashlight to illuminate the loopy. & 0/3 & 0/3 & 1/3 & \textbf{1/3}\\
    Put the pen into the pen container. & 0/3 & 1/3 & 0/3 & \textbf{1/3} \\
    Pour out chips from the chips cylinder to the plate. & 0/3 & 1/3 & 1/3 & \textbf{1/3} \\
    Total success rate& 20.0\% & 26.7\% & 33.3\% & \textbf{48.9\%} \\
    \bottomrule[0.95pt]
\end{tabular}
}
\end{table*}
\vspace{-10pt}
\begin{figure}[t!]
\centering
\includegraphics[width=1.0\linewidth]{figs/src/real_obj.pdf}
\captionof{figure}{\textbf{The real-world assets used in our real-world experiments.} More than 100 diverse objects are used in our 6-DoF rearrangement experiments.}
\label{fig:real_obj}
\vspace{-10pt}
\end{figure}


\input{tabs/PointSO_configurations}
\usepackage{graphicx}
\usepackage{url}
\usepackage{color, soul}
\usepackage{bm}
\usepackage{amsmath} % for the equation* environment
\usepackage{lineno}

\subsection{PointSO Model Details}\label{app:pointso_details}
For PointSO, we utilize FPS + KNN to perform patchify and employ a small PointNet~\cite{PointNet} as the patch encoder. Subsequently, a standard Transformer encoder is adopted as the backbone, followed by a single linear layer to map the output to a three-dimensional vector space. All parameter configurations follow prior work on point cloud representation learning~\cite{ACT23,ReCon23,ShapeLLM24}. Detailed hyperparameter and model configurations are provided in \cref{tab:hyper_params,tab:PointSO_configs}.

\subsection{SoFar-LLaVA Model Details}\label{app:model_details}
\begin{figure*}[t!]
\begin{center}
\includegraphics[width=0.89\linewidth]{figs/src/sofar_llava.pdf}
\caption{\textbf{Pipeline of \sofar-LLaVA}, a fine-tuned vision language model based on visual instruction tuning.
}
\label{fig:sofar_llava}
\vspace{-5pt}
\end{center}
\end{figure*}

In addition to leveraging the extensive knowledge and strong generalization capabilities of closed-source/open-source pretrained VLMs~\cite{ChatGPT22,gemini23,qwenvl23} for zero-shot or in-context learning, \sofar~can also enhance the planning performance of open-source models through visual instruction tuning for rapid fine-tuning. The pipeline of the model is illustrated in \cref{fig:sofar_llava}. A JSON-formatted 6-DoF scene graph, processed through a text tokenizer, along with the image refined by SoM~\cite{SoM23}, is fed into an LLM (\eg, LLaMA~\cite{LLaMA23,LLaMA2_23}) for supervised fine-tuning~\cite{LLaVA23}.
In the Open6DOR~\cite{Open6DOR24} task, we supplement the training dataset with additional samples retrieved and manually annotated from Objaverse~\cite{objaverse23}, ensuring alignment with the object categories in the original benchmark. This dataset includes approximately 3,000 6-DoF object manipulation instructions. Using this data, we construct dialogue-style training data based on ChatGPT and train the \sofar-LLaVA model. The training hyperparameters are detailed in \cref{tab:hyper_params}. Similarly, we finetune PointSO on this training dataset and achieve superior performance on the Open6DOR task.

\subsection{ChatGPT API Costs}
The knowledge of OrienText300K is derived from the annotations of 3D modelers on Sketchfab, combined with ChatGPT's filtering and comprehension capabilities. To generate semantic direction annotations, we filter the 800K dataset of Objaverse~\cite{objaverse23} and apply ChatGPT to approximately 350K of the filtered data to generate semantic text-view index pairs. The OpenAI official API was used for these calls, with the GPT-4o version set to 2024-08-06 and the output format configured as JSON. The total cost for debugging and execution amounted to approximately \$10K.


\section{Additional Benchmark Statistic Analysis}
\subsection{6-DoF SpatialBench Analysis}
We conduct a statistical analysis of the manually constructed 6-DoF SpatialBench, with category comparisons and word cloud visualizations shown in \cref{fig:spatialvqa_statistic}. We collect diverse image data from the internet, encompassing scenes such as indoor, outdoor, and natural landscapes. The questions may involve one or multiple objects, with varying levels of uncertainty in image resolution. Most importantly, we are the first to propose a VQA benchmark for orientation understanding, focusing on both quantitative and qualitative evaluation of orientation.


\subsection{Open6DOR V2 Analysis}
Open6DOR V2 builds upon Open6DOR V1 by removing some incorrectly labeled data and integrating assets and metrics into Libero, enabling closed-loop policy evaluation. The detailed number of tasks is presented in \cref{tab:open6dorv2_statistic}, comprising over 4,500 tasks in total. Notably, we remove level 2 of the position track in Open6DOR V1~\cite{Open6DOR24} because it requires manual inspection, which is not conducive to open-source use and replication by the community. Besides, due to the randomness of object drops in the scene, approximately 8\% of the samples already satisfy the evaluation metrics in their initial state.

\vspace{3pt}
\section{Additional Related Works}\label{app:related_work}
\subsection{3D Representation Learning}
Research on 3D Representation Learning encompasses various methods, including point-based~\cite{PointNet,PointNet++}, voxel-based~\cite{voxelnet15}, and multiview-based approaches~\cite{MVCNN3D15,MVTN}. 
Point-based methods~\cite{PointNext,PointTrans21} have gained prominence in object classification~\cite{ModelNet15,ScanObjectNN19} due to their sparsity yet geometry-informative representation. On the other hand, voxel-based methods~\cite{voxelrcnn21,SyncSpecCNN17,VPP23} offer dense representation and translation invariance, leading to a remarkable performance in object detection~\cite{ScanNet17} and segmentation~\cite{ShapeNetPart16, S3DIS16}.
The evolution of attention mechanisms~\cite{AttentionIsAllYouNeed,ReKo23} has also contributed to the development of effective representations for downstream tasks, as exemplified by the emergence of 3D Transformers~\cite{PointTrans21,groupfree21, voxeltransformer21}. Notably, 3D self-supervised representation learning has garnered significant attention in recent studies. PointContrast~\cite{PointContrast20} utilizes contrastive learning across different views to acquire discriminative 3D scene representations. Innovations such as Point-BERT~\cite{PointBERT} and Point-MAE~\cite{PointMAE} introduce masked modeling~\cite{MAE,BERT} pretraining into the 3D domain. 
ACT~\cite{ACT23} pioneers cross-modal geometry understanding through 2D or language foundation models such as CLIP~\cite{CLIP} or BERT~\cite{BERT}. 
Following ACT, {\scshape ReCon}~\cite{ReCon23} further proposes a learning paradigm that unifies generative and contrastive learning. PPT~\cite{ppt24} highlights the significance of positional encoding in 3D representation learning
Additionally, leveraging foundation vision-language models like CLIP~\cite{ACT23,CLIP} has spurred the exploration of a new direction in open-world 3D representation learning. This line of work seeks to extend the applicability and adaptability of 3D representations in diverse and open-world/vocabulary scenarios~\cite{OpenScene23,CLIPFO3D23,PLA23,Lowis3D23,OVIR3D23,PointGCC23}.

\section{Additional Discussions}
\subsection{Relation to Affordance \& 6-DoF Pose Estimation}
Conceptually, this semantic orientation is a counterpart of \textit{affordance}~\citep{Affordance77,AffordanceHRI16,MoveWithAffordanceMaps20,HandsAsAffordancesProbes22} but beyond,
as SO and affordance all present potential actions and interactions with objects.
However, SO also contains the spatial understanding of intra-object part-level attributes more than affordance learning.
Compared to vanilla 6-DoF pose estimation, our proposed SO combined with the 3-DoF translation understanding has the same DoF completeness.
The difference is, our proposed SO is grounded by languages, making it useful for open-world manipulation requiring complicated spatial reasoning~\cite{RobotsThatUseLanguage20,SayCan22,Open6DOR24}. 
In addition, our Semantic Orientation can be auto-labeled from Internet 3D data that achieves higher scalability, introduced in the next section.
\begin{figure}[t!]
\begin{center}
% \includegraphics[width=0.85\linewidth]{figs/src/spatialvqa_statistic.pdf}
\includegraphics[width=\linewidth]{figs/src/spatialvqa_statistic.pdf}
\vspace{-15pt}
\caption{\textbf{6-DoF SpatialBench statistics}. (a) Statistical analysis of the task type, question type, and object relation. (b) Word cloud visualization.}
\label{fig:spatialvqa_statistic}
\end{center}
\end{figure}



\subsection{Comparison to Concurrent Works}
\begin{figure*}[t!]
\includegraphics[width=\linewidth]{figs/src/simpler_visual.pdf}
\vspace{-15pt}
\caption{An example of \ours~how to finish ``move near'' task in SIMPLER~\cite{simplerenv24}.}
\label{fig:simpler_visual}
\end{figure*}

\subsubsection{Comparison with ReKep~\cite{ReKep24}}
Recently, ReKep has succeeded in executing complex robotic tasks, such as long-horizon manipulation, based on the relationships and constraints between spatial key points. 
Its structural design offers many insights that \sofar~can draw upon, yet it also presents several issues: 
(1) Overly customized prompt engineering. ReKep requires manually designed complex system prompts for each task during inference. 
While this approach may be described as ``no training'', it cannot be considered a true zero-shot transfer. In contrast, \sofar~achieves genuine zero-shot transfer by eliminating the need for any human involvement during inference; (2) Using constraints based solely on key points fails to capture the full 6-DoF pose integrity of objects. For example, in the ``pouring water'' task, merely bringing the spout of the kettle close to the cup may lead to incorrect solutions, such as the kettle overturning; (3) ReKep requires all key points to be present in the first frame, and each step of the process—from mask extraction to feature dimensionality reduction, clustering, and filtering—introduces additional hyperparameters.

\subsubsection{Comparison with Orient Anything~\cite{orient_anything24}}
Recently, Orient Anything also highlighted the importance of orientation in spatial perception and adopted a training data construction approach similar to Our PointSO. Our primary distinction lies in semantic orientation, which is language-conditioned orientation. In contrast, Orient Anything is limited to learning basic directions such as ``front'' and ``top''. By aligning with textual information, semantic orientation better enhances spatial perception, understanding, and robotic manipulation.

\subsection{Future Works}
Future work includes further expanding the OrienText300K with larger datasets like Objaverse-XL~\cite{ObjaverseXL23}, enhancing the performance of semantic orientation through self-supervised learning and pretraining methods~\cite{MAE,CLIP,ACT23,ReCon23}, and demonstrating its effectiveness in a broader range of robotic scenarios, such as navigation~\cite{GOAT24}, mobile manipulation~\cite{homerobot23}, lifelong learning~\cite{LIBERO23}, spatio-temporal reasoning~\cite{ReKep24,LeaFLF23,CrossVideoSC24,thinking24}, humanoid~\cite{OmniH2O24,SmoothHumanoidLCP24,Exbody24,humanup25}, and human-robot interaction~\cite{HOI4D22,InteractiveHO23}.

\begin{table*}[t!]
\centering
\caption{\textbf{Statistics of Open6DOR V2 Benchmark.} The entire benchmark comprises three independent tracks, each featuring diverse tasks with careful annotations. The tasks are divided into different levels based on instruction categories, with statistics demonstrated above.}
\label{tab:open6dorv2_statistic}
\resizebox{\textwidth}{!}{
\setlength{\tabcolsep}{3.5pt}
    \begin{tabular}{c|ccccc|cc|c|c|c|c|c}
    \toprule
        Track & \multicolumn{7}{c|}{Position-track} & \multicolumn{3}{c|}{Rotation-track} & 6-DoF-track & Totel\\
        \midrule
        Level  & \multicolumn{5}{c|}{Level 0} & \multicolumn{2}{c|}{Level 1}  & Level 0 & Level 1 & Level 2 & - & -\\
        \midrule
        Task Catog. &  Left & Right  & Top &  Behind &  Front & Between  & Center & Geometric & Directional & Semantic & - & - \\
        \midrule
        Task Stat. & 296 & 266 & 209 & 297 & 278 & 193 & 159 & 318 & 367 & 134 & 1810 & 4535\\
        
        \midrule
        Benchmark Stat. &\multicolumn{7}{c|}{1698} & \multicolumn{3}{c|}{1027} & 1810 & 4535\\
    \bottomrule 
    \end{tabular}
}
\end{table*}

\section{Additional Visualizations}\label{app:visualization}

\subsection{Robotic Manipulation}
As shown in \cref{fig:simpler_visual}, we present a visualization of executing a task named ``move near''.
According to the input image and task instruction - ``\textit{move blue plastic bottle near pepsi can}'', \ours~can predict the center coordinate of the target object (bottle) and relative target (pepsi can), and it would infer the place coordinate and produce a series of grasp pose.

\subsection{6-DoF SpatialBench}
To further evaluate 6-DoF spatial understanding, we construct a 6-DoF SpatialBench.
We present examples of question-answer pairs from the 6-DoF SpatialBench, with quantitative and qualitative questions shown in \cref{fig:spatialbench_show1,fig:spatialbench_show2}, respectively. 
The benchmark we constructed is both challenging and practical, potentially involving calculations based on the laws of motion, such as ``\textit{Assuming a moving speed of 0.5 m/s, how many seconds would it take to walk from here to the white flower?}'' Moreover, it covers a wide range of spatially relevant scenarios across both indoor and outdoor environments.


\subsection{System Prompts}
Prompt engineering significantly enhances ChatGPT's capabilities. The model's understanding and reasoning abilities can be greatly improved by leveraging techniques such as Chain-of-Thought~\cite{CoT22} and In-Context Learning~\cite{GPT3_20}. \cref{fig:filter_prompt,fig:instruction_prompt} illustrate the system prompt we used in constructing OrienText300K.
\cref{fig:open6dor_prompt}, \cref{fig:manip_prompt}, and \cref{fig:vqa_prompt} illustrate the system prompt we used when evaluating \sofar on Open6DOR (simulation), object manipulation (both simulation and real worlds), and VQA, respectively.
Note that different from previous methods~\cite{VoxPoser23,ReKep24}, \sofar does not require complicated in-context examples.


\begin{figure*}[h!]
\centering
\includegraphics[width=0.97\linewidth]{figs/src/show1.pdf}
\captionof{figure}{\textbf{Visualization example of 6-DoF SpatialBench data samples.}
% 6-DoF SpatialBench includes complex spatial reasoning of absolute numbers.
}
\label{fig:spatialbench_show1}
\end{figure*}

\begin{figure*}[h!]
\centering
\includegraphics[width=0.97\linewidth]{figs/src/show2.pdf}
\captionof{figure}{\textbf{Visualization example of 6-DoF SpatialBench data samples.}}
\label{fig:spatialbench_show2}
\end{figure*}

\begin{figure*}[h!]
\centering
\includegraphics[width=0.97\linewidth]{figs/src/filter_prompt.pdf}
\captionof{figure}{\textbf{The system prompt of ChatGPT-4o used for filtering Objaverse data.}}
\label{fig:filter_prompt}
\end{figure*}

\begin{figure*}[h!]
\centering
\includegraphics[width=0.97\linewidth]{figs/src/instruction_prompt.pdf}
\captionof{figure}{\textbf{The system prompt of ChatGPT-4o used for generating Semantic Direction-Index pairs.}}
\label{fig:instruction_prompt}
\end{figure*}

\begin{figure*}[h!]
\centering
\includegraphics[width=0.97\linewidth]{figs/src/open6dor_prompt.pdf}
\captionof{figure}{\textbf{The system prompt of Open6DOR tasks.}}
\label{fig:open6dor_prompt}
\end{figure*}

\begin{figure*}[h!]
\centering
\includegraphics[width=0.97\linewidth]{figs/src/manip_prompt.pdf}
\captionof{figure}{\textbf{The system prompt of general manipulation tasks.}}
\label{fig:manip_prompt}
\end{figure*}

\begin{figure*}[h!]
\centering
\includegraphics[width=0.97\linewidth]{figs/src/vqa_prompt.pdf}
\captionof{figure}{\textbf{The system prompt of visual-question-answering tasks.}}
\label{fig:vqa_prompt}
\end{figure*}



\end{document}
