\documentclass{article} % For LaTeX2e
% \usepackage{iclr2025_conference,times}

\usepackage[preprint]{icml2025}
% \usepackage[accepted]{icml2025}

% Optional math commands from https://github.com/goodfeli/dlbook_notation.
%%%%% NEW MATH DEFINITIONS %%%%%

% \usepackage{amsmath,amsfonts,bm}
\usepackage{amsmath,amsfonts}

\usepackage{pifont}


\newcommand{\R}{\mathbb{R}}


\def\va{{\mathbf{a}}}
\def\vg{{\mathbf{g}}}

% Sets
\def\sR{\mathbb{R}}
\def\sC{\mathbb{C}}
\def\sZ{\mathbb{Z}}
\def\sN{\mathbb{N}}
\def\sQ{\mathbb{Q}}

\def\sS{\mathcal{S}}



% Vectors
\def\vzero{{\mathbf{0}}}
\def\vone{{\mathbf{1}}}
\def\vmu{{\mathbf{\mu}}}
\def\vtheta{{\mathbf{\theta}}}
\def\va{{\mathbf{a}}}
\def\vb{{\mathbf{b}}}
\def\vc{{\mathbf{c}}}
\def\vd{{\mathbf{d}}}
\def\ve{{\mathbf{e}}}
\def\vf{{\mathbf{f}}}
\def\vg{{\mathbf{g}}}
\def\vh{{\mathbf{h}}}
\def\vi{{\mathbf{i}}}
\def\vj{{\mathbf{j}}}
\def\vk{{\mathbf{k}}}
\def\vl{{\mathbf{l}}}
\def\vm{{\mathbf{m}}}
\def\vn{{\mathbf{n}}}
\def\vo{{\mathbf{o}}}
\def\vp{{\mathbf{p}}}
\def\vq{{\mathbf{q}}}
\def\vr{{\mathbf{r}}}
\def\vs{{\mathbf{s}}}
\def\vt{{\mathbf{t}}}
\def\vu{{\mathbf{u}}}
\def\vv{{\mathbf{v}}}
\def\vw{{\mathbf{w}}}
\def\vx{{\mathbf{x}}}
\def\vy{{\mathbf{y}}}
\def\vz{{\mathbf{z}}}
\def\vzeta{{\mathbf{\zeta}}}

% Matrix
\def\mA{{\mathbf{A}}}
\def\mB{{\mathbf{B}}}
\def\mC{{\mathbf{C}}}
\def\mD{{\mathbf{D}}}
\def\mE{{\mathbf{E}}}
\def\mF{{\mathbf{F}}}
\def\mG{{\mathbf{G}}}
\def\mH{{\mathbf{H}}}
\def\mI{{\mathbf{I}}}
\def\mJ{{\mathbf{J}}}
\def\mK{{\mathbf{K}}}
\def\mL{{\mathbf{L}}}
\def\mM{{\mathbf{M}}}
\def\mN{{\mathbf{N}}}
\def\mO{{\mathbf{O}}}
\def\mP{{\mathbf{P}}}
\def\mQ{{\mathbf{Q}}}
\def\mR{{\mathbf{R}}}
\def\mS{{\mathbf{S}}}
\def\mT{{\mathbf{T}}}
\def\mU{{\mathbf{U}}}
\def\mV{{\mathbf{V}}}
\def\mW{{\mathbf{W}}}
\def\mX{{\mathbf{X}}}
\def\mY{{\mathbf{Y}}}
\def\mZ{{\mathbf{Z}}}
\def\mBeta{{\mathbf{\beta}}}
\def\mPhi{{\mathbf{\Phi}}}
\def\mLambda{{\mathbf{\Lambda}}}
\def\mSigma{{\mathbf{\Sigma}}}


% Expectation
% \def\eE{\mathop{\mathbb{E}}\limits}
\def\eE{\mathbb{E}}

% Probability
\def\pP{\mathbb{P}}

% Tilde
\def\tf{\tilde{f}}
\def\tS{\tilde{S}}
\def\wtF{\widetilde{\mathcal{F}}}
\def\whR{\widehat{R}}
\def\tvx{\tilde{\mathbf{x}}}
\def\ty{\tilde{y}}


\def\defeq{\overset{\textup{def}}{=}}
% \def\defeq{\overset{.}{=}}
\def\defone{\overset{\text{\ding{172}}}{=}}
\def\deftwo{\overset{\text{\ding{173}}}{=}}
\def\leqone{\overset{\text{\ding{172}}}{\leq}}
\def\leqtwo{\overset{\text{\ding{173}}}{\leq}}
\def\leqthree{\overset{\text{\ding{174}}}{\leq}}
\def\leqfour{\overset{\text{\ding{175}}}{\leq}}
\def\eqone{\overset{\text{\ding{172}}}{=}}
\def\eqtwo{\overset{\text{\ding{173}}}{=}}
\def\eqthree{\overset{\text{\ding{174}}}{=}}
\def\eqfour{\overset{\text{\ding{175}}}{=}}
\def\geqfive{\overset{\text{\ding{176}}}{\geq}}
\input{useful_macros}

\usepackage{hyperref}
\usepackage{url}
\usepackage{graphicx}
\usepackage{tikz-cd}
\usepackage{amsthm}
\usepackage{subcaption}
% \newtheorem{theorem}{Theorem}
% \newtheorem{definition}[theorem]{Definition}
% \newtheorem*{notation}{Notation}
\usepackage{wrapfig}
\usepackage{bbm}
\usepackage{booktabs}


% \newcommand{\todo}[1]{\textcolor{red}{\textbf{TODO: #1}}}

% \definecolor{myblue}{RGB}{0, 102, 204}

\icmltitlerunning{Meta-Statistical Learning}

\begin{document}

\twocolumn[
\icmltitle{Meta-Statistical Learning: Supervised Learning of Statistical Inference}



\begin{icmlauthorlist}
\icmlauthor{Maxime Peyrard}{xxx}
\icmlauthor{Kyunghyun Cho}{yyy,zzz}
\end{icmlauthorlist}

\icmlaffiliation{xxx}{Université Grenoble Alpes, CNRS, Grenoble INP, LIG}
\icmlaffiliation{yyy}{New York University}
\icmlaffiliation{zzz}{Genentech}

\icmlcorrespondingauthor{Maxime Peyrard}{maxime.peyrard@univ-grenoble-alpes.fr}

% You may provide any keywords that you
% find helpful for describing your paper; these are used to populate
% the "keywords" metadata in the PDF but will not be shown in the document
\icmlkeywords{Machine Learning, ICML}

\vskip 0.3in
]


\printAffiliationsAndNotice{\icmlEqualContribution} % otherwise use the standard text.

\begin{abstract}
This work demonstrates that the tools and principles driving the success of large language models (LLMs) can be repurposed to tackle distribution-level tasks, where the goal is to predict properties of the data-generating distribution rather than labels for individual datapoints. These tasks encompass statistical inference problems such as parameter estimation, hypothesis testing, or mutual information estimation. 
Framing these tasks within traditional machine learning pipelines is challenging, as supervision is typically tied to individual datapoint.
% 
We propose \textit{meta-statistical learning}, a framework inspired by multi-instance learning that reformulates statistical inference tasks as supervised learning problems. In this approach, entire datasets are treated as single inputs to neural networks, which predict distribution-level parameters. Transformer-based architectures, without positional encoding, provide a natural fit due to their permutation-invariance properties.
% 
By training on large-scale synthetic datasets, meta-statistical models can leverage the scalability and optimization infrastructure of Transformer-based LLMs. We demonstrate the framework’s versatility with applications in hypothesis testing and mutual information estimation, showing strong performance, particularly for small datasets where traditional neural methods struggle. 
\end{abstract}


\section{Introduction}
\label{sec:introduction}
\begin{figure}[ht]
    \centering
    \includegraphics[width=0.8\linewidth]{graphs/greater_than_naive.pdf}
    \vspace{0.5cm}
    \includegraphics[width=0.8\linewidth]{graphs/p1_bottom.png}
    \vspace{-5pt}
    \caption{\textcolor{positional}{Positional} vs.\ \textcolor{nonpositional}{non-positional} circuits. In a \textcolor{nonpositional}{non-positional} circuit, the same edges must be included at all positions. A \textcolor{positional}{positional} circuit can distinguish between the same edge at different positions. This specificity yields better trade-offs between circuit size and faithfulness. It can also increase both precision and recall.}
    \label{fig:p1}
    \vspace{-5pt}
\end{figure}

\section{Introduction}

\looseness=-1
A primary goal of interpretability research is to characterize the internal mechanisms in language models (LMs) and other NLP models. 
A core approach in this area is \textbf{circuit discovery}---identifying the minimal subgraph within the model's computation graph that performs a specific task \citep{olah2021framework,olah-mech}.
Typically, the nodes of a circuit represent model components (e.g., attention heads, neurons, or layers).
While manual circuit discovery methods can yield position-specific insights \citep{wanginterpretability,goldowskydill2023localizingmodelbehaviorpath}, \emph{automatic methods often overlook positional information}, treating components as uniformly relevant across all input token positions \citep{conmytowards,syed2023attribution}. 
For instance, if an attention head is included in a circuit, it is assumed to contribute equally to the computation for every position in the input sequence.
The assumption that circuits are position-invariant ignores the fact that different positions often require distinct computations.
By ignoring positions, current methods limit their ability to capture mechanisms that operate across positions, such as interactions between attention heads across positions.

In this study, we start by demonstrating that positional agnosticism is a significant limitation (\S\ref{sec:motivating}). Then, to address these limitations, we introduce a new approach: position-aware edge attribution patching (PEAP; \S\ref{sec:full_circ_discovery}; Figure~\ref{fig:p1}). Current approaches  assume that if an edge is in a circuit, then the same edge will be in the circuit at all positions, thus leading to low precision. It is also assumed that an edge's importance should be aggregated across positions before deciding whether it should be included in the circuit; this can lead to cancellation effects, and thus low recall. PEAP instead allows us to compute the importance of cross-positional edges, and separately evaluates edge importance at each position. We show that this leads to smaller and more accurate circuits; see Figure~\ref{fig:p1}.

Incorporating positional information into circuit discovery is straightforward when inputs have the same length and structure across examples.

However, realistic datasets are not nearly this templatic.
How, then, can we incorporate positional information into automatic circuit discovery?
To address this challenge, we propose \textbf{schemas} (\S\ref{sec:schema}). 
Schemas assign semantic labels to spans of tokens, enabling information aggregation across examples even when the spans differ in length.

For example, in the input ``The \textcolor{positional}{war} lasted from 1453 to 14\underline{\hspace{1em}},'' the span ``\textcolor{positional}{war}'' could be labeled as ``\emph{Subject}''.
This enables handling spans with varying lengths: the phrase ``\textcolor{positional}{Black Plague}'' in another example can be treated as a single positional span with the same role as ``\textcolor{positional}{war}''.
In experiments with two LMs and three tasks, we find that circuits discovered using schemas achieve a better trade-off between circuit size and faithfulness to the model's behavior than position-agnostic circuits.
Importantly, position-aware circuits offer a more precise representation of the underlying mechanisms, providing a more concise foundation for mechanistic explanations.

We also present a fully automated pipeline for schema generation and application (\S\ref{sec:schema-generation}) using large language models (LLMs). 
We evaluate the quality of the generated schemas and their utility in discovering position-aware circuits (\S\ref{sec:schema-eval}).
Notably, circuits derived using automatically generated and applied schemas achieve comparable faithfulness scores to circuits discovered with human-designed and manually applied schemas.

We summarize our contributions as follows:
\begin{itemize}[noitemsep,leftmargin=*,topsep=1pt,parsep=1pt]
    \item Introduce a position-aware circuit discovery method, which obtains better faithfulness than position-agnostic discovery.  
    \item Introduce dataset schemas,  facilitating positional circuit discovery in more naturalistic settings. 
    \item Develop an automated schema generation and application pipeline with LLMs, yielding schemas that are comparable to manually-annotated ones.
\end{itemize}


\section{Meta-Statistical Learning}
\label{sec:meta_ml}
%% TODO more details in methodology and data processing
% merge methodology and 
\section{Framework for Analyzing Emotion}
In this section, we present our framework for analyzing emotion. We first establish a basic understanding of emotion polarity by determining the sentiment valence of each root tweet and comment. We then use multi-label emotion detection to predict the emotion categories associated with each post. Based on this data, we explore the interactive nature of emotions, by identifying common patterns in emotion transition pairs between temporally-adjacent posts. Finally we investigate the emotional trajectory within threads to understand how emotional intensity and type shift over time, by aggregating the predicted labels for posts at each time stamp in a given thread. As part of this, we contrast rumour with non-rumour threads, to gain a holistic understanding of emotional expression in rumours and non-rumours on Twitter.

% elaborate a bit on why we choose EmoLLM, compared with other automatic emotion detection methods
\paragraph{Affective Computing: Automatic Emotion Detection}
Manually annotating emotions is both costly and time-consuming, so we use an LLM-based emotion detection model, EmoLLM~\citep{liu2024emollms}, which is specifically designed for sentiment analysis and emotion detection. The model was instruction-tuned on SemEval 2018 Task1 using a comprehensive emotion labeling scheme grounded in established theoretical frameworks. We prompt the model to perform Valence Ordinal Classification (V-oc), Emotion Classification (E-c), and Emotion Intensity regression (E-i). Detailed prompts are shown in \Cref{tab:emollm_ins}.

\paragraph{Categorical Emotion Labeling Scheme} \label{para:emotion_label}
Numerous emotion label sets  have been proposed~\citep{Ekman1992AnAF, Plutchik1980AGP, Russell1980ACM}. According to \citet{Ekman1992AnAF, Plutchik1980AGP}, certain emotions, such as joy, fear, and sadness, are considered more fundamental than others, both physiologically and cognitively. The Valence-Arousal-Dominance (VAD) model \citep{Russell1980ACM} categorizes emotions within a three-dimensional space of valence (positivity-negativity), arousal (active-passive), and dominance (dominant-submissive). Inspired by \citet{mohammad-etal-2018-semeval}, we incorporate elements from both basic emotion theories and the VAD model, and further ground EmoLLM emotion predictions to develop the following emotion label schemes: (1) \textit{neutral or no emotion}; (2) \textit{negative emotions}: anger (also includes annoyance and rage),  disgust (also includes disinterest, dislike, and loathing), fear (also includes apprehension, anxiety, and terror), pessimism (also includes cynicism, and no confidence), sadness (also includes pensiveness and grief); 3) \textit{positive emotions}: joy (also includes serenity and ecstasy), love (also includes affection), optimism (also includes hopefulness and confidence), anticipation (also includes interest and vigilance), surprise (also includes distraction and amazement) and trust (also includes acceptance, liking, and admiration). 


\paragraph{Emotion Polarity: Sentiment Valence} 
To understand the basic emotion polarity expressed in rumour and non-rumour content, we begin with sentiment valence analysis. Sentiment valence aims to capture the overall emotional tone conveyed by a post, in terms of how positive or negative it is~\citep{liu2024emosurvey}. We frame the sentiment valence task as ordinal regression~\citep{mohammad-etal-2018-semeval}. As shown in \Cref{tab:emollm_ins}, for a given tweet post, we classify it into one of seven ordinal levels of sentiment intensity, spanning varying degrees of positive and negative valence, that best represents the tweeter's mental state. The tweet posts within a thread can be divided into two categories: root tweets, which are posted by the publisher, and follow posts, which include all subsequent replies under the root post. We begin by conducting sentiment valence analysis on each post within the thread conversation. 
% TJB: confused by how comments can include all subsequent replies; we seem to be overloading the terminology, for comments to be both individual posts and series of posts
% RX: yes, I am unifiying all terms.
For each category, we compute the mean sentiment valence to enable further investigation into the specific emotions associated with different sentiment valences over a thread.
% TJB: clarify for comments whether the classification is done over the combined meta-document (i.e. the root + all comments to that point) or individually over the separate documents and then combined ... or over individual documents, in which case the statement about "all replies" needs clarification
% RX: we separate root and comments for each tweet conversation, the former is the root tweet posted by the publisher while the rest are comments. "all replies" mean all comments under root tweet, we aggregate them by computing the mean sentiment, and then average over each part.

\paragraph{Emotion Distribution} 
Following sentiment valence analysis, we then examine specific emotions and their distribution in rumour and non-rumour tweet posts.
Motivated by the fact that a certain tweet might exhibit more than one emotion, we frame the task as multi-label emotion detection problem. As shown as V-oc in \Cref{tab:emollm_ins}, given a tweet, we classify it into one of seven ordinal classes, corresponding to various levels of positive and negative sentiment intensity. To reduce noise from automatic emotion detectors, we take the top-three predicted emotions for each tweet. We then aggregate and plot the emotion distribution to provide an overview of dominant emotional trends across the rumour and non-rumour posts. Given that the follow posts make up the majority of the data compared to the root posts, we will focus on using follow posts in our next analysis.
% TJB: what is the basis of saying that the signal is richer? simply that there are more reply posts than root posts? clarify
% RX: yes, and we are more interested in interaction in comments.

\paragraph{Emotion Transitions} 
Emotions are contagious and highly interactive~\citep{Ferrara_2015}. When publishers write tweets that convey their emotions, readers are likely to respond with emotional reactions of their own~\citep{Ferrara_2015,emotion_dynamics}. In this part, we model this interactive nature of emotions in the form of emotion transition pairs, which are built from two chronologically-adjacent tweets. In each pair, the first element represents the emotion inferred from the initial content published at a given time, and the second element represents the emotion inferred from the reply content published immediately after. For example, if the first tweet exhibits \textit{joy} \textit{trust} and \textit{anticipation}, and the second tweet shows \textit{anger}, \textit{disgust} and \textit{surprise}, we form the pairs (\textit{joy}, \textit{anger}), (\textit{joy}, \textit{disgust}), (\textit{joy}, \textit{surprise}), (\textit{trust}, \textit{surprise}), (\textit{trust}, \textit{surprise}), (\textit{trust}, \textit{disgust}), (\textit{anticipation}, \textit{anger}), (\textit{anticipation}, \textit{surprise}) and (\textit{anticipation}, \textit{disgust}). We create transitions for all combinations of emotion pairs and explore the likelihood of emotion transition pairs occurring in rumour and non-rumour content. Exploring emotion transitions allows us to understand the emotional flow in social media conversations and uncover typical patterns of rumour and non-rumour content, and any differences between the two.

\paragraph{Emotion Trajectories} 
We explore the cumulative trajectory of emotion over time to observe how emotions evolve during the conversational thread. We collect all detected emotion labels for each tweet from both rumour and non-rumour content, then track cumulative emotion counts at each chronological step. Finally, we visualize these trends and apply regression models to analyze the growth of emotions over time. This temporal analysis reveals how emotions accumulate or intensify across time, offering insight into the trajectory of emotions in rumour and non-rumour content.

\begin{table*}[!h]
    \centering
    \small
    \begin{tabular}{cccccccccccc}
        \toprule
        \textbf{Setting} & \textbf{Ru} & \textbf{Non} & \textbf{p} & \textbf{\#Ru/Non} & \textbf{T} & \textbf{F} & \textbf{U} & \textbf{$p$ (U vs T)} & \textbf{$p$ (U vs F)} & \textbf{\#T/\#F/\#U} \\
        \midrule
        \textbf{PHEME root} & \textbf{$-$0.25} & $-$0.17 & 0.00 & 2602/2602 & $-$0.21 & $-$0.11 & \textbf{$-$0.39} & 7.75e-11 & 4.41e-11 & 629/629/629 \\
        \textbf{PHEME follow} & \textbf{$-$0.33} & $-$0.26 & 6.47e-09 & & $-$0.35 & $-$0.20 & \textbf{$-$0.39} & 0.03 & 8.38e-15 & \\
        \textbf{Twitter15 root} & \textbf{$-$0.26} & $-$0.01 & 3.51e-05 & 372/372 & $-$0.21 & $-$0.20 & \textbf{$-$0.34} & 0.01 & 0.01 & 359/359/359 \\
        \textbf{Twitter15 follow} & \textbf{$-$0.27} & $-$0.06 & 1.65e-09 & & $-$0.24 & $-$0.25 & \textbf{$-$0.30} & 0.16 & 0.21 & \\
        \textbf{Twitter16 root} & \textbf{$-$0.18} & \z0.07 & 0.00 & 205/205 & \z0.11 & $-$0.22 & \textbf{$-$0.30} & 1.35e-06 & 0.18 & 63/63/63 \\
        \textbf{Twitter16 follow} & \textbf{$-$0.31} & $-$0.12 & 9.19e-06 & & $-$0.30 & \textbf{$-$0.36} & $-$0.27 & 0.67 & 0.90 & \\
        % \textbf{CoAID root} & \textbf{$-$0.34} & $-$0.16 & 0.01 & 167/167 & - & - & - & - & - & - \\
        % \textbf{CoAID follow} & \textbf{$-$0.24} & $-$0.13 & 0.01 & & - & - & - & - & - & \\
        \bottomrule
    \end{tabular}
    \caption{Valence Ordinal Regression results for all datasets. root = root posts, follow = follow posts, Ru = rumour, Non = Non-rumour, T = True rumour, F = False rumour, U = Unverified rumour; $p$ values indicates significance of a one-tailed t-test.}
\label{tab:voc_results}
\end{table*}

\begin{algorithm}[ht!] 
\caption{PC Algorithm}
\label{pc}
\begin{algorithmic}[1] 
\State \textbf{Input:} Data $\mathbf{X}$, significance level $\alpha$
\State \textbf{Output:} Completed Partially Directed Acyclic Graph (CPDAG)

\State Initialize a complete undirected graph $G$ with all variables as nodes.

\State \textbf{Step 1: Skeleton Identification}
\For{each pair of variables $(X, Y)$ in $G$}
    \State Find the subset $S \subseteq \text{Adj}(X, G) \setminus \{Y\}$ such that 
    $X \indep Y \mid S$ with significance $\alpha$.
    \If{such a subset $S$ exists}
        \State Remove the edge $X - Y$ from $G$.
    \EndIf
\EndFor

\State \textbf{Step 2: Edge Orientation}
\For{each triple of variables $(X, Y, Z)$ in $G$ where $X - Z - Y$ and $X, Y$ are not adjacent}
    \If{$Z \notin S$ for all separating sets $S$ for $X$ and $Y$}
        \State Orient as $X \to Z \leftarrow Y$ (identify a collider).
    \EndIf
\EndFor

\While{possible}
    \For{each edge $(X - Y)$ in $G$}
        \If{there exists a directed path $X \to \dots \to Z$ such that $Z - Y$}
            \State Orient as $X \to Y$ (acyclicity rule).
        \ElsIf{orienting $X - Y$ as $X \to Y$ creates a new v-structure}
            \State Orient as $X \to Y$ (v-structure rule).
        \EndIf
    \EndFor
\EndWhile

\State \textbf{return} the CPDAG representing the equivalence class of causal graphs.

% how we frame the task, compute the emotion intensity, how to aggregate on conversation level

\end{algorithmic}
\end{algorithm}


\paragraph{Causal Relationship of Emotions in Rumour \& Non-Rumour Threads}
To gain a deeper insight into the relationship between rumours and the emotions underlying them, we extend our analysis beyond statistical correlation by conducting a causal analysis. Specifically, we apply the Peter-Clark (PC) algorithm \cite{Spirtes2000}, a classical constraint-based causal discovery algorithm on the three merged datasets. 

Uncovering causal relations between variables of interest is never an easy problem. Under the fundamental assumption of \textit{causal Markov condition} that a variable is conditionally independent of all its non-effects given its direct cause, \textit{faithfulness} ensures that the casual graph exactly encodes the independence and conditional independence relations among variables. These two assumptions allow us to infer causal relationships from observed statistical independencies, forming the cornerstone of constraint-based causal discovery methods. 

The PC algorithm identifies causal relationships among the variables of interest, represented as a directed acyclic graph (DAG), by numerating the independence and conditional independence relationships. The algorithm consists of two main steps: 
\begin{enumerate}
    \item \textbf{Skeleton Identification}: Starting with a complete undirected graph where all variables are connected, edges are iteratively removed based on conditional independence and independence relationships among variables, inferred by a conditional independence test. This step returns an undirected graph, which we call a skeleton. 
    \item \textbf{Edge Orientation}: After constructing the skeleton, edges are oriented by a set of predefined rules (Meek's Rule \cite{meek1997graphical}) to avoid cycles and orient collider structures.
\end{enumerate}

The complete PC algorithm is provided in algorithm \ref{pc}. It returns a  completed partially directed acyclic graph (CPDAG), which represents an equivalence class of causal graphs that are consistent with the observed data’s independence and conditional independence relations. In our implementation, we adopt the  Fisher-z test \cite{fisher_probable_1921} to infer the conditional independence relations.

%%% Local Variables:
%%% mode: latex
%%% TeX-master: "../main_anonymous"
%%% End:


\section{A Framework for Measuring \ENDow{}}
\label{sec_framework}

%Inspired by the vast research conducted on investigating the effect of transcription noise on downstream NLU tasks, our goal is to formulate a framework that systematically analyzes SLU pipelines. 
With the purpose of systematically analyzing SLU pipelines,
our framework's objective is to describe the behavior of downstream tasks as a function of the noise score (e.g., WER, which we use throughout the paper, but any transcription noise metric can be applied) and the type of noise in transcripts. 
%, and what cleaning methods effectively boost the outcomes.

The \textbf{input} to the framework is an SLU dataset $D = (T, O)$, where $T$ is a set of reference transcripts and $O$ are the respective expected outcomes. For example, a set of meetings and their respective summaries, for the task of meeting summarization.
%Note that some datasets also provide audio files along with the transcripts, denoted $A$. 

The framework consists of a pipeline (illustrated in \autoref{fig_framework}) which includes a text-to-speech (TTS) model to generate audio files for $T$; the acoustic noising method and intensity to apply on the audio; an ASR system for audio transcription; the transcript cleaning technique; the downstream task model; and the evaluation metrics for the task. The components in the pipeline are flexibly set according to the use-case being analyzed.

The framework \textbf{output}s a report on the behavior of the SLU pipeline at the different noise levels and with the cleaning techniques assessed (\S{\ref{sec_framework_analysis}}).

% \paragraph{Framework input.}
% The framework receives two inputs:

% \noindent
% 1. An SLU dataset $D = (T, O)$ that contains reference transcripts ($T$) and the respective expected outcome of each transcript instance ($O$). For example, a set of meetings and their summaries. Note that some datasets also provide audio files along with the transcripts, denoted $A$.

% \noindent
% 2. A configuration specifying the components of the SLU pipeline (\S{\ref{sec_framework_noise}} and \S{\ref{sec_framework_task}}). The pipeline includes: A text-to-speech (TTS) model to generate audio files for transcripts; the noising method to apply on the audio, and number of noising intensities to apply; an ASR system for audio transcription; the transcript cleaning techniques; the downstream task model; and the evaluation metrics for the task.

% \paragraph{Framework output.}
% The framework outputs a 
% %quantitative
% report on the behavior of the SLU pipeline at different noise levels and cleaning techniques (\S{\ref{sec_framework_analysis}}).


\subsection{Preparing Transcripts with Varying Noise}
\label{sec_framework_noise}

\paragraph{Creating initial audio files.}
Audio files are first created for the input transcripts, in case the SLU dataset lacks them (or when using a non-SLU dataset), or to begin the analysis with clean audio\footnote{That is, with clear speech, without background noise or overlapping speakers.} for greater control over the subsequent noising process.
%Audio files are first prepared in several scenarios: (1) when the input SLU dataset provides transcripts without source audio files; (2) when a non-SLU dataset is used to enrich data availability; (3) to start the analysis with clean audio files\footnote{That is, with clear speech, no background environmental sounds, and without overlapping speakers.} for greater control over the subsequent noising process.
%Many SLU datasets provide transcripts without source audio files. Analogously, non-SLU datasets can be leveraged for our setting to enrich analyses.
%In some SLU datasets, the source audio files are not available.
%, and only the transcripts are on hand.
%Also, starting the analysis with clean audio files\footnote{That is, with clear speech, no background environmental sounds, and without overlapping speakers.} can provide greater control over the noising process.
%Alternatively, one may wish to initiate the \ENDow{} analysis with clean audio files\footnote{That is, with clear speech, no background sounds, and without multiple speakers at a time.} in order to have more control over the noising process.
%In such scenarios,
The TTS system is executed on each input (transcript) in dataset $D$, resulting in the corresponding set of audio files $A$.
%transcript in $T$ of input dataset $D$. The result of this step is the set of audio files $A$.
%\footnote{Technical details for the pipeline are in Appendix \ref{sec_appendix_implementation}.}

\paragraph{Adding noise to audio files.}
Given the audio files $A$, each is acoustically impaired at $k$ levels to increase transcription difficulty, preferably under realistic acoustic conditions.
%Given the set of audio files $A$, each audio file is to be acoustically impaired at $k$ levels, with the purpose of making the speech difficult to transcribe at different severities, preferably based on realistic acoustic settings. 
To that end, reverberation (i.e., sound reflection, like echoing) is applied, and background sounds are added with increasing intensity (signal-to-noise ratio) \citep{wang2018speechsep}.
%, according to the input configuration.
This stage yields a collection of audio sets $\{A_i\}_{i=1}^k$ (and we define $A_0 = A$), where the severity of impairment increases as $i$ increases.

\paragraph{Transcribing audio files.}
The ASR model is then executed on the audio files in sets $\{A_i\}_{i=0}^k$, resulting in respective transcripts $\{\widehat{T_i}\}_{i=0}^k$. Overall, there are $k+2$ sets of transcripts for dataset $D$: the $k+1$ ASR-generated sets and reference set $T$. It is expected that as $i$ increases, $\widehat{T_i}$ will have a higher WER score (more errors) with respect to $T$.
%For each $A_i$, the configured ASR model is run on each of $A_i$'s audio files to generate the respective transcript. This stage results in a collection of ASR-generated transcripts $\{\widehat{T_i}\}_{i=0}^k$. Overall, there are $k+2$ sets of transcripts for dataset $D$: the $k+1$ ASR-generated sets, and the reference set $T$. It is expected that as $i$ increases, $\widehat{T_i}$ will have a higher WER score (more errors) with respect to $T$.

\paragraph{Cleaning transcripts.}
Each non-reference transcript (in all sets $\widehat{T_i}$) is partially repaired using one of $m$ cleaning techniques. This culminates in sets $\{\{\widehat{T_{i_j}}\}_{j=1}^m\}_{i=0}^k$, and $\widehat{T_{i_0}} = \widehat{T_i}$ (when no cleaning is performed on $\widehat{T_i}$), encompassing $(k+1)*(m+1)$ different levels and types of transcript noise.
%Each non-reference transcript (those in sets $\widehat{T_i}$) is partially repaired using $m$ different cleaning techniques, as configured. In all, there are $(k+1)*(m+1)$ non-reference transcript sets with different levels and types of noise. I.e., the prepared sets are $\{\{\widehat{T_{i_j}}\}_{j=0}^m\}_{i=0}^k$, where $\widehat{T_{i_0}} = \widehat{T_i}$ (when no cleaning is performed on $\widehat{T_i}$).


\subsection{Executing the Downstream Task}
\label{sec_framework_task}
Next, the task model is executed on each of the transcripts in the prepared transcript sets, producing the respective predicted outputs $\{\{\widehat{O_{i_j}}\}_{j=0}^m\}_{i=0}^k$, and $\widehat{O}$ for the reference transcripts $T$. The predicted outputs in each set $\widehat{O_{i_j}}$ and $\widehat{O}$ are then evaluated against the respective expected outcomes in $O$. 
Finally, this process culminates with the overall score of each dataset variant $\{\{s_{i_j}\}_{j=0}^m\}_{i=0}^k$ and $s$.\footnote{To clarify, $s$ is the score obtained on reference transcripts $T$, portraying a standard execution of the SLU task on input dataset $D$. Score $s_{i_j}$ is for one of the noisy dataset variants.}
%(with respective margins of error).

In addition, the WER score is computed for each transcript set $\widehat{T_{i_j}}$ with respect to references $T$. Accordingly, this produces WER scores $\{\{w_{i_j}\}_{j=0}^m\}_{i=0}^k$ (see Appendix \ref{sec_appendix_implementation_wer} for details). Notice that $T$'s WER is $0$. With the task scores and respective WER scores, we can now assess and compare the performance of the dataset variants.


\subsection{Analyzing the Results}
\label{sec_framework_analysis}

Each of the WER and task score-pairs $(w_{i_j}, s_{i_j})$ is a data point that can be plotted on a graph.
The curve $l_j = [(0, s)] \cdot [(w_{i_j}, s_{i_j})]_{i=0}^k$ describes the behavior of a task model as noise increases in the transcripts (as $i$ increases), when applying cleaning technique $j$ (or when no cleaning is enforced, at $j=0$).
%See an illustration of the graph in \autoref{fig_framework}.
These curves form a basis for analyzing the configured SLU pipeline, as explained next.
(See \autoref{fig_framework_graph} in the Appendix for visualization.)
%The line $[(0, s)]; [(w_{i_0}, s_{i_0})]_{i=0}^k$ (when $j=0$) describes the behavior of the SLU pipeline as noise increases in the transcripts, and without any enforced cleaning. For each cleaning technique $j$, the line $[(0, s)]; [(w_{i_j}, s_{i_j})]_{i=0}^k$ describes the corresponding behavior when cleaning is enforced. For example... \todo{forward reference to an example plot}.

\paragraph{Model performance vs. noise level.}
%A curve $l_j$ describes the behavior of the NLU task model as transcripts bear more errors.
%As noise accumulates the results on the downstream task exacerbate.
As transcript noise accumulates, NLU task model performance is expected to degrade.
One question to ask is: \textit{how much transcript noise can the task model tolerate before its performance is jeopardized?} To that end, we define the \textbf{\textit{noise-toleration point}} (NTP) as follows. For curve $l_j$, described by function\footnote{Note that the curve is not continuous since it is made up of several discrete segments. See Appendix \ref{sec_appendix_implementation_ntp} for details on how the noise-toleration point is computed.} $f_j$, and the respective upper and lower bound functions $f_j^{\text{upper}}$ and $f_j^{\text{lower}}$ (based on the margins-of-error), we define $l_j$'s noise-toleration point, $w^t_j$, as the WER score when $f_j^{\text{lower}}(0) = f_j^{\text{upper}}(w^t_j)$, i.e., 
the lowest WER at which the task score becomes statistically significantly lower than when transcripts have no noise, indicating a notable drop in task-model performance due to noise.
%the minimum WER score where the task score is statistically significantly smaller than the task score when there is no noise in the transcripts, marking a significant drop in task performance.
%See the illustration in \autoref{fig_framework}.
%Nevertheless, a task model has a \textit{noise-toleration point}. This is the extent of noise until which the task-scores are relatively stable, i.e., the task model can tolerate that amount of errors in the transcripts. This point also marks the onset of a more substantial downward trend of task-scores. We define the noise-toleration point as the noise level until which the task-score does not fluctuate more than a specified threshold. \todo{Formalize with an equation?} For example, \todo{refer to a pplot in the results and say what the point is...}

Another question to ask about the SLU pipeline is: \textit{how do different models behave comparatively, with respect to noise level?} The general behavior is approximated with the \textbf{\textit{area-under-the-curve}} (AUC), which can be compared between curves to judge which model is generally more tolerant to noise. Furthermore, by focusing on a certain region in the graph, the localized behavior is comparable. For example, in \autoref{fig_noclean_graphs}a, the GPT model is the better model at lower WER levels, but drops to the bottom rank at high WER levels.\footnote{The reliability of the analyses increases with the number of points constructing a curve (increasing $k$) and with a broader coverage of the WER score range (between 0 and 1).}
%more levels of noise that enrich a model's curve (larger value of $k$ and a broader range of word-error rates).

%The behavior of a curve changes between tasks, models and noise types, and the examination described yields a practical understanding of what to expect from a tested SLU solution.


\paragraph{Comparing cleaning techniques.}
Applying a cleaning technique on transcripts decreases the noise, and consequently shifts the plots leftward. Cleaning a transcript also essentially means that the \textit{type} of noise changes, and therefore the task model reacts differently to the errors in the transcripts, potentially altering the behavior of the curves altogether.
The point $(w_{i_j}, s_{i_j})$ with respect to point $(w_{i_0}, s_{i_0})$ portrays how much ``effort'' is required (the decrease in WER: $w_{i_0} - w_{i_j}$) in order to change the task score from $s_{i_0}$ to $s_{i_j}$. The effect of each cleaning method $j$ varies, and therefore all $l_j$s are compared with respect to $l_0$ (e.g., see \autoref{fig_cleaning_graphs}). Ultimately, an effective cleaning technique should increase the task scores with minimum effort.

Formally, let $\Delta w_{i_j} = w_{i_0} - w_{i_j}$ be the change in WER for noising level $i$ and cleaning method $j$, and $\delta s_{i_j} = (s_{i_0} - s_{i_j}) / s$ be the respective relative\footnote{The change in task-score is normalized by the score at WER=0 to get the relative change. The change in WER is already on a 0-to-1 scale, and is not further normalized.} change in the task-score. The pointwise effectiveness score of cleaning technique $j$ at noise-level $i$ is measured as $e_{i_j} = \delta s_{i_j} / \sqrt{\Delta w_{i_j} + \epsilon}$.\footnote{We applied a square root transformation on the \textit{effort} ($\Delta w_{i_j}$) to reduce the impact of the larger changes at noisier levels, and to increase the weight of the change in task score ($\delta s_{i_j}$). $\epsilon$ is a minuscule value to prevent division by zero.} Finally, we measure the \textit{\textbf{cleaning-effectiveness score}} (CES) of cleaning method $j$ with the average: $\frac{1}{k+1} \sum_{i=0}^k e_{i_j}$. The higher the score, the better the overall improvement in the downstream task with a lower effort of cleaning. A score of 0 means that the cleaning procedure had no effect on the task-model's results, and a negative score means that there was a deterioration of task results, on average. 

The CES metric captures the two objectives of a cleaning technique: heightened task results for lesser effort.
%\todo{try out the acceleration of the changes -- get the derivative of the regression line formed by the delta\_y / delta\_x line -- the larger the acceleration, the better.}
The metric suggests how comparably effective a cleaning method is for the data and task-model in question. As such, it compares the effects of different \textit{types} of noise in the transcripts, as we exemplify in our experiments in Section \ref{sec_results}.




%The more levels of impairment, the more precise the \ENDow{} assessment will be

\section{Experimental Setup}
\label{sec:experimental_setup}
\section{Experimental setup}
\label{experimental_setup}


\paragraph{Models}
% main models
% variety of models with moderate model size: Llama 3.2 3B, Gemma 2 2B, Qwen2.5 3B
% \cite{dubey2024llama, team2024gemma, yang2024qwen2}
% math-specific models
% {yang2024qwen2math, shao2024deepseekmath}
% we use math specific models to verify the robustness of methods when applied to models that undergo extensive post-training.
% we also conduct scaling study on the llama 3 family of models, for 1B (3.2), 3B (3.2) and 8B (3.1).
% main models
% variety of models with moderate model size: Llama 3.2 3B, Gemma 2 2B, Qwen2.5 3B
% \cite{dubey2024llama, team2024gemma, yang2024qwen2}
% math-specific models
% {yang2024qwen2math, shao2024deepseekmath}
% we use math specific models to verify the robustness of methods when applied to models that undergo extensive post-training.
% we also conduct scaling study on the llama 3 family of models, for 1B (3.2), 3B (3.2) and 8B (3.1).
To account for realistic task-specific deployment settings, we select recent moderately sized post-trained models.
We also consider math-specialized models to evaluate on models that have been optimized for specific task domains.
We consider five main models for our key experiments: Llama-3.2-3B \small\cite{dubey2024llama}\normalsize, Gemma-2-2B \small\cite{team2024gemma}\normalsize, Qwen2.5-3B \small\cite{yang2024qwen2}\normalsize, Qwen2.5-Math-1.5B \small\cite{yang2024qwen2math}\normalsize, and DeepSeekMath-7B \small\cite{shao2024deepseekmath}\normalsize.
We investigate scaling on Llama-3.2-\{1B, 3B\} and Llama-3.1-8B.

\paragraph{Tasks and datasets}
% - focus on math reasoning tasks because they require shortform answers, and CoT reasoning is used to enhance the accuracy of answers. It is desirable to reduce reasoning in such task to reduce inference cost, whereas it may not be desirable to reduce output length in tasks that require longform answers such as creative writing.
% focus on sufficiently difficult reasoning tasks where CoT reasoning is beneficial for performance.
% we use GSM8K \cite{cobbe2021training} and MATH \cite{hendrycks2021measuring}, where our models achieve 40--90% and 20--70% accuracy, respectively.
We focus on challenging reasoning tasks where (1) CoT reasoning significantly improves model performance, (2) only the final answer is relevant, and (3) models achieve moderate performance.
Reasoning length reduction is desirable under the first and second conditions as it can reduce inference latency without affecting utility.
The third condition is necessary to assess accuracy preservation.
We consider two mathematical reasoning datasets: GSM8K \cite{cobbe2021training} and MATH \cite{hendrycks2021measuring}, where the models achieve accuracies of 40--90\% and 20--70\%, respectively.
For evaluation, we utilize the test set of the GSM8K and MATH500 dataset~\citep{lightman2023let}.
We explain details in \autoref{appx_datasets}.

\paragraph{Evaluation metrics}
We evaluate methods using two primary metrics: \textit{accuracy} and \textit{length}.
Accuracy is evaluated using Python-based parsing code, described in \autoref{appx_answer_parsing}.
Length is defined as the average number of output tokens in all reasoning paths, including incorrect ones, as output tokens incur inference costs in deployment scenarios regardless of their correctness.
We focus on number of \textit{output} tokens for clear comparison, as number of input tokens are similar across methods, and output tokens affect wall-clock latency to a higher degree  \cite{agrawal2024taming}.
% Namgyu - it seems the first reason is more important. Input token prefilling time is not insignificant, as shown by zero-shot vs few-shot generation time.
We further justify this choice in \autoref{appx_justification_length}.
% Consistent with our preliminary analysis in \autoref{sec:preliminary},
We also employ \textit{relative accuracy} and \textit{relative length} metrics to better evaluate how well each method elicits concise reasoning while maintaining accuracy.
% Concretely, relative accuracy refers to the given method's accuracy divided by the baseline accuracy. The same goes for relative length.
Specifically, relative accuracy is the ratio of the given method's accuracy to the baseline accuracy, and relative length is the ratio of the given method's average length to the baseline average length, using a strong baseline zero-shot prompt \cite{pang2024iterative}.
We use greedy decoding throughout evaluation to ensure reproducibility\footnote{
Note that this contrasts with \autoref{sec:preliminary} where we took the average length of \textit{correct} samples over \textit{stochastic} generations per question, for precise question-level length normalization.
}.
% We evaluate methods using two primary metrics: \textit{accuracy} and \textit{length}. We assess concise reasoning ability in line with our preliminary analysis in \autoref{sec:preliminary} and also employ \textit{relative accuracy} and \textit{relative length} metrics.
% \textit{Accuracy} is evaluated using Python-based parsing code, which is detailed in \autoref{appx_answer_parsing}.
% \textit{Relative accuracy} is the ratio of the method's accuracy to the baseline accuracy.
% \textit{Length} is defined as the average number of output tokens in reasoning paths obtained through greedy decoding across the entire evaluation set. 
% % This differs from the calculation in \autoref{sec:preliminary} as it includes incorrect reasoning paths.
% % In real-world deployments, the tokens in all reasoning paths contribute to inference costs, regardless of correctness.
% This differs from the calculation in \autoref{sec:preliminary} as it also includes incorrect reasoning paths, since in real-world deployments, tokens of all reasoning paths contribute to inference costs regardless of correctness.
% We focus on the number of \textit{output} tokens for a clearer comparison, as this correlates to real-world wall-clock latency, whereas input tokens are typically processed with significantly higher throughput \cite{agrawal2024taming} (\autoref{appx_justification_length}).
% \textit{Relative length} is the ratio of the method's average length to the baseline's average length.

\paragraph{Baseline methods}
For baselines, we consider zero-shot prompting (\autoref{zero_shot_prompting}) and fine-tuning directly on ground-truth answers, and with external supervision from human- and GPT-4o-generated concise reasoning paths.
For existing fine-tuning methods on concise reasoning, we reproduce Rational Metareasoning (RM) \cite{de2024rational}, which follows a similar approach to our naive BoN method but with two key differences: (1) a reward function that balances efficiency and utility (i.e., output length and accuracy), and (2) iterative training via expert iteration \cite{zelikman2022star}.
% In contrast, our naive BoN method selects the shortest samples and is trained in a single step.

\paragraph{Self-training budget allocation}  
% The cost structure of self-training methods consists of two components: generation and fine-tuning. We generate $N = 16$ paths per question for BoN methods. In our reproduction of RM \cite{de2024rational}, we generate $N = 4$ paths in each of their 4 iterations. For FS-BoN, our default setting includes 16 few-shot-conditioned and 16 naive BoN-conditioned paths. Additionally, we explore a `Budget-Matched' setting, where we use 8 paths from few-shot conditioning and 8 from naive BoN, as shown in Table \ref{tab:main_results}. We focus on the generation budget since selecting the most concise reasoning path for each question keeps the fine-tuning cost minimal across all methods. Detailed experimental settings are provided in~ \autoref{appx_experimental_details}.
The cost of self-training methods can be broken down into (1) generation and (2) fine-tuning.
We aim to match the budget across methods in terms of generation (number of paths generated per question) since fine-tuning costs are relatively negligible (see Table \ref{tab:generation_training_time}).
For naive BoN, we generate 16 paths per question.
For RM \cite{de2024rational}, we generate $N = 4$ paths per iteration across 4 iterations.
We generate a single path for standalone few-shot conditioning (FS) and 16 paths for FS-BoN, both augmented with 16 additional paths from the default output distribution (used for naive BoN).
Considering the large budget of FS-BoN with augmentation, we evaluate a \textit{Budget-Matched} setting with 8 paths each from the few-shot conditioned distribution and default distribution, as shown in Table \ref{tab:main_results}.
Details are provided in~\autoref{appx_generation_and_fine_tuning}.
% Namgyu - BoN conditioning is not a valid expression in our paper

\section{Experiments on Descriptive Tasks}
\label{sec:experiments_desc}
In descriptive tasks, the label \( y \) of a dataset \( \mathcal{D} \) is the output of an algorithm \( A \) applied to \( \mathcal{D} \), i.e., \( y = A(\mathcal{D}) \). Simple tasks like median or correlation serve as unit testing of meta-statistical models. However, for more computationally intensive algorithms, such as optimal transport, meta-statistical models could serve as fast approximations. For datasets \( \mathcal{D} \in \mathbb{R}^{n \times m} \), we consider four descriptive tasks: the \textbf{per-column median} label \( y \in \mathbb{R}^m \) consists of the medians of each column. The \textbf{Pearson correlation} coefficient \( y \in \mathbb{R} \) is computed between the two columns. The \textbf{win rate} (Bradley-Terry) is the fraction of rows where the value in the first column exceeds that in the second: \( y = \frac{1}{n} \sum_{i=1}^n \mathbb{I}(\mathcal{D}_{i,1} > \mathcal{D}_{i,2})\), where \( \mathbb{I}(\cdot) \) is the indicator function. Finally, the 1D \textbf{optimal transport} (OT) label \( y \in \mathbb{R} \) is the optimal transport cost between the empirical distributions of the two columns.


\xhdr{Meta-Dataset Generation}
To construct the meta-dataset, we sample datasets \( D \) from predefined probability distributions as described in \Secref{sec:experimental_setup}. Once a dataset is sampled we simply compute the target label \( y \) by applying the target algorithm. We experiment with various numbers of columns $m$. By having $k > 1$, we produce $k$ computation in parallel with one forward pass (independently of the batch dimension). We observe no significant difference when varying $k$ and fix $k=2$ in the experiments. The meta-dataset contains $30K$ training meta datapoints per task, with dataset sizes sampled from $n \in [5, 300]$. Details about meta-datasets and which distribution families are in- or out-of-meta-distribution are provided in \Appref{app:desc_details}.


\xhdr{Meta-Statistical Models}  
After optimizing hyperparameters and architecture choices (e.g., pooling mechanisms and head-to-dimensionality ratio) on a small validation set of 1K meta datapoints, we compare four meta-statistical model variants: LSTM, Vanilla Transformer (VT), and two ST2 variants with 16 or 32 inducing points. ST2(16) is the fastest model for both training and inference. In \Appref{app:eff}, we show that VT scales quadratically, while LSTM and ST2 scale linearly, with better slopes for ST2. Additionally, ST2(16) achieves a 12x faster training time per batch normalized by parameters compared to VT, meaning an ST2(16) model with 12 times more parameters can be trained in the same time as VT. However, for consistency in reporting, we compare models with approximately the same number of parameters (\( \sim 10K \) in this section).


\begin{figure}[t] 
    \centering
    \begin{subfigure}[t]{0.49\textwidth}
        \centering
        \includegraphics[width=\textwidth]{images/per_column_median.pdf} 
        \caption{\texttt{Median} prediction}
        \label{fig:subfig_b}
    \end{subfigure}
    \vspace{0.5cm} 
    \begin{subfigure}[t]{0.49\textwidth}
        \centering
        \includegraphics[width=\textwidth]{images/correlation.pdf}
        \caption{\texttt{Correlation} prediction}
        \label{fig:subfig_c}
    \end{subfigure}
    \hfill
    \vspace{-0.5cm}
    \caption{\textbf{Generalization across dataset lengths and meta-distributions.} The left panel shows MSE as a function of dataset length for in-meta-distribution datasets, while the right panel displays the same for out-of-meta-distribution datasets. The vertical red line marks the largest dataset seen during training ($n = 300$). LSTM is excluded due to its errors being an order of magnitude higher. Additional tasks can be found in \Appref{app:gen_plots}.}
    \label{fig:generalization}
\end{figure}

\xhdr{In-meta-distribution performance}  
\Tabref{tab:performance_comparison} shows the MSE of the four meta-statistical models on a test set sampled from the same meta-distribution as the training data. All models approximate the descriptive tasks well, but the LSTM-based model, lacking permutation invariance, performs worse than attention-based models. 
Notably, ST2, despite being much faster than VT, narrowly outperforms it. Given its strength and efficiency, ST2(16) is our main model in the rest of the paper, with VT considered as an alternative baseline.

\xhdr{Generalization Performance}  
We evaluate meta-statistical models' generalization capabilities on two aspects:  (i) \textbf{Out-of-Meta-Distribution (OoMD)}: Datasets from unseen distributions. (ii) \textbf{Length Generalization}: Datasets with lengths outside the training range. \Figref{fig:generalization} shows strong length generalization, where models maintain their performance for larger datasets than seen during training, both IMD and OoMD. 
They are also robust to OoMD datasets despite a small performance degradation. Manual inspection reveals that the degradation mainly comes from cases where the magnitude of the input values exceeds the range seen during training. This is discussed further in \Secref{sec:discussion}.
Additional results and generalization plots are provided in \Appref{app:desc_details}.


\begin{table}[t]
\centering
\resizebox{0.95\columnwidth}{!}{
\begin{tabular}{@{}l|c|c|c|c@{}}
\toprule
& \textbf{Median} & \textbf{Corr} & \textbf{WinRate (BT)} & \textbf{OT (1D)} \\ 
\midrule
\midrule
LSTM
& $2.9e^{-1}$ \scriptsize{$\pm 0.8$}
& $5.9e^{-2}$ \scriptsize{$\pm 1.5$}
& $4.4e^{-2}$ \scriptsize{$\pm 0.9$}
& $8.5e^{-2}$ \scriptsize{$\pm 2.9$}\\
VT
& $\mathbf{6.0e^{-2}}$ \scriptsize{$\pm 1.9$}
& $\mathbf{9.2e^{-3}}$ \scriptsize{$\pm 4.6$}
& $7.1e^{-3}$ \scriptsize{$\pm 1.5$}
& $\mathbf{5.5e^{-2}}$ \scriptsize{$\pm 1.4$}\\
ST2(16)
& $\mathbf{4.2e^{-2}}$ \scriptsize{$\pm 1.7$}
& $\mathbf{7.5e^{-3}}$ \scriptsize{$\pm 2.8$}
& $\mathbf{2.9e^{-3}}$ \scriptsize{$\pm 1.2$}
& $\mathbf{4.5e^{-2}}$ \scriptsize{$\pm 1.9$}\\
ST2(32)
& $\mathbf{4.4e^{-2}}$ \scriptsize{$\pm 0.9$}
& $\mathbf{9.1e^{-3}}$ \scriptsize{$\pm 5.1$}
& $\mathbf{1.6e^{-2}}$ \scriptsize{$\pm 0.5$}
& $\mathbf{3.0e^{-2}}$ \scriptsize{$\pm 1.5$}\\
\bottomrule

\end{tabular}
}
\caption{Performance comparison meta-statistical models across tasks, measured by Mean Squared Error with respect to correct output on the test set. \textbf{Bold} indicates no significant difference with the best model.}
\label{tab:performance_comparison}
\end{table}


\section{Experiments on Inferential Tasks}
\label{sec:experiments_inf}
In inferential tasks, the label \( y \) represents a property \( g \) of the underlying distribution \( P_X \) from which a dataset \( \mathcal{D} \) is sampled: \( y = g(P_X) \). We illustrate the meta-statistical framework with three such tasks: standard deviation estimation, normality testing, and mutual information estimation. Details on meta-dataset creation and models are in \Appref{app:std}. For all tasks in this section, the dataset sizes during training are sampled from $n \in [5, 150]$, depicted by vertical red lines in the plots. 

\subsection{Standard Deviation Estimation}  
The standard deviation (\( \sigma = \sqrt{\mathbb{E}[(X - \mathbb{E}[X])^2]}\)) quantifies the spread of a distribution \( P_X \). Unlike the mean or variance, estimating \( \sigma \) is non-trivial due to the square root's non-linearity \cite{gurland1971simple,gupta1952estimation}. In fact, no universal unbiased estimator exists across all distributions \cite{gurland1971simple,fenstad1980robust}. We use this task to show meta-statistical learning in action.  


\xhdr{Meta-Dataset} To create the meta-dataset, we follow the procedure outlined in \Secref{sec:experimental_setup}, keeping different distribution families for in- and out-of-meta-distribution. We use 100K meta datapoints for training.

\begin{figure}
    \centering
    \includegraphics[width=0.85\linewidth]{images/std_line_plot.pdf}
    \caption{MSE of $\sigma$ estimators as a function of dataset sizes, for dataset sampled \textbf{out-of-meta-distribution}.}
    \label{fig:std_line_plot}
\end{figure}


\xhdr{Meta-Statistical Model}  
We train two ST2-based models: ST2$_{\text{std}}$, which predicts the standard deviation \( \sigma \), and ST2$_{\text{fsd}}$, which estimates the finite sample correction to apply to the sample standard deviation, defined as \( y = \sigma - \text{np.std}(X) \). This allows constructing a corrected estimator by adjusting \texttt{np.std} with ST2$_{\text{fsd}}$'s predictions. Both models share the same architecture: 16 inducing points, five hidden layers (128 dimensions), and 12 attention heads per layer, totaling around 950K parameters.  


\xhdr{Results} 
We compare the ST2-based estimator to the sample standard deviation (\texttt{np.std} with Bessel’s correction) across dataset lengths for out-of-meta-distribution scenarios in \Figref{fig:std_line_plot}. ST2$_{\text{std}}$ achieves strong MSE performance, converging to a low error in high-sample sizes. Also, the learned correction from ST2$_{\text{fsd}}$ further reduces the bias of \texttt{np.std}, effectively capturing finite sample errors. Notably, ST2$_{\text{fsd}}$ also lowers variance of \texttt{np.std}, suggesting the correction is data-dependent rather than a fixed offset. 
Full tables of bias, variance, and MSE across distributions and dataset lengths are provided in \Appref{app:std}, confirming these observations.


\subsection{Normality Testing}
The task is now to determine whether a dataset \( \mathcal{D} \sim P_X \) originates from a normal distribution, formulated as a binary classification task: \( y = 1 \) if \( P_X \) is normal, \( y = 0 \) otherwise. Normality testing is crucial in hypothesis testing, model selection, and preprocessing \cite{shapiro1965analysis, razali2011power}, particularly before applying t-tests, linear regression, or ANOVA with small samples \cite{altman1990practical, das2016brief, doi:10.4078/jrd.2019.26.1.5}. However, standard tests struggle in low-sample settings \cite{razali2011power}. We propose to train meta-statistical models for normality classification, aiming for robust generalization in such regimes. Details on meta-dataset creation and model properties are in \Appref{app:norm_test}.

\begin{figure}[t]
    \centering
    \includegraphics[width=0.85\linewidth]{images/norm_test_oomd.pdf}
    \caption{Accuracy of normality classifiers as a function of dataset sizes. The non-normal distributions are sampled \textbf{out-of-meta distribution} for meta-statistical models.}
    \label{fig:norm_t_oomd}
    \vspace{-0.5cm}
\end{figure}


\xhdr{Meta-dataset creation}  
We construct a balanced meta-dataset of normally and not normally distributed datasets following the process described in \Secref{sec:experimental_setup}. For non-normality, we choose an alternative distribution from a predefined set detailed in \Appref{app:norm_test}. We use 40K meta datapoints for training. 

\xhdr{Estimators}  
We transform traditional normality tests into binary classifiers by thresholding their \( p \)-values, optimizing the threshold on the training meta-dataset for maximum classification accuracy. We consider four widely used tests: the \textit{Shapiro-Wilk test} \cite{shapiro1965analysis}, known to be effective for small samples \cite{razali2011power}; the \textit{D'Agostino-Pearson test} \cite{d1973tests}, which combines skewness and kurtosis; the \textit{Kolmogorov-Smirnov test} \cite{massey1951kolmogorov}, a non-parametric test based on cumulative distribution differences; and the \textit{Jarque-Bera test} \cite{jarque1987test}, which assesses skewness and kurtosis deviations from theoretical expectations.

We then train two meta-statistical models: one based on VT and another on ST2 with 16 inducing points. Both use four layers, a hidden dimensionality of 32, and 12 attention heads. The classification head is a single-layer MLP with 32 neurons, totaling approximately 50K parameters per model.

\begin{table}[h]
\centering
\begin{tabular}{l|cccc}
\toprule
 & Accuracy $\uparrow$ & AuROC $\uparrow$ & Brier $\downarrow$ & BT $\uparrow$ \\
\midrule
KS & $0.88$ {\scriptsize± $0.01$} & $0.93$ {\scriptsize± $0.01$} & $0.09$ {\scriptsize± $0.01$} & $0.12$ {\scriptsize± $0.02$} \\
SW & $0.89$ {\scriptsize± $0.01$} & $0.95$ {\scriptsize± $0.01$} & $0.18$ {\scriptsize± $0.01$} & $0.16$ {\scriptsize± $0.03$} \\
JB & $0.88$ {\scriptsize± $0.01$} & $0.93$ {\scriptsize± $0.01$} & $0.16$ {\scriptsize± $0.01$} & $0.13$ {\scriptsize± $0.02$} \\
AP & $0.90$ {\scriptsize± $0.01$} & $0.95$ {\scriptsize± $0.01$} & $0.18$ {\scriptsize± $0.01$} & $0.17$ {\scriptsize± $0.03$} \\
ST2 & $\mathbf{0.92}$ {\scriptsize± $0.01$} & $\mathbf{0.97}$ {\scriptsize± $0.01$} & $\mathbf{0.06}$ {\scriptsize± $0.01$} & $\mathbf{0.25}$ {\scriptsize± $0.04$} \\
VT & $0.91$ {\scriptsize± $0.01$} & $\mathbf{0.97}$ {\scriptsize± $0.01$} & $\mathbf{0.07}$ {\scriptsize± $0.01$} & $0.17$ {\scriptsize± $0.03$} \\
\bottomrule
\end{tabular}
\caption{Normality test classifiers with datasets drawn from Gaussian or Uniform distributions of sizes $n \in [10, 300]$. AuROC refers to the area under the ROC curve, Brier loss is the calibration error, and BT measures the relative strengths of classifiers in a paired evaluation.}
\label{tab:normality_test}
\end{table}


\xhdr{Results}  
\Figref{fig:norm_t_oomd} summarizes the accuracy of the proposed meta-statistical models, in settings where negative labels correspond to datasets sampled from distribution families unseen during training. Consistent with prior comparisons of normality tests, Shapiro-Wilk and D'Agostino-Pearson perform best among the baselines \cite{razali2011power}. Across all dataset sizes, meta-statistical models consistently and largely outperform baselines, with particularly strong gains in small-sample settings (\( n < 100 \)), making them highly relevant for biomedical applications \cite{doi:10.4078/jrd.2019.26.1.5}. Meta-statistical models achieve near-perfect accuracy (\( > 0.98 \)) as \( n \) increases demonstrating their consistency. Overall, this task seems relatively easy for meta-statistical models, which generalize smoothly out-of-meta distribution. However, note that the training of meta-statistical models could be harder if the input datasets are standardized during training (see \Secref{sec:discussion}). 

While classification lacks a direct bias-variance formulation, we analyze false positive and false negative rates as well as precision and recall in \Appref{app:prec_rec}, showing more balanced error profiles for meta-statistical estimators. In \Tabref{tab:normality_test}, we present key metrics for evaluating classifier performance: the Area Under the Receiver Operating Characteristic Curve (AuROC), the Brier Score, and the Bradley-Terry (BT) scores from a paired evaluation.
The AuROC measures a classifier's ability to discriminate between positive and negative classes across different decision thresholds. A higher AuROC indicates better separability. Unlike accuracy, AuROC provides a threshold-independent measure of performance. The Brier loss \cite{brier1950verification} quantifies the calibration of a model’s predicted probabilities. Lower values indicate better calibration. The Bradley-Terry (BT) score \cite{bradley1952rank,NIPS2004_825f9cd5} ranks models based on pairwise comparisons, assessing how often one classifier outperforms another across test instances \cite{peyrard-etal-2021-better,colombo-etal-2023-glass}. The accuracy scores are lower than those of \Figref{fig:norm_t_oomd} because the uniform is among the hardest out-of-meta-distribution to recognize as non-Gaussian. Still, across metrics, the meta-statistical estimators perform strongly. In particular, we find it interesting that they are particularly well-calibrated.


\subsection{Mutual Information Estimation}
Mutual information (MI) quantifies the dependency between two random variables \(X\) and \(Y\) and is defined as:
\[
\mathrm{MI}(X; Y) = \int \int P_{X,Y}(x, y) \log \frac{P_{X,Y}(x, y)}{P_X(x) P_Y(y)} \, dx \, dy.
\]
Here, \(P_X\) and \(P_Y\) denote the marginal distributions of \(X\) and \(Y\), respectively.

MI possesses key properties such as invariance to homeomorphisms and adherence to the \textit{Data Processing Inequality}, making it fundamental in machine learning and related fields \cite{inv_bottleneck, pmlr-v80-belghazi18a, repr_learning, tishby2000informationbottleneckmethod}. However, MI estimation remains challenging, particularly for small sample sizes and non-Gaussian distributions \cite{song2020understandinglimitationsvariationalmutual, pmlr-v108-mcallester20a, NEURIPS2023_36b80eae}. 

We adopt a meta-statistical approach, training models to predict \( y = \mathrm{MI}(X; Y) \) between two dataset columns. Focusing on low-sample, non-Gaussian settings, but we restrict experiments to the one-dimensional case for simplicity. Details on meta-dataset creation, models, and extra results are provided in \Appref{app:mi_details}.


\xhdr{Meta-dataset Creation}
We construct a meta-dataset inspired by the benchmark methodology in \cite{NEURIPS2023_36b80eae}, where distributions with ground-truth MI are generated in two steps: (i) by sampling a distribution with known MI, (ii) optionally applying MI-preserving transformations. This process creates complex distributions and datasets with known MI. For generating meta-dataset in this way, we again follow the process described in \Secref{sec:experimental_setup} using different base-distribution and MI-preserving transformations between in-meta-distribution and out-of-meta-distribution. We use 50K meta datapoints for training.


\xhdr{Estimators}
We compare our approach with the best-performing 1D estimators from \cite{NEURIPS2023_36b80eae}, including Kraskov-St\"ogbauer-Grassberger (KSG) \cite{PhysRevE.69.066138}, Canonical Correlation Analysis (CCA) \cite{pml2Book}, and three neural estimators: MINE \cite{pmlr-v80-belghazi18a}, InfoNCE \cite{repr_learning}, and NWJE \cite{NIPS2007_72da7fd6, NIPS2016_cedebb6e, pmlr-v97-poole19a}. We train two meta-statistical models: one based on Vanilla Transformer (VT) and the other on Set Transformer 2 (ST2). Both models consist of five layers, with a hidden dimensionality of 256 and 12 attention heads. The regression head is a single hidden-layer MLP with 128 neurons, resulting in models with approximately 1M parameters.


\xhdr{Estimation Performance}
The mean squared error (MSE) results for both in- and out-of-meta-distribution testing are shown in \Tabref{tab:mi_test}. Meta-statistical models outperform baseline estimators across all sample sizes, with significant advantages in low-sample scenarios. Baseline models, particularly neural ones, struggle with small sample sizes, while only KSG and CCA begin to match meta-statistical models for sample sizes greater than 100 in the out-of-meta-distribution regime.

\xhdr{Bias and Variance of MI Estimators}
We examine the bias and variance of MI estimators by resampling datasets from fixed distributions and measuring the variance and bias of the estimates. In \Figref{fig:mi_bias_var}, we visualize the bias and variance for a challenging distribution identified by previous works \cite{NEURIPS2023_36b80eae} (additive noise). Even at a sample size of $n = 100$, meta-statistical models show clear improvements in both bias (estimates centered around 0) and variance. A more detailed analysis of bias and variance is available in \Appref{app:mi_details} (\Tabref{tab:mi_bias_variance}). Compared to baseline estimators, meta-statistical models demonstrate significantly lower bias, close to zero, and lower or comparable variance. These results are promising, suggesting that further scaling could create even more robust meta-statistical MI estimators. Currently, the ST2 model can be trained in less than an hour on a single GPU, with inference orders of magnitude faster than existing neural baselines.


\begin{figure}
    \centering
    \includegraphics[width=0.9\linewidth]{images/additive-noise.pdf}
    \caption{We estimate statistics for MI estimators over 150 resampled datasets of size \( n = 100 \) from a fixed distribution (additive noise \cite{NEURIPS2023_36b80eae}). Each dot represents the difference between an estimate and the true mutual information (MI).}
    \label{fig:mi_bias_var}
\end{figure}

\begin{table}[t]
\centering
\resizebox{0.95\columnwidth}{!}{
\begin{tabular}{@{}l|cc|cc@{}}
\toprule
& \multicolumn{2}{c|}{IMD} & \multicolumn{2}{c}{OoMD} \\ 
% \cmidrule(lr){2-4} \cmidrule(lr){5-7}
$n \in $ & $[10, 100]$ & $[100, 200]$ & $[10, 100]$ & $[100, 200]$ \\ 
\midrule
\midrule
CCA
& $7.4e^{-2}$ \scriptsize{$\pm 9.3$}
& $1.4e^{-2}$ \scriptsize{$\pm 1.1$}
% ----
& $1.3e^{-1}$ \scriptsize{$\pm 1.2$}
& $4.9e^{-2}$ \scriptsize{$\pm 3.3$}\\
KSG
& $2.9e^{-2}$ \scriptsize{$\pm 1.5$}
& $7.8e^{-3}$ \scriptsize{$\pm 3.5$}
% ----
& $\mathbf{1.2e^{-2}}$ \scriptsize{$\pm 0.4$}
& $\mathbf{7.2e^{-3}}$ \scriptsize{$\pm 2.3$}\\
MINE
& $2.5e^{0}$ \scriptsize{$\pm 2.4$}
& $2.8e^{-2}$ \scriptsize{$\pm 1.2$}
% ---
& $5.4e^{0}$ \scriptsize{$\pm 7.7$}
& $1.6e^{-1}$ \scriptsize{$\pm 2.1$}\\
NWJE
& -
& $7.3e^{-2}$ \scriptsize{$\pm 4.7$}
% --
& $6.6e^{0}$ \scriptsize{$\pm 7.5$}
& $6.3e^{-2}$ \scriptsize{$\pm 5.4$}\\
InfoNCE
& $1.5e^{1}$ \scriptsize{$\pm 2.2$}
& $1.9e^{-2}$ \scriptsize{$\pm 0.7$}
% ---
& $2.3e^{1}$ \scriptsize{$\pm 3.4$}
& $3.4e^{-1}$ \scriptsize{$\pm 4.5$}\\
\midrule
VT
& $\mathbf{4.6e^{-3}}$ \scriptsize{$\pm 2.4$}
& $\mathbf{2.5e^{-3}}$ \scriptsize{$\pm 1.3$}
% ---
& $\mathbf{1.5e^{-2}}$ \scriptsize{$\pm 0.8$}
& $\mathbf{7.7e^{-3}}$ \scriptsize{$\pm 3.2$}\\
ST2(16)
& $\mathbf{6.2e^{-3}}$ \scriptsize{$\pm 3.0$}
& $\mathbf{2.4e^{-3}}$ \scriptsize{$\pm 1.1$}
% ---
& $\mathbf{1.3e^{-2}}$ \scriptsize{$\pm 0.7$}
& $\mathbf{8.5e^{-3}}$ \scriptsize{$\pm 3.1$}\\
\bottomrule
\end{tabular}
}
\caption{MSE loss of mutual information estimators both in- and out-of-meta-distribution. \textbf{Bold} indicates no significant difference with the best estimator.}
\label{tab:mi_test}
\end{table}

\section{Discussion}
\label{sec:discussion}
\section{Discussion}

In this paper, we explored the relationship between human evaluations and NLP benchmarks of chat-finetuned language models (chat LMs). Our work is motivated by the recent shift towards human evaluations as the primary means of assessing chat LM performance, and the desire to determine the role that NLP benchmarks should play.

Through a large-scale study of the Chat Llama 2 model family on a diverse set of human and NLP evaluations, we demonstrated that NLP benchmarks are generally well-correlated with human judgments of chat LM quality. However, our analysis also reveals some notable exceptions to this overall trend. In particular, we find that adversarial and safety-focused evaluations, as well as language assistance and open question answering tasks, exhibit weaker or negative correlations respectively with NLP benchmarks. We also explored predicting human evaluation scores from NLP evaluation scores using overparameterized linear regression models. Our results suggest that NLP benchmarks can indeed be used to predict aggregate human preferences, although we caution that the limited sample size and the assumptions of our models may limit the generalizability of these findings. Our results suggest that NLP benchmarks can serve as fast and cheap proxies of slower and expensive human evaluations in assessing chat LMs.

Additionally, our work highlights the need for further research into NLP evaluations that can effectively capture important aspects of LM behavior, such as safety, robustness to adversarial inputs, and performance on complex, open-ended tasks. It is possible that new NLP benchmarks can provide signals on these topics, e.g., \citep{wang2023decodingtrust}. Of particular interest is developing human-interpretable and scaling-predictable evaluation processes, e.g., \citep{schaeffer2024emergent, ruan2024observational,schaeffer2024predictingdownstreamcapabilitiesfrontier}. Developing and refining such evaluation methods \citep{madaan2024quantifyingvarianceevaluationbenchmarks}, as well as detecting whether evaluations scores faithfully capture models' true performance \citep{oren2023proving,schaeffer2023pretrainingtestsetneed,roberts2023cutoff,jiang2024investigatingdatacontaminationpretraining,zhang2024careful,duan2024uncoveringlatentmemoriesassessing} will be crucial for ensuring that LMs are safe, reliable, and beneficial as they become increasingly integrated into society.

% In conclusion, our study provides insights into the relationship between human evaluations and NLP benchmarks of chat language models. By leveraging the complementary strengths of both human and NLP benchmarks, we can build a more complete understanding of LM capabilities and behaviors, ultimately enabling the development of models more capable, trustworthy, and beneficial to society.



% \section*{Acknowledgments}

\section*{Impact Statements}
This paper presents work whose goal is to advance the field of Machine Learning. Meta-statistical learning aims to enhance inference in low-sample settings, benefiting applied fields of Science like medicine and economics by improving estimator reliability. Learned estimators may inherit biases from the data they are trained on, potentially leading to misleading conclusions if not carefully validated. Further, as with any data-driven methodology, interpretability remains a challenge; understanding why a model makes a particular statistical inference is crucial for scientific rigor. 

% This paper presents work whose goal is to advance the field of Machine Learning. There are many potential societal consequences of our work, none which we feel must be specifically highlighted here.


% \clearpage



% \section*{Glossary of Important Terms}
% \label{sec:glossary}
% \input{__glossary}


% Entries for the entire Anthology, followed by custom entries
% \clearpage
\bibliography{anthology,main}
\bibliographystyle{acl_natbib}

% 
\appendix
\clearpage

\subsection{Additional details for Theorem~1}\label{app:decomp_lattice_proof}
We provide details that were omitted in the proof of Theorem~1. First, we derive the matrix $P$.
Given the planes $H$ and  $H_0:=\{x_{d+1}=0\}$, we wish  to find a plane $H_{ref}$ that is half-way (angle-wise) between $H$ and $H_0$. This would allow to reflect points in $H$ onto $H_0$ through $H_{ref}$ where the reflection is achieved using the Householder matrix $P:=I-2\hat{n}_{ref}\hat{n}^t_{ref}$, where $\hat{n}_{ref}\in \dR^{d+1}$ is the normal of $H_{ref}$~\cite{householder1958unitary}. That is, we reflect a lattice point $p\in \dR^{d+1}$ by computing the value \({p_{\text{reflected}}=P\cdot p}\).

Next, we show that the normal 
\begin{equation*}
    \hat{n}_{ref}:=\frac{1}{\sqrt{2-\frac{2}{\sqrt{d+1}}}}\cdot \left(-\tfrac{1}{\sqrt{d+1}},\dots,-\tfrac{1}{\sqrt{d+1}},1-\tfrac{1}{\sqrt{d+1}}\right)
\end{equation*}
satisfies those requirements.\footnote{We obtained the expression for $\hat{n}_{ref}$ by first considering $d=2$, where the task is more tangible, and then generalizing to higher dimensions.}  Consider the Householder matrix
\begin{align}\label{eq:reflection}
 P&=% I-2\hat{n}\hat{n}^t= 
 I-2\hat{n}_{ref}\hat{n}^t_{ref}\nonumber\\
 &=\begin{pNiceArray}{cw{c}{1cm}c|c}[margin]
            \Block{3-3}<\Large>{I_d - \frac{1}{D-\sqrt{D}}\mathds{1}} 
            & & & \dfrac{1}{\sqrt{D}} \\
            & & & \Vdots \\
            & & & \dfrac{1}{\sqrt{D}} \\
            \hline
            \dfrac{1}{\sqrt{D}} & \dots& \dfrac{1}{\sqrt{D}} & \dfrac{1}{\sqrt{D}}
        \end{pNiceArray},
    \end{align}
where $D:=d+1$, $I_d$ is an $d\times d$ identity matrix, and $\mathds{1}$ is the $d\times d$ matrix with $1$s in all its entries. 

Consider a point $p\in A^*_d$. Next, we show that it is reflected onto the plane  $H_0$, i.e., for $v=P\cdot p$, we get $v_{d+1}=0$. To do that, we move to the basis of the integer vector space, and show that for all $1\leq i\leq d$, taking the base element $e_i=(0,\dots,1,\dots,0)$, the $(d+1)$th element of $v=PG^t\cdot e_i$ (i.e., using the generator and then the reflector) is zero. First, for all $i<d$ it holds that 
    \begin{align*}
        PG^t\cdot e_i=P\cdot
        \begin{pmatrix}
        1 &  0&  \dots&  0& -1& 0 &\dots& 0
            % 1 \\
            % 0 \\
            % \vdots \\
            % 0 \\
            % -1 \\
            % 0 \\
            % \vdots \\
            % 0
        \end{pmatrix}^t.
    \end{align*}
Now, considering that the elements of the final row of $P$ are all equal to $1/\sqrt{D}$, we obtain a zero in the $(d+1)$th dimension. It remains to calculate the expression resulting from multiplying with $e_d$:
        \begin{align*}
        PG^t\cdot e_d=P\cdot
        \begin{pmatrix}
        -\frac{D-1}{D} &  \frac{1}{D}&  \dots&  \frac{1}{D}
        \end{pmatrix}^t.
    \end{align*}
    Looking specifically at the last element, we see that it is equal to     \begin{align*}
        \frac{1}{\sqrt{D}}\cdot\frac{1-D}{D} + (D-1)\frac{1}{D}\frac{1}{\sqrt{D}}=\frac{1-D+D-1}{D\sqrt{D}}=0.
    \end{align*}

    That is, by applying the transformation $P$ on the lattice points, we reflect them onto the $x_{d+1}=0$ plane. It remains to get rid of the $(d+1)$th dimension. This is accomplished by the mapping
\begin{align*}
        E=
        \begin{pmatrix}
            1 & 0 & \dots & 0 & 0 \\
            0 & 1 & \dots & 0 & 0 \\
            \vdots & \vdots & \ddots & \vdots & 0 \\
            0 & 0 & \dots & 1 & 0
        \end{pmatrix}_{d\times(d+1)}.
    \end{align*}
    
It remains to compute the  explicit embedding $T(g):=EPG^t(g)$, for $g\in \dZ^d$. We first calculate 
    \begin{align*}
        \left(EP\right)^t=
        \begin{pNiceArray}{cw{c}{1cm}c}[margin]
            \Block{3-3}<\Large>{I_d - \frac{1}{D-\sqrt{D}}\mathds{1}} 
            & &  \\
            & &  \\
            & &  \\
            \hline
            \dfrac{1}{\sqrt{D}} & \dots & \dfrac{1}{\sqrt{D}}
        \end{pNiceArray}_{d\times(d+1)}.
    \end{align*}
 Next, it can be shown that
    \begin{align}
        T^t&=G\left(EP\right)^t\nonumber\\
        &=\begin{pmatrix}
            1 & -1 &  0  & \dots & 0 \\
            1 & 0  &  -1 & \dots & 0 \\
            \vdots & \vdots  &  \vdots  & \ddots & \vdots \\
            1 & 0  &  0  & \dots & -1 \\
            \frac{1}{D - \sqrt{D}} - 1 & \frac{1}{D - \sqrt{D}} & \frac{1}{D - \sqrt{D}} & \dots & \frac{1}{D - \sqrt{D}}
        \end{pmatrix}_{d\times(d+1)}\!\!\!\!\!\!.
    \end{align}

    %      \begin{figure}[H]
    %     \centering
    %     \begin{subfigure}[b]{0.3\textwidth}
    %         \includegraphics[width=\textwidth]{Images/EPGt_visual_explanation1.png}
    %         %\caption{Sample points in $H_0$}
    %         %\label{fig:epgt_visual1}
    %     \end{subfigure}
    %     \hfill
    %     \begin{subfigure}[b]{0.3\textwidth}
    %         \includegraphics[width=\textwidth]{Images/EPGt_visual_explanation2.png}
    %         %\caption{$H_0$ rotated to the "floor"}
    %         %\label{fig:epgt_visual2}
    %     \end{subfigure}
    %     \hfill
    %     \begin{subfigure}[b]{0.3\textwidth}
    %         \includegraphics[width=\textwidth]{Images/EPGt_visual_explanation3.png}
    %         %\caption{The samples as they look in $\dR^2$}
    %         %\label{fig:epgt_visual3}
    %     \end{subfigure}
    %     \caption{Visualization of embedding the lattice  $A_2^*$ originally defined in $\dR^3$ onto $\dR^2$ via the mapping $T$. [Left] The blue rectangle represents the plane $H$, where the corresponding $A_2^*$ lattice points are drawn in red. The points are generated by taking integer vectors in $\dR^d$ and applying the mapping $G^t$.  [Center] $H$ and $A_2^*$ is reflected onto the plane $H_0=\{x_3=0\}$ using the mapping $PG^t$. [Right] The third dimension is removed, via the mapping $E$, to yield the embedding of $A_2^*$ in $\dR^2$.}
    %     \label{fig:egpt_visual}
    % \end{figure}


\subsection{Additional details for Theorem~3}\label{app:CC}
We provide details omitted from the main body of the text. 
We start with a simplified derivation of a single annulus, which would inform the more advanced construction. Fix ${0<r_1<
  r^*}$ forced it to a single line, and define $\btheta_{r'} := \frac{r'}{{\beta^*}}f_\Lambda$, and observe that 
\begin{align}
  CC_\X&\leq  r^*\cdot
\left|\X\cap (\B_{r^*}\setminus \B_{r_1})\right| + r_1\cdot
\left|\X\cap \B_{r_1}\right|\nonumber \\ & = r^*\left(|\X\cap \B_{r^*}|-|\X \cap \B_{r_1}|\right) + r_1 \left|\X\cap \B_{r_1}\right| \nonumber\\
& = r^*|\X\cap \B_{r^*}|+ (r_1-r^*) |\X\cap \B_{r_1}| \nonumber \\  
  & = r^*\frac{\partial(B_1)}{\sqrt{\det(\Lambda)}}\btheta^d_{r^*} +r^* P_d(\btheta_{r^*}) + (r_1-r^*) \frac{\partial(B_1)}{\sqrt{\det(\Lambda)}}\btheta^d_{r_1}\nonumber\\& + (r_1-r^*) P_d(\btheta_{r_1}) \nonumber
\\
& = \frac{\partial(B_1)}{\sqrt{\det(\Lambda)}}\theta^d\left({r^*}^{d+1}+{r_1}^{d+1}-r^*{r_1}^{d}\right)\nonumber\\&+ r P_d(\btheta_{r^*}) + (r_1-r^*) P_d(\btheta_{r_1})\nonumber \\ & = \frac{\partial(B_1)}{\sqrt{\det(\Lambda)}}\theta^d\left({r^*}^{d+1}+{r_1}^{d+1}-r^*{r_1}^{d}\right)+ r^* P_d(\btheta_{r^*}), \label{eq:CC1}
\end{align}
where the sample complexity bound in Equation~(5) is used. For simplicity, we bound throughout the error term with $r^* P_d(\btheta_{r^*})$.
Next, we optimize the value $r_1$ to minimize the expression in Equation~\eqref{eq:CC1}.

Consider the function $f(r_1)={r^*}^{d+1}-{r^*} r^d_1 + {r^*}^{d+1}_1$. We look for the minimum of $f(r_1)$ by requiring that
\begin{align*}
            f'(r_1)=-{r^*} dr_1^{d-1}+(d+1)r_1^d=0,
\end{align*}
which yields the value $r'_1:=\frac{d}{d+1}{r^*}$. This value is  a minimum since
\begin{align*}
 f^{(2)}(r_1)|_{r'_1}=&\left(-{r^*}(d-1)r_1^{d-2}+d(d+1)r_1^{d-1}\right)|_{r_1'}\\
        =&{r^*}^{d-1}\left(\frac{d^d}{(d+1)^{d-2}}-\frac{d^{d-2}(d-1)}{(d+1)^{d-2}}\right)\\
        =&{r^*}^{d-1}d^{d-2}\frac{d^2-d+1}{(d+1)^{d-2}},
    \end{align*}
    and we know that $d^2-d+1>0$ for all $d\geq 2$.%, then $r_1=\frac{d}{d+1}r$ indeed minimizes $f(r_1)$.
    
Now, we apply the above line of reasoning in a recursive manner by considering a sequence of $k+1\geq 2$ radii ${0<r_k<\ldots<r_0={r^*}}$ where $r_i:=\td^i r^*$, where $\td:=\frac{d}{d+1}$. This leads to the bound
\begin{align}
\label{eq:cc_eval_app}
CC_\X&\leq \sum_{i=0}^{k-1}r_i |\X\cap (\B_{r_i}\setminus \B_{r_{i+1}})| + r_k|\X\cap \B_{r_k}|\nonumber\\
  &= \frac{\partial(B_1)}{\sqrt{\det(\Lambda)}} \left(\underbrace{{r^*} \btheta^d_{r^*} + \sum_{i=1}^k(r_i-r_{i-1}) \btheta^d_{r_i}}_{:=\gamma}\right) + {r^*} P_d(\btheta_{r^*}).
\end{align}

We show that 
\[\gamma:=r \btheta^d_{r^*} + \sum_{i=1}^k(r_i-r_{i-1}) \btheta^d_{r_i}= {r^*} \btheta^d_{r^*} \left(1 - \frac{\xi^{d+2} - \xi}{ d\xi - (d+1)}\right),\]
where $r_i=\td^i {r^*},\td:=\frac{d}{d+1}, \btheta_{r_i}= r_i\frac{\btheta_{r^*}}{r^*}, k=d,$ and $\xi:=\td^d=\left(\frac{d}{d+1}\right)^d$. In particular,
\begin{align}
  \gamma &={r^*} \btheta^d_{r^*} + \sum_{i=1}^k(r_i-r_{i-1}) r_i^d\frac{\btheta^d_{r^*}}{{r^*}^d} \nonumber\\
  & = {r^*} \btheta^d_{r^*} + \sum_{i=1}^k{r^*}\td^{i-1}(\td-1) \td^{di} {r^*}^d\frac{\btheta^d_{r^*}}{{r^*}^d} \nonumber\\ 
  &=  {r^*} \btheta^d_{r^*} + \sum_{i=1}^k{r^*} (\td-1) \td^{di+ i -1} \btheta^d_{r^*} \nonumber\\ 
  &=   {r^*} \btheta^d_{r^*} \left(1 + \sum_{i=1}^k (\td-1) \td^{di+ i -1} \right)\nonumber\\
  &= {r^*} \btheta^d_{r^*} \left(1 + \frac{\td-1}{\td}\sum_{i=1}^k \td^{(d+1)i} \right)\nonumber\\
  &= {r^*} \btheta^d_{r^*} \left(1 + \frac{\td-1}{\td}\frac{\left(\td^{d+1}\right)^{k+1} - \td^{d+1}}{\td^{d+1} - 1} \right)\nonumber\\
  &= {r^*} \btheta^d_{r^*} \left(1 + \td^d(\td-1)\frac{\left(\td^{d+1}\right)^k - 1}{\td^{d+1} - 1} \right).\nonumber\\
  % & = r \btheta^d_{r} + \sum_{i=1}^kr\td^{i-1}(\td-1) \td^{di} r^d\frac{\btheta^d_{r}}{r^d} \\ &  =  r \btheta^d_{r} + \sum_{i=1}^kr (\td-1) \td^{di+ i -1} \btheta^d_{r} \\ & =   r \btheta^d_{r} \left(1 + \sum_{i=1}^k (\td-1) \td^{di+ i -1} \right)\\
\end{align}

Taking $k=d$ results in $r_k=(\frac{d}{d+1})^d {r^*}\approx\frac{1}{e}{r^*}$. 
To use the sample set analysis, we need a large enough $r$ value, so assuming the original $r$ is large enough, we can deduce safely that $\frac{r}{e}$ is also large enough. 
Notice also that $\td - 1 = \frac{-1}{d+1}$, and thus $(d+1)\td=d$, so returning to our expression, and substituting $\xi:=\td^d=\left(\frac{d}{d+1}\right)^d$, we obtain 
\begin{align*}
    \gamma&={r^*} \btheta^d_{r^*} \left(1 - \frac{\td^d\left(\td^{d(d+1)} - 1\right)}{(d+1)(\td^{d+1} - 1)}\right)\\
    &= {r^*} \btheta^d_{r^*} \left(1 - \frac{\xi\left(\xi^{d+1} - 1\right)}{(d+1)(\td \xi - 1)}\right) 
    \\&= {r^*} \btheta^d_{r^*} \left(1 - \frac{\xi^{d+2} - \xi}{ d\xi - (d+1)}\right)
    :={r^*} \btheta^d_{r^*}\zeta.
\end{align*}

We finish this section with a plot of the value $\gamma$ in Figure~\ref{fig:annuli_bound:app}.

\begin{figure}[thb]
\centering  
\includegraphics[width=0.9\columnwidth]{Images/annuli_bound.pdf}
\caption{Plot of the improvement factor $\gamma$.}
\label{fig:annuli_bound:app}
\end{figure}

\subsection{Additional experimental results}
Additional scenarios, which were omitted from the main paper, are given in Figure~\ref{fig:scenarios:app}. Extended results comparing lattice-based samples using the \loc algorithm are provided in Table~\ref{tbl:lattice_comparison:app}.

\begin{figure*}[tbh]
  \centering
%     \hspace*{-0.66cm}
% \subfloat[Zigzag-bypass (long)]{\includegraphics[width=2.18\columnwidth,clip]{Images/Scenarios/ZZB3H_scenario.png}
%     %\label{fig:3d_lattices:da}
%     }
%     \newline
\subfloat[Zigzag-bypass (short)]{\includegraphics[width=1.15\columnwidth,clip]{Images/Scenarios/ZZB2H_scenario.png}
    %\label{fig:3d_lattices:da}
    }
\subfloat[Narrow (more scenarios)]{\includegraphics[width=0.465\columnwidth,clip]{Images/Scenarios/N1_scenarios.png}
    %\label{fig:3d_lattices:da}
    }
  \caption{Additional scenarios used in the experiments. The scenario ZZB3, which is not illustrated here, is similar to ZZB2, only that the horizontal hallways are twice as long.}
  \label{fig:scenarios:app}
\end{figure*}

\begin{table}[tbh]
\caption{Extended comparison of running time and solution length using lattices-based sample sets (where the underlying lattice is denoted in the table) in the iA*-\loc algorithm. Solution length is normalized with respect to the length obtained using $\XA$. }
\centering
\label{tbl:lattice_comparison:app}
\begin{tabular}{|c||ccc|cc|}
\hline
 & \multicolumn{3}{c|}{\cellcolor[HTML]{EFEFEF} Time (s)} & \multicolumn{2}{c|}{\cellcolor[HTML]{EFEFEF} Length (r)} \\ \cline{2-6} 
\multirow{-2}{*}{\begin{tabular}[c]{@{}c@{}}Scenario\\ (robot \#)\end{tabular}} & \multicolumn{1}{c|}{\cellcolor[HTML]{FFFFC7}$\ZN$} & \multicolumn{1}{c|}{\cellcolor[HTML]{FFFFC7}$\DN$} & \cellcolor[HTML]{FFFFC7}$\AN$ & \multicolumn{1}{c|}{\cellcolor[HTML]{FFFFC7}$\ZN$} & \cellcolor[HTML]{FFFFC7}$\DN$ \\ \hline \hline
\cellcolor[HTML]{ECF4FF}N4(2) & \multicolumn{1}{c|}{0.00} & \multicolumn{1}{c|}{0.00} & 0.00 & \multicolumn{1}{c|}{0.62} & 0.74 \\
\cellcolor[HTML]{ECF4FF}N1(5) & \multicolumn{1}{c|}{165.35} & \multicolumn{1}{c|}{4.59} & 0.36 & \multicolumn{1}{c|}{0.65} & 0.79 \\
\cellcolor[HTML]{ECF4FF}N2(5) & \multicolumn{1}{c|}{62.68} & \multicolumn{1}{c|}{1.81} & 0.41 & \multicolumn{1}{c|}{0.85} & 0.95 \\
\cellcolor[HTML]{ECF4FF}N3(5) & \multicolumn{1}{c|}{142.27} & \multicolumn{1}{c|}{2.91} & 0.59 & \multicolumn{1}{c|}{0.65} & 0.87 \\
\cellcolor[HTML]{ECF4FF}N5(5) & \multicolumn{1}{c|}{dnf} & \multicolumn{1}{c|}{4.82} & 3.32 & \multicolumn{1}{c|}{dnf} & 0.82 \\
\cellcolor[HTML]{ECF4FF}N1B(6) & \multicolumn{1}{c|}{dnf} & \multicolumn{1}{c|}{328.30} & 15.08 & \multicolumn{1}{c|}{dnf} & 0.89 \\ \hline
\cellcolor[HTML]{ECF4FF}BT4(2) & \multicolumn{1}{c|}{0.04} & \multicolumn{1}{c|}{0.01} & 0.01 & \multicolumn{1}{c|}{0.69} & 0.85 \\
\cellcolor[HTML]{ECF4FF}BT10(2) & \multicolumn{1}{c|}{-} & \multicolumn{1}{c|}{1.20} & 0.30 & \multicolumn{1}{c|}{-} & 0.92 \\
\cellcolor[HTML]{ECF4FF}BT5(3) & \multicolumn{1}{c|}{0.54} & \multicolumn{1}{c|}{0.14} & 0.06 & \multicolumn{1}{c|}{0.38} & 0.51 \\
\cellcolor[HTML]{ECF4FF}BT1(4) & \multicolumn{1}{c|}{146.69} & \multicolumn{1}{c|}{50.81} & 3.51 & \multicolumn{1}{c|}{0.95} & 1.03 \\
\cellcolor[HTML]{ECF4FF}BT6(4) & \multicolumn{1}{c|}{dnf} & \multicolumn{1}{c|}{153.40} & 12.36 & \multicolumn{1}{c|}{dnf} & 1.04 \\
\cellcolor[HTML]{ECF4FF}BT7(4) & \multicolumn{1}{c|}{240.88} & \multicolumn{1}{c|}{5.38} & 4.36 & \multicolumn{1}{c|}{0.95} & 0.96 \\ \hline
\cellcolor[HTML]{ECF4FF}K1(3) & \multicolumn{1}{c|}{32.31} & \multicolumn{1}{c|}{4.97} & 1.37 & \multicolumn{1}{c|}{0.82} & 0.89 \\ \hline
\cellcolor[HTML]{ECF4FF}UM4(2) & \multicolumn{1}{c|}{-} & \multicolumn{1}{c|}{8.47} & 2.43 & \multicolumn{1}{c|}{-} & 0.90 \\
\cellcolor[HTML]{ECF4FF}UM1(3) & \multicolumn{1}{c|}{482.17} & \multicolumn{1}{c|}{25.15} & 6.68 & \multicolumn{1}{c|}{0.84} & 1.16 \\
\cellcolor[HTML]{ECF4FF}UM2(3) & \multicolumn{1}{c|}{13.35} & \multicolumn{1}{c|}{1.22} & 0.04 & \multicolumn{1}{c|}{1.04} & 1.52 \\
\cellcolor[HTML]{ECF4FF}UM4B3(3) & \multicolumn{1}{c|}{99.35} & \multicolumn{1}{c|}{1.03} & 0.66 & \multicolumn{1}{c|}{1.59} & 0.89 \\
\cellcolor[HTML]{ECF4FF}UM3(4) & \multicolumn{1}{c|}{236.31} & \multicolumn{1}{c|}{223.87} & 64.57 & \multicolumn{1}{c|}{0.63} & 0.97 \\ \hline
\cellcolor[HTML]{ECF4FF}ZZB1(2) & \multicolumn{1}{c|}{1.93} & \multicolumn{1}{c|}{1.01} & 0.44 & \multicolumn{1}{c|}{0.94} & 0.94 \\
\cellcolor[HTML]{ECF4FF}ZZB2(2) & \multicolumn{1}{c|}{2.91} & \multicolumn{1}{c|}{0.93} & 0.71 & \multicolumn{1}{c|}{0.94} & 0.94 \\
\cellcolor[HTML]{ECF4FF}ZZB3(2) & \multicolumn{1}{c|}{2.26} & \multicolumn{1}{c|}{0.84} & 0.47 & \multicolumn{1}{c|}{0.95} & 0.95 \\ \hline\end{tabular}
\end{table}

\begin{table*}[tbh]
\centering
\begin{tabular}{|c|cccl|ccl|cl|cl|}
\hline
 & \multicolumn{4}{c|}{\cellcolor[HTML]{EFEFEF} Total time (s)} & \multicolumn{3}{c|}{\cellcolor[HTML]{EFEFEF} Search time (s)} & \multicolumn{2}{c|}{\cellcolor[HTML]{EFEFEF}Length (r)} & \multicolumn{2}{c|}{\cellcolor[HTML]{EFEFEF}Success (\%)} \\ \cline{2-12} 
\multirow{-2}{*}{\begin{tabular}[c]{@{}c@{}}Scenario\\ (Robot \#)\end{tabular}} & \multicolumn{1}{c|}{\cellcolor[HTML]{FFFFC7}\begin{tabular}[c]{@{}c@{}}$\AN$\\ \loc\end{tabular}} & \multicolumn{1}{c|}{\cellcolor[HTML]{FFFFC7}\begin{tabular}[c]{@{}c@{}}$\AN$\\ \glo\end{tabular}} & \multicolumn{1}{c|}{\cellcolor[HTML]{FFFFC7}\begin{tabular}[c]{@{}c@{}}\rnd\\ \glo\end{tabular}} & \multicolumn{1}{c|}{\cellcolor[HTML]{FFFFC7}\begin{tabular}[c]{@{}c@{}}\rndm\\ \glo\end{tabular}} & \multicolumn{1}{c|}{\cellcolor[HTML]{FFFFC7}\begin{tabular}[c]{@{}c@{}}$\AN$\\ \glo\end{tabular}} & \multicolumn{1}{c|}{\cellcolor[HTML]{FFFFC7}\begin{tabular}[c]{@{}c@{}}\rnd\\ \glo\end{tabular}} & \multicolumn{1}{c|}{\cellcolor[HTML]{FFFFC7}\begin{tabular}[c]{@{}c@{}}\rndm\\ \glo\end{tabular}} & \multicolumn{1}{c|}{\cellcolor[HTML]{FFFFC7}\begin{tabular}[c]{@{}c@{}}\rnd\\ \glo\end{tabular}} & \multicolumn{1}{c|}{\cellcolor[HTML]{FFFFC7}\begin{tabular}[c]{@{}c@{}}\rndm\\ \glo\end{tabular}} & \multicolumn{1}{c|}{\cellcolor[HTML]{FFFFC7}\begin{tabular}[c]{@{}c@{}}\rnd\\ \glo\end{tabular}} & \multicolumn{1}{c|}{\cellcolor[HTML]{FFFFC7}\begin{tabular}[c]{@{}c@{}}\rndm\\ \glo\end{tabular}} \\ \hline
\cellcolor[HTML]{ECF4FF}N1(5) & \multicolumn{1}{c|}{0.36} & \multicolumn{1}{c|}{3.05} & \multicolumn{1}{c|}{4.16} & 3.40 & \multicolumn{1}{c|}{0.84} & \multicolumn{1}{c|}{3.37} & 2.59 & \multicolumn{1}{c|}{1.48} & 1.46 & \multicolumn{1}{c|}{80.00} & 90 \\
\cellcolor[HTML]{ECF4FF}N2(5) & \multicolumn{1}{c|}{0.41} & \multicolumn{1}{c|}{2.67} & \multicolumn{1}{c|}{2.74} & 4.28 & \multicolumn{1}{c|}{0.82} & \multicolumn{1}{c|}{2.11} & 3.62 & \multicolumn{1}{c|}{2.43} & 3.31 & \multicolumn{1}{c|}{65.00} & 95 \\
\cellcolor[HTML]{ECF4FF}N3(5) & \multicolumn{1}{c|}{0.59} & \multicolumn{1}{c|}{3.83} & \multicolumn{1}{c|}{5.44} & 4.22 & \multicolumn{1}{c|}{1.72} & \multicolumn{1}{c|}{4.65} & 3.39 & \multicolumn{1}{c|}{2.02} & 1.56 & \multicolumn{1}{c|}{85.00} & 85 \\
\cellcolor[HTML]{ECF4FF}N5(5) & \multicolumn{1}{c|}{3.32} & \multicolumn{1}{c|}{31.48} & \multicolumn{1}{c|}{23.42} & 26.19 & \multicolumn{1}{c|}{20.02} & \multicolumn{1}{c|}{18.62} & 21.14 & \multicolumn{1}{c|}{0.89} & 0.88 & \multicolumn{1}{c|}{100.00} & 100 \\ \hline
\cellcolor[HTML]{ECF4FF}BT9(2) & \multicolumn{1}{c|}{0.13} & \multicolumn{1}{c|}{0.13} & \multicolumn{1}{c|}{0.77} & 0.42 & \multicolumn{1}{c|}{0.13} & \multicolumn{1}{c|}{0.77} & 0.42 & \multicolumn{1}{c|}{1.10} & 1.41 & \multicolumn{1}{c|}{95.00} & 40 \\
\cellcolor[HTML]{ECF4FF}BT10(2) & \multicolumn{1}{c|}{0.30} & \multicolumn{1}{c|}{0.31} & \multicolumn{1}{c|}{1.16} & 0.46 & \multicolumn{1}{c|}{0.31} & \multicolumn{1}{c|}{1.16} & 0.46 & \multicolumn{1}{c|}{1.13} & 1.27 & \multicolumn{1}{c|}{95.00} & 75 \\
\cellcolor[HTML]{ECF4FF}BT1B(3) & \multicolumn{1}{c|}{34.83} & \multicolumn{1}{c|}{47.58} & \multicolumn{1}{c|}{118.27} & 62.88 & \multicolumn{1}{c|}{47.27} & \multicolumn{1}{c|}{118.11} & 62.70 & \multicolumn{1}{c|}{0.93} & 0.99 & \multicolumn{1}{c|}{100.00} & 100 \\
\cellcolor[HTML]{ECF4FF}BT2(3) & \multicolumn{1}{c|}{5.62} & \multicolumn{1}{c|}{7.08} & \multicolumn{1}{c|}{22.67} & 28.17 & \multicolumn{1}{c|}{6.97} & \multicolumn{1}{c|}{22.61} & 28.11 & \multicolumn{1}{c|}{0.93} & 1.12 & \multicolumn{1}{c|}{100.00} & 95 \\
\cellcolor[HTML]{ECF4FF}BT2B(3) & \multicolumn{1}{c|}{9.58} & \multicolumn{1}{c|}{14.40} & \multicolumn{1}{c|}{41.67} & 21.23 & \multicolumn{1}{c|}{14.13} & \multicolumn{1}{c|}{4.36} & 21.09 & \multicolumn{1}{c|}{1.00} & 1.05 & \multicolumn{1}{c|}{100.00} & 95 \\
\cellcolor[HTML]{ECF4FF}BT3(3) & \multicolumn{1}{c|}{5.38} & \multicolumn{1}{c|}{14.15} & \multicolumn{1}{c|}{62.22} & 32.10 & \multicolumn{1}{c|}{12.80} & \multicolumn{1}{c|}{61.54} & 31.39 & \multicolumn{1}{c|}{1.05} & 1.11 & \multicolumn{1}{c|}{100.00} & 100 \\
\cellcolor[HTML]{ECF4FF}BT5(3) & \multicolumn{1}{c|}{0.06} & \multicolumn{1}{c|}{0.38} & \multicolumn{1}{c|}{0.27} & 0.19 & \multicolumn{1}{c|}{0.12} & \multicolumn{1}{c|}{0.14} & 0.05 & \multicolumn{1}{c|}{0.57} & 0.57 & \multicolumn{1}{c|}{100.00} & 85 \\
\cellcolor[HTML]{ECF4FF}BT8(3) & \multicolumn{1}{c|}{12.17} & \multicolumn{1}{c|}{19.31} & \multicolumn{1}{c|}{169.32} & 79.52 & \multicolumn{1}{c|}{18.87} & \multicolumn{1}{c|}{169.12} & 79.31 & \multicolumn{1}{c|}{1.00} & 1.05 & \multicolumn{1}{c|}{100.00} & 100 \\
\cellcolor[HTML]{ECF4FF}BT8B(3) & \multicolumn{1}{c|}{3.17} & \multicolumn{1}{c|}{3.60} & \multicolumn{1}{c|}{41.63} & 24.23 & \multicolumn{1}{c|}{3.55} & \multicolumn{1}{c|}{41.60} & 24.20 & \multicolumn{1}{c|}{1.16} & 1.20 & \multicolumn{1}{c|}{100.00} & 100 \\
\cellcolor[HTML]{ECF4FF}BT11(3) & \multicolumn{1}{c|}{17.21} & \multicolumn{1}{c|}{35.21} & \multicolumn{1}{c|}{51.19} & 31.23 & \multicolumn{1}{c|}{34.34} & \multicolumn{1}{c|}{50.77} & 30.80 & \multicolumn{1}{c|}{0.88} & 0.95 & \multicolumn{1}{c|}{100.00} & 100 \\
\cellcolor[HTML]{ECF4FF}BT1(4) & \multicolumn{1}{c|}{3.51} & \multicolumn{1}{c|}{97.33} & \multicolumn{1}{c|}{63.69} & 68.39 & \multicolumn{1}{c|}{13.88} & \multicolumn{1}{c|}{18.83} & 22.18 & \multicolumn{1}{c|}{1.02} & 1.04 & \multicolumn{1}{c|}{100.00} & 100 \\
\cellcolor[HTML]{ECF4FF}BT6(4) & \multicolumn{1}{c|}{12.36} & \multicolumn{1}{c|}{124.16} & \multicolumn{1}{c|}{106.73} & 95.75 & \multicolumn{1}{c|}{43.87} & \multicolumn{1}{c|}{61.96} & 50.18 & \multicolumn{1}{c|}{1.01} & 1.03 & \multicolumn{1}{c|}{100.00} & 100 \\
\cellcolor[HTML]{ECF4FF}BT7(4) & \multicolumn{1}{c|}{4.36} & \multicolumn{1}{c|}{95.89} & \multicolumn{1}{c|}{60.65} & 56.06 & \multicolumn{1}{c|}{15.06} & \multicolumn{1}{c|}{15.24} & 9.29 & \multicolumn{1}{c|}{1.00} & 1.01 & \multicolumn{1}{c|}{100.00} & 100 \\ \hline
\cellcolor[HTML]{ECF4FF}UM4(2) & \multicolumn{1}{c|}{2.43} & \multicolumn{1}{c|}{2.93} & \multicolumn{1}{c|}{12.71} & 2.26 & \multicolumn{1}{c|}{2.90} & \multicolumn{1}{c|}{12.69} & 2.24 & \multicolumn{1}{c|}{0.96} & 1.41 & \multicolumn{1}{c|}{70.00} & 5 \\
\cellcolor[HTML]{ECF4FF}UM4B1(2) & \multicolumn{1}{c|}{4.81} & \multicolumn{1}{c|}{5.68} & \multicolumn{1}{c|}{17.38} & 5.03 & \multicolumn{1}{c|}{5.64} & \multicolumn{1}{c|}{17.35} & 5.01 & \multicolumn{1}{c|}{0.86} & 1.12 & \multicolumn{1}{c|}{90.00} & 45 \\
\cellcolor[HTML]{ECF4FF}UM1(3) & \multicolumn{1}{c|}{6.68} & \multicolumn{1}{c|}{58.62} & \multicolumn{1}{c|}{49.35} & 31.92 & \multicolumn{1}{c|}{47.14} & \multicolumn{1}{c|}{42.58} & 24.86 & \multicolumn{1}{c|}{0.98} & 1.09 & \multicolumn{1}{c|}{100.00} & 100 \\
\cellcolor[HTML]{ECF4FF}UM2(3) & \multicolumn{1}{c|}{0.04} & \multicolumn{1}{c|}{2.94} & \multicolumn{1}{c|}{4.49} & 3.44 & \multicolumn{1}{c|}{0.21} & \multicolumn{1}{c|}{2.97} & 1.83 & \multicolumn{1}{c|}{1.95} & 2.23 & \multicolumn{1}{c|}{75.00} & 40 \\
\cellcolor[HTML]{ECF4FF}UM5(3) & \multicolumn{1}{c|}{2.87} & \multicolumn{1}{c|}{31.79} & \multicolumn{1}{c|}{29.71} & 21.31 & \multicolumn{1}{c|}{19.01} & \multicolumn{1}{c|}{22.18} & 13.63 & \multicolumn{1}{c|}{1.05} & 1.26 & \multicolumn{1}{c|}{95.00} & 85 \\ \hline
\cellcolor[HTML]{ECF4FF}ZZB1(2) & \multicolumn{1}{c|}{0.44} & \multicolumn{1}{c|}{0.49} & \multicolumn{1}{c|}{10.44} & 0.47 & \multicolumn{1}{c|}{0.48} & \multicolumn{1}{c|}{10.43} & 0.45 & \multicolumn{1}{c|}{0.89} & 1.01 & \multicolumn{1}{c|}{100.00} & 5 \\
\cellcolor[HTML]{ECF4FF}ZZB2(2) & \multicolumn{1}{c|}{0.71} & \multicolumn{1}{c|}{2.51} & \multicolumn{1}{c|}{272.70} & 2.66 & \multicolumn{1}{c|}{1.22} & \multicolumn{1}{c|}{271.96} & 1.83 & \multicolumn{1}{c|}{0.89} & 2.62 & \multicolumn{1}{c|}{100.00} & 65 \\
\cellcolor[HTML]{ECF4FF}ZZB3(2) & \multicolumn{1}{c|}{0.47} & \multicolumn{1}{c|}{7.69} & \multicolumn{1}{c|}{341.84} & 5.33 & \multicolumn{1}{c|}{1.18} & \multicolumn{1}{c|}{338.15} & 1.45 & \multicolumn{1}{c|}{0.88} & 2.58 & \multicolumn{1}{c|}{100.00} & 65 \\ \hline
\end{tabular}
\caption{Comparison of running time and solution length between $\XA$ (using \loc and \glo) and uniform random sampling. For random sampling we report the average values in terms of running and solution length (the latter is given as normalized value with  respect to the solution length with $\XA$). }
\label{tbl:lattice_vs_random:app}
\end{table*}

\begin{figure*}[th]
  \centering
%     \hspace*{-0.66cm}
% \subfloat[Zigzag-bypass (long)]{\includegraphics[width=2.18\columnwidth,clip]{Images/Scenarios/ZZB3H_scenario.png}
%     %\label{fig:3d_lattices:da}
%     }
%     \newline
\subfloat{\includegraphics[width=\columnwidth,clip]{Images/tuning1.pdf}
    }
    \subfloat{\includegraphics[width=\columnwidth,clip]{Images/tuning2.pdf}
    }
    \newline
\subfloat{\includegraphics[width=\columnwidth,clip]{Images/tuning3.pdf}
    }
\subfloat{\includegraphics[width=\columnwidth,clip]{Images/tuning4.pdf}
    }
      \caption{Effect of the parameters $\delta,\epsilon$ on the performance of \loc with $\XA$ for $\delta=2.5$ (left) and $\delta=4$ (right). We report the running time (top) and solution length (bottom). The absence of data points for the parameters $\delta=4, \eps\in \{2,4,5\}$ indicates a solution failure.  
  }
  \label{fig:parameters:app}
\end{figure*}

\subsection{Comparison with Random Sampling}
Extended results where $\XA$-samples are compared with \rnd are given in Table~\ref{tbl:lattice_vs_random:app}. Here, we consider two versions of random sampling. The first version, denoted by \rnd, which is identical to the one considered in the main paper, uses random sampling together with the asymptotically optimal connection radius $r_{\textup{rnd}}(n)$, which is commonly used in practice. The second version, denoted by \rndm uses the radius as ${r^*}$ used for lattice-based sampling. The latter is used to further emphasize the inferiority of uniform random sampling as compared to $\XA$ due to identical parameters between $\XA$-\glo and \rndm-\glo (except for the sampling distribution). In particular, the move to \rndm  severely reduces the success rates in some of the scenarios.

Another addition in Table~\ref{tbl:lattice_vs_random:app} is the running time of the search algorithm (under "search time"). Recall that the total running time for \glo consists of the (i) construction of the sample set and the nearest-neighbor data structure and the (ii) running the search algorithm. Although both $\XA$-\glo and \rnd use the same number of samples, the construction time is usually larger in the former due to an additional step of constructing the lattice samples over the entire configuration space, which is currently implemented in a naive and unoptimized manner. In this sense, the comparison between $\XA$-\glo and \rnd is not entirely fair. Thus, we also report the running time of the search algorithm, which can be the computational bottleneck, especially for more complicated robot geometries where the collision-check operation is more expensive~\cite{KleinbortSH16}. Although the search time for $\XA$-\glo is usually lower for most scenarios, we argue that with more expensive collision checks, the advantage of lattice-based sample sets would be even more prominent.

\subsection{Effect of parameter choice}
We report the effect of the choice of the $\delta$ and $\eps$ parameters on solution length and running time for the \loc algorithm using $\XA$ sampling. We specifically focus on the ZZB3 scenario due to the availability of several homotopy classes for the solution, where each class has a different length and level of difficulty. For instance, in one class, the robots use the rightmost part of the workspace, which consists of a winding path, and exchange positions halfway between---leading to a relatively short solution length. In a second class, the robots use the long passage to the left, which consists of long straight-line motions and yields a significantly longer solution length.


We set $\delta\in \{2.75,4\}$ and report the solution length and running time in Figure~\ref{fig:parameters:app} for $\eps\in \{0.5,0.6,\ldots,1,2,\ldots,10\}$. Observe that for $\delta=2.75$ the planner obtains a low-length solution already for high $\eps$ values, whereas $\delta=4$ initially uncovers an inefficient solution length-wise but eventually settles on the better homotopy class when $\eps$ is reduced. From values of $\eps\leq 1$ the length relatively stabilizes, while the runtime jumps at several orders of magnitude, which highlights the exponential dependence of sample and collision-check complexity on the value $\eps$. Finding the middle-ground $\eps$ value is an important goal, which we leave for future work. 

Notice that the planner fails to find a solution for $\delta=4$ and $\eps\in\{2,4,5\}$. Due to our \decomps result, this implies that no $4$-clear solution exists. Despite this, the planner does succeed for some values of $\eps$, which suggests that our sufficient conditions for \decomps are not necessary. The success could also be explained by the specific arrangement of the points in $\XA$, which coincidentally induces a connected component via the second homotopy class for this specific scenario. It should also be noted that the sample set $\X_{\AN}^{4,\eps}$ can be viewed (via Lemma~1) as the sample set $\X_{\AN}^{2.5,\eps'}$ for $\eps$ small enough, which explains the success of the planner with  $\delta=4$ and smaller $\eps$ values. 

% Please add the following required packages to your document preamble:
% \usepackage{multirow}
% \usepackage[table,xcdraw]{xcolor}
% Beamer presentation requires \usepackage{colortbl} instead of \usepackage[table,xcdraw]{xcolor}




\end{document}
