\section{Related Work}
\label{sec:related-work}

% CH: CSF PC members most likely to review this
%     https://csf2025.ieee-security.org/committee.html
% - Benjamin Gregoire: Coq, side-channels, Jasmin, etc. [DONE]
% - Lennart Beringer: Coq, some IFC [NOTHING]
% - Andrei Popescu: Isabelle; relative security paper at CSF'24 [DONE]
% - Deian Stefan: https://dblp.org/pid/91/6118.html [DONE]
%   + SoK, Blade, Swivel, FaCT, spectre era x 2
% - Lesly-Ann Daniel: speculative side-channels [DONE]
%   + https://dblp.org/pid/222/9411.html
% - Sébastien Bardin [DONE]
%   + Binsec/Haunted: https://dblp.org/rec/conf/ndss/DanielBR21.html?view=bibtex
%   + ProSpeCT: https://dblp.org/rec/conf/uss/DanielBNBRP23.html?view=bibtex -- unrelated!
% - Tegan Brennan: side-channels (in SE venues)
% - Michael Emmi:
%   + ct-fuzz, Fuzzing for Timing leaks https://dblp.org/rec/conf/icst/HeEC20.html?view=bibtex
% - various IFC people: Owen Arden, Aslan Askarov, Abhishek Bichhawat,
%     Daniel Hedin, Michael Hicks, Elisavet Kozyri, Piotr Mardziel,
%     Scott Moore, Alejandro Russo

As already mentioned, SLH was originally proposed and implemented in
LLVM____ as a defense against Spectre v1 attacks.
%
____ noticed that SLH is not strong enough to achieve one of
their relative security notions and proposed Strong SLH, which uses the
misspeculation flag to also mask all branch conditions and all addresses of
memory accesses.
%
____ later noticed that the inputs of all variable-time
instructions also have to be masked for security and implemented everything as
the Ultimate SLH program transformation in the x86 backend of LLVM.
%
They experimentally evaluated Ultimate SLH on the SPEC2017 benchmarks and
reported overheads of around 150\%, which is large, but still smaller than
adding fences.

Taking inspiration in prior static analysis work____, both ____ and ____ propose
to use relative security to assess the security of their transformations, by
requiring that the transformed program does not leak speculatively more than
{\em the transformed program itself} leaks sequentially.\ch{Should double check
  that this is the case for ____, but it's difficult given the
  mess they do.}
%
While this is a sensible way to define relative security in the absence of a
program transformation____,\ch{as discussed earlier, hopefully}
%
when applied only to the transformed program this deems secure a transformation
that introduces sequential leaks, instead of removing speculative ones.
%
Instead, we use a security definition that is more suitable for transformations
and compilers, which requires that any transformed program running with
speculation does not leak more than what the {\em source} program leaks sequentially.
%
Finally, our definition only looks at 4 executions (as opposed to 8
executions____) and our proofs are fully mechanized in Rocq.
%
\ifanon\else
A similar relative security definition was independently discovered by
Jonathan Baumann, who verified fence-based Spectre countermeasures in the Rocq
prover using hypersimulations____.
\fi

In a separate line of research, \SelvSLH was proposed as an efficient way to
protect cryptographic code against Spectre v1 in the Jasmin
language____.
%
____ proved in a simplified setting that an automatic
\SelvSLH transformation (discussed in \autoref{sec:FvSLH}) achieves speculative
CT security.
%
Yet for the best efficiency, Jasmin programmers may have to reorganize their code
and manually insert a minimal number SLH protections, with a static analysis
just checking that the resulting code is secure____.
%
In more recent work, ____ showed how speculative CT security can
be preserved by compilation in a simplified model of the Jasmin compiler, which
they formalized in Rocq.
%
These preservation proofs seem easily extensible to the stronger relative
security definition we use in this paper, yet providing security guarantees to
non-CCT code does not seem a goal for the Jasmin language, which is specifically
targeted at cryptographic implementations.

The simple and abstract speculative semantics we use in \autoref{sec:defs} is
taken from the paper of ____, who credit prior work by
____ and ____ for the idea of a
``forwards'' semantics that takes attacker directions and does no rollbacks.
%
This style of speculative semantics seems more suitable for higher-level
languages, for which the concept of a ``speculation window'' that triggers
rollbacks would require exposing too many low-level details about the
compilation and target architecture, if at all possible to accommodate.
%
____ also prove that in their setting any speculative CT
attack against a semantics with rollbacks also exist in their ``forwards''
semantics with directions.