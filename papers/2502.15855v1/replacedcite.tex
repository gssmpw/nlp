\section{Related Work}
\subsection{Peptide generation}
% \paragraph{Peptide generation}
% \textbf{Peptide generation}
Peptide generation is a complex task in drug discovery, aiming to design peptides with specific biological functions, such as binding to target proteins or modulating enzymatic activity. 
Generative models have significantly advanced peptide design by exploring large sequence and structural spaces with greater flexibility. For example, RFDiffusion____, designed for protein design, inspired the application of similar diffusion techniques for peptide generation. 
AMP-diffusion____ utilizes the capabilities of protein large language model ESM-2____ to regenerate functional antimicrobial peptides (AMPs). 
MMCD____ employs multi-modal contrastive learning in diffusion, exploiting the integration of both sequence and structural information to produce peptides with high functional relevance.

Peptide design is a subtask of peptide generation, involving several key components. 
% Backbone design methods____, such as PepFlow____ and PPFlow____, use flow matching to simulate dynamic conformational changes and optimize peptide properties.
One important aspect is backbone design, where methods like PepFlow____ and PPFlow____ use flow matching to simulate dynamic conformational changes and optimize peptide properties. 
% For side-chain packing, methods like RED-PPI____ focus on protein-protein complexes, while DiffPack____ targets peptide-protein interactions.
Another aspect is side-chain packing, with methods such as RED-PPI____ focusing on protein-protein complexes and DiffPack____ targeting peptide-protein interactions.

\subsection{Protein–ligand docking}
% \paragraph{Protein–ligand docking}
Protein–ligand docking aims to predict the binding pose and affinity of a ligand to a target protein. 
Traditional docking methods often rely on rigid-body simulations and predefined scoring functions, which can struggle to handle flexible ligands and complex interactions____.
Recent years, deep learning (DL) approaches have significantly improved docking accuracy____. Models like AtomNet____ and OnionNet____ use convolutional neural networks (CNNs) to capture complex molecular features, significantly enhancing binding affinity predictions.
Further developments have incorporated graph-based models, such as GraphSite____ and DGraphDTA____, which represent protein-ligand interactions as graphs to better model flexible docking. 
Additionally, advances in protein structure characterization, notably via AlphaFold____, have enhanced understanding of protein flexibility, fueling the development of structure-based docking prediction methods____.
However, challenges still remain in fully capturing the dynamic and flexible nature of ligands, highlighting the need for further improvements in docking models.