% ICCV 2025 Paper Template

\documentclass[10pt,twocolumn,letterpaper]{article}

%%%%%%%%% PAPER TYPE  - PLEASE UPDATE FOR FINAL VERSION
% \usepackage{iccv}              % To produce the CAMERA-READY version
% \usepackage[review]{iccv}      % To produce the REVIEW version
\usepackage[pagenumbers]{iccv} % To force page numbers, e.g. for an arXiv version

% Import additional packages in the preamble file, before hyperref
%
% --- inline annotations
%
\newcommand{\red}[1]{{\color{red}#1}}
\newcommand{\todo}[1]{{\color{red}#1}}
\newcommand{\TODO}[1]{\textbf{\color{red}[TODO: #1]}}
% --- disable by uncommenting  
% \renewcommand{\TODO}[1]{}
% \renewcommand{\todo}[1]{#1}



\newcommand{\VLM}{LVLM\xspace} 
\newcommand{\ours}{PeKit\xspace}
\newcommand{\yollava}{Yo’LLaVA\xspace}

\newcommand{\thisismy}{This-Is-My-Img\xspace}
\newcommand{\myparagraph}[1]{\noindent\textbf{#1}}
\newcommand{\vdoro}[1]{{\color[rgb]{0.4, 0.18, 0.78} {[V] #1}}}
% --- disable by uncommenting  
% \renewcommand{\TODO}[1]{}
% \renewcommand{\todo}[1]{#1}
\usepackage{slashbox}
% Vectors
\newcommand{\bB}{\mathcal{B}}
\newcommand{\bw}{\mathbf{w}}
\newcommand{\bs}{\mathbf{s}}
\newcommand{\bo}{\mathbf{o}}
\newcommand{\bn}{\mathbf{n}}
\newcommand{\bc}{\mathbf{c}}
\newcommand{\bp}{\mathbf{p}}
\newcommand{\bS}{\mathbf{S}}
\newcommand{\bk}{\mathbf{k}}
\newcommand{\bmu}{\boldsymbol{\mu}}
\newcommand{\bx}{\mathbf{x}}
\newcommand{\bg}{\mathbf{g}}
\newcommand{\be}{\mathbf{e}}
\newcommand{\bX}{\mathbf{X}}
\newcommand{\by}{\mathbf{y}}
\newcommand{\bv}{\mathbf{v}}
\newcommand{\bz}{\mathbf{z}}
\newcommand{\bq}{\mathbf{q}}
\newcommand{\bff}{\mathbf{f}}
\newcommand{\bu}{\mathbf{u}}
\newcommand{\bh}{\mathbf{h}}
\newcommand{\bb}{\mathbf{b}}

\newcommand{\rone}{\textcolor{green}{R1}}
\newcommand{\rtwo}{\textcolor{orange}{R2}}
\newcommand{\rthree}{\textcolor{red}{R3}}
\usepackage{amsmath}
%\usepackage{arydshln}
\DeclareMathOperator{\similarity}{sim}
\DeclareMathOperator{\AvgPool}{AvgPool}

\newcommand{\argmax}{\mathop{\mathrm{argmax}}}     



% It is strongly recommended to use hyperref, especially for the review version.
% hyperref with option pagebackref eases the reviewers' job.
% Please disable hyperref *only* if you encounter grave issues, 
% e.g. with the file validation for the camera-ready version.
%
% If you comment hyperref and then uncomment it, you should delete *.aux before re-running LaTeX.
% (Or just hit 'q' on the first LaTeX run, let it finish, and you should be clear).
\definecolor{iccvblue}{rgb}{0.21,0.49,0.74}
\usepackage[pagebackref,breaklinks,colorlinks,allcolors=iccvblue]{hyperref}

%%%%%%%%% PAPER ID  - PLEASE UPDATE
\def\paperID{****} % 7521 Enter the Paper ID here
\def\confName{ICCV}
\def\confYear{2025}

\newcommand{\methodname}{UniGaze\xspace}

%%%%%%%%% TITLE - PLEASE UPDATE
\title{\methodname: Towards Universal Gaze Estimation via Large-scale Pre-Training}

%%%%%%%%% AUTHORS - PLEASE UPDATE
\author{Jiawei Qin$^{1}$, Xucong Zhang$^{2}$, Yusuke Sugano$^{1}$ \\
\normalsize $^1$ Institute of Industrial Science, The University of Tokyo, Komaba 4-6-1, Tokyo, Japan \\
\normalsize $^2$ Computer Vision Lab, Delft University of Technology, Mekelweg 5, Delft, Netherlands \\
{
\tt\small \{jqin, sugano\}@iis.u-tokyo.ac.jp}\\
\tt\small xucong.zhang@tudelft.nl
}


\begin{document}
% \maketitle


\twocolumn[{%
\renewcommand\twocolumn[1][]{#1}%
\maketitle
\vspace{-25pt}
\begin{center}
  % \captionsetup{type=figure}
  \includegraphics[width=0.9\linewidth]{figs/teaser_v6.pdf}
  % \fbox{\rule{0pt}{2.5in} \rule{0.95\linewidth}{0pt}}
  \captionof{figure}{
  Leveraging self-supervised pre-training on large-scale facial data, the proposed \methodname demonstrates strong generalization to unseen in-the-wild face images under diverse conditions, including facial appearance, lighting conditions, variant head poses, and face resolutions. We draw the estimated gaze direction with green arrows. More examples are shown in the supplementary materials.
  }
  \label{fig:teaser}
\end{center}%
}]


\begin{abstract}
Despite decades of research on data collection and model architectures, current gaze estimation models encounter significant challenges in generalizing across diverse data domains. 
Recent advances in self-supervised pre-training have shown remarkable performances in generalization across various vision tasks.
However, their effectiveness in gaze estimation remains unexplored.
We propose \methodname, for the first time, leveraging large-scale in-the-wild facial datasets for gaze estimation through self-supervised pre-training.
Through systematic investigation, we clarify critical factors that are essential for effective pre-training in gaze estimation. 
Our experiments reveal that self-supervised approaches designed for semantic tasks fail when applied to gaze estimation, while our carefully designed pre-training pipeline consistently improves cross-domain performance. 
Through comprehensive experiments of challenging cross-dataset evaluation and novel protocols including leave-one-dataset-out and joint-dataset settings, we demonstrate that \methodname significantly improves generalization across multiple data domains while minimizing reliance on costly labeled data. 
Source code and model are available at \url{https://github.com/ut-vision/UniGaze}.
\end{abstract}

  
\section{Introduction}
Backdoor attacks pose a concealed yet profound security risk to machine learning (ML) models, for which the adversaries can inject a stealth backdoor into the model during training, enabling them to illicitly control the model's output upon encountering predefined inputs. These attacks can even occur without the knowledge of developers or end-users, thereby undermining the trust in ML systems. As ML becomes more deeply embedded in critical sectors like finance, healthcare, and autonomous driving \citep{he2016deep, liu2020computing, tournier2019mrtrix3, adjabi2020past}, the potential damage from backdoor attacks grows, underscoring the emergency for developing robust defense mechanisms against backdoor attacks.

To address the threat of backdoor attacks, researchers have developed a variety of strategies \cite{liu2018fine,wu2021adversarial,wang2019neural,zeng2022adversarial,zhu2023neural,Zhu_2023_ICCV, wei2024shared,wei2024d3}, aimed at purifying backdoors within victim models. These methods are designed to integrate with current deployment workflows seamlessly and have demonstrated significant success in mitigating the effects of backdoor triggers \cite{wubackdoorbench, wu2023defenses, wu2024backdoorbench,dunnett2024countering}.  However, most state-of-the-art (SOTA) backdoor purification methods operate under the assumption that a small clean dataset, often referred to as \textbf{auxiliary dataset}, is available for purification. Such an assumption poses practical challenges, especially in scenarios where data is scarce. To tackle this challenge, efforts have been made to reduce the size of the required auxiliary dataset~\cite{chai2022oneshot,li2023reconstructive, Zhu_2023_ICCV} and even explore dataset-free purification techniques~\cite{zheng2022data,hong2023revisiting,lin2024fusing}. Although these approaches offer some improvements, recent evaluations \cite{dunnett2024countering, wu2024backdoorbench} continue to highlight the importance of sufficient auxiliary data for achieving robust defenses against backdoor attacks.

While significant progress has been made in reducing the size of auxiliary datasets, an equally critical yet underexplored question remains: \emph{how does the nature of the auxiliary dataset affect purification effectiveness?} In  real-world  applications, auxiliary datasets can vary widely, encompassing in-distribution data, synthetic data, or external data from different sources. Understanding how each type of auxiliary dataset influences the purification effectiveness is vital for selecting or constructing the most suitable auxiliary dataset and the corresponding technique. For instance, when multiple datasets are available, understanding how different datasets contribute to purification can guide defenders in selecting or crafting the most appropriate dataset. Conversely, when only limited auxiliary data is accessible, knowing which purification technique works best under those constraints is critical. Therefore, there is an urgent need for a thorough investigation into the impact of auxiliary datasets on purification effectiveness to guide defenders in  enhancing the security of ML systems. 

In this paper, we systematically investigate the critical role of auxiliary datasets in backdoor purification, aiming to bridge the gap between idealized and practical purification scenarios.  Specifically, we first construct a diverse set of auxiliary datasets to emulate real-world conditions, as summarized in Table~\ref{overall}. These datasets include in-distribution data, synthetic data, and external data from other sources. Through an evaluation of SOTA backdoor purification methods across these datasets, we uncover several critical insights: \textbf{1)} In-distribution datasets, particularly those carefully filtered from the original training data of the victim model, effectively preserve the model’s utility for its intended tasks but may fall short in eliminating backdoors. \textbf{2)} Incorporating OOD datasets can help the model forget backdoors but also bring the risk of forgetting critical learned knowledge, significantly degrading its overall performance. Building on these findings, we propose Guided Input Calibration (GIC), a novel technique that enhances backdoor purification by adaptively transforming auxiliary data to better align with the victim model’s learned representations. By leveraging the victim model itself to guide this transformation, GIC optimizes the purification process, striking a balance between preserving model utility and mitigating backdoor threats. Extensive experiments demonstrate that GIC significantly improves the effectiveness of backdoor purification across diverse auxiliary datasets, providing a practical and robust defense solution.

Our main contributions are threefold:
\textbf{1) Impact analysis of auxiliary datasets:} We take the \textbf{first step}  in systematically investigating how different types of auxiliary datasets influence backdoor purification effectiveness. Our findings provide novel insights and serve as a foundation for future research on optimizing dataset selection and construction for enhanced backdoor defense.
%
\textbf{2) Compilation and evaluation of diverse auxiliary datasets:}  We have compiled and rigorously evaluated a diverse set of auxiliary datasets using SOTA purification methods, making our datasets and code publicly available to facilitate and support future research on practical backdoor defense strategies.
%
\textbf{3) Introduction of GIC:} We introduce GIC, the \textbf{first} dedicated solution designed to align auxiliary datasets with the model’s learned representations, significantly enhancing backdoor mitigation across various dataset types. Our approach sets a new benchmark for practical and effective backdoor defense.



\section{Related Work}
\label{sec:related-works}
\subsection{Novel View Synthesis}
Novel view synthesis is a foundational task in the computer vision and graphics, which aims to generate unseen views of a scene from a given set of images.
% Many methods have been designed to solve this problem by posing it as 3D geometry based rendering, where point clouds~\cite{point_differentiable,point_nfs}, mesh~\cite{worldsheet,FVS,SVS}, planes~\cite{automatci_photo_pop_up,tour_into_the_picture} and multi-plane images~\cite{MINE,single_view_mpi,stereo_magnification}, \etal
Numerous methods have been developed to address this problem by approaching it as 3D geometry-based rendering, such as using meshes~\cite{worldsheet,FVS,SVS}, MPI~\cite{MINE,single_view_mpi,stereo_magnification}, point clouds~\cite{point_differentiable,point_nfs}, etc.
% planes~\cite{automatci_photo_pop_up,tour_into_the_picture}, 


\begin{figure*}[!t]
    \centering
    \includegraphics[width=1.0\linewidth]{figures/overview-v7.png}
    %\caption{\textbf{Overview.} Given a set of images, our method obtains both camera intrinsics and extrinsics, as well as a 3DGS model. First, we obtain the initial camera parameters, global track points from image correspondences and monodepth with reprojection loss. Then we incorporate the global track information and select Gaussian kernels associated with track points. We jointly optimize the parameters $K$, $T_{cw}$, 3DGS through multi-view geometric consistency $L_{t2d}$, $L_{t3d}$, $L_{scale}$ and photometric consistency $L_1$, $L_{D-SSIM}$.}
    \caption{\textbf{Overview.} Given a set of images, our method obtains both camera intrinsics and extrinsics, as well as a 3DGS model. During the initialization, we extract the global tracks, and initialize camera parameters and Gaussians from image correspondences and monodepth with reprojection loss. We determine Gaussian kernels with recovered 3D track points, and then jointly optimize the parameters $K$, $T_{cw}$, 3DGS through the proposed global track constraints (i.e., $L_{t2d}$, $L_{t3d}$, and $L_{scale}$) and original photometric losses (i.e., $L_1$ and $L_{D-SSIM}$).}
    \label{fig:overview}
\end{figure*}

Recently, Neural Radiance Fields (NeRF)~\cite{2020NeRF} provide a novel solution to this problem by representing scenes as implicit radiance fields using neural networks, achieving photo-realistic rendering quality. Although having some works in improving efficiency~\cite{instant_nerf2022, lin2022enerf}, the time-consuming training and rendering still limit its practicality.
Alternatively, 3D Gaussian Splatting (3DGS)~\cite{3DGS2023} models the scene as explicit Gaussian kernels, with differentiable splatting for rendering. Its improved real-time rendering performance, lower storage and efficiency, quickly attract more attentions.
% Different from NeRF-based methods which need MLPs to model the scene and huge computational cost for rendering, 3DGS has stronger real-time performance, higher storage and computational efficiency, benefits from its explicit representation and gradient backpropagation.

\subsection{Optimizing Camera Poses in NeRFs and 3DGS}
Although NeRF and 3DGS can provide impressive scene representation, these methods all need accurate camera parameters (both intrinsic and extrinsic) as additional inputs, which are mostly obtained by COLMAP~\cite{colmap2016}.
% This strong reliance on COLMAP significantly limits their use in real-world applications, so optimizing the camera parameters during the scene training becomes crucial.
When the prior is inaccurate or unknown, accurately estimating camera parameters and scene representations becomes crucial.

% In early works, only photometric constraints are used for scene training and camera pose estimation. 
% iNeRF~\cite{iNerf2021} optimizes the camera poses based on a pre-trained NeRF model.
% NeRFmm~\cite{wang2021nerfmm} introduce a joint optimization process, which estimates the camera poses and trains NeRF model jointly.
% BARF~\cite{barf2021} and GARF~\cite{2022GARF} provide new positional encoding strategy to handle with the gradient inconsistency issue of positional embedding and yield promising results.
% However, they achieve satisfactory optimization results when only the pose initialization is quite closed to the ground-truth, as the photometric constrains can only improve the quality of camera estimation within a small range.
% Later, more prior information of geometry and correspondence, \ie monocular depth and feature matching, are introduced into joint optimisation to enhance the capability of camera poses estimation.
% SC-NeRF~\cite{SCNeRF2021} minimizes a projected ray distance loss based on correspondence of adjacent frames.
% NoPe-NeRF~\cite{bian2022nopenerf} chooses monocular depth maps as geometric priors, and defines undistorted depth loss and relative pose constraints for joint optimization.
In earlier studies, scene training and camera pose estimation relied solely on photometric constraints. iNeRF~\cite{iNerf2021} refines the camera poses using a pre-trained NeRF model. NeRFmm~\cite{wang2021nerfmm} introduces a joint optimization approach that simultaneously estimates camera poses and trains the NeRF model. BARF~\cite{barf2021} and GARF~\cite{2022GARF} propose a new positional encoding strategy to address the gradient inconsistency issues in positional embedding, achieving promising results. However, these methods only yield satisfactory optimization when the initial pose is very close to the ground truth, as photometric constraints alone can only enhance camera estimation quality within a limited range. Subsequently, 
% additional prior information on geometry and correspondence, such as monocular depth and feature matching, has been incorporated into joint optimization to improve the accuracy of camera pose estimation. 
SC-NeRF~\cite{SCNeRF2021} minimizes a projected ray distance loss based on correspondence between adjacent frames. NoPe-NeRF~\cite{bian2022nopenerf} utilizes monocular depth maps as geometric priors and defines undistorted depth loss and relative pose constraints.

% With regard to 3D Gaussian Splatting, CF-3DGS~\cite{CF-3DGS-2024} also leverages mono-depth information to constrain the optimization of local 3DGS for relative pose estimation and later learn a global 3DGS progressively in a sequential manner.
% InstantSplat~\cite{fan2024instantsplat} focus on sparse view scenes, first use DUSt3R~\cite{dust3r2024cvpr} to generate a set of densely covered and pixel-aligned points for 3D Gaussian initialization, then introduce a parallel grid partitioning strategy in joint optimization to speed up.
% % Jiang et al.~\cite{Jiang_2024sig} proposed to build the scene continuously and progressively, to next unregistered frame, they use registration and adjustment to adjust the previous registered camera poses and align unregistered monocular depths, later refine the joint model by matching detected correspondences in screen-space coordinates.
% \gjh{Jiang et al.~\cite{Jiang_2024sig} also implemented an incremental approach for reconstructing camera poses and scenes. Initially, they perform feature matching between the current image and the image rendered by a differentiable surface renderer. They then construct matching point errors, depth errors, and photometric errors to achieve the registration and adjustment of the current image. Finally, based on the depth map, the pixels of the current image are projected as new 3D Gaussians. However, this method still exhibits limitations when dealing with complex scenes and unordered images.}
% % CG-3DGS~\cite{sun2024correspondenceguidedsfmfree3dgaussian} follows CF-3DGS, first construct a coarse point cloud from mono-depth maps to train a 3DGS model, then progressively estimate camera poses based on this pre-trained model by constraining the correspondences between rendering view and ground-truth.
% \gjh{Similarly, CG-3DGS~\cite{sun2024correspondenceguidedsfmfree3dgaussian} first utilizes monocular depth estimation and the camera parameters from the first frame to initialize a set of 3D Gaussians. It then progressively estimates camera poses based on this pre-trained model by constraining the correspondences between the rendered views and the ground truth.}
% % Free-SurGS~\cite{freesurgs2024} matches the projection flow derived from 3D Gaussians with optical flow to estimate the poses, to compensate for the limitations of photometric loss.
% \gjh{Free-SurGS~\cite{freesurgs2024} introduces the first SfM-free 3DGS approach for surgical scene reconstruction. Due to the challenges posed by weak textures and photometric inconsistencies in surgical scenes, Free-SurGS achieves pose estimation by minimizing the flow loss between the projection flow and the optical flow. Subsequently, it keeps the camera pose fixed and optimizes the scene representation by minimizing the photometric loss, depth loss and flow loss.}
% \gjh{However, most current works assume camera intrinsics are known and primarily focus on optimizing camera poses. Additionally, these methods typically rely on sequentially ordered image inputs and incrementally optimize camera parameters and scene representation. This inevitably leads to drift errors, preventing the achievement of globally consistent results. Our work aims to address these issues.}

Regarding 3D Gaussian Splatting, CF-3DGS~\cite{CF-3DGS-2024} utilizes mono-depth information to refine the optimization of local 3DGS for relative pose estimation and subsequently learns a global 3DGS in a sequential manner. InstantSplat~\cite{fan2024instantsplat} targets sparse view scenes, initially employing DUSt3R~\cite{dust3r2024cvpr} to create a densely covered, pixel-aligned point set for initializing 3D Gaussian models, and then implements a parallel grid partitioning strategy to accelerate joint optimization. Jiang \etal~\cite{Jiang_2024sig} develops an incremental method for reconstructing camera poses and scenes, but it struggles with complex scenes and unordered images. 
% Similarly, CG-3DGS~\cite{sun2024correspondenceguidedsfmfree3dgaussian} progressively estimates camera poses using a pre-trained model by aligning the correspondences between rendered views and actual scenes. Free-SurGS~\cite{freesurgs2024} pioneers an SfM-free 3DGS method for reconstructing surgical scenes, overcoming challenges such as weak textures and photometric inconsistencies by minimizing the discrepancy between projection flow and optical flow.
%\pb{SF-3DGS-HT~\cite{ji2024sfmfree3dgaussiansplatting} introduced VFI into training as additional photometric constraints. They separated the whole scene into several local 3DGS models and then merged them hierarchically, which leads to a significant improvement on simple and dense view scenes.}
HT-3DGS~\cite{ji2024sfmfree3dgaussiansplatting} interpolates frames for training and splits the scene into local clips, using a hierarchical strategy to build 3DGS model. It works well for simple scenes, but fails with dramatic motions due to unstable interpolation and low efficiency.
% {While effective for simple scenes, it struggles with dramatic motion due to unstable view interpolation and suffers from low computational efficiency.}

However, most existing methods generally depend on sequentially ordered image inputs and incrementally optimize camera parameters and 3DGS, which often leads to drift errors and hinders achieving globally consistent results. Our work seeks to overcome these limitations.

\section{Study Design}
% robot: aliengo 
% We used the Unitree AlienGo quadruped robot. 
% See Appendix 1 in AlienGo Software Guide PDF
% Weight = 25kg, size (L,W,H) = (0.55, 0.35, 06) m when standing, (0.55, 0.35, 0.31) m when walking
% Handle is 0.4 m or 0.5 m. I'll need to check it to see which type it is.
We gathered input from primary stakeholders of the robot dog guide, divided into three subgroups: BVI individuals who have owned a dog guide, BVI individuals who were not dog guide owners, and sighted individuals with generally low degrees of familiarity with dog guides. While the main focus of this study was on the BVI participants, we elected to include survey responses from sighted participants given the importance of social acceptance of the robot by the general public, which could reflect upon the BVI users themselves and affect their interactions with the general population \cite{kayukawa2022perceive}. 

The need-finding processes consisted of two stages. During Stage 1, we conducted in-depth interviews with BVI participants, querying their experiences in using conventional assistive technologies and dog guides. During Stage 2, a large-scale survey was distributed to both BVI and sighted participants. 

This study was approved by the University’s Institutional Review Board (IRB), and all processes were conducted after obtaining the participants' consent.

\subsection{Stage 1: Interviews}
We recruited nine BVI participants (\textbf{Table}~\ref{tab:bvi-info}) for in-depth interviews, which lasted 45-90 minutes for current or former dog guide owners (DO) and 30-60 minutes for participants without dog guides (NDO). Group DO consisted of five participants, while Group NDO consisted of four participants.
% The interview participants were divided into two groups. Group DO (Dog guide Owner) consisted of five participants who were current or former dog guide owners and Group NDO (Non Dog guide Owner) consisted of three participants who were not dog guide owners. 
All participants were familiar with using white canes as a mobility aid. 

We recruited participants in both groups, DO and NDO, to gather data from those with substantial experience with dog guides, offering potentially more practical insights, and from those without prior experience, providing a perspective that may be less constrained and more open to novel approaches. 

We asked about the participants' overall impressions of a robot dog guide, expectations regarding its potential benefits and challenges compared to a conventional dog guide, their desired methods of giving commands and communicating with the robot dog guide, essential functionalities that the robot dog guide should offer, and their preferences for various aspects of the robot dog guide's form factors. 
For Group DO, we also included questions that asked about the participants' experiences with conventional dog guides. 

% We obtained permission to record the conversations for our records while simultaneously taking notes during the interviews. The interviews lasted 30-60 minutes for NDO participants and 45-90 minutes for DO participants. 

\subsection{Stage 2: Large-Scale Surveys} 
After gathering sufficient initial results from the interviews, we created an online survey for distributing to a larger pool of participants. The survey platform used was Qualtrics. 

\subsubsection{Survey Participants}
The survey had 100 participants divided into two primary groups. Group BVI consisted of 42 blind or visually impaired participants, and Group ST consisted of 58 sighted participants. \textbf{Table}~\ref{tab:survey-demographics} shows the demographic information of the survey participants. 

\subsubsection{Question Differentiation} 
Based on their responses to initial qualifying questions, survey participants were sorted into three subgroups: DO, NDO, and ST. Each participant was assigned one of three different versions of the survey. The surveys for BVI participants mirrored the interview categories (overall impressions, communication methods, functionalities, and form factors), but with a more quantitative approach rather than the open-ended questions used in interviews. The DO version included additional questions pertaining to their prior experience with dog guides. The ST version revolved around the participants' prior interactions with and feelings toward dog guides and dogs in general, their thoughts on a robot dog guide, and broad opinions on the aesthetic component of the robot's design. 

\section{Experiments}
\label{sec:exp}
\subsection{Experimental settings}

\noindent\textbf{Benchmark.}  We conduct experiments on two established 3D occupancy benchmarks: (i) nuScenes~\cite{nuScenes}, which provides instance-level annotations with manually labeled 3D bounding boxes (position/size/orientation) for dynamic objects, and (ii) Occ3D~\cite{Occ3D}, which generates voxel-level occupancy labels (0.4m resolution) through automated LiDAR point cloud aggregation and mesh reconstruction, including occlusion states. Both benchmarks share identical scene configurations of 1,050 driving scenes, each containing up to 40 timestamped frames. Every frame includes six synchronized camera views (front, front-left, front-right, back, back-left, back-right) at 1600$\times$900 resolution. In our experiments, we extend single-frame baselines~\cite{MonoScene,surroundOcc,viewformer} by aggregating features from $N$ historical keyframes. Additionally, we extract unlabeled intermediate frames from the ``sweeps'' folder~\cite{nuScenes} to provide implicit motion cues, enabling self-supervised temporal consistency learning.

\noindent\textbf{Implementation details.} For the nuScenes benchmark~\cite{nuScenes}, we follow the parameter settings of SurroundOcc~\cite{surroundOcc}, using $Cam=6$, $p=6$, $v=32$,$L=116$, and $W=200$. For the Occ3D benchmark~\cite{Occ3D}, we adopt ViewFormer's~\cite{viewformer} standard setup with $Cam=6$, $p=6$, $v=32$, $L=32$, and $W=88$. The output of the occupancy result on both benchmarks is formatted into a vector with dimensions $[200, 200, 16]$. In this vector, the first two dimensions (200 and 200) represent the length and width, while the third (16) indicates the height. The occupancy result covers a range from -50 meters to 50 meters in both width and length, and the vertical height varies from -5 meters to 3 meters. Each voxel corresponds to a cube measuring 0.5 meters on each side. Occupied voxels are categorized into one of 17~\cite{nuScenes,surroundOcc} and 18~\cite{Occ3D} semantic classes.
More details on implementation can be found in the supplementary material.

\subsection{Evaluation Metrics}

To validate the temporal consistency and occupancy accuracy of moving and static objects, objects are divided into two general classes~\cite{Cam4docc}: General Moving Objects (GMO) and General Static Objects (GSO). Detailed classification classes are introduced in the supplementary material.

\noindent\textbf{Occupancy Accuracy Metric.} To ensure rigorous evaluation across different benchmarks, we employ both Intersection over Union (IoU) and Mean Intersection over Union (mIoU) metrics. These metrics are widely adopted in 3D semantic occupancy prediction tasks~\cite{PASCAL, Microsoft_COCO, Cityscapes_dataset, Mask_R_CNN}. The mIoU are calculated separately for three category groups: All classes, GMO classes, and GSO classes.

\noindent\textbf{Temporal Consistency Metric.} \label{para:consistency_metric} To evaluate the effect achieved by integrating \ours\ with baseline models, we propose a temporal consistency metric. We aim to detect and measure changes in a scene from one frame to the next. This metric reflects the stability of prediction results, which directly impacts the user's visual experience. Let $\sigma_{i,n}^{(x,y,z)}$ denote the semantic label of the $n$-th voxel point (with coordinates $(x,y,z)$) in frame $i$, and define the indicator function $\delta(e_1,e_2) = \mathbb{I}(e_1 \neq e_2)$. 

In the occupancy results of frames $i$ and $j$, voxels at corresponding positions may undergo changes, which are categorized into two types: ``Static Object Change"~(SOC) and ``Moving Object Change"~(MOC). The definitions of these changes are as table \ref{tab:moc-soc}.

\begin{wrapfigure}[7]{l}{80mm}
\centering
% \setlength{\tabcolsep}{8pt}
\captionsetup{type=table}
    \begin{tabular}{c|c}
    \toprule
    \textbf{Type} & \textbf{Condition} \\  
    \midrule
    MOC & $\sigma_{i,n}^{(x,y,z)} \in \text{GMO} \lor \sigma_{j,n}^{(x,y,z)} \in \text{GMO}$ \\
    % \midrule
    SOC & $\sigma_{i,n}^{(x,y,z)} \wedge \sigma_{j,n}^{(x,y,z)} \in \text{GSO}$ \\  
    \bottomrule
    \end{tabular}
    \vspace{8pt}
    \caption{Definition of MOC and SOC. $N_{mc}$/$N_{sc}$ denote the number of MOC/SOC voxels, respectively.}
    \label{tab:moc-soc}
    % \vspace{-10pt}
\end{wrapfigure}

Based on these definitions, we can define disparity metrics~($\Delta_{m}$/$\Delta_{s}$) to quantify temporal inconsistencies across frames~($i$ and $j$). The process is defined as:
\begin{equation}
\begin{dcases}
\Delta_{m}(i,j) = \dfrac{1}{N_{mc}} \sum\limits_{n=1}^{N_{mc}} \delta\left(\sigma_{i,n}^{(x,y,z)}, \sigma_{j,n}^{(x,y,z)}\right) \\
\Delta_{s}(i,j) = \dfrac{1}{N_{sc}} \sum\limits_{n=1}^{N_{sc}} \delta\left(\sigma_{i,n}^{(x,y,z)}, \sigma_{j,n}^{(x,y,z)}\right).
\end{dcases}
\label{eq:disparity}
\end{equation}

The temporal consistency metrics -- $S_m$ (moving) and $S_s$ (static) -- are derived through aggregation of $\Delta_{m}$ and $\Delta_{s}$ across sequential frames. Formally, we have:
\begin{equation}
S_{m/s} = 1 - \dfrac{1}{M-1} \sum\limits_{k=1}^{M-1} \Delta_{m/s}(k,k+1),
\label{eq:consistency_scores}
\end{equation}
where $M$ is the scene's total frame count. Final metrics $\overline{S_m}$/$\overline{S_s}$ average across all scenes. A higher temporal consistency score indicates that the predictions within the scene are smoother and more consistent over time.

\begin{table}[t]
    % \centering
    \footnotesize 
    \begin{minipage}[t]{0.48\textwidth}
        \centering
        % \captionsetup{justification=centering, singlelinecheck=false}
        \setlength{\tabcolsep}{2pt}
        \renewcommand{\arraystretch}{1.25}
        \begin{tabular}{r|cccc|cc}
            \toprule
            \multicolumn{1}{r|}{\multirow{2}{*}[-0.4em]{Method}} & \multicolumn{1}{c|}{\multirow{2}{*}[-0.4em]{IoU~$\uparrow$}} & \multicolumn{3}{c|}{mIoU~$\uparrow$} & \multicolumn{1}{c}{\multirow{2}{*}[-0.4em]{$\overline{S_m}\uparrow$}} & \multicolumn{1}{c}{\multirow{2}{*}[-0.4em]{$\overline{S_s}\uparrow$}} \\ \cmidrule(lr){3-5}
            \multicolumn{1}{c|}{} & \multicolumn{1}{c|}{} & All & GMO & GSO & \multicolumn{1}{c}{} & \multicolumn{1}{c}{} \\ 
            \midrule        
            Atlas~\cite{Atlas} & 28.66 & 15.00 & 12.64 & 17.35 & \text{--} & \text{--}  \\
            BEVFormer{~\cite{BEVFormer}} & 30.50 & 16.75 & 14.17 & 19.33 & \text{--} & \text{--}  \\
            TPVFormer~\cite{TPVFormer} & 30.86 & 17.10 & 14.04 & 20.15 & \text{--} & \text{--} \\
            BEVDet4D-Occ~\cite{bevdet4d} & 24.26 & 14.22 & 11.10 & 17.34 & \text{--} & \text{--} \\
            MonoScene~\cite{MonoScene} & 10.04 & 1.15 & 0.24 & 2.07 & 46.53 & 81.77 \\
            % Cam4DOcc~\cite{Cam4docc} & 23.92 & 7.12 & 4.71 & 10.17 & 60.34 & 91.15 \\
            SurroundOcc~\cite{surroundOcc} & 31.49 & 20.30  & \cellcolor{gray!20}18.39 & 22.20 & 58.33 & 91.71 \\
            \midrule
            \makecell[r]{MonoScene \\ \textbf{+\ours}} & \makecell{13.10\\\textbf{+3.06}} & \makecell{1.69\\\textbf{+0.54}}  & \makecell{0.34\\\textbf{+0.10}} & \makecell{3.04\\\textbf{+0.98}} & \makecell{54.21\\\textbf{+7.68}} & \makecell{83.84\\\textbf{+2.07}} \\
            \midrule
            \makecell[r]{SurroundOcc \\ \textbf{+\ours}} & \cellcolor{gray!20}\makecell{33.12 \\ \textbf{+1.63}} & \cellcolor{gray!20}\makecell{20.67\\\textbf{+0.37}}  & \makecell{18.26\\-0.13} & \cellcolor{gray!20}\makecell{23.08\\\textbf{+0.88}} & \cellcolor{gray!20}\makecell{60.64\\\textbf{+2.31}} & \cellcolor{gray!20}\makecell{92.54\\\textbf{+0.83}} \\
            \bottomrule
        \end{tabular}
        \vspace{2mm}
        \caption{Occupancy prediction accuracy on \textbf{nuScenes benchmark~\cite{nuScenes}}. For a fair comparison, we ensure that all models have uniform input data. The best performance is highlighted in gray.}
        \label{tab:main-res-a}
    \end{minipage}\hfill
    \begin{minipage}[t]{0.48\textwidth}
        \centering
        % \captionsetup{justification=centering, singlelinecheck=false}
        \setlength{\tabcolsep}{2pt}
        \begin{tabular}{r|cccc|cc}
            \toprule
            \multicolumn{1}{r|}{\multirow{2}{*}[-0.4em]{Method}} & \multicolumn{1}{c|}{\multirow{2}{*}[-0.4em]{IoU~$\uparrow$}} & \multicolumn{3}{c|}{mIoU~$\uparrow$} & \multicolumn{1}{c}{\multirow{2}{*}[-0.4em]{$\overline{S_m}\uparrow$}} & \multicolumn{1}{c}{\multirow{2}{*}[-0.4em]{$\overline{S_s}\uparrow$}} \\ \cmidrule(lr){3-5}
            \multicolumn{1}{c|}{} & \multicolumn{1}{c|}{} & All & GMO & GSO & \multicolumn{1}{c}{} & \multicolumn{1}{c}{} \\ 
            \midrule    
            MonoScene~\cite{MonoScene} & \text{--} & 6.06 & 5.36 & 6.68 & \text{--} & \text{--} \\
            OccFormer~\cite{OccFormer} & \text{--} & 21.93 & 21.78 & 22.06 & \text{--} & \text{--} \\
            % CTF-Occ~\cite{Occ3D} & 28.53 & 27.42 & 29.52 & \text{--} & \text{--} \\
            FB-OCC~\cite{fb_occ} & \text{--} & 39.11  & 33.74 & 43.88 & \text{--} & \text{--} \\
            SparseOcc~\cite{SparseOcc_Liu} & \text{--} & 30.10  & \text{--} & \text{--} & \text{--} & \text{--} \\
            BEVDet4D-Occ~\cite{bevdet4d} & \text{--} & 39.30  & 29.09 & 42.16 & \text{--} & \text{--} \\ 
            OPUS-L~\cite{opus} & \text{--} & 36.20  & 31.25 & 40.44 & \text{--} & \text{--} \\      
            SurroundOcc~\cite{surroundOcc} & 51.89 & 7.24  & 0.36 & 13.35 & 65.35 & 89.54 \\
            ViewFormer~\cite{viewformer} & 70.39 & 40.46  & 33.73 & 46.45 & 67.26 & 86.06 \\
            \midrule
            \makecell[r]{SurroundOcc \\ \textbf{+\ours}} & \makecell{52.13\\\textbf{+0.24}}& \makecell{10.33\\\textbf{+3.09}}  & \makecell{1.98\\\textbf{+1.62}} & \makecell{17.76\\\textbf{+4.41}} & \makecell{69.60\\\textbf{+4.25}} & \cellcolor{gray!20}\makecell{90.91\\\textbf{+1.37}} \\
            \midrule
            \makecell[r]{ViewFormer \\ \textbf{+\ours}} & \cellcolor{gray!20}\makecell{70.63\\\textbf{+0.24}} & \cellcolor{gray!20}\makecell{41.30\\\textbf{+0.84}}  & \cellcolor{gray!20}\makecell{34.33\\\textbf{+0.60}} & \cellcolor{gray!20}\makecell{47.50\\\textbf{+1.05}} & \cellcolor{gray!20}\makecell{70.13\\\textbf{+2.87}} & \makecell{87.10\\\textbf{+1.04}} \\
            \bottomrule
        \end{tabular}
        \vspace{2mm}
        \caption{Occupancy prediction accuracy on \textbf{Occ3D benchmark~\cite{Occ3D}}. For a fair comparison, we ensure that all models have uniform input data. The best performance is highlighted in gray.}
        \label{tab:main-res-b}
    \end{minipage}
    \vspace{-3mm}
    % \caption{Occupancy prediction accuracy on two benchmarks. The best performance is highlighted in gray.}
    \label{tab:main-res}
    \vspace{-5mm}
\end{table}

\subsection{Comparison Results}

\noindent\textbf{Occupancy accuracy on nuScenes.} We compare our method against several SOTA models, including Atlas~\cite{Atlas}, BEVFormer~\cite{BEVFormer}, TPVFormer~\cite{TPVFormer}, MonoScene~\cite{MonoScene}, and SurroundOcc~\cite{surroundOcc}. For a fair comparison, all methods are trained on the same ground truth and follow the same training procedure. By combining methods such as MonoScene~\cite{MonoScene} and SurroundOcc~\cite{surroundOcc} with \ours, we evaluate the effect of \ours\ in performance enhancement. The results presented in \cref{tab:main-res-a} show that our performance improvement is significant. Notably, the incorporation of \ours\ into SurroundOcc~\cite{surroundOcc} has led to improved metrics that surpass those of all other models listed in this table. The results are improved by 1.63\% and 0.37\% compared with SurroundOcc~\cite{surroundOcc} in IoU and mIoU~(All), respectively.

\noindent\textbf{Occupancy accuracy on Occ3D.} We also conduct experiments on Occ3D~\cite{Occ3D} in \cref{tab:main-res-b}. To validate \ours, we conducted two sets of experiments: First, integrating \ours\ with the 3D VONs~\cite{surroundOcc,viewformer} improved one of the original models'~\cite{viewformer} performance by 0.24\% in IoU and 0.84\% in mIoU. Second, \ours\ consistently outperforms existing history-aware VONs~\cite{opus,bevdet,SparseOcc_Liu,fb_occ} by over 2\% mIoU, demonstrating the efficacy of the \ours.


\noindent\textbf{Temporal Consistency.} The results of $\overline{S_m}$ and $\overline{S_s}$ shown in \cref{tab:main-res} indicate that the integration of \ours\ improved the temporal consistency of occupancy across all frames in all scenes for all models, demonstrating \ours' effectiveness. This enhancement can be attributed to the incorporation of previous keyframes from the dataset~\cite{nuScenes,Occ3D}, along with the addition of intermediate frames from the ``sweeps''~\cite{nuScenes} directory for the SFE and MFE modules. These elements provide critical historical information and motion clues for the model.

\subsection{Ablation study}

Our ablation experiments are all conducted on the nuScenes benchmark~\cite{nuScenes}. The results are presented in~\cref{tab:ablation-studies}.

\begin{wrapfigure}{l}{90mm}
\centering
\captionsetup{type=table}
    \begin{subtable}[t]{0.48\textwidth}
    \centering
    \footnotesize
    \begin{tabular}{c|ccc|cccc}
    \toprule 
    Idx. & Pre & Cur & Mid & IoU$\uparrow$ & mIoU$\uparrow$ & $\overline{S_m}\uparrow$ & $\overline{S_s}\uparrow$ \\
    \midrule
    \textbf{M0} & \ding{55} & \ding{55} & \ding{55} & 31.49 & 20.30 & 58.33 & 91.71 \\
    \textbf{M1} & \ding{55} & \ding{51} & \ding{51} & 33.04 & 20.04 & 60.59 & 92.25 \\
    \textbf{M2} & \ding{51} & \ding{55} & \ding{51} & 33.05 & 19.98 & 60.09 & 92.44 \\
    \textbf{M3} & \ding{51} & \ding{51} & \ding{55} & 32.88 & 20.10 & 60.24 & 92.24 \\
    \textbf{M4} & \ding{51} & \ding{51} & \ding{51} & 31.97 & 20.11 & 60.19 & 92.01 \\
    \textbf{M5} & \ding{51} & \ding{51} & \ding{51} & \textbf{33.12} & \textbf{20.67} & \textbf{60.64} & \textbf{92.54} \\
    \bottomrule
    \end{tabular}
    \vspace{1mm}
    \caption{Ablation study of \ours. \textbf{Cur}, \textbf{Pre} and \textbf{Mid} represent the $f_{cur}$, $f_{pre}$ and $f_{mid}$ input, respectively, in the MSI.}
    \label{tab:ablation-modules}
    \end{subtable}
% \hfill
    \vspace{2mm}
    
    \begin{subtable}[t]{0.48\textwidth}
    \centering
    \footnotesize
    \setlength{\tabcolsep}{7pt}
    \begin{tabular}{c|c|cccc}
    \toprule 
    Idx. & \makecell{Type of\\motion info.} & IoU~$\uparrow$ & mIoU~$\uparrow$ & $\overline{S_m}\uparrow$ & $\overline{S_s}\uparrow$ \\
    \midrule
    \textbf{I0} & -  & 31.49 & 20.30 & 58.33 & 91.71 \\
    \textbf{I1} & Raw Image  & 32.39 & 19.45 & 59.01 & 91.15 \\
    \textbf{I2} & Optical Flow  & 32.80 & 20.27 & 60.53 & 92.13 \\
    \textbf{I3} & Frame Diff.  & \textbf{33.12} & \textbf{20.67}  & \textbf{60.64} & \textbf{92.54} \\
    \bottomrule
    \end{tabular}
    \vspace{1mm}
    \caption{Effect of different types of motion information.}
    \label{tab:ablation-motion}
    \end{subtable}
    % \vspace{-2mm}
\caption{Ablation studies on \ours\ modules and motion information. Best results are \textbf{bolded}.}
\label{tab:ablation-studies}
\end{wrapfigure}

\noindent\textbf{Different combinations of \ours.} \label{para:aba-comb}\cref{tab:ablation-modules} presents the performance results of different combination of \ours's components for $N$=$1$. In \cref{tab:ablation-modules}, there are 6 different combinations: \textbf{M0} shows results from SurroundOcc~\cite{surroundOcc}, which represents the basic model without our method. \textbf{M1} means the model variant in which the part responsible for processing previous keyframes is removed, thereby excluding the input data \(F_{pre}^{1}\). \textbf{M2} refers to the model variant that omits the current feature \(F_{cur}\). \textbf{M3} indicates the model configuration that has \(F_{mid}^{1}\) removed. \textbf{M4} indicates that $I_{cur}$ is used to compute MHAM's query, while $I_{pre}$ and $I_{mid}$ are utilized to compute MHAM's key and value, which differs from the standard design. \textbf{M5} represents the full model with all components included.
% 验证了我们三种输入的必要性
\cref{tab:ablation-modules} clearly demonstrates that the removal of any single input from \ours\ module significantly reduces performance both in prediction accuracy and in temporal consistency. This validates the necessity of the three inputs. Furthermore, the comparison between \textbf{M4} and \textbf{M5} confirms that the cues provided by the previous keyframes and the intermediate frames are crucial for occupancy prediction.

\noindent\textbf{Impact of different types of motion information.} \label{para:motion-extracting} This experiment was conducted on the MFE module to investigate the effects of various types of motion information for $N=1$. The results are presented in \cref{tab:ablation-motion}. Specifically, \textbf{I0} served as the base model~\cite{surroundOcc} without using any motion information. \textbf{I1} employed raw intermediate frames as the input for the MFE. \textbf{I2} used optical flow~\cite{OpticalFlow} as the motion information input. \textbf{I3} used frame difference~\cite{Frame_difference} to capture motion information. It is clear that \textbf{I1} surpasses \textbf{I0} in terms of IoU metrics; however, it exhibits the lowest performance in mIoU, $\overline{S_m}$, and $\overline{S_s}$ metrics compared with \textbf{I1}, \textbf{I2}, and \textbf{I3}. This discrepancy is mainly because of the substantial amount of irrelevant information in the raw, intermediate frames, which complicates the extraction of motion features by the MFE. In addition, the results show that \textbf{I3} significantly outperforms \textbf{I2} in both IoU and mIoU metrics and slightly improves in $\overline{S_m}$ and $\overline{S_s}$ metrics. This indicates that frame difference more effectively captures sudden changes in a scene, such as the abrupt appearance of pedestrians or vehicles exiting intersections, while optical flow may experience delays in processing these sudden events. Furthermore, given the lightweight design of \ours, the frame difference method~\cite{Frame_difference} reduces data processing complexity by only processing simple differential data, thereby contributing to computing speed.

\begin{wrapfigure}[22]{l}{90mm}
    \centering
    \includegraphics[width=90mm]{assets/case_single_light.png}
    \caption{
    Several challenging scenarios are presented: pedestrians are partially occluded by vehicles in the first and second columns, and the road boundary appears visually obscure in the third column. Ours achieves more accurate predictions, while SOTA methods display significant artifacts.
    }
    \label{fig:case-extra}
\end{wrapfigure}

\noindent\textbf{Impact of different numbers of previous keyframes.}\label{para:ntrack} We conduct ablation experiments on $N$ to explore the performance of the model when $N$=$0$, $N$=$1$ and $N$=$2$. $N$=$0$ represents SurroundOcc~\cite{surroundOcc}, which does not use any previous keyframes. Detailed experiment results are documented in the supplementary material.

\begin{figure}[!t]
\centering
% \fbox{\rule{0pt}{2in} \rule{0.9\linewidth}{0pt}}
\includegraphics[width=\linewidth]{assets/case_big_light.png}
% \vspace{-7mm}
\caption{
Comparison under a T-junction scenario, where a pedestrian is partially and dynamically occluded in certain frames. Ours showcases robust predictions, with the pedestrian being consistently tracked, while SOTA methods show a flickering phenomenon.
}
\label{fig:case-study-big}
\vspace{-20pt}
\end{figure}

\subsection{Case analysis}
To visually evaluate the effectiveness of our method~(SurroundOcc+\ours), we compare it with the SOTA 3D VONs~\cite{surroundOcc} and the SOTA history-aware VONs~\cite{bevdet4d}.

\noindent\textbf{Temporal visualization case.} As shown in~\cref{fig:case-study-big} (Scene 277, Frames \#7-\#11), a pedestrian traversing the sidewalk parallel to the ego-motion trajectory is intermittently occluded by roadside vegetation. SurroundOcc~\cite{surroundOcc} exhibits severe instability in predictions (missing in Frames \#7/\#9), revealing fundamental limitations in temporal modeling. BEVDet4D-Occ~\cite{bevdet4d} alleviates this issue through data fusion but still suffers from occasional inconsistencies, such as detection dropout in Frame \#8. In contrast, our method completely eliminates flickering artifacts and maintains consistent detection across all occlusion states.

\noindent\textbf{Extra single frame visualization case.} ~\cref{fig:case-extra} highlights challenging scenarios: 
(i) Vehicle-pedestrian occlusion (Scene-0911 Frame \#15, Scene-0928 Frame \#14): Both SurroundOcc~\cite{surroundOcc} and BEVDet4D-Occ~\cite{bevdet4d} fail to recover the occluded pedestrian’s occupancy, while our method successfully localizes the target with precise geometry.
(ii) Curved road prediction (Scene-0923 Frame \#28): Our approach correctly anticipates the right-turn road geometry where baselines produce fragmented or erroneous occupancy, achieving superior shape consistency with real-world conditions.




\subsection{Overhead analysis}

For a fair comparison, all overhead analysis experiments are performed on a single NVIDIA L20 GPU.

\begin{figure}[htbp]
    \centering
    \begin{minipage}{0.48\textwidth}
        \footnotesize
        \begin{tabular}{r|ccc}
            \toprule
            Model & mIoU~$\uparrow$ & \makecell{Memory (MB)\\ Train~/~Test}~$\downarrow$ & Latency~$\downarrow$ \\
            \midrule
            FB-Occ~\cite{fb_occ} & 39.11 & 32,915~/~5,933 & 0.09s \\
            % SparseOcc~\cite{SparseOcc_Liu} & 30.10 & $>$49,140~/~7,147 & \cellcolor{Gray} 0.05s \\
            OPUS-L~\cite{opus} & 36.20 & OOM~/~10,579 & 0.16s \\
            OPUS-T~\cite{opus} & 33.20 & 48,532~/~6,711 & \textbf{0.03s} \\
            BEVDet4D-Occ~\cite{bevdet4d} & 39.30 & 22,833~/~4,689 & 0.26s \\
            \midrule
            ViewFormer+Ours & \textbf{41.30} & \textbf{16,619~/~4,687} & 0.12s \\
            \bottomrule
        \end{tabular}
        \vspace{2mm}
        \caption{Comparison of computational overhead. All models are benchmarked with ResNet-50 backbones. Our result (ViewFormer+\ours) in this table is measured for $N = 1$. OOM indicates out of CUDA memory. Best results are \textbf{bolded}.}
        \label{tab:efficiency}
    \end{minipage}\hfill
    \begin{minipage}{0.48\textwidth}
        \centering
        \includegraphics[width=\linewidth]{assets/bubble_0304.png}  % 替换成你的图片
        % \vspace{-8mm}
        \caption{Comparison of memory and latency overheads. Lower-left positions indicate superior performance with reduced memory consumption and faster inference. Large circles indicate better mIoU quality.}
        \label{fig:bubble}
    \end{minipage}
\end{figure}



As illustrated in \cref{tab:efficiency} and \cref{fig:bubble}, we conducted a comparative study to evaluate the computational overhead of our model against existing temporal methods~\cite{bevdet4d,opus,fb_occ}. The analysis focuses on GPU memory consumption during the training/testing phases and per-sample inference latency. The result shows that our method establishes an optimal accuracy-memory balance, achieving state-of-the-art mIoU while maintaining minimal GPU memory consumption alongside sustained computational efficiency that avoids runtime bottlenecks. For quantitative benchmarking, we compare two baseline frameworks:
\begin{itemize}
    \item ViewFormer on Occ3D: (i) Training memory: ViewFormer+\ours\ requires 16 GB of GPU memory, with the \ours\ module consuming only 0.22 GB, accounting for \textbf{1.4\%} of total usage; (ii) Inference latency: Full sample processing takes 0.1218s, where \ours\ contributes merely 0.0043s, accounting for \textbf{3.5\%} of total computation.
    \item SurroundOcc on nuScenes: (i) Training memory: SurroundOcc+\ours\ consumes 39 GB of GPU memory, with \ours\ occupying only 0.69 GB, which is \textbf{1.8\%} of total memory; (ii) Inference latency: Complete sample inference requires 0.9200s, while \ours\ takes 0.0065s, contributing to \textbf{0.7\%} of total latency.
\end{itemize}

These measurements confirm that our architecture introduces negligible computational overhead while delivering competitive performance.
\section{Conclusion}
We introduce a novel approach, \algo, to reduce human feedback requirements in preference-based reinforcement learning by leveraging vision-language models. While VLMs encode rich world knowledge, their direct application as reward models is hindered by alignment issues and noisy predictions. To address this, we develop a synergistic framework where limited human feedback is used to adapt VLMs, improving their reliability in preference labeling. Further, we incorporate a selective sampling strategy to mitigate noise and prioritize informative human annotations.

Our experiments demonstrate that this method significantly improves feedback efficiency, achieving comparable or superior task performance with up to 50\% fewer human annotations. Moreover, we show that an adapted VLM can generalize across similar tasks, further reducing the need for new human feedback by 75\%. These results highlight the potential of integrating VLMs into preference-based RL, offering a scalable solution to reducing human supervision while maintaining high task success rates. 

\section*{Impact Statement}
This work advances embodied AI by significantly reducing the human feedback required for training agents. This reduction is particularly valuable in robotic applications where obtaining human demonstrations and feedback is challenging or impractical, such as assistive robotic arms for individuals with mobility impairments. By minimizing the feedback requirements, our approach enables users to more efficiently customize and teach new skills to robotic agents based on their specific needs and preferences. The broader impact of this work extends to healthcare, assistive technology, and human-robot interaction. One possible risk is that the bias from human feedback can propagate to the VLM and subsequently to the policy. This can be mitigated by personalization of agents in case of household application or standardization of feedback for industrial applications. 


{
    \small
    \bibliographystyle{ieeenat_fullname}
    \bibliography{main}
}



\clearpage
\setcounter{page}{1}
\setcounter{figure}{0}
\setcounter{table}{0}
\setcounter{equation}{0}

\appendix


\begin{center}
    {\Large \bfseries Supplementary Materials}\\[1em]
\end{center}




\begin{table*}[t]
    \centering
    \resizebox{1.0\linewidth}{!}{ %< auto-adjusts font size to fill line
    \setlength{\tabcolsep}{4.5pt}
         \begin{tabular}{l|ccccc}
        \toprule
        \textbf{Training Data \textbackslash~Test}  & \textbf{X\textsubscript{test}} & \textbf{M\textsubscript{test}} & \textbf{GC\textsubscript{test}} & \textbf{E\textsubscript{test}} & \textbf{G360\textsubscript{test}} \\
        \midrule
        \rowcolor{LightGray} \multicolumn{6}{l}{ResNet-50}  \\
         \textit{same-domain}       & 5.25  & 5.11   & 3.49  &  8.51 & 11.87\\
         \textit{leave-one-dataset-out}    & 16.31 (\increase{210.7}) & 6.23 (\increase{21.9}) & 6.35 (\increase{82.0}) & 8.25 (\reduce{3.1}) & 20.38 (\increase{71.7}) \\
         \textit{joint-dataset} & \textbf{5.04} (\reduce{4.0}) & 5.88 (\increase{15.1}) & 3.59 (\increase{2.9}) & \textbf{6.04} (\reduce{29.0}) & \textbf{10.55} (\reduce{11.1}) \\
         \midrule
        \rowcolor{LightGray}
        \multicolumn{6}{l}{\methodname-H} \\
        \textit{same-domain}   & 4.62 & 5.19 & 3.01 & 6.11 & 9.44  \\
         \textit{leave-one-dataset-out} & 11.29 (\increase{144.4}) & 5.22 (\increase{0.6}) & 5.13 (\increase{70.4}) & 6.14 (\increase{0.5}) & 13.12 (\increase{39.0}) \\
          \textit{joint-dataset} & \textbf{4.46} (\reduce{3.5}) & \textbf{5.08} (\reduce{2.1}) & 3.20 (\increase{6.3}) & \textbf{5.16} (\reduce{15.6}) & \textbf{9.07} (\reduce{3.9})  \\
        \bottomrule
        \end{tabular}
        }
    \caption{
        Comparison of different training data configurations for gaze estimation.
        Each column represents a specific test dataset: XGaze Test, MPIIFaceGaze Test, GazeCapture Test, EYEDIAP Test, and Gaze360 Test.
        Each row corresponds to a training configuration:
        \textit{Same-domain} means training on the same domain as the test set,
        \textit{leave-one-dataset-out} means training on the remaining four datasets other than the test set, and \textit{joint-dataset} means training on the aggregated Train split of all five datasets.
        The percentages in parentheses indicate the reduction or increment compared to the \textit{same-domain} results, where lower errors indicate better performance.
        For the \textit{leave-one-dataset-out} configuration, the errors reported here are on the Test splits, while the main paper reports errors on the entire dataset.
    }
\label{table:comprehensive}
\end{table*}



\noindent In this supplementary material, we first provide an analysis of the effect of combining multiple domains.
Then, we include additional ablations to investigate the effects of color-jitter augmentation and pixel normalization during the pre-training.
Finally, we present qualitative results, highlighting images captured under diverse and challenging conditions.



\section{Analysis on Combining Multiple Domains}\label{sec:supp_data}

We analyze the effect of different training data configurations on gaze estimation performance.
Specifically, we compare three configurations: training on the same domain (\textit{same-domain}), training on multiple domains excluding the testing domain (\textit{leave-one-dataset-out}), and training on multiple domains including the testing domain (\textit{joint-dataset}).

\Cref{table:comprehensive} shows the comparison of gaze errors for these configurations.
Each column corresponds to a specific test dataset: XGaze Test, MPIIFaceGaze Test, GazeCapture Test, EYEDIAP Test, and Gaze360 Test, while each row represents a training configuration.
This \textit{same-domain} setting is different from the \textit{within-dataset} in the main paper. 
We use the splits defined in Sec.~4.1 of the main paper.
Especially, please note that for MPIIFaceGaze dataset, we train the model on the first 10 subjects and test on the remaining five subjects, different from the typical leave-one-subject-out protocol~\cite{abdelrahman2023l2cs,shi2024agent,ververas20253dgazenet}.

The percentages in parentheses indicate the reduction or increment compared to the \textit{same-domain} results, where lower errors indicate better performance.
Note that, for the \textit{leave-one-dataset-out} configuration, errors on the entire left-out dataset are reported in the main paper, but here we present errors on the Test split to align with the other configurations that require dataset splits.



\paragraph{\textit{Same-domain}}
In general, training and testing on the same domain (\textit{same-domain}) yields the best results, even though datasets combined from multiple domains have the potential to be more diverse.
This emphasizes the persistent challenge of achieving optimal performance when using data from different domains.
The exception observed for E\textsubscript{test} with the ResNet-50 backbone may be attributed to the limited number of samples in the EYEDIAP Train split.

\paragraph{\textit{Leave-one-dataset-out}}
In the \textit{leave-one-dataset-out} configuration, we observe varying tendencies across different test datasets. 
Some datasets achieve errors comparable to the \textit{same-domain} results, while others remain challenging. 
For instance, for M\textsubscript{test} and E\textsubscript{test}, which are relatively less complex, the remaining four datasets provide sufficient information to achieve good performance.
In contrast, for X\textsubscript{test}, GC\textsubscript{test}, and G360\textsubscript{test}, the remaining four datasets fail to fully capture the critical factors required for optimal performance.
This variation highlights the strong dependence of performance on the attributes of the training data.

Importantly, our \methodname-H demonstrates a smaller performance gap compared to ResNet-50 in most cases, with the only exception being EYEDIAP, where the difference is marginal. 
This suggests that \methodname-H is better equipped to learn gaze representations from out-of-domain data with less overfitting, underscoring its enhanced generalization capability.


\paragraph{\textit{Joint-dataset}}

Overall, the \textit{joint-dataset} configuration demonstrates significant promise, creating a single model robust across multiple test domains. 
For \methodname-H, the only exception is GC\textsubscript{test}, where the \textit{joint-dataset} configuration produces a slightly higher error (3.01$\rightarrow$3.20). 
Although this suggests some negative effects from the other four datasets, the effects remain marginal.
While the improvement percentages for \methodname-H are smaller compared to ResNet-50, the absolute errors are consistently lower.


\begin{table}[t]
    \centering
    \resizebox{0.8\linewidth}{!}{ %< auto-adjusts font size to fill line
    \centering
    \setlength{\tabcolsep}{6pt}
        \begin{tabular}{ccc|cccc}
        \toprule
        \textbf{\textit{Real}} & \textbf{\textit{Syn.}} & \textbf{\textit{NV.}}  & \textbf{M} & \textbf{GC} & \textbf{E} & \textbf{G360} \\
        \midrule
        \checkmark &            &            & 6.79 & 7.81 & 6.86 & 12.93 \\
        \checkmark & \checkmark &            & 6.57 & 7.37 & \textbf{6.51} & 13.23 \\
        \checkmark & \checkmark & \checkmark & \textbf{6.21} & \textbf{7.35} & 6.64 & \textbf{12.18} \\
        \bottomrule
        \end{tabular}
    }
    \caption{
    We ablate the pre-training facial datasets by comparing real, synthetic, and novel-rendered images.
    The comparison is performed on the \methodname-B network, followed by training on XGaze.
    The last row represents the full-dataset setting.
    }
\label{table:ablation_data_type}
\end{table}


\begin{figure*}[t]
  \centering
  \begin{subfigure}{0.99\linewidth}
    \centering
    \includegraphics[width=0.99\linewidth]{figs/supp/MAE_without_pixel_norm.pdf}
    \caption{ MAE reconstruction examples without pixel normalization.}
    \label{fig:no_pixel_norm}
  \end{subfigure}
  % \hfill
  \begin{subfigure}{0.99\linewidth}
    \centering
    \includegraphics[width=0.99\linewidth]{figs/supp/MAE_with_pixel_norm.pdf}
    \caption{ MAE reconstruction examples with pixel normalization (Proposed).}
    \label{fig:w_pixel_norm}
  \end{subfigure}
  \caption{Examples comparison of the pixel normalization during the MAE pre-training.
  The left, middle, and right columns show the original image, masked input, and the reconstructed image, respectively.
    }
  \label{fig:pixel_norm_samples}
\end{figure*}


\begin{figure}[t]
  \centering
    \includegraphics[width=0.9\linewidth]{figs/data_diversity/leave_perc_change.pdf}
  \caption{
        Effect of MAE pre-training dataset composition on downstream gaze estimation performance.  
        The horizontal axis represents the incremental accumulation of datasets, while the vertical axis shows the percentage reduction in error relative to the first CelebV-Text dataset~\cite{yu2023celebv}.
    }
  \label{fig:plot_err_vs_data_composition}
\end{figure}


\section{Additional Ablation Studies on Pre-Training}


\paragraph{Effect of Pre-Training Dataset Composition}


Beyond the overall pre-training dataset size, the composition of the dataset also plays a critical role in learning effective gaze representations.
To investigate the impact of different facial dataset components, we conduct an experiment where we incrementally accumulate datasets during the MAE pre-training stage and analyze their effect on the downstream gaze estimation performance.
Starting with CelebV-Text~\cite{yu2023celebv}, we progressively add datasets for pre-training and evaluate model separately on gaze estimation.
Each pre-trained model is subsequently trained on gaze datasets using the same \textit{leave-one-dataset-out} protocol.
\Cref{fig:plot_err_vs_data_composition} illustrates the error change across different test sets as more datasets are included in pre-training.



Overall, the results indicate that adding more diverse data during pre-training generally enhances gaze generalization.
However, there are exceptions that adding a model can result in increased error for specific test sets.
For example, adding VFHQ increases the error on the XGaze Test set from 12.65 to 13.32, while including SFHQ-T2I causes performance fluctuations across different benchmarks.
This suggests that certain dataset attributes may not align well with particular test distributions, leading to suboptimal transferability.
On the other hand, datasets such as VGGFace2 and XGaze-Dense provide performance improvements on most test sets.
Additionally, performance gains becomes marginal as the dataset number increases, aligning with the analysis of pre-training data size in main paper.


In conclusion, dataset diversity plays a crucial role in improving MAE pre-training for gaze estimation.
A more detailed analysis of dataset attributes and their impact on gaze estimation remains an open research question, which we leave for future work.
Nonetheless, our empirical results suggest that increasing data diversity in pre-training tends to improve model performance across various test domains.



\paragraph{Effect of Novel-View Synthesis Data in Pre-Training}\label{sec:ablation_novel_view_data}
To examine the effect of novel-view synthesis in pre-training data, we conduct further experiments separating these two elements.
In \cref{table:ablation_data_type}, we conduct an ablation study by varying data subsets during the pre-training: real datasets (CelebV-Text, VGGFace2, and VFHQ), synthetic datasets (FaceSynthetics and SFHQ-T2I), and novel-view-rendered datasets (FFHQ-NV and XGaze-Dense).
We use the \methodname-B to conduct the experiment due to its time efficiency.
After pre-training, we train on XGaze and test on the rest of the four datasets. 

The results further clarify the effect of different data types on the model's generalizability.
Adding synthetic data (\textit{Real + Syn.}) reduces errors in several test domains compared to using only real data, suggesting the variability of the synthetic data contributes to generalization.
Further incorporating novel-view data (\textit{Real + Syn. + NV}) provides additional performance gains, especially in head-pose generalization, likely due to the expanded range of facial orientations.
This finding supports the idea that a mix of real, synthetic, and novel-view data in MAE pre-training strengthens ViT’s representation learning.



\paragraph{Effect of Pixel Normalization}\label{sec:ablation_pixel_norm}
The patch normalization technique is applied during the MAE pre-training as suggested in~\cite{he2022masked} which is different from reconstructing the natural image, as shown in \cref{fig:pixel_norm_samples}.
We compare models pre-trained with and without patch normalization to investigate its impact.

\paragraph{Effect of Color-Jitter Augmentation}\label{sec:ablation_augmentation}
Color jittering introduces randomness in brightness, contrast, saturation, and hue to simulate diverse lighting conditions, enhancing the robustness of learned features.
We compare models pre-trained with and without color-jitter augmentation to investigate its impact.

\paragraph{Results}
We use the \methodname-B model as the backbone and compare different pre-training settings, followed by training on XGaze and testing on the remaining four datasets.
\Cref{table:ablation_colorjitter_pixelnorm} demonstrates that both color-jitter augmentation and pixel normalization contribute to improved gaze estimation performance, highlighting their benefits for the generalization of the pre-trained model.
Notably, pixel normalization consistently improves performance across all test datasets, aligning with the observations in the original MAE paper~\cite{he2022masked}, which showed that pixel normalization enhances representation learning.



\begin{table}[t]
    \centering
    \resizebox{0.97\linewidth}{!}{ %< auto-adjusts font size to fill line
    \centering
    \setlength{\tabcolsep}{6pt}
        \begin{tabular}{cc|cccc}
        \toprule
        \textbf{Color-Jitter} & \textbf{Pixel Norm.} & \textbf{M} & \textbf{GC} & \textbf{E} & \textbf{G360} \\
        \midrule
         \XSolidBrush & \XSolidBrush     & 7.52 & 8.01 &  8.56 & 14.14 \\
        \checkmark & \XSolidBrush   &  7.17 & 8.23 & 8.03 & 14.03 \\
        \XSolidBrush   & \checkmark &  7.18 & 7.94 & 8.05 & 13.66 \\
        \checkmark & \checkmark & \textbf{6.21} & \textbf{7.35} & \textbf{6.64} & \textbf{12.18} \\
        \bottomrule
        \end{tabular}
    }
    \caption{
        Ablation studies on the pre-training, comparing the effect of the color-jitter augmentation and the pixel normalization.
        During the gaze estimation training, we train the model using XGaze and test on the other four datasets to evaluate the generalizability.
    }
\label{table:ablation_colorjitter_pixelnorm}
\end{table}




\section{Comparison with the SOTAs}
3DGazeNet~\cite{ververas20253dgazenet} collects in-the-wild face images with pseudo gaze labels and applies multi-view synthesis to obtain an augmented dataset ITWG-MV.
To account for the difference in test data settings, we compare 3DGazeNet with \methodname-H separately in \cref{table:supp_sota_3dgn}.
The results demonstrate that \methodname-H outperforms 3DGazeNet in all domain generalization settings.

\paragraph{Re-implementation}
In the main paper, we compared our \methodname-H model with state-of-the-art (SOTA) methods using their reported results. 
It is important to note that minor discrepancies may arise due to differences in our data pre-processing compared to prior work~\cite{cheng2022puregaze,xu2023learning,zhao2024improving}. 
To ensure a fair comparison, we re-implemented ResNet-18 and PureGaze~\cite{cheng2022puregaze} using our pre-processed datasets, aligning them with the reported results~\cite{cheng2022puregaze,zhao2024improving}. 
The re-implementation results, alongside the reported values, are summarized in \cref{table:supp_sota}.

While minor differences exist between our re-implementation and the reported values, the improvements achieved by our \methodname-H model remain significant, demonstrating its superior performance across all domain generalization tasks. 



\begin{table}[t]
\begin{center}
    \resizebox{0.99\linewidth}{!}{ %< auto-adjusts font size to fill line
    \setlength{\tabcolsep}{4pt}
        \begin{tabular}{l|cccc}
        \toprule
        \textbf{Models}  & \textbf{X}$\rightarrow$\textbf{M} & \textbf{X}$\rightarrow$\textbf{GC}& \textbf{G360}$\rightarrow$\textbf{M} & \textbf{G360}$\rightarrow$\textbf{GC} \\
        \midrule
        \rowcolor{LightGray}
        3DGazeNet$^{\dagger}$~\cite{ververas20253dgazenet} & 6.0 & 7.8 & 6.3 & 8.0  \\
        \midrule
        \methodname-H & \textbf{5.57} &  \textbf{6.56} & \textbf{5.43} & \textbf{6.48} \\
        \bottomrule
        \end{tabular}
}
\end{center}
\caption{
    Domain generalization compared with SOTA methods. 
    The results marked with $^{\dagger}$ are directly cited from previous studies~\cite{ververas20253dgazenet}.
}
\label{table:supp_sota_3dgn}
\end{table}


\begin{table}[t]
\begin{center}
    \resizebox{0.99\linewidth}{!}{ %< auto-adjusts font size to fill line
    \setlength{\tabcolsep}{4pt}
        \begin{tabular}{l|cccc}
        \toprule
        \textbf{Models}  & \textbf{X}$\rightarrow$\textbf{M} & \textbf{X}$\rightarrow$\textbf{E\textsubscript{CS}}& \textbf{G360}$\rightarrow$\textbf{M} & \textbf{G360}$\rightarrow$\textbf{E\textsubscript{CS}} \\
        \midrule
        ResNet-18   & 7.57 & 9.54 & 9.24  & 8.07  \\
        \rowcolor{LightGray}
        ResNet-18$^{\dagger}$~\cite{zhao2024improving}  & 8.02 & 9.11 & 8.04 & 9.20 \\
        PureGaze & 6.68 & 7.62 & 8.87 & 10.53 \\ %% the paper used pure50 for XGaze, pure18 for Gaze360
        \rowcolor{LightGray}
        PureGaze$^{\dagger}$~\cite{cheng2022puregaze}  & 7.08 & 7.48 & 9.28 & 9.32 \\
        \midrule
        \methodname-H & \textbf{5.57} &  \textbf{4.65} & \textbf{5.43} & \textbf{5.35} \\
        \bottomrule
        \end{tabular}

}
\end{center}
\caption{
    Domain generalization compared with SOTA methods and their re-implementations. 
    The results marked with $^{\dagger}$ are cited from previous studies~\cite{cheng2022puregaze,zhao2024improving}, and the rest of the results are based on our implementation.
}
\label{table:supp_sota}
\end{table}




\section{Implementation Details}

\paragraph{Novel-Rendered Data Preparation}

To render images from novel views, we follow the rendering approach described in~\cite{qin2022learning}.
To control the head pose, we randomly generate target head poses and compute the corresponding rotation matrices to apply to the 3D face models. 
During the rendering process, 40\% of the images are assigned a random background color, while the remaining 60\% use random scene images from the Places365 dataset~\cite{zhou2017places} as background. 
Additionally, to simulate varied lighting conditions, half of the rendered images are adjusted to have lower ambient light intensity, ranging from $0.2$ to $0.75$.

All face images in our method are in the size of \num{224} $\times$ \num{224} after the data normalization process~\cite{zhang2018revisiting}.
When the camera parameters are unknown, we use a camera matrix with focal length $f$ set to the image width and principal point $(c_x, c_y)$ set to half the image height and width.

\paragraph{Pre-Training}
We apply random color jitter augmentation with a probability of 0.5 and the following parameters: hue in the range $[-0.15, 0.15]$, saturation in $[0.8, 1.2]$, contrast in $[0.4, 1.8]$, and brightness in $[0.7, 1.3]$.
We apply random grayscale with a probability of 0.05 on all images.

\paragraph{Gaze Estimation Training}

We use the Adam optimizer~\cite{kingma2014adam} with a learning rate of \num{1e-4} and a weight decay of \num{1e-6} for all experiments.
For experiments with ResNet-50 and GazeTR-50, we set the batch size to 128 and decay the learning rate by 0.1 every five epochs, with a total of 12 epochs.
For cross-dataset evaluation with \methodname-H, we use a batch size of 128 and train the model for eight epochs with the one-cycle learning rate schedule~\cite{smith2019super}.
For \textit{leave-one-dataset-out} and \textit{joint-dataset} evaluations, we set the batch size to 160 with 12 epochs.

\section{Qualitative Results}
In this section, we present additional qualitative results using the \methodname-H model trained on the aggregated datasets under the \textit{joint-dataset} setting. 
We employ an off-the-shelf facial landmark detector~\cite{bulat2017far} to extract landmarks and perform data normalization.
Gaze estimation is conducted on the normalized images, and the results are de-normalized back to the original image for visualization.
For reference, we also include the normalized faces alongside the original images.

\Cref{fig:results_qual_wild_one_person} and \cref{fig:results_qual_wild} showcase examples from various in-the-wild videos captured under challenging conditions, including large head poses and diverse lighting environments.
Notably, we also include a synthetic example from URAvatar~\cite{li2024uravatar} (bottom row in \cref{fig:results_qual_wild}), which generates faces with controlled viewpoints and lighting.
Furthermore, \cref{fig:results_qual_VAT} presents examples from the gaze-following dataset VideoAttentionTarget~\cite{chong2020detecting}, a collection of diverse samples extracted from movies. 
This dataset provides annotated gaze targets, which are visualized when annotated within the image frame, as some targets may be out of frame.

These examples highlight the model's ability to predict gaze direction accurately in unseen environments, even under extreme head poses, challenging lighting conditions, and synthetic appearances.




\begin{table*}[ht!]
\centering
\footnotesize
\renewcommand{\arraystretch}{1.2}
\begin{tabular}{p{2.15cm} p{14.05cm}}
\toprule
 & \includegraphics[height=2.6cm]{figure/qualitative_example.pdf}\\
\midrule
Original \phantom{00000} Description & Bromazepam binds to the GABA receptor GABA-A, causing a conformational change and \textbf{increasing inhibitory effects of GABA}. Other neurotransmitters are not influenced. One of the \textbf{benzodiazepines} that is \textbf{used in the treatment of anxiety disorders.} \\
\midrule
User & Explain the features of the given molecule that are most distinct from those of other molecules. \\
\midrule
Mol-LLaMA (Ours) & The given molecule is a member of the \textcolor{tabblue}{benzodiazepine class}, which is characterized by a fusion of a benzene ring and a seven-membered diazepine ring. This structure is crucial for its biological activity, particularly its role as \textcolor{tabblue}{a sedative and hypnotic agent}. ...$<$omitted$>$...\\

& The given molecule is also characterized by a ketone group within the diazepine ring, which is a common feature in benzodiazepines. \textcolor{tabblue}{This functional group is important for the interaction with the GABA-A receptor, enhancing the inhibitory effects of GABA}. ...$<$omitted$>$...\\

\midrule
GPT-4o & 1. **Quinazoline Core**: The molecule contains a \textcolor{tabred}{quinazoline core}, ...$<$omitted$>$... \\
 & Overall, the combination of a \textcolor{tabred}{quinazoline} structure with unique substitution patterns, including bromine halogenation and the presence of a pyridinyl group, ...$<$omitted$>$... . \\
\midrule
LLaMo & The molecule has a \textcolor{tabblue}{benzodiazepine} structure with a bromo substituent at the 7-position and \textcolor{tabred}{a methyl group at the 1-position}. It is distinct from other molecules in that it contains a bromine atom, a nitrogen atom, and a methyl group, which are not present in the other molecules. \\
\midrule
3D-MoLM & ...$<$omitted$>$... It belongs to the class of compounds called \textcolor{tabred}{quinazolines}, which are characterized by a bicyclic structure consisting of a benzene ring fused to a pyrimidine ring. ...$<$omitted$>$...\\
& The molecule's structure suggests potential applications in medicinal chemistry, as quinazolines have been found to possess various biological activities, including \textcolor{tabred}{antitumor, antimicrobial, and anti-inflammatory properties.} \\
\midrule
Mol-Instructions & The molecule is a \textcolor{tabred}{quinoxaline derivative}. \\
\bottomrule
\end{tabular}
\vspace{-0.1in}
\caption{\small Case study to compare molecular understanding and reasoning ability. Mol-LLaMA accurately understands the molecular features, answering a correct molecular taxonomy and providing its distinct properties that are relevant to the given molecule.}
\label{tab:qualitative}
\vspace{-0.1in}
\end{table*}

\section{Ethical Considerations}

Our research involves the use of existing facial and gaze datasets.
In accordance with ethical guidelines, we rely on the fact that these datasets were originally collected and published following relevant ethical and data protection standards, including obtaining consent, and we do not generate or collect additional new data.
Our experimental protocols involve only image content, with no identifiable personal information or links to other personal data.




\end{document}
