\documentclass[sigconf]{acmart}
%\microtypecontext{spacing=nonfrench} %microtype package used by usenix
%\setlength {\marginparwidth }{2cm} 
\usepackage{algorithm}
\usepackage{algorithmic}
\usepackage{cryptocode}
\usepackage{multirow}
\usepackage{threeparttable}

\usepackage{subfig}

\usepackage{colortbl}
 
%\usepackage{todonotes}

\usepackage{amsmath,amsfonts}
\usepackage{amsthm}

\usepackage{graphicx}
\usepackage{tikz}
\usepackage{xspace}

% restatement of theorems
\usepackage{thmtools}
\usepackage{thm-restate}



\definecolor{ao}{rgb}{0.0,0.65,0.0}
\definecolor{lightred}{rgb}{0.95,0.45,0.5}

\newcommand{\tikzcircle}[2][red,fill=red]{\tikz[baseline=-0.5ex]\draw[#1,radius=#2] (0,0) circle ;}
%\newcommand{\rv}[1]{{\color{teal} [A: #1]}}
%\newcommand{\rvv}[1]{{\color{orange} [B: #1]}}
\newcommand\yuz[1]{{\color{orange}{\textbf{\{Yu: {\em#1}\}}}}}

\renewcommand{\algorithmicrequire}{\textbf{Input:}}
\renewcommand{\algorithmicensure}{\textbf{Output:}}
\renewcommand{\algorithmiccomment}{ // }

%!TEX root = gcn.tex
\def\ie{\textit{i.e.}\xspace} 
\def\etc{\textit{etc.}\xspace}
\def\eg{\textit{e.g.}\xspace}
\def\etal{\textit{et~al.}\xspace}
\def\cf{\textit{cf.}\xspace}

\definecolor{grayL}{RGB}{255,242,248}
\definecolor{colora}{RGB}{147,112,219}
\definecolor{colorb}{RGB}{219,112,147}

\def\textcgnn{VIRGOS}
\def\cgnn{\textnormal{\textsc{Virgos}}\xspace}
\def\osmm{\textnormal{(SM)$^2$}\xspace}

\def\oom{\mathbin{\text{\textcircled{M}}}}

\def\adjmat{\mathsf{A}}
\def\degmat{\mathsf{D}}
\def\actmat{\mathsf{H}}
\def\weimat{\mathsf{W}}
\def\feamat{\mathsf{X}}

\def\fucact{\mathsf{F}_{\mathsf{act}}}
\def\adjout{\adjmat_{\mathsf{out}}}
\def\adjin{\adjmat_{\mathsf{in}}}

\def\graph{\mathcal{G}}
\def\graphout{\mathcal{G}_{\mathsf{out}}}
\def\graphin{\mathcal{G}_{\mathsf{in}}}
\def\vertex{\mathcal{V}}
\def\edge{\mathcal{E}}
\def\numnode{|\vertex|}
\def\numedge{|\edge|}
\def\numnonzero{t}

%\def\filter{\mathsf{fil}}
%\def\eigenmat{\mathsf{U}}
%\def\normmat{\mathsf{I}}
\def\gracon{\mathsf{GConv}}
\def\node{\mathsf{Node}}

\def\ssp{\mathsf{OP}}
\def\SM{\mathsf{OSM}}

\def\OS{\mathsf{OS}}
\def\cop{\mathsf{CoP}}

\def\promult{\Pi_{\mathsf{Mult}}}
\def\smm{\mathsf{(SM)^2}}
\def\prosmm{\Pi_{\smm}}

\def\l{\langle}
\def\r{\rangle}

\def\type{\mathsf{type}}
\def\raw{\mathsf{plain}}
\def\shared{\mathsf{shared}}
\def\negl{\mathsf{negl}}

\def\bvec{\mathsf{B}}
\def\uvec{\mathsf{U}}
\def\xvec{\mathsf{X}}

\def\g{\mathbf{g}}
\def\bitlen{L} %\log \mathbb{R}

\def\pinput{\mathsf{input}}
\def\view{\mathsf{view}}

\def\ideal{\mathsf{Ideal}}
\def\real{\mathsf{Real}}
\def\Scal{\mathcal{S}}
\def\view{\mathsf{view}}
\def\output{\mathsf{output}}
\def\hyb{\mathsf{Hyb}}

\def\A{\mathcal{A}}
\def\F{\mathcal{F}}

\def\pp{\mathcal{P}}
\def\tt{\mathcal{T}}

\def\Rcal{\mathcal{R}}

\def\Abb{\mathbb{A}}
\def\Bbb{\mathbb{B}}
\def\Mbb{\mathbb{M}}
\def\Sbb{\mathbb{S}}
\def\Zbb{\mathbb{Z}}

\def\Asf{\mathsf{A}}
\def\Bsf{\mathsf{B}}
\def\Esf{\mathsf{E}}
\def\Msf{\mathsf{M}}
\def\Psf{\mathsf{P}}
\def\Qsf{\mathsf{Q}}
\def\Xsf{\mathsf{X}}
\def\Ysf{\mathsf{Y}}
\def\Zsf{\mathsf{Z}}

\def\prf{\mathsf{PRF}}
\def\key{\mathsf{key}}
\def\ctr{\mathsf{ctr}}

\def\nrow{n_{\mathsf{row}}}
\def\ncol{n_{\mathsf{col}}} 
\def\gammaout{\Gamma_{\mathsf{out}}}
\def\gammain{\Gamma_{\mathsf{in}}} 

% Define a compact column vector
% #1 - The repeated entry in the vector
% #2 - Number of repetitions
\newcommand{\cv}[2]{\overbrace{\begin{matrix} #1 \\ \vdots \\ #1 \end{matrix}}^{#2\ \text{times}}}

\iffalse
\makeatletter
% Redefine the font-size changing commands to throw errors
\renewcommand{\small}{\@latex@error{Don't use \string\small\space here}\@ehc}
\renewcommand{\scriptsize}{\@latex@error{Don't use \string\scriptsize\space here}\@ehc}
\renewcommand{\tiny}{\@latex@error{Don't use \string\tiny\space here}\@ehc}
\makeatother
\fi

%\captionsetup{skip=3pt}
 
\newtheorem{theorem}{Theorem}
\newtheorem{definition}{Definition}
\newtheorem{lemma}{Lemma}
%\newtheorem{proof}{Proof}

\newenvironment{protocol}[1][htb]{%
	\floatname{algorithm}{Protocol}% Update algorithm name
	\begin{algorithm}[#1]%
	}{\end{algorithm}}

\newenvironment{functionality}[1][htb]{%
	\floatname{algorithm}{Functionality}% Update algorithm name
	\begin{algorithm}[#1]%
	}{\end{algorithm}}

% Disable all inline math for bookmarks
\pdfstringdefDisableCommands{%
	\def\({}	% Disable opening math mode
	\def\){}	% Disable closing math mode
	\def\${}	% Ignore dollar signs used for math mode
	\def\_{sub }
}


 

\def\UrlBreaks{%
  \do\/%
  \do\a\do\b\do\c\do\d\do\e\do\f\do\g\do\h\do\i\do\j\do\k\do\l%
     \do\m\do\n\do\o\do\p\do\q\do\r\do\s\do\t\do\u\do\v\do\w\do\x\do\y\do\z%
  \do\A\do\B\do\C\do\D\do\E\do\F\do\G\do\H\do\I\do\J\do\K\do\L%
     \do\M\do\N\do\O\do\P\do\Q\do\R\do\S\do\T\do\U\do\V\do\W\do\X\do\Y\do\Z%
  \do0\do1\do2\do3\do4\do5\do6\do7\do8\do9\do=\do/\do.\do:%
  \do\*\do\-\do\~\do\'\do\"\do\-}

    \setcopyright{acmlicensed}
\copyrightyear{2025}
\acmYear{2025}
\acmDOI{XXXXXXX.XXXXXXX}
%% These commands are for a PROCEEDINGS abstract or paper.
\acmConference[Conference acronym 'XX]{Make sure to enter the correct
  conference title from your rights confirmation email}{June 03--05,
  2025}{Woodstock, NY}
\acmISBN{978-1-4503-XXXX-X/18/06}


%%
%% \BibTeX command to typeset BibTeX logo in the docs
\AtBeginDocument{%
  \providecommand\BibTeX{{%
    Bib\TeX}}}

%%
%% Submission ID.
%% Use this when submitting an article to a sponsored event. You'll
%% receive a unique submission ID from the organizers
%% of the event, and this ID should be used as the parameter to this command.
\acmSubmissionID{123-A56-BU3}

%\settopmatter{printfolios=true}
\begin{document}
\title{$\cgnn$: Secure Graph Convolutional Network on Vertically Split Data from Sparse Matrix Decomposition}
 
\author{Yu Zheng}
%\authornotemark[1]
\authornote{Yu and Qizhi share the co-first authorship.}
\email{yu.zheng@uci.edu}
 \affiliation{%
 University of California, Irvine
 % \institution{}
%  \city{Dublin}
%  \state{Ohio}
  \country{ }
}

\author{Qizhi Zhang}
\authornotemark[1]
\email{zqz.math@gmail.com}
\affiliation{%
Morse Team, Ant Group
 % \institution{Institute for Clarity in Documentation}
 % \city{Dublin}
%  \state{Ohio}
  \country{ }
}

\author{Lichun Li}
\email{lichun.llc@antgroup.com}
\affiliation{%
Morse Team, Ant Group
 % \institution{Institute for Clarity in Documentation}
 % \city{Dublin}
%  \state{Ohio}
  \country{ }
}

\author{Kai Zhou}
\email{kaizhou@polyu.edu.hk}
\affiliation{%
Hong Kong Polytechnic University
 % \institution{Institute for Clarity in Documentation}
 % \city{Dublin}
%  \state{Ohio}
  \country{ }
}

\author{Shan Yin}
\email{yinshan.ys@antgroup.com}
\affiliation{%
Morse Team, Ant Group
 % \institution{Institute for Clarity in Documentation}
 % \city{Dublin}
%  \state{Ohio}
  \country{ }
}
 
\begin{abstract}
Securely computing graph convolutional networks (GCNs) is critical for applying their analytical capabilities to privacy-sensitive data like social/credit networks. 
Multiplying a sparse yet large adjacency matrix of a graph in GCN---a core operation in training/inference---poses a performance bottleneck in secure GCNs. 
Consider a GCN with $\numnode$ nodes and $\numedge$ edges; it incurs a large $O(\numnode^2)$ communication overhead.

Modeling bipartite graphs and leveraging the monotonicity of non-zero entry locations, we propose a co-design harmonizing secure multi-party computation (MPC) with matrix sparsity.
Our sparse matrix decomposition transforms an arbitrary sparse matrix into a product of structured matrices.
Specialized MPC protocols for oblivious permutation and selection multiplication are then tailored, enabling our secure sparse matrix multiplication (\osmm) protocol, optimized for secure multiplication of these structured matrices.
Together, these techniques take $O(\numedge)$ communication in constant rounds.
Supported by \osmm, we present \cgnn\footnote{
Vertically-split Inference \& Reasoning on GCNs Optimized by Sparsity. 
}, a secure $2$-party framework that is communication-efficient and memory-friendly on standard vertically-partitioned graph datasets. 
Performance of \cgnn has been empirically validated across diverse network conditions. 
%\footnote{\label{footnote:artifact}\url{https://anonymous.4open.science/r/usenixsec-317-BD74}}
\end{abstract}

\begin{CCSXML}
<ccs2012>
   <concept>
       <concept_id>10002978.10002979</concept_id>
       <concept_desc>Security and privacy~Cryptography</concept_desc>
       <concept_significance>500</concept_significance>
       </concept>
   <concept>
       <concept_id>10010147.10010257</concept_id>
       <concept_desc>Computing methodologies~Machine learning</concept_desc>
       <concept_significance>500</concept_significance>
       </concept>
 </ccs2012>
\end{CCSXML}

\ccsdesc[500]{Security and privacy~Cryptography}
\ccsdesc[500]{Computing methodologies~Machine learning}
%%
%% Keywords. The author(s) should pick words that accurately describe
%% the work being presented. Separate the keywords with commas.
\keywords{Secure Sparse Matrix Computation, Secure Graph Learning, Secure Multiparty Computation.}
%% A "teaser" image appears between the author and affiliation
%% information and the body of the document, and typically spans the
%% page.



\maketitle
 

\section{Introduction}
\label{sec:intro}

Foundational models (FMs)~\cite{zhang2024data, zhou2023comprehensive} have shown remarkable progress in the healthcare domain, enabling professional-like assessment of disease diagnosis, treatment decision-making, and monitoring~\cite{zhang2023text, wang2022medclip, lu2023mi-zero}. 
Examples include LLaVA-Med~\cite{li2023llava}, Med-PaLM Multimodal~\cite{tu2024towards}, and Med-Flamingo~\cite{moor2023med}, have demonstrated their capacity on question answering, medical image analysis, and report generation.
These studies follow a predominant top-down model development strategy that requires upstream developers to collect data and train models for downstream tasks. 
Consequently, the developed model capabilities are heavily dependent on the training data, limiting their generalization performance in diverse clinical scenarios. 
For instance, Med-Gemini~\cite{yang2024advancing} reveals promising general capabilities in report generation while it lags behind state-of-the-art (SoTA) models on classification tasks, especially for out-of-domain applications. 
This indicates that while the generalizability of the foundation model is promising, more solutions are expected to meet the various specialized clinical needs.

To address these challenges, multi-center data centralization becomes essential to enhance model capacity and robustness across varied clinical scenarios~\cite{rajpurkar2022ai}. 
Centralizing distributed data can significantly improve model training and inference performance.
However, the process of medical data storage, transfer, and aggregation among centers requires extra efforts to ensure data security and system interoperability~\cite{bradford2020international}.
Moreover, a growing concern for patient privacy makes large-scale multi-center data sharing particularly challenging. 
While efforts like federated learning~\cite{wen2023survey, li2020review} can achieve good model performance on local data, the need for synchronized system coordination presents significant challenges, as clients are unable to update asynchronously. This limitation greatly restricts the practical capability of such approaches.
As a result, without a flexible collaboration, medical community still struggles to fully utilize the isolated data and local computation resources for comprehensive medical AI model development. 
To address this dilemma, open-source platforms encourage public data sharing and knowledge integration~\cite{markiewicz2021openneuro, zenodo}.
However, these platforms focus solely on raw data sharing while seldom providing collaborative model training or cooperation between different institutions.
Recently, collaborative learning has emerged as a viable approach for enhancing multi-model robustness~\cite{boulemtafes2020review}. 
For instance, software-like model development~\cite{raffel2023building} mimics software engineering practices by introducing structured workflows, enabling merging, version control, and continuous model integration.
Under this design, model ability can be strengthened with incremental knowledge updates similar to the version updating in software development. 

Although collaborative learning provides a multi-model collaboration, two key challenges remain in the leakage of raw data during collaboration~\cite{huang2023lorahub} and the synchronization of multiple collaborators~\cite{mcmahan2017communication} in the medical AI community. It is still challenging to integrate decentralized, privacy-sensitive data across institutions, leading to under-utilized insights and fragmented knowledge sharing~\cite{kaissis2020secure, rajpurkar2022ai, abdullah2021ethics}.
 To address these challenges, inspired by the collaborative software development, we propose \textbf{Med}ical \textbf{Fo}undation Models Me\textbf{rg}ing (\textbf{MedForge}), a cooperative workflow enabling continuously community-driven foundation model (FM) development.
MedForge enables a lightweight manner for individual centers to share their knowledge among multiple centers, minimizing the burden of data transmission and integration while enhancing model robustness.
Meanwhile, MedForge facilitates asynchronous and flexible collaboration, allowing individual centers to continuously update and improve medical FMs without the need for real-time synchronization.
Similar to open-source software development, MedForge incrementally updates medical knowledge and follows a sustainable model development scheme. 
This key design emphasizes a bottom-up construction of a multi-task medical FM, allowing downstream users to collaboratively build, refine, and update the upstream model according to their local resources. Our major contributions of MedForge are as below: 
\begin{enumerate}
    \item[$\bullet$] We introduce a collaborative workflow to promote the merging scheme of open-source software development. Our proposed MedForge allows distributed clinical centers to asynchronously contribute to comprehensive medical model construction while reducing transmitting costs among centers and avoiding the leakage of raw data, thus enhancing the utilization of private resources in the healthcare system. 
    \item[$\bullet$] We propose two effective knowledge-merging strategies for the asynchronous branch contribution. The MedForge-Fusion strategy updates the plugin module parameters of the main model during the merging phase, whereas the MedForge-Mixture strategy integrates the output of the plugin module by memorizing each contributor's coefficient. These strategies make MedForge more flexible and versatile. MedForge-Fusion is friendly to implement, while the MedForge-Mixture offers better performance and robustness.
    \item[$\bullet$]  We comprehensively evaluate model merging strategies to accumulate medical knowledge among multiple branch plugin modules. MedForge yields superior performance on medical classification tasks compared to other collaborative baselines across multiple datasets. We demonstrate the robustness of MedForge by shuffling the task order and evaluating various configurations of plugin modules and dataset distillation methods.
\end{enumerate}



\section{Preliminaries}

\paragraph{Task Formulation.}
\label{sec:task_form}
This work investigates how LLM-based agents tackle long-horizon tasks within specific environments through interactions.
Following previous studies~\citep{song2024trial, xiong2024watch}, we formalize these agentic tasks as a partially observable Markov decision process (POMDP), which contains the key elements ($\mathcal U, \mathcal S, \mathcal A, \mathcal O, \mathcal T, \mathcal R$). Here, $\mathcal U$ denotes the instruction space, $\mathcal S$ the state space, $\mathcal A$ the action space, $\mathcal O$ the observation space, $\mathcal T$ the transition function ($\mathcal T: \mathcal S \times \mathcal A \rightarrow \mathcal S$), and $\mathcal R$ the reward function ($\mathcal R: \mathcal S \times \mathcal A \rightarrow [0, 1]$). Since the task planning capability of LLM agents is our main focus, $\mathcal U, \mathcal A, \mathcal O$ are subsets of natural language space.

Given a task instruction $u\in\mathcal{U}$, the LLM agent $\pi_{\theta}$ at time step $t$ takes an action $a_t\sim \pi_\theta(\cdot|u, e_{t-1})$ and receives the environmental feedback as the observation $o_t\in\mathcal{O}$. $e_{t-1}$ denotes the historical interaction trajectory $(a_1, o_1, ... , a_{t-1}, o_{t-1})$. Each action $a_t$ incurs the environment state to $s_t\in\mathcal{S}$. The interaction loop terminates when either the agent completes the task or the maximum step is reached.
The final trajectory is $e_m = (u, a_1, o_1, ..., a_m, o_m)$, where $m$ denotes the trajectory length. The outcome reward $r_o(u, e_m) \in [0, 1]$ indicates the success or failure of the task.


\begin{figure*}[t!]
    \centering
    \includegraphics[width=0.95\textwidth]{figure/Fig_overview.pdf}
    \caption{Overview of the \textbf{S}tep-level \textbf{T}raj\textbf{e}ctory \textbf{Ca}libration (\textbf{STeCa}) framework for LLM agent learning.}
    \label{fig:overview}
\end{figure*}


\paragraph{Step-level Reward Acquisition.}
\label{sec:step_reward_compute}

It is crucial to acquire step-level rewards as feedback to improve decision-making for LLM agents.
Following prior work~\citep{kakade2002approximately,salimans2018learning,xiong2024watch}, we leverage expert trajectories as demonstrations and ask an LLM agent to begin exploration from the initial state $s_0\in\mathcal{S}$ toward the target state for a given demonstration. At each $t$-step, the agent's policy $\pi_\theta$ generates an action $a_t$, and we define a step-level reward $r_s(s_t, a_t)$ to quantify the contribution of $a_t$ to future success. 
Specifically, at $t$-th step, the agent generates $N$ new subsequent trajectories $\{e_{t+1:m}^{(i)}\}_{i=1}^{N}$ using the widely-used Monte Carlo sampling, conditioned on the historical trajectory $e_t$. Each trajectory receives an outcome reward $r_o(u,e_m)$ from the environment. The step-level reward $r_s(s_t, a_t)$ is computed as the expected value of these outcome rewards:
\begin{equation}
    r_s(s_t, a_t) = \mathbb E_{e_m \sim \pi_{\theta}(e_{t+1:m}|e_{t})} [r_o(u, e_m)].
\end{equation}


\paragraph{Normalized Dynamic Time Warping.}
\label{sec:distance_measure}
The normalized Dynamic Time Warping (nDTW) algorithm~\cite{muller2007dynamic}, implemented via dynamic programming (DP), effectively measures the distance between two trajectories containing multiple time steps. Formally, given a pair of trajectories $(x, y)$, this computation process is computed as:
\begin{align}
    D(i,j) = d(x_i, y_j) + 
        \min \begin{cases} 
            D(i-1, j) \\ 
            D(i, j-1),\\ 
            D(i-1, j-1)
        \end{cases}
\label{eq:nDTW}
\end{align}
where $d(x_i,y_i)$ denotes a distance function such as $L_2$ or cosine distance, $D(0,0)=d(x_0,y_0)$. $x_i$ denotes the action at the $i$-th step in the trajectory $x$, while $y_j$ denotes the action at the $j$-th step in the trajectory $y$. With a normalization operation, the nDTW distance $d_{\text{nDTW}}$ is given by:
\begin{equation}
    d_{\text{nDTW}}(x,y) = \frac{D(x-1, y-1)}{\sqrt{n_{x}^2 + n_{y}^2}},
\end{equation}
where $d_{\text{nDTW}}\in [0,1]$, $n_{x}$ and $n_{y}$ denote the number of steps in the trajectory $x$ and $y$, respectively. 

% !TeX root = ../camera.tex

\begin{figure*}[tbp]
  \centering
  \includegraphics[width=\linewidth]{architecture_v8.pdf}
  \caption{
    Overview of the IC-Mamba Architecture for social media engagement prediction. 
    (left panel) The model first takes three types of inputs (interval-censored social engagement, post content, and user metadata). 
    These inputs are tokenized through a linear tokenization layer. 
    The tokenized sequence (combination of temporal embedding, positional embeddings and user embeddings) is processed through N-stacked \icmamba blocks.
    (right panel) Each \icmamba block contains a selective SSM mechanism and parallel Conv1d operations to handle input and time-interval vectors simultaneously. 
    Lastly, the processed features go through normalization and linear layers to generate the final social engagement predictions.
  }
  \label{fig:overall_archi}
\end{figure*}

\section{Interval-Censored Mamba (IC-Mamba)}
\label{sec: ic-mamba-method} 

This section introduces \icmamba, our proposed approach for engagement prediction illustrated in \cref{fig:overall_archi}.
We begin with the problem statement (\cref{subsec:problem-statement}) and then detail the key components of our architecture:
% We present the overall architecture of \icmamba in \cref{fig:overall_archi}.
the time-aware positional embeddings (\cref{subsec:time-aware-embed}), 
the content and sequence embeddings (\cref{subsec:context-sequence-embed}), 
interval-censored state space modeling (\cref{subsec:interval-censored-SSM}), 
the pretraining strategies (\cref{subsec:ic-mamba-pretrain}), and 
the two-tier architecture that enables predictions at both post and opinion levels (\cref{subsec:two-tier-arch}).

\subsection{Problem Statement}
\label{subsec:problem-statement}
Let $\mathcal{E}$ denote a social outbreak event with associated posts $\mathcal{P} = \{p_1, p_2, \dots, p_N\}$. For each post $p \in \mathcal{P}$, we define a tuple $(t_0, x, u, o, H)$ where $t_0$ denotes the original posting time; $x$ represents the textual content; $u$ captures the user metadata; $o \in \mathcal{O}$ indicates the opinion class from the set of possible opinions $\mathcal{O}$; and the interval-censored engagement history is defined as $H = \{(t_j, e_j)\}_{j=1}^{m}$, with 
$m$ as the total number of observation intervals. Each $e_j$ is a $d$-dimensional vector capturing different types of engagement at observation time $t_j$, with intervals $\Delta t_j = t_{j+1} - t_j$ between consecutive observations -- see also \cref{fig:sample_ic_mamba} for how these quantities interact.
See \cref{tab:notations} for a complete reference of mathematical notations used in this work.


Given an observation window $\tau_{obs}$ (e.g., 1 day), let $H_{\tau_{obs}}(p) = \{(t, e) \in H \mid t_0 \leq t \leq t_0 + \tau_{obs}\}$ denote the initial interval-censored engagement history. Let $\Delta t$ be a fixed time interval (e.g.,\ 5 minutes) and $T$ be the prediction horizon (e.g.,\ 28 days). Our goal is to predict the engagement trajectory at regular intervals: $\{\hat{e}(t_0 + \tau_{obs} + k\Delta t)\}_{k=1}^{K}$, where $K = \lfloor T/\Delta t \rfloor$ represents the number of prediction points.

% {
% Let $\mathcal{E}$ denote a social outbreak event. 
% Within this event, we consider a set of $N$ social media posts, denoted as $\mathcal{P} = \{p_1, p_2, \dots, p_N\}$. 
% For each post $p_i \in \mathcal{P}$, we define the following -- see also \cref{fig:sample_ic_mamba} for how these quantities interact.
% \begin{itemize} 
% \item $t_{i,0}$: The original posting time of post $p_i$. 
% \item $T_i = \{t_{i,1}, t_{i,2}, \dots, t_{i,m_i}\}$: The observed times of engagement events for post $p_i$, where $t_{i,j}$ is the time of the $j$-th engagement event, and $m_i$ is the total number of observed engagement events.
% \item $[t_{i,j-1}, t_{i,j}]$: The interval-censored period between consecutive observations, \hl{where engagement counts are unknown}\mar{Unknown or known? Also, how does this link to $\tau_j$ in \cref{fig:sample_ic_mamba}}.
% \item $e_{i,j} \in \mathbb{N}^d$: The $d$-dimensional vector of social engagements (likes, comments, shares, emojis) received at time $t_{i,j}$.
% \item $x_i$: The textual content of post $p_i$. 
% \item $u_i$: The user metadata associated with post $p_i$ (e.g., user profile information). 
% \item $o_i \in \mathcal{O}$: The opinion expressed in post $p_i$, where $\mathcal{O}$ is a set of opinion classes. 
% \end{itemize}}
% \mar{Now I really think we should transform this into a notation table instead and refer to it when necessary.}

%$\mathcal{H}_{i}(\tau_k)$
% \replace{
Using this setup, we address two primary tasks. 
(1) Social Engagement Prediction: We predict engagement at both individual and collective levels. 
\emph{Post level}: Predict the engagement trajectory $\hat{e}(t_0 + \tau_{\text{obs}} + k\,\tau_{\text{step}})_{k=1}^{K}$ at regular intervals $\tau_{\text{step}}$ up to horizon $T$ (with $K = \lfloor T/\tau_{\text{step}} \rfloor$), as well as the total cumulative engagement over $T$. 
\emph{Opinion level:} For a given opinion \(o\), predict the collective trajectory ${\hat{E}_o(t_0 + \tau_{obs} + k\tau_{\text{step}})}_{k=1}^{K}$ , where $\hat{E}_o$ is the sum of engagements across all posts $\mathcal{P}_o$ expressing $o$. 
(2) Opinion Classification: We learn a mapping $f: (x, u, H_{\tau_{obs}}) \mapsto \mathcal{O}$ that assigns a post to an opinion class based on its content $x$, user metadata $u$, and engagement history $H_{\tau_{obs}}$.



% \noindent Using this setup, we address two main learning tasks:
% \begin{enumerate}
%     \item Social Engagement Prediction:
%     Predicting social engagement at both individual and collective levels:
%     \begin{itemize}
%         \item At post level:  Predicting both the engagement trajectory ${\hat{e}(t_0 + \tau_{obs} + k\tau_{\text{step}})}_{k=1}^{K}$ at regular intervals $\tau_{\text{step}}$ up to horizon $T$, where $K = \lfloor T/\tau_{\text{step}} \rfloor$, and the total cumulative engagement over horizon $T$.
%         \item At opinion level:  For a given opinion $o$, predict the collective engagement trajectory ${\hat{E}_o(t_0 + \tau_{obs} + k\tau_{\text{step}})}_{k=1}^{K}$ of all posts $\mathcal{P}_o$ expressing that opinion, where $\hat{E}_o$ represents the sum of engagement across posts in $\mathcal{P}_o$ at each timestep.
%     \end{itemize}
%     \item Opinion Classification:
%     The task involves determining a mapping $f: (x, u, H_{\tau_{obs}}) \mapsto \mathcal{O}$ that assigns each post to an opinion class based on its content $x$, user metadata $u$, and engagement history $H_{\tau_{obs}}$.
% \end{enumerate}
% }
% {Given prediction time points $\mathcal{T}_p = \{\tau_1, \tau_2, \dots, \tau_K\}$\mar{What are these? You never defined them. I thought that $\tau$ were lengths of intervals} and interval-censored engagement history $\mathcal{H}_{i,k} = \{(t_{i,j}, e_{i,j}) \mid t_{i,0} \leq t_{i,j} \leq t_{i,0} + \tau_k\}$, we address three tasks:
% \begin{enumerate}
% \item Post-level Engagement Prediction: For each post $p_i$, predict its future cumulative engagement vector $\hat{E}{i}(t)$ at time $t > t{i,0} + \tau_k$, given its censored history up to $\tau_k$.
% \item Opinion-level Engagement Prediction: For each opinion class $o \in \mathcal{O}$, predict its collective engagement influence at time $t > t_{i,0} + \tau_k$, using the censored histories up to $\tau_k$ of all posts expressing that opinion.
% \item Opinion Classification: For each post $p_i$, predict its opinion class $o_i$ based on its content $x_i$, user metadata $u_i$, and engagement history $\mathcal{H}_{i,k}$.
% \end{enumerate}
% }
% \TODO{MAR}{I wonder if it does not make more sense to group the two types of engagement prediction (post, opinion) together and the opinion classification separately.} 

\subsection{Time-aware Positional Embeddings}
\label{subsec:time-aware-embed}

The temporal dynamics of social media engagement operate at multiple scales -- from rapid initial spread to long-term influence patterns. 
To capture these multi-scale dynamics, we introduce a dual strategy featuring Relative Temporal Encoding (RTE) and Absolute Temporal Encoding (ATE).
RTE captures temporal relationships between two time points $t$ and $t_{ref}$ as
$RTE(t, t_{ref}) = \sin\left(\frac{t - t_{ref}}{\sigma}\right)$, where $\sigma$ is a learnable parameter that allows the model to adapt to varying engagement velocities.
ATE is capturing predictions to the global event timeline by mapping each time point $t$ into a sinusoidal embedding space:
\begin{equation*}
ATE(t) = \left[\sin\left(\frac{t}{10000^{2i/d}}\right), \cos\left(\frac{t}{10000^{2i/d}}\right)\right]_{i=0}^{d/2-1}.
\end{equation*}

These embeddings combine through a learnable projection:
\begin{equation*}
PE(t, t_{ref}) = W_p \begin{bmatrix} RTE(t, t_{ref}) \\ ATE(t) \end{bmatrix},
\end{equation*}
which is then modulated by observed engagement
$EPE(t, t_{ref}, e) = PE(t, t_{ref}) \odot \bigl(1 + \log\left(1 + e\right)\bigl)$,
 where $\odot$ denotes element-wise multiplication, and $e$ is the engagement vector at time $t$.

This engagement-sensitive embedding enables the model to learn characteristic temporal patterns associated with different levels of social impact.
For each post $p \in \mathcal{P}$ and a prediction time $\tau_k$, we construct a \textit{time-aware embedding sequence} $TE^k(p) \in \mathbb{R}^{(m_k + 1) \times d}$ as $
TE^k(p) = \left[EPE(t_{j}, \tau_k, e_{j}) \mid (t_j, e_j) \in H_{\tau_{obs}}(p)\right] \cup \left[PE(\tau_k, \tau_k, 0)\right]$, where $H_{\tau_{obs}}(p) = \{(t, e) \in H \mid t_0 \leq t \leq t_0 + \tau_{obs}\}$ is the observed engagement history within the observation window $\tau_{obs}$.

\subsection{Content and Sequence Embedding}
\label{subsec:context-sequence-embed}

To create a unified representation of social media posts, we must handle both textual content and temporal patterns. 
We use a byte-level BPE tokenizer~\citep{black2022gptneoxb} to process the social media text, enabling us to embed the multi-modal information (content, user metadata, and temporal dynamics) into a single sequence representation:
$SE(p) = {Encoder}([CLS] \oplus [x] \oplus [SEP] \oplus [u] \oplus [SEP] \oplus [T] \oplus [SEP] \oplus [{e_j}])$.
Here, $Encoder$ is a transformer-based function, $x$ is the post text, $u$ is user metadata, $T = \{t_0, t_1, \dots, t_{m}\}$ is the post's timeline of engagement events, $\{e_j\}$ are engagement counts, and $[CLS]$ and $[SEP]$ are special tokens.
Note that the $Encoder$ function maps the input sequence to a fixed-dimensional space $\mathbb{R}^d$, where $d$ is the embedding dimension. 
This allows for building uniform representations regardless of the posts' content or engagement history length.

\subsection{Interval-Censored State Space Modeling}
\label{subsec:interval-censored-SSM}

Here, we extend the Mamba architecture to incorporate time intervals within the state space model. 
Standard SSMs assume regular sampling intervals, which fails to capture social media engagement's irregular and censored nature (see \cref{fig:sample_ic_mamba}). 
We address this through three key components: interval-aware state representation, time-dependent transitions, and selective state updates.

\noindent\textbf{Interval-aware State Representation.}
For each observation time $t_j$ in the engagement history $H_{\tau_{obs}}(p)$, we construct an interval-aware vector $v_j \in \mathbb{R}^{4d}$:
\begin{equation*}
        v_j = [\Delta t_j^-; \log(1 + e_j); \Delta t_j^+; \log(1 + \hat{e}_{j+1})],
\end{equation*}
where $\Delta t_{j}^- = t_{j} - t_{j-1}$ captures the time since the last observation, 
$e_{j}$ is the current engagement vector, 
$\Delta t_{j}^+ = t_{j+1} - t_{j}$ is the forward interval length, and 
$\hat{e}_{j+1}$ is the predicted next engagement vector.

To maintain a consistent representation when transitioning from variable-length historical intervals to fixed-length prediction intervals, at each prediction time point $\tau_k$, we construct: $v_k = [\tau_k - t_j; \log(1 + e_j); \tau_{k+1} - \tau_k; \log(1 + \hat{e}_k)]$,
using the last observed engagement $(t_{j}, e_{j})$ in $H_{\tau_{obs}}$.

\noindent\textbf{Time-Dependent State Transitions.}
We handle varying-length censored intervals by modifying the standard SSM architecture to incorporate time-dependent state transitions. 
For a hidden state dimension $D_h$ and input dimension $D$, our model becomes:
\begin{align*}
    \mathbf{A}_t(\Delta t) &= \exp(\Delta t \cdot \tilde{\mathbf{A}}_t) \in \mathbb{R}^{D_h \times D_h}, \\ 
    \mathbf{h}_t &= \mathbf{A}_t(\Delta t) \mathbf{h}_{t-1} + \mathbf{B}_t \mathbf{x}_t, \quad\quad \mathbf{y}_t = \mathbf{C}_t^T \mathbf{h}_t,
\end{align*}
where $\mathbf{h}_t \in \mathbb{R}^{D_h}$ is the hidden state at time $t$, $\mathbf{x}_t \in \mathbb{R}^D$ is derived from the interval-aware vector $v_j$, and the matrix exponential $\exp(\Delta t \cdot \tilde{\mathbf{A}}_t)$ enables smooth interpolation across censored intervals.

\noindent\textbf{Selective State Processing.}
We integrate the temporal embeddings (${TE}^k(p)$) and interval-aware vectors through parallel pathways:
\begin{equation*}
        [\mathbf{X}, \boldsymbol{\Delta}, \mathbf{B}, \mathbf{C}] = \text{Projection}\left(\mathbf{V}, {TE}^k(p)\right) \in \mathbb{R}^{L \times (D + 1 + 2N)}
\end{equation*}
where $L$ is the sequence length, $\mathbf{V} \in \mathbb{R}^{L \times 4d}$ is the sequence of interval-aware vectors, and ${TE}^k(p)$ provides temporal context. 
The selective SSM mechanism then processes as follows:
\begin{equation*}
\mathbf{Y} = \text{SSM}(\tilde{\mathbf{A}}, \mathbf{B}, \mathbf{C}, \mathbf{X}, \boldsymbol{\Delta}) \in \mathbb{R}^{L \times D},
\end{equation*}

The final output is modulated through a gating mechanism:
\begin{equation*}
\text{Output} = \mathbf{Y} \odot \sigma \bigl(\text{Conv1d}(\mathbf{X})\bigl) \in \mathbb{R}^{L \times D},
\end{equation*}
where $\sigma$ is the Silu activation function~\citep{elfwing2018sigmoid} and Conv1d~\citep{gu2021combining} captures local engagement patterns.



\subsection{IC-Mamba Pretraining}
\label{subsec:ic-mamba-pretrain}

% To leverage the full potential of our interval-censored architecture, we introduce a pretraining strategy that enables the model to learn general temporal dynamics from large-scale social media data.
Creating labeled sets of misinformation and disinformation campaigns is a human-time-intensive process, and often, the resulting training sets are too small to allow training an architecture such as \icmamba from scratch.
%We introduce $\mathcal{D}$ a pretraining strategy that leverages $1.78$ million posts and their social engagement timelines, totaling over $153$ million timelines, from the two datasets we use in this paper: SocialSense~\citep{kong2022slipping} and DiN (detailed in Section~\ref{subsec:datasets}):
\cref{alg:pretraining} outlines the pretraining procedure for \icmamba.
We introduce $\mathcal{D} = \{(p_i, H_i, x_i, u_i)\}_{i=1}^M$, a pretraining dataset comprising $1.78$ million posts and their associated social engagement timelines -- totaling over $153$ million timelines -- collected from the two datasets SocialSense~\citep{kong2022slipping} and DiN (detailed in Section~\ref{subsec:datasets}).
Here $M$ is the number of posts and $H_i = \{(t_{i,n}, e_{i,n})\}_{n=1}^{m_i}$ with $|H_i| = m_i$ represents the complete engagement history for post $p_i$.

%Remove this part for now, looks duplciate.
%\subsubsection{Interval-Censored Sequence Construction}
%For each post's engagement history $\mathcal{H}_i$ of length $m_i$, we compute interval-aware vectors:
%\begin{equation}
%v_{i,j} = [\Delta t_{i,j}^-; \log(1 + e_{i,j}); \Delta t_{i,j}^+; \log(1 + e_{i,j+1})]
%\end{equation}
%where intervals and engagements are as defined in Section 3.3.


\noindent\textbf{Objective Function.}
We define two objective functions that we combine for pretraining.

\emph{Engagement Prediction Loss.} 
For each post, we train the model to predict the next engagement vector:
\begin{equation}
  \label{eq:loss_next_pred}
        \mathcal{L}_\text{pred} = \frac{1}{|\mathcal{P}|} \sum_{p \in \mathcal{P}} \sum_{j=0}^{m-1} \|\hat{e}_{j+1} - e_{j+1}\|^2 \enspace,
\end{equation}
where $\hat{e}_{j+1} \in \mathbb{R}^d$ is the predicted engagement vector.

\emph{Temporal Coherence Loss.} 
We enforce consistent state transitions across intervals:
\begin{equation}
  \label{eq:loss_interval}
        \mathcal{L}_\text{temp} = \frac{1}{|\mathcal{P}|} \sum_{p \in \mathcal{P}} \sum_{j=0}^{m-1} \|\mathbf{h}_{j+1} - \exp(\Delta t_j^+ \cdot \tilde{\mathbf{A}}_t)\mathbf{h}_j\|^2 \enspace,
\end{equation}
where $\mathbf{h}_j \in \mathbb{R}^{D_h}$ is the hidden state at time $t_j$ and the exponential term comes from our SSM formulation.%todo:ref

The pretraining loss combines these objectives from \cref{eq:loss_next_pred} and \cref{eq:loss_interval} as $\mathcal{L}_\text{total} = \mathcal{L}_\text{pred} + \lambda \mathcal{L}_\text{temp}$,
where $\lambda$ is a hyperparameter balancing the two losses.
% 
% \noindent\textbf{Training Procedure.}


\begin{algorithm}[t]
\caption{\icmamba Pretraining}
\label{alg:pretraining}
\begin{algorithmic}[1]
\State Initialize parameters $\theta = \{\tilde{\mathbf{A}}, \mathbf{B}, \mathbf{C}, \mathbf{W}_p, \theta_\text{Encoder}\}$
\For{epoch $= 1$ to $N_\text{epochs}$}
    \For{batch $\mathcal{B} \subset \mathcal{D}$}
        \State Construct interval-aware vectors $\{\mathbf{v}_{j}\}_{j \in \mathcal{B}}$ 
        \State Compute temporal embeddings $\{TE^k(p)\}_{p \in \mathcal{B}}$
        \State $[\mathbf{X}, \boldsymbol{\Delta}, \mathbf{B}, \mathbf{C}] \gets \text{Projection}(\{\mathbf{v}_{j}\}, \{TE^k(p)\})$
        \State $\mathbf{H} \gets \text{SSM}(\tilde{\mathbf{A}}, \mathbf{B}, \mathbf{C}, \mathbf{X}, \boldsymbol{\Delta})$
        \State $\hat{\mathbf{E}} \gets \text{MLP}(\mathbf{H})$
        \State Compute $\mathcal{L}_\text{pred}$ (\cref{eq:loss_next_pred}) and $\mathcal{L}_\text{temp}$ (\cref{eq:loss_interval})
        \State Update $\theta$ using $\nabla_\theta(\mathcal{L}_\text{pred} + \lambda \mathcal{L}_\text{temp})$
    \EndFor
\EndFor
\State \Return $\theta$
\end{algorithmic}
\end{algorithm}

  
\subsection{Two-Tier \icmamba Architecture}
\label{subsec:two-tier-arch}

\begin{figure}[t]
  \centering
  \includegraphics[width=0.85\linewidth]{images/2-tier_v8.pdf}
  \caption{
    Two-Tier \icmamba Architecture. The bottom-tier model ($\text{IC-Mamba}_{1}$) learns post-level representations from historical ($H$), content ($x$), and user ($u$) features, while the top-tier model ($\text{IC-Mamba}_2$) captures temporal dependencies across intervals $\delta t$ to jointly predict individual post virality and aggregate narrative engagement dynamics.
    %\TODO{MAR}{New comment: figure caption is uninformative. Please detail it!}
  }
  \label{fig:2tier_ic-mamba}
\end{figure}

It is desirable to model and predict the engagement dynamics of a group of posts expressing the same opinion -- dubbed \emph{the engagement of an opinion}.
% To capture both individual post dynamics and collective opinion influence, 
We propose a hierarchical two-tier architecture, showcased in \cref{fig:2tier_ic-mamba}.
The intuition of the two-tier \icmamba model is that the first tier ($\text{IC-Mamba}_{1}$) models the arrival of engagement on an individual post. 
The second tier ($\text{IC-Mamba}_2$) models the arrival of posts within an opinion.

\noindent\textbf{Post-Level Processing.}
In the first tier, for each opinion $o$, we process all posts $p_i \in \mathcal{P}_o$ individually using the $\text{IC-Mamba}{1}$ model:
\begin{equation}
\label{eq:tier-1-icmamba}
        \mathbf{h}_i = \text{IC-Mamba}_{1}(H_{\tau_{obs}}(p_i), x_i, u_i), \quad \forall p_i \in \mathcal{P}_o
\end{equation}
where $H{\tau_{obs}}(p_i)$ is the interval-censored engagement over observation window and $\mathbf{h}_i$ the hidden state representation of post $p_i$.

\noindent\textbf{Group-Level Dynamics.}
In the second tier, we model the temporal interactions between posts sharing opinion $o$. 
By ordering posts in $\mathcal{P}_o$ chronologically by posting time $t_i^{\mathrm{p}}$, we capture the inter-post intervals $\delta t_i = t_{i+1}^{\mathrm{p}} - t_i^{\mathrm{p}}$ between posts in the group. 
The group-level dynamics are modeled using $\text{IC-Mamba}_2$ with $\mathbf{h}_i$ from \cref{eq:tier-1-icmamba}:
\begin{equation*}
\mathbf{z}_o = \text{IC-Mamba}_2({(\mathbf{h}_i, \delta t_i)}).
\end{equation*}


%!TEX root = gcn.tex
\iffalse
\begin{framed}
\noindent ``\textit{It is relatively well-known that given an arbitrary graph $\mathcal{G}$, one can construct a variety of bipartite graphs which \underline{faithfully} represent $\mathcal{G}$.}'' 

$\quad$ \textit{\hfill\textemdash Stephen T.~Hedetniemi}
\end{framed}
\fi
\section{Sparse Matrix Decomposition}
\label{sec::sgc}
Graph convolution layers $\adjmat\feamat\weimat$ encode the graph structures in $\adjmat$ into GCNs.
Graph convolution is then computed by SMM $\adjmat\feamat$.
By bridging computations of matrices and graphs, we detail how to decompose a sparse matrix $\adjmat$ into a product of special matrices for more efficient SMM.
In essence, we \emph{revisit linear algebra} relations to \emph{faithfully} capture the graph.

\subsection{Bipartite Graph Representation}\label{subsec::sme}
We represent graph $\graph$ corresponding to $\adjmat$ as bipartite graphs, and decompose $\adjmat$ into matrices.
This bipartite representation enables the identification of structured patterns that facilitate efficient SMM aligning with our $2$PC protocols.

\paragraph{Graph Decomposition via Edges.}
Non-zero entries in $\adjmat$ correspond to edges between nodes in $\graph$.
By representing $\graph$ as two bipartite graphs---$\graphout$ (the out-degree node-to-edge relation) and $\graphin$ (the in-degree edge-to-node relation)---we can decompose $\adjmat$ into the product $\adjout \adjin$, where $\adjout$ and $\adjin$ reflect the respective bipartite structures 
are sparse matrices correspond to $\graphout$ and $\graphin$, respectively.
%
Consider graph~$\graph$ (with arbitrary-sparse $\adjmat$) in Figure~\ref{fig::nn_relation_diff}.
We label each edge and treat them as imaginary nodes (`$\diamond$' drawn by dotted lines) to
%build the bridge in the decomposed graph.
%We
construct $\graphout$ and $\graphin$ as in Figure~\ref{fig::nen_relation_diff}.
This representation decomposes $\adjmat$ into $\adjout \adjin$ as in Figure~\ref{fig::adjmat_decom_init}.

%The upper one represents the out-degree node-to-edge relation, and the other represents the in-degree edge-to-node relation.
%Arbitrary connection among different nodes leads to an arbitrary sparsity.
%We first perform analysis for a general representation of arbitrary topological relations.
%Decomposition with edges.Figure~\ref{fig::graph_decom_diff} shows the process of decomposing a graph of an arbitrary topology.
%For any graph, such a graph decomposition always builds the relation between source, imaginary, and target nodes.

\begin{figure}[!t]
 \subfloat[Node-Node Graph $\graph$]{
 \includegraphics[width = 0.21\textwidth]{./fig_and_tab/before_graph_diff.png}
 \label{fig::nn_relation_diff}
 }
 \hspace{2mm}
 \subfloat[Node-Edge-Node Graph $\graphout, \graphin$]{
 \includegraphics[width = 0.225\textwidth]{./fig_and_tab/after_graph_diff.png}
 \label{fig::nen_relation_diff}
 }
	\caption{Graph Decomposition through Edges}
	\label{fig::graph_decom_diff}
\end{figure}
%\vspace{-2mm}

\begin{figure}[!t]

\centering
	\includegraphics[width = 0.47\textwidth]{./fig_and_tab/graph_Aoutin.png}
\iffalse
$$
\begin{bmatrix}
0 & 1 & 0 & 0 & 0 \\
0 & 0 & 0 & 0 & 1 \\
1 & 0 & 0 & 1 & 0 \\
0 & 1 & 0 & 0 & 1 \\
0 & 0 & 0 & 0 & 0
\end{bmatrix} 
% refers to $\graph$
=
\begin{bmatrix}
1 & 0 & 0 & 0 & 0 & 0 \\
0 & 0 & 0 & 0 & 1 & 0 \\
0 & 1 & 0 & 1 & 0 & 0 \\
0 & 0 & 1 & 0 & 0 & 1 \\
0 & 0 & 0 & 0 & 0 & 0
\end{bmatrix}
% refers to $\graph_out$
\begin{bmatrix}
0 & 1 & 0 & 0 & 0 \\
1 & 0 & 0 & 0 & 0 \\
0 & 1 & 0 & 0 & 0 \\
0 & 0 & 0 & 1 & 0 \\
0 & 0 & 0 & 0 & 1 \\
0 & 0 & 0 & 0 & 1 \\
\end{bmatrix}
% refers to $\graph_in$
$$
\fi
	\caption{Matrix Decomposition Equivalent to Figure~\ref{fig::graph_decom_diff}}
	\label{fig::adjmat_decom_init}
\end{figure}
 
\begin{figure*}[!t]
%	\vspace{-5mm}
	\centering
	\includegraphics[width = 0.98\textwidth]{./fig_and_tab/graph_psigq.png}
	\caption{Graph/Matrix Decomposition with Monotonicity}
	\label{graph_psigq}
\end{figure*}

\subsection{Permutation for Monotonicity}\label{subsec::initdecom}
$\adjout$ and $\adjin$ are still unstructured sparse matrices, challenging further decomposition.
We then permute the columns of $\adjout$ and the rows of $\adjin$ to yield permuted matrices $\adjout'$ and $\adjin'$ with monotonically non-decreasing (row-index, column-index) coordinates for non-zero positions as shown in Figure~\ref{graph_psigq}.
Definitions~\ref{q_type} and~\ref{p_type} formulate $\Psf$-type and $\Qsf$-type
sparse matrices to capture these monotonic relations,
where $\Psf$-type matrices have exactly one non-zero value in each column, and
$\Qsf$-type matrices have exactly one non-zero value in each row.
%
Note that $\adjmat \neq \adjout'\adjin'$ as the imaginary nodes
%(representing edges in $\graph$) 
in $\adjout'$ and $\adjin'$ are ordered differently.
We use a permutation $\sigma_3$ 
(before defining $\sigma_1,\sigma_2$)
to map these nodes
%in $\adjout'$ and $\adjin'$ 
for preserving the topology among edges and 
%obtaining a decomposition of 
decomposing 
$\adjmat$, given by $\adjout'\sigma_3\adjin'$.

%\footnote{
%Both $\Psf$-type and $\Qsf$-type matrices have monotonically non-decreasing (row-index, column-index) coordinates for non-zero positions.
%}


%Observe that $\adjout$ have exactly a ``1" in each column, while $\adjin$ have exactly a ``1" in each row.
%Our next step is to permute them so that their (row-index, column-index) coordinates

\iffalse
\textbf{Matrix representation}.
We capture the regularity in the graph for the algebra relations in an arbitrary graph.
Figure~\ref{fig::nen_relation_diff} can be decomposed into two bipartite graphs without cross-edges and a permutation operation.
As shown in Figure~\ref{graph_psigq}, a permutation $\sigma_3$ always exists in the middle to guarantee that no cross-edge exists in the upper and lower bipartite graphs, which respectively represent out-degree nodes to directed edges and directed edges to in-degree nodes.
Then, we capture the graph property of no cross-edges to construct a monotonic relation of coordinates stored in the sparse matrix.
Accordingly, we construct the $\Psf$-type (formalized in Definition~\ref{q_type}) and $\Qsf$-type (formalized in Definition~\ref{p_type}) sparse matrices to formulate the monotone relation of (row-index, column-index) coordinates by marking the existence of directed edges.
\fi
%Technically, we formulate the graphic topology in three bipartite graphs as $\Psf$-type matrix, permutation, and $\Qsf$-type matrix.
%$\mapsto$ denotes the ``maps to'' relation, while ``row-id'' and ``column-id'' represent the row and column indexes, respectively.
%In \yuzh{Appendix XXXX}, and formally define $\Qsf$-type matrix and $\mathsf{P}$-type matrix, respectively.

%Next, we decompose the upper-half graph of Figure~\ref{fig::nen_relation} by constructing the permutational relation between imaginary nodes (previously directed edges in Figure~\ref{fig::nn_relation}), which is essentially re-sorting imaginary nodes.
%We can now decompose any graph to a composition of three bipartite graphs as Figure~\ref{graph_psigq}, including out-degree nodes to directed edges, directed edges to resorted directed edges, and directed edges to in-degree nodes.
%In particular, a permutation operation always exists in the middle to guarantee that no cross-edge exists in the upper and lower parts.

% My understanding of the intuitions of P- and Q- type matrix:
% P-type matrix: a matrix with precisely a ``1" in each column and monotonically non-decreasing (row-index, column-index) coordinates of the "1"s
% Q-type matrix: a matrix with precisely a ``1" in each row and monotonically non-decreasing (row-index, column-index) coordinates of the "1"s


%Relation between graph and matrix.
%Although we know any graph topology can be decomposed into a series of bipartite graphs, how to achieve graph computation from the protocol design perspective is still unclear.
%Recall that topological relation in an $n$-node graph can be represented by $n\times n$ adjacency matrix, in which the entry is $1$ if there is an edge or $0$ otherwise.

%So far, we assume all non-zero values in $\adjmat$ are $1$.
Recall that $\adjmat$ is a normalized adjacency matrix, \ie, its non-zero values may not be $1$.
To account for this, we introduce a diagonal matrix $\Lambda$ in the decomposition to store the non-zero edge weights.
Theorem~\ref{the::p_sig_q} (proven in $\S$\ref{sec::proof_p_sig_q})
shows that any sparse matrix $\adjmat$ can be decomposed to a $\Psf$-type matrix, a diagonal matrix,
% $\Lambda$ (\ie, the weight matrix of edges)
a permutation, and a $\Qsf$-type matrix.

%\noindent\textbf{Sparse matrix decomposition.}
%As for an arbitrary matrix $\Msf$ (\ie, representing an arbitrary graph), w
%We present 


%Previously, we thought the value contained in each node was $1$ in default; then, the node value could be achieved by multiplying a scalar stored in a diagonal matrix.
%The last step is integrating the matrices and their transformations, abstracting a general theorem for representing arbitrary sparsity.
%Thus, we formulate 

\begin{restatable}{theorem}{initdecom}
\label{the::p_sig_q}
Let $\adjmat\in \Mbb_{m,n}(\Rcal)$ be an $m\times n$ matrix, where each entry is an element from ring $\Rcal$.
The elements of $\adjmat$ are $0$'s except $\numnonzero$ of them.
There exists a matrix decomposition $\adjmat = \adjout' \Lambda \sigma_3 \adjin'$, where $\adjout' \in \Mbb_{m, \numnonzero}(\Rcal)$ is a $\Psf$-type matrix, $\Lambda \in \Mbb_{\numnonzero,\numnonzero}(\Rcal)$ is a diagonal matrix, $\sigma_3 \in \Sbb_{\numnonzero}$ is a permutation,
%\footnote{There exists a natural injection homomorphism from $e$-permutation group $\Sbb_e$ into $\Mbb_{e,e}(\Rcal)$.
%$\sigma \in \Sbb_e$ can be viewed as a matrix $\{(\sigma(i), i, 1): i=1, \ldots, n\}\in \Mbb_{e, e}$.}
and $\adjin' \in \Mbb_{\numnonzero, n}(\Rcal)$ is a $\Qsf$-type matrix.
\end{restatable}

\subsection{Re-decomposition to Basic Operations}
\label{sec::the_supp_smm}
Given the permuted matrices $\adjout'$ and $\adjin'$ with the %above
monotonicity properties, we can re-decompose them into a product of permutation, diagonal, and constant matrices.
Due to the page limit, we focus on the intuition of re-decomposing $\adjin'$ (Q-type matrix)\footnote{
We can view a P-type matrix (\eg, $\adjout'$) as a transpose of a Q-type matrix and perform re-decomposition similarly.
} and the general theorem.
Implementation details and proofs can be found in Appendices~\ref{sec:matrix_found_sparse} and~\ref{sec::alg}.


We consider two constant lower triangular matrices:
\\
1) a ``summation matrix'' $\Sigma \in \Mbb_{\numnonzero,\numnonzero}(\Rcal)$ with $\Sigma[i, j]=1$ if $i \geq j$ or $0$ otherwise;
\\
2) a ``difference matrix'' $\delta_k \in \Mbb_{k,k}(\Rcal)$ with $\delta_k[i,j]=1$ for $i=j$ or $-1$ for $j=i-1$, or $0$ otherwise.

Intuitively, when multiplying with a (column) vector, $\Sigma$ sums values on or above each element, while $\delta_k$ computes a difference between each element and its previous one.

Based on the above intuition, it is not hard to decompose $\adjin'$ into a product of $\Sigma$ and another matrix $\delta'$ (Figure~\ref{fig::Q-decom_1}).
Interestingly,
%after the decomposition, 
we observe that the resulting matrix $\delta'$ ``contains'' a difference matrix (with size equals the number of non-zero columns in $\adjin'$) on its left-top corner (after permuting its rows and columns).
This relation can be characterized by expressing $\delta'$ into a product of permutation ($\sigma_1$, $\sigma_2$), diagonal ($\gammain$), and difference ($\delta$) matrices, as in Figure~\ref{fig::q-type_decom}.
 
\begin{figure}[!t]
	\centering
	\includegraphics[width = 0.47\textwidth]{./fig_and_tab/Q-decom_1.png}\caption{Decomposition of $\adjin' = \Sigma \delta'$}
	\label{fig::Q-decom_1}
\end{figure}

\begin{figure*}[!t]
\centering
\includegraphics[width = 0.99\textwidth]{./fig_and_tab/q-type_decom.png}	\caption{Re-decomposition of $\adjin'$ (Q-type Matrix)}
	\label{fig::q-type_decom}
\end{figure*}
 

\paragraph{General Theorem.}
%The transpose of $\Psf$-type matrix can be regarded as a $\Qsf$-type matrix, thus we do not repeat the similar contents.
Combining Theorem~\ref{the::p_sig_q} and matrix decomposition of $\Qsf$-type (Theorem~\ref{the::q_decom} proved in $\S$\ref{sec::proof_q}) and $\Psf$-type (Theorem~\ref{the::p_decom} proved in $\S$\ref{sec::proof_p}) matrices, Theorem~\ref{the::general_dec_main} concludes the general matrix decomposition (proof in $\S$\ref{sec::proof_general}).
%Colloquially speaking, graph computation 
Essentially, an arbitrary-sparse matrix can be transformed into a %sequential composition 
sequence of permutation and matrix multiplication.

\begin{restatable}[Sparse Matrix Decomposition]{theorem}{finaldecom}
\label{the::general_dec_main}
Let an $m\times n$ sparse matrix $\adjmat\in \Mbb_{m,n}(\Rcal)$ contain $\nrow$ non-zero rows, $\ncol$ non-zero columns, and $\numnonzero$ non-zero elements.
Then, there exists a matrix decomposition
$\adjmat = \sigma_5 \delta_m ^{\top} \gammaout \sigma_4 \Sigma ^{\top} \Lambda \sigma_3 \Sigma \sigma_2 \gammain \delta _n \sigma_1$,
where $\sigma _5 \in \Sbb_m$, $\sigma_4 \in \Sbb_{\numnonzero}, \sigma_3 \in \Sbb_{\numnonzero}, \sigma_2 \in \Sbb_{\numnonzero}, \sigma_1 \in \Sbb_n$, and,
\\
1) $\Sigma=(\Sigma[i, j])_{i,j=1}^{\numnonzero}$ is the left-down triangle matrix such that $\Sigma[i, j]=1$ if $i \geq j$ or $0$ otherwise,
\\
2) $\delta_k=(\delta_k[i,j])_{i,j=1}^{k}$ is the left-down triangle matrix such that $\delta_k[i,j]=1$ for $i=j$ or $-1$ for $j=i-1$, or $0$ otherwise,
\\
3) $\gammain =(\gammain[i,j])_{i=1,j=1}^{\numnonzero,n}$ is a matrix such that $\gammain[i,j]=1$ for $1\leq i=j\leq \ncol$ or $0$ otherwise,
\\
4) $\gammaout =(\gammaout[i,j])_{i=1,j=1}^{m,\numnonzero}$ is a matrix such that $\gammaout[i,j]=1$ for $1\leq i=j\leq {\nrow}$ or $0$ otherwise.
\end{restatable}
%1) $\Sigma=(a_{ij})_{i,j=1}^e$ is the left-down triangle matrix such that $a_{ij}=1$ if $i \geq j$ or $0$ otherwise,\\
%2) $\delta_n=(a_{ij})_{i,j=1}^n$ is the left-down triangle matrix such that $a_{ij}=1$ for $i=j$ or $-1$ for $j=i-1$, or $0$ otherwise,\\
\iffalse
3) $J_{k_2}=\begin{psmallmatrix}
I_{k_2} & O_{k_2,n-k_2} \\
O_{e-k_2,k_2} & O_{e-k_2,n-k_2}
\end{psmallmatrix}=(a_{ij})_{i=1,j=1}^{e,n}$ is a matrix such that $a_{ij}=1$ for $1\leq i=j\leq k_2$ or $0$ otherwise,\\
4) $\Gamma_{k_4}=\hspace{-1mm}
\begin{psmallmatrix}
I_{k_4} & O_{{k_4},e-{k_4}} \\
O_{m-{k_4},{k_4}} & O_{m-{k_4},e-{k_4}}
\end{psmallmatrix} 
=(a_{ij})_{i=1,j=1}^{m,e}$ is a matrix such that $a_{ij}=1$ for $1\leq i=j\leq {k_4}$ or $0$ otherwise.
\fi

 
\subsection{Reasoning from Graph Perspective}
\begin{figure}[!t]
	\centering
	\includegraphics[width = 0.477\textwidth]{./fig_and_tab/graph_qx.png}
	\caption{Recover In-degrees in $\graphin$ through $\adjin'\Xsf$}
	\label{graph_psigqx_diff}
		%\end{minipage}
\end{figure}

To illustrate the sparse matrix decomposition underlying Theorem~\ref{the::general_dec_main} for arbitrary topology,
Figure~\ref{graph_psigqx_diff} shows the directed edges \textit{in a reversed direction} represented by the decomposed matrices in Figure~\ref{fig::q-type_decom} and recovers the original $\graphin$.
Consider a vector
$\Xsf = [x_1, x_2, x_3, x_4, x_5]^\top$, 
which passes $5$ values through graph $\graphin$ (equivalent to SMM $\adjin'\Xsf$).
After $\sigma_1$ operation, 
$\Xsf$ passes from $\{\tikzcircle[fill=red]{3pt},\tikzcircle[fill=orange]{3pt},\tikzcircle[fill=yellow]{3pt},\tikzcircle[fill=ao]{3pt},\tikzcircle[fill=cyan]{3pt}\}$ to the re-ordered $\{\tikzcircle[fill=red]{3pt},\tikzcircle[fill=orange]{3pt},\tikzcircle[fill=ao]{3pt},\tikzcircle[fill=cyan]{3pt},\tikzcircle[fill=yellow]{3pt}\}$.
Then, the $\delta$ operation computes the difference of values stored in the neighboring source nodes to obtain the target nodes.

The $\gammain$ operation extracts the effective message passing to the subsequent graph computation by classifying the nodes with or without in-degree edges in $\graphin$.
Thus, all interdependent nodes
$\{x_1 (\tikzcircle[fill=red]{3pt}), 
x_2 (\tikzcircle[fill=orange]{3pt}), 
x_4 (\tikzcircle[fill=ao]{3pt}), 
x_5 (\tikzcircle[fill=cyan]{3pt})\}$ 
in $\graphin$ are recovered, \ie, those nodes containing one or multiple in-degree edges.

After the $\sigma_2$ operation, nodes
%plus independent nodes 
are rearranged in order
$\{x_1 (\tikzcircle[fill=red]{3pt}),\allowbreak 
x_2 (\tikzcircle[fill=orange]{3pt}), 
x_3 (\mathrm{None}), 
x_4 (\tikzcircle[fill=ao]{3pt}), 
x_5 (\tikzcircle[fill=cyan]{3pt})\}$.
Interestingly, the imaginary nodes (`$\diamond$' drawn by dotted lines) reflect the $\delta'$ matrix in Figure~\ref{fig::Q-decom_1}.
Next, the $\Sigma$ operation takes the sum of source nodes to get target nodes.
Finally, we get the results
$\{\tikzcircle[fill=red]{3pt},
\tikzcircle[fill=orange]{3pt},
\tikzcircle[fill=orange]{3pt},
\tikzcircle[fill=ao]{3pt},
\tikzcircle[fill=cyan]{3pt},
\tikzcircle[fill=cyan]{3pt}\}$, 
which recover the (permuted) in-degree edges (represented by $\adjin'$) matching the correct nodes in~$\graphin$.
 

%!TEX root = gcn.tex
\section{Secure Sparse Matrix Multiplication}
\label{sec::smm_protocol}
Given the sparse matrix decomposition from Theorem~\ref{the::general_dec_main}, SMM can be %obtained by multiplication $\adjmat\feamat$ as Theorem~\ref{the::smm_main} (proof in $\S$\ref{sec::proof_smm}).
%The multiplication $\adjmat\feamat$ is 
transformed into an ordered sequence of basic operations from right to left as Theorem~\ref{the::smm_main} (proof in $\S$\ref{sec::proof_smm}).
If we expect to compute $\feamat\adjmat$, the linear transformations should be performed sequentially from left to right.
For a sparse matrix that is multiplied by another sparse matrix, we can combine the sequential computation of $\adjmat\feamat$ and $\feamat\adjmat$.

\begin{restatable}[Sparse Matrix Multiplication]{theorem}{thmsmm}
\label{the::smm_main}
Consider a sparse matrix $\adjmat$ and a dense matrix $\feamat$.
Computing $\adjmat\feamat =
\allowbreak \sigma_5 \delta_m ^{\top} \gammaout \sigma_4 \Sigma ^{\top} \Lambda \sigma_3\allowbreak \Sigma \sigma_2 \gammain \delta_n \sigma_1 \feamat$ requires an ordered sequence of permutation group action, element-wise multiplication, %cut-off or padding with $0$, 
and constant matrix multiplication from right to left.
\end{restatable}

For secure MPC, the graph owner $\pp_0$ first decomposes its graph to obtain matrices $\sigma_1, \sigma_2, \sigma_3, \sigma_4, \sigma_5, \gammaout, \gammain, \Lambda$.
These matrices are privacy-sensitive and should not be learned by the feature owner $\pp_1$.
The summation matrix $\Sigma$ and difference matrices $\delta_m, \delta_n$ are constants (given dimensionality of~$\adjmat$) and thus are public to both $\pp_0$ and $\pp_1$.
Next, $\pp_0$ and $\pp_1$ jointly execute the MPC protocols of SMM, which multiplies the above matrices described in Theorem~\ref{the::smm_main}.

We first present an oblivious permutation protocol (for secure permutation operations based on $\sigma_1, \ldots, \sigma_5$) in Section~\ref{sec::op_pro}
and then an oblivious selection-multiplication protocol (for privately multiplying $\gammaout$ and $\gammain$) in Section~\ref{sec::osm_pro}.
Finally, we describe how to realize our \osmm protocol using our OP and OSM protocols in Section~\ref{subsec:prosmm}.
 
\iffalse
For the MPC computation, the graph holder $\pp_0$ first decomposes the graph computation into the corresponding linear transformations, essentially an ordered sequence of permutation group actions as explained in Section~\ref{sec::priv_graph}.
Next, $\pp_0$ and $\pp_1$ execute the MPC protocols of secure sparse matrix multiplication (\osmm) by sequentially executing the secure version of linear transformations as in Theorem~\ref{the::smm_main}.
To realize \osmm, we first present an \osmm protocol that reveals the sparsity in Section~\ref{subsec:prosmm} by adopting the oblivious permutation protocol in Section~\ref{sec::op_pro}.
Moreover, we build obliviously selection-multiplication in Section~\ref{sec::osm_pro} for hiding the sparsity degrees in Section~\ref{sec::smm::degree}.

\subsection{Explication of Private Graph}
\label{sec::priv_graph}

After the graph holder $\pp_0$ decomposes the private graph, $\pp_0$ obtains $\sigma_1, \sigma_2, \sigma_3, \sigma_4, \sigma_5$, which are privacy-sensitive and essentially represent the topological relations in the graph.
Thus, we design the oblivious permutation protocol $\Pi_\ssp$ in Section~\ref{sec::op_pro} to protect the permutation operations $\sigma_1, \sigma_2, \sigma_3, \sigma_4, \sigma_5$.

$k_4$ in $\Gamma$ denotes the number of non-zero rows, and $k_2$ in $J$ denotes the number of non-zero rows.
We protect $k_2,k_4$ using oblivious selection-multiplication ($\SM$) protocol in Section~\ref{sec::osm_pro} as the elements in diagonal matrices $\Gamma,J$ are either $1$ and $0$.
We design the $\SM$ protocol to achieve binary-arithmetic multiplication, which requires less communication costs than secure multiplication.

 Oblivious selection-multiplication is used for $J \times$ (see Theorem~\ref{the::q_decom}) and $\Gamma \times$ (see Theorem~\ref{the::p_decom}) since the entries in $J, \Gamma$ are binary (essentially representing the sparsity degree).
In the MPC domain, binary-arithmetic multiplication can not be achieved by arithmetic-arithmetic multiplication.


Since $\Xsf$ is dense, the entry values in $\Xsf$ are irrelevant to topological relations of non-zero products (\ie, only matter to non-zero product values).
Accordingly, the joining party who owns the private graph can derive permutation operations corresponding to its graph.
$\Sigma, \delta$ are public matrices (\ie, no need for protection) as they are identical for an arbitrary graph.

\fi

\subsection{Oblivious Permutation}
\label{sec::op_pro}
Protocol~$\Pi_{\ssp}$
is our oblivious permutation protocol.
%(\sigma, \xvec, \type)$
Given $\pp_0$'s private permutation $\sigma\in\Sbb_k$ and $\pp_1$'s private $k$-dimensional vector $\xvec\in\Zbb^k_{2^L}$, $\Pi_{\ssp}$ generates a secret share $\l \sigma\xvec \r_i$ for $\pp_i \in \{\pp_0, \pp_1\}$ without revealing $\sigma$ or $\xvec$.
%$\pp_0$ and $\pp_1$ jointly compute the permuted output $\sigma\xvec\in\Zbb^e_{2^L}$ 
%in a way that $\pp_0$ knows nothing about $\xvec$ and $\pp_1$ learns nothing about $\sigma$.
%by modifying a three-party protocol proposed for the private set intersection.
%In Protocol~\ref{fig::sspcons:main}, 
The protocol parameter $\type \in \{\raw, \shared\}$ specifies the type of input vector $\xvec$.
If $\type$ is $\raw$, $\xvec$ is initially owned by $\pp_1$; otherwise, it is secret-shared among $\pp_0$ and $\pp_1$.
%which means that $\xvec$ is initially owned by $\pp_1$ or shared-owned by $\pp_0, \pp_1$.
%After joint protocol execution, $\pp_0$ obtains the share $\l\sigma(x)\r_0\in\Zbb^e_{2^L}$ and $\pp_1$ obtains the share $\l\sigma(x)\r_1\in\Zbb^e_{2^L}$ such that $\l\sigma(x)\r_0 + \l\sigma(x)\r_1 =\ sigma(x)$.

\begin{protocol}[!t]
\caption{$\Pi_{\ssp}$: Oblivious Permutation}\label{fig::sspcons:main}
\begin{algorithmic}[1]
	\item[\textbf{Parameter:} $\pp_0$ and $\pp_1$ know $\type \in \{\raw, \shared\}$.]
	\REQUIRE $\pp_0$ inputs $\sigma$ and $\pp_1$ inputs $\xvec$ if $\type == \raw$;
	\\~~~~~otherwise, $\pp_0$ inputs $(\sigma, \l\xvec\r_0)$ and $\pp_1$ inputs $\l\xvec\r_1$.
	\ENSURE $\pp_0$ gets $\l \sigma\xvec \r_0$ and $\pp_1$ gets $\l \sigma\xvec \r_1$.
	\STATE {\color{gray}\COMMENT{Offline Phase: Generate Correlated Randomness}}
	%\STATE $\tt$: Generates random permutation $\pi$ and matrix $\uvec$
	%\STATE $\tt$: Computes $\pi\uvec$ and generates $\l \pi\uvec \r_0, \l \pi\uvec \r_1$
	%\STATE $\tt$: Sends $\pi, \l \pi\uvec \r_0$ to $\pp_0$ and $\l \pi\uvec \r_1$ to $\pp_1$
	\STATE $\tt, \pp_0$: Get $\pi, \l \pi\uvec \r_0 \leftarrow \prf(\key_0, \ctr)$
	\STATE $\tt, \pp_1$: Get $\uvec \leftarrow \prf(\key_1, \ctr)$
	\STATE $\tt$: Send $\l \pi\uvec \r_1 = \pi\uvec - \l \pi\uvec \r_0$ to $\pp_1$
	\STATE {\color{gray}\COMMENT{Online Phase: Compute $\l \sigma\xvec \r$ in 1 Round}}
	\STATE $\pp_0$: Send $\delta_{\sigma} = \sigma \cdot \pi^{-1}$ to $\pp_1$\label{pro:op:masksigma}
	\IF{ $\type == \raw $}
		\STATE $\pp_1$: Send $\delta_{\xvec} = \xvec-\uvec$ to $\pp_0$\label{pro:op:maskx}
		\STATE $\pp_0$: Compute $\l \sigma\xvec\r_0={\color{colora}\sigma\delta_{\xvec} + \delta_{\sigma} \l \pi\uvec \r_0}$\label{pro:op:reconst0raw}
	\ELSE
		\STATE $\pp_1$: Send $\delta_{\l \xvec \r_1} = \l \xvec \r_1-\uvec$ to $\pp_0$\label{pro:op:sendmaskedx1}
		\STATE $\pp_0$: Compute $\l \sigma\xvec\r_0 = {\color{colora} \sigma\delta_{\l \xvec\r_1} + \delta_{\sigma} \l \pi\uvec \r_0 + \sigma \l \xvec \r_0}$\label{pro:op:reconst0shared}
	\ENDIF
	\STATE $\pp_1$: Compute $\l \sigma\xvec\r_1= {\color{colorb}\delta_{\sigma} \l \pi\uvec \r_1}$\label{pro:op:reconst1}
	\RETURN $\l \sigma\xvec \r$
\end{algorithmic}
\end{protocol}
\begin{table}[!t]
\centering
		\caption{Communication for Oblivious Permutation}
				\setlength\tabcolsep{6pt}
			\begin{tabular}{l|c|c|c}
			 \hline
\textbf{Protocol} & \textbf{Offline} & \textbf{Online} & \textbf{Round} \\
\hline
Asharov~\etal~\cite{ccs/AsharovHIKNPTT22} &$0$&$6k\bitlen$ &$3$\\\hline
OLGA~\cite{ccs/AttrapadungH0MO21} & $2k(\bitlen + 32)$ &$2k\bitlen $ & $1$\\\hline
Araki~\etal~\cite{ccs/Araki0OPRT21} & $0$ & $4k\bitlen$& 2 \\ \hline 
\rowcolor{grayL}$\Pi_\ssp$ & $k\bitlen$& $ k\bitlen + k\log k$ & $1$
\\\hline
%		 \hline 
			\end{tabular}\\
$\bitlen$: bit-length of data,
$k$: degree of the permutation group 
\label{table:op_comm}
\end{table} 

\paragraph{Offline Phase.}
The commodity server $\tt$ assists $\pp_0$ and $\pp_1$ to generate a random permutation $\pi\in\Sbb_k$, a random vector $\uvec\in\Zbb^k_{2^L}$, and correlated randomnesses $\l \pi \uvec \r_0\in\Zbb^k_{2^L}, \l \pi \uvec \r_1\in\Zbb^k_{2^L}$.
%correlated randomness $\bvec\in\Zbb^e_{2^L},h\in\Sbb_e, \uvec\in\Zbb^e_{2^L}, \l \uvec\r_0\in\Zbb^e_{2^L}, \l \uvec\r_1\in\Zbb^e_{2^L}$.
%, where $h$ satisfies uniform distribution on $\Sbb$.

\iffalse
\begin{protocol}[!t]
 %\vspace{-3mm}
		\begin{algorithmic}[1]
\REQUIRE $\pp_0$ inputs $\sigma$ and $\pp_1$ inputs $\xvec$ if $\type == \raw$; otherwise $\pp_0$ inputs $\sigma, \l\xvec\r_0$ and $\pp_1$ inputs $\l\xvec\r_1$ if $\type == \shared$.
\ENSURE $\pp_0$ gets $\l \sigma(\xvec)\r_0$ and $\pp_1$ gets $\l \sigma(\xvec)\r_1$.
		\STATE {\color{gray}\COMMENT{Offline Phase: Generate Correlated Randomness}}
		\STATE $\tt, \pp_0$: Get $h, \langle \uvec\rangle_0\leftarrow\prf_0(\mathsf{key}_0 ) $ 
		\STATE $\tt, \pp_1$: Get $\bvec\leftarrow\prf_1(\mathsf{key}_1 ) $
		%
		\STATE $\tt$: Get $\langle \uvec\rangle_1=h(\bvec)-\langle \uvec\rangle_0$ and send $\langle \uvec\rangle_1$ to $\pp_1$
		\STATE {\color{gray}\COMMENT{Online Phase: Reconstruct $\sigma(\xvec)$ in 1 Round}}
		\STATE $\pp_0$: Get $\delta_{\sigma} = \sigma \cdot h^{-1}$ and {send $\delta_{\sigma}$} to $\pp_1$\label{pro:op:l6}
			\IF{ $\type == \raw $}
		\STATE $\pp_1$: Get $\delta_{\xvec} = \xvec-\bvec$ and {send $\delta_{\xvec}$} to $\pp_0$ \label{pro:op:l8}
		\STATE $\pp_0$: Get $\l \sigma(\xvec)\r_0={\color{cyan}\sigma(\delta_{\xvec}) + \delta_{\sigma} (\langle \uvec\rangle_0)}$ \label{pro:op:l9}
%and output 
	\ENDIF
	\IF{$\type == \shared$} 
		\STATE $\pp_1$: Get $\delta_{\l \xvec \r_1} = \l \xvec \r_1-\bvec$, {send $\delta_{\l \xvec \r_1}$} to $\pp_0$\label{pro:op:l12}
		 \STATE $\pp_0$: Get $\l \sigma(\xvec)\r_0 = {\color{cyan} \sigma(\delta_{\langle \xvec\rangle_1}) + \delta_{\sigma} (\langle \uvec\rangle_0) + \sigma (\langle \xvec \rangle_0)}$\label{pro:op:l13}
	\ENDIF
		\STATE $\pp_1$: Get $\l \sigma(\xvec)\r_1= {\color{ao}\delta_{\sigma} (\langle \uvec\rangle_1)}$ \label{pro:op:l15}
		\end{algorithmic}
	\caption{$\Pi_{\ssp}$: Oblivious Permutation}\label{fig::sspcons:main}%\vspace{-4mm}
\end{protocol}
 \fi

\paragraph{Online Phase.}
%In the $\raw$ case, 
$\pp_0$ masks $\sigma$ using random $\pi^{-1}$ (\ie, inverse permutation of $\pi$) to get random {$\delta_\sigma$ ({Line~\ref{pro:op:masksigma}})}.
If $\type$ is $\raw$, $\pp_1$ masks $\xvec$ using random $\uvec$ to get $\delta_\xvec$ (Line~\ref{pro:op:maskx}).
%Line~\ref{pro:op:l6} is essentially to protect $\sigma$ since $\pp_1$ can only view $\delta_\sigma$ and knows nothing about $\sigma, \pi$.
%$\pp_1$ masks $\xvec$ using random $\bvec$ to get random {$\delta_\xvec$ (Line~\ref{pro:op:l8})}.
%Line~\ref{pro:op:l8} is essentially to protect $\xvec$ since $\pp_0$ can only view $\delta_\xvec$ and know nothing about $\xvec, \bvec$.
If $\type$ is $\shared$, 
%In the $\shared$ case, $\pp_0$ masks $\sigma$ using random $h^{-1}$, which is identical to the $\raw$ case.
$\pp_1$ needs not mask $\xvec$ since $\l \xvec \r_0$ is kept by $\pp_0$ %all the time 
as a part of computing $\l \sigma\xvec\r_0$ (Line~\ref{pro:op:reconst0shared}).
In this case, $\pp_1$ masks $\l \xvec\r_1$ using random $\uvec$ to get random {$\delta_{\l\xvec\r_1}$ (Line~\ref{pro:op:sendmaskedx1})}.
%Line~\ref{pro:op:sendmaskedx1} is essentially to protect $\l\xvec\r_1$ since $\pp_0$ can only view $\delta_{\l\xvec\r_1}$ and know nothing about ${\l\xvec\r_1}, \bvec$.
%$\pp_0$ uses the offline-generated randomness $h$ to mask the permutation $\sigma$.
%If the input type of $\xvec$ is $\raw$, $\pp_1$ masks $\Xsf$ using offline-generated randomness $\bvec$, otherwise masking $\l\Xsf\r_1$.
$\pp_0$ and $\pp_1$ can then obtain the respective secret shares $\l \sigma\xvec\r_0, \l \sigma\xvec\r_1$.
%such that $\sigma(\xvec) = \l \sigma(\xvec)\r_0 + \l \sigma(\xvec)\r_1$.

\paragraph{Correctness.}
%The correctness of $\Pi_\ssp$ is to 
Here, we verify that %$\pp_0$ and $\pp_1$ jointly compute the functionality $\sigma(\xvec)$ (defined in Figure~\ref{func::ssp}) such that 
${\color{colora}\l\sigma\xvec\r_0} + {\color{colorb}\l\sigma\xvec\r_1} = \sigma\xvec$.
If $\type$ is $\raw$, it holds that $\sigma\Xsf = \sigma (\Xsf-\uvec + \uvec) = \sigma(\delta_{\Xsf} + \uvec) = \sigma\delta_{\Xsf} + \sigma\uvec = \sigma\delta_{\Xsf} + \sigma \pi^{-1}\pi\uvec = \sigma\delta_{\xvec} + \delta_{\sigma} \pi \uvec={\color{colora}\sigma\delta_{\xvec} + \delta_{\sigma} \l\pi \uvec\r_0} + {\color{colorb}\delta_{\sigma} \l \pi \uvec\r_1}$.

If $\Xsf$'s $\type$ is $\shared$,
$\sigma\Xsf = \sigma (\l \Xsf \r_0 + \l \Xsf \r_1-\uvec + \uvec) = \sigma(\l \Xsf \r_0 + \delta_{\l\Xsf\r_1} + \uvec) = \sigma \l \Xsf \r_0 + \sigma\delta_{\l\Xsf\r_1} + \sigma \pi^{-1} \pi\uvec = \sigma \l \Xsf \r_0 + \sigma\delta_{\l\Xsf\r_1} + \delta_{\sigma}\pi \uvec={\color{colora}\sigma \l \Xsf \r_0 + \sigma\delta_{\l\Xsf\r_1} + \delta_{\sigma}\l\pi \uvec\r_0} + {\color{colorb}\delta_{\sigma}\l\pi \uvec\r_1}$.

\paragraph{Communication.}
Since $\sigma \in \Sbb_k$, $\log k$ bits are enough to represent $k$ elements.
The online phase of $\Pi_\ssp$ requires communication of $k\log k + kL$ bits (\ie, sending $\delta_\sigma, \delta_\Xsf$ in the $\raw$ case or sending $\delta_\sigma, \delta_{\l\Xsf\r_1}$ in the $\shared$ case) in $1$ round.

\paragraph{Comparison to Existing Works.}
%\todo[inline]{Move Table~\ref{table:op_comm} and related text here}
%\ak{How about writing a bit to ``explain'' the Table and show the improvement of our protocols over the existing work? E.g., XXX requires communication of YY bits in ZZ rounds...Our work reduces online communication by XX\% and saves YY online rounds...}
Asharov~\etal~\cite{ccs/AsharovHIKNPTT22} spend $6kL$~bits online in three rounds.
Araki~\etal~\cite{ccs/Araki0OPRT21}'s oblivious shuffle requires $4kL$~bits in two rounds for $k$-element permutation.
The OLGA protocol~\cite{ccs/AttrapadungH0MO21} is $1$-round but communicates $2k(L + 32)$~bits offline and $2kL$~bits online.
Our $\Pi_{\ssp}$ protocol is also $1$-round, communicates $kL$~bits offline and $kL + k\log k$~bits online.
Particularly, $k$ equals to $\numedge$ or $\numnode$ for GCN.
%is equal to $\numedge$ for $\sigma_2, \sigma_3, \sigma_4$ and $\numnode$ for $\sigma_1, \sigma_5$.
In practice, $\log k$ is much smaller than $L$, \eg, for a $10^6$-node graph, $\log k= 20< L=64$.

\subsection{Oblivious Selection-Multiplication}
\label{sec::osm_pro}

We design the oblivious selection-multiplication %($\SM$) 
protocol $\Pi_{\SM}$ %(s,x)$ over private $s$ 
in Protocol~\ref{fig:osm-main}.
It takes a private bit (called ``selector'') 
$s\in\Zbb_{2}$ from $\pp_0$ and a secret share $\l x \r$ of an arithmetic number $x\in\Zbb_{2^L}$ owned by $\pp_0$ and $\pp_1$.
$\Pi_{\SM}$ generates a secret share of $0$ if $s=0$ or share of $x$ otherwise without disclosing $s$ or $x$.

\begin{protocol}[!t]
\caption{$\Pi_{\SM}$: Oblivious Selection-Multiplication}\label{fig:osm-main}
\begin{algorithmic}[1]
	%\item[\textbf{Parameter:} No extra parameter.]
	\REQUIRE $\pp_0$ inputs $(s, \l x \r_0)$ and $\pp_1$ inputs $\l x \r_1$.
	\ENSURE $\pp_0$ gets $\l sx \r_0$ and $\pp_1$ gets $\l sx \r_1$.
	\STATE {\color{gray}\COMMENT{Offline Phase: Generate Correlated Randomness}}
	%\STATE $\tt$: Generates random bit $b$ and scalar $u$
	%\STATE $\tt$: Computes $bu$ and generates $\l bu \r_0, \l bu \r_1$
	%\STATE $\tt$: Sends $b$, $\l bu \r_0$ to $\pp_0$ and $\l bu \r_1$ to $\pp_1$
	\STATE $\tt, \pp_0$: Get $(b, \l u \r_0, \l bu \r_0) \leftarrow \prf(\key_0, \ctr)$
	\STATE $\tt, \pp_1$: Get $\l u \r_1 \leftarrow \prf(\key_1, \ctr)$
	\STATE $\tt$: Send $\l bu \r_1 = bu - \l bu \r_0$ to $\pp_1$
	\STATE {\color{gray}\COMMENT{Online Phase: Compute $\l sx \r$ in 1 Round}}
	\STATE $\pp_0$: Send $\delta_s= s-b$ to $\pp_1$\label{osm::masks}%\label{osm::l13}\label{osm::l14}\label{osm::l16} 
	\STATE $\pp_1$: Send $\delta_{\l x\r_1} = \l x\r_1 - \l u\r_1$ to $\pp_0$\label{osm::maskx1}%\label{osm::l15}
	\STATE $\pp_0$: Compute $\delta_x =\l x \r_0 - \l u\r_0 + \delta_{\l x\r_1}$
	\STATE $\pp_0$: Compute $\langle sx \rangle_0 = {\color{colora}s \delta_x + \delta_s \l u\r_0 + (-1)^{\delta_s} \l bu\r_0}$\label{osm::l18}
	\STATE $\pp_1$: Compute $\langle sx \rangle_1 = {\color{colorb}\delta_s\l u\r_1 + (-1)^{\delta_s}\l bu\r_1}$\label{osm::l19}
	\RETURN $\l sx \r$
\end{algorithmic}
\end{protocol}
\begin{table}[!t]
\centering
		\caption{Communication for Oblivious Selection-Mult.}
			\setlength\tabcolsep{12pt}
			\begin{tabular}{l|c|c|c}
 		\hline
\textbf{Protocol} & \textbf{Offline} & \textbf{Online} & \textbf{Round} \\
\hline
$\promult$~\cite{crypto/Beaver91a} & $\bitlen$&$2\bitlen$& $1$\\\hline
OT~\cite{pkc/Tzeng02} &$\bitlen$&$2\bitlen + 1$&$1$\\\hline
%B2A~\cite{ndss/ABY15} &-&$(\bitlen+1)\bitlen/2$& $\log L$\\\hline
\rowcolor{grayL}$\Pi_\SM$ & $\bitlen$&$\bitlen + 1$& $1$\\
 	 \hline 
			\end{tabular}

 $\bitlen$: bit-length of data
 % $e$ is the degree of the permutation group.
\label{table:osm_comm}
\end{table} 

%$s\in\Zbb_{2}$ is a binary number, independently owned by $\pp_0$.
%(\ie, $\type =\ raw$) or secret-shared (\ie, $\type =\ shared$) between two parties.
%$x\in\Zbb_{2^L}$ is an arithmetic number, which is shared-owned by two parties because $\SM$ is executed in the intermediate process of secure training.
%If $s=0$, $sx$ returns $0$; if $s=1$, $sx$ returns $x\in\Zbb_{2^L}$.
%We call $s$ the selector since $s$ is the role of choosing a value from $x$ and $0$.
%In the execution of $\Pi_\SM$, $\pp_0$ and $\pp_1$ jointly compute $sx$ in a way that $\pp_0$ does not know $\pp_1$'s inputs and $\pp_1$ does not know $\pp_0$'s inputs.
%After executing the protocol $\Pi_\SM$, $\pp_0$ obtains the share $\l sx\r_0\in\Zbb_{2^L}$, and $\pp_1$ obtains the share $\l sx\r_1\in\Zbb_{2^L}$ such that $\l sx\r_0 + \l sx\r_1=sx$.

%For functionality, $\SM$ is identical to AB-to-B multiplication
%$s b=c$ that outputs the arithmetic share of $c$
%\yuzh{*************}
%Consider two parties $\pp_0$ and $\pp_1$.
%In the $\raw$ case (i.e., the $\type$ of $s$ is $\raw$), $\pp_0$ owns the private $s$ and the share $\l x\r_0$, while $\pp_1$ owns the share $\l x\r_1$.
%Here,
%The shares of $x$ are the output shares from the prior protocol.

\paragraph{Offline Phase.}
The commodity server $\tt$ assists $\pp_0, \pp_1$ to generate a random bit $b\in\Zbb_{2}$, a secret share of a random number $u\in\Zbb_{2^L}$, and correlated randomness $\l bu \r_0 \in \Zbb_{2^L}, \l bu \r_1 \in \Zbb_{2^L}$

\iffalse
correlated randomness $h, \l h\r_0, \l h\r_1\in\Zbb_2$ and $b, \l b \r_0, \l b \r_1,u , \l u \r_0, \l u \r_1\in\Zbb_{2^\bitlen}$ such that
$u=hb,h =\ l h\r_0 + \l h\r_1, b =\ l b\r_0 + \l b\r_1, u = \l u\r_0 + \l u\r_1$.
These randomnesses can be realized by 1) generating the random $\l u\r_0, \l h\r_0, \l h\r_1, \l b\r_0, \l b\r_1$, 2) computing $u=hb$, and 3) splitting $u$ to $\l u\r_0, \l u\r_1$.
\fi

\paragraph{Online Phase.}
%In the $\raw$ case, 
$\pp_0$ masks $s$ using random $b$ to generate random $\delta_s$ (Line~\ref{osm::masks}).
%%, and masks $\l x\r_0$ using $\l b\r_0$ to generate random $\delta_{\l x\r_0}$ (Line~\ref{osm::l13}).
%Line~\ref{osm::l13} in $\Pi_\SM$ is essentially to protect the $\pp_0$'s inputs from viewing by $\pp_1$ since
%Then, $\pp_1$ can only view the masked random values of $\delta_s$ sent by $\pp_0$ in Line~\ref{osm::l15}.
%Thus, $\pp_1$ does not know $s$ and $x$.
%, i.e., the selector denoting which value to choose.
$\pp_1$ masks $\l x\r_1$ using random $\l u\r_1$ to generate random $\delta_{\l x\r_1}$ (Line~\ref{osm::maskx1}).
%Line~\ref{osm::l14} in $\Pi_\SM$ is essentially to protect the $\pp_1$'s inputs from viewing by $\pp_0$ since 
%Thus, $\pp_0$ can only view the masked random values of $\delta_{\l x\r_1}$ sent by $\pp_1$ in Line~\ref{osm::l16}, \ie, not knowing $x$.
%Thus, neither $\pp_0$ or $\pp_1$ knows $x$, i.e., only knowing one share of $x$.
After receiving the masked $\l x\r_1$ and $s$, $\pp_0$ and $\pp_1$ can respectively compute the shares $\l sx\r_0, \l sx\r_1$.


%Finally, after execution of $\Pi_{\SM}(s,x, \type)$, $\pp_0$ obtains the share $\l sx\r_0$, and $\pp_1$ obtains the share $\l sx\r_1$.
%Neither $\pp_0, \pp_1$ knows $s,x$ as they know the only one share of $s,x$.

%In other words, $\pp_0$ selects a value from $x$ (in the form of $\l x \r_0, \l x \r_1$, which are shared-owned) and $0$, while $\pp_1$ does not know which value $\pp_0$ chooses.

%Simultaneously, $\pp_0$ does not know the exact value of $\l x\r_1$ (thus also does not know $x$).


%In the $\shared$ case, the input $s$ is shared-owned by $\pp_0$ and $\pp_1$.
%That is, before the execution of $\Pi_{\SM}$, $\pp_0$ owns the shares $\l s\r_0, \l x\r_0$, meanwhile,
%$\pp_1$ owns the shares $\l s\r_1, \l x\r_1$ such that $\l s\r_0 + \l s\r_1=s, \l x\r_0 + \l x\r_1=x$.
%During the protocol execution, $\pp_0$ and $\pp_1$ jointly compute $sx$.
%After the protocol execution, $\pp_0$ does not know $\pp_1$'s inputs $\l s\r_1, \l x\r_1$ (thus also does not know $s$ and $x$); $\pp_1$ does not know $\pp_0$'s inputs $\l s\r_0, \l x\r_0$ (thus also does not know $s$ and $x$).


%\in\Zbb_{2^L}$ on the binary input of $s\in\Zbb_2$ and the arithmetic share of $b\in\Zbb_{2^L}$.
%
%Our $\SM$ protocol adopts the functionality of oblivious selection~\cite{ccs/AttrapadungH0MO21} for obliviously selecting $s$ from $\{0,1\}$, acting as 1-out-of-2 oblivious transfer (OT).





\iffalse
\begin{definition}[Obliviously-Selected Multiplication]
\label{def::osm}
	Let $\pp_0$ hold a private selection $s\in\{0,1\}$ and an arithmetic share $\l x \r_0$.
	Let $\pp_1$ hold an arithmetic share $\l x\r_1$.
	Obliviously-selected multiplication $\SM$ is that $\pp_0$ and $\pp_1$ collaboratively compute $s\cdot x$ without revealing any inputs, and output $\l s x\r_0, \l sx\r_1$.
\end{definition}
\fi

\iffalse
\begin{protocol}[!t]
	\caption{$\Pi_{\SM}$: Oblivious Selection-Multiplication}\label{fig:osm-main}
		\begin{algorithmic}[1]
		 \STATE {\color{gray}\COMMENT{Offline Phase: Generate Correlated Randomness}}
	\STATE $\tt, \pp_0$: Get $h, \l b\r_0, \langle u \rangle_0 \leftarrow \mathsf{PRF}_0(\mathsf{key}_0 ) $ 
\STATE $\tt, \pp_1$: Get $\l b\r_1, \langle u \rangle_1\leftarrow\mathsf{PRF}_1(\mathsf{key}_1 ) $
\STATE $\tt$: Get $\langle u \rangle_1= hb - \langle u \rangle_0$ and send $\langle u \rangle_1$ to $\pp_1$
		\STATE {\color{gray}\COMMENT{Online Phase: Reconstruct $sx$ in 1 Round}}
		\STATE $\pp_0$: Get $\delta_s= s- h, \delta_{\l x\r_0} =\ l x\r_0 -\l b\r_0$\label{osm::l13}
\STATE $\pp_1$: Get $\delta_{\l x\r_1} = \l x\r_1 - \l b\r_1$\label{osm::l14} 
\STATE $\pp_0$: Send $\delta_s$ to $\pp_1$\label{osm::l15}
\STATE $\pp_1$: Send $\delta_{\l x\r_1}$ to $\pp_0$\label{osm::l16} 
\STATE $\pp_0$: Get $\delta_x= \delta_{\l x\r_0} + \delta_{\l x\r_1}$
\STATE $\pp_0$: Get $\langle sx \rangle_0 = {\color{cyan}s \delta_x + \delta_s \l b\r_0 + (-1)^{\delta_s} \l u\r_0}$\label{osm::l18}
\STATE $\pp_1$: Get $\langle sx \rangle_1 = {\color{ao}\delta_s\l b\r_1 + (-1)^{\delta_s}\l u\r_1}$\label{osm::l19}

		\end{algorithmic}
	%\vspace{-4mm}
\end{protocol}
 \fi

\paragraph{Correctness.}
Here, we 
%Correctness of $\Pi_\SM$ is to 
verify that %the output shares 
${\color{colora}\l sx\r_0} + {\color{colorb}\l sx\r_1}=sx$ %building upon 
by using Lemma~\ref{lem::sxb}
(proven in Appendix~\ref{sec:proof_lemma}).
%Protocol~\ref{fig:osm-main} is which happens to group action of binary element multiplying arithmetic element.

\begin{restatable}{lemma}{sxb}
\label{lem::sxb}
Let $\Abb$ be an Abelian group and $\Bbb =\{0, 1\}$ be the binary group.
Let map $f: \Bbb \times \Abb \rightarrow \Abb$ be defined as $f(s, x) = x \mbox{ if } s = 1 \mbox{ else } 0$.
Then, for any $s,b\in \Bbb$ and $x,u\in\Abb$:
\begin{itemize}
	\setlength{\itemsep}{0pt}
	\setlength{\parskip}{0pt}
	\setlength{\parsep}{0pt}
\item[(i)] $f(s, x + u) = f(s, x) + f(s, u)$.
\item[(ii)] $f(s + b, x) = f(s, x) + (-1)^s f(b, x)$.
\end{itemize}
\iffalse
\begin{equation*}
\begin{array}{l}
(i).\ f(s, x + u) = f(s, x) + f(s, u) \\
%f(s-\tilde{s}, x) = (-1)^{\tilde{s}} (f(s,x)-f(\tilde{s}, x))
(ii).\ f(s + b, x) = f(s, x) + (-1)^s f(b, x).
\end{array}
\end{equation*}
\fi
\end{restatable}

%Take $f$ to be multiplication.
Let $f: \Bbb \times \Abb \rightarrow \Abb$ be the same $f$ as above.
Using Lemma~\ref{lem::sxb}, we have
%the case of $\raw $ can be verified by 
$f(s,x) =f(s, x- u) + f(s, u) 
=f(s, \delta_x) + f(s - b + b, u ) 
=f(s, \delta_x) + f(\delta_s, u) + (-1) ^{\delta_s }f(b, u)=s\delta_x + \delta_s u + (-1)^{\delta_s} (bu)=s\delta_x + \delta_s \l u\r_0 + \delta_s \l u\r_1 + (-1)^{\delta_s} (bu)={\color{colora}s\delta_x + \delta_s \l u\r_0} + {\color{colorb}\delta_s \l u\r_1} + {\color{colora}(-1)^{\delta_s} \l bu\r_0} + {\color{colorb}(-1)^{\delta_s} \l bu\r_1}$.
%When taking $f$ to be multiplication, we attain $sx= s\delta_x + \delta_s b + (-1)^{\delta_s} (hb)$, where $u=hb$.
%The case for `$\shared$' is similar, which is proven in Appendix~\ref{sec::proof_osm_corr}.

\paragraph{Communication.}
$\Pi_\SM$ requires communicating $L + 1$ bits (\ie, $65$ bits for sending $ \delta_{\l x\r_1}, \delta_s$) online in $1$ round.
%in the $\raw$ case.
%or $2L + 2$ sending $\delta_{\l x\r_0}, \delta_{\l x\r_1}, \delta_{\l s\r_0}, \delta_{\l s\r_1}$ in the $\shared$ case.
Except for $\Pi_\SM$, OT based~\cite{pkc/Tzeng02} protocol (by using two OT instances to select $\l x\r$ and $0$) and standard arithmetic multiplication (by transforming binary $s\in\Zbb_2$ into arithmetic $s\in\Zbb_{2^\bitlen}$) can also realize the functionality of section-multiplication.
As compared in Table~\ref{table:osm_comm}, our protocol saves about $50\%$ of communication while having the same round complexity compared to the OT-based~\cite{pkc/Tzeng02} protocol and standard Beaver-triple-based~\cite{crypto/Beaver91a} ($\promult$).
%by regarding $s$ as an arithmetic number 
%, \ie, requiring $2L$ bits online.
%in the $\raw$ case or $4L$ in the $\shared$ case.}


\subsection{Construction of \texorpdfstring{\osmm}{OSMM}}
\label{subsec:prosmm}
Based on our sparse matrix decomposition and protocols for OP and OSM, we present our \osmm protocol $\prosmm$ in Protocol~\ref{fig:osmm}.
It takes a sparse matrix $\adjmat \in \Mbb_{m,n}(\Rcal)$ from $\pp_0$ and a dense matrix $\feamat \in \Mbb_{n,d}(\Rcal)$ from $\pp_1$.
$\prosmm$ generates a secret share $\l \adjmat \feamat \r_i$ for $\pp_i \in \{\pp_0, \pp_1\}$ without leaking $\adjmat$ or $\feamat$.

\paragraph{\osmm Realization.}
Following Theorem~\ref{the::smm_main}, $\prosmm$ essentially performs an ordered sequence of linear transformations ($\sigma_5 \delta_m ^{\top} \gammaout \sigma_4 \Sigma ^{\top} \allowbreak\Lambda \sigma_3 \Sigma \sigma_2 \gammain \delta_n \sigma_1$) from right to left over $\pp_1$'s private input $\feamat$.
%Secure sparse matrix multiplication (\osmm) can be realized by sequentially (\ie, Theorem~\ref{the::smm_main}) integrating the $\ssp$ protocols and local multiplication as Figure~\ref{fig:osmm1}.
%In this way, $\pp_0$'s adjacency matrix is regarded as an ordered sequence of linear transformations ($\sigma_5 \delta_m ^{\top} \Gamma_{k_4} \sigma_4 \Sigma ^{\top} \sigma_3 \Sigma \sigma_2 J_{k_2} \delta_n \sigma_1$) over $\pp_1$'s $\feamat$.
Multiplying public matrices $\delta_n, \Sigma, \Sigma^{\top}, \delta_m^T$ can be done non-interactively on secret shares %$\pp_0$'s and $\pp_1$'s share 
(Lines~\ref{osmm::deltan},~\ref{osmm::Sigma},~\ref{osmm::Sigmat},~\ref{osmm::deltam}).

Permuting the rows of input $\feamat$ or intermediate output $\Ysf$ based on $\sigma_1, \ldots, \sigma_5$ are performed by invoking $d$ parallel ${\Pi}_{\ssp}$ instances as ${\Pi}_{\ssp}$ takes a column vector as input, 
but $\feamat$ and $\Ysf$ are matrices with $d$ columns (Lines~\ref{osmm::sigma1},~\ref{osmm::sigma2},~\ref{osmm::sigma3},~\ref{osmm::sigma4},~\ref{osmm::sigma5}).

Since $\gammain$ and $\gammaout$ are diagonal matrices with binary values, multiplications of them (Lines~\ref{osmm::gammain} and~\ref{osmm::gammaout}) can be done by $nd$ and $md$ parallel $\Pi_\SM$ instances, respectively.\footnote{
In practice, the parties need to pad zero values (non-interactively) before invoking the first $\Pi_\SM$ and cutting off zero values after invoking the last $\Pi_\SM$ to ensure consistent matrix dimensionality.
For simplicity, we omit this step in our $\prosmm$ protocol presentation.
}
% todo: explain why $nd$ and $md$ (tedious details) if have time and space
Multiplication of $\Lambda$, a diagonal matrix with arithmetic values, is performed similarly,
%as above 
but we can use the standard Beaver-triple-based multiplication protocol $\promult$ (Line~\ref{osmm::lambda}) instead of $\Pi_\SM$.

%Secure SMM via Beaver triples requires $O(n^2)$ costs of communication or random-access memory, while plaintext sparse MM mostly targets fast computation instead of low communication.
%Rather than considering each entry value as $\pp_0$'s private inputs,

\begin{protocol}[!t]
\caption{$\prosmm$: Secure Sparse Matrix Multiplication}\label{fig:osmm}
\begin{algorithmic}[1]
	%\item[\textbf{Parameter:} No extra parameter.]
	\REQUIRE $\pp_0$ inputs $\adjmat$ and $\pp_1$ inputs $\feamat$.
	\ENSURE $\pp_0$ gets $\l \Ysf \r_0$ and $\pp_1$ gets $\l \Ysf \r_1$ where $\Ysf = \adjmat\feamat$.
	\STATE $\pp_0$: Decomposes $\adjmat= \sigma_5 \delta_m ^{\top} \gammaout \sigma_4 \Sigma ^{\top} \Lambda \sigma_3 \Sigma \sigma_2 \gammain \delta_n \sigma_1$\label{osmm::decom}
	\STATE $\pp_0, \pp_1$: Invoke $\l \Ysf \r \leftarrow {\Pi}_{\ssp}(\sigma_1;\Xsf)$ \COMMENT{$\type == \raw$}\label{osmm::sigma1}
	\STATE $\pp_0, \pp_1$: Locally compute $\l \Ysf \r = \delta_n \l \Ysf \r$\label{osmm::deltan}
	\STATE $\pp_0, \pp_1$: Invoke $\l \Ysf \r \leftarrow \Pi_\SM(\gammain, \l \Ysf \r_0; \l \Ysf \r_1)$\label{osmm::gammain}
	\STATE $\pp_0, \pp_1$: Invoke $\l \Ysf \r \leftarrow {\Pi}_{\ssp}(\sigma_2, \l \Ysf \r_0; \l \Ysf \r_1)$\label{osmm::sigma2}
	\STATE $\pp_0, \pp_1$: Locally compute $\l \Ysf \r = \Sigma \l \Ysf \r$\label{osmm::Sigma}
	\STATE $\pp_0, \pp_1$: Invoke $\l \Ysf \r \leftarrow {\Pi}_{\ssp}(\sigma_3, \l \Ysf \r_0; \l \Ysf \r_1)$\label{osmm::sigma3}
	\STATE $\pp_0, \pp_1$: Invoke $\l \Ysf \r \leftarrow \promult(\Lambda, \l \Ysf \r_0; \l \Ysf \r_1)$\label{osmm::lambda}
	\STATE $\pp_0, \pp_1$: Locally compute $\l \Ysf \r = \Sigma^\top \l \Ysf \r$\label{osmm::Sigmat}
	\STATE $\pp_0, \pp_1$: Invoke $\l \Ysf \r \leftarrow {\Pi}_{\ssp}(\sigma_4, \l \Ysf \r_0; \l \Ysf \r_1)$\label{osmm::sigma4}
	\STATE $\pp_0, \pp_1$: Invoke $\l \Ysf \r \leftarrow \Pi_\SM(\gammaout, \l \Ysf \r_0; \l \Ysf \r_1)$\label{osmm::gammaout}
	\STATE $\pp_0, \pp_1$: Locally compute $\l \Ysf \r = \delta_m^\top \l \Ysf \r$\label{osmm::deltam}
	\STATE $\pp_0, \pp_1$: Invoke $\l \Ysf \r \leftarrow {\Pi}_{\ssp}(\sigma_5, \l \Ysf \r_0; \l \Ysf \r_1)$\label{osmm::sigma5}
	\RETURN $\l \Ysf \r$
\end{algorithmic}
\end{protocol}

\paragraph{Graph Protection and  Dimensions.}
All entries of $\sigma_5, \sigma_4, \sigma_3, \sigma_2, \sigma_1, \allowbreak\gammaout, \gammain,$ and $\Lambda$ are protected in $\prosmm$.
The dimensions of $\sigma_1$ and $\sigma_5$ are $\numnode$, corresponding to the number of rows in $\Asf$ and $\Xsf$. Since $\Asf$ and $\Xsf$ are held by $\pp_0$ and $\pp_1$, respectively, the dimensions of $\sigma_1$ and $\sigma_5$ are considered reasonable public knowledge.
The dimensions of $\sigma_4, \sigma_3, \sigma_2,$ and $\Lambda$ are $\numedge$, representing tolerable leakage useful for sparsity exploration.
$\gammain$ has dimensions of $\numedge \times \numnode$, while $\gammaout$ is $\numnode \times \numedge$, both of which do not incur extra graph leakage beyond $\numedge$ or $\numnode$.
Importantly, such general statistical information  about the graph does not compromise the privacy of specific nodes or edges or incur identifiable risks.

\paragraph{Correctness.}
$\prosmm$ follows the same sequence of transformations as in Theorem~\ref{the::smm_main}, which shows the correctness of our sparse matrix decomposition.
Since the underlying ${\Pi}_{\ssp}$ and $\Pi_\SM$ protocols are correct, so does
our $\prosmm$ protocol.

%derived by correctly integrating $\ssp$ protocol in an ordered sequence (Theorem~\ref{the::smm_main}).
%Each sub-routing $\ssp$ protocol is correct, and the integration of basic operations is faithful.

\paragraph{Communication.}
$\prosmm$ invokes $5$ ${\Pi}_{\ssp}$, $2$ $\Pi_\SM$, and $1$ $\promult$, communicating $2(2\numnonzero + m + n)dL$ bits offline and $((5\numnonzero + 2m + 2n)\bitlen + 3\numnonzero \log \numnonzero + m\log m + n\log n + m + n)d$ bits online in $8$ rounds (see Table~\ref{table:SMM_comm} for a breakdown).
The Beaver-triple-based protocol $\promult$~\cite{crypto/Beaver91a} for (dense) matrix multiplication communicates $mndL$ bits offline and $2mndL$ bits online.
% in $1$ round.
%Yupeng Zhang's sp17 makes it O(mn+nd) using the same masks for each vector in the matrix.
%We can use the trivial version for simplification.
% t parallel $\promult$ instance at one time, modified to 1 for consistency

In practical GCN usages, we have $m = n = \numnode < t = \numedge \ll \numnode^2 = mn$ and $\log m, \log n, \log t < L = 64$.
Also, $d$ is a relatively small constant.
So, the communication cost of $\prosmm$ is simplified to $O(\numedge)$, rather than $O(\numnode^2)$, by directly using $\promult$ for each entry in SMM.
%If omitting the parameters (\ie, bit-length of data and feature dimensions) irrelevant to the graph topology, $\prosmm$'s communication costs are simplified to $O(m+n+t)$, where $m,n\ll t<mn$ normally, thus equivalent to $O(t)$ in computational complexity.

\begin{table}[!t]
	\centering
	\caption{Cost for \osmm on $\adjmat \in \Mbb_{m,n}(\Rcal)$ and $\feamat \in \Mbb_{n,d}(\Rcal)$}
	\setlength\extrarowheight{2pt}
	\setlength\tabcolsep{1pt}
	\begin{tabular}{l|c|c|c}
	\hline
	\textbf{Protocol} & \textbf{Offline} & \textbf{Online} & \textbf{Rd} \\
	\hline
	\multirow{2}{*}{$5\ {\Pi}_{\ssp}$} & \multirow{2}{*}{$(3t + m + n)d\bitlen$} & $((3t + m + n)\bitlen + 3t \log t$ & \multirow{2}{*}{$5$} \\
	& & $ + m \log m + n \log n)d$ & \\
	\hline
	$2\ \Pi_\SM$ & $(m + n)d\bitlen$ & $((m + n)\bitlen + m + n)d$ & $2$ \\
	\hline
	$1\ \promult$ & $td\bitlen$ & $2td\bitlen$ & $1$ \\
	\hline
	\rowcolor{grayL} & & $((5\numnonzero + 2m + 2n)\bitlen + 3\numnonzero\log \numnonzero$ & \\
	\rowcolor{grayL}\multirow{-2}{*}{$\prosmm$} & \multirow{-2}{*}{$2(2\numnonzero + m + n)dL$} & $ + m\log m + n\log n + m + n)d$ & \multirow{-2}{*}{$8$} \\\hline
	\end{tabular}\\
	%\begin{tablenotes}[para, flushleft]
		$t$: number of non-zero elements in $\adjmat$,
	$\bitlen$: bit-length of data,\\
	$d$: node feature dimensionality, Rd: round
	%\end{tablenotes}
	\label{table:SMM_comm}
\end{table}

%!TEX root = gcn.tex
\section{End-to-End GCN Inference and Training}
\label{sec::secgcn}

\paragraph{Implementation.}
%of $\cgnn$.
$\cgnn$ adopts classical GCN~\cite{iclr/KipfW17} in the transductive setting with
%, which contains 
two graph convolution layers ($\mathsf{GConv}$) 
followed by ReLU and softmax function.
We implement $\cgnn$ using Python
in the TensorFlow framework.
\begin{equation*}
\begin{aligned}
\Xsf,\Asf
	\rightarrow 
\boxed{
	\stackrel{\textstyle{\mathsf{GConv}}}{\boxed{\weimat_{1}}}
}
	\rightarrow 
 	\boxed{ {\mathsf{{ReLU}}}}
	\rightarrow 
\boxed{
	\stackrel{\textstyle{\mathsf{GConv}}}{\boxed{\weimat_{2}}}
}
	\rightarrow
	\Ysf
	\rightarrow 
 	 	\boxed{ {\mathsf{{Softmax}}}} 
 	 	\end{aligned}
 	\end{equation*}
We implemented all the above protocols (detailed in Appendix~\ref{sec::alg}) in $\cgnn$
%, in which all lines of code build 
upon secure computation with secret shares.
Since GCN inherits conventional neural networks, we still rely on similar functions and layers.
Following the ideas of \cite{sp/MohasselZ17,ccs/RatheeRKCGRS20,uss/WatsonWP22,acsac/0021ZCPTLY23,popets/AttrapadungHIKM22},
%in private training, 
we re-implemented the relevant protocols under the 2PC setting in $\cgnn$ for ReLU,
%pooling, 
softmax, Adam optimization, and more.
For non-sparse multiplication, $\cgnn$ still uses Beaver triples~\cite{crypto/Beaver91a}.

\paragraph{Forward Propagation.}
Recall that $\pp_0$ holds the (normalized) adjacency matrix $\Asf$ and $\pp_1$ holds the features $\Xsf$.
The first GCN layer is defined as $\Zsf =\mathsf{ReLU}(\Asf\Xsf \weimat)$, thus $\pp_0$ and $\pp_1$ jointly execute $\Pi_\smm$, $\promult$, and ReLU protocols (combining PPA~\cite{arith/Beaumont-SmithL01}, GMW~\cite{acssc/Harris03}, and $\SM$).
$\pp_0$ and $\pp_1$ will get $\l \Zsf\r_0$ and $\l \Zsf\r_1$, respectively.
%In the forward propagation of the second layer, $\pp_0$ holds $\Asf$ and the share $\l Z_1\r_0$, while $\pp_1$ has the share $\l Z_1\r_1$.
Then, $\pp_0$ and $\pp_1$ securely compute $\Ysf= \Asf \Zsf \weimat$ and $\mathsf{Softmax}(\Ysf)$ in the second layer.
In the output layer, $\pp_0$ and $\pp_1$ jointly execute secure softmax protocol~\cite{acsac/0021ZCPTLY23}.

\paragraph{Backward Propagation.}
$\pp_0$ and $\pp_1$ securely compute $\mathsf{softmax} (\Ysf)-\Ysf'$ using cross-entropy loss to get $\frac{\partial loss}{\partial \Ysf}$, where $\Ysf'$ is the label matrix.
%$Z \in \Mbb_{n, k}$ 
%, i.e., every line of $Z$ is the one-hot vector of the class of the node, and softmax is computed for every line.
Then, we compute the gradient of a graph layer $\frac{\partial loss}{\partial \weimat}=\Zsf^{\top}\Asf^{\top} \frac{\partial loss}{\partial \Ysf}$, where the multiplication of $\Zsf^{\top}\Asf^{\top}$ is \osmm.
If we use the SGD optimizer, $\weimat$ is updated to be $\weimat \leftarrow \weimat - \eta \frac{\partial loss}{\partial W}$,
where $\eta$ is the learning rate.
If we use the Adam optimizer~\cite{iclr/KingmaB14}, $\weimat$ is updated by following the computation of Adam given $\frac{\partial loss}{\partial W}$.
The last step is to securely compute the gradient of ReLU and graph layer similarly.
%If we use other method of optimization, we update $W$ as $W \leftarrow (W, \frac{\partial loss}{\partial W})$
%similarly.

\paragraph{GCN Inference and Training.} 
As for end-to-end secure GCN computations, $\pp_0$ and $\pp_1$ collaboratively execute a sequence of protocols to run a single forward propagation (for inference) or forward and backward propagation iteratively (for training).
%As for end-to-end secure GCN training, $\pp_0$ and $\pp_1$ collaboratively execute a sequence of protocols in the forward and backward propagation iteratively.
$\cgnn$ supports both single-server simulation for multiple hosts and multiple-server execution in a distributed setting.
Using $\cgnn$, researchers and practitioners can realize various GCNs
%models 
using the template of $\texttt{class SGCN}$ (Appendix~\ref{sec::alg}),
%The usage of $\cgnn$ is 
similar to using the TensorFlow framework except that \emph{all computations are over secret shares}.
%!TEX root = gcn.tex

\section{Experiments and Evaluative Results}

We evaluate the performance of our \osmm protocol and $\cgnn$'s private GCN inference/training on three Ubuntu servers with $16$-core Intel(R) Xeon(R) Platinum 8163 2.50GHz CPUs of $62$GB
RAM and NVIDIA-T4 GPU of $16$GB RAM.
We aim to answer the three questions below.

\noindent\textbf{Q1.} \emph{How much communication/memory-efficient and accurate for \cgnn?} (\S\ref{sec::comm_compare_gcn}, \S\ref{sec::smmmemory}, \S\ref{sec::acc_compare_gcn})

\noindent\textbf{Q2.} \emph{How do different network conditions impact the running time of \cgnn's inference and training?} (\S\ref{sec::time_net})

\noindent\textbf{Q3.} \emph{How much efficiency has been improved by \osmm?} (\S\ref{sec:ablation})

%\noindent\textbf{Q3.} \textit{How much communication costs are reduced for secure Adam compared with AdamPriv?} (\S\ref{sec::comp_adam})

{Graph Datasets.}
We consider three publication datasets widely adopted in GCN training: Citeseer~\cite{dl/GilesBL98}, Cora~\cite{aim/SenNBGGE08}, and Pubmed~\cite{ijcnlp/DernoncourtL17}.
Their statistics are summarized in Table~\ref{tab:datasets}.
%Following GCN~\cite{iclr/KipfW17}, the number of samples in training sets is $140, 120$, and $60$, respectively; the samples in test sets are $1000$.
%Each publication is described by a $0/1$-valued word vector indicating the absence/presence of the corresponding word from the dictionary.
\begin{table}[!t]
\centering
\caption{Dataset Statistics}
\setlength\tabcolsep{2pt}
\begin{tabular}{l|rrrr|rr}
\hline
\multicolumn{1}{c|}{\textbf{Dataset}} & \textbf{Node} & \textbf{Edge} & \textbf{Feature} & \textbf{Class} &\textbf{\# Train} &\textbf{\# Test}
\\ \hline
Cora 	 & $2,708$ 	 & $5,429$ 	 & $1,433$ & $7$ & $140$ & $1,000$ \\
Citeseer 	 	 & $3,312$ 	 & $4,732$ 	 & $3,703$ 	 & $6$ & $120$ & $1,000$ \\
Pubmed 	 	 & $19,717$ 	 	 & $44,338$ 	 	 & $500$ 	 & $3$ 	 & $60$ & $1,000$ \\
\hline
\end{tabular}
\begin{tablenotes}
\item \# Train/Test: number of samples in training/test dataset
\end{tablenotes}
\label{tab:datasets}
\end{table}

\iffalse
\begin{itemize}
\item CiteSeer dataset contains $3312$ scientific publications and $6$ classes.
The citation network consists of $4732$ links.
The dictionary consists of $3703$ unique words.

\item Cora dataset contains $2708$ scientific publications and $7$ classes.
The citation network consists of $5429$ links, while the dictionary contains $1433$ unique words.

\item PubMed dataset consists of $19717$ scientific publications from the PubMed database classified into one of three classes.
The citation network consists of $44338$ links.
Each publication in the dataset is described by a TF/IDF weighted word vector from a dictionary consisting of $500$ unique words.
\end{itemize}
\fi


\subsection{Communication of GCN}
\label{sec::comm_compare_gcn}
To evaluate communication costs in $\cgnn$, we record the transmitting data, including frame and MPC-related data in both online and offline phases, across the servers or ports.
%We tested the communication for inference and training by measuring the transmitting data among the servers.
The inference refers to
%one-time 
a forward propagation, while the training involves an epoch of training.
Unlike classical CNN training over independent data points, GCN training feeds up the whole graph (\ie, $1$ batch) in each training epoch, thus no benchmarking for batch sizes.

\paragraph{Secure Training.} 
SecGNN~\cite{tsc/WangZJ23} and  CoGNN~\cite{ccs/ZouLSLXX24} are the only two open-sourced works for secure training with MPC.
SecGNN~\cite{tsc/WangZJ23} is the first work, meanwhile
 CoGNN~\cite{ccs/ZouLSLXX24} and its optimized version CoGNN-Opt are the most recent advances.
Thus, we choose them as \cgnn's counterparts for comparison.
Table~\ref{table:comm_on_off_gcn} shows their comparison results.
In general, $\cgnn$ uses ${\leq}1.3$GB in all cases.
Using SGD, $\cgnn$ uses $0.3075$GB, $0.5400$GB, and $1.2567$GB for training over Cora, Citeseer, and Pubmed.
With Adam, $\cgnn$ costs slightly higher communication due to SGD not needing $1/\sqrt{x}.$
%division and square roots.
The additional
%communication 
costs are $6.2\%$, $3.7\%$, and $0.8\%$ for
%one-epoch 
training.
These differences are related to the sparsity of
%training 
data and the times of gradient update.
All the cases above require less communication costs than CoGNN and SecGNN.
%SecGNN~\cite{tsc/WangZJ23} is the only open-source framework for secure GCN inference and training to our knowledge.
%Thus, we compared accuracy and communication with SecGNN.
%From the SecGNN implementation~\cite{tsc/WangZJ23}, 
%Its communication costs are computed by counting the number of multiplication.
%For fair comparisons, we multiply $64$ by the number of arithmetic multiplications and then count the boolean multiplication to derive the overall communication costs of SecGNN.




\paragraph{Secure Inference.}
Except for CoGNN and SecGNN, we additionally compare \cgnn with the most recent secure inference  work -- OblivGNN~\cite{uss/XuL0AYY24} for comprehensiveness.
Table~\ref{table:comm_on_off_inf} compares the communication costs.
$\cgnn$ requires the lowest communication costs in all cases, reducing by  $\sim50\%$ of OblivGNN and $\sim 80\%$ of CoGNN-Opt.


\iffalse
\begin{figure}[!t]
% \vspace{-5mm}
 	\centering
 	\includegraphics[width = 0.38\textwidth]{./fig_and_tab/fig_comm_inf}
 	\caption{ Communication for Inference}
 	\label{fig_comm_on_off_inf}
 	 	%\end{minipage}
\end{figure}
\fi

\begin{table}[!t]
 	\centering
 	\caption{Communication (GB/epoch) for Training}
 	\label{table:comm_on_off_gcn}
 \setlength\tabcolsep{9pt}
 	\begin{tabular}{l|ccc}
 	\hline
 	\multirow{2}{*}{\textbf{Framework}} & \multicolumn{3}{c}{\textbf{Dataset}}
	\\\cline{2-4}
 & \textbf{Cora} 	 & \textbf{Citeseer} & \textbf{Pubmed}\\\hline
 	SecGNN & 	$18.99$ & $48.21$ & $31.74$\\
    CoGNN & $86.99$ & $202.81$ &$273.25$ \\
    CoGNN-Opt & $0.82$ & $1.4$ & $4.33$\\\hline
 \rowcolor{grayL}$\cgnn$ (SGD) & $0.3075$ & $0.5400$ & $1.2567$ \\
 \rowcolor{grayL}$\cgnn$ (Adam) & $0.3265$ & $0.5600$ & $1.2667$ \\
 	 	\hline
 	 	\end{tabular}
 	\end{table}


\begin{table}[!t]
 	\centering
 	\caption{Communication for Inference}
 	\label{table:comm_on_off_inf}
\setlength\tabcolsep{11pt}
 	\begin{tabular}{l|ccc}
 	\hline
 	\multirow{2}{*}{\textbf{Framework}} & \multicolumn{3}{c}{\textbf{Dataset}}
	\\\cline{2-4}
 & \textbf{Cora} 	 & \textbf{Citeseer} & \textbf{Pubmed}\\\hline
	SecGNN & 	$1$GB & $1.7$GB & $2.5$GB\\
    CoGNN & $85.63$GB & $201.29$GB &$263.59$GB\\
 CoGNN-Opt & $0.5$GB &$0.91$GB & $2.02$GB\\\hline
 OblivGNN-B & $34.32$GB & $61.81$GB &$16.33$GB\\
 OblivGNN &$0.29$GB&$0.41$GB& $1.65$GB\\\hline
    \rowcolor{grayL}$\cgnn$ & $114$MB & $274$MB & $602$MB \\
 	 	\hline
 	 	\end{tabular}
\end{table}

 \iffalse
 	\begin{figure}[!t]
% \vspace{-5mm}
 	\centering
 	\includegraphics[width = 2.5in, height=1.7in]{./fig_and_tab/fig_acc}
 	\caption{Model Accuracy}
 	\label{fig_acc}
 	 	%\end{minipage}
\end{figure}
\fi

\subsection{Memory Usage}
\label{sec::smmmemory}
To avoid extra irrelevancy (\eg, communication), we tested the memory usage
%of all parties 
on a single server, recording the largest observed value.
Table~\ref{table:mem_smm} reports memory usage for training with $\prosmm$ and the standard $\promult$ using Beaver triple.
Both protocols show acceptable results for
%the relatively small 
smaller Cora and Citeseer datasets.
%Simultaneously, 
Yet, $\prosmm$ saves $14.5\%, 20.8\%$ memory with secure SGD,
and $10.5\%, 18.2\%$ memory with secure Adam.

The maximum memory SGD training uses is slightly lower than with Adam, as Adam's optimization requires more memory.
When training over the larger Pubmed dataset,
%we failed to attain reasonable results 
an out-of-memory (OOM) error occurs (marked by $^{\oom}$) when using Beaver triple, whereas the \cgnn with $\prosmm$ supports the stable use (${<}2.7$GB) of memory for all datasets.

\begin{table}[!t]
 	\caption{Maximum Memory Usage (GB) for Training}
 	\label{table:mem_smm}
 	\setlength\tabcolsep{3pt}
 	\centering
 	\begin{tabular}{c|c|c|c|c}
 	\hline
	\textbf{Optimizer} & 	\textbf{Dataset} & \textbf{Protocol} & \textbf{Memory} & \textbf{Reduction} 
	\\
 	\hline 
	\multirow{6}{*}{\textbf{SGD}} &\multirow{2}{*}{\textbf{Cora}} & Beaver & $1.31$ &
	\\
 	 & & \cellcolor{grayL}$\prosmm$ & 	\cellcolor{grayL}$1.12$ & \cellcolor{grayL}$14.5\%$
 	\\\cline{2-5}
 	 & \multirow{2}{*}{\textbf{Citeseer}} & Beaver & $2.07$ &
	\\
 	 & & \cellcolor{grayL}$\prosmm$ & \cellcolor{grayL}$1.64$ & \cellcolor{grayL}$20.8\%$ 
	\\\cline{2-5}
	 & \multirow{2}{*}{\textbf{Pubmed}} & Beaver & ${>}28.82^{\oom}$ & 
	\\
	 & & \cellcolor{grayL}$\prosmm$ & \cellcolor{grayL}$1.94$ & \cellcolor{grayL}${>}93.3\%$
	\\\hline
 	\multirow{6}{*}{\textbf{Adam}} & \multirow{2}{*}{\textbf{Cora}} & Beaver & $1.91$ &
	\\
 	 & & \cellcolor{grayL}$\prosmm$ & 	\cellcolor{grayL}$1.71$ & \cellcolor{grayL}$10.5\%$
	\\\cline{2-5}
 	 & \multirow{2}{*}{\textbf{Citeseer}} & Beaver & $2.75$ &
	\\
 	 & & \cellcolor{grayL}$\prosmm$ & 	\cellcolor{grayL}$2.25$ & \cellcolor{grayL}$18.2\%$ 
	\\\cline{2-5}
	 & \multirow{2}{*}{\textbf{Pubmed}} & Beaver & ${>}28.02^{\oom}$ & 
	\\
	 & & \cellcolor{grayL}$\prosmm$ & \cellcolor{grayL}$2.69$ & \cellcolor{grayL}${>}90.4\%$ 
	\\\hline
 \end{tabular}
 \begin{tablenotes}
 \item  ${\oom}$:  out-of-memory (OOM) error occurs.
 \end{tablenotes}
\end{table}

\subsection{Model Accuracy}
\label{sec::acc_compare_gcn}
We trained the GCN over different datasets from random initialization for $300$ epochs using Adam~\cite{iclr/KingmaB14} with a $0.001$ learning rate.
Our configuration of model parameters (\ie, the dimensionality of hidden layers and the number of samples) follow the original setting~\cite{iclr/KipfW17}.
Since model accuracy is meaningful only for identical partitioning strategy, we compare the accuracy of secure training with plaintext in the same contexts in Table~\ref{tab:acc_inf_tra}.
Our results show that $\cgnn$'s accuracy is comparable to that of plaintext~training.
Specifically, $\cgnn$ achieves $\{73.5\%, 64.4\%, 75.4\%\}$ after $100$ epochs and $\{76.0\%, 65.1\%, 75.2\%\}$ after $300$ epochs.
Due to fluctuated training convergence, fixed-point representation, and non-linear approximation, model accuracy is slightly different.

%%%%%% Keep it in case reviewers ask
%In SecGNN, the number of samples in training sets is $40$ per class, and the number of samples in the test set is $500$.}, which reduces the training difficulty.
%instead of the standard graph datasets as in plaintexts, which reduces the training difficulty.
%Specifically, in plaintext training and $\cgnn$, the number of samples in training sets is $140, 120$, and $60$, respectively;
%the samples in test sets are $1000$.
%In SecGNN, the number of samples in training sets is $40$ per class, and the number of samples in the test set is $500$.
 
\begin{table}[!t]
 	\centering
 	\caption{Model Accuracy}
 	\label{tab:acc_inf_tra}
 	\setlength\tabcolsep{7pt}
 	\begin{tabular}{l|ccc}
	\hline
 	\multirow{2}{*}{\textbf{Framework}} & \multicolumn{3}{c}{\textbf{Dataset}}
	\\\cline{2-4}
 & \textbf{Cora} &\textbf{Citeseer} & \textbf{Pubmed}\! \\
 	 	\hline
	{Plaintext}
	%\tnote{*}
	& $75.7\%$ 	 & $65.4\%$ & $74.5\%$\! \\
% 	SecGNN & $78.0\%$ & $68.3\%$ & $78.6\%$ 	\\
 	\rowcolor{grayL}$\cgnn$ ($100$ Epochs)\! & $73.5\%$ & $64.4\%$ & \textbf{75.4\%}\! \\
 	\rowcolor{grayL}$\cgnn$ ($300$ Epochs)\! & \textbf{76.0\%} & \textbf{65.1\%} & $75.2\%$\! \\
 	 	\hline
 	 	\end{tabular}
 	\iffalse
 	\begin{tablenotes}
 	\item {The results of ``plaintext'' are from running dgl at {\url{https://github.com/dmlc/dgl}}.
 	The results of ``SecGNN'' are from running the code at \url{https://github.com/songleiW/SecGNN.git}.
 	}
% In Plaintext and SuGar, the accuracy is after $300$-epoch training.
%In SecGNN, the accuracy is after $30$-epoch training.
 	\end{tablenotes}
 	\fi
\end{table}

\subsection{Running Time in Different Networks}
\label{sec::time_net}
We simulate real-world deployment under different network conditions for \osmm, private inference, and private training.
In particular, we consider a normal network condition ($800$Mbps, $0.022$ms) and two poor network conditions, including a narrow-bandwidth (N.B.) network ($200$Mbps, $0.022$ms) and high-latency (H.L.) network ($800$Mbps, $50$ms).
Additionally,  TCP transmission involves the process of three-step handshake, data transmission, congestion control, and connection termination, thus practical time delay of $\osmm$ is varied below under different network conditions.
\begin{table*}[!t]
	\centering
		\caption{$1$-Epoch Training Time (seconds) in Normal, Narrow-Bandwidth, or High-Latency Networks}
		\label{table:tra_net}
  \setlength\tabcolsep{8pt}
			\begin{tabular}{l|l|r|r|r|r|r|r}
				\hline
     \multirow{2}{*}{\textbf{Dataset}} & \multirow{2}{*}{\textbf{{Protocol}}} & \multicolumn{2}{c|}{\textbf{Normal ($800$Mbps, $0.022$ms)}}& \multicolumn{2}{c|}{\textbf{N.B. ($200$Mbps, $0.022$ms)}}& \multicolumn{2}{c}{\textbf{H.L. ($800$Mbps, $50$ms)}}
     \\\cline{3-8}
     & & \phantom{sherman}{\textbf{SGD}} 
     & {\textbf{Adam}}
     & \phantom{sherman}{\textbf{SGD}}
     & \phantom{123}{\textbf{Adam}}
     & \phantom{sherman}{\textbf{SGD}}
     & \phantom{123}{\textbf{Adam}}\\
     %& &\textbf{Time}&\textbf{Time}&\textbf{Time}&\textbf{Time}&\textbf{Time}&\textbf{Time}\\
				\hline
    \multirow{3}{*}{\textbf{Cora}} & Beaver &$6.55 $ &$7.89 $  &$25.98 $   &$27.55 $   &$11.70 $  &$19.57 $  \\
	  &$\prosmm$&$4.20 $  &  $ 5.55$  &  $13.29 $  &  $14.88 $  
   & \raggedleft $9.11$
   &  $16.72 $   \\ 
     & (Saving) 
     & \cellcolor{grayL}{$35.9\%$} 
     & \cellcolor{grayL}{$29.7\%$} 
     & \cellcolor{grayL}{\textbf{48.8\%}} 
     & \cellcolor{grayL}{\textbf{46.0\%}} 
     & \cellcolor{grayL}{$22.1\%$}
     & \cellcolor{grayL}{$14.6\%$}
     \\\hline
     
  
   \multirow{3}{*}{\textbf{Citeseer}} & Beaver &$11.66 $   &$ 13.20$  &$46.31 $  &$48.75 $  &$ 18.53$  &$27.93 $  \\
   
    & $\prosmm$
    & \raggedleft $6.77$ 
    & \raggedleft $8.35$  & $24.00 $  & $26.44 $  & $ 13.47$  & $ 21.26$  \\
      & (Saving) & \cellcolor{grayL}{$41.9\%$} &\cellcolor{grayL}{$36.7\%$} & \cellcolor{grayL}{\textbf{48.2\%}} & \cellcolor{grayL}{\textbf{45.8\%}}& \cellcolor{grayL}{$27.3\%$}&\cellcolor{grayL}{$23.9\%$} \\\hline
    
     \multirow{2}{*}{\textbf{Pubmed}} & Beaver &OOM  &OOM& OOM& OOM& OOM& OOM  \\    
 & $\prosmm$  &$ 22.87 $   &$24.45 $ & $63.69 $   &$63.58 $   &$32.00 $  &$39.86 $ \\
	\hline
	\end{tabular} 
\end{table*}

\paragraph{Inference Time.}
Table~\ref{table:inf_time_net} compares the private inference time,
%in the varying networks.
%Inference time contains the time of simulating a practical inference on the cloud server, 
including TensorFlow-Graph construction and forward-propagation computation of GCN in varying network conditions
%We conducted one-time inference experiments in each network type 
over multiple
%Cora, Citeseer, and Pubmed 
datasets.
Compared to adopting Beaver triples, $\cgnn$ via $\prosmm$ is ${\sim}7\%$-$19\%$ faster in the normal network, ${\sim}35\%$-$45\%$ quicker in the narrow-bandwidth one, and saves ${\sim}6\%$-$17\%$ time in the high-latency setting.
%In the normal network, $\cgnn$ via $\prosmm$ is ${\sim}7\%$-$19\%$ faster than adopting Beaver triples.
%In the narrow-bandwidth network, $\cgnn$ via $\prosmm$ is ${\sim}35\%$-$45\%$ quicker, while $\cgnn$ via $\prosmm$ is ${\sim}6\%$-$17\%$ faster in the high-latency setting.
The OOM problem prevents us from evaluating inference over Pubmed using Beaver triples, 
%For inference over Pubmed, the OOM problem prevents us from obtaining evaluation using Beaver triples, 
while $\prosmm$ takes ${\sim}30$-$50$s.

%!TEX root = PPGraphNN.tex
\begin{table}[!t]
    \centering
        \caption{Inference Time (seconds) in Varying Networks}
        \label{table:inf_time_net}
  \setlength\tabcolsep{5pt}
            \begin{tabular}{c|c|r|r|r}
                \hline
     \textbf{Dataset} & \textbf{Protocol} & \textbf{Normal} & \textbf{N.B.} & \textbf{H.L.}\\
     %\multirow{2}{*}{\textbf{Dataset}} & \multirow{2}{*}{\textbf{Protocol}} & {\textbf{Normal}}&  {\textbf{N.B.}}&  {\textbf{H.L.}}\\
     %& &\textbf{Time} &\textbf{Time} &\textbf{Time} \\
                \hline
    \multirow{3}{*}{\textbf{Cora}} & Beaver &$17.48$  &  $28.34$ &   $24.22$    \\
       &$\prosmm$&$16.27 $  &  $ 21.06$  &  $22.76 $   \\ 
       & (Saving) &{$6.9\%$}& \cellcolor{grayL}{$25.7\%$} & {$ 6.0\%$}  \\\hline
   \multirow{3}{*}{\textbf{Citeseer}}& Beaver & $24.57 $    &$ 44.39$  &$33.32$    \\
     &$\prosmm$&$ 20.58 $    &$30.57 $    &$28.49 $     \\ 
     & (Saving) & \cellcolor{grayL}{$16.2\%$} & \cellcolor{grayL}{$31.3\%$} & \cellcolor{grayL}{$14.5\%$}   \\\hline
     \multirow{2}{*}{\textbf{Pubmed}} & Beaver & OOM  &  OOM    & OOM  \\
 & $\prosmm$ &$ 29.38$     &$49.93 $  &$38.40 $  \\
    \hline
            \end{tabular}

N.B.: narrow bandwidth,
H.L.: high latency
\end{table}


\input{./fig_and_tab/tab_tra_network.tex}

\paragraph{Training Time.}
Table~\ref{table:tra_net} compares the private training time
%equipped 
with SGD/Adam in varying network conditions over different datasets.
%the varying networks.
%For reliable results, w
We tested $10$ epochs and got the average.
%In each type of network, we tested the time of one-epoch training over Cora, Citeseer, and Pubmed datasets.
In the normal network, $\cgnn$ via $\prosmm$ is ${\sim}56\%$-$73\%$ faster with SGD and ${\sim}42\%$-$58\%$ faster with Adam.
In the narrow-bandwidth network, $\cgnn$ via $\prosmm$ is ${\sim}93\%$-$95\%$ quicker with SGD and ${\sim}84\%$-$85\%$ quicker with Adam.
Besides, $\cgnn$ via $\prosmm$ is ${\sim}28\%$-$38\%$ faster with SGD and ${\sim}17\%$-$32\%$ faster with Adam in the high-latency setting.
%For private training over Pubmed, we encountered the same issue as that was happening in private inference.

\begin{table}[!t]
 	\centering
 	\caption{Communication Costs (MB) for SMM}
 	\label{table:comm_smm}
 	\setlength\tabcolsep{7pt}
 	 	\begin{tabular}{c|c|r|r|r}
 	 	\hline
 	\textbf{$\#$E/N} & \textbf{$\#$Node} & \textbf{Beaver} & ${\prosmm}$ & \textbf{Saving} \\
 	 	\hline
 	$1$ & $1000$ & $25.1$ & $0.8$ & \cellcolor{grayL}{$ 96.8\%$} \\
 	$1$ & $2000$ & $ 100.3$ & $1.3$ & \cellcolor{grayL}{$ 98.7\%$}\\
% 	$1$ & $3000$ & $225.5$ & $1.8$ & 	{$ 99.2\%$} \\
% $1$ & $4000$ & $400.7 $ & $2.3$ & {$ 99.4\%$} \\
 	$1$ & $5000$ & $626.1$ & $2.8$ & \cellcolor{grayL}{\textbf{99.6\%}} 	\\
 	\hline
 	$2$ & $1000$ & $25.1$ & $1.0$ & \cellcolor{grayL}{$ 95.9\%$} \\
 	$2$ & $2000$ & $100.3$ & $1.8$ & \cellcolor{grayL}{$ 98.2\%$} \\
% $2$ & $3000$ & $225.5$ & $2.5$ & 	{$ 98.9\%$} \\
% $2$ & $4000$ & $400.7$ & $3.2$ & {$ 99.2\%$} \\
 	$2$ & $5000$ & $626.0$ & $3.9$ & 	\cellcolor{grayL}{\textbf{99.4\%}} \\
 	\hline
 	$3$ & $1000$ & $25.1$ & $1.3$ & \cellcolor{grayL}{$ 95.0\%$} \\
 	$3$ & $2000$ & $100.3$ & $2.2$ & \cellcolor{grayL}{$ 97.8\%$} \\
% $3$ & $3000$ & $225.5$ & $3.2$ & {$ 98.6\%$} 	\\
% $3$ & $4000$ & $400.8$ & $4.1$ & {$ 99.0\%$} 	\\
 	$3$ & $5000$ & $626.1$ & $5.1$ & \cellcolor{grayL}{\textbf{99.2\%}} 	\\
 	\hline
 	 	\end{tabular} 
 	 	
 	$\#$E/N: ratio of edges per node,
	\\``Beaver'': using Beaver triples for SMM
\end{table}

\begin{table}[!t]
 	\centering
 	\caption{Memory Usage (MB) Given Varying Sparsity}
 	\label{table:mem_smm_sparse}
 \setlength\tabcolsep{7pt}
 	 	\begin{tabular}{c|c|r|r|r}
 	 	\hline \textbf{$\#$E/N} & \textbf{$\#$Node} & \textbf{Beaver} & ${\prosmm}$ & \textbf{Saving} \\
 	 	\hline
 	$1$ & $1000$ & $688.6$ & $572.7$ & \cellcolor{grayL}{$ 16.83\%$} \\
 	$1$ & $2000$ & $1236.5$ & $575.9 $ & \cellcolor{grayL}{$ 53.42\%$}\\
% $1$ & $3000$ & $1111.2$ & $ 578.5$ & 	{$ 47.94\%$} \\
% $1$ & $4000$ & $1553.1$ & $581.4 $ & {$ 62.57\%$} \\
 	$1$ & $5000$ & $2136.0$ & $ 583.7$ & \cellcolor{grayL}{\textbf{72.67\%}} \\
 	\hline
 	$2$ & $1000$ & $680.6$ & $575.1 $ & \cellcolor{grayL}{$ 15.50\%$} 	\\
 	$2$ & $2000$ & $1173.4$ & $ 579.8$ & \cellcolor{grayL}{$ 50.59\%$} \\
% $2$ & $3000$ & $1111.6$ & $582.8 $ & 	{$ 47.57\%$} \\
% 	$2$ & $4000$ & $1552.4$ & $ 588.9$ & {$ 62.07\%$} \\
 	$2$ & $5000$ & $2135.8$ & $596.1 $ & 	\cellcolor{grayL}{\textbf{72.09\%}} \\
 	\hline
 	$3$ & $1000$ & $719.2$ & $ 578.5$ & \cellcolor{grayL}{$ 19.56\%$} 	\\
 	$3$ & $2000$ & $1142.0$ & $582.2 $ & \cellcolor{grayL}{$ 49.02\%$} \\
% $3$ & $3000$ & $1111.3$ & $590.1 $ & {$ 46.90\%$} \\
% $3$ & $4000$ & $1553.7$ & $ 599.8$ & {$ 61.40\%$} \\
 	$3$ & $5000$ & $2136.8$ & $ 605.3$ & \cellcolor{grayL}{\textbf{71.67\%}} \\
 	\hline
 	 	\end{tabular}
% 	$\#$E/N: ratio of edges per node.
%\\``Beaver'': using Beaver triples for SMM.
\end{table}
\subsection{Ablation Study for \texorpdfstring{\osmm}{OSMM}}
We perform extensive experiments 
\label{sec:ablation}
%\textit{with or without \osmm} 
to study the computational, communication, and memory costs saved by \osmm.
Due to saving space, we defer some experimental results to Appendix~\ref{sec::more_exp}.

\paragraph{Communication.} 
\label{sec::smmcomm}
Table~\ref{table:comm_smm} reports communication comparison given varying sparsity of matrices with ${\sim}1000$-$5000$ nodes, each with $\{1, 2, 3\}$ edges on average.
In a training epoch, Beaver triples cost ${\sim}25$-$626$MB for sparse MM, whereas $\prosmm$ spends relatively stable costs of roughly
\mbox{${\sim}1$-$5$MB.}
$\prosmm$ reduces $95\%+$ communication compared with standard MM in all cases.
At best, $\prosmm$ costs only $0.4\%$ communication of standard one when $\#$Node is $5000$ with $1$ edge on average (also the sparsest case in Table~\ref{table:comm_smm}).



%\ak{Maybe can present results only for $1000$, $3000$, and $5000$ nodes (or $1000$, $2000$, and $5000$) to save space as there is nothing surprising for $2000$, $3000$, and $4000$ nodes.
%Similarly for Table~\ref{table:mem_smm_sparse} and~\ref{table:time_smm_net_10}}

\paragraph{Memory Usage.}
Table~\ref{table:mem_smm_sparse} shows how sparsity affects memory usage.
$\#$Node is the total number of nodes and $\#$Edge/Node is the average number of edges connected per node.
Memory usage via Beaver triples scales with $\#$Node, whereas $\prosmm$ maintains relatively stable use.
In detail, $\prosmm$ reduces ${\sim}16\%$-$73\%$ memory for ${\sim}1000$-$5000$ nodes.


\paragraph{Running Time under Varying Network Conditions.}
Table~\ref{table:time_smm_net_10} reports the running time of $10$ epochs of \osmm in the normal, narrow-bandwidth, and high-latency networks.
In the normal network, $\prosmm$ achieves a ${\sim}1.1$-$26.3\times$ speed-up compared with Beaver triples.
In the narrow-bandwidth network, $\prosmm$ is ${\sim}1.53$-$41.96\times$ faster, showing a higher speed-up than the normal network.
In the high-latency network, $\prosmm$ shows a slightly lower speed-up than the normal network.
 The reason is that $\prosmm$ uses more rounds of communication than Beaver triples.
 It would be interesting to explore reducing round complexity for \osmm in the future.

%If we compute the multiplication over a relatively dense matrix in the high-latency network, Beaver triples are slightly better since they require one round only.


\paragraph{Running Time with Varying Dimensionality.}
In practice, feature dimensionality (\eg, salary, life cost) is not very high.
We vary it across $\{10, 20, 50\}$ over the Citeseer dataset in Figures~\ref{fig:time_fea_dim_inference_cite},~\ref{fig:time_fea_dim_train_cite} (results for other datasets are in Appendix~\ref{sec::more_exp}).
%\ref{table:time_fea_dim_infer_cora},\ref{table:time_fea_dim_train_cora},\ref{table:time_fea_dim_infer_pubmed},\ref{table:time_fea_dim_train_pubmed}.
We test both inference and training times.
The fewer feature dimensions, the higher the percentage of costs is from SMM.
Roughly, the time costs have been reduced by ${\sim}50$-$75\%$.



\begin{figure}[!t]
	\centering
	\includegraphics[width = 0.42\textwidth]{./fig_and_tab/inference_time_plot.png}
		\caption{Inference Time with Feature Dimensionality}
 	\label{fig:time_fea_dim_inference_cite}
		%\end{minipage}
\end{figure}
\begin{figure}[!t]
	\centering
	\includegraphics[width = 0.42\textwidth]{./fig_and_tab/training_time_plot_updated.png}
		\caption{Training Time with  Feature Dimensionality}
 	\label{fig:time_fea_dim_train_cite}
		%\end{minipage}
\end{figure}


\input{8-related}

\section{Conclusion}
%Our work sheds new light on optimizing the communication-intensive sparse matrix multiplication operation in private GCNs over vertically partitioned data.
%handling \osmm in the GCN scenario, where arbitrary sparsity can be decomposed into a sequence of linear transformations.

We propose $\cgnn$, a secure $2$PC framework for GCN inference and training over vertically partitioned data, a neglected MPC scenario motivated by cross-institutional business collaboration.
It is supported by our \osmm protocol using a sparse matrix decomposition method for converting an arbitrary-sparse matrix into a sequence of linear transformations and employing $1$-round MPC protocols of oblivious permutation and selection-multiplication for efficient secure evaluation of these linear transformations.

Our work provides an open-source baseline and extensive benchmarks for practical usage.
Theoretical and empirical analysis demonstrate $\cgnn$'s superior communication and memory efficiency in private GCN computations.
Hopefully, our insight could motivate further research on private graph learning.
%, such as graph transformers.
%, by referring to the transformer layer as a fully connected ``word'' graph.

\section*{Acknowledgments}

Yu Zheng sincerely appreciates the valuable discussion, editorial helps or comments from  Andes Y.L. Kei, Zhou Li, Sherman S.M. Chow, Sze Yiu Chau, and Yupeng Zhang. 
%Motivated by business requirements in practice, our research explores a new MPC scenario in which different participating parties own diverse types of private inputs, such as graph structures or features.

% Acknowledge: Zhou Li and Yupeng Zhang Andes Kei, Sherman Chow
\iffalse
\section*{Ethics Considerations and Compliance with the Open Science Policy}
To our knowledge, this work does not raise ethical issues.
In particular, all data used in this paper are publicly available.

Our code\textsuperscript{\ref{footnote:artifact}} will be open-sourced for \textit{future non-profitable academic research}.
Part of the implementation was productized at an anonymized company.
\fi

\bibliography{reference}
\bibliographystyle{ACM-Reference-Format}


\appendix

%\clearpage
%\newpage

%!TEX root = gcn.tex

\section{More Experimental Results}
\label{sec::more_exp}
Tables~\ref{table:time_fea_dim_infer_cora},\ref{table:time_fea_dim_infer_cite},\ref{table:time_fea_dim_infer_pubmed},\ref{table:time_fea_dim_train_cora},\ref{table:time_fea_dim_train_cite},\ref{table:time_fea_dim_train_pubmed} present the inference and training time with varying feature dimensionality over Cora and Pubmed datasets.
The results align with our conclusion in Section~\ref{sec:ablation}.
 

\begin{table*}[!t]
 	\centering
 	\caption{Inference Time (seconds) with Varying Feature Dimensionality over Cora}
 	\label{table:time_fea_dim_infer_cora}
 \setlength\tabcolsep{8pt}
 	 	\begin{tabular}{c|c|c|c|c|c|c|c|c|c}
 	 	\hline
 \multirow{2}{*}{\textbf{$\#$Dim}} &\multicolumn{3}{c|}{\textbf{$50$Mbps}} &\multicolumn{3}{c|}{\textbf{$100$Mbps}}&\multicolumn{3}{c}{\textbf{$200$Mbps}}
 \\\cline{2-10}
 	& \textbf{Beaver} & ${\prosmm}$ & \textbf{Saving} & \textbf{Beaver} & ${\prosmm}$ 	& \textbf{Saving} & \textbf{Beaver} & ${\prosmm}$ & \textbf{Saving} \\
 	 	\hline
 	 $1433$&$64.18$&$33.53$&$47.8\%$&$35.99$&$20.74$&$ 42.4\%$&$22.03$& $14.62$& $33.6\%$\\
 	 $10$&$51.96$&$19.55$&$62.4\%$&$29.56$&$14.33$&$51.5\%$&$18.89$& $10.80$& $42.8\%$\\
 	 $20$&$52.59$&$19.06$&$63.8\%$&$30.25$&$13.86$&$54.2\%$&$18.48$& $10.62$& $42.5\%$\\
 	 $50$&$52.04$&$19.42$&$62.7\%$&$29.71$&$13.96$&$53.0\%$&$18.69$& $11.38$& $39.1\%$\\\hline
 	%$100$&$52.38$&$20.37$&$61.1\%$&$30.53$&$14.49$&$52.5\%$&$18.39$& $10.67$& $42.0\%$\\\hline
 	 \end{tabular}
 	
 	$\#$E/N: ratio of edges per node,
    ``Beaver'': using Beaver triples for MM.
\end{table*}

\begin{table*}[!t]
 	\centering
 	\caption{Inference Time (seconds) with Varying Feature Dimensionality over Citeseer}
 	\label{table:time_fea_dim_infer_cite}
 \setlength\tabcolsep{8pt}
 	 	\begin{tabular}{c|c|c|c|c|c|c|c|c|c}
 	 	\hline
 \multirow{2}{*}{\textbf{$\#$Dim}} &\multicolumn{3}{c|}{\textbf{$50$Mbps}} &\multicolumn{3}{c|}{\textbf{$100$Mbps}} &\multicolumn{3}{c}{\textbf{$200$Mbps}}
 \\\cline{2-10}
 	 & \textbf{Beaver} & ${\prosmm}$ & \textbf{Saving} & \textbf{Beaver} & ${\prosmm}$ 	 & \textbf{Saving} & \textbf{Beaver} & ${\prosmm}$ & \textbf{Saving}
 	\\\hline
 	$3703$ &$117.83$ &$62.36$ &$47.1\%$ &$64.04$ &$35.78$ &$44.1\%$ &$37.36$ & $23.29$ & $37.7\%$\\
	$10$ &$77.18$ &$20.69$ &$73.2\%$ &$43.22$ &$14.79$ &$65.8\%$ &$26.31$ & $10.66$ & \textbf{59.5\%}\\
 	$20$ &$76.81$ &$20.11$ &\textbf{73.8\%} &$43.59$ &$14.39$ &$67.0\%$ &$25.95$ & $11.06$ & $57.4\%$\\
 	 $50$ &$77.60$ &$20.78$ &$73.2\%$ &$43.69$ &$ 14.31$ &\textbf{67.2\%} &$26.71$ & $11.41$ & $57.3\%$\\\hline
 	%$100$ &$77.90$ &$21.23$ &$72.7\%$ &$43.81$ &$ 14.50$ &$66.9\% $ &$26.89$ & $11.47$ & $46.1\%$\\\hline
 	 \end{tabular}
% 	$\#$E/N: ratio of edges per node.
%``Beaver'': using Beaver triples for MM.
\end{table*}

 \begin{table}[!t]
 	\centering
 	\caption{Inference Time (seconds)
  %with Varying Feature Dimensionality
  over Pubmed}
 	\label{table:time_fea_dim_infer_pubmed}
 \setlength\tabcolsep{2pt}
 	 	\begin{tabular}{c|c|c|c|c|c|c}
 	 	\hline
 \multirow{2}{*}{\textbf{$\#$Dim}} &\multicolumn{2}{c|}{\textbf{$50$Mbps}} &\multicolumn{2}{c|}{\textbf{$100$Mbps}}&\multicolumn{2}{c}{\textbf{$200$Mbps}}
 \\\cline{2-7}
 	& \textbf{Beaver} & ${\prosmm}$ & \textbf{Beaver} & ${\prosmm}$ 	& \textbf{Beaver} & ${\prosmm}$ \\
 	 	\hline
 	 	 $500$&OOM&$132.06$ &OOM&$69.66$&OOM& $42.37$\\
 	 $10$&OOM&$92.70$&OOM &$53.35$&OOM& $32.77$\\
 	 $20$&OOM&$92.94$&OOM &$53.88$&OOM& $34.22$\\
 	 $50$&OOM&$93.87$&OOM &$54.67$&OOM& $33.49$\\\hline
 % $100$&OOM&$96.99$&OOM &$56.37$&OOM& $34.45$\\\hline
 	 \end{tabular}
\end{table}


\begin{table*}[!t]
 	\centering
 	\caption{Training Time (seconds) with Varying Feature Dimensionality over Cora}
 	\label{table:time_fea_dim_train_cora}
\setlength\tabcolsep{8pt}
 	 	\begin{tabular}{c|c|c|c|c|c|c|c|c|c}
 	 	\hline
 \multirow{2}{*}{\textbf{$\#$Dim}} &\multicolumn{3}{c|}{\textbf{$50$Mbps}} &\multicolumn{3}{c|}{\textbf{$100$Mbps}}&\multicolumn{3}{c}{\textbf{$200$Mbps}}
 \\\cline{2-10}
 	& \textbf{Beaver} & ${\prosmm}$ & \textbf{Saving} & \textbf{Beaver} & ${\prosmm}$ 	& \textbf{Saving} & \textbf{Beaver} & ${\prosmm}$ & \textbf{Saving} \\
 	 	\hline
 	 	 $1433$&$106.31$&$50.50$&$52.5\%$&$53.19$&$25.79$&$51.5\%$&$22.03$& $13.58$& $38.4\%$\\
 	 $10$&$86.05$&$29.75$&$65.4\%$&$43.05$&$15.62$&$63.7\%$&$18.89$& $8.42$& $55.4\%$\\
 	 $20$&$86.18$&$29.89$&$65.3\%$&$43.13$&$15.47$&$64.1\%$&$18.48$& $8.48$& $54.1\%$\\
 	 $50$&$86.62$&$30.30$&$65.0\%$&$43.34$&$15.77$&$63.6\%$&$18.69$& $8.65$& $53.7\%$\\\hline
 	%$100$&$87.34$&$31.02$&$64.5\%$&$43.66$&$16.20$&$62.9\%$&$18.39$& $8.77$& $52.3\%$\\\hline
 	 \end{tabular}
\end{table*}

\begin{table*}[!t]
 	\centering
 	\caption{Training Time (seconds) with Varying Feature Dimensionality over Citeseer}
 	\label{table:time_fea_dim_train_cite}
 \setlength\tabcolsep{8pt}
 	 	\begin{tabular}{c|c|c|c|c|c|c|c|c|c}
 	 	\hline
 \multirow{2}{*}{\textbf{$\#$Dim}} &\multicolumn{3}{c|}{\textbf{$50$Mbps}} &\multicolumn{3}{c|}{\textbf{$100$Mbps}} &\multicolumn{3}{c}{\textbf{$200$Mbps}}
 \\\cline{2-10}
 	 & \textbf{Beaver} & ${\prosmm}$ & \textbf{Saving} & \textbf{Beaver} & ${\prosmm}$ 	 & \textbf{Saving} & \textbf{Beaver} & ${\prosmm}$ & \textbf{Saving} \\
 	 	\hline
 	 	 $3703$ &$190.04$ &$94.60$ &$50.2\%$ &$95.06$ &$47.86$ &$49.7\%$ &$47.65$ & $24.58$ & $48.4\%$\\
 	 $10$ &$125.78$ &$30.08$ &\textbf{76.1\%} &$62.91$ &$14.79$ &\textbf{76.5\%} &$31.54$ & $8.35$ & \textbf{73.5}\%\\
 	 $20$ &$125.97$ &$30.23$ &$76.0\%$ &$63.03$ &$15.75$ &$75.0\%$ &$31.57$ & $8.46$ & $73.2\%$\\
 	 $50$ &$126.47$ &$30.83$ &$75.6\%$ &$63.29$ &$15.98$ &$74.8\%$ &$31.68$ & $8.58$ & $72.9\%$\\\hline
%$100$ &$127.33$ &$31.67$ &$75.1\%$ &$63.68$ &$16.33$ &$ 74.4\%$ &$31.91$ & $8.82$ & $72.4\%$\\\hline
 	 \end{tabular}
% 	$\#$E/N: ratio of edges per node.
%``Beaver'': using Beaver triples for MM.
\end{table*}


\begin{table}[!t]
 	\centering
 	\caption{Training Time (seconds) over Pubmed}
 	\label{table:time_fea_dim_train_pubmed}
 \setlength\tabcolsep{3pt}
 	 	\begin{tabular}{c|c|c|c|c|c|c}
 	 	\hline
 \multirow{2}{*}{\textbf{$\#$Dim}} &\multicolumn{2}{c|}{\textbf{$50$Mbps}} &\multicolumn{2}{c|}{\textbf{$100$Mbps}}&\multicolumn{2}{c}{\textbf{$200$Mbps}}
 \\\cline{2-7}
 	& \textbf{Beaver} & ${\prosmm}$ & \textbf{Beaver} & ${\prosmm}$ & \textbf{Beaver} & ${\prosmm}$ \\
 	 	\hline
 	 	 $500$&OOM&$233.53$ &OOM&$118.23$&OOM& $64.16$\\
 	 $10$&OOM&$175.14$&OOM &$92.90$&OOM& $51.27$\\
 	 $20$&OOM&$176.75$&OOM &$92.52$&OOM& $51.72$\\
 	 $50$&OOM&$180.08$&OOM &$93.53$&OOM& $52.71$\\\hline
% 	$100$&OOM&$184.61$&OOM &$97.68$&OOM& $53.74$\\\hline
 	 \end{tabular}
$\#$E/N: edges per node ratio,
``Beaver'': Beaver triples for MM
\end{table}
\begin{table*}[!t]
 	\centering
 	\caption{$10$-Epoch Running Time (seconds) of \osmm in Varying Networks}
 	\label{table:time_smm_net_10}
 \setlength\tabcolsep{7pt}
 	 	\begin{tabular}{c|c|r|r|r|r|r|r|r|r|r}
 	 	\hline
 	\multirow{2}{*}{\textbf{$\#$E/N}} & 
 	\multirow{2}{*}{\textbf{$\#$Node}} & 
 	\multicolumn{3}{c|}{\textbf{Normal ($800$Mbps, $0.022$ms)}} & 
 	\multicolumn{3}{c|}{
 	\textbf{N.B.~($200$Mbps, $0.022$ms)}} & 
 	\multicolumn{3}{c}{
 	\textbf{H.L.~($800$Mbps, $50$ms)}}
 	\\\cline{3-11}
 	 & & \textbf{Beaver} & \osmm & \textbf{Saving} &
 	\textbf{Beaver} & \osmm & \textbf{Saving} &
 	\textbf{Beaver} & \osmm & \textbf{Saving} \\
 	 	\hline
 	$1$ & $1000$ & $3.79$ & $2.92 $ & {$23.0\%$} & $8.89$ &$3.07$ & {$65.5\%$} & $7.55$ &$8.57$ & {$11.9\% $} \\
% $1$ & $2000$ & $16.12 $ & $3.79$ & {$76.5\%$} & $ 36.50$ & $4.06$ & {$88.9\%$} & $23.69$ & $9.24$ & {$61.0\%$}\\
 	$1$ & $3000$ & $60.67$ & $4.70$ & {$92.3\%$} & $106.39$ & $5.15$ & {$95.2\%$} & $74.54$ & $10.00$ & {$86.6\%$} \\
% $1$ & $4000$ & $ 105.92$ & $5.46$ & \cellcolor{grayL}{$94.8\% $} & $188.24$ & $5.96$ & \cellcolor{grayL}{$96.8\%$} & $130.44$ & $10.69$ & \cellcolor{grayL}{$91.8\%$}\\
 	$1$ & $5000$ & $165.35$ & $6.30$ & {$\mathbf{96.2\%}$} & $294.11$ & $7.01$ & {$\mathbf{97.6\%}$} & $203.77$ & $11.53$ & {$\mathbf{94.3\%}$} \\
 	\hline
 	$2$ & $1000$ & $5.95$ & $5.54$ & {$6.9\%$} & $10.92$ & $5.72$ & {$47.6\%$} & $9.17$ & $10.88$ & {$-$} \\
% $2$ & $2000$ & $17.84$ & $6.73$ & {$62.3\%$} & $38.05$ & $7.16$ & {$81.2\%$} & $25.33$ & $11.98$ & {$52.7\%$}\\
 	$2$ & $3000$ & $62.48$ & $7.99$ & {$87.2\%$} & $108.42$ & $8.57$ & {$92.1\%$} & $76.64$ & $13.24$ & {$82.7\%$} \\
% $2$ & $4000$ & $108.65$ & $9.22$ & \cellcolor{grayL}{$91.5\%$} & $190.44$ & $10.06$ & \cellcolor{grayL}{$94.7\%$} & $133.24$ & $14.46$ & {$89.1\%$} \\
 	$2$ & $5000$ & $168.06$ & $10.57$ & {$\mathbf{93.7\%}$} & $296.50$ & $11.71$ & {$\mathbf{96.1\%}$} & $206.38$ & $15.82$ & {$\mathbf{92.3\%}$} \\
 	\hline
 	$3$ & $1000$ & $8.27$ & $8.62$ & {$4.2\%$} & $13.61$ & $8.89$ & {$34.7\%$} & $11.83$ & $13.93$ & {$-$} \\
% $3$ & $2000$ & $20.95$ & $10.23$ & {$51.2\%$} & $40.11$ & $10.69$ & {$73.3\%$} & $27.73$ & $15.45$ & {$44.3\%$} \\
 	$3$ & $3000$ & $64.45$ & $11.92$ & {$81.5\%$} & $111.71$ & $12.74$ & {$88.6\%$} & $79.58$ & $17.19$ & {$78.4\%$} \\
% $3$ & $4000$ & $111.72$ & $13.94$ & {$87.5\%$} & $192.50$ & $15.28$ & \cellcolor{grayL}{$92.1\%$} & $136.09$ & $18.85$ & {$86.1\%$} \\
 	$3$ & $5000$ & $170.01$ & $15.28$ & {$\mathbf{91.0\%}$} & $299.57$ & $16.77$ & {$\mathbf{94.4\%}$} & $209.03$ & $20.64$ & {$\mathbf{90.1\%}$} \\
 	\hline
 	 	\end{tabular}
        
 	 $\#$E/N: ratio of edges per node,
	``Beaver'': using Beaver triples
%\\N.B.~denotes Narrow Bandwidth,
%H.L.~means High Latency.
\end{table*}

%!TEX root = gcn.tex

\section{Proofs related to Sparsity}
\label{sec:matrix_found_sparse}


\begin{definition}[$\Qsf$-type matrix]
\label{q_type}
%Let $e>n$.
A $(0,1)$-matrix $\Msf$ of size $m\times n$ is a $\Qsf$-type matrix iff there exists a monotonically non-decreasing %function 
$f:\Zbb/m\Zbb\rightarrow \Zbb/n\Zbb$ s.t.~$\Msf[i,j]=1$ iff $j=f(i)$.
\end{definition}

\begin{definition}[$\Psf$-type matrix] 
\label{p_type}
% Let $e>m$.
A $(0,1)$-matrix $\Msf$ of size $m\times n$ is a $\Psf$-type matrix iff there exists a monotonically non-decreasing %function 
$f:\Zbb/n\Zbb\rightarrow \Zbb/m\Zbb$ s.t.~$\Msf[i,j]=1$ iff $i=f(j)$.
\end{definition}

\subsection{Proof of Theorem~\ref{the::p_sig_q}}
\label{sec::proof_p_sig_q}
%\yuz{modifying all $t$-to-$e$ seems easier than all $e$-to-$t$? which one do you prefer? I can modify them}
%\ak{I prefer $t$ over $e$ as it is slightly easy to mix up with the base of the natural log.
%I can also help modify this later (after the main content)}
%\initdecom*
\begin{proof}
Suppose the sparse representation of $\adjmat$ is $\{(i_k, j_k, \lambda_k): k=1, \ldots, t\}$ where
%the map 
$k \mapsto i_k$ is monotonically non-decreasing.
Then, we have $\adjmat=\adjout'\Lambda \adjin$, where $\adjout'\in \Mbb_{m,t}(\Rcal)$ is a sparse matrix represented by $\{(i_k, k, 1): k=1, \ldots, t\}$, $\Lambda=\mathsf{diag}(\lambda _1, \ldots, \lambda _t)$, $\adjin\in \Mbb_{t,n}(\Rcal)$ is a 
sparse matrix represented by $\{(k, j_k, 1): k=1, \ldots, t\}$.
Now, $\adjout'$ is a $\Psf$-type matrix, but $\adjin$ is not a $\Qsf$-type matrix as $\adjin$ does not satisfies ``$k \mapsto j_k$ is monotonically non-decreasing''.
We permute the lines of $\adjin$ as $\adjin'=\{ (\sigma_3^{-1}(k), j_k, 1): k=1, \ldots, t\}=\{ (k, j_{\sigma_3(k)}, 1): k=1\ldots, t\}$ such that $k \mapsto j_{\sigma _3(k)}$ is monotonically non-decreasing.
After that, we have $\adjin= \{ (k, \sigma_3^{-1}(k), 1): k=1, \ldots, t \} \times \adjin' = \{ (\sigma_3(k), k, 1): k=1, \ldots, t \} \times \adjin' =\sigma_3 \adjin'$, where $\adjin'$ is a $\Qsf$-type matrix.
Hence, $\adjmat=\adjout' \Lambda \sigma _3 \adjin'$.
\end{proof}

\subsection{Proof of Theorem~\ref{the::general_dec_main}} 
\label{sec::proof_general}
\iffalse
\begin{theorem}[Restatement of Theorem~\ref{the::general_dec_main}] 
Let an $m\times n$ sparse matrix $\adjmat\in \Mbb_{m,n}(\Rcal)$ contains $\nrow$ non-zero rows, $\ncol$ non-zero columns, and $\numnonzero$ non-zero elements.
Then, there exists a matrix decomposition
$\adjmat= \sigma_5 \delta_m ^{\top} \gammaout \sigma _4 \Sigma ^{\top} \Lambda \sigma _3 \Sigma \sigma_2 \gammain \delta _n \sigma _1$,
where $\sigma _5 \in \Sbb_m$, $\sigma_4 \in \Sbb_{\numnonzero}, \sigma_3 \in \Sbb_{\numnonzero},\sigma _2 \in \Sbb_{\numnonzero},\sigma _1 \in \Sbb_n$, and,\\
1) $\Sigma=(\Sigma[i, j])_{i,j=1}^{\numnonzero}$ is the left-down triangle matrix such that $\Sigma[i, j]=1$ if $i \geq j$ or $0$ otherwise,\\
2) $\delta_k=(\delta_k[i,j])_{i,j=1}^{k}$ is the left-down triangle matrix such that $\delta_k[i,j]=1$ for $i=j$ or $-1$ for $j=i-1$, or $0$ otherwise,\\
3) $\gammain =(\gammain[i,j])_{i=1,j=1}^{\numnonzero,n}$ is a matrix such that $\gammain[i,j]=1$ for $1\leq i=j\leq \ncol$ or $0$ otherwise,\\
4) $\gammaout =(\gammaout[i,j])_{i=1,j=1}^{m,\numnonzero}$ is a matrix such that $\gammaout[i,j]=1$ for $1\leq i=j\leq {\nrow}$ or $0$ otherwise.
\end{theorem}
\fi
%\finaldecom*

\begin{proof}
The proof is straightforward by composing $\adjmat = \adjout' \Lambda \sigma_3 \adjin'$ from Theorem \ref{the::p_sig_q}, $\adjin'=\Sigma \sigma_2 \gammain \delta _n \sigma _1$ from Theorem~\ref{the::q_decom}, and $\adjout'=\sigma_5 \delta_m ^\top \gammaout \sigma_4 \Sigma ^\top$ from Theorem~\ref{the::p_decom}.
%composing group actions of $\Psf\Lambda\sigma_3\Qsf$ and equivalently to get $\sigma_5 \delta_m ^{\top} \gammaout \sigma _4 \Sigma ^{\top} \Lambda \sigma _3 \Sigma \sigma_2 \gammain \delta _n \sigma _1$.
\end{proof}

\subsection{Proof of Theorem~\ref{the::smm_main}}
\label{sec::proof_smm}
\iffalse
\begin{theorem}[Restatement of Theorem~\ref{the::smm_main}]
Let $\Msf$ be a sparse matrix and $\xvec$ be a dense vector/matrix.
Computing sparse matrix multiplication that $\Msf\xvec=\sigma_5 \delta _m ^\top \Gamma _{\nrow} \sigma _4 \Sigma ^\top \Lambda \sigma _3 \Sigma \sigma_2 J_{\ncol} \delta _n \sigma _1 \xvec$ requires an ordered sequential of permutation group action, element-wise multiplication, cut-off and padding with $0$, and constant matrix multiplication from right to left.
\end{theorem}
\fi
%\thmsmm*
\begin{proof}
It is straightforward to prove by observing that:\\
1) Permutations on $\sigma _i$ calls permutation group action;\\
2) Multiplying %$\delta_n$, $\delta_m^{\top}$, $\Sigma$, $\Sigma^{\top}$ 
$\delta$, $\Sigma$ calls constant matrix multiplication;\\
3) Multiplying $\gammain$, $\gammaout$ calls element-wise multiplication (with cut-off and padding of zero values);\\
4) Multiplying $\Lambda$ calls element-wise multiplication.	
\end{proof}

\subsection{Proof of Theorem~\ref{the::q_decom}}


\label{sec::proof_q}

\begin{theorem}
\label{the::q_decom}
Let $\adjin' \in \Mbb_{t, n}$ be a $\Qsf$-type matrix with $\ncol$ non-zero columns.
Then, there exists a matrix decomposition $\adjin'=\Sigma \sigma_2 \gammain \delta _n \sigma _1$ where $\sigma _1 \in \Sbb_n, \sigma _2 \in \Sbb_t$, and, \\
1) $\Sigma=(\Sigma[i, j])_{i,j=1}^{\numnonzero}$ is the left-down triangle matrix such that $\Sigma[i, j]=1$ if $i \geq j$ or $0$ otherwise,\\
2) $\delta_n=(\delta_n[i,j])_{i,j=1}^{n}$ is the left-down triangle matrix such that $\delta_n[i,j]=1$ for $i=j$ or $-1$ for $j=i-1$, or $0$ otherwise,\\
3) $\gammain =(\gammain[i,j])_{i=1,j=1}^{\numnonzero,n}$ is a matrix such that $\gammain[i,j]=1$ for $1\leq i=j\leq \ncol$ or $0$ otherwise.
\iffalse
1) $\Sigma=(a_{ij})_{i,j=1}^t$ is the left-down triangle matrix such that $a_{ij}=1$ if $i \geq j$ or $0$ otherwise;
2) $\delta_n=(a_{ij})_{i,j=1}^n$ is the left-down triangle matrix such that $b_{ij}=1$ for $i=j$ or $-1$ for $j=i-1$ or $0$ otherwise;
3) $\gammain=\begin{psmallmatrix}
I_k & O_{k,n-k} \\
O_{t-k,k} & O_{t-k,n-k}
\end{psmallmatrix}=(a_{ij})_{i=1,j=1}^{t,n}$ is a matrix such that $a_{ij}=1$ for $1\leq i=j\leq k$ or $0$ otherwise.
\fi
\end{theorem}
 
\begin{proof}
Here, we prove that $\adjin' \Xsf=\Sigma \sigma _2 \gammain \delta_n \sigma _1 \Xsf $ holds for any $\Xsf\in \Rcal^{(n)}$.
Firstly, we can use column transformation to transform matrix $\adjin'$ to a new matrix $\tilde{\adjin'}$ of $\Qsf$-type such that all the $n-\ncol$ zero-columns of $\tilde{\adjin'}$ lie in the last columns.
Hence, we have $\adjin'=\tilde{\adjin'}\sigma _1$, and $\tilde{\adjin'}$ is in the form of,
\[
\tilde{\adjin'}=\left(
\begin{array}{ccccccc}
1 &&& & 0 & \cdots & 0\\
\vdots &&& \\
1 &&& & \vdots & & \vdots \\
 & 1 && \\
 & \vdots && & \vdots & & \vdots \\
 & 1 && \\
 & & \ddots & & \vdots & & \vdots \\
 && & 1 \\
 && & \vdots \\
 &&& 1 & 0 & \cdots & 0\\
\end{array}
\right)
\]
Let $\tilde{\Xsf} = \sigma_1 \Xsf$,
then we have $\adjin' \Xsf$ equal to:
\begin{equation*}
	\begin{aligned}
		\tilde{\adjin'}\tilde{\Xsf}&=
\left(
\begin{array}{c}
\tilde{x}_1 \\
\tilde{x}_1 \\
\vdots \\
\tilde{x}_1 \\
\tilde{x}_2 \\
\tilde{x}_2 \\
\vdots \\
\tilde{x}_2 \\
\vdots \\
\vdots \\
\tilde{x}_{\ncol} \\
\tilde{x}_{\ncol} \\
\vdots \\
\tilde{x}_{\ncol} \\
\end{array}
\right) = \Sigma
\left(
\begin{array}{c}
\tilde{x}_1 \\
0 \\
\vdots \\
0 \\
\tilde{x}_2-\tilde{x}_1 \\
0 \\
\vdots \\
0 \\
\vdots \\
\vdots \\
\tilde{x}_{\ncol}-\tilde{x}_{\ncol-1} \\
0 \\
\vdots \\
0 \\
\end{array}
\right)
\end{aligned}
\end{equation*}
\begin{equation*}
\begin{aligned}
& =
\Sigma \sigma _2
\left(
\begin{array}{c}
\tilde{x}_1 \\
\tilde{x}_2-\tilde{x}_1 \\
\vdots \\
\tilde{x}_{\ncol}-\tilde{x}_{\ncol-1} \\
0 \\
\vdots \\
0 \\
\end{array}
\right)
 = \Sigma \sigma _2 \gammain
\left(
\begin{array}{c}
\tilde{x}_1 \\
\tilde{x}_2-\tilde{x}_1 \\
\vdots \\
\tilde{x}_{\ncol}-\tilde{x}_{\ncol-1} \\
\tilde{x}_{\ncol+1}-\tilde{x}_{\ncol} \\
\vdots \\
\tilde{x}_{n}-\tilde{x}_{n-1} \\
\end{array}
\right)
\\
&
= \Sigma \sigma _2 \gammain \delta_n \tilde{\Xsf}
= \Sigma \sigma _2 \gammain \delta_n \sigma _1 \Xsf
\rlap{\qquad \qquad \qquad \qquad \qquad \qquad \qedhere}.
\end{aligned}
\end{equation*}
\end{proof}

\subsection{Proof of Theorem~\ref{the::p_decom}}

\label{sec::proof_p}
\begin{theorem} 
\label{the::p_decom}
Let $\adjout' \in \Mbb_{m,t}(\Rcal)$ be a $\Psf$-type matrix with $\nrow$ non-zero rows.
Then, there exists a matrix decomposition $\adjout'=\sigma_5 \delta_m ^\top \gammaout \sigma_4 \Sigma ^\top$ where $\sigma _5 \in \Sbb_m$, $\sigma_4 \in \Sbb_t$, and, \\
1) $\Sigma=(\Sigma[i, j])_{i,j=1}^{\numnonzero}$ is the left-down triangle matrix such that $\Sigma[i, j]=1$ if $i \geq j$ or $0$ otherwise,\\
2) $\delta_m=(\delta_m[i,j])_{i,j=1}^{m}$ is the left-down triangle matrix such that $\delta_m[i,j]=1$ for $i=j$ or $-1$ for $j=i-1$, or $0$ otherwise,\\
3) $\gammaout =(\gammaout[i,j])_{i=1,j=1}^{m,\numnonzero}$ is a matrix such that $\gammaout[i,j]=1$ for $1\leq i=j\leq {\nrow}$ or $0$ otherwise.
\iffalse
1) $\Sigma=(a_{ij})_{i,j=1}^t$ is the left-down triangle matrix such that $a_{ij}=1$ if $i \geq j$ or $0$ otherwise;
2) $\delta_m=(a_{ij})_{i,j=1}^m$ is the left-down triangle matrix such that $b_{ij}=1$ for $i=j$ or $-1$ for $j=i-1$ or $0$ otherwise;
3) $\Gamma_{\nrow}=
\begin{psmallmatrix}
I_{\nrow} & O_{{\nrow},t-{\nrow}} \\
O_{m-{\nrow},{\nrow}} & O_{m-{\nrow},t-{\nrow}}
\end{psmallmatrix} 
=(a_{ij})_{i=1,j=1}^{m,t}$ is the matrix such that $a_{ij}=1$ for $1\leq i=j\leq {\nrow}$ or $0$ otherwise.
\fi
\end{theorem}

\begin{proof}
Note that $(\adjout')^\top \in \Mbb_{t,m}$ is a $\Qsf$-type matrix, we have the matrix decomposition $(\adjout')^\top=\Sigma \sigma_2 \gammain \delta_m \sigma _1$ by Theorem~\ref{the::q_decom}, where $\sigma_1 \in \Sbb_m$ and $\sigma_2\in \Sbb_t$.
Let $\sigma_5=\sigma_1^\top$, $\sigma_4=\sigma_2^\top$, $\gammaout=\gammain^\top$, we have $\adjout'=\sigma_5 \delta_m ^\top \gammaout \sigma_4 \Sigma ^\top$.
\end{proof}
 


%!TEX root = gcn.tex
 
\section{Algorithm Realization}
\label{sec::alg}

Guided by Theorem~\ref{the::p_sig_q}, our algorithm of decomposing the adjacency matrix $\Asf$ can be implemented as below.
$\Asf$ is a \texttt{SparseMatrix} consists of all $0$-$1$ elements represented by coordinates of nonzero values,
\ie, [(\texttt{row-index}, \texttt{column-index}), (\texttt{row-index}, \texttt{column-index}), $\ldots$].
In Figure~\ref{fig::code_psigq_de}, we draw the pairs of (row-index, column-index) of $\Asf$, \eg, $[(2, 3), (3, 1), \ldots]$.
Step~1 is to split the pairs in $\Asf$ to construct two sparse matrices (Graph language in Figure~\ref{fig::nen_relation_diff}), called $\Psf'$ and $\Qsf'$ (correspond to $\adjout$ and $\adjin$ in Section~\ref{subsec::sme}).
In Steps~2 and~3, the row-indices of $\Psf'$ are sorted and generate $\sigma_3^{\Psf}$, and the column-indices of $\Qsf'$ are sorted and generate $\sigma_3^{\Qsf}$.
Then, $\sigma_3$ is obtained by sparse-matrix multiplying $\sigma_3^{\Psf}$ and $\sigma_3^{\Qsf}$, thus $\sigma_3=\sigma_3^{\Psf}\cdot\sigma_3^{\Qsf}$.
%\yuz{done}
Now, we can see that the resulting matrices $\Psf$ and $\Qsf$ (correspond to $\adjout'$ and $\adjin'$ in Section~\ref{subsec::initdecom}) have monotonically non-decreasing row/column indices.
Moreover, $\Psf$ contain exactly one $1$ in every column and $\Qsf$ contains exactly one $1$ in every row.
% monotonically non-decreasing (1, 1, 1, 1, 2) is not strictly equal to monotonically increasing (1,2,3,4,5).
In summary, the pseudocode of \texttt{decompose\_row\_column(A)} below describes the extraction of $\Psf$, $\sigma_3$ and $\Qsf$ as previously displayed in Figure~\ref{graph_psigq}.

% \vspace{2mm}
% \noindent
% \begin{frame}{\texttt{def decompose\_row\_column(A):}}
%	\\\texttt{\#A: a list contains \{(i, j): a\_{\{i,j\}}=1\}}
% \\\indent\texttt{P = A[:, 0]}
% \\\indent\texttt{Q\_trans = A[:, 1]}
% \\\indent\texttt{sigma3 = tf.argsort(Q\_trans)}
% \\\indent\texttt{sigma3 = Permutations(sigma3)}
% \\\indent\texttt{inv\_sigma3 = $\sim$sigma3}
% \\\indent\texttt{Q\_trans = inv\_sigma3.act(Q\_trans)}
% \\\indent\texttt{return P, sigma3, Q\_trans}
% \end{frame}	

\begin{figure}[!t]
%	\vspace{-5mm}
	\centering
	\includegraphics[width = 0.47\textwidth]{./fig_and_tab/code_psq.png}
	\caption{Illustration of $\Psf\sigma_3\Qsf$ Decomposition}
	 \label{fig::code_psigq_de}
		%\end{minipage}
\end{figure}



\noindent
\begin{frame}{\texttt{def decompose\_row\_column(A):}}
 \\\texttt{\#A: a list of (row\_id, col\_id) for non-zero element in sparse matrix}
\\\indent\texttt{P=[(A[i, 0], i) for i in range(len(A))]}
\\\indent\texttt{Q=[(i, A[i, 1]) for i in range(len(A))]}
\\\indent\texttt{P=sorted(P, key=lambda x: x[0])}
\\\indent\texttt{Q=sorted(Q, key=lambda x: x[1])}
\\\indent\texttt{P=[(P[i,0], i) for i in range(len(A))]}
\\\indent\texttt{sigma3P=[(i, P[i,1]) for i in range(len(A))]}
\\\indent\texttt{sigma3Q=[(Q[i,0], i) for i in range(len(A))]}
\\\indent\texttt{Q=[(i,Q[i,1]) for i in range(len(A))] }
\\\indent\texttt{sigma3 = sigma3P*sigma3Q}
\\\indent\texttt{return P, sigma3, Q}
\end{frame}	


\noindent\textbf{Pseudocode of re-decomposition}.
%Compared with the difference of coordinates, we extract $\ncol$, $\sigma_1$, and $\sigma_2$ one by one from $\Qsf$.
In Figure~\ref{fig::code_q_compos}, we re-draw $\Qsf$ and describe its decomposition.
Step~1 extracts the unique column indices using $\text{set}$ function in Python, and their quantity is $\ncol$.
Then, the corresponding row indices are extracted by comparing whether the neighboring column indices are identical.
Step~2 constructs $\sigma_1$ and $\sigma_2$ by keeping the first $\ncol$ elements and padding the elements in numerical order to a permutation in $\sigma_1\in\Sbb_n,\sigma_2\in\Sbb_t$.
The code of decomposing $\Qsf$ is outlined below.
To derive $\sigma_4$, $\nrow$, and $\sigma_5$, the $\Psf$-type matrix decomposition follows a similar logic.

\begin{figure}[!t]
%	\vspace{-5mm}
	\centering
	\includegraphics[width = 0.42\textwidth]{./fig_and_tab/code_qx.png}
	\caption{Illustration of $\Qsf$ Decomposition}
	 \label{fig::code_q_compos}
		%\end{minipage}
\end{figure}

\begin{frame}{\texttt{def decompose\_Q(Q, e, n):}}
% \\\indent\texttt{assert len(Q)=e}
% \\\indent\texttt{assert Q[i:0]=i for i in range(e)}
 \\\indent\texttt{unique\_col\_ids = set(Q[:,1])} 
 \\\indent\texttt{step\_row\_ids=[Q[i,0] for Q[i, 1)!=Q[i-1, 1) or i=0]}
 \\\indent\texttt{k2 = len(unique\_col\_ids)}
 \\\indent\texttt{sigma1 = [ (i, unique\_col\_ids[i]	) for i in range(k2)]}
 \\\indent\texttt{sigma2 = [(step\_row\_ids[i], i) for i in range(k2)] }
\\\indent\texttt{sigma1=pad\_perm(sigma1, n)} 
\\\indent\texttt{sigma2=pad\_perm(sigma2, e)} 
\\\indent\texttt{return sigma2, k2, sigma1}
\end{frame}
 
 
The \texttt{class PrivateSparseMatrix} below contains the realization of \osmm protocols.
Before secure training, the graph owner locally decomposes $\Asf$ into $\Psf$, $\Qsf$, and then into the corresponding basic operations.
%before secure training with another party.
\osmm let $\pp_0$ and $\pp_1$ jointly execute secure multiplications on $\Psf$ and $\Qsf$, and $\ssp$ on $\sigma_3$.
%, which boils down to $\ssp$ and secure multiplication essentially.

\vspace{2mm}
\noindent
\begin{frame}{\texttt{class PrivateSparseMatrix:}}
\\\indent\texttt{def \_\_init\_\_(self, A ...):}
\\\indent\indent\texttt{self.owner = get\_device()}
\\\indent\indent\texttt{with tf.device(self.owner):}
\\\indent\indent\indent\texttt{P, s3, Q = decompose\_row\_column()}
\\\indent\indent\indent\texttt{s5, k4, s4 = decompose\_P()}
\\\indent\indent\indent\texttt{s2, k2, s1 = decompose\_Q()}
\\\indent\texttt{def sm\_2(self, x: Union[PrivateTensor,} \\\indent\indent\indent\indent\ \indent\indent\indent\indent\texttt{SharedPair]):}
\\\indent\indent\texttt{Qx = Q\_mult(s2, k2, s1, e, x)}
\\\indent\indent\texttt{x3 = s3.act(Qx)}
\\\indent\indent\texttt{Ax = P\_mult(s5, k4, s4, x3, m)}
\\\indent\indent\texttt{return Ax}

\end{frame}

{Class of Secure GCN.}
The implementation of \cgnn follows the plaintext-training repository (\url{https://github.com/dmlc/dgl}) in the transductive setting.
Accordingly, the graph decomposition can be performed once with the fixed graph before secure training.
The \texttt{class SGCN} inherits the conventional \texttt{NN} (the template of neural network).
Secure GCN training is thus composed of secure graph convolution and secure activation layers.
%, in particular.
The $\Asf\Xsf$ in graph convolution layers is realized by the $\Pi_\smm$ protocol.
Below, we extract the code of implementing the \texttt{class SGCN} with respect to the plaintext GCN.
%model~\cite{iclr/KipfW17}.
Notably, all the inputs \texttt{feature}, \texttt{label}, and \texttt{adj\_matrix} are secret shares in the form of fixed-pointed numbers.
The functions \texttt{GraphConv, ReLU, SoftmaxCE} are the MPC protocols executed by two parties.


\begin{frame}{\texttt{class SGCN(NN):}}
\\\texttt{def \_\_init\_\_(self, feature: PrivateTensor, 
\\\indent\indent\indent\indent\indent\indent label: PrivateTensor, 
\\\indent\indent\indent\indent\indent\indent dense\_dims: List[int],
\\\indent\indent\indent\indent\indent\indent adj\_matrix: Union[...],
\\\indent\indent\indent\indent\indent\indent train\_mask, loss=...):}
\\\indent\texttt{super(SGCN, self).\_\_init\_\_()}
\\\indent\texttt{layer = Input(dim, feature)}
\\\indent\texttt{self.addLayer(layer)}
\\\indent\texttt{input\_layers = [layer]}
\\\indent\texttt{for i in range(1, len(dense\_dims)):}
\\\indent\indent\texttt{layer = GraphConv()}
\\\indent\indent\texttt{self.addLayer(ly=layer)}
\\\indent\indent\texttt{if i < len(dense\_dims) - 1:}
\\\indent\indent\indent\texttt{layer = ReLU()} 
\\\indent\indent\indent\texttt{self.addLayer(ly=layer)}
\\\indent\texttt{layer\_label = Input(dim, label)}
\\\indent\texttt{self.addLayer(layer\_label)}
\\\indent\texttt{if loss == "SoftmaxCE":} 
\\\indent\indent\texttt{layer\_loss = SoftmaxCE()}
\\\indent\indent\texttt{self.addLayer(ly=layer\_loss)}
\\\indent\texttt{else:}
\\\indent\indent\texttt{...\# use other layer/loss}
\end{frame}


%!TEX root = gcn.tex
\section{Selection-Multiplication's Correctness}
\label{sec:proof_lemma}
\begin{proof}[Proof of Lemma~\ref{lem::sxb}]
We analyze the two cases of $s=0$ and $s=1$ for a complete proof where $s \in \{0, 1\}$.

%\begin{itemize}
%	\setlength{\itemsep}{0pt}
%	\setlength{\parskip}{0pt}
%	\setlength{\parsep}{0pt}
%\item[(i)] 
(i). When $s=0$, the equivalence of the first property turns to be $f(0,x+u)=f(0,x)+f(0,u) \Leftrightarrow 0\cdot(x+u) = 0\cdot x+ 0\cdot u \Leftrightarrow 0=0$.
When $s=1$, we have $f(1,x+u)=f(1,x)+f(1,u) \Leftrightarrow 1\cdot(x+u) = 1\cdot x+ 1\cdot u \Leftrightarrow x+u=x+u$.
%\item[(ii)] 

(ii). When $s=0$, the %equivalence of the 
second property %turns to be 
becomes $f(0+b,x)=f(0,x)+(-1)^0f(b,x) \Leftrightarrow bx = 0\cdot x+ 1\cdot bx \Leftrightarrow bx = bx$.
When $s=1$, we have $f(1+b,x)=f(1,x)+(-1)^1f(b,x) \Leftrightarrow (1+b)\cdot x = 1\cdot x - bx$.
If $b=0$, we get $x=x$.
If $b=1$, we get $0=0$ since $(1+b)\!\!\!\!\mod 2=0$.
%\end{itemize}
\end{proof}
%!TEX root = gcn.tex
\section{Security Analysis}
\label{app::fullproof}
We prove the semi-honest security of our protocols under the real/ideal-world simulation paradigm~\cite{sp/17/Lindell17} with a hybrid argument.
%techniques.
%For clarification, we divided the protocol dependence into three branches, including $\Pi_\ssp$-related Protocols, $\Pi_\SM$-related Protocols, and Adam-related Protocols.
As our protocols satisfy the stand-alone model without malicious assumption, we adopt the standard simulation proof technique instead of the UC framework that adds an additional ``environment" representing an interactive distinguisher.

%\subsection{Security Definition}
We consider the $2$PC executed by $\pp_0$ and $\pp_1$ in the presence of static semi-honest adversaries $\A$ that control one of the parties at the beginning, follow the protocol specification, and try to learn information about the honest party's private input.
Definition~\ref{def:semidef} states the semi-honest security so that simulated and real execution are computationally indistinguishable ("$\equiv$") for~$\A$.
That is, the simulator $\Scal$ can generate the view of a party in the execution, implying the party learns nothing beyond what they can derive from their input and prescribed output.
For simplicity, we assume $\mathsf{PRF}$ to be secure and exclude its standard proof here.

\begin{restatable}[Semi-honest Security~\cite{sp/17/Lindell17}]{definition}{semidef}
\label{def:semidef} 
	Let $\lambda$ be a security parameter.
	A protocol $\Pi$ securely realizes a functionality $\F = (\F_0, \F_1)$ on input $\l x \r = (\l x\r_0, \l x\r_1)$ against %in the presence of 
	static semi-honest adversaries if there exist PPT simulators $\Scal_0, \Scal_1$ s.t.
 \begin{equation*}
     \begin{aligned}
         \{\Scal_0(1^\lambda, \l x\r_0, \F_0), \F(\l x \r)\} \equiv \{\view_0^\Pi, \output^{\Pi, \lambda}(\l x \r)\},\\
         \{\Scal_1(1^\lambda,\l x\r_1,\F_1), \F(\l x \r)\} \equiv \{\view_1^\Pi,\output^{\Pi,\lambda}(\l x \r)\}.
     \end{aligned}
 \end{equation*}
\end{restatable}
%\semidef*

\subsection{Security of \texorpdfstring{$\Pi_\ssp$}{πOP}}
\label{sec::detail_op}

We divide the analyses into `raw' and `shared' cases.
%to differentiate none of the share or zero share owned by one party.
Functionality~\ref{func::ssp} presents the ideal functionality $\F_\ssp$.
%where either party owns the permutation.
It contains two cases in which the input vector/matrix $\Xsf$ is owned by one party or secret-shared among two parties.
The functionality $\F_\ssp$ of both cases outputs additive shares of permutation over~$\Xsf$, \ie, $\l \sigma\xvec\r_0 +\l \sigma\xvec\r_1 =\sigma\xvec$.

\setcounter{algorithm}{0}

\begin{functionality}[!t]
	\caption{$\F_\ssp$: Ideal Functionality of $\Pi_{\ssp}$}\label{func::ssp} 
\begin{algorithmic}[1]
	\item[\textbf{Parameter:} Type of input $\type \in \{\raw, \shared\}$.]
	\REQUIRE $\sigma\in\Sbb,\xvec\in \Zbb^m$ if $\type == \raw$; \\ ~~~~~otherwise $\sigma\in \Sbb,\l\xvec\r_0\in\Zbb^m, \l\xvec\r_1\in\Zbb^m$.
	\ENSURE $\l \sigma\xvec\r_0,\l \sigma\xvec\r_1$.
	\IF{$\type == \shared$} 
	\STATE Reconstruct $\Xsf = \l \xvec \r_0+\l \xvec \r_1$
	\ENDIF
	\STATE Compute and generate random shares of $\sigma\xvec$
	\RETURN $\l \sigma\xvec \r$
\end{algorithmic}
\end{functionality}

\begin{theorem}
\label{the::ssp}
	The protocol $\Pi_\ssp$ securely realizes the ideal functionality $\F_\ssp$ against static semi-honest adversaries.
\end{theorem}
 
\begin{proof}
	%To attain the semi-honest security in Theorem~\ref{the::ssp}, we are essentially to prove that there exists simulator $\Scal$ satisfying Definition~\ref{def:semidef}.
	%Concisely, regarding Definition~\ref{def:semidef}, w
	We define the following $\ideal$ and $\real$ experiments:
	\begin{equation*}
			\begin{aligned}
			\real^{1^\lambda,\A}_{\Pi_\ssp}=&\ \{\{(\view_0^{\Pi,\raw},\l\sigma\Xsf\r_0), (\view_1^{\Pi,\raw},\l\sigma\Xsf\r_1)\} \text{ or }\\
			&\ \ \ \{(\view_0^{\Pi,\shared},\l\sigma\Xsf\r_0), (\view_1^{\Pi,\shared},\l\sigma\Xsf\r_1)\}
			\}\\
			\ideal^{1^\lambda, \A}_{\Scal, \F}=&\ \{\{\Scal(\raw,1^\lambda,\sigma,\Xsf,\F_\ssp),\F_\ssp(\sigma,\Xsf)\}\text{ or }\\
			&\ \ \ \{\Scal(\shared,1^\lambda,\sigma,\l\Xsf\r_0,\l\Xsf\r_1,\F_\ssp),
			\\&\quad\F_\ssp(\sigma,\l\Xsf\r_0,\l\Xsf\r_1)\}\}
		\end{aligned}
	\end{equation*}
	where $\pp_0$'s view is either $\view_0^{\Pi,\raw}$, which is
	$(\sigma, \pi, \l \pi\uvec\r_0, \delta_{\sigma}, \delta_{\Xsf})$ or $\view_0^{\Pi,\shared}$,
	which is $(\sigma, \l \Xsf\r_0, \pi, \l \pi\uvec \r_0, \delta_{\sigma}, \delta_{\l\Xsf\r_1})$,
	and $\pp_1$'s view is either $\view_1^{\Pi,\raw}=(\Xsf, \l \pi\uvec \r_1,\delta_{\sigma},\delta_{\Xsf})$ or $\view_1^{\Pi,\shared}=(\l\Xsf\r_1, \l \pi\uvec \r_1,\delta_{\sigma},\allowbreak\delta_{\l\Xsf\r_1})$.
	$\real^{1^\lambda,\A}_{\Pi_\ssp}$ represents real protocol execution.
	In the $\ideal$ world, the simulators $\Scal=\{\Scal_0,\Scal_1\}$ can indistinguishably simulate the view of each honest party in the protocol given only that party's input.
 
	Now, we argue that $\real^{1^\lambda,\A}_{\Pi_\ssp}\equiv\ideal^{1^\lambda, \A}_{\Scal, \F}$ for any PPT $\A$ using the multi-step hybrid-argument technique.
	 \begin{itemize}
		 \item[$\hyb_0$:] It is identical to the real protocol execution $\real^{1^\lambda,\A}_{\Pi_\ssp}$.
		 \item[$\hyb_{1}$:] It is identical to $\hyb_0$ except that $\delta_\sigma,\delta_{\Xsf}$ are randomly generated for the case of $\raw$ and $\delta_\sigma,\delta_{\l\Xsf\r_1}$ are randomly generated for the case of $\shared$.
		 \\
		 i) In the first case that $\Xsf$'s type is $\raw$, any PPT $\A$ cannot distinguish $\delta_\sigma,\delta_{\Xsf}$ in $\real^{1^\lambda,\A}_{\Pi_\ssp}$ experiment and $\Tilde{\delta}_\sigma,\Tilde{\delta}_{\Xsf}$ in $\hyb_1$ since $\delta_\sigma,\delta_{\Xsf}$ are computed by $\pi,\uvec$, which are generated by $\prf$.
		 If $\A$ can distinguish $\hyb_{1}$ and $\real^{1^\lambda,\A}_{\Pi_\ssp}$ with non-negligible advantage, $\A$ can break the security of $\prf$, which contradicts the assumption.
		 \\
		 ii) For the second case that $\Xsf$'s type is $\shared$, any PPT $\A$ cannot distinguish $\delta_\sigma,\delta_{\l\Xsf\r_1}$ in $\real^{1^\lambda,\A}_{\Pi_\ssp}$ experiment and $\Tilde{\delta}_\sigma,\Tilde{\delta}_{\l\Xsf\r_1}$ in $\hyb_1$ since $\delta_\sigma,\delta_{\l\Xsf\r_1}$ are computed by $\pi,\uvec$, which are generated by $\prf$.
		 If $\A$ can distinguish $\hyb_{1}$ and $\real^{1^\lambda,\A}_{\Pi_\ssp}$ with non-negligible advantage, $\A$ can break the $\prf$ security, contradicting the assumption.
		 \\
		 $\Rightarrow \hyb_1\equiv\hyb_0$.
		 \item[$\hyb_{2}$:] It is identical to $\ideal^{1^\lambda, \A}_{\Scal, \F}$, \ie, all $\view_0,\view_1$ of two parties are simulated by $\Scal_0,\Scal_1$.
		 The randomness of $\pi,\l \pi\uvec \r_0, \delta_{\sigma}, \delta_{\Xsf}$ for $\raw$ and $\pi,\l \pi\uvec \r_0, \delta_{\sigma}, \delta_{\l\Xsf\r_1}$ for $\shared$ ensures no non-negligible $\A$'s advantage of distinguishability to $\Scal_0$'s view.
		 Similarly, the randomness of $\l \pi\uvec \r_1, \delta_{\sigma}, \delta_{\Xsf}$ for $\raw$ and $\l \pi\uvec \r_1, \allowbreak\delta_{\sigma}, \delta_{\l\Xsf\r_1}$ for $\shared$ ensures no non-negligible $\A$'s advantage of distinguishability to $\Scal_1$'s view.
		 Now, $\pp_0$ cannot obtain $\pp_1$ inputs, while $\pp_1$ cannot obtain $\pp_0$ inputs since $\{\Scal_i\}_{i\in\{0,1\}}$ cannot obtain $\{\Scal_{1-i}\}_{i\in\{0,1\}}$'s inputs using the $\{\view_i\}_{i\in\{0,1\}}$.\\
		 $\Rightarrow \hyb_2\equiv\hyb_1$.
	 \end{itemize}
Thus, for both cases, %of $\raw$ and $\shared$, 
we have $\hyb_2\equiv\hyb_1\equiv\hyb_0$, equivalent to $\real^{1^\lambda,\A}_{\Pi_\ssp}\equiv\ideal^{1^\lambda, \A}_{\Scal, \F}$ for any PPT semi-honest $\A$.
\end{proof}


\subsection{Security of 
\texorpdfstring{$\Pi_\SM$}{πOSM}
} 
\label{sec::proof_sxb}

%\sxb*

Functionality~\ref{func::osm} gives the ideal functionality of $\Pi_{\SM}$, defining the multiplication $sx\in\Zbb_{2^n}$ between %binary element 
$s\in\Zbb_2$ and %arithmetic element 
$x\in\Zbb_{2^n}$.
%Protocol~\ref{fig:osm-main} ($\Pi_\SM$) realizes $\F_\ssp$ in semi-honest security model, as given in Theorem~\ref{the::osm}.
%Now, we define ideal/real experiments using simulation technique to prove Theorem~\ref{the::osm} against any honest-but-curious $\A$.

\begin{functionality}[!t]
	\caption{$\F_{\SM}$: Ideal Functionality of $\Pi_{\SM}$}\label{func::osm}
\begin{algorithmic}[1]
	\REQUIRE $s\in\Zbb_2, \l x\r_0\in\Zbb_{2^n}, \l x\r_1\in\Zbb_{2^n}$.
	\ENSURE $\l sx \r_0, \l sx \r_1$.
	\STATE Reconstruct $s=\l s\r_0+\l s\r_1$
	\STATE Compute and generate random shares of $sx$
	\RETURN $\l sx \r$
\end{algorithmic}
\end{functionality}

\begin{theorem}
\label{the::osm}
	The protocol $\Pi_\SM$ securely realizes the ideal functionality $\F_\SM$ against static semi-honest adversaries.
\end{theorem}

\begin{proof}
%In terms of Definition~\ref{def:semidef}, w
We define the $\ideal$ and $\real$ experiments:
	\begin{equation*}
		\begin{aligned}
			\real^{1^\lambda,\A}_{\Pi_\SM}=&\ \{(\view_0^{\Pi},\l sx\r_0), (\view_1^{\Pi},\l sx\r_1)\} \\
			\ideal^{1^\lambda, \A}_{\Scal, \F}=&\ \{\Scal(1^\lambda,s,\l x\r_0,\l x\r_1,\F_\SM),\\&\ \ \ \ \F_\SM(s,\l x\r_0,\l x\r_1)\}
		\end{aligned}
	\end{equation*}
	where 
	%$\pp_0$'s view is 
	$\view_0^{\Pi}=(s, \l x\r_0, b, \l u \r_0, \l bu \r_0,\delta_s, \allowbreak\delta_{\l x\r_0},\delta_{\l x\r_1},\delta_x)$, and 
	%$\pp_1$'s view is 
	$\view_1^{\Pi}=(\l x\r_1, \l bu \r_1, \l u \r_1, \delta_{\l x\r_1}, \delta_s)$.
	$\real^{1^\lambda,\A}_{\Pi_\SM}$ represents real protocol execution.
	The simulators $\Scal=\{\Scal_0,\Scal_1\}$ are indistinguishably simulating the view of each honest party in the protocol given only that party's input.
 
	Now, we prove that $\real^{1^\lambda,\A}_{\Pi_\SM}\equiv\ideal^{1^\lambda, \A}_{\Scal, \F}$ for any PPT $\A$ with a series of hybrid-arguments.
	The hybrid games can be sequentially formulated as follows.
\begin{itemize}
		 \item[$\hyb_0$:] It is identical to the real protocol execution $\real^{1^\lambda,\A}_{\Pi_\SM}$.
		 \item[$\hyb_{1}$:] It is identical to $\hyb_0$ except that $\delta_{\l x\r_0},\delta_{\l x\r_1},\delta_{x}$ are randomly generated 
		 %for the case of $\raw$ 
		 by $\Scal_0$.
		 Since $\delta_{\l x\r_0},\delta_{\l x\r_1}$ in $\real^{1^\lambda,\A}_{\Pi_\SM}$ experiment are computed by $\l u \r_0,\l u \r_1$, which are outputted by $\prf$.
		 Thus, any PPT $\A$ cannot distinguish $\delta_{\l x\r_0},\delta_{\l x\r_1}$ in $\real^{1^\lambda,\A}_{\Pi_\SM}$ experiment and $\Tilde{\delta}_{\l x\r_0},\Tilde{\delta}_{\l x\r_1}$ in $\hyb_1$, guaranteed by $\prf$'s security.
		 The value of $\delta_x$, added by $\delta_{\l x\r_0}$ and $\delta_{\l x\r_1}$, is also distinguishable to $\Tilde{\delta}_x$ simulated by $\Scal_0$.
		 Overall, if $\A$ can distinguish $\Tilde{\delta}_{\l x\r_0},\Tilde{\delta}_{\l x\r_1},\Tilde{\delta}_x$ with $\delta_{\l x\r_0},\delta_{\l x\r_1},\delta_x$ in $\real^{1^\lambda,\A}_{\Pi_\SM}$ with non-negligible advantage, $\A$ can break the security of $\prf$.
		 %, which yet contradicts the fact.
		 \\
		 $\Rightarrow \hyb_1\equiv\hyb_0$.
		 \item[$\hyb_{2}$:] It is identical to $\hyb_1$ except that $\delta_{s}$ are randomly generated.
		 Since $\delta_s$ in $\hyb_1$ experiment are computed by $h$, which are generated by $\prf$.
		 Thus, any PPT $\A$ cannot distinguish $\delta_s$ and $\Tilde{\delta}_{s}$, given the security of $\prf$.
		If $\A$ has the non-negligible advantage to guess the real $s$, then the $\A$ can distinguish $\Tilde{\delta}_{s}$ and $\delta_s$ with non-negligible probability, which breaks $\prf$.
		\\
		 $\Rightarrow \hyb_2\equiv\hyb_1$.
		 \item[$\hyb_{3}$:] It is identical to $\ideal^{1^\lambda, \A}_{\Scal, \F}$, \ie, all the $\view_0,\view_1$ 
		 %in $\raw$ case 
		 are simulated by $\Scal_0,\Scal_1$.
		 The randomness of $b, \l u \r_0, \l bu \r_0,\delta_s, \allowbreak\delta_{\l x\r_0},\delta_{\l x\r_1},\allowbreak\delta_x$ 
		 %for $\raw$ 
		 guarantees no non-negligible $\A$'s advantage of distinguishability to $\Scal_0$'s view.
		 Similarly, the randomness of $\l bu \r_1, \l u \r_1, \allowbreak\delta_{\l x\r_1}, \delta_s$ guarantees no non-negligible $\A$'s advantage of distinguishability to $\Scal_1$'s view.
		 Now, $\pp_0$ cannot obtain $\l x\r_1 $, while $\pp_1$ cannot obtain $s,\l x\r_0$ since $\{\Scal_i\}_{i\in\{0,1\}}$ cannot obtain $\{\Scal_{1-i}\}_{i\in\{0,1\}}$'s inputs using $\{\view_i\}_{i\in\{0,1\}}$.\\
		 $\Rightarrow \hyb_3\equiv\hyb_2$.
	 \end{itemize}
So,  $\real^{1^\lambda,\A}_{\Pi_\SM}\equiv\ideal^{1^\lambda, \A}_{\Scal, \F}$ for any PPT semi-honest $\A$.	
\end{proof}

\subsection{Security of 
\texorpdfstring{$\prosmm$}{π(SM)2}
}
Correctness has been checked using theoretical foundation for sparse-matrix in Appendix~\ref{sec:matrix_found_sparse}.
Functionality~\ref{func::mmult} defines arbitrary-sparse matrix multiplication without %really performing 
decomposition.


\begin{functionality}[!t]
	\caption{$\F_{\text{\osmm}}$: Ideal Functionality of $\prosmm$}\label{func::mmult}
\begin{algorithmic}[1]
	\REQUIRE $\adjmat\in \Mbb_{m,n}(\Rcal), \feamat\in \Mbb_{n,d}(\Rcal)$.
	\ENSURE $\l \adjmat\feamat \r_0, \l \adjmat\feamat \r_1$.
	\STATE Compute and generate random shares of $\adjmat\feamat$
	\RETURN $\l \adjmat\feamat \r$
\end{algorithmic}
\end{functionality}


\begin{theorem}[Security of $\prosmm$]
\label{secthe::smm}
Let $\Asf$ be a sparse %adjacency 
matrix and $\Xsf$ be any vector/matrix.
The protocol $\prosmm$ realizes the functionality $\F_{\text{\osmm}}$ %(Functionality~\ref{func::mmult}) 
against static semi-honest adversaries.
\end{theorem}


\begin{proof}
The $\prosmm$ protocol sequentially call the independent subroutines of $5\ {\Pi}_{\ssp}, 2\ \Pi_\SM$, and $1\ \promult$ protocols that have been proved to be semi-honest secure.
The sequential composition theorem~\cite{joc/Canetti00} guarantees that security is closed under composition.
So, $\prosmm$ is semi-honest secure.
% If a protocol is secure in the stand-alone model, it remains semi-honest secure under the sequential composition.
%Thus, we finish the proof of Theorem~\ref{secthe::smm}.
\end{proof}
 
\subsection{Security of 
\texorpdfstring{$\cgnn$}{\textcgnn}
}

\begin{theorem}
\label{secthe::sec_swan}
\cgnn securely realizes the functionality of GCN (Figure~\ref{graph_imple}) against static semi-honest adversaries.
\end{theorem}

\begin{proof}
    $\cgnn$ integrates the semi-honest protocols for all elementary operations like graph convolution and activation layers.
To obtain the secure inference or training protocol, we can sequentially compose the relevant protocols.
Correctness and %ideal functionality 
security of private inference or training follow the integration of underlying sub-protocols.
By the sequential composition theorem~\cite{joc/Canetti00}, 
%The sequential composition theorem~\cite{joc/Canetti00} guarantees the semi-honest security of secure inference or training.
%Thus, 
$\cgnn$ is semi-honest secure.
\end{proof}


\end{document}
