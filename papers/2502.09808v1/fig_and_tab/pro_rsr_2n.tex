%!TEX root = PPGraphNN.tex

\begin{protocol}[!t] 
	\caption{\texorpdfstring{$\prorsrn$: Reciprocal of Square Root of $2^k$}}
	\label{pro::rsr_2n}
	\begin{center}
		\setlength\tabcolsep{5pt}
		\begin{tabular}{c|c|c|c}
			{$\pp_{0}$ Input} & \multicolumn{3}{c}{$\langle x^{(0)} \rangle_0, \langle x^{(1)} \rangle_0, \langle x^{(2)} \rangle_0, \ldots, \langle x^{(n-1)} \rangle_0$} \\
           {$\pp_{1}$ Input} & \multicolumn{3}{c}{$\langle x^{(0)} \rangle_1, \langle x^{(1)} \rangle_1, \langle x^{(2)} \rangle_1, \ldots, \langle x^{(n-1)} \rangle_1$} \\\hline
            $\pp_0$ Output & $\langle 1/\sqrt{x} \rangle_0 $& $\pp_1$ Output & $\langle  1/\sqrt{x} \rangle_1 $  \\\hline
		\end{tabular}
	\end{center}
	\begin{algorithmic}[1]
 	\STATE $\pp_0$: Reverse $\l  \Vec{x}\r_0$  to get $\l  \Vec{y}_{[0,n-1]}\r_0=\l  \Vec{x}_{[n-1,0]}\r_0$
  \STATE $\pp_1$: Reverse $\l  \Vec{x}\r_1$  to get $\l  \Vec{y}_{[0,n-1]}\r_1=\l  \Vec{x}_{[n-1,0]}\r_1$
	\STATE $\pp_0$: $\langle z[j] \rangle_0 = \sum_i \langle y[4i+j] \rangle_0 \cdot 2^{i-f/4}$ for $j=0,1,2,3$
    \STATE $\pp_1$: $\langle z[j] \rangle_1 = \sum_i \langle y[4i+j] \rangle_1 \cdot 2^{i-f/4}$ for $j=0,1,2,3$
    \STATE $\pp_0$:$\langle w\rangle_0\hspace{-1mm}=\hspace{-1mm}\langle z[0] \rangle_0 + \sqrt[4]{2} \langle z[1] \rangle_0+ \sqrt{2}\langle z[2] \rangle_0+  \sqrt[4]{8}\langle z[3] \rangle_0$
    \STATE $\pp_1$:$\langle w\rangle_1\hspace{-1mm}=\hspace{-1mm}-\langle z[0] \rangle_1-\hspace{-1mm}\sqrt[4]{2}\langle z[1] \rangle_1-\hspace{-1mm} \sqrt{2}\langle z[2] \rangle_1-  \sqrt[4]{8}\langle z[3] \rangle_1$
    \STATE $\pp_0,\pp_1$: Execute $\langle 1/\sqrt{x} \rangle_0, \langle 1/\sqrt{x} \rangle_1=\prosqr(\langle w\rangle_0; \langle w\rangle_1)$ 
 %   \STATE $\pp_0$: $\langle \sqrt{x} \rangle_0=\langle w\rangle_0^2 + 2\langle q \rangle_0$ 
    %    \STATE $\pp_1$: $\langle \sqrt{x} \rangle_1=\langle w\rangle_1^2 + 2\langle q  \rangle_1$   
	\end{algorithmic}
\end{protocol}

\subsection{Protocols for \texorpdfstring{$1/\sqrt{x}$}{1/sqrt{x}} Variants and Adam}
\label{sec::sec_adam_app}

\noindent\textbf{Reciprocal of square root.}
Protocol~\ref{pro::esr_x} describes the process of solving $1/\sqrt{x}=\frac{1}{\sqrt{x'}}\cdot\frac{1}{\sqrt{2^k}}$ for $x'\in[1,2)$.
Combining Line~\ref{rsrx::l1} (\ie, find the power of $2$ detailed in Appendix~\ref{sec::sec_adam_app}) and Line~\ref{rsrx::l3},  $\pp_0$ and $\pp_1$ get the shares of $v={1}/{\sqrt{2^k}}$.
%In Line~\ref{rsrx::l6} and Line~\ref{rsrx::l7}, $\pp_0$ and $\pp_1$ get the shares of $z=\frac{1}{\sqrt{x'}}$.
Combining Line~\ref{rsrx::l1}, Line~\ref{rsrx::l2}, and Line~\ref{rsrx::l4}, $\pp_0$ and $\pp_1$ get the shares of $x'=\frac{x}{2^k}$ and enforce $x'\in [1,2)$.
From Line~\ref{rsrx::l5} to Line~\ref{rsrx::l7}, $\pp_0$ and $\pp_1$ get the shares of $1/\sqrt{x'}$ via quadratic polynomial approximation.
That is, $z= (4.63887*(m/4)^2 - 5.77789*(m/4) + 3.14736)/2$.
We use $a', b', c'$ to denote the scalar for simplification, which requires the truncation~\cite{sp/MohasselZ17} after multiplying the shares.
Here, both quadratic polynomial and Newton's iteration method can approximate $1/\sqrt{m}$ for $m\in[1,2)$.
We pick quadratic polynomial due to its high efficiency and accuracy, whereas Newton's iteration method requires $4$ rounds of iteration for $64$-bit fixed-point numbers with the $16$-bit fractional part.
The last step in Line~\ref{rsrx::l8} is multiplying $\frac{1}{\sqrt{x'}}$ and $\frac{1}{\sqrt{2^k}}$ to get the shares of $1/\sqrt{x}$.

%!TEX root = PPGraphNN.tex
\begin{protocol}[!t]
\caption{\texorpdfstring{$\prorsr$}{$\Pi_{RSR}$}: Reciprocal of Square Root}
	\label{pro::esr_x}
	\begin{center}
		\setlength\tabcolsep{5pt}
		\begin{tabular}{c|c|c|c}
		{$\pp_{0}$ Input} & {$ \langle x \rangle _0  $} & {$\pp_{1}$ Input} & {$\langle x \rangle _1 $} \\\hline
            $\pp_0$ Output& $\langle \frac{1}{\sqrt{x}} \rangle_0 $& $\pp_1$ Output& $\langle  \frac{1}{\sqrt{x}} \rangle_1 $  \\\hline
		\end{tabular}
	\end{center}
	\begin{algorithmic}[1]
 	\STATE $\pp_0,\pp_1$: Execute $\l y\r_0; \l y\r_1 =\pronptwon (\l x\r_0 ; \l x\r_1)$\label{rsrx::l1}
  \STATE $\pp_0,\pp_1$: Execute $\l u\r_0, \l u\r_1=\prortwon(\l y\r_0; \l y\r_1)$\label{rsrx::l2}
  \STATE $\pp_0,\pp_1$: Execute $\l v\r_0, \l v\r_1=\prorsrn(\l y\r_0; \l y\r_1)$\label{rsrx::l3}
  \STATE $\pp_0,\pp_1$: $\l m\r_0,\l m\r_1 =\promult( \l x\r_0,\l u\r_0;\l x\r_1,\l u\r_1)$\label{rsrx::l4}
 \STATE $\pp_0,\pp_1$: $\l h\r_0,\l h\r_1 =\prosqr( \l m\r_0;\l m\r_1)$\label{rsrx::l5}
  \STATE $\pp_0$: $\l z\r_0=a' \l h\r_0 +b' \l m\r_0 +c'$\label{rsrx::l6}
  \STATE $\pp_1$: $\l z\r_0=a' \l h\r_1 +b' \l m\r_1$\label{rsrx::l7}
  \STATE   $\pp_0,\pp_1$: $\l \frac{1}{\sqrt{x}}\r_0,\l \frac{1}{\sqrt{x}}\r_1 =\promult( \l z\r_0,\l v\r_0;\l z\r_1,\l v\r_1)$\label{rsrx::l8}
	\end{algorithmic}
\end{protocol}




%\smallskip
\noindent\textbf{Reciprocal of $\sqrt{x+\epsilon^2}$.}
Protocol~\ref{pro::reci_sqrt_eps} solves secure ${1}/{\sqrt{\bar{v}_t+\epsilon^2}}$ efficiently in the crypto domain.
Compared with $\prorsr$, there is an if-else condition for whether the denominator $\bar{v}_t+ \epsilon ^2$ equals $0$.
If $s=0$, which means that the highest non-zero bit of $\bar{v}_t+\epsilon^2$ is zero, then $\Pi_\SM$ returns $1/\epsilon$; otherwise returns $1/{\sqrt{\bar{v}_t+\epsilon^2}}$.


 \begin{protocol}[!t] 
	\caption{$\prorsre$: Reciprocal of Square Root with epsilon}
	\label{pro::reci_sqrt_eps}
	\begin{center}
		\setlength\tabcolsep{5pt}
		\begin{tabular}{c|c|c|c}
		{$\pp_{0}$ Input} & {$ \langle \bar{v}_t \rangle _0 $} 
          & {$\pp_{1}$ Input} & {$\langle \bar{v}_t \rangle _1 $} \\\hline
           \multicolumn{2}{c|}{Public Parameter} & \multicolumn{2}{c}{$\epsilon$} \\
           \hline
            $\pp_0$ Output & $\langle \frac{1}{\sqrt{\bar{v}_t+\epsilon ^2}} \rangle_0 $& $\pp_1$ Output & $\langle  \frac{1}{\sqrt{\bar{v}_t+\epsilon ^2}} \rangle_1 $  \\\hline
		\end{tabular}
	\end{center}
	\begin{algorithmic}[1]
 \STATE $\pp_0,\pp_1$: $\l x\r_0, \l x\r_1 \leftarrow \bar{v}_t+ \epsilon ^2$\label{rsr:l1}
 \STATE $\pp_0,\pp_1$:$\l z\r_0, \l z\r_1\leftarrow\prorsr(\l x\r_0,\hspace{-1mm}\l x\r_1)$\hfill\COMMENT{$\prorsr$'s Lines~\ref{rsrx::l6},\ref{rsrx::l7}}
  \STATE $\pp_0$: $\l s\r_0= \sum _{i=0} ^{L-1} \l y\r_0$\label{rsr:l3} \hfill\COMMENT{$\l y\r_0$ from Line~\ref{rsrx::l1} in $\prorsr$}
  \STATE $\pp_1$: $\l s\r_1= \sum _{i=0} ^{L-1} \l y\r_1$\label{rsr:l4} \hfill\COMMENT{$\l y\r_1$ from Line~\ref{rsrx::l1} in $\prorsr$}
  \STATE $\pp_0,\pp_1$: Execute $\Pi_\SM(\l s\r_0, \l s\r_1, \Pi_{\mathsf{EMult}}(\cdot), \frac{1}{\epsilon})$ to attain 
 $\langle \frac{1}{\sqrt{\bar{v}_t+\epsilon ^2}} \rangle_0,\langle \frac{1}{\sqrt{\bar{v}_t+\epsilon ^2}} \rangle_1$\label{rsr:l11}
	\end{algorithmic}
\end{protocol}

%\smallskip
\noindent\textbf{Composing secure protocol for Adam.}
To design protocols for Adam, we essentially  protect ${\bar{u}_t}/{(\sqrt{\bar{v}_t}+\epsilon)}$, while other computations could be done either by $\prosqr$ or by local computation.
The computation of ${\bar{u}_t}/{(\sqrt{\bar{v}_t}+\epsilon)}$ can be decomposed to multiplication and $1/{(\sqrt{\bar{v}_t}+\epsilon)}$.
We note that the existence of $\epsilon$ is to avoid the denominator vanishing from the root cause; that is, the exact value of $\epsilon$ does not matter much.
Thus, from a crypto perspective, we replace  $1/{(\sqrt{\bar{v}_t}+\epsilon)}$ with $ {1}/\sqrt{\bar{v}_t+\epsilon^2}$ for simplifying secret-shared computation. 
Then, we use the suite of designed protocols to approximate ${1}/\sqrt{\bar{v}_t+\epsilon^2}$.
%, by first computing $\frac{1}{\sqrt{2^k}}$ and then $\frac{1}{\sqrt{x'}}$ such that
%    $\frac{1}{\sqrt{2^k}}\cdot\frac{1}{\sqrt{x'}}=\frac{1}{\sqrt{\bar{v}_t+\epsilon^2}}, {x'}\in [1,2]$.
Empirically, we observe that quadratic polynomial approximation for ${1}/{\sqrt{x''}}$ is accuracy-sufficient for private GCN, thus removing the dependence on $t$ iterations in AdamPriv.
More details are put in Appendix~\ref{sec::sec_adam_app}.

