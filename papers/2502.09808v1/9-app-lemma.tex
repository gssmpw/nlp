%!TEX root = gcn.tex
\section{Selection-Multiplication's Correctness}
\label{sec:proof_lemma}
\begin{proof}[Proof of Lemma~\ref{lem::sxb}]
We analyze the two cases of $s=0$ and $s=1$ for a complete proof where $s \in \{0, 1\}$.

%\begin{itemize}
%	\setlength{\itemsep}{0pt}
%	\setlength{\parskip}{0pt}
%	\setlength{\parsep}{0pt}
%\item[(i)] 
(i). When $s=0$, the equivalence of the first property turns to be $f(0,x+u)=f(0,x)+f(0,u) \Leftrightarrow 0\cdot(x+u) = 0\cdot x+ 0\cdot u \Leftrightarrow 0=0$.
When $s=1$, we have $f(1,x+u)=f(1,x)+f(1,u) \Leftrightarrow 1\cdot(x+u) = 1\cdot x+ 1\cdot u \Leftrightarrow x+u=x+u$.
%\item[(ii)] 

(ii). When $s=0$, the %equivalence of the 
second property %turns to be 
becomes $f(0+b,x)=f(0,x)+(-1)^0f(b,x) \Leftrightarrow bx = 0\cdot x+ 1\cdot bx \Leftrightarrow bx = bx$.
When $s=1$, we have $f(1+b,x)=f(1,x)+(-1)^1f(b,x) \Leftrightarrow (1+b)\cdot x = 1\cdot x - bx$.
If $b=0$, we get $x=x$.
If $b=1$, we get $0=0$ since $(1+b)\!\!\!\!\mod 2=0$.
%\end{itemize}
\end{proof}