\section{Introduction}
Consider a basic question in complexity theory: How can we determine for which problems an essentially quadratic-time algorithm is best possible? If a given problem $A$ admits an algorithm running in $n^{2+o(1)}$ time, and it is known that $A$ cannot be solved in time $O(n^{2-\epsilon})$ for any $\epsilon>0$, then clearly the $n^{2+o(1)}$ algorithm has \emph{optimal} runtime, up to subpolynomial factors.
This question can be asked more generally for any $k\geq 1$ and time $n^{k \pm o(1)}$.
To this day, the theoretical computer science community is far from able to resolve this question unconditionally. However, a surge of results over recent years uses
conditional lower bounds based on plausible hardness assumptions to shed some light on why some problems seemingly cannot be solved in time $O(n^{k-\epsilon})$ for any $\epsilon >0$.
Most notably, reductions from $k$-OV, $k$-SUM and the weighted $k$-clique problem have been used to establish $n^{k-o(1)}$-time conditional lower bounds, often matching known algorithms; see~\cite{williams2018some} for a detailed survey.

In this context, the $3$-SUM hypothesis is arguably the first -- and particularly central -- 
hardness assumption for conditional lower bounds. Initially introduced to explain various quadratic-time barriers
observed in computational geometry~\cite{DBLP:journals/comgeo/GajentaanO95}, it has since been used to 
show quadratic-time hardness for a wealth of problems from various 
fields~\cite{DBLP:journals/siamcomp/WilliamsW13, DBLP:conf/stoc/Patrascu10,DBLP:conf/focs/AbboudW14, DBLP:conf/soda/KopelowitzPP16, DBLP:conf/stoc/0001GS20, DBLP:conf/stoc/AbboudBKZ22, DBLP:conf/stoc/ChanWX22}.
Its generalization, the $k$-SUM\footnote{The $k$-SUM problem asks, given sets $A_1,\dots ,A_k$ of $n$ numbers, whether there exist $a_1 \in A_1 ,\dots ,a_k \in A_k$ such that $\sum_{i=1}^{k}a_i=0$.
The $k$-SUM hypothesis states that for no $\epsilon>0$ there exists a $O(n^{\lceil k/2 \rceil-\epsilon})$ time algorithm that solves $k$-SUM.} hypothesis, has led to further conditional lower bounds beyond the quadratic-time
regime~\cite{DBLP:journals/siamcomp/Erickson99, DBLP:conf/icalp/AbboudL13, DBLP:conf/focs/AbboudBBK17, DBLP:conf/nips/AbboudBBK20, DBLP:conf/focs/Kunnemann22}. For a more comprehensive overview, we refer to~\cite{williams2018some}.

The centrality of the $3$-SUM hypothesis for understanding quadratic-time barriers begs an interesting question: Does $3$-SUM fully capture quadratic-time solvability, in the sense that it is hard for the entire class $\mathsf{DTIME}(n^2)$? Alas, Bloch, Buss, and Goldsmith~\cite{bloch1994hard} give evidence that we are unlikely to prove this: If $3$-SUM is hard for $\mathsf{DTIME}(n^2)$ under quasilinear reductions, then $\mathsf{P} \ne \mathsf{NP}$.
%~\cite{DBLP:conf/cccg/Barrera96,DBLP:journals/ijcga/BarequetH01,DBLP:journals/comgeo/GajentaanO95} \geri{mehr}.
Thus, to understand precisely the role of $3$-SUM to understand quadratic-time computation, the more reasonable question to ask is: 

\begin{center}
\em What is the largest class $\mathcal{C}$ of problems such that $3$-SUM is $\mathcal{C}$-hard?\footnote{Note that there are different reasonable notions of reductions to consider. Rather than the quasilinear reductions used by Bloch et al., we will consider the currently more commonly used notion of fine-grained reductions; see Section~\ref{sec:contributions} for details on the notion of completeness that we will use.}
\end{center}

Finding a large class $\mathcal{C}$ for which $3$-SUM is hard may be seen as giving evidence for the $3$-SUM hypothesis. Furthermore, such a result may clarify the true expressive power of the $3$-SUM hypothesis, much like the NP-completeness of 3-SAT highlights its central role for polynomial intractability.

\subsection{Our approach}
We approach our central question from a descriptive complexity perspective. This line of research
has been initiated by Gao et al.~\cite{GaoIKW19}, who establish the sparse OV problem as complete for the class of model checking first-order properties.
One can interpret this result as showing that the OV problem expresses relational database queries in the sense that a truly subquadratic algorithm for OV would improve the fine-grained data 
complexity of such queries (see~\cite{GaoIKW19} for details). Related works further delineate the fine-grained hardness of model checking first-order properties and related 
problem classes~\cite{DBLP:conf/csl/Williams14, DBLP:conf/coco/BringmannFK19, DBLP:conf/approx/BringmannCFK21, DBLP:journals/algorithmica/AnGIJKN22, DBLP:conf/icalp/BringmannCFK22,FischerKR24}, see Section~\ref{sec:relatedWork} for more discussion. %uniform version of the circuit complexity class $\mathsf{AC}^0$. 

Towards continuing the line of research on fine-grained completeness theorems,
we introduce a class of problems corresponding to deciding formulas in linear integer arithmetic over finite sets of integers. Specifically, consider the vectors $$x_1 =(x_1[1],\dots, x_1[d_1]) , \dots, x_k=(x_k[1],\dots, x_k[d_k])$$ as quantified variables, and let $t_1,\dots ,t_l$ be free variables.
Moreover, let 
\[X:=\{x_1[1],\dots, x_1[d_1], \dots x_k[1], \dots, x_k[d_k], t_1,\dots ,t_l\},\]
and let $\psi$ be a quantifier-free linear arithmetic formula over variables in $X$.
We consider the model-checking problem for formulas $\phi$ in the prenex normal form 
\[ \phi:=Q_1 x_1 \dots Q_k x_k: \psi, \]
where the quantifiers $Q_1, \dots, Q_k \in \{ \exists, \forall \}$ are arbitrary. Formally, for such a $\phi$, we define the model checking problem $\mathsf{FOP}_{\mathbb{Z}}(\phi)$ as follows\footnote{Below, we use the notation $\psi[(t_1,\dots, t_l) \backslash (\hat{t_1},\dots, \hat{t_l})]$ to denote the substitution of the variables $t_1,\dots,t_l$ by $\hat{t_1},\dots,\hat{t_l}$ respectively.}
\begin{align}
	\mathsf{FOP}_{\mathbb{Z}}(\phi):  & \label{model_check}\\
\textbf{Input:}& \text{ Finite sets }A_1 \subseteq \mathbb{Z}^{d_1} ,\dots, A_k \subseteq  \mathbb{Z}^{d_k}\text{ and }\hat{t_1},\dots, \hat{t_l} \in \mathbb{Z}. \notag \\
\textbf{Problem:} &\text{ Does } Q_1 x_1 \in A_1 \dots Q_k x_k \in A_k : \psi[(t_1,\dots, t_l) \backslash (\hat{t_1},\dots, \hat{t_l})] \text{ hold?} \notag
\end{align}
We let $n := \max_i\{|A_i|\}$ denote the input size and will assume throughout the paper that all input numbers (i.e., the coordinates of the vectors in $A_1,\dots, A_k$ and the values $\hat{t_1},\dots, \hat{t_l}$) are chosen from a polynomially sized universe, i.e., $\{-U, \dots, U\}$ with $U\le n^{c}$ for some~$c$.
Let $\mathsf{FOP}_{\mathbb{Z}}$ be the union of all $\mathsf{FOP}_{\mathbb{Z}}(\phi)$ problems, where $\phi$ has at least 3 quantifiers.\footnote{It is not too difficult to see that all formulas with 2 quantifiers can be model-checked in near-linear time; see the full version for details.}
Besides $3$-SUM, a variety of interesting problems is contained in $\mathsf{FOP}_{\mathbb{Z}}$; we discuss a few notable examples below and in the full version.

Frequently, we will distinguish formulas in $\mathsf{FOP}_{\mathbb{Z}}$ using their quantifier structure; e.g., $\mathsf{FOP}_{\mathbb{Z}}(\exists \exists \forall)$ describes
the class of model checking problems $\mathsf{FOP}_{\mathbb{Z}} (\phi)$ where in $\phi$ we have $Q_1=Q_2=\exists $ and $Q_3=\forall.$ 
Furthermore, we let $\mathsf{FOP}_{\mathbb{Z}}^k$ be the union of all $\mathsf{FOP}_{\mathbb{Z}}(\phi)$ problems, where $\phi$ consists of precisely $k$ quantifiers, regardless of their quantifier structure.
For a quantifier $Q \in \{\exists,\forall\}$, we write $Q^k$ for the repetition $\underbrace{Q \dots Q}_{k \text{ times}}$.
Finally, we remark that a small subset of $\mathsf{FOP}_{\mathbb{Z}}$ has already been studied by An et al. \cite{DBLP:journals/algorithmica/AnGIJKN22}, for a discussion see Section \ref{sec:relatedWork}.
%Finally, let $\mathsf{FOP}_{\mathbb{Z}}$ be the class of all model checking problems for valid $\phi$. Let us consider an application of 
%$\exists^3 \mathsf{FOP}_{\mathbb{Z}}$. \geri{The counting versions of the problems are naturally defined, take the model checking problem and ask for the number of solutions.}
%\geri{Then just make the classes analogously.}
%In the first part of the paper we focus on the class $\exists^k \mathsf{FOP}_{\mathbb{Z}}$ for $k\geq 3$.
%These classes of problems are able to model a variety of tasks. For instance, assuming the 3-uniform hyperclique hypothesis,

\subsection{Our Contributions}\label{sec:contributions}
%In the following, we go through our contributions at a high level. 
%For more technical results, and a detailed description of our results for each class with precisely $3$ quantifiers, see the Technical Overview Section. 


We seek to determine completeness results for the class $\mathsf{FOP}_{\mathbb{Z}}$.
In particular: What are the largest fragments of this class for which 3-SUM (or more generally, $k$-SUM) is complete? Is there a problem that is complete for the entire class?

Intuitively, we say that a $T_A(n)$-time solvable problem $A$ is \emph{(fine-grained) complete} for a $T_{\mathcal{C}}(n)$-time solvable class of problems $\mathcal{C}$, if the existence of an $O(T_A(n)^{1-\epsilon})$-time algorithm for~$A$ with $\epsilon>0$ implies that for \emph{all} problems $C$ in $\mathcal{C}$ there exists $\delta > 0$ such that $C$ can be solved in time $O(T_{\mathcal{C}}(n)^{1-\delta})$. We extend this notion to completeness of a \emph{family} of problems, since strictly speaking, any (geometric) problem over $\mathbb{Z}^d$ expressible in linear integer arithmetic corresponds to a family of formulas $\mathsf{FOP}_{\mathbb{Z}}$ (one for each $d\in \mathbb{N}$).
Formally, consider a family of problems $\mathcal{P}$ with an associated time bound $T_{\mathcal{P}}(n)$ and a class of problems~$\mathcal{C}$ with an associated time bound $T_{\mathcal{C}}(n)$; usually $T_{\mathcal{P}}(n), T_{\mathcal{C}}(n)$ denote the running time of the fastest known algorithm solving all problems in $\mathcal{P}$ or $\mathcal{C}$, respectively (often, we omit these time bounds, as they are clear from context).\footnote{Here, we use \emph{family} and \emph{class} as a purely semantic and intuitive distinction: A family consists of a small set of similar problems, and a class consists of a large and diverse variety of problems.}
We say that $\mathcal{P}$ is \emph{(fine-grained) complete} for $\mathcal{C}$, if
\begin{enumerate}
\item the family $\mathcal{P}$ is a subset of the class $\mathcal{C}$, and
\item if for all problems $P$ in $\mathcal{P}$ there exists $\epsilon>0$ such that $P$ can be solved in time $O(T_{\mathcal{P}}(n)^{1-\epsilon})$, then for all problems $C$ in $\mathcal{C}$ there exists some $\delta>0$ such that we can solve $C$ in time $O(T_{\mathcal{C}}(n)^{1-\delta})$.
\end{enumerate}
That is, a polynomial-factor improvement for solving the problems in $\mathcal{P}$ would lead
to a polynomial-factor improvement in solving \emph{all} problems in $\mathcal{C}$.
If a singleton family $\mathcal{P} = \{P\}$ is fine-grained complete for $\mathcal{C}$, then we also say that $P$ is fine-grained complete for $\mathcal{C}$.
We work with standard hypotheses and problems encountered in fine-grained complexity; for detailed definitions of these, we refer to the full version of this article.

\subsubsection{$k$-SUM is complete for the existential fragment of $\mathsf{FOP}_{\mathbb{Z}}$}
Consider first the existential fragment of $\FOPZ$, i.e., formulas exhibiting only existential quantifiers.
Any $\mathsf{FOP}_{\mathbb{Z}}$ formula with $k$ existential quantifiers can be decided using a standard meet-in-the-middle approach, augmented by orthogonal range search, in time $\Tilde{O}(n^ {\lceil k/2 \rceil})$\footnote{We use the notation $\Tilde{O}(T):=T \log^{O(1)}(T)$ to hide polylogarithmic factors.}, see the full version of the paper for details. Since $k$-SUM is a member of $\FOPZ(\exists^k)$, this running time is optimal up to subpolynomial factors, assuming the $k$-SUM Hypothesis.
As our first contribution, we provide a converse reduction. Specifically, we show that a polynomially improved $k$-SUM algorithm would give a polynomially improved algorithm for solving the entire class. In our language, we show that $k$-SUM is fine-grained complete for formulas of $\mathsf{FOP}_{\mathbb{Z}}$ with $k$ existential quantifiers.

\begin{restatable}[$k$-SUM is $\mathsf{FOP}_{\mathbb{Z}}(\exists^k)$-complete]{theorem}{threesumcompl}
	Let $k\geq 3$ and assume that $k$-SUM can be solved in time $T_{k\mathrm{SUM}}(n)$. For any problem $P$ in $\mathsf{FOP}_{\mathbb{Z}}(\exists^k )$, there exists some $c$ such that $P$ can be solved in time $O(T_{k\mathrm{SUM}}(n) \log^c n)$. 
\label{existential_complete}
\end{restatable}

Thus, if there are $k\geq 3$ and $\epsilon>0$ such that we can solve $k$-SUM in time $O(n^{\lceil k/2 \rceil-\epsilon})$, then we can solve all problems in $\mathsf{FOP}_{\mathbb{Z}}(\exists^k )$ in time  $O(n^{\lceil k/2 \rceil-\epsilon'})$ for any $0 < \epsilon' < \epsilon$.
By a simple negation argument, we conclude that $k$-SUM is also complete for the class of problems $\mathsf{FOP}_{\mathbb{Z}}(\forall^k)$.

The above theorem generalizes and unifies previous reductions from problems expressible as $\FOPZ(\exists^k)$ formulas to 3-SUM, using different proof ideas: Jafargholi and Viola~\cite[Lemma 4]{DBLP:journals/algorithmica/JafargholiV16} give a simple randomized linear-time reduction from triangle detection in sparse graphs to 3-SUM, and a derandomization via certain combinatorial designs.
Dudek, Gawrychowski, and Starikovskaya~\cite{DBLP:conf/stoc/0001GS20} study the family of 3-linear degeneracy testing (3-LDT), which constitutes a large and interesting subset of $\FOPZ(\exists \exists \exists)$: This family includes, for any $\alpha_1,\alpha_2,\alpha_3, t\in \mathbb{Z}$, the \emph{3-partite} formula $\exists a_1 \in A_1 \exists a_2\in A_2 \exists a_3\in A_3: \alpha_1 a_1 + \alpha_2 a_2 + \alpha_3 a_3 = t$ and the \emph{1-partite} formula $\exists \alpha_1, \alpha_2, \alpha_3 \in A: \alpha_1 a_1 + \alpha_2 a_2 + \alpha_3 a_3 = t \wedge a_1\ne a_2 \wedge a_2\ne a_3 \wedge a_1\ne a_3$.
The authors show that each such formula is either trivial or subquadratic \emph{equivalent} to $3$-SUM.
For 3-partite formulas, a reduction to $3$-SUM is essentially straightforward.
For 1-partite formulas, Dudek et al.~\cite{DBLP:conf/stoc/0001GS20} use color coding.\footnote{We remark that the reverse direction, i.e., $3$-SUM-hardness of non-trivial formulas, is technically much more involved and can be regarded as the main technical contribution of~\cite{DBLP:conf/stoc/0001GS20}.} %At this point we would like to differentiate between the aim of our paper and the work of Dudek et al. .

As further examples for reductions from $\FOPZ$ problems to $k$-SUM, we highlight a reduction from Vector $k$-SUM to $k$-SUM~\cite{DBLP:journals/corr/AbboudLW13} as well as a reduction from $(\min,+)$-convolution to $3$-SUM (see~\cite{BackursIS17,DBLP:journals/talg/CyganMWW19}) based on a well-known bit-level trick due to Vassilevska Williams and Williams~\cite{DBLP:journals/siamcomp/WilliamsW13}, which allows us to reduce inequalities to equalities. 

Perhaps surprisingly in light of its generality and applicability, Theorem~\ref{existential_complete} is obtained via a very simple, deterministic reduction that combines the tricks from~\cite{DBLP:journals/corr/AbboudLW13,DBLP:journals/siamcomp/WilliamsW13}. This generality comes at the cost of polylogarithmic factors (which we do not optimize), which depend on the number of inequalities occurring in the considered formula; for the details see Section \ref{sec:existentialpa} and the full version of the paper.






\subsubsection{Completeness for counting witnesses}
We provide a certain extension of the above completeness result to the problem class of \emph{counting} witnesses to existential $\FOPZ$ formulas\footnote{A witness for a $\FOPZ(\exists^k)$ formula $\exists a_1 \in A_1 \dots \exists a_k \in A_k: \varphi$ with $\hat{t_1}, \dots, \hat{t_l} \in \mathbb{Z}$ is a tuple $(a_1,\dots, a_k)\in A_1\times \cdots \times A_k$ that satisfies the formula $\varphi[(t_1,\dots,t_l) \backslash (\hat{t_1},\dots,\hat{t_l})]$.}. Counting witnesses is an important task particularly in database applications (usually referred to as model counting). Furthermore, we will make use of witness counting to \emph{decide} certain quantified formulas in subsequent results detailed below. In Section \ref{sec:countwitn}, we will obtain the following result. 

\begin{restatable}{theorem}{countComplete}
	Let $k\geq3$ be odd. If there is $\epsilon > 0$ such that we can count the number of witnesses for $k$-SUM in time $O(n^{\lceil k/2 \rceil -\epsilon})$, then for all problem $P$ in $\FOPZ(\exists^k)$, there is some $\epsilon'> 0$ such that we can count the number of witnesses for $P$ in time $O(n^{\lceil k/2 \rceil -\epsilon'})$.
	\label{thm:counting-witnesses}
\end{restatable}

Leveraging the recent breakthrough by~\cite{DBLP:journals/corr/abs-2303-14572} that 3-SUM is subquadratic equivalent to counting witnesses of 3-SUM, we obtain the corollary that \emph{3-SUM is hard even for counting witnesses of $\FOPZ(\exists^3)$}.


\begin{restatable}{corollary}{threecountcomplete}
	For all problems $P$ in $\mathsf{FOP}_{\mathbb{Z}}(\exists^3)$, there is some $\epsilon_P >0$ such that we can count the number of witnesses for $P$ in randomized time $O(n^{2-\epsilon_P})$ if and only if there is some $\epsilon' > 0$ such that
    $3$-SUM can be solved in randomized time $O(n^{2-\epsilon'})$. 
\label{Count3COMP}
\end{restatable}


\subsubsection{Completeness for general quantifier structures of $\mathsf{FOP}_{\mathbb{Z}}$ }

In light of our first completeness result, one might wonder whether $k$-SUM is complete for deciding all $k$-quantifier formulas in $\mathsf{FOP}_{\mathbb{Z}}$,
regardless of the quantifier structure of the formulas.
Note that for these general quantifier structures, a baseline algorithm with running time $\Tilde{O}(n^{k-1})$ can be achieved with a combination of brute-force and orthogonal range queries; see the full version for details. 

However, by \cite[Theorem 15]{DBLP:journals/algorithmica/AnGIJKN22} there exists a $\FOPZ(\exists^{k-1} \forall)$-formula $\phi$ that cannot be solved in time $O(n^{k-1-\epsilon})$-time unless the 3-uniform hyperclique hypothesis is false (see the discussion in Section~\ref{sec:relatedWork}). Thus, proving that 3-SUM is complete for all 3-quantifier formulas would establish that the 3-uniform hyperclique hypothesis implies the $3$-SUM hypothesis -- this would be a novel tight reduction among important problems/hypotheses in fine-grained complexity theory. For $k\ge 4$, it becomes even more intricate: the conditionally optimal running time of $n^{k-1\pm o(1)}$ for $\FOPZ(\exists^k \forall)$ formulas exceeds the conditionally optimal running time of $n^{\lceil \frac{k}{2} \rceil \pm o(1)}$ for $\FOPZ(\exists^k)$ formulas.  

We are nevertheless able to obtain a completeness result for general quantifier structures: Specifically, we show that if two geometric problems over $\mathbb{Z}^d$ can be solved in time $O(n^{2-\epsilon_d})$ where $\epsilon_d > 0$ for all $d$, then each $k$-quantifier formula in $\FOPZ$ can be decided in time $O(n^{k-1-\epsilon})$ for some $\epsilon > 0$. These problems are (1) a variation of the Hausdorff distance that we call \emph{Hausdorff distance under $n$ Translations} and (2) the Pareto Sum problem; the details are covered in Section \ref{sec:GeneralQuantifier}.

\paragraph*{Hausdorff Distance under $n$ Translations}
Among the most common translation-invariant distance measures for given point sets $B$ and $C$ is the Hausdorff Distance under Translation~\cite{chew1992improvements, DBLP:conf/compgeom/BringmannN21, DBLP:conf/compgeom/Chan23, DBLP:journals/dcg/ChewDEK99, Nusser21, HuttenlocherK90}. 
To define it, we denote the directed Hausdorff distance under the $L_{\infty}$ metric by $\delta_{\overrightarrow{H}}(B,C):= \max_{ b \in B} \min_{c \in C} \|b-c\|_{\infty}$.\footnote{Since we will exclusively consider the \emph{directed} Hausdorff distance under Translation, we will drop ``directed'' throughout the paper.} The Hausdorff distance under translation $\delta_{\overrightarrow{H}}^T(B,C)$ is defined as the minimum Hausdorff distance of $B$ and an arbitrary translation of $C$, i.e.,
\[ \delta_{\overrightarrow{H}}^T(B,C) \coloneqq \min_{\tau \in \mathbb{R}^d} \delta_{\overrightarrow{H}}(B,C+\{\tau\}) = \min_{\tau \in \mathbb{R}^d} \max_{b\in B} \min_{c\in C} \|b-(c+\tau)\|_{\infty}.\]
For $d=2$, Bringmann et al.~\cite{DBLP:conf/compgeom/BringmannN21} were able to show a $(|B||C|)^{1-o(1)}$ time lower bound based on the orthogonal vector hypothesis, and there exists a matching $\Tilde{O}(|B||C|)$ upper bound by Chew et al.~\cite{ChewK92}.

We shall establish that restricting the translation vector to be among a set of $m$ candidate vectors yields a central problem in $\FOPZ$. Specifically, we define the Hausdorff distance under Translation in $A$, denoted as $\delta_{\overrightarrow{H}}^{T(A)}(B,C)$, by
 \[\delta_{\overrightarrow{H}}^{T(A)}(B,C) \coloneqq \min_{\tau \in A} \delta_{\overrightarrow{H}}(B,C+\{\tau\}) = \min_{\tau \in A} \max_{b\in B} \min_{c\in C} \|b-(c+\tau)\|_{\infty}.\]
Correspondingly, we define the problem \emph{Hausdorff distance under $m$ Translations} as: Given $A, B, C\subseteq \mathbb{Z}^d$ with $|A|\le m$, $|B|,|C| \le n$ and a distance value $\gamma \in \mathbb{N}$, determine whether $\delta_{\overrightarrow{H}}^{T(A)}(B,C) \le \gamma$.
Note that this can be rewritten as a $\FOPZ(\exists\forall\exists)$-formula, see the full version of the paper for details.

The Hausdorff distance under $m$ Translations occurs naturally when approximating the Hausdorff distance under translation: Specifically, common algorithms compute a set~$A$ of $|A|= f(\epsilon)$ translations such that $\delta_{\overrightarrow{H}}^{T(A)}(B,C) \le (1+\epsilon)\delta_{\overrightarrow{H}}^{T}(B,C)$.
Generally, this problem is then solved by performing $|A|$ computations of the Hausdorff distance, which yields $\tOh(|A|n)=\tOh(f(\epsilon)n)$-time algorithms \cite{Wenk03}. Improving over the $\tOh(m n)$-time baseline for Hausdorff Distance under $m$ Translations would thus lead to immediate improvements for approximating the Hausdorff Distance under Translation.
Our results will establish additional consequences of fast algorithms for this problem: an $O(n^{2-\epsilon_d})$-time algorithm with $\epsilon_d > 0$ for Hausdorff distance under $n$ Translations would give an algorithmic improvement for the classes of $\FOPZ(\exists \forall \exists)$- and $\FOPZ(\forall \exists \forall)$-formulas.





\paragraph*{Verification of Pareto Sums}

Our second geometric problem is a verification version of computing \emph{Pareto sums}: Given point sets $A,B \subseteq \mathbb{Z}^{d}$, the Pareto sum $C$ of $A,B$ is defined as the Pareto front of their sumset $A+B=\{a+b\mid a\in A,b\in B\}$. Put differently, the Pareto sum of $A,B$ is a set of points $C$ satisfying (1) $C \subseteq A+B$,  (2) for every $a \in A$ and $b \in B$, the vector $a+b$ is dominated\footnote{We consider the usual domination notion: A vector $u\in \mathbb{Z}^d$ is dominated by some vector $v\in \mathbb{Z}^d$ (written $u \leq v$) if and only if in all dimensions $i\in [d]$ it holds that $u[i]\leq v[i]$.} by some $c \in C$ and (3) there are no distinct $c,c' \in C$ such that $c'$ dominates $c$. 
The task of computing Pareto sums appears in various multicriteria optimization settings~\cite{artigues2013state,schulze2019multi,ehrgott2000survey,lust2014variable}; fast output-sensitive algorithms (both in theory and in practice) have recently been investigated by Hespe, Sanders, Storandt, and Truschel~\cite{DBLP:conf/esa/Hespe0ST23}. 

We consider the following problem as \emph{Pareto Sum Verification}: Given $A,B,C\subseteq \mathbb{Z}^d$, determine whether
\begin{align*}
\forall a \in A \forall b \in B \exists c \in C: a+b \le c.
\end{align*}
The complexity of Pareto Sum Verification\footnote{We remark that our problem definition only checks a single of the three given conditions, specifically, condition~(2). However, in Section~\ref{sec:ParetoSum}, we will establish that the verifying \emph{all three} conditions reduces to verifying this single condition. More specifically, for sets $A,B,C$ of size at most $n$, we obtain that if we can solve Pareto Sum Verification in time $T(n)$, then we can check whether $C$ is the Pareto sum of $A,B$ in time $O(T(n))$.} is tightly connected to output-sensitive algorithms for Pareto Sum. Specifically, solving Pareto Sum Verification reduces to \emph{computing} the Pareto sum $C$ when given inputs $A,B$ of size at most $n$ with the promise that $|C| =\Theta(n)$; see Section~\ref{sec:ParetoSum} for details. The work of Hespe et al.~\cite{DBLP:conf/esa/Hespe0ST23} gives a practically fast $O(n^2)$-time algorithm in this case for $d=2$; note that for $d\ge 3$, we still obtain an $\tOh(n^2)$-time algorithm via our Baseline Algorithm, which is described in the full version of the paper.



\paragraph{A problem pair that is complete for $\FOPZ$}
As a pair, these two geometric problems turn out to be fine-grained complete for the class $\mathsf{FOP}_{\mathbb{Z}}$.
  \begin{restatable}{theorem}{completenessclass}
	There is a function $\epsilon(d)>0$ such that 
	both of the following problems can be solved in time $O(n^{2-\epsilon(d)})$
	\begin{itemize}
	\item Pareto Sum Verification,
	\item Hausdorff distance under $n$ Translations,
	\end{itemize}
	if and only if for each problem $P$ in $\mathsf{FOP}_{\mathbb{Z}}^k$ with $k\geq3$ there exists an $\epsilon_P>0$ such that $P$ can be solved in time $O(n^{k-1-\epsilon_P})$. 
	\label{completenesswholeFOP3}
	\end{restatable}

The above theorem shows that a single pair of natural problems captures the fine-grained complexity of the expressive and diverse class $\FOPZ$. As an illustration just how expressive this class is, we observe the following barriers:\footnote{The first three statements follow from $\FOPZ$ generalizing the class $PTO$ studied in~\cite{DBLP:journals/algorithmica/AnGIJKN22}, see Section~\ref{sec:relatedWork}. The remaining statements rely on the additive structure of $\FOPZ$.}
\begin{enumerate}
	\item If there is some $\epsilon >0$ such that all problems in $\FOPZ(\exists \exists \forall)$ (or $\FOPZ(\forall \forall \exists)$) can be solved in time $O(n^{2-\epsilon})$, then OVH (and thus SETH) is false \cite[Theorem 16]{DBLP:journals/algorithmica/AnGIJKN22}.
	\item If there is some $\epsilon >0$ such that all problems in $\FOPZ(\exists \forall \exists)$ (or $\FOPZ(\forall \exists \forall)$) can be solved in time $O(n^{2-\epsilon})$, then the Hitting Set Hypothesis is false \cite[Theorem 12]{DBLP:journals/algorithmica/AnGIJKN22}.
	\item If for all problems $P$ in $\FOPZ(\exists \exists \forall)$ (or $\FOPZ(\forall \forall \exists)$), there exists some $\epsilon>0$ such that we can solve $P$ in  $O(n^{2-\epsilon})$, then the 3-uniform Hyperclique Hypothesis is false \cite[Theorem 15]{DBLP:journals/algorithmica/AnGIJKN22}.
	\item If for all problems $P$ in $\FOPZ(\exists \exists \exists)$ ($\FOPZ(\forall \forall \forall), \FOPZ(\forall \forall \exists)$, or $\FOPZ(\exists \exists \forall)$), there exists some $\epsilon>0$ such that we can solve $P$ in time $O(n^{2-\epsilon})$, then the $3$-SUM Hypothesis is false (Theorem \ref{existential_complete} with Lemma \ref{verif-complete-three}).
\end{enumerate}
Theorem \ref{completenesswholeFOP3} raises the question whether for any constant dimension $d$, the Hausdorff distance under $n$ Translations admits a subquadratic reduction to Pareto Sum Verification. A positive answer would establish Pareto Sum Verification as complete for the \emph{entire} class $\FOPZ$. We elaborate on this in Section~\ref{sec::ende}.
 


\subsubsection{$3$-SUM is complete for $\mathsf{FOP}_{\mathbb{Z}}$ formulas of low inequality dimension}

Returning to our motivating question, we ask: Since it appears unlikely to prove completeness of 3-SUM for all $\FOPZ$ formulas (as this requires a tight 3-uniform hyperclique lower bound for 3-SUM), can we at least identify a large fragment of $\FOPZ$ for which $3$-SUM is complete? In particular, can we extend our first result of Theorem~\ref{existential_complete} from existentially quantified formulas to substantially different problems in $\FOPZ$, displaying other quantifier structures?
 
Surprisingly, we are able to show that $3$-SUM is complete for \emph{low-dimensional} $\FOPZ$ formulas, \emph{independent of their quantifier structure}.
To formalize this, we introduce the \emph{inequality dimension} of a $\FOPZ$ formula as the smallest number of linear inequalities required to model it. More formally, consider a $\FOPZ$ formula $\phi = Q_1 x_1\in A_1, \dots, Q_k x_k\in A_k: \psi$ with $A_i\subseteq \mathbb{Z}^{d_i}$. The \emph{inequality dimension} of $\phi$ is the smallest number $s$ such that there exists a Boolean function $\psi' :\{0,1\}^s \to \{0,1\}$ and (strict or non-strict) linear inequalities $L_1, \dots, L_s$ in the variables $\{x_i[j] : i\in \{1,\dots,k\} ,j\in \{1,\dots,d_i\} \}$ and the free variables such that $\psi(x_1,\dots, x_k)$ is equivalent to $\psi'(L_1,\dots, L_s)$. As an example, the 3-SUM formula $\exists a\in A \exists b\in B\exists c\in C: a+b=c$ has inequality dimension 2, as $a+b=c$ can be modelled as conjunction of the two linear inequalities $a+b \le c$ and $a+b \ge c$, whereas no single linear inequality can model $a+b=c$.

We show that $3$-SUM is fine-grained complete for model-checking $\mathsf{FOP}^3_{\mathbb{Z}}$ formulas with inequality dimension at most $3$.
This result is our perhaps most interesting technical contribution and intuitively combines our result that $3$-SUM is hard for counting $\FOPZ$ witnesses (Corollary~\ref{Count3COMP}) with a geometric argument, specifically, that the union of $n$ unit cubes in $\mathbb{R}^3$ can be decomposed into the union of $O(n)$ pairwise interior- and exterior-disjoint axis-parallel boxes. To this end, we extend a result from \cite{DBLP:journals/dcg/ChewDEK99}, which constructs pairwise interior-disjoint axis-parallel boxes, to also achieve exterior-disjointness. For more details, see the Technical Overview below and Section~\ref{sec:IneqDimension}. 
\begin{restatable}[]{theorem}{threesumineq}
	There is an algorithm deciding $3$-SUM in randomized time $O(n^{2-\epsilon})$ for an $\epsilon>0$,
	if and only if for each problem $P$ in $\mathsf{FOP}^{k}_{\mathbb{Z}}$ with $k\geq 3$ and inequality dimension at most $3$, there exists some $\epsilon > 0$ such that we can solve $P$ in randomized time $O(n^{k-1-\epsilon})$.
	\label{three-sum-Completeness-all-quantifer}
\end{restatable}

Note that this fragment of $\FOPZ$ contains a variety of interesting problems. A general example is given by comparisons of sets defined using the sumset arithmetic\footnote{The sumset arithmetic uses the sumset operator $X+Y$ to denote the sumset $\{x+y\mid x\in X, y\in Y\}$ and $\lambda X$ to denote $\{\lambda x \mid x\in X\}$.}, which correspond to formulas of inequality dimension at most 2: E.g., checking, given sets $A,B,C\subseteq \mathbb{Z}$ and $t\in \mathbb{Z}$, whether $C$ is an additive $t$-approximation of the sumset $A+B$ is equivalent to verifying the conjunction of the $\FOPZ(\forall \forall \exists)$ problem\footnote{Note that the corresponding formula is $\forall a\in A\forall b\in B \exists c\in C: (c\le a+b) \wedge (a+b \le c+t)$, which clearly has inequality dimension at most 2.} $A+B \subseteq C+\{0,\dots,t\}$ and (2) the $\FOPZ(\forall \exists \exists)$ problem\footnote{Note that the corresponding formula is $\forall c\in C\exists a\in A\exists b\in B: a+b = c$, which clearly has inequality dimension at most 2.} $C\subseteq A+B$. Likewise, this extends to $\lambda$-multiplicative approximations of sumsets. Furthermore, the problems corresponding to general sumset comparisons like $\alpha_1 A_1 + \cdots + \alpha_i A_i \subseteq \alpha_{i+1} A_{i+1} + \cdots + \alpha_k A_{k} + \{-\ell, \dots, u\}$ have inequality dimension at most~$2$ as well.

Our results of Theorems~\ref{completenesswholeFOP3} and~\ref{three-sum-Completeness-all-quantifer} suggests to view Pareto Sum Verification as a geometric, high-dimensional generalization of $3$-SUM. Furthermore, it remains an interesting problem to establish the highest $d$ such that 3-SUM is complete for $\FOPZ$ formulas of inequality dimension at most $d$; for a discussion see Section \ref{sec::ende}.




\paragraph*{Further Applications}
As an immediate application of our first completeness theorem, we obtain a simple proof of a $n^{4/3-o(1)}$ lower bound for the 4-SUM problem based on the the $3$-uniform hyperclique hypothesis; see the full version of the paper for details. Specifically, by Theorem~\ref{existential_complete}, it suffices to model the 3-uniform 4-hyperclique problem as a problem in $\mathsf{FOP}_{\mathbb{Z}}(\exists \exists \exists \exists)$. The resulting conditional lower bound is implicitly known in the literature, as it can alternatively be obtained by combining a 3-uniform hyperclique lower bound for $4$-cycle given in~\cite{DBLP:conf/soda/LincolnWW18} with a folklore reduction from $4$-cycle to $4$-SUM (see~\cite{DBLP:journals/algorithmica/JafargholiV16} for a deterministic reduction from $3$-cycle to $3$-SUM).
\begin{restatable}{theorem}{lowerbound} \label{thm:4SUMlb}
    If there is some $\epsilon > 0$ such that 4-SUM can be solved in time $O(n^{\frac{4}{3}-\epsilon})$, then the 3-uniform hyperclique hypothesis fails. 
\end{restatable}
 Similarly, we can also give a simple proof for a known lower bound for $3$-SUM.

Another application of our results is to establish class-based conditional bounds. As a case in point, consider the problem of computing the Pareto sum of $A,B\subseteq \mathbb{Z}^d$: Clearly, this problem can be solved in time $\tOh(n^2)$ by explicitly computing the sumset $A+B$ and computing the Pareto front using any algorithm running in near-linear time in its input, e.g.~\cite{DBLP:conf/stoc/GabowBT84}. We prove the following conditional optimality results already in the case when the desired output (the Pareto sum of $A,B$) has size $\Theta(n)$. 
\begin{restatable}[Pareto Sum Computation Lower Bound]{theorem}{lowerboundPS}
	The following conditional lower bounds hold for output-sensitive Pareto sum computation:
	\begin{enumerate}
		\item If there is $\epsilon > 0$ such that we can compute the Pareto sum $C$ of $A,B\subseteq \mathbb{Z}^2$, whenever $C$ is of size $\Theta(n)$, in time $O(n^{2-\epsilon})$, then the $3$-SUM hypothesis fails (thus, for any $\FOPZ^k$ formula $\phi$ of inequality dimension at most 3, there is $\epsilon'>0$ such that $\phi$ can be decided in time $O(n^{k-1-\epsilon'})$). 
		\item If for all $d\ge 2$, there is $\epsilon > 0$ such that we can compute the Pareto sum $C$ of $A,B\subseteq \mathbb{Z}^d$, whenever $C$ is of size $\Theta(n)$, in time $O(n^{2-\epsilon})$, then there is some $\epsilon' > 0$ such that we can decide all $\FOPZ$ formulas with $k$ quantifiers not ending in $\exists \forall \exists$ or $\forall \exists \forall$ in time $O(n^{k-1-\epsilon'})$. 
	\end{enumerate}
	\end{restatable}

Our lower bound for 2D strengthens a quadratic-time lower bound found by Funke et al.~\cite{HespeSST24-PC} based on the $(\min,+)$-convolution hypothesis to hold already under the weaker (i.e., more believable) $3$-SUM hypothesis. For higher dimensions, we furthermore strengthen the conditional lower bound via its connection to $\FOPZ$.

We conclude with remaining open questions in Section \ref{sec::ende}.







\subsection{Further Related Work}\label{sec:relatedWork}
To our knowledge, the first investigation of the connection between classes of model-checking problems and central problems in fine-grained complexity was given by Williams~\cite{DBLP:conf/csl/Williams14}, who shows that the $k$-clique problem is complete for the class of existentially-quantified first order graph properties, among other results. As important follow-up work, Gao et al.~\cite{GaoIKW19} establish OV as complete problem for model-checking any first-order property.

Subsequent results include classification results for $\exists^k \forall$-quantified first-order graph properties~\cite{DBLP:conf/coco/BringmannFK19}, fine-grained upper and lower bounds for counting witnesses of first-order properties~\cite{DBLP:conf/icalp/DellRW19}, completeness theorems for multidimensional ordering properties~\cite{DBLP:journals/algorithmica/AnGIJKN22} (discussed below), completeness and classification results for optimization classes~\cite{DBLP:conf/approx/BringmannCFK21,DBLP:conf/icalp/BringmannCFK22} as well as an investigation of sparsity for monochromatic graph properties~\cite{FischerKR24}.

We remark that An et al.~\cite{DBLP:journals/algorithmica/AnGIJKN22} study completeness results for a strict subset of $\FOPZ$ formulas: Specifically, they introduce a class $\PTO_{k,d}$ of $k$-quantifier first-order sentences over inputs $\mathbb{N}^d$ (or, without loss of generality $\{1,\dots, n\}^d$) that may only use \emph{comparisons} of coordinates (and constants).
Note that such sentences lack additive structure, and indeed the fine-grained complexity differs decisively from $\FOPZ$: E.g., for $\PTO(\exists \exists \exists)$ formulas, they establish the sparse triangle detection problem as complete,  establishing a conditionally tight running time of $m^{2\omega/(\omega+1)\pm o(1)}$.
This is in stark contrast to $\FOPZ(\exists \exists \exists)$ formulas, for which we establish $3$-SUM as complete problem, yielding a conditionally optimal running time of $n^{2\pm o(1)}$. In particular, for each 3-quantifier structure $Q_1Q_2Q_3$, a $O(n^{2-\epsilon})$-time algorithm for all $\FOPZ(Q_1Q_2Q_3)$ problems would break a corresponding hardness barrier\footnote{Specifically, an $O(n^{2-\epsilon})$ time algorithm for problems in $\mathsf{FOP}_{\mathbb{Z}}(\exists \exists \exists), \FOPZ(\forall \forall \forall), \mathsf{FOP}_{\mathbb{Z}}(\forall \forall \exists)$, or $\FOPZ(\exists \exists \forall)$ with $\epsilon>0$ would
refute the $3$-SUM hypothesis.
Furthermore, an $O(n^{2-\epsilon})$ time algorithm for problems in $\mathsf{FOP}_{\mathbb{Z}}(\forall \exists \exists)$, $\FOPZ(\exists \forall \forall)$, $\mathsf{FOP}_{\mathbb{Z}}(\exists \forall \exists)$, or $\FOPZ(\forall \exists \forall)$ with $\epsilon>0$ would
immediately yield an improvement for the MaxConv lower bound problem \cite{DBLP:journals/talg/CyganMWW19}; for details see the full version of the paper.}.




Since any $\PTO_{k,d}$ formula is also a $\FOPZ$ formula with the same quantifier structure, any hardness result in~\cite{DBLP:journals/algorithmica/AnGIJKN22} for $\PTO(Q_1, \dots, Q_k)$ carries over to $\FOPZ(Q_1,\dots, Q_k)$. On the other hand, any of our algorithmic results for $\FOPZ(Q_1,\dots, Q_k)$ transfers to its subclass $\PTO(Q_1,\dots, Q_k)$.



