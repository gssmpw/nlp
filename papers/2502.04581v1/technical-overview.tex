\section{Technical Overview}
\label{sec:TechnicalOverview}

In this section, we sketch the main ideas behind our proofs.
\paragraph*{Completeness of $k$-SUM for $\mathsf{FOP_{\mathbb{Z}}}(\exists^k)$}
With the right ingredients, proving that $k$-SUM is complete for $\FOPZ$ formulas with $k$ existential quantifiers (Theorem~\ref{existential_complete}) is possible via a simple approach: We observe that any $\FOPZ(\exists^k)$ formula $\phi$ can be rewritten such that we may assume that $\phi$  is a conjunction of $m$ inequalities. We then use a slight generalization of a bit-level trick of~\cite{DBLP:journals/siamcomp/WilliamsW13} to reduce each inequality to an equality, incurring only $O(\log n)$ overhead per inequality (intuitively, we need to guess the most significant bit position at which the left-hand side and the right-hand side differ).
Thus, we obtain $O(\log^m n)$ conjunctions of $m$ equalities; each such conjunction can be regarded as an instance of Vector $k$-SUM. Using a straightforward approach for reducing Vector $k$-SUM to $k$-SUM given in~\cite{DBLP:journals/corr/AbboudLW13}, the reduction to $k$-SUM follows. We give all details in Section~\ref{sec:existentialpa} and the full version of the paper.

 

\paragraph*{Counting witnesses and handling multisets}
While the reduction underlying Theorem \ref{existential_complete} preserves the existence of solutions, it fails 
to preserve the number of solutions. The challenge is that when applying the bit-level trick to reduce inequalities to equalities, we need to make sure that for each witness of a $\FOPZ(\exists^k)$ formula $\phi$, there is a unique witness in the $k$-SUM instances produced by the reduction. While it is straightforward to ensure that we do not produce multiple witnesses, the subtle issue arises that distinct witnesses for $\phi$ may be mapped to the same witness in the $k$-SUM instances. It turns out that it suffices to solve a \emph{multiset} version of \#$k$-SUM, i.e., to count all witnesses in a $k$-SUM instance in which each input number may occur multiple times. 

Thus, to obtain Theorem~\ref{thm:counting-witnesses}, we show a fine-grained equivalence of Multiset \#$k$-SUM and \#$k$-SUM, for all odd $k\ge 3$. This fine-grained equivalence, which we prove via a heavy-light approach, might be of independent interest.\footnote{We remark that it is plausible that the proof of the subquadratic equivalence of $3$-SUM and \#$3$-SUM due to Chan et al.~\cite{DBLP:journals/corr/abs-2303-14572} could be extended to establish subquadratic equivalence with Multiset \#$3$-SUM as well. Note, however, that a fine-grained equivalence of \#$k$-SUM and $k$-SUM is not known for any $k\ge 4$.}
Combining this equivalence with an inclusion-exclusion argument, we may thus lift Theorem~\ref{existential_complete} to a counting version for all odd $k \ge 3$.

In the reductions below, we will make crucial use of the immediate corollary of Theorem~\ref{thm:counting-witnesses} and~\cite{DBLP:journals/corr/abs-2303-14572} that for each $\FOPZ(\exists \exists \exists)$ formula $\phi$, there exists a subquadratic reduction from counting witnesses for $\phi$ to $3$-SUM (Corollary~\ref{Count3COMP}).




\paragraph*{On general quantifier structures}
We perform a systematic study on the different quantifier structures for $k=3$. 
Due to simple negation arguments, we only have to perform a systematic study on the classes of problems
$\mathsf{FOP}_{\mathbb{Z}}(\exists \exists \exists)$,
$\mathsf{FOP}_{\mathbb{Z}}(\forall \exists \exists)$,
$\mathsf{FOP}_{\mathbb{Z}}(\forall \forall \exists)$,
$\mathsf{FOP}_{\mathbb{Z}}(\exists \forall \exists)$.

First, we state a simple lemma establishing syntactic complete problems for the classes above.
\begin{lemma}[Syntactic Complete problems (Informal Version)]
Let $Q_1,Q_2 \in \{\exists,\forall\}$. We can reduce every formula of the class $\mathsf{FOP}_{\mathbb{Z}}(Q_1Q_2\exists)$ to the formula  
$$Q_1 \Tilde{a_1} \in \Tilde{A_1} Q_2 \Tilde{a_2}  \in \Tilde{A_2} \exists \Tilde{a_3}  \in \Tilde{A_3}: \Tilde{a_1} +\Tilde{a_2}  \leq \Tilde{a_3}.  $$
\end{lemma}

\paragraph*{On the quantifier change $\mathsf{FOP}_{\mathbb{Z}}(\forall \exists \exists) \to \mathsf{FOP}_{\mathbb{Z}}(\exists \exists \exists) $.}
We rely on the subquadratic equivalence between $3$-SUM and a functional version of $3$-SUM called All-ints $3$-SUM, which asks to determine 
for every $a \in A$  whether there is a solution involving $a$. A randomized subquadratic equivalence was given in~\cite{DBLP:conf/focs/WilliamsW10}, which can be turned deterministic~\cite{DBLP:conf/icalp/LincolnWWW16}.

This equivalence allows us to use the bit-level trick to turn inequalities to equalities, despite it seemingly not interacting well with the quantifier structure $\forall \exists \exists$ at first sight.
This results in a proof of the following hardness result.
\begin{restatable}{lemma}{allintshard}
	If $3$-SUM can be solved in time $O(n^{2-\epsilon})$ for an $\epsilon>0$,
	then all problems $P$ of $\mathsf{FOP}_{\mathbb{Z}}(\forall \exists \exists)$
	can be solved in time $O(n^{2-\epsilon_{P}})$ for an $\epsilon_{P}>0$. 
	\label{fopaee}
	\end{restatable}

\paragraph*{On the quantifier change $\mathsf{FOP}_{\mathbb{Z}}(\exists \exists \exists) \to \mathsf{FOP}_{\mathbb{Z}}(\forall \forall \exists) $.}
As a first result for the class $\mathsf{FOP}_{\mathbb{Z}}(\forall \forall \exists)$, 
we are able to show equivalence to $3$-SUM for a specific problem in this class,
thus introducing a $3$-SUM equivalent problem
with a different quantifier structure in comparison to $3$-SUM.
Specifically, we consider the problem of verifying additive $t$-approximation of sumsets.
We are able to precisely characterize the fine-grained complexity depending on $t$.

Formally, we show the following theorem.
\begin{restatable}{theorem}{sumsetapproxchar}
	Consider the Additive Sumset Approximation problem of deciding, given $A,B,C\subseteq \mathbb{Z}, t\in \mathbb{Z}$, whether
	$$A+B \subseteq C +\{0,\dots,t\}.$$
	This problem is
	\begin{itemize}
	\item solvable in time $O(n^{2-\delta})$ with $\delta>0$, whenever $t=O(n^{1-\epsilon})$ for any $\epsilon>0,$
	\item not solvable in time $O(n^{2-\epsilon})$, whenever $t =\Omega(n)$ assuming the Strong $3$-SUM hypothesis.
	\end{itemize}
	  Furthermore, subquadratic hardness holds under the standard 3-SUM Hypothesis if no restriction on $t$ is made.
	  \label{sumsetapproxTHM}
	\end{restatable}

The above theorem is essentially enabling a quantifier change transforming the $\exists \exists \exists$ quantifier structure for which $3$-SUM is complete 
into a subquadratic equivalent problem with a quantifier structure $\forall \forall \exists$.
Moreover, the $3$-SUM hardness is a witness to the hardness of the class $\mathsf{FOP}_{\mathbb{Z}}(\forall \forall \exists)$.

Let us remark a few interesting aspects: The algorithmic part follows from sparse convolution techniques
going back to Cole and Hariharan~\cite{DBLP:conf/stoc/ColeH02}, see~\cite{DBLP:journals/corr/abs-2107-07625} for a recent account and also \cite{DBLP:conf/stoc/ChanL15,DBLP:conf/icalp/BringmannN21,DBLP:conf/stoc/BringmannFN21}. Specifically, whenever $t = O(n^{1-\epsilon})$, it holds that $|C + \{0, \dots, t\}| = O(n^{2-\epsilon})$ and intuitively,
we can use an output-sensitive convolution algorithm to compute $A+B$ and compare it to $C+\{0, \dots, t\}$.\footnote{The argument is slightly more subtle, since we need to avoid computing $A+B$ if its size exceeds $O(n^{2-\epsilon})$.}
Our result indicates that an explicit construction of $C+\{0, \dots, t\}$ is required, since once it may get as large as $\Omega(n^2)$, we obtain a $n^{2-o(1)}$-time lower
bound assuming the Strong $3$-SUM Hypothesis.  

The lower bound follows from describing the $3$-SUM problem alternatively as $(A+B) \cap C \neq \emptyset$, which is equivalent to the negation of $(A+B)\subseteq \Bar{C}$, where $\Bar{C}$ denotes the complement of $C$.
Thus, we aim to cover the complement of $C$ by intervals of length $t$. While this appears impossible for $3$-SUM, we employ the subquadratic equivalence of $3$-SUM and its convolutional version due to Patrascu \cite{DBLP:conf/stoc/Patrascu10}. This problem will deliver us the necessary structure to represent this complement with the addition of few auxilliary points.


The reverse reduction from Additive Sumset Approximation to $3$-SUM follows from Theorem~\ref{three-sum-Completeness-all-quantifer} (as Additive Sumset Approximation has inequality dimension $2$).

\paragraph*{On completeness results for $\mathsf{FOP}_{\mathbb{Z}}^k$}
The above ingredients establish our completeness theorems by exhaustive search over remaining quantifiers. Specifically, by a combination of Theorem~\ref{sumsetapproxTHM}, which shows that Additive Sumset Approximation is $3$-SUM hard, and a combination of Lemma~\ref{fopaee} and Theorem~\ref{existential_complete},
we get:
\begin{restatable}[]{lemma}{verifhard}
There is a function $\epsilon(d)>0$ such that 
the Verification of Pareto Sum problem can be solved in time $O(n^{2-\epsilon(d)})$
if and only if all problems $P$ in the classes 
\begin{itemize}
\item $\mathsf{FOP}_{\mathbb{Z}}(Q_1\dots Q_{k-3}\exists \exists \exists)$,$\mathsf{FOP}_{\mathbb{Z}}(Q_1\dots Q_{k-3}\forall \forall \forall),$
\item $\mathsf{FOP}_{\mathbb{Z}}(Q_1\dots Q_{k-3}\forall \exists \exists)$,$\mathsf{FOP}_{\mathbb{Z}}(Q_1\dots Q_{k-3}\exists \forall \forall),$
\item $\mathsf{FOP}_{\mathbb{Z}}(Q_1\dots Q_{k-3}\forall \forall \exists)$,$\mathsf{FOP}_{\mathbb{Z}}(Q_1\dots Q_{k-3}\exists \exists \forall),$
\end{itemize}
where $Q_1, \dots Q_{k-3} \in \{ \exists, \forall \}$ and $k\geq 3$,
can be solved in time $O(n^{k-1-\epsilon_P})$ for an $\epsilon_P>0$.
\label{verif-complete-three}
\end{restatable}

Similarly, for quantifier structures ending in $\exists \forall \exists$ and $\forall \exists \forall$, we obtain the following completeness result.

\begin{restatable}[]{lemma}{hunthard}
	There is a function $\epsilon(d)>0$ such that 
	the Hausdorff Distance under $n$ Translations problem can be solved in time $O(n^{2-\epsilon(d)})$
	if and only if all problems $P$ in the classes
	\begin{itemize}
    \item $\mathsf{FOP}_{\mathbb{Z}}(Q_1\dots Q_{k-3}\exists \forall \exists), \mathsf{FOP}_{\mathbb{Z}}(Q_1\dots Q_{k-3}\forall \exists \forall),$
	\end{itemize}
	where $Q_1, \dots Q_{k-3} \in \{ \exists, \forall \}$ and $k\geq 3$,
	can be solved in time $O(n^{k-1-\epsilon_P})$ for an $\epsilon_P>0$.
	\label{hunthard}
	\end{restatable}

The combination of Lemma \ref{verif-complete-three} and Lemma \ref{hunthard}, thus suffice to 
prove Theorem \ref{completenesswholeFOP3}.
\paragraph*{The $3$-SUM completeness of formulas with inequality dimension at most $3$}

As a first idea, one could try to solve problems of different quantifier structures
by just counting witnesses. Consider in the following the example $\FOPZ(\forall \forall \exists)$. 

Assume we are promised that the formula $\forall a\in A \forall b\in B \exists c\in C \psi(a,b,c)$ 
satisfies a kind of \emph{disjointness} property, specifically that for every $(a,b) \in A \times B$ there exists at most one $c \in C$ such that $\psi(a,b,c)$. Then satisfying the formula boils down to checking whether the number of witnesses $(a,b,c)$ satisfiying $\psi(a,b,c)$ equals to $|A|\cdot |B|$.

To create this \emph{disjointness} effect, we use the following geometric approach: 
We show that one can re-interpret the formula as $\forall a\in A \forall b\in B: a+b\in \bigcup_{c'\in C'} V(c')$, where $A,B,C' \subseteq \mathbb{Z}^3$, $C'$ is a set of size $O(n)$ and $V(c')$ is an orthant associated to $c'$. Using an adapted variant of~\cite{DBLP:journals/dcg/ChewDEK99}, we decompose this union of orthants in $\mathbb{R}^3$ (which we may equivalently view as sufficiently large congruent cubes) into a set $\mathcal{R}$ of $O(n)$ \emph{disjoint} boxes. Thus, it remains to notice that the resulting problem  -- i.e., for all $a\in A, b\in B$ is there a box $R\in \mathcal{R}$ such that $a+b$ is contained in $R$ -- is a $\FOPZ(\forall \forall \exists)$ formula with the desired disjointness property, which can be handled as argued above.  
For the class $\mathsf{FOP}_{\mathbb{Z}}(\exists \forall \exists)$, we perform a slightly more involved argument.
The classes $\mathsf{FOP}_{\mathbb{Z}}(\exists \exists \exists)$ and $ \mathsf{FOP}_{\mathbb{Z}}(\forall \exists \exists)$
reduce to $3$-SUM regardless of the inequality dimension due to Theorem \ref{existential_complete} and Lemma \ref{fopaee}.











