\section{Preliminaries}\label{sec:prelim}
Let us first state problems and hypotheses we work with, starting with the $k$-SUM problem. 
\begin{definition}[The $k$-SUM problem]
Given $k$ finite sets of at most $n$ integers $A_1, A_2, \dots, A_k\subseteq \{-n^{k},\dots,0, \dots ,n^{k} \},$ and an integer $t$, determine if 
$\exists a_1 \in A_1 \exists a_2 \in A_2 \dots \exists a_k \in A_k: \sum_{i=1}^{k}a_i=t.$
\end{definition}

There is a simple meet-in-the-middle approach to solve the above problem in time $O(n^{\lceil k/2 \rceil+o(1)})$. It is widely
believed that this is optimal, as stated in the following hypothesis.
\begin{hypothesis}[The $k$-SUM Hypothesis]
Let $k\geq3$. There is no algorithm solving the $k$-SUM problem in time $O(n^{\lceil \frac{k}{2} \rceil -\epsilon})$ for any $\epsilon>0$.
\end{hypothesis}
For discussion of its plausibility, we refer to the survey \cite{williams2018some}.
We work in the standard word RAM model with words of $O(\log(n))$ bits. As we aim to relate runtime between quadratic time problems, we need 
a different notion of reduction than those used to prove NP completeness.

Fine-grained reductions were first introduced for subcubic runtimes in
\cite{DBLP:conf/focs/WilliamsW10}. A general definition can be found in \cite{williams2018some}.
In this paper, the following definition of fine-grained reduction, also used in \cite{DBLP:conf/approx/BringmannCFK21} suffices. A more detailed description on fine-grained reductions
can be found in \cite{DBLP:conf/innovations/CarmosinoGIMPS16}.

\begin{definition}[Fine-grained reductions]
Consider problems $P_1, P_2$ with presumed time complexities $T_1,T_2,$ respectively.
A fine-grained reduction from $P_1$ to $P_2$ is an algorithm $\mathcal{A}$ with oracle access to $P_2$. Whenever $\mathcal{A}$
uses an $O(T_2(n)^{1-\epsilon})$ algorithm for oracle calls to $P_2$ (for an $\epsilon>0$), there exists an $\epsilon'>0$,
such that $\mathcal{A}$ runs in time $O(T_1(n)^{1-\epsilon'})$. We write this as $(P_1,T_1) \leq (P_2,T_2).$
\end{definition}
We introduce for problems $P,Q$ the notation $P \leq_{c}Q$ if and only if $(P,n^c) \leq (Q,n^c)$ holds. Furthermore, $P$ and $Q$ are fine-grained equivalent, denoted by
$P \equiv_{c}Q$, if and only if $(P,n^c) \leq (Q,n^c)$ and $(Q,n^c) \leq (P,n^c) $ holds. We call $P_1,P_2$ subquadratic equivalent if $P_1 \equiv_2 P_2$.


A convolution of two vectors $x:=(x_0, \dots,x_{n-1})$ $y:=(y_0, \dots, y_{n-1})$
is defined as $z=(z_0, \dots,z_{2n-2})$, where $$z[k]:= \sum_{i,j \in [n-1], i+j=k} x_i \cdot y_j.$$ It is a well known fact that one can compute 
the sumset $A+B$ by computing the convolution of the characteristic vectors of $A$ and $B$.
In particular, we also make use of the theorem from Bringmann et al. \cite{DBLP:journals/corr/abs-2107-07625} on sparse convolutions.
\begin{theorem}
There is a deterministic algorithm to compute the convolution of two nonnegative vectors $A,B \in \mathbb{N}^n$ in time $O(t \text{polylog}(n \Delta)),$ where 
$t$ denotes the number of non-zero entries in the output convolution vector and $\Delta$ denotes the maximum entry size of the vectors $A,B$.
\label{sprase_conv}
\end{theorem}

The convolutional $3-$SUM problem asks for three sequences $A,B,C$ of size $n$, whether $\exists 0\leq i, j, k \leq n-1: a[i]+b[j]=c[k]$.
It is well known by a reduction from Pătrașcu \cite{DBLP:conf/stoc/Patrascu10} that there is no $O(n^{2-\epsilon})$ for an $\epsilon>0$ algorithm  for the Convolutional 3-SUM problem if and only if the 3-SUM hypothesis holds. 
The strong convolutional $3$-SUM hypothesis asks this question over a linearly sized universe \cite{DBLP:conf/icalp/AmirCLL14}, that is $A,B,C \subseteq \{-n,\dots,n\}.$ 


Let us fix the notation throughout the paper. We denote the sumset $U+V:= \{u+v: u \in U,v  \in V\}$. The $i$-th entry
of a vector $v$ is denoted by $v[i]$. For natural numbers $n$, we denote by $n[i]$ the $i$-th bit of $n$, where $n[0]$ is the least significant bit. Furthermore, we say the $i$-th bit of a natural number is set iff $n[i]=1.$
For vectors $u,v$ we write $u \leq v$ if and only if for all dimensions $i$ it holds that $u[i] \leq v[i]$. If $u \leq v$, we say $u$ is dominated by $v$.

For a unary function $f: U \to M$ and a set $A \subseteq U$ denote $f(A):= \{f(a): a\in A\}$. We abbreviate by $[t]:= \{0,\dots,t\}$. 
The notation $\Tilde{O}(T):=T \log^{O(1)}T$ is used
to hide poly-logarithmic factors. We denote the cardinality of a set $A$ by $\# A$ or $|A|$. Linear Integer arithmetic refers to the first order logic over the domain $\mathbb{Z}$ with
vocabulary: equality $(=)$,  inequality $(<,>,\geq,\leq)$,  and addition $(+)$. We use $\binom{V}{k}$ to denote all $k$-element subsets of $V$.

The $(M,d)$-vector $k$-SUM problem is defined as follows \cite{DBLP:journals/corr/AbboudLW13}.
For given $k$ sets $A_1 ,\dots ,A_k$ of size at most $n$ where each $A_i \subseteq \{-M,\dots, M\}^d$ and a target $t \in \{-M,\dots, M\}^d,$
do there exist $a_1 \in A_1, \dots, a_k \in A_k: a_1+\dots+ a_k=t$ 
Through the standard technique of interpreting vectors as integers we get:
\begin{lemma}
  The $(M,d)$-vector $k$-SUM problem can be reduced to the $k$-SUM problem with universe size $\{0,\dots, (kM+1)^d\}$ in time $O(n \log M).$
\label{trans_vec_to_k}
\end{lemma}
For a proof of the above see the proof of Abboud et al. \cite{DBLP:journals/corr/AbboudLW13} or our multiset adaptation of the proof in Lemma \ref{vector_to_one_red}.
A functional version of the $3$-SUM we will require is the following
\begin{definition}[All-ints $3$-SUM]
Given sets $A,B,C$ of at most $n$ integers $A,B,C \subseteq \{-n^k, \dots, n^k\}$ for each $a \in A$ determine,
whether there exist $b \in B$ and $c \in C$ such that $a+b+c=t$.
\end{definition}
\begin{lemma}[All-ints $3$-SUM $\equiv_2$ $3$-SUM \cite{DBLP:conf/focs/WilliamsW10}]
There exists a $O(n^{2-\epsilon})$ time algorithm for the All-ints $3$-SUM problem for an $\epsilon>0$
if and only if there exists a $O(n^{2-\epsilon'})$ time algorithm for $3$-SUM for an $\epsilon'>0$.
\label{allintsequiv}
\end{lemma}
For a proof see results from Williams et al. \cite{DBLP:conf/focs/WilliamsW10}.
It is known that the reduction can be made deterministic for instance using the $3$-SUM self reduction from Lincoln et al. \cite{DBLP:conf/icalp/LincolnWWW16} combined with the technique
introduced by Williams et al.\cite{DBLP:conf/focs/WilliamsW10}.
We continue with a $3$-SUM version, which aims to count witnesses.

  \begin{definition}[$\#3$-SUM]
    Given sets $A,B,C$ of at most $n$ integers $A,B,C \subseteq \{-n^k, \dots, n^k\}$.  The $\#3$-SUM problem asks for the number of triplets $(a,b,c) \in A \times B \times C$ such that $a+b+c=0$. \footnote{This version of $3$-SUM is equivalent to the version where $a+b+c=t$  for an integer $t$ is asked. In particular by setting $C':=C-\{t\}$, we get a reduction that preserves all solutions.}
    \end{definition}
  
  \begin{definition}[All-ints \#$3$-SUM]
    Given sets $A,B,C$ of at most $n$ integers $A,B,C \subseteq \{-n^k, \dots, n^k\}$. The All-ints $\#3$-SUM problem, asks to determine for each $a \in A$ the number of $(b,c) \in B \times C$ such that $a+b+c=0$.
  \end{definition}
    In our paper, we make use of the following recent powerful result from Chan et al. \cite{DBLP:journals/corr/abs-2303-14572}
    \begin{theorem}[\cite{DBLP:journals/corr/abs-2303-14572}]
    The following problems are all subquadratic equivalent
    \begin{itemize}
      \item $\#$All-ints $3$-SUM,
      \item $3$-SUM,
      \item $\#3$-SUM.
    \end{itemize}
    \label{Count3}
    \end{theorem}
As a remark, the non-trivial reductions rely on randomization techniques and to date it is not known whether it can be made deterministic.



    \begin{definition}[3-Uniform $k$-hyperclique problem]
      Given a $k$-partite 3-uniform hypergraph $G=(V,E)$, that is the vertices are a disjoint union of sets $V_1, V_2, \dots, V_k$ of size $n$ each, and $E$ is a set of edges
      of the form ${v_a,v_b,v_c}$ where $a,b,c$ are distinct and $v_a \in V_a, v_b \in V_b , v_c \in V_c$. The problem asks if there exists a $k-$Clique in $G$, that is vertices $v_1 \in V_1, \dots, v_k \in V_k$,
      such that for all $$a,b,c \in \binom{\{1,\dots,k\}}{3}$$ there exists an edge $\{v_a, v_b,v_c \}.$
      \end{definition}
      There is a naive algorithm deciding the above problem in $O(n^k)$. It is strongly believed
      that this runtime is optimal. One reason is that matrix multiplication techniques that speed up clique detections in graphs (rather than hypergraphs) 
      seem to fail, see for instance \cite{DBLP:conf/soda/LincolnWW18}. Furthermore, a faster algorithm would lead to an exponential improvement 
      over current $2^{n-o(n)}$ algorithms for MAX 3-SAT, see \cite{DBLP:conf/sat/ChenS15,DBLP:conf/focs/AlmanCW16}.
      Thus, we work with the following hypothesis.
      \begin{hypothesis}[3-Uniform Hyperclique Hypothesis]
      There is no $O(n^{k-\epsilon})$ algorithm solving the 3-Uniform $k$-hyperclique problem for $k\geq 4$ and an $\epsilon >0$. 
      \label{hyperclique_conj}
      \end{hypothesis}
    

      In \cite {DBLP:journals/talg/CyganMWW19}, Cygan et al. studied the MaxConv lower bound problem, but were unable to give a nontrivial upperbound.
      They managed to only show a reduction from this problem to the $L_p$ necklace alignment problem.
      The problem is defined as follows:
      \begin{definition}[MaxConv lower bound]
      \label{def:maxconvlb}
      Given integer arrays $A,B,C$ of length $n$. Determine whether 
      $C[k] \leq \max_{i+j=k} (A[i]+B[j])$ holds.
      \end{definition}


A key in our proofs will be a slightly generalized version of a lemma, whose aim it is to reduce inequality checking to a logarithmic amount of equality checks \cite{DBLP:journals/siamcomp/WilliamsW13}. 
\begin{lemma}[Bit-trick]
  For any non-negative integers $x_1,x_2, \dots, x_k, z \in \{0,\dots,U\},$ we have the following equivalence:
  \begin{align*}
      x_1+ \dots +x_k >z \iff & \text{There are } \ell \in \{1,\dots, [\lceil \log_2(U) \rceil] \}, b\in \{1,2, \dots, k\}:\\
      & pre_\ell(x_1)+\dots +pre_\ell(x_k)=pre_\ell(z)+b,
  \end{align*}
  where $pre_\ell(x)$ denotes the number remaining when taking the first $l$ bits of $x$, where the most significant bit is considered the first bit. Formally, for $z \in \{0,\dots,U\}$ and $\ell \in \{0,\dots, \lceil \log_2(U) \rceil\}$ $pre_\ell(z):=\lfloor \frac{z}{2^{B-\ell}} \rfloor,$ where $B$ denotes the number of bits in $z$.
  \label{bit_trick}
  \end{lemma}
  \begin{proof}
    If  $pre_\ell(x_1)+pre_\ell(x_2)+ \dots +pre_\ell(x_k)=pre_\ell(z)+b$, for $B$-bit integers holds, we have
    \begin{align*}
       & \sum_{i=B-\ell}^{B-1}2^{i}x_{1}[i] + \dots +\sum_{i=B-l}^{B-1}2^{i}x_{k}[i]  = \sum_{i=B-\ell}^{B-1}2^{i}z[i]+2^{B-l}b \\
        \implies &\sum_{i=B-\ell}^{B-1}2^{i}(x_1[i]+x_2[i] +\dots + x_k[i]) =\sum_{i=B-\ell}^{B-1}2^{i}z[i]+2^{B-l}b\\
        \implies & \sum _{i=1}^k x_i=z+ \sum_{i=0}^{B-\ell-1}(x_1[i]+x_2[i] +\dots + x_k[i]-z[i])+2^{B-\ell}b\\
        \implies &\sum _{i=1}^k x_i >z.
    \end{align*}
    Assume now that $x_1+x_2 + \dots + x_k >z$ holds, let $$ \hat{\ell} :=\min\{i:1\leq i\leq B \land  pre_i(x_1)+pre_i(x_2)+\dots +pre_i(x_k)>pre_i(z)\}. $$ If $b \leq k$, the statement holds, for the sake of a contradiction assume $$pre_{\hat{\ell}}(x_1)+ pre_{\hat{\ell} }(x_2)+ \dots + pre_{\hat{\ell} }(x_k) \geq pre_{\hat{\ell} }(z)+k+1. $$
    We know that the following holds for any $B$-bit integer $x$ and any $1 \leq l\leq B$
    $$2pre_{\ell-1}(x) \leq pre_\ell(x) \leq 2pre_{\ell-1}(x)+1. $$
    Thus, we have
    \begin{align*}
        2\left(pre_{\hat{\ell} -1}(x_1)+ \dots + pre_{\hat{\ell} -1}(x_k)   \right) +k & \geq  pre_{\hat{\ell} }(x_1)+ \dots + pre_{\hat{\ell}}(x_k) \\
        & \geq pre_{\hat{\ell} } (z) +k +1\\
        & \geq 2pre_{\hat{\ell} -1}(z)+k+1 .
        \end{align*}
    Concluding $$ pre_{\hat{\ell} -1}(x_1)+ pre_{\hat{\ell} -1}(x_2)+ \dots + pre_{\hat{\ell} -1}(x_k)  \geq pre_{\hat{\ell} -1}(z) +1/2,$$ and thereby also
    $$ pre_{\hat{\ell} -1}(x_1)+ pre_{\hat{\ell} -1}(x_2)+ \dots + pre_{\hat{\ell} -1}(x_k)   > pre_{\hat{\ell} -1}(z), $$ contradicting the minimality of $\hat{\ell} $.
    \end{proof}
We conclude this section with an observation.
\begin{observation}
 The choices of $\ell$ and $b$ in Lemma \ref{bit_trick} are not unique, but there is a unique $\ell \in \{1,\dots, \lceil \log_2(U) \rceil \}$ and $b \in \{1,\dots,k\}$ such that the following holds
 \begin{align*}
  \sum_{i=1}^{k}x_i >z \iff \sum_{i=1}^{k}pre_\ell(x_i)=pre_l(z)+b \land \sum_{i=1}^{k}pre_{\ell-1}(x_i)=pre_{\ell-1}(z).
 \end{align*}
 \label{uniqueness_bit}
\end{observation}
