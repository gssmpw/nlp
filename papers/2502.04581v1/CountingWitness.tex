\section{On counting witnesses in $\mathsf{FOP}_{\mathbb{Z}}$}\label{sec:countwitn}
  In this section, we show reductions from counting witnesses of $\FOPZ(\exists^k)$ formulas to $\#k$-SUM, specifically, we prove Theorem~\ref{thm:counting-witnesses}. To do so, we adapt the proof of Theorem~\ref{existential_complete} given in Section~\ref{sec:existentialpa} to a counting version. As discussed in Section~\ref{sec:TechnicalOverview}, this requires us to work with a multiset version of \#$k$-SUM. Handling multisets is thus the main challenge addressed in this section.
   Formally, we say that a multiset is a set $A$ together with a function $f:A\to \mathbb{N}$. For $a \in A$, we abbreviate $n_a:= f(a)$ as the multiplicity
   of $a$. To measure multiset sizes, we still think of each $a$ to have $n_a$ copies in the input, i.e. the size of $A$ is $\sum_{a \in A}n_a$.
   Almost all proofs in this section are deferred to the full version of the paper.
   \begin{definition}[$(U,d)$-vector Multiset $\#k$-SUM]
   Let $ X:=\{-U,\dots,U\}^d$. Given $k$ multisets $A_1,\dots,A_k \subseteq X$ and $t \in X$, we ask for the total number of $k$-SUM witnesses, that is 
   \begin{align*}
   \sum_{\substack{a_1+\dots +a_k=t, \\ a_1 \in A_1, \dots, a_k \in A_k}}\prod_{i=1}^k n_{a_i}.
   \end{align*}
   \end{definition}
   Furthermore, define Multiset $\#k$-SUM as $(U,1)$-vector Multiset $\# k$-SUM and $M$-multiplicity $\# k$-SUM as Multiset $\#k$-SUM with the additional restriction that the multiplicity of each element is limited, that is for all $a \in A_1 \cup \dots \cup A_k: n_a \leq M$ holds. 
   Lastly, $\# k$-SUM is defined as $1$-Multiplicity $\# k$-SUM and $(U,d)$-vector $\#k$-SUM is $(U,d)$-vector Multiset $\#k$-SUM where for all $a \in A_1 \cup \dots \cup A_k: n_a = 1$ holds.

   For the case of $\FOPZ^3$ we will also introduce the $\#$All-ints version of the above problems, which asks to determine, for each $a_1 \in A_1$, the number of witnesses involving $a_1$.
   

The (deferred) proof of the following lemma is analogous to the proof of Abboud et al.~\cite{DBLP:journals/corr/AbboudLW13} to reduce Vector $k$-SUM to $k$-SUM.
\begin{lemma}[$(U,d)$-vector Multiset $\#k$-SUM $\leq_{\lceil k/2 \rceil }$ Multiset $\#k$-SUM]
If Multiset $\#k$-SUM can be solved in time~$T(n)$ then $(U,d)$-vector Multiset $\#k$-SUM can be solved in time $O(nd \log(U)+T(n)).$
\label{vector_to_one_red}
\end{lemma}

Next, we give a simple approach to solve Multiset $\# k$-SUM when all multiplicities are comparably small.

\begin{lemma}[$M$-multiplicity $\# k$-SUM$\leq_{\lceil k/2 \rceil}$ $\# k$-SUM]
  If $\#k$-SUM can be solved in time~$T(n)$, then $M$-multiplicity $\#k$-SUM can be solved in time $\Tilde{O}(T(nM^{k-1}))$.
\label{reduction_t}
\end{lemma}

For later purposes, we will need the following version of the above lemma.
\begin{observation}
If $\#$All-ints $3$-SUM can be solved in time $T(n)$,
then we can solve $\#$All-ints $M$-multiplicity $3$-SUM in time $\Tilde{O}(T(nM^{2})).$
\label{All-ints-count}
\end{observation}

We can finally prove the main result of this section. 

\begin{lemma}
    For odd $k\geq 3$, if there exists an algorithm for the $\#k$-SUM problem running in time $O(n^{\lceil k/2 \rceil-\epsilon})$ for an $\epsilon>0$, then there exists an algorithm for the Multiset $\#$k-SUM problem running in time $O(n^{\lceil k/2 \rceil-\epsilon'})$ for an $\epsilon'>0$.
  \label{multisetToSet}  
  \end{lemma}
    \begin{proof}
    We proceed with a heavy-light approach.
    Assume there exists an $O(n^{\lceil k/2 \rceil -\epsilon})$ algorithm for the $ \#k$-SUM problem.
    Set $c:=(k-1)(\lceil k/2 \rceil)$.
    Firstly, we count the number of solutions $(a_1,\dots,a_k) \in A_1 \times \dots \times A_k$, where $n_{a_1},\dots, n_{a_k}\leq n^{ \epsilon/c}$ using  Lemma \ref{reduction_t}. This takes time
    \begin{align*}
	    \Tilde{O}\left( (n \cdot (n^{ \epsilon/c})^{k-1})^{\lceil k/2 \rceil -\epsilon} \right)&=\Tilde{O} \left(\left(n^{1+\frac{\epsilon}{\lceil k/2 \rceil }} \right )^{ \lceil k/2 \rceil -\epsilon}\right) \\
                                                              &= \Tilde{O} \left(n^{ {\lceil k/2 \rceil -\epsilon} + \epsilon  -\frac{ \epsilon ^2}{\lceil k/2 \rceil } }  \right) \\
                                                              &= O \left(n^ {\lceil k/2 \rceil -\epsilon'} \right),
    \end{align*}
    where $\epsilon'>0$.
    It remains to calculate the number of witnesses $(a_1,\dots,a_k)$, where for at least one $i \in \{1,\dots,k\}$, we have high multiplicity, meaning $n_{a_i}>n^{\epsilon/c}$ holds.
    Consider the case that $a_1 \in A_1$ is a high-multiplicity number (the case where $a_i \in A_i$ with $i\neq 1$  is a high-multiplicity number is analogous).
    For each high-multiplicity number $a_1$ in $A_1$ we do the following.
    Solve the $(k-1)$-SUM instance with sets $A_2,\dots ,A_k$ and target $t-a_1$.
    There are at most $n^{1-(\epsilon/c)}$ many high-multiplicity numbers in $A_1$, and solving the $(k-1)$-SUM instance takes time $O(n^{(k-1)/2})$, since $k$ is odd.
    We get a total runtime of 
    \begin{align*}
      n^{1-\frac{\epsilon}{c}} \cdot \Tilde{O}(n^{(k-1)/2})&=\Tilde{O}(n^{1-(\epsilon/c) +(k-1)/2})\\
                    &=\Tilde{O}(n^{(k+1)/2 -(\epsilon/c)})\\
                    &=O(n^{\lceil k/2 \rceil -\epsilon''}),
    \end{align*}
    where $\epsilon''>0$, which concludes the proof.
    \end{proof}



    \countComplete*


By combining the subquadratic equivalence between $3$-SUM and $\#3$-SUM due to Chan et al.~\cite{DBLP:journals/corr/abs-2303-14572} and the above theorem, we obtain the following corollary.
\threecountcomplete*

The above proof can also be adapted for the special case $k=3$ to count for each $a_1 \in A_1$ the number of witnesses involving $a_1$, by plugging in the appropriate All-ints versions; see the full version of the paper for details.
Together with the equivalence between $\#$All-ints $3$-SUM and $3$-SUM of Chan et al. \cite{DBLP:journals/corr/abs-2303-14572}, we get
\begin{corollary}
For all problems $P$ in $\mathsf{FOP}_{\mathbb{Z}}(\exists^3)$, we are able to count for each $a_1 \in A_1$ the number of witnesses involving $a_1$ in randomized time $O(n^{2-\epsilon})$ for an $\epsilon>0$,
if $3$-SUM can be solved in randomized time $O(n^{2-\epsilon'})$ for an $\epsilon'>0$.
\label{count-all-ints}
\end{corollary}


