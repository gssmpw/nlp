\section{Examples of problems in $\mathsf{FOP}_{\mathbb{Z}}$} \label{sec:Examples}
In the following, we give a plethora of examples of problems in $\mathsf{FOP}_{\mathbb{Z}}$.
\begin{example}[3-Average free set]
Consider a finite set of integers $A$. Is there no arithmetic progression of length $3$? 
We can use the negation of the following sentence to obtain an answer.

	\[ \exists a_1 \in A \exists  a_2 \in A \exists a_3 \in A: a_1 <a_2<a_3 \land a_1 +a_3=2a_2.\]
\end{example}
We use the condition $a_1 <a_2 <a_3$ as to avoid $a_1=a_2=a_3$.
The 3-average free set problem is known to be subquadratic equivalent to $3$-SUM~\cite{DBLP:conf/stoc/0001GS20}, which itself is another example.
\begin{example}[3-SUM and Conv-3SUM]
Consider a finite set of integers $A$. Are there three numbers in $A$ summing up to $0$? I.e., 
	\[ \exists a_1 \in A \exists  a_2 \in A \exists a_3 \in A: a_1 +a_2+a_3 = 0.\]
	We can also express its subquadratic equivalent formulation convolutional $3$-SUM\footnote{For further discussion, see Section~\ref{sec:prelim}.} as a two-dimensional formula:
	Representing a  sequence $X[0 \dots n-1]$ as $X'=\{ (i,X[i]) \mid 0 \le i < n\}$, we ask whether given sequences $A[0\dots n-1],B[0 \dots n-1],C[0 \dots n-1]$ satisfy:
	\[ \exists (i,a[i])\in A' \; \exists  (j,b[j]) \in B' \; \exists (k,c[k]) \in C': a[i]+b[j]=c[k] \land i+j=k\]
\end{example}

More generally, we can view these problems as natural database queries with numerical data, e.g.: 
\begin{example}[Orthogonal Range and Database queries]
    Consider a database $\mathcal{D}$, where each entity consists of a tuple $(id,age,income)$. 
    The following query is part of the class $\mathsf{FOP}_{\mathbb{Z}}(\exists^3 )$.\\
    \emph{Do there exist three different people in $\mathcal{D}$, whose average age is below 30, whose income is in the range $[10000,20000]$, and
    whose incomes accumulate to more than $50000$?}
\end{example}

All the above examples use existentially quantified variables. This existential fragment will be of particular interest for us.

Other quantifier structures also give rise to natural algorithmic problems. 

\begin{example}[Universal $3$-SUM]
Given sets $A,B,C \subseteq \mathbb{Z}$. Does for all $a \in A $ exist $b \in B$ and $c \in C$ such that 
$c=a+b$. Clearly the problem is in $\FOPZ(\forall \exists \exists)$. Alternatively, one can also view the problem as a sumset expression, namely
$C \subseteq A+B$.
\end{example}
We note that Universal $3$-SUM seems to be a weaker version of All-ints $3$-SUM and that in general the class $\FOPZ(\forall \exists \exists)$ seems
to be the most likely to admit a subquadratic algorithm, as it has the weakest known lower bound barrier.

\begin{example}[Verification of $\Delta$-approximations of sumsets]\label{ex:addApx}
Bringmann and Nakos introduce a notion of additive approximation of sumsets in \cite{DBLP:conf/soda/BringmannN21}. We consider the problem of verifying whether a set additively approximates a sumset.  
	Specifically, let $A+B:=\{a +b: a \in A, b \in B\}$ denote the sumset of $A$ and $B$. We say that a set $C$ is an additive $\Delta$-approximation\footnote{We remark that we use here a simplified notion of additive approximation that is closely related to the ones used by Bringmann et al.~\cite{DBLP:conf/soda/BringmannN21}.} of $A+B$ whenever %for is used in the context of sumsets $A+B:=\{a +b: a \in A, b \in B\}.$

	\[
\forall a \in A \forall b  \in B\exists c \in C: c \leq a+b \leq c+\Delta \iff A+B \subseteq C+\{0,\dots,\Delta\}.
	\]
Note that we could also require each number in $C$ to be a sum itself by taking the conjunction with the Universal $3$-SUM formula:
	\[
		\forall c \in C \exists a \in A \exists b \in B: a+b=c \iff  C \subseteq A+B.\\
	\]
\end{example}
The above example makes use of the free variables in the definition of the problem $\mathsf{FOP}_{\mathbb{Z}}(\phi)$, in this case the additive approximation constant $\Delta$ is a free variable and can be instantiated by an input natural number.
Interestingly, the conjunction of both conditions $A+B \subseteq C+\{0,\dots,\Delta\}$ and $C \subseteq A+B$ can be reduced to $3$-SUM,
as $A+B \subseteq C+\{0,\dots,\Delta\}$ is in $\FOPZ(\forall \forall \exists)$ and has inequality dimension 2, and $C\subseteq A+B$ is in $\FOPZ(\forall \exists \exists)$.
\begin{example}[Hausdorff Distance under $n$ Translations]\label{ex:hunt}
We recall the definition of the Hausdorff Distance under $n$ Translations problem over $d$-dimensional sets $A,B,C$, and a value $\gamma \in \mathbb{N}$, where we ask whether the following holds
 \[\delta_{\overrightarrow{H}}^{T(A)}(B,C) = \min_{\tau \in A} \max_{b\in B} \min_{c\in C} \|b-(c+\tau)\|_{\infty} \leq \gamma.\]


Clearly, $\|b-(c+\tau)\|_{\infty} \leq \gamma$ if and only if for all dimensions $i \in \{1,\dots d\}$, we have $b[i] -(c[i]+\tau[i]) \leq \gamma$ and $(c[i]+\tau[i])-b[i]\leq \gamma. $
Thus, it remains to check 

\begin{align*}
\exists \tau \in A \forall b \in B \exists c \in C: \bigwedge_{i=1}^{d} b[i] -(c[i]+\tau[i]) \leq \gamma \land (c[i]+\tau[i])-b[i]\leq \gamma.
\end{align*}
\end{example}
\begin{example}[Triangle Detection]
\label{example:triangle}
Let $E$ be the set of edges in a directed graph. Each edge has an id and consider $\alpha,\omega$ to be functions that denote the start and endpoint of an edge respectively.
Thus $E'=\{(e_{id},\alpha(e),\omega(e)):e \in E\}$.
To detect a triangle we can simply ask 
\begin{align*}
\exists e_1 \in E \exists e_2 \in E \exists e_3 \in E:& e_1[0] \neq e_2[0] \land e_2[0] \neq e_3[0]\land e_1[0] \neq e_3[0] \land \\
&e_1[2]=e_2[1] \land e_2[2]=e_3[1] \land e_3[2]=e_1[1].
\end{align*}
Thus a $\Tilde{O}(n^{1+\epsilon})$ time algorithm for $3$-SUM implies a $\Tilde{O}(m^{1+\epsilon})$
time algorithm for triangle detection. 
\end{example}

Let us now take a look at the MaxConv lower bound problem (see Definition \ref{def:maxconvlb}).
\begin{lemma}
    The MaxConv lower bound problem is a member of the class $\mathsf{FOP}_{\mathbb{Z}}(\exists \forall \exists)$,
    and a member of the class $\mathsf{FOP}_{\mathbb{Z}}(\forall \exists \exists)$.
    \label{maxconvlb}
  \end{lemma}
  \begin{proof}
    \begin{enumerate}
  \item 
  Let 
  \begin{align*}
  &A'=\{ (A[i],i): i \in [n-1] \} \\
  &B'=\{(B[j],j): j\in [n-1] \} \cup \{(-M, -j): j \in \{1,\dots,n-1\} \}\\
  &C'=\{(C[k],k): k \in [n-1] \},
  \end{align*}
  where $M=3\cdot \max{(A\cup B \cup C)}.$
  We ask 
  $$\exists c' \in C' \forall a' \in A' \exists b' \in B': (i+j=k \land C'[k]>A'[i]+B'[j]).$$
  Thus, we can formulate MaxConv lower bound in $\mathsf{FOP}_{\mathbb{Z}}(\exists \forall \exists).$
  
  We give a short proof to this equivalence in the following. We have that $C[k]$ is a non witness to MaxConv lower bound if and only if  $\forall i \in [n-1]: C[k]>A[i]+B[k-i]$.
  Now, let us make a case distinction on $i$.
  Consider the case $i\leq k$, then only $j=k-i \in [n-1]$ fulfills $i+j=k$, and due to the fact 
  that $c[k]$ is a non witness $C[k]>A[i]+B[j]$.
  For the case $i>k$,  we have $j$ negative, which trivially fulfills $C[k]>A[i]+B[j]$.
  
  For the other direction notice that there exists a $k \in [n-1]$ which fulfills for all indices $i+j=k$ that $C[k]>A[i]+B[j]$.
  
  \item For the following constructed sets:
  \begin{align*}
  &A'=\{ (A[i],i): i \in [n-1] \}\\
  &B'=\{(B[j],j): j\in [n-1] \}\\
  &C'=\{(C[k],k): k \in [n-1] \} \cup \{(-M, k): k \in \{n,...,2n\} \},
  \end{align*}
  where $M=3\cdot \max A\cup B \cup C .$
  It can be easily seen that the following formula in $\mathsf{FOP}_{\mathbb{Z}}(\forall \exists \exists)$ models the MaxConv lower bound problem, 
  $$
  \forall c' \in C' \exists a \in A' \exists b' \in B': i+j=k \land C[k]\leq A[i]+B[j].
  $$
  \end{enumerate}
  \end{proof}
  The above Lemma presents a witness to the hardness of the classes $\mathsf{FOP}_{\mathbb{Z}}(\forall \exists \exists)$ and 
  $\mathsf{FOP}_{\mathbb{Z}}(\exists \forall \exists)$.
  
