%\appendix 
\section{Baseline Algorithms} \label{sec:baseline}
\begin{lemma}
Every problem in the class $ \mathsf{FOP}_{\mathbb{Z}}(\exists^k)$ can be decided in time $\tilde{O}(n^{\lceil k/2 \rceil})$.
\label{baseline_exist}
\end{lemma}
\begin{proof}
    To this end, after substituting the free variables $\hat{t_1}, \dots \hat{t_l}$ in the quantifier free part of $\phi$, without loss 
    of generality, by the same arguments used in the proof of Theorem \ref{existential_complete}, we can transform every such linear arithmetic formula $\phi[(t_1,\dots,t_l)\backslash(\hat{t_1},\dots,\hat{t_l})]$ into the following form:
    $$\exists a_1 \in A_1 \dots \exists a_k \in A_k:\bigvee_{h=1}^{H}\bigwedge_{i=1}^{m} \sum_{j=1}^{k}c_{h,i,j}^T  a_j \geq S_{h,i}.$$
    Due to the commutativity of disjunction and existential quantifiers, we can rewrite the above in the form:
    $$ \bigvee_{h=1}^{H}\exists a_1 \in A_1  \dots \exists a_k \in A_k:\bigwedge_{i=1}^{m} \sum_{j=1}^{k}c_{h,i,j}^T  a_j \geq S_{h,i}.$$
    Now it suffices to have a look at $H$ instances of the following problem, whose results are combined disjunctively
    $$\exists a_1 \in A_1  \dots \exists a_k \in A_k:\bigwedge_{i=1}^{m} \sum_{j=1}^{k}c_{i,j}^T  a_j \geq S_i,$$
    where $S_i \in \mathbb{Z},c_{i,j} \in \mathbb{Z}^{d_j}.$
    Consider summing up the tuples by their corresponding coefficients, that is we consider the sets 
\begin{align*}
  A_j^{'}:=\left\{ \left(\begin{array}{c}
  c_{1,j}^T a  \\
  c_{2,j}^T a \\
  \vdots  \\
  c_{m,j}^T a
  \end{array}\right)   : a \in A_j \right\},
\end{align*}
and define $S:=(S_1,S_2,\dots,S_m)^T$.
We are now left with the following problem
\begin{align*}
\exists a_1' \in A_1' \dots \exists a_k' \in A_k': \sum_{j=1}^{\lceil k/2 \rceil }a_j' \geq \sum_{j=\lceil k/2 \rceil +1}^{ k } a_j' + S
\end{align*}
Precompute the left-hand side sums and the right-hand side sums in $O(n^{\lceil k/2 \rceil })$. Store all possible sums in the left-hand side in a range tree, see \cite{DBLP:conf/focs/AlstrupBR00}.  Iterating over the precomputed sums in the right-hand side and querying them in the range tree gives us an algorithm in time $\Tilde{O}(n^{\lceil k/2 \rceil })$.
\end{proof}

\begin{lemma}
Every formula $\phi$ in one of the classes $$\FOPZ(\exists \exists),\FOPZ(\forall \forall),\FOPZ(\exists \forall ),\FOPZ(\forall \exists),$$
can be decided in time $\Tilde{O}(n)$.  \label{baseline:twoquant}
\end{lemma}
\begin{proof}
If $\phi$ is in the class $\FOPZ(\exists \exists)$, we can conclude by the same algorithm as in Lemma \ref{baseline_exist}.
If $\phi$ is in the class $\FOPZ(\forall \exists)$, after substituting the free variables, by the same arguments used in the proof of Theorem \ref{existential_complete}, we can transform $\phi$ to an equivalent formula of the type:
$$\forall a_1 \in A_1 \exists a_2 \in A_2:\bigvee_{h=1}^{H}\bigwedge_{i=1}^{m} c_{h,i,1}^T  a_1+ c_{h,i,2}^T a_2 \geq S_{h,i},$$
which is equivalent to 
$$\forall a_1 \in A_1 \bigvee_{h=1}^{H} \exists a_2 \in A_2:\bigwedge_{i=1}^{m} c_{h,i,1}^T  a_1+ c_{h,i,2}^T a_2 \geq S_{h,i}.$$
Consider 
\begin{align*}
  A_2^{(h)}:=\left\{ \left(\begin{array}{c}
  c_{h,1,2}^T a  \\
  c_{h,2,2}^T a \\
  \vdots  \\
  c_{h,m,2}^T a
  \end{array}\right)   : a \in A_2 \right\},
\end{align*}
$S^{(h)}:=(S_{h,1},\dots,S_{h,m})^T$, and finally $a_1^{(h)}:=(c_{h,1,1}^T a_1,c_{h,2,1}^T a_1, c_{h,m,1}^T a_1)$.
Then, it remains to solve:
$$\forall a_1 \in A_1 \bigvee_{h=1}^{H} \exists a_2 \in A_2^{(h)}: a_2 \geq S^{(h)} -a_1^{(h)}.$$

Thus, we insert in $H$ different orthogonal range trees, the elements of $A_2^{(h)}$.

By iterating over all the elements in $A_1$, and checking whether for any $h \in \{1,\dots,H\}$, the vector $S^{(h)}-a_1^{(h)}$ is dominated by some $a_2 \in A_2^{(h)}$ via a range tree query,
we can conclude. 

Finally, by simply negating the formula and pushing the negation inwards, we can always get a formula in $\FOPZ(\exists \exists)$ or $\FOPZ(\forall \exists)$.
\end{proof}
\begin{lemma}
  Every problem in the class $\mathsf{FOP}_{\mathbb{Z}}^k$ can be decided in time $\Tilde{O}(n^{k-1})$.
\label{baseline} 
\end{lemma}
  \begin{proof}
  Let $\phi$ be in $\mathsf{FOP}_{\mathbb{Z}}$.
  After substituting the free variables, we brute force the first $k-2$ quantifiers. It remains to solve a formula 
 $\varphi$ with 2 quantifiers. We can conclude by a simple application of Lemma \ref{baseline:twoquant}
\end{proof}


%\section{Continuation of Preliminaries} \label{appencontinuation}

\section{Proof of Lemma \ref{Ineq_to_eq}}\label{lemma_conj_ineq}
\begin{proof}
    We define the transformations for $f^{\ell,\psi}_j$ with $j \in \{1,\dots,k\}$ and the transformation $g^{\ell,W,\psi}$  as the following vectors with $2m$ dimensions:
     \begin{flalign*}
     \underbrace{ \left(\begin{array}{c}
       pre_{\ell[1]}(M+p_{1,j}( a_j))  \\
       \vdots  \\
       pre_{\ell[m]}(M+p_{m,j}( a_j))\\
       pre_{\ell[1]-1}(M+p_{1,j}( a_j)) \\
       \vdots \\
       pre_{\ell[m]-1}(M+p_{m,j}( a_j))
       \end{array}\right)}_{=:f^{\ell,\psi}_j(a_j)} \quad\quad       \underbrace{ \left(\begin{array}{c}
        pre_{\ell[1]}\left(S_1-1+ kM \right)+W[1] \\
        \vdots  \\
        pre_{\ell[m]}(S_m-1+ kM)+W[m]\\
        pre_{\ell[1]-1}\left(S_1-1+ kM \right) \\
        \vdots  \\
        pre_{\ell[m]-1}(S_m-1+ kM)
        \end{array}\right).}_{=:g^{\ell,W,\psi}(S_1,\dots,S_m)}
     \end{flalign*}

     Notice that $pre_\ell$ is applied on non-negative integers due to the choice of $M.$
     For all $\ell \in \{1,\dots,\lceil \log_2(M) \rceil \}^m, W \in \{1,\dots,k\}^m$, we have that 
     \begin{align*}
       & f^{\ell,\psi}_1 (a_1)+\dots + f^{\ell,\psi}_k (a_k)=g^{\ell,\psi,W} (S_1,\dots,S_m) \\
       \iff &  \underbrace{\bigwedge_{i=1}^m \sum_{j=1}^{k} pre_{\ell[i]}(M+p_{i,j}( a_j))= pre_{\ell[i]}\left(S_i-1+ kM \right)+W[i]}_{=:\varphi_1(\ell,W)} \\
       \land & \underbrace{\bigwedge_{i=1}^m \sum_{j=1}^{k} pre_{\ell[i]-1}(M+p_{i,j}( a_j))= pre_{\ell[i]-1}\left(S_i-1+ kM \right)}_{=:\varphi_2(\ell,W)}.
     \end{align*}
     By Lemma \ref{bit_trick} (also Observation \ref{uniqueness_bit}), there exist unique $ \ell' \in \{1,\dots,\lceil \log_2(M) \rceil \}^m, W' \in \{1,\dots,k\}^m$ such that 
     \begin{align*}
       \varphi_1(\ell',W') \land \varphi_2(\ell',W')\iff \bigwedge_{i=1}^m\sum_{j=1}^{k} M+p_{i,j}(a_j) >S_i -1 + kM,
     \end{align*}
     which is equivalent to the desired conjunction of inequalities.
     The function of the dimensions $m+1,\dots,2m,$ and in particular $\varphi_2(\ell',W')$ are to ensure the uniqueness for the choices of $b$ and $\ell$ in Lemma \ref{bit_trick}.
   \end{proof}

\section{Proof of Disjoint boxes Lemma} \label{sec:disjointboxes}

%\begin{definition}
  %We call a collection of interior disjoint rectangles lonely if and only if each rectangle has at most one neighbor in each direction.
 % \end{definition}
  %\begin{lemma}
  %The union of $n$ overlapping squares can be decomposed into a lonely decomposition of $O(n)$ rectangles in time $O(n \log n)$.
  %\label{lemma:decomp}
  %\end{lemma}
  \begin{lemma}
	  We can decompose a rectilinear shape of complexity $n$ in $O(n)$ interior- and exterior-disjoint rectangles, some of which may be degenerate, in time $O(n \log n)$.
  \label{lem:2d}
  \end{lemma}
  \begin{proof}
  At the beginning, we aim to decompose the rectilinear shape into interior-disjoint rectangles.
  For this purpose, we perform a sweep-line algorithm. The sweep-line goes through the $x$-direction. 
  Firstly, we maintain a collection of components given by a segment in $y$ direction, which will intuitively denote the opening of components(or rectangles) which are not overlapping.
  Each vertex event, will now either create a new component, will enlarge an existing component, will shrink an existing component, or will make it disjoint. By shooting a ray up and down from this event point vertex
  we can create a new rectangle, together with the current existing component. The vertices intersected by this ray, will form the starting points for the new component.
  For details see the figure below. As each vertex will create at most one new rectangle, we conclude that the number of rectangles is in $O(n)$.
  \begin{figure}[H]
  \includegraphics[width=\textwidth, height=6cm]{decomposition.pdf}
  \caption{To the left we can find the rectilinear shape. To the right we find the decomposition into interior disjoint rectangles.}
  \end{figure}
  The rectangles formed by the above procedure are kept as open $2$-boxes.
  We now turn to the face segments of these rectangles. If a vertex event point splits a segment, we 
  decompose the segment at this point. Clearly at most $O(n)$ segments (1-boxes) can be created in this way.
  The $1$-boxes we will keep as open, and the vertices as closed $0$-boxes (or points).
\end{proof}



\begin{lemma}
	The union of $n$ unit cubes can be decomposed into $O(n)$ interior- and exterior-disjoint axis-aligned boxes, some of which may be degenerate, in time $O(n \log^2 n)$.
\end{lemma}
\begin{proof}
We adapt the algorithm of \cite{DBLP:journals/dcg/ChewDEK99} to also create exterior-disjoint boxes.
Slice the three-dimensional space by planes parallel to the $z$ axis for planes $z=1, z=2, z=3 \dots$ (without loss of generality, we can assume the cubes to be starting at $z=1$).
Consider the slab of cubes generated by cutting the cubes by  the plane $z=i$ and $z=i+1$, where we 
denote by $n_i$ the number of cubes cut by the plane $z=i$. The complexity of the slab is in $O(n_i+n_{i+1})$ by \cite{DBLP:journals/dcg/BoissonnatSTY98}.

We repeat the following for each slab.

Let $E$ be the portion of cubes that lie within the slab bounded by $z=i$ and $z=i+1$, and let the 
silhouette of $E$ be the projection on both $z=i$ and $z=i+1$ of all vertical lines whose intersection with
	$E$ is one unit long -- we refer to this vertical length as \emph{height}.

We firstly construct a decomposition of the silhouette of $E$ into interior- and exterior-disjoint boxes.
To achieve this we perform a decomposition as given in Lemma \ref{lem:2d} of the projection of the silhouette onto the plane $z=i$.
After achieving this decomposition, we create $3$-boxes out of the rectangles, with height 1, which we keep open.
Out of the segments, we create open $2$-boxes with height $1$, and from the vertices we create open $1$-boxes of height $1$.

Let $S$ be the projection of the silhouette of $E$ onto the plane $z=i$, and $S'$ be the projection of the silhouette of $E$ onto the plane $z=i+1$.
	Moreover, let $E'$ be the intersection of $E$ with the plane $z=i$, and let $F'$ denote the points on $E'$ whose vertical line length (i.e., height) changes.
	Computing $F'$ (with the associated heights) costs time $O((n_i+n_{i+1}) \log^2 (n_i+ n_{i+1}))$ by computing the union of the cubes in the slab by \cite{AgarwalS21}.


Consider $F'$ without $S$, where $S$ is the union of the formed rectangle decomposition of the silhouette of $E$.
We perform the Lemma \ref{lem:2d} decomposition onto this rectilinear shape, with the restriction that the rays shot up and down do not cross 
	$S$. Create the open $3$-boxes just like in the step for the silhouette, with the difference being that we use the height of each produced rectangle (which by construction is uniform over the rectangle).
We ignore the segments and vertices which lie adjacent to $S$ (they have height $1$ and are already taken care of).
For the remaining segments and vertices, we create open $2$-boxes and closed $1$-boxes respectively, where we take the height of the larger adjacent box. Finally, in contrast to the silhouette we also create open $2$-boxes for the faces, which lie on top of the created open $3$-boxes (or rather the face lying on the height of the $3$-box).

Proceed analogously for $E''$ which is the intersection of $E$ onto the plane $z=i+1$ and $F''$, the points on $E''$ whose vertical line length changes.

Finally, it remains to treat the rectilinear shapes $E'$ and $E''$ separately, as none of the boxes include the faces on the plane $z=i$ and $z=i+1$.
For this again, we perform the Lemma \ref{lem:2d} decomposition for $E'$ and $E''$ respectively, where we keep the rectangles and segments, as open $2$ and $1$-boxes respectively, and the vertices 
as closed $0$-boxes which lie on the respective plane.
\end{proof}


  %\section{Proofs for section 6}\label{Appendix6}
  
  %\paragraph*{Proof of Theorem \ref{HCsumset}}
  %\begin{proof}
   % On a high level the search space $[t]$ will denote all potential cliques. The reduction will transform a YES instance of $3$-uniform $6$-hyperclique into a NO instance of $[t]\subseteq A+B$.  
    %We consider the graph to be $6$-partite with node sets $V_0, \dots, V_5.$ In each of $V_0, \dots, V_5$,
    %we add two dummy vertices. The dummy nodes are not connected by an edge to any other node.
    %We number the vertices in each partition $V_0, \dots ,V_5$ from $0$ to $|V|+1$, where node $0$ and node $1$ are the dummy nodes. 
    %Consider $N :=|V|+2$, and each clique $C$ to be a number in basis $N$, for $x_{i} \in \{0,\dots,|V|+1\}$ that means:
    %$$ C:= \sum_{i=0}^{5} x_{i}\cdot N^i.$$
    %Consider $$A=\left\{\sum_{i \in I} v_i \cdot N^i: I \in \binom{[5]}{3}, \{v_i: i \in I\} \not \in E  \right\},$$
    %which is the set of numbers describing non-edges.
   % Furthermore, consider 
    %$$ B:= \left\{\sum_{i \in I}x_i\cdot N^{i} : I \in \binom{[5]}{3}, x_i \in \left[ |V|+1 \right]\right\} .$$
    %The size of $A$ is due to the size of the edges in $\Theta(n^3)$. The size of $B$ is also clearly in $\Theta(n^3)$.
    %Set $t:=N^6-1$. 
    %For the correctness let $C \in [t]$, which represents a potential clique $(c_0,\dots,c_5)$, with $c_i \in V_i$.
    %We claim that $C\in A+B$ if and only if $(c_0,\dots,c_5)$ does not form a clique.
    %Assume $C \in [t]$ represents $(c_0,\dots,c_5)$ that do not form a clique, then $C$ contains a missing edge represented by an $a \in A$ and there also exists $b \in B$ such that $C=a+b$ by construction.

    %Assume now there exists $a=\sum_{i=0}^{5} a_{i}\cdot N^i \in A, b=\sum_{i=0}^{5} b_{i}\cdot N^i \in B$ and $C= \sum_{i=0}^{5} c_{i}\cdot N^i \in [t] $ such that $a+b=C$ for some $a_i,b_i \in \{0,\dots,|V|+1\}$.
    %If there was an overflow when adding $a$ and $b$, that is $a_i+b_i>N-1$ for an $i \in [5]$. Then one of the following cases happens, there is a $j \in [5]$ with $c_j=0$, or we have some $k \in [5]$ with $a_k+b_k>N-1$ and $c_{k+1}=1$.
    %As these cases consist of at least one dummy node, $C$ is not a clique. 
    %Assume now there was no overflow when adding, that is for all $i \in [5]$ we have $a_i+b_i \leq N-1$, then there exist disjoint $I_1,I_2 \subseteq [5]$, with $I_1 \cup I_2=[5].$
    %Clearly, $\{v_i: i \in I_1\}$ is a non-edge and thereby $C$ is not a clique.
    %Assuming, there was an algorithm solving $[t] \subseteq A+B$ in time $O(n^{2-\epsilon})$ for an $\epsilon>0$, then we would get an algorithm in time $O((n^3)^{2-\epsilon})=O(n^{6-\epsilon'})$ for an $\epsilon'>0$, for the 
    %$3$-uniform $6$-hyperclique problem refuting the hypothesis.
    %\end{proof}

   % \begin{lemma}[$3$-SUM $\leq_2$ Verification of $2$-dimensional Pareto Sum]
    %  If Verification of $2D$ Pareto SUM can be solved in time $O(n^{2-\epsilon})$ for an $\epsilon>0$, then we can solve $3$-SUM in time $O(n^{2-\epsilon'})$
    %  for an $\epsilon'>0$.
    %  \end{lemma}
    %\begin{proof}
    %We reduce from the problem 
    %$$ A+B\subseteq C +[t],$$ which we have shown to be $3$-SUM hard. We define the sets $$A':= \left \{ \left(
    % \begin{array}{cc}
    %    a \\
    %    -a \\
    %    \end{array}  \right)
    %    :a \in A  \right \}.$$
   % 
    %    $$B':= \left \{ \left(
     %\begin{array}{cc}
      %  b \\
       % -b \\
        %\end{array}  \right) : b \in B
        % \right \}.$$
    
        %$$C':= \left \{ \left(
         %   \begin{array}{cc}
          %     c+t \\
           %    -c \\
            %   \end{array}  \right) : c \in C
             %   \right \}.$$
    %The quantifier structures match, and we conclude correctness by rewriting $c \leq a+b \leq c+t \iff a+b \leq c+t \land -a-b\leq -c $.
    %\end{proof}

    %\section{Continuation of Count chapter}
    


