In the last three decades, the $k$-SUM hypothesis has emerged as a satisfying explanation of long-standing time barriers for a variety of algorithmic problems. Yet to this day, the literature knows of only few proven consequences of a refutation of this hypothesis. Taking a descriptive complexity viewpoint, we ask: What is the largest logically defined class of problems \emph{captured} by the $k$-SUM problem?

	To this end, we introduce a class $\mathsf{FOP}_{\mathbb{Z}}$ of problems corresponding to deciding sentences in Presburger arithmetic/linear integer arithmetic over finite subsets of integers.
	We establish two large fragments for which the $k$-SUM problem is complete under fine-grained reductions:
	\begin{enumerate}
		\item The $k$-SUM problem is complete for deciding the sentences with $k$ existential quantifiers.
		\item The $3$-SUM problem is complete for all $3$-quantifier sentences of $\mathsf{FOP}_{\mathbb{Z}}$ expressible using at most $3$ linear inequalities.
	\end{enumerate}
	Specifically, a faster-than-$n^{\lceil k/2 \rceil \pm o(1)}$ algorithm for $k$-SUM (or faster-than-$n^{2 \pm o(1)}$ algorithm for $3$-SUM, respectively) directly translate to polynomial speedups of a general algorithm for \emph{all} sentences in the respective fragment.
	
	Observing a barrier for proving completeness of $3$-SUM for the entire class $\FOPZ$, we turn to the question which other -- seemingly more general -- problems are complete for $\FOPZ$. In this direction, we establish $\FOPZ$-completeness of the \emph{problem pair} of Pareto Sum Verification and Hausdorff Distance under $n$ Translations under the $L_\infty$/$L_1$ norm in $\mathbb{Z}^d$. In particular, our results invite to investigate Pareto Sum Verification as a high-dimensional generalization of 3-SUM.

	