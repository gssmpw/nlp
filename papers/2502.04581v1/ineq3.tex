\section{The $3$-SUM problem is complete for $\mathsf{FOP}_{\mathbb{Z}}$ formulas with Inequality Dimension at most 3}\label{sec:IneqDimension}

In this section, we show that $3$-SUM problem captures an interesting subclass of $\FOPZ$ formulas with arbitrary quantifier structure, namely the formulas of sufficiently small \emph{inequality dimension}.  Let us recall the notion of inequality dimension.
\begin{definition}[Inequality Dimension of a Formula]
  Let $\phi = Q_1 x_1\in A_1, \dots, Q_k x_k\in A_k: \psi$ be a $\FOPZ$ formula with $A_i\subseteq \mathbb{Z}^{d_i}$.
  
  The \emph{inequality dimension} of $\phi$ is the smallest number $s$ such that there exists a Boolean function $\psi' :\{0,1\}^s \to \{0,1\}$ and (strict or non-strict)
  linear inequalities $L_1, \dots, L_s$ in the variables $\{x_i[j] : i\in \{1,\dots,k\} ,j\in \{1,\dots,d_i\} \}$ and 
  the free variables such that $\psi(x_1,\dots, x_k)$ is equivalent to $\psi'(L_1,\dots, L_s)$.  
\end{definition}
In the following, we look at the class of problems $\mathsf{FOP}_{\mathbb{Z}}^k$ with the restriction of 
inequality dimension at most $3$.
We use the following naming convention for boxes.
\begin{definition}
  A $d$-box in $\mathbb{R}^d$ is the cartesian product of $d$ proper intervals $s_1 \times \dots \times s_d$,
  where $s_i$ is an open, closed or half-open interval. We call a cartesian product of only closed intervals a closed box and
  a cartesian product of only open intervals an open box.
  \end{definition}
  Given a set $R$ of $n$ closed boxes (represented as $2d$ integer coordinates), and $d$-dimensional points $a \in A ,b \in B$, we can express in $\FOPZ(\exists \exists \exists)$ whether $a+b$ lies in one of the boxes as follows:
  $$ \exists a \in A \exists b \in B \exists r \in R: \bigwedge_{i=1}^{d} r[i] \leq a[i]+b[i] \land a[i]+b[i]  \leq r[d+i]. $$ 
  In fact, we are not limited to closed boxes, if a box is open or half open in a dimension, one can adjust the inequalities in this dimension appropriately.

In order to prove our main theorem in this section, we need to partition the union of $n$ unit cubes in $\mathbb{R}^3$ into pairwise interior- and exterior-disjoint boxes.
While Chew et al.~\cite{DBLP:journals/dcg/ChewDEK99} studied such a decomposition of unit cubes  with the requirement of only interior-disjoint boxes, we need an extension of their result to guarantee disjoint exteriors.

\begin{lemma}[Disjoint decomposition of the union of cubes in $\mathbb{R}^3$]
Let $\mathcal{C}$ be a set of $n$ axis-aligned congruent cubes in $\mathbb{R}^3$. The union of these cubes,
can be decomposed into $O(n)$ boxes whose interiors and exteriors are disjoint in time $O(n \log^2 n)$.
\label{Cubes_in_space}
\end{lemma}
The proof is deferred to the full version.
\begin{theorem}
  There is an algorithm deciding $3$-SUM in randomized time $O(n^{2-\epsilon})$ for an $\epsilon>0$ if and only if
  for each problem $P$ in the classes $\mathsf{FOP}_{\mathbb{Z}}(\forall \forall \exists )$ and $\mathsf{FOP}_{\mathbb{Z}}(\exists \forall \exists )$ of inequality dimension at most $3$ there exists some $\epsilon'>0$ such that we can solve $P$ in randomized time $O(n^{2-\epsilon'})$.
  \label{ineq3}
\end{theorem}
\begin{proof}
  For the first direction due to Theorem \ref{sumsetapproxTHM}, we can reduce $3$-SUM to an instance of Additive Sumset Approximation,
  $$ \forall a \in A \forall b \in B \exists c \in C: c \leq a+b \land a+b \leq c+t,$$
  which has inequality dimension 2. Let us continue with the other direction.
  Let $\phi:=Q_1 a \in A \forall b \in B \exists c \in C: \varphi$, where $Q_1 \in \{\exists,\forall \}$ and $\varphi$ is a quantifier free linear arithmetic formula with inequality dimension $3$.
  Let $L_1:=\alpha_{1}^T a + \beta_{1}^T b \leq \gamma_{1}^T c +S_{1}$,
   $L_2:= \alpha_{2}^T a + \beta_{2}^T b \leq \gamma_{2}^T c +S_{2}$ and
   $L_3:= \alpha_{3}^T a + \beta_{3}^T b \leq \gamma_{3}^T c +S_{3}$ after replacing the free variables.
  Assume that the formula $\varphi$ is given in DNF, thus each co-clause
  has at most $3$ atoms, chosen from $L_1,L_2,L_3$ and their negations. 
  Let 
  \begin{align*}
   A':= \left \{ \left( \begin{array}{cc}
    \alpha_{1}^T a \\
    \alpha_{2}^T a \\
    \alpha_{3}^T a
    \end{array}  \right): a \in A \right \},
    B':=\left \{ \left( \begin{array}{cc}
      \beta_{1}^T b \\
      \beta_{2}^T b \\
      \beta_{3}^T b 
      \end{array}  \right):b \in B \right \},
      C':=\left \{ \left( \begin{array}{cc}
        \gamma_{1}^T c +S_{1} \\
        \gamma_{2}^T c +S_{2}\\
        \gamma_{3}^T c+ S_{3}
        \end{array}  \right):c \in C \right \}
  \end{align*}
  Thus each co-clause consists of conjunctions of a subset of the following set
  \begin{align*}
    \{ & a'[0]+b'[0] \leq c'[0] ,a'[0]+b'[0] \geq c'[0]+1, 
                    a'[1]+b'[1] \leq c'[1], \\
                   & a'[1]+b'[1] \geq  c'[1]+1, 
                    a'[2]+b'[2] \leq c'[2], a'[2] +b'[2] \geq c'[2] +1
    \}.
  \end{align*}
  Let the co-clauses of $\varphi$ be $V_1, \dots ,V_h$. Thus, we aim to decide a formula of the form:
  \begin{equation}\label{eq:Dnfcurr}
  Q_1 a' \in A' \forall b' \in B' \exists c' \in C': \bigvee_{i=1}^{h} V_{i} 
  \end{equation} 
  For each co-clause $V_i$, $i \in \{1,\dots,h\} $ it holds that $V_i$ is of the form
  $$ \bigwedge_{k \in V_i^K} L_k  \land \bigwedge_{j \in V_i^J} \lnot L_j,$$
  for some $V_i^J,V_i^K \subseteq \{1,2,3\}$ and $V_i^J\cap V_i^K =\emptyset$. 

  Let us consider for each fixed $c' \in C'$ the following possibly empty orthant in $\mathbb{R}^3$.
  $$\mathcal{S}(V_{i},c'):=\{x \in \mathbb{R}^3 : \bigwedge_{k \in V_i^K}x[k] \leq c'[k] \land \bigwedge_{j \in V_i^J}x[j] \geq c'[j]+1   \}.$$

  By construction, it is immediate that for a fixed $c'$  and $(a',b') \in A' \times B'$ that $(a',b',c')$ fulfill the co-clause $V_i $ if and only if $a'+b' \in \mathcal{S}(V_{i},c')$.
  Thus, equivalently to \eqref{eq:Dnfcurr}, we ask
	\[ Q_1 a' \in A' \forall b' \in B' \exists c' \in C': \bigvee_{i=1}^{h} \left( a'+b' \in S(V_i,c') \right).\]
  Having a closer look, $\bigvee_{i=1}^{h} \left( a'+b' \in S(V_i,c') \right)$ is true if and only if $a'+b'$ lies in one of the orthants $S(V_i,c')$.

	We argue that we may represent the orthant $ \mathcal{S}(V_{i},c')$ as an appropriately chosen cube in $\mathbb{R}^3$. To this end, let $M:=2 \cdot \max \{\|a \|_1+ \|b \|_1+ \|c \|_1: a' \in A', b' \in B', c' \in C'\}$ be a sufficiently large number. We can interpret $\mathcal{S}(V_{i},c')$ as a cube of the type $\mathcal{C}_{i,c'} = [m_0,m'_0] \times [m_1,m_1'] \times [m_2,m'_2]$,
  where for $u \in \{0,1,2\}$, we define:
  \begin{align*}
  m_u:=\begin{cases} 
    -M & u \not \in V_i^K , u \not \in V_i^J,\\
    -2M+c[u] & u \in V_i^K, \\
    c[u]+1 & u \in V_i^J,
 \end{cases} \quad   m_{u}' := \begin{cases} 
  M & u \not \in V_i^K , u \not \in V_i^J,\\
  c[u] & u \in V_i^K, \\
  2M+c[u]+1 & u \in V_i^J. 
\end{cases}
\end{align*}
The cubes are axis-aligned and have side length $2M$. Due to the large size of the cube we get for fixed $c' \in C'$ that
$a'+b' \in \mathcal{S}(V_i,c')$ if and only if $a'+b'$ lies inside the cube $\mathcal{C}_{i,c'}$.

By Lemma \ref{Cubes_in_space}, we can decompose the collection of cubes $\mathcal{C}_{i,c'}$ for $i \in \{1,\dots,H \}, c' \in C'$ into $l = O(n)$ disjoint boxes $\mathcal{R}:=\{R_1, \dots, R_l\}$ in time $O( n \log^2 n)$.
Let us now go through a case distinction based on the first quantifier. 
  \begin{itemize}
  \item If $Q_1=\forall$, equivalent to $\phi$ we ask 
  $$\forall a' \in A' \forall b' \in B' \exists i \in \{1,\dots,l\}: a'+b' \text{ lies in } R_i.$$
  
		  By replacing each $i \in \{1,\dots,l\}$ by a 6-tuple denoting the dimensions of the box $R_i$, we can reduce counting the number of $(a',b',R_i)$ with $a'+b' \in R_i$ to 3-SUM using Corollary~\ref{Count3COMP}.
		  Due to the disjointness of the boxes $R_i$, we know that no $(a',b')$ can be in different boxes $R_i, R_{i'}$ with $i\ne i'$.

  Thus, we can decide our original question by checking whether the number of such witnesses equals $|A'| \cdot |B'|$, concluding the fine-grained reduction to 3-SUM.


  \item Assume now that $Q_1=\exists$. Thus, equivalently to $\phi$, we ask.
  $$\exists a' \in A' \forall b' \in B' \exists i \in \{1,\dots,l\}: a'+b' \text{ lies in } R_i.$$

		  We can now make use of Corollary~\ref{count-all-ints}. Count for each $a' \in A'$ the number of \emph{witnesses} $(a',b',R_i)$ with $a'+b'\in R'$. We claim that it remains to check
whether there is some $a'$ that is involved in $|B'|$ witnesses. 
		  To see this, note that due to the disjointness of the $R_{i}$'s, for any $a'\in A'$ we have that the number of $(b',R_i)$ with $a'+b'\in R_i$ is equal to the number of $b'$ such that there exists $R_i$ with $a'+b'\in R_i$. Again, the desired reduction to 3-SUM follows. \qedhere
  \end{itemize}  
\end{proof}

We remark that, by \cite{DBLP:journals/dcg/BoissonnatSTY98}, we know that the complexity of the union of orthants in $\mathbb{R}^d$ has worst case complexity $O(n^{\lfloor d/2 \rfloor})$.
Thus, the above proof does not seem directly generalizable to inequality dimensions larger than 3.
We can extend Theorem \ref{ineq3} to $k$-quantifiers by the following theorem.


\threesumineq*

  The above theorem gives us immediate reductions to $3$-SUM for many seemingly unrelated problems of different 
  quantifier structures and semantics.

  For instance, as a direct application of the above theorem we can conclude the equivalence of the 
  Additive Sumset Approximation problem to $3$-SUM, together with Theorem \ref{sumsetapproxTHM}.

  \begin{lemma}[Additive Sumset Approximation $\leq_2$ $3$-SUM]
    If the $3$-SUM problem can be solved in randomized time $O(n^{2-\epsilon})$ for an $\epsilon>0$
    then Additive Sumset Approximation problem can be solved in randomized time 
    $O(n^{2-\epsilon'})$ for an $\epsilon'>0$.
    \end{lemma}
