\section{Completeness Theorems for General Quantifier Structures}\label{sec:GeneralQuantifier}
As Theorem \ref{existential_complete} establishes $3$-SUM as the complete problem for the class $\mathsf{FOP}_{\mathbb{Z}}(\exists \exists \exists)$,
we would like to similarly explore complete problems for other quantifier structures. All proofs in this section are deferred to the full version.
Let us recall our main geometric problems.
    \begin{definition}[Verification of $d$-dimensional Pareto Sum]
        Given sets $A,B,C \subseteq \mathbb{Z}^d$. Does the set $C$ dominate $A+B$, that is does for all $a \in A ,b \in B$ exist a $c \in C$, with $c \geq a+b$ ?
      \end{definition}
It is easy to see that Verification of $d$-dimensional Pareto Sum is in $\mathsf{FOP}_{\mathbb{Z}}(\forall \forall \exists)$.

\begin{definition}[Hausdorff Distance under $n$ Translations]
  Given sets $A,B,C \subseteq \mathbb{Z}^d$ with at most $n$ elements and a $\gamma\in \mathbb{N}$, the Hausdorff distance under $n$ Translations problem asks whether the following holds:
   \[\delta_{\overrightarrow{H}}^{T(A)}(B,C) \coloneqq \min_{\tau \in A} \delta_{\overrightarrow{H}}(B,C+\{\tau\}) = \min_{\tau \in A} \max_{b\in B} \min_{c\in C} \|b-(c+\tau)\|_{\infty}\leq \gamma.\]
  \end{definition}

We show the following result firstly, which allows us to assume without loss of generality a certain normal form.
\begin{lemma}
A general $\FOPZ(Q_1Q_2\exists)$ formula, with input set $A_1 \subseteq \mathbb{Z}^{d_1}, A_2 \subseteq \mathbb{Z}^{d_2},A_3 \subseteq \mathbb{Z}^{d_3} $, where $|A_1|=|A_2|=|A_3|=n$,
can be reduced to the $\FOPZ(Q_1Q_2\exists)$ formula
$$Q_1 a_1'\in A_1' Q_2 a_2' \in A_2' \exists a_3' \in A_3':a_1'+a_2' \leq a_3'$$ in time $O(n)$, where
$|A_1'|=|A_2'|=n$ and $|A_3'|=O(n)$. 
\label{normalform}
\end{lemma}
The above lemma immediately gives us complete syntactic problems for our classes.
It remains to establish connections between the different quantifier structure classes, and explore natural variants of the syntactic problems.

The syntactic complete problem for the class $\mathsf{FOP}_{\mathbb{Z}}(\exists \forall \exists)$ turns out to be equivalent to Hausdorff Distance under $n$ Translations. We obtain:

\begin{restatable}[Hausdorff Distance under $n$ Translations is complete for $\mathsf{FOP}_{\mathbb{Z}}(\exists \forall \exists)$]{lemma}{hausdorffcompl}
	There is a function $\epsilon(d)>0$ such that Hausdorff Distance under $n$ Translations can be solved in time $O(n^{2-\epsilon(d)})$ if and only if all problems $P$ in 
	$\mathsf{FOP}_{\mathbb{Z}}(\exists \forall \exists)$ can be solved in time $O(n^{2-\epsilon_P})$ for an $\epsilon_P>0$.
	\label{Hausdorff-Completeness}
\end{restatable}
Similarly, the Verification of Pareto Sum problem is complete for the class $\mathsf{FOP}_{\mathbb{Z}}(\forall \forall \exists)$.
\begin{restatable}[Verification of Pareto Sum is complete for $\mathsf{FOP}_{\mathbb{Z}}(\forall \forall \exists)$]{lemma}{verifcompl}
	There is a function $\epsilon(d)>0$ such that Verification of Pareto Sum can be solved in time $O(n^{2-\epsilon(d)})$ if and only if all problems $P$ in 
	$\mathsf{FOP}_{\mathbb{Z}}(\forall \forall \exists)$ can be solved in time $O(n^{2-\epsilon_P})$ for an $\epsilon_P>0$.
	\label{Verif-Completeness}
  \end{restatable}


\subsection{$\mathsf{FOP}_{\mathbb{Z}}(\forall \exists \exists) \to \mathsf{FOP}_{\mathbb{Z}}(\exists \exists \exists)$ }

We continue with handling the class $\mathsf{FOP}_{\mathbb{Z}}(\forall \exists \exists)$.
By simply making use of Corollary \ref{count-all-ints}, one can easily prove that $3$-SUM is hard for the class $\mathsf{FOP}_{\mathbb{Z}}(\forall \exists \exists)$.
We can also show a deterministic proof, as Corollary \ref{count-all-ints} makes use of 
the subquadratic equivalence between $3$-SUM and $\#$All-ints $3$-SUM, which relies on randomization techniques.

\allintshard*

%We remark a small observation, which will be important to us.
%\begin{observation}
 % The above proof can be generalized so that we can count the number of witnesses for each $a \in A$,
 % by making a call to $\#$All ints $3$-SUM instead of All ints $3$-SUM. This holds due to the preservation of the witnesses.
 % (Does this need an equivalence from all ints $3$-SUM auf sets und all ints $3$-SUM auf multisets ?)
%\end{observation}


 





\subsection{$\mathsf{FOP}_{\mathbb{Z}}(\exists \exists \exists) \to \mathsf{FOP}_{\mathbb{Z}}(\forall \forall \exists)$}
We explore the connection between the problem Additive Sumset Approximation, which is a member of the class 
$\mathsf{FOP}_{\mathbb{Z}}(\forall \forall \exists)$, and the $3$-SUM problem.
The following theorem will play a key role to enable the discovery of the relationship between $3$-SUM and other quantifier structures.
\sumsetapproxchar*
The proof is deferred to the full version of the paper.
\subsection{Completeness results for the class $\mathsf{FOP}_{\mathbb{Z}}^k$}
We turn to combining the above insights to establish (a pair of) complete problems for the class $\mathsf{FOP}_{\mathbb{Z}}$. The proofs in this section are deferred to the full version of the paper.


\verifhard*

\hunthard*
We finally obtain  our completeness theorem for the whole class $\mathsf{FOP}_{\mathbb{Z}}^k$.
\completenessclass*

Essentially, these two problems capture the complexity of the class $\mathsf{FOP}_{\mathbb{Z}}^3$ and can be seen as the most important problems in $\mathsf{FOP}_{\mathbb{Z}}^{k}$. 
%\begin{definition}[All-ints $\exists \forall$ domination]
%Given $A,B,C \subseteq \mathbb{Z}^d$ determine for each $a \in A$ whether there exists $b \in B$ such 
%that for all $c \in C$ the inequality $a+b \leq c$ holds.
%\end{definition}

%\dominancehard*
%\begin{proof}
%We bruteforce the first $k-3$ quantifiers.
%By application of Lemma \ref{normalform}, it remains to solve a formula $\phi:=Q_1 a_1 \in A_1 Q_2 a_2 \in A_2 Q_3 a_3 \in A_3: a_1 + a_2 \leq a_3$.
%By a possible negation, we are able to achieve one of the following four cases.
%\begin{enumerate}
%\item If $Q_1=Q_2\forall$ and $Q_3=\exists$, then assuming we can solve All-ints $\forall \exists$ domination in time $O(n^{2-\epsilon_d})$,
%we can conclude by checking for each $a_1$ whether the rest of the formula is satisfied .
%\item If $Q_1=Q_3\exists$ and $Q_2=\forall$, then assuming we can solve All-ints $\forall \exists$ domination in time $O(n^{2-\epsilon_d})$,
%we can conclude by checking whether for atleast one $a_1 \in A_1$ the rest of the formula is satisfied.
%\item If $Q_1=Q_2=Q_3=\exists$ or $Q_1=\forall$ and $Q_2=Q_3=\exists$, we reduce $\phi$ to a $3$-SUM instance firstly, and then to an instance of the additive aproximation problem.
%Thus, we again end up in Case 1.
%\end{enumerate}
%\end{proof}



