% This is samplepaper.tex, a sample chapter demonstrating the
% LLNCS macro package for Springer Computer Science proceedings;
% Version 2.20 of 2017/10/04
%
\documentclass[runningheads]{llncs}
%
\usepackage{url}
\usepackage{graphicx}
 \usepackage{nicematrix}

%\usepackage[small]{caption}
\usepackage{booktabs}
\urlstyle{same}
\usepackage{multirow}
\usepackage{color, colortbl}

\usepackage{pbox}
\usepackage{mathrsfs}
\usepackage[lined, ruled, linesnumbered, noend, scleft, nofillcomment]{algorithm2e}
\SetKwRepeat{Do}{do}{while}
\usepackage{amsmath,amsfonts}
\usepackage{algorithmic}
\usepackage{graphicx}
\usepackage{textcomp}
\newcommand*\rot{\rotatebox{90}}

\let\Bbbk\relax
\usepackage{newtxmath}

%-------------------
\usepackage{caption}
\usepackage[utf8]{inputenc}
\usepackage{url}
\usepackage{amsmath}
\DeclareMathOperator*{\argmax}{arg\,max}
\DeclareMathOperator*{\argmin}{arg\,min}
\usepackage{siunitx}
\usepackage{svg}
\usepackage{parcolumns} % đặt trước \begin{document}

\newcommand\I{\ensuremath{\mathcal{I}}}
\newcommand\A{\ensuremath{\mathcal{A}}}
\newcommand\T{\ensuremath{\mathcal{T}}}
\newcommand\N{\ensuremath{\mathcal{N}}}
\newcommand\ClT{\ensuremath{\mathcal{C}l}_\mathcal{T}}
\newcommand\Cl{\ensuremath{\mathcal{C}l}}
\newcommand\IRCC{\ensuremath{\mathcal{S}}}
%\newcommand\OELB{\mathcal{O}_{\mathcal{EL}_\bot}}
\newcommand\OELB{\mathcal{O}}
\newcommand\Onto{\ensuremath{\mathcal{O}}}
\newcommand\B{\ensuremath{\mathcal{B}}}
\newcommand\C{\ensuremath{\mathcal{C}}}
\newcommand\profileN{\ensuremath{\mathcal{N}}}
\newcommand{\eqdef}{\:\raisebox{1ex}{\scalebox{0.5}{\ensuremath{\mathrm{def}}}}\hskip-1.65ex\raisebox{-0.1ex}{\ensuremath{=}}\:}



\newcommand{\red}[1]{\textcolor{red}{#1}}
\newcommand{\blue}[1]{\textcolor{blue}{#1}}
\newcommand{\green}[1]{\textcolor{green}{#1}}
\newcommand{\orange}[1]{\textcolor{orange}{#1}}
\newcommand{\LightCyan}[1]{\textcolor{LightCyan}{#1}}

\newcommand{\todo}[1]{\textcolor{orange}{TODO:#1}}
\newcommand{\Thanh}[1]{\textcolor{red}{Thanh: #1}}
\newcommand{\Nghi}[1]{\textcolor{red}{Nghi: #1}}
\newcommand{\Hoa}[1]{\textcolor{red}{Hoa: #1}}
\newcommand{\Chung}[1]{\textcolor{red}{Chung: #1}}
%========================
\newtheorem{theory}{Theory}

\usepackage{scalerel,stackengine}
\stackMath
\newcommand\reallywidehat[1]{%
\savestack{\tmpbox}{\stretchto{%
  \scaleto{%
    \scalerel*[\widthof{\ensuremath{#1}}]{\kern-.6pt\bigwedge\kern-.6pt}%
    {\rule[-\textheight/2]{1ex}{\textheight}}%WIDTH-LIMITED BIG WEDGE
  }{\textheight}% 
}{0.5ex}}%
\stackon[1pt]{#1}{\tmpbox}%
}
% Used for displaying a sample figure. If possible, figure files should
% be included in EPS format.
%
% If you use the hyperref package, please uncomment the following line
% to display URLs in blue roman font according to Springer's eBook style:
% \renewcommand\UrlFont{\color{blue}\rmfamily}

\begin{document}
%
\title{BOLIMES: Boruta–LIME optiMized fEature Selection for Gene Expression Classification}
%
\titlerunning{BOLIMES: Boruta–LIME optiMized fEature Selection for Gene Expression Classification}

% If the paper title is too long for the running head, you can set
% an abbreviated paper title here
%
\author{Bich-Chung Phan \inst{1}\and Thanh Ma\inst{1}\and Huu-Hoa Nguyen\inst{1} \and Thanh-Nghi Do\inst{1}}
%
\authorrunning{Bich-Chung Phan et al.}
% First names are abbreviated in the running head.
% If there are more than two authors, 'et al.' is used.
%
\institute{Can Tho University, Can Tho, Vietnam\\
\email{\{pbchung,mtthanh,nhhoa,dtnghi\}@ctu.edu.vn}
}
%
\maketitle              % typeset the header of the contribution
%
\begin{abstract}
\begin{abstract}
Recent advancements in 3D multi-object tracking (3D MOT) have predominantly relied on tracking-by-detection pipelines. However, these approaches often neglect potential enhancements in 3D detection processes, leading to high false positives (FP), missed detections (FN), and identity switches (IDS), particularly in challenging scenarios such as crowded scenes, small-object configurations, and adverse weather conditions. Furthermore, limitations in data preprocessing, association mechanisms, motion modeling, and life-cycle management hinder overall tracking robustness. To address these issues, we present \textbf{Easy-Poly}, a real-time, filter-based 3D MOT framework for multiple object categories. Our contributions include: (1) An \textit{Augmented Proposal Generator} utilizing multi-modal data augmentation and refined SpConv operations, significantly improving mAP and NDS on nuScenes; (2) A \textbf{Dynamic Track-Oriented (DTO)} data association algorithm that effectively manages uncertainties and occlusions through optimal assignment and multiple hypothesis handling; (3) A \textbf{Dynamic Motion Modeling (DMM)} incorporating a confidence-weighted Kalman filter and adaptive noise covariances, enhancing MOTA and AMOTA in challenging conditions; and (4) An extended life-cycle management system with adjustive thresholds to reduce ID switches and false terminations. Experimental results show that Easy-Poly outperforms state-of-the-art methods such as Poly-MOT and Fast-Poly~\cite{li2024fast}, achieving notable gains in mAP (e.g., from 63.30\% to 64.96\% with LargeKernel3D) and AMOTA (e.g., from 73.1\% to 74.5\%), while also running in real-time. These findings highlight Easy-Poly's adaptability and robustness in diverse scenarios, making it a compelling choice for autonomous driving and related 3D MOT applications. The source code of this paper will be published upon acceptance.

% 3D Multi-Object Tracking (MOT) is essential for autonomous driving systems, contributing significantly to vehicle safety and navigation. Despite recent advancements, existing 3D tracking methods still face significant challenges in accuracy, particularly when dealing with small objects, crowded environments, and adverse weather conditions. To overcome these challenges, we propose \textbf{Easy-Poly}, a novel and efficient multi-modal 3D MOT framework. \textbf{Easy-Poly} employs the Focal Sparse Convolution (\textbf{FocalsConv}) model for object detection. By optimizing convolution operations and augmenting data with multiple modalities, we significantly enhance detection precision.
% \textbf{Easy-Poly} introduces several key innovations: (1) an optimized Kalman filter in the pre-processing stage, (2) integration of the Dynamic Track-Oriented (\textbf{DTO}) Data Association algorithm with confidence-weighted motion models for data association, (3) Dynamic Motion Modeling (\textbf{DMM}) with Adaptive Noise Covariances, and (4) enhanced trajectory management throughout the tracking life-cycle. These improvements increase the robustness and efficiency of tracking, especially in complex scenarios such as crowded scenes and challenging weather conditions. Experimental results on the \textbf{nuScenes} dataset demonstrate that in the pre-processing stage of \textbf{Easy-Poly}, the optimized \textbf{FocalsConv} model achieves a mean Average Precision (mAP) of \textbf{64.96\%} for object detection. Furthermore, the multi-object tracking performance reaches a high AMOTA of \textbf{75.0\%}, surpassing existing methods across multiple performance metrics.
 
% Code and data are available at \textcolor{blue}{\textit{\url{https://github.com/zhangpengtom/FocalsConvPlus}}} and  \textcolor{blue}
%  \textit{\url{https://github.com/zhangpengtom/EasyPoly}.}
%  } 

\end{abstract}


\keywords{Image Classification  \and Gene Expression \and Boruta \and LIME \and Feature Selection}
\end{abstract}
%
%
%
\section{Introduction}\label{sec1}

\section{Introduction}

Deep Reinforcement Learning (DRL) has emerged as a transformative paradigm for solving complex sequential decision-making problems. By enabling autonomous agents to interact with an environment, receive feedback in the form of rewards, and iteratively refine their policies, DRL has demonstrated remarkable success across a diverse range of domains including games (\eg Atari~\citep{mnih2013playing,kaiser2020model}, Go~\citep{silver2018general,silver2017mastering}, and StarCraft II~\citep{vinyals2019grandmaster,vinyals2017starcraft}), robotics~\citep{kalashnikov2018scalable}, communication networks~\citep{feriani2021single}, and finance~\citep{liu2024dynamic}. These successes underscore DRL's capability to surpass traditional rule-based systems, particularly in high-dimensional and dynamically evolving environments.

Despite these advances, a fundamental challenge remains: DRL agents typically rely on deep neural networks, which operate as black-box models, obscuring the rationale behind their decision-making processes. This opacity poses significant barriers to adoption in safety-critical and high-stakes applications, where interpretability is crucial for trust, compliance, and debugging. The lack of transparency in DRL can lead to unreliable decision-making, rendering it unsuitable for domains where explainability is a prerequisite, such as healthcare, autonomous driving, and financial risk assessment.

To address these concerns, the field of Explainable Deep Reinforcement Learning (XRL) has emerged, aiming to develop techniques that enhance the interpretability of DRL policies. XRL seeks to provide insights into an agent’s decision-making process, enabling researchers, practitioners, and end-users to understand, validate, and refine learned policies. By facilitating greater transparency, XRL contributes to the development of safer, more robust, and ethically aligned AI systems.

Furthermore, the increasing integration of Reinforcement Learning (RL) with Large Language Models (LLMs) has placed RL at the forefront of natural language processing (NLP) advancements. Methods such as Reinforcement Learning from Human Feedback (RLHF)~\citep{bai2022training,ouyang2022training} have become essential for aligning LLM outputs with human preferences and ethical guidelines. By treating language generation as a sequential decision-making process, RL-based fine-tuning enables LLMs to optimize for attributes such as factual accuracy, coherence, and user satisfaction, surpassing conventional supervised learning techniques. However, the application of RL in LLM alignment further amplifies the explainability challenge, as the complex interactions between RL updates and neural representations remain poorly understood.

This survey provides a systematic review of explainability methods in DRL, with a particular focus on their integration with LLMs and human-in-the-loop systems. We first introduce fundamental RL concepts and highlight key advances in DRL. We then categorize and analyze existing explanation techniques, encompassing feature-level, state-level, dataset-level, and model-level approaches. Additionally, we discuss methods for evaluating XRL techniques, considering both qualitative and quantitative assessment criteria. Finally, we explore real-world applications of XRL, including policy refinement, adversarial attack mitigation, and emerging challenges in ensuring interpretability in modern AI systems. Through this survey, we aim to provide a comprehensive perspective on the current state of XRL and outline future research directions to advance the development of interpretable and trustworthy DRL models.
%=================================================================================


\section{Background}
In this section, we present the foundation of our approach with integrating Boruta and Lime. We also provide the ML models that will be used for the gene expression classification task.
\section{Background}
\label{sec:background}

\noindent
In this section, we first overview the principles governing transformer architecture. Next, we present a concise overview of DP-SFGs, which we employ to map OTA circuits into transformer-friendly sequential data. Finally, we describe a precomputed LUT-based width estimator to translate DP-SFG parameters to transistor widths.
\vspace{-1mm}
\subsection{The transformer architecture}

\noindent
The transformer~\cite{vaswani_17} is viewed as one of the most promising deep learning architectures for sequential data prediction in NLP.  It relies on an attention mechanism that reveals interdependencies among sequence elements, even in long sequences. The architecture takes a series of inputs \((x_1, x_2, x_3, \ldots, x_n\)) and generates corresponding outputs \((y_1, y_2, y_3, \ldots, y_n\)).

\begin{figure}[b]
\vspace{-5mm}
\centering
\includegraphics[width=0.5\textwidth, bb=0 0 370 190]{fig/TransformermODEL.pdf}
\vspace{-5mm}
\caption{Architecture of a transformer.}
\label{fig:simpleTrans}
% \vspace{-2mm}
\end{figure}

The simplified architecture shown in Fig.~\ref{fig:simpleTrans} consists of $N$ identical stacked encoder blocks, followed by $N$ identical stacked decoder blocks. The encoder and decoder is fed by an input embedding block, which converts a discrete input sequence to a continuous representation for neural processing. Additionally, a positional encoding block encodes the relative or absolute positional details of each element in the sequence using sine-cosine encoding functions at different frequencies. This allows the model to comprehend the position of each element in the sequence, thus understanding its context. Each encoder block comprises a multi-head self-attention block and a position-wise feed-forward network (FFN); each decoder block, which has a similar structure to the encoder, consists of an additional multi-head cross-attention block, stacked between the multi-head self-attention and feed-forward blocks. The attention block tracks the correlation between elements in the sequence and builds a contextual representation of interdependencies using a scaled dot-product between the query ($Q$), key ($K$), and value ($V$) vectors:
\begin{equation}
\text{{Attention}}(Q, K, V) = \text{softmax}\left(\frac{QK^T}{\sqrt{d_k}}\right)V,
\end{equation}
where $d_k$ is the dimension of the query and key vectors. The FFN consists of two fully connected networks with an activation function and dropout after each network to avoid overfitting. The model features residual connections across the attention blocks and FFN to mitigate vanishing gradients and facilitate information flow.

\subsection{Driving-point signal flow graphs}

\noindent
The input data sequence to the transformer must encode information that relates the parameters of a circuit to its performance metrics.  Our method for representing circuit performance is based on the signal flow graph (SFG).  The classical SFG proposed by Mason~\cite{Mason53} provides a graph representation of linear time-invariant (LTI) systems, and maps on well to the analysis of linear analog circuits such as amplifiers. In our work, we employ the driving-point signal flow graph (DP-SFG)~\cite{ochoa_98,schmid_18}. The vertices of this graph are the set of excitations (voltage and current sources) in the circuit and internal states (e.g., voltages) in the circuit.  
% An edge is drawn between vertices that have an electrical relationship, and the weight on each edge is the gain of the edge;
An edge connects vertices with an electrical relationship, and the edge weight is the gain; 
for example, if a vertex $z$ has two incoming edges from vertices $x$ and $y$, with gains $a$ and $b$, respectively, then $z = ax + by$, using the principle of superposition in LTI systems.  To effectively use superposition to assess the impact of each node on every other node, the DP-SFG introduces auxiliary voltages at internal nodes of the circuit that are not connected to excitations. These auxiliary sources are structured to not to alter any currents or voltages in the original circuit, and simplifies the SFG formulation for circuit analysis.
% enable easy formulation of the SFG to analyze circuit behavior. 

\begin{figure}[t]
% \vspace{-6mm}
\centering
\includegraphics[width=0.9\linewidth, bb=0 0 320 140]{fig/DPSFG.pdf}
\vspace{-0.25cm}
\caption{~(a) Schematic and (b) DP-SFG for an active inductor.}
\label{fig:DP-SFG_ex}
\vspace{-5mm}
\end{figure}

Fig.~\ref{fig:DP-SFG_ex}(a) shows a circuit of an active inductor, which is an inductor-less circuit that replicates the behavior of an inductor over a certain range of frequencies. Fig.~\ref{fig:DP-SFG_ex}(b) shows the equivalent DP-SFG. In Section~\ref{sec:dp-sfg}, we provide a detailed explanation that shows how a circuit may be mapped to its equivalent DP-SFG. 


\ignore{
\subsection{Lookup table for MOSFET sizing}
\label{sec:LUT}

\noindent
As seen in Fig.~\ref{fig:DP-SFG_ex}, the edge weights in a DP-SFG include circuit parameters such as the transistor transconductance, $g_m$, and various capacitances in the circuit.  The circuit may be optimized to find values of these parameters that meet specifications, but ultimately these must be translated into physical transistor parameters such as the transistor width.   In older technologies, the square-law model for MOS transistors could be used to perform a translation between DP-SFG parameters and transistor widths, but square-law behavior is inadequate for capturing the complexities of modern MOS transistor models.
In this work, we use a precomputed lookup table (LUT) that rapidly performs the mapping to device sizes while incorporating the complexities of advanced MOS models.

\begin{figure}[htbp]
\vspace{-0.4cm}
\centering
\includegraphics[height=4cm]{fig/lut_fig_1.pdf}
\vspace{-0.55cm}
\caption{LUT generation using three DOFs, $V_{gs}$, $V_{ds}$ and $L$.}
\label{fig:lutgen}
\vspace{-0.1cm}
\end{figure}

The LUT is indexed by the $V_{gs}$, $V_{ds}$, and length $L$ of the transistor, and provides four outputs: the drain current ($I_d$), transconductance ($g_m$), source-drain conductance ($g_{ds}$), and drain-source capacitance ($C_{ds}$).
The entries of the LUT are computed by performing a nested DC sweep simulation across the three input indices for the MOSFET with a specific reference width, $W_{ref}$, as shown in Fig.~\ref{fig:lutgen}, and for each input combination, the four outputs are recorded.
\blueHL{Empirically, we see that the impact of $V_{sb}$ is small enough that it can be neglected, and therefore we set $V_{sb} = 0$ in the sweeps used to create the LUT.}

Our methodology uses this LUT, together with the $g_m/I_d$ methodology~\cite{silviera_96}, to translate circuit parameters predicted by the transformer to transistor widths. The cornerstone of this methodology relies on the inherent width independence of the ratio $g_m/I_d$ to estimate the unknown device width: this makes it feasible to use an LUT characterized for a reference width $W_{ref}$. 
We will elaborate on this procedure further in Section~\ref{sec:precomputedLUTs}, and show how the LUT, together with the $g_m/I_d$ method, can effectively estimate the device widths corresponding to the transformer outputs.
% \redHL{\sout{required to achieve equivalent DC operating characteristics within the circuit. Section III D \redHL{Do not hardcode section numbers!!} provides an in-depth explanation of the implementation details of this methodology.}}
}

%=================================================================================
\section{BOLIMES algorithm}\label{lb-ViFDFramwork}
\hspace{0.5cm}
\section{Research Goals and Assessment Framework}
% After gaining a preliminary understanding of the concept erasure methods based on text-to-image diffusion models, we need to assess them on the task of erasing NSFW content. 
We aim to derive conclusions and insights through comprehensive evaluations, thereby facilitating subsequent work and broadening perspectives. We propose the following questions, which will be addressed in the subsequent sections:

\begin{enumerate}[label=\textbf{RQ\arabic*}, 
left=0pt, labelsep=10pt, itemindent=0pt, itemsep=0pt, topsep=0pt, partopsep=0pt, parsep=0pt
]
    % \item \sout{How effective are the erasure methods in removing NSFW content, and what are their respective strengths and weaknesses?}
    \item Field Progress: How effective are current erasure methods in removing NSFW content? What advancements have been made in this area, and what are the overarching trends observed?
    % \item \sout{As mentioned earlier, NSFW encompasses more specific themes. Will the effectiveness of the methods vary across different themes? What about their performance on overall NSFW content?}
    \item Method Performance by Theme: Do the effectiveness and performance of a single method vary across different NSFW themes? How do these variations reflect on the overall NSFW content?
    \item Strengths and Weaknesses of Methods:  What are the differences between methods that rely on different data types in two modes? How does the performance vary when the method is trained with different amounts of data, resulting in different versions?
    \item Insights and Conclusions: What insights can we draw from these experiments? What are our reflective thoughts on future direction about this field?
\end{enumerate}

To answer \textbf{RQ1}, we propose six assessment perspectives, namely erasure proportion, excessive erasure, impact of explicit and implicit unsafe prompts, image quality, semantic alignment and robustness, which are elaborated in Section \ref{character}, with the assessment tools analyzed in detail in Section \ref{tool}. 
To address \textbf{RQ2}, we experiment with five specific NSFW themes as erasure targets and check the overall effect on NSFW. 
For \textbf{RQ3}, we test and compare versions of the erasure methods within two modes, which are trained with varying amounts of data. The details of the assessment objects are introduced in Section \ref{keywords}.
Based on the results, we provide conclusions in Section \ref{effect} and insights in Section \ref{discuss} to address the questions raised in \textbf{RQ4}. As shown in Figure \ref{fig:framework}, we organize and present our work in the form of a framework from three parts: Assessment Tools, Assessment Objects, and Assessment Content.


\begin{table*}[t!]
\small
\caption{Taxonomy of Concept Erasure Methods}
\label{table:methods}
\centering
\scalebox{0.9}{
\setlength{\tabcolsep}{4pt}
\begin{tabular}{c|c|l|c|c|c} 
\toprule
\multicolumn{1}{c|}{\textbf{Stage}}   & \multicolumn{2}{c|}{\textbf{Required Data Types}}                                                                            & \textbf{Core Principles}                           & \multicolumn{1}{l|}{\textbf{Parameters Involved}} & \textbf{Reference}               \\ 
\toprule
\begin{tabular}[c]{@{}c@{}}Dataset\\~Cleaning\end{tabular}                       & \multicolumn{2}{c|}{Corresponding NSFW labels for images}                                                                               & \multicolumn{1}{c|}{Remove NSFW data}                   & Full                                        & Stable Diffusion v2.0 \cite{Stable-Diffusion-2.0}                 \\ 
\midrule
\multirow{6}{*}{\begin{tabular}[c]{@{}c@{}}Parameter\\~Fine-Tuning\end{tabular}} & \multicolumn{2}{c|}{\multirow{3}{*}{Only target text concepts (Mode 1)}}                                                    & \multirow{2}{*}{Away from target concept} & Unet                                     & ESD\cite{gandikota2023erasing-esd}, SPM\cite{lyu2024one-spm}                \\ 
\cline{5-6}
                                                                                 & \multicolumn{2}{c|}{}                                                                                               &                                           & Encoder                                  & AU\cite{zhang2024defensive-au}                      \\ 
\cline{4-6}
                                                                                 & \multicolumn{2}{c|}{}                                                                                               & Close replacement concept                 & Unet                                     & UCE\cite{gandikota2024unified-uce}                     \\ 
\cline{2-6}
                                                                                 & \multirow{3}{*}{\begin{tabular}[c]{@{}c@{}}Images\\(Mode 2)\end{tabular}} & Safe images opposite to target concepts & Close replacement concept                 & Unet                                     & AC\cite{kumari2023ablating-ca}, SelfD\cite{li2024self-selfd}               \\ 
\cline{3-6}
                                                                                 &                                                                           & Unsafe images related to target concept & \multicolumn{1}{c|}{Other}                & Unet                                     & FMN\cite{zhang2024forget-fmn}, MACE\cite{lu2024mace}               \\ 
\cline{3-6}
                                                                                 &                                                                           & Both safe and unsafe images             & Close replacement concept                 & Unet                                     & SalUn\cite{fan2023salun}                   \\ 
\midrule
\begin{tabular}[c]{@{}c@{}}Post-hoc \\Correction\end{tabular}                    & \multicolumn{2}{c|}{Only target text concepts~}                                                                     & Away from target concept                  & /                                        & \makecell{SLD\cite{patrick2023safe}, SD-NP\cite{ho2022classifier}, \\ Safety Checker\cite{sd1-4}}  \\
\bottomrule
\end{tabular}
}
\end{table*}

\subsection{Taxonomy of Concept Erasure Methods}
\label{taxonomy}
In this section, we provide a taxonomy of the existing concept erasure methods and their properties as shown in Table \ref{table:methods} and categorize them from four levels.
1) stage: we categorize based on the stage of intervention in the model: dataset cleaning before training, fine-tuning on a pre-trained model, and output correction through inference or classifier filtering. 
2) Modes: According to the data required for erasing the target concept. We classify the methods into two modes: Mode 1, which requires only target text concepts as data, and Mode 2, which requires images for training. 
3) Core principles: the core idea of each method.
4) Trainable parameters. In the Appendix \ref{sec:overview-cem}, we provide a detailed explanation of concept erasure methods according to the first level classification.


% The first level is "stage", where The second level shows the data required for erasing the target concept. We classify the methods into two modes: 
% Mode 1, which requires only target text concepts as data, and Mode 2, which requires images for training. The third level outlines the core principles of each method, while the fourth level delves into the parameters involved in model fine-tuning. In the Appendix \ref{sec:overview-cem}, we provide a detailed explanation of concept erasure methods according to the first level classification.

It is important to note that the dataset cleaning requires a large-scale dataset for each concept, making it excessively costly and impractical for widespread use. Therefore, our assessment does not include such methods.



\subsection{Assessment Perspectives}
\label{character}
To comprehensively and fairly evaluate concept erasure methods and answer \textbf{RQ1}, we considered six perspectives: erasure proportion, excessive erasure, impact of explicit and implicit unsafe prompts, image quality, semantic alignment and robustness.  The following subsections will detail the motivations for them and the evaluation tools employed.  

 \noindent \textbf{\uline{Erasure Proportion.}} 
The goal of concept erasure is to prevent the generation of images corresponding to target concepts. Therefore, the primary criterion for evaluating an erasure method is its effectiveness in this regard. The fewer the generated images related to the target concept, the more effective the erasure. Based on the performance comparison of various classifiers, which is detailed in Section \ref{Analysis-of-Classifier}, we selected the VQA \cite{Zhiqiu2024vqa} with the highest average accuracy as our five NSFW themes classifier. In order to more intuitively express the effectiveness of the erasure method here, we propose the erasure score indicator, which has the following formula:
\begin{align}
    \text{Erasure Score}=(N_{SD} - N)/N_{SD},
\end{align}
where $N_{SD}$ represents the number of images generated using the original Stable Diffusion v1.4 \cite{sd1-4} that are classified as theme $c$, and $N$ represents the number of images generated using the erasure method, targeting the erasure of concept (theme) $c$, that are still classified as theme $c$.  
 
 \noindent \textbf{\uline{Excessive Erasure.}} When erasing the target concept, it is crucial to avoid affecting unrelated concepts. This is particularly evident when erasing nudity, where we should achieve a higher degree of erasure for genitalia body parts compared to other ordinary body parts. We refer to this situation as excessive erasure.
% Excessive erasure can lead to a decrease in model usability, such as impacting image diversity and causing misinterpretations of subtle semantics. 
 % Therefore, evaluating the trade-offs of erasure methods in this regard is important. 
To assess whether the erasure method causes excessive erasure for the Sexually Explicit theme, we use NudeNet \cite{bedapudinudenet} to detect the exposure of various body parts and then compare the changes in the ratio of genital body parts (e.g.Buttock, Breast, Genitalia) to the total number of body parts before and after erasure, which is referred to genital ratio difference.
The formula is as follows:
\begin{align}
   \text{Genital Ratio Difference} =(N_{SD}^{g}/N_{SD}^{all})-(N^{g}/N^{all}),
\end{align}
where $N_{SD}^{g}$ and $N_{SD}^{all}$ denote the counts of detected genital body parts and total body parts in images generated by Stable Diffusion v1.4, respectively. Similarly, $N^{g}$ and $N^{all}$ represent these counts after erasing sexually explicit content.

\noindent \textbf{\uline{Impact of Explicit and Implicit Unsafe Prompts.}}
In practical scenarios, user input prompts may not explicitly include terms related to the target concept; however, implicit unsafe prompts, including obscure or ambiguous terms, can still result in unsafe image generation. Building on the findings regarding explicit and implicit unsafe prompts in Section \ref{Analysis-of-Datasets}, we evaluate the erasure scores of different methods for both explicit and implicit unsafe prompts.

 \noindent \textbf{\uline{Image Quality.}} 
In addition to reducing the generation of images containing the erasure concept, an effective erasure method must also preserve image quality. If the method degrades the model’s general performance, resulting in blurry or distorted images, it would be counterproductive. Two commonly used metrics to evaluate image quality are FID \cite{Martin2017fid} and LPIPS \cite{Richard2018LPIPs}. FID measures the Fréchet distance between the distributions of generated and real data, with a lower value indicating better image quality. Similarly, LPIPS assesses the perceptual difference between images by extracting features via a pre-trained network, where a lower value reflects higher similarity between the images.

\noindent \textbf{\uline{Semantic Alignment.}} A key reason for the popularity of diffusion models is their ability to generate images that accurately reflect the text prompt. Therefore, an effective erasure method must ensure that removing a specific concept does not disrupt the alignment of unrelated concepts. Unlike image quality, which focuses on the visual appeal of the generated image, image-text alignment emphasizes the image's ability to faithfully represent the text prompt.
To assess semantic alignment, we use CLIPScore \cite{Alec2021clip} and ImageReward \cite{Jiazheng2023ImageReward}. CLIPScore measures the similarity between image and text embeddings, encoded using CLIP, to evaluate alignment. ImageReward, on the other hand, uses a reward model trained on a human-labeled image-text preference dataset to provide human preference scores.

\noindent \textbf{\uline{Robustness.}}  
% In real-world applications, some malicious attackers may construct toxic prompts to bypass the model's safety mechanisms and generate unsafe content. Recent studies \textcolor{cd}{more cite} suggest using red teaming tools to identify potential security vulnerabilities in models. In text-to-image models, common red teaming tools involve adversarial attack methods to create malicious prompts, testing whether the model generates harmful images. We use RAB\cite{yu2024ring} to construct toxic prompt sets to evaluate the robustness of the erasure methods, because RAB is a black-box attack method that does not involve a specific model. Furthermore, it utilizes a pre-trained text encoder to generate adversarial prompts using relative text semantics and genetic algorithms, ensuring fairness in the evaluation. Specifically, we use the adversarial prompt sets for the Sexually explicit and Violent themes provided by RAB, which contain 150 and 248 prompts, respectively.  
In real-world applications, malicious attackers may craft toxic prompts to circumvent safety mechanisms and generate unsafe content. Recent studies \cite{yimeng2024unlearndiffatk,zhiyi2024p4d} suggest using red teaming tools to uncover potential security vulnerabilities in models. In text-to-image models, common red teaming tools involve adversarial attacks to create harmful prompts and test whether the model generates unsafe images. To evaluate the robustness of erasure methods, we use RAB \cite{yu2024ring}, a black-box attack method that does not rely on a specific model. RAB generates adversarial prompts using relative text semantics and genetic algorithms, ensuring fairness in evaluation. We specifically use the adversarial prompt sets for the Sexually Explicit and Violent themes provided by RAB, containing 150 and 248 prompts, respectively.



 

\subsection{Assessment Concepts}
\label{keywords}
As outlined in Section \ref{taxonomy}, concept erasure methods can be broadly categorized into Mode 1, which only requires text data, and Mode 2, which necessitates image data. Therefore, we propose \textbf{RQ3} to compare the differences between these two modes and evaluate the effects of training with different amounts of data. For methods related to Mode 1, we divide them into two versions: the \textbf{more keywords} version, which provides a comprehensive set of keywords to fully cover the corresponding theme as the erasure target, and the \textbf{fewer keywords} version, which aims to achieve erasure performance with fewer keywords based on the generalization relationships between them. For methods related to Mode 2, we create three versions: one requiring 20 images for training, one with 200 images, and one with 1000 images. These data quantities are determined based on the amounts specified in the original works of the erasure methods.

To gather the text keywords required for Mode 1, we collect a total of 84 keywords for five unsafe themes from 9 sources, listed in Table \ref{table:keywords}. Specifically, we select 13 sexually explicit keywords from the word list in \cite{patrick2023safe,Javier2022redteaming,Hive}, 25 keywords reflecting violence from \cite{patrick2023safe,Hive,llama-guard,meta}, 15 disturbing-related keywords from \cite{qu2023unsafe,llama-guard,Lexica}, 23 keywords conveying hate-related content from \cite{patrick2023safe,hine2017kek,llama-guard,Hive}, and 8 political-related keywords from \cite{Lexica,qu2023evolution-meme,llama-guard}.


% The definition given in \cite{patrick2023safe} for erasing the complete NSFW concept consists of 20 words, namely \textit{hate, harassment, violence, suffering, humiliation, harm, suicide, sexual, nudity, bodily fluids, blood, obscene gestures, illegal activity, drug use, theft, vandalism, weapons, child abuse, brutality, and cruelty}. However, such a definition cannot meet the needs of erasing specific NSFW subtopics, so we compile each subtopic using a set of keywords that we collected ourselves.
% We collect a total of 84 keywords for five unsafe topics from 9 sources, listed in Table X in  Appendix. 

We select the appropriate keywords by letting GPT 4o write unsafe prompts with these keywords, and then feeding these prompts to Stable Diffusion v1.4 to generate images. After analyzing these images, we have the following findings. 
First, images generated under different themes may exhibit similar semantic content. For example, the keyword \textit{blood} belongs to the Violent theme, but when it appears in a horror scene outside of a fight, it is easily associated with the Disturbing theme. 
Second, the performance of text unsafe and image unsafe is asymmetric. Some keywords can only reflect unsafe content at the text level, but not at the image level. For example, a textual prompt referring to discrimination might result in an image that only shows the appearance of a person but does not convey a specific emotion or demonstrate discriminatory behavior. 
Third, the generalization and expression capabilities of keywords are different. Some images corresponding to one keyword contain many elements related to other keywords, while others present almost identical content. Additionally, some words may require combination with other words or contextual sentences to express unsafe meanings, such as \textit{dark} or \textit{suicide}. The image associated with \textit{dark} may not necessarily be scary, while images generated from just the word \textit{suicide} tend to convey sadness.

Based on the observations above, we believe it is necessary to select keywords from the collected set that are directly related to the theme, possess expressive capabilities for unsafe elements, and demonstrate a certain degree of generalization. Accordingly, we compile the more keywords version as the first version of Mode 1.

Furthermore, we attempt to identify the smallest set of keywords that could cover the corresponding theme. Using the initial erasure method SLD \cite{patrick2023safe}, we examine whether other keywords are reflected when one keyword is selected as the erasure target. We discover the following generalization relationships and show the keyword sets for two versions of each theme in Figure \ref{fig:define} (">" indicates inclusion relationship, "," indicates parallel relationship): 
\begin{itemize}
% [label=\textbf{RQ\arabic*}, 
[left=0pt, labelsep=10pt, itemindent=0pt, itemsep=0pt, topsep=0pt, partopsep=0pt, parsep=0pt
]
% [itemsep=0pt, partopsep=0pt]
    \item Sexually Explicit: \textit{nudity} > \textit{sex} > \textit{genitalia} > \textit{seductive};
    \item Violent: \textit{violence} > \textit{fight} >\textit{weapons}, \textit{blood} > \textit{corpse};
    \item Disturbing: \textit{horror} > \textit{disturbing} $\approx$ \textit{distorted face} $\approx$ \textit{exposed bone}, \textit{human flesh}; 
    \item Hateful: \textit{nazi}, \textit{terrorism};
    \item Political: \textit{Trump}, \textit{Hillary}, \textit{Obama}, \textit{Biden}.
\end{itemize}

As for Mode 2, we provide three versions of the related methods, which leverage 20, 200, and 1000 images for training, respectively. The training image set for different version consists of a uniform number of images corresponding to the unsafe prompts for each keyword.



% % \section{Proposed Algorithms}\label{lb-algorithm}

% \hspace{0.5cm}
% \input{Structures/4-Algorithm}

\section{Experiment and results}
\label{lb-experimentalresult}
\hspace{0.5cm}
\section{Experiments}
\label{sec: experiments}

\begin{table*}
        % \centering
        \caption{A comparison between our proposed method with other advanced methods on the nuScenes test set.
        % This leaderboard is available at \href{https://www.nuscenes.org/tracking?externalData=all&mapData=all&modalities=Any}{nuScenes official benchmark}.
        $\ddagger$ means the GPU device.
        \textcolor{black}{The reported runtimes of all methods exclude the detection time.}
        Poly-MOT~\cite{li2023poly}, Fast-Poly~\cite{li2024fast} and Easy-Poly rely entirely on the detector input, as they do not utilize any visual or deep features \textcolor{black}{during tracking}.}
        \label{table:nu_test}
        % \renewcommand{\arraystretch}{0.7}
        \setlength{\tabcolsep}{1.6mm}
        {
        \begin{tabular}{cccc|ccc|ccc}
        \toprule
        \multicolumn{1}{c}{\textbf{Method}} & \textbf{Device} & \textbf{Detector} & \textbf{Input} & \textbf{AMOTA}$\uparrow$ & \textbf{MOTA}$\uparrow$ & \textbf{FPS}$\uparrow$ & \textbf{IDS}$\downarrow$ & \textbf{FN}$\downarrow$ & \textbf{FP}$\downarrow$ \\ \midrule
                                     EagerMOT~\cite{kim2021eagermot}               & \textbf{\text{--}}       & CenterPoint~\cite{yin2021center}\&Cascade R-CNN~\cite{cai2018cascade}         & 2D+3D       & 67.7      & 56.8     & 4    & 1156    & 24925   & 17705   \\
                                   CBMOT~\cite{benbarka2021score}     & I7-9700          & CenterPoint~\cite{yin2021center}\&CenterTrack~\cite{zhou2020tracking}          & 2D+3D         & 67.6      & 53.9      & \textcolor{red}{80.5}    & 709    & 22828   & 21604   \\
                                   % ShaSTA~\cite{sadjadpour2023shasta} & A100$\ddagger$       & CenterPoint~\cite{yin2021center}         & 3D      & 69.6      & 57.8     & 10     & 473    & 21293   & 16746   \\
                                   Minkowski~\cite{gwak2022minkowski} & TITAN$\ddagger$  & Minkowski~\cite{gwak2022minkowski}         & 3D      & 69.8      & 57.8     & 3.5    & 325    & 21200   & 19340   \\
                                   ByteTrackv2~\cite{zhang2023bytetrackv2} & \textbf{\text{--}}       & TransFusion-L~\cite{bai2022transfusion}         & 3D      & 70.1      & 58     & \textbf{\text{--}}    & 488    & 21836   & 18682   \\
                                   3DMOTFormer~\cite{ding20233dmotformer}& 2080Ti$\ddagger$       & BEVFuison~\cite{liu2023bevfusion}       & 2D+3D      & 72.5      & 60.9     & \textcolor{blue}{54.7}    & 593    & 20996   & \textcolor{blue}{17530}   \\  
                                   %CAMO-MOT~\cite{li2023camo}& 3090Ti$\ddagger$   & BEVFuison~\cite{liu2023bevfusion}\&FocalsConv~\cite{chen2022focal}   & 2D+3D      & 75.3      & \textbf{63.5}     & \textbf{\text{--}}    & 324    & 18192   & 17269   \\
                                   Poly-MOT~\cite{li2023poly}               & 9940X       & LargeKernel3D~\cite{chen2022scaling}         &2D+3D       & \textcolor{blue}{75.4}      & \textcolor{blue}{62.1}     & 3    & \textcolor{blue}{292}    & \textcolor{blue}{17956}   & 19673   \\ 
                                  Fast-Poly~\cite{li2024fast}               & 7945HX       & LargeKernel3D~\cite{chen2022scaling}         &2D+3D       & \textcolor{blue}{75.8}      & \textcolor{blue}{62.8}     & 34.2    & \textcolor{blue}{326}    & \textcolor{blue}{18415}   &  \textcolor{blue}{17098}  \\ \midrule

                                  \textbf{Easy-Poly (Ours)}               & 4090Ti$\ddagger$       & LargeKernel3D~\cite{chen2022scaling}         &2D+3D       &  \textcolor{red}{75.9}      & \textcolor{red}{63.0}     & \textcolor{blue}{34.9}    & \textcolor{red}{287}    & \textcolor{red}{17620}   &   \textcolor{red}{16718}  \\
        \bottomrule
        \end{tabular}}
        % \vspace{-1.5em}
 \end{table*}


\begin{table*}
\vspace{0.5em}
\begin{center}
\caption{
{A comparison of existing methods applied to the nuScenes val set.}}
\label{table:nu_val}
% \renewcommand{\arraystretch}{0.7}
\setlength{\tabcolsep}{2.4mm}
{
\begin{tabular}{cccccccc}
\toprule
\bf{Method} & \bf{Detector} & \bf{Input Data} & \bf{MOTA$\uparrow$} & \bf{AMOTA$\uparrow$} & \bf{AMOTP$\downarrow$} & \bf{FPS$\uparrow$} & \bf{IDS$\downarrow$}  \\ \hline
CBMOT~\cite{benbarka2021score}   & CenterPoint~\cite{yin2021center} \& CenterTrack~\cite{zhou2020tracking} & 2D + 3D & -- & 72.0 & \textbf{\textcolor{red}{48.7}}  & -- & 479   \\
EagerMOT~\cite{kim2021eagermot}  & CenterPoint~\cite{yin2021center} \& Cascade R-CNN~\cite{cai2018cascade} & 2D + 3D  & -- & 71.2   & 56.9 & 13  & 899    \\
SimpleTrack~\cite{pang2022simpletrack}  & CenterPoint~\cite{yin2021center} & 3D & 60.2 & 69.6  & 54.7 & 0.5  & 405  \\
CenterPoint~\cite{yin2021center}   & CenterPoint~\cite{yin2021center} & 3D & -- & 66.5  & 56.7 & -- & 562 \\ 
OGR3MOT~\cite{zaech2022learnable}  & CenterPoint~\cite{yin2021center} &3D & 60.2 & 69.3  & 62.7 & 12.3 & \textbf{\textcolor{blue}{262}}  \\ \hline

\textbf{Poly-MOT}~\cite{li2023poly}      & CenterPoint~\cite{yin2021center} & 3D & 61.9 & 73.1   & \textbf{\textcolor{blue}{52.1}} & 5.6 & 281   \\ 
% \textbf{Poly-MOT}~\cite{li2023poly}      & LargeKernel3D-L~\cite{chen2022scaling}  & 3D   & \textbf{\textcolor{red}{75.2}}    & 54.1   & \textbf{\textcolor{red}{252}} \\ 

\textbf{Poly-MOT}~\cite{li2023poly}      & LargeKernel3D-L~\cite{chen2022scaling}  & 3D & 54.1 & \textbf{\textcolor{red}{75.2}}    & 54.1  & 8.6 & 252 \\

\textbf{Fast-Poly}~\cite{li2024fast}      & CenterPoint~\cite{yin2021center} & 3D & \textbf{\textcolor{blue}{63.2}}  & 73.7   &  -- & \textbf{\textcolor{blue}{28.9}} & 414   \\ \hline
 
\textbf{Easy-Poly (Ours)}       & CenterPoint~\cite{yin2021center} & 2D + 3D & \textbf{\textcolor{blue}{64.4}} & \textbf{\textcolor{blue}{74.5}}   & 54.9  & \textbf{\textcolor{blue}{34.6}} & \textbf{\textcolor{blue}{272}}   \\
\textbf{Easy-Poly (Ours)}       & LargeKernel3D~\cite{chen2022scaling}  & 2D + 3D  & \textbf{\textcolor{red}{64.8}} & \textbf{\textcolor{blue}{75.0}}    & \textbf{\textcolor{blue}{53.6}}  & \textbf{\textcolor{red}{34.9}}  & \textbf{\textcolor{red}{242}} \\ \hline
% \vspace{-4.5em}
% \setlength{\abovecaptionskip}{3pt}
% \setlength{\belowcaptionskip}{3pt}
\end{tabular}}
\end{center}
\end{table*}

\subsection{Datasets}

% The nuScenes dataset~\cite{caesar2020nuscenes} consists of 850 training and 150 test sequences, capturing a wide range of driving scenarios, including challenging weather conditions and nighttime environments. Each sequence contains approximately 40 frames, with keyframes sampled at 2Hz and fully annotated. In addition to this, it provides annotations for object-level attributes such as visibility, activity, pose, and more. It includes a large volume of RGB and point-cloud data (in PCD format). The official evaluation protocol utilizes \textbf{AMOTA}, \textbf{MOTA}, and \textbf{sAMOTA}~\cite{weng20203d} as primary metrics, evaluating performance across seven object categories: Car (\textit{Car}), Bicycle (\textit{Bic}), Motorcycle (\textit{Moto}), Pedestrian (\textit{Ped}), Bus (\textit{Bus}), Trailer (\textit{Tra}), and Truck (\textit{Tru}). Given the substantial size of the complete nuScenes dataset, users often prefer the nuScenes-mini dataset. Notably, Poly-MOT, Fast-Poly, and our proposed Easy-Poly methods exclusively utilize keyframes for tracking tasks.


The nuScenes dataset~\cite{caesar2020nuscenes} consists of 850 training and 150 test sequences, capturing a wide range of driving scenarios, including challenging weather conditions and nighttime environments. Each sequence contains approximately 40 frames, with keyframes sampled at 2Hz and fully annotated. The official evaluation protocol utilizes \textbf{AMOTA}, \textbf{MOTA}, and \textbf{sAMOTA}~\cite{weng20203d} as primary metrics, evaluating performance across seven object categories: Car (\textit{Car}), Bicycle (\textit{Bic}), Motorcycle (\textit{Moto}), Pedestrian (\textit{Ped}), Bus (\textit{Bus}), Trailer (\textit{Tra}), and Truck (\textit{Tru}). Notably, Poly-MOT, Fast-Poly, and our proposed Easy-Poly methods exclusively utilize keyframes for tracking tasks.

\subsection{Implementation Details}

% Our tracking method is implemented in Python under the Nvidia 4090X GPU.  Hyperparameters are chosen based on the best AMOTA identified in the validation set. SF thresholds are category-specific and detector-specific, which are (\textit{Bic}: 0.15; are (\textit{Car}: 0.16; are (\textit{Moto}: 0.16; \textit{Bus}: 0.12; \textit{Tra}: 0.13; \textit{Tru}: 0; \textit{Ped}: 0.13) on nuScenes on Waymo. The NMS thresholds are 0.08 on all categories and datasets. We also employ Scale-NMS~\cite{huang2021bevdet} on (\textit{Bic}, \textit{Ped}) on nuScenes. With default $IoU_{bev}$ in NMS, we additionally utilize our proposed $A\text{-}gIoU_{bev}$ to describe similarity for (\textit{Bic}, \textit{Ped}, \textit{Bus}, \textit{Tru}) on nuScenes. The motion models and filters are consistent with~\cite{li2023poly}. The lightweight filter is implemented by the median filter with $l_{lw}=5$ on all datasets. The association metrics are all implemented by $A\text{-}gIoU$ on all datasets. The first association thresholds $\theta_{fm}$ are category-specific, which are (\textit{Bic, Moto, Bus}: 1.6; \textit{Car, Tru}: 1.2; \textit{Tra}: 1.16;\textit{Ped}: 1.78) on nuScenes. Voxel mask size $\theta_{vm}$ is 5\textit{m} on nuScenes.  The count-based and output file strategies are consistent with~\cite{li2023poly}. In the confidence-based part, decay rates $\sigma$ are category-specific, which are (\textit{Ped}: 0.18; \textit{Car}: 0.26; \textit{Tru, Moto}: 0.28; \textit{Tra}: 0.22; \textit{Bic,Bus}: 0.24) on nuScenes. The delete threshold $\theta_{dl}$ are (\textit{Bus}: 0.08,  \textit{Ped}: 0.1 and 0.04 for other categories) on nuScenes.

Our tracking framework is implemented in Python and executed on an Nvidia 4090X GPU. Hyperparameters are optimized based on the highest AMOTA achieved on the validation set. The following category-specific and SF thresholds are employed for nuScenes: (\textit{Bic}: 0.15; \textit{Car}: 0.16; \textit{Moto}: 0.16; \textit{Bus}: 0.12; \textit{Tra}: 0.13; \textit{Tru}: 0; \textit{Ped}: 0.13). NMS thresholds are uniformly set to 0.08 across all categories and datasets. Additionally, we implement Scale-NMS~\cite{huang2021bevdet} for (\textit{Bic}, \textit{Ped}) categories on nuScenes.
In conjunction with the default in NMS, we introduce our novel  metric to enhance similarity assessment for (\textit{Bic}, \textit{Ped}, \textit{Bus}, \textit{Tru}) categories on nuScenes. Motion models and filters are consistent with those described in~\cite{li2023poly}. A lightweight filter, implemented as a median filter with , is applied across all datasets. Association metrics universally employ  across all datasets.
Category-specific first association thresholds  for nuScenes are as follows: (\textit{Bic, Moto, Bus}: 1.6; \textit{Car, Tru}: 1.2; \textit{Tra}: 1.16; \textit{Ped}: 1.78). The voxel mask size  is set to 5\textit{m} on nuScenes. Count-based and output file strategies align with those presented in~\cite{li2023poly}.
In the confidence-based component, category-specific decay rates for nuScenes are: (\textit{Ped}: 0.18; \textit{Car}: 0.26; \textit{Tru, Moto}: 0.28; \textit{Tra}: 0.22; \textit{Bic, Bus}: 0.24). The delete thresholds  are set as follows: (\textit{Bus}: 0.08, \textit{Ped}: 0.1, and 0.04 for all other categories) on nuScenes.

In the object tracking phase of 3D MOT, Easy-Poly exhibits exceptional performance following a series of optimizations. These enhancements include pre-processing, Kalman filtering, motion modeling, and tracking cycle refinements. The integration of these techniques significantly improves the algorithm's effectiveness in complex 3D environments.

Easy-Poly exhibits exceptional performance on the test set, achieving a \textbf{75.9\%} AMOTA score, surpassing the majority of existing 3D MOT methods. As shown in Table \ref{table:nu_test}, Easy-Poly attains a remarkably low IDS count of \textbf{287} while maintaining the highest AMOTA (\textbf{75.9\%}) among all modal methods. This underscores Easy-Poly's ability to maintain stable tracking without compromising recall. Notably, Easy-Poly achieves state-of-the-art performance without relying on additional image data input. Easy-Poly significantly outperforms competing algorithms in the critical 'Car' category. With minimal computational overhead, it delivers impressive results, highlighting its potential for integration into real-world autonomous driving systems. The False Negative and False Positive metrics in Table \ref{table:nu_test} further demonstrate Easy-Poly's robust continuous tracking capability while maintaining high recall.

For validation set experiments in Table \ref{table:nu_val}, we utilize CenterPoint~\cite{yin2021center} as the detector to ensure fair comparisons. As illustrated in Table \ref{table:nu_val}, Easy-Poly significantly outperforms most deep learning-based methods in both tracking accuracy (\textbf{75.0\%} AMOTA, \textbf{64.8\%} MOTA) and computational efficiency (\textbf{34.9} FPS). Compared to the baseline FastPoly~\cite{li2024fast}, Easy-Poly achieves substantial improvements of \textbf{+1.3\%} in MOTA and \textbf{+1.6\%} in AMOTA, while operating \textbf{1.5x} faster under identical conditions. When integrated with the high-performance LargeKernel3D~\cite{chen2022scaling} detector, Easy-Poly demonstrates even more impressive detection and tracking capabilities. Furthermore, when employing the multi-camera detector \textcolor{black}{DETR3D}~\cite{DETR3D} with constrained performance, Easy-Poly exhibits robust real-time performance without compromising accuracy. Furthermore, the lower AMOTP and IDS metrics demonstrate Easy-Poly's exceptional capability in tracking small objects and maintaining performance in complex scenarios and adverse weather conditions. These results underscore the algorithm's robustness across diverse and challenging environments.
% It is noteworthy that achieving optimal AMOTA necessitates a stringent score filter threshold, which marginally reduces the latency advantage of our method. Nevertheless, Easy-Poly maintains a favorable balance between accuracy and computational efficiency.

\begin{table}
% \vspace{0.5em}
        \caption{Comparing different data association algorithms using CenterPoint and (lines 1-7) and LargeKernel3D (lines 8-11) methods on nuScenes val set. Among them, the algorithms lines 1-3 are the Fast-Poly framework and in lines 4-11 are the latest our Easy-Poly framework.}

        \label{table:nus_assoc}
        % \renewcommand{\arraystretch}{0.7}
        \setlength{\tabcolsep}{0.1mm}
        \begin{tabular}{cccccc}
        \toprule
        \multicolumn{1}{c}{\textbf{Algorithms}} & \textbf{MOTA}$\uparrow$ & \textbf{AMOTA}$\uparrow$ & \textbf{AMOTP}$\downarrow$ & \textbf{IDS}$\downarrow$ & \textbf{FN}$\downarrow$\\ 
        
        \midrule
         %  Mutual Nearest Neighbor (MNN)
         MNN  & 62.2  & 72.5 & 52.4  & 433 & 16644  \\  
         Greedy    & 62.3  & 72.7 &  53.4  & 428  & 17647 \\
         Hungarian & \textbf{\textcolor{blue}{63.2}}  & \textbf{\textcolor{blue}{73.7}} & \textbf{\textcolor{blue}{52.1}} & \textbf{\textcolor{blue}{414}}  & \textbf{\textcolor{blue}{15996}}  \\
        
        \midrule  
        % 多模态+数据增强
         %  Mutual Nearest Neighbor (MNN)
        \textbf{MNN (Ours)}  & 63.7  & 73.6 & 54.8  & 406 &  \textbf{\textcolor{red}{15873 }}\\  
        \textbf{Greedy (Ours)}    &  64.0  & 73.7  &  54.6  & 368  & 16736 \\
        \textbf{Hungarian (Ours)}  & 64.3  & 74.3 & \textbf{\textcolor{red}{54.3}} &  335 & 16892  \\

        \textbf{DTO (Ours)}   & \textbf{\textcolor{red}{64.4}}  &   \textbf{\textcolor{red}{74.5}} & 54.9 &  \textbf{\textcolor{red}{272}} & 16982  \\

        
         \midrule
         \textbf{MNN (Ours)}  & 64.1  & 73.9 & 54.0  & 370 & 15865  \\  
         \textbf{Greedy (Ours)}    & 64.5  & 74.3 & 53.7  & 307  & 16014 \\
        \textbf{Hungarian (Ours)}  & 64.7  & 74.8 & 53.9  & 291  & 15923 \\

        \textbf{DTO (Ours)}  & \textbf{\textcolor{red}{64.8}}  & \textbf{\textcolor{red}{75.0}} & \textbf{\textcolor{red}{53.6}}  & \textbf{\textcolor{red}{242}}  & \textbf{\textcolor{red}{15488}} \\

    \bottomrule
\end{tabular}
\end{table}



% \begin{table}
% \caption{The ablation study of whether or not to use Score Filter and Non-Maximum Suppression, including the Run-Time, which represents the execution time of the Pre-processing Module. We compared Poly-MOT~\cite{li2023poly} (rows 1-3) with our proposed Easy-Poly method (rows 4-6).}
% \label{table:nu_NMSsf}
% % \renewcommand{\arraystretch}{0.7}
% \setlength{\tabcolsep}{2.7mm}
% {
% \begin{tabular}{cccc}
% \toprule

% \textbf{Variable} & \textbf{AMOTA$\uparrow$} & \textbf{IDS$\downarrow$} &
% \textbf{Run-Time (s) $\downarrow$}
% \\ 
% \midrule
% NMS + SF & \textbf{\textcolor{blue}{73.1}}   & \textbf{\textcolor{blue}{281}}  & 0.055    \\
% NMS     & 71.8  & 320  & 0.093     \\
% SF       & 68.6  & 354  & \textbf{\textcolor{blue}{0.008}}     \\ 
% \midrule
% \textbf{NMS + SF (Ours)} & \textbf{\textcolor{red}{75.0}}   & \textbf{\textcolor{red}{242}}  & 0.037    \\
% \textbf{NMS (Ours)}      & 73.6   & 273  & 0.068    \\
% \textbf{SF (Ours)}       & 71.2   & 308  & \textbf{\textcolor{red}{0.008}}     \\\hline
% % \vspace{-3.2em}
% \end{tabular}}
% \end{table}



\subsection{Comparative Evaluations} 
\label{sec: Comparative}

% Our proposed method, Easy-Poly, achieves state-of-the-art performance on the nuScenes validation set, demonstrating \textbf{75.0\%} AMOTA at \textbf{34.9} FPS, surpassing existing approaches. Utilizing an identical detector, Easy-Poly outperforms Fast-Poly \cite{li2024fast} across nearly all key metrics, with notable improvements in accuracy (\textbf{+1.3\%} AMOTA, \textbf{+1.6\%} MOTA) and speed (\textbf{+6.0 FPS}). While marginally slower than CBMOT \cite{benbarka2021score} and 3DMOTFormer \cite{ding20233dmotformer}, Easy-Poly significantly exceeds their accuracy while maintaining robust real-time performance. Our open-source implementation establishes a strong baseline for 3D MOT, providing a solid foundation for future advancements in the field.

In this study, we conduct a comprehensive evaluation of the association stage, focusing on four algorithms: Hungarian, Greedy, MNN, and the novel DTO. Our extensive experiments, summarized in Table~\ref{table:nus_assoc}, reveal that the Easy-Poly consistently outperforms Fast-Poly across both CenterPoint and LargeKernel3D frameworks. Notably, LargeKernel3D demonstrates superior performance over CenterPoint, particularly in complex tracking scenarios. Among the association algorithms, Hungarian and DTO consistently yield superior results, underscoring their robustness and efficacy in diverse multi-object tracking contexts. Compared to the Hungarian algorithm, DTO not only achieves similarly excellent AMOTA values but also provides more robust and accurate tracking performance, especially in challenging scenarios involving occlusions, missed detections, or false positives. These findings highlight the critical role of algorithm selection and model optimization in advancing the state-of-the-art in 3D object tracking.

\begin{figure*}[t]
    \centering
    \includegraphics[width=0.95\linewidth]{Images/Ablation_study_line_chart_for_NMS.pdf}
    \vspace{-8pt}
    \caption
    {
      The ablation study of whether or not to use Score Filter and Non-Maximum Suppression, including the Run-Time, which represents the execution time of the Pre-processing Module. We compared Poly-MOT with our proposed Easy-Poly method.
     }
     \Description{}
    \label{fig: NMSFS}
\end{figure*}


\textcolor{black}{Table \ref{table:nu_life} demonstrates the efficacy of our proposed methods. The Fast-Poly tracklet termination strategy (line 3) significantly outperforms baseline score refinement~\cite{benbarka2021score} (line 2), yielding a \textbf{2.7\%} improvement in AMOTA and reducing FN by \textbf{2366}. This enhancement mitigates tracker vulnerabilities in mismatch scenarios, including occlusions. Further performance gains are achieved through smoother score prediction (line 4), resulting in additional improvements of \textbf{0.4\%} AMOTA, \textbf{0.1\%} MOTA, and a reduction of \textbf{926 FN}.}
Our Easy-Poly model exhibits even more substantial performance enhancements. The tracklet termination strategy (line 7) surpasses the baseline score refinement (line 6) by \textbf{2.8\%} in AMOTA while decreasing FN by \textbf{1844}. Furthermore, the integration of smoother score prediction (line 8) further boosts the tracking performance, resulting in improvements of \textbf{0.4\%} in AMOTA and \textbf{0.6\%} in MOTA, along with a reduction of \textbf{575} FN.


\begin{table}
% \vspace{0.5em}
        \caption{A comparison on distinct life-cycle modules on nuScenes val set.
        \textbf{Average} means using the online average score to delete.
        \textbf{Latest} means using the latest score to delete.
        \textbf{Max-age} means using the continuous mismatch time to delete.
        Other settings are under the best performance. Methods in lines 1-4 and lines 5-8 use Fast-Poly~\cite{li2024fast} and Our Easy-Poly.
        }
        \label{table:nu_life}
        % \renewcommand{\arraystretch}{0.7}
        \setlength{\tabcolsep}{0.1mm}
        \begin{tabular}{ccccc}
        \toprule
        \multicolumn{1}{c}{\textbf{Strategy}} & \textbf{AMOTA}$\uparrow$ & \textbf{MOTA}$\uparrow$ & \textbf{FPS}$\uparrow$ & \textbf{FN}$\downarrow$\\ \midrule
 
         Count                \& Max-age            & 73.3      & 62.9     & 23.0  & 16523    \\
         %predict:normal update:normal+delete = -100
         Confidence~\cite{benbarka2021score} \& Latest       & 70.6      & 63.2     & \textbf{\textcolor{blue}{45.8}}  & 19192   \\%predict:minus update:multi+latest
         Confidence~\cite{benbarka2021score} \& Average      & 73.3      & 63.1     & 28.3 & 16826   \\%predict:minus update:multi+average
         Confidence  \& Average      & \textbf{\textcolor{blue}{73.7}}      & \textbf{\textcolor{blue}{63.2}}     & 28.9  & \textbf{\textcolor{blue}{15900}}   \\%predict:normal update:multi+average 27.7or28.9
         \midrule
         \textbf{Count \& Max-age (Ours)}          & 74.3      & 63.8     & 34.3  & 16024    \\
         %predict:normal update:normal+delete = -100
         \textbf{Confidence~\cite{benbarka2021score} \& Latest (Ours)}        & 71.8      & 63.7     & \textbf{\textcolor{red}{50.2}}  & 17907   \\%predict:minus update:multi+latest
         \textbf{Confidence~\cite{benbarka2021score} \& Average (Ours)}       & 74.6      & 64.2     & 35.6 & 16063   \\%predict:minus update:multi+average
         \textbf{Confidence \& Average (Ours)}      & \textbf{\textcolor{red}{75.0}}      & \textbf{\textcolor{red}{64.8}}     & 36.9  & \textbf{\textcolor{red}{15488}}   \\%predict:normal update:multi+average 36.9
        \bottomrule
        \end{tabular}
    \end{table}


\subsection{Ablation Studies}

% The Sf can filter out low-score bounding boxes while NMS can remove duplicate bounding boxes with high confidence, which makes the remaining bounding boxes have superior quality. For the Poly-MOT, using SF before NMS brings inference 40\% reduction in pre-processing inference time while boosting AMOTA by 1.3\% compared with only using NMS. The Fast-Poly has better performance, using SF before NMS brings inference 50\% reduction in pre-processing inference time while boosting AMOTA by 1.4\% compared with only using NMS, as demonstrate in Table~\ref{table:nu_NMSsf}.  It is clear from the table that when only SF is used without NMS, although the running time is fast, both AMOTA and IDS values become lower and the values drop very significantly.

We optimize two-stage filtering approach that synergistically combines SF and NMS to significantly enhance bounding box quality in multi-object tracking scenarios. This novel method effectively integrates SF to eliminate low-score detections and NMS to remove high-confidence duplicates, resulting in a set of superior quality bounding boxes. Our comprehensive experimental results, presented in Figure~\ref{fig: NMSFS}, demonstrate substantial improvements in both computational efficiency and tracking accuracy across multiple state-of-the-art models. For the Poly-MOT model, our approach of applying SF before NMS yields a remarkable \textbf{40\%} reduction in pre-processing inference time while simultaneously improving AMOTA by \textbf{1.3\%} compared to using NMS alone. These results highlight the significant potential of our method in improving real-time tracking capabilities without compromising accuracy. The Easy-Poly model exhibits even more impressive performance gains, further validating the scalability and effectiveness of our approach. By applying SF before NMS, we achieve a substantial \textbf{50\%} reduction in pre-processing time coupled with a \textbf{1.4\%} improvement in AMOTA. This notable improve in both speed and accuracy underscores the robustness of our method across different model architectures. Importantly, our analysis reveals critical insight into the interplay between SF and NMS. Although using SF without NMS accelerates processing, it leads to significant degradation in both AMOTA and Identity Switches IDS metrics. This observation underscores the crucial importance of our combined SF-NMS approach in maintaining an optimal balance between processing speed and tracking accuracy. The synergistic effect of SF and NMS not only enhances the quality of bounding boxes but also optimizes the trade-off between computational efficiency and tracking performance. This balance is particularly vital in real-world applications where both speed and accuracy are paramount, such as autonomous driving and surveillance systems. 
% The consistent improvements observed across different models suggest broad applicability and potential for integration into various tracking frameworks, paving the way for more efficient and accurate tracking systems in complex, real-world environments.


% \subsection{Visualization}





\section{Conclusion and future work}\label{sec13}
This study presents BOLIMES, a novel feature selection algorithm that combines the robustness of Boruta with the interpretability of LIME to address the challenges in gene expression classification. The proposed method has demonstrated superior performance, achieving higher accuracy compared to alternative solutions, while significantly reducing training time. Furthermore, we have identified the optimal number of features necessary to achieve the best possible accuracy. Looking ahead, future work could focus on integrating additional interpretability methods with LIME for a more comprehensive feature relevance assessment. Extending BOLIMES to multi-omics data, including proteomics and metabolomics, could enhance its applicability. Investigating deep learning-based classifiers and validating the approach on larger, diverse datasets and clinical samples would be essential for improving performance and ensuring robustness in real-world applications.














\noindent\textbf{Acknowledgements:}
Thanh MA and Thanh-Nghi DO are received support from the European Union's Horizon research and innovation program under the MSCA-SE (Marie Sk\l{}odowska-Curie Actions Staff Exchange) grant agreement 101086252; Call: HORIZON-MSCA-2021-SE-01; Project title: STARWARS.
% This work has received support from the College of Information Technology at Can-Tho University. In addition, the authors would like to thank the Big Data and Mobile Computing Laboratory very much.
% ---- Bibliography ----

\bibliographystyle{splncs04}
\bibliography{Bolimes}
\end{document}
