% This is samplepaper.tex, a sample chapter demonstrating the
% LLNCS macro package for Springer Computer Science proceedings;
% Version 2.20 of 2017/10/04
%
\documentclass[runningheads]{llncs}
%
\usepackage{url}
\usepackage{graphicx}
 \usepackage{nicematrix}

%\usepackage[small]{caption}
\usepackage{booktabs}
\urlstyle{same}
\usepackage{multirow}
\usepackage{color, colortbl}

\usepackage{pbox}
\usepackage{mathrsfs}
\usepackage[lined, ruled, linesnumbered, noend, scleft, nofillcomment]{algorithm2e}
\SetKwRepeat{Do}{do}{while}
\usepackage{amsmath,amsfonts}
\usepackage{algorithmic}
\usepackage{graphicx}
\usepackage{textcomp}
\newcommand*\rot{\rotatebox{90}}

\let\Bbbk\relax
\usepackage{newtxmath}

%-------------------
\usepackage{caption}
\usepackage[utf8]{inputenc}
\usepackage{url}
\usepackage{amsmath}
\DeclareMathOperator*{\argmax}{arg\,max}
\DeclareMathOperator*{\argmin}{arg\,min}
\usepackage{siunitx}
\usepackage{svg}
\usepackage{parcolumns} % đặt trước \begin{document}

\newcommand\I{\ensuremath{\mathcal{I}}}
\newcommand\A{\ensuremath{\mathcal{A}}}
\newcommand\T{\ensuremath{\mathcal{T}}}
\newcommand\N{\ensuremath{\mathcal{N}}}
\newcommand\ClT{\ensuremath{\mathcal{C}l}_\mathcal{T}}
\newcommand\Cl{\ensuremath{\mathcal{C}l}}
\newcommand\IRCC{\ensuremath{\mathcal{S}}}
%\newcommand\OELB{\mathcal{O}_{\mathcal{EL}_\bot}}
\newcommand\OELB{\mathcal{O}}
\newcommand\Onto{\ensuremath{\mathcal{O}}}
\newcommand\B{\ensuremath{\mathcal{B}}}
\newcommand\C{\ensuremath{\mathcal{C}}}
\newcommand\profileN{\ensuremath{\mathcal{N}}}
\newcommand{\eqdef}{\:\raisebox{1ex}{\scalebox{0.5}{\ensuremath{\mathrm{def}}}}\hskip-1.65ex\raisebox{-0.1ex}{\ensuremath{=}}\:}



\newcommand{\red}[1]{\textcolor{red}{#1}}
\newcommand{\blue}[1]{\textcolor{blue}{#1}}
\newcommand{\green}[1]{\textcolor{green}{#1}}
\newcommand{\orange}[1]{\textcolor{orange}{#1}}
\newcommand{\LightCyan}[1]{\textcolor{LightCyan}{#1}}

\newcommand{\todo}[1]{\textcolor{orange}{TODO:#1}}
\newcommand{\Thanh}[1]{\textcolor{red}{Thanh: #1}}
\newcommand{\Nghi}[1]{\textcolor{red}{Nghi: #1}}
\newcommand{\Hoa}[1]{\textcolor{red}{Hoa: #1}}
\newcommand{\Chung}[1]{\textcolor{red}{Chung: #1}}
%========================
\newtheorem{theory}{Theory}

\usepackage{scalerel,stackengine}
\stackMath
\newcommand\reallywidehat[1]{%
\savestack{\tmpbox}{\stretchto{%
  \scaleto{%
    \scalerel*[\widthof{\ensuremath{#1}}]{\kern-.6pt\bigwedge\kern-.6pt}%
    {\rule[-\textheight/2]{1ex}{\textheight}}%WIDTH-LIMITED BIG WEDGE
  }{\textheight}% 
}{0.5ex}}%
\stackon[1pt]{#1}{\tmpbox}%
}
% Used for displaying a sample figure. If possible, figure files should
% be included in EPS format.
%
% If you use the hyperref package, please uncomment the following line
% to display URLs in blue roman font according to Springer's eBook style:
% \renewcommand\UrlFont{\color{blue}\rmfamily}

\begin{document}
%
\title{BOLIMES: Boruta–LIME optiMized fEature Selection for Gene Expression Classification}
%
\titlerunning{BOLIMES: Boruta–LIME optiMized fEature Selection for Gene Expression Classification}

% If the paper title is too long for the running head, you can set
% an abbreviated paper title here
%
\author{Bich-Chung Phan \inst{1}\and Thanh Ma\inst{1}\and Huu-Hoa Nguyen\inst{1} \and Thanh-Nghi Do\inst{1}}
%
\authorrunning{Bich-Chung Phan et al.}
% First names are abbreviated in the running head.
% If there are more than two authors, 'et al.' is used.
%
\institute{Can Tho University, Can Tho, Vietnam\\
\email{\{pbchung,mtthanh,nhhoa,dtnghi\}@ctu.edu.vn}
}
%
\maketitle              % typeset the header of the contribution
%
\begin{abstract}
\begin{abstract}
Advancements in DNA sequencing technologies have significantly improved our ability to decode genomic sequences. However, the prediction and interpretation of these sequences remain challenging due to the intricate nature of genetic material. Large language models (LLMs) have introduced new opportunities for biological sequence analysis. Recent developments in genomic language models have underscored the potential of LLMs in deciphering DNA sequences. Nonetheless, existing models often face limitations in robustness and application scope, primarily due to constraints in model structure and training data scale. To address these limitations, we present \textbf{Gener}\textit{ator}, a generative genomic foundation model featuring a context length of 98k base pairs (bp) and 1.2B parameters. Trained on an expansive dataset comprising 386B bp of eukaryotic DNA, the \textbf{Gener}\textit{ator} demonstrates state-of-the-art performance across both established and newly proposed benchmarks. The model adheres to the central dogma of molecular biology, accurately generating protein-coding sequences that translate into proteins structurally analogous to known families. It also shows significant promise in sequence optimization, particularly through the prompt-responsive generation of enhancer sequences with specific activity profiles. These capabilities position the \textbf{Gener}\textit{ator} as a pivotal tool for genomic research and biotechnological advancement, enhancing our ability to interpret and predict complex biological systems and enabling precise genomic interventions. Implementation details and supplementary resources are available at \url{https://github.com/GenerTeam/GENERator}.
\keywords{DNA, Genomics, Foundation model, Generative model}
\vspace{12pt}
\end{abstract}





\keywords{Image Classification  \and Gene Expression \and Boruta \and LIME \and Feature Selection}
\end{abstract}
%
%
%
\section{Introduction}\label{sec1}

\section{Introduction}

Chain-of-Thought (CoT) prompting~\cite{Nye:2021, cot, Kojima:2022cotzero} has emerged as a cornerstone strategy for enhancing Large Language Models (LLMs) in complex reasoning tasks. By eliciting step-by-step inference, CoT enables LLMs to decompose intricate problems into manageable subtasks, thereby improving their problem-solving performance~\cite{Yao:2023tot, Wang:2023self-consistency, Zhou:2023least, Shinn:2023Reflexion}. Recent advancements, such as OpenAI's o1~\cite{o1} and DeepSeek-R1~\cite{deepseekr1}, further demonstrate that scaling up CoT lengths from hundreds to thousands of reasoning steps could continuously improve LLM reasoning. These breakthroughs have underscored CoT’s potential to advance LLM capabilities, expanding the boundaries of AI-driven problem-solving.

\begin{figure}[t]
\centering
    \includegraphics[width=0.95\columnwidth]{fig/intro.pdf}
    \caption{In contrast to vanilla CoT that generates all reasoning tokens sequentially, \method enables LLMs to \textit{skip} tokens with less semantic importance (\textit{e.g.,} \includegraphics[width=7pt]{fig/token.pdf}~) and learn shortcuts between critical reasoning tokens, facilitating controllable CoT compression.}
    \label{fig:intro}
\end{figure}

Despite its effectiveness, the increased length of CoT sequences introduces substantial computational overhead. Due to the autoregressive nature of LLM decoding, longer CoT outputs lead to proportional increases in both inference latency and memory footprints of key-value cache. Additionally, the quadratic computational cost of attention layers further exacerbates this burden. These issues become particularly pronounced when CoT sequences extend into thousands of reasoning steps, resulting in significant computational costs and prolonged response times. While prior research has explored methods for selectively skipping reasoning steps~\cite{Ding:2024cotshortcut, liu2024skipstep}, recent findings~\cite{jin:2024cotlength, Merrill:2024cotlength} suggest that such reductions may conflict with test-time scaling~\cite{o1-blog, snell2025scaling}, ultimately impairing LLM reasoning performance. Therefore, striking an optimal balance between CoT efficiency and reasoning accuracy remains a critical open challenge.

In this work, we delve into CoT efficiency and seek the answer to an important question: \textit{``Does every token in the CoT output contribute equally to deriving the answer?''} We empirically analyze the semantic importance of tokens within CoT outputs and reveal that their contributions to the reasoning performance vary, as depicted in Figure 2. Building on this insight, we introduce \method, a simple yet effective approach that enables LLMs to \textit{skip} less important tokens within CoT sequences and learn shortcuts between critical reasoning tokens, thereby allowing for controllable CoT compression with adjustable ratios. Specifically, as shown in Figure~\ref{fig:intro}, \method constructs compressed CoT training data with various compression ratios, by pruning unimportance tokens from original LLM CoT trajectories. Then, it conducts a general supervised fine-tuning process on target LLMs with this training data, facilitating LLMs to automatically trim redundant tokens during reasoning.

We conduct extensive experiments across various models, including LLaMA-3.1-8B-Instruct and the Qwen2.5-Instruct series, using two widely recognized math reasoning benchmarks: GSM8K and MATH-500. The results validate the effectiveness of \method in compressing CoT outputs while maintaining robust reasoning performance. Notably, Qwen2.5-14B-Instruct exhibits almost \textbf{NO} performance drop (less than $0.4\%$) with a $\bm{40\%}$ reduction in token usage on GSM8K. On the challenging MATH-500 dataset, LLaMA-3.1-8B-Instruct effectively reduces CoT token usage by $\bm{30}\%$ with a performance decline of less than $4\%$, resulting in a $\bm{1.4}\times$ inference speedup. Further analysis underscores the coherence of \method in specified compression ratios and its potential scalability with stronger compression techniques.

\method is distinguished by its low training cost. For Qwen2.5-14B-Instruct, \method fine-tunes only 0.2\% of the model's parameters using LoRA. The size of the compressed CoT training data is no larger than that of the original training set, with 7,473 examples in GSM8K and 7,500 in MATH. The training is completed in approximately 2 hours for the 7B model and 2.5 hours for the 14B model on two 3090 GPUs. These characteristics make \method an efficient and reproducible approach, suitable for use in efficient and cost-effective LLM deployment.

To sum up, our key contributions are:
\begin{enumerate}
    \item To the best of our knowledge, this work is the \textit{first} to investigate the potential of enhancing CoT efficiency through \textit{token skipping}, inspired by the varying semantic importance of tokens in CoT trajectories of LLMs.
    \item We introduce \method, a simple yet effective approach that enables LLMs to skip redundant tokens within CoTs and learn shortcuts between critical tokens, facilitating CoT compression with adjustable ratios.
    \item Our experiments validate the effectiveness of \method. When applied to Qwen2.5-14B-Instruct, \method reduces reasoning tokens by $40\%$ (from 313 to 181) on GSM8K, with less than a $0.4\%$ performance drop.
\end{enumerate}

%=================================================================================


\section{Background}
In this section, we present the foundation of our approach with integrating Boruta and Lime. We also provide the ML models that will be used for the gene expression classification task.
\subsection{Gene Expression Classification with ML models}
Gene expression classification \cite{do2024enhancing,do2023ensemble,huynh2019novel} lies at the forefront of biomedical research, offering profound insights into the molecular mechanisms underlying various diseases. ML models have become indispensable in this domain, as they can uncover complex patterns within vast and high-dimensional gene expression datasets. However, these datasets often contain a plethora of features, many of which are redundant or irrelevant, potentially obscuring the most critical biological signals and leading to overfitting. Consequently, feature selection becomes imperative—it refines the dataset by isolating the most informative genes, thereby enhancing model accuracy, interpretability, and computational efficiency. By focusing solely on the pivotal biomarkers, this research is able to achieve more reliable predictive outcomes. In this paper, we investigate and evaluate the classification with various ML techniques. Namely, we experiment our selected features with ML algorithms, i.e., SVM \cite{vapnik1995support}, Random Forest \cite{breiman2001random}, XGB \cite{chen2015xgboost}, Gradient Boosting \cite{friedman2002stochastic}.

\begin{definition}[Classification]
Let \( D = (X, y) \) be a dataset where \( X \subseteq \mathbb{R}^n \) is the feature space and \( y \in \mathcal{Y} = \{1,2,\dots,k\} \) represents the class labels. A classifier is a function
\[
f: X \to \mathcal{Y},
\]
that assigns a predicted label \( \hat{y} = f(x) \) to each input \( x \in X \). The function \( f \) is learned from the labeled examples
\[
D = \{(x_i, y_i) \mid x_i \in X,\; y_i \in \mathcal{Y},\; i = 1, \dots, N\},
\]
by minimizing a loss function \( \ell: \mathcal{Y} \times \mathcal{Y} \to \mathbb{R}_{\ge 0} \) that quantifies the error between the predicted and true labels. Once trained, \( f \) is used to classify new, unseen inputs.
\end{definition}

% \begin{definition}[Classification Using Machine Learning]
% Let \( D_{\text{selected}} = (X_{\text{selected}}, y) \) be the dataset with features \( X_{\text{selected}} \subseteq X^* \) as determined by LIME. A classifier is a function 
% \[
% f: X_{\text{selected}} \to \mathcal{Y},
% \]
% that assigns a predicted label \( \hat{y} = f(x) \) to each input \( x \in X_{\text{selected}} \). The classifier is trained on the labeled examples
% \[
% D_{\text{selected}} = \{(x_i, y_i) \mid x_i \in X_{\text{selected}},\; y_i \in \mathcal{Y},\; i = 1, \dots, N\},
% \]
% by minimizing a loss function \( \ell: \mathcal{Y} \times \mathcal{Y} \to \mathbb{R}_{\ge 0} \) that measures the discrepancy between the predicted and true labels. The trained classifier is then used to predict the classes of new, unseen instances.
% \end{definition}


Feature selection is crucial before classification begins. Our study focuses on two techniques: Boruta and LIME. 
% Boruta is chosen for its robustness in identifying all relevant features in high-dimensional datasets, ensuring no important predictor is missed. LIME is used for its ability to provide interpretable, local explanations of model predictions, which is essential for evaluating feature importance. 
We now introduce Boruta and LIME in the following sections.

\subsection{Leveraging Boruta for Robust Feature Extraction}
Boruta \cite{kursa2010boruta,zhou2023diabetes} is a powerful wrapper-based feature selection algorithm designed to identify all truly relevant variables in a dataset. By comparing the importance of actual features with that of randomly generated ``shadow'' features, Boruta systematically filters out irrelevant variables while preserving essential predictors. This rigorous selection process is particularly valuable in high-dimensional applications, such as gene expression classification, where capturing meaningful signals is crucial. For clarity, we formally define Boruta as follows:
\begin{definition}[Boruta Feature Selection]
Let \( D = (X, y) \) be a dataset with features \( X = \{x_1, x_2, \dots, x_p\} \) and target \( y \). The Boruta algorithm identifies all relevant features in \( X \) as follows:
\begin{enumerate}
    \item \textbf{Shadow Feature Generation:} For each \( x_i \in X \), create a shadow feature \( x_i^{\text{shadow}} \) by randomly permuting its values, forming the set \( X^{\text{shadow}} \).
    \item \textbf{Importance Estimation:} Train a classifier (e.g., Random Forest) on the combined set \( X \cup X^{\text{shadow}} \) and compute the importance score \( I(z) \) for each \( z \).
    \item \textbf{Feature Comparison:} For each \( x_i \), define
    \[
    I^{\text{shadow}}_{\max} = \max_{z \in X^{\text{shadow}}} I(z).
    \]
    Then classify \( x_i \) as \emph{relevant} if \( I(x_i) \) is significantly greater than \( I^{\text{shadow}}_{\max} \), \emph{irrelevant} if significantly lower, or \emph{tentative} otherwise.
    \item \textbf{Iteration:} Remove irrelevant and tentative features and repeat until all features are decisively classified.
\end{enumerate}
The final selected subset \( X^* \subseteq X \) comprises all features deemed relevant.
\end{definition}

After applying the Boruta algorithm, we retain only the relevant features (confirmed) and excluded the tentative and irrelevant features (rejected). To further enhance the selection of features in \(X^*\), we employed the AI explanation technique outlined in the following section.

% \begin{definition}[Boruta Feature Selection]
% Given a dataset \( D = (X, y) \) with original features \( X = \{ x_1, x_2, \dots, x_p \} \), Boruta augments \( X \) by creating shadow features \( X^{\text{shadow}} = \{ x_1^{\text{shadow}}, \dots, x_p^{\text{shadow}} \} \) via random permutation. A model \( M \) (e.g., Random Forest) is then trained on \( X \cup X^{\text{shadow}} \) to compute importance scores \( I(z) \) for every feature \( z \). For each \( x_i \in X \), if \( I(x_i) \) is significantly greater than the maximum shadow importance \( I^{\text{shadow}}_{\max} = \max_{z \in X^{\text{shadow}}} I(z) \), then \( x_i \) is marked as relevant; otherwise, it is rejected or considered tentative. Iterating this process yields the final set of selected features \( X^* \subseteq X \).
% \end{definition}

\subsection{XAI for Feature Selection}
Explainable AI (XAI) \cite{dwivedi2023explainable,zacharias2022designing} represents a forefront of AI research, aiming to elucidate the decision-making processes of complex models. In the context of gene expression classification, where feature selection is pivotal to model performance and interpretability, our study leverages LIME—Local Interpretable Model-Agnostic Explanations—to demystify and select critical features. LIME approximates the behavior of a sophisticated, black-box model with a simpler, locally interpretable surrogate, thereby pinpointing the most influential predictors in the vicinity of a given instance. This approach enhances the transparency of the model's predictions and facilitates a more informed and rigorous feature selection process, ultimately contributing to both improved accuracy and trustworthiness of the classification system.  Now, we provide a formal definition of LIME as follows:

% \begin{definition}[LIME-based Feature Selection]
% Let \( D = (X, y) \) be a dataset and \( f: X \to \mathcal{Y} \) a trained black-box classifier, where \( X \subseteq \mathbb{R}^p \) and \( \mathcal{Y} = \{1,2,\dots,k\} \). For a given instance \( x \in X \), LIME constructs an interpretable surrogate model \( g \) from a simple model class \( G \) (typically linear), expressed as
% \[
% g(z) = w_0 + \sum_{j=1}^{p} w_j z_j.
% \]
% The surrogate \( g \) is fitted by minimizing the weighted loss
% \[
% \min_{g \in G} \sum_{z \in Z_x} \pi_x(z) \left( f(z) - g(z) \right)^2 + \Omega(g),
% \]
% where \( Z_x \) is a set of perturbed samples around \( x \), \( \pi_x(z) \) is a proximity measure between \( z \) and \( x \), and \( \Omega(g) \) is a regularization term enforcing simplicity. The absolute coefficients \( |w_j| \) quantify the local importance of each feature, thus guiding feature selection.
% \end{definition}
\begin{definition}[LIME-based Feature Selection]
Let \( D^* = (X^*, y) \) be the dataset resulting from Boruta, where \( X^* \subseteq \mathbb{R}^{p^*} \) is the set of relevant features. Given a trained black-box classifier \( f: X^* \to \mathcal{Y} \) and an instance \( x \in X^* \), LIME constructs an interpretable surrogate model \( g \in G \) (typically linear), expressed as
\[
g(z) = w_0 + \sum_{j=1}^{p^*} w_j z_j,
\]
by solving the optimization problem
\[
\min_{g \in G} \sum_{z \in Z_x} \pi_x(z) \left( f(z) - g(z) \right)^2 + \Omega(g),
\]
where \( Z_x \) is a set of perturbed samples in the neighborhood of \( x \), \( \pi_x(z) \) is a proximity measure, and \( \Omega(g) \) enforces simplicity. The absolute coefficients \( |w_j| \) indicate the local importance of each feature, enabling a further refined selection \( X_{\text{selected}} \subseteq X^* \) for classification.
\end{definition}


To clarify, our choice of LIME for feature selection arises from the critical question of determining the optimal number of features for the model. In this context, assessing the local importance of each vector proves to be the most effective strategy, leading us to introduce the BOLIMES algorithm. The following section will provide a comprehensive explanation of the BOLIMES algorithm and its application.

%--------------------






%=================================================================================
\section{BOLIMES algorithm}\label{lb-ViFDFramwork}
\hspace{0.5cm}
Our methodology begins with applying the Boruta algorithm to sift through the high-dimensional gene expression dataset, effectively filtering out irrelevant features and isolating those that are truly significant. This initial reduction is critical because LIME, our subsequent interpretability tool, involves generating numerous perturbed samples and calculating distances—a computationally intensive process, especially in large feature spaces. By narrowing the feature set, we improve LIME's precision and enhance the model's interpretability, ultimately boosting classification performance.

However, the key question remains: how many features should be selected for optimal classification? In our proposed BOLIMES algorithm, which integrates Boruta and LIME, we first reduce the high-dimensional feature set by eliminating irrelevant variables with Boruta. Then, we further refine this subset using LIME to assess the local importance of each feature. Finally, we determine the optimal number of features by evaluating classification performance—selecting the subset that yields the highest accuracy for model training. In general, our model is both efficient and robust, relying only on the most informative features for gene expression classification.

\begin{algorithm}[h]
\scriptsize
    \DontPrintSemicolon
    \SetKwInOut{Input}{Input}
    \SetKwInOut{Output}{Output}
    
    \Input{Dataset \( D = (X, y) \) with \( X \subseteq \mathbb{R}^p \) and class labels \( y \).}
    \Output{Optimal feature subset \( X_{\text{opt}} \) and trained classifier \( f_{\text{opt}} \).}
    
    \Begin{
        \( X^* \leftarrow \text{Boruta}(D) \) \tcp*{\scriptsize Identify relevant features from \( X \)}
        
        \( \mathcal{I} \leftarrow \text{LIME}(f, X^*) \) \tcp*{\scriptsize Compute local importance scores on \( X^* \)}
        
        \( X^{*}_R \gets \{ x^*_{(i)} \}_{i=1}^{|X^*|} \) \quad  \( \mathcal{I}(x^*_{(1)}) \geq \dots \geq \mathcal{I}(x^*_{(|X^*|)}) \) \tcp*{\tiny \textbf{Rank} features in \( X^* \) in descending order of \( \mathcal{I} \)}
        
        \( \text{best\_acc} \leftarrow 0 \), \( k^* \leftarrow 0 \)\;
        
        \For{\( k = 10 \) \KwTo \( |X^*| \)}{
            \( S_k \gets \{ x^*_{(i)} \}_{i=1}^{k} \) \tcp*{\scriptsize Select top-\( k \) features from \( X^*_R \)}
            \( f_k \leftarrow \text{TrainClassifier}(D_{S_k}) \) \tcp*{\scriptsize Train classifier on \( D_{S_k} \)}
            \( \text{acc} \leftarrow \text{Evaluate}(f_k, D_{S_k}) \) \tcp*{\scriptsize Compute classification accuracy}
            
            \If{\( \text{acc} > \text{best\_acc} \)}{
                \( \text{best\_acc} \leftarrow \text{acc} \)\;
                \( k^* \leftarrow k \)\;
                \( f_{\text{opt}} \leftarrow f_k \)\;
            }
        }
        
        \( X_{\text{opt}} \gets \{ x^*_{(i)} \}_{i=1}^{k^*} \) \tcp*{\scriptsize Select top-\( k^* \) features from \( X^*_R \)}
        
        \Return{\( X_{\text{opt}}, f_{\text{opt}} \)}
    }
    
    \caption{Optimal Feature Selection for Classification using Boruta and LIME}
    \label{alg:BOLIMES}
\end{algorithm}

% The algorithm optimally selects features for classification by integrating Boruta and LIME. First, Boruta extracts a relevant subset \( X^* \) from \( X \). LIME then assigns importance scores \( \mathcal{I} \) to \( X^* \), producing a ranked set \( X^*_R \) where \( \mathcal{I}(x^*_{(1)}) \geq \dots \geq \mathcal{I}(x^*_{(|X^*|)}) \).  For \( k = 1 \) to \( |X^*| \), the algorithm selects the top-\( k \) features \( S_k \), trains a classifier \( f_k \), and evaluates its accuracy. The best-performing model determines the optimal feature count \( k^* \), yielding \( X_{\text{opt}} = \{ x^*_{(i)} \}_{i=1}^{k^*} \) and classifier \( f_{\text{opt}} \). This method ensures feature selection maximizes classification accuracy while maintaining interpretability.


Algorithm \ref{alg:BOLIMES} presents an optimal feature selection framework by integrating Boruta and LIME to refine feature subsets for classification. Given a dataset \( D = (X, y) \), it first applies Boruta to extract a subset \( X^* \) of relevant features. LIME then computes local importance scores \( \mathcal{I} \) for \( X^* \), producing a ranked set \( X^*_R \) where
\[
X^*_R \gets \{ x^*_{(i)} \}_{i=1}^{|X^*|}, \quad \mathcal{I}(x^*_{(1)}) \geq \dots \geq \mathcal{I}(x^*_{(|X^*|)}).
\]
An iterative search determines the optimal number of features \( k^* \) by selecting the top-\( k \) ranked features, training a classifier \( f_k \), and evaluating its accuracy:
\[
S_k \gets \{ x^*_{(i)} \}_{i=1}^{k}, \quad f_k \leftarrow \text{TrainClassifier}(D_{S_k}).
\]
Note that we begin with 
$k=10$ and increase it until 
$|X^*|$, as using a smaller number of vectors would be inefficient. The best-performing classifier defines \( k^* \), yielding the final subset \( X_{\text{opt}} \) and trained model \( f_{\text{opt}} \). 


The algorithm's complexity is driven by three key stages: Boruta for feature selection, LIME for ranking, and iterative classification. Boruta, relying on multiple iterations of Random Forest, has a worst-case complexity of \( O(T \cdot p^2 \log p) \). LIME, which perturbs \( m \) samples per feature, contributes \( O(m \cdot p) \). The final stage trains classifiers iteratively over \( p^* \) ranked features, leading to an overhead of \( O(n p^*{}^2) \) assuming a model with \( O(n k) \) complexity. Thus, the total complexity is:
\(
O(T \cdot p^2 \log p) + O(m \cdot p) + O(n p^*{}^2),
\)
where \( p^* \ll p \) in practice, making the approach feasible for moderate-dimensional data but computationally intensive for extremely large \( p \).







% % \section{Proposed Algorithms}\label{lb-algorithm}

% \hspace{0.5cm}
% \input{Structures/4-Algorithm}

\section{Experiment and results}
\label{lb-experimentalresult}
\hspace{0.5cm}
\section{Experiments}

\subsection{Models}
To comprehensively evaluate on LMMs, we conducted zero-shot inference across both commercial and open-source models. Our evaluation suite includes leading commercial models GPT-4o~\cite{hurst2024gpt40} and Gemini1.5-Pro~\cite{Gemini} alongside state-of-the-art open-source alternatives of varying scales: Qwen2.5-VL~\cite{qwen2.5-VL}, Qwen2-VL~\cite{wang2024qwen2}, LLaVA-v1.6~\cite{liu2023llava}, CogVLM~\cite{wang2023cogvlm}, MiniCPM-o-2.6~\cite{yao2024minicpm}, mPlug-Owl2~\cite{ye2023mplugowl2}, InternVL2v5~\cite{chen2024internvl},LLaVA-NEXT-Video~\cite{zhang2024llavanextvideo} and Cambrian~\cite{tong2024cambrian1}. Besides, Janus-Pro~\cite{chen2025januspro}, which unifies multimodal understanding and generation, is included to test the abilities between Unified Model and Vision Language Model.  This diverse selection enables us to analyze how model scale, architecture, and training approaches influence comic comprehensive capabilities. 


% Detailed specifications and inference configurations for each model are provided in Appendix~\ref{appendix:model}.

\subsection{Experimental Details}
% 隐含含义理解和预测帧内容这两个任务都是选择题,因为他们的标准答案是不唯一的很难衡量。如果模型选择了正确的答案选项,则认为是正确的,也就是说accuracy是主要的metric。 而对于排序任务,这一个任务形式非常新颖,对于一个comic strip,其输入顺序可以随便打乱,且答案是确切的。所以这个任务我们既采用了选择题的题型又使用了问答题。
The task prompts is displayed in Table ~\ref{prompt}. For visual narrative comprehension task, model is provided with the whole image. But for next-frame prediction and multi-frame sequence reordering task, LMMs infer with image sequences.
The hyper-parameters for each LMMs in the experiments including possible settings are detailed in Appendix~\ref{appendix:hyper-param}. Furthermore, to assess human capabilities in these tasks, we randomly select 100 questions from the dataset for each task and instruct human evaluators to answer. This allows us to benchmark the performance of human participants against our models, offering a thorough comparison of both human and LMMs proficiency in these specific tasks. 


\subsection{Main Results}
Our comprehensive evaluation reveals that while LMMs show promising capabilities in comprehension and prediction tasks, they significantly underperformed in sequence reordering tasks. Moreover, there remains a substantial performance gap between current models and human performance across all tasks. Unified Model underperformed than Vision Language Model.

\paragraph{Contextual Frame Prediction}
The frame prediction task appears to be the most tractable among the three tasks. GPT-4o achieves the highest score of $69.95\%$, followed closely by Qwen2-VL at $64.00\%$.This demonstrates that the performance gap between closed and open-source models is relatively small for this task. However, Janus-Pro perform notably below expectations ($27.50\%$), possibly due to its unified model architectural.

\paragraph{Visual Narrative Comprehension}
For visual narrative comprehension, we observe a similar pattern but with generally lower scores. GPT-4o leads with $61.60\%$, while other models show varying degrees of capability. 

\paragraph{Temporal Narrative Reordering}
The frame reordering task proves to be the most challenging, with all models performing significantly below human capability. Even the best-performing models struggle to exceed $30\%$ accuracy, with many achieving scores around $25-26\%$, which is slightly higher than random selection. 
Notably, several models (marked with *) are unable to perform this task due to their architectural limitations in processing multiple images simultaneously. For these models, we attempted to accommodate their single-image constraint by concatenating multiple frames horizontally into a single image, with white margins serving as frame boundaries. However, this workaround appears to be suboptimal, as these models likely struggle to properly distinguish individual frame boundaries and maintain the semantic independence of each frame, ultimately leading to their poor performance on the reordering task. 

The poor performance on reordering task suggests that current LMMs, regardless of their scale or architecture, have not yet developed robust capabilities for understanding temporal relationships and sequential logic in visual narratives.








\section{Conclusion and future work}\label{sec13}
This study presents BOLIMES, a novel feature selection algorithm that combines the robustness of Boruta with the interpretability of LIME to address the challenges in gene expression classification. The proposed method has demonstrated superior performance, achieving higher accuracy compared to alternative solutions, while significantly reducing training time. Furthermore, we have identified the optimal number of features necessary to achieve the best possible accuracy. Looking ahead, future work could focus on integrating additional interpretability methods with LIME for a more comprehensive feature relevance assessment. Extending BOLIMES to multi-omics data, including proteomics and metabolomics, could enhance its applicability. Investigating deep learning-based classifiers and validating the approach on larger, diverse datasets and clinical samples would be essential for improving performance and ensuring robustness in real-world applications.














\noindent\textbf{Acknowledgements:}
Thanh MA and Thanh-Nghi DO are received support from the European Union's Horizon research and innovation program under the MSCA-SE (Marie Sk\l{}odowska-Curie Actions Staff Exchange) grant agreement 101086252; Call: HORIZON-MSCA-2021-SE-01; Project title: STARWARS.
% This work has received support from the College of Information Technology at Can-Tho University. In addition, the authors would like to thank the Big Data and Mobile Computing Laboratory very much.
% ---- Bibliography ----

\bibliographystyle{splncs04}
\bibliography{Bolimes}
\end{document}
