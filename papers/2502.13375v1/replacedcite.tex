\section{Related work}
\label{sec:related}

In the last decade the random process introduced by Schelling has been theoretically analyzed  in the computer science literature, e.g.,  
____. The expected size of the resulting segregated neighbourhoods was shown to be polynomial in the size of the neighbourhood on the line ____ and exponential in its size on the grid ____. 
Zhang ____ proposed to study the changes in a neighbourhood as a strategic game played by two types of agents  rather than as a random process. Different utility functions for the agents were studied in ____, and generalizations to $k$ types of agents were studied in ____.
Agarwal et al. ____ studied jump games  for $k$ types of agents in which agents always seek to increase the fraction of agents of their own type. They modelled location preferences by considering two 
classes of agents: those in the first class  are strategic and aim to maximize their utility, while those in the second class are stubborn, not 
changing their initial location.
The utility function considered by Agarwal et al. ____ for swap games is similar to that of Elkind et al. ____. 
A new measure of global diversity, called the  {\em degree of integration}, was introduced there 
and  results on the price of anarchy and stability with respect to this measure were given. 
Kreisel et al. ____ showed that determining the existence of equilibria is NP hard in jump and swap games even if all the agents are strategic. The influence of the underlying topology and locality in swap games was studied in Bil{\`o} et al. ____.

Diversity hedonic games were studied in ____ in which some agents have homophilic preferences, while others have heterophilic preferences. However, hedonic games are different to our model as in hedonic games agents form pairwise disjoint coalitions while in our model, the neighbourhoods of different agents may overlap. 


 A few recent papers  ____,
like our work, are motivated by the observation that real-world agents 
can favour diverse neighbourhoods. Swap games with a single-peaked utility function were studied in ____. The utility of an agent  increases monotonically with the fraction of
 same-type neighbours in the interval $[0, \lambda]$ for some $0<\lambda<1$, and decreases monotonically afterwards. Results on the existence of equilibria for specific classes of graphs for different ranges of $\lambda$, as well as tight bounds on the price of anarchy and stability with respect to the degree of integration defined in ____ were shown. 
Using the  single peak utility function for jump games, the authors of ____ investigated the existence of equilibria. 
They showed that improving response cycles exist independently of the peak value,
 even for graphs with very simple structures. They also showed that while the existence of equilibria are not guaranteed  even on rings for $\lambda\geq 1/2$, there are  conditions under which they are guaranteed to exist, depending on the size  of the independent set, the number of empty nodes, and numbers of agents of each type set.
In addition, bounds on the price of stability and anarchy with respect to the degree of integration and some hardness results were given. 
Kanellopoulos et al. ____ considered jump games with $k \geq 2$ types of agents and  an implicit ordering of the types.  The utility of an agent is affected by the distance between the types in the given ordering  whenever there are agents of those types in its neighborhood. 
In ____, the authors studied jump games in which the utility of an agent is the fraction of neighbours of a different type from itself.  It is shown that  to determine the existence of an equilibrium in the presence of stubborn agents is NP-hard, and the game is a potential game in regular graphs, and spider graphs, while there is always an equilibrium in trees.  











%%%%%%%%%%%%%%%%%%%%%%%%%%