\section{Related Work}
%  Suggested change ==> Split the Related Work into two distinct parts: attribute prediction and profile updatin

User profiling is essential for personalized systems \cite{shen2005implicit, yao2020employing, zhu2008impact}, enabling platforms to deliver tailored recommendations \cite{balog2019transparent, lu2015exploiting, middleton2004ontological} and computational social media analysis \cite{arunachalam2013new, bamman2014gender, tang2015learning, al2020comparative}. Over time, profiling methods have evolved from rule-based systems to machine learning models, and now to LLMs, which offer improved accuracy and adaptability \cite{bloedorn1998using, wu2024understanding}. 

Early research treated user profiling as a \textit{classification problem}, focusing on predicting fixed attributes like gender \cite{liu2012using, rao2010classifying, liu2013s, sakaki2014twitter, priadana2020gender}, age \cite{rosenthal2011age, sap2014developing, chen2015comparative, fang2015relational, mac2017demographic, bessarab2025social}, and political views \cite{rao2010classifying, demszky2019analyzing, hettiachchi2021us}. used hierarchical classification to predict gender from Twitter data, leveraging linguistic features. \cite{liu2013using, ciot2013gender} expanded these methods to demographic and multilingual contexts, demonstrating the generalizability of text-based profiling. However, these models were limited by their static nature, as they couldn’t capture evolving user behaviors.

To address this, researchers began integrating social network data. \citet{al2012homophily} used \textit{homophily}—the tendency of people with similar traits to connect—to predict political affiliations. \citet{onikoyi2023gender} combined social interactions with text features to enhance gender prediction accuracy. Despite these advancements, these approaches still struggled to adapt to changes in user preferences over time.

To improve profiling, hybrid model emerged, combing textual, social and behavioral data. \citet{agichtein2006improving} showed that clickstream data and user interactions could enhance search algorithms. \citet{agarwal2013collaborative} and \citet{priadana2020gender} improved friend recommendation systems by integrating both content and network features. In marketing, \citet{bucklin2003model} and \citet{shah2021marketing} used browsing and purchasing behaviors to target ads effectively.

While dynamic profiling addressed many challenges, the introductions of LLMs revolutionized the field. LLMs can generate context-aware, nuanced profiles and adapt to evolving user behaviors. \citet{wu2024understanding} analyzed the role of user profiles in personalizing large language models, revealing that historical personalized responses are key to effective personalization and that profile placement at the beginning of the input context has a greater impact.

While significant progress has been made in developing model based techniques in user profiling, such as deep learning and reinforcement learning approaches, there is a notable lack of open source datasets to support comprehensive evaluation and benchmarking. This gaps limits reproducibility and hinders further advancements in the field. To address this, we propose tow novel open source datasets: one for user profile construction, providing diverse user information for generating accurate profile, and another for profile updating, capturing temporal changes in user behavior to evaluate dynamic profiling models. These datasets aim to facilitate transparent, reproducible research and drive innovation in user profiling.


\begin{figure}[ht]
  \centering
  \includegraphics[width=\linewidth]{words.pp.png}
  \caption{Top 10 Most Frequent Words. This bar chart illustrates the ten most frequently occurring words in the dataset, highlighting key terms such as "university," "born," and "first." The word "university" appears most frequently (4,545 times), followed by "born" (3,932) and "first" (3,614). The distribution suggests a strong focus on biographical and educational information within the user profiles.}
  \label{fig:bar_org}
\end{figure}