ComposeOn is not merely an innovative tool for music composition; it represents a shift in the intersection of music education and technology. Through its user-friendly design and multi-level theoretical guidance, ComposeOn effectively lowers the barriers traditionally associated with music creation, making the process accessible to those without formal musical training. For users lacking a background in music theory, ComposeOn offers a platform that fosters incremental learning, enabling them to transform simple melodic ideas into fully developed compositions through interactive guidance.

Our findings indicate that ComposeOn demonstrates clear superiority over generative AI-based tools such as Suno in terms of musical structure and coherence. While Suno may generate sonically complex and multi-layered compositions, its output often lacks logical progression and structural integrity, especially upon repeated listening. In contrast, ComposeOn produces music that is not only more consistent and cohesive but also aligns with established principles of music theory. Moreover, the high level of explainability embedded in ComposeOn significantly enhances users' understanding of fundamental concepts such as chord progressions and rhythmic patterns, blending theory with practice to provide a more immersive and effective learning experience.

A key strength of ComposeOn lies in its ability to elevate users' confidence and motivation to create music. Through its highly visual and interactive interface, users can actively engage with the composition process, taking control of melodic development and exploring various harmonic and rhythmic options. The system’s flexibility allows for real-time adjustments, empowering users to make informed creative decisions. This interactive approach not only increases user engagement but also transforms them from passive recipients of AI-generated music into active composers with a deeper sense of creative ownership.

Furthermore, ComposeOn’s multi-tiered explanation system caters to users of varying skill levels, offering appropriate theoretical insights at each stage of their development. This layered approach ensures that users, whether beginners or intermediate learners, can progress at their own pace, reinforcing long-term knowledge retention through the immediate application of theoretical concepts. The integration of music theory into the creative process bridges the gap between abstract learning and practical application, thereby enhancing the overall educational value of the platform.
