In the post-quesionnaire, we analyzed the music theory correctness and all the subjective assessments in the form of 5-point scores and free comments, Figure~\ref{fig:result} reflects the results of scores for ComposeOn and the baseline method.
\begin{figure}[h]
\centering
\includegraphics[width=0.8\linewidth]{figs/musicresult.png}
\caption{5-point result on the music theory correctness and the subjective assessments.}
\label{fig:result}
\end{figure}

\subsection{ComposeOn develops better music than Suno}
First, the music generated by Suno exhibits more richness and complexity in terms of weaving and orchestration. As noted in P2, P4, and P10, this complexity makes Suno's music potentially more appealing on first listen. One participant might describe it this way, “Suno's music feels rich on first listen, with multiple layers of sound and rich orchestration, giving a first impression of great depth. However, after multiple listens, Suno's music feels less logical. In contrast, ComposeOn's music, although given only melodies with no instrumental layers, feels comfortable after multiple listens because it is very logical.”

However, ComposeOn received high ratings for other aspects of music quality. Multiple participants (P1, P2, P9) emphasized the better structure of the music generated by ComposeOn.P1's comment was particularly specific: “The music generated by Suno lacks structure and sounds like randomly assembled pieces. It sounds like randomly assembled fragments because it has no clear development or ending. In contrast, I could hear (and see) the beginning and end of each phrase of ComposeOn's music, and each phrase ended in a similar pattern, making the music feel more holistic.”

In terms of musical coherence, several participants (P3, P6, P8) noted that ComposeOn showed better consistency in musical sequences.P3's observation was particularly insightful: “Sometimes Suno repeats chords that have been entered previously, but when you hear certain parts of the music, Suno sometimes generates random segments that have little to do with the previous text. This interspersing makes for very little musical consistency in this continuation. By contrast, the musical consistency in ComposeOn's continuation is stable; instead of outputting all the input melody at once and writing it all by itself at once, he assigns features of the input melody to different phrases, so that each phrase has parts that relate to the input melody but are not entirely consistent with it.” This approach makes the music generated by ComposeOn more coherent and natural. In addition, P6 points out the stability of the tempo: “The music written by Suno sometimes accelerates or decelerates suddenly, whereas ComposeOn's tempo is more stable.” This is further evidence of ComposeOn's strength in maintaining musical consistency.

Finally, ComposeOn shows more variation and innovation in its music writing, as articulated in P5's review, “Suno continues music that feels like it's repeating the same chords and melodies without advancing or developing; ComposeOn at least gives me a sense that the music is evolving because it doesn't all sound too much like the previous input.ComposeOn at least gives me ComposeOn at least gives me a sense that the music is evolving, as it doesn't all sound too much like the previous input. It demonstrates ComposeOn's ability to introduce new elements while maintaining musical coherence, making the music more layered and diverse.”

\subsection{ComposeOn increased participants' willingness and confidence to create music}

Our findings indicate that ComposeOn significantly enhanced participants' willingness and confidence in music creation. This improvement can be attributed to two key features of the system: high visualization and interactivity, and high explainability.

\subsubsection{High Visualization and Interactivity}

ComposeOn's interface provides a highly visual and interactive experience, which proved particularly beneficial for novice users. The system displays both user input and AI-recommended melodies on a traditional musical staff, offering real-time visual cues during playback. This feature allows users, especially beginners, to gain a deeper understanding of musical structure, including how notes form chords and how chord progressions work.

The system's interactivity is evident in its user-guided recommendation process. Participants can control the flow of suggestions using the "continue" and "end" buttons, giving them agency over the composition process. Furthermore, ComposeOn offers contextually appropriate chord and rhythm options for each measure, allowing users to experiment with different musical elements. For instance, one participant reported successfully changing a 1-4-5-1 chord progression to a 1-2-5-1 progression, demonstrating the system's flexibility and its capacity to facilitate learning through experimentation.

\subsubsection{High Explainability}

ComposeOn's high explainability feature significantly contributes to users' understanding and confidence. The system provides detailed explanations for its musical continuations, covering aspects such as chord progression, rhythm, and ornamental notes. These explanations are tailored to three proficiency levels - beginner, intermediate, and expert - ensuring that users receive information appropriate to their knowledge level.

The granularity of explanations is noteworthy; users can request explanations for individual notes, measures, or entire phrases. This feature not only clarifies the system's recommendations but also reinforces fundamental musical concepts. As one participant noted, "This feature not only helped explain why certain recommendations were made, but it also helped me remember basic information like chord composition more clearly. I could associate the sound of a chord in a phrase with its composition, an experience that's hard to achieve when learning chords in isolation."

Additionally, the integrated "music theory mentor" feature provides quick access to theoretical knowledge, offering suggested terms and concepts like the circle of fifths. This feature's effectiveness was demonstrated when a participant (P1) first encountered an explanation of the I-VI-V-I chord progression and then used the music theory mentor to gain a deeper understanding of why this progression is harmonically pleasing.

In conclusion, ComposeOn's combination of high visualization, interactivity, and explainability creates a supportive environment for music creation. By demystifying the composition process and providing immediate, context-specific feedback, the system empowers users to engage more confidently with music creation, regardless of their initial skill level. This approach not only facilitates the creation of music but also enhances users' overall musical understanding, potentially contributing to long-term skill development and musical appreciation.