Our system, ComposeOn, is designed to facilitate music extension and learning for users with little or no musical theory background. By analyzing the results of the formative study mentioned in Section \ref{sec:Formative Study}, we identified three design goals for ComposeOn:

\begin{itemize}
\item Melody Expansion: Enable users to easily extend and develop their melodic ideas into complete compositions.
\item Easy Editing: Allow users to effortlessly edit and control their music creations.
\item Music Theory Integration: Incorporate music theory learning to boost engagement and interest.
\end{itemize}

To generate the extended part of music based on the input melody, our system consists of three main components: the input \& analysis module, the generation module, and the output \& explanation module. Figure \ref{fig:system_design_diagram} shows a high-level diagram of the system design.
\begin{figure}[h]
\centering
\includegraphics[width=0.8\linewidth]{figs/system_design.png}
\caption{ComposeOn System Design Diagram: the input and analysis module, represented by light-blue color; the generation module, represented by light-green color; and the output and explanation module, represented by light-yellow color.}
\Description{ComposeOn System Design Diagram with three modules}
\label{fig:system_design_diagram}
\end{figure}

\subsection{ComposeOn Database}

The ComposeOn database consists of two main parts: common chord progressions and common rhythm patterns. These data provide ComposeOn with fundamental materials and references.

The \textbf{chord progression section} contains 39 different chord sequences across seven categories, covering various music styles from basic to advanced. These chord progressions are selected based on the principles of Functional Harmony and Tonal Harmony in music theory\cite{kostka2008harmonia,clendinning2016musician}. Specifically, it includes (1) 9 classic chord progressions, such as \texttt{I–IV–V–I} and \texttt{vi–IV–V–I}, common in pop and rock music; (2) 9 extended chord progressions, like \texttt{Imaj7–ii7–V7–Imaj7}, suitable for jazz and blues; (3) 4 diminished triad progressions, such as \texttt{i–iidim–V7–i}, used to create tense or dissonant effects; (4) 4 augmented fourth chord progressions, like \texttt{I–IV–aug4–I}, used to add harmonic color; (5) 5 mixed chord progressions, such as \texttt{Imaj7–ii7–V7–IVmaj7}, blending different types of chords; (6) 4 substitute chord progressions, like \texttt{Imaj7–bIImaj7–V7–Imaj7}, common in modern jazz; and (7) 4 cycle chord progressions, such as \texttt{Imaj7–ii7–V7–iii7}, used to create repeating sections or build-ups. The selection of these chord progressions also references popular music composition practices and jazz theory\cite{levine2011jazz,mulholland2013berklee}.

The \textbf{rhythm pattern section} contains 16 different rhythm types, each composed of specific combinations of notes and rests~\cite{r9}. These patterns cover various popular music styles, including pop, rock, reggae, ska, jazz, funk, blues, and country music. For example, the first pattern \texttt{[(1, 'rest'), (1, 'note'), (1, 'rest'), (1, 'note')]} represents a simple, regular rhythm, while the seventh pattern \texttt{[(1/3, 'note'), (1/3, 'note'), (1/3, 'note')] * 4} represents a more complex, triplet-based rhythm. These patterns provide ComposeOn with a rich selection of rhythms, enabling it to generate melodies that conform to specific musical styles.

\subsection{Input and Analysis Module}

To allow users to easily express their musical ideas, we have designed a flexible input module. Users can input their melodies by singing, humming, or playing an instrument. To ensure an accurate capture of users' musical creativity, we employ the advanced Basic Pitch library~\cite{r10}. This neural network-based pitch detection model can accurately convert users' audio input into standard MIDI file format, handling even complex polyphonic melodies with precision.

Once the user's input is converted into a MIDI file, the analysis module begins its work, delving deep into the musical characteristics of the melody. We use the powerful MusicPy~\cite{r11} library to process this MIDI file, extracting rich musical information. First, ComposeOn identifies the implied \textbf{chord progressions} in the melody, for example, the chords present in the user's input, recognizing them as D and G.. This provides insight into the harmonic foundation of the melody. Next, ComposeOn uses the detect\_scale function to determine the \textbf{scales} most likely used in the melody, for example, the system determines that the scale being used is D major.. To standardize the melody's harmonic structure, ComposeOn converts the identified chords into \textbf{scale degrees}, within the context of D major: D is identified as the I (tonic) chord, and G is identified as the IV (subdominant) chord. This process entails matching chords with detected scales, determining their position within the scale, and addressing special cases for minor scales. This conversion provides a more abstract, theoretical representation of the harmonic structure, allowing for a unified generation logic to be applied regardless of the original scale or key.

Finally, ComposeOn integrates all these analysis results, creating a detailed profile of the input melody's characteristics, and serving as a basis of the explanations. The standardized scale degree representation enables ComposeOn to apply consistent generation algorithms across various musical contexts.

\subsection{Generation Module}

The Generation Module is responsible for creating new musical content based on the analysis of the user's input. This process is divided into three main steps: recommending chord progressions, recommending rhythms for right-hand rhythm, and adding ornaments and variations. The system consults its chord progression database and identifies a common progression pattern that incorporates the input chords: [I - IV - V - I];[I - IV - ii - V - I];  [I - IV - vii dim - I];  [I - IV - V7 - I] etc.

\textbf{Step 1: Recommending Chord Progressions}

In the first step, we begin with chord degrees from the previous step of the Analysis Module and find the recommended chord degrees. When the user clicks the "Continue" button in the UI, we sequentially identify the most similar progressions, utilizing the SequenceMatcher class from the difflib library ~\cite{pythondifflib} to calculate the sequence similarity between the input chord progression and our predefined progressions through their chord degrees.  Each click recommends a complete progression. This recommended progression is then combined with the original input progression to form a new progression base. On subsequent "continue" clicks, this combined progression serves as the new input. 

Then, we convert these degrees into absolute chords based on the scales retained from earlier steps of the Analysis Module. For instance, if our recommended chord degree sequence is [I, IV, V, I], and the retained scale is D major, the resulting chord progression would be [D, G, A, D]. This determined chord progression forms the basis for both the left-hand and right-hand melodies.

For the left-hand part, we utilize the triads of these chords, playing them as whole notes for each measure. In the case of [D, G, A, D], the left hand would play D-F\#-A (D major triad) for the first measure, G-B-D (G major triad) for the second measure, A-C\#-E (A major triad) for the third measure, and D-F\#-A (D major triad) for the fourth measure. This approach provides a solid harmonic foundation using simple, sustained chords in the left hand.

Meanwhile, the same chord progression serves as the foundation for developing the right-hand melody, allowing for more intricate melodic and rhythmic patterns that complement the underlying harmonic structure. For example, the right-hand melody might incorporate arpeggios or scalar passages derived from the D major scale, with emphasis on the chord tones of each underlying harmony.

This method of chord realization and melody generation demonstrates how the system can translate abstract music theory concepts into concrete musical elements. By providing a clear harmonic foundation in the left hand and a related but more elaborate melody in the right hand, the system creates a balanced and musically coherent output that is accessible to novice users while still adhering to established musical principles.

Furthermore, this approach can be extended to the other progression options identified earlier:

\begin{enumerate}
    \item For [I, IV, ii, V, I] in D major: [D, G, Em, A, D]
    \item For [I, IV, vii°, I] in D major: [D, G, C\# dim, D]
    \item For [I, IV, V7, I] in D major: [D, G, A7, D]
\end{enumerate}


To introduce uniqueness, we then apply variations to these selected progressions. This variation technique is based on the concept of chord substitution in music theory. According to Levine (2011), chords can often be substituted with chords that share common tones or have a similar function within the key\cite{levine2011jazz}. Our implementation focuses on diatonic substitutions, where chords are replaced by others from the same key, maintaining harmonic coherence while introducing variety. It's worth noting that we treat each complete chord progression as a musical phrase, providing a structural basis for subsequent melody generation.

\textbf{Step 2: Recommending Rhythms for the right-hand melody}

In the second step, we employ the rhythm pattern pool introduced in the Section of the ComposeOn database. We begin by fitting the input rhythm to this rhythm pattern database to find the closest match and randomly chose two more patterns in our rhythm pattern pool. For each complete chord progression, we then apply the following strategy for rhythm patterns: The first measure of each phrase always uses the rhythm pattern fitted to the input. This ensures that the generated music retains the rhythmic characteristics of the original input, maintaining musical coherence. For subsequent measures within the phrase, we select the randomly chosen two rhythm patterns. This approach both maintains a connection to the original input and introduces new variations, enhancing the richness and diversity of the music.

\textbf{Step 3: Adding Ornaments and Variations}

In this final step, we enrich the melody by adding musical ornaments and variations. We focus on adding ornaments to only the right-hand melody, randomly selecting 5\% of the notes for embellishment. The ornaments are chosen to be as close as possible to the chord tones of the current triad. This approach includes ornaments such as appoggiaturas (grace notes creating brief dissonance before resolving to the main note), mordents (rapid alternations between the main note and an adjacent note), and trills (quick alternations between two adjacent notes)~\cite{adler1989study}. To implement this, we first determine the total number of notes in the right-hand melody and calculate 5\% of this total to decide how many notes will receive ornaments. We then randomly select these positions within the melody. For each chosen note, we identify the nearest chord tone based on the current harmony and select an appropriate ornament type. When adding the ornaments, we ensure they complement the harmonic structure and maintain the overall flow of the melody. This method effectively increases the expressiveness and complexity of the melody while preserving its essential character and harmonic integrity. Care is taken to use ornaments judiciously, avoiding overuse that might disrupt the melody's fluency, and to ensure their application aligns with the specific musical style and period being emulated.

By combining these three steps - chord progression generation with variations, flexible rhythm pattern application, and ornament addition - our Generation Module creates musically rich and varied content based on the user's input. This multi-faceted approach ensures that the generated music is harmonically sound, rhythmically interesting with a balance of familiarity and novelty, and melodically expressive. This provides users with inspiring and unique musical ideas that both respect the original input and introduce new musical elements.
\subsection{Output and Explanation Module}

The output module writes the generated melody to a new MIDI file, which can be played back to the user or further edited and refined. The explanation component of this module provides insights into the composition at three levels of complexity: Beginner, Intermediate, and Advanced. These explanations focus on three key aspects of the generated melody: chord progressions, rhythm patterns, and embellishments.

\textbf{Beginner Level} The explanation starts with an introduction to basic musical concepts. For chord progressions, it introduces the concept of chords as groups of notes played together, explaining the difference between major and minor chords, and showing how the melody notes relate to these underlying chords. The rhythm explanation at this level covers basic note durations such as quarter notes and eighth notes, and introduces common time signatures like 4/4 and 3/4. It demonstrates how the melody's rhythm fits into these basic patterns. Regarding embellishments, the beginner explanation introduces the concept as extra notes that decorate the main melody, providing simple examples like grace notes or trills.

\textbf{Intermediate Level} This level deepens the musical analysis. For chord progressions, it explains common sequences (e.g., I-IV-V-I) and introduces the concept of harmonic function (tonic, dominant, subdominant), discussing how the chosen progression supports the melody. The rhythm explanation at this level delves into how the input rhythm was matched to one of the 16 predefined patterns and how the other three random variations were created. It also discusses how these rhythms relate to different musical styles. The embellishment explanation introduces more complex decorative techniques like arpeggios or turns, and explains how these relate to the underlying harmony.

\textbf{Advanced Level} At this level, the explanation provides a sophisticated analysis of the composition. The chord progression explanation discusses advanced harmonic concepts such as secondary dominants or modal interchange, explains any modulations or key changes, and analyzes how the chord progression contributes to the overall structure of the piece. The rhythm explanation covers complex concepts like syncopation or polyrhythms, explaining how the rhythm interacts with the harmonic rhythm and contributes to the overall feel or genre of the piece. For embellishments, the advanced explanation discusses techniques like counterpoint or voice leading, explaining how embellishments can create tension and release in the melody and how they contribute to the overall expressiveness of the piece.

What's more, ComposeOn also incorporates a \textbf{Music Theory Mentor Chatbot}, powered by the advanced ChatGPT-4 model, to enhance the user's learning experience and provide on-demand musical expertise. Within the recommendation rationales provided by the system, specialized musical terminology is hyperlinked. When a user clicks on one of these hyperlinked terms, the query is automatically populated in the Music Theory Mentor's input field, situated in a dedicated section of the interface. This mechanism facilitates immediate access to additional information, enabling users to explore complex musical concepts without disrupting their creative flow.

\subsection{ComposeOn User Interface (UI)}
The ComposeOn UI, as shown in Figure~\ref{fig:musicUI}  is a user-centric platform designed to facilitate seamless interaction between users and the ComposeOn composition system. This interface integrates melody continuation, editing, and explanatory functionalities, providing users with a comprehensive music creation and learning environment.

\begin{figure}[h]
\centering
\includegraphics[width=0.8\linewidth]{figs/musicUIillu.png}
\caption{ComposeOn UI illustration. Step1-2, choose a file, and process the file as MIDI. Step3-4, when click on the continue button, a new progression will be added. Step 5-6, click one bar or some notes to check the explanation. Step 7, click the hyperlink for quick chat with a chatbot powered by ChatGPT 4. Step8-9, check the alternative rhythms and chords suitable for the selected bar.}
\Description{ComposeOn System Design Diagram with three modules}
\label{fig:musicUI}
\end{figure}

At the top of the interface, users can dynamically adjust two primary parameters: Beats Per Minute (BPM) and Explanation Level. These controls allow real-time modification of playback speed and the depth of musical analysis provided, respectively.

The composition process begins with the user uploading an audio file in either .mp3 or .wav format, initiating the Upload Module. Subsequently, activating the "Process MIDI" function triggers the Analysis Module, which visualizes the uploaded melody as a musical score on the interface. This score is interactive, allowing playback with synchronized highlighting of the current measure. Concurrently, a virtual piano display in the lower left corner visually represents the notes being played.

Upon user initiation of the "Continue" function, the Generate Module extends the composition, producing a complete chord progression. This extension is seamlessly integrated into the existing musical score visualization. When a user selects any measure of the generated continuation, the system displays explanations calibrated to the preset Explanation Level. These explanations feature hyperlinked musical terminology, enabling users to access more detailed information in the Music Theory Mentor section on the right side of the interface.

For users wishing to edit the generated music, the interface offers measure-specific editing capabilities. By selecting a measure, users can access dropdown menus for "Common Progression Degrees" and "Common Rhythms". These menus present contextually appropriate chord progression degrees and rhythmic patterns, respectively. As users modify these parameters, the system dynamically updates both the musical score and the accompanying explanations, providing immediate feedback on the musical implications of their choices.
