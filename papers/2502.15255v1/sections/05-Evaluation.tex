A user study was conducted to evaluate the effectiveness and educational value of ComposeOn. The purpose of the study was to compare the quality and experience of two groups of people - those with no knowledge of music theory and those with some knowledge of music theory - when using ComposeOn and a benchmark method (the Suno music continuation feature) for English lyrics continuation. We paid particular attention to changes in participants' knowledge of music theory. This study aims to answer the following question:
\textbf{Q1:} Does ComposeOn \textbf{\textit{develop}} music better than Suno?
\textbf{Q2:} Does ComposeOn increase more \textbf{\textit{willingness and confidence}} of participants to develop and compose music? 

To better collect and validate the results, we had participants fill out both a pre-study and a post-study questionnaire. the pre-study questionnaire included their demographic information, a simple test of their level of knowledge of music theory, as well as their confidence and motivation about composing and learning to compose. the post-study questionnaire included their judgment of the quality of the continued music, as well as a few indicators of their judgment in SUS ~\cite{r23}. The post-study questionnaire included a quality rating of the continued music, as well as SUS indicator questions, a music theory questionnaire similar to the pre-study questionnaire, and a change in their motivation and confidence about composing.

\subsection{Participant}

\begin{table}[h]
\centering
\begin{tabular}{|c|c|c|c|}
\hline
Participant ID & Music Theory Level & Compose Freq. & Compose Willingness \\
\hline
1 & Beginner & Never & High \\
2 & Intermediate & Rarely & Medium \\
3 & Advanced & Weekly & Very High \\
4 & Beginner & Monthly & Low \\
5 & Intermediate & Daily & High \\
6 & Beginner & Yearly & Medium \\
7 & Advanced & Weekly & High \\
8 & Intermediate & Never & Low \\
9 & Beginner & Rarely & Very High \\
10 & Advanced & Daily & Medium \\
\hline
\end{tabular}
\caption{Participant Information on Music Theory and Composition}
\label{tab:musicinfo}
\end{table}

\subsection{Procedure}
\textbf{Pre-Study Questionnaire}
A pre-study questionnaire is a survey that is used prior to the start of a study or program to gather background information and initial musical knowledge about the participant. The questionnaire usually contains the following questions: \textbf{Basic information}: e.g., name, age, etc. \textbf{Relevant experience}: e.g. previous experience in music making. \textbf{Assessment of prior knowledge}: Tests the participant's current knowledge of the research topic, such as music theory. \textbf{Interest and Motivation}: To find out the participants' level of interest in the topic and their motivation to learn. \textbf{Self-efficacy}: To assess the participant's confidence in his/her ability in the domain.

\textbf{Main Study}
is to have the participants randomly pick one of the 9 melodies we prepared, and then have the participants continue the melody using the ComposeOn and baseline methods. Our baseline is the Suno 3.5 model, participants need to upload the melody file, and Suno will generate at most 4 minutes of continuation. In addition, participants can use ComposeOn to continue the melody, there is no time limit requirement, and users can check the reason for continuing the melody during the process of continuing the melody, make changes to the melody, check the knowledge of music theory through the music theory mentor, and so on.

\textbf{Post-Study Questionnaire}
The post-study questionnaire contained the same \textbf{music theory questions} as the pre-study questionnaire, which was designed to test whether the user's knowledge of music theory increased after using ComposeOn and suno for melodic continuation. In addition, there are questions about the ability of the two instruments to increase compositional \textbf{confidence and motivation}, as well as questions about the use of ComposeOn in the \textbf{SUS framework}, which are about user experience.


