To understand the needs and challenges faced by individuals with little to no music theory knowledge in composing music, we conducted a formative study with six participants (FP1-FP6), aiming at exploring their willingness and confidence in music composition, as well as their experiences with existing music composition tools, the demographics and basic music composition information are shown in \ref{tab:music_composition}.
\begin{table}[h]
\centering
\resizebox{\textwidth}{!}{%
\begin{tabular}{|c|c|c|c|p{5cm}|}
\hline
\textbf{Labels} & \textbf{Music Theory Level} & \textbf{Compose Willingness} & \textbf{Compose Confidence} & \textbf{Compose Tools} \\
\hline
FP1 & Intermediate & Moderate & Moderate & Suno AI \\

FP2 & Beginner & High & Low & GarageBand, Suno AI \\

FP3 & Beginner & High & Low & Logic Pro, Fruity Loops \\

FP4 & Intermediate & High & Moderate & Suno AI \\

FP5 & Beginner & Moderate & Low & Never \\

FP6 & Intermediate & High & Low & Udio \\
\hline
\end{tabular}%
}
\caption{Music Composition Profile of Participants}
\label{tab:music_composition}
\end{table}

Most participants expressed a strong desire to compose music, with many having attempted to use various music composition tools in the past. However, they encountered significant barriers due to the tools' requirements for \textbf{basic music theory knowledge}. For instance, FP2 mentioned, \textit{"I tried GarageBand~\cite{r1}, but I struggled with identifying piano keys and understanding orchestration techniques".} Other composition tools, such as Logic Pro~\cite{r2} ~\cite{r4}and Fruity Loops Studio~~\cite{r3}~\cite{r5}, were also cited as challenging for beginners due to their requirements for music theory knowledge.

Recent developments in AI-powered music generation, such as Udio~\cite{r8} and Suno AI~\cite{r7}~\cite{r6}, were acknowledged by participants as interesting music composition tools. However, users expressed concerns about the \textbf{low interpretability} and \textbf{limited control} in AI music generation, leading to a sense of disconnection from the creative process. FP2 remarked, \textit{"It's fascinating to see what the AI can produce, but I often feel like I'm just pressing buttons rather than truly composing."} Rather than feeling like they were composing music themselves, participants viewed these tools more as a form of entertainment or game. FP1 elaborated on this sentiment, stating, \textit{"It's fun to play around with, but I don't feel like I'm learning or improving my musical skills."} These characteristics also \textbf{hindered users' from learning music theory} from AI-generated compositions and did not increase their their own music composition confidence. FP4 noted, \textit{"Suno can generate and extend music, but I feel limited in my ability to adjust the output or incorporate my own creative ideas".} This sentiment highlights a gap between the capabilities of AI-generated music and the desire for personal creative input. FP6 further emphasized this point, saying, \textit{"I want to understand why certain musical choices are made, but the AI doesn't provide that insight. It's like being given a finished painting without learning how to mix colors or use brushstrokes."} Several other participants echoed similar concerns, emphasizing the importance of maintaining creative agency and the ability to learn and grow as musicians through the composition process.

Despite their lack of formal music theory knowledge, all participants reported frequently experiencing melodic inspirations. FP5 expressed, \textit{"I often have tunes in my head that I'd love to develop into full songs, but I don't know where to start".} This desire to \textbf{expand on their melodic ideas and potentially create complete songs} was a common theme among participants. However, the participants generally lacked the motivation to undertake extensive music theory study, from understanding notes to learning chord progressions, as a prerequisite to composition. FP3 stated, \textit{"The idea of studying music theory from scratch before I can start composing is daunting and discouraging".} Interestingly, when presented with the concept of learning music theory gradually through the composition process, participants showed increased enthusiasm. FP5 commented, \textit{"If I could learn about music while actually creating something, I'd be much more motivated to stick with it".}

This formative study revealed a clear need for a music composition tool that caters to beginners or intermediate music theory learners, allowing them to \textbf{translate their melodic ideas into compositions} without extensive prior knowledge. Additionally, the generated music should be \textbf{easy for users to edit}, giving them control over their creations. Furthermore, it highlighted the potential value of integrating \textbf{music theory learning} into the composition process itself, potentially increasing user engagement and long-term commitment to music creation.