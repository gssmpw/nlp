Music composition has long been recognized as a significant and widely appreciated art form. With the advent of the digital age, there has been a growing desire among individuals lacking formal musical training to express themselves through music creation. This demand has driven the development of various digital audio workstations (DAWs) and music production software. However, despite technological advancements, many existing tools still present substantial barriers to entry and fail to adequately address the needs of novice users.

Current systems and software often fall short of providing sufficient support for novice users in music composition. Traditional DAWs, while powerful, can be overwhelmingly complex for learners with limited musical skills ~\cite{r16}. These tools typically require users to possess a solid foundation in music theory, familiarity with musical notation, and proficiency in navigating intricate software interfaces. This complexity gives rise to three primary challenges. Firstly, the high technical threshold makes it difficult for beginners to master these tools ~\cite{r14}. Secondly, existing tools often struggle to effectively translate novice users' creative ideas into musical compositions  ~\cite{r15}. Lastly, these software applications frequently fail to effectively impart additional music theory knowledge to users ~\cite{r17}. Collectively, these factors contribute to a significant gap between individuals' desire to create music and their actual ability to do so, particularly for those without formal musical training ~\cite{r13}. This disparity not only impedes personal creative expression but may also potentially diminish the diversity and innovation in music creation  ~\cite{r21}. Consequently, the development of more intuitive, user-friendly, and learning-supportive music creation tools has emerged as a crucial research direction  ~\cite{r16}.

To address this challenge, we present ComposeOn, a music theory-based tool designed specifically for users with little to no music knowledge. ComposeOn empowers users to easily extend and develop their melodic ideas into complete compositions. Additionally, the tool offers easy editing, which allows users to effortlessly edit and control their music creations. By integrating music theory, ComposeOn explains the music creation from the music theory perspectives in three different levels: beginner, intermediate, and advanced.

Our user study (N=10) compared participants composing music using ComposeOn and the baseline method, Suno AI, to explore the effectiveness of ComposeOn in helping novice users create music. The results show that ComposeOn provides a more accessible and enjoyable composing and learning experience for individuals with little musical skill. 

In our discussion, we explored how ComposeOn leverages music theory to enhance music creation for beginners. Unlike generative AI tools, ComposeOn provides a structured, educational approach, encouraging users to actively engage with musical concepts. The tool's melody input feature allows users to develop their ideas into complete compositions, fostering creativity and personal expression. Additionally, ComposeOn's multi-tiered explainability enhances learning by offering tailored insights into music theory, promoting deeper understanding and critical thinking. This innovative approach positions ComposeOn as both a composition aid and a powerful music education platform, bridging the gap between theory and practice.

This study highlights the contributions of ComposeOn in addressing user dissatisfaction with generative music tools. Our user study findings indicate that many users find generative music lacking in coherence and personal expression. ComposeOn bridges this gap by integrating music theory with practical composition, allowing users to engage deeply with the creative process. The tool's focus on melody input and explainability supports users in developing their musical ideas while enhancing their understanding of music theory. This approach not only improves user satisfaction but also fosters a more meaningful music creation experience.

