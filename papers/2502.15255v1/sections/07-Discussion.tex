\subsection{Generative and Music Theory-Based Composition}

The emergence of generative AI music tools like Suno has transformed digital music creation. These tools use deep learning algorithms to generate complex, multi-layered compositions with minimal user input ~\cite{briot2020deep}. They excel at creating diverse musical textures, particularly for background music or ambient sounds. However, our findings align with recent research indicating that while generative tools may produce impressive initial results, they often lack the musical coherence and logical structure many listeners expect, especially upon repeated listening ~\cite{carnovalini2020computational,r22}.

The challenges faced by generative AI in music creation stem from fundamental differences between music and language structures. While AI has made significant progress in natural language processing and generation, music composition presents unique challenges. As Herremans et al. (2017) point out, musical patterns, particularly in chord progressions and melodic structures, are more constrained and rule-bound than language patterns ~\cite{herremans2017functional,r13}. This inherent structure in music makes it crucial for composers, especially beginners, to have a basic understanding of these musical rules and structures.

Compared to generative approaches, music theory-based tools like ComposeOn are better suited for users who want to understand and actively participate in the composition process. These tools are particularly beneficial for beginners and intermediate learners who wish to develop musical skills while creating. ComposeOn's structured approach follows established music theory principles, which may produce simpler but more coherent and logically clear compositions. This approach aligns with pedagogical research emphasizing the importance of active learning in music education ~\cite{wright2010informal,paule2017music}.

Recent studies highlight the potential of theory-based composition tools in enhancing musical understanding and creativity. For example, Paule-Ruiz et al. (2017) found that students using theory-based composition software showed significant improvements in understanding musical concepts and creating original compositions ~\cite{paule2017music}. This supports the view that tools like ComposeOn can serve not only as composition aids but also as effective learning platforms.

While generative AI tools have opened new possibilities for music creation, our analysis suggests that theory-based approaches retain important value, particularly for educational purposes and those seeking to create more structured compositions. As the field evolves, we may see increasing convergence of these approaches, potentially leading to tools that offer both the creative freedom of generative AI and the guiding structure of music theory.

\subsection{The Importance of Melody Input for Beginners}

ComposeOn's melody input feature plays a crucial role in making music creation accessible to beginners. By allowing users to input their melodic ideas, whether through singing or simple note input, ComposeOn bridges the gap between musical inspiration and realization. This feature addresses a key finding from our preliminary research: many novice composers have melodic ideas but lack the technical knowledge to develop them ~\cite{r14,r15}.

Melody's input serves as a starting point, providing the system with a framework to build upon. It allows users to see their ideas transformed into structured musical works, enhancing their sense of ownership and creative engagement. Moreover, this feature helps users understand how their initial ideas fit into larger musical structures, promoting learning about harmony, rhythm, and form in a practical, hands-on way ~\cite{r19}.

By using user-provided melodies as the basis for composition, ComposeOn also ensures that the final product retains personal character, addressing concerns about AI-generated music lacking individual creativity ~\cite{r20}. This approach strikes a balance between AI assistance and user creativity, particularly suitable for beginners who want to learn while creating.

The importance of melody input for novice composers has been recognized in various music creation systems. For instance, the MusicMaker system developed by Chuan and Chew ~\cite{chuan2007} allows users to input melodies via MIDI keyboard or singing, which are then automatically harmonized. Similarly, the Hyperscore system ~\cite{farbood2004} enables users to draw melodic contours that are then converted into musical phrases, making composition accessible to those without formal musical training ~\cite{r18}.

In the mobile app domain, SongSmith ~\cite{simon2008} and Bean Academy ~\cite{r16} pioneered the approach of inputting melodies through singing, automatically generating accompaniments based on the user's vocal input. More recently, AI-driven systems like AIVA ~\cite{hadjeres2017} have also incorporated melody input features, allowing users to provide initial ideas that are then expanded by AI.

These examples from the literature highlight the widespread recognition of melody input as a key feature in making music creation more accessible to beginners. ComposeOn builds on this established principle, combining it with advanced AI capabilities to provide a comprehensive learning and creation environment for novice composers.

\subsection{The Role of Explainability in Music Learning}

A significant advantage of ComposeOn over other music creation and learning tools is its high level of explainability. This feature not only functionally distinguishes ComposeOn from other generative AI tools but, more importantly, transforms the system into an interactive music learning environment. Explainability has become increasingly important in AI and machine learning, especially in educational and creative applications ~\cite{gunning2019,r17}. ComposeOn applies this trend to music education, pioneering a new paradigm of learning.

ComposeOn's explainability features serve a dual purpose: they help users understand the composition process while teaching them music theory. This approach closely aligns with modern teaching theories, particularly those emphasizing learning by doing and understanding the reasons behind rules and conventions ~\cite{kolb2014experiential}. By providing detailed, context-specific explanations, ComposeOn enables users to see the decisions made during the composition process and understand the musical logic behind these decisions. This transparency not only enhances users' understanding of music creation but also cultivates their critical thinking skills, enabling a deeper comprehension of musical structures and theory.

Another key advantage of ComposeOn is its multi-tiered explanation feature (beginner, intermediate, expert). This design allows the system to remain relevant as users' musical knowledge progresses, enabling ComposeOn to serve as a long-term companion for music learning and composition. This approach aligns with cognitive load theory, which emphasizes that learning materials should be adjusted according to the learner's expertise level to optimize learning outcomes ~\cite{sweller2019cognitive}. By providing explanations tailored to the user's current level, ComposeOn ensures that the learning process is challenging but not overwhelming, creating an ideal learning environment.

Furthermore, ComposeOn's explainability features not only impart technical knowledge but also foster users' musical intuition and creativity. By explaining why certain musical choices work well, the system helps users develop their own creative judgment. This approach supports constructivist learning theory, which posits that learners actively construct knowledge by combining new information with existing knowledge structures ~\cite{ormrod2020human}. In ComposeOn, users are not passive recipients of information but active participants in the knowledge construction process, contributing to deeper and more lasting learning ~\cite{r21}.

ComposeOn also effectively bridges the gap between music theory and practice through its explainability features. By explaining theoretical concepts in the context of actual composition, users can immediately see the application of these concepts. This immediate feedback and application helps deepen understanding and significantly improves knowledge retention ~\cite{hattie2007power}. By connecting abstract music theory concepts with concrete music creation practices, ComposeOn creates a comprehensive learning environment that facilitates the translation of theoretical knowledge into practical skills.

The multi-tiered explanation system of ComposeOn allows users to customize their learning experience according to their needs and interests. This personalized approach aligns with self-determination theory, which emphasizes learners' agency and autonomy in the learning process ~\cite{ryan2020self}. By allowing users to choose the complexity of explanations, ComposeOn empowers learners to control their own learning pace, enhancing learning motivation and engagement.

ComposeOn's explainability features not only enhance its value as a composition tool but also transform it into a powerful music learning platform. By seamlessly integrating theoretical knowledge with practical application, providing personalized learning experiences, and fostering critical thinking and creativity, ComposeOn offers users a comprehensive music education environment. This innovative approach not only supports long-term learning and creation processes from beginner to expert but also has the potential to revolutionize how music education is delivered, making the learning process more interactive, personalized, and effective.