%%%% Small single column format, used for CIE, CSUR, DTRAP, JACM, JDIQ, JEA, JERIC, JETC, PACMCGIT, TAAS, TACCESS, TACO, TALG, TALLIP (formerly TALIP), TCPS, TDSCI, TEAC, TECS, TELO, THRI, TIIS, TIOT, TISSEC, TIST, TKDD, TMIS, TOCE, TOCHI, TOCL, TOCS, TOCT, TODAES, TODS, TOIS, TOIT, TOMACS, TOMM (formerly TOMCCAP), TOMPECS, TOMS, TOPC, TOPLAS, TOPS, TOS, TOSEM, TOSN, TQC, TRETS, TSAS, TSC, TSLP, TWEB.
% \documentclass[acmsmall]{acmart}

%%%% Large single column format, used for IMWUT, JOCCH, PACMPL, POMACS, TAP, PACMHCI
% \documentclass[acmlarge,screen]{acmart}

%%%% Large double column format, used for TOG
% \documentclass[acmtog, authorversion]{acmart}

\documentclass[manuscript]{acmart}
%\documentclass[sigconf]{acmart}

\usepackage{float}
\usepackage{subfigure}
%% \BibTeX command to typeset BibTeX logo in the docs
\AtBeginDocument{%
  \providecommand\BibTeX{{%
    \normalfont B\kern-0.5em{\scshape i\kern-0.25em b}\kern-0.8em\TeX}}}
\usepackage{tabularx}
\usepackage{bbding}
\usepackage{graphicx}
\usepackage{geometry}
\usepackage{subfigure}
\usepackage{amsmath}
\usepackage{makecell}
\usepackage{float}
\usepackage{color}
\usepackage{booktabs}
\usepackage{caption}
\usepackage{tabu}
\usepackage{hyperref}

%% Rights management information.  This information is sent to you
%% when you complete the rights form. CHANGE THIS:
\copyrightyear{2025}
\acmYear{2025}
\setcopyright{acmlicensed}\acmConference[CHI '25]{Proceedings of the 2025 CHI Conference on Human Factors in Computing Systems}{April 26-- May 1, 2025}{Yokohama, Japan}
\acmBooktitle{Proceedings of the 2025 CHI Conference on Human Factors in Computing Systems (CHI '25), April 26-- May 1, 2025, Yokohama, Japan}
\acmDOI{XXXXXXX.XXXXXXX}

%% These commands are for a PROCEEDINGS abstract or paper.
\newcommand{\remove}[1]{{\color{red} \sout{#1}}}
%\newcommand{\change}[1]{{\color{blue} \uwave{#1}}}
\newcommand{\add}[1]{{\color{blue} #1}}
%%
%% Submission ID.
%% Use this when submitting an article to a sponsored event. You'll
%% receive a unique submission ID from the organizers
%% of the event, and this ID should be used as the parameter to this command.
%\acmSubmissionID{123-A56-BU3}
% \author{Zheng Chen$^{*}$}

% \email{zchenin@connect.ust.hk}

% \affiliation{
%   \institution{Hong Kong University of Science and Technology}
%   \city{Hong Kong}
%   \country{Hong Kong}
% }
\author{Hongxi Pu$^{*}$}
\email{hongxi@umich.edu}

\affiliation{
  \institution{University of Michigan}
  \city{Ann Arbor}
  \country{United States}
}

\author{Futian Jiang}


\affiliation{
  \institution{University of Hong Kong}
  \city{Hong Kong}
  \country{Hong Kong}
}



\author{Zihao Chen}



\affiliation{
  \institution{Brooklyn College}
  \city{NYC}
  \country{United States}
}

\author{Xingyue Song}

% \email{xysong1213@gmail.com}

\affiliation{
  \institution{San Francisco Conservatory of Music}
  \city{San Francisco}
  \country{United States}
}

\renewcommand{\shortauthors}{}
%%
%% end of the preamble, start of the body of the document source.

%%\citestyle{acmauthoryear}
\begin{document}
%%\citestyle{authoryear}
%%
%% The "title" command has an optional parameter,
%% allowing the author to define a "short title" to be used in page headers.
\title[ComposeOn Academy]{ComposeOn Academy: Transforming Melodic Ideas into Complete Compositions Integrating Music Learning}

\begin{abstract}
Music composition has long been recognized as a significant art form. However, existing digital audio workstations and music production software often present high entry barriers for users lacking formal musical training. To address this, we introduce ComposeOn, a music theory-based tool designed for users with limited musical knowledge. ComposeOn enables users to easily extend their melodic ideas into complete compositions and offers simple editing features. By integrating music theory, it explains music creation at beginner, intermediate, and advanced levels. Our user study (N=10) compared ComposeOn with the baseline method, Suno AI, demonstrating that ComposeOn provides a more accessible and enjoyable composing and learning experience for individuals with limited musical skills. ComposeOn bridges the gap between theory and practice, offering an innovative solution as both a composition aid and music education platform. The study also explores the differences between theory-based music creation and generative music, highlighting the former's advantages in personal expression and learning.
\end{abstract}

%%
%% The code below is copied from, generated by the tool at http://dl.acm.org/ccs.cfm.
\begin{CCSXML}
<ccs2012>
   <concept>
       <concept_id>10003120.10003130.10011762</concept_id>
       <concept_desc>Human-centered computing~Interactive System and Tools</concept_desc>
       <concept_significance>500</concept_significance>
       </concept>
 </ccs2012>
\end{CCSXML}
\ccsdesc[500]{Human-centered computing~Interactive System and Tools}

%%
%% Keywords.
\keywords{Sound and music computing, Music composition, Music theory E-learning, Digital Audio Workstation, Novice Users }

\begin{teaserfigure}
\centering
\includegraphics[width=0.5\linewidth]{figs/musicbegin.png}
\caption{ComposeOn takes the melody input to compose complete music, teach music theory and allow music edition.}
\Description{ComposeOn System Design Diagram with three modules}
\label{fig:system_design_diagram}
\end{teaserfigure}
% \begin{teaserfigure}
% \centering
% \subfigure[]{
% \includegraphics[width=0.305\textwidth, trim= 0 0 40 3]{figs/cover1.JPG}\label{fig1a}
% }\hspace{1mm}
% \subfigure[]{
% \includegraphics[width=0.305\textwidth, trim= 0 0 50 0]{figs/cover2.JPG}\label{fig1b}
% }\hspace{1mm}
% \subfigure[]{
% \includegraphics[width=0.315\textwidth, trim= 0 0 50 0]{figs/cover3.JPG}
% \label{fig:01}
% }
% \caption{Caption}
% \Description{Caption}
% \end{teaserfigure}

%%
%% This command processes the author and affiliation and title
%% information and builds the first part of the formatted document.
\maketitle

\section{Introduction}\label{sec:Introduction}
\section{Introduction}

\section{Motivation}
\label{sec:motivation}



% In LLM inference, not only does weight matter, but the memory requirements of the KV Cache are also considerable.
In this section, we first demonstrate that the emerging paradigm of group quantization demands a high level of adaptivity, which current adaptive methods lack.
We then discuss how adapting these methods to group quantization could compromise their efficiency.
Given that LLMs generate KV caches during runtime, real-time quantization capability is crucial.
These challenges lead to our proposal of a mathematical adaptive numerical type (\texttt{MANT}), which we will detail later.



\begin{figure}[t]
    \centering
    \begin{minipage}[t]{0.48\columnwidth}
      \centering
      \includegraphics[width=\columnwidth]{fig/moti_group_ppl.pdf}
      \caption{LLM accuracy with different quantization granularities. We report the perplexity (PPL) metric (lower is better).}\label{fig:moti_group_ppl} 
    \end{minipage}
    \hspace{2pt}
    \begin{minipage}[t]{0.48\columnwidth}
      \centering
      \includegraphics[width=\columnwidth]{fig/motivation_adaptive_ppl.pdf}
      \caption{Accuracy loss for \texttt{INT}, \texttt{ANT}, and Ideal (clustering algorithm K-Means) adaptive methods in group quantization. }\label{fig:moti_ppl} 
    \end{minipage}
    % \vspace*{-0.3cm}
\end{figure}




\subsection{Group Quantization Accuracy Analysis}
\label{sec:acc_analysis}

In this subsection, we begin by comparing the accuracy of traditional channel-wise quantization with group-wise quantization~\cite{shao2024omniquant,zhao2023atom,liu2024kivi,sheng2023flexgen,lin2023awq,zhao2023atom}, establishing the baseline for group-wise quantization in this study.
We then delve into the use of various adaptive data types in group quantization, emphasizing the necessity for full adaptivity.



\Fig{fig:moti_group_ppl} illustrates the perplexity when quantizing the LLaMA-7B model~\cite{touvron2023llama} with various granularities using the \texttt{INT4}-based symmetric quantization.
Channel-wise quantization significantly worsens the perplexity of the examined LLM, increasing it from 5.68 to 6.85.
Conversely, group-wise quantization mitigates this loss in perplexity with a group size of 128, corresponding to an average of 4.125 bits per element (16-bit scaling factor).
Additionally, we observe that a smaller group size of 32 offers only a slight improvement in perplexity, but the scaling factor overhead increases by $4\times$.



Given this analysis, we adopt a group size of 128 as our standard configuration for the remainder of this section.
Previous research indicates that the \texttt{INT} data type is not optimal for accuracy since tensors or channels exhibit varied distributions, leading to the proposal of various adaptive data types~\cite{guo2022ant, guo2023olive, zadeh2020gobo, zadeh2022mokey}.
We evaluate their efficacy in the context of group quantization, which falls into two main categories: data-type-based and clustering-based.



\textbf{Data-type-based adaptive methods} select data types from discrete sets based on tensor data distribution.
ANT~\cite{guo2022ant} is a representative example of the data-type-based method.
ANT packages several different data types for selection, including \texttt{INT} for the uniform distribution, \texttt{PoT} (Power of Two) for the Laplace distribution, and \texttt{flint} for the Gaussian distribution.
%ANT designed \texttt{flint} for Gaussian distributions.

\textbf{Clustering-based adaptive methods} utilize clustering algorithms to generate centroids that align with the data distribution and provide considerable adaptivity. 
Mokey~\cite{zadeh2022mokey} and GOBO~\cite{zadeh2020gobo} exemplify this approach, though they focus on tensor- or channel-wise quantization. In our study, we adapt them to group quantization through per-group clustering.

%Clustering-based methods employ clustering algorithms to generate centroids that fit the data distribution, demonstrating sufficient adaptivity.
%Mokey~\cite{zadeh2022mokey} and GOBO~\cite{zadeh2020gobo} are such presentative works, but only target tensor- or channel-wise quantization.
%In our work, we modify those works to support group quantization by performing per-group clustering.
\Fig{fig:moti_ppl} compares the accuracy of the methods described above for the LLaMA-7B model under 4-bit group-wise quantization. 
The group-wise \texttt{ANT} method outperforms the \texttt{INT} type by dynamically selecting from three data types to better match the value distribution, resulting in reduced perplexity (PPL) loss. 
Moreover, per-group clustering adjusts more effectively to the value distribution of each group, establishing itself as the accuracy-optimal and ideal adaptive method. 
This approach achieves nearly lossless 4-bit quantization, equivalent to 16 centroids per group. 
However, this ideal scenario is impractical due to the significant overhead associated with storing per-group centroids, effectively rendering it a 6-bit quantization.

\begin{figure}[t] 
    \centering 
    \includegraphics[width=1.0\linewidth]{fig/intro_cdf.pdf}  
    \caption{The cumulative distribution function (CDF) of the tensor, channel, and group, respectively. The tensor data were taken from layers 8 to 23, while the 16 channel and group data were sampled from one tensor with specific strides.}\label{fig:moti_dist} 
    %  \vspace*{-0.3cm}
\end{figure}

To illustrate the group-wise diversity in data distribution, we sampled the weights of the Q and V tensors in LLaMA-7B model. 
We normalized all sampled data to their absolute maximum values, which ranged from -1 to 1. \Fig{fig:moti_dist} displays the cumulative distribution function (CDF) for the tensor, channel, and group levels, respectively. 
We observed that the diversity at the group level is significantly higher than at the tensor level. 
In simpler terms, while different tensors exhibit similar distributions, groups can have markedly different distributions. This finding underscores the necessity for full adaptivity in group quantization to fully realize its potential.
\paragraph{Takeaway 1.} The group quantization is an emerging paradigm to accelerate LLMs, and the significant group-level diversity requires a high level of adaptivity to fully unleash its potential.

\subsection{Group Quantization Efficiency Analysis}
\label{subsec:efficiency}


In this subsection, we provide a detailed efficiency analysis for the above adaptive quantization methods.
In \Tbl{intro:dtype}, we compare OliVe~\cite{guo2023olive}, ANT~\cite{guo2022ant}, GOBO~\cite{zadeh2020gobo}, and Mokey~\cite{zadeh2022mokey} with \texttt{INT} regarding the efficiency of computation, encoding, and decoding. 
In this paper, we use the term encoding (decoding) interchangeably with quantization (dequantization).
 

Data-type-based adaptive methods such as ANT~\cite{guo2022ant} and Olive~\cite{guo2023olive} achieve computational efficiency comparable to \texttt{INT}. 
Both utilize specialized decoders that decode these data types prior to computation, resulting in high decoding efficiency. 
However, as previously demonstrated, these methods suffer from limited adaptivity in the group quantization paradigm. 
A straightforward approach to enhance adaptivity is to expand their set of data types. 
However, incorporating new data types necessitates additional decoders, escalating hardware design costs. 
Additionally, compatibility issues between new and existing data types may reduce computational efficiency. 
For instance, the \texttt{NF4} data type~\cite{dettmers2023qlora} requires an FP16 MAC unit, which is incompatible with existing \texttt{ANT} data types.


\paragraph{Takeaway 2.} Enhancing the data-type-based adaptive method for group quantization is challenging and requires a careful balance for the computation and decoding efficiency.

Clustering-based adaptive methods like GOBO~\cite{zadeh2020gobo} and Mokey~\cite{zadeh2022mokey} can sufficiently adapt to various distributions at the group level. 
However, they require codebooks for quantization and dequantization, leading to high adaptivity at the expense of encoding and computational efficiency. 
For instance, a 16-entry codebook with 8 bits per entry requires 128 bits per group, creating an inevitable trade-off between adaptivity and memory overhead. GOBO~\cite{zadeh2020gobo} employs the K-means algorithm to quantize weights and requires dequantization to \texttt{FP16} using a codebook lookup table before computation, resulting in high adaptivity but low computational efficiency. 
Conversely, Mokey~\cite{zadeh2022mokey} enhances the computation of clustering-based methods by using indices for centroid values via approximate calculations, though matrix multiplication still relies on floating-point units, increasing overhead compared to integer units. 
Furthermore, Mokey creates one \texttt{golden dictionary} for all activations and weights, akin to using a single data type in quantization, thus reducing adaptivity.


\paragraph{Takeaway 3.} Deploying the clustering-based adaptive methods under group quantization is challenging owing to the low encoding and computation efficiency. 


\begin{table}[t]
    \centering
    \small
    \renewcommand{\arraystretch}{1.2}
    \caption[]{Features of DNN accelerators with adaptive and flexible data types are summarized. Here, `Effi.' stands for efficiency, `Med.' for medium, and `LUT' for lookup table.}
  
    \resizebox{1.0\columnwidth}{!}{
      \begin{tabular}{c|cc|ccc|cc|c}
        \Xhline{1.2pt}
        \multirow{2}{*}{Architecture} & \multicolumn{2}{c|}{Encode} & \multicolumn{3}{c|}{Computation} & \multicolumn{2}{c|}{Decode} & \multirow{2}{*}{Adaptivity} \\ \cline{2-8}
        & Method & Effi. & Method & Bit & Effi. & Method & Effi. \\
        \Xhline{1.2pt}
        \texttt{INT} & Round & High & INT & 4 \& 8 & High & Calculation & High & Low \\ 
        OliVe~\cite{guo2023olive} & Search & Med. & INT & 4 \& 8 & High & Decoder & High & Med. \\ 
        ANT~\cite{guo2022ant} & Search & Med. & INT & 4 \& 8 & High & Decoder & High & Med. \\ 
        Mokey~\cite{zadeh2022mokey} & Cluster & Med. & Float & 4 \& 8 & Med. & Calculation & Med. & Low \\ 
        GOBO~\cite{zadeh2020gobo} & Cluster & Low & Float & 16 & Low & LUT & Med. & High \\ 
        \hline
        \multirow{2}{*}{\proj}  & Search  & Med.  & \multirow{2}{*}{INT} & \multirow{2}{*}{4 \& 8} & \multirow{2}{*}{High} & \multirow{2}{*}{Calculation} & \multirow{2}{*}{High} & \multirow{2}{*}{High} \\ \cline{2-3}
        &  Map &  High &  &&&\\ 
        \Xhline{1.2pt}
    \end{tabular}
    }
    \vspace*{0.1cm}
    \label{intro:dtype}
    \vspace*{-0.2cm}
  \end{table}

\subsection{Support for Real-time Quantization}
\label{sec:moti_kvcache}

The above group-wise diversity presents a challenge for both weights and KV cache.
In addition, KV cache faces challenges in real-time group-wise quantization because the KV cache is generated dynamically during LLM inference.


To facilitate low-precision computation in group-wise quantization, it is necessary to quantize K and V along the inner dimension. 
This requirement stems from the support for matrix inner product operations in most GPUs and TPUs. 
During these operations, the group-wise scaling factor can be extracted from the multiply-accumulate process. 
\Fig{fig:kv_process} depicts the computation process of K and V during the decode stage. We define the dimension used for matrix inner product operations as the inner dimension. 
The inner dimensions of the K and V caches differ; the K cache requires a transpose operation, whereas the V cache does not, complicating the situation.


In the prefill stage, K and V can easily compute the scaling factor for each group. 
During the decode stage, the newly generated K vector is concatenated along the inner dimension of the K cache, enabling immediate quantization. 
However, the newly generated V vector is associated with different groups, with only one element per group produced per iteration. This process prevents the scaling factor for the entire group from being obtained in a single iteration, posing a significant challenge for the real-time quantization of the V cache.


\begin{figure}[t] 
  \centering 
  % \includegraphics[width=1.0\linewidth]{fig/dse_kv_process.pdf}  
  \includegraphics[width=0.9\linewidth]{fig/moti_kv_dimension.pdf}  
  \caption{\small Comparison of group-wise K and V cache quantization. They have different inner dimensions due to the transposition of K (key).}

  \label{fig:kv_process}
  % \vspace*{-0.4cm}
\end{figure}


Given those challenges, we propose \proj with a mathematical encoding format that can fuse with integer computation and enhance the decoding efficiency.
In addition, this encoding format provides sufficient adaptivity for group-wise quantization.
Regarding the challenge in KV cache, \proj employs a real-time quantization engine that ensures efficient encoding and decoding for KV cache.
By addressing these challenges, \proj enables efficient low-bit group-wise quantization.



Indoor scene design requires a comprehensive consideration of space partition, functional arrangement, and aesthetic creativity to determine the object selection and placement to form the scene layout. There has been a vast amount of research on indoor scene synthesis, ranging from layout optimization~\cite{Yu2011MakeIH, Merrell2011InteractiveFL, Qi2018HumanCentricIS} to various conditional scene synthesis~\cite{Wang2018DeepCP, Paschalidou2021ATISSAT, Gao2023SceneHGNHG, Tang2023DiffuSceneSG}. The goal is to automatically produce plausible, realistic, and diverse 3D indoor scenes, especially
given arbitrary user requirements. However, due to the complexity of indoor scenes, most of them are limited within the scope of training data and cannot be generalized to arbitrary conditions.

Some pioneer works~\cite{Feng2023LayoutGPTCV, Wen2023AnyHomeOG, Yang2023HolodeckLG} make use of the promising generalization ability of pre-trained large language models (LLM) to address the open-vocabulary scene synthesis task, where the LLM is responsible for interpreting any textual requirement into detailed scene configurations. The challenge lies in obtaining reasonable and physically feasible scene layouts from LLM outputs. Directly using LLM to output numerical layouts~\cite{Feng2023LayoutGPTCV} causes unreliable results with heavy object overlap and out-of-boundary, since LLM fails to understand the spatial relationship with numerical layouts. On the other hand, LLM shows good performance in generating detailed text descriptions of various scenes, but still require an approach to convert the textual descriptions into numerical layouts while maintaining the generalization of the entire pipeline. The existing methods~\cite{Wen2023AnyHomeOG, Yang2023HolodeckLG} have to pre-define textual phrases and numerical rules for several types of spatial relations to obtain the layouts. However, dense spatial relations often lead to incompatible object arrangements while coarse relations fail to capture diverse spatial placements, causing misalignment between LLM-generated configuration and the results. 

In this paper, we propose to use hierarchical scene descriptions as the intermediate representation in the LLM-assisted scene synthesis pipeline. The hierarchical structure has three levels, with the entire scene as root node, functional areas as internal nodes, and objects as leaf nodes, as illustrated in Figure~\ref{fig:hierarchy}. Our approach contains three stages. First, we prompt the pre-trained LLM to generate the hierarchical structure with text descriptions. Second, we train a hierarchy-aware network to further infer the fine-grained relative placements between objects with textual spatial relations. Taking the hierarchical structure as grounding, the network can infer reasonable relative placements in an open-vocabulary setting. Third, we develop a divide-and-conquer optimization, which optimizes each functional area separately and then arranges them to form the entire scene, to solve for the physically feasible scene layouts effectively.

The advantages of using hierarchically structured scene representation are two-fold. First, the hierarchical structure provides a rough grounding for object arrangement, which alleviates contradictory placements with dense relations and enhances the generalization ability of the network to infer fine-grained placements. Second, it naturally supports the divide-and-conquer optimization to more effectively solve for a feasible layout that matches with the LLM-generated descriptions. We perform extensive comparison experiments and ablation studies with both qualitative and quantitative evaluations. Our approach generates more reasonable and physically feasible scenes that align better with the user requirements and LLM arrangements. In addition, we show the results of open-vocabulary scene synthesis and interactive scene design as practical applications of our approach.

Our contributions are summarized as follows:
\begin{itemize}
\item We propose an LLM-assisted hierarchically-structured scene synthesis pipeline, which uses a three-level hierarchical structure to infer reasonable object arrangements.
\item We develop the hierarchy-aware network and the divide-and-conquer optimization, which takes advantage of the hierarchical structure for effective layout synthesis.
\item We conducted extensive comparison and ablation study experiments to validate the effectiveness of our approach, as well as two applications to show its practical usage.
\end{itemize}



\section{Related Work}\label{sec:Related Work}
\subsection{Making Arguments in Online Forums}
Online communities are inherently heterogeneous and multi-faceted, with goals that include entertaining, information exchange, social support, and prestige~\cite{kairam_how_2024, moore_redditors_2017}. Given the diverse range of discussion topics, there are online communities focused on politics~\cite{papakyriakopoulos_upvotes_2023, hua_characterizing_2020, lyu_exploring_2023}, fan fiction~\cite{campbell_thousands_2016}, sports~\cite{zhang_this_2018,kim_social_2015, zhang_intergroup_2019, wang_making_2023}, career mentoring~\cite{tomprou_career_2019}, Korean popular music (K-pop) groups~\cite{park_armed_2021}, virtual communities~\cite{fu_i_2023}, live streaming~\cite{lu_you_2018,lu_i_2019,lu_more_2021}, and so on. As online communities can vary greatly in purpose, scope, and topic~\cite{hwang_why_2021}, our research focuses on argument-based and forum-style communities. These types of forums are identified as essential places for people to voice their opinions and engage in debates with each other~\cite{qiu_modeling_2015}.

Even though the motivations for establishing communities are always to benefit their members and form a bond among them~\cite{kairam_how_2024, matthews_goals_2014}, dissonance may arise in forum discussions as part of the community activities. Specifically, political forums may be inherently more prone to incivility than forums about other topics~\cite{efstratiou_non-polar_2022}, as research has suggested that "interactions between ideologically opposed users were significantly more negative than like-minded ones"~\cite{marchal_be_2022}. Nevertheless, another study challenged this popular belief by suggesting that intra-group members holding the same sides of the political spectrum can have an even higher amount of polarizing and aggressive comments compared to inter-group members~\cite{efstratiou_non-polar_2022}.

Similar debates can also occur in sports communities. Sports fans who support different players may treat each other as enemies, and their attitudes can vary according to team performances~\cite{zhang_this_2018}. In these circumstances, expressing emotions can easily turn into aggressive posts, trolling behaviors, and even a vicious circle by down-voting and spreading negative feelings~\cite{wang_making_2023}. Another work revealed that members with higher inter-group contact levels tended to use more negative words, swear words, and produce more hate speech comments in their affiliated group discussions compared to those who only had single-group identity~\cite{zhang_intergroup_2019}.

In light of this, online debates constitute a pivotal component of online interactions among forum members. Unlike traditional face-to-face debates, the persuasiveness and effectiveness of online debates predominantly rely on a form of designing for persuasive influence~\cite{lc_designing_2021, lc_designing_2022}, thereby highlighting the importance of persuasive writing.


\subsection{Persuasive Writing}
It is common for people holding different views to try to persuade others when discussing online~\cite{xia_persua_2022,tan_winning_2016}. Historically, rhetoric and argumentation can be traced back to Aristotle's modes of persuasion~\cite{wang_argulens_2020}. Contemporary rhetoric studies also focus on argumentation, the audience, and the conditions for rational debates~\cite{herrick_history_2020}. Toulmin's model~\cite{toulmin_uses_2003}, one of the most influential argumentation models~\cite{wang_argulens_2020}, proposed six fundamental argumentative components including claim, ground, warrant, qualifier, rebuttal, and backing~\cite{wang_argulens_2020,bentahar_taxonomy_2010,toulmin_uses_2003}. Previous research has widely adopted Toulmin's model as a foundation to improve the persuasiveness of usability feedback~\cite{norgaard_evaluating_2008}, unveil community opinions on usability~\cite{wang_argulens_2020}, and support system building to enhance argumentation~\cite{zhang_using_2016,wambsganss_modeling_2022}. Compared to other models, Toulmin's model and extensions have distinct advantages in specifying various components of the argument structure, their interconnections, and the inference rules for constructing textual arguments~\cite{bentahar_taxonomy_2010}.

More persuasion models have been developed to explain how people respond to persuasive attempts in marketing and advertising. For example, the Heuristic-Systematic Model (HSM) of persuasion describes how people process persuasive messages through heuristic and systematic processing~\cite{reimer_use_2004}. The Persuasive Knowledge Model (PKM) addresses how people recognize, evaluate, and respond to persuasive content~\cite{friestad_persuasion_1994}.

Building on Toulmin's model~\cite{toulmin_uses_2003}, researchers have established a framework that includes claims, evidence (the information or data that support the claim), and reasoning (a justification that shows why the data count as evidence to support the claim)~\cite{berland_making_2009}. Claims can be further classified into different types, including definitive and descriptive ones~\cite{van_der_wall_statement_2012}. In addition to claims, evidence also comes in various categories such as numerical data~\cite{berland_making_2009}, observations~\cite{berland_making_2009}, facts~\cite{berland_making_2009}, examples~\cite{southerland_examples_2017}, and counterexamples~\cite{johnson-laird_how_2008}. In terms of reasoning, besides typical techniques such as rebuttal~\cite{toulmin_uses_2003} and analogy~\cite{winebrenner_argumentation_1991}, some fallacies can lead to misunderstanding and even deceive readers. Fallacies in reasoning can take many forms, such as hasty generalization~\cite{van_eemeren_argumentation_2016,kord_grey_2021}, ad hominem attacks~\cite{van_eemeren_argumentation_2016, kord_grey_2021}, straw man arguments~\cite{van_eemeren_argumentation_2016}, misplacing the burden of proof~\cite{kord_grey_2021}, and irrelevant conclusion~\cite{kord_grey_2021}.

\subsection{Co-Writing with AI Assistants}

Unlike writing alone, collaborative writing, with either human or AI assistance, is common and has been applied in various aspects of our daily life~\cite{storch_collaborative_2005, li_computer-mediated_2018, barile_computer-mediated_2002}. With the support of AI writing assistants such as Grammarly~\footnote{Grammarly:~\url{https://www.grammarly.com/}}, the writing quality can be significantly improved~\cite{fitria_grammarly_2021}. In 2022, the release of ChatGPT by OpenAI represented a pivotal advancement in the field of human-AI collaborative writing, drawing substantial attention from various research communities, such as Human-Computer Interaction (HCI), Natural Language Processing (NLP), and Computational Social Science (CSS)  ~\cite{lee_design_2024}. Beyond general writing purposes, human-AI co-writing is widely adopted in specific use cases such as fiction writing~\cite{zhong_fiction-writing_2023,yang_ai_2022}, poetry writing~\cite{lc_imitations_2022}, theater script writing~\cite{mirowski_co-writing_2023}, science and scientific writing~\cite{gero_sparks_2022, kim_metaphorian_2023, shen_convxai_2023}, etc. Prior research has also highlighted the promising future of human-AI co-writing across various application scenarios~\cite{luther_teaming_2024}.

In the HCI community, people have designed various human-AI co-writing tools to explore new writing paradigms. For example, Dramatron, derived from a large language model, enables participants to collaborate with AI systems to create theater scripts and screenplays, proving especially useful for hierarchical text generation~\cite{mirowski_co-writing_2023}. Similarly, CoPoet is tailored to assist human writers in crafting poems, enhancing the final outcomes~\cite{chakrabarty_help_2022}. Wordcraft, an interface designed for story writing, allows AI to serve various roles such as idea generator, scene interpolator, and copy editor~\cite{yuan_wordcraft_2022}. Audiences prefer specific modes with fine-grained control over generated text, often expressing satisfaction~\cite{zhong_fiction-writing_2023}. Wan et al.~\cite{wan_it_2024} investigated human-AI co-creativity in the prewriting scenario to shift the focus from convergent to divergent thinking.

Previous research shows that the AI mediator can enhance critical thinking, which helps in bursting filter bubbles and depolarizing online communities~\cite{govers_ai-driven_2024, tanprasert_debate_2024, lin_case_2024}. However, online debates are inherently adversarial, often thriving on polarization to stimulate engagement and argumentation. This contrast motivates the exploration of how the use of generative AI can be adapted to support such a polarized and competitive context effectively.

\section{Formative Study}\label{sec:Formative Study}
\section{Formative Study}\label{sec:formative}
We conducted a formative study to understand the workflow of paper-cutting design and users' challenges during the design.
% Formative study: 

%.相反,我们最初感兴趣的是了解采用复杂动画的数据驱动故事是如何创作的,以及创作者的工作流程中存在的挑战。为了实现这一目标,我们与专家进行了一项基于访谈的研究,这些专家在使用可视化和动画讲故事方面拥有丰富的经验。由于动画数据故事通常是由一组创建者创建的,因此我们招募了不同角色的专家来全面了解此类工作流。通过这些访谈,我们发现,在受访者创建的各种类型的数据故事中,由AUV组成的数据故事被认为既引人注目又具有挑战性,因为它们加剧了复杂动画创作过程中的许多痛点。

\subsection{Participants}
As a work targeted to public users, we initially sought to examine the paper-cutting design process and the challenges encountered without GenAI. Subsequently, we aimed to assess the role and limitations of integrating GenAI into paper-cutting design. To achieve this, we recruited participants with varying levels of expertise in both paper-cutting and GenAI, as experts are expected to have insights into design workflows, and each participant group faces different challenges in paper-cutting design. \rrtext{Participants were categorized into four levels of paper-cutting expertise, with further clarification regarding the criteria for paper-cutting expertise provided in~\autoref{expert clarification}:}
(1) \textbf{Masters}: who have over 20 years of professional experience in paper-cutting creation are officially recognized as ICH inheritors; (2) \textbf{Practitioners}: who have 10-20 years of experience in paper-cutting-related work; \revisedtext{(3) \textbf{Amateurs}: who have 1-3 years of experience in creating paper-cuttings without systematic training;} (4) \textbf{Novices}: who never engaged in paper-cutting design or creation. Additionally, we defined three levels of GenAI expertise: (1) \textbf{Professionals}: researchers in the multi-modal machine learning field; 
% (2) \textbf{Knowledgeable Users}: who have previously used GenAI; (3) \textbf{Novices}: who are only vaguely familiar with or unfamiliar with GenAI.
\revisedtext{(2) \textbf{Knowledgeable Users}: who regularly integrate GenAI into their professional, educational, or personal activities and have experience with basic prompt engineering. (3) \textbf{Novices}: who have minimal exposure to GenAI, may have heard of it without engaging in its use, or are entirely unacquainted with it.}

\revisedtext{Seven participants (3 females and 4 males; age M=34.43, SD=12.79) were recruited for the semi-structured interviews through online postings on various social media platforms (e.g., Douyin\footnote{\url{https://www.douyin.com/}} and Bilibili\footnote{\url{https://www.bilibili.com/}}), and the detailed information of the participants is shown in \autoref{table:formative participants}.} Among them were three paper-cutting masters (P1, P4, P5), two practitioners (P2, P3), one amateur (P7), and one novice (P6). Regarding GenAI expertise, the group included one GenAI professional (P6), two knowledgeable users (P2, P7), and four novices (P1, P3, P4, P5). Each participant is compensated with 100 CNY (approximately 14 USD).
% \begin{itemize}
%     \item[1] Masters: who have over 30 years of professional experience in paper-cutting creation are officially recognized as ICH inheritors.
%     \item[2] Practitioners: who have 10-20 years of experience in paper-cutting-related work.
%     \item[3] Amateurs: who have 1-3 years of experience in creating paper-cuttings.
%     \item[4] Novices: who never engaged in paper-cutting design or creation.
% \end{itemize}  
% Additionally, we defined three levels of GenAI expertise: 
% \begin{itemize}
%     \item[1] Professionals: multi-modal machine learning researchers.
%     \item[2] Knowledgeable Users: who have previously used GenAI;
%     \item[3] Novices: who are only vaguely familiar with or unfamiliar with GenAI.
% \end{itemize} 
% 我们面向的是所有想要通过GenAI辅助进行剪纸创作的用户,无论是专家还是新手。因此我们招募了3 professional inheritors 2个市级 (40年, 20年),一个省级(40年);2 paper-cutting practitioners, 都有超过10年的的从业经验;1 novices,1个纯新人和; 1爱好者 自己剪纸剪纸3年的人。

\subsection{Procedure}
The semi-structured interview included two parts: (1) individual design and (2) GenAI-aided design.
% In the beginning, we provided a 10-minute background introduction for novices in the fields of paper-cutting or GenAI.
\revisedtext{At the beginning, we provided separate 10-minute background introductions for novices in the fields of paper-cutting and GenAI, amounting to a total of 20 minutes.}
In the initial part of the study, each participant was tasked with providing a design concept description for paper-cuttings and giving a sketch of paper-cutting, including the areas to cut out, selecting randomly from two main themes:  ''\textit{Dragon Boat Festival}'' and ''\textit{Wedding,}'' for 30 minutes. We then interviewed them to explore participants' personal understanding and perspectives on the paper-cutting design process and examined the key aspects of the design process, including steps, core factors, and challenges faced.
Then, based on the aforementioned themes, participants were asked to engage with a Large Language Model (LLM) to assist in the design process and use a Text-to-Image model to generate paper-cutting image in 10 minutes. After that, we collected feedback from participants on GenAI-aided paper-cutting design, including the shortcomings of the results, challenges in the design process, and their expectations and suggestions for the GenAI-aided system.

% 1. 以“龙”, “平安富贵”, “多子多福”,5min to think the requirement and primary content in this paper-cutting can match these given topics; 2. 10 min to interactive with Stable Diffusion to generate the paper-cuttings. They were asked to describe how their paper-cutting's primary object/content matched the given topics. After that, we first conducted a semi-structured interview to 
% 我们的访谈分为两部分,第一部分,我们首先基于两个主题:“迎接春天”和"婚礼"来询问它们设计思路。然后我们向受访者询问了每人对于剪纸设计过程的理解和看法,最后向每个人询问了有关设计过程一般是怎样组成的,有哪些关键要素需要考虑,有哪些步骤是具有挑战的。 第二部分,首先我们继续基于同样的两个主题来让用户使用LLM辅助进行设计,并结合输出的结果来让T2I模型生成最终的剪纸内容。 然后,我们向受访者询问了ai-辅助剪纸作品的设计的不足,整个过程存在的挑战,以及它们对于系统可以如何改进GenAI,来改善设计的期待和关心

% We summarized our findings regarding the core factors of workflow and challenges in paper-cutting creation with GenAI assistance.

% \begin{figure*}[!htbp]
% \centering
% \includegraphics[width=0.95\textwidth]{Images/workflow.pdf}
% \caption{\label{figure1}
% A general workflow for GenAI-aided paper-cutting design is outlined from the 2-step formative study, with the main challenges in the workflow labeled on corresponding stages. Based on the workflow, and challenges, the design goals are solidified to the pipeline and interface of HarmonyCut}
% \Description{This Figure shows a general workflow for GenAI-aided paper-cutting design is outlined from the 2-step formative study, with the main challenges in the workflow labeled on corresponding stages. Based on the workflow, and challenges, the design goals are solidified to develop HarmonyCut}
% \end{figure*}


\subsection{Findings}

Through the semi-structured interviews and literature review, we identified the workflow of paper-cutting design: ideation and composition. We found that ideation in paper-cutting design presents challenges to all participants, albeit to varying degrees. It is tedious to both participants and GenAI in the composition phase. In GenAI-aided designs, the recommended and generated content is often undesirable, leading to an uncontrollable overall process that cannot modify the output.

\subsubsection{Workflow of Paper-cutting Design}
% 根据专家的意见,剪纸设计主要分为两部,第一是将需求转变为一个 coceptual idea,这个idea 一般需要从几个方面来来考虑(在section 4会提及);第二个是将conceptual idea 落实为视觉的内容(即用剪刀进行剪裁前的设计图)
\revisedtext{Drawing on feedback from interviews, the suggested process of paper-cutting creation, especially in paper-cutting education~\cite{Lin:1974:howtopapercutting, Li:1998:monopapercutting, Li:2011:PatternandDesign, Zhang:1982:discusspapaercut}, and the design steps from Hubka et al.~\cite{Hubka:1992:engineeringdesign}, paper-cutting design primarily involves two main steps, as shown in \autoref{figure1}~(A).} The first step is transforming intents into a conceptual idea~\cite{Choi:2024:creativeconnect}.
% , which generally needs to be considered from several aspects (as will be discussed in \autoref{sec:content}). 
The second step is translating the conceptual idea into visual form (a design blueprint before using scissors for cutting).
\begin{itemize}
\item \textbf{Ideation.} The first step involves transforming intents with a theme into a conceptual idea~\cite{Li:1998:monopapercutting}. 
\rrtext{Based on the experts' feedback, several preliminary dimensions were mentioned, including \textbf{function and style}. These dimensions suggest that ideation should be approached from multiple dimensions (4 factors as detailed in \autoref{sec:4_2}) to determine the core components of the design.} This idea will later be translated into a visual design in the next step.
\item \textbf{Composition.} During the composition step, designers (1) select the shapes of the elements based on the idea, (2) arrange and combine the selected contours, and (3) decide on cut-out regions (unit patterns) for future creation~\cite{Lin:1974:howtopapercutting, Li:2011:PatternandDesign}.
\end{itemize}

\subsubsection{Challenges in Paper-cutting Design}
\revisedtext{Based on the observation in the formative study and literature review related to design and paper-cutting, we refined and summarized 5 challenges in paper-cutting design with GenAI assistance.}
\begin{itemize}
\item[\textbf{C1.}] \textbf{Challenges in Translating Intent to Ideas.}
% Novices and amateurs (P6 and P7) spent a considerable amount of time on the first part of the study, and each could only provide 2 vague descriptions of their design concept for each theme. This difficulty stems from their lack of fundamental knowledge in paper-cutting, including an understanding of the design workflow and the aspects that should be considered to fulfill the required theme. 
% \revisedtext{What's more, the observation of struggling to translate intent to idea aligns with Li's discussion on the cognitive approach in paper-cutting: they lack mapping thinking—a process that enables them to creatively link learned knowledge to the attributes of natural objects, assign relevant concepts, and employ structural methods to express their idea of paper-cutting designs~\cite{Li:1998:monopapercutting}. }
Novices and amateurs (P6 and P7) spent significant time on the first part of the study but could only provide two vague descriptions of their design concepts for each theme.
\revisedtext{This difficulty arises not only from their limited knowledge of paper-cutting, including familiarity with the design workflow and the essential elements needed to address the theme but also from the absence of a cognitive approach (i.e., ``mapping thinking'') described by Li~\cite{Li:1998:monopapercutting}. This approach enables them to creatively link their knowledge to the attributes of natural objects, assign meaningful concepts, and employ structural methods to effectively express their paper-cutting design ideas. All these competencies are essential to learn and apply for paper-cutting creation~\cite{Lin:1974:howtopapercutting, Zhang:2021:papercuttingteaching}.}
As noted by P7, ``\textit{It is easy to cut a paper-cutting based on a sketched outline, but besides the content, I am unsure of which dimensions need consideration in the design process.}'' \revisedtext{Besides, P6 stated, ``\textit{I don't know what content in paper-cutting can appropriately map to those creation intents.}''} Consequently, it is challenging to establish a clear direction for their ideas.

\item[\textbf{C2.}] \textbf{Lack of Creativity and Multiple Variations in Ideation.}
% Novices and amateurs face difficulty selecting multiple elements that align with their themes due to a limited understanding of paper-cutting subjects and their associated meanings and connotations.
% \revisedtext{Novices and amateurs (P6 and P7) struggled to select multiple elements that aligned with their themes in the first part of the study, primarily due to a limited understanding of paper-cutting subjects and their associated meanings and connotations. }
% \revisedtext{The cases in studies of Bloom~\cite{Bloom:1985:knowledgecreative}, Ericsson et al.~\cite{Ericsson:1994:knowledgecreative}, and Gardner~\cite{Gardner:1993:knowledgecreative} clearly indicate that long-term immersion in a discipline is a crucial prerequisite for creative ability, and knowledge serves as an indispensable cornerstone for innovative ideas~\cite{Weisberg:1999:creativityandknowledge}. Based on them, we infer that in the field of paper-cutting design, novices and amateurs who lack systematic knowledge of paper-cutting will face challenges in creative design. The observation in our study further validated this inference: P6 and P7 with limited knowledge of paper-cutting found it particularly difficult to select elements aligned with the theme, and their proposed design ideas tended to be repetitive.}
\revisedtext{Novices and amateurs (P6 and P7), who possessed limited knowledge of paper-cutting, struggled to select elements aligned with their themes, and their proposed ideas appeared repetitive. These findings are consistent with prior research~\cite{Ericsson:1994:knowledgecreative, Gardner:1993:knowledgecreative}, which highlight that long-term immersion in a discipline is essential for developing creative ability. Furthermore, knowledge is identified as a critical foundation for fostering innovative ideas~\cite{Weisberg:1999:creativityandknowledge}, supporting the finding that a lack of systematic knowledge hinders the creative potential of novices and amateurs in paper-cutting design.}
Conversely, experts and practitioners, although proficient in the creative process and capable of rapid ideation, tend to rely heavily on their accumulated experience and local cultural influences for themes and content. This dependence can lead to fixation on a single idea. As P4 said, ``\textit{Paper-cutting is highly regional, with the meaning of specific elements differing significantly even within the same province. Although there is diversity in paper-cutting, I am only familiar with the themes and elements from my region, resulting in more fixed forms and content across many themes.}''

\item[\textbf{C3.}] \textbf{Challenges in Converting Ideas into Concrete Visual Representations.}
Novices have limited drawing skills, making translating content in their ideas directly into visual forms challenging. Additionally, composition requires considering not only the spatial arrangement and structure of elements but also deciding which areas should be cut-out (pattern) and in what shapes during creation. It is labor-intensive for both novices and experts. As mentioned by P1, ``\textit{The arrangement of specific content and the shapes created through cut-outs (pattern) best reflect personal style. However, translating an idea into a visual expression is still laborious.}''

\item[\textbf{C4.}] \textbf{Challenges in Exploring Suitable and Rational GenAI Results.} 
For the given themes, the model struggles to grasp the user's unique design idea, often providing overly broad suggestions, which cannot assist the user in avoiding fixation even increase the load to user in exploration. Additionally, the finally generated paper-cutting images often do not match the text description, especially in the spatial arrangement of content. Moreover, many parts of the model-generated paper-cuttings are irrational or irrelevant, such as generating some random clusters of stripes as patterns in paper-cutting.

\item[\textbf{C5.}] \textbf{Challenges in Controlling and Editing GenAI Results.} 
Regarding the above issue with GenAI, it is difficult to directly adjust errors in the generated results. Participants can only try to improve the output by revising the input descriptions. However, because the model struggles to understand the knowledge of contents with nuanced meaning and composition, iterative changes to the input often yield minimal improvement.

\end{itemize}

% 用户难以获得剪纸相关的知识,从而进行符合需求的构思和创作:
% 首先,对于新手,它们缺乏对剪纸的基本知识,如创作流程和创作要素。 用户对创作需求只有模糊的想法,不知道该从哪些方面来考虑内容选取,从而满足创作需求。
% 而对于专家,虽然他们可以快速的明确创作流程和创作思路,但对于具体该选用哪些主题,内容和纹样来进行符合创作需求的作品,是具有挑战的。即使是专家,对于一些常见的主题,他们的创作内容也会陷入到固步自封(fixation)。
% 在GenAI辅助创作上,符合用户需求的剪纸的内容和纹样是多样的,而同一内容在不同情境下,或多种纹样的不同搭配方式都具有不同的意义。这些都是模型难以理解的。对于某种事物的剪纸内容生成往往限于固定的几种模式; 同时,在模型生成的剪纸内容中,有很多纹样是无意义、不合理,甚至只是一团胡乱的条纹。系统没有办法理解作为剪纸最核心的符号语言,纹样。
% 而如果让模型直接通过深度生成模型从文本输入到生成整个图像,结果很大程度上由人工智能主导,用户缺乏参与感,更无法对模型产生的问题直接做出修改与调整。正能迭代的重新生成

% 形成性研究发现,剪纸构思与创作过程存在显著挑战,主要体现在以下两个方面:
% 一、用户难以获取知识限制构思
% 新手用户普遍缺乏剪纸艺术的基本知识,包括创作流程和关键要素,导致难以根据创作需求进行有效构思。
% 专家用户虽能迅速明确创作流程与思路,但在选择适合创作需求的主题、内容及纹样时易陷入它们自己的创作定式(fixation)。
% 二、GenAI辅助创作中难以对结果修改,难以理解深层的领域知识
% 在利用GenAI进行剪纸创作时,模型难以全面理解剪纸艺术中与文化相关的复杂知识,如内容在不同情境下的意义变化及纹样搭配的深层含义。它们都导致GenAI生成的剪纸内容常受限于固定模式,结果中的纹样可能缺乏意义与合理性、甚至只是胡乱的条纹,未能充分体现纹样作为剪纸中符号语言的精髓。
% 当模型直接从文本输入生成剪纸图像时,用户参与度低,难以对生成结果进行即时修改与调整,限制了创作的灵活性与个性化。



\section{System Design}\label{sec:System Design}
Our system, ComposeOn, is designed to facilitate music extension and learning for users with little or no musical theory background. By analyzing the results of the formative study mentioned in Section \ref{sec:Formative Study}, we identified three design goals for ComposeOn:

\begin{itemize}
\item Melody Expansion: Enable users to easily extend and develop their melodic ideas into complete compositions.
\item Easy Editing: Allow users to effortlessly edit and control their music creations.
\item Music Theory Integration: Incorporate music theory learning to boost engagement and interest.
\end{itemize}

To generate the extended part of music based on the input melody, our system consists of three main components: the input \& analysis module, the generation module, and the output \& explanation module. Figure \ref{fig:system_design_diagram} shows a high-level diagram of the system design.
\begin{figure}[h]
\centering
\includegraphics[width=0.8\linewidth]{figs/system_design.png}
\caption{ComposeOn System Design Diagram: the input and analysis module, represented by light-blue color; the generation module, represented by light-green color; and the output and explanation module, represented by light-yellow color.}
\Description{ComposeOn System Design Diagram with three modules}
\label{fig:system_design_diagram}
\end{figure}

\subsection{ComposeOn Database}

The ComposeOn database consists of two main parts: common chord progressions and common rhythm patterns. These data provide ComposeOn with fundamental materials and references.

The \textbf{chord progression section} contains 39 different chord sequences across seven categories, covering various music styles from basic to advanced. These chord progressions are selected based on the principles of Functional Harmony and Tonal Harmony in music theory\cite{kostka2008harmonia,clendinning2016musician}. Specifically, it includes (1) 9 classic chord progressions, such as \texttt{I–IV–V–I} and \texttt{vi–IV–V–I}, common in pop and rock music; (2) 9 extended chord progressions, like \texttt{Imaj7–ii7–V7–Imaj7}, suitable for jazz and blues; (3) 4 diminished triad progressions, such as \texttt{i–iidim–V7–i}, used to create tense or dissonant effects; (4) 4 augmented fourth chord progressions, like \texttt{I–IV–aug4–I}, used to add harmonic color; (5) 5 mixed chord progressions, such as \texttt{Imaj7–ii7–V7–IVmaj7}, blending different types of chords; (6) 4 substitute chord progressions, like \texttt{Imaj7–bIImaj7–V7–Imaj7}, common in modern jazz; and (7) 4 cycle chord progressions, such as \texttt{Imaj7–ii7–V7–iii7}, used to create repeating sections or build-ups. The selection of these chord progressions also references popular music composition practices and jazz theory\cite{levine2011jazz,mulholland2013berklee}.

The \textbf{rhythm pattern section} contains 16 different rhythm types, each composed of specific combinations of notes and rests~\cite{r9}. These patterns cover various popular music styles, including pop, rock, reggae, ska, jazz, funk, blues, and country music. For example, the first pattern \texttt{[(1, 'rest'), (1, 'note'), (1, 'rest'), (1, 'note')]} represents a simple, regular rhythm, while the seventh pattern \texttt{[(1/3, 'note'), (1/3, 'note'), (1/3, 'note')] * 4} represents a more complex, triplet-based rhythm. These patterns provide ComposeOn with a rich selection of rhythms, enabling it to generate melodies that conform to specific musical styles.

\subsection{Input and Analysis Module}

To allow users to easily express their musical ideas, we have designed a flexible input module. Users can input their melodies by singing, humming, or playing an instrument. To ensure an accurate capture of users' musical creativity, we employ the advanced Basic Pitch library~\cite{r10}. This neural network-based pitch detection model can accurately convert users' audio input into standard MIDI file format, handling even complex polyphonic melodies with precision.

Once the user's input is converted into a MIDI file, the analysis module begins its work, delving deep into the musical characteristics of the melody. We use the powerful MusicPy~\cite{r11} library to process this MIDI file, extracting rich musical information. First, ComposeOn identifies the implied \textbf{chord progressions} in the melody, for example, the chords present in the user's input, recognizing them as D and G.. This provides insight into the harmonic foundation of the melody. Next, ComposeOn uses the detect\_scale function to determine the \textbf{scales} most likely used in the melody, for example, the system determines that the scale being used is D major.. To standardize the melody's harmonic structure, ComposeOn converts the identified chords into \textbf{scale degrees}, within the context of D major: D is identified as the I (tonic) chord, and G is identified as the IV (subdominant) chord. This process entails matching chords with detected scales, determining their position within the scale, and addressing special cases for minor scales. This conversion provides a more abstract, theoretical representation of the harmonic structure, allowing for a unified generation logic to be applied regardless of the original scale or key.

Finally, ComposeOn integrates all these analysis results, creating a detailed profile of the input melody's characteristics, and serving as a basis of the explanations. The standardized scale degree representation enables ComposeOn to apply consistent generation algorithms across various musical contexts.

\subsection{Generation Module}

The Generation Module is responsible for creating new musical content based on the analysis of the user's input. This process is divided into three main steps: recommending chord progressions, recommending rhythms for right-hand rhythm, and adding ornaments and variations. The system consults its chord progression database and identifies a common progression pattern that incorporates the input chords: [I - IV - V - I];[I - IV - ii - V - I];  [I - IV - vii dim - I];  [I - IV - V7 - I] etc.

\textbf{Step 1: Recommending Chord Progressions}

In the first step, we begin with chord degrees from the previous step of the Analysis Module and find the recommended chord degrees. When the user clicks the "Continue" button in the UI, we sequentially identify the most similar progressions, utilizing the SequenceMatcher class from the difflib library ~\cite{pythondifflib} to calculate the sequence similarity between the input chord progression and our predefined progressions through their chord degrees.  Each click recommends a complete progression. This recommended progression is then combined with the original input progression to form a new progression base. On subsequent "continue" clicks, this combined progression serves as the new input. 

Then, we convert these degrees into absolute chords based on the scales retained from earlier steps of the Analysis Module. For instance, if our recommended chord degree sequence is [I, IV, V, I], and the retained scale is D major, the resulting chord progression would be [D, G, A, D]. This determined chord progression forms the basis for both the left-hand and right-hand melodies.

For the left-hand part, we utilize the triads of these chords, playing them as whole notes for each measure. In the case of [D, G, A, D], the left hand would play D-F\#-A (D major triad) for the first measure, G-B-D (G major triad) for the second measure, A-C\#-E (A major triad) for the third measure, and D-F\#-A (D major triad) for the fourth measure. This approach provides a solid harmonic foundation using simple, sustained chords in the left hand.

Meanwhile, the same chord progression serves as the foundation for developing the right-hand melody, allowing for more intricate melodic and rhythmic patterns that complement the underlying harmonic structure. For example, the right-hand melody might incorporate arpeggios or scalar passages derived from the D major scale, with emphasis on the chord tones of each underlying harmony.

This method of chord realization and melody generation demonstrates how the system can translate abstract music theory concepts into concrete musical elements. By providing a clear harmonic foundation in the left hand and a related but more elaborate melody in the right hand, the system creates a balanced and musically coherent output that is accessible to novice users while still adhering to established musical principles.

Furthermore, this approach can be extended to the other progression options identified earlier:

\begin{enumerate}
    \item For [I, IV, ii, V, I] in D major: [D, G, Em, A, D]
    \item For [I, IV, vii°, I] in D major: [D, G, C\# dim, D]
    \item For [I, IV, V7, I] in D major: [D, G, A7, D]
\end{enumerate}


To introduce uniqueness, we then apply variations to these selected progressions. This variation technique is based on the concept of chord substitution in music theory. According to Levine (2011), chords can often be substituted with chords that share common tones or have a similar function within the key\cite{levine2011jazz}. Our implementation focuses on diatonic substitutions, where chords are replaced by others from the same key, maintaining harmonic coherence while introducing variety. It's worth noting that we treat each complete chord progression as a musical phrase, providing a structural basis for subsequent melody generation.

\textbf{Step 2: Recommending Rhythms for the right-hand melody}

In the second step, we employ the rhythm pattern pool introduced in the Section of the ComposeOn database. We begin by fitting the input rhythm to this rhythm pattern database to find the closest match and randomly chose two more patterns in our rhythm pattern pool. For each complete chord progression, we then apply the following strategy for rhythm patterns: The first measure of each phrase always uses the rhythm pattern fitted to the input. This ensures that the generated music retains the rhythmic characteristics of the original input, maintaining musical coherence. For subsequent measures within the phrase, we select the randomly chosen two rhythm patterns. This approach both maintains a connection to the original input and introduces new variations, enhancing the richness and diversity of the music.

\textbf{Step 3: Adding Ornaments and Variations}

In this final step, we enrich the melody by adding musical ornaments and variations. We focus on adding ornaments to only the right-hand melody, randomly selecting 5\% of the notes for embellishment. The ornaments are chosen to be as close as possible to the chord tones of the current triad. This approach includes ornaments such as appoggiaturas (grace notes creating brief dissonance before resolving to the main note), mordents (rapid alternations between the main note and an adjacent note), and trills (quick alternations between two adjacent notes)~\cite{adler1989study}. To implement this, we first determine the total number of notes in the right-hand melody and calculate 5\% of this total to decide how many notes will receive ornaments. We then randomly select these positions within the melody. For each chosen note, we identify the nearest chord tone based on the current harmony and select an appropriate ornament type. When adding the ornaments, we ensure they complement the harmonic structure and maintain the overall flow of the melody. This method effectively increases the expressiveness and complexity of the melody while preserving its essential character and harmonic integrity. Care is taken to use ornaments judiciously, avoiding overuse that might disrupt the melody's fluency, and to ensure their application aligns with the specific musical style and period being emulated.

By combining these three steps - chord progression generation with variations, flexible rhythm pattern application, and ornament addition - our Generation Module creates musically rich and varied content based on the user's input. This multi-faceted approach ensures that the generated music is harmonically sound, rhythmically interesting with a balance of familiarity and novelty, and melodically expressive. This provides users with inspiring and unique musical ideas that both respect the original input and introduce new musical elements.
\subsection{Output and Explanation Module}

The output module writes the generated melody to a new MIDI file, which can be played back to the user or further edited and refined. The explanation component of this module provides insights into the composition at three levels of complexity: Beginner, Intermediate, and Advanced. These explanations focus on three key aspects of the generated melody: chord progressions, rhythm patterns, and embellishments.

\textbf{Beginner Level} The explanation starts with an introduction to basic musical concepts. For chord progressions, it introduces the concept of chords as groups of notes played together, explaining the difference between major and minor chords, and showing how the melody notes relate to these underlying chords. The rhythm explanation at this level covers basic note durations such as quarter notes and eighth notes, and introduces common time signatures like 4/4 and 3/4. It demonstrates how the melody's rhythm fits into these basic patterns. Regarding embellishments, the beginner explanation introduces the concept as extra notes that decorate the main melody, providing simple examples like grace notes or trills.

\textbf{Intermediate Level} This level deepens the musical analysis. For chord progressions, it explains common sequences (e.g., I-IV-V-I) and introduces the concept of harmonic function (tonic, dominant, subdominant), discussing how the chosen progression supports the melody. The rhythm explanation at this level delves into how the input rhythm was matched to one of the 16 predefined patterns and how the other three random variations were created. It also discusses how these rhythms relate to different musical styles. The embellishment explanation introduces more complex decorative techniques like arpeggios or turns, and explains how these relate to the underlying harmony.

\textbf{Advanced Level} At this level, the explanation provides a sophisticated analysis of the composition. The chord progression explanation discusses advanced harmonic concepts such as secondary dominants or modal interchange, explains any modulations or key changes, and analyzes how the chord progression contributes to the overall structure of the piece. The rhythm explanation covers complex concepts like syncopation or polyrhythms, explaining how the rhythm interacts with the harmonic rhythm and contributes to the overall feel or genre of the piece. For embellishments, the advanced explanation discusses techniques like counterpoint or voice leading, explaining how embellishments can create tension and release in the melody and how they contribute to the overall expressiveness of the piece.

What's more, ComposeOn also incorporates a \textbf{Music Theory Mentor Chatbot}, powered by the advanced ChatGPT-4 model, to enhance the user's learning experience and provide on-demand musical expertise. Within the recommendation rationales provided by the system, specialized musical terminology is hyperlinked. When a user clicks on one of these hyperlinked terms, the query is automatically populated in the Music Theory Mentor's input field, situated in a dedicated section of the interface. This mechanism facilitates immediate access to additional information, enabling users to explore complex musical concepts without disrupting their creative flow.

\subsection{ComposeOn User Interface (UI)}
The ComposeOn UI, as shown in Figure~\ref{fig:musicUI}  is a user-centric platform designed to facilitate seamless interaction between users and the ComposeOn composition system. This interface integrates melody continuation, editing, and explanatory functionalities, providing users with a comprehensive music creation and learning environment.

\begin{figure}[h]
\centering
\includegraphics[width=0.8\linewidth]{figs/musicUIillu.png}
\caption{ComposeOn UI illustration. Step1-2, choose a file, and process the file as MIDI. Step3-4, when click on the continue button, a new progression will be added. Step 5-6, click one bar or some notes to check the explanation. Step 7, click the hyperlink for quick chat with a chatbot powered by ChatGPT 4. Step8-9, check the alternative rhythms and chords suitable for the selected bar.}
\Description{ComposeOn System Design Diagram with three modules}
\label{fig:musicUI}
\end{figure}

At the top of the interface, users can dynamically adjust two primary parameters: Beats Per Minute (BPM) and Explanation Level. These controls allow real-time modification of playback speed and the depth of musical analysis provided, respectively.

The composition process begins with the user uploading an audio file in either .mp3 or .wav format, initiating the Upload Module. Subsequently, activating the "Process MIDI" function triggers the Analysis Module, which visualizes the uploaded melody as a musical score on the interface. This score is interactive, allowing playback with synchronized highlighting of the current measure. Concurrently, a virtual piano display in the lower left corner visually represents the notes being played.

Upon user initiation of the "Continue" function, the Generate Module extends the composition, producing a complete chord progression. This extension is seamlessly integrated into the existing musical score visualization. When a user selects any measure of the generated continuation, the system displays explanations calibrated to the preset Explanation Level. These explanations feature hyperlinked musical terminology, enabling users to access more detailed information in the Music Theory Mentor section on the right side of the interface.

For users wishing to edit the generated music, the interface offers measure-specific editing capabilities. By selecting a measure, users can access dropdown menus for "Common Progression Degrees" and "Common Rhythms". These menus present contextually appropriate chord progression degrees and rhythmic patterns, respectively. As users modify these parameters, the system dynamically updates both the musical score and the accompanying explanations, providing immediate feedback on the musical implications of their choices.


\section{Evaluation}\label{sec:Evaluation}
A user study was conducted to evaluate the effectiveness and educational value of ComposeOn. The purpose of the study was to compare the quality and experience of two groups of people - those with no knowledge of music theory and those with some knowledge of music theory - when using ComposeOn and a benchmark method (the Suno music continuation feature) for English lyrics continuation. We paid particular attention to changes in participants' knowledge of music theory. This study aims to answer the following question:
\textbf{Q1:} Does ComposeOn \textbf{\textit{develop}} music better than Suno?
\textbf{Q2:} Does ComposeOn increase more \textbf{\textit{willingness and confidence}} of participants to develop and compose music? 

To better collect and validate the results, we had participants fill out both a pre-study and a post-study questionnaire. the pre-study questionnaire included their demographic information, a simple test of their level of knowledge of music theory, as well as their confidence and motivation about composing and learning to compose. the post-study questionnaire included their judgment of the quality of the continued music, as well as a few indicators of their judgment in SUS ~\cite{r23}. The post-study questionnaire included a quality rating of the continued music, as well as SUS indicator questions, a music theory questionnaire similar to the pre-study questionnaire, and a change in their motivation and confidence about composing.

\subsection{Participant}

\begin{table}[h]
\centering
\begin{tabular}{|c|c|c|c|}
\hline
Participant ID & Music Theory Level & Compose Freq. & Compose Willingness \\
\hline
1 & Beginner & Never & High \\
2 & Intermediate & Rarely & Medium \\
3 & Advanced & Weekly & Very High \\
4 & Beginner & Monthly & Low \\
5 & Intermediate & Daily & High \\
6 & Beginner & Yearly & Medium \\
7 & Advanced & Weekly & High \\
8 & Intermediate & Never & Low \\
9 & Beginner & Rarely & Very High \\
10 & Advanced & Daily & Medium \\
\hline
\end{tabular}
\caption{Participant Information on Music Theory and Composition}
\label{tab:musicinfo}
\end{table}

\subsection{Procedure}
\textbf{Pre-Study Questionnaire}
A pre-study questionnaire is a survey that is used prior to the start of a study or program to gather background information and initial musical knowledge about the participant. The questionnaire usually contains the following questions: \textbf{Basic information}: e.g., name, age, etc. \textbf{Relevant experience}: e.g. previous experience in music making. \textbf{Assessment of prior knowledge}: Tests the participant's current knowledge of the research topic, such as music theory. \textbf{Interest and Motivation}: To find out the participants' level of interest in the topic and their motivation to learn. \textbf{Self-efficacy}: To assess the participant's confidence in his/her ability in the domain.

\textbf{Main Study}
is to have the participants randomly pick one of the 9 melodies we prepared, and then have the participants continue the melody using the ComposeOn and baseline methods. Our baseline is the Suno 3.5 model, participants need to upload the melody file, and Suno will generate at most 4 minutes of continuation. In addition, participants can use ComposeOn to continue the melody, there is no time limit requirement, and users can check the reason for continuing the melody during the process of continuing the melody, make changes to the melody, check the knowledge of music theory through the music theory mentor, and so on.

\textbf{Post-Study Questionnaire}
The post-study questionnaire contained the same \textbf{music theory questions} as the pre-study questionnaire, which was designed to test whether the user's knowledge of music theory increased after using ComposeOn and suno for melodic continuation. In addition, there are questions about the ability of the two instruments to increase compositional \textbf{confidence and motivation}, as well as questions about the use of ComposeOn in the \textbf{SUS framework}, which are about user experience.




\section{Result}\label{sec:Result}
In the post-quesionnaire, we analyzed the music theory correctness and all the subjective assessments in the form of 5-point scores and free comments, Figure~\ref{fig:result} reflects the results of scores for ComposeOn and the baseline method.
\begin{figure}[h]
\centering
\includegraphics[width=0.8\linewidth]{figs/musicresult.png}
\caption{5-point result on the music theory correctness and the subjective assessments.}
\label{fig:result}
\end{figure}

\subsection{ComposeOn develops better music than Suno}
First, the music generated by Suno exhibits more richness and complexity in terms of weaving and orchestration. As noted in P2, P4, and P10, this complexity makes Suno's music potentially more appealing on first listen. One participant might describe it this way, “Suno's music feels rich on first listen, with multiple layers of sound and rich orchestration, giving a first impression of great depth. However, after multiple listens, Suno's music feels less logical. In contrast, ComposeOn's music, although given only melodies with no instrumental layers, feels comfortable after multiple listens because it is very logical.”

However, ComposeOn received high ratings for other aspects of music quality. Multiple participants (P1, P2, P9) emphasized the better structure of the music generated by ComposeOn.P1's comment was particularly specific: “The music generated by Suno lacks structure and sounds like randomly assembled pieces. It sounds like randomly assembled fragments because it has no clear development or ending. In contrast, I could hear (and see) the beginning and end of each phrase of ComposeOn's music, and each phrase ended in a similar pattern, making the music feel more holistic.”

In terms of musical coherence, several participants (P3, P6, P8) noted that ComposeOn showed better consistency in musical sequences.P3's observation was particularly insightful: “Sometimes Suno repeats chords that have been entered previously, but when you hear certain parts of the music, Suno sometimes generates random segments that have little to do with the previous text. This interspersing makes for very little musical consistency in this continuation. By contrast, the musical consistency in ComposeOn's continuation is stable; instead of outputting all the input melody at once and writing it all by itself at once, he assigns features of the input melody to different phrases, so that each phrase has parts that relate to the input melody but are not entirely consistent with it.” This approach makes the music generated by ComposeOn more coherent and natural. In addition, P6 points out the stability of the tempo: “The music written by Suno sometimes accelerates or decelerates suddenly, whereas ComposeOn's tempo is more stable.” This is further evidence of ComposeOn's strength in maintaining musical consistency.

Finally, ComposeOn shows more variation and innovation in its music writing, as articulated in P5's review, “Suno continues music that feels like it's repeating the same chords and melodies without advancing or developing; ComposeOn at least gives me a sense that the music is evolving because it doesn't all sound too much like the previous input.ComposeOn at least gives me ComposeOn at least gives me a sense that the music is evolving, as it doesn't all sound too much like the previous input. It demonstrates ComposeOn's ability to introduce new elements while maintaining musical coherence, making the music more layered and diverse.”

\subsection{ComposeOn increased participants' willingness and confidence to create music}

Our findings indicate that ComposeOn significantly enhanced participants' willingness and confidence in music creation. This improvement can be attributed to two key features of the system: high visualization and interactivity, and high explainability.

\subsubsection{High Visualization and Interactivity}

ComposeOn's interface provides a highly visual and interactive experience, which proved particularly beneficial for novice users. The system displays both user input and AI-recommended melodies on a traditional musical staff, offering real-time visual cues during playback. This feature allows users, especially beginners, to gain a deeper understanding of musical structure, including how notes form chords and how chord progressions work.

The system's interactivity is evident in its user-guided recommendation process. Participants can control the flow of suggestions using the "continue" and "end" buttons, giving them agency over the composition process. Furthermore, ComposeOn offers contextually appropriate chord and rhythm options for each measure, allowing users to experiment with different musical elements. For instance, one participant reported successfully changing a 1-4-5-1 chord progression to a 1-2-5-1 progression, demonstrating the system's flexibility and its capacity to facilitate learning through experimentation.

\subsubsection{High Explainability}

ComposeOn's high explainability feature significantly contributes to users' understanding and confidence. The system provides detailed explanations for its musical continuations, covering aspects such as chord progression, rhythm, and ornamental notes. These explanations are tailored to three proficiency levels - beginner, intermediate, and expert - ensuring that users receive information appropriate to their knowledge level.

The granularity of explanations is noteworthy; users can request explanations for individual notes, measures, or entire phrases. This feature not only clarifies the system's recommendations but also reinforces fundamental musical concepts. As one participant noted, "This feature not only helped explain why certain recommendations were made, but it also helped me remember basic information like chord composition more clearly. I could associate the sound of a chord in a phrase with its composition, an experience that's hard to achieve when learning chords in isolation."

Additionally, the integrated "music theory mentor" feature provides quick access to theoretical knowledge, offering suggested terms and concepts like the circle of fifths. This feature's effectiveness was demonstrated when a participant (P1) first encountered an explanation of the I-VI-V-I chord progression and then used the music theory mentor to gain a deeper understanding of why this progression is harmonically pleasing.

In conclusion, ComposeOn's combination of high visualization, interactivity, and explainability creates a supportive environment for music creation. By demystifying the composition process and providing immediate, context-specific feedback, the system empowers users to engage more confidently with music creation, regardless of their initial skill level. This approach not only facilitates the creation of music but also enhances users' overall musical understanding, potentially contributing to long-term skill development and musical appreciation.

\section{Discussion}\label{sec:Discussion}
\section{Discussion}
In this section, we discuss the key implications of our findings based on our two overarching research questions. Overall, our findings open up interesting opportunities for future research and implications for the industry as a whole.

\subsection{Resilience Of Adaptive Hate Speech Detection Against Poisoning Attacks - RQ1}
Lexicon-based approaches to hate speech detection systems are prone to poisoning attacks.
In a poisoning attack, an adversary intentionally uses safe words in place of toxic words that can cause the model to produce incorrect or biased outputs.
For example, in 2016, 4Chan's /pol/ launched a deliberate attack against Google's Perspective API via the so-called ``Operation Google''~\cite{hine2017kek,operationgoogle}. This attack was designed to poison models by replacing slurs with the names of various tech companies.
For example, instead of saying a slur for a black person, you would say ``Google,'' or instead of a slur for a Jew, you would say ``Skype'' etc.
Poisoning attacks can be challenging to mitigate because they exploit vulnerabilities in the training process of machine learning models.
However, our proposed system can be used to expose it specifically because our goal is to discover how toxic behavior and aggression attacks change over time.
The basic core of our approach is to identify words that are used in a \textit{similar fashion} as known toxic and toxic ones.
We adopt a similarity-based approach thus for each word appearing in our dataset we calculate its vector embedding, extracted from the models built as part of the previous step. We compare this vector with the vector embeddings for all the words in our seed dataset. If the vector for a word has a high similarity (e.g., cosine similarity) with a known toxic word, it is very likely that this word is itself toxic – this is because the two words are used in similar contexts on social media. The output of this phase is a set of words that are likely to be toxic, or used in a toxic way.

\subsection{Hybrid Approach to Hate Speech Detection - RQ2}
By combining the strengths of lexicon-based detection as well as BERT methodologies into a hybrid model we can effectively identify and analyze hate speech in various domains with improved accuracy and contextual understanding. The lexicon-based analysis component leverages pre-defined word lists and sentiment analysis techniques to identify toxic words and sentiments associated with them.
This approach provides a good foundation for detecting explicit risk indicators and capturing straightforward and easily identifiable risk factors.
It allows for quick identification of keywords and phrases commonly associated with risk, enabling efficient detection in real-time scenarios.
On the other hand, the BERT approach, which utilizes a deep learning neural network model, brings contextual understanding and semantic analysis to the hybrid system. BERT enables the model to comprehend the context and nuances of language, capturing the subtleties and complexities of risk factors that may not be explicitly expressed.
This contextual understanding helps the hybrid model to identify implicit risks, detect sarcasm, and recognize risks that might be disguised through various linguistic techniques.
The combination of these two approaches creates a comprehensive risk detection system that combines the advantages of both methods. Secondly, our model also detects implicit hate found in most text online. Unlike explicit hate speech, which uses overtly offensive words or phrases, implied hate speech is more subtle and can be embedded within seemingly innocuous language. Our hybrid model mitigates this limitation by leveraging BERT's contextual understanding.

\subsection{Limitations and Future Work}
\textcolor{black}{In our research, we propose an adaptive methodology to detect toxic language through the utilization of word embeddings. However, it is important to acknowledge that our hybrid approach, despite its numerous strengths, does possess certain limitations.
One notable limitation lies within the lexicon-based analysis employed in our methodology itself. However, our approach reduces this dependency by employing adaptive techniques, allowing for the detection of new toxic words with minimal manual input. This significantly enhances scalability compared to traditional lexicon-based methods. Future work could explore removing older lexicons, and real-time language monitoring to fully automate lexicon updates and improve adaptability to evolving language trends.
Our evaluation also primarily focuses on English-language content, which is a limitation given the global nature of online hate speech. While this allowed us to deeply analyze our approach within a single language, adapting the method to multilingual contexts is crucial for real-world applicability. Hate speech varies significantly across languages and cultures, both in content and contextual nuances. Future work will explore the use of multilingual embeddings (e.g., mBERT, XLM-R) and cross-lingual transfer learning to adapt the approach to other languages. Additionally, we aim to incorporate culturally diverse datasets and expert input to address cross-cultural variations in hate speech detection. Furthermore, it is worth mentioning that recent limitations imposed on using Twitter's APIs have impacted the availability and accessibility of data for research purposes. These limitations may pose challenges in acquiring the necessary data for training and evaluating our model, however, our approach can be mapped to other text-based social media applications especially Threads which is a Meta-owned platform similar in design to Twitter.}





\section{Conclusion}\label{sec:Conclusion}
\section{Conclusion}
In summary, our work takes an adaptive approach to advance hate speech detection approaches on social media.
First, we present our adaptive method to update hate speech lexicons.
We test our approach on existing lexicon-based machine learning models and show that the updated lexicons are better at detecting hate speech.
Then we introduce our hybrid approach that combines the powers of lexicon-based hate speech detection with that of BERT-based models.

% \begin{acks}
% thanks.
% \end{acks}

\bibliographystyle{ACM-Reference-Format}
\bibliography{references}

\end{document}

%% End of file "main.tex".
