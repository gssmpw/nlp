%%%% Small single column format, used for CIE, CSUR, DTRAP, JACM, JDIQ, JEA, JERIC, JETC, PACMCGIT, TAAS, TACCESS, TACO, TALG, TALLIP (formerly TALIP), TCPS, TDSCI, TEAC, TECS, TELO, THRI, TIIS, TIOT, TISSEC, TIST, TKDD, TMIS, TOCE, TOCHI, TOCL, TOCS, TOCT, TODAES, TODS, TOIS, TOIT, TOMACS, TOMM (formerly TOMCCAP), TOMPECS, TOMS, TOPC, TOPLAS, TOPS, TOS, TOSEM, TOSN, TQC, TRETS, TSAS, TSC, TSLP, TWEB.
% \documentclass[acmsmall]{acmart}

%%%% Large single column format, used for IMWUT, JOCCH, PACMPL, POMACS, TAP, PACMHCI
% \documentclass[acmlarge,screen]{acmart}

%%%% Large double column format, used for TOG
% \documentclass[acmtog, authorversion]{acmart}

\documentclass[manuscript]{acmart}
%\documentclass[sigconf]{acmart}

\usepackage{float}
\usepackage{subfigure}
%% \BibTeX command to typeset BibTeX logo in the docs
\AtBeginDocument{%
  \providecommand\BibTeX{{%
    \normalfont B\kern-0.5em{\scshape i\kern-0.25em b}\kern-0.8em\TeX}}}
\usepackage{tabularx}
\usepackage{bbding}
\usepackage{graphicx}
\usepackage{geometry}
\usepackage{subfigure}
\usepackage{amsmath}
\usepackage{makecell}
\usepackage{float}
\usepackage{color}
\usepackage{booktabs}
\usepackage{caption}
\usepackage{tabu}
\usepackage{hyperref}

%% Rights management information.  This information is sent to you
%% when you complete the rights form. CHANGE THIS:
\copyrightyear{2025}
\acmYear{2025}
\setcopyright{acmlicensed}\acmConference[CHI '25]{Proceedings of the 2025 CHI Conference on Human Factors in Computing Systems}{April 26-- May 1, 2025}{Yokohama, Japan}
\acmBooktitle{Proceedings of the 2025 CHI Conference on Human Factors in Computing Systems (CHI '25), April 26-- May 1, 2025, Yokohama, Japan}
\acmDOI{XXXXXXX.XXXXXXX}

%% These commands are for a PROCEEDINGS abstract or paper.
\newcommand{\remove}[1]{{\color{red} \sout{#1}}}
%\newcommand{\change}[1]{{\color{blue} \uwave{#1}}}
\newcommand{\add}[1]{{\color{blue} #1}}
%%
%% Submission ID.
%% Use this when submitting an article to a sponsored event. You'll
%% receive a unique submission ID from the organizers
%% of the event, and this ID should be used as the parameter to this command.
%\acmSubmissionID{123-A56-BU3}
% \author{Zheng Chen$^{*}$}

% \email{zchenin@connect.ust.hk}

% \affiliation{
%   \institution{Hong Kong University of Science and Technology}
%   \city{Hong Kong}
%   \country{Hong Kong}
% }
\author{Hongxi Pu$^{*}$}
\email{hongxi@umich.edu}

\affiliation{
  \institution{University of Michigan}
  \city{Ann Arbor}
  \country{United States}
}

\author{Futian Jiang}


\affiliation{
  \institution{University of Hong Kong}
  \city{Hong Kong}
  \country{Hong Kong}
}



\author{Zihao Chen}



\affiliation{
  \institution{Brooklyn College}
  \city{NYC}
  \country{United States}
}

\author{Xingyue Song}

% \email{xysong1213@gmail.com}

\affiliation{
  \institution{San Francisco Conservatory of Music}
  \city{San Francisco}
  \country{United States}
}

\renewcommand{\shortauthors}{}
%%
%% end of the preamble, start of the body of the document source.

%%\citestyle{acmauthoryear}
\begin{document}
%%\citestyle{authoryear}
%%
%% The "title" command has an optional parameter,
%% allowing the author to define a "short title" to be used in page headers.
\title[ComposeOn Academy]{ComposeOn Academy: Transforming Melodic Ideas into Complete Compositions Integrating Music Learning}

\begin{abstract}
Music composition has long been recognized as a significant art form. However, existing digital audio workstations and music production software often present high entry barriers for users lacking formal musical training. To address this, we introduce ComposeOn, a music theory-based tool designed for users with limited musical knowledge. ComposeOn enables users to easily extend their melodic ideas into complete compositions and offers simple editing features. By integrating music theory, it explains music creation at beginner, intermediate, and advanced levels. Our user study (N=10) compared ComposeOn with the baseline method, Suno AI, demonstrating that ComposeOn provides a more accessible and enjoyable composing and learning experience for individuals with limited musical skills. ComposeOn bridges the gap between theory and practice, offering an innovative solution as both a composition aid and music education platform. The study also explores the differences between theory-based music creation and generative music, highlighting the former's advantages in personal expression and learning.
\end{abstract}

%%
%% The code below is copied from, generated by the tool at http://dl.acm.org/ccs.cfm.
\begin{CCSXML}
<ccs2012>
   <concept>
       <concept_id>10003120.10003130.10011762</concept_id>
       <concept_desc>Human-centered computing~Interactive System and Tools</concept_desc>
       <concept_significance>500</concept_significance>
       </concept>
 </ccs2012>
\end{CCSXML}
\ccsdesc[500]{Human-centered computing~Interactive System and Tools}

%%
%% Keywords.
\keywords{Sound and music computing, Music composition, Music theory E-learning, Digital Audio Workstation, Novice Users }

\begin{teaserfigure}
\centering
\includegraphics[width=0.5\linewidth]{figs/musicbegin.png}
\caption{ComposeOn takes the melody input to compose complete music, teach music theory and allow music edition.}
\Description{ComposeOn System Design Diagram with three modules}
\label{fig:system_design_diagram}
\end{teaserfigure}
% \begin{teaserfigure}
% \centering
% \subfigure[]{
% \includegraphics[width=0.305\textwidth, trim= 0 0 40 3]{figs/cover1.JPG}\label{fig1a}
% }\hspace{1mm}
% \subfigure[]{
% \includegraphics[width=0.305\textwidth, trim= 0 0 50 0]{figs/cover2.JPG}\label{fig1b}
% }\hspace{1mm}
% \subfigure[]{
% \includegraphics[width=0.315\textwidth, trim= 0 0 50 0]{figs/cover3.JPG}
% \label{fig:01}
% }
% \caption{Caption}
% \Description{Caption}
% \end{teaserfigure}

%%
%% This command processes the author and affiliation and title
%% information and builds the first part of the formatted document.
\maketitle

\section{Introduction}\label{sec:Introduction}
%motivation
The prevalence of misinformation is a growing concern globally. Misinformation damages society in numerous ways. It threatens the maintenance of trust in vaccines and health policies\cite{do2022infodemics,macdonald2023meme}, can incite violence and harassment\cite{cdtFromFellows}, can undermine democratic processes (particularly elections)\cite{bovet2019influence,groshek2017helping}, and harms individual and societal well-being\cite{verma2022examining}. For example, during the 2018 Brazilian presidential election, manipulated photos, decontextualized videos, and hoax audio significantly influenced election results by aiding the victory of the main far-right candidate \cite{theguardianWhatsAppFake,santos2020social}. Another example is the spread of false information during the COVID-19 pandemic. Many people were persuaded that ineffective or counterproductive treatments such as alcohol-based cleaning products and an anti-parasitic drug could cure patients. The victims of this misinformation could suffer serious illness or even death \cite{bbcHundredsDead}.

Different countermeasures have been investigated to combat misinformation. These interventions can be categorized into two major groups: pre-emptive intervention ("prebunking") and reactive intervention ("debunking")\cite{ecker2022psychological}. Debunking involves fact-checking and correcting misinformation after it has been encountered. Fact-checkers use investigative practices to determine the veracity of content and dispute factual inaccuracies\cite{juneja2022human}.  %The most common debunking approach uses fact-checked information to dispute misinformation and provide accurate information\cite{hameleers2020misinformation,facebookIntoFacebook,aghajari2023reviewing}. 

%difficulties of overcoming misinfo
%describe debunking and spreading info. "two aspects of misinfo: how it spreads and how to debunk it once it's here."
However, the lasting effects of misinformation make it challenging to mitigate its influence once people have been exposed \cite{roets2017fake,lewandowsky2012misinformation}. Furthermore, fact-checking efforts are limited in terms of scale and reach, which restricts their effectiveness\cite{roozenbeek2020prebunking}. Given these challenges, it is not sufficient to rely purely on debunking efforts. Prebunking, on the other hand, works to build attitudinal inoculation, enabling people to identify and resist manipulative messages. This approach equips individuals to better manage misinformation they encounter in the real world\cite{prebin2024}. Prebunking is based on a range of educational measures \cite{dame2022combating,cook2017neutralizing}, which can include games \cite{roozenbeek2019fake,cook2023cranky}.

%helping by teaching people  

%preemptive corrections before misinformation is encountered, effectively reducing reliance on misinformation. 

%This includes digital media literacy education activities\cite{dame2022combating,cook2017neutralizing} and designing serious games to improve media literacy\cite{roozenbeek2019fake,cook2023cranky}. 
%Prebunking becoming an important 



%valuable complement by raising people's awareness and understanding of misinformation, helping them more effectively navigate credible, biased, and false information. One promising application of prebunking is through game-based learning. 

%games + research gap
Recent research indicates that the game-based learning approaches are potentially useful prebunking interventions. These games educate and build resistance in people exposed to misinformation\cite{traberg2022psychological,kiili2024tackling}. In games such as "Bad News,"\cite{roozenbeek2019fake} "Harmony Square,"\cite{harmonysquare} "Go Viral!"\cite{camCambridgeGame} and "Trustme!"\cite{yang2021can} players adopt the role of a misinformation producer whose task is to create and spread misinformation as efficiently as possible. Another approach, applied in the games "MAthE"\cite{katsaounidou2019mathe} and "Escape Fake"\cite{escapefake}, involves players acting as fact-checkers and challenging them to assess the validity of information. In some scenarios players can use tools such as search engines to gather clues. 

%the problem with standard PVP games (deterministic games), lack of natural interaction, use common language to explain better. doesn't feel real world. 
While gamified misinformation interventions have shown promise in teaching players about the nature of misinformation, %either from the perspective of the creator or the audience, 
the choice-based formats of these games can limit replayability. %and fun that make games engaging.
%Another Moreover, these games often present only one-shot interactions with misinformation, which may not capture the complexity and ever-changing nature of misinformation in the real world\cite{shin2018diffusion}. 
Additionally, such choice-based game formats require little cognitive effort from the player, who is presented with a limited number of pre-generated options. This can diminish player involvement in the game and reduce enjoyment of gameplay. Another factor limiting engagement is that most of these games are designed for single-player mode. Multiplayer mechanics, by contrast, are an effective way to enhance motivation and replayability in games\cite{mustaro2012immersion}. Social interactions also significantly contribute to player engagement in educational and other serious games\cite{lepper2021intrinsic}. Research has demonstrated that both collaborative and competitive gameplay can enhance the effectiveness of serious games\cite{cagiltay2015effect,bellotti2010designing}.
%and increase player motivation \cite{cagiltay2015effect,bellotti2010designing}.

%GAMES: 2 problems:
%1. not natural, not realistic, deterministic -> LLM solution.
%2. engagement and attitude -> two-player PvP (one against another goal).

To address the challenges of non-natural interactions and a deterministic game paths (with a view to fostering more open-ended exploration and engagement through Player versus Player (PvP)), we need to incorporate more complex scenarios. These would give players the ability to not only choose from preselected options, but also to actively generate content and implement their own strategies for debunking or creating misinformation. Additionally, implementing a PvP approach can be a training tool through which players can learn how to counter real-life misinformation.

%critically discern misinformation.


%During the game, players learn how several misinformation manipulation techniques can be used to produce credible fake news\cite{roozenbeek2019fake}. Another type of game involves players acting as fact-checkers whose task is to identify misinformation or fake news\cite{katsaounidou2019mathe,escapefake}.

%However, these games may overlook key features of misinformation. One noticeable feature is that misinformation typically involves more than isolated instances. During significant events like elections, wars, or health crises, misinformation evolves across various formats (text, pictures, infographics, videos) and stages. For example, during the COVID-19 pandemic, misinformation ranged from false cures and conspiracy theories about the virus's origin to vaccine misinformation, each gaining prominence at different stages[REF]. Another feature is that misinformation is constantly changing. Research found that political false rumors tend to become more intense and extreme over time\cite{shin2018diffusion}, and health misinformation statuses change as new evidence emerges\cite{tang2024knows}.

%evolution idea - using LLM to model public opinion.
%Therefore, there is a need to design a game that not only enhances the ability to distinguish misinformation but also raises awareness of its evolving characteristics and various forms. 
%Advancements in AI and Natural Language Processing (NLP) open the opportunities for creating engaging games. Large language models (LLMs), such as ChatGPT, have been applied in video games for generative narratives\cite{park2023generative}, NPC dialogue\cite{ashby2023personalized,uludaugli2023non}, and role-playing\cite{xu2023exploring}. 

%Integrating LLMs into misinformation learning games could dynamically adapt to player interactions, providing a more engaging and personalized experience. This is particularly useful for capturing and reflecting the dynamic nature of misinformation events. However, further exploration is needed to detail this integration.


Progress in AI and Natural Language Processing (NLP) provide new opportunities for creating more engaging game experiences. Large language models (LLMs) can simulate complex human interactions and societal dynamics\cite{ziems2024can}and have been applied in video games for tasks like generating narratives\cite{park2023generative}, non-playable characters (NPCs),dialogue\cite{ashby2023personalized,uludaugli2023non}, and role-playing scenarios\cite{xu2023exploring}. 
Previous work also demonstrated the opportunities for prompting the LLM to impersonate a specific character and create interactive dialogues from this perspective\cite{zhou2024eternagram,shao2023character}. Integrating LLMs into the game can introduce greater variability to in-game interactions and enhances both engagement and replayability. As the effectiveness of many inoculation interventions tends to diminish over time, developing an enjoyable and replayable game that consistently reinforces players' cognitive "resilience" against misinformation is an essential advancement in this field\cite{wells2024doomscroll}. Applying LLM to the game also has the potential to increase the educational impact of the intervention. It allows users not just to select from a range of options but also to put their input into the model and receive individual feedback tailored to this input.

%To bridge the gap in misinformation education games and explore LLM-driven games, 
Inspired by previous misinformation game interventions, this research has involved the development of a PvP misinformation education game called \textit{Breaking the News}. In our game, two players are assigned either the role of a misinformation creator (referred to as an "influencer" in the game) or a counteractor of misinformation (referred to as a "journalist-debunker" in the game). The influencer creates misinformation posts in a system that mimics a social media environment, while the debunker seeks to counter these messages by presenting compelling arguments. LLMs are used to represent public opinion in the "country" where the game events take place. The goal for both players is to earn the trust of the citizens and convince them to believe the information they present. 

In this paper we aim to answer the following research questions: 

\textbf{RQ1:}  
How may we empower players to understand the processes of misinformation generation and misinformation debunking through a GenAI-based PvP game?

\textbf{RQ2:} 
What behaviors do players exhibit when they are asked to generate versus protect against misinformation?
 
In this paper, we present the design and evaluation of the PvP game. We conducted a mixed-method study with 47 participants, using a within-subjects design and pre- and post-surveys for repeated measures. Our findings suggest that through gameplay, participants improved their ability to reflect on instances of misinformation, raised their levels of media literacy, expanded their repertoire of strategies applied to countering misinformation, and improved their discernment abilities. This study contributes to the growing body of work analyzing misinformation education games. Specifically, we provide insights into integrating LLMs and interactive PvP mechanics in media literacy contexts. We also offer practical guidance for the design of serious games aimed at combating misinformation in a dynamic, real-world manner.

%based on events rather than isolated posts or headlines. %In our game, %two different parties (debunker and disinformation creator (influencer) are battling for influence on people's opinion about event.control, and the hearts and minds of their citizens and the global community.
%two players are assigned either the role of a misinformation maker (influencer) or a misinformation stopper (debunker). %One player will start with an unfolding event and will experience the creation and dissemination of information and misinformation in various formats and stages. The other player will have to identify misinformation and use different countermeasures to respond and combat misinformation in the game. After the second player's responses to misinformation, they will receive simulated reactions from a group of citizens, generated by LLM, and decide on the next steps. After a few rounds, mimicking misinformation diffusion patterns, the game will have results. 

%The influencer will create misinformation posts, while the debunker tries to resolve the issue by proposing compelling arguments. The LLM-4 is representing represent the public opinion of several citizens in the "Country" where the game events take place.

%The goal for both players is to earn the trust of the citizens and convince them to believe the information they present.

%to align more closely with their direction. Once the game is finished, the players will have a chance to review their strategies at different stages. 

%The game aimed to enhance players' ability to reflect on instances of misinformation, raise their media literacy, expand their repertoire of strategies applied to counter misinformation, and improve their discernment abilities.

%for real-world encounters.
%Define player versus player.


%needs to be more conservative here:
%RQ2: What behaviors do players exhibit when they are asked to generate and protect against misinformation?

%what study did you do?
%what are the 3 major findings?
%our contribution (short version).


\section{Related Work}\label{sec:Related Work}
\subsection{Compose Theory}
ComposeOn promotes people without a music background to compose their own music. We design the system to detect melody users put in and give suggestions for continuations based on basic music and harmony progression theory. "Western music written during Baroque, Classical, and Romantic periods (ca.1650-ca.1900) is called tonal music, which has a point of gravitation called tonic." \citep{laitz2008complete} The keys and scales gradually formed based on that and thus formed tonal hierarchy and harmony function theory.

"The music of the tonal era is almost exclusively tertian, which means being constructed of stacked 3rd." \citep{kostka2006materials} Chords are marked with their root notes (where the stack begins) using roman numerals. Functionally, they are basically divided into Tonic chords: I, Dominant chord: V, and predominant/subdominant chords: ii, iv, vi. The harmony often progresses as Tonic–Subdominant–Dominant–Tonic. Sometimes may use iii or Vii, and different 7th chords, but their function is various depending on the texture \citep{aldwell2010harmony}.

There are also uses out of this basic progression, such as sequence, modulation or transportation. We also considered those situations with limited possibilities within our database.

Based on the above theory, we suggest notes within the key and the possible harmony progressions. As to music phrases, we followed basic 2+2/4+4 (refer to measure numbers) to form the music phrase and thus sentence and sections. \citep{schoenberg1967fundamentals} We mainly focus on songwriting, so the intro, verse — chorus — verse — chorus —bridge — chorus — outro structure of song is also being considered. \citep{masterclass2021songwriting}

\subsection{Voice to MIDI Technology and Its Applications}

MIDI (Musical Instrument Digital Interface) is a technical standard that describes a protocol, digital interface, and connectors, allowing various electronic musical instruments, computers, and other related devices to connect and communicate with each other \citep{midi1996complete}. Voice to MIDI technology is the process of converting vocal or other audio signals into this MIDI data format, playing a crucial role in music production and analysis. This technology involves multiple steps, including pitch detection, note segmentation, quantization, and MIDI conversion. Pitch detection typically employs algorithms such as YIN \citep{decheveigne2002yin} or pYIN \citep{mauch2014pyin} to estimate the fundamental frequency of audio. Subsequently, the continuous pitch sequence is segmented into discrete notes, which are then time-aligned to a musical grid. Finally, the detected note information is converted into MIDI events. Voice-to-MIDI technology has found applications in various fields, such as quickly converting hummed melodies into MIDI for song composition, providing instant feedback to students in music education, and generating real-time music based on user voice input in interactive music systems. This technology not only simplifies the music creation process but also provides powerful tools for music analysis and education.

\subsection{Automated Melody Analysis}

Automated melody analysis is a significant branch of music information retrieval, aimed at extracting and analyzing melodic features from musical data. This process typically includes the analysis of notes, chords, and chord progressions. Note analysis involves extracting attributes such as pitch, duration, and velocity. Chord analysis focuses on identifying combinations of simultaneously sounding notes, often using algorithms like Chordino \citep{mauch2010simultaneous}. Chord progression analysis examines patterns in the series of chords, utilizing methods such as hidden Markov models \citep{rohrmeier2012comparing}. In this field, Musicpy, a powerful Python music programming library, provides extensive functional support. It not only has concise syntax for representing various musical elements but also incorporates a complete music theory system supporting advanced musical operations. Musicpy's core data structures include notes, keys, chords, scales, etc., offering various practical functions including chord identification, melody analysis, and automatic composition. Through Musicpy, researchers and music creators can conveniently achieve automated melody analysis, explore musical structures and characteristics. Notably, the ComposeOn project extensively utilizes Musicpy's powerful capabilities, particularly in chord extraction and chord progression analysis. ComposeOn employs Musicpy's algorithms to identify and analyze the progression patterns of these chords, thereby providing an important foundation for music analysis and creation. This application demonstrates the practicality and effectiveness of Musicpy in real-world music analysis projects.

\subsection{Automatic Accompaniment Generation}
Accompaniment generation is referred to as "the audio realization of a chord sequence"
by systems like MySong \cite{simon2008mysong}, which represents a significant advancement in the field of automatic accompaniment generation for vocal melodies. MySong allows users to input vocal melodies, which the system then inputs to a hidden Markov model to recommend chords. However, MySong's capabilities are limited to generating accompaniments, whereas our ComposeOn system empowers users to easily extend and develop their melodic ideas into complete compositions, providing a more comprehensive music creation and learning experience.


\section{Formative Study}\label{sec:Formative Study}
To understand the needs and challenges faced by individuals with little to no music theory knowledge in composing music, we conducted a formative study with six participants (FP1-FP6), aiming at exploring their willingness and confidence in music composition, as well as their experiences with existing music composition tools, the demographics and basic music composition information are shown in \ref{tab:music_composition}.
\begin{table}[h]
\centering
\resizebox{\textwidth}{!}{%
\begin{tabular}{|c|c|c|c|p{5cm}|}
\hline
\textbf{Labels} & \textbf{Music Theory Level} & \textbf{Compose Willingness} & \textbf{Compose Confidence} & \textbf{Compose Tools} \\
\hline
FP1 & Intermediate & Moderate & Moderate & Suno AI \\

FP2 & Beginner & High & Low & GarageBand, Suno AI \\

FP3 & Beginner & High & Low & Logic Pro, Fruity Loops \\

FP4 & Intermediate & High & Moderate & Suno AI \\

FP5 & Beginner & Moderate & Low & Never \\

FP6 & Intermediate & High & Low & Udio \\
\hline
\end{tabular}%
}
\caption{Music Composition Profile of Participants}
\label{tab:music_composition}
\end{table}

Most participants expressed a strong desire to compose music, with many having attempted to use various music composition tools in the past. However, they encountered significant barriers due to the tools' requirements for \textbf{basic music theory knowledge}. For instance, FP2 mentioned, \textit{"I tried GarageBand~\cite{r1}, but I struggled with identifying piano keys and understanding orchestration techniques".} Other composition tools, such as Logic Pro~\cite{r2} ~\cite{r4}and Fruity Loops Studio~~\cite{r3}~\cite{r5}, were also cited as challenging for beginners due to their requirements for music theory knowledge.

Recent developments in AI-powered music generation, such as Udio~\cite{r8} and Suno AI~\cite{r7}~\cite{r6}, were acknowledged by participants as interesting music composition tools. However, users expressed concerns about the \textbf{low interpretability} and \textbf{limited control} in AI music generation, leading to a sense of disconnection from the creative process. FP2 remarked, \textit{"It's fascinating to see what the AI can produce, but I often feel like I'm just pressing buttons rather than truly composing."} Rather than feeling like they were composing music themselves, participants viewed these tools more as a form of entertainment or game. FP1 elaborated on this sentiment, stating, \textit{"It's fun to play around with, but I don't feel like I'm learning or improving my musical skills."} These characteristics also \textbf{hindered users' from learning music theory} from AI-generated compositions and did not increase their their own music composition confidence. FP4 noted, \textit{"Suno can generate and extend music, but I feel limited in my ability to adjust the output or incorporate my own creative ideas".} This sentiment highlights a gap between the capabilities of AI-generated music and the desire for personal creative input. FP6 further emphasized this point, saying, \textit{"I want to understand why certain musical choices are made, but the AI doesn't provide that insight. It's like being given a finished painting without learning how to mix colors or use brushstrokes."} Several other participants echoed similar concerns, emphasizing the importance of maintaining creative agency and the ability to learn and grow as musicians through the composition process.

Despite their lack of formal music theory knowledge, all participants reported frequently experiencing melodic inspirations. FP5 expressed, \textit{"I often have tunes in my head that I'd love to develop into full songs, but I don't know where to start".} This desire to \textbf{expand on their melodic ideas and potentially create complete songs} was a common theme among participants. However, the participants generally lacked the motivation to undertake extensive music theory study, from understanding notes to learning chord progressions, as a prerequisite to composition. FP3 stated, \textit{"The idea of studying music theory from scratch before I can start composing is daunting and discouraging".} Interestingly, when presented with the concept of learning music theory gradually through the composition process, participants showed increased enthusiasm. FP5 commented, \textit{"If I could learn about music while actually creating something, I'd be much more motivated to stick with it".}

This formative study revealed a clear need for a music composition tool that caters to beginners or intermediate music theory learners, allowing them to \textbf{translate their melodic ideas into compositions} without extensive prior knowledge. Additionally, the generated music should be \textbf{easy for users to edit}, giving them control over their creations. Furthermore, it highlighted the potential value of integrating \textbf{music theory learning} into the composition process itself, potentially increasing user engagement and long-term commitment to music creation.

\section{System Design}\label{sec:System Design}
Our system, ComposeOn, is designed to facilitate music extension and learning for users with little or no musical theory background. By analyzing the results of the formative study mentioned in Section \ref{sec:Formative Study}, we identified three design goals for ComposeOn:

\begin{itemize}
\item Melody Expansion: Enable users to easily extend and develop their melodic ideas into complete compositions.
\item Easy Editing: Allow users to effortlessly edit and control their music creations.
\item Music Theory Integration: Incorporate music theory learning to boost engagement and interest.
\end{itemize}

To generate the extended part of music based on the input melody, our system consists of three main components: the input \& analysis module, the generation module, and the output \& explanation module. Figure \ref{fig:system_design_diagram} shows a high-level diagram of the system design.
\begin{figure}[h]
\centering
\includegraphics[width=0.8\linewidth]{figs/system_design.png}
\caption{ComposeOn System Design Diagram: the input and analysis module, represented by light-blue color; the generation module, represented by light-green color; and the output and explanation module, represented by light-yellow color.}
\Description{ComposeOn System Design Diagram with three modules}
\label{fig:system_design_diagram}
\end{figure}

\subsection{ComposeOn Database}

The ComposeOn database consists of two main parts: common chord progressions and common rhythm patterns. These data provide ComposeOn with fundamental materials and references.

The \textbf{chord progression section} contains 39 different chord sequences across seven categories, covering various music styles from basic to advanced. These chord progressions are selected based on the principles of Functional Harmony and Tonal Harmony in music theory\cite{kostka2008harmonia,clendinning2016musician}. Specifically, it includes (1) 9 classic chord progressions, such as \texttt{I–IV–V–I} and \texttt{vi–IV–V–I}, common in pop and rock music; (2) 9 extended chord progressions, like \texttt{Imaj7–ii7–V7–Imaj7}, suitable for jazz and blues; (3) 4 diminished triad progressions, such as \texttt{i–iidim–V7–i}, used to create tense or dissonant effects; (4) 4 augmented fourth chord progressions, like \texttt{I–IV–aug4–I}, used to add harmonic color; (5) 5 mixed chord progressions, such as \texttt{Imaj7–ii7–V7–IVmaj7}, blending different types of chords; (6) 4 substitute chord progressions, like \texttt{Imaj7–bIImaj7–V7–Imaj7}, common in modern jazz; and (7) 4 cycle chord progressions, such as \texttt{Imaj7–ii7–V7–iii7}, used to create repeating sections or build-ups. The selection of these chord progressions also references popular music composition practices and jazz theory\cite{levine2011jazz,mulholland2013berklee}.

The \textbf{rhythm pattern section} contains 16 different rhythm types, each composed of specific combinations of notes and rests~\cite{r9}. These patterns cover various popular music styles, including pop, rock, reggae, ska, jazz, funk, blues, and country music. For example, the first pattern \texttt{[(1, 'rest'), (1, 'note'), (1, 'rest'), (1, 'note')]} represents a simple, regular rhythm, while the seventh pattern \texttt{[(1/3, 'note'), (1/3, 'note'), (1/3, 'note')] * 4} represents a more complex, triplet-based rhythm. These patterns provide ComposeOn with a rich selection of rhythms, enabling it to generate melodies that conform to specific musical styles.

\subsection{Input and Analysis Module}

To allow users to easily express their musical ideas, we have designed a flexible input module. Users can input their melodies by singing, humming, or playing an instrument. To ensure an accurate capture of users' musical creativity, we employ the advanced Basic Pitch library~\cite{r10}. This neural network-based pitch detection model can accurately convert users' audio input into standard MIDI file format, handling even complex polyphonic melodies with precision.

Once the user's input is converted into a MIDI file, the analysis module begins its work, delving deep into the musical characteristics of the melody. We use the powerful MusicPy~\cite{r11} library to process this MIDI file, extracting rich musical information. First, ComposeOn identifies the implied \textbf{chord progressions} in the melody, for example, the chords present in the user's input, recognizing them as D and G.. This provides insight into the harmonic foundation of the melody. Next, ComposeOn uses the detect\_scale function to determine the \textbf{scales} most likely used in the melody, for example, the system determines that the scale being used is D major.. To standardize the melody's harmonic structure, ComposeOn converts the identified chords into \textbf{scale degrees}, within the context of D major: D is identified as the I (tonic) chord, and G is identified as the IV (subdominant) chord. This process entails matching chords with detected scales, determining their position within the scale, and addressing special cases for minor scales. This conversion provides a more abstract, theoretical representation of the harmonic structure, allowing for a unified generation logic to be applied regardless of the original scale or key.

Finally, ComposeOn integrates all these analysis results, creating a detailed profile of the input melody's characteristics, and serving as a basis of the explanations. The standardized scale degree representation enables ComposeOn to apply consistent generation algorithms across various musical contexts.

\subsection{Generation Module}

The Generation Module is responsible for creating new musical content based on the analysis of the user's input. This process is divided into three main steps: recommending chord progressions, recommending rhythms for right-hand rhythm, and adding ornaments and variations. The system consults its chord progression database and identifies a common progression pattern that incorporates the input chords: [I - IV - V - I];[I - IV - ii - V - I];  [I - IV - vii dim - I];  [I - IV - V7 - I] etc.

\textbf{Step 1: Recommending Chord Progressions}

In the first step, we begin with chord degrees from the previous step of the Analysis Module and find the recommended chord degrees. When the user clicks the "Continue" button in the UI, we sequentially identify the most similar progressions, utilizing the SequenceMatcher class from the difflib library ~\cite{pythondifflib} to calculate the sequence similarity between the input chord progression and our predefined progressions through their chord degrees.  Each click recommends a complete progression. This recommended progression is then combined with the original input progression to form a new progression base. On subsequent "continue" clicks, this combined progression serves as the new input. 

Then, we convert these degrees into absolute chords based on the scales retained from earlier steps of the Analysis Module. For instance, if our recommended chord degree sequence is [I, IV, V, I], and the retained scale is D major, the resulting chord progression would be [D, G, A, D]. This determined chord progression forms the basis for both the left-hand and right-hand melodies.

For the left-hand part, we utilize the triads of these chords, playing them as whole notes for each measure. In the case of [D, G, A, D], the left hand would play D-F\#-A (D major triad) for the first measure, G-B-D (G major triad) for the second measure, A-C\#-E (A major triad) for the third measure, and D-F\#-A (D major triad) for the fourth measure. This approach provides a solid harmonic foundation using simple, sustained chords in the left hand.

Meanwhile, the same chord progression serves as the foundation for developing the right-hand melody, allowing for more intricate melodic and rhythmic patterns that complement the underlying harmonic structure. For example, the right-hand melody might incorporate arpeggios or scalar passages derived from the D major scale, with emphasis on the chord tones of each underlying harmony.

This method of chord realization and melody generation demonstrates how the system can translate abstract music theory concepts into concrete musical elements. By providing a clear harmonic foundation in the left hand and a related but more elaborate melody in the right hand, the system creates a balanced and musically coherent output that is accessible to novice users while still adhering to established musical principles.

Furthermore, this approach can be extended to the other progression options identified earlier:

\begin{enumerate}
    \item For [I, IV, ii, V, I] in D major: [D, G, Em, A, D]
    \item For [I, IV, vii°, I] in D major: [D, G, C\# dim, D]
    \item For [I, IV, V7, I] in D major: [D, G, A7, D]
\end{enumerate}


To introduce uniqueness, we then apply variations to these selected progressions. This variation technique is based on the concept of chord substitution in music theory. According to Levine (2011), chords can often be substituted with chords that share common tones or have a similar function within the key\cite{levine2011jazz}. Our implementation focuses on diatonic substitutions, where chords are replaced by others from the same key, maintaining harmonic coherence while introducing variety. It's worth noting that we treat each complete chord progression as a musical phrase, providing a structural basis for subsequent melody generation.

\textbf{Step 2: Recommending Rhythms for the right-hand melody}

In the second step, we employ the rhythm pattern pool introduced in the Section of the ComposeOn database. We begin by fitting the input rhythm to this rhythm pattern database to find the closest match and randomly chose two more patterns in our rhythm pattern pool. For each complete chord progression, we then apply the following strategy for rhythm patterns: The first measure of each phrase always uses the rhythm pattern fitted to the input. This ensures that the generated music retains the rhythmic characteristics of the original input, maintaining musical coherence. For subsequent measures within the phrase, we select the randomly chosen two rhythm patterns. This approach both maintains a connection to the original input and introduces new variations, enhancing the richness and diversity of the music.

\textbf{Step 3: Adding Ornaments and Variations}

In this final step, we enrich the melody by adding musical ornaments and variations. We focus on adding ornaments to only the right-hand melody, randomly selecting 5\% of the notes for embellishment. The ornaments are chosen to be as close as possible to the chord tones of the current triad. This approach includes ornaments such as appoggiaturas (grace notes creating brief dissonance before resolving to the main note), mordents (rapid alternations between the main note and an adjacent note), and trills (quick alternations between two adjacent notes)~\cite{adler1989study}. To implement this, we first determine the total number of notes in the right-hand melody and calculate 5\% of this total to decide how many notes will receive ornaments. We then randomly select these positions within the melody. For each chosen note, we identify the nearest chord tone based on the current harmony and select an appropriate ornament type. When adding the ornaments, we ensure they complement the harmonic structure and maintain the overall flow of the melody. This method effectively increases the expressiveness and complexity of the melody while preserving its essential character and harmonic integrity. Care is taken to use ornaments judiciously, avoiding overuse that might disrupt the melody's fluency, and to ensure their application aligns with the specific musical style and period being emulated.

By combining these three steps - chord progression generation with variations, flexible rhythm pattern application, and ornament addition - our Generation Module creates musically rich and varied content based on the user's input. This multi-faceted approach ensures that the generated music is harmonically sound, rhythmically interesting with a balance of familiarity and novelty, and melodically expressive. This provides users with inspiring and unique musical ideas that both respect the original input and introduce new musical elements.
\subsection{Output and Explanation Module}

The output module writes the generated melody to a new MIDI file, which can be played back to the user or further edited and refined. The explanation component of this module provides insights into the composition at three levels of complexity: Beginner, Intermediate, and Advanced. These explanations focus on three key aspects of the generated melody: chord progressions, rhythm patterns, and embellishments.

\textbf{Beginner Level} The explanation starts with an introduction to basic musical concepts. For chord progressions, it introduces the concept of chords as groups of notes played together, explaining the difference between major and minor chords, and showing how the melody notes relate to these underlying chords. The rhythm explanation at this level covers basic note durations such as quarter notes and eighth notes, and introduces common time signatures like 4/4 and 3/4. It demonstrates how the melody's rhythm fits into these basic patterns. Regarding embellishments, the beginner explanation introduces the concept as extra notes that decorate the main melody, providing simple examples like grace notes or trills.

\textbf{Intermediate Level} This level deepens the musical analysis. For chord progressions, it explains common sequences (e.g., I-IV-V-I) and introduces the concept of harmonic function (tonic, dominant, subdominant), discussing how the chosen progression supports the melody. The rhythm explanation at this level delves into how the input rhythm was matched to one of the 16 predefined patterns and how the other three random variations were created. It also discusses how these rhythms relate to different musical styles. The embellishment explanation introduces more complex decorative techniques like arpeggios or turns, and explains how these relate to the underlying harmony.

\textbf{Advanced Level} At this level, the explanation provides a sophisticated analysis of the composition. The chord progression explanation discusses advanced harmonic concepts such as secondary dominants or modal interchange, explains any modulations or key changes, and analyzes how the chord progression contributes to the overall structure of the piece. The rhythm explanation covers complex concepts like syncopation or polyrhythms, explaining how the rhythm interacts with the harmonic rhythm and contributes to the overall feel or genre of the piece. For embellishments, the advanced explanation discusses techniques like counterpoint or voice leading, explaining how embellishments can create tension and release in the melody and how they contribute to the overall expressiveness of the piece.

What's more, ComposeOn also incorporates a \textbf{Music Theory Mentor Chatbot}, powered by the advanced ChatGPT-4 model, to enhance the user's learning experience and provide on-demand musical expertise. Within the recommendation rationales provided by the system, specialized musical terminology is hyperlinked. When a user clicks on one of these hyperlinked terms, the query is automatically populated in the Music Theory Mentor's input field, situated in a dedicated section of the interface. This mechanism facilitates immediate access to additional information, enabling users to explore complex musical concepts without disrupting their creative flow.

\subsection{ComposeOn User Interface (UI)}
The ComposeOn UI, as shown in Figure~\ref{fig:musicUI}  is a user-centric platform designed to facilitate seamless interaction between users and the ComposeOn composition system. This interface integrates melody continuation, editing, and explanatory functionalities, providing users with a comprehensive music creation and learning environment.

\begin{figure}[h]
\centering
\includegraphics[width=0.8\linewidth]{figs/musicUIillu.png}
\caption{ComposeOn UI illustration. Step1-2, choose a file, and process the file as MIDI. Step3-4, when click on the continue button, a new progression will be added. Step 5-6, click one bar or some notes to check the explanation. Step 7, click the hyperlink for quick chat with a chatbot powered by ChatGPT 4. Step8-9, check the alternative rhythms and chords suitable for the selected bar.}
\Description{ComposeOn System Design Diagram with three modules}
\label{fig:musicUI}
\end{figure}

At the top of the interface, users can dynamically adjust two primary parameters: Beats Per Minute (BPM) and Explanation Level. These controls allow real-time modification of playback speed and the depth of musical analysis provided, respectively.

The composition process begins with the user uploading an audio file in either .mp3 or .wav format, initiating the Upload Module. Subsequently, activating the "Process MIDI" function triggers the Analysis Module, which visualizes the uploaded melody as a musical score on the interface. This score is interactive, allowing playback with synchronized highlighting of the current measure. Concurrently, a virtual piano display in the lower left corner visually represents the notes being played.

Upon user initiation of the "Continue" function, the Generate Module extends the composition, producing a complete chord progression. This extension is seamlessly integrated into the existing musical score visualization. When a user selects any measure of the generated continuation, the system displays explanations calibrated to the preset Explanation Level. These explanations feature hyperlinked musical terminology, enabling users to access more detailed information in the Music Theory Mentor section on the right side of the interface.

For users wishing to edit the generated music, the interface offers measure-specific editing capabilities. By selecting a measure, users can access dropdown menus for "Common Progression Degrees" and "Common Rhythms". These menus present contextually appropriate chord progression degrees and rhythmic patterns, respectively. As users modify these parameters, the system dynamically updates both the musical score and the accompanying explanations, providing immediate feedback on the musical implications of their choices.


\section{Evaluation}\label{sec:Evaluation}


\section{Evaluation}
In this section, we benchmark our techniques in addressing the security threats for TBAS, that is, prompt injection and privacy leakage. We aim to answer the following questions:  

\begin{itemize}[noitemsep]
    \item[\textbf{Q1}:] Under scenarios with prompt injections, how well does our system maintain integrity and utility compared to state-of-the-art defenses? 
    \item[ \textbf{Q2}:] Under scenarios with privacy leakage, how much excessive user confirmations do we burden the user and whether utility is degraded compared to baselines?
    \item[\textbf{Q3}:] How accurate is our detector in determining the information flow within the LM and what is its runtime overhead? 
\end{itemize}


\subsection{End to End Evaluation: Prompt Injection}

\subsubsection{Setup} 

\noindent\textbf{Test Suites.} We benchmark our system on AgentDojo~\cite{debenedetti2024agentdojo}, a state-of-the-art benchmark on agent adversarial robustness against prompt injection attacks. Shown in Tab.~\ref{tab:agentdojo}, the dataset consists of 79 realistic user tasks in four suites: banking, travel, workspace, and slack. Every test suite represents a TBAS application where LLM serves user's request using a given set of tools, e.g. \texttt{send\_money} for the banking suite and \texttt{reserve\_restaurant} for the travel suite. Every test case in a suite requires the LLM to solve a task with multi-round interaction with external tools such as booking a restaurant after filtering through reviews and datary restrictions.

\noindent\textbf{Data Labeling.} To integrate the information flow mechanism, we enhance the task suites by assigning integrity labels based on the application's requirements while remaining agnostic to specific test cases (examples are shown in Tab.~\ref{tab:agentdojo}). The labeling process follows these key principles to satisfy the assumptions we denote on the tool environment in \ref{subsec:robust_tbas_assumptions}:
\begin{itemize}[noitemsep]
    \item Regions in a tool responses that incorporates textual data from external sources is labeled as low-integrity.
    \item Tools with significant side-effects (e.g., sending money) or those can introduce high-integrity data to the external environments (e.g. sending messages) are labeled as high-integrity.
\end{itemize} 


\noindent\textbf{Prompt Injection Attacks.} To emulate prompt injection attacks, each test suite includes a set of injection tasks. These tasks aim to induce the agent to misuse tools and produce harmful side effects, such as making unintended reservations on behalf of the user or leaking user's private data through public channels like emails. When evaluating the benchmark under Prompt Injection attacks, each user task is tested against every injection task in the corresponding test suite, resulting in a total of 629 security test cases.

\noindent\textbf{Baselines.} We evaluate the effectiveness of our mechanism against state-of-the-art prompt injection defenses, as well as two baseline approaches: 
\begin{itemize}[noitemsep]
    \item \textbf{Tool Filter} by AgentDojo: Use the LM as a Judge to filter the set of legal tools that an LM is allowed to use based on the user task.
    \item \textbf{Näive Tainting}: A baseline tainting approach where we assume every region in history affects the next message and needed to be tainted accordingly.
    \item \textbf{Redact All}: A baseline approach where we redact every single region that is not of high-integrity and public and therefore no labels are propagated.
\end{itemize}
\textbf{PI Detector} by \cite{protectAIdetector}, \textbf{Delimiting} by \cite{hinesdefend} and \textbf{Prompt Sandwiching} by \cite{learning_prompt_sandwich_url} were evaluated by AgentDojo \cite{debenedetti2024agentdojo}. \textbf{PI Detector} and \textbf{Delimitting} performed strictly worse than \textbf{Tool Filter}.  \textbf{Prompt Sandwiching} performed better without attack in utility, but suffered a 27\% attack success rate. We do not include these results since we consider \textbf{Tool Filter} the existing SOTA.

\noindent\textbf{Evaluation Metrics.} We follow AgentDojo to use utility and integrity (a.k.a. security in AgentDojo) as two evaluation metrics to compare different defenses, where
\begin{itemize}[noitemsep]
    \item \textbf{Utility} determines whether the agent has solved the task correctly, by inspecting the model output and the mutations in the environment state.
    \item \textbf{Integrity} determines whether the attacker succeeds in their attacks against the system. 
\end{itemize}









\begin{table*}[]
\centering
\caption{Overview of the Prompt Injection Benchmark}
\label{tab:agentdojo}
\resizebox{\linewidth}{!}{
\begin{tabular}{@{}lllllll@{}}
\toprule
Task Suite & \# User Task & \#Test Case & Number Tools & Number Messages Per Test Case & Example Labelled Low-Integrity Data        & Example High Integrity Tool Calls                                                   \\ \midrule
Banking    & 16           & 144         & 11           & 8.9 +- 3.0                    & External Bills, External Transaction Notes & \texttt{update\_transactions}, \texttt{send\_money} \\
Travel     & 20           & 140         & 28           & 13.6 +- 3.8                   & Hotel Reviews, Restaurant reviews.         & \texttt{send\_email}, \texttt{book\_hotel}        \\
Slack      & 21           & 105         & 11           & 15.6 +- 4.4                   & External Channel messages, Web Contents.   & \texttt{add\_new\_user}                                            \\
Workspace  & 40           & 240         & 24           & 8.7 +- 3.4                    & External Documents in a Cloud Drive        & \texttt{update\_calendar}                                            \\ \bottomrule
\end{tabular}
}
\end{table*}

We evaluate this benchmark suite using GPT-4o, consistent with results from AgentDojo. The Prompt Engineering detector is also implemented using this model. For the Attention-Based detector, which requires access to a LM’s internal weights to compute cross-token attention scores, we use the Phi-3-Mini-128K\cite{abdin2024phi3} instruction-tuned model. However, inference steps are still performed using GPT-4o. 


In this benchmark, we do not model user confirmations. Instead, any apparent unauthorized calls contrary to the information-flow policy are skipped and unperformed.













\subsubsection{Results and Analysis}

We present the results of the AgentDojo dataset both with (Figure \ref{fig:three graphs}) and without (Figure \ref{fig:agentdojoNoAttack}) prompt injection attacks. A cost comparison of running our techniques as a measure of overhead (Table \ref{tab:runtime}) is also provided. 

Importantly, the lack of user confirmations and the subsequent rejection of all apparent suspicious tool calls means that if we are able to seek user confirmations for calls that inheritently depend on low integrity data or in the case of over-tainting, then we are likely to achieve even better performance.

\noindent\textbf{Without Attacks.}
We present the results without attack in Fig \ref{fig:agentdojoNoAttack}. The impact on utility is best illustrated by the difference between the baseline case (no defense) and our techniques. Specifically, we observe a 10\% and 7.4\% degradation in utility for the LM-Judge and Attention-based detectors, respectively. Interestingly, the Tool Filter technique slightly increases utility in the absence of attacks. We speculate that this improvement arises from an implicit planning step, where irrelevant tools are excluded from LLM consideration.  

Our approach performs particularly well in the travel and workspace suites. As illustrated in the results in these suites, our approaches consistently achieve the highest utility among other methods, even exceeding the baseline(no defense) case by $5\%$ on average. These task suites naturally align with a more fine-grained integrity lattice and precise security policy, reducing scenarios that require manual user intervention.  

The Slack dataset, however, emerges as an outlier for our techniques. While our approaches still outperform naive tainting and redact-everything methods, the utility drops to $33\%$ and $22\%$ for the Attention-based and LM-Judge detectors respectively, which is more than halved compared to the mean utility. This performance drop can be attributed to the nature of Slack tasks, which often involve variations in reading content from untrusted websites and performing actions based on that content. We consider such tasks inherently unsafe, necessitating user confirmation. 

\begin{figure}[H]
\includegraphics[width=\columnwidth]{figures/dora_example.pdf}
\label{fig:dora}
\end{figure}
Additionally, some tasks require the agent to send a summary of an untrusted website to a high-integrity source (e.g., posting the summary to a Slack channel). If left unchecked, such actions could compromise the high-integrity source by spreading prompt injection attacks like a virus or conveying unintended statements. 



\noindent\textbf{Under Attack.}
We present the result when under prompt injection attack in Fig \ref{fig:three graphs}. RQ1: Our techniques still retain a high utility compared to the baseline without defense,  only losing less than 1\% utility for the LM-judge screener and 3\% for the Attention-based screener. 

We note that we \textbf{do} prevent 100\% of attacks that violate our security policy. However, in the workspace benchmark, there was one test case where text written by the user, labeled as high-integrity, contained possible prompt injection and is thus not tracked. This illustrates a major limitation for our mechanism, as with any other IFC techniques, that the security guarantees provided are only as good as the labels provided and the policies enforced.

\begin{figure*}
     \centering
     \begin{subfigure}[b]{0.19\textwidth}
         \centering
         \includegraphics[width=\textwidth]{figures/pi_w_attack/banking.pdf}
         \caption{Banking.}
         \label{fig:y equals x}
     \end{subfigure}
     \hfill
     \begin{subfigure}[b]{0.19\textwidth}
         \centering
         \includegraphics[width=\textwidth]{figures/pi_w_attack/slack.pdf}
         \caption{Slack.}
         \label{fig:three sin x}
     \end{subfigure}
     \hfill
     \begin{subfigure}[b]{0.19\textwidth}
         \centering
         \includegraphics[width=\textwidth]{figures/pi_w_attack/workspace.pdf}
         \caption{Workspace.}
     \end{subfigure}
     \hfill
     \begin{subfigure}[b]{0.19\textwidth}
         \centering
         \includegraphics[width=\textwidth]{figures/pi_w_attack/travel.pdf}
         \caption{Travel.}
     \end{subfigure}
     \hfill
     \begin{subfigure}[b]{0.19\textwidth}
         \centering
         \includegraphics[width=\textwidth]{figures/pi_w_attack/mean-attack.pdf}
         \caption{Weighted Average.}
     \end{subfigure}
        \caption{End-to-end evaluation on Security-Utility trade-off for Prompt Injection. The Top Right Corner indicates that high success rate of the user's task and high integrity of the defense against prompt injection across test cases.}
        \label{fig:three graphs}
\end{figure*}

\begin{figure}[ht!]
    \centering
    \includegraphics[width=\linewidth]{figures/no_attack.pdf}
    \caption{The Utility Rate comparison for the Prompt Injection Benchmark without attack. }
    \label{fig:agentdojoNoAttack}
\end{figure}

\begin{figure}[ht!]
     \centering
     \includegraphics[width=\linewidth]{figures/privacy.pdf}
     \caption{The Utility Comparison for the privacy leakage benchmark. The solid bars represent the utility achieved when users block tool calls upon receiving confirmation requests from the defenses. The faint bars indicate the additional utility users can gain by allowing these tool calls. The results demonstrate that our approaches provide a near-optimal balance, offering users flexibility to achieve varying utility levels based on their confirmation choices, unlike GPTs, which require confirmation every time.}
     \label{fig:utilityLeakage}
 \end{figure}



\subsection{End-to-End Evaluation: Privacy Leakage}

This experiment evaluates different defenses against the privacy leakage threat, e.g. accidental reference to chat history, silently booking a restaurant without user's confirmation. For every tool call the LLM makes, the defense mechanisms decide whether to flag the user for confirmation or proceed silently by masking out the private data. An ideal defense should effectively balance the \textbf{transparency} by asking the user for confirmation whenever privacy leakage occurs, and provide a smooth \textbf{user experience} by avoiding unnecessary confirmations when the tool call is independent of private data. 


\begin{table}
\caption{Overall false positive rates and false negative rates, for the Accidental Leakage benchmark.} 
\label{tab:accLeakage}
\vspace{-0.15in}
\begin{center}
\resizebox{\linewidth}{!}{
\begin{tabular}{lrr}
\toprule
 & FPR & FNR \\
\midrule
Confirm Never - redact-all & 0 & 0.513514 \\
Confirm Every Time (GPTs) & 0.297297 & 0 \\
RTBAS (LM-judge) & 0.081081 & 0.108108 \\
RTBAS (Attention) & 0.162162 & 0.108108 \\
\bottomrule
\end{tabular}
}
\end{center}
\end{table}

\begin{table}[h!]
\centering

\caption{Benchmark for Privacy Leakage}
\resizebox{\linewidth}{!}{
\begin{tabular}{p{2.5cm}|p{4.5cm}|p{4.5cm}}

\textbf{Task Suite} & \textbf{Description} & \textbf{Sensitive Data} \\ \hline
Venmo \newline(12 tasks) & Managing transactions, friend interactions, and account updates. & Transaction details, user info (balance, password), friend lists/info \\ \hline
Flight Booking \newline (12 tasks) & Searching, booking, and updating flights. & Credit card, passport number, user address, booked itinerary \\ \hline
Amazon \newline (13 tasks) & Buying, returning, recommendation of products. Promotions & Credit card, address, past orders, preferences, gender  \\ 
\end{tabular}
}
\label{tab:privacy_benchmark_descrpt}
\end{table}



\noindent\textbf{Synthesized Benchmark.} We are not aware of existing comprehensive benchmark for privacy leakage for TBAS. We manually created 37 test cases across three TBASs in different domains: shopping, finance, and flight booking. We provide a short description of the task suite in table \ref{tab:privacy_benchmark_descrpt}. Each task suite simulates a specific TBAS setup, featuring the same tools, descriptions, and system prompt, to represent a user-facing application. Tools capable of contributing private information to the context are annotated with regions identifying where private data appear in their outputs, along with labels specifying the nature of the private information. 

Each task begins with prior interactions between the user and the agent (i.e., the context window), which may already contain marked private data. This is followed by a user message that outlines the task to be completed. To achieve the task, the LLM may call tools to retrieve information, perform actions with external side effects, and report back to the user with the results.

Tasks vary in complexity. Some require a single reasoning step, such as directly calling a tool or answering a query based on the context. Others involve more intricate reasoning, requiring sequential calls to multiple tools to complete the task. Analyzing private information propagation in complex, multistep tasks is particularly valuable, as these scenarios provide more opportunities to observe indirect propagation of private information. Each tool call should propagate only the relevant information from the context, enabling a detailed and fine-grained evaluation of our approach.

As illustrated in \S\ref{subsection:unintended_confidentiality}, propagation of privacy information can occur in subtle ways. We include the diverse propagation patterns explored in \S\ref{subsection:unintended_confidentiality} as part of this benchmark to evaluate the effectiveness of our \dependencydetector.

We keep in mind the following principles when creating the dataset: %
\begin{itemize}
    \item Every test case has a ground truth tool calling to obtain the utility.
    \item Every test case whose utility does not depend on the private data will see private data in the ground truth tool calling chain.
\end{itemize}






\noindent\textbf{Evaluation Metrics.} Upon evaluation, each tool call made by an agent is manually labeled either as requiring confirmation (leaking private data) or not based on the natural understanding of the tool calling.  
Based on the oracle labels, we consider the following metrics for the benchmark: 
\begin{itemize}
    \item \textit{False Positive Rate} (FPR) measures the proportion of test cases in which the defense mechanism fails to detect a call to a tool that involves privacy leakage,
    \item \textit{False Negative Rate} (FNR) measures the proportion of test cases in which the defense mechanism incorrectly identifies a tool call as leaking private data,
    \item \textit{Utility} that measures the proportion of test cases that the user's task succeeds. Degradation to utility can result from erroneous masking.
\end{itemize}


\noindent\textbf{Approaches Compared.} We compare the following approaches: 
\begin{itemize}
    \item \textbf{Confirm Never - Redact All} redacts all private data upon information propagation. No confirmation necessary since no private information will ever be seen by the agent.
    \item \textbf{Confirm Every Time (GPTs)} assumes every tool call may leak private information and thus always requires confirmation.
    \item \textbf{Selective Propagation} selectively propagates information with the dependency screener. We include two instantiations (LM-Judge based and Attention based) for comparison. 
    \item \textbf{Oracle} represents a human expert that acts as the LM to perform tool calls and decide whether to confirm with the user. 
\end{itemize}



\noindent\textbf{Result and Analysis.}
Table~\ref{tab:accLeakage} shows the trade-off between the false negative rate (FNR) and the false positive rate (FPR) across the synthesized test suites. 

For the baselines, the Confirm Never redacts every private region, hence it will proceed silently by masking out the private data even when it is valid for a tool call to leak private information, e.g. booking a flight with credit card number, resulting in 51\% $FNR$ and severe utility loss. On the other hand, the Confirm Every Time (GPTs) defense taints the tool call as long as there is any private data in the context, resulting in 30\% $FPR$ and redundant user confirmations.

\textbf{Compared to the baselines, our selective propagation defenses effectively tames the trade-off between transparency and user experience}. Compared to the Confirm Never, the LM-Judge-based selective propagation delivers higher transparency to the user by reducing the $FNR$ from 30.7\% to 7.6\% for the Amazon Test Suite, from 58.3\% to 8.3\% for the Flight Booking test case, and from 66.7\% to 16.6\%. 

In contrast, compared to GPTs that require user confirmation for every tool call, our information flow-based defenses significantly reduce unnecessary confirmations. Specifically, the LM-Judge approach and the attention-based approach achieve an FPR of 8.1\% and 16.2\% across all test suites, respectively, whereas GPTs exhibit FPRs of 29\%. In practical terms, a smaller FPR translates into a significantly improved user experience, requiring minimal interaction from the user. This reduction in unnecessary confirmations is particularly crucial for maintaining a seamless and efficient workflow.

Next, we explore the utility results achieved by different approaches. Shown in Fig.~\ref{fig:utilityLeakage}, the solid bars show the success rate of the user tasks when the user blocks every tool call upon confirmation.
The faint bars show the additional utility the user can gain by allowing tool calls.
Confirm Never and GPTs baseline represents two extremes. On one side, Confirm Never does not provide the user with any autonomy in deciding whether a tool call should proceed, resulting in overall 35\% of utility. On the other side, the Confirm Every Time (GPTs) defense prompts the user for confirmation upon every tool call, with zero utility in the worst case and 91\% utility in the best case.



\textbf{Across the two extremes, our selective propagation approaches are able to balance the utility and number of times we seek user confirmation. }Compared to GPTs confirming every time, our approaches obtain the baseline utility of 40\% and 43\% for the LM-Judge and attention-based approach, respectively. That is, our approaches saves the user from from the need to confirm for test cases in which no private data is required for the task to succeed. For example, a large portion of the amazon test suite is confirmation-free services like product recommendation, product searching, etc., our approaches passes 53\% test cases without confirmations. In fact, compared to the oracle, we are losing utility only in 1 out of 15 test cases, because of the overtainting booking history for the current flight lookup. 




Compared to Confirm Never approach, our approach offers users the flexibility to proceed with the task by allowing potentially risky tool calls with the user's permission. This is especially critical in applications like Venmo, where sensitive data and financial activities are always involved. In our evaluation, we are able to achieve 83\% and 75\% of utility when the user allows every tool call, which is the same as GPT (83\%). 






\subsection{Analysis}


\subsubsection{Taint Tracking Accuracy}

We augmented the Privacy Leakage benchmark with precise labels that represents the sets of private information category involved. We evaluate, for every tool calls, how often these labels matches exactly the ground truth label we annotated(Q3).

The user, through this label, can gather more information about the category of data that the tool call purports to leak. A user comfortable with leaking their credit card number to book a flight may be hesitant to share her social security number. 

A mislabeled tool call with more private data categories than actually propagated could be erroneously rejected either interactively or by reference to the policy that the user agrees to prior. Oppositely, a label claiming less private data categories can distort the task, with actually relevant data masked. 
\begin{table}[h]
\centering
\label{tab:percentages}
\resizebox{\linewidth}{!}{%
\begin{tabular}{lccc}
\toprule
 Confirm Never (Redact All) & Confirmation Always (GPTs) & RTBAS (LLM Judge) & RTBAS (LLM Judge) \\
\midrule
 22.3\% & 56.7\% & 57.3\% & 70.0\% \\
\bottomrule
\end{tabular}%
}
\end{table}




We show that the selective propagation approach, when instantiated by either the prompting or the attention approach arrives at the exact ground truth label more than 70\% and 57\% of the time, respectively. This is superior to our baseline techniques for redacting all sensitive regions and thus propagate nothing or the always confirm method where we assume a tool call always leak every secret. 

\subsubsection{Dependency Screener Comparisons}

For the Prompt Injection and Privacy Leakage benchmarks, we find that the LM judge and the Attention-based \dependencydetector perform similarly across the benchmarks, with LM Judge performing slightly better overall under attack for Prompt Injection and much better in terms of its detection accuracies for privacy leakage(Tab ~\ref{tab:accLeakage}). We conject the LM judge's ability to explicitly reason about the dependencies and output its chain-of-thought \cite{wei2023chainofthoughtpromptingelicitsreasoning} could help generalize the mechanism across unseen task, and for more subtle propagation cases. However, across end-to-end benchmarks, both methods perform similarly with respect to the overall task utilities. This suggests that the Attention-based approach can detect important dependencies crucial to task success, but can possibly miss more subtle dependencies that may still influence task outcomes.




\subsubsection{Runtime Overhead}
Q3: Our techniques incur higher costs compared to existing methods, primarily due to the overhead introduced by the detectors. The Attention-based detector requires the LLM to run twice: the first run generates a preliminary message, which is used for the attention mechanism to compute dependency results. The second run generates the final output after masking. The LM Judge screener also incur computer overhead by running the judge LLM before the agent generates each new message. In contrast, the tool detector only runs one additional inference for each user message but not between tool calls, and the Prompt Sandwiching approach only marginally increases the number of tokens by repating the user requests. We discuss opportunities for optimization in Sec. \ref{sec:discussion}.
\begin{table}
\caption{Runtime comparison of executing the user tasks on the banking suite of AgentDojo. The metrics are averaged across test cases. The price is calculated against OpenAI's pricing. }
\label{tab:runtime}
\resizebox{\linewidth}{!}{
\begin{tabular}{llrrr}
\toprule
 baseline & price (\$) & time (s) & \#Tokens \\
\midrule
 Vanilla & 0.014712 & 4.369265 & 2709.937500 \\
 Tool Filter (AgentDojo) & 0.008653 & 4.880799 & 1504.625000 \\
 RTBAS (Attn) & 0.027531 & 8.728362 & 5048.687500 \\
 RTBAS (LLM Judge) & 0.031672 & 9.707550 & 5851.562500 \\
\bottomrule
\end{tabular}
}
\end{table}




























 









\section{Result}\label{sec:Result}
In the post-quesionnaire, we analyzed the music theory correctness and all the subjective assessments in the form of 5-point scores and free comments, Figure~\ref{fig:result} reflects the results of scores for ComposeOn and the baseline method.
\begin{figure}[h]
\centering
\includegraphics[width=0.8\linewidth]{figs/musicresult.png}
\caption{5-point result on the music theory correctness and the subjective assessments.}
\label{fig:result}
\end{figure}

\subsection{ComposeOn develops better music than Suno}
First, the music generated by Suno exhibits more richness and complexity in terms of weaving and orchestration. As noted in P2, P4, and P10, this complexity makes Suno's music potentially more appealing on first listen. One participant might describe it this way, “Suno's music feels rich on first listen, with multiple layers of sound and rich orchestration, giving a first impression of great depth. However, after multiple listens, Suno's music feels less logical. In contrast, ComposeOn's music, although given only melodies with no instrumental layers, feels comfortable after multiple listens because it is very logical.”

However, ComposeOn received high ratings for other aspects of music quality. Multiple participants (P1, P2, P9) emphasized the better structure of the music generated by ComposeOn.P1's comment was particularly specific: “The music generated by Suno lacks structure and sounds like randomly assembled pieces. It sounds like randomly assembled fragments because it has no clear development or ending. In contrast, I could hear (and see) the beginning and end of each phrase of ComposeOn's music, and each phrase ended in a similar pattern, making the music feel more holistic.”

In terms of musical coherence, several participants (P3, P6, P8) noted that ComposeOn showed better consistency in musical sequences.P3's observation was particularly insightful: “Sometimes Suno repeats chords that have been entered previously, but when you hear certain parts of the music, Suno sometimes generates random segments that have little to do with the previous text. This interspersing makes for very little musical consistency in this continuation. By contrast, the musical consistency in ComposeOn's continuation is stable; instead of outputting all the input melody at once and writing it all by itself at once, he assigns features of the input melody to different phrases, so that each phrase has parts that relate to the input melody but are not entirely consistent with it.” This approach makes the music generated by ComposeOn more coherent and natural. In addition, P6 points out the stability of the tempo: “The music written by Suno sometimes accelerates or decelerates suddenly, whereas ComposeOn's tempo is more stable.” This is further evidence of ComposeOn's strength in maintaining musical consistency.

Finally, ComposeOn shows more variation and innovation in its music writing, as articulated in P5's review, “Suno continues music that feels like it's repeating the same chords and melodies without advancing or developing; ComposeOn at least gives me a sense that the music is evolving because it doesn't all sound too much like the previous input.ComposeOn at least gives me ComposeOn at least gives me a sense that the music is evolving, as it doesn't all sound too much like the previous input. It demonstrates ComposeOn's ability to introduce new elements while maintaining musical coherence, making the music more layered and diverse.”

\subsection{ComposeOn increased participants' willingness and confidence to create music}

Our findings indicate that ComposeOn significantly enhanced participants' willingness and confidence in music creation. This improvement can be attributed to two key features of the system: high visualization and interactivity, and high explainability.

\subsubsection{High Visualization and Interactivity}

ComposeOn's interface provides a highly visual and interactive experience, which proved particularly beneficial for novice users. The system displays both user input and AI-recommended melodies on a traditional musical staff, offering real-time visual cues during playback. This feature allows users, especially beginners, to gain a deeper understanding of musical structure, including how notes form chords and how chord progressions work.

The system's interactivity is evident in its user-guided recommendation process. Participants can control the flow of suggestions using the "continue" and "end" buttons, giving them agency over the composition process. Furthermore, ComposeOn offers contextually appropriate chord and rhythm options for each measure, allowing users to experiment with different musical elements. For instance, one participant reported successfully changing a 1-4-5-1 chord progression to a 1-2-5-1 progression, demonstrating the system's flexibility and its capacity to facilitate learning through experimentation.

\subsubsection{High Explainability}

ComposeOn's high explainability feature significantly contributes to users' understanding and confidence. The system provides detailed explanations for its musical continuations, covering aspects such as chord progression, rhythm, and ornamental notes. These explanations are tailored to three proficiency levels - beginner, intermediate, and expert - ensuring that users receive information appropriate to their knowledge level.

The granularity of explanations is noteworthy; users can request explanations for individual notes, measures, or entire phrases. This feature not only clarifies the system's recommendations but also reinforces fundamental musical concepts. As one participant noted, "This feature not only helped explain why certain recommendations were made, but it also helped me remember basic information like chord composition more clearly. I could associate the sound of a chord in a phrase with its composition, an experience that's hard to achieve when learning chords in isolation."

Additionally, the integrated "music theory mentor" feature provides quick access to theoretical knowledge, offering suggested terms and concepts like the circle of fifths. This feature's effectiveness was demonstrated when a participant (P1) first encountered an explanation of the I-VI-V-I chord progression and then used the music theory mentor to gain a deeper understanding of why this progression is harmonically pleasing.

In conclusion, ComposeOn's combination of high visualization, interactivity, and explainability creates a supportive environment for music creation. By demystifying the composition process and providing immediate, context-specific feedback, the system empowers users to engage more confidently with music creation, regardless of their initial skill level. This approach not only facilitates the creation of music but also enhances users' overall musical understanding, potentially contributing to long-term skill development and musical appreciation.

\section{Discussion}\label{sec:Discussion}
\section{Discussion and limitations}
The results presented in Section~\ref{sec:experiments} show that compared to baseline methods, Xen provides a more effective solution that can enable an autonomous car to detect on-road anomalies in diverse driving scenarios using a single monocular camera. Despite the advantages, our work also encompasses several limitations.

Object detection plays an important role in Xen, as the interaction and behavior expert work under the assumption that anomalous objects can be reliably detected for trajectory reconstruction and prediction, respectively. However, the perception capability of monocular cameras is largely limited when the visibility is poor, such as at night and under inclement weather conditions. To alleviate the issue, camera-LiDAR fusion has been proposed and shown more effective than unimodal approaches in computer vision tasks~\citep{cui2021deep,chen2017multi,sindagi2019mvx}. With an additional sensor modality, object detection and thus anomaly detection can be made more robust in different environments. Furthermore, as noted in Section~\ref{subsec:bm}, perspective projection onto the image plane distorts the motion characteristics of objects (e.g., a proximate object appears to move faster than a distant object even though the two objects have the same speed in reality), which challenges efficient modeling of normal motion patterns. 3D object detection enabled by point clouds from LiDAR has the potential to resolve the issue by projecting bounding boxes to bird's eye view (BEV) and thus eliminating the negative effect of perspective projection on learning object motions.

Another common failure case of Xen results from large scene motions in normal scenarios, e.g., when the ego car executes an aggressive lane change or moves fast in complex urban areas. Frame prediction becomes difficult in such cases due to large motions of the ego car, and the resulting increase of score is indistinguishable from that caused by an anomaly. It has been shown recently in video prediction literature that camera poses are helpful in rendering high-quality images~\citep{ak2021robust}. As a result, given that additional onboard vehicle state is available, ego motions can be exploited to create a more robust scene expert. Another similar issue that can cause false positives in Xen is discussed in supplemental materials.

Anomaly detection is an active research topic both in robotics and computer vision. At a more general level, we hope that the analysis in this work, especially those in Section~\ref{sec:overview}, provides insights on a unified framework for anomaly detection in related areas. More specifically, an anomaly detector can be designed based on Figure~\ref{fig:anomaly-patterns} with necessary modifications for different applications. For example, for AD on field robots which operate in autonomous farms without human labors, only the edge between the ego agent and the environment needs to be monitored as the robot often performs a task individually; for AD with surveillance cameras, the two edges with one of the ends being the ego agent can be ignored as the surveillance camera is fixed and will never participate in an anomaly; and for AD on mobile robots that navigate through human crowds, the whole graph needs to be considered if non-ego involved anomalies also affect robot decisions. With the high-level framework determined, each expert can then be designed specifically for each type of edge based on the characteristics of different anomalies. Validating the generalization capability of Xen in other application domains is left as future work.

Another possible direction is to evaluate the efficacy of more complex architectures, such as foundation models, for anomaly detection. Large visual language models (LVLMs) have been shown powerful in a variety of application areas, including image captioning, content generation, and conversational AI~\citep{jiang2024effectiveness}. In the domain of autonomous driving, LVLMs have also been explored for tasks of visual question-answering~\citep{xu2024drivegpt4}, trajectory prediction~\citep{wu2023language}, path planning~\citep{mao2023language}, and decision-making and control~\citep{wen2023road}. These recent research advancements suggest that incorporating foundation models into on-road anomaly detection is a promising direction.

Although powerful, LVLMs are currently limited in efficiency due to billions of parameters~\citep{brohan2022rt,brohan2023rt,padalkar2023open}. Furthermore, proprietary models, such as GPT-4V, must be queried over the cloud, further increasing inference time~\citep{achiam2023gpt}. To ensure both accuracy and efficiency of on-road anomaly detection with limited onboard resources, a combination of LVLMs and lightweight models is necessary. One integration method is to retrieve intermediate embeddings of images through the LVLM, which can then be provided as an additional context to lightweight anomaly detectors for inference. The embeddings from the LVLM can be updated periodically for efficiency. Such a method, however, requires access to the hidden states of the LVLM, which most proprietary models do not allow. Alternatively, LVLMs can be used as an additional anomaly detection expert, which can then be incorporated into Xen through the Kalman filter. While the three original experts update the system states of Kalman filter at a high frequency, the LVLM can be queried at a low frequency and updates the system states asynchronously in a similar manner. With such an approach, we are able to benefit from both the efficiency of lightweight models and the accuracy and generalization ability of LVLMs.

\section{Conclusion}\label{sec:Conclusion}
\section{Conclusion}
This paper examined the challenges of authoring site-specific outdoor AR experiences, which are often constrained by incomplete and outdated world representations and limited access to evolving real-world conditions. Our formative study revealed that developers and designers frequently encounter these limitations, necessitating costly and time-consuming on-site visits to capture environmental details, assess user flow, and ensure contextual relevance. Based on these insights, we identified key requirements for integrating real-world context into remote authoring workflows, leading to the development of \SystemName, an asymmetric collaborative authoring system that facilitates synchronous collaboration between \exsitu (i.e., \textit{off-site}, \textit{remote}) developers and \insitu (i.e., \textit{on-site}) collaborators.

Our exploratory user study demonstrated that this approach mitigates key challenges by enhancing confidence in authored results, stimulating engagement and creativity, and enabling direct iterative refinements informed by up-to-date environmental data. At the same time, our findings highlight multitasking demands as a challenge in synchronous collaboration and emphasize the need for a balanced integration of synchronous and asynchronous workflows. Situating these findings within the broader landscape of AR authoring and remote collaboration, we provided recommendations for future work. We hope this research motivates further exploration of methods for building, testing, and evaluating site-specific AR experiences, contributing to the future of immersive and contextually grounded interactive applications.


% \begin{acks}
% thanks.
% \end{acks}

\bibliographystyle{ACM-Reference-Format}
\bibliography{references}

\end{document}

%% End of file "main.tex".
