\section{Related Work}
\label{sec:ReWork}
% {\color{red}The literature review should be rewritten. It should be organized as four parts:
% \begin{enumerate}
%     \item method based on representative structure, including some research on strategies about applying it to large scale problems.
%     \item analytic placement for VLSI, including algorithms for floorplanning and global placement of standard unit.
%     \item PCB placement algorithm
%     \item automatic placement based on machine learning, including deep learning accerleration method and pure deep learning method
% \end{enumerate}

% Note: Control the number of references to 20-30. select representative publications that are cited a lot or published in top journals and conferences.
% }

% In order to search for optimization algorithms for PCB placement, we refer to a large amount of literature, which includes different scenarios and algorithms. In a broad sense, Layout scenarios can be broadly divided into three categories: floorplanning for mixed-size modules, placement for standard units and PCB placement. At the same time, there are a large number of methods to solve the layout problem, such as the representative structure-based layout method, the analytic placement method and the reinforcement learning method. For the first two methods, they have been introduced in Section I. What needs to be further supplemented is the research of these two algorithms for larger scale and different scenarios.

%To improve the efficiency of meteherusitics on combinatorial model of placement, Lee \emph{et al.}~\cite{lee2003multilevel} proposed a multi-level floorplanning/placement method based on B*-tree. Through multi-level optimization and efficient space management, the method significantly improves layout quality and computational efficiency, especially in reducing wirelength and optimizing design area. Chen and Chang~\cite{chen2005modern} presented an improved fast simulated annealing algorithm (FSA), which provides faster calculation speed and higher solution quality when dealing with large-scale layout problems and has significant advantages over standard simulated annealing algorithms.



Analytic methods are popularly employed in floorplanning/placement scenarios of very-larege scale integrated circuit (VLSI). In order to overcome the difficulty of the standard quadratic programming method, Nam \emph{et al.} \cite{nam2006fast} proposed an innovative fast hierarchical quadratic programming layout algorithm (HQP). Chen and Zhu \cite{6238405} proposed a multilevel algorithm for placement of standard cells and VLSI, which employs a nonlinear programming technique and a best-choice clustering algorithm to take a global view of the whole netlist and placement information, and then uses an iterative local refinement technique during the declustering stage to further distribute the cells and reduce the wirelength. By modeling the placement task as an optimzation problem constrained by the Poisson equation, Lu \emph{et al.}  \cite{Lu2015} developed the analytic algorithm \emph{ePlace} for the cell placement of VLSI, and  Li \emph{et al.} \cite{li2022pef} proposed an efficient large-scale floorplanning algorithm for the task of large-scale floorplanning with fixed-outline. To address the challenge of developing a faster mixed-size placer without hardware acceleration and loss of solution quality, Peng and Zhu \cite{10266769} proposed a mixed-size placement algorithm based on a novel definition of potential energy  and a fast approximate computation scheme for partial derivatives of the potential energy for the Poisson’s equation.

 Since the anlaytic global placement cannot eliminate overlaps between modules, the legalization process is always introduced at the following stage.
 Peter \emph{et al.}~\cite{Peter2008} proposed a legalization algorithm for placement in VLSI, in which standard cells can be aligned to rows. Lin \emph{et al.}~\cite{Lin2016} use polygon to approximate the curve of the area of every soft module and solve an ILP for legalization, which is time consuming and is difficult to scale up for large-scale floorplanning problem. Moffitt \emph{et al.}~\cite{Moffitt2006} proposed the repair method \emph{Floorist} based on constraint graphs, where the insertion of constraints could result in dense constraint graphs that is expensive to visit all arcs. Li \emph{et al.} \cite{li2022pef} improved the graph-based legalization method based on \emph{Floorist}, where both insertion and deletion of constraints are performed to legalize the floorplan efficiently. Sun \emph{et al.} \cite{main_ref} performed the legalization by optimizing the commposite objective with increasing weight of overlap violation, which contributes to faster convergence and small wirelength of the final result.



 %The constraint graph method\cite{li2022pef} can consider the interaction between components in the global scope and avoid the unreasonable global layout caused by local overlap or local optimization. However, building an accurate constraint graph and managing all the constraints in it is a complex process. If the initial layout is not properly set, the effect of graph optimization may be affected, resulting in a slow convergence rate during optimization, or even an effective legal layout cannot be found. In order to eliminate the above problems, this paper proposes an innovative legalization algorithm. It is designed to expand the overlapping components from the middle to the periphery, which effectively maintains the relative position of the components.
 Machine learning is also implemented in the floorplanning/placement scenarios. Hu \emph{et al.}~\cite{hu2020graph2plan} presented the Graph2Plan, which utilizes deep learning to automatically generate VLSI floorplans. The method works by modeling a floorplan problem as a graph structure and using a graph neural network (GNN) to learn how to generate an optimized layout from graph data. Vassallo and Bajada~\cite{10546526} explores the use of Reinforcement Learning (RL) to automate learning and optimize circuit placement. They propose a new reinforcement learning framework that utilizes adaptive reward mechanisms to improve the performance of layout algorithms, especially in large-scale and complex VLSI designs. Lin \emph{et al.}~\cite{10.1145/3316781.3317803} proposed an accelerated algorithm named as  DREAMPlace, which approximates the objective function in the traditional layout optimization problem by using the neural network model, and making full use of parallel computing power to accelerate the layout process. Mirhoseini \emph{et al.}~\cite{mirhoseini2020chip} solved the chip layout problem using Deep Reinforcement Learning (DRL), where each step of the layout is seen as an action, and the strategy function associated with the high dimensional space and complex structure of chip layout is approximated by a deep neural network (DNN). The experimental results show that the deep reinforcement learning method is superior to the traditional heuristic layout method in the optimization of multiple design objectives, which provides a new idea for VLSI design.




% {\color{red} Machine learning is also implemented in the floorplanning/placement scenarios. Hu \emph{et al.}~\cite{hu2020graph2plan} presented the - Graph2Plan, which utilizes deep learning to automatically generate VLSI (very large scale integrated circuit) floorplans. The method works by modeling a floorplan problem as a graph structure and using a graph neural network (GNN) to learn how to generate an optimized layout from graph data. \cite{10546526} explores the use of Reinforcement Learning (RL) to automate learning and optimize circuit placement. They propose a new reinforcement learning framework that utilizes adaptive reward mechanisms to improve the performance of layout algorithms, especially in large-scale and complex VLSI designs. }
%\cite{vashisht2020placement} combines cyclic reinforcement learning with simulated annealing for intelligent layout optimization. By means of reinforcement learning guidance and simulated annealing random search, the method can converge quickly and find better solutions in complex VLSI layout problems.






Flexible rotation of modules is critical for improvement of design quality in floorplanning/placement. In some works based on simulated annealing implementation of placement~\cite{sheng2012simulated}, The rotation Angle of each component (e.g., 0°, 90°, 180°, 270°) is discrete and can be adjusted by selecting a different rotation angle at each iteration. Besides its expensive complexity, the efficiency of simulated annealing algorithm largely depends on its neighborhood solution generation strategy, which may ledas to its inefficient exploration in the solution space. Another component rotation strategy is based on particle swarm optimization\cite{sun2006floorplanning,kumar2020review}, its implementation is relatively simple, fewer parameters, easy to understand and implement. However, when the solution space is large or very complex, the particles may fall into local optimal solutions, resulting in the final result not being globally optimal. In some cases, particle populations may not have enough diversity in the search space because some particles are clustered in a small area, which can cause the search to become limited.
Huang \emph{et al.}~\cite{Huang2023} employed an analytical method to optimize the variable module rotation degree driven by wirelength, which allows the modeules to be placed in desired orietations.  Sun \emph{et al.}~\cite{main_ref} proposed to address the discerete solution space of orietation by DEA-PPM, which provides an efficient framework for coevolution of both discerete orietation and continuous coordinate.

For automatic placement of PCB components, Wang \emph{et al.}\cite{10652855} discussed how to realize automatic PCB layout under multiple constraints, and Li \emph{et al.}\cite{10652825} proposed a centralized placement method based on the sequence pair representation.
Various types of algorithms are widely used in PCB placement. ML-related algorithms play a significant role in PCB placement.  Based on a reinforcement learning -based agent for layout inference and
fine-tuning and a large language model -based agent for interactive optimization, Chen \emph{et al.}\cite{Chen2025} developed a novel agent-based framework that automatically generates PCB layouts meeting industrial constraints
through user interactions. To accelerate the placement process of PCB, Zhang \emph{et al.} \cite{zhang2025cypress} proposed a scalable GPU-accelerated PCB placement method inspired by VLSI. It incorporates
tailored cost functions, constraint handling, and optimized techniques adapted for PCB layouts. Taking the netlist of the already placed and unplaced circuits as input and abstract them into graphs, Chen \emph{et al.} \cite{10617495} proposed a subgraph matching based reference placement algorithm to achieve PCB placement reuse, thereby improving placement efficiency.

%Chen \emph{et al.}\cite{10652635} adopted Deep Q-Learning (DQN) algorithm as the core method of reinforcement learning. Through Q-learning, the agent updated the Q value after each placement operation according to the reward obtained, thus optimizing the placement strategy.
%Heuristic algorithms are also often applied in this scenario. %Sheng \emph{et al.}\cite{5992466} propose a heuristic approach, inspired by Relay races, that continuously improves layout quality through multiple "relay" stages. By simulating a competition mechanism similar to a race, it continuously "transmits" improved information at different stages and gradually optimizes the layout.

% {\color{blue}references for PCB placement: general method (optimization-based, ML/DL-based), flexible rotation, centralized placement}

% The main contributions of this paper are summarized as follows:
% \begin{itemize}
%     \item In order to solve the problem of large-scale components placement, we propose a components clustering algorithm based on DBSCAN, and give the initial placement and global placement system based on clustering.
%     \item For the global placement based on the original DEA-PPM and CSA algorithms, we introduce them into the clustering system, and give a complete set of parameters adjustment scheme for different scale clustering and the placement scheme of single cluster components.
%     \item We propose a novel legalization algorithm, which can effectively eliminate overlap while preserving component position relationship and fast convergence.
% \end{itemize}

% The rest of this paper is organized as follows. Section \ref{pre} provides some preliminary information. Then, in Section \ref{init} and \ref{global placement}, the algorithms developed for the top and bottom layer placement problems are presented. Section \ref{legalization} legalizes the placement results of the previous section. In Section \ref{experiment}, numerical experiments are conducted to prove the competitiveness of the proposed algorithm, and Section \ref{conclusion} summarizes the paper.