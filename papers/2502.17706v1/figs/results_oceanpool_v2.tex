\begin{figure}
\setlength{\lineskip}{0pt}
\centering
\setlength\tabcolsep{1.pt}
\renewcommand{\arraystretch}{0.5}
  \begin{tabular}{c@{\extracolsep{0.1cm}}c@{\extracolsep{0.1cm}}c}
  \includegraphics[trim=0 60 0 58 , clip, width=0.3\linewidth]{imgs/can_starfish_cup_raw.png} &
         \includegraphics[trim=0 60 0 58 , clip, width=0.3\linewidth]{imgs/can_starfish_cup_out.png} &
           \includegraphics[width=0.3\linewidth]{imgs/can_starfish_cup_gopro.png}\\
 \includegraphics[trim=0 118 0 0 , clip, width=0.3\linewidth]{imgs/pool_bottlebag6_raw_e.png}
&
     \includegraphics[trim=0 118 0 0 , clip, width=0.3\linewidth]{imgs/pool_bottlebag6_out_e.png} &

     \includegraphics[width=0.3\linewidth]{imgs/pool_bottlebag6_gopro_e.png}
  \\
    
    \footnotesize{(a) Raw input images} & \footnotesize{(b) Detection results} & \footnotesize{(c) Detection response using LEDs} \\

  \end{tabular}
  \caption{The first row consists of real images from the Caribbean Sea. The second row contains real images from a pool. These images are used as input for testing the detector network, which is trained on images generated using IBURD. (a) shows the raw images from the perspective of the camera on the LoCO AUV, (b) visualizes the segmentation mask and bounding box predicted by the detector. The network detects the objects correctly, with high confidence scores, and (c) shows the LED response in real-time. In the ocean image (first row) in (c) cup, starfish, and can are visible with their respective LED colors blue, magenta, and green. In the pool image (second row) in (c) bag and bottle are present with their LED indicating their class colors, white and red.} 
\label{fig:images}
\end{figure}