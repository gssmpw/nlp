\section{Related Work}
\label{sec:related}

\paragraph{\textbf{Sketch algorithms.}} Methods such as the Count-Min-Sketch____, the Count Sketch____, and Universal Sketch____ are composed of fixed-sized shared counter arrays with multiple hash functions. Such a common structure is common across many sketch algorithms for multiple types of approximate statistics.  Specifically, the Universal Sketch____  can estimate a large family of functions of the frequency vector, allowing the user to determine the metric of interest on the fly, or Cold Filter____ that optimizes accuracy and speed by filtering out cold items preventing \mbox{them from updating the sketch. }

Sketch algorithms that solve the same analytic problem (e.g., heavy hitters) are comparable by their update speed, throughput, and space/accuracy tradeoff. Namely, the more counters we fit into their counter arrays, the higher the accuracy is. Examples include Nitro____ and Randomized Counter Sharing____ that trade space efficiency for update speed. As well as algorithms like
Counter Braids____ and Counter Tree____ trade update speed for more accuracy where the desired approach and optimization criteria depend on the specific domain. Nitro and Randomized Counter Sharing use a form of a random sample to accelerate the runtime. In short, they perform fewer counter updates, which increases the error rate but still provides statistical guarantees. Methods that optimize space usually include a hierarchical structure that allows for dynamic sizing of counters. Examples include Counter Braids, Counter Tree, and Pyramid sketch____, where overflowing counters are extended to a secondary data structure which allows us to size counters for the average case rather than for the worst case. However, such a hierarchical approach usually slows the computation as we perform more memory accesses and hash calculations.  
\begin{table}[t]
\resizebox{.999\columnwidth}{!}{%
\begin{tabular}{|c||l|}
\hline
\textbf{Symbol}   & \textbf{Meaning}                                                                          \\ \hline\hline
\textbf{$n$}      & The number of bits per pool.                                                  \\ \hline
\textbf{$k$} & The number of counters per pool.                                                     \\ \hline
\multirow{2}{*}{\textbf{$SnB(n,k)$}} & The number of options to place $n$ identical \\   & balls into $k$ distinguishable bins.                                                     \\ \hline

\textbf{$x_1,\ldots,x_k$} & An encoded configuration.                                                     \\ \hline


\multirow{2}{*}{\textbf{$\xi$}} & The number of stars-and-bars configuration \\ & whose first entry is smaller than $x_0$.                                                     \\ \hline

\multirow{1}{*}{$T[a,b,c]$} & A lookup table value of $\sum_{j=0}^{c-1} SnB(a-j,b-1)$.                                                     \\ \hline

\multirow{1}{*}{$C$} & A configuration number.                                                     \\ \hline

\multirow{1}{*}{$\rho$} & The first decoded value given $C,n,k$.                                                     \\ \hline

\multirow{1}{*}{$\psi$} & The other decoded values given $C,n,k$.                                                     \\ \hline

\multirow{1}{*}{$s$} & The starting size of each counter.                                                    \\ \hline
\multirow{1}{*}{$i$} & The number of bits added when a counter overflows.                                                    \\ \hline
\multirow{1}{*}{$w$} & The weight added to a counter in an Increment.                                                    \\ \hline

\multirow{2}{*}{$L$} & maps each $C\in SnB(64,4)$ into an array of $k=4$ \emph{offsets}\\ & of where each counter starts in the pool's memory block.                                                    \\ \hline
\end{tabular}
}
\vspace{0mm}
\caption{The notations used in the paper.}\label{tbl:notations}
\vspace{-4mm}
\end{table}

Other algorithms suggest ways for overflowing counters to grow at the expense of other counters; for example, in ABC____, an overflowing counter `steals' a bit from its subsequent counter, whereas in SALSA____, it merges with a neighboring counter forming a double-sized counter. Our work follows this flavor but is more ambitious than ABC and SALSA. Namely, we distribute the bits of a pool according to demand rather than stealing bits from a neighboring counter (which then can steal more bits from its neighbor). Such an on-demand approach yields a lower chance of overflow and faster runtime. SALSA's approach of merging overflowing counters only applies to approximate shared counter solutions, whereas our approach applies to accurate streaming algorithms. Even when considering the approximate algorithms, merging counters increases the error of the largest heavy hitter flows, which is usually undesirable.
In contrast,  up-sizing counters without a merge does not increase the error. However, unlike SALSA, our approach yields overflows that must be handled (e.g., by employing secondary structures). Here, we consider multiple options and explain their impact. The secret sauce is in balancing these \mbox{approaches to yield an attractive tradeoff. }

Other approaches use multiple forms of sampling to reduce the counter length by only incrementing counters with a certain probability which is either fixed____ or depends on the specific counter value____. Such approaches save space by counting to larger numbers with a small counter size, but their use of sampling does not apply to all measurement tasks. Moreso, they complement our approach because even sampled counters need to be stored in memory and are expected to have variations in values.

\paragraph{\textbf{Histograms.}} To the best of our knowledge, the task of computing (an exact) histogram succinctly is novel to our paper, and current approaches typically use a simple hash table (such as the Robin Hash table____ for that purpose. Also, to have an additional strong baseline, we extend the Perfect Cuckoo Filter algorithm____ to support holding values.