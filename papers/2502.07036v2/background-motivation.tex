One key question arises on the use of LLMs is - can we trust an LLM's responses? Reliability of an LLM is dependent on its consistency. 

\subsection{Consistency of Responses}
Consistency of LLMs is defined based on how consistent an LLM's responses to different prompts are. It is about whether the responses that an LLM returns to the same or semantically identical prompts are sufficiently similar or identical to each other. Such responses may be syntactically or structurally different but semantically identical. If a set of responses are not sufficiently similar or identical to each other then we refer to it as "inconsistent responses".   

The prompts are processed by the LLM over a duration of time $\Delta_t$, during which the LLM model is assumed to remain stable and unchanged. In this paper, by the clause - "responses of an LLM to a prompt" means "responses of an LLM to the same prompt or semantically identical prompts issued multiple times". Each issuance of a prompt to an LLM is called as a query.

\subsection{Consistency and other Properties}
Some of the key implications of consistency and inconsistency in the context of LLMs are as follows:
\begin{itemize}    
    \item Accuracy $A$ $\implies$ Consistency $C$: if an LLM is stated to  be accurate in its responses to a prompt, then such responses will be consistent with respect to each other. In turn, lack of consistency implies lack of accuracy: $\neg C \implies \neg A$. 
    \item Consistency $C$ $\centernot\implies$ Accuracy $A$: Semantic consistency of responses does not always imply accuracy of the responses, primarily because an LLM may respond with the semantically equivalent responses for the same prompt or semantically equivalent prompts each time it is queried, but guaranteeing the semantic accuracy of the responses is beyond the problem of maintaining consistency.
    \item Consistency of LLMs are related to the hallucination of LLMs~\cite{mcdonald2024reducing}: The more inconsistent an LLM is in its responses, the more the LLM may hallucinate simply because the LLM is responding with semantically different responses to the same prompt over multiple queries. Section~\ref{sec:hallucination} presents some related findings on hallucination. However, a detailed discussion and analysis of such a relationship is beyond the scope of this paper.
\end{itemize}

\subsection{Formal Definition}

Let $L_i$ refer to a large language model (LLM). Let $p_k$ refer to a prompt in one or more languages that is/are supported by an $L_i$. Let $s_j$ represent an active user session of a $w$'th user entity $u_w$. Let $t_l$ represent the point in time at which a query is made to an LLM or the time at which an LLM responds to a query such that the response is complete (not in the process of incremental output). 
Let $r_v$ $\xleftarrow{}$ $q_v(L_i, s_j, p_k, t_l, u_w)$ refer to a unique query the user $u$ using prompt $p_j$ to $L_i$ leading to response $r_v$. The terms "Prompt" and "Query" are used interchangeably throughout this paper.

Let $L_i$ $\in$ $\mathcal{L}$, which is a set of  LLMs, where $L_i$ $\in$ $\mathcal{L}$, refer to a large language model (LLM). Let $R_v$ $\xleftarrow{}$ $Q_v(\mathcal{L}_i, S_j, P_k, t_l, U_w)$ refer to a unique set of queries the user $u$ using prompt $p_j$ to a subset of LLMs $\mathcal{L}_i$ $\subseteq$ $\mathcal{L}$ leading to a set of responses $R_v$ that are received by the set of users $U_w$, over a set of active sessions $S_j$, where there is an one-to-one mapping of all the elements across the sets of responses, queries, LLMs, sessions, prompts, timestamps of completion of queries, and users.


Prompts $p_x$ and $p_y$ are semantically equivalent, i.e. they are identical, sufficiently identical or sufficiently similar to each other semantically, which is represented by $p_x$ $\equiv$ $p_y$. Responses $r_v$ and $r_w$ are semantically equivalent, i.e. they are identical, sufficiently identical, or sufficiently similar to each other semantically, which is represented by $r_v$ $\equiv$ $r_w$.\\ 

\textbf{Definition 1: Consistency of Responses -- One LLM.}\\
    Given two queries $r_1$  $\xleftarrow{}$ $q_1(L_1, s_1, p_x, t_1, u_1)$ and $r_2$  $\xleftarrow{}$ $q_2(L_1, s_1, p_y, t_3, u_1)$, where $p_x \equiv p_y$, the responses are consistent if $r_1$ $\equiv$ $r_2$.\\ 

\textbf{Definition 2: Consistency of Responses -- Multiple LLMs.}\\ 
    Given two queries $R_1$  $\xleftarrow{}$ $Q_1(\mathcal{L}_1, S_1, p_x, T_1, u_1)$ and $R_2$  $\xleftarrow{}$ $Q_2(\mathcal{L}_1, S_1, p_y, T_3, u_1)$, where $p_x \equiv p_y$, the responses are consistent if for all pairs $<d_v, e_v>$, where $d_v \in R_1$  and $e_v \in R_2$,  $x_v$ $\equiv$ $y_v$.

In the rest of the paper, on the basis of this formal model, we will study and carry out experiments on black-box LLMs. 
