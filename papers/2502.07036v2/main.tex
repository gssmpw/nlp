\documentclass[conference]{IEEEtran}
\IEEEoverridecommandlockouts
\usepackage{amsmath,amssymb,amsfonts}
\usepackage{graphicx}
\usepackage{textcomp}
\usepackage{xcolor}
\usepackage{comment}
\usepackage{algorithm}
\usepackage[noend]{algpseudocode}
\usepackage{xcolor}
\usepackage{fontawesome}
\usepackage{tikz}
\usepackage{comment}
\usepackage{centernot}

\usepackage{subfig,hyperref}
\def\BibTeX{{\rm B\kern-.05em{\sc i\kern-.025em b}\kern-.08em
    T\kern-.1667em\lower.7ex\hbox{E}\kern-.125emX}}
\begin{document}

\title{Automated Consistency Analysis of LLMs\\
}

\author{\IEEEauthorblockN{Aditya Patwardhan}
\IEEEauthorblockA{\textit{Department of Computer Science} \\
\textit{Stony Brook University}\\
Stony Brook, USA \\
aapatwardhan@cs.stonybrook.edu}
\and
\IEEEauthorblockN{Vivek Vaidya}
\IEEEauthorblockA{\textit{Department of Computer Science} \\
\textit{Rutgers University}\\
New Brunswick, USA \\
vivek.vaidya@rutgers.edu}
\and
\IEEEauthorblockN{Ashish Kundu}
\IEEEauthorblockA{\textit{Cisco Research} \\
San Jose, USA \\
ashkundu@cisco.com}
}

\maketitle

\begin{abstract}
\begin{abstract}  
Test time scaling is currently one of the most active research areas that shows promise after training time scaling has reached its limits.
Deep-thinking (DT) models are a class of recurrent models that can perform easy-to-hard generalization by assigning more compute to harder test samples.
However, due to their inability to determine the complexity of a test sample, DT models have to use a large amount of computation for both easy and hard test samples.
Excessive test time computation is wasteful and can cause the ``overthinking'' problem where more test time computation leads to worse results.
In this paper, we introduce a test time training method for determining the optimal amount of computation needed for each sample during test time.
We also propose Conv-LiGRU, a novel recurrent architecture for efficient and robust visual reasoning. 
Extensive experiments demonstrate that Conv-LiGRU is more stable than DT, effectively mitigates the ``overthinking'' phenomenon, and achieves superior accuracy.
\end{abstract}  
\end{abstract}

\begin{IEEEkeywords}
 Cybersecurity,  Generative AI. Large Language Models, Agents, Consistency, Trustworthiness, Validity, Reliability, Hallucination
\end{IEEEkeywords}

\section{Introduction}
\section{Introduction}


\begin{figure}[t]
\centering
\includegraphics[width=0.6\columnwidth]{figures/evaluation_desiderata_V5.pdf}
\vspace{-0.5cm}
\caption{\systemName is a platform for conducting realistic evaluations of code LLMs, collecting human preferences of coding models with real users, real tasks, and in realistic environments, aimed at addressing the limitations of existing evaluations.
}
\label{fig:motivation}
\end{figure}

\begin{figure*}[t]
\centering
\includegraphics[width=\textwidth]{figures/system_design_v2.png}
\caption{We introduce \systemName, a VSCode extension to collect human preferences of code directly in a developer's IDE. \systemName enables developers to use code completions from various models. The system comprises a) the interface in the user's IDE which presents paired completions to users (left), b) a sampling strategy that picks model pairs to reduce latency (right, top), and c) a prompting scheme that allows diverse LLMs to perform code completions with high fidelity.
Users can select between the top completion (green box) using \texttt{tab} or the bottom completion (blue box) using \texttt{shift+tab}.}
\label{fig:overview}
\end{figure*}

As model capabilities improve, large language models (LLMs) are increasingly integrated into user environments and workflows.
For example, software developers code with AI in integrated developer environments (IDEs)~\citep{peng2023impact}, doctors rely on notes generated through ambient listening~\citep{oberst2024science}, and lawyers consider case evidence identified by electronic discovery systems~\citep{yang2024beyond}.
Increasing deployment of models in productivity tools demands evaluation that more closely reflects real-world circumstances~\citep{hutchinson2022evaluation, saxon2024benchmarks, kapoor2024ai}.
While newer benchmarks and live platforms incorporate human feedback to capture real-world usage, they almost exclusively focus on evaluating LLMs in chat conversations~\citep{zheng2023judging,dubois2023alpacafarm,chiang2024chatbot, kirk2024the}.
Model evaluation must move beyond chat-based interactions and into specialized user environments.



 

In this work, we focus on evaluating LLM-based coding assistants. 
Despite the popularity of these tools---millions of developers use Github Copilot~\citep{Copilot}---existing
evaluations of the coding capabilities of new models exhibit multiple limitations (Figure~\ref{fig:motivation}, bottom).
Traditional ML benchmarks evaluate LLM capabilities by measuring how well a model can complete static, interview-style coding tasks~\citep{chen2021evaluating,austin2021program,jain2024livecodebench, white2024livebench} and lack \emph{real users}. 
User studies recruit real users to evaluate the effectiveness of LLMs as coding assistants, but are often limited to simple programming tasks as opposed to \emph{real tasks}~\citep{vaithilingam2022expectation,ross2023programmer, mozannar2024realhumaneval}.
Recent efforts to collect human feedback such as Chatbot Arena~\citep{chiang2024chatbot} are still removed from a \emph{realistic environment}, resulting in users and data that deviate from typical software development processes.
We introduce \systemName to address these limitations (Figure~\ref{fig:motivation}, top), and we describe our three main contributions below.


\textbf{We deploy \systemName in-the-wild to collect human preferences on code.} 
\systemName is a Visual Studio Code extension, collecting preferences directly in a developer's IDE within their actual workflow (Figure~\ref{fig:overview}).
\systemName provides developers with code completions, akin to the type of support provided by Github Copilot~\citep{Copilot}. 
Over the past 3 months, \systemName has served over~\completions suggestions from 10 state-of-the-art LLMs, 
gathering \sampleCount~votes from \userCount~users.
To collect user preferences,
\systemName presents a novel interface that shows users paired code completions from two different LLMs, which are determined based on a sampling strategy that aims to 
mitigate latency while preserving coverage across model comparisons.
Additionally, we devise a prompting scheme that allows a diverse set of models to perform code completions with high fidelity.
See Section~\ref{sec:system} and Section~\ref{sec:deployment} for details about system design and deployment respectively.



\textbf{We construct a leaderboard of user preferences and find notable differences from existing static benchmarks and human preference leaderboards.}
In general, we observe that smaller models seem to overperform in static benchmarks compared to our leaderboard, while performance among larger models is mixed (Section~\ref{sec:leaderboard_calculation}).
We attribute these differences to the fact that \systemName is exposed to users and tasks that differ drastically from code evaluations in the past. 
Our data spans 103 programming languages and 24 natural languages as well as a variety of real-world applications and code structures, while static benchmarks tend to focus on a specific programming and natural language and task (e.g. coding competition problems).
Additionally, while all of \systemName interactions contain code contexts and the majority involve infilling tasks, a much smaller fraction of Chatbot Arena's coding tasks contain code context, with infilling tasks appearing even more rarely. 
We analyze our data in depth in Section~\ref{subsec:comparison}.



\textbf{We derive new insights into user preferences of code by analyzing \systemName's diverse and distinct data distribution.}
We compare user preferences across different stratifications of input data (e.g., common versus rare languages) and observe which affect observed preferences most (Section~\ref{sec:analysis}).
For example, while user preferences stay relatively consistent across various programming languages, they differ drastically between different task categories (e.g. frontend/backend versus algorithm design).
We also observe variations in user preference due to different features related to code structure 
(e.g., context length and completion patterns).
We open-source \systemName and release a curated subset of code contexts.
Altogether, our results highlight the necessity of model evaluation in realistic and domain-specific settings.







\section{Consistency in the context of LLMs} \label{sec:def}
One key question arises on the use of LLMs is - can we trust an LLM's responses? Reliability of an LLM is dependent on its consistency. 

\subsection{Consistency of Responses}
Consistency of LLMs is defined based on how consistent an LLM's responses to different prompts are. It is about whether the responses that an LLM returns to the same or semantically identical prompts are sufficiently similar or identical to each other. Such responses may be syntactically or structurally different but semantically identical. If a set of responses are not sufficiently similar or identical to each other then we refer to it as "inconsistent responses".   

The prompts are processed by the LLM over a duration of time $\Delta_t$, during which the LLM model is assumed to remain stable and unchanged. In this paper, by the clause - "responses of an LLM to a prompt" means "responses of an LLM to the same prompt or semantically identical prompts issued multiple times". Each issuance of a prompt to an LLM is called as a query.

\subsection{Consistency and other Properties}
Some of the key implications of consistency and inconsistency in the context of LLMs are as follows:
\begin{itemize}    
    \item Accuracy $A$ $\implies$ Consistency $C$: if an LLM is stated to  be accurate in its responses to a prompt, then such responses will be consistent with respect to each other. In turn, lack of consistency implies lack of accuracy: $\neg C \implies \neg A$. 
    \item Consistency $C$ $\centernot\implies$ Accuracy $A$: Semantic consistency of responses does not always imply accuracy of the responses, primarily because an LLM may respond with the semantically equivalent responses for the same prompt or semantically equivalent prompts each time it is queried, but guaranteeing the semantic accuracy of the responses is beyond the problem of maintaining consistency.
    \item Consistency of LLMs are related to the hallucination of LLMs~\cite{mcdonald2024reducing}: The more inconsistent an LLM is in its responses, the more the LLM may hallucinate simply because the LLM is responding with semantically different responses to the same prompt over multiple queries. Section~\ref{sec:hallucination} presents some related findings on hallucination. However, a detailed discussion and analysis of such a relationship is beyond the scope of this paper.
\end{itemize}

\subsection{Formal Definition}

Let $L_i$ refer to a large language model (LLM). Let $p_k$ refer to a prompt in one or more languages that is/are supported by an $L_i$. Let $s_j$ represent an active user session of a $w$'th user entity $u_w$. Let $t_l$ represent the point in time at which a query is made to an LLM or the time at which an LLM responds to a query such that the response is complete (not in the process of incremental output). 
Let $r_v$ $\xleftarrow{}$ $q_v(L_i, s_j, p_k, t_l, u_w)$ refer to a unique query the user $u$ using prompt $p_j$ to $L_i$ leading to response $r_v$. The terms "Prompt" and "Query" are used interchangeably throughout this paper.

Let $L_i$ $\in$ $\mathcal{L}$, which is a set of  LLMs, where $L_i$ $\in$ $\mathcal{L}$, refer to a large language model (LLM). Let $R_v$ $\xleftarrow{}$ $Q_v(\mathcal{L}_i, S_j, P_k, t_l, U_w)$ refer to a unique set of queries the user $u$ using prompt $p_j$ to a subset of LLMs $\mathcal{L}_i$ $\subseteq$ $\mathcal{L}$ leading to a set of responses $R_v$ that are received by the set of users $U_w$, over a set of active sessions $S_j$, where there is an one-to-one mapping of all the elements across the sets of responses, queries, LLMs, sessions, prompts, timestamps of completion of queries, and users.


Prompts $p_x$ and $p_y$ are semantically equivalent, i.e. they are identical, sufficiently identical or sufficiently similar to each other semantically, which is represented by $p_x$ $\equiv$ $p_y$. Responses $r_v$ and $r_w$ are semantically equivalent, i.e. they are identical, sufficiently identical, or sufficiently similar to each other semantically, which is represented by $r_v$ $\equiv$ $r_w$.\\ 

\textbf{Definition 1: Consistency of Responses -- One LLM.}\\
    Given two queries $r_1$  $\xleftarrow{}$ $q_1(L_1, s_1, p_x, t_1, u_1)$ and $r_2$  $\xleftarrow{}$ $q_2(L_1, s_1, p_y, t_3, u_1)$, where $p_x \equiv p_y$, the responses are consistent if $r_1$ $\equiv$ $r_2$.\\ 

\textbf{Definition 2: Consistency of Responses -- Multiple LLMs.}\\ 
    Given two queries $R_1$  $\xleftarrow{}$ $Q_1(\mathcal{L}_1, S_1, p_x, T_1, u_1)$ and $R_2$  $\xleftarrow{}$ $Q_2(\mathcal{L}_1, S_1, p_y, T_3, u_1)$, where $p_x \equiv p_y$, the responses are consistent if for all pairs $<d_v, e_v>$, where $d_v \in R_1$  and $e_v \in R_2$,  $x_v$ $\equiv$ $y_v$.

In the rest of the paper, on the basis of this formal model, we will study and carry out experiments on black-box LLMs. 


\section{Consistency Validation Framework} \label{sec:framework}

In this paper, we propose a comprehensive evaluation framework that incorporates multiple algorithms for evaluating consistency and accuracy, providing a holistic metric of how trustworthy an LLM is. In this paper, we measure LLM consistency in the context of cybersecurity applications.

The rest of the paper follows the formal model and definitions. However, the notations used may differ in order to provide more readability for the algorithms and discussion.

\subsection{Consistency}

To be trustworthy, an LLM has to return a similar answer every time it’s prompted with the same question, so different users don’t get different answers or explanations to answers when researching the same topic. Our consistency algorithm gives an LLM the same prompt $n$ times and evaluates the similarity between responses using multiple metrics such as Jaccard Index~\cite{article}, Cosine Similarity~\cite{cosine_similarity}, Sequence Matcher~\cite{python_doc}, and Levenshtein distance~\cite{levenshtein_distance}, all standardized to a scale of 0 to 100. 

The Consistency algorithm (Algorithm \ref{alg:consistency}) operates in three modes: low, medium, and high, where higher settings require progressively greater consistency in the metrics for the model to be considered consistent. For each question, the algorithm collects $k$ model responses and then calculates pairwise consistency scores using the four metrics for every possible pair of responses, including consecutive responses. If the metric score is higher than a certain threshold, that pair passes for that metric. Therefore, while consecutive comparisons are part of the pairwise evaluation, the algorithm ensures a comprehensive assessment by comparing all responses in the set. 

If the pair passes $x$ out of 4 consistency score metrics, it is considered to pass overall. If 80\% of pairs pass, the model is considered consistent for that question. If 80\% of questions pass, the model is considered consistent overall.

Instead of keeping the same percentages to pass each metric, we have implemented low, medium, and high settings to further bring out the differences between the models. Under these settings, the percentage required for a pair to "pass" a certain consistency metric changes. 

For the low threshold, Jaccard and Cosine have to be 70\%, and Sequence Matcher and Levenshtein have to be 20\%. For medium, Jaccard and Cosine have to be 80\%, and Sequence Matcher and Levenshtein have to be 40\%. For high, Jaccard and Cosine have to be 90\%, and Sequence Matcher and Levenshtein have to be 60\%. Sequence Matcher and Levenshtein Distance similarity take the order of characters into account as opposed to the two, so they tend to be more critical of responses that are roughly the same but worded differently. Due to this, their required percentages are significantly lower than the other two.

\begin{algorithm}[tb]
\caption{Consistency Analysis}
\label{alg:consistency}
\begin{algorithmic}[1]
\Statex \textbf{Input:} \textit{LLM} $L_i$ -  LLM to perform consistency analysis
\Statex \textbf{Input:} \textit{Prompts/Queries} -  list of queries to be validated
\Statex \textbf{Input:} \textit{k} - The number of repetitions for validation
\Statex \textbf{Input:} \textit{simthreshold} - The threshold checked to determine similarity: low, medium, high
\Statex \textbf{Input:} \textit{qthreshold} - The minimum fraction of questions for which the LLM's answers need to be consistent
\Statex
\Statex \textbf{Output:} True/False
\Statex
\Procedure{Consistency\_Analysis}{\textit{LLM, Queries, k, simthresh, qthreshold}}
\State $qcnt \gets 0$
\State $npt \gets 0.8 * k * (k-1) / 2$
\For{each $q \in Queries$}
  \State $Resp \gets [~]$
  \For{$i \in 1 \dots k$}
    \State $Resp_i \gets LLM\_Api(q)$
  \EndFor
  \State $SS\_cnt,LS\_cnt,JS\_cnt,CS\_cnt \gets 0$
  \For{$i \in 1 \dots k-1$}
    \For{$j \in i+1 \dots k$}
        \State $SS \gets SeqMatcher(Resp_i,Resp_j)$
        \State $LS \gets LevenDist(Resp_i,Resp_j)$
        \State $JS \gets JaccardCoef(Resp_i,Resp_j)$
        \State $CS \gets CosineSim(Resp_i,Resp_j)$
        \State $SS\_cnt$ += $SS \ge simthresh$ ? 1 : 0
        \State $LS\_cnt$ += $LS \ge simthresh$ ? 1 : 0
        \State $JS\_cnt$ += $JS \ge simthresh$ ? 1 : 0
        \State $CS\_cnt$ += $CS \ge simthresh$ ? 1 : 0
    \EndFor
  \EndFor
  \If{$SS\_cnt,LS\_cnt,JS\_cnt,CS\_cnt \ge npt$}
    %\If{$SS\_cnt \ge npt$ \&\& $LS\_cnt \ge npt$ \&\& $JS\_cnt \ge npt$ \&\& $CS\_cnt \ge npt$}
    \State $qcnt \gets qcnt + 1$
  \EndIf
\EndFor
\If{$qcnt / |Queries| \ge qthreshold$}
  \State \Return{$true$}
\Else
  \State \Return{$false$}
\EndIf
\EndProcedure
\end{algorithmic}
\end{algorithm}


\subsection{Agreement}
To be trustworthy an LLM has to return the correct answer to a question. To determine if the LLMs agree on whether a certain answer is correct or not, our framework uses two algorithms. The first is Self-Validation, where an LLM checks it's own answer to a question. The second is Cross-Validation, where an LLM's answer to a question is checked by every other LLM. An LLM must be considered accurate by both these algorithms to be considered trustworthy in terms of information accuracy.

\subsubsection{Self-Validation}

\begin{figure}[t]
    \centering
    \includegraphics[width=\linewidth]{Self-Valid.png}
    \caption{Self-Validation Architecure}
    \label{fig:Self-Valid-Alg-Fig}
\end{figure}

Figure~\ref{fig:Self-Valid-Alg-Fig} illustrates the Self-Validation framework for Large Language Models (LLMs). In this process, the LLM generates a "Response List" with repeated responses to the same question. That same LLM is then asked whether the generated responses are the correct answer to the original query. If it agrees with enough of its own responses, it is considered factually consistent by self-validation.

The Self-Validation Algorithm (Algorithm \ref{Self Validation}) has an LLM to evaluate the accuracy of its answers as shown in \ref{fig:Self-Valid-Alg-Fig} . It prompts the LLM with its answer to a question and asks if it is the correct answer to that question. This is done $k$ times for every question. If the LLM responds "yes" 80\% of the time, that question is considered correct by this metric. If the number of correct questions divided by the total number of questions is greater than $qthreshold$, the LLM is considered accurate overall by this metric.


\begin{algorithm}[tb]
\caption{Self Validation}
\label{Self Validation}
\begin{algorithmic}[1]
\Statex \textbf{Input:} \textit{LLM} $L_i$ - The LLM to validate
\Statex \textbf{Input:} \textit{Queries} - The list of queries to be validated
\Statex \textbf{Input:} \textit{k} - The number of repetitions for validation
\Statex \textbf{Input:} \textit{qthreshold} - The minimum fraction of questions for which the LLM needs to accept its own answer to be considered successful
\Statex
\Statex \textbf{Output:} True/False
\Statex
\Procedure{Self\_Validation}{\textit{LLM, Queries, k, qthreshold}}
\State $qcnt \gets 0$
\For{each $q \in Queries$}
  \State $Orig\_Resp \gets LLM\_Query\_Api(q)$
  %\Statex \Comment{Get a yes or no response from the LLM for whether the answer it provides is correct?}
  \State $svq \gets $q + Orig\_Resp + ``correct? yes or no''
  \State $valcnt \gets 0$
  \For{$i \in 1 \dots k$}
    \State $Resp_i \gets LLM\_Api(svp)$
    \If{$Resp_i$ = Yes}
        \State $valcnt \gets valcnt + 1$
    \EndIf
  \EndFor
  \If{$valcnt > 0.8 * k$}
    \State $qcnt \gets qcnt + 1$
  \EndIf
\EndFor
\If{$qcnt / |Queries| \ge qthreshold$}
  \State \Return{$true$}
\Else
  \State \Return{$false$}
\EndIf
\EndProcedure
\end{algorithmic}
\end{algorithm}

\subsubsection{Cross-Validation}

\ref{fig:Cross-Valid-Alg-Fig} illustrates the cross-validation framework for evaluating the consistency of a Large Language Model (LLM). This framework fact checks LLM responses with other LLMs. Each LLM generates a response to a prompt, and then every other LLM is asked whether that LLM's response to the original prompt is correct. If there is enough agreement between LLMs, that LLM is considered factually consistent by cross-validation

\begin{figure}[!t]
    \centering
    \includegraphics[width=\linewidth]{Cross-Valid.png}
    \caption{Cross-Validation Architecture}
    \label{fig:Cross-Valid-Alg-Fig}
\end{figure}

If an LLM has unreliable information caused by biased training data, it may not be able to recognize that in the self-validation step. To remedy that we propose the Cross-Validation Algorithm (Algorithm \ref{fig:Cross-Valid-Alg-Fig}), which cross-validates an LLM's responses with the other LLMs as shown in \ref{fig:cross-valid}. To begin with, the algorithm is provided with a list of one LLMs responses to a set of questions. For each response, all the other LLMs are asked whether it is the correct response to the respective question. If 80\% of the other LLMs' responses are yes to a question, that question is considered correct by this metric. If the number of correct questions divided by the total number of questions is greater than $qthreshold$, the LLM is considered factually consistent overall by this metric.

\begin{algorithm}[tb]
\caption{Cross Validation}
\label{Cross Validation}
\begin{algorithmic}[1]
\Statex \textbf{Input:} \textit{LLMs} $\mathcal{L}$ - The list of LLMs to cross-validate
\Statex \textbf{Input:} \textit{Queries} - The list of queries to be validated
\Statex \textbf{Input:} \textit{k} - The number of repetitions for validation
\Statex \textbf{Input:} \textit{qthreshold} - The minimum fraction of questions for which the other LLMs needs to accept any LLM's answer for it to be considered successful
\Statex
\Statex \textbf{Output:} cv\_llm - a boolean list; $cv\_llm_i$ indicates whether $llm_i$ is cross validated
\Statex
\Procedure{Cross\_Validation}{\textit{LLMs, Queries, k, qthreshold}}
\State $cv\_llm \gets \phi$
\For{each $llm_i \in LLMs$}
    \State $qcnt \gets 0$
    \For{each $q \in Queries$}
      \State $Orig\_Resp \gets llm_i\_Query\_Api(q)$
      %\Statex \Comment{Get a yes or no response from the LLM for whether the answer it provides is correct?}
      \State $svq \gets $q + Orig\_Resp + ``correct? yes or no''
        \State $llmcnt \gets 0$
        \For{each $llm_j \in LLMs$, s.t. $llm_j \neq llm_i$}
          \State $valcnt \gets 0$
          \For{$i \in 1 \dots k$}
            \State $Resp_i \gets LLM_j\_Api(svp)$
            \If{$Resp_i$ = Yes}
                \State $valcnt \gets valcnt + 1$
            \EndIf
          \EndFor
          \If{$valcnt > 0.8 * k$}
              \State $llmcnt \gets llmcnt + 1$
          \EndIf
        \EndFor
        \If{$llmcnt > 0.66 * |LLMs|$}
            \State $qcnt \gets qcnt + 1$
        \EndIf
    \EndFor
    \If{$qcnt / |Queries| \ge qthreshold$}
      \State $cv\_llm_i \gets$ {$true$}
    \Else
      \State $cv\_llm_i \gets$ {$false$}
    \EndIf
\EndFor
\State \Return {cv\_llm}
\EndProcedure
\end{algorithmic}
\end{algorithm}



\section{Empirical Analysis on LLMs using this Framework} \label{sec:experiments}
\subsection{Benchmarks}
To test the LLMs, we developed a benchmark on 40 Cybersecurity interview questions from a popular list of cybersecurity interview questions and answers~\cite{Hiremath_2024}. In the benchmark, the questions are divided into two types as follows.

\subsubsection{Information Questions} 
The first 33 questions are basic information questions, with a well-known correct answer. For example, ``what is cryptography", or ``what is the CIA triad in Cybersecurity"? These should theoretically be the easiest for LLMs to answer and check the answers of. They are listed in Table \ref{tbl:infq}.

\subsubsection{Situation Question}
The last 7 questions are Cybersecurity situation questions. They place the reader in a situation and ask what they should do. For example, "You receive an email from your bank telling you there is a problem with your account. The email provides instructions and a link so you can log into your account and fix the problem. What should you do?" These questions are more open to interpretation and should be harder for LLMs to answer and check. They are listed in Table \ref{tbl:sitq}.

\subsection{Consistency}
Gemini~\cite{geminiteam2024geminifamilyhighlycapable} and Bloom~\cite{workshop2023bloom176bparameteropenaccessmultilingual} are both deterministic models, and return the exact same response when the same prompt is passed. Therefore, they are perfectly consistent for all thresholds. Since their behavior is deterministic, the consistency check does not provide any useful information and they aren't included in the plots. 

Figures \ref{fig:consistency-analysis-low}-\ref{fig:consistency-analysis-high} depict the results for the low, medium, and high thresholds respectively for the 33 information questions. Figures \ref{fig:consistency-analysis-low-sit}-\ref{fig:consistency-analysis-high-sit} provide the corresponding results for the 7 situational questions. In each figure, there are 4 sets of bars corresponding to each LLM, providing the results of the consistency analysis for the cases where 1 or more, 2 or more, 3 or more, or all 4 of the consistency metrics evaluate to true.

Looking at the results for Low threshold with information questions (Figure \ref{fig:consistency-analysis-low}), the majority of LLMs seem to pass when 1-3 similarity metrics are sufficient to pass for a response pair to be considered consistent. When all 4 similarity metrics are needed, the percentage of questions that pass drops drastically, and not a single model is above 80\%. Comparing this with the situational questions (Figure \ref{fig:consistency-analysis-low-sit}), all the results are noticeably lower, as expected. As opposed to all LLMs passing at least once under the Low threshold, none pass under the High threshold. With the information questions (Figure \ref{fig:consistency-analysis-high}), GPT 4o mini performs the best once again, with GPT 3.5 not far behind. 3.5 and 4o are the only ones to have any questions pass when questions have to pass 3/4 metrics, and 4o mini is the only one to have any questions pass when all 4 metrics are used. The difference between the situation questions (Figure \ref{fig:consistency-analysis-high}) and the information questions is also greatest under the High threshold. Interestingly, GPT 3.5 outperforms 4o mini here, and it and Gemini are the only models to have any questions pass when only 1/4 of metrics are used.


\begin{figure}[!t]
    \centering
    \includegraphics[width=\linewidth]{ConsistencyAnalysisLow.png}
    \caption{Consistency Analysis for Low threshold}
    \label{fig:consistency-analysis-low}
\end{figure}

\begin{figure}[!t]
    \centering
    \includegraphics[width=\linewidth]{ConsistencyAnalysisMed.png}
    \caption{Consistency Analysis for Medium threshold}
    \label{fig:consistency-analysis-med}
\end{figure}

\begin{figure}[!h]
    \centering
    \includegraphics[width=\linewidth]{ConsistencyAnalysisHigh.png}
    \caption{Consistency Analysis for High threshold}
    \label{fig:consistency-analysis-high}
\end{figure}

The Medium threshold has the most interesting results. With the information questions (Figure \ref{fig:consistency-analysis-med}), GPT 4o Mini and GPT 3.5 are the only models to pass when 2/4 metrics are used. With 1 metric, Gemini passes as well. With the situation questions (Figure \ref{fig:consistency-analysis-med-sit}) GPT 4o Mini was the only model to pass when only 1/4 metrics are used.



\begin{figure}[tb]
    \centering
    \includegraphics[width=\linewidth]{ConsistencyAnalysisLowSit.png}
    \caption{Low threshold for Situational Questions}
    \label{fig:consistency-analysis-low-sit}
\end{figure}

\begin{figure}[tb]
    \centering
    \includegraphics[width=\linewidth]{ConsistencyAnalysisMedSit.png}
    \caption{for Medium threshold for Situational Questions}
    \label{fig:consistency-analysis-med-sit}
\end{figure}

\begin{figure}[!h]
    \centering
    \includegraphics[width=\linewidth]{ConsistencyAnalysisHighSit.png}
    \caption{High threshold for Situational Questions}
    \label{fig:consistency-analysis-high-sit}
\end{figure}



Not counting Bloom and Meta OPT~\cite{zhang2022optopenpretrainedtransformer} since they're inherently 100\% consistent, the most consistent models are GPT 4o Mini, GPT 3.5, and Google Gemini, in that order. When putting the average metric scores into tables, in both the regular questions (Table \ref{tab:sim_tab}) and the situation questions (Table \ref{tab:sim_tab}) GPT 4o Mini consistently scores higher in Sequence Matcher and Levenshtein Distance, while 3.5 consistently scores higher on Jaccard Index and Cosine similarity. Sequence and Levenshtein take order into account, so it can be inferred that 4o mini has more variation in the way it words its responses as compared to the older 3.5. Along with that, there is very little variation in the Cosine Similarity and to a lesser degree Jaccard Index scores. This increases a bit on the situation questions but is still considerably lower than the other two. This shows that these two metrics are less useful for comparing LLMs to each other.





\begin{table}[tb]
    \centering
        \caption{Average Similarity Scores per LLM}
    \label{tab:sim_tab}
    \begin{tabular}{|p{1.5cm} | p{1.5cm} | p{1.5cm} | p{1.3cm} | p{1.1cm} |}
        \hline
         Model & Sequence Matcher &  Levenshtein Distance & Jaccard Index & Cosine Similarity \\
         \hline
         GPT 4o Mini	& 30.21 & 46.54 & 89.75 & 89.29 \\ 
         \hline
         GPT 3.5 & 30.82 & 50.56 & 89.49 & 84.63 \\
         \hline
         Gemini & 10.22 & 32.5 & 86.62 & 82.1 \\
         \hline
         Cohere & 13.33 & 33.35 & 79.45 & 81.03 \\
         \hline
         Llama3 & 14.2 & 33.88 & 84.97 & 81.24 \\
         \hline
    \end{tabular}

\end{table}

\begin{table}[tb]
    \centering
        \caption{Average Similarity Scores per LLM Situation Questions}
    \label{tab:sim_tab_sit}
    \begin{tabular}{|p{1.5cm} | p{1.5cm} | p{1.5cm} | p{1.3cm} | p{1.1cm} |}
        \hline
         Model & Sequence Matcher &  Levenshtein Distance & Jaccard Index & Cosine Similarity \\
         \hline
         GPT 4o Mini & 23.31 & 43.62 & 86.79 & 81.84 \\
         \hline
         GPT 3.5 & 33.02 & 46.68 & 84.22 & 81.51 \\
         \hline
         Gemini & 12.31 & 34.24 & 83.9 & 79.69 \\
         \hline
         Cohere & 10.6 & 31.9 & 73.87 & 72.57 \\
         \hline
         Llama3 & 13.86 & 32.09 & 80.63 & 73.39 \\
         \hline
    \end{tabular}

\end{table}

\begin{table}[!h]
\caption{Difference between the average Similarity Scores for Information vs Situation Questions}
    \label{tab:sim_tab_diff}
    \begin{tabular}{|p{1.5cm} | p{1.5cm} | p{1.5cm} | p{1.3cm} | p{1.1cm} |}
\hline
Model           & Sequence Matcher & Levenshtein Distance & Jaccard Index & Cosine Similarity \\ \hline
GPT 4o Mini & 6.9              & 2.92                 & 2.96          & 7.45              \\ \hline
GPT 3.5     & -2.2             & 3.88                 & 5.27          & 3.12              \\ \hline
Gemini   & -2.09            & -1.74                & 2.72          & 2.41              \\ \hline
Cohere          & 2.73             & 1.45                 & 5.58          & 8.46              \\ \hline
Llama3          & 0.34             & 1.79                 & 4.34          & 7.85              \\ \hline
\end{tabular}
\end{table}



\subsubsection{How Types of Questions affect LLM Consistency}
When looking at the raw similarity scores for each question for each LLM, we noticed some patterns. The question "What are the response codes that can be received from a Web Application?" yielded the highest similarity scores for 3 out of 4 metrics on GPT 4o Mini, 1 out of 4 metrics on GPT 3.5, and 3 out of 4 metrics on Gemini. Interestingly, this same question yielded the lowest similarity score on 2 out of 4 of the metrics for Cohere, specifically Sequence Matcher and Levenshtein Distance, the two metrics that take order into account. On the other two metrics, that question scored fairly high for Cohere. It scored a little less than Cohere's highest recorded Jaccard index and in the middle of its highest and lowest recorded Cosine similarity scores. Looking at the responses themselves, they provide the same explanations, but with varying degrees of elaboration. This shows Cohere is not factually inconsistent but still inconsistent in its responses. Overall, there has been very little variation in Cosine and Jaccard scores, which shows relative factual consistency, but a lot of variation in Sequence Matcher and Levenshtein scores which shows that all the LLMs studied display this to some degree.

If we use Sequence Matcher and Levenshtein Distance scores to represent consistency in explanations and Jaccard Index and Cosine Similarity scores to represent informational consistency, looking at a table of the differences between the average Similarity Scores
for Information vs Situation Questions (Table \ref{tab:sim_tab_diff}), informational consistency drops much more for the situation questions than consistency in explanation. In some cases, the average Sequence Matcher and Levenshtein Distance scores are even higher for situational questions, most notably with Gemini, where they are both higher. This shows that LLMs are less factually accurate when given more open to interpretation situational questions, but are sometimes more consistent in explaining their responses to those questions.


\subsection{Agreement}
For agreement, Meta OPT~\cite{metaopt}'s responses had to be left out, because it refuses to give a yes or no answer on whether a response is correct or not, usually responding with "I think it's a good question", or just answering the question again, no matter what prompt we used. This means it cannot be used for Self-Validation, and cannot Cross-Validate the other models. So it isn't included in the Self-Validation plots, and on the Cross-Validation plots every other model only has 5 other LLMs to check its answer instead of 6, Meta OPT is not included.


\subsubsection{Self-Validation}
For Self-Validation, whether it's informational (Figure \ref{fig:self-valid}) or situational questions (Figure \ref{fig:self-valid-sit}), most of the models agree with themselves more than 80\% of the time. Bloom is the only model that is critical of itself. Interestingly, it appears to agree with itself more on the situational questions than the informational questions.


\subsubsection{Cross-Validation}
For Cross-Validation, the plots are organized by the number of LLMs that agree with one LLM's response. The first bar depicts the case when 2 of the 5 LLMs agree with a response, and the last bar depicts the case when all 6 agree. The last bar is only applicable for Meta OPT since Meta OPT itself cannot be used to cross-validate the other models, they all have 5 models checking their responses, while Meta OPT has all 6.


Interestingly, despite being the only 100\% consistent models, Meta OPT and Bloom score the lowest in both the informational questions (Figure \ref{fig:cross-valid}) and the situational questions (Figure \ref{fig:cross-valid-sit}). They actually score higher on the situational questions than the informational ones. Along with Gemini, they're the only models to behave this way.


\subsubsection{Model Hallucination and Validation} \label{sec:hallucination}
When asked to "Explain CIA triad in Cybersecurity", Meta OPT provided an incorrect answer, talking about the Central Intelligence Agency instead of the CIA triad. In the Cross-Validation step, 4 of the 6 LLMs responded that the answer was incorrect, which marks the overall question as incorrect by our parameters. While that showcases the strengths of Cross-Validation with other LLMs, two LLMs still got the validation wrong, those being GPT 3.5 and Cohere. More interestingly, Cohere answered "Yes, you explained the CIA triad, which stands for confidentiality, integrity, and availability, in cybersecurity", generating the correct answer itself but wrongfully believing it was provided in the prompt. Future work on refining the effectiveness of the Cross-Validation algorithm could be on marking the LLMs which are most accurate when validating answers and removing those that are the least. 

Detecting a hallucination is more difficult with situational questions. When prompted with "A friend of yours sends an e-card to your mail. You have to click on the attachment to get the card. What do you do? Justify your answer", Meta OPT simply responded with "I have to click on the attachment to get the card", where the correct answer would be to confirm that it is not malicious first. In the cross-validation step, only GPT 4o Mini identified this as the incorrect answer. Despite Gemini, GPT 3.5, Cohere and Llama3 all giving the correct answer when directly responding to the question, they still believed the incorrect answer was correct despite correctly identifying the security risks in their own responses. This shows the weakness in using LLM agreement to double-check responses for more abstract questions, as an LLM's response to a situation can be a possible course of action, but not necessarily a correct one.

\begin{figure}[!tb]
    \centering
    \includegraphics[width=\linewidth]{SelfValidation.png}
    \caption{Self Validation for Information Questions}
    \label{fig:self-valid}
\end{figure}

\begin{figure}[!tb]
    \centering
    \includegraphics[width=\linewidth]{SelfValidationSit.png}
    \caption{Self Validation for Situational Questions}
    \label{fig:self-valid-sit}
\end{figure}
\begin{figure}[!tb]
    \centering
    \includegraphics[width=\linewidth]{CrossValidation.png}
    \caption{Cross Validation for Information Questions}
    \label{fig:cross-valid}
\end{figure}


\begin{figure}[!tb]
    \centering
    \includegraphics[width=\linewidth]{CrossValidationSit.png}
    \caption{Cross Validation for Situational Questions}
    \label{fig:cross-valid-sit}
\end{figure}

\begin{comment}
    
\end{comment}


\section{Related Work} \label{sec:related}
\putsec{related}{Related Work}

\noindent \textbf{Efficient Radiance Field Rendering.}
%
The introduction of Neural Radiance Fields (NeRF)~\cite{mil:sri20} has
generated significant interest in efficient 3D scene representation and
rendering for radiance fields.
%
Over the past years, there has been a large amount of research aimed at
accelerating NeRFs through algorithmic or software
optimizations~\cite{mul:eva22,fri:yu22,che:fun23,sun:sun22}, and the
development of hardware
accelerators~\cite{lee:cho23,li:li23,son:wen23,mub:kan23,fen:liu24}.
%
The state-of-the-art method, 3D Gaussian splatting~\cite{ker:kop23}, has
further fueled interest in accelerating radiance field
rendering~\cite{rad:ste24,lee:lee24,nie:stu24,lee:rho24,ham:mel24} as it
employs rasterization primitives that can be rendered much faster than NeRFs.
%
However, previous research focused on software graphics rendering on
programmable cores or building dedicated hardware accelerators. In contrast,
\name{} investigates the potential of efficient radiance field rendering while
utilizing fixed-function units in graphics hardware.
%
To our knowledge, this is the first work that assesses the performance
implications of rendering Gaussian-based radiance fields on the hardware
graphics pipeline with software and hardware optimizations.

%%%%%%%%%%%%%%%%%%%%%%%%%%%%%%%%%%%%%%%%%%%%%%%%%%%%%%%%%%%%%%%%%%%%%%%%%%
\myparagraph{Enhancing Graphics Rendering Hardware.}
%
The performance advantage of executing graphics rendering on either
programmable shader cores or fixed-function units varies depending on the
rendering methods and hardware designs.
%
Previous studies have explored the performance implication of graphics hardware
design by developing simulation infrastructures for graphics
workloads~\cite{bar:gon06,gub:aam19,tin:sax23,arn:par13}.
%
Additionally, several studies have aimed to improve the performance of
special-purpose hardware such as ray tracing units in graphics
hardware~\cite{cho:now23,liu:cha21} and proposed hardware accelerators for
graphics applications~\cite{lu:hua17,ram:gri09}.
%
In contrast to these works, which primarily evaluate traditional graphics
workloads, our work focuses on improving the performance of volume rendering
workloads, such as Gaussian splatting, which require blending a huge number of
fragments per pixel.

%%%%%%%%%%%%%%%%%%%%%%%%%%%%%%%%%%%%%%%%%%%%%%%%%%%%%%%%%%%%%%%%%%%%%%%%%%
%
In the context of multi-sample anti-aliasing, prior work proposed reducing the
amount of redundant shading by merging fragments from adjacent triangles in a
mesh at the quad granularity~\cite{fat:bou10}.
%
While both our work and quad-fragment merging (QFM)~\cite{fat:bou10} aim to
reduce operations by merging quads, our proposed technique differs from QFM in
many aspects.
%
Our method aims to blend \emph{overlapping primitives} along the depth
direction and applies to quads from any primitive. In contrast, QFM merges quad
fragments from small (e.g., pixel-sized) triangles that \emph{share} an edge
(i.e., \emph{connected}, \emph{non-overlapping} triangles).
%
As such, QFM is not applicable to the scenes consisting of a number of
unconnected transparent triangles, such as those in 3D Gaussian splatting.
%
In addition, our method computes the \emph{exact} color for each pixel by
offloading blending operations from ROPs to shader units, whereas QFM
\emph{approximates} pixel colors by using the color from one triangle when
multiple triangles are merged into a single quad.



\section{Conclusions and Future Work} \label{sec:conclusions}
In this paper, we ask the following question -- "how consistent are LLM responses, both in the information provided and factually" especially in the context of their use in cybersecurity. LLMs are expected to be used for various security operations in the industry~\cite{gennari2024considerations}. Before we put significant reliance on the black-box LLMs (which the industry has already started), can we evaluate such models on how consistent they are in their responses, and can security stakeholders such as CISOs make decisions on whether and how to use which LLM for tasks as important as cybersecurity operations?

We have carried out an extensive set of experiments and analyzed the consistency of LLMs for their responses against a benchmark of cybersecurity questions. Our experiments demonstrate that LLMs have made significant strides in improving consistency and reducing hallucination in the past couple of years, as newer models like GPT 4o Mini and Meta Llama3 outperform older ones like Meta OPT and Bloom. Despite that, LLMs still have quite a way to go before they become usable for important cybersecurity operations. When confronted with more abstract situational questions, there is a clear drop in consistency and agreement between LLMs. While our self and cross-validation algorithms have been effective at detecting LLM hallucination, they become less reliable the more abstract a question is. In the future, we plan to conduct a more supervised analysis of how accurate LLMs are to specifically select the ones with the best track record for response validation.

We plan to explore the relations between consistency and hallucination in detail, as well as carry out further experiments on classifying LLMs to different cybersecurity tasks. Further understanding of how inconsistent are the responses and how they are generated based on an analysis of the internal states of the models and attention layers may help us fine-tune the models better. 


\bibliographystyle{IEEEtranS}
\bibliography{references}
\vspace{12pt}

\newpage

\section*{Appendix}

\begin{table}[!h]
\caption{List of Cybersecurity Information Questions}\label{tbl:infq}
\begin{tabular}{lp{3.2in}}
Q1:  & What is the difference between   VA(Vulnerability Assessment) and PT (Penetration Testing)? \\
Q2:  & What is a three-way handshake?                                                             \\
Q3:  & What are the response codes that can be received from a Web Application?                   \\
Q4:  & What is the difference between HIDS and NIDS?                                              \\
Q5:  & What are the steps to set up a firewall?                                                   \\
Q6:  & Explain SSL Encryption                                                                     \\
Q7:  & What steps will you take to secure a server?                                               \\
Q8:  & Explain Data Leakage                                                                       \\
Q9:  & What is Port Scanning?                                                                     \\
Q10: & What are the different layers of the OSI model?                                            \\
Q11: & What is a VPN?                                                                             \\
Q12: & What do you understand by Risk, Vulnerability \& Threat in a network?                      \\
Q13: & How can identity theft be prevented?                                                       \\
Q14: & What are black hat, white hat, and grey hat hackers?                                        \\ Q15: & How often should you perform Patch management?                                             \\
Q16: & How would you reset a password-protected BIOS configuration?                               \\
Q17: & What is an ARP and how does it work?                                                       \\
Q18: & What is port blocking within LAN?                                                          \\
Q19: & What are salted hashes?                                                                    \\
Q20: & Explain SSL and TLS                                                                        \\
Q21: & What is 2FA and how can it be implemented for public websites?                             \\
Q22: & What is Cognitive Cybersecurity?                                                           \\
Q23: & What is the difference between VPN and VLAN?                                               \\
Q24: & What is cryptography?                                                                      \\
Q25: & What is the difference between VPN and VLAN?                                               \\
Q26: & What is the difference between Symmetric and Asymmetric encryption?                        \\
Q27: & What is the difference between IDS and IPS?                                                \\
Q28: & Explain the CIA triad in cybersecurity.                                                        \\
Q29: & How is Encryption different from Hashing?                                                  \\
Q30: & What is a Firewall and why is it used?                                                     \\
Q31: & What are some of the common Cyberattacks?                                                  \\
Q32: & What protocols fall under the TCP/IP internet layer?                                           \\
Q33: & What is a Botnet?                                                                         
\end{tabular}

\vspace{0.3cm}
\caption{List of Cybersecurity Situation Questions}\label{tbl:sitq}

\begin{tabular}{lp{3.2in}}
Q1: & A friend of yours sends an e-card to your mail. You have to click on the attachment to get the card. What do you do?   Justify your answer                                                                                                                                                                                                                                                  \\
Q2: & In our computing labs, print billing is often tied to the user’s login.   Sometimes people call to complain about bills for printing they never did only to find out that the bills are, indeed, correct. What do you infer from this situation? Justify                                                                                                                                  \\
Q3: & Which of the following passwords meets UCSC’s password requirements?   a).@\#\$)*\&\textasciicircum{}\% b).akHGksmLN c).UcSc4Evr! d).Password1                                                                                                                                                                                                                                                \\
Q4: & You receive an email from your bank telling you there is a problem with your account. The email provides instructions and a link so you can log into your account and fix the problem. What should you do?                                                                                                                                                                                \\
Q5: & A while back, the IT folks got a number of complaints that one of our campus computers was sending out Viagra spam. They checked it out, and the reports were true: a hacker had installed a program on the computer that made it automatically send out tons of spam emails without the computer owner’s knowledge. How do you think the hacker got into the computer to set this up? \\
Q6: & There is this case that happened in my computer lab. A friend of mine used their Yahoo account at a computer lab on campus. She ensured that her account was not left open before she left the lab. Someone came after her and used the same browser to re-access her account. and they started sending emails from it. What do you think might be going on here?                     \\
Q7: & Two different offices on campus are working to straighten out an error in an employee’s bank account due to a direct deposit mistake. Office \#1 emails the correct account and deposit information to Office \#2, which promptly fixes the problem. The employee confirms with the bank that everything has,   indeed, been straightened out. What is wrong here?                     
\end{tabular}

\end{table}

\end{document}
