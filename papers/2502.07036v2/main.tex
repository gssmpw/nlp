\documentclass[conference]{IEEEtran}
\IEEEoverridecommandlockouts
\usepackage{amsmath,amssymb,amsfonts}
\usepackage{graphicx}
\usepackage{textcomp}
\usepackage{xcolor}
\usepackage{comment}
\usepackage{algorithm}
\usepackage[noend]{algpseudocode}
\usepackage{xcolor}
\usepackage{fontawesome}
\usepackage{tikz}
\usepackage{comment}
\usepackage{centernot}

\usepackage{subfig,hyperref}
\def\BibTeX{{\rm B\kern-.05em{\sc i\kern-.025em b}\kern-.08em
    T\kern-.1667em\lower.7ex\hbox{E}\kern-.125emX}}
\begin{document}

\title{Automated Consistency Analysis of LLMs\\
}

\author{\IEEEauthorblockN{Aditya Patwardhan}
\IEEEauthorblockA{\textit{Department of Computer Science} \\
\textit{Stony Brook University}\\
Stony Brook, USA \\
aapatwardhan@cs.stonybrook.edu}
\and
\IEEEauthorblockN{Vivek Vaidya}
\IEEEauthorblockA{\textit{Department of Computer Science} \\
\textit{Rutgers University}\\
New Brunswick, USA \\
vivek.vaidya@rutgers.edu}
\and
\IEEEauthorblockN{Ashish Kundu}
\IEEEauthorblockA{\textit{Cisco Research} \\
San Jose, USA \\
ashkundu@cisco.com}
}

\maketitle

\begin{abstract}
\begin{abstract}
Retrieval-Augmented Generation (RAG) is often used with Large Language Models (LLMs) to infuse domain knowledge or user-specific information. In RAG, given a user query, a retriever extracts chunks of relevant text from a knowledge base. These chunks are sent to an LLM as part of the input prompt. Typically, any given chunk is repeatedly retrieved across user questions. However, currently, for every question, attention-layers in LLMs fully compute the key values (KVs) repeatedly for the input chunks, as state-of-the-art methods cannot reuse KV-caches when chunks appear at arbitrary locations with arbitrary contexts. Naive reuse leads to output quality degradation.  This leads to potentially redundant computations on expensive GPUs and increases latency. In this work, we propose \sys, a system for managing and reusing precomputed KVs corresponding to the text chunks (we call \textit{chunk-caches}) in RAG-based systems. We present how to identify \hl{\textit{chunk-caches} that are reusable}, how to efficiently perform a small fraction of recomputation to \textit{fix} the cache to maintain output quality, and how to efficiently store and evict \textit{chunk-caches} in the hardware for maximizing reuse while masking any overheads. With real production workloads as well as synthetic datasets, we show that \sys reduces redundant computation by \textbf{51\%} over SOTA prefix-caching and \textbf{75\%} over full recomputation.
\hl{Additionally, with continuous batching on a real production workload, we get a \textbf{1.6$\times$} speedup in throughput and a \textbf{2$\times$} reduction in end-to-end response latency over prefix-caching while maintaining quality, for both the \llama-3-8B and \llama-3-70B models. 
}
\end{abstract}





\end{abstract}

\begin{IEEEkeywords}
 Cybersecurity,  Generative AI. Large Language Models, Agents, Consistency, Trustworthiness, Validity, Reliability, Hallucination
\end{IEEEkeywords}

\section{Introduction}
\section{Introduction}
\label{sec:intro}

\begin{figure*}[tb]
    \centering
    \includegraphics[width=0.848\linewidth]{figs/circuitnn.pdf} 
    \caption{Illustration of differentiable CircuitNN. CircuitNN is designed based on differentiable NAND gates. After DAS is guided by PI and PO pairs of the truth table, CircuitNN can get the precise circuit architecture logic equivalent to the truth table.}
    \label{fig:circuitnn}
\end{figure*}

% 1. Describe the importance of logic synthesis
% 2. Existing Problems
% (a) Neural Architecture Search: Unstable, Predefined Setting, etc.
% (b) Circuit Generation: Probabilistic Model, Logic Equivalence

With the rapid advancement of technology, the scale of integrated circuits (ICs) has expanded exponentially. 
This expansion has introduced significant challenges in chip manufacturing, particularly concerning power and area metrics.
A primary objective in IC design is achieving the same circuit function with fewer transistors, thereby reducing power usage and area occupancy.

Logic synthesis~\cite{hachtel2005logicsynth}, a critical step in electronic design automation (EDA), transforms behavioral-level circuit designs into optimized gate-level circuits, ultimately yielding the final IC layout. 
The primary goal of logic synthesis is to identify the physical implementation with the fewest gates for a given circuit function. 
This task constitutes a challenging NP-hard combinatorial optimization problem. 
Current logic synthesis tools~\cite{brayton2010abc, wolf2013yosys} rely on human-designed heuristics, often leading to sub-optimal outcomes.

Differentiable architecture search (DAS) techniques~\cite{liu2018darts, chu2020darts} offer novel perspectives on addressing challenges in this problem.
Circuit functions can be represented through truth tables, which map binary inputs to their corresponding outputs. 
Truth tables provide a precise representation of input-output relationships, ensuring the design of functionally equivalent circuits.
Inspired by this, researchers~\cite{deepmind2024ai4sys, wang2024tnet} have begun exploring the application of DAS to synthesize circuits directly from truth tables.
Specifically, \citet{deepmind2024ai4sys} proposed CircuitNN, a framework that learns differentiable connection structures with logic gates, enabling the automatic generation of logic circuits from truth tables.
This approach significantly reduces the complexity of traditional circuit generation. 
Building on this, \citet{wang2024tnet} introduced T-Net, a triangle-shaped variant of CircuitNN, incorporating regularization techniques to enhance the efficiency of DAS.

Despite these advancements, several challenges remain. 
The computational complexity of DAS grows quadratically with the number of gates, posing scalability issues.
Although triangle-shaped architecture~\cite{wang2024tnet} partially mitigates this problem, redundancy persists. 
%Additionally, DAS is susceptible to converging to local optima, limiting the ability to search architectures that satisfy the given truth tables~\cite{liu2018darts}. 
%Furthermore, hyperparameters (network depth and layer width) require extensive searches, introducing complexity and prolonging the synthesis process. 
Additionally, DAS is susceptible to converging to local optima~\cite{liu2018darts} and hyperparameters (network depth and layer width) require extensive searches. 
The challenges arise from the vast search space in DAS. 
% Even with predefined settings for CircuitNN, finding a configuration that meets the truth table requires extensive trial and error during the DAS process. 
Intuitively, limiting the search space through predefined parameters (network depth, gates per layer, and connection probabilities) can significantly reduce the complexity.

Recent advances~\cite{openai2023gpt4, abramson2024alphafold3, esser2024sd3, li2024mar} in conditional generative models have demonstrated remarkable performance across language, vision, and graph generation tasks. 
Motivated by these developments, we propose a novel approach to circuit generation that generates preliminary circuit structures to guide DAS in generating refined circuits matching specified truth tables. 
Firstly, we introduce CircuitVQ, a tokenizer with a discrete codebook for circuit tokenization. 
Built upon our Circuit AutoEncoder framework~\cite{hou2022graphmae,li2023maskgae,wu2025mgvga}, CircuitVQ is trained through a circuit reconstruction task. 
Specifically, the CircuitVQ encoder encodes input circuits into discrete tokens using a learnable codebook, while the decoder reconstructs the circuit adjacency matrix based on these tokens.
Subsequently, the CircuitVQ encoder serves as a circuit tokenizer for CircuitAR pretraining, which employs a masked autoregressive modeling paradigm~\cite{chang2022maskgit, li2023mage}. 
In this process, the discrete codes function as supervision signals. 
After training, CircuitAR can generate discrete tokens progressively, which can be decoded into initial circuit structures by the decoder of the CircuitVQ. 
These prior insights can guide DAS in producing refined circuits that match the target truth tables precisely.

Our key contributions can be summarized as follows:
\begin{itemize}
\item We introduce CircuitVQ, a circuit tokenizer that facilitates graph autoregressive modeling for circuit generation, based on our Circuit AutoEncoder framework;
\item Develop CircuitAR, a model trained using masked autoregressive modeling, which generates initial circuit structures conditioned on given truth tables;
\item Propose a refinement framework that integrates differentiable architecture search to produce functionally equivalent circuits guided by target truth tables;
\item Comprehensive experiments demonstrating the scalability and capability emergence of our CircuitAR and the superior performance of the proposed circuit generation approach.
\end{itemize}

% Motivation
% (a) Diffusion (Vision, Graph), Autoregressive (Language, Vision)
% (b) Circuit Generation for Predefined Setting
% (c) Neural Architecture Search for Strict Logic Equivalence

% Contribution
% (a) Circuit Tokenizer (new transformer arch, training strategy)
% (b) CircuitAR (train and gen strategies, post-ar strategy)
% (c) Extensive Evaluation including BitD (Bit Distance) for Scalability



\section{Consistency in the context of LLMs} \label{sec:def}
One key question arises on the use of LLMs is - can we trust an LLM's responses? Reliability of an LLM is dependent on its consistency. 

\subsection{Consistency of Responses}
Consistency of LLMs is defined based on how consistent an LLM's responses to different prompts are. It is about whether the responses that an LLM returns to the same or semantically identical prompts are sufficiently similar or identical to each other. Such responses may be syntactically or structurally different but semantically identical. If a set of responses are not sufficiently similar or identical to each other then we refer to it as "inconsistent responses".   

The prompts are processed by the LLM over a duration of time $\Delta_t$, during which the LLM model is assumed to remain stable and unchanged. In this paper, by the clause - "responses of an LLM to a prompt" means "responses of an LLM to the same prompt or semantically identical prompts issued multiple times". Each issuance of a prompt to an LLM is called as a query.

\subsection{Consistency and other Properties}
Some of the key implications of consistency and inconsistency in the context of LLMs are as follows:
\begin{itemize}    
    \item Accuracy $A$ $\implies$ Consistency $C$: if an LLM is stated to  be accurate in its responses to a prompt, then such responses will be consistent with respect to each other. In turn, lack of consistency implies lack of accuracy: $\neg C \implies \neg A$. 
    \item Consistency $C$ $\centernot\implies$ Accuracy $A$: Semantic consistency of responses does not always imply accuracy of the responses, primarily because an LLM may respond with the semantically equivalent responses for the same prompt or semantically equivalent prompts each time it is queried, but guaranteeing the semantic accuracy of the responses is beyond the problem of maintaining consistency.
    \item Consistency of LLMs are related to the hallucination of LLMs~\cite{mcdonald2024reducing}: The more inconsistent an LLM is in its responses, the more the LLM may hallucinate simply because the LLM is responding with semantically different responses to the same prompt over multiple queries. Section~\ref{sec:hallucination} presents some related findings on hallucination. However, a detailed discussion and analysis of such a relationship is beyond the scope of this paper.
\end{itemize}

\subsection{Formal Definition}

Let $L_i$ refer to a large language model (LLM). Let $p_k$ refer to a prompt in one or more languages that is/are supported by an $L_i$. Let $s_j$ represent an active user session of a $w$'th user entity $u_w$. Let $t_l$ represent the point in time at which a query is made to an LLM or the time at which an LLM responds to a query such that the response is complete (not in the process of incremental output). 
Let $r_v$ $\xleftarrow{}$ $q_v(L_i, s_j, p_k, t_l, u_w)$ refer to a unique query the user $u$ using prompt $p_j$ to $L_i$ leading to response $r_v$. The terms "Prompt" and "Query" are used interchangeably throughout this paper.

Let $L_i$ $\in$ $\mathcal{L}$, which is a set of  LLMs, where $L_i$ $\in$ $\mathcal{L}$, refer to a large language model (LLM). Let $R_v$ $\xleftarrow{}$ $Q_v(\mathcal{L}_i, S_j, P_k, t_l, U_w)$ refer to a unique set of queries the user $u$ using prompt $p_j$ to a subset of LLMs $\mathcal{L}_i$ $\subseteq$ $\mathcal{L}$ leading to a set of responses $R_v$ that are received by the set of users $U_w$, over a set of active sessions $S_j$, where there is an one-to-one mapping of all the elements across the sets of responses, queries, LLMs, sessions, prompts, timestamps of completion of queries, and users.


Prompts $p_x$ and $p_y$ are semantically equivalent, i.e. they are identical, sufficiently identical or sufficiently similar to each other semantically, which is represented by $p_x$ $\equiv$ $p_y$. Responses $r_v$ and $r_w$ are semantically equivalent, i.e. they are identical, sufficiently identical, or sufficiently similar to each other semantically, which is represented by $r_v$ $\equiv$ $r_w$.\\ 

\textbf{Definition 1: Consistency of Responses -- One LLM.}\\
    Given two queries $r_1$  $\xleftarrow{}$ $q_1(L_1, s_1, p_x, t_1, u_1)$ and $r_2$  $\xleftarrow{}$ $q_2(L_1, s_1, p_y, t_3, u_1)$, where $p_x \equiv p_y$, the responses are consistent if $r_1$ $\equiv$ $r_2$.\\ 

\textbf{Definition 2: Consistency of Responses -- Multiple LLMs.}\\ 
    Given two queries $R_1$  $\xleftarrow{}$ $Q_1(\mathcal{L}_1, S_1, p_x, T_1, u_1)$ and $R_2$  $\xleftarrow{}$ $Q_2(\mathcal{L}_1, S_1, p_y, T_3, u_1)$, where $p_x \equiv p_y$, the responses are consistent if for all pairs $<d_v, e_v>$, where $d_v \in R_1$  and $e_v \in R_2$,  $x_v$ $\equiv$ $y_v$.

In the rest of the paper, on the basis of this formal model, we will study and carry out experiments on black-box LLMs. 


\section{Consistency Validation Framework} \label{sec:framework}

\section{Verification and validation}
\label{sec:validation}
This section describes verification and validation efforts taken to ensure that the ETKidney simulator closely mimics ETKAS and ESP. Under model verification, we understand efforts taken to \textit{\enquote{ensure that the computer program of the computerized model and its implementation are correct}} \cite{sargent2020}. Under model validation, we understand efforts taken to assess that the model \textit{\enquote{possesses a satisfactory range of accuracy consistent with the intended application of the model}} \cite{sargent2020}. 

\subsection{Verification of the ETKidney simulator}
\label{section:verification}
We built unit tests to ensure that the behavior of the ETKidney's simulator modules is in agreement with their intended behavior. For instance, unit tests were constructed to check whether HLA match qualities returned by the HLA system module matched HLA match qualities of actual ETKAS match lists. Unit tests were also used to ascertain that the HLA system module returned the correct mismatch probabilities and vPRAs. %To ensure that the correct serological definitions are used by the HLA system module, antigen definitions were exported directly from the match tables published by the ETRL. %(see \href{https://etrl.eurotransplant.org/resources/hla-a/}{https://etrl.eurotransplant.org/resources/hla-a/}). 

\par
Simulation of the graft offering process and of post-transplant survival is based on statistical models, which were estimated in R. Unit tests were constructed to ensure that predicted probabilities in the ETKidney simulator matched predicted probabilities in R for offer acceptance decisions and for post-transplant survival.

\subsection{Validation of the ETKidney simulator}
\label{section:validation}
To ensure face validity of the model, medical doctors from Eurotransplant were actively involved in the development and conceptual design of the simulator. We also had meetings with ETKAC and the ETRL, and we presented the model at the Eurotransplant Annual Meeting to collect feedback on the conceptual model from other major stakeholders, such as medical doctors and transplantation coordinators from the transplantation centers.
\par 
We use input-output validation to assess how closely the model can approximate outcomes of ETKAS and ESP. For this, we simulate kidney allocation between 01-04-2021 and 01-01-2024 under the actual allocation rules used within this simulation window. For the donor input stream, we used all 4,326 donors with kidneys transplanted through ETKAS or ESP in the simulation period. For the candidate input stream, we used all 9,589 candidates who were on the waiting list on 01-04-2021, and all 14,333 candidates who were activated on the waiting list in the simulation period. To enable accurate simulation of the ETKAS balances, international transplantations in the AM program or via combined transplantations were exported from the Eurotransplant database. In simulations, we schedule donors, candidate status updates, and balance update events on the dates these were actually reported to Eurotransplant. Our input-output validation exercise thus keeps the inputs as close as possible to reality, and assesses whether outputs of the ETKidney simulator are comparable to the actual outputs of ETKAS and ESP.
\par
Important is that outputs of the simulator depend on several stochastic processes (offer acceptance behavior, re-listing, and non-standard allocation). To give insight into the resulting variability in simulator outputs, we simulate ETKAS and ESP allocation 200 times over the simulation window and report \enquote{95\%-interquantile ranges} for relevant summary statistics. These 95\%-IQRs are obtained by simulating allocation 200 times and reporting the 2.5th and 97.5th percentiles of simulation outputs. For each of these 200 simulation runs, we use a different set of imputed status trajectories (see Section \ref{subsection:input_streams}). We say that the ETKidney simulator is \textit{well-calibrated} for a quantity of interest if the actually observed summary statistic falls within the 95\%-IQR of the 200 simulations. We do not test statistically whether the mean outcome over the simulations is different from the actually observed outcome, because such tests can always be made statistically significant by increasing the number of simulation runs.

\subsubsection*{Results of input-output validation}
\label{subsection:val_wl_outc}
Table \ref{tab:waitlist_validation} reports input-output validation results for outcomes on the kidney waiting list. The ETKidney simulator is well-calibrated for almost all summary statistics: the total number of transplantations, the number of dual kidney transplantations, the number of ETKAS / ESP transplantations, the number of re-listings, and the number of waiting list deaths per country in all countries. We only observe miscalibration for the number of waiting list deaths in Hungary (-11\%) and the active waiting list size at simulation termination (+1.8\% too many candidates have an active waitlist status).
\begin{table}[h]
	\caption{Input-output validation of waiting list outcomes between 01-04-2021 to 01-01-2024. For simulations, the numbers shown are averages and 95\%-IQR of outcomes over 200 simulations. Ranges are displayed in bold if the simulator is not well-calibrated, i.e. if the actual statistic does not fall within the 95\%-IQR.}
	\input{tables/validation/2025-01-09/summary_waitlist_results}
	\label{tab:waitlist_validation}
\end{table}
\par
The left side of Table \ref{tab:transplant_validation} reports input-output validation results for ETKAS transplantations. The simulator is well-calibrated for the number of transplantations placed by allocation mechanism (standard or non-standard), transplantations by candidate age group, and transplantations in repeat transplantation candidates. The simulator is also well-calibrated for the number of transplantations by HLA match quality, with only the number of 0-mismatched transplantations overestimated ($\Plus$5\%). The number of transplantations in candidates with vPRAs exceeding 95\% is underestimated (-17\%). Such miscalibration seems to have been the result of the introduction of the virtual crossmatch in January 2023 (see Supplementary Table \ref{tab:vpra_vxm}), potentially because the number of positive recipient center crossmatches has decreased after introduction of the virtual crossmatch \cite{Heidt2024}. The simulator is also well-calibrated for the number of transplantations per country, with only a slight overestimation observed in Croatia ($\Plus 3\%)$ and a slight underestimation observed in Hungary (-2\%). Geographical sharing within ETKAS is underestimated, with 5\% extra local or regional transplantations, 11\% fewer interregional and 11\% fewer international transplantations.
\par
The right side of Table \ref{tab:transplant_validation} shows validation results for ESP transplantations. The simulator is again well-calibrated for most relevant outcomes: the number of transplantations in primary and repeat kidney transplantation candidates, and the number of transplantations by HLA match quality and immunization status. The simulator does overestimate the number of kidneys transplanted by allocation mechanism in ESP, with on average 23\% too many kidneys placed via non-standard allocation. An apparent consequence of this is that the number of kidneys allocated to candidates aged below 65 is overestimated ($\Plus$28\%), particularly in Belgium, the Netherlands, and Slovenia (see Supplementary Table \ref{tab:esp}) where centers appear to be reluctant to transplant a candidate aged under 65 with an ESP donor. Finally, the simulator is well-calibrated for the number of transplantations by recipient country and match geography, with the only exception Germany where 2\% too many ESP kidneys are transplanted.
\par
Overall, the ETKidney simulator appears to be well-calibrated for most outcomes of ETKAS and ESP allocation. The results of this input-output validation exercise were discussed with medical doctors from Eurotransplant and ETKAC, who deemed differences small enough to make the simulator useful for determining the impact of alternative kidney allocation policies. We illustrate this with case studies in the next section. 
\begin{table}[h]
	\caption{Validation of the number of transplantations between 01-04-2021 and 01-01-2024. For simulations, shown numbers are averages and 95\%-IQRs over 200 simulations. 
		Statistics are displayed in bold if the actual statistic does not fall within the 95\%-IQR.}
	\input{tables/validation/2025-01-09/summary_transplants.tex}
	\label{tab:transplant_validation}
\end{table}
\FloatBarrier

%\subsubsection*{Discussion of input-output validation results}
%Our input-output validation exercise thus shows that the ETKidney simulator is able to closely approximate most outcomes of ETKAS and ESP allocation. One outcome on which the simulator is not well-calibrated is the number of transplantations in candidates aged below 65 through ESP. Supplementary Table \ref{tab:esp} shows the number of such transplantations by country of listing, and shows that the simulator overestimates the incidence of such transplantations in Belgium (on average 14 vs. 3 in reality), the Netherlands (18 vs. 1) and Slovenia (5 vs. 1). This suggests that there are country differences in willingness to transplant ESP donors in candidates aged below 65, which are not captured by the used graft offer acceptance models.
%\par
%Another important area in which the simulator is miscalibrated for ETKAS is the number of transplantations in immunized candidates with vPRAs exceeding 95\%.  A likely explanation for this is that the virtual crossmatch has reduced the incidence of positive recipient center crossmatches \cite{Heidt2024}. Such crossmatches were considered as an offer decline in developing the graft offer acceptance models, which may explain why the models overestimate decline rates in immunized candidates after introduction of the virtual crossmatch.
%\par
%The simulator also appears slightly miscalibrated in a few other areas. Such areas of miscalibration were discussed with medical doctors from Eurotransplant and ETKAC, who deemed differences small enough to make the simulator useful for determining the impact of alternative kidney allocation policies. We illustrate this with case studies in the next section. 
\FloatBarrier


%\subsubsection*{Post-transplant outcomes}


%In the previous subsections, we reported validation results for the ETKidney simulator. One important finding of this validation exercise is that relatively too few ESP-aged donors (-4.5\%) are placed through the ESP program, and instead allocated through extended / rescue allocation. This may explain why the simulator appears to be poorly calibrated for many summary statistics for the ESP program. A second important finding is that the ETKidney simulator tends to overestimate the number of transplantations in repeat transplantation candidates and in immunized candidates. A potential explanation for this finding could be that re-listed / immunized candidates are more likely to have a positive crossmatch, even after accounting for the vPRA. Calibration of the ETKidney simulator could potentially be improved by using a logistic model to predict the chance of having a positive crossmatch based on donor and candidate characteristics. We did not pursue this for the ETKidney simulator, since preliminary data shows that repeat transplantation / immunized candidates are no longer more likely to have positive crossmatches in Eurotransplant kidney allocation after introduction of the virtual crossmatch in January 2023.



\section{Empirical Analysis on LLMs using this Framework} \label{sec:experiments}
\subsection{Benchmarks}
To test the LLMs, we developed a benchmark on 40 Cybersecurity interview questions from a popular list of cybersecurity interview questions and answers~\cite{Hiremath_2024}. In the benchmark, the questions are divided into two types as follows.

\subsubsection{Information Questions} 
The first 33 questions are basic information questions, with a well-known correct answer. For example, ``what is cryptography", or ``what is the CIA triad in Cybersecurity"? These should theoretically be the easiest for LLMs to answer and check the answers of. They are listed in Table \ref{tbl:infq}.

\subsubsection{Situation Question}
The last 7 questions are Cybersecurity situation questions. They place the reader in a situation and ask what they should do. For example, "You receive an email from your bank telling you there is a problem with your account. The email provides instructions and a link so you can log into your account and fix the problem. What should you do?" These questions are more open to interpretation and should be harder for LLMs to answer and check. They are listed in Table \ref{tbl:sitq}.

\subsection{Consistency}
Gemini~\cite{geminiteam2024geminifamilyhighlycapable} and Bloom~\cite{workshop2023bloom176bparameteropenaccessmultilingual} are both deterministic models, and return the exact same response when the same prompt is passed. Therefore, they are perfectly consistent for all thresholds. Since their behavior is deterministic, the consistency check does not provide any useful information and they aren't included in the plots. 

Figures \ref{fig:consistency-analysis-low}-\ref{fig:consistency-analysis-high} depict the results for the low, medium, and high thresholds respectively for the 33 information questions. Figures \ref{fig:consistency-analysis-low-sit}-\ref{fig:consistency-analysis-high-sit} provide the corresponding results for the 7 situational questions. In each figure, there are 4 sets of bars corresponding to each LLM, providing the results of the consistency analysis for the cases where 1 or more, 2 or more, 3 or more, or all 4 of the consistency metrics evaluate to true.

Looking at the results for Low threshold with information questions (Figure \ref{fig:consistency-analysis-low}), the majority of LLMs seem to pass when 1-3 similarity metrics are sufficient to pass for a response pair to be considered consistent. When all 4 similarity metrics are needed, the percentage of questions that pass drops drastically, and not a single model is above 80\%. Comparing this with the situational questions (Figure \ref{fig:consistency-analysis-low-sit}), all the results are noticeably lower, as expected. As opposed to all LLMs passing at least once under the Low threshold, none pass under the High threshold. With the information questions (Figure \ref{fig:consistency-analysis-high}), GPT 4o mini performs the best once again, with GPT 3.5 not far behind. 3.5 and 4o are the only ones to have any questions pass when questions have to pass 3/4 metrics, and 4o mini is the only one to have any questions pass when all 4 metrics are used. The difference between the situation questions (Figure \ref{fig:consistency-analysis-high}) and the information questions is also greatest under the High threshold. Interestingly, GPT 3.5 outperforms 4o mini here, and it and Gemini are the only models to have any questions pass when only 1/4 of metrics are used.


\begin{figure}[!t]
    \centering
    \includegraphics[width=\linewidth]{ConsistencyAnalysisLow.png}
    \caption{Consistency Analysis for Low threshold}
    \label{fig:consistency-analysis-low}
\end{figure}

\begin{figure}[!t]
    \centering
    \includegraphics[width=\linewidth]{ConsistencyAnalysisMed.png}
    \caption{Consistency Analysis for Medium threshold}
    \label{fig:consistency-analysis-med}
\end{figure}

\begin{figure}[!h]
    \centering
    \includegraphics[width=\linewidth]{ConsistencyAnalysisHigh.png}
    \caption{Consistency Analysis for High threshold}
    \label{fig:consistency-analysis-high}
\end{figure}

The Medium threshold has the most interesting results. With the information questions (Figure \ref{fig:consistency-analysis-med}), GPT 4o Mini and GPT 3.5 are the only models to pass when 2/4 metrics are used. With 1 metric, Gemini passes as well. With the situation questions (Figure \ref{fig:consistency-analysis-med-sit}) GPT 4o Mini was the only model to pass when only 1/4 metrics are used.



\begin{figure}[tb]
    \centering
    \includegraphics[width=\linewidth]{ConsistencyAnalysisLowSit.png}
    \caption{Low threshold for Situational Questions}
    \label{fig:consistency-analysis-low-sit}
\end{figure}

\begin{figure}[tb]
    \centering
    \includegraphics[width=\linewidth]{ConsistencyAnalysisMedSit.png}
    \caption{for Medium threshold for Situational Questions}
    \label{fig:consistency-analysis-med-sit}
\end{figure}

\begin{figure}[!h]
    \centering
    \includegraphics[width=\linewidth]{ConsistencyAnalysisHighSit.png}
    \caption{High threshold for Situational Questions}
    \label{fig:consistency-analysis-high-sit}
\end{figure}



Not counting Bloom and Meta OPT~\cite{zhang2022optopenpretrainedtransformer} since they're inherently 100\% consistent, the most consistent models are GPT 4o Mini, GPT 3.5, and Google Gemini, in that order. When putting the average metric scores into tables, in both the regular questions (Table \ref{tab:sim_tab}) and the situation questions (Table \ref{tab:sim_tab}) GPT 4o Mini consistently scores higher in Sequence Matcher and Levenshtein Distance, while 3.5 consistently scores higher on Jaccard Index and Cosine similarity. Sequence and Levenshtein take order into account, so it can be inferred that 4o mini has more variation in the way it words its responses as compared to the older 3.5. Along with that, there is very little variation in the Cosine Similarity and to a lesser degree Jaccard Index scores. This increases a bit on the situation questions but is still considerably lower than the other two. This shows that these two metrics are less useful for comparing LLMs to each other.





\begin{table}[tb]
    \centering
        \caption{Average Similarity Scores per LLM}
    \label{tab:sim_tab}
    \begin{tabular}{|p{1.5cm} | p{1.5cm} | p{1.5cm} | p{1.3cm} | p{1.1cm} |}
        \hline
         Model & Sequence Matcher &  Levenshtein Distance & Jaccard Index & Cosine Similarity \\
         \hline
         GPT 4o Mini	& 30.21 & 46.54 & 89.75 & 89.29 \\ 
         \hline
         GPT 3.5 & 30.82 & 50.56 & 89.49 & 84.63 \\
         \hline
         Gemini & 10.22 & 32.5 & 86.62 & 82.1 \\
         \hline
         Cohere & 13.33 & 33.35 & 79.45 & 81.03 \\
         \hline
         Llama3 & 14.2 & 33.88 & 84.97 & 81.24 \\
         \hline
    \end{tabular}

\end{table}

\begin{table}[tb]
    \centering
        \caption{Average Similarity Scores per LLM Situation Questions}
    \label{tab:sim_tab_sit}
    \begin{tabular}{|p{1.5cm} | p{1.5cm} | p{1.5cm} | p{1.3cm} | p{1.1cm} |}
        \hline
         Model & Sequence Matcher &  Levenshtein Distance & Jaccard Index & Cosine Similarity \\
         \hline
         GPT 4o Mini & 23.31 & 43.62 & 86.79 & 81.84 \\
         \hline
         GPT 3.5 & 33.02 & 46.68 & 84.22 & 81.51 \\
         \hline
         Gemini & 12.31 & 34.24 & 83.9 & 79.69 \\
         \hline
         Cohere & 10.6 & 31.9 & 73.87 & 72.57 \\
         \hline
         Llama3 & 13.86 & 32.09 & 80.63 & 73.39 \\
         \hline
    \end{tabular}

\end{table}

\begin{table}[!h]
\caption{Difference between the average Similarity Scores for Information vs Situation Questions}
    \label{tab:sim_tab_diff}
    \begin{tabular}{|p{1.5cm} | p{1.5cm} | p{1.5cm} | p{1.3cm} | p{1.1cm} |}
\hline
Model           & Sequence Matcher & Levenshtein Distance & Jaccard Index & Cosine Similarity \\ \hline
GPT 4o Mini & 6.9              & 2.92                 & 2.96          & 7.45              \\ \hline
GPT 3.5     & -2.2             & 3.88                 & 5.27          & 3.12              \\ \hline
Gemini   & -2.09            & -1.74                & 2.72          & 2.41              \\ \hline
Cohere          & 2.73             & 1.45                 & 5.58          & 8.46              \\ \hline
Llama3          & 0.34             & 1.79                 & 4.34          & 7.85              \\ \hline
\end{tabular}
\end{table}



\subsubsection{How Types of Questions affect LLM Consistency}
When looking at the raw similarity scores for each question for each LLM, we noticed some patterns. The question "What are the response codes that can be received from a Web Application?" yielded the highest similarity scores for 3 out of 4 metrics on GPT 4o Mini, 1 out of 4 metrics on GPT 3.5, and 3 out of 4 metrics on Gemini. Interestingly, this same question yielded the lowest similarity score on 2 out of 4 of the metrics for Cohere, specifically Sequence Matcher and Levenshtein Distance, the two metrics that take order into account. On the other two metrics, that question scored fairly high for Cohere. It scored a little less than Cohere's highest recorded Jaccard index and in the middle of its highest and lowest recorded Cosine similarity scores. Looking at the responses themselves, they provide the same explanations, but with varying degrees of elaboration. This shows Cohere is not factually inconsistent but still inconsistent in its responses. Overall, there has been very little variation in Cosine and Jaccard scores, which shows relative factual consistency, but a lot of variation in Sequence Matcher and Levenshtein scores which shows that all the LLMs studied display this to some degree.

If we use Sequence Matcher and Levenshtein Distance scores to represent consistency in explanations and Jaccard Index and Cosine Similarity scores to represent informational consistency, looking at a table of the differences between the average Similarity Scores
for Information vs Situation Questions (Table \ref{tab:sim_tab_diff}), informational consistency drops much more for the situation questions than consistency in explanation. In some cases, the average Sequence Matcher and Levenshtein Distance scores are even higher for situational questions, most notably with Gemini, where they are both higher. This shows that LLMs are less factually accurate when given more open to interpretation situational questions, but are sometimes more consistent in explaining their responses to those questions.


\subsection{Agreement}
For agreement, Meta OPT~\cite{metaopt}'s responses had to be left out, because it refuses to give a yes or no answer on whether a response is correct or not, usually responding with "I think it's a good question", or just answering the question again, no matter what prompt we used. This means it cannot be used for Self-Validation, and cannot Cross-Validate the other models. So it isn't included in the Self-Validation plots, and on the Cross-Validation plots every other model only has 5 other LLMs to check its answer instead of 6, Meta OPT is not included.


\subsubsection{Self-Validation}
For Self-Validation, whether it's informational (Figure \ref{fig:self-valid}) or situational questions (Figure \ref{fig:self-valid-sit}), most of the models agree with themselves more than 80\% of the time. Bloom is the only model that is critical of itself. Interestingly, it appears to agree with itself more on the situational questions than the informational questions.


\subsubsection{Cross-Validation}
For Cross-Validation, the plots are organized by the number of LLMs that agree with one LLM's response. The first bar depicts the case when 2 of the 5 LLMs agree with a response, and the last bar depicts the case when all 6 agree. The last bar is only applicable for Meta OPT since Meta OPT itself cannot be used to cross-validate the other models, they all have 5 models checking their responses, while Meta OPT has all 6.


Interestingly, despite being the only 100\% consistent models, Meta OPT and Bloom score the lowest in both the informational questions (Figure \ref{fig:cross-valid}) and the situational questions (Figure \ref{fig:cross-valid-sit}). They actually score higher on the situational questions than the informational ones. Along with Gemini, they're the only models to behave this way.


\subsubsection{Model Hallucination and Validation} \label{sec:hallucination}
When asked to "Explain CIA triad in Cybersecurity", Meta OPT provided an incorrect answer, talking about the Central Intelligence Agency instead of the CIA triad. In the Cross-Validation step, 4 of the 6 LLMs responded that the answer was incorrect, which marks the overall question as incorrect by our parameters. While that showcases the strengths of Cross-Validation with other LLMs, two LLMs still got the validation wrong, those being GPT 3.5 and Cohere. More interestingly, Cohere answered "Yes, you explained the CIA triad, which stands for confidentiality, integrity, and availability, in cybersecurity", generating the correct answer itself but wrongfully believing it was provided in the prompt. Future work on refining the effectiveness of the Cross-Validation algorithm could be on marking the LLMs which are most accurate when validating answers and removing those that are the least. 

Detecting a hallucination is more difficult with situational questions. When prompted with "A friend of yours sends an e-card to your mail. You have to click on the attachment to get the card. What do you do? Justify your answer", Meta OPT simply responded with "I have to click on the attachment to get the card", where the correct answer would be to confirm that it is not malicious first. In the cross-validation step, only GPT 4o Mini identified this as the incorrect answer. Despite Gemini, GPT 3.5, Cohere and Llama3 all giving the correct answer when directly responding to the question, they still believed the incorrect answer was correct despite correctly identifying the security risks in their own responses. This shows the weakness in using LLM agreement to double-check responses for more abstract questions, as an LLM's response to a situation can be a possible course of action, but not necessarily a correct one.

\begin{figure}[!tb]
    \centering
    \includegraphics[width=\linewidth]{SelfValidation.png}
    \caption{Self Validation for Information Questions}
    \label{fig:self-valid}
\end{figure}

\begin{figure}[!tb]
    \centering
    \includegraphics[width=\linewidth]{SelfValidationSit.png}
    \caption{Self Validation for Situational Questions}
    \label{fig:self-valid-sit}
\end{figure}
\begin{figure}[!tb]
    \centering
    \includegraphics[width=\linewidth]{CrossValidation.png}
    \caption{Cross Validation for Information Questions}
    \label{fig:cross-valid}
\end{figure}


\begin{figure}[!tb]
    \centering
    \includegraphics[width=\linewidth]{CrossValidationSit.png}
    \caption{Cross Validation for Situational Questions}
    \label{fig:cross-valid-sit}
\end{figure}

\begin{comment}
    
\end{comment}


\section{Related Work} \label{sec:related}

\section{Related Work} \label{sec:related}

% \textbf{Adversarial Attack}
\textbf{Attacks on SLAM.} 
%With the rise of machine learning, 
The robustness of computer vision systems is being actively investigated. With the emergence of adversarial images in the digital domain by adding optimized noise directly to images~\cite{szegedy2013intriguing,carlini2017towards}, researchers find that such attacks also exist physically in the real world \cite{eykholt2018robust,song2018physical,zhao2019seeing}. To fill the gap between attacks in the digital and physical worlds, recent studies have demonstrated that attacks on real-world computer vision systems are practical \cite{eykholt2018robust,li2019adversarial,man2020ghostimage,sharif2016accessorize,zhao2019seeing,zhou2018invisible}. However, attacks on traditional computer vision methods such as SLAM are relatively less explored. \cite{yoshida2022adversarial} proposes an attack against the scan matching algorithm in LiDAR-based SLAM, while most SLAMs in AR/VR devices rely on different sensors like RGB/depth cameras and IMUs. \cite{ikram2022perceptual} and \cite{chen2024adversary} mislead visual SLAM by poisoning the images with special patterns, and \cite{wang2021can} causes the camera to fail using infrared light. In our work, we demonstrate attacks on Visual-Inertial SLAM (VI-SLAM) by perturbing the IMU readings, rather than cameras, and showing its impact on XR user experience. 

\textbf{Acoustic Injection Attacks.} Among various physical attacks, acoustic injection attacks are attractive due to their low cost. Son~\etal~\cite{son2015rocking} were the first to introduce acoustic attacks on MEMS gyroscopes, demonstrating how these attacks could lead to sensor denial-of-service and result in drone crashes. WALNUT~\cite{trippel2017walnut} expanded on this by developing output biasing and control attacks that enable precise manipulation of MEMS accelerometer outputs using modulated sound waves. Wang et al.~\cite{wang2017sonic} demonstrated a sonic gun, showcasing the vulnerability of various smart devices (\eg drones and self-balancing vehicles) to acoustic attacks. Tu et al. \cite{tu2018injected} designed side-swing and switching attacks to alter the outputs of MEMS gyroscopes and accelerometers. Furthermore, Ji et al. \cite{ji2021poltergeist} fool the object detectors by applying acoustic attack to the image stabilizers commonly used in modern cameras. However, none of the existing works study the relationship between the acoustic injections and SLAM outputs on recent XR devices. 

% \zijian{Do we need one session about security in AR/VR?}
% \yicheng{TODO}
%\jiasi{cite the AIVR paper (UMass Amherst?) paper is we have not already. They add IMU perturbation but w/o SLAM, iirc} \yicheng{Cited}

\textbf{XR Security and Privacy.} 
%Security and privacy concerns in XR systems have gained significant attention. 
For single-user XR systems, researchers have demonstrated various side-channel attacks to extract sensitive information (\eg keystrokes) through video feeds~\cite{ling2019know}, head movements~\cite{nair2023unique, slocum2023going}, architectural hints~\cite{zhang2023its,shang2020arspy}, power usage~\cite{li2024dangers}, and EM side-channel leakages~\cite{al2021vr}. In multi-user XR systems, Su et al.~\cite{su2024remote} use avatar motion data to infer keystrokes in shared VR environments. Slocum et al.~\cite{slocum2024doesn} reveal vulnerabilities in the shared state frameworks of multi-user AR. Similarly, Lebeck et al.~\cite{lebeck2017securing} highlight risks like deceptive virtual objects and emphasize access control for managing shared physical and virtual spaces. Ruth et al.~\cite{ruth2019secure} further propose a secure multi-user AR framework focusing on content sharing and permissions.
Chandio et al.~\cite{chandio2024stealthy} %introduced a multi-modal spatiotemporal attack that 
simultaneously manipulated visual and inertial sensors to disrupt XR pose estimation. However, their study evaluated the attack using offline datasets and assumed the attacker's capability to manipulate IMU data streams through acoustic means, without real experiments. Ours is the first to demonstrate acoustic injection attacks on recent XR devices, like the Hololens 2, in the real world.
 



\section{Conclusions and Future Work} \label{sec:conclusions}
In this paper, we ask the following question -- "how consistent are LLM responses, both in the information provided and factually" especially in the context of their use in cybersecurity. LLMs are expected to be used for various security operations in the industry~\cite{gennari2024considerations}. Before we put significant reliance on the black-box LLMs (which the industry has already started), can we evaluate such models on how consistent they are in their responses, and can security stakeholders such as CISOs make decisions on whether and how to use which LLM for tasks as important as cybersecurity operations?

We have carried out an extensive set of experiments and analyzed the consistency of LLMs for their responses against a benchmark of cybersecurity questions. Our experiments demonstrate that LLMs have made significant strides in improving consistency and reducing hallucination in the past couple of years, as newer models like GPT 4o Mini and Meta Llama3 outperform older ones like Meta OPT and Bloom. Despite that, LLMs still have quite a way to go before they become usable for important cybersecurity operations. When confronted with more abstract situational questions, there is a clear drop in consistency and agreement between LLMs. While our self and cross-validation algorithms have been effective at detecting LLM hallucination, they become less reliable the more abstract a question is. In the future, we plan to conduct a more supervised analysis of how accurate LLMs are to specifically select the ones with the best track record for response validation.

We plan to explore the relations between consistency and hallucination in detail, as well as carry out further experiments on classifying LLMs to different cybersecurity tasks. Further understanding of how inconsistent are the responses and how they are generated based on an analysis of the internal states of the models and attention layers may help us fine-tune the models better. 


\bibliographystyle{IEEEtranS}
\bibliography{references}
\vspace{12pt}

\newpage

\section*{Appendix}

\begin{table}[!h]
\caption{List of Cybersecurity Information Questions}\label{tbl:infq}
\begin{tabular}{lp{3.2in}}
Q1:  & What is the difference between   VA(Vulnerability Assessment) and PT (Penetration Testing)? \\
Q2:  & What is a three-way handshake?                                                             \\
Q3:  & What are the response codes that can be received from a Web Application?                   \\
Q4:  & What is the difference between HIDS and NIDS?                                              \\
Q5:  & What are the steps to set up a firewall?                                                   \\
Q6:  & Explain SSL Encryption                                                                     \\
Q7:  & What steps will you take to secure a server?                                               \\
Q8:  & Explain Data Leakage                                                                       \\
Q9:  & What is Port Scanning?                                                                     \\
Q10: & What are the different layers of the OSI model?                                            \\
Q11: & What is a VPN?                                                                             \\
Q12: & What do you understand by Risk, Vulnerability \& Threat in a network?                      \\
Q13: & How can identity theft be prevented?                                                       \\
Q14: & What are black hat, white hat, and grey hat hackers?                                        \\ Q15: & How often should you perform Patch management?                                             \\
Q16: & How would you reset a password-protected BIOS configuration?                               \\
Q17: & What is an ARP and how does it work?                                                       \\
Q18: & What is port blocking within LAN?                                                          \\
Q19: & What are salted hashes?                                                                    \\
Q20: & Explain SSL and TLS                                                                        \\
Q21: & What is 2FA and how can it be implemented for public websites?                             \\
Q22: & What is Cognitive Cybersecurity?                                                           \\
Q23: & What is the difference between VPN and VLAN?                                               \\
Q24: & What is cryptography?                                                                      \\
Q25: & What is the difference between VPN and VLAN?                                               \\
Q26: & What is the difference between Symmetric and Asymmetric encryption?                        \\
Q27: & What is the difference between IDS and IPS?                                                \\
Q28: & Explain the CIA triad in cybersecurity.                                                        \\
Q29: & How is Encryption different from Hashing?                                                  \\
Q30: & What is a Firewall and why is it used?                                                     \\
Q31: & What are some of the common Cyberattacks?                                                  \\
Q32: & What protocols fall under the TCP/IP internet layer?                                           \\
Q33: & What is a Botnet?                                                                         
\end{tabular}

\vspace{0.3cm}
\caption{List of Cybersecurity Situation Questions}\label{tbl:sitq}

\begin{tabular}{lp{3.2in}}
Q1: & A friend of yours sends an e-card to your mail. You have to click on the attachment to get the card. What do you do?   Justify your answer                                                                                                                                                                                                                                                  \\
Q2: & In our computing labs, print billing is often tied to the user’s login.   Sometimes people call to complain about bills for printing they never did only to find out that the bills are, indeed, correct. What do you infer from this situation? Justify                                                                                                                                  \\
Q3: & Which of the following passwords meets UCSC’s password requirements?   a).@\#\$)*\&\textasciicircum{}\% b).akHGksmLN c).UcSc4Evr! d).Password1                                                                                                                                                                                                                                                \\
Q4: & You receive an email from your bank telling you there is a problem with your account. The email provides instructions and a link so you can log into your account and fix the problem. What should you do?                                                                                                                                                                                \\
Q5: & A while back, the IT folks got a number of complaints that one of our campus computers was sending out Viagra spam. They checked it out, and the reports were true: a hacker had installed a program on the computer that made it automatically send out tons of spam emails without the computer owner’s knowledge. How do you think the hacker got into the computer to set this up? \\
Q6: & There is this case that happened in my computer lab. A friend of mine used their Yahoo account at a computer lab on campus. She ensured that her account was not left open before she left the lab. Someone came after her and used the same browser to re-access her account. and they started sending emails from it. What do you think might be going on here?                     \\
Q7: & Two different offices on campus are working to straighten out an error in an employee’s bank account due to a direct deposit mistake. Office \#1 emails the correct account and deposit information to Office \#2, which promptly fixes the problem. The employee confirms with the bank that everything has,   indeed, been straightened out. What is wrong here?                     
\end{tabular}

\end{table}

\end{document}
