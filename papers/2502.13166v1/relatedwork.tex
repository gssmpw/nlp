\section{Related Work}
\label{sec:rewk}
\citet{McClean2018landscapes} first investigated barren plateau (BP) phenomenons and demonstrated that under the assumption of the 2-design Haar distribution, gradient variance in \qnns\ will exponentially decrease to zero during training as the model size increases. In recent years, enormous studies have been devoted to mitigating BP issues in \qnns~\cite{qi2023barren}. \citet{cunningham2024investigating} categorize most existing studies into the following five groups.
(i) Initialization-based strategies initialize model parameters with various well-designed distributions in the initialization stage~\cite{grant2019initialization, sack2022avoiding, mele2022avoiding, grimsley2023adaptive, liu2023mitigating, park2024hamiltonian}.
(ii) Optimization-based strategies address BP issues and further enhance trainability during optimization~\cite{ostaszewski2021structure, suzuki2021normalized, heyraud2023estimation, liu2024mitigating, sannia2024engineered}. %wu2021mitigating, gharibyan2023hierarchical, sciorilli2024towards, falla2024graph
(iii) Model-based strategies attempt to mitigate BPs by proposing new model architectures~\cite{li2021vsql, bharti2021simulator, du2022quantum, selvarajan2023dimensionality, tuysuz2023classical, Kashif2024resQNets}. %zhang2022quark, shin2024layerwise
(iv) To address both BPs and saddle points, \citet{zhuang2024improving} regularize \qnns' model parameters via Bayesian approaches.
(v) \citet{rappaport2023measurement} measure BP phenomenon via various informative metrics.