
\documentclass{article} % For LaTeX2e
\usepackage{iclr2025_conference,times}

% Optional math commands from https://github.com/goodfeli/dlbook_notation.
%%%%% NEW MATH DEFINITIONS %%%%%

% \usepackage{amsmath,amsfonts,bm}
\usepackage{amsmath,amsfonts}

\usepackage{pifont}


\newcommand{\R}{\mathbb{R}}


\def\va{{\mathbf{a}}}
\def\vg{{\mathbf{g}}}

% Sets
\def\sR{\mathbb{R}}
\def\sC{\mathbb{C}}
\def\sZ{\mathbb{Z}}
\def\sN{\mathbb{N}}
\def\sQ{\mathbb{Q}}

\def\sS{\mathcal{S}}



% Vectors
\def\vzero{{\mathbf{0}}}
\def\vone{{\mathbf{1}}}
\def\vmu{{\mathbf{\mu}}}
\def\vtheta{{\mathbf{\theta}}}
\def\va{{\mathbf{a}}}
\def\vb{{\mathbf{b}}}
\def\vc{{\mathbf{c}}}
\def\vd{{\mathbf{d}}}
\def\ve{{\mathbf{e}}}
\def\vf{{\mathbf{f}}}
\def\vg{{\mathbf{g}}}
\def\vh{{\mathbf{h}}}
\def\vi{{\mathbf{i}}}
\def\vj{{\mathbf{j}}}
\def\vk{{\mathbf{k}}}
\def\vl{{\mathbf{l}}}
\def\vm{{\mathbf{m}}}
\def\vn{{\mathbf{n}}}
\def\vo{{\mathbf{o}}}
\def\vp{{\mathbf{p}}}
\def\vq{{\mathbf{q}}}
\def\vr{{\mathbf{r}}}
\def\vs{{\mathbf{s}}}
\def\vt{{\mathbf{t}}}
\def\vu{{\mathbf{u}}}
\def\vv{{\mathbf{v}}}
\def\vw{{\mathbf{w}}}
\def\vx{{\mathbf{x}}}
\def\vy{{\mathbf{y}}}
\def\vz{{\mathbf{z}}}
\def\vzeta{{\mathbf{\zeta}}}

% Matrix
\def\mA{{\mathbf{A}}}
\def\mB{{\mathbf{B}}}
\def\mC{{\mathbf{C}}}
\def\mD{{\mathbf{D}}}
\def\mE{{\mathbf{E}}}
\def\mF{{\mathbf{F}}}
\def\mG{{\mathbf{G}}}
\def\mH{{\mathbf{H}}}
\def\mI{{\mathbf{I}}}
\def\mJ{{\mathbf{J}}}
\def\mK{{\mathbf{K}}}
\def\mL{{\mathbf{L}}}
\def\mM{{\mathbf{M}}}
\def\mN{{\mathbf{N}}}
\def\mO{{\mathbf{O}}}
\def\mP{{\mathbf{P}}}
\def\mQ{{\mathbf{Q}}}
\def\mR{{\mathbf{R}}}
\def\mS{{\mathbf{S}}}
\def\mT{{\mathbf{T}}}
\def\mU{{\mathbf{U}}}
\def\mV{{\mathbf{V}}}
\def\mW{{\mathbf{W}}}
\def\mX{{\mathbf{X}}}
\def\mY{{\mathbf{Y}}}
\def\mZ{{\mathbf{Z}}}
\def\mBeta{{\mathbf{\beta}}}
\def\mPhi{{\mathbf{\Phi}}}
\def\mLambda{{\mathbf{\Lambda}}}
\def\mSigma{{\mathbf{\Sigma}}}


% Expectation
% \def\eE{\mathop{\mathbb{E}}\limits}
\def\eE{\mathbb{E}}

% Probability
\def\pP{\mathbb{P}}

% Tilde
\def\tf{\tilde{f}}
\def\tS{\tilde{S}}
\def\wtF{\widetilde{\mathcal{F}}}
\def\whR{\widehat{R}}
\def\tvx{\tilde{\mathbf{x}}}
\def\ty{\tilde{y}}


\def\defeq{\overset{\textup{def}}{=}}
% \def\defeq{\overset{.}{=}}
\def\defone{\overset{\text{\ding{172}}}{=}}
\def\deftwo{\overset{\text{\ding{173}}}{=}}
\def\leqone{\overset{\text{\ding{172}}}{\leq}}
\def\leqtwo{\overset{\text{\ding{173}}}{\leq}}
\def\leqthree{\overset{\text{\ding{174}}}{\leq}}
\def\leqfour{\overset{\text{\ding{175}}}{\leq}}
\def\eqone{\overset{\text{\ding{172}}}{=}}
\def\eqtwo{\overset{\text{\ding{173}}}{=}}
\def\eqthree{\overset{\text{\ding{174}}}{=}}
\def\eqfour{\overset{\text{\ding{175}}}{=}}
\def\geqfive{\overset{\text{\ding{176}}}{\geq}}

\usepackage{hyperref}
\usepackage{url}


\title{Elucidating the Preconditioning in Consistency Distillation}

% Authors must not appear in the submitted version. They should be hidden
% as long as the \iclrfinalcopy macro remains commented out below.
% Non-anonymous submissions will be rejected without review.

\author{%
  Kaiwen Zheng$^{\dagger1}$\thanks{Work done during an internship at Shengshu; \quad $^\dagger$Equal contribution; \quad $^\ddagger$The corresponding author.}, \quad Guande He$^{\dagger1}$\footnotemark[1], \quad  Jianfei Chen$^1$, \quad  Fan Bao$^{12}$, \quad Jun Zhu$^{\ddagger123}$\\
  $^1$Dept. of Comp. Sci. \& Tech., Institute for AI, BNRist Center, THBI Lab\\
  $^1$Tsinghua-Bosch Joint ML Center, Tsinghua University, Beijing, China\\
  $^2$Shengshu Technology, Beijing \quad $^3$Pazhou Lab (Huangpu), Guangzhou, China \\
  \texttt{zkwthu@gmail.com; guande.he17@outlook.com;}\\
  \texttt{fan.bao@shengshu.ai; \{jianfeic, dcszj\}@tsinghua.edu.cn} \\
}

% The \author macro works with any number of authors. There are two commands
% used to separate the names and addresses of multiple authors: \And and \AND.
%
% Using \And between authors leaves it to \LaTeX{} to determine where to break
% the lines. Using \AND forces a linebreak at that point. So, if \LaTeX{}
% puts 3 of 4 authors names on the first line, and the last on the second
% line, try using \AND instead of \And before the third author name.

\newcommand{\fix}{\marginpar{FIX}}
\newcommand{\new}{\marginpar{NEW}}

\iclrfinalcopy % Uncomment for camera-ready version, but NOT for submission.
\begin{document}


\maketitle

\begin{abstract}
Consistency distillation is a prevalent way for accelerating diffusion models adopted in consistency (trajectory) models, in which a student model is trained to traverse backward on the probability flow (PF) ordinary differential equation (ODE) trajectory determined by the teacher model. Preconditioning is a vital technique for stabilizing consistency distillation, by linear combining the input data and the network output with pre-defined coefficients as the consistency function. It imposes the boundary condition of consistency functions without restricting the form and expressiveness of the neural network. However, previous preconditionings are hand-crafted and may be suboptimal choices. In this work, we offer the first theoretical insights into the preconditioning in consistency distillation, by elucidating its design criteria and the connection to the teacher ODE trajectory. Based on these analyses, we further propose a principled way dubbed \textit{Analytic-Precond} to analytically optimize the preconditioning according to the consistency gap (defined as the gap between the teacher denoiser and the optimal student denoiser) on a generalized teacher ODE. We demonstrate that Analytic-Precond can facilitate the learning of trajectory jumpers, enhance the alignment of the student trajectory with the teacher's, and achieve $2\times$ to $3\times$ training acceleration of consistency trajectory models in multi-step generation across various datasets.
\end{abstract}

\section{Introduction}
\label{sec:intro}

Foundational models (FMs)~\cite{zhang2024data, zhou2023comprehensive} have shown remarkable progress in the healthcare domain, enabling professional-like assessment of disease diagnosis, treatment decision-making, and monitoring~\cite{zhang2023text, wang2022medclip, lu2023mi-zero}. 
Examples include LLaVA-Med~\cite{li2023llava}, Med-PaLM Multimodal~\cite{tu2024towards}, and Med-Flamingo~\cite{moor2023med}, have demonstrated their capacity on question answering, medical image analysis, and report generation.
These studies follow a predominant top-down model development strategy that requires upstream developers to collect data and train models for downstream tasks. 
Consequently, the developed model capabilities are heavily dependent on the training data, limiting their generalization performance in diverse clinical scenarios. 
For instance, Med-Gemini~\cite{yang2024advancing} reveals promising general capabilities in report generation while it lags behind state-of-the-art (SoTA) models on classification tasks, especially for out-of-domain applications. 
This indicates that while the generalizability of the foundation model is promising, more solutions are expected to meet the various specialized clinical needs.

To address these challenges, multi-center data centralization becomes essential to enhance model capacity and robustness across varied clinical scenarios~\cite{rajpurkar2022ai}. 
Centralizing distributed data can significantly improve model training and inference performance.
However, the process of medical data storage, transfer, and aggregation among centers requires extra efforts to ensure data security and system interoperability~\cite{bradford2020international}.
Moreover, a growing concern for patient privacy makes large-scale multi-center data sharing particularly challenging. 
While efforts like federated learning~\cite{wen2023survey, li2020review} can achieve good model performance on local data, the need for synchronized system coordination presents significant challenges, as clients are unable to update asynchronously. This limitation greatly restricts the practical capability of such approaches.
As a result, without a flexible collaboration, medical community still struggles to fully utilize the isolated data and local computation resources for comprehensive medical AI model development. 
To address this dilemma, open-source platforms encourage public data sharing and knowledge integration~\cite{markiewicz2021openneuro, zenodo}.
However, these platforms focus solely on raw data sharing while seldom providing collaborative model training or cooperation between different institutions.
Recently, collaborative learning has emerged as a viable approach for enhancing multi-model robustness~\cite{boulemtafes2020review}. 
For instance, software-like model development~\cite{raffel2023building} mimics software engineering practices by introducing structured workflows, enabling merging, version control, and continuous model integration.
Under this design, model ability can be strengthened with incremental knowledge updates similar to the version updating in software development. 

Although collaborative learning provides a multi-model collaboration, two key challenges remain in the leakage of raw data during collaboration~\cite{huang2023lorahub} and the synchronization of multiple collaborators~\cite{mcmahan2017communication} in the medical AI community. It is still challenging to integrate decentralized, privacy-sensitive data across institutions, leading to under-utilized insights and fragmented knowledge sharing~\cite{kaissis2020secure, rajpurkar2022ai, abdullah2021ethics}.
 To address these challenges, inspired by the collaborative software development, we propose \textbf{Med}ical \textbf{Fo}undation Models Me\textbf{rg}ing (\textbf{MedForge}), a cooperative workflow enabling continuously community-driven foundation model (FM) development.
MedForge enables a lightweight manner for individual centers to share their knowledge among multiple centers, minimizing the burden of data transmission and integration while enhancing model robustness.
Meanwhile, MedForge facilitates asynchronous and flexible collaboration, allowing individual centers to continuously update and improve medical FMs without the need for real-time synchronization.
Similar to open-source software development, MedForge incrementally updates medical knowledge and follows a sustainable model development scheme. 
This key design emphasizes a bottom-up construction of a multi-task medical FM, allowing downstream users to collaboratively build, refine, and update the upstream model according to their local resources. Our major contributions of MedForge are as below: 
\begin{enumerate}
    \item[$\bullet$] We introduce a collaborative workflow to promote the merging scheme of open-source software development. Our proposed MedForge allows distributed clinical centers to asynchronously contribute to comprehensive medical model construction while reducing transmitting costs among centers and avoiding the leakage of raw data, thus enhancing the utilization of private resources in the healthcare system. 
    \item[$\bullet$] We propose two effective knowledge-merging strategies for the asynchronous branch contribution. The MedForge-Fusion strategy updates the plugin module parameters of the main model during the merging phase, whereas the MedForge-Mixture strategy integrates the output of the plugin module by memorizing each contributor's coefficient. These strategies make MedForge more flexible and versatile. MedForge-Fusion is friendly to implement, while the MedForge-Mixture offers better performance and robustness.
    \item[$\bullet$]  We comprehensively evaluate model merging strategies to accumulate medical knowledge among multiple branch plugin modules. MedForge yields superior performance on medical classification tasks compared to other collaborative baselines across multiple datasets. We demonstrate the robustness of MedForge by shuffling the task order and evaluating various configurations of plugin modules and dataset distillation methods.
\end{enumerate}



\section{Background} \label{sec:background}

% \subsection{Capture the Flag (CTF) Challenges}

% CTF challenges simulate real-world cyber-attack scenarios and have emerged as a popular medium for practical cybersecurity training, evaluation, and research. These challenges can simulate real-world attack and defense scenarios and thus assist competitors in developing practical skills in areas such as cryptography, binary exploitation, and reverse engineering. 
% Evaluation of autonomous LLM agents works best with jeopardy-style CTF challenges that focus on standalone software that must be compromised \cite{shao2024nyu,pieterse2024friend}.
% The standalone software may be a binary that can be reverse engineered or exploited, encrypted data that can be decrypted, or a web server whose authentication can be bypassed. After successfully compromising the software, a unique ``flag'' string is either found or revealed by the software server.
% The unique flag string is a concrete indicator of the success of a CTF challenge.
% Recent studies use benchmarks of CTF challenges to evaluate LLM agents on their ability to solve complex tasks and demonstrate practical skills in cybersecurity \cite{shao2024nyu,shao2024empirical,abramovich2024enigma, muzsai2024hacksynth, zhang2024cybenchframeworkevaluatingcybersecurity,yang2023language,turtayev2024hacking}
% Platforms like PicoCTF~\cite{picoctf}, TryHackMe~\cite{tryhackme}, CTFTime~\cite{ctftime} and HackTheBox~\cite{hackthebox} have popularized these formats by providing structured challenges for learners at various skill levels.

% Research indicates that CTF challenges can foster cybersecurity expertise and serve as tools for evaluating facility with cybersecurity skills~\cite{chicone2018using}. They are widely used in academia to enhance learning outcomes in cybersecurity education, with studies demonstrating their effectiveness in promoting analytical thinking and teamwork~\cite{hanafi2021ctf,leune2017using,vykopal2020benefits}. Furthermore, the integration of CTF challenges into research environments enables benchmarking of advanced AI systems like LLMs. .

% Yet, challenges in CTF design persist. These include achieving significant performance, preserving context across tasks, and handling complex, dynamic CTFs that rely on multidisciplinary approaches. Implementing strategies to address these issues enhances problem-solving efficiency, enabling more accurate, adaptive, and effective responses to evolving challenges within CTF environments.


% \subsection{Prompt Engineering}
% \subsection{Prompt Engineering for CTF}
% \subsection{LLM Agents}

% As the use of LLMs to solve CFT challenges expands, prompt engineering is becoming a critical technique for enhancing performance. Various methods have been explored to craft prompts that effectively guide LLMs to the solution of complex cybersecurity problems. Each of these solutions have their own unique strengths and limitations.
%\meet{add more references for LLM agents in other domains, like SWE-Agent, also talk about use of function calling}
Text-based LLMs take a text prompt as input from the user, and produce a text output that follows the user prompt.
LLMs have a finite length of text tokens that they can process called the context.
An alternating sequence of user prompts and LLM outputs makes a conversation and is the basis of chat-based LLM interfaces like ChatGPT.
To remove the user from the loop and create autonomous agents, a feedback mechanism is added based on the LLM outputs, so that the LLM can autonomously continue the conversation.
\citet{yang2023intercode} introduce iterative feedback prompting where the LLM is tasked with writing a piece of code, and the code's compilation and execution logs are provided as feedback, which the LLM uses to iteratively refine it's output.
Recent LLMs support function calling, a way to provide a set of actions to the LLM that it may choose to ``call'' as a function.
In this manner, LLM agents can be provided with many ``tools'' such as a command line, web search, file editing, and code execution \cite{wang2024surveyllmagents}, so that they can autonomously perform various tasks like software development \cite{yang2024sweagent}, web browsing \cite{yoran2024assistantbench}, or solve CTF challenges~\cite{shao2024nyu, abramovich2024enigma}.

With access to the command line and file editing tools, LLM agents can autonomously solve many tasks, but they still struggle on complex long-horizon tasks such as CTF challenges that require multiple steps.
Plan-and-solve prompting \cite{wang2023planandsolve} enhances long-term focus of the agent by incorporating a planning phase before iterative execution. This helps agents tackle ambiguous or complex tasks by structured strategies \cite{turtayev2024hacking}.
ReAct (reasoning + action) \cite{yao2022react} combines step-by-step reasoning with action, allowing the agent to adjust dynamically through iterative cycles. ReWOO (Reasoning without Observation) \cite{xu2023rewoo} separates the reasoning process from tool outputs and observations, allowing it to handle multi-step reasoning tasks efficiently while maintaining focus.
The prompting methods in these agents involve static hard-coded templates where environment and task information is filled in.
While static prompts provide straightforward guidance, they often fail to adapt to different problems and complex tasks, limiting their effectiveness.
Auto-prompting~\cite{shin-etal-2020-autoprompt, zhou-etal-2023-revisiting, zhang2023automatic} is a technique to allow the LLM itself to generate a highly-relevant prompt. Auto-prompting invokes more factual responses and reduces hallucinations in LLMs.
D-CIPHER incorporates auto-prompting as a separate agent that can explore the environment and generate a better prompt.
%Based on the given prompt, LLM agents make a decision and proceed further to find flags.  To address this gap, we propose \textbf{dynamic prompting}, where the LLM agent autonomously generates prompts based on the CTF challenge's context and stage.
%include a static template which needs to be given to LLM to solve the CTF challenges. For instance, the NYU CTF framework provides a static prompt as \emph{``Please proceed to the next step using your best judgment"} for each decision making point. 

% To address this gap, we introduce a novel approach where the LLM agent generates the next prompt autonomously based on the current context and stage of the CTF challenge, a technique we call \textbf{dynamic prompting}.


Expanding on single LLM agents, multi-agent LLM systems are a powerful approach to enhance problem-solving by simulating team-based collaboration. Specialized agents, each with distinct objectives, work together to tackle different aspects of complex tasks \cite{guo2024largelanguagemodelbased}
Multi-agent systems are effective in cybersecurity applications. For instance, Audit-LLM~\cite{song2024audit} deploys a  multi-agent system for insider threat detection by employing agents to decompose tasks, build tools, and use collaborative reasoning to enhance detection accuracy. Liu~\cite{liu2024multi} explores multi-agent systems to enhance incident response in cybersecurity by examining centralized, decentralized, and hybrid team structures to assess how LLM agents can improve decision-making, adaptability, and coordination during cyber-attack scenarios. AutoSafeCoder~\cite{nunez2024autosafecoder} enhances the security of code generated by LLMs by incorporating a coding agent for code generation, a static analyzer agent that identifies vulnerabilities, and a fuzz testing agent for dynamic testing to detect runtime errors. Division of responsibilities among different agents allows AutoSafeCoder to produce secure, functionally correct code. 

% With the growing use of LLMs in CTF challenges, prompt engineering is key to enhancing performance. Various methods guide LLMs in solving complex cybersecurity tasks, each with distinct strengths and limitations.

% \textbf{Single Turn (Zero-Shot Prompting)} involves providing the model with a one-time task description that outputs  an immediate solution. This is efficient for straightforward tasks~\cite{yang2023intercode}. In contrast, \textbf{Try Again (Iterative Feedback Prompting)} uses iterative feedback to refine responses over multiple attempts, mimicking real-world problem-solving~\cite{yang2023intercode}. The \textbf{Plan \& Solve} enhances adaptability by incorporating a planning phase before iterative execution. This helps models tackle ambiguous or complex tasks by  structured strategies~\cite{turtayev2024hacking}. Additionally, \textbf{ReAct (Reasoning + Action)} combines step-by-step reasoning with action, allowing the model to adjust dynamically through iterative cycles. This makes it particularly effective for evolving and complex challenges like CTFs~\cite{yao2023react}. 
% These prompting techniques highlight diverse approaches to optimizing LLM performance in cybersecurity tasks. 

% Multi-agents!


%\meet{Add references for auto-prompting, and shorten this para}
%\nanda{Maybe we can add this to previous paragraphs which discusses other prompting methods such as plan-and-solve and ReAct method}
% All of these prompting methods include a static template which needs to be given to LLM to solve the CTF challenges. For instance, the NYU CTF framework provides a static prompt as \emph{``Please proceed to the next step using your best judgment"} for each decision making point. 
% Based on the given prompt, LLM agents make a decision and proceed further to find flags. While static prompts provide straightforward guidance, they often fail to account for the evolving nature of complex tasks, limiting their effectiveness in multi-step or ambiguous CTF challenges. To address this gap, we propose \textbf{dynamic prompting}, where the LLM agent autonomously generates prompts based on the CTF challenge's context and stage.
% % To address this gap, we introduce a novel approach where the LLM agent generates the next prompt autonomously based on the current context and stage of the CTF challenge, a technique we call \textbf{dynamic prompting}.
% Dynamic prompting adapts instructions to task progress, ensuring instructions are contextually relevant and reflective of the specific obstacles encountered. By iterating based on feedback and intermediate outputs, it continuously refines the LLM’s approach, enhancing problem-solving for dynamic tasks like CTFs.
% This adaptive process not only mirrors how humans tackle complex problems but also improves the model’s ability to handle unpredictable scenarios, making it particularly advantageous for cybersecurity tasks like CTFs where conditions change dynamically.


% The very first prompt type used in several applications is \textbf{Single Turn (Zero-Shot Prompting)}~\cite{yang2023intercode}. In single-turn prompting, the model receives a one-time, straightforward task description and is expected to generate a complete response without further interaction. The initial output is directly assessed, making this approach efficient for tasks where minimal feedback or iteration is required. This method tests the model’s ability to understand and respond to tasks immediately, relying heavily on the model's pre-trained knowledge and generalization capabilities.

% Along with this, The prompting method named \textbf{Try Again (Iterative Feedback Prompting)}~\cite{yang2023intercode} has been also used in several appreciations specially to solve CTF challenges. It is an iterative prompting method involves continuous interaction, where the model is provided with feedback after each attempt. The model can refine its responses over multiple turns based on the observations or execution results from previous outputs. This iterative process continues until the task is successfully completed or a maximum number of interactions is reached. This approach closely mirrors real-world problem-solving, where adjustments are made iteratively based on evolving circumstances or feedback.

% Some application are also using \textbf{Plan \& Solve}~\cite{turtayev2024hacking} prompting method which enhances problem-solving by dividing the process into a planning phase followed by execution. Initially, the model formulates a strategy based on the task description and available information, allowing for a structured approach to ambiguous or complex problems. This plan guides the subsequent execution phase, where the model carries out actions iteratively, refining its approach based on feedback. In more challenging scenarios, re-planning mid-task further improves adaptability and performance. This method proves effective in tasks like CTF challenges, where vague instructions require careful analysis and step-by-step resolution.

% Further some application are also using \textbf{ReAct (Reasoning + Action)}~\cite{yao2023react} prompting method blends reasoning with action by guiding the model to think through tasks step-by-step before executing actions. At each step, the model generates a thought based on the task and observations, which informs the next action. The action is executed, and the resulting feedback refines the model’s understanding for the next cycle. This continuous process helps the model adapt dynamically to complex tasks, making it effective for CTF challenges where logical reasoning and step-by-step execution are essential.

\section{Related Works} \label{sec:related_work}


\begin{table}[htpb]
    \centering
    \caption{Feature comparison of LLM agents for solving CTFs.}
    \label{tab:related_work_comparison}
    \begin{tabular}{lcccccc}
    \toprule
         \textbf{Study} & \rotatebox{90}{\textbf{\# CTFs}} & \rotatebox{90}{\textbf{Open bench}} & \rotatebox{90}{\textbf{Tool use}}  & \rotatebox{90}{\textbf{Autonomous}} & \rotatebox{90}{\textbf{Multi-agent}} &\rotatebox{90}{\textbf{Auto-prompt}} \\
    \cmidrule{2-7}
     % \textbf{Study} & \textbf{Dynamic} & \textbf{Used} & \textbf{Multi-} & \textbf{Automatic} & \textbf{Tool} & \textbf{\# of} \\
         Tann et al. \cite{tann2023using} &  $7$ & \purplecross & \purplecross & \purplecross & \purplecross & \purplecross  \\
         Shao et al. \cite{shao2024empirical} & $26$ & \purplecross & \tealcheck & \tealcheck & \purplecross & \purplecross  \\
         InterCode-CTF\cite{yang2023language} & $100$ & \tealcheck & \tealcheck & \tealcheck & \purplecross & \purplecross   \\
         NYU CTF Bench \cite{shao2024nyu} & $200$ & \tealcheck & \tealcheck & \tealcheck & \purplecross & \purplecross \\
         Turtayev et al. \cite{turtayev2024hacking} & $100$ & \tealcheck & \tealcheck & \tealcheck & \purplecross & \purplecross\\
         Cybench \cite{zhang2024cybenchframeworkevaluatingcybersecurity} & $40$ & \tealcheck & \tealcheck & \tealcheck & \purplecross & \purplecross \\
         EnIGMA \cite{abramovich2024enigma} & $350$ & \tealcheck & \tealcheck & \tealcheck & \purplecross & \purplecross\\
         HackSynth \cite{muzsai2024hacksynth} & $200$ & \tealcheck & \tealcheck & \tealcheck & \tealcheck & \purplecross \\
         \textbf{D-CIPHER (ours)} & $290$ & \tealcheck & \tealcheck & \tealcheck & \tealcheck & \tealcheck \\
    \bottomrule
    \end{tabular}
\end{table}



% \subsection{LLMs on Cybersecurity}
% \subsection{LLM Agents for CTF}

%LLMs have a vast knowledge base that can be tapped for cybersecurity use.
Tann et al.~\cite{tann2023using} evaluate early LLMs such as ChatGPT and Google Bard in solving CTF challenges and answering professional certification questions, showing that LLM responses contain key task information.
%Many works extend the LLM capabilities by providing them access to programming and command execution tools, to form autonomous agents. 
The InterCode-CTF agent~\cite{yang2023intercode} reveals that LLM agents demonstrate basic cybersecurity skills, however they struggle with more complex tasks.
The NYU CTF baseline agent~\cite{shao2024empirical} integrates external tools into the LLM's function-calling features and demonstrate improved potential of tool-assisted LLMs to solve CTFs, however it exhausts the LLM context length when command output history becomes very long. InterCode-CTF manages this issue by truncating the history to only show the LLM the last few iterations. Even so, LLM agents face issues with longer tasks.
%NYU CTF Bench~\cite{shao2024nyu}, a benchmark of 200 CTF challenges, presents a baseline agent with specialized reverse engineering tools and category-specific prompts, demonstrating their importance to solve CTFs.
% The NYU CTF baseline agent faces issues of LLM context length when complex tasks run for several iterations and the entire command and output history becomes longer than the LLM's context window size. The InterCode agent manages this issue by truncating the history to only show the LLM the last few iterations.


Excessive tool availability and verbose interfaces can overwhelm agents, leading to inefficiencies. Agents perform better with a focused set of tools with well-defined interfaces~\cite{yang2024sweagent}.
EnIGMA~\cite{abramovich2024enigma} agent incorporates interactive tools and in-context learning techniques to achieve state-of-the-art results. % on the NYU CTF Bench, HackTheBox, and Cybench benchmarks.
For better context management, EnIGMA also uses an LLM summarizer that summarizes the command outputs for the main agent.

HackSynth~\cite{muzsai2024hacksynth}, an LLM agent for autonomous penetration testing, shows that iterative planning and feedback summarization stages help the agent finish multiple tasks and improves overall problem solving.
Similarly, Cybench~\cite{zhang2024cybenchframeworkevaluatingcybersecurity} introduces a benchmark of 40 CTF challenges augmented with step-by-step tasks, demonstrating better focus of LLM agents on smaller tasks, leading to improved success and alleviating the context length issue.
\citet{turtayev2024hacking} expand on InterCode-CTF by implementing plan-and-solve prompting, achieve significant improvement on the InterCode-CTF benchmark. They show that prompting techniques can improve performance even with simple toolsets.
% . Their baseline agent is evaluated in unguided mode (i.e. fully autonomous), and guided mode where the agent is given one task at a time. Their results indicate that providing smaller tasks to the LLM agents improve their focus yielding improved success on complex challenges while .

These works highlight that LLM agents excel at implementing code and executing commands to accomplish small concrete tasks when provided with dynamic feedback and task-specific toolsets. While these works  involved using multiple LLMs with different tasks such as planning and summarizing along-side a main agent, D-CIPHER is the first work to formulate a multi-agent system where there is a bifurcation of responsibilities between agents and meaningful well-defined interactions for dynamic feedback.
Table~\ref{tab:related_work_comparison} shows a feature comparison of D-CIPHER with related works on LLM agents for autonomous CTF solving.
%\meet{some description of the feature comparison?}
% Recent research has focused on enable autonomous solving of CTF challenges~\cite{shao2024empirical,shao2024nyu,abramovich2024enigma}. These agents typically operate in containerized environments to ensure reproducibility and modularity. 

% As an early effort, Tann et al.~\cite{tann2023using} evaluated the effectiveness of LLMs, such as OpenAI's ChatGPT, Google Bard, and Microsoft Bing, in solving cybersecurity CTF challenges and answering professional certification questions. 
% % Their study results show that LLMs performed well on $7$ CTF test cases, with ChatGPT solving $6$, Bard $2$, and Bing $1$. 
% The study shows that LLM responses often contain key information essential for solving tasks.

% The InterCode framework~\cite{yang2023intercode} approaches coding as an interactive process and uses execution feedback to improve code generation. As described in Yang et al.~\cite{yang2023intercode}, InterCode-CTF integrates CTF benchmarks into a reinforcement learning environment that can evaluate the cybersecurity capabilities of language agents. It features $100$ tasks that tapskills such as reverse engineering, forensics, and binary exploitation. While existing language agents demonstrate basic cybersecurity skills, evaluations indicate they struggle with more complicated complex tasks unless the system is fine-tuned or given external support. 
% cite Intercode: Standardizing and benchmarking interactive coding with execution feedback

% Another notable example is an LM agent developed by Shao et al. specifically to automate CTF tasks. 
% Shao et al.~\cite{shao2024empirical} developed a LM agent to automate CTF tasks.
% % They report an accuracy rate of  $46\%$ on $26$ CTF challenges sourced from CSAW'23 Qualifying round competition using GPT-4.
% By effectively combining LLM capabilities with external tools, the researchers demonstrated the potential of tool-assisted LLMs to solve complex problems. Building on this, the team incorporated a broader range of cybersecurity tools and interfaces that enhance both accuracy and versatility. 
% Empirical results show their system outperforms baselines on both the InterCode CTF benchmark and the NYU CTF benchmark.

% Shao et al.~\cite{shao2024nyu} presented a diverse, open-source database of CTF challenges that can be used to benchmark an LLM's ability to solve cybersecurity problems.
% It provides a scalable platform for developing and testing AI-driven approaches for vulnerability detection and resolution, facilitating advancements in automated cybersecurity tasks. The benchmark database and automated framework were successfully applied to the performance of five LLMs. 

% The Cybench benchmark~\cite{zhang2024cybenchframeworkevaluatingcybersecurity} provides another significant contribution by creating a framework tailored to solving CTF challenges. % Cybench: A framework for evaluating cybersecurity capabilities and risk
% % Their benchmark environment achieves an accuracy of $17.5\%$ using Claude 3.5 Sonnet. 
% Such frameworks operate in Linux-based containerized environments, such as Kali Linux, which includes pre-installed cybersecurity tools. However, excessive tool availability can overwhelm agents, leading to inefficiencies. Research indicates that agents perform better with a focused set of tools that have well-defined interfaces~\cite{yang2024sweagent}. % Swe-agent: Agent-computer interfaces enable automated software engineering



% Muzsai et al. introduced HackSynth~\cite{muzsai2024hacksynth}, an LLM-based agent for autonomous penetration testing. It uses a dual-module architecture that consists of a Planner and a Summarizer, allowing for iterative command generation and feedback processing. The framework is evaluated using two benchmark sets from platforms like PicoCTF~\cite{picoctf} and OverTheWire~\cite{overthewire}. These benchmarks address $200$ challenges drawn from various domains and difficulty levels. Results of their study show that HackSynth, especially with the GPT-4o model, achieves the best performance. This highlights the potential of LLM-based agents in advancing autonomous penetration testing.
 % Using basic prompting techniques and expanding tool availability, the study highlights how straightforward approaches can unlock the latent potential of LLMs for cybersecurity tasks. Their work emphasizes that simple LLM designs can effectively solve CTF challenges, and thus broaden the number of cybersecurity applications without the need for advanced engineering.

% \begin{table*}[]
%     \centering
%     \begin{tabular}{|c|c|>{\centering\arraybackslash}p{4.5cm}|c|c|c|c|c|c|}
%     \hline
%          \textbf{Study} & \textbf{Dynamic} & \textbf{Used} & \textbf{Multi-} & \textbf{Open} & \textbf{Automatic} & \textbf{Tool} & \textbf{\# of} & \textbf{\# of} \\
%          & \textbf{Prompt} & \textbf{Benchmarks} & \textbf{Agents} & \textbf{Dataset} & \textbf{Framework} & \textbf{Use} & \textbf{LLMs} & \textbf{CTFs}\\
%          \hline
%          Tann et al.~\cite{tann2023using} & \purplecross & Manual collected & \purplecross & \purplecross & \purplecross & \purplecross & $3$ & $7$ \\
%          \hline
%          InterCode-CTF~\cite{yang2023language} & \purplecross &  PicoCTF~\cite{picoctf} & \purplecross & \purplecross& \purplecross & \purplecross & $1$ & $100$  \\
%          \hline
%          Shao et al.~\cite{shao2024empirical} & \purplecross & CSAW 2023 & \purplecross & \purplecross & \tealcheck & \tealcheck & $4$ & $26$ \\
%          \hline
%          Shao et al.~\cite{shao2024nyu} & \purplecross & NYU CTF~\cite{shao2024nyu} & \purplecross & \tealcheck & \tealcheck & \tealcheck & $5$ & $200$ \\
%          \hline
%          Cybench~\cite{zhang2024cybenchframeworkevaluatingcybersecurity} & \purplecross & Cybench~\cite{zhang2024cybenchframeworkevaluatingcybersecurity}  & \purplecross & \tealcheck & \tealcheck & & $8$ & $40$ \\
%          \hline
%          EnIGMA~\cite{abramovich2024enigma} & \purplecross & NYU CTF~\cite{shao2024nyu}, InterCode-CTF~\cite{yang2023language},  HackTheBox~\cite{hackthebox} & \purplecross & \purplecross & \tealcheck & \tealcheck & $3$ & $350$ \\
%          \hline
%          HackSynth~\cite{muzsai2024hacksynth} & \purplecross & PicoCTF~\cite{picoctf}, OverTheWire~\cite{overthewire} & \tealcheck & \tealcheck & \tealcheck & \tealcheck & $8$ & $200$ \\
%          \hline
%          Turtayev et al.~\cite{turtayev2024hacking} & \purplecross & InterCode-CTF~\cite{yang2023language} & \purplecross & \purplecross & \purplecross & \purplecross & $4$ & $100$ \\
%          \hline
%          \textbf{D-CIPHER (Proposed)} & \tealcheck & NYU CTF~\cite{shao2024nyu}, Cybench \cite{zhang2024cybenchframeworkevaluatingcybersecurity}, HackTheBox \cite{hackthebox} & \tealcheck & \tealcheck & \tealcheck & \tealcheck & 5 & 290 \\
%          \hline
%     \end{tabular}
%     \caption{Comparison with LLM-based CTF solving Literature}
%     \label{tab:related_work_comparison}
% \end{table*}




% \subsection{Multi-agent framework}

% The use of multi-agent LLM systems in Capture the Flag (CTF) challenges is emerging as a powerful approach to enhance cybersecurity problem-solving. Multi-agent frameworks mimic team-based collaboration, where multiple LLM agents, each with specialized expertise, work together to tackle complex tasks. This approach reflects real-world cybersecurity operations, where success often depends on coordinated efforts from teams with diverse skills, each addressing different components of a security challenge.
% Multi-agent LLM systems are emerging as a powerful approach to enhance cybersecurity problem-solving by simulating team-based collaboration. Specialized agents, each with distinct objectives, work together to tackle different aspects of complex security tasks. This mirrors real-world cybersecurity operations, where coordinated efforts and diverse skills are essential for addressing evolving threats and vulnerabilities.

% CTF challenges cover a wide range of domains, including cryptography, reverse engineering, forensics, and web exploitation. Multi-agent systems can distribute the workload by assigning agents to handle specific tasks. This enables parallel problem-solving and emulates the collaborative nature of human teams. For example, one agent may specialize in guiding the fellow agents to what needs to be done, while another executes the instructions, ensuring that tasks are addressed without losing the context, and implementing reasoning from multiple LLMs. This division of labor boosts efficiency and enables problem-solving from multiple perspectives.
% This division of labor enhances efficiency and allows the system to approach problems from multiple perspectives, reflecting the interdisciplinary approach often used in cybersecurity teams.

% Guo et al.~\cite{guo2024largelanguagemodelbased} highlight the strengths of multi-agent LLMs in complex, multi-step tasks where different agents handle specific roles The framework HackSynth~\cite{muzsai2024hacksynth} is a multi-agent penetration testing framework in which agents operate collaboratively to address vulnerabilities in staged environments. Their work emphasizes that when agents work as a cohesive team, they outperform single-agent approaches. This is particularly true when facing layered, iterative challenges. 
% This team-based model of problem-solving aligns closely with how cybersecurity professionals approach real-world security incidents and penetration testing exercises.

% Multi-agent LLM systems have shown effectiveness in various other applications. For instance,  Audit-LLM~\cite{song2024audit} presents a multi-agent framework for insider threat detection using log analysis. It employs agents to decompose tasks, build tools, and use collaborative reasoning to enhance detection accuracy. Liu~\cite{liu2024multi} explores the application of LLM-based multi-agent systems to enhance incident response (IR) in cybersecurity. Utilizing the ``Backdoors \& Breaches" tabletop game as a simulation environment, the study examines centralized, decentralized, and hybrid team structures to assess how LLM agents can improve decision-making, adaptability, and coordination during cyberattack scenarios. AutoSafeCoder~\cite{nunez2024autosafecoder} is a multi-agent system designed to enhance the security of code generated by LLMs. The framework comprises three agents: a Coding Agent responsible for code generation, a Static Analyzer Agent that identifies vulnerabilities through static analysis, and a Fuzzing Agent that performs dynamic testing using mutation-based fuzzing to detect runtime errors. By integrating both static and dynamic testing in an iterative process, AutoSafeCoder aims to produce secure, functionally correct code. 

% To enhance CTF-solving by promoting team-based specialization, we employ a multi-agent CTF solving agent. Within this framework, agents tackle tasks aligned with their strengths. Tasks are executed in parallel, improving efficiency and accelerating progress. Agents share insights, adapt refining strategies based on feedback, and overcome obstacles collectively. This collaborative approach boosts scalability, adaptability, and and resilience, and improves performance in complex challenges.

% This paper presents a comprehensive comparison of D-CIPHER with existing LLM-based CTF-solving literature, as shown in Table~\ref{tab:related_work_comparison}.
% This paper documents the results of  our comprehensive comparison of D-CIPHER with existing LLM-based CTF-solving literature. These results are presented in Table~\ref{tab:related_work_comparison}.
\vspace{-5pt}
\section{Method}
\label{sec:method}
\begin{figure*}[t]
\begin{center}
\includegraphics[width=.85\linewidth]{fig_overview_v3.pdf}
\end{center}
\caption{
FastAtlas Overview: In each frame, we compute charts spanning fully or partially visible triangles (a), determine texture space bounding boxes for the visible portions of the view-space projections of each chart, and tightly pack these boxes into atlases (b, here $2K \times 2K$). We simultaneously bijectively parameterize and shade the charts into their atlas boxes, obtaining high quality texture space shading (c), and use this shading to render the shaded frames (d).}
\label{fig:overview}
\label{fig:alg_overview}
\end{figure*}

\section{Overview}
\label{sec:overview}
Our work has two core contributions: a real-time, GPU-based algorithm for tight packing of general parameterized charts into compact atlases; and a real-time TSS method that
utilizes this packing.  

\paragraph*{FastAtlas Packing.}
FastAtlas runs entirely on the GPU as a series of compute shaders. It takes the bounding boxes of parameterized charts as input, and packs them into an atlas (Fig~\ref{fig:overview}b, Sec.~\ref{sec:pack}). As such, the only input it requires are the dimensions of the bounding boxes.
Its outputs are deterministic; identical input charts are packed into identical atlases. This is critical for TSS and similar applications, as it ensures that consecutive frames taken from the same camera view have the same shading. Even minute shading differences across such frames can cause sampling jitter, leading to undesirable flicker \cite{baker2012rock}. 
While prior methods such as \cite{mueller2018shading,hladky2019tessellated,hladky2021snakebinning,Neff2022MSA} cap the dimensions of the charts that can be packed as-is for a given atlas size, and scale down all charts that exceed these dimensions, we scale all charts by the same factor, and do so only when strictly necessary to achieve packing success (Figs~\ref{fig:atlas},~\ref{fig:sas_issues}). 

\paragraph*{TSS using FastAtlas.}
Our end-to-end TSS atlas generation method combines the packing method above with a novel approach for computing seamless per-frame charts. 
We define our charts as the connected components of the visible surfaces in each frame (Fig.~\ref{fig:overview}a), and efficiently compute them using a parallel union-find algorithm (Sec.~\ref{sec:visible}). Since the boundaries of these charts coincide with the contours of the rendered surface, they are {\em invisible} to the viewer. This approach 
eliminates the artifacts caused by shading discontinuities along visible seams (Fig.~\ref{fig:seams}). 

\begin{parWithWrapFigure}
\begin{wrapfigure}{l}{.27\columnwidth}%
\includegraphics[width=\linewidth]{fig_inset_view_plane.pdf}%
\end{wrapfigure}
We bijectively parametrize the {\em visible portions} of our charts by projecting them to view space (inset). This maps a constant number of texels to each pixel in the final rendered output, evenly distributing residual undersampling error across all image pixels. While conceptually straightforward, efficiently parameterizing charts containing partially visible triangles using viewspace projection is non-trivial, as the visible portions may no longer be triangular (e.g. green triangle in the inset); applying naive projection to triangles with vertices behind the camera may produce ill-posed results. Clipping triangles before projection is both computationally expensive and significantly complicates downstream operations. We avoid explicit clipping by observing that all that is required for atlas packing is the dimensions of, potentially conservative, bounding boxes of these projected visible portions. We compute such bounding boxes without explicit chart clipping by adapting a conservative screen coverage estimator \shortcite{Blinn:CalculatingScreenCoverage} (Sec.~\ref{sec:box}). We then pack the computed boxes using FastAtlas. 
\end{parWithWrapFigure}

Finally, we shade the visible portion of each chart into its corresponding atlas bounding box (Fig~\ref{fig:overview}c). 
The resulting texture is then used during rasterization as a standard texture map (Fig. ~\ref{fig:overview}d). 
Our framework is compatible with all existing approaches for texture space shading, including forward shading, raytraced illumination, or deferred shading in texture space \cite{baker:2016}. In the examples shown, we use the standard forward shading based rendering pipeline included in the G3D Innovation Engine \cite{G3D17}, a commercial grade renderer.


Our goal is to increase the robustness of T2I models, particularly with rare or unseen concepts, which they struggle to generate. To do so, we investigate a retrieval-augmented generation approach, through which we dynamically select images that can provide the model with missing visual cues. Importantly, we focus on models that were not trained for RAG, and show that existing image conditioning tools can be leveraged to support RAG post-hoc.
As depicted in \cref{fig:overview}, given a text prompt and a T2I generative model, we start by generating an image with the given prompt. Then, we query a VLM with the image, and ask it to decide if the image matches the prompt. If it does not, we aim to retrieve images representing the concepts that are missing from the image, and provide them as additional context to the model to guide it toward better alignment with the prompt.
In the following sections, we describe our method by answering key questions:
(1) How do we know which images to retrieve? 
(2) How can we retrieve the required images? 
and (3) How can we use the retrieved images for unknown concept generation?
By answering these questions, we achieve our goal of generating new concepts that the model struggles to generate on its own.

\vspace{-3pt}
\subsection{Which images to retrieve?}
The amount of images we can pass to a model is limited, hence we need to decide which images to pass as references to guide the generation of a base model. As T2I models are already capable of generating many concepts successfully, an efficient strategy would be passing only concepts they struggle to generate as references, and not all the concepts in a prompt.
To find the challenging concepts,
we utilize a VLM and apply a step-by-step method, as depicted in the bottom part of \cref{fig:overview}. First, we generate an initial image with a T2I model. Then, we provide the VLM with the initial prompt and image, and ask it if they match. If not, we ask the VLM to identify missing concepts and
focus on content and style, since these are easy to convey through visual cues.
As demonstrated in \cref{tab:ablations}, empirical experiments show that image retrieval from detailed image captions yields better results than retrieval from brief, generic concept descriptions.
Therefore, after identifying the missing concepts, we ask the VLM to suggest detailed image captions for images that describe each of the concepts. 

\vspace{-4pt}
\subsubsection{Error Handling}
\label{subsec:err_hand}

The VLM may sometimes fail to identify the missing concepts in an image, and will respond that it is ``unable to respond''. In these rare cases, we allow up to 3 query repetitions, while increasing the query temperature in each repetition. Increasing the temperature allows for more diverse responses by encouraging the model to sample less probable words.
In most cases, using our suggested step-by-step method yields better results than retrieving images directly from the given prompt (see 
\cref{subsec:ablations}).
However, if the VLM still fails to identify the missing concepts after multiple attempts, we fall back to retrieving images directly from the prompt, as it usually means the VLM does not know what is the meaning of the prompt.

The used prompts can be found in \cref{app:prompts}.
Next, we turn to retrieve images based on the acquired image captions.

\vspace{-3pt}
\subsection{How to retrieve the required images?}

Given $n$ image captions, our goal is to retrieve the images that are most similar to these captions from a dataset. 
To retrieve images matching a given image caption, we compare the caption to all the images in the dataset using a text-image similarity metric and retrieve the top $k$ most similar images.
Text-to-image retrieval is an active research field~\cite{radford2021learning, zhai2023sigmoid, ray2024cola, vendrowinquire}, where no single method is perfect.
Retrieval is especially hard when the dataset does not contain an exact match to the query \cite{biswas2024efficient} or when the task is fine-grained retrieval, that depends on subtle details~\cite{wei2022fine}.
Hence, a common retrieval workflow is to first retrieve image candidates using pre-computed embeddings, and then re-rank the retrieved candidates using a different, often more expensive but accurate, method \cite{vendrowinquire}.
Following this workflow, we experimented with cosine similarity over different embeddings, and with multiple re-ranking methods of reference candidates.
Although re-ranking sometimes yields better results compared to simply using cosine similarity between CLIP~\cite{radford2021learning} embeddings, the difference was not significant in most of our experiments. Therefore, for simplicity, we use cosine similarity between CLIP embeddings as our similarity metric (see \cref{tab:sim_metrics}, \cref{subsec:ablations} for more details about our experiments with different similarity metrics).

\vspace{-3pt}
\subsection{How to use the retrieved images?}
Putting it all together, after retrieving relevant images, all that is left to do is to use them as context so they are beneficial for the model.
We experimented with two types of models; models that are trained to receive images as input in addition to text and have ICL capabilities (e.g., OmniGen~\cite{xiao2024omnigen}), and T2I models augmented with an image encoder in post-training (e.g., SDXL~\cite{podellsdxl} with IP-adapter~\cite{ye2023ip}).
As the first model type has ICL capabilities, we can supply the retrieved images as examples that it can learn from, by adjusting the original prompt.
Although the second model type lacks true ICL capabilities, it offers image-based control functionalities, which we can leverage for applying RAG over it with our method.
Hence, for both model types, we augment the input prompt to contain a reference of the retrieved images as examples.
Formally, given a prompt $p$, $n$ concepts, and $k$ compatible images for each concept, we use the following template to create a new prompt:
``According to these examples of 
$\mathord{<}c_1\mathord{>:<}img_{1,1}\mathord{>}, ... , \mathord{<}img_{1,k}\mathord{>}, ... , \mathord{<}c_n\mathord{>:<}img_{n,1}\mathord{>}, ... , $
$\mathord{<}img_{n,k}\mathord{>}$,
generate $\mathord{<}p\mathord{>}$'', 
where $c_i$ for $i\in{[1,n]}$ is a compatible image caption of the image $\mathord{<}img_{i,j}\mathord{>},  j\in{[1,k]}$. 

This prompt allows models to learn missing concepts from the images, guiding them to generate the required result. 

\textbf{Personalized Generation}: 
For models that support multiple input images, we can apply our method for personalized generation as well, to generate rare concept combinations with personal concepts. In this case, we use one image for personal content, and 1+ other reference images for missing concepts. For example, given an image of a specific cat, we can generate diverse images of it, ranging from a mug featuring the cat to a lego of it or atypical situations like the cat writing code or teaching a classroom of dogs (\cref{fig:personalization}).
\vspace{-2pt}
\begin{figure}[htp]
  \centering
   \includegraphics[width=\linewidth]{Assets/personalization.pdf}
   \caption{\textbf{Personalized generation example.}
   \emph{ImageRAG} can work in parallel with personalization methods and enhance their capabilities. For example, although OmniGen can generate images of a subject based on an image, it struggles to generate some concepts. Using references retrieved by our method, it can generate the required result.
}
   \label{fig:personalization}\vspace{-10pt}
\end{figure}
\section{Related Work}
\paragraph{Fast Diffusion Sampling} Fast sampling of diffusion models can be categorized into training-free and training-based methods. The former typically seek implicit sampling processes~\citep{song2020denoising,zheng2024masked,zheng2024diffusion} or dedicated numerical solvers to the differential equations corresponding to diffusion generation, including Heun's methods~\citep{karras2022elucidating}, splitting numerical methods~\citep{wizadwongsa2023accelerating}, pseudo numerical methods~\citep{liu2021pseudo} and exponential integrators~\citep{zhang2022fast,lu2022dpm,zheng2023dpm,gonzalez2023seeds}. They typically require around 10 steps for high-quality generation. In contrast, training-based methods, particularly adversarial distillation~\citep{sauer2023adversarial} and consistency distillation~\citep{song2023consistency,kim2023consistency}, have gained prominence for their ability to achieve high-quality generation with just one or two steps. While adversarial distillation proves its effectiveness in one-step generation of text-to-image diffusion models~\citep{sauer2024fast}, it is theoretically less transparent than consistency distillation due to its reliance on adversarial training. Diffusion models can also be accelerated using quantized or sparse attention~\citep{zhang2025sageattention,zhang2024sageattention2,zhang2025spargeattn}.
\paragraph{Parameterization in Diffusion Models} 
Parameterization is vital in efficient training and sampling of diffusion models. Initially, the practice involved parameterizing a noise prediction network \citep{ho2020denoising,song2020score}, which outperformed direct data prediction. A notable subsequent enhancement is the introduction of "v" prediction \citep{salimans2022progressive}, which predicts the velocity along the diffusion trajectory, and is proven effective in applications like text-to-image generation \citep{esser2024scaling} and density estimation \citep{zheng2023improved}. EDM~\citep{karras2022elucidating} further advances the field by proposing a preconditioning technique that expresses the denoiser function as a linear combination of data and network, yielding state-of-the-art sample quality alongside other techniques. However, the parameterization in consistency distillation remains unexplored.

\section{Experiments}

\subsection{Datasets}

\textbf{MSMARCO}.
We utilized the MS MARCO Passage Ranking dataset as the data source to evaluate the ability of our method to improve document rankings under more challenging topic-query tasks. Specifically, we assessed whether our method could significantly enhance the ranking of documents by the retrieval model within a RAG system.

To construct topic-lists for evaluation, we applied a K-means clustering algorithm to group similar queries, forming topics that each contained a series of related queries. To further evaluate the performance of our method under extreme topic-query scenarios, we applied an intra-topic similarity filtering process. Only topics with queries exhibiting high semantic diversity and containing a sufficient number of queries were retained.

This process resulted in 29 topics, with each topic containing an average of 22.28 queries. The average similarity score within each topic was approximately 0.5, indicating sufficient diversity among queries to ensure a rigorous evaluation. This curated dataset enabled us to test the robustness of our method in handling highly diverse and challenging topic-query tasks within a RAG system.

\textbf{PROCON}.
To conduct our experiments, we utilized controversial topic data scraped from the PROCON.ORG website. This dataset includes over 80 topics covering various domains, such as society, health, government, and education. Each topic is discussed from two stance labels \{\textit{PRO (support), CON (oppose)}\}, with passages arguing from these perspectives.

To simulate real-world user interactions with a RAG system, we instructed a large language model (GPT-4o) to act as a user and generate 40 potential sub-queries for each topic. These sub-queries were designed to reflect the diverse questions and concerns users might raise when exploring a specific controversial topic. 

After generating the sub-queries, we applied a similarity filtering process to ensure diversity by retaining only those with a similarity score below approximately 0.85. The filtering step effectively removed redundant queries while preserving a wide range of perspectives. As a result, the final set of topic-queries achieved an average similarity score of approximately 0.7, ensuring that the queries were sufficiently diverse yet semantically relevant. This process provided a robust and balanced sub-queries set for evaluation.


\subsection{Experiment Details}
The specific setting details for the Topic-queries RAG manipulation experiment are as follows:

(1) Black-box RAG. We represent the black-box RAG process as \( \text{RAG}_{\text{black}} \). The RAG framework is Conversational RAG from LangChain. The LLMs adopted in RAG are the open-source models Meta-Llama-3.1-8B-Instruct (Llama3.1), Qwen-2.5-7B-Instruct(Qwen2.5). The system prompt and additional detailed descriptions are provided in Appendix~\ref{exp-detail}.

(2) Retrieval Model Specification. We benchmark three dominant dense retrievers—Contriever \cite{gao2021unsupervised}, DPR \cite{karpukhin-etal-2020-dense}, and ANCE \cite{xiong2020approximate}.By convention, we use dot product between the embedding vectors of questions(queries) and candidate documents as their similarity score \(R\) in the ranking. 


\label{opinion-classfication}
(3) Opinion classification. We use Qwen2.5-Instruct-72B as the opinion classifier. Qwen2.5-Instruct-72B, due to its large parameter size, is capable of accurately identifying and classifying opinions within text.

(4) Experimental parameters. In knowledge-guided attack process, we set the maximum editing distance $\epsilon$ to 0.2, the semantic similarity threshold $\lambda$ to 0.85, and the number of iterations $N$ to 5. For adversarial trigger generation, we used a beam size of 3, top-$k$ of 10, a batch size of 32, a temperature of 1.0, a learning rate of 0.005, and a sequence length of 10. In RAG\textsubscript{black}, $k$ (the number of retrieved documents) is set to 3, with the LLM temperature also fixed at 1.0 to mirror real-world conditions.

(5) Poisoned Target. For the PROCON dataset, to investigate the manipulation performance under more challenging conditions, we performed relevance ranking for the documents with respect to each topic-query set $Q$ and the target stance $S_t$ . From the ranked list, we selected the last five documents (i.e., those with the lowest relevance) as the target poisoned documents.
For the MS MARCO dataset, we utilized the top-1000 relevance-ranked passage list for each topic-query set. From this list, we selected the passage with the lowest average rank as the target passage. This approach ensures that the evaluation focuses on passages that are least relevant to the target queries, thus providing a more rigorous benchmark for the proposed method.

(6) Experimental environment. All our experiments are conducted in Python 3.8 environment and run on a NVIDIA DGX H100 GPU. 

\subsection{Research Questions}

We propose four research questions to evaluate the effectiveness of our method in the topic-queries task, focusing on black-box NRM attacks and opinion manipulation to RAGs.

\textbf{RQ1}: Can Topic-FlipRAG significantly enhance the rankings of target documents in the RAG retriever for topic-queries?

\textbf{RQ2}: To what extent does Topic-FlipRAG affect the answers generated by the target RAG systems?

\textbf{RQ3}: Does topic-oriented opinion manipulation significantly impact users' perceptions of controversial topics?

\textbf{RQ4}: How robust does Topic-FlipRAG against exisiting mitigation mechanism?

\subsection{Baseline Settings}
To assess the effectiveness of our proposed method, we compare it against adversarial attack baselines designed for black-box, topic-oriented RAG scenarios, ensuring minimal modifications to the original documents. We exclude BadRAG\cite{xue2024badrag}, a backdoor RAG attack limited to white-box scenarios, and topic-IR-attack\cite{liu2023topic}, as its incomplete implementation prevents reliable replication.
For the selected baseline methods, we adapted them to meet the requirements of our task while preserving their core components. A brief overview of the baseline methods is provided below, with detailed descriptions available in Appendix~\ref{baselines-details}.

\textbf{PoisonedRAG.}
Zou et al.\cite{zou2024poisonedrag} propose an approach adaptable to both black-box and white-box settings. For our task, we employ its black-box strategy by inserting a randomly chosen query from the topic-queries set \( Q \) at the beginning of each document.

\textbf{PAT.}
This gradient-based adversarial retrieval attack uses a pairwise loss function to generate triggers that meet fluency and coherence constraints. We adapt PAT to produce triggers \( T_{\text{pat}} \) for target documents within the topic-queries set, evaluating their effectiveness under black-box conditions.


\textbf{Collision.}
This method generates adversarial paragraphs (collisions) via gradient-based optimization to produce content semantically aligned with the target query. In a topic-queries context, we create collisions for the entire topic-queries set and examine their transfer performance on black-box RAG retrievers.

These baseline methods provide benchmarks for comparing the efficacy of our approach in a fully black-box, topic-oriented RAG attack scenario.

\subsection{Evaluation Metrics}

For \textbf{RQ1}, we focus on ranking manipulation. We measure the average proportion of target opinions in top-3 rankings before and after manipulation (\(\text{Top3}_{\text{ori}}, \text{Top3}_{\text{att}}\)) and define top3-v as their difference. We also employ the Ranking Attack Success Rate (RASR), reflecting how often target documents are successfully boosted, and Boost Rank (BRank), denoting the average rank improvement for all target documents. Lastly, we report the proportion of target documents in the Top-50 and Top-500 positions to indicate how effectively they are pushed toward higher rankings.

\textbf{top3-v.} Computed by subtracting \(\text{Top3}_{\text{ori}}\) from \(\text{Top3}_{\text{att}}\), top3-v ranges from -1 to 1. A positive value signifies a successful increase of the target opinion in top-3 results, while a negative value indicates a detrimental effect.

\textbf{Ranking Attack Success Rate (RASR).} RASR captures how frequently target documents are successfully boosted in each query’s ranking. Higher values indicate greater attack effectiveness.

\textbf{Boost Rank (BRank).} BRank is the average rank improvement for all target documents under each query. A target document contributes negatively if its rank is unintentionally lowered.

\textbf{Top-50, Top-500.} These metrics represent the percentage of target documents that move into specific ranking thresholds in the MS MARCO Dataset after manipulation. Higher percentages imply more effective promotion of target documents. 


For \textbf{RQ2}, we employ Average Stance Variation (ASV) to assess how significantly our opinion manipulation influences the LLM’s responses in a black-box RAG. To address the natural variability of controversial topics and the inherent instability of large language models, we also propose Real Adjusted ASV (\(\Delta\)-ASV).

\textbf{Average Stance Variation (ASV).}
ASV is defined as the absolute difference between the manipulated opinion score and the original opinion score assigned to an LLM response (0 = opposing, 1 = neutral, 2 = supporting). A higher ASV signifies a more pronounced shift in polarity and hence greater manipulation effectiveness.

\textbf{Real Adjusted ASV ($\Delta$-ASV)}. To account for the inherent variability of controversial topics and the instability of large language models, we measure the baseline ASV in a clean state, denoted as ASV\textsubscript{clean} (calculated without adversarial manipulation). $\Delta$-ASV is then obtained by subtracting ASV\textsubscript{clean} from the manipulated ASV, i.e., \( \text{$\Delta$-ASV} = \text{ASV} - \text{ASV\textsubscript{clean}} \). This adjustment ensures that $\Delta$-ASV reflects the true impact of adversarial manipulation by eliminating the influence of natural stance variation. It reflects the extent to which the polarity of the RAG-system outputs is affected by the manipulation.  A positive $\Delta$-ASV indicates a significant shift in the opinion polarity due to manipulation, with larger values representing greater manipulation effectiveness.

\section{Conclusion}
We introduced \methodname, an effective training framework defending against MIAs for LLMs. The extensive experiments demonstrate its robustness in protecting privacy while maintaining strong language modeling performance across various datasets and architectures. Although our study focuses on fine-tuning due to computational constraints, \methodname can be seamlessly applied to large-scale pretraining, as done in prior selective pretraining work~\cite{lin2024not}. By categorizing tokens and treating them appropriately, \methodname opens a novel pathway for MIA defense. Future work can explore improved token selection strategies and multi-objective training approaches.

\section*{Acknowledgments}
This work was supported by the National Natural Science Foundation of China (Nos. 62350080, 62106120, 92270001), Tsinghua Institute for Guo Qiang, and the High Performance
Computing Center, Tsinghua University; J. Zhu was also supported by the XPlorer Prize.


\bibliography{iclr2025_conference}
\bibliographystyle{iclr2025_conference}
\newpage
\appendix
\section{Proofs}
\label{appendix:proof}
\subsection{Proof of Proposition~\ref{proposition:gap}}
\label{appendix:proof1}
\begin{proof}
Denote $\{\x_\tau\}_{\tau=s}^t$ as data points on the same teacher ODE trajectory. The generalized ODE in Eqn.~\eqref{eq:ODE-with-l-s} can be reformulated as an integral:
\begin{equation}
    L_s\x_s-L_t\x_t=\int_{\eta_t}^{\eta_s}\hv_\phi(\x_{t_{\lambda_\eta}},t_{\lambda_\eta})\dm\eta
\end{equation}
where $\hv_\phi(\x_t,t)\coloneqq\frac{\gv_\phi(\x_t,t)}{S_t}$, and $\gv_\phi$ is defined by the teacher denoiser $\Dv_\phi$ in Eqn.~\eqref{eq:ODE-with-l}. On the other hand, by replacing the teacher denoiser $\Dv_\phi$ with the student denoiser $\Dv_\theta$ in the Euler discretization (Eqn.~\eqref{eq:generalize-discretization-not-rearranged}), the optimal student $\theta^*$ should satisfy
\begin{equation}
    L_s\x_s-L_t\x_t=(\eta_s-\eta_t)\hv_{\theta^*}(\x_t,t,s)
\end{equation}
where $\hv_\theta,\gv_\theta$ are defined similarly to $\hv_\phi,\gv_\phi$ as
\begin{equation}
    \hv_{\theta}(\x_t,t,s)=\frac{\gv_{\theta}(\x_t,t,s)}{S_t},\quad \gv_{\theta}(\x_t,t,s)=\Dv_{\theta}(\x_{t},t,s)-(1-l_{t})\x_{t}
\end{equation}
Combining the above equations, we have
\begin{equation}
\begin{aligned}
    \Dv_{\theta^*}(\x_{t},t,s)-\Dv_\phi(\x_t,t)&=S_t(\hv_{\theta^*}(\x_t,t,s)-\hv_\phi(\x_t,t))\\
    &=S_t\left(\frac{L_s\x_s-L_t\x_t}{\eta_s-\eta_t}-\hv_\phi(\x_t,t)\right)\\
    &=\frac{S_t}{\eta_s-\eta_t}\int_{\eta_t}^{\eta_s}\hv_\phi(\x_{t_{\lambda_\eta}},t_{\lambda_\eta})-\hv_\phi(\x_t,t)\dm\eta
\end{aligned}
\end{equation}
According to the mean value theorem, there exists some $\tau\in [t,t_{\lambda_\eta}]$ satisfying
\begin{equation}
\|\hv_\phi(\x_{t_{\lambda_\eta}},t_{\lambda_\eta})-\hv_\phi(\x_t,t)\|_2\leq (\eta-\eta_t)\left\|\frac{\dm \hv_\phi(\x_\tau,\tau)}{\dm\eta_\tau}\right\|_2
\end{equation}
Besides, the derivative $\frac{\dm\hv_\phi}{\dm\eta}$ can be calculated as
\begin{equation}
\begin{aligned}
    \frac{\dm \hv_\phi(\x_\tau,\tau)}{\dm\eta_\tau}&=\frac{\dm \hv_\phi(\x_\tau,\tau)}{\dm\lambda_\tau}\frac{1}{\dm\eta_\tau/\dm\lambda_\tau}\\
    &=\left(\frac{1}{S_\tau}\frac{\dm\gv_\phi(\x_\tau,\tau)}{\dm\lambda_\tau}-\frac{\dm\log S_\tau}{\dm\lambda_\tau}\frac{\gv_\phi(\x_\tau,\tau)}{S_\tau}\right)\frac{1}{L_\tau S_\tau}\\
    &=\frac{\frac{\dm\gv_\phi(\x_\tau,\tau)}{\dm\lambda_\tau}-s_\tau\gv_\phi(\x_\tau,\tau)}{L_\tau S_\tau^2}
\end{aligned}
\end{equation}
where we have used $\frac{\dm \log S_{\tau}}{\dm\lambda_\tau}=s_\tau$ and $\frac{\dm\eta_\tau}{\dm\lambda_\tau}=L_\tau S_\tau$. Therefore,
\begin{equation}
\label{eq:proof1-1}
\begin{aligned}
\|\Dv_{\theta^*}(\x_{t},t,s)-\Dv_\phi(\x_t,t)\|_2&\leq \frac{S_t}{\eta_s-\eta_t}\max_{s\leq \tau\leq t} \left\|\frac{\dm\gv_\phi(\x_\tau,\tau)}{\dm\lambda_\tau}-s_\tau\gv_\phi(\x_\tau,\tau)\right\|\int _{\eta_t}^{\eta_s}\frac{\eta-\eta_t}{L_\tau S_\tau^2}\dm\eta\\
&\leq \max_{s\leq \tau\leq t} \left\|\frac{\dm\gv_\phi(\x_\tau,\tau)}{\dm\lambda_\tau}-s_\tau\gv_\phi(\x_\tau,\tau)\right\|\int_{\lambda_t}^{\lambda_s}\frac{L_{t_\lambda}S_{t_\lambda}S_t}{L_\tau S_\tau^2}\dm\lambda
\end{aligned}
\end{equation}
Since we assumed $|l_t|,|s_t|\leq c$, according to $\tau\in[t,t_{\lambda}]$, we have
\begin{equation}
    \frac{L_{t_\lambda}}{L_\tau}=e^{\int_{\lambda_\tau}^{\lambda}l_{t_\lambda'}\dm\lambda'}\leq e^{c(\lambda-\lambda_t)},\quad\frac{S_{t_\lambda}}{S_\tau}\leq e^{c(\lambda-\lambda_t)},\quad\frac{S_t}{S_\tau}\leq e^{c(\lambda-\lambda_t)}
\end{equation}
Therefore,
\begin{equation}
\label{eq:proof1-2}
\int_{\lambda_t}^{\lambda_s}\frac{L_{t_\lambda}S_{t_\lambda}S_t}{L_\tau S_\tau^2}\dm\lambda\leq \int_{\lambda_t}^{\lambda_s}e^{3c(\lambda-\lambda_t)}\dm\lambda=\frac{e^{3c(\lambda_s-\lambda_t)}-1}{3c}=\frac{(t/s)^{3c}-1}{3c}
\end{equation}
Substituting Eqn.~\eqref{eq:proof1-2} in Eqn.~\eqref{eq:proof1-1} completes the proof.
\end{proof}
\section{Experiment Details}
\label{appendix:exp-details}
\begin{table}[t]
\vskip -0.2in
	\caption{Experimental configurations.}
	\label{tab:configurations}
	%\footnotesize
	\centering
 % \resizebox{1.0\linewidth}{!}{
	\begin{tabular}{lcccc}
		\toprule
		Configuration & \multicolumn{2}{c}{CIFAR-10}& FFHQ 64$\times$64 & ImageNet 64$\times$64 \\\cmidrule(lr){2-3}\cmidrule(lr){4-4}\cmidrule(lr){5-5}
        & Uncond & Cond & Uncond & Cond \\
        \cmidrule(lr){1-1}\cmidrule(lr){2-2}\cmidrule(lr){3-3}\cmidrule(lr){4-4}\cmidrule(lr){5-5}
        Learning rate & 0.0004 & 0.0004 & 0.0004 & 0.0004 \\
        Student's stop-grad EMA parameter & 0.999 & 0.999 & 0.999 & 0.999 \\
        $N$ & 18 & 18 & 18 & 40 \\
        ODE solver & Heun & Heun & Heun & Heun \\
        Max. ODE steps & 17 & 17 & 17 & 20 \\
        EMA decay rate & 0.999 & 0.999 & 0.999 & 0.999 \\
        Training iterations & 200K & 150K & 150K & 60K \\
        Mixed-Precision (FP16) & True & True & True & True \\
        Batch size & 256 & 512 & 256 & 2048 \\
        Number of GPUs & 4 & 8 & 8 & 32 \\
        Training Time (A800 Hours) &490&735&900&6400\\
            \bottomrule
	\end{tabular}
 % }
 \vskip -0.1in
\end{table}
\subsection{Coefficients Computing}
\label{appendix:compute-ems}
At every time $t$, the parameters $l_t$ and $s_t$ can be directly computed according to Eqn.~\eqref{eq:l-solution} and Eqn.~\eqref{eq:s-solution}, relying solely on the teacher denoiser model $\Dv_\phi$. The computation of $l_t$ involves evaluating $\tr(\nabla_{\x_t}\Dv_\phi(\x_t,t))$, which is the trace of a Jacobian matrix. Utilizing Hutchinson’s trace estimator, it can be unbiasedly estimated as $\frac{1}{N}\sum_{n=1}^N \vv^\top\nabla_{\x_t}\Dv_\phi(\x_t,t)\vv$, where $\vv$ obeys a $d$-dimensional distribution with zero mean and unit covariance. Thus, only the Jacobian-vector product (JVP) $\Dv_\phi(\x_t,t)\vv$ is required, achievable in $\Oc(d)$ computational cost via automatic differentiation. Once $l_t$ is obtained, the function $\gv_\phi(\x_t,t)=\Dv_\phi(\x_t,t)-(1-l_t)\x_t$ is determined. The computation of $s_t$ involves evaluating $\frac{\dm\gv_\phi(\x_t,t)}{\dm\lambda_t}$, which expands as follows:
\begin{equation}
\begin{aligned}
    \frac{\dm\gv_\phi(\x_t,t)}{\dm\lambda_t}&=\frac{\dm\Dv_\phi(\x_t,t)}{\dm\lambda_t}+\frac{\dm l_t}{\dm\lambda_t}\x_t-(1-l_t)\frac{\dm\x_t}{\dm\lambda_t}\\
    &=\frac{\dm\Dv_\phi(\x_t,t)}{\dm\lambda_t}+\frac{\dm l_t}{\dm\lambda_t}\x_t-(1-l_t)(\Dv_\phi(\x_t,t)-\x_t)\\
\end{aligned}
\end{equation}
where $\frac{\dm\Dv_\phi(\x_t,t)}{\dm\lambda_t}$ can also be calculated in $\Oc(d)$ time by by automatic differentiation.

For the CIFAR-10 and FFHQ 64$\times$64 datasets, we compute $l_t$ and $s_t$ across 120 discrete timesteps uniformly distributed in log space, with 4096 samples used to estimate the expectation $\E_{q(\x_t)}$. For ImageNet 64$\times$64, computations are performed across 160 discretized timesteps following EDM's scheduling $(t_{\max}^{1/\rho}+\frac{i}{N}(t_{\min}^{1/\rho}-t_{\max}^{1/\rho}))^\rho$, using 1024 samples to estimate the expectation $\E_{q(\x_t)}$. The total computation times for CIFAR-10, FFHQ 64$\times$64, and ImageNet 64$\times$64 on 8 NVIDIA A800 GPU cards are approximately 38 minutes, 54 minutes, and 38 minutes, respectively.

% CIFAR-10 8.3s/10.45s, 512

% FFHQ 64$\times$64 13.01s/13.84s, 512

% ImageNet 64$\times$64 6.75s/7.35s, 128
\subsection{Training Details}
Throughout the experiments, we follow the training procedures of CTMs. The teacher models are the pretrained diffusion models on the corresponding dataset, provided by EDM. The network architecture of the student models mirrors that of their respective teachers, with the addition of a time-conditioning variable $s$ as input. Training of the student models involves minimizing the consistency loss outlined in Eqn.~\ref{eq:loss-ctm} and the denoising score matching loss $\E_t\E_{p_{\data}(\x_0)q(\x_t|\x_0)}[w(t)\|\Dv_\theta(\x_t,t,t)-\x_0\|_2^2]$. For the consistency loss, we use LPIPS~\citep{zhang2018unreasonable} as the distance metric $d(\cdot,\cdot)$, which is also the choice of CMs. $t$ and $s$ in the consistency loss are chosen from $N$ discretized timesteps determined by EDM's scheduling $(t_{\max}^{1/\rho}+\frac{i}{N}(t_{\min}^{1/\rho}-t_{\max}^{1/\rho}))^\rho$. The Heun sampler in EDM is employed as the solver in Eqn.~\ref{eq:loss-ctm}. The number of sampling steps, determined by the gap between $t$ and $s$, is restricted to avoid excessive training time. For CIFAR-10 and FFHQ 64$\times$64, we select $N=18$ and the maximum number of sampling steps as 17, i.e., not restricting the range of jumping from $t$ to $s$. For ImageNet 64$\times$64, we set $N=40$ and the maximum number of sampling steps to 20, so that the jumping range is at most half of the trajectory length. $\sg(\theta)$ in Eqn.~\ref{eq:loss-ctm} is an exponential moving average stop-gradient version of $\theta$, updated by
\begin{equation}
    \sg(\theta)=\texttt{stop-gradient}(\mu\sg(\theta)+(1-\mu)\theta)
\end{equation}
We follow the hyperparameters used in EDM, setting $\sigma_{\min}=\epsilon=0.002,\sigma_{\max}=T=80.0,\sigma_{\data}=0.5$ and $\rho=7$. The training configurations are summarized in Table~\ref{tab:configurations}.

We run the experiments on a cluster of NVIDIA A800 GPU cards. For CIFAR-10 (unconditional), we train the model with a batch size of 256 for 200K iterations, which takes 5 days on 4 GPU cards. For CIFAR-10 (conditional), we train the model with a batch size of 512 for 150K iterations, which takes 4 days on 8 GPU cards. For FFHQ 64$\times$64 (unconditional), we train the model with a batch size of 256 for 150K iterations, which takes 5 days on 8 GPU cards. For ImageNet 64$\times$64 (conditional), we train the model with a batch size of 2048 for 60K iterations, which takes 8 days on 32 GPU cards.
\subsection{Evaluation Details}
For single-step as well as multi-step sampling of CTMs, we utilize their deterministic sampling procedure by jumping along a set of discrete timesteps $T=t_0\rightarrow t_{1}\rightarrow\dots t_{N-1}\rightarrow t_N=\epsilon$ with the consistency function, formulated as the updating rule $\x_{t_{n}}=\fv_\theta(\x_{t_{n-1}},t_{n-1},t_{n})$. The timesteps $\{t_i\}_{i=0}^N$ are distributed according to EDM's scheduling $(t_{\max}^{1/\rho}+\frac{i}{N}(t_{\min}^{1/\rho}-t_{\max}^{1/\rho}))^\rho$, where $t_{\min}=\epsilon,t_{\max}=T$. We generate 50K random samples with the same seed and report the FID on them.
\subsection{License}
\label{appendix:license}
\begin{table}[ht]
    \centering
    \caption{\label{tab:license}The used datasets, codes and their licenses.}
    \vskip 0.1in
    \resizebox{\textwidth}{!}{% <------ Don't forget this %
    \begin{tabular}{llll}
    \toprule
    Name&URL&Citation&License\\
    \midrule
    CIFAR-10&\url{https://www.cs.toronto.edu/~kriz/cifar.html}&\citep{krizhevsky2009learning}&$\backslash$\\
    FFHQ&\url{https://github.com/NVlabs/ffhq-dataset}&\citep{karras2019style}&CC BY-NC-SA 4.0\\
    ImageNet&\url{https://www.image-net.org}&\citep{deng2009imagenet}&$\backslash$\\
    EDM&\url{https://github.com/NVlabs/edm}&\citep{karras2022elucidating}&CC BY-NC-SA 4.0\\
    CM&\url{https://github.com/openai/consistency_models_cifar10}&\citep{song2023consistency}&Apache-2.0\\
    CTM&\url{https://github.com/sony/ctm}&\citep{kim2023consistency}&MIT\\
    \bottomrule
    \end{tabular}% <------ Don't forget this %
    }
    \vspace{-0.1in}
\end{table}

We list the used datasets, codes and their licenses in Table~\ref{tab:license}.
\section{Additional Samples}
\label{appendix:additional-samples}
\begin{figure}[t]

\begin{minipage}{0.47\textwidth}
    \centering
\resizebox{1\textwidth}{!}{
\includegraphics{figures/fig/cifar10_uncond_crop.jpg}}
\small{(a) CTM (CIFAR-10, Uncond)}
\end{minipage}
\hfill
\begin{minipage}{0.47\textwidth}
\centering
\resizebox{1\textwidth}{!}{
\includegraphics{figures/fig/cifar10_uncond_new_crop.jpg}
}
\small{(b) CTM + Ours (CIFAR-10, Uncond)}
\end{minipage}
\medskip

\begin{minipage}{0.47\textwidth}
    \centering
\resizebox{1\textwidth}{!}{
\includegraphics{figures/fig/cifar10_cond_crop.jpg}}
\small{(c) CTM (CIFAR-10, Cond)}
\end{minipage}
\hfill
\begin{minipage}{0.47\textwidth}
\centering
\resizebox{1\textwidth}{!}{
\includegraphics{figures/fig/cifar10_cond_new_crop.jpg}
}
\small{(d) CTM + Ours (CIFAR-10, Cond)}
\end{minipage}
\medskip

\begin{minipage}{0.47\textwidth}
    \centering
\resizebox{1\textwidth}{!}{
\includegraphics{figures/fig/ffhq_crop.jpg}}
\small{(e) CTM (FFHQ $64\times64$, Uncond)}
\end{minipage}
\hfill
\begin{minipage}{0.47\textwidth}
\centering
\resizebox{1\textwidth}{!}{
\includegraphics{figures/fig/ffhq_new_crop.jpg}
}
\small{(f) CTM + Ours (FFHQ $64\times64$, Uncond)}
\end{minipage}
\medskip

\begin{minipage}{0.47\textwidth}
    \centering
\resizebox{1\textwidth}{!}{
\includegraphics{figures/fig/imagenet_crop.jpg}}
\small{(g) CTM (ImageNet $64\times64$, Cond)}
\end{minipage}
\hfill
\begin{minipage}{0.47\textwidth}
\centering
\resizebox{1\textwidth}{!}{
\includegraphics{figures/fig/imagenet_new_crop.jpg}
}
\small{(h) CTM + Ours (ImageNet $64\times64$, Cond)}
\end{minipage}
\medskip

\caption{\label{fig:samples-more}Random samples produced by CTM and CTM + Analytic-Precond (Ours) with NFE=2.}
\end{figure}


\end{document}
