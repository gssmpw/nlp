\section{Related Works}
The integration of AI tools in education has garnered significant attention, with several studies exploring their impact on student learning outcomes and pedagogical effectiveness. A comprehensive review of ChatGPT applications in higher education highlighted its potential for enhancing student engagement and promoting active learning across various disciplines \cite{rasul2023role}. Similarly, research examining the effectiveness of GenAI tools in computer science education demonstrated improved learning outcomes through AI-assisted programming exercises and code review processes \cite{cubillos2025generative}.

In cybersecurity education specifically, research has increasingly focused on innovative teaching methodologies that bridge theoretical knowledge with practical application. Studies have shown that interactive learning environments significantly enhance students' understanding of complex security frameworks and regulatory requirements \cite{marsa2013design}. Building on this, recent investigations into the role of AI in developing cybersecurity policies highlighted both the benefits and limitations of AI-generated security documentation in educational contexts \cite{anandita2023role}.

The challenges of integrating emerging technologies into cybersecurity curricula have been well-documented. Analysis of the pedagogical implications of AI adoption in security education emphasised the need for structured guidance and critical evaluation frameworks \cite{shchavinsky2023application}. This aligns with research identifying potential risks of over-reliance on AI tools and proposing strategies for maintaining academic rigour while leveraging AI capabilities \cite{zhai2024effects}.

Recent studies have also examined the effectiveness of case-based learning in cybersecurity education \cite{anderson2024case}. Research has demonstrated how real-world scenarios and industry engagement enhance students' practical understanding of security concepts, particularly emphasising the importance of regulatory compliance and risk management in cybersecurity training \cite{alnajim2023exploring}. This aspect forms a foundation for our study's AI-assisted learning approaches.
The role of critical thinking in cybersecurity education has been extensively studied \cite{nowduri2018critical}, with proposed frameworks for evaluating student engagement with security concepts and regulatory requirements. These findings suggest that structured analytical exercises significantly improve students' ability to assess and mitigate security risks \cite{crabb2024critical}, a principle we incorporate through AI-assisted policy development and refinement.
Concerning regulatory compliance in cybersecurity education, research has examined how educational programs can better prepare students for industry requirements. Studies highlighted the importance of incorporating frameworks such as NIST and GDPR into practical exercises \cite{hajny2021framework}, an approach we extend through AI-assisted learning activities.

While existing literature demonstrates the potential of AI in education \cite{10837597,10837608,10837679}, there remains a gap in understanding how GenAI tools can be effectively integrated into cybersecurity curricula while maintaining educational integrity and ensuring regulatory compliance \cite{sandu2024role}. This study addresses this gap by presenting a structured framework for embedding AI tools within cybersecurity education, emphasising critical evaluation, research-based refinement, and practical application of security concepts.
Based on the literature review and identified research gaps, this study addresses the following research questions:
\begin{itemize}
    \item RQ1: How can GenAI tools be effectively integrated into cybersecurity education to enhance students' critical thinking and practical application skills?
    \item RQ2: What are the key challenges and success factors in implementing AI-assisted learning in cybersecurity education?
    \item RQ3: How does the integration of GenAI tools impact students' understanding of regulatory compliance and security policy development?
\end{itemize}


%%%%%%%%%%%%%%%%%%%%%%%%%%%%%%%%%%%%%%%%%%