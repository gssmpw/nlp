\documentclass[lettersize,journal]{IEEEtran}
\usepackage{amsmath,amssymb,amsfonts}
\usepackage{algorithm}
\usepackage{array}
\usepackage[caption=false,font=normalsize,labelfont=sf,textfont=sf]{subfig}
\usepackage{textcomp}
\usepackage{stfloats}
\usepackage{url}
\usepackage{verbatim}
\usepackage{graphicx}
\usepackage{cite}
\usepackage{caption}
\captionsetup[table]{position=top}
\usepackage[noend]{algpseudocode}
\usepackage{tikz}
\usepackage{xcolor}
\usepackage{footnote}
\usepackage{multirow}
\usepackage{tabulary}
\makesavenoteenv{algorithm}
\usepackage{subfig}
\usepackage{subcaption}
\usepackage{booktabs}
\usepackage{soul}

\usepackage[compact]{titlesec}
\usepackage{enumitem}

\renewcommand{\baselinestretch}{0.98}
\titlespacing*{\section}{0pt}{*0.3}{*0.2} % Adjust as needed
\setlist{nosep}

%%%%
\setlength{\textfloatsep}{5pt} % Space between floats and text
\setlength{\floatsep}{5pt}     % Space between two floats
\setlength{\intextsep}{5pt}    % Space between floats in text
%%%%

\title{A Modal-Based Approach for System Frequency Response and Frequency Nadir Prediction%
\thanks{This work was supported in part by the National Science Foundation (NSF) under Grant No. 2044629, and in part by CURENT, which is an NSF Engineering Research Center funded by NSF and the Department of Energy under NSF Award EEC-1041877.}
\thanks{Zelaya-Arrazabal, Martinez-Lizana, and Pulgar-Painemal are with the Department of Electrical Engineering and Computer Science, University of Tennessee, Knoxville, TN 37996, USA. e-mail: fzelayaa@vols.utk.edu, smartinezlizana@ieee.org, hpulgar@utk.edu}}

\author{%
Francisco Zelaya-Arrazabal,~\IEEEmembership{Graduate Student Member,~IEEE,} 
Sebastian Martinez-Lizana,~\IEEEmembership{Graduate Student Member,~IEEE,} 
Héctor Pulgar-Painemal,~\IEEEmembership{Senior Member,~IEEE}%
\vspace{-3em}} % Reduce space here

\begin{document}

\maketitle
\vspace{-1.5em} % Reduce space after maketitle
\begin{abstract}
This letter introduces a novel approach for predicting system frequency response and frequency nadir by leveraging modal information. It significantly differentiates from traditional methods rooted in the average system frequency model. The proposed methodology targets system modes associated with the slower dynamics of the grid, enabling precise predictions through modal decomposition applied to the full system model. This decomposition facilitates an analytical solution for the frequency at the center of inertia, resulting in highly accurate predictions of both frequency response and nadir.
Numerical results from a 39-bus, 10-machine test system verify the method's effectiveness and accuracy. This methodology represents a shift from observing a simplified average system frequency response to a more detailed analysis focusing on system dynamics.
\end{abstract}
\begin{IEEEkeywords}
Frequency nadir, frequency response, nadir estimation, primary frequency regulation, power system dynamics. 
\end{IEEEkeywords}
\vspace{-1.0em}
%%%%%%%%%%%%%%%%%%%%%%%%%%%%%%%%%%%%%%%%%%%%%%%%%%%%%%%%%%%%%%%%%%%%%%%%%%%%%%%%
\section{Introduction}
\IEEEPARstart{I}{n} the rapidly evolving power grid, estimating accurately frequency deviations is essential for effective resource planning, control strategies, and protective measures to prevent instability. This has driven the development of methods for computing the system frequency response and frequency nadir---the maximum frequency excursion---which are used to adjust frequency relays, estimate spinning reserves, limit load reconnection, and calculate safety limits for renewable energy, among others. These methods typically rely on the Average System Frequency (ASF) model \cite{chan1972dynamic} or the low-order ASF \cite{anderson1990low}, which capture the average response of generators after load-generation power imbalances and simplify large-scale nonlinear simulations.

State-of-the-art approaches generally simplify the ASF model to obtain analytical expressions for nadir estimation. This has been done by approximating turbine/governor models as first-order transfer functions or assuming a ramp or parabolic response for the frequency excursion \cite{egido2009maximum,liu2020analytical,niu2022analytical}. More accurate descriptions have also been sought by seeking governor's high-order transfer functions through polynomial fitting \cite{liu2020analytical}. However, these methods neglect complex turbine/governor models, control dynamics, power flow, electromechanical oscillations, and load dynamics leading to oversimplified results. 

A more robust approach maintains the ASF model's control loop for all turbines/governors and aggregates them for improved estimation \cite{shi2018analytical}. However, this approach primarily applies to power grids with

thermal power plants and identical turbine/governor models. Recent efforts based on the ASF model aim to include various primary resource dynamics and governor deadbands but still do not address the limitations of earlier methods \cite{niu2022analytical, chengwei2017minimum}.

This letter proposes a modal-based approach to accurately estimate frequency response and its nadir while avoiding common oversimplifications that can lead to inaccuracies. The main contributions of this letter are: (a) presenting the process to derive a frequency response expression from the full nonlinear model of the power grid, and (b) providing a concise method for accurate estimation of both the system frequency response and its nadir. It is shown that a reduced set of eigenvalues is required for estimation. This methodology can streamline planning studies and open new avenues for utilizing modal information in frequency response and nadir prediction for assessment, control, and monitoring objectives.

\section{Frequency Response Estimation}\label{ADCDC}
Consider a power system modeled through a set of differential-algebraic equations \cite{sauer2017power} such as $\dot{x} = f(x,y)$ and
$0 = g(x,y)$. Here, $x \in \mathbb{R}^n$, and $y \in \mathbb{R}^l$ are the vectors of state variables and algebraic variables, respectively. Assume the system is in steady-state at $x_0$ when a power imbalance occurs. This creates a frequency excursion and the system stabilizes at a new equilibrium point $x_e$. When linearized around $x_e$, the model becomes
$\Delta\dot{x} = J_{1}\Delta x + J_{2}\Delta y$ and $0 = J_{3}\Delta x + J_{4}\Delta y$. By eliminating algebraic equations through Kron reduction, the following simplified model is obtained,
%
\begin{equation}
\Delta \dot{x} = [J_{1} - J_{2}J_{4}^{-1}J_{3}]\Delta x  = A_s \Delta x
\end{equation}
%
with initial condition $\Delta x(0) = \Delta x_0 = x_0-x_e$, which is the difference between the pre- and post-fault states. To obtain the solution, the following similarity transformation is used: $V^{-1}A_sV = W^{T} A_sV = \Lambda$, where $V =\{v_i\}\in \mathbb{C}^{n \times n}$ is the right eigenvectors matrix of $A_s$, $W =\{w_i\}\in \mathbb{C}^{n \times n}$ is the left eigenvectors matrix of  $A_s$, $\Lambda = \text{diag} \{\lambda_i\} \in \mathbb{C}^{n \times n}$ is a diagonal matrix containing the eigenvalues of $A_s$, and the index $i \in \mathcal{I}=\{1,2,...,n\}$. Note that the set of right and left eigenvectors are considered to be orthonormal, i.e., $\langle w_i,v_k\rangle=w_i^Tv_k=\delta_{ik}~\forall~ i,k \in \mathcal{I}$, thus, $W^T=V^{-1}$. For an explicit solution of $\Delta \dot{x} = A_s\Delta x$, the transformed vector of state variables is defined as $q =V^{-1} \Delta x$ to obtain:
%
\begin{align}
\Delta \dot{x} &= A_s \Delta x,~~\Delta x(t_0)=\Delta x_0 \\
\Rightarrow V^{-1} \Delta \dot{x} &= V^{-1} A_s V q ,~~V^{-1}\Delta x(t_0)=V^{-1}\Delta x_0\\
\Rightarrow \dot{q} &= \Lambda q,~~q(t_0)=W^T\Delta x_0=q_0 \label{eq:Transf_model}
\end{align}
%
The explicit solution of Eq. \eqref{eq:Transf_model} is given by $q(t) = e^{\Lambda t} q_0$, where $q_0=W^T \Delta x_0=[w_1,...,w_n]^T \Delta x_0=[\langle w_1,\Delta x_0\rangle,...,\langle w_n,\Delta x_0\rangle]^T$. As $e^{\Lambda t}$ is a diagonal matrix, the solution of Eq. \eqref{eq:Transf_model} can be rewritten as, 
\begin{equation}
\label{Eq_decoupled}
q(t) = 
\begin{bmatrix}
e^{\lambda_1 t}\langle w_1, \Delta x_0\rangle \\
\vdots \\
e^{\lambda_n t}\langle w_n, \Delta x_0\rangle
\end{bmatrix},~~\forall~ t \ge t_0
\end{equation}
%
As a result, the solution for each transformed state variable $i$ becomes $q_i(t) = e^{\lambda_i t} \langle w_i, \Delta x_0\rangle,~ \forall~ i \in \mathcal{I}$. Given that $\Delta x = Vq=\sum_{i \in \mathcal{I}} v_iq_i$, the vector of state variables can be explicitly defined as $\Delta x(t) = \sum _{i \in \mathcal{I}} e^{\lambda_{i} t} \langle w_{i},\Delta x_0\rangle v_{i}$. Note that on the right hand side of this equation, all terms are scalars except for $v_i\in \mathbb{C}^n$. Finally, the explicit solution for a single state variable $k$ corresponds to:
%
\begin{equation}
\Delta x_{k}(t) = \sum _{i \in \mathcal{I}} e^{\lambda_{i} t} \langle w_i,\Delta x_0\rangle v_{i,k}  \  , \  \ \  \forall~ t \geq 0
\end{equation}
%
where $v_{i,k}$ is the $k^{th}$ term of the $i^{th}$ right eigenvector.

Let $\mathcal{Z} \subset \mathcal{I}$ be the index set related to the speed of the generators. Let $H_z$ be the inertia of the generator whose speed index is $z \in \mathcal{Z}$. The total inertia is $H_t=\sum_{z \in \mathcal{Z}} H_z$, and the frequency of the center of inertia (COI) becomes,
%
\begin{equation}
\label{wcoi}
\Delta \omega_{coi} = \sum\limits_{z \in \mathcal{Z}} C_z \Delta x_z=
\sum\limits_{z \in \mathcal{Z}} C_z
\sum _{i \in \mathcal{I}} e^{\lambda_{i} t} \langle w_i,\Delta x_0\rangle  v_{i,z}
\end{equation}

where $C_z = H_z /H_t$. This is an explicit and general expression for the response of the COI under any initial condition. As we are interested in the frequency excursion after a power imbalance, note that only those modes associated with the slow dynamic response of the rotating masses and governors are required for an accurate approximation. Let $\mathcal{M} \subset \mathcal{I}$ be the index set of the most significant modes related to the frequency excursion. Thus, the frequency of the COI can be approximated as,
%
\begin{equation}
\label{eq:w_coi_approx}
\Delta \omega_{coi} \approx
\sum\limits_{z \in \mathcal{Z}}
\sum\limits_{m \in \mathcal{M}} C_z e^{\lambda_{m} t} \langle w_m,\Delta x_0\rangle  v_{m,z} 
\end{equation}
%
\textcolor{black}{This expression provides the system's frequency response in terms of the initial condition} $\Delta x_0 = x_0 - x_e$, where $x_0$ represents the pre-fault steady-state equilibrium point, and $x_e$ must be estimated based on the power imbalance caused by the disturbance. 

Based on modal analysis and the use of participation factors, we argue that estimating only a few variables in $\Delta x_0$ is sufficient to accurately determine $\Delta \omega_{coi}$. These variables include the new steady-state frequency and the governor-related variables. For example, using the IEESGO model for turbines and governors \cite{Zelaya}, and assuming $n_g$ generators, the estimation of the new frequency and governor-related variables after a $\Delta P$ power imbalance, as described in \cite{kundur2007power}, is given by:
%
\begin{align}
\Delta \omega&=-\Delta P/\sum_{z \in \mathcal{Z}} \tfrac{1}{R_{D,z}} = f_{0}-f_{e}\\
y_{1,i}&=y_{3,i} = -\Delta \omega/R_{D,i}~~\forall~i \in \{1,2,...,n_g\}\\
T_{m,i}&=P_{C,i}-\Delta \omega /R_{D,i}~~\forall~i \in \{1,2,...,n_g\}
\end{align}

%%%%%%%%%%%%%%%%%%%%%%%%%%%%%%%%%%%%%%%%%%%%%%%%%%%%%%%%%%%%%%%%%%%%%%
%%%%%%%%%%%%%%%%%%%%%%%%%%%%%%%%%%%%%%%%%%%%%%%%%%%%%%%%%%%%%%%%%%%%%%
\section{Frequency Nadir Prediction}
The frequency nadir can be found through a numerical procedure that minimizes $\Delta \omega_{coi}$ given by Eq. \eqref{eq:w_coi_approx}. However, an analytical approach can simplify and expedite this prediction. For instance, to generalize, consider that only three modes are involved in the frequency excursion dynamics: a complex conjugate pair of eigenvalues, $\lambda_{c1} = \lambda_{c2}^* \in \mathbb{C}$, and a real eigenvalue, $\lambda_r \in \mathbb{R}$. Thus, the frequency response estimation becomes:
%
\begin{equation}
\label{eq:1conjugated_1real}
    \Delta \omega_{coi} \approx e^{\lambda_{c_1} t} \gamma_{c_1} + e ^{\lambda_{c_2} t} \gamma_{c_2} + e ^{\lambda_{r} t} \gamma_{r}\ 
\end{equation}
%
where $\gamma_{m} = \sum_{z \in \mathcal{Z}} C_z \langle w_{m}, \Delta x_0 \rangle v_{m,z},~\forall~m \in \mathcal{M}= \{c1, c2, r\}$. Given the conjugated nature of the first two terms on the right-hand side, these can be compactly expressed in terms of the cosine function $2e^{\alpha t} r_c cos(\beta t + \theta)$, where $\lambda_{c_1}=\alpha + j \beta $, $\gamma_{c_1}=E+jF$, $j=\sqrt{-1}$, $r_c = \sqrt{E^2 + F^2}$, and $\theta = \arctan{(F/E)}$. Thus, $\Delta \omega_{coi}$ and its derivative become,
\begin{align}
    \Delta \omega_{coi} &\approx 2e^{\alpha t} r_c cos(\beta t + \theta) + e ^{\lambda_{r} t} \gamma_{r}\label{eq:1conjugated_1real}\\
    \frac{d \Delta \omega_{coi}}{dt} &= R e^{\alpha t} cos(\beta t + \theta + \phi ) + \lambda_{r} e ^{\lambda_{r} t} \gamma_{r}\label{eq:derivativewcoi}
\end{align}
with $R = 2 r_c \sqrt{\alpha^2 + \beta ^2}$ and $\phi = \arctan{(\beta/\alpha)}$. The time occurrence of the frequency nadir corresponds to the time when Eq. \eqref{eq:derivativewcoi} vanishes. Note that this equation involves two transcendental functions, making it challenging to solve it analytically. Thus, a second order Taylor expansion around an initial guess, $\tau$, can be used to approximate the right-hand side of Eq. \eqref{eq:derivativewcoi}, resulting in the following expression,
\begin{align}
\label{eq:wcoi_der}
     \frac{d \Delta\omega_{\text{coi}}}{dt} =  \left(a_1 t^2 + a_2 t + a_3 \right) + \left( a_4 t^2 + a_5 t + a_6 \right) = 0 
\end{align}
See appendix for the coefficients definition. Therefore, the estimated time occurrence of the frequency nadir time is,
\begin{equation}
    t_{nadir} = \frac{-(a_2+a_5) + \sqrt{(a_2+a_5)^2 - 4(a_1+a_4)(a_3+a_6)}}{2(a_1+a_4)}
\end{equation}
and the per unit frequency nadir prediction can be obtained through direct evaluation as $f_{nadir} = f_{e} + \Delta \omega_{coi} (t_{nadir})$, where $f_{e}$ is the post-disturbance steady-state frequency and can be extracted from Eq. 9. If additional modes are required, either complex conjugated or real, the number of terms in Eqs. \eqref{eq:1conjugated_1real}-\eqref{eq:wcoi_der} will be increased; still, the solution for $t_{nadir}$ will be always the solution of a quadratic function. Results show that the frequency response estimation requires a small set of modes. For instance, the authors validated this idea on a 9-bus, a 14-bus, and a 39-bus test systems. For accurate estimation, the first system requires a pair of conjugate modes and two real modes, whereas the other two systems require a pair of conjugate modes and one real mode. Due to space limitations, further details, results and analysis are presented only for the 39-bus test system in the following section.

%%%%%%%%%%%%%%%%%%%%%%%%%%%%%%%%%%%%%%%%%%%%%%%%%%%%%%%%%%%%%%%%%%%%%%%
%%%%%%%%%%%%%%%%%%%%%%%%%%%%%%%%%%%%%%%%%%%%%%%%%%%%%%%%%%%%%%%%%%%%%%%

\section{Case Study: IEEE 39-bus test system}
\label{ADCDC}
Each SG is represented by a two-axis model, an IEEE Type-1 exciter, and an IEESGO turbine/governor model. The system model has a total of 139 state variables. To observe a larger nadir, the inertia has been reduced by half. \textcolor{black}{System data can be obtained from \cite{sauer2017power}.} The slow dynamics involved in the frequency response are led by the turbine/governor of each generator; therefore, the key modes are identified as those with the turbine/governor state variables having the largest participation factors (PFs). For practical purposes, all state variables with PFs above a predefined threshold can be considered for estimation purposes; in this application, PFs above 0.001 are considered. For this system, the following modes are identified: $\lambda_r = -2.9163$, and $\lambda_{c_1,c_2}=-0.1553 \pm j0.1507$.
%%%%%%%%%%%%%%%%%%%%%%%%%%%%%%%%%%%%%%%%
\subsection{Frequency response estimation and nadir prediction}
Two cases are studied. The first one considers a $20\%$ load step-change at bus 15, equivalent to $2.5\%$ of the total system load. The second case involves the trip of SG1 at bus 30, equivalent to a $4\%$ generation decrease in the system. Fig. \ref{ls1} shows a comparison of the estimated frequency response and the actual response from a nonlinear time-domain simulation. Note that the modal-based approach is able to reproduce the results with high fidelity. A similar situation occurs with the nadir prediction displayed in Table \ref{tab:nadir_data} for both cases.

\begin{figure}[h!]
    \centering
    \begin{tikzpicture}
     \hspace{-0.1 in}
    \node at (0.8,0.1) [anchor=center] {\includegraphics[width=1.8in]{figures/Load_step_red.eps}};
    \node at (4.9,0.1) [anchor=center] {\includegraphics[width=1.65in]{figures/Gen_trip_red.eps}};
    \node at (2.4,-0.35) [anchor=center] {\footnotesize(a)};
    \node at (6.5,-0.35) [anchor=center] {\footnotesize(b)};
    \end{tikzpicture}
    \caption{SFR: (a) Load step at bus 15 and (b) SG1 trip.}
    \label{ls1}
\end{figure}
\begin{table}[htbp]
    \centering
    \scriptsize
     \caption{Comparison of frequency nadir timing, magnitude, and absolute percentage error (APE).}
    \begin{tabular}{lcc|cc|c|c}
        \toprule
        Case & \multicolumn{2}{c}{Actual} & \multicolumn{2}{c}{Prediction} & \multicolumn{2}{c}{APE} \\
        \cmidrule{2-7}
        & Nadir & Nadir time & Nadir & Nadir time & Nadir  & Time \\
        & (Hz) & (s) & (Hz) & (s) & (\%) &  (\%)\\
        \midrule
        Case 1 & 59.60 & 8.51 & 59.60 & 8.62 & 0 & 1.29 \\
        Case 2 & 59.09 & 9.31 & 59.11 & 9.14 & 0.03 & 1.82 \\
        \bottomrule
    \end{tabular}
    \label{tab:nadir_data}
\end{table}

%%%%%%%%%%%%%%%%%%%%%%%%%%%%%%%%%%%%%%%%
\subsection{Modal sensitivity}
\textcolor{black}{The slow dynamics of the system frequency response following a power imbalance are linked to turbine/governor models, where changes occur only in the power reference set points when the loading level varies. These adjustments have minimal impact on the modes involved in the estimation and do not significantly affect the results.}
This is demonstrated in Fig. 2, where a sensitivity analysis of the modes involved in the estimation has been carried out. The load and generation have been scaled from 50\% to 150\% of the base case in steps of 10\%. The analysis shows (a) minimal variation in the frequency and damping of the modes—one of the complex conjugate modes is depicted in red and the real mode in black—and (b) the nadir estimation remains highly accurate despite using the modal information of the base case for estimating all other cases under different loading conditions. The largest absolute error between the real nadir and its estimation occurs with a 130\% scaling factor and is under 0.003\%.
\begin{figure}[t]
    % \centering
    \begin{tikzpicture}
     \hspace{-0.1 in}
    \node at (0.16,0.1) [anchor=center] {\includegraphics[width=3.4in]{figures/eigenplot.eps}};
    \node at (0.16,-2.4) [anchor=center] {\includegraphics[width=3.2in]{figures/Load_step_eval2.eps}};
    % \node at (0.16,0.1) [anchor=center] {\includegraphics[width=3.4in]{figures/eigenplot.eps}};
    % \node at (0.16,-2.6) [anchor=center] {\includegraphics[width=3.2in]{figures/Load_step_eval2.eps}};
    \node at (3.2,0.7) [anchor=center] {\footnotesize(a)};
    \node at (3.2,-1.85) [anchor=center] {\footnotesize(b)};
    \end{tikzpicture}
    % \vspace{-0.7cm}
    \caption{(a) Eigenvalues sensitivity, (b) Nadir comparison and absolute percentage error (APE) for different scaling factors.}
\end{figure}
\textcolor{black}{This suggest that is not necessary to linearize the grid model for every potential operating condition. The modal information from a base case remains highly effective in providing accurate predictions across a range of loading conditions.}
%%%%%%%%%%%%%%%%%%%%%%%%%%%%%%%%%%%%%%%%%%%%%%%%%%%%%%%%%%%%%%%%%%%%%%%%%%%%%%%%
\section{Conclusion}
\textcolor{black}{This paper proposes an analytical solution for system frequency response and nadir prediction. It uses the modal information associated with the slow frequency dynamics of synchronous generators and turbine/governors while neglecting faster dynamics. Numerical simulations validate the approach, showing that system frequency response and nadir can be accurately estimated using a small set of eigenvalues and their modal information. Tests on a 39-bus system reveal minimal error in the magnitude and timing of the frequency nadir estimation. Additionally, results indicate low sensitivity of the relevant dynamics to operating conditions, providing a fast and accurate method for evaluating disturbances across various loading levels using the same modal information.} 
\textcolor{black}{Next steps in this research will focus on its practical implementation to streamline planning analyses and enhance real-time security assessment.}
\bibliographystyle{IEEEtran}
\bibliography{Biblio}
\appendix

\begin{align*}
    \psi &= \beta \tau+ \theta + \phi, \ \quad K_1 = R e^{\alpha \tau}, \quad K_2 = \sqrt{\alpha^{2}+\beta^{2}}\\
    a_1 &= K_1 \left( \frac{(\alpha^2 - \beta^2)}{2} \cos(\psi) - \alpha \beta \sin(\psi) \right) \\
    a_2 &= K_1 \left( K_2 \cos(\psi+\phi) - \right. 
    \\
    & \quad \left. ((\alpha^2 -\beta^2)cos(\psi)-2\alpha\beta\sin(\psi))\tau \right) \\ 
    a_3 &= K_1 \left(\cos(\psi)-K_2\cos(\psi+\phi)\tau + \right. 
    \\
    & \quad \left. ( \frac{(\alpha^2 - \beta^2)}{2} \cos(\psi) - \alpha \beta \sin(\psi))\tau^2  \right)  \\ 
    a_4 &= \frac{\lambda_{r}^3 e^{\lambda_{r} t_0} \gamma_{r}}{2}, \ \  a_5 = \lambda_{r}^2 e^{\lambda_{r} \tau} \left(1 - \lambda_{r} \tau \right) \gamma_{r},  \\
    a_6 &= \lambda_{r} e^{\lambda_{r} \tau}\left(1 - \lambda_{r} \tau + \frac{\lambda_{r}^2 \tau^2}{2} \right)\gamma_{r}.
\end{align*}
%%%%%%%%%%%%%%%%%%%%%%%%%%%%%%%%%%%%%%%%%%%%%%%%%%%%%%%%%%%%%%%%%%%%%%%%%%%%%%%%
\end{document}
