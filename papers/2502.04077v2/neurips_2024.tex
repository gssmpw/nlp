\documentclass{article}
\pdfoutput=1
% if you need to pass options to natbib, use, e.g.:
%     \PassOptionsToPackage{numbers, compress}{natbib}
% before loading neurips_2024


% ready for submission
% \usepackage{neurips_2024}


% to compile a preprint version, e.g., for submission to arXiv, add add the
% [preprint] option:
    \usepackage[preprint]{neurips_2024}


% to compile a camera-ready version, add the [final] option, e.g.:
%     \usepackage[final]{neurips_2024}


% to avoid loading the natbib package, add option nonatbib:
%    \usepackage[nonatbib]{neurips_2024}


\usepackage[utf8]{inputenc} % allow utf-8 input
\usepackage[T1]{fontenc}    % use 8-bit T1 fonts
\usepackage{hyperref}       % hyperlinks
\usepackage{url}            % simple URL typesetting
\usepackage{booktabs}       % professional-quality tables
\usepackage{amsfonts}       % blackboard math symbols
\usepackage{nicefrac}       % compact symbols for 1/2, etc.
\usepackage{microtype}      % microtypography
\usepackage{xcolor}         % colors

\usepackage{algorithm}
\usepackage{algorithmic}
\usepackage{amsmath}
\usepackage{amssymb}
\usepackage{mathtools}
\usepackage{amsthm}
\usepackage{subcaption}
\usepackage{multicol}
\usepackage{multirow}
\usepackage{makecell}
\usepackage{colortbl}
\setlength{\tabcolsep}{0.25em}
\usepackage{wrapfig}
\usepackage{natbib}

\title{AttentionPredictor: Temporal Pattern Matters \\ for Efficient LLM Inference}


% The \author macro works with any number of authors. There are two commands
% used to separate the names and addresses of multiple authors: \And and \AND.
%
% Using \And between authors leaves it to LaTeX to determine where to break the
% lines. Using \AND forces a line break at that point. So, if LaTeX puts 3 of 4
% authors names on the first line, and the last on the second line, try using
% \AND instead of \And before the third author name.
\newcommand{\thought}[1]{{\color[rgb]{0.2,0.39,0.66}(#1)}}
\newcommand{\todo}[1]{{\color[rgb]{1.0,0.0,0.0}(#1)}}
\newcommand{\hsh}[1]{{\color{green!50!black} Henrik: #1}}
\newcommand{\st}[1]{{\color{red!50!black} Sebastian: #1}}

\newcommand{\ulm}[1]{_{\scaleto{\mathrm{#1}}{3pt}}}
\newcommand\at[2]{\left.#1\right|_{#2}}











\newtheorem{assumption}{Assumption}

\DeclareMathOperator*{\argmax}{arg\,max}
\DeclareMathOperator*{\argmin}{arg\,min}

\newcommand{\swname}[1]{\texttt{#1}}
\newcommand{\ie}{i\/.\/e\/.,\/~}
\newcommand{\eg}{e\/.\/g\/.,\/~}
\newcommand{\cf}{cf\/.\/~}

\newcommand{\fig}{Fig\/.\/~}
\newcommand{\defn}{Def\/.\/~}
\newcommand{\sect}{Sec\/.\/~}
\newcommand{\tabl}{Tab\/.\/~}
\newcommand{\algo}{Algorithm~}
\newcommand{\theo}{Theorem~}

\newcommand{\bnnl}{3 hidden layers}
\newcommand{\bnnn}{50 neurons}
\newcommand{\bnna}{tanh activations}

\newcommand{\capt}[1]{\mdseries{\emph{#1}}}

\newcommand{\videolink}{at \url{https://youtu.be/_d7AqTRjz6g}}
\newcommand{\codelink}{\url{https://github.com/wheelbot/mini-wheelbot}}

\newcommand{\fakepar}[1]{\vspace{0mm}\noindent\textbf{#1.}}

\newcommand{\needref}{\textcolor{red}{[REF]}}

\newcommand{\plotfontsize}{9pt}


\author{
    Qingyue Yang$^1$\thanks{This work was done when Qingyue Yang was an intern at Huawei.}
  \And
   Jie Wang$^1$\thanks{Corresponding author. Email: jiewangx@ustc.edu.cn.}
   \And
   Xing Li$^2$
   \And
   Zhihai Wang$^1$
   \And
   Chen Chen$^2$
   \And
   Lei Chen$^2$ 
   \And 
   Xianzhi Yu$^2$
   \And
   Wulong Liu$^2$
   \And
   Jianye Hao$^{2,3}$
   \And
   Mingxuan Yuan$^2$
   \And
   Bin Li$^1$
   \AND
   $^1$MoE Key Laboratory of Brain-inspired Intelligent Perception and Cognition,\\
   University of Science and Technology of China\\
   $^2$Noah's Ark Lab, Huawei Technologies\\
   $^3$College of Intelligence and Computing, Tianjin University
}


\begin{document}


\maketitle


\begin{abstract}
With the development of large language models (LLMs), efficient inference through Key-Value (KV) cache compression has attracted considerable attention, especially for long-context generation.
To compress the KV cache, recent methods identify critical KV tokens through heuristic ranking with attention scores.
However, these methods often struggle to accurately determine critical tokens as they neglect the \textit{temporal patterns} in attention scores, resulting in a noticeable
degradation in LLM performance. 
To address this challenge, we propose \ours, which is the first learning-based critical token identification approach.
Specifically, \ours learns a lightweight convolution model to capture spatiotemporal patterns and predict the next-token attention score.
An appealing feature of \ours is that it accurately predicts the attention score while consuming negligible memory.
Moreover, we propose a cross-token critical cache prefetching framework that hides the token estimation time overhead to accelerate the decoding stage.
By retaining most of the attention information, \ours achieves 16$\times$ KV cache compression with comparable LLM performance, significantly outperforming the state-of-the-art. The code is available at \href{https://github.com/MIRALab-USTC/LLM-AttentionPredictor}{https://github.com/MIRALab-USTC/LLM-AttentionPredictor}.
\end{abstract}


\section{Introduction}

Large language models (LLM) like OpenAI o1~\citep{openai2024o1} have shown impressive scalability and effectiveness in tackling complex tasks through long chain-of-thought (CoT) reasoning and multi-turn conversations~\citep{minaee2024large, huang2024understanding}. 
However, these long-context tasks require LLMs to handle extremely lengthy contexts, presenting computational and memory challenges for LLMs~\citep{zhou2024survey}. 
Specifically, the key-value cache (KV cache), which holds the attention keys and values during generation to prevent re-computations, consumes huge GPU memory~\citep{li2024survey}. 
For example, for a model with 7 billion parameters, the parameters consume only 14 GB of memory whereas the KV cache requires around 72 GB with the 128K prompt length~\citep{yang2024pyramidinfer}. 
As the decoding latency and memory footprint scale with the KV cache, it is important to compress the KV cache as the prompt expands. 

\begin{figure}[t]
    \centering
    \includegraphics[width=0.7\textwidth]{figures/intro.pdf}
    \caption{A comparison of H2O, Quest, and \ours for identifying critical tokens in the next step with history attention score. Our learning-based spatiotemporal predictor captures the dynamic attention patterns and accurately predicts next-step attention scores.}
    \vspace{-0.7cm}
    
    \label{fig:intro_comparison}
\end{figure}

Many attention sparsity-based methods are proposed to compress the KV cache in the sequence dimension. 
Previous works have shown that a small portion of tokens dominate the attention distribution,  substantially determining token generation precision \citep{liu2023dejavu,zhang2023h2o, ge2024fastgen}.
Therefore, we can dramatically reduce the cache size by computing attention sparsely with only the critical tokens, while maintaining LLM performance. 
Cache eviction methods \citep{zhang2023h2o, xiao2024streamingllm,li2024snapkv} use heuristic ranking with attention scores to identify critical keys and values and evict the less relevant ones. However, heuristic scoring methods can only model static patterns and struggle to identify critical tokens accurately, causing these methods to suffer from a degradation in LLM performance.
Cache retrieval methods \citep{tang2024quest, zhang2024pqcache} identify critical tokens by approximating attention using compressed keys and the current query.
Although effective,
these methods demand substantial computational resources, and model accuracy tends to degrade as the compression ratio and retrieval granularity increase. 
In particular, \citet{tang2024quest} experiences a sharp 11\% accuracy drop when the retrieving page size increases from 16 to 64.
Additionally, retrieval-based methods rely on the query token in the current step, limiting their ability to overlap the estimation time overhead with actual LLM inference computation in cache prefetching systems. 
Thus, it is essential to accurately and rapidly identify critical tokens of the KV cache.



In our paper, we explore the importance of temporal patterns in attention scores for identifying critical KV cache. Previous research has indicated that attention scores exhibit repetitive patterns \citep{jiang2024minference, ge2024fastgen} and are intrinsic to the LLM. 
These findings suggest that identifying attention patterns enables the prediction of critical tokens, making cache compression possible while preserving most information as in multi-tier cache management systems \citep{hashemi2018learning, li2021block}.
We further observe that attention patterns have temporal characteristics, such as re-access, sequential, and seasonal, illustrated in \autoref{fig:attention_heatmap}.
To capture the attention patterns, we introduce \ours, a time-series prediction approach to predict attention scores for critical token identification, as shown in \autoref{fig:intro_comparison}. 
Since temporal attention patterns are dynamic and difficult to capture with existing heuristic methods, our approach is the first learning-based method to address this challenge.
We frame the prediction of attention scores as a spatio-temporal forecasting problem, treating token positions as spatial patterns. 
We then train a convolutional model to predict next-step attention scores based on historical attention time series, enabling us to accurately identify the most critical tokens during LLM decoding.
Notably, the lightweight \ours allows for accurate predictions with negligible memory consumption.
Additionally, we employ distribution error calibration by periodic computing dense attention,
and we apply block-wise attention compression to further improve the efficiency of prediction. 
By accurately identifying critical tokens, \ours retains most of the attention information after the KV cache compression.

To further accelerate LLM decoding, we apply \ours to our proposed KV cache management framework, a cross-token KV cache prefetching system.
Current LLM serving systems offload the KV cache to the CPU to reduce the GPU memory usage of long-context generation, but CPU-GPU transfer latency of KV cache becomes a significant bottleneck.  
KV cache prefetching presents a solution to hide cache loading time by asynchronously loading the required KV cache in advance.
In contrast to the existing cross-layer approach \citep{lee2024infinigen}, our cross-token method can capitalize on longer transfer times, and adapt to more sophisticated and accurate methods for identifying critical tokens. 
Additionally, the transfer data for each token is better organized, leading to improved I/O utilization.


We evaluated our method on several representative LLMs with varying KV cache budgets. On the widely used LongBench dataset, our approach achieves 
% the LLM accuracy drop of no more than 0.5\% 
comparable accuracy
with 16× KV cache compression, outperforming the state-of-the-art by 41\%.  With 32K context, our prefetching framework achieves a 1.4$\times$ speedup of per token decoding latency.
% Additionally, on the CoT task with 16K prompt length, we achieved a 14.86\% improvement in accuracy over other methods at the same compression ratio.
In summary, we make the following contribution:
\begin{itemize}
\item 
Based on our observation of the temporal pattern in the attention score, we propose \ours, the first learning-based critical token identification approach.
\item We propose the first cross-token prefetching framework, which effectively mitigates prediction and transfer delays in the LLM decoding stage. 
\item Experiments on the long-context datasets demonstrate that our approach achieves comparable accuracy of LLM with 16× KV cache compression.

\end{itemize}

\section{Related works}
\subsection{Efficient LLM Inference}
Several efforts have optimized pre-trained model inference efficiency from different perspectives.
Speculative decoding methods~\citep{cai2024medusa, li2024eagle, sun2024triforce, liu2024deepseekv3} use smaller draft models to accelerate the auto-regressive decoding process of LLMs. 
Specifically, these methods employ a draft model to efficiently predict several subsequent tokens as generation results, which are then validated in parallel using the target LLM.
In contrast, our approach focuses on predicting the attention scores of the next token, serving as an estimation for KV cache compression.
Other methods include model compression~\citep{frantar2023gptq, lin2024awq} and inference systems~\citep{kwon2023vllm, sheng2023flexgen, song2024tackling, 2023lmdeploy, zheng2024sglang, ye2025flashinfer}.


\subsection{KV Cache Compression}
Many methods are dedicated to compressing the KV cache while retaining as much information of attention as possible.
Many works have found that attention scores are sparse, so sparse computation can be performed on high-score positions by estimating attention scores.

\textbf{Cache eviction.} These methods use heuristic approaches to identify key-value pairs of high importance and evict the less relevant ones.
StreamingLLM~\citep{xiao2024streamingllm} observes that the earliest tokens have high attention scores during inference, so it only retains the initial and the recent few tokens.
H2O~\citep{zhang2023h2o} accumulates all historical attention scores as an estimation, but suffers from the accumulation error of scores from the first few tokens being too frequent. 
SnapKV~\citep{li2024snapkv} accumulates attention scores within a recent window as an estimate and performs one-time filtering during the prefill stage.
MInference~\citep{jiang2024minference} inductively defines three attention patterns and pre-assigns each head a fixed compression method, determining hyperparameters by attention scores within a recent window and staying static during the decoding stage.
All these methods are heuristic-based and statically model attention patterns. 
They struggle to capture the dynamic temporal patterns within attention scores accurately.

\textbf{Cache retrieval.} These methods aim to retrieve the most critical tokens with approximate attention scores.
Quest~\citep{tang2024quest} achieves this by using the current query and paged key to approximate the attention score. However, Quest is sensitive to the page size, and the accuracy significantly drops with large page sizes and small budgets. 
Similarly, PQCache~\citep{zhang2024pqcache} applies key quantization during the prefilling stage and reuses these quantized keys in decoding to approximate attention.
Since these methods require information from the current step, they cannot use asynchronous computation to cover long estimation durations. 
Other methods involving KV cache quantization~\citep{liu2024kivi, hooper2024kvquant, zhang2024qhitter} and KV cache budget allocation~\citep{cai2024pyramidkv, yang2024pyramidinfer, feng2024adakv} are orthogonal to our token scoring approach and can be combined with our \ours.

\textbf{KV cache cross-layer prefetching.} These methods hide part of the cache transfer time based on offloading the cache to the CPU. However, as the sequence length increases, the transfer time is too long to be hidden.
InfiniGen~\citep{lee2024infinigen} combines cache retrieval and prefetching by approximating the attention score for the next layer to load the critical cache.
However, the estimation time increases significantly as the sequence grows, and the inference time for a single layer is insufficient to cover this. 
In contrast, our cross-token prefetching framework can hide longer estimation and transfer time within the per-token inference time.

\section{Observation}
\subsection{Attention Temporal Patterns}
Our research is motivated by the observation that attention exhibits three distinct temporal patterns: re-access, sequential, and seasonal. 
As shown in \autoref{fig:attention_heatmap}, the re-access pattern occurs as vertical lines, which shows repeated attention to specific tokens. 
The sequential pattern is marked by diagonal lines, which show the attention moves sequentially to the next tokens.
Meanwhile, the seasonal pattern is characterized by the periodic recurrence of critical tokens, which appears in \autoref{fig:attention_heatmap} as alternating regions of high and uniform attention scores. 
These patterns are consistently observed across different models and datasets. 

\subsection{Predict Attention by Time Series Methods}
Capturing attention patterns enables accurate prediction of subsequent attention, suggesting that attention is inherently predictable.
Given that LLMs are autoregressive, the process of token generation can be naturally modeled as a time series. 
Consequently, we propose to model attention scores as a spatiotemporal sequence, where they form a time series along the inference dimension and a spatial sequence along the token dimension. 
This frames attention prediction as a time series forecasting task, leveraging temporal prediction techniques to forecast the next-step attention. 
In contrast, existing approaches rely on static attention modeling, limiting their ability to adapt to dynamic temporal variations.

\begin{figure}[t]
    \centering
    \includegraphics[width=0.7\linewidth] %, height=0.5\linewidth
        {figures/observation/larger.png}
    \caption{Visualization of three temporal attention patterns. \textbf{Re-access}  shows repeated attention to specific tokens. \textbf{Sequential} shows attention progresses toward the next tokens. \textbf{Seasonal} exhibits periodic recurrence as alternating bands of high and uniform attention scores.}
    \vspace{-0.4cm}
    \label{fig:attention_heatmap}
\end{figure}


\subsection{Association Between Attention Patterns and Query}
To better understand why attention follows temporal patterns, we investigate the underlying causes. We find that the high continuity between queries plays a central role in shaping attention behavior. In particular, the query sequence exhibits a cosine autocorrelation of 87\% at a one-step lag (in App.~\ref{appendix:q_similarity}), which highlights its strong similarity. High query autocorrelation, as observed in our work, is also reported in \citet{lee2024infinigen}.
Based on this, we further analyze the correlation between attention in step $i$ and step $i+1$.
Focusing on the logits before $\text{Softmax}$, the attention computation at step $i$ is given as $A_i = \frac{1}{\sqrt{D}} \mathbf{Q_i} \mathbf{K_i}^\top = \frac{1}{\sqrt{D}} q_i k_{1:i}^\top$,
where $q_i, k_i \in \mathbb{R}^{D} $ represent the query and key vectors of the token $i$, respectively.
Assuming the query at step $i+1$ satisfies the relationship 
$q_{i+1} = q_i + \Delta q$ ,
the attention computation at step $i+1$ can then be expressed as:\vspace{-5pt}
\begin{equation}
\begin{aligned}
A_{i+1} &= \frac{1}{\sqrt{D}} \mathbf{Q_{i+1}} \mathbf{K_{i+1}}^\top  \\
&= \frac{1}{\sqrt{D}} q_{i+1} k_{1:i+1}^\top   \\
&=\frac{1}{\sqrt{D}} (q_{i}+\Delta q) k_{1:i+1}^\top   \\
&=\frac{1}{\sqrt{D}} (q_{i} k_{1:i+1}^\top + \Delta q k_{1:i+1}^\top) \\
\end{aligned}
\end{equation}
Focusing on the values of $A_{i+1}$ in the first $i$ positions,
\begin{equation}
\begin{aligned}
A_{i+1}[1:i] &=\frac{1}{\sqrt{D}} q_{i} k_{1:i}^\top + \frac{1}{\sqrt{D}}\Delta q k_{1:i}^\top \\
&= A_i + \Delta A \\
\end{aligned}
\end{equation}

Therefore the difference between $A_i$ and $A_{i+1}$ is dominated by $\Delta q$. Since $q_i$ and $q_{i+1}$ have high similarity, $\Delta q$ is relatively small. Consequently, $\Delta A$ is small and $A_i \approx A_{i+1}$. Therefore, the critical tokens of adjacent steps are similar, which is suitable for cross-token prefetching.






% \subsection{Notations}  %% commented out as we do not use them

% The notations used throughout this paper are summarized in Table ~\ref{t:notations}.

% \begin{table}
%     \centering
%     \small % Reduce font size for the table (optional)
%     \begin{tabular}{|l|c|}
%         \hline
%         \textbf{Notation} & \textbf{Description}  \\
%         \hline
%         $X_{\text{tr}}$ & Training set inputs (messages) 
%         \\\hline
%         $y_{\text{tr}}^{\text{gold}}$ & Gold labels for $X_{\text{tr}}$\\
%         \hline
%         $y_{\text{tr}}^{\text{llm}}$ & Synthetic labels for $X_{\text{tr}}$ \\ \hline
%          $X_{\text{val}}$ & Validation set inputs \\
%          \hline
%         $y_{\text{val}}^{\text{gold}}$ & Gold labels for $X_{\text{val}}$ \\
%         \hline
%         $y_{\text{val}}^{\text{llm}}$ & Synthetic labels for $X_{\text{val}}$ \\ \hline
%         $X_{\text{test}}$ & Test set inputs \\
%         \hline
%         $y_{\text{test}}^{\text{gold}}$ & Gold labels for $X_{\text{test}}$ \\
%         \hline
%         $y_{\text{test}}^{\text{llm}}$ & Synthetic labels for $X_{\text{test}}$ \\ \hline
%         $(X, y)_{\text{tr}}^{\text{llm}}$ & Synthetic training data \\
%         \hline
%          $(X, y)_{\text{val}}^{\text{llm}}$ & Synthetic validation data \\  
%          \hline
%     \end{tabular}
%     % ACL style has the caption below the table or figure
%     \caption{Summary of notations used in the paper}
%     \label{t:notations} 
% \end{table}


\subsection{Overview of Scenarios}

%In this study, we
We investigate the role of LLMs in CB detection, focusing on their utility under varying data availability conditions
and under the assumption that direct use of LLMs as a classifier is too expensive due to the high volume of messages to be checked.
%To establish
As a baseline for comparison, we %first
evaluate a scenario in which a
lightweight, BERT-based
classifier is trained exclusively on gold-standard, manually labeled authentic data without %any
LLM involvement.
We then define three additional scenarios with different data availability
and that use LLMs in different ways.
%, each illustrating how LLMs can aid in CB detection depending on the availability and quantity of authentic data.
%The scenarios are as follows.

%To establish a baseline for comparison, in the first scenario, we evaluate a setup that relies exclusively on training a classifier using gold-standard, manually labeled authentic data with no LLM involvement. We then define three other distinct scenarios, each corresponding to
% %% JW: The following is unneccesary vague as the scenarios are more specifically
% %% about the way the synthetic data is used, apart from the zero-shot LLM.
% a unique way LLMs can be integrated into the detection pipeline.
% These scenarios range from directly serving as classifiers to generating synthetic data or labels for training. 

\paragraph{Scenario 1: Baseline}

This scenario represents the ideal situation where sufficient
%manually labeled (
gold-standard data is available for fine-tuning %a classic encoder such as
BERT.
It serves as the benchmark for evaluating the effectiveness of other approaches.
In this setup, no synthetic data or LLMs are involved.
%The system relies entirely on human annotations.
This scenario is feasible if resources such as time, budget and expert annotators are abundant. However, it often proves impractical due to the
%high costs and scalability
challenges of manual labeling.



\paragraph{Scenario 2: LLM as Classifier}  \label{s:m:sc2}

This scenario applies when labeled authentic data is unavailable, and there is no intention to train a separate classifier for CB detection. Instead, an instruction-tuned
LLM is used directly as a classifier, leveraging its pre-trained knowledge and its ability to follow instructions
to identify CB instances.
%This approach is particularly useful in contexts that require rapid deployment or when computational or time resources are limited for training a new model. 
The primary advantage of this method is its elimination of the need for labeled data and training time. However, there are trade-offs. While an LLM can handle nuanced language patterns, it may be less efficient and incur higher computational costs
compared to simpler BERT-based classifiers with a classification head and fine-
tuned on a labeled dataset.
%% JW: add reference to large zero-shot study in NLP
We explore two prompting strategies for generating synthetic labels:
\textit{(a)} guideline-enhanced (GE) prompts, guiding the LLM with detailed labeling instructions and
\textit{(b)} guidelne-free (GF) prompts, allowing the LLM to generate labels without such guidelines.

\paragraph{Scenario 3: Fully Synthetic Data}

In this scenario, only a small set of manually labeled gold data is available for testing, with no access to authentic data for training or validation.
%To address this, we
We
use an LLM to generate a fully synthetic dataset, consisting of both synthetic messages and corresponding labels, for training and validation.
This approach is particularly valuable in low-resource domains or emerging tasks where authentic data is scarce or difficult to collect.
It is especially useful in situations where creating authentic datasets is costly, time-consuming, or ethically challenging, such as annotating harmful or sensitive content or working with vulnerable populations.
The effective

%Commented for indusrty track \subsubsection{Scenario 4: Data Augmentation with Synthetic Data}
%This scenario assumes the availability of a moderate amount of gold-labeled data for training and validation, which may be insufficient to achieve optimal performance. To augment the dataset, we use an LLM to generate additional synthetic data, which is then combined with the gold-labeled data during training and validation. The experiment systematically varies the ratio of synthetic-to-gold data to evaluate its impact on model performance. This scenario explores how LLMs can supplement authentic data, striking a balance between scalability and accuracy.


\paragraph{Scenario 4: Synthetic Labels for Unlabeled Data} \label{s:m:sc4}

This scenario addresses the common situation where resources for manual annotation are limited. Here, gold-standard labeled data is available only for the test set, while a significant amount of unlabeled authentic data is available for training and validation.
%This scenario demonstrates the utility of LLMs in resource-constrained settings, enabling cost-effective dataset creation from unannotated corpora.
To utilize the unlabeled data, we label it using the best prompting strategy (GE or GF) from scenario~2.

% \subsubsection{Summary of Scenarios}
% Table~\ref{t:scenario-summary} presents an overview of the data used in the baseline system and each scenario, specifying the datasets utilized for training, validation, and testing. For Scenario 2, where no classifier is trained and the LLM is used directly as a classifier, only the test set is included.
% \begin{table}
%     \centering
%     \small % Reduce font size for the table (optional)
%     \begin{tabularx}{\columnwidth}{|X|X|X|X|}
%         \hline
%         \textbf{Scenario} & \textbf{Train} & \textbf{Validation} & \textbf{Test} \\
%         \hline
%          1 & $X_{\text{tr}}, y_{\text{tr}}^{\text{gold}}$ & $X_{\text{val}}, y_{\text{val}}^{\text{gold}}$ &  $X_{\text{test}}, y_{\text{test}}^{\text{gold}}$ \\
%         \hline
%            2 & - & - & $X_{\text{test}}, y_{\text{test}}^{\text{gold}}$ \\
%         \hline
%           3 & $(X, y)_{\text{tr}}^{\text{llm}}$ & $(X, y)_{\text{val}}^{\text{llm}}$ & $X_{\text{test}}, y_{\text{test}}^{\text{gold}}$ \\
%         \hline
%          4 & $X_{\text{tr}}, y_{\text{tr}}^{\text{llm}}$ & $X_{\text{val}}, y_{\text{val}}^{\text{llm}}$ &  $X_{\text{test}}, y_{\text{test}}^{\text{gold}}$ \\ \hline
       
%         % 4 & $X_{\text{tr}}, y_{\text{tr}}^{\text{llm}} + X_{\text{tr}}, y_{\text{tr}}^{\text{gold}}$ & $X_{\text{val}}, y_{\text{val}}^{\text{llm}}+ X_{\text{val}}, y_{\text{val}}^{\text{gold}}$ & $X_{\text{test}}, y_{\text{test}}^{\text{gold}}$ \\ \hline
%         % 4 & $(X, y)_{\text{tr}}^{\text{llm}} + X_{\text{tr}}, y_{\text{tr}}^{\text{gold}}$ & $(X, y)_{\text{val}}^{\text{llm}}+ X_{\text{val}}, y_{\text{val}}^{\text{gold}}$ & $X_{\text{test}}, y_{\text{test}}^{\text{gold}}$ \\ \hline
%     \end{tabularx}
%     % ACL style has the caption below the table or figure
%     \caption{Overview of data used in each scenario}
% \label{t:scenario-summary} 
% \end{table}





% \subsection{Intrinsic Evaluation Metrics}

% Intrinsic evaluation examines the inherent qualities of datasets, enabling the assessment of linguistic diversity, emotional tone, and conversational structure independently from task-specific performance. For our CB detection task, we utilize \textbf{four} categories of intrinsic metrics to compare the authentic dataset with LLM-generated synthetic data. These categories are: 1) lexical and linguistic characteristics, %including metrics such as Mean Words per Message, Mean Word Length, and Type-Token Ratio; 
% 2) content and CB indicators, 
% %such as rate of Harmful Messages, Bully Messages, Victim Messages, and Toxicity; 
% 3) sentiment and emotional tone, 
% %which classifies messages into negative, positive, or neutral; 
% and 4) dialogue act distribution.
% %categorizing messages into types such as Question, Statement, Greeting, Accept/Reject, and Other. 
% These categories are critical for understanding the fundamental differences between authentic and synthetic data in the context of CB detection, as they provide insight into how well the synthetic data replicates the linguistic, emotional, and conversational behaviors that are typically present in real-world online interactions.

% To ensure a fair comparison between the authentic and synthetic datasets, we first normalize both dataset by employing pre-processing techniques such as tokenization using NLTK \cite{loper-bird-2002-nltk} and punctuation handling. Additionally, data is segmented into equal-sized token slices to account for metrics that are influenced by corpus size.

% Sentiment scores are measured using VADER \cite{hutto2014vader}, a sentiment analysis tool optimized for short social media texts. Dialogue acts are classified using a Naive Bayes model trained on the NLTK \texttt{nps-chat} corpus,
% following \newcite[Chp.~6, Sec.~2.2]{bird2009natural}.\footnote{
%     While no citation is provided by \newcite{bird2009natural}, the source
%     of this corpus seems to be
%     \newcite{forsyth-martell-2007-lexical,forsyth-etal-2010-nps}.
% }
%



% Natural Language Processing with Python, by Steven Bird, Ewan Klein and Edward Loper
% Chapter 6, section 2.2 "Identifying Dialogue Act Types"
% refers to Chapter 2, section 1.2 "Web and Chat Text", for the
% NPS Chat Corpus but provides no source or citation.
%  
% An unrelated 2011 paper cites an "NPS Chat Corpus of North American English chat
% conversations (Forsyth and Martell 2007)".
%   * Forsyth, Eric. M. and Craig H. Martell (2007), Lexical and discourse analysis
%     of online chat dialog, Proceedings of the First IEEE International Conference
%     on Semantic Computing (ICSC) 2007, pp. 19–26.
%   * data collected in 2006
%   * approximately 500,000 chat posts gathered from various online services
%   * 10,567 posts tagged in Release 1.0
%   * available on http://faculty.nps.edu/cmartell/NPSChat.htm (page no longer
%     exists but is archived, e.g. on
%     http://web.archive.org/web/20190510121556/http://faculty.nps.edu/cmartell/NPSChat.htm
%        - "If you want just the data, you can get it through the Linguistic Data
%          Consortium.  It is catalog number LDC2010T05."
%        - This page asked for the 2007 paper above to be cited "when referring to
%          the NPS Chat Corpus".
%
% There is a 2010 thesis from Naval Postgraduate School, Monterey, California, by
% J. R. Hitt entitled "Implementation and Performance exploration of a cross-genre
% part of speech tagging methodology to determine dialog act tags in the chat
% domain".
%   * credits Lin and Forsyth
%


% Type-Token Ratio (TTR), which is calculated by dividing the number of unique words by the total tokens in fixed-size slices, serves as a normalized measure of vocabulary diversity. Toxicity scores, which represent the ratio of messages containing profanity, are derived using a publicly available profanity list \cite{surge2023profanity}.


\subsection{Evaluation Metrics}

We choose accuracy of label prediction for development decisions and reporting since the labels are reasonably balanced in the authentic test data with 30.3\% items labeled with the minority
label.\footnote{In the appendix, we further report macro average F1 scores that are also widely used in the area of harmful content detection.}
In scenarios 1, 3 and 4,
we train BERT\_base\_uncased \cite{devlin-etal-2019-bert}, a 110M parameter transformer model, with a linear classification head
% using
% the HuggingFace transformers library \cite{wolf-etal-2020-huggingface}
to detect harm, assigning binary labels to text messages.
To address noise from randomness in training, we train at least 45 models for each setting and report average accuracy and standard deviation.
\section{Empirical Evaluation}
\begin{table*}[!ht]
    \centering
    \resizebox{0.88\textwidth}{!}{    
    \begin{tabular}{r|cccccc|cccccc}
        \toprule 
        & \multicolumn{6}{c}{\textbf{LLaVA-1.5-7B}} & \multicolumn{6}{c}{\textbf{LLaVA-1.5-13B}} \\ 
        \cmidrule(lr){2-7}\cmidrule(lr){8-13}
        & \multicolumn{3}{c}{\textbf{MM-SafetyBench}} & \multicolumn{3}{c|}{\textbf{MOSSBench}} & \multicolumn{3}{c}{\textbf{MM-SafetyBench}} & \multicolumn{3}{c}{\textbf{MOSSBench}} \\
        \textbf{Method} & \textbf{DSR}$\uparrow$ & \textbf{RR}$\uparrow$ & \textbf{Avg}$\uparrow$ & \textbf{DSR}$\uparrow$ & \textbf{RR}$\uparrow$ & \textbf{Avg}$\uparrow$ & \textbf{DSR}$\uparrow$ & \textbf{RR}$\uparrow$ & \textbf{Avg}$\uparrow$ & \textbf{DSR}$\uparrow$ & \textbf{RR}$\uparrow$ & \textbf{Avg}$\uparrow$\\
        \midrule
        w/o Defense          & 0.06  & 0.98  & 0.52  & 0.14  & 0.97  & 0.55  & 0.10  & 0.97  & 0.53  & 0.30  & 0.96  & 0.63  \\
        \midrule
        \multicolumn{13}{c}{Baseline} \\
        \midrule
        Responsible          & 0.12  & 0.96  & 0.54  & 0.32  & 0.96  & 0.64  & 0.18  & 0.96  & 0.57  & 0.47  & 0.92  & 0.70  \\
        Policy               & 0.08  & 0.96  & 0.52  & 0.18  & 0.98  & 0.58  & 0.12  & 0.97  & 0.55  & 0.34  & 0.97  & 0.65  \\
        Demonstration        & 0.15  & 0.97  & 0.56  & 0.37  & 0.95  & 0.66  & 0.25  & 0.96  & 0.60  & 0.52  & 0.92  & \textbf{0.72}  \\
        SFT                  & 0.20  & 0.95  & 0.58  & 0.50  & 0.88  & 0.69  & 0.13  & 0.98  & 0.55  & 0.49  & 0.88  & 0.68 \\
        SafeDecoding         & 0.08  & 0.97  & 0.53  & 0.31  & 0.94  & 0.62  & 0.12  & 0.96  & 0.54  & 0.42  & 0.93  & 0.68  \\
        Caption              & 0.09  & 0.98  & 0.53  & 0.21  & 0.98  & 0.60  & 0.12  & 0.97  & 0.55  & 0.27  & 0.94  & 0.60  \\
        Caption (w/o image)  & 0.16  & 0.95  & 0.55  & 0.34  & 0.94  & 0.64  & 0.22  & 0.93  & 0.57  & 0.45  & 0.89  & 0.67 \\
        Intention            & 0.07  & 0.98  & 0.53  & 0.20  & 0.99  & 0.59  & 0.11  & 0.96  & 0.54  & 0.26  & 0.97  & 0.61  \\
        \midrule
        % \multicolumn{13}{c}{} \\
        % \midrule
        \midrule
        \multicolumn{13}{c}{SR++} \\
        \midrule        
        Responsible-Demonstration & 0.18 & 0.95 & 0.57 & 0.40 & 0.94 & 0.67 & 0.29 & 0.96 & 0.62 & 0.58 & 0.85 & \textbf{0.72} \\
        Responsible-Policy & 0.12 & 0.96 & 0.54 & 0.27 & 0.97 & 0.62 & 0.18 & 0.96 & 0.57 & 0.46 & 0.94 & 0.70 \\
        Policy-Demonstration & 0.13 & 0.96 & 0.55 & 0.37 & 0.97 & 0.67 & 0.20 & 0.96 & 0.58 &0.51 & 0.93 & \textbf{0.72}\\
        Responsible-Policy-Demonstration & 0.15 & 0.96 & 0.55 & 0.38 & 0.95 & 0.66 & 0.25 & 0.97 & 0.61 & 0.53 & 0.88 & 0.70\\
        \midrule
        \multicolumn{13}{c}{SR+MO} \\
        \midrule     
        Responsible-SFT & 0.56 & 0.93 & \textbf{0.75} & 0.61 & 0.72 & 0.67 & 0.35 & 0.96 & 0.65 & 0.74 & 0.62 & 0.68 \\
        Responsible-SafeDecoding & 0.30 & 0.96 & 0.63 & 0.54 & 0.87 & \underline{0.70} & 0.23 & 0.96 & 0.59 & 0.63 & 0.79 & 0.71\\
        Demonstration-SFT & 0.60 & 0.90 & \textbf{0.75} & 0.65 & 0.77 & \textbf{0.71} & 0.56 & 0.92 & \textbf{0.74} & 0.67 & 0.70 & 0.68\\
        Demonstration-SafeDecoding & 0.38 & 0.96 & \underline{0.67} & 0.55 & 0.87 & \textbf{0.71} & 0.40 & 0.96 & \underline{0.68} & 0.62 & 0.78 & 0.70\\
        \midrule
        \multicolumn{13}{c}{QR++} \\
        \midrule   
        Caption-Intention & 0.09 & 0.97 & 0.53 & 0.20 & 0.98 & 0.59 & 0.14 & 0.95 & 0.55 & 0.26 & 0.96 & 0.61\\
        % Caption-Intention (w/o image) & 0.18 & 0.96 & 0.57 & 0.32 & 0.95 & 0.64 & 0.25 & 0.92 & 0.59 & 0.45 & 0.92 & 0.68\\
        \midrule
        % \multicolumn{13}{c}{} \\
        % \midrule
        \midrule
        \multicolumn{13}{c}{QR\textbar{}SR} \\
        \midrule   
        Caption-Responsible & 0.34 & 0.96 & 0.65 & 0.53 & 0.79 & 0.66 & 0.33 & 0.96 & 0.65 & 0.50 & 0.82 & 0.66\\
        Intention-Responsible & 0.36 & 0.97 & \underline{0.67} & 0.51 & 0.86 & 0.68 & 0.27 & 0.96 & 0.61 & 0.49 & 0.90 & 0.70\\
        Caption-Responsible (w/o image) & 0.96 & 0.25 & 0.60 & 0.93 & 0.16 & 0.55 & 0.60 & 0.80 & \underline{0.70} & 0.72 & 0.72 & \textbf{0.72}\\
        % Responsible-Intention (w/o image) & 0.99 & 0.06 & 0.52 & 0.95 & 0.17 & 0.56 & 0.61 & 0.81 & 0.71 & 0.68 & 0.77 & 0.72\\
        \midrule
        \multicolumn{13}{c}{QR\textbar{}MO} \\
        \midrule
        Caption-SafeDecoding & 0.20 & 0.96 & 0.58 & 0.39 & 0.88 & 0.64 & 0.33 & 0.94 & 0.63 & 0.40 & 0.90 & 0.65 \\
        Intention-SFT & 0.28 & 0.97 & 0.62 & 0.43 & 0.78 & 0.61 & 0.25 & 0.96 & 0.60 & 0.50 & 0.88 & 0.69\\
        Caption-SafeDecoding (w/o image) & 0.24 & 0.95 & 0.60 & 0.41 & 0.89 & 0.65 & 0.36 & 0.85 & 0.61 & 0.56 & 0.84 & 0.70\\
        \bottomrule
    \end{tabular}}
    \caption{Comparison results of ensemble strategies with the corresponding individual defenses. \textbf{Bold} indicates the best overall performance, while \underline{underlined} highlights the top three methods.} % and the full score is 100\%
    \label{tab:en_inter_results}
\end{table*}


\subsection{Experimental Setup}
We empirically evaluate various defense methods and their ensemble strategies on LLaVA-1.5-7B and LLaVA-1.5-13B~\cite{liu2024visual} to validate their effectiveness in standard settings. Using MM-SafetyBench and MOSSBench datasets, we assess safety and helpfulness by measuring defense success rate (DSR) on harmful queries and response rate (RR) on benign queries. We evaluate 28 defense methods, including system reminders, optimization techniques, query refactoring, and noise injection, as well as inter- and intra-mechanism ensembles. Detailed descriptions of defense methods and experimental setups are provided in Appendix~\ref{sec:defense strategies} and~\ref{sec:experiment_detail}. 
For a broader evaluation, we add more experiments in Appendix~\ref{sec:utility}, ~\ref{sec:diverse_attacks} and~\ref{sec:time}, including evaluation with the MM-Vet dataset for testing the quality of model's response on general queries, tests on JailbreakV-28K for more diverse and complex attack scenarios, and a comparison of inference time for different defense methods.

\subsection{Individual Defense Results}

Table~\ref{tab:indi_results} shows results of individual defense methods across four categories. Most methods, except for noise injection, effectively improve model safety across different models and datasets, as evidenced by increased defense success rates. This aligns with our analysis in Figure~\ref{fig:analysis results} where system reminder, model optimization and query refactoring lead to an overall increase in refusal probabilities. 

\paragraph{Safety shift defenses compromise helpfulness.} System reminder and model optimization methods generally reduce response rates on the benign subset while increasing defense success rates on the harmful subset. This confirms that safety shift tend to compromise helpfulness. This is more pronounced in MOSSBench than MM-SafetyBench due to the more apparent harmfulness and concealed harmlessness in MOSSBench queries.

\paragraph{Harmfulness discrimination defenses mitigate over-defense.} Query refactoring methods, except for Caption (w/o image), generally achieve the highest response rates on the benign subset, particularly for MOSSBench with misleadingly benign queries. This validates that harmfulness discrimination improves the model's ability to distinguish between truly harmful and benign queries. Notably, the removal of images in the Caption (w/o image) significantly reduces response rates for both harmful and benign queries, highlighting the crucial role images play in jailbreaking LVLMs.
% \paragraph{Image matters.} The removal of images in the Caption (w/o image) and Intention (w/o image) defenses leads to significant improvements in DSR compared to their image-included counterparts, underscoring the crucial role that images play in jailbreaking LVLMs.

\paragraph{Multimodal defense is challenging.}
However, all individual defense methods still exhibit limited defense success rates. While larger-scale LVLMs (i.e., LLaVA-1.5-13B) tend to achieve slightly higher success rates, they are also more susceptible to over-defense. This underscores the inherent challenges of jailbreak defense for LVLMs, especially when relying on individual defense methods. 

\subsection{Ensemble Defense Results}
Table~\ref{tab:en_inter_results} provides the empirical evaluation of both inter-mechanism and intra-mechanism ensemble strategies, leading to the following insights:

\paragraph{Ensembles improve safety.} Compared to individual methods, most ensemble strategies effectively enhance safety across both datasets and model sizes, showing increased defense success rates, especially in \textit{SR+MO} and \textit{QR\textbar{}SR} methods.

\paragraph{Inter-mechanism ensembles amplify.} Our evaluation shows most \textit{SR++} and \textit{SR+MO} ensembles improve defense success rates while reducing responses rates, whereas the \textit{QR++} ensemble better maintain responses rates. This confirms that inter-mechanism ensembles can amplify a single defense mechanism. Specifically, safety shift ensembles would further enhance model safety at the expense of helpfulness, while harmfulness discrimination ensemble better preserves helpfulness. Among inter-mechanism ensembles, those combining different types of specific methods (e.g., SR+MO) show a more pronounced amplification effect than those combining the same type (e.g., SR++). 
Notably, the Demonstration-SFT method excels in defense strength, utility, and response rate. Its success comes from combining two strong safety shift defenses, Demonstration and SFT, which complement each other and boost overall performance.

\paragraph{Intra-mechanism ensembles complement.} Compared to inter-mechanism ensembles, most \textit{QR\textbar{}SR} and \textit{QR\textbar{}MO} methods—except those without input images—can simultaneously maintain decent defense success rates and stable response rates,
compared to the undefended model and individual defense methods. This demonstrates that intra-mechanism ensemble can complement each other to achieve a more balanced trade-off. Additionally, the removal of input images offering a most conservative ensemble for multimodal defense while still maintaining certain helpfulness.
% In contrast, the defenses in intra-mechanism ensemble complement each other, strengthening safety while maintaining a stable level of helpfulness.
% In contrast, intra-mechanism ensembles combine the strengths of both mechanisms to achieve a more balanced trade-off. Specifically, \textit{QR\textbar{}SR} and \textit{QR\textbar{}MO} increase the refusal probability for harmful queries, while maintaining or even decreasing the refusal probability for benign queries, thereby improving the model's ability to distinguish between benign and harmful queries. This makes them a better choice for general scenarios where balancing safety and helpfulness is essential. 


\subsection{How Do Fine-tuning Affect Model Safety?}
We examine how different fine-tuning methods impact the safety of LVLMs by training LLaVA-1.5-7B using DPO and SFT with two datasets: SPA-VL~\cite{zhang2024spa} and VLGuard~\cite{zong2024safety}. SPA-VL focuses on safety discussions, while VLGuard emphasizes query rejection. We also test the effect of adding 5000 general instruction-following data from LLaVA.  

Table~\ref{tab:training_dataset_results} shows that DPO with SPA-VL and LLaVA provides a slight safety boost without significantly changing response behavior. In contrast, SFT has a stronger impact, but its effectiveness depends on the dataset. SPA-VL improves safety while maintaining helpfulness, though it may miss some harmful cases. VLGuard, however, makes the model overly defensive, rejecting too many queries. Adding LLaVA data helps balance safety and helpfulness, reducing excessive refusals.  


\begin{table}[ht]
    \centering
    \resizebox{0.49\textwidth}{!}{
    \begin{tabular}{r|cccccc}
        \toprule 
        & \multicolumn{3}{c}{\textbf{MM-SafetyBench}} & \multicolumn{3}{c}{\textbf{MOSSBench}} \\
        \textbf{Method} & \textbf{DSR}$\uparrow$ & \textbf{RR}$\uparrow$ & \textbf{Avg}$\uparrow$ & \textbf{DSR}$\uparrow$ & \textbf{RR}$\uparrow$ & \textbf{Avg}$\uparrow$ \\
        \midrule
        w/o Defense          & 0.06  & 0.98  & 0.52  & 0.14  & 0.97  & 0.55 \\
        \midrule
        \multicolumn{7}{c}{DPO} \\
        \midrule
        \multicolumn{1}{l|}{SPA-VL + LLaVA}          & 0.06  & 0.97  & 0.52  & 0.28  & 0.97  & 0.63  \\
        \midrule
        \multicolumn{7}{c}{SFT} \\
        \midrule
        \multicolumn{1}{l|}{SPA-VL}          & 0.24  & 0.96  & 0.60  & 0.58  & 0.78  & 0.68  \\
        + LLaVA     & 0.20  & 0.95  & 0.58  & 0.50  & 0.88  & 0.69  \\
        \midrule
        \multicolumn{1}{l|}{VLGuard}          & 1.00  & 0.09  & 0.55  & 0.90  & 0.21  & 0.55  \\
        + LLaVA     & 0.97  & 0.43  & 0.70  & 0.76  & 0.58  & 0.67  \\
        \bottomrule
    \end{tabular}}
    \caption{Comparison of varying fine-tuning settings.} % and the full score is 100\%
    \label{tab:training_dataset_results}
\end{table}

% In this work, we propose WildLong, a novel framework for synthesizing diverse, scalable, and realistic instruction-response datasets designed for long-context tasks. Our approach addresses key challenges in dataset creation by leveraging meta-information extraction from real-world user queries, graph-based modeling of co-occurrence relationships, and adaptive instruction-response generation.
% WildLong is built on the principles of diversity, scalability, and realism, enabling it to support complex reasoning tasks such as cross-document comparison, and aggregation, which are essential for real-world applications. By integrating meta-information into the data generation process and systematically exploring new combinations through graph-based modeling, WildLong generates diverse datasets that reflect the complexity of extended contexts.
% Experimental results demonstrate that WildLong significantly improves long-context task performance, surpassing other open-source long-context-optimized models across multiple benchmarks. Importantly, this improvement is achieved without requiring supplementary short-context instruction tuning, highlighting the robustness and generalizability of our approach.
% The success of WildLong highlights the potential of structured, meta-information-driven data synthesis to enhance the capabilities of LLMs for complex, real-world tasks. By addressing the critical gaps in long-context dataset diversity and quality, WildLong sets a new standard for long-context instruction tuning and paves the way for further advancements in equipping LLMs to tackle the challenges of extended-context reasoning.
% We propose WildLong, a framework for synthesizing diverse, scalable, and realistic instruction-response datasets for long-context tasks. By leveraging meta-information extraction, graph-based modeling, and adaptive instruction generation, WildLong generates long-context instruction-tuning data with real-world complexity.
% Experiments show improved long-context task performance while retaining short-context performance without additional short-context fine-tuning, demonstrating its robustness and generalizability. We hope WildLong provides insights into generalizing instruction tuning and inspires further advancements in long-context reasoning for LLMs.
We propose WildLong, a framework for synthesizing diverse, scalable, and realistic instruction-response datasets for long-context tasks. 
It integrates meta-information extraction to ensure realistic complexity, graph-based modeling for systematic instruction expansion, and adaptive instruction generation for enhanced contextual relevance.
Our fine-tuned models consistently outperform baselines and maintain short-context performance without mixing short-context data. Notably, our finetuned Llama-3.1-8B model surpasses most open-source long-context models on Longbench-Chat and demonstrates competitive performances with even larger models across benchmarks.
WildLong enables the synthesis of instruction-tuning data that produces robust models capable of handling diverse long-context tasks. Extending beyond synthetic QA and summarization, it bridges the gap to more complex, realistic challenges, advancing the effectiveness of long-context LLMs.
We hope WildLong provides insights into generalizing synthetic data and inspires further progress in long-context reasoning for LLMs.

\newpage

\bibliography{neurips_2024}
\bibliographystyle{neurips_2024}



%%%%%%%%%%%%%%%%%%%%%%%%%%%%%%%%%%%%%%%%%%%%%%%%%%%%%%%%%%%%

\appendix

\bibliographystyle{ACM-Reference-Format}
\bibliography{reference}

\appendix

\newpage

\section{Appendix A: Researchers Prompt Examples}
\label{appendix:a}

Below we provided researchers prompts examples, age and research experience (Exp), grouped by education level.

\begin{longtable}{|p{1cm}|p{2cm}|p{10cm}|}
\hline
\textbf{PID} & \textbf{Description} & \textbf{Prompt Example} \\ \hline
\multicolumn{3}{|c|}{\textbf{Master Students}} \\ \hline
P3 & Yr-1 Master \newline Age: 23 \newline Exp: 0.5 Yrs & I'm [anonymized] with 2 years of design experience, I fuse data, user insights, and business objectives to craft empowering user experiences, one interaction at a time. \newline I am interested in VR/AR and accessibility. I want to build a portfolio website focused on research projects. This website's color is based on orange and blue. \\ \hline
P8 & Yr-1 Master \newline Age: 23 \newline Exp: 1 Yrs & My name is [anonymized], I'm a first-year Human-Computer Interaction Master student. I studied Computer Science and Psychology, also really like cognitive science and drawing. I especially like comics, so I want my website to have some American comics styles. I want a personal website to display my front-end projects, my drawing works, and my research. Each project category will be a book (Book1, Book2, Book3...), when I click on the book, it will deliver me to the specific project page. \\ \hline
P11 & Yr-1 Master \newline Age: 23 \newline Exp: 0 Yrs & This is my homepage of my personal website. It includes the navigation bar, introduction of myself, and other basic things of the website page. The background color will be black, and the style of the website will be simple but also creative. My name is [anonymized], I used to study computer science, but now I changed into a design major. I think my website can show both of the knowledge about coding and design. The website will include pages for my research, my design, and my coding work, as well as a page for my personal life because I want to show my personality to the interviewer. \\ \hline
P4 & Yr-2 Master \newline Age: 23 \newline Exp: 3 Yrs & I'm [anonymized], a second-year master student of the human-computer interaction program. I studied psychology before and have some research projects related to that. I worked as a UX design intern in a few places. I'm graduating in May 2025 and want to apply for UX designer jobs. \\ \hline
\multicolumn{3}{|c|}{\textbf{Yr-1 PhD Students}} \\ \hline
P1 & Yr-1 PhD \newline Age: 25 \newline Exp: 3 Yrs & My name is [anonymized], and I'm a first-year PhD student at [anonymized] and a practicing speech-language pathologist. My research interests include AI integration when developing communication tools for AAC users. \\ \hline
P2 & Yr-1 PhD \newline Age: 25 \newline Exp: 4 Yrs & My name is [anonymized]. I'm now a first-year PhD student in [anonymized], working with Dr. [anonymized]. I have an education background in both electrical engineering and design. My research now focuses on AI-based assistive technologies, especially personalization systems for blind users. \\ \hline
P7 & Yr-1 PhD \newline Age: 29 \newline Exp: 4 Yrs & My website should showcase my current affiliation and my research publications. I wish it to be of vibrant color but with simplistic design. There should be different tabs to hold various content. \\ \hline
P10 & Yr-1 PhD \newline Age: 24 \newline Exp: 3 Yrs & Name: [anonymized] \newline Academic background: PhD student [anonymized] \newline Research interest: Data science for education using natural language processing tools \newline Personal hobbies: drawing, piano, cooking (pictures of dishes I cooked) \newline Style: minimalism \newline Base color: white \\ \hline
P13 & Yr-1 PhD \newline Age: 27 \newline Exp: 3.5 Yrs & My name is [anonymized], and I am an accessibility and UX researcher. I use he/him pronouns. I want to create an accessibility and data visualization portfolio website. I want to have a dark background with white text. The font size of the text should have high contrast and be very readable. \\ \hline
\multicolumn{3}{|c|}{\textbf{Yr-3/4 PhD Students}} \\ \hline
P5 & Yr-3 PhD \newline Age: 28 \newline Exp: 5.5 Yrs & I am [anonymized], a PhD student starting my third year in [anonymized]. My work is at the intersection of Human-Computer Interaction, Aging and Accessibility, and Personal and Health Informatics. My research focuses on investigating technologies for collecting and sharing personal health information among underrepresented populations, including older adults and people with mild cognitive impairment and dementia. Recently I have been working on supporting older adults in the data labeling process for training their personalized activity trackers. My work informs strategies that engage older adults as end-users in machine learning. \\ \hline
P6 & Yr-3 PhD \newline Age: 31 \newline Exp: 10 Yrs & This is my homepage for a website that I can use to showcase my credentials, blogging, and consulting work. I am [anonymized] and would like to introduce myself as a broadly trained social/behavioral scientist now working at the intersection of metascience and human-computer interaction. \\ \hline
P12 & Yr-3 PhD \newline Age: 29 \newline Exp: 4 Yrs & I want to build a research website showcasing my interests and publications. I am a 3rd year PhD student named [anonymized], my pronouns are [anonymized], and my research interests are broadly in multilingual NLP, human-centered NLP, authorship analysis, and explainability. \\ \hline
P9 & Yr-4 PhD \newline Age: 31 \newline Exp: 8 Yrs & I want the circles to be interlinked like a network and when you click on one I want it to expand and highlight more information and the rest to pull back to the sides. I would want to group them thematically with an overarching team science page. Each bubble you click on opens. The top right about corner would be static. \\ \hline
\end{longtable}

\newpage

\section{Appendix B: Prompts for Agentic Pipeline}
\label{appendix:b}

\subsection{Prompt for PRD generation}

\begin{lstlisting}
Please generate a Product Requirements Document (PRD) targeting the creation of a modern and user-friendly personal website for Junior Researchers based on the following user's sketch (the picture I sent you) and prompt.
User's prompt: ${userPrompt}
In the PRD, specify what images are needed and where they should be placed (e.g., hero image, profile image, etc.) using the format: [term(size)], please use concrete keywords like [(profile-picture)medium] instead of vague descriptions like [image1(small)].
There are 3 keywords for the size (small, medium, large, landscape, or portrait). Remember this only applies to images; for icons, you can just define them without the expected format.
Example: [portfolio-preview(landscape)]`
\end{lstlisting}


\subsection{Prompt for website code generation}

\begin{lstlisting}

You are a design engineer tasked with creating a user interface for junior researcher based on a user's wireframe sketch. Prioritize the user's considerations as design preferences while ensuring the design adheres to these principles:
1. Apply shadows judiciously enough to create depth but not overly done.
2. Use the Gestalt principles (proximity, similarity, continuity, closure, and connectedness) to enhance visual perception and organization.
3. Ensure accessibility, particularly in color choices; use contrasting colors for text, such as white text on suitable background colors, to ensure readability. Feel free to use gradients if they enhance the design's aesthetics and functionality.
4. Maintain consistency across the design.
5. Establish a clear hierarchy to guide the user's eye through the interface.
Additional considerations:
2. Utilize a CSS icon library Font Awesome in your <head> tag to include vector glyph icons.
3. Ensure all elements that can be rounded, such as buttons and containers, have consistent rounded corners to maintain a cohesive and modern visual style.
Based on the following Product Requirements Document (PRD) and User Prompt.
Product Requirements Document (PRD): ${storedPRD}
User's prompt: ${userPrompt}
Please incorporate the following images as specified:
${imageInsertionInstructions}
Please provide your output in HTML, CSS, and JavaScript without any explanations and natural languages(only code),with an emphasis on JavaScript for dynamic user interactions such as clicks and hovers.`;
      
\end{lstlisting}

\subsection{Prompt for code iteration idea}

\begin{lstlisting}

Based on the previously generated code, generate 3-5 ideas to improve the website design:
Previously Generated Code:
${previousCode}
Based on the previous design, please provide optimizations and enhancements focusing on:
1. Visual Consistency: Ensure a cohesive look and feel across the entire interface.
2. Unique Imagery: Suggest diverse and non-repetitive images that align with the theme of each section.
3. Component Refinement: Enhance the details of each UI component, considering:
- Button designs (hover states, shadows, etc.)
- Input field styles and interactions
- Card layouts and information hierarchy
4. Layout Improvements: Propose better ways to organize content for improved readability and user flow.
5. Color Scheme: Refine the color palette to improve contrast and visual appeal.
6. Typography: Suggest improvements in font choices, sizes, and text formatting for better readability.
7. Responsive Design: Ensure the layout adapts well to different screen sizes.
8. Interaction Design: Add subtle animations or transitions to improve user experience.
9. Accessibility: Suggest improvements to make the design more inclusive and easier to use for all users.
10. Performance Optimization: If applicable, propose ways to optimize the code for faster loading and rendering.
Please provide concise, innovative ideas that could enhance the user experience, visual appeal, or functionality of the website. Consider the existing code and suggest improvements or new features.`

\end{lstlisting}




%%%%%%%%%%%%%%%%%%%%%%%%%%%%%%%%%%%%%%%%%%%%%%%%%%%%%%%%%%%%


%%% END INSTRUCTIONS %%%




\end{document}