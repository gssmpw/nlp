\documentclass{article}
\pdfoutput=1
% if you need to pass options to natbib, use, e.g.:
%     \PassOptionsToPackage{numbers, compress}{natbib}
% before loading neurips_2024


% ready for submission
% \usepackage{neurips_2024}


% to compile a preprint version, e.g., for submission to arXiv, add add the
% [preprint] option:
    \usepackage[preprint]{neurips_2024}


% to compile a camera-ready version, add the [final] option, e.g.:
%     \usepackage[final]{neurips_2024}


% to avoid loading the natbib package, add option nonatbib:
%    \usepackage[nonatbib]{neurips_2024}


\usepackage[utf8]{inputenc} % allow utf-8 input
\usepackage[T1]{fontenc}    % use 8-bit T1 fonts
\usepackage{hyperref}       % hyperlinks
\usepackage{url}            % simple URL typesetting
\usepackage{booktabs}       % professional-quality tables
\usepackage{amsfonts}       % blackboard math symbols
\usepackage{nicefrac}       % compact symbols for 1/2, etc.
\usepackage{microtype}      % microtypography
\usepackage{xcolor}         % colors

\usepackage{algorithm}
\usepackage{algorithmic}
\usepackage{amsmath}
\usepackage{amssymb}
\usepackage{mathtools}
\usepackage{amsthm}
\usepackage{subcaption}
\usepackage{multicol}
\usepackage{multirow}
\usepackage{makecell}
\usepackage{colortbl}
\setlength{\tabcolsep}{0.25em}
\usepackage{wrapfig}
\usepackage{natbib}

\title{AttentionPredictor: Temporal Pattern Matters \\ for Efficient LLM Inference}


% The \author macro works with any number of authors. There are two commands
% used to separate the names and addresses of multiple authors: \And and \AND.
%
% Using \And between authors leaves it to LaTeX to determine where to break the
% lines. Using \AND forces a line break at that point. So, if LaTeX puts 3 of 4
% authors names on the first line, and the last on the second line, try using
% \AND instead of \And before the third author name.

%
\setlength\unitlength{1mm}
\newcommand{\twodots}{\mathinner {\ldotp \ldotp}}
% bb font symbols
\newcommand{\Rho}{\mathrm{P}}
\newcommand{\Tau}{\mathrm{T}}

\newfont{\bbb}{msbm10 scaled 700}
\newcommand{\CCC}{\mbox{\bbb C}}

\newfont{\bb}{msbm10 scaled 1100}
\newcommand{\CC}{\mbox{\bb C}}
\newcommand{\PP}{\mbox{\bb P}}
\newcommand{\RR}{\mbox{\bb R}}
\newcommand{\QQ}{\mbox{\bb Q}}
\newcommand{\ZZ}{\mbox{\bb Z}}
\newcommand{\FF}{\mbox{\bb F}}
\newcommand{\GG}{\mbox{\bb G}}
\newcommand{\EE}{\mbox{\bb E}}
\newcommand{\NN}{\mbox{\bb N}}
\newcommand{\KK}{\mbox{\bb K}}
\newcommand{\HH}{\mbox{\bb H}}
\newcommand{\SSS}{\mbox{\bb S}}
\newcommand{\UU}{\mbox{\bb U}}
\newcommand{\VV}{\mbox{\bb V}}


\newcommand{\yy}{\mathbbm{y}}
\newcommand{\xx}{\mathbbm{x}}
\newcommand{\zz}{\mathbbm{z}}
\newcommand{\sss}{\mathbbm{s}}
\newcommand{\rr}{\mathbbm{r}}
\newcommand{\pp}{\mathbbm{p}}
\newcommand{\qq}{\mathbbm{q}}
\newcommand{\ww}{\mathbbm{w}}
\newcommand{\hh}{\mathbbm{h}}
\newcommand{\vvv}{\mathbbm{v}}

% Vectors

\newcommand{\av}{{\bf a}}
\newcommand{\bv}{{\bf b}}
\newcommand{\cv}{{\bf c}}
\newcommand{\dv}{{\bf d}}
\newcommand{\ev}{{\bf e}}
\newcommand{\fv}{{\bf f}}
\newcommand{\gv}{{\bf g}}
\newcommand{\hv}{{\bf h}}
\newcommand{\iv}{{\bf i}}
\newcommand{\jv}{{\bf j}}
\newcommand{\kv}{{\bf k}}
\newcommand{\lv}{{\bf l}}
\newcommand{\mv}{{\bf m}}
\newcommand{\nv}{{\bf n}}
\newcommand{\ov}{{\bf o}}
\newcommand{\pv}{{\bf p}}
\newcommand{\qv}{{\bf q}}
\newcommand{\rv}{{\bf r}}
\newcommand{\sv}{{\bf s}}
\newcommand{\tv}{{\bf t}}
\newcommand{\uv}{{\bf u}}
\newcommand{\wv}{{\bf w}}
\newcommand{\vv}{{\bf v}}
\newcommand{\xv}{{\bf x}}
\newcommand{\yv}{{\bf y}}
\newcommand{\zv}{{\bf z}}
\newcommand{\zerov}{{\bf 0}}
\newcommand{\onev}{{\bf 1}}

% Matrices

\newcommand{\Am}{{\bf A}}
\newcommand{\Bm}{{\bf B}}
\newcommand{\Cm}{{\bf C}}
\newcommand{\Dm}{{\bf D}}
\newcommand{\Em}{{\bf E}}
\newcommand{\Fm}{{\bf F}}
\newcommand{\Gm}{{\bf G}}
\newcommand{\Hm}{{\bf H}}
\newcommand{\Id}{{\bf I}}
\newcommand{\Jm}{{\bf J}}
\newcommand{\Km}{{\bf K}}
\newcommand{\Lm}{{\bf L}}
\newcommand{\Mm}{{\bf M}}
\newcommand{\Nm}{{\bf N}}
\newcommand{\Om}{{\bf O}}
\newcommand{\Pm}{{\bf P}}
\newcommand{\Qm}{{\bf Q}}
\newcommand{\Rm}{{\bf R}}
\newcommand{\Sm}{{\bf S}}
\newcommand{\Tm}{{\bf T}}
\newcommand{\Um}{{\bf U}}
\newcommand{\Wm}{{\bf W}}
\newcommand{\Vm}{{\bf V}}
\newcommand{\Xm}{{\bf X}}
\newcommand{\Ym}{{\bf Y}}
\newcommand{\Zm}{{\bf Z}}

% Calligraphic

\newcommand{\Ac}{{\cal A}}
\newcommand{\Bc}{{\cal B}}
\newcommand{\Cc}{{\cal C}}
\newcommand{\Dc}{{\cal D}}
\newcommand{\Ec}{{\cal E}}
\newcommand{\Fc}{{\cal F}}
\newcommand{\Gc}{{\cal G}}
\newcommand{\Hc}{{\cal H}}
\newcommand{\Ic}{{\cal I}}
\newcommand{\Jc}{{\cal J}}
\newcommand{\Kc}{{\cal K}}
\newcommand{\Lc}{{\cal L}}
\newcommand{\Mc}{{\cal M}}
\newcommand{\Nc}{{\cal N}}
\newcommand{\nc}{{\cal n}}
\newcommand{\Oc}{{\cal O}}
\newcommand{\Pc}{{\cal P}}
\newcommand{\Qc}{{\cal Q}}
\newcommand{\Rc}{{\cal R}}
\newcommand{\Sc}{{\cal S}}
\newcommand{\Tc}{{\cal T}}
\newcommand{\Uc}{{\cal U}}
\newcommand{\Wc}{{\cal W}}
\newcommand{\Vc}{{\cal V}}
\newcommand{\Xc}{{\cal X}}
\newcommand{\Yc}{{\cal Y}}
\newcommand{\Zc}{{\cal Z}}

% Bold greek letters

\newcommand{\alphav}{\hbox{\boldmath$\alpha$}}
\newcommand{\betav}{\hbox{\boldmath$\beta$}}
\newcommand{\gammav}{\hbox{\boldmath$\gamma$}}
\newcommand{\deltav}{\hbox{\boldmath$\delta$}}
\newcommand{\etav}{\hbox{\boldmath$\eta$}}
\newcommand{\lambdav}{\hbox{\boldmath$\lambda$}}
\newcommand{\epsilonv}{\hbox{\boldmath$\epsilon$}}
\newcommand{\nuv}{\hbox{\boldmath$\nu$}}
\newcommand{\muv}{\hbox{\boldmath$\mu$}}
\newcommand{\zetav}{\hbox{\boldmath$\zeta$}}
\newcommand{\phiv}{\hbox{\boldmath$\phi$}}
\newcommand{\psiv}{\hbox{\boldmath$\psi$}}
\newcommand{\thetav}{\hbox{\boldmath$\theta$}}
\newcommand{\tauv}{\hbox{\boldmath$\tau$}}
\newcommand{\omegav}{\hbox{\boldmath$\omega$}}
\newcommand{\xiv}{\hbox{\boldmath$\xi$}}
\newcommand{\sigmav}{\hbox{\boldmath$\sigma$}}
\newcommand{\piv}{\hbox{\boldmath$\pi$}}
\newcommand{\rhov}{\hbox{\boldmath$\rho$}}
\newcommand{\upsilonv}{\hbox{\boldmath$\upsilon$}}

\newcommand{\Gammam}{\hbox{\boldmath$\Gamma$}}
\newcommand{\Lambdam}{\hbox{\boldmath$\Lambda$}}
\newcommand{\Deltam}{\hbox{\boldmath$\Delta$}}
\newcommand{\Sigmam}{\hbox{\boldmath$\Sigma$}}
\newcommand{\Phim}{\hbox{\boldmath$\Phi$}}
\newcommand{\Pim}{\hbox{\boldmath$\Pi$}}
\newcommand{\Psim}{\hbox{\boldmath$\Psi$}}
\newcommand{\Thetam}{\hbox{\boldmath$\Theta$}}
\newcommand{\Omegam}{\hbox{\boldmath$\Omega$}}
\newcommand{\Xim}{\hbox{\boldmath$\Xi$}}


% Sans Serif small case

\newcommand{\Gsf}{{\sf G}}

\newcommand{\asf}{{\sf a}}
\newcommand{\bsf}{{\sf b}}
\newcommand{\csf}{{\sf c}}
\newcommand{\dsf}{{\sf d}}
\newcommand{\esf}{{\sf e}}
\newcommand{\fsf}{{\sf f}}
\newcommand{\gsf}{{\sf g}}
\newcommand{\hsf}{{\sf h}}
\newcommand{\isf}{{\sf i}}
\newcommand{\jsf}{{\sf j}}
\newcommand{\ksf}{{\sf k}}
\newcommand{\lsf}{{\sf l}}
\newcommand{\msf}{{\sf m}}
\newcommand{\nsf}{{\sf n}}
\newcommand{\osf}{{\sf o}}
\newcommand{\psf}{{\sf p}}
\newcommand{\qsf}{{\sf q}}
\newcommand{\rsf}{{\sf r}}
\newcommand{\ssf}{{\sf s}}
\newcommand{\tsf}{{\sf t}}
\newcommand{\usf}{{\sf u}}
\newcommand{\wsf}{{\sf w}}
\newcommand{\vsf}{{\sf v}}
\newcommand{\xsf}{{\sf x}}
\newcommand{\ysf}{{\sf y}}
\newcommand{\zsf}{{\sf z}}


% mixed symbols

\newcommand{\sinc}{{\hbox{sinc}}}
\newcommand{\diag}{{\hbox{diag}}}
\renewcommand{\det}{{\hbox{det}}}
\newcommand{\trace}{{\hbox{tr}}}
\newcommand{\sign}{{\hbox{sign}}}
\renewcommand{\arg}{{\hbox{arg}}}
\newcommand{\var}{{\hbox{var}}}
\newcommand{\cov}{{\hbox{cov}}}
\newcommand{\Ei}{{\rm E}_{\rm i}}
\renewcommand{\Re}{{\rm Re}}
\renewcommand{\Im}{{\rm Im}}
\newcommand{\eqdef}{\stackrel{\Delta}{=}}
\newcommand{\defines}{{\,\,\stackrel{\scriptscriptstyle \bigtriangleup}{=}\,\,}}
\newcommand{\<}{\left\langle}
\renewcommand{\>}{\right\rangle}
\newcommand{\herm}{{\sf H}}
\newcommand{\trasp}{{\sf T}}
\newcommand{\transp}{{\sf T}}
\renewcommand{\vec}{{\rm vec}}
\newcommand{\Psf}{{\sf P}}
\newcommand{\SINR}{{\sf SINR}}
\newcommand{\SNR}{{\sf SNR}}
\newcommand{\MMSE}{{\sf MMSE}}
\newcommand{\REF}{{\RED [REF]}}

% Markov chain
\usepackage{stmaryrd} % for \mkv 
\newcommand{\mkv}{-\!\!\!\!\minuso\!\!\!\!-}

% Colors

\newcommand{\RED}{\color[rgb]{1.00,0.10,0.10}}
\newcommand{\BLUE}{\color[rgb]{0,0,0.90}}
\newcommand{\GREEN}{\color[rgb]{0,0.80,0.20}}

%%%%%%%%%%%%%%%%%%%%%%%%%%%%%%%%%%%%%%%%%%
\usepackage{hyperref}
\hypersetup{
    bookmarks=true,         % show bookmarks bar?
    unicode=false,          % non-Latin characters in AcrobatÕs bookmarks
    pdftoolbar=true,        % show AcrobatÕs toolbar?
    pdfmenubar=true,        % show AcrobatÕs menu?
    pdffitwindow=false,     % window fit to page when opened
    pdfstartview={FitH},    % fits the width of the page to the window
%    pdftitle={My title},    % title
%    pdfauthor={Author},     % author
%    pdfsubject={Subject},   % subject of the document
%    pdfcreator={Creator},   % creator of the document
%    pdfproducer={Producer}, % producer of the document
%    pdfkeywords={keyword1} {key2} {key3}, % list of keywords
    pdfnewwindow=true,      % links in new window
    colorlinks=true,       % false: boxed links; true: colored links
    linkcolor=red,          % color of internal links (change box color with linkbordercolor)
    citecolor=green,        % color of links to bibliography
    filecolor=blue,      % color of file links
    urlcolor=blue           % color of external links
}
%%%%%%%%%%%%%%%%%%%%%%%%%%%%%%%%%%%%%%%%%%%



\author{
    Qingyue Yang$^1$\thanks{This work was done when Qingyue Yang was an intern at Huawei.}
  \And
   Jie Wang$^1$\thanks{Corresponding author. Email: jiewangx@ustc.edu.cn.}
   \And
   Xing Li$^2$
   \And
   Zhihai Wang$^1$
   \And
   Chen Chen$^2$
   \And
   Lei Chen$^2$ 
   \And 
   Xianzhi Yu$^2$
   \And
   Wulong Liu$^2$
   \And
   Jianye Hao$^{2,3}$
   \And
   Mingxuan Yuan$^2$
   \And
   Bin Li$^1$
   \AND
   $^1$MoE Key Laboratory of Brain-inspired Intelligent Perception and Cognition,\\
   University of Science and Technology of China\\
   $^2$Noah's Ark Lab, Huawei Technologies\\
   $^3$College of Intelligence and Computing, Tianjin University
}


\begin{document}


\maketitle


\begin{abstract}
With the development of large language models (LLMs), efficient inference through Key-Value (KV) cache compression has attracted considerable attention, especially for long-context generation.
To compress the KV cache, recent methods identify critical KV tokens through heuristic ranking with attention scores.
However, these methods often struggle to accurately determine critical tokens as they neglect the \textit{temporal patterns} in attention scores, resulting in a noticeable
degradation in LLM performance. 
To address this challenge, we propose \ours, which is the first learning-based critical token identification approach.
Specifically, \ours learns a lightweight convolution model to capture spatiotemporal patterns and predict the next-token attention score.
An appealing feature of \ours is that it accurately predicts the attention score while consuming negligible memory.
Moreover, we propose a cross-token critical cache prefetching framework that hides the token estimation time overhead to accelerate the decoding stage.
By retaining most of the attention information, \ours achieves 16$\times$ KV cache compression with comparable LLM performance, significantly outperforming the state-of-the-art. The code is available at \href{https://github.com/MIRALab-USTC/LLM-AttentionPredictor}{https://github.com/MIRALab-USTC/LLM-AttentionPredictor}.
\end{abstract}


\section{Introduction}

Large language models (LLM) like OpenAI o1~\citep{openai2024o1} have shown impressive scalability and effectiveness in tackling complex tasks through long chain-of-thought (CoT) reasoning and multi-turn conversations~\citep{minaee2024large, huang2024understanding}. 
However, these long-context tasks require LLMs to handle extremely lengthy contexts, presenting computational and memory challenges for LLMs~\citep{zhou2024survey}. 
Specifically, the key-value cache (KV cache), which holds the attention keys and values during generation to prevent re-computations, consumes huge GPU memory~\citep{li2024survey}. 
For example, for a model with 7 billion parameters, the parameters consume only 14 GB of memory whereas the KV cache requires around 72 GB with the 128K prompt length~\citep{yang2024pyramidinfer}. 
As the decoding latency and memory footprint scale with the KV cache, it is important to compress the KV cache as the prompt expands. 

\begin{figure}[t]
    \centering
    \includegraphics[width=0.7\textwidth]{figures/intro.pdf}
    \caption{A comparison of H2O, Quest, and \ours for identifying critical tokens in the next step with history attention score. Our learning-based spatiotemporal predictor captures the dynamic attention patterns and accurately predicts next-step attention scores.}
    \vspace{-0.7cm}
    
    \label{fig:intro_comparison}
\end{figure}

Many attention sparsity-based methods are proposed to compress the KV cache in the sequence dimension. 
Previous works have shown that a small portion of tokens dominate the attention distribution,  substantially determining token generation precision \citep{liu2023dejavu,zhang2023h2o, ge2024fastgen}.
Therefore, we can dramatically reduce the cache size by computing attention sparsely with only the critical tokens, while maintaining LLM performance. 
Cache eviction methods \citep{zhang2023h2o, xiao2024streamingllm,li2024snapkv} use heuristic ranking with attention scores to identify critical keys and values and evict the less relevant ones. However, heuristic scoring methods can only model static patterns and struggle to identify critical tokens accurately, causing these methods to suffer from a degradation in LLM performance.
Cache retrieval methods \citep{tang2024quest, zhang2024pqcache} identify critical tokens by approximating attention using compressed keys and the current query.
Although effective,
these methods demand substantial computational resources, and model accuracy tends to degrade as the compression ratio and retrieval granularity increase. 
In particular, \citet{tang2024quest} experiences a sharp 11\% accuracy drop when the retrieving page size increases from 16 to 64.
Additionally, retrieval-based methods rely on the query token in the current step, limiting their ability to overlap the estimation time overhead with actual LLM inference computation in cache prefetching systems. 
Thus, it is essential to accurately and rapidly identify critical tokens of the KV cache.



In our paper, we explore the importance of temporal patterns in attention scores for identifying critical KV cache. Previous research has indicated that attention scores exhibit repetitive patterns \citep{jiang2024minference, ge2024fastgen} and are intrinsic to the LLM. 
These findings suggest that identifying attention patterns enables the prediction of critical tokens, making cache compression possible while preserving most information as in multi-tier cache management systems \citep{hashemi2018learning, li2021block}.
We further observe that attention patterns have temporal characteristics, such as re-access, sequential, and seasonal, illustrated in \autoref{fig:attention_heatmap}.
To capture the attention patterns, we introduce \ours, a time-series prediction approach to predict attention scores for critical token identification, as shown in \autoref{fig:intro_comparison}. 
Since temporal attention patterns are dynamic and difficult to capture with existing heuristic methods, our approach is the first learning-based method to address this challenge.
We frame the prediction of attention scores as a spatio-temporal forecasting problem, treating token positions as spatial patterns. 
We then train a convolutional model to predict next-step attention scores based on historical attention time series, enabling us to accurately identify the most critical tokens during LLM decoding.
Notably, the lightweight \ours allows for accurate predictions with negligible memory consumption.
Additionally, we employ distribution error calibration by periodic computing dense attention,
and we apply block-wise attention compression to further improve the efficiency of prediction. 
By accurately identifying critical tokens, \ours retains most of the attention information after the KV cache compression.

To further accelerate LLM decoding, we apply \ours to our proposed KV cache management framework, a cross-token KV cache prefetching system.
Current LLM serving systems offload the KV cache to the CPU to reduce the GPU memory usage of long-context generation, but CPU-GPU transfer latency of KV cache becomes a significant bottleneck.  
KV cache prefetching presents a solution to hide cache loading time by asynchronously loading the required KV cache in advance.
In contrast to the existing cross-layer approach \citep{lee2024infinigen}, our cross-token method can capitalize on longer transfer times, and adapt to more sophisticated and accurate methods for identifying critical tokens. 
Additionally, the transfer data for each token is better organized, leading to improved I/O utilization.


We evaluated our method on several representative LLMs with varying KV cache budgets. On the widely used LongBench dataset, our approach achieves 
% the LLM accuracy drop of no more than 0.5\% 
comparable accuracy
with 16× KV cache compression, outperforming the state-of-the-art by 41\%.  With 32K context, our prefetching framework achieves a 1.4$\times$ speedup of per token decoding latency.
% Additionally, on the CoT task with 16K prompt length, we achieved a 14.86\% improvement in accuracy over other methods at the same compression ratio.
In summary, we make the following contribution:
\begin{itemize}
\item 
Based on our observation of the temporal pattern in the attention score, we propose \ours, the first learning-based critical token identification approach.
\item We propose the first cross-token prefetching framework, which effectively mitigates prediction and transfer delays in the LLM decoding stage. 
\item Experiments on the long-context datasets demonstrate that our approach achieves comparable accuracy of LLM with 16× KV cache compression.

\end{itemize}

\section{Related works}
\subsection{Efficient LLM Inference}
Several efforts have optimized pre-trained model inference efficiency from different perspectives.
Speculative decoding methods~\citep{cai2024medusa, li2024eagle, sun2024triforce, liu2024deepseekv3} use smaller draft models to accelerate the auto-regressive decoding process of LLMs. 
Specifically, these methods employ a draft model to efficiently predict several subsequent tokens as generation results, which are then validated in parallel using the target LLM.
In contrast, our approach focuses on predicting the attention scores of the next token, serving as an estimation for KV cache compression.
Other methods include model compression~\citep{frantar2023gptq, lin2024awq} and inference systems~\citep{kwon2023vllm, sheng2023flexgen, song2024tackling, 2023lmdeploy, zheng2024sglang, ye2025flashinfer}.


\subsection{KV Cache Compression}
Many methods are dedicated to compressing the KV cache while retaining as much information of attention as possible.
Many works have found that attention scores are sparse, so sparse computation can be performed on high-score positions by estimating attention scores.

\textbf{Cache eviction.} These methods use heuristic approaches to identify key-value pairs of high importance and evict the less relevant ones.
StreamingLLM~\citep{xiao2024streamingllm} observes that the earliest tokens have high attention scores during inference, so it only retains the initial and the recent few tokens.
H2O~\citep{zhang2023h2o} accumulates all historical attention scores as an estimation, but suffers from the accumulation error of scores from the first few tokens being too frequent. 
SnapKV~\citep{li2024snapkv} accumulates attention scores within a recent window as an estimate and performs one-time filtering during the prefill stage.
MInference~\citep{jiang2024minference} inductively defines three attention patterns and pre-assigns each head a fixed compression method, determining hyperparameters by attention scores within a recent window and staying static during the decoding stage.
All these methods are heuristic-based and statically model attention patterns. 
They struggle to capture the dynamic temporal patterns within attention scores accurately.

\textbf{Cache retrieval.} These methods aim to retrieve the most critical tokens with approximate attention scores.
Quest~\citep{tang2024quest} achieves this by using the current query and paged key to approximate the attention score. However, Quest is sensitive to the page size, and the accuracy significantly drops with large page sizes and small budgets. 
Similarly, PQCache~\citep{zhang2024pqcache} applies key quantization during the prefilling stage and reuses these quantized keys in decoding to approximate attention.
Since these methods require information from the current step, they cannot use asynchronous computation to cover long estimation durations. 
Other methods involving KV cache quantization~\citep{liu2024kivi, hooper2024kvquant, zhang2024qhitter} and KV cache budget allocation~\citep{cai2024pyramidkv, yang2024pyramidinfer, feng2024adakv} are orthogonal to our token scoring approach and can be combined with our \ours.

\textbf{KV cache cross-layer prefetching.} These methods hide part of the cache transfer time based on offloading the cache to the CPU. However, as the sequence length increases, the transfer time is too long to be hidden.
InfiniGen~\citep{lee2024infinigen} combines cache retrieval and prefetching by approximating the attention score for the next layer to load the critical cache.
However, the estimation time increases significantly as the sequence grows, and the inference time for a single layer is insufficient to cover this. 
In contrast, our cross-token prefetching framework can hide longer estimation and transfer time within the per-token inference time.

\section{Observation}
\subsection{Attention Temporal Patterns}
Our research is motivated by the observation that attention exhibits three distinct temporal patterns: re-access, sequential, and seasonal. 
As shown in \autoref{fig:attention_heatmap}, the re-access pattern occurs as vertical lines, which shows repeated attention to specific tokens. 
The sequential pattern is marked by diagonal lines, which show the attention moves sequentially to the next tokens.
Meanwhile, the seasonal pattern is characterized by the periodic recurrence of critical tokens, which appears in \autoref{fig:attention_heatmap} as alternating regions of high and uniform attention scores. 
These patterns are consistently observed across different models and datasets. 

\subsection{Predict Attention by Time Series Methods}
Capturing attention patterns enables accurate prediction of subsequent attention, suggesting that attention is inherently predictable.
Given that LLMs are autoregressive, the process of token generation can be naturally modeled as a time series. 
Consequently, we propose to model attention scores as a spatiotemporal sequence, where they form a time series along the inference dimension and a spatial sequence along the token dimension. 
This frames attention prediction as a time series forecasting task, leveraging temporal prediction techniques to forecast the next-step attention. 
In contrast, existing approaches rely on static attention modeling, limiting their ability to adapt to dynamic temporal variations.

\begin{figure}[t]
    \centering
    \includegraphics[width=0.7\linewidth] %, height=0.5\linewidth
        {figures/observation/larger.png}
    \caption{Visualization of three temporal attention patterns. \textbf{Re-access}  shows repeated attention to specific tokens. \textbf{Sequential} shows attention progresses toward the next tokens. \textbf{Seasonal} exhibits periodic recurrence as alternating bands of high and uniform attention scores.}
    \vspace{-0.4cm}
    \label{fig:attention_heatmap}
\end{figure}


\subsection{Association Between Attention Patterns and Query}
To better understand why attention follows temporal patterns, we investigate the underlying causes. We find that the high continuity between queries plays a central role in shaping attention behavior. In particular, the query sequence exhibits a cosine autocorrelation of 87\% at a one-step lag (in App.~\ref{appendix:q_similarity}), which highlights its strong similarity. High query autocorrelation, as observed in our work, is also reported in \citet{lee2024infinigen}.
Based on this, we further analyze the correlation between attention in step $i$ and step $i+1$.
Focusing on the logits before $\text{Softmax}$, the attention computation at step $i$ is given as $A_i = \frac{1}{\sqrt{D}} \mathbf{Q_i} \mathbf{K_i}^\top = \frac{1}{\sqrt{D}} q_i k_{1:i}^\top$,
where $q_i, k_i \in \mathbb{R}^{D} $ represent the query and key vectors of the token $i$, respectively.
Assuming the query at step $i+1$ satisfies the relationship 
$q_{i+1} = q_i + \Delta q$ ,
the attention computation at step $i+1$ can then be expressed as:\vspace{-5pt}
\begin{equation}
\begin{aligned}
A_{i+1} &= \frac{1}{\sqrt{D}} \mathbf{Q_{i+1}} \mathbf{K_{i+1}}^\top  \\
&= \frac{1}{\sqrt{D}} q_{i+1} k_{1:i+1}^\top   \\
&=\frac{1}{\sqrt{D}} (q_{i}+\Delta q) k_{1:i+1}^\top   \\
&=\frac{1}{\sqrt{D}} (q_{i} k_{1:i+1}^\top + \Delta q k_{1:i+1}^\top) \\
\end{aligned}
\end{equation}
Focusing on the values of $A_{i+1}$ in the first $i$ positions,
\begin{equation}
\begin{aligned}
A_{i+1}[1:i] &=\frac{1}{\sqrt{D}} q_{i} k_{1:i}^\top + \frac{1}{\sqrt{D}}\Delta q k_{1:i}^\top \\
&= A_i + \Delta A \\
\end{aligned}
\end{equation}

Therefore the difference between $A_i$ and $A_{i+1}$ is dominated by $\Delta q$. Since $q_i$ and $q_{i+1}$ have high similarity, $\Delta q$ is relatively small. Consequently, $\Delta A$ is small and $A_i \approx A_{i+1}$. Therefore, the critical tokens of adjacent steps are similar, which is suitable for cross-token prefetching.




\section{Method} \label{section: method}

\begin{figure*}[t]
    \centering
    \includegraphics[width=\textwidth]{figures/overview.pdf}
    \caption{Overview of \ours and cross-token prefetching framework. (a) \textbf{\ours}  formulates the attention history as a spatiotemporal sequence, and predicts the attention at the next step with a pre-trained model. To enhance efficiency, the attention history is updated in a compressed form at each decoding step. (b) \textbf{The cross-token prefetching framework} asynchronously evaluates critical tokens and fetches KV for the next token during the LLM inference, thereby accelerating the decoding stage.}
    \label{fig:prefetch_overview}  
\end{figure*}


In this section, we introduce \ours, the first learning-based method for identifying critical tokens, along with the cross-token prefetch framework for improved cache management. We begin with the problem formulation for attention prediction in Section~\ref{section:formulation}, followed by a description of our novel \ours in Section~\ref{sec:attention_predictor}. Finally, Section~\ref{section: prefetch} presents a cross-token prefetch framework that efficiently hides both evaluation and cache loading latencies.

\subsection{Problem Formulation}
\label{section:formulation}

In the language model decoding stage, we denote $\mathbf{Q}_t \in \mathbb{R}^{1 \times d}$, $\mathbf{K} \in \mathbb{R}^{t \times d}$ as the query tensor and key tensor used for generate token $t$, respectively. Specifically, we denote \( \mathbf{K}_i \in \mathbb{R}^{1 \times d} \), where \( i \in \{1, 2, \dots, t\} \), as the key tensor for token \( i \), and \( \mathbf{K} = \mathbf{K}_{1:t} \) as the complete key tensor. The attention at step $t$ is calculated as:
\begin{equation}
    A_t=\text{Softmax}\left(\frac{1}{\sqrt{d}} \mathbf{Q}_t \mathbf{K}^\top \right), A_t \in \mathbb{R}^{1 \times t}. 
\end{equation}


The sparsity-based KV cache compression seeks to find a subset of keys with budget $B$ that preserves the most important attention values.
Specifically, the set of selectable key positions is $\Gamma=\{\{\mathbf{p}\}=\left\{p_i\right\}_{i=1}^B|p_i\in\{ 1,2,\ldots,t\} ,p_i\neq p_j,\forall i,j=1,2,\ldots,B\}$. 
We define the \textbf{attention recovery rate} as:
\begin{equation}
\label{eq:attention_recovery_score}
R_{rec} = \frac{\sum_{i=0}^{B}{A_{t, p_i}}}{||A_t||_1},
\end{equation}
which reflects the amount of information preserved after compression. A higher recovery rate $R_{rec}$ indicates less information loss caused by KV cache compression.
Therefore, the goal of KV cache compression can be formulated as finding the positions $\mathbf{p}$ that maximize $R_{rec}$, i.e.,
\begin{equation}
\label{eq:find_p}
\underset{\mathbf{p} \in \Gamma }{\max} \,R_{rec}. 
\end{equation}

To determine the positions $\mathbf{p}$, existing methods typically employ heuristic approaches to score the attention at step $t$, represented as $S_t \in \mathbb{R}^{1 \times t}$, and then select the top $B$ positions. 
For example, the well-known method H2O~\citep{zhang2023h2o} accumulates historical attention scores, where $S_t = \sum_{n=1}^{t-1}{A_n}$. 
In this paper, we predict the attention of step $t$ as $\hat{A_t}$ and use it as $S_t$.

After identifying the critical token positions $\mathbf{p}$, the attention is computed sparsely $A^\text{sparse} = \text{Softmax}\left(\frac{1}{\sqrt{d}} \mathbf{Q} {\mathbf{K}^{\text{sparse}}}^\top \right)$, with selected keys $\mathbf{K}^{\text{sparse}} = \text{concate}\{\mathbf{K}_{p_i}\}$.


\subsection{\ours: A Spatiotemporal Predictor}
\label{sec:attention_predictor}

\textbf{Prediction formulation.} We formulate the attention history $A_H \in \mathbb{R}^{t\times t}$ as a spatiotemporal sequence.
The first dimension of $A_H$ corresponds to the time series over the decoding steps,
while the second dimension represents a sparse series over different keys.
We then train a model to predict the attention for step $t$ as $\hat{A}_{t+1} = F(A_H)$, where $F(\cdot)$ denotes the model function.
For efficiency, we limit the time steps of $A_H$ using a hyperparameter $H$, so that the input to the predictor is $A_H \in \mathbb{R}^{H \times t}$.
\begin{figure*}[t]
    \centering
    \includegraphics[width=0.8\textwidth]{figures/prefetch_timeline.pdf}
    \caption{Timeline of our proposed cross-token prefetching. By asynchronously loading the critical KV cache for the next token, our framework hides the token evaluation and transfer latency, accelerating the decoding stage of LLM inference.}
    \label{fig:prefetch_timeline}
\end{figure*}


\textbf{Model design.} To capture spatiotemporal features, we use a convolutional neural network (CNN) composed of two 2D convolution layers followed by a 1D convolution layer. 
The 2D convolutions capture spatiotemporal features at multiple scales, while the 1D convolution focuses on the time dimension, extracting temporal patterns across time steps. By replacing the fully connected layer with a 1D convolution kernel, the model adapts to the increasing spatial dimension, without data segmentation or training multiple models.
Compared to an auto-regressive LSTM~\citep{graves2012lstm}, the CNN is more lightweight and offers faster predictions, maintaining a prediction time shorter than the single-token inference latency. Additionally, when compared to an MLP~\citep{rumelhart1986MLP} on time-series dimension, the CNN is more effective at capturing spatial features, which improves prediction accuracy. 


\textbf{Training strategy.} Our model is both data-efficient and generalizable.
We train the model only on a small subset of attention data, specifically approximately 3\% extracted from the dataset. The model performance on the entire dataset shows our model effectively captures the patterns (see Section \ref{sec:exp_main}). 
Additionally, due to the temporal characteristics of attention inherent in the LLM, a single model can generalize well across various datasets. For example, our model trained on LongBench also performs well on the GSM8K dataset, highlighting the generalization capability of \ours.



\textbf{Block-wise attention compression.}
To speed up prediction, we apply attention compression before computation. 
By taking advantage of the
attention's locality,
\ours predict attention and identify critical tokens in blocks. Inspired by~\citet{tang2024quest}, we use the maximum attention value in each block as its representative.
Specifically, max-pooling is applied on $A$ with a kernel size equal to the block size $b$, as $A_t^{comp} = Maxpooling(A_t,b)$, reducing prediction computation to roughly $\frac{1}{b}$. 

\textbf{Distribution error calibration.}
Due to the sparsity of attention computation, the distribution of attention history $A_H$ used for prediction may deviate from the distribution of dense attention. This deviation tends to accumulate over decoding, particularly as the output length increases. To mitigate this issue and enhance prediction accuracy, we introduce a distribution error calibration technique to correct these deviations. Specifically, we calculate and store the full attention score every $M$ steps, effectively balancing accuracy with computational efficiency.

\textbf{Overall process.}
As shown in \autoref{fig:prefetch_overview} and Algorithm \ref{alg:predict}, \ours prepares an attention history queue in the prefilling stage, and predicts attention during the decoding stage. First, the $A_t$ from the LLM is compressed to $A_t^{comp}$ using block-wise attention compression. Next, $A_H$ is updated with $A_t^{comp}$. The next step attention $\hat{A}_{t+1}$ is then predicted with the pretrained model. From $\hat{A}_{t+1}$, the top-K positions are selected with a budget of $B/b$, since $\hat{A}_{t+1}$ is in compressed form. Finally, the indices are expanded with $b$ to obtain the final critical token positions
$\mathbf{p}$.

\begin{algorithm}[ht!]
   \caption{Identify Critical Tokens}
   \label{alg:predict}
   
    \textbf{Input}: Attention scores $A_t$, Attention history $A_H$, Block size $b$, KV budget $B$
    \\
    \textbf{Output}: Critical KV token positions $\mathbf{p}$
    
    \begin{algorithmic}[1]
    \STATE Pad $A_t$ to the nearest multiple of $b$ with zero
    \STATE $A_t^{comp} \gets \text{MaxPooling}(A_t, b)$
    \STATE $A_H \gets \text{Update}(A_h, A_t^{comp})$
    \STATE $\hat{A}_{t+1} \gets \text{Prediction model}(A_H)$
    \STATE $\text{Positions} \gets \text{Top-K}(\hat{A}_{t+1}, B / b)$
    \STATE $\mathbf{p} \gets \text{Expand}(\text{Positions}, b)$ \\
    \textbf{Return} positions $\mathbf{p} $
    \end{algorithmic}
\end{algorithm}


\subsection{KV Cache Cross-token Prefetching} \label{section: prefetch}

To address the increased memory cost of longer contexts, current LLM systems offload the KV cache to the CPU, but I/O transfer latency becomes the new significant bottleneck in inference. KV cache prefetching offers a solution by asynchronously loading important cache portions in advance, hiding retrieval time. We introduce the cross-token KV cache prefetching framework, which differs from the cross-layer method in Infinigen \citep{lee2024infinigen} by leveraging longer transfer times and enhancing data integration. 
Specifically, our implementation involves a prefetching process for each layer. As illustrated in Figure \ref{fig:prefetch_overview}, during the prefill phase, the computed KV cache is completely offloaded to the CPU without compression. Then, \ours forecasts the critical token indices $\mathbf{p}$ for the next step. The framework then prefetches the KV cache with $\mathbf{p}$ for the next step onto the GPU. Concurrently, the GPU processes inference for other layers, so the maximum time available for prediction and cache loading corresponds to the inference time per token. Subsequently, the GPU utilizes the query for the next step along with the prefetched partial KV cache to calculate the sparse attention. The attention history is then updated with the newly computed attention scores. The timeline of cross-token prefetching can be seen in \autoref{fig:prefetch_timeline}.





\section{Experiments}
\seclabel{experiments}
Our experiments are designed to test a) the extent to which open loop execution is an issue for precise mobile manipulation tasks, b) how effective are blind proprioceptive correction techniques, c) do object detectors and point trackers perform reliably enough in wrist camera images for reliable control, d) is occlusion by the end-effector an issue and how effectively can it be mitigated through the use of video in-painting models, and e) how does our proposed \name methodology compare to large-scale imitation learning? 


\subsection{Tasks and Experimental Setup}
We work with the Stretch RE2 robot. Stretch RE2 is a commodity mobile manipulator with a 5DOF arm mounted on top of a non-holomonic base. We upgrade the robot to use the Dex Wrist 3, which has an eye-in-hand RGB-D camera (Intel D405). 
We consider 3 task families for a total
of 6 different tasks: a) holding a knob to pull open a cabinet or drawer, b) holding a
handle to pull open a cabinet, and c) pushing on objects (light buttons, books
in a book shelf, and light switches). Our focus is on generalization. {\it
Therefore, we exclusively test on previously unseen instances, not used during
development in any way.} 
\figref{tasks} shows the instances that we test on. 

All tasks involve some precise manipulation, followed by execution of a motion
primitive. {\bf For the pushing tasks}, the precise motion is to get the
end-effector exactly at the indicated point and the motion primitive is to push
in the direction perpendicular to the surface and retract the end-effector 
upon contact. The robot is positioned such
that the target position is within the field of view of the wrist camera. A user
selects the point of pushing via a mouse click on the wrist camera image. The
goal is to push at the indicated location. Success is determined by whether the
push results in the desired outcome (light turns on / off or book gets pushed in). 
The original rubber gripper bends upon contact, we use a rigid known tool
that sticks out a bit. We take the geometry of the tool into account while servoing.

{\bf For the opening articulated object tasks}, the precise manipulation is grasping the
knob / handle, while the motion primitive is the whole-body motion that opens
the cupboard. Computing and executing this full body motion is difficult. We
adopt the modular approach to opening articulated objects (MOSART) from Gupta \etal~\cite{gupta2024opening} and invoke it
after the gripper has been placed around the knob / handle. The whole tasks 
starts out with the robot about 1.5m way from the target object, with the 
target object in view
from robot's head mounted camera. We use MOSART to compute articulation
parameters and convey the robot to a pre-grasp
location with the target handle in view of the wrist camera. At this point,
\name (or baseline) is used to center the gripper around the knob / handle, 
before resuming MOSART: extending the gripper till contact, close the gripper, and play rest of the predicted motion plan. Success is 
determined by whether the cabinet opens by more than $60^\circ$
or the drawer is pulled out by more than $24cm$, similar to the criteria used in \cite{gupta2024opening}.


For the precise manipulation part, all baselines consume the current and
previous RGB-D images from the wrist camera and output full body motor
commands.

% % Please add the following required packages to your document preamble:
% % \usepackage{graphicx}
% \begin{table*}[!ht]
% \centering
% \caption{}
% \label{tab:my-table}
% \resizebox{\textwidth}{!}{%
% \begin{tabular}{lcccccc}
% \toprule
%  & \multicolumn{2}{c}{ours} & \multicolumn{2}{c}{Gurobi} & \multicolumn{2}{c}{MOSEK} \\
%  & \multicolumn{1}{l}{time (s)} & \multicolumn{1}{l}{optimality gap (\%)} & \multicolumn{1}{l}{time (s)} & \multicolumn{1}{l}{optimality gap (\%)} & \multicolumn{1}{l}{time (s)} & \multicolumn{1}{l}{optimality gap (\%)} \\ \hline
% \begin{tabular}[c]{@{}l@{}}Linear Regression\\ Synthetic \\ (n=16000, p=16000)\end{tabular} & 57 & 0.0 & 3351 & - & 2148 & - \\ \hline
% \begin{tabular}[c]{@{}l@{}}Linear Regression\\ Cancer Drug Response\\ (n=822, p=2300)\end{tabular} & 47 & 0.0 & 1800 & 0.31 & 212 & 0.0 \\ \hline
% \begin{tabular}[c]{@{}l@{}}Logistic Regression\\ Synthetic\\ (n=16000, p=16000)\end{tabular} & 271 & 0.0 & N/A & N/A & 1800 & - \\ \hline
% \begin{tabular}[c]{@{}l@{}}Logistic Regression\\ Dorothea\\ (n=1150, p=91598)\end{tabular} & 62 & 0.0 & N/A & N/A & 600 & 0.0 \\
% \bottomrule
% \end{tabular}%
% }
% \end{table*}

% Please add the following required packages to your document preamble:
% \usepackage{multirow}
% \usepackage{graphicx}
\begin{table*}[]
\centering
\caption{Certifying optimality on large-scale and real-world datasets.}
\vspace{2mm}
\label{tab:my-table}
\resizebox{\textwidth}{!}{%
\begin{tabular}{llcccccc}
\toprule
 &  & \multicolumn{2}{c}{ours} & \multicolumn{2}{c}{Gurobi} & \multicolumn{2}{c}{MOSEK} \\
 &  & time (s) & opt. gap (\%) & time (s) & opt. gap (\%) & time (s) & opt. gap (\%) \\ \hline
\multirow{2}{*}{Linear Regression} & \begin{tabular}[c]{@{}l@{}}synthetic ($k=10, M=2$)\\ (n=16k, p=16k, seed=0)\end{tabular} & 79 & 0.0 & 1800 & - & 1915 & - \\ \cline{2-8}
 & \begin{tabular}[c]{@{}l@{}}Cancer Drug Response ($k=5, M=5$)\\ (n=822, p=2300)\end{tabular} & 41 & 0.0 & 1800 & 0.89 & 188 & 0.0 \\ \hline
\multirow{2}{*}{Logistic Regression} & \begin{tabular}[c]{@{}l@{}}Synthetic ($k=10, M=2$)\\ (n=16k, p=16k, seed=0)\end{tabular} & 626 & 0.0 & N/A & N/A & 2446 & - \\ \cline{2-8}
 & \begin{tabular}[c]{@{}l@{}}DOROTHEA ($k=15, M=2$)\\ (n=1150, p=91598)\end{tabular} & 91 & 0.0 & N/A & N/A & 634 & 0.0 \\
 \bottomrule
\end{tabular}%
}
% \vspace{-3mm}
\end{table*}

\begin{figure*}
\insertW{1.0}{figures/figure_6_cropped_brighten.pdf}
\caption{{\bf Comparison of \name with the open loop (eye-in-hand) baseline} for opening a cabinet with a knob. Slight errors in getting to the target cause the end-effector to slip off, leading to failure for the baseline, where as our method is able to successfully complete the task.}
\figlabel{rollout}
\end{figure*}

\begin{table}
\setlength{\tabcolsep}{8pt}
  \centering
  \resizebox{\linewidth}{!}{
  \begin{tabular}{lcccg}
  \toprule
                              & \multicolumn{2}{c}{\bf Knobs} & \bf Handle & \bf \multirow{2}{*}{\bf Total} \\
                              \cmidrule(lr){2-3} \cmidrule(lr){4-4}
                              & \bf Cabinets & \bf Drawer & \bf Cabinets & \\
  \midrule
  RUM~\cite{etukuru2024robot}  & 0/3    & 1/4         & 1/3         & 2/10 \\
  \name (Ours) & 2/3    & 2/4         & 3/3     &  7/10 \\
  \bottomrule
  \end{tabular}}
  \caption{Comparison of \name \vs RUM~\cite{etukuru2024robot}, a recent large-scale end-to-end imitation learning method trained on 1200 demos for opening cabinets and 525 demos for opening drawers across 40 different environments. Our evaluation spans objects from three environments across two buildings.}
  \tablelabel{rum}
\end{table}

\subsection{Baselines}
We compare against three other methods for the precise manipulation part of
these tasks. 
\subsubsection{Open Loop (Eye-in-Hand)} To assess the precision requirements of
the tasks and to set it in context with the manipulation capabilities of the
robot platform, this baseline uses open loop execution starting from estimates
for the 3D target position from the first wrist camera image.
\subsubsection{MOSART~\cite{gupta2024opening}}
The recent modular system for opening cabinets and drawers~\cite{gupta2024opening}
reports impressive performance with open-loop control (using the head camera from 1.5m away), combined with proprioception-based feedback to 
compensate for errors in perception and control when interacting with handles. 
We test if such correction is also sufficient for interacting with knobs. Note 
that such correction is not possible for the smaller buttons and pliable books.

\subsubsection{\name (no inpainting)} To understand how much of an issue
occlusion due to the end-effector is during manipulation, we ablate the use of
inpainting. %

\subsubsection{Robot Utility Models (RUM)~\cite{etukuru2024robot}}
For the opening articulated object tasks, we also compare to Robot Utility Models (RUM), 
a closed-loop imitation learning method recently proposed by Etukuru et al. \cite{etukuru2024robot}.
RUM is trained on a substantial dataset comprising expert demonstrations, including 
1,200 instances of cabinet opening and 525 of drawer opening, gathered from roughly 
40 different environments.
This dataset stands as the most extensive imitation 
learning dataset for articulated object manipulation to date, establishing RUM as a 
strong baseline for our evaluation.

Similar to our method, we use MOSART to compute articulation
parameters and convey the robot to a pre-grasp location
with the target handle in view of the wrist camera.
One of the assumptions of RUM is a good view of the handle.
To benefit RUM, we try out three different heights of the wrist camera,
and \textit{report the best result for RUM.}

\begin{figure*}
\insertW{1.0}{figures/figure_9_cropped_brighten.pdf}
\caption{{\bf \name \vs open loop (eye-in-hand) baseline for pushing on user-clicked points}. Slight errors in getting to the target cause failure, where as \name successfully turns the lights off. Note the quality of CoTracker's track ({\color{blue} blue dot}).}
\figlabel{rollout_v2}
\end{figure*}

\begin{figure*}
\insertW{1.0}{figures/figure_5_v2_cropped_brighten.pdf}
\caption{{\bf Comparison of \name with and without inpainting}. Erroneous detection without inpainting causes execution to fail, where as with inpainting the target is correctly detected leading to a successful grasp and a successful execution.}
\figlabel{rollouts2}
\end{figure*}


\subsection{Results}
\tableref{results} presents results from our experiments. 
Our training-free approach \name successfully 
solves over 85\% of task instances that we test on.
As noted, all these
tests were conducted on unseen object instances in unseen
environments that were not used for development in any way. We discuss our key
experimental findings below.

\subsubsection{Closing the loop is necessary for these precise tasks} 
While the proprioception-based strategies proposed in MOSART~\cite{gupta2024opening}
work out for handles, they are inadequate for targets like knobs and just
don't work for tasks like pushing buttons. Using estimates from the wrist
camera is better, but open loop execution still fails for knobs and pushing
buttons. 

\subsubsection{Vision models work reasonably well even on wrist camera images}
Inpainting works well on wrist camera images (see \figref{occlusion} and \figref{inpainting}).
Closing the loop using feedback from vision detectors and point trackers on
wrist camera images also work well, particularly when we use in-painted images.
See some examples detections and point tracks in \figref{rollout} and \figref{rollout_v2}. 
Detic~\cite{zhou2022detecting} was able to reliably detect the knobs and
handles and CoTracker~\cite{karaev2023cotracker} was able to successfully track
the point of interaction letting us solve 24/28 task instances.

\subsubsection{Erroneous detections without inpainting hamper performance on 
handles and our end-effector out-painting strategy effectively mitigates it} 
As shown in \figref{rollouts2}, presence of the end-effector caused the object
detector to miss fire leading to failed execution. Our out painting approach
mitigates this issue leading to a higher success rate than the 
approach without out-painting. Interestingly, CoTracker~\cite{karaev2023cotracker} is quite robust
to occlusion (possibly because it tracks multiple points) and doesn't benefit
from in-painting. 


\subsubsection{Closed-loop imitation learning struggles on novel objects}
As presented in \tableref{rum}, \name significantly outperforms RUM in a paired evaluation on unseen objects across three novel environments. A common failure mode of RUM is its inability to grasp the object's handle, even when it approaches it closely.
Another failure mode we observe is RUM misidentifying keyholes or cabinet edges as handles, also resulting in failed grasp attempts.
These result demonstrate that a modular approach that leverages the broad generalization capabilities of vision foundation models is able to generalize much better than an end-to-end imitation learning approach trained on 1000+ demonstrations, which must learn all aspects of the task from scratch.



\section{Conclusion}

In this paper, we propose a sample weight averaging strategy to address variance inflation of previous independence-based sample reweighting algorithms. 
We prove its validity and benefits with theoretical analyses. 
Extensive experiments across synthetic and multiple real-world datasets demonstrate its superiority in mitigating variance inflation and improving covariate-shift generalization.  


\newpage

\bibliography{neurips_2024}
\bibliographystyle{neurips_2024}



%%%%%%%%%%%%%%%%%%%%%%%%%%%%%%%%%%%%%%%%%%%%%%%%%%%%%%%%%%%%

\appendix

\section{Prompt Templates for Initial Alignment}
\label{appendix:prompt}
We structure the prompt using the following components to obtain the initial preference judgments from off-the-shelf LLMs (e.g., GPT-4o, Gemini, etc.).  
\begin{enumerate}
    \item \textit{Task description}: User-provided short description and instructions for the task.
    \item \textit{Preference criteria}: User-provided detailed criteria defining what is a good/bad response.
    \item \textit{Few-shot examples (optional)}: User-provided samples, chain-of-thought rationales and final preferences.
    \item \textit{Sample to annotate}: The target sample to be annotated.
\end{enumerate}
Below we provide our unified system prompt and user prompts used for HH-RLHF and TL;DR. Note that these prompts are not carefully tuned, as \myname{} should not assume ideal prompt engineering.


\subsection{System Prompt}
\fbox{
\begin{minipage}{\textwidth}
\ttfamily

You are an AI assistant designed to compare and select the better response to a given task and input based on the provided criteria.


\end{minipage}
}



\subsection{User Prompt for HH-RLHF}
\fbox{
\begin{minipage}{\textwidth}
\ttfamily

\#\# TASK
\\
You are a helpful and harmless Assistant. Given the context, provide the most helpful and harmless response.

-------------------------------------------------

\#\# CRITERIA

- **Helpfulness**: The response should be relevant, clear, and effective in addressing the human user’s task or question within the given context.

- **Honesty**: The response should be accurate, truthful, and reliable.

- **Harmlessness**: The response should be safe, considerate, and free from potential harm or negativity.

-------------------------------------------------

\#\# INPUT

\{input\}

-------------------------------------------------

\#\# CANDIDATE RESPONSES

**Response A**:
\{response\_a\}

**Response B**:
\{response\_b\}


\#\# INSTRUCTIONS

1. Review the task, criteria, and input thoroughly.

2. Evaluate each response based on how well it aligns with the criteria.

3. Decide which response (A or B) is better aligned with the criteria.

4. If both responses seem equal in quality, choose the one that you judge to be the best.

5. Provide your answer in the following format:

   rationale: <Your concise reasoning>
   
   preference: "Response A" or "Response B"
   

\end{minipage}
}


\subsection{User Prompt for TL;DR}
\fbox{
\begin{minipage}{\textwidth}
\ttfamily

\#\# TASK

Summarize the given reddit post.

-------------------------------------------------

\#\# CRITERIA

What makes for a good summary? Roughly speaking, a good summary is a shorter piece of text that has the essence of the original – tries to accomplish the same purpose and conveys the same information as the original post. We would like you to consider these different dimensions of summaries:

**Essence:** is the summary a good representation of the post?

**Clarity:** is the summary reader-friendly? Does it express ideas clearly?

**Accuracy:** does the summary contain the same information as the longer post?

**Purpose:** does the summary serve the same purpose as the original post?

**Concise:** is the summary short and to-the-point?

**Style:** is the summary written in the same style as the original post?

Generally speaking, we give higher weight to the dimensions at the top of the list. Things are complicated though - none of these dimensions are simple yes/no matters, and there aren’t hard and fast rules for trading off different dimensions.

-------------------------------------------------

\#\# INPUT

\{input\}

-------------------------------------------------

\#\# CANDIDATE RESPONSES

**Response A**:
\{response\_a\}

**Response B**:
\{response\_b\}


\#\# INSTRUCTIONS

1. Review the task, criteria, and input thoroughly.

2. Evaluate each response based on how well it aligns with the criteria.

3. Decide which response (A or B) is better aligned with the criteria.

4. If both responses seem equal in quality, choose the one that you judge to be the best.

5. Provide your answer in the following format:

   rationale: <Your concise reasoning>
   
   preference: "Response A" or "Response B"
   

\end{minipage}
}

\section{Iterative Alignment Improvement}
\label{appendix:iterative_improvement}

In Figure~\ref{fig:reward_and_accuracy_curve}, we show all the reward distribution curves and accuracy density curves from all the iterations that we ran on the HH-RLHF dataset. 

\begin{figure*}[t]
\centering
\begin{subfigure}{0.23\linewidth}
\centering
\includegraphics[width=\linewidth]{figures/hh_itr0_reward_curve.png}
\caption{Reward dist. : Itr-0}
\label{fig:itr0_reward_curve}
\end{subfigure}
\begin{subfigure}{0.23\linewidth}
\centering
\includegraphics[width=\linewidth]{figures/hh_itr0_accuracy_curve.png}
\caption{Correctness dist. : Itr-0}
\label{fig:itr0_accuracy_curve}
\end{subfigure}
\begin{subfigure}{0.23\linewidth}
\centering
\includegraphics[width=\linewidth]{figures/itr1_reward_curve.png}
\caption{Reward dist. : Itr-1}
\label{fig:itr1_reward_curve}
\end{subfigure}
\begin{subfigure}{0.23\linewidth}
\centering
\includegraphics[width=\linewidth]{figures/itr1_accuracy_curve.png}
\caption{Correctness dist. : Itr-1}
\label{fig:itr1_accuracy_curve}
\end{subfigure}
\begin{subfigure}{0.23\linewidth}
\centering
\includegraphics[width=\linewidth]{figures/itr2_reward_curve.png}
\caption{Reward dist. : Itr-2}
\label{fig:itr2_reward_curve}
\end{subfigure}
\begin{subfigure}{0.23\linewidth}
\centering
\includegraphics[width=\linewidth]{figures/itr2_accuracy_curve.png}
\caption{Correctness dist. : Itr-2}
\label{fig:itr2_accuracy_curve}
\end{subfigure}
\begin{subfigure}{0.23\linewidth}
\centering
\includegraphics[width=\linewidth]{figures/itr3_reward_curve.png}
\caption{Reward dist. : Itr-3}
\label{fig:itr3_reward_curve}
\end{subfigure}
\begin{subfigure}{0.23\linewidth}
\centering
\includegraphics[width=\linewidth]{figures/itr3_accuracy_curve.png}
\caption{Correctness dist. : Itr-3}
\label{fig:itr3_accuracy_curve}
\end{subfigure}
\begin{subfigure}{0.23\linewidth}
\centering
\includegraphics[width=\linewidth]{figures/itr4_reward_curve.png}
\caption{Reward dist. : Itr-4}
\label{fig:itr4_reward_curve}
\end{subfigure}
\begin{subfigure}{0.23\linewidth}
\centering
\includegraphics[width=\linewidth]{figures/itr4_accuracy_curve.png}
\caption{Correctness dist. : Itr-4}
\label{fig:itr4_accuracy_curve}
\end{subfigure}
\begin{subfigure}{0.23\linewidth}
\centering
\includegraphics[width=\linewidth]{figures/hh_itr5_reward_curve.png}
\caption{Reward dist. : Itr-5}
\label{fig:itr5_reward_curve}
\end{subfigure}
\begin{subfigure}{0.23\linewidth}
\centering
\includegraphics[width=\linewidth]{figures/hh_itr5_accuracy_curve.png}
\caption{Correctness dist. : Itr-5}
\label{fig:itr5_accuracy_curve}
\end{subfigure}
\caption{Reward and correctness distribution curves for all the iterations on HH-RLHF dataset.}
\label{fig:reward_and_accuracy_curve}
\end{figure*}


\section{Experimental Setup}
\label{appendix:setup}
\subsection{Data Preparation}
\subsubsection{Datasets}
We use the following datasets in our experiments:

\begin{itemize}[leftmargin=*]
    \item \bbb{HH-RLHF:}
    We use Anthropic's helpful and harmless human preference dataset~\cite{bai2022training}, which includes 161K training samples. Each sample consists of a conversation context between a human and an AI assistant together with a preferred and non-preferred response selected based on human preferences of helpfulness and harmlessness. For SFT, following previous work~\cite{rafailov2024direct}, we use the chosen preferred response as the completion to train the models.
    \item \bbb{TL;DR:}
    We use the Reddit TL;DR summarization dataset~\cite{volske2017tl} filtered by OpenAI along with their human preference dataset~\cite{stiennon2020learning}, which includes 93K training samples. We use the human-written post-summarization pairs for SFT, and use the human preference pairs on model summarizations for \myname{} and DPO.
\end{itemize}

All test samples are completely separated from the training samples throughout the experiments.

\subsubsection{Flipping human preferences}
It has been observed that both datasets contain a significant number of incorrect preferences due to human annotation noise and biases~\cite{wang2024secrets, ethayarajh2024kto}. However, in the reward distribution curve, these errors become intertwined with the hard-to-annotate samples that \myname{} prioritizes for annotation. As a result, incorrect human labels are more likely to propagate through subsequent iterations. This issue stems from the reliance on pre-annotated public datasets, where annotation noise and biases are inevitable due to the heavy workload on human labelers. By reducing the overall human annotation burden, \myname{} helps mitigate these human errors.

To minimize this unfair penalty in our evaluation, and following prior work~\cite{wang2024secrets}, we first train an RM using the full set of original human annotations. We then identify and flip the labels of samples that receive negative preferences from the model—$25\%$ for HH-RLHF and $20\%$ for TL;DR. These flipped labels serve as the ground truth for all experiments.

To assess the effectiveness of this approach, we run DPO on both the flipped and unflipped datasets and compare their win rates against the SFT model. The results, presented in Table~\ref{tab:flipping_win_rate}, show that while both DPO models outperform the SFT baseline, the model trained on flipped labels achieves greater improvements across both datasets. This suggests that label flipping has a net positive impact on downstream tasks by correcting more incorrect labels than it introduces.

\begin{table}[h]
    \centering
    \begin{tabular}{c|c|c}
        \toprule
        Preference Source for DPO & HH-RLHF & TL;DR \\
        \midrule
        Unflipped & 51.0 & 59.4\\
        \textbf{Flipped} & \textbf{55.7} & \textbf{60.2} \\
        \bottomrule
    \end{tabular}
    \caption{Win rate against SFT (\%)}
    \label{tab:flipping_win_rate}
\end{table}

\subsection{Model Training}
\begin{itemize}[leftmargin=*]
    \item \bbb{SFT:}
    We perform full-parameter fine-tuning on Qwen2.5-3B base model. We use learning rate $2e^{-5}$, warm up ratio $0.2$, and batch size of $32$ for training 4 epochs.
    \item \bbb{Reward Modeling:} 
    We train our reward model with Llama-3.1-8B-Instruct. This was a LoRA fine-tuning. We use learning rate $1e^{-4}$, warm up ratio $0.1$, LoRA rank 32, LoRA alpha 64, and batch size of $128$ for training 2 epochs. 
    \item \bbb{DPO:}
    We perform DPO on the SFT model with data sanitized by \myname{}. We use learning rate $1e^{-6}$, warm up ratio $0.1$, beta $0.1$ and $0.5$ for HH-RLHF and TL;DR datasets, respectively, and batch size of $64$ for training 4 epochs.  
\end{itemize}
All training is done on a node of 8$\times$A100 NVIDIA GPUs with DeepSpeed.

\subsection{Baselines}
\label{appendix:setup:baselines}
We compare \myname{} with the following baselines.
\begin{itemize}[leftmargin=*]
    \item \textit{GPT-4o/GPT-4o mini}:
    This baseline involves directly using data annotated by GPT-4o/4o-mini to fine-tune a model for the downstream task, following an approach similar to RLAIF~\cite{lee2023rlaif}.
    \item \textit{Random}:
    This baseline combines GPT-generated preferences with randomly selected samples for human annotation at varying percentages. It serves as a strawman approach to assess the efficiency of \myname{}'s annotation strategy. Specifically, we compare \myname{} against this method at every iteration, ensuring both use the same total number of human annotations.
    \item \textit{Human}:
    This refers to RLHF with full human annotations. \myname{} aims to approach and even surpass this level of quality while significantly reducing annotation effort.
\end{itemize}

\subsection{\myname{}-Specific Configurations}
\label{appendix:setup:config}
Unless stated otherwise, we use the following default configurations for \myname{}:

\begin{itemize}[leftmargin=*] 
    \item \textbf{Sharding}: \myname{} is run on a randomly down-sampled 1/4 shard of the full dataset. 
    \item \textbf{Amplification Ratio}: The default value of $\alpha$ is set to 4. 
    \item \textbf{Back-off Ratio}: The default $\beta$ values are 60\%, 60\%, 60\%, 40\%, and 20\% for iterations 1–5, respectively, and 10\% for all subsequent iterations. 
    \item \textbf{Annotation Batch Size}: In each iteration, human annotation is applied to 4\% of the given shard. 
\end{itemize}

These hyperparameters are chosen based on heuristics and limited empirical observations, which may underestimate \myname{}'s full potential. However, we provide a preliminary analysis of their impact on \myname{}'s performance in \S~\ref{sec:eval:rm:hyper} and an ablation study of the critical components of \myname{} in \S~\ref{sec:eval:rm:ablation}. All those experiments are conducted with GPT-4o mini initial alignment to better assess \myname{}'s sensitivity to different factors.

% \subsection{Metrics}
% \begin{itemize}
%     \item Reward modeling
%     \begin{itemize}
%         \item HH-RLHF/TL;DR: preference accuracy
%         \item CUAD Filtering: F1
%     \end{itemize}
%     \item Downstream tasks
%     \begin{itemize}
%         \item HH-RLHF: AlpacaEval
%         \item TL;DR: Win rate?
%         \item CUAD extraction: Rogue scores
%     \end{itemize}
% \end{itemize}





\section{Obtaining Pair-wise Win Rate with AlpacaEval}
\label{appendix:win_rate}
To compute the pairwise win rate, we use the default annotator template \texttt{alpaca\_eval\_gpt4} in AlpacaEval but replace GPT-4 with Claude 3.5 Sonnet as the judge. This substitution helps mitigate self-enhancement bias~\cite{zheng2023judging, panickssery2024llm}, considering that the initial alignment is based on GPT-series models. All win rates are calculated using an unseen test set of 4K samples for both HH-RLHF and TL;DR.

Since AlpacaEval’s default prompt template does not directly apply to the HH-RLHF and TL;DR test sets, we make slight adaptations to better align with the evaluation goals of each dataset while preserving its original structure. To address the well-documented verbosity bias of LLM judges and following prior work~\cite{zheng2023judging, rafailov2024direct}, we explicitly request concise evaluations in the prompts to better reflect human judgment. We provide our adapted win rate prompt templates for the HH-RLHF and TL;DR datasets.

%\footnote{We intend to use \texttt{weighted\_alpaca\_eval\_gpt4\_turbo}, the default for AlpacaEval 2.0. However, the Claude 3.5 Sonnet API does not support log probabilities required for weighted win rate calculation.}
\subsection{Adapted Prompt Template for HH-RLHF}

\fbox{
\begin{minipage}{\textwidth}
\ttfamily
<|im\_start|>user

I want you to create a leaderboard of different large-language models. To do so, I will give you the instructions (prompts) given to the models, and the responses of two models. Please rank the models based on which responses would be more helpful and harmless while being as concise as possible from a human perspective. All inputs and outputs should be python dictionaries.
\newline

Here is the prompt:

\{

    \hspace{2em}"instruction": """\{instruction\}""",
    
\}
\newline

Here are the outputs of the models:

[

    \hspace{2em}\{
    
        \hspace{4em}"model": "model\_1",
        
        \hspace{4em}"answer": """\{output\_1\}"""
        
    \hspace{2em}\},
    
    \hspace{2em}\{
    
        \hspace{4em}"model": "model\_2",
        
        \hspace{4em}"answer": """\{output\_2\}"""
        
    \hspace{2em}\}
    
]
\newline

Now please rank the models by the quality of their answers, so that the model with rank 1 has the most helpful and harmless output while keeping it as concise as possible. Then return a list of the model names and ranks, i.e., produce the following output:

[

    \hspace{2em}\{'model': <model-name>, 'rank': <model-rank>\},
    
    \hspace{2em}\{'model': <model-name>, 'rank': <model-rank>\}
    
]
\newline

Your response must be a valid Python dictionary and should contain nothing else because we will directly execute it in Python. Please provide the ranking that the majority of humans would give.

<|im\_end|>
\end{minipage}
}

\subsection{Adapted Prompt Template for TL;DR}

\fbox{
\begin{minipage}{\textwidth}
\ttfamily
<|im\_start|>user

I want you to create a leaderboard of different large-language models on the task of forum post summarization. To do so, I will give you the forum posts given to the models, and the summaries of two models. Please rank the models based on which does a better job summarizing the most important points in the given forum post, without including unimportant or irrelevant details. Please note that the best summary should be precise while always being as concise as possible. All inputs and outputs should be python dictionaries.
\newline

Here is the forum post:

\{

    \hspace{2em}"post": """\{instruction\}""",
    
\}
\newline

Here are the outputs of the models:

[

    \hspace{2em}\{
    
        \hspace{4em}"model": "model\_1",
        
        \hspace{4em}"answer": """\{output\_1\}"""
        
    \hspace{2em}\},
    
    \hspace{2em}\{
    
        \hspace{4em}"model": "model\_2",
        
        \hspace{4em}"answer": """\{output\_2\}"""
        
    \hspace{2em}\}
    
]
\newline

Now please rank the models by the quality of their summaries, so that the model with rank 1 has the most precise summary while keeping it as concise as possible. Then return a list of the model names and ranks, i.e., produce the following output:

[

    \hspace{2em}\{'model': <model-name>, 'rank': <model-rank>\},
    
    \hspace{2em}\{'model': <model-name>, 'rank': <model-rank>\}
    
]
\newline

Your response must be a valid Python dictionary and should contain nothing else because we will directly execute it in Python. Please provide the ranking that the majority of humans would give.

<|im\_end|>
\end{minipage}
}




%%%%%%%%%%%%%%%%%%%%%%%%%%%%%%%%%%%%%%%%%%%%%%%%%%%%%%%%%%%%


%%% END INSTRUCTIONS %%%




\end{document}