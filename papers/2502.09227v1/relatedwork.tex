\section{Related Works}
In this section, it is explored the state-of-art and the related works in each field relevant to my research.

Concerning weather prediction, traditional methods such as numerical weather prediction models \cite{Kalnay2002}, supplemented by statistical and machine learning models \cite{rasp2018deep,mcgovern2017using,weyn2019can,weyn2020improving,li2024generative}, achieve high accuracy. However, understanding models like neural networks remains challenging. Explainable AI (XAI) techniques are being developed to address this issue. For instance, Labe et al. \cite{labe2023changes} used XAI with ANNs to explain rising summer temperatures in the USA. The idea of my research is, instead, to use ILP since it is able to give explainable results. I am specifically interested in ILASP, which have been already exploited in many study \cite{law2014inductive} and FastLAS \cite{baugh2023neuro,cunnington2023symbolic}. FastLAS, effective in prioritizing specific rules over general ones \cite{policyfastlas}, has also been integrated with neural networks \cite{cunnington2023ffnsl}. Within methods providing explainability, Alviano et al. \cite{alviano2023advancements} and Cabalar et al. \cite{cabalar2014causal} contributed with some methods such as XASP.

In the legal field, Allen \cite{Allen57} pioneered legal document interpretation using symbolic logic. Then, Kowalski and Sergot \cite{DBLP:conf/ijcai/KowalskiS85} classified legal rules, applying logic programming to model British laws. Golshani \cite{DBLP:journals/ijis/Golshani91} emphasized argument construction in automated legal reasoning while a more recent work by Sartor et al. \cite{DBLP:conf/iclp/SartorDBCPK22} utilized Logical English and top-down ASP solvers for legal encoding.

In image recognition, the YOLO network \cite{redmon2016you} excels in real-time object detection, such as applied in autonomous driving \cite{huang2022rd}. Of particular interest are instead, the study by Chen et al. \cite{chen2024multi} and the work by Yang et al. \cite{yang2024research} where YOLO is applyed for cell and cancer detection.

Summing up, the usage of ILP in weather prediction provides significant advantages over other methods. Specifically, ILP allows learning concepts that can be interpretable and adaptable to changing conditions. Moreover, unlike related works in the legal and image recognition domains, which primarily focus on statistical and deep learning approaches, our application of ILP introduces a level of transparency and interpretability that these other techniques lack. For example, in the legal domain, ILP can model complex rules and account for vagueness, while in image recognition, it complements neural networks by providing explainable reasons behind classification results, a feature rarely seen in standard approaches.