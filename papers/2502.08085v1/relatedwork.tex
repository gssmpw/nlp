\section{Related Work}
\subsection{Simulated Patient}
Unlike classroom materials, which are often theoretical and abstract, the simulated patient approach - having professional certified actors roleplay as a patient in various standardized scenarios, allows trainees to engage with lifelike cases, enhancing their ability to recognize and assess pain in real patients \cite{ma2023standardized}. However, employing human actors requires significant resources to compensate, which leads to the rise of virtual simulated patient development. Mixed reality solutions arise as a cheaper and more sustainable solution in which a virtual avatar is programmed with standardized scenarios for training purposes. Most recent notable mixed reality platforms for nursing training, such as GigXR's Holopatient \cite{kang2023effect} or SimX's VR \cite{shah2023orchestrating}, employ VR headsets as the main medium to project the patient avatar to users. However, due to current technology limitations, VR headset is reported to hinder the training experience: most VR head-mounted display supports only single users, and VR glasses also cause discomfort, isolating feelings with extended use. On the other hand, a multi-view 3D display can naturally support group settings while maintaining realistic interaction between the training and the patient avatar.