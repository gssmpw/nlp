

\section{Discussion and system limitations}

Our system employs compact nano-UAVs for both greenhouses and outdoor applications. 
Indoor environments provide ideal conditions with no weather constraints, while outdoor functionality is limited by weather conditions (e.g., wind speeds up to ~\SI{3.5}{\meter/\second}~\cite{9811834} and no rain). 
Though primarily demonstrated for pest detection, the system can also be applied to tasks like dry plant detection, crop monitoring, and counting. 
For Popillia japonica, optimal deployment conditions are sunny days with light winds and temperatures around \SI{29}{\celsius}~\cite{eppo-popillia} that fit the nano-UAVs' ideal operating conditions.

\begin{wrapfigure}{R}{0.48\textwidth}
\centering
\includegraphics[width=0.48\textwidth]{images/pop_sizes.png}
\caption{Example images depicting \textit{Popillia japonica} specimens at different scales, relative to image size. From left to right, bounding boxes occupy, on average, 0.9\%, 23.9\% and 41.9\% of the image's total pixels.}
\label{fig:popillia_sizes}
\end{wrapfigure}


Nano-UAVs are of particular interest for pest detection because they have a lower environmental impact than standard-sized drones that are currently used for this application.
In fact, nano-UAVs produce noise up to 40 dB~\cite{10.1145/3666025.3699337} while their bigger counterpart reaches up to 75 dB~\cite{ijerph18126202}, as such nano-UAVs provide an interesting solution that reduces the impact on the environment.

We test the system across varying hotspot densities (0 to 50) and three crop arrangements (environments 1, 2, and 3), highlighting its advantages over ground robots for early pest detection. 
Performance is influenced more by obstacle density than by the overall environment layout, thanks to the reliance on local path planning for obstacle avoidance.
We now analyze in detail the limitations of our insect detector and of the routing algorithm.


\subsection{Insect detector}

The dataset used in our work contains images gathered online rather than images taken directly from a nano-drone. 
This implies the presence of both pictures where the insects appear in close proximity, and pictures where the insects are much smaller relative to image size. 
On average, target insects cover 17.1\% of an image’s total pixels (8.9\% for \textit{Popillia japonica}, 19.8\% for \textit{Phyllopertha horticola}, 22.5\% for \textit{Cetonia aurata}). 
Figure~\ref{fig:popillia_sizes} provides a visual reference for this. From left to right, bounding boxes occupy, on average, 0.9\%, 23.9\% and 41.9\% of the image's total pixels.
We believe nano-drones can produce similar images, given their small size and capability of flying between vineyards' rows, close to vines. 
However, we point out that a real-world deployment would likely benefit from a model trained directly on images acquired by the drones during exploration.


\subsection{Routing}

\begin{figure}[tb!]
\centering
\includegraphics[width=1.0\columnwidth]{images/oscillation_ver_02.jpg}
\caption{An obstacle covers the entire depth sensor FoV, causing the blockage.}
\label{fig:deadlock}
\end{figure}


Blockages are a typical problem when relying on local planning strategies, they can occur when obstacles cover the entire FoV of the depth sensor, as reported in Figure~\ref{fig:deadlock}.
In fact, in this condition, the local planner provides a solution that passes through the uncertain region of the map.
However, when the UAV starts moving toward the uncertainty region, the depth sensor detects the presence of an obstacle that intersects the new locally planned path.
This causes a new iteration of the local planner, which provides a new solution that belongs to the uncertainty region that will result in a collision as soon as the UAV moves towards it.

The first solution involves maintaining a record of the surroundings based on the current depth measurement. 
This approach uses the same local planning algorithm proposed in this work, namely A*, applied to a local map that includes the current depth measurement along with all previous measurements within a 4$\times$\SI{4}{\meter} area used for local planning. 
This method gradually maps objects larger than the sensor’s FoV, which might otherwise obscure entirely the area ahead of the sensor and cause blockages. 
While this approach reduces the frequency of blockages, it does not eliminate them, as objects exceeding \SI{4}{\meter} in size can still lead to blockages in the current implementation.
Other local solutions to mitigate the blockages issue rely on reinforcement learning and swarm cooperation. 
Still, in our use case, that does not involve communication between UAVs and limits the knowledge of the map to a local instance of 4$\times$\SI{4}{\meter} obstacles that occlude the entire FoV may always result in blockages.
To avoid the blockage issue, a reliable solution is to perform obstacle avoidance on the global map, which cannot be done on our nano-UAVs due to the platform's computational constraints.




