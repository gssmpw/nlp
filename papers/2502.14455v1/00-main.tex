%%
%% This is file `sample-acmsmall.tex',
%% generated with the docstrip utility.
%%
%% The original source files were:
%%
%% samples.dtx  (with options: `all,journal,bibtex,acmsmall')
%% 
%% IMPORTANT NOTICE:
%% 
%% For the copyright see the source file.
%% 
%% Any modified versions of this file must be renamed
%% with new filenames distinct from sample-acmsmall.tex.
%% 
%% For distribution of the original source see the terms
%% for copying and modification in the file samples.dtx.
%% 
%% This generated file may be distributed as long as the
%% original source files, as listed above, are part of the
%% same distribution. (The sources need not necessarily be
%% in the same archive or directory.)
%%
%%
%% Commands for TeXCount
%TC:macro \cite [option:text,text]
%TC:macro \citep [option:text,text]
%TC:macro \citet [option:text,text]
%TC:envir table 0 1
%TC:envir table* 0 1
%TC:envir tabular [ignore] word
%TC:envir displaymath 0 word
%TC:envir math 0 word
%TC:envir comment 0 0
%%
%%
%% The first command in your LaTeX source must be the \documentclass
%% command.
%%
%% For submission and review of your manuscript please change the
%% command to \documentclass[manuscript, screen, review]{acmart}.
%%
%% When submitting camera ready or to TAPS, please change the command
%% to \documentclass[sigconf]{acmart} or whichever template is required
%% for your publication.
%%
%%
\documentclass[acmsmall]{acmart}

%%
%% \BibTeX command to typeset BibTeX logo in the docs
\AtBeginDocument{%
  \providecommand\BibTeX{{%
    Bib\TeX}}}
\usepackage{siunitx}
\usepackage{natbib}
\usepackage{multirow}
\usepackage{wrapfig}
\DeclareSIUnit{\nothing}{\relax}
\newcommand{\firstreviewer}[1]{\textit{\color{blue}~#1}}%LS
\newcommand{\secondreviewer}[1]{\noindent\textit{\color{brown}~#1}}%LS
\newcommand{\thirdreviewer}[1]{\noindent\textit{\color{red}~#1}}%LS
\sisetup{detect-all=true}    % In the abstract, makes numbers and SIUnit in SI{} bold
%% Rights management information.  This information is sent to you
%% when you complete the rights form.  These commands have SAMPLE
%% values in them; it is your responsibility as an author to replace
%% the commands and values with those provided to you when you
%% complete the rights form.
\setcopyright{acmlicensed}
\copyrightyear{2025}
\acmYear{2025}
\acmDOI{10.1145/3719210}


%%
%% These commands are for a JOURNAL article.
\acmJournal{JACM}
\acmVolume{1}
\acmNumber{1}
\acmArticle{1}
\acmMonth{1}

%%
%% Submission ID.
%% Use this when submitting an article to a sponsored event. You'll
%% receive a unique submission ID from the organizers
%% of the event, and this ID should be used as the parameter to this command.
%%\acmSubmissionID{123-A56-BU3}

%%
%% For managing citations, it is recommended to use bibliography
%% files in BibTeX format.
%%
%% You can then either use BibTeX with the ACM-Reference-Format style,
%% or BibLaTeX with the acmnumeric or acmauthoryear sytles, that include
%% support for advanced citation of software artefact from the
%% biblatex-software package, also separately available on CTAN.
%%
%% Look at the sample-*-biblatex.tex files for templates showcasing
%% the biblatex styles.
%%

%%
%% The majority of ACM publications use numbered citations and
%% references.  The command \citestyle{authoryear} switches to the
%% "author year" style.
%%
%% If you are preparing content for an event
%% sponsored by ACM SIGGRAPH, you must use the "author year" style of
%% citations and references.
%% Uncommenting
%% the next command will enable that style.
%%\citestyle{acmauthoryear}


%%
%% end of the preamble, start of the body of the document source.
\usepackage{mathtools}
\DeclarePairedDelimiter\floor{\lfloor}{\rfloor}
\DeclareMathOperator{\arctantwo}{arctan2}

%%% BEGIN ARXIV MARKER %%%
% Command to write a header to say "paper accepted at such conference"
\definecolor{somegray}{rgb}{0.5, 0.5, 0.5}
\newcommand{\darkgrayed}[1]{\textcolor{somegray}{#1}}
\makeatletter
\newcommand*\titleheader[1]{\gdef\@titleheader{#1}}
\AtBeginDocument{%
  \let\st@red@title\@title
  \def\@title{%
    \vskip-2.0em
    \bgroup\normalfont\large\centering\@titleheader\par\egroup
    \vskip0.0em\st@red@title}
}

\makeatother

% Here goes the MESSAGE THAT YOU WANT TO APPEAR above the paper title
\titleheader{\darkgrayed{This paper has been accepted for publication in the 2025 ACM Journal on Autonomous Transportation Systems\\\copyright 2025 ACM.}}
%%% END ARXIV MARKER %%%

\usepackage[firstpage]{draftwatermark} % Apply watermark only to the first page
\SetWatermarkText{This paper has been accepted for publication in the 2025 ACM Journal on Autonomous Transportation Systems\\\copyright 2025 ACM.} % The watermark text
\SetWatermarkScale{0.075} % Adjust size of the watermark
\SetWatermarkColor[gray]{0.85} % Light gray watermark
\SetWatermarkAngle{0} % Rotate the watermark for a diagonal appearance
\SetWatermarkVerCenter{0.1\paperheight} % Move watermark vertically (higher value moves it down)

\begin{document}



%%
%% The "title" command has an optional parameter,
%% allowing the author to define a "short title" to be used in page headers.
\title[An Efficient Ground-aerial Transportation System for Pest Control Enabled by Autonomous AI-based Nano-UAVs]{An Efficient Ground-aerial Transportation System for Pest Control Enabled by AI-based Autonomous Nano-UAVs}

% An Efficient Ground-aerial Transportation System Enabled by AI-based Autonomous Nano-drones for Pest Detection

% Seek and Transport: An Efficient Ground-aerial Transportation System for Ultra-low-power AI-based Pest Detection and Graph-based Obstacle Avoidance on Miniaturized Drones

% An AI-based Efficient Ground-aerial Transportation System for Pest Detection and Graph-based Obstacle Avoidance aboard Miniaturized Drones

% Onboard Graph-based Obstacle Avoidance for Autonomous Nano-drone Pest Detection in Vineyards

%%
%% The "author" command and its associated commands are used to define
%% the authors and their affiliations.
%% Of note is the shared affiliation of the first two authors, and the
%% "authornote" and "authornotemark" commands
%% used to denote shared contribution to the research.
\author{Luca Crupi}
\email{luca.crupi@supsi.ch}
\affiliation{%
  \institution{IDSIA, SUPSI}
  \city{Lugano}
  \country{Switzerland}
}
%\orcid{1234-5678-9012}
\author{Luca Butera}
\email{luca.butera@usi.ch}
\affiliation{%
  \institution{IDSIA, USI}
  \city{Lugano}
  \country{Switzerland}
}
\author{Alberto Ferrante}
\email{alberto.ferrante@usi.ch}
\affiliation{%
  \institution{IDSIA, USI}
  \city{Lugano}
  \country{Switzerland}
}
\author{Alessandro Giusti}
\email{alessandro.giusti@usi.ch}
\affiliation{%
  \institution{IDSIA, SUPSI}
  \city{Lugano}
  \country{Switzerland}
}
\author{Daniele Palossi}
\email{daniele.palossi@supsi.ch}
\affiliation{%
  \institution{IDSIA, SUPSI}
  \city{Lugano}
  \country{Switzerland}
}
\affiliation{%
  \institution{IIS, ETH Z\"urich}
  \city{Z\"urich}
  \country{Switzerland}
}


%%
%% By default, the full list of authors will be used in the page
%% headers. Often, this list is too long, and will overlap
%% other information printed in the page headers. This command allows
%% the author to define a more concise list
%% of authors' names for this purpose.

%%
%% The abstract is a short summary of the work to be presented in the
%% article.

\begin{abstract}


Efficient crop production requires early detection of pest outbreaks and timely treatments; we consider a solution based on a fleet of multiple autonomous miniaturized unmanned aerial vehicles (nano-UAVs) to visually detect pests and a single slower heavy vehicle that visits the detected outbreaks to deliver treatments.
To cope with the extreme limitations aboard nano-UAVs, e.g., low-resolution sensors and sub-\SI{100}{\milli\watt} computational power budget, we design, fine-tune, and optimize a tiny image-based convolutional neural network (CNN) for pest detection. 
Despite the small size of our CNN (i.e., \SI{0.58}{\giga Ops/inference}), on our dataset, it scores a mean average precision (mAP) of 0.79 in detecting harmful bugs, i.e., 14\% lower mAP but 32$\times$ fewer operations than the best-performing CNN in the literature. 
Our CNN runs in real-time at~\SI{6.8}{frame/\second}, requiring~\SI{33}{\milli\watt} on a GWT GAP9 System-on-Chip aboard a Crazyflie nano-UAV. 
Then, to cope with in-field unexpected obstacles, we leverage a global+local path planner based on the A* algorithm. 
The global path planner determines the best route for the nano-UAV to sweep the entire area, while the local one runs up to~\SI{50}{\hertz} aboard our nano-UAV and prevents collision by adjusting the short-distance path. 
Finally, we demonstrate with in-simulator experiments that once a 25 nano-UAVs fleet has combed a 200$\times$\SI{200}{\meter} vineyard, collected information can be used to plan the best path for the tractor, visiting all and only required hotspots. 
In this scenario, our efficient transportation system, compared to a traditional single-ground vehicle performing both inspection and treatment, can save up to~\SI{20}{\hour} working time.

%\firstreviewer{Timely collection of critical information to plan optimized strategies, such as the best path, best scheduling, or best production flow, is certainly key in increasing efficiency in civil and industrial transportation systems. In this context, we address, as an example and without loss of generality, an agribusiness use case.
%Prompt, precise, and efficient treatments are crucial in crop production to prevent pest outbreaks but require timely and fine-grained treatments. In this context, accurate pest detection and optimal planning of routes for slow machinery are of the essence, i.e., our transportation problem.}

%\firstreviewer{Our general solution presents a two-level autonomous transportation system.
%The forefront is represented by a fleet of agile miniaturized autonomous unmanned aerial vehicles (UAVs) as big as the palm of one hand (i.e., nano-UAVs), while a slow and bulky tractor acts as the backbone for the heavy-duty job.
% Thanks to their agility, nano-UAVs can quickly act as probes to comb vast cultivated areas, looking for the first signs of pest outbreaking, leaving the fine-grained treatment to the bulky and slow tractor. 
%Autonomous nano-UAVs can rely only on the limited resources aboard, which means low-resolution sensors and sub-\SI{100}{\milli\watt} power budget for processing.
%To cope with these limitations, we design, fine-tune, and optimize a tiny image-based SSDLite-MobileNetV3 convolutional neural network (CNN) for pest detection.
%Despite the small size of our CNN, on a custom testing dataset of 660 images, it scores a mean average precision of 0.79 in detecting harmful bugs (i.e., Popillia japonica), only 14\% less than the best-performing CNN in the literature that requires 32$\times$ more operations.
%Our CNN runs in real-time (i.e., \SI{6.8}{frame/\second}) and requires only \SI{33}{\milli\watt} on a Greenwaves Technologies GAP9 multi-core System-on-Chip, which we employ aboard a Crazyflie nano-UAV.
%To cope with the unexpected dynamic obstacles in the field, we leverage a global+local path planner based on the A* algorithm.
%The global path planner, executed ahead of the mission, determines the best route for the nano-UAV to sweep the entire area; the local one runs in real-time (up to \SI{50}{\hertz}) aboard our nano-UAV and prevents collision by adjusting the short-distance path.}

%\firstreviewer{We demonstrate with in-simulator experiments that once a 25 nano-UAVs fleet has combed all areas of a 200$\times$\SI{200}{\meter} vineyard, collected information can be used to plan the best path for the tractor, visiting all and only required hotspots.
%In this scenario, our efficient transportation system, compared to a traditional single-ground vehicle performing both inspection and treatment, can save up to \SI{20}{\hour} working time.}
\end{abstract}


%%
%% The code below is generated by the tool at http://dl.acm.org/ccs.cfm.
%% Please copy and paste the code instead of the example below.
%%
\begin{CCSXML}
<ccs2012>
<concept>
<concept_id>10010520.10010553.10010554.10010557</concept_id>
<concept_desc>Computer systems organization~Robotic autonomy</concept_desc>
<concept_significance>500</concept_significance>
</concept>
</ccs2012>
\end{CCSXML}

\ccsdesc[500]{Computer systems organization~Robotic autonomy}

%\ccsdesc[500]{Do Not Use This Code~Generate the Correct Terms for Your Paper}
%\ccsdesc[300]{Do Not Use This Code~Generate the Correct Terms for Your Paper}
%\ccsdesc{Do Not Use This Code~Generate the Correct Terms for Your Paper}
%\ccsdesc[100]{Do Not Use This Code~Generate the Correct Terms for Your Paper}

%%
%% Keywords. The author(s) should pick words that accurately describe
%% the work being presented. Separate the keywords with commas.
\keywords{routing, path planning, UAV, nano-UAV, drones, nano-drones, CNN, neural networks, pest detection, transportation}

%\received{xxx xxx xxx}
%\received[revised]{xxx xxx xxx}
%\received[accepted]{xxx xxx xxx}

%%
%% This command processes the author and affiliation and title
%% information and builds the first part of the formatted document.
\maketitle


\begin{acks}
We thank Hanna Müller, Victor Kartsch, and Luca Benini, authors of~\cite{müller2024gap9shield150gopsaicapableultralow} for providing us with a GAP9Shield prototype.
\end{acks}


% humans are sensitive to the way information is presented.

% introduce framing as the way we address framing. say something about political views and how information is represented.

% in this paper we explore if models show similar sensitivity.

% why is it important/interesting.



% thought - it would be interesting to test it on real world data, but it would be hard to test humans because they come already biased about real world stuff, so we tested artificial.


% LLMs have recently been shown to mimic cognitive biases, typically associated with human behavior~\citep{ malberg2024comprehensive, itzhak-etal-2024-instructed}. This resemblance has significant implications for how we perceive these models and what we can expect from them in real-world interactions and decisionmaking~\citep{eigner2024determinants, echterhoff-etal-2024-cognitive}.

The \textit{framing effect} is a well-known cognitive phenomenon, where different presentations of the same underlying facts affect human perception towards them~\citep{tversky1981framing}.
For example, presenting an economic policy as only creating 50,000 new jobs, versus also reporting that it would cost 2B USD, can dramatically shift public opinion~\cite{sniderman2004structure}. 
%%%%%%%% 图1:  %%%%%%%%%%%%%%%%
\begin{figure}[t]
    \centering
    \includegraphics[width=\columnwidth]{Figs/01.pdf}
    \caption{Performance comparison (Top-1 Acc (\%)) under various open-vocabulary evaluation settings where the video learners except for CLIP are tuned on Kinetics-400~\cite{k400} with frozen text encoders. The satisfying in-context generalizability on UCF101~\cite{UCF101} (a) can be severely affected by static bias when evaluating on out-of-context SCUBA-UCF101~\cite{li2023mitigating} (b) by replacing the video background with other images.}
    \label{fig:teaser}
\end{figure}


Previous research has shown that LLMs exhibit various cognitive biases, including the framing effect~\cite{lore2024strategic,shaikh2024cbeval,malberg2024comprehensive,echterhoff-etal-2024-cognitive}. However, these either rely on synthetic datasets or evaluate LLMs on different data from what humans were tested on. In addition, comparisons between models and humans typically treat human performance as a baseline rather than comparing patterns in human behavior. 
% \gabis{looks good! what do we mean by ``most studies'' or ``rarely'' can we remove those? or we want to say that we don't know of previous work doing both at the same time?}\gili{yeah the main point is that some work has done each separated, but not all of it together. how about now?}

In this work, we evaluate LLMs on real-world data. Rather than measuring model performance in terms of accuracy, we analyze how closely their responses align with human annotations. Furthermore, while previous studies have examined the effect of framing on decision making, we extend this analysis to sentiment analysis, as sentiment perception plays a key explanatory role in decision-making \cite{lerner2015emotion}. 
%Based on this, we argue that examining sentiment shifts in response to reframing can provide deeper insights into the framing effect. \gabis{I don't understand this last claim. Maybe remove and just say we extend to sentiment analysis?}

% Understanding how LLMs respond to framing is crucial, as they are increasingly integrated into real-world applications~\citep{gan2024application, hurlin2024fairness}.
% In some applications, e.g., in virtual companions, framing can be harnessed to produce human-like behavior leading to better engagement.
% In contrast, in other applications, such as financial or legal advice, mitigating the effect of framing can lead to less biased decisions.
% In both cases, a better understanding of the framing effect on LLMs can help develop strategies to mitigate its negative impacts,
% while utilizing its positive aspects. \gabis{$\leftarrow$ reading this again, maybe this isn't the right place for this paragraph. Consider putting in the conclusion? I think that after we said that people have worked on it, we don't necessarily need this here and will shorten the long intro}


% If framing can influence their outputs, this could have significant societal effects,
% from spreading biases in automated decision-making~\citep{ghasemaghaei2024understanding} to reducing public trust in AI-generated content~\citep{afroogh2024trust}. 
% However, framing is not inherently negative -- understanding how it affects LLM outputs can offer valuable insights into both human and machine cognition.
% By systematically investigating the framing effect,


%It is therefore crucial to systematically investigate the framing effect, to better understand and mitigate its impact. \gabis{This paragraph is important - I think that right now it's saying that we don't want models to be influenced by framing (since we want to mitigate its impact, right?) When we talked I think we had a more nuanced position?}




To better understand the framing effect in LLMs in comparison to human behavior,
we introduce the \name{} dataset (Section~\ref{sec:data}), comprising 1,000 statements, constructed through a three-step process, as shown in Figure~\ref{fig:fig1}.
First, we collect a set of real-world statements that express a clear negative or positive sentiment (e.g., ``I won the highest prize'').
%as exemplified in Figure~\ref{fig:fig1} -- ``I won the highest prize'' positive base statement. (2) next,
Second, we \emph{reframe} the text by adding a prefix or suffix with an opposite sentiment (e.g., ``I won the highest prize, \emph{although I lost all my friends on the way}'').
Finally, we collect human annotations by asking different participants
if they consider the reframed statement to be overall positive or negative.
% \gabist{This allows us to quantify the extent of \textit{sentiment shifts}, which is defined as labeling the sentiment aligning with the opposite framing, rather then the base sentiment -- e.g., voting ``negative'' for the statement ``I won the highest prize, although I lost all my friends on the way'', as it aligns with the opposite framing sentiment.}
We choose to annotate Amazon reviews, where sentiment is more robust, compared to e.g., the news domain which introduces confounding variables such as prior political leaning~\cite{druckman2004political}.


%While the implications of framing on sensitive and controversial topics like politics or economics are highly relevant to real-world applications, testing these subjects in a controlled setting is challenging. Such topics can introduce confounding variables, as annotators might rely on their personal beliefs or emotions rather than focusing solely on the framing, particularly when the content is emotionally charged~\cite{druckman2004political}. To balance real-world relevance with experimental reliability, we chose to focus on statements derived from Amazon reviews. These are naturally occurring, sentiment-rich texts that are less likely to trigger strong preexisting biases or emotional reactions. For instance, a review like ``The book was engaging'' can be framed negatively without invoking specific cultural or political associations. 

 In Section~\ref{sec:results}, we evaluate eight state-of-the-art LLMs
 % including \gpt{}~\cite{openai2024gpt4osystemcard}, \llama{}~\cite{dubey2024llama}, \mistral{}~\cite{jiang2023mistral}, \mixtral{}~\cite{mistral2023mixtral}, and \gemma{}~\cite{team2024gemma}, 
on the \name{} dataset and compare them against human annotations. We find  that LLMs are influenced by framing, somewhat similar to human behavior. All models show a \emph{strong} correlation ($r>0.57$) with human behavior.
%All models show a correlation with human responses of more than $0.55$ in Pearson's $r$ \gabis{@Gili check how people report this?}.
Moreover, we find that both humans and LLMs are more influenced by positive reframing rather than negative reframing. We also find that larger models tend to be more correlated with human behavior. Interestingly, \gpt{} shows the lowest correlation with human behavior. This raises questions about how architectural or training differences might influence susceptibility to framing. 
%\gabis{this last finding about \gpt{} stands in opposition to the start of the statement, right? Even though it's probably one of the largest models, it doesn't correlate with humans? If so, better to state this explicitly}

This work contributes to understanding the parallels between LLM and human cognition, offering insights into how cognitive mechanisms such as the framing effect emerge in LLMs.\footnote{\name{} data available at \url{https://huggingface.co/datasets/gililior/WildFrame}\\Code: ~\url{https://github.com/SLAB-NLP/WildFrame-Eval}}

%\gabist{It also raises fundamental philosophical and practical questions -- should LLMs aim to emulate human-like behavior, even when such behavior is susceptible to harmful cognitive biases? or should they strive to deviate from human tendencies to avoid reproducing these pitfalls?}\gabis{$\leftarrow$ also following Itay's comment, maybe this is better in the dicsussion, since we don't address these questions in the paper.} %\gabis{This last statement brings the nuance back, so I think it contradicts the previous parapgraph where we talked about ``mitigating'' the effect of framing. Also, I think it would be nice to discuss this a bit more in depth, maybe in the discussion section.}






\vspace{-0.02in}
We discuss three lines of related work: chart-to-code generation, multi-agent framework, and test-time scaling research.
\vspace{-0.01in}
\subsection{Chart Generation with VLMs}

Chart generation, or chart-to-code generation,  is an emerging task aimed at automatically translating visual representations of charts into corresponding visualization code \cite{shi2024chartmimic, wu2024plot2code}. This task is inherently challenging as it requires both visual understanding and precise code synthesis, often demanding complex reasoning over visual elements.

Recent advances in Vision-Language Models (VLMs) have expanded the capabilities of language models in tackling complex multimodal problem-solving tasks, such as visually-grounded code generation.
% \kw{It's a bit unclear to me what do you mean by multimodal reasoning and how it is related to code generation}
Leading proprietary models, such as GPT-4V~\cite{GPT4V}, Gemini~\cite{Gemini}, and Claude-3~\cite{Claude}, have demonstrated impressive capabilities in understanding complex visual patterns.
% \kw{what do you mean by structured outputs here?} 
The open-source community has contributed models like LLaVA~\cite{xu2024llava-uhd, li2024llavanext-strong}, Qwen-VL~\cite{Qwen-VL}, and DeepSeek-VL~\cite{lu2024deepseekvl}, which provide researchers with greater flexibility for specific applications like chart generation.

Despite these advancements, current VLMs often struggle with accurately interpreting chart structures and faithfully reproducing visualization code. 

\begin{figure*}[ht]
    \centering
    \includegraphics[width=1\linewidth]{figs/method.pdf}
    \caption{Overview of \model{}: A multi-agents system that consists of four specialized agents working in an iterative pipeline: (1) Generation Agent creates initial Python code to reproduce the reference chart, (2) Visual Critique Agent identifies visual discrepancies between the generated and reference charts, (3) Code Critique Agent analyzes the code and provides specific improvement guidelines, and (4) Revision Agent modifies the code based on the critiques. The process iterates until either reaching the verification score or maximum attempts limit.} %Each agent has a clearly defined objective and operates on specific inputs and outputs, enabling systematic improvement of the generated visualization.}
    \vspace{-0.1in}
    \label{fig:method}
\end{figure*}


\subsection{Multi-Agents Framework}

Many researchers have suggested a paradigm shift from single monolithic models to compound systems comprising multiple specialized components~\cite{compound-ai-blog, du2024compositional}. One prominent example is the multi-agent framework.

LLMs-driven multi-agent framework  has been widely explored in various domains, including narrative generation~\cite{huot2024agents}, financial trading~\cite{xiao2024tradingagents}, and cooperative problem-solving~\cite{du2023improving}.

Our work investigates the application of multi-agent framework to the visually-grounded code generation task.



\subsection{Test-Time Scaling} 
% \kw{I feel this subsection is a bit irrelevant }

Inference strategies have been a long-studied topic in the field of language processing. Traditional approaches include greedy decoding \cite{teller2000speech}, beam search \cite{graves2012sequence}, and Best-of-N.

Recent research has explored test-time scaling law for language model inference. For example, \citet{wu2024inference} empirically demonstrated that optimizing test-time compute allocation can significantly enhance problem-solving performance, while \citet{zhang2024scaling} and \citet{snell2024scaling} highlighted that dynamic adjustments in sample allocation can maximize efficiency under compute constraints. Although these studies collectively underscore the promise of test-time scaling for enhancing reasoning performance of LLMs, its existence in other contexts, such as different model types and application to cross-modal generation,  remains under-explored. 

\section{System}
\begin{figure*}[h]
    \centering
    \includegraphics[width=.85\textwidth]{fig/SYSTEM_IMAGE_TEST_FLIPPED.png}
    \caption{HumorSkills System Diagram. Given an image, the system first extracts visual details with a visual language model, then performs visual humor ideation to analyze the image and propose humorous angles. It then generates ten potential conflicts that could be used to extrapolate the image into a relatable experience. The system then generates humor with and without the narratives, for diversity. A separate instance of the LLM trained to rank gen-Z humor ranks all the captions and returns the top five.}
    \Description{HumorSkills System Diagram}
    \label{fig:system}
\end{figure*}

HumorSkills is a system that takes an input image and outputs 5 image captions. 
The architecture has three key steps that mimic human skills needed for humor. \textit{Visual Detail Extraction}, is a step that describes the image in depth in order to make non-obvious observations about it. \textit{Narrative and Conflict Extrapolation} is a step that finds narratives not in the image that could be related to it, to expand the topic of jokes to things that are not just in the image but also analogous to it.  \textit{Fine-tuning} the joke generator with examples of good Gen-Z humor helps the jokes be more relatable to the target audience by using references, slang, topics, and insecurities that resonate with this group.

% first, 
% a \textbf{divergent stage} where the image is analysed and multiple observations, angle, alternative narratives and humorous angles are generated. 
% Second, a \textbf{generation stage} where two types of captions are generated: 1) captions focusing on image content directly 2) captions that bring in an outside narrative to the image, often bringing in outside references. It generates 15 captions of each type. (Figure 1 has examples Of the Content Focused, and Narrative Expanded Captions). Finally, a \textbf{ranking stage} where a separate AI agent selects the top 5 captions

The system generates two types of captions: image-focused captions which common directly on the content in the image, and narrative-driven captions. Variety is important to humor. Humor relies on surprise, and jokes that are too similar start to become more predictable. Additionally, with an infinite set of input images with different subjects and situations, there are more strategies needed to find a humorous angle that fits the content. 

% With caption-based humor, often the humor can be focused on finding something in the image that is inherently interesting. 

% For example, the caption ”little man really thought he could escape bedtime” relies solely on information in the image. However, some images don’t have something funny in and of themselves, and it’s easier to make a joke by bringing in a new unrelated angle. For example, ``the police chasing me when I'm broke and in debt to the tune of \$100,000 for student loans''. Generally, Images with people doing interesting things lend themselves to visual humor because there are many stories one could tell about it. However, for images with only static objects, it's more difficult to tell a story on only the objects, so bringing a new story in is another way to find humor. 

\subsection{AI Humor Generation Walk Through}
Figure \ref{fig:system} contains a visual diagram and example of intermediate outputs when generating captions for an image. We describe each phase and implementation in detail.  
% The main contribution of this paper is the evaluation, rather than the system, but it is still it is important to understand the mechanism used to generate humor.
% Although the individual components of the system are not totally, the combination of features including the

\subsubsection{Visual Detail Extraction}

The first phase of the system’s workflow involves the Visual Detail Extraction component, which utilizes GPT-4o’s vision capabilities to analyze the input image. This system incorporates a prompt that asks for a detailed paragraph that explains the who/what/where of the image, distinguishing between identifying the subject of the image, the main action of the image if it exists, and the background elements of the image. This component is responsible for extracting key visual elements such as objects, human expressions, background settings, and any notable aspects that could serve as the foundation for humor.

For instance, in the demolition site example from the system diagram, the system identifies a large industrial demolition excavator and a person with a hose spraying the demolition site. 

\subsubsection{Visual Humor Ideation}

On top of the visual detail extraction, the system ideates on possible humorous elements from the visual of the image. This incorporates an additional prompt using GPT-4o that intakes the image and asks it to identify and ideate on potential humorous visual elements in the image, whether they are directly humorous elements, such as funny facial expressions, or more analogous elements. For example, for the system diagram image, the system noted the visual contrast of the excavator and person, reminiscent of a David versus Goliath scenario, which provides a foundational metaphor for generating humorous captions. 

\begin{figure*}[b]
    \centering
    \includegraphics[width=.95\textwidth]{fig/Workflow.png}
    \caption{A diagram for how narrative extrapolation works}
    % \Description{}
    \label{fig:systemLines}
\end{figure*}

\subsubsection{Narrative and Conflict Extrapolation}

In this next step, the system generates a narrative and conflict framework by drawing upon common and relatable Gen Z experiences such as work, school, family, social interactions, relationships, and more. 
The system chains together the results of the previous steps, into a new prompt sent to GPT-4o. 
% The system prompts GPT-4o to 
% GPT-4o is utilized by incorporating the text description of the image and potential humorous elements of the image, then being prompted to generate relatable scenarios applicable to the image description from a list of common Gen Z experiences. 
The prompt contains the visual details, the visual humor ideation, and a list of common Gen Z experiences,  and the instruction to "generate narratives that reflect the essence of the image that is set within the framework of the Gen Z experience."
This narrative generation adds depth to the humorous captions by applying relatable themes and conflicts to the visual elements identified earlier.

For instance, our system diagram generates narratives such as “Tackling student loans”, "Group Project Disaster", and “Relationship Issues” based on the image, both of which are common experiences among those who identify as Gen Z. These particular narratives are likely inspired by the imagery of a disaster site, referring to how the effort of paying off student loans, attempting to complete group projects during school, or addressing relationship -- all of which can feel like disaster clean up. These relatable conflicts can transform the visual of a demolition scene -- a setting that is not particularly relatable -- into a relatable scenario that has the potential for humor, thereby expanding the humorous possibilities by connecting the visual input with broader life experiences.



\subsubsection{Humorous Caption Generation}

Following the narrative and conflict extrapolation, the system generates humorous captions in the generation stage using a fine-tuned version of GPT-3.5 trained on humorous Instagram comments. This involves producing captions through two distinct strategies: one focused on the visual humor of the image, and the other by bringing in the previously generated external narratives. Caption generation is segmented into two separate prompts utilizing the fine-tuned GPT-3.5 model. For captions without generated external narratives, the prompt asks to generate 15 humorous captions in the style of Gen Z that bases the generation off the visual extraction and visual humor ideation of the input image. For captions with the external narratives, the prompt also asks to generate 15 humorous captions in the style of Gen Z that bases the generation off the visual extraction and visual humor ideation of the input image, but also asks the system to incorporate the list of generated narrative/conflict extrapolations to base the humorous captions off of.

Image-focused captions rely solely on the visual details in the image, such as “bro out here getting paid \textdollar8 an hour to spray some water on some bricks,” which references the direct visual elements in the scene in order to generate a caption. This particular caption directly references the humor of the image, poking fun at the minimal impact of the person spraying water on bricks while an excavator clearly has more impact on the demolition site. Narrative-driven captions, on the other hand, introduce external references to add humor. For instance, a caption like “The entitled bro you tried to make the group presentation with” introduces an outside, exaggerated, interpretation of the scene from earlier, "Group Project Disaster." This caption takes the group project narrative and pairs it with the visual of the image, analogizing the person spraying the hose with minimal impact on the demolition site to an entitled person who has not done much to complete the group project. 

This variety between visual humor and narrative-driven humor is crucial because jokes that are too similar become predictable, losing their element of surprise. Additionally, humor strategies need to adapt to the varying content in input images. Some images lend themselves to humor based on their inherent visual details, while others require bringing in outside references to create a joke. For instance, an image of static objects might not be inherently funny, such as the demolition image shown in the system diagram, but a caption introducing an unrelated, exaggerated narrative, such as “Eboy doin' his part to stop climate change” can inject humor and absurdity by making an unexpected connection.

\subsubsection{Caption Ranking using Gen Z Agent}

The final component of the system architecture is the Caption Ranking and Filtering Agent, a GPT-4o-based agent fine-tuned to evaluate humor from a Gen Z perspective. This agent receives the list of 30 total captions from the narrative and visual humor-based caption generations and ranks the captions generated in the previous stage based on humor, relatability, and alignment with the image and narrative.

As illustrated in our system diagram, this agent ranks captions such as “Me mopping up my last relationship” and “me pulling the emotional weight of the friend group” based on their relevance to Gen Z humor. Captions that fail to meet the humor threshold are filtered out, such as "Demolition worker really said 1v1 me bro," because although the phrase like "1v1 me bro" invokes Gen Z phrases, the content of the caption seems less relevant and relatable than a caption talking about school or relationships, ensuring that only the most effective and relatable captions are presented to the user.

\subsubsection{Fine-tuning}

To fine-tune a GPT-3.5 model, a dataset of 80 humorous comments were extracted from popular Instagram images. From three popular Instagram meme pages with over 400,000 followers, the top five comments of each image post were collected. All fit the style of Gen-Z humor. 
% The fine-tuning process ensured that the generated captions align with the humor style favored by Gen Z. 
Examples of the visual description of the images in addition to an explanation of potential humorous elements of the image were written in the fine-tuning prompts, then followed by the actual comment itself. This reflected the visual extraction and humor ideation being incorporated into the prompt of our current system.



\label{evaluation-results}
% \setlength{\tabcolsep}{4.6pt}
% \begin{table*}[t]
% \centering
% \footnotesize
% \begin{tabular}{rcccccc}
% \toprule
%                                & \multicolumn{2}{c}{\textbf{DDxPlus}} & \multicolumn{2}{c}{\textbf{iCraft-MD}} & \multicolumn{2}{c}{\textbf{RareBench}} \\ \cmidrule(lr){2-3} \cmidrule(lr){4-5} \cmidrule(lr){6-7}
%                                & \textbf{GTPA@1 $\uparrow$}          & \textbf{Avg Rank $\downarrow$}   & \textbf{GTPA@1 $\uparrow$}       & \textbf{Avg Rank $\downarrow$}       & \textbf{GTPA@1 $\uparrow$}        & \textbf{Avg Rank $\downarrow$}       \\\midrule
%                                & \multicolumn{6}{c}{\textbf{GPT-4o}}                                                                 \\\midrule
% \textcolor{cyan}{Zero-shot}                      &                &            &             &                &              &                \\
% \textcolor{cyan}{Few-shot (Standard, Dyn\_BAII)} &                &            &             &                &              &                \\
% \textcolor{cyan}{Few-shot (CoT, Dyn\_BAII)}      &                &            &             &                &              &                \\
% History Taking (\textit{n}=5)         & 0.45           & 4.13       & 0.40        & 5.58           & 0.11         & 7.84           \\
% %History Taking (\textit{n}=10)        & 0.59           & 3.16       & 0.45        & 5.35           & 0.24         & 6.67           \\
% History Taking (\textit{n}=15)        & 0.69           & 2.47       & 0.46        & 5.23           & 0.36         & 5.49           \\
% Retrieval (PubMed) \textcolor{red}{rerun/ignore?}                   & 0.69           & 2.27       & 0.68        & 3.23           & 0.45         & 3.92           \\
% MEDDxAgent (\textbf{Ours})         &                &            &             &                &              &                \\
% \textit{iter} =  1                       & 0.74           & 1.91       & 0.52        & 4.93           & 0.51         & 4.37           \\
% \textit{iter} =  2                       & 0.78           & 1.56       & \textbf{0.54}        & \textbf{4.71}           & \textbf{0.56}         & 4.10           \\
% \textit{iter} =  3                       & \textbf{0.86}           & \textbf{1.29}       & \textbf{0.54}        & 4.80           & 0.50         & \textbf{4.09}           \\\midrule
%                                & \multicolumn{6}{c}{\textbf{Llama3.1-70B}}                                                           \\ \midrule
% \textcolor{cyan}{Zero-shot}                      &                &            &             &                &              &                \\
% \textcolor{cyan}{Few-shot (Standard, Dyn\_BAII)} &                &            &             &                &              &                \\
% \textcolor{cyan}{Few-shot (CoT, Dyn\_BAII)}      &                &            &             &                &              &                \\
% History Taking (\textit{n}=5)         & 0.45           & 4.15       & 0.29        & 6.48           & 0.30         & 6.04           \\
% %History Taking (\textit{n}=10)        & 0.58           & 3.12       & 0.33        & 5.82           & 0.36         & 4.51           \\
% History Taking (\textit{n}=15)        & 0.56           & 3.50       & 0.36        & 5.36           & 0.31         & 4.80           \\
% Retrieval (PubMed)  \textcolor{red}{rerun/ignore?}                 & 0.56           & 3.42       & 0.44        & 4.72           & 0.38         & 3.96           \\
% MEDDxAgent (\textbf{Ours})         &                &            &             &                &              &                \\
% \textit{iter} =  1                       & 0.61           & 2.91       & 0.29        & 7.05           & 0.39         & 5.05           \\
% \textit{iter} =  2                       & \textbf{0.71}   & \textbf{2.20}       & 0.37        & \textbf{6.26}           & \textbf{0.48}         & 4.48           \\
% \textit{iter} =  3                       & 0.68   & 2.30       & \textbf{0.42}        & 6.31           & \textbf{0.48}         & \textbf{4.30}           \\\midrule
%                                & \multicolumn{6}{c}{\textbf{Llama3.1-8B}}                                                            \\\midrule
% \textcolor{cyan}{Zero-shot}                      &                &            &             &                &              &                \\
% \textcolor{cyan}{Few-shot (Standard, Dyn\_BAII)} &                &            &             &                &              &                \\
% \textcolor{cyan}{Few-shot (CoT, Dyn\_BAII)}     &                &            &             &                &              &                \\
% History Taking (\textit{n}=5)         & 0.23           & 6.85       & 0.10        & 8.78           & 0.05         & 8.38           \\
% %History Taking (\textit{n}=10)        & 0.35           & 5.46       & 0.12        & 8.39           & \textbf{0.13}         & 8.25           \\
% History Taking (\textit{n}=15)        & 0.40           & 5.44       & 0.11        & 8.30           & \textbf{0.11}        & 8.95           \\
% Retrieval (PubMed)  \textcolor{red}{rerun/ignore?}                 & 0.42           & 4.50       & 0.29        & 6.93           & 0.35         & 5.33           \\
% MEDDxAgent (\textbf{Ours})         &                &            &             &                &              &                \\
% \textit{iter} =  1                       & 0.34           & 5.25       & 0.11        & 9.38           & 0.08         & 8.47           \\
% \textit{iter} =  2                       & 0.56           & 3.59       & \textbf{0.14}        & 9.22           & 0.09         & \textbf{8.11}           \\
% \textit{iter} =  3                       & \textbf{0.58}           & \textbf{3.10}       & 0.12        & \textbf{9.07}           & 0.07         & 8.56        \\  
% \bottomrule
%     \end{tabular}
%     \caption{Iterative experiment performance across 3 datasets. \textcolor{red}{The \textbf{best results} are based on ignoring the Pubmed retrieval results!}}
%     \label{tab:iterative_overall}
% \end{table*}

\setlength{\tabcolsep}{3.8pt}
\begin{table*}[ht]
\centering
\scriptsize
\begin{tabular}{rccccccccc}
\toprule
                               & \multicolumn{3}{c}{\textbf{DDxPlus}} & \multicolumn{3}{c}{\textbf{iCraft-MD}} & \multicolumn{3}{c}{\textbf{RareBench}} \\ \cmidrule(lr){2-4} \cmidrule(lr){5-7} \cmidrule(lr){8-10}
                               & \textbf{GTPA@1 $\uparrow$}          & \textbf{Avg Rank $\downarrow$}   & \textbf{$\Delta$ Progress} & \textbf{GTPA@1 $\uparrow$}       & \textbf{Avg Rank $\downarrow$}     & \textbf{$\Delta$ Progress}   & \textbf{GTPA@1 $\uparrow$}        & \textbf{Avg Rank $\downarrow$}   & \textbf{$\Delta$ Progress}     \\\midrule
                               & \multicolumn{9}{c}{\textbf{GPT-4o}}                                                                 \\\midrule
%\textcolor{cyan}{Zero-shot}                      &     0.69           &    2.21        &      -      &       0.68         &     3.37         &         -       &       0.46       & 3.99            &   -              \\
%\textcolor{cyan}{Zero-shot (CoT)}                      &     0.71          &    2.10        &      -      &       0.68         &     3.35         &         -       &       0.47       & 4.02            &   -              \\
%\textcolor{cyan}{Few-shot (CoT, Dyn\_BAII)} &                &            &     -        &                &              &          -      &              &             &            -     \\
%\textcolor{cyan}{Few-shot (CoT, Dyn\_BERT/Dyn\_BAII)}      &                &            &       -      &                &              &          -      &              &             &           -      \\
%\textit{Single-Turn}      &                &            &       -      &                &              &          -      &              &             &           -      \\
KR (\textit{n}=0)                &      0.18      & 7.33  &  -  &   0.15      &    8.27     & -  &       0.07   &  9.07  &    -   \\
DS (\textit{n}=0)     &  0.27    &    6.01        &       -      &      0.18          &      7.87        &          -      &       0.11       &     8.38        &           -      \\
%SDS (\textit{n}=5)         & 0.45           & 4.13     & - & 0.40        & 5.58     &    -  & 0.11         & 7.84     &  -    \\
KR (\textit{n}=5)  &      0.52      & 3.32   &  -  &  0.49  &  5.36       & -  &     0.40   &   5.27 &    -   \\
DS (\textit{n}=5)  &    0.72       &  2.14 &  -  &  0.40 &    5.55   & -  &   0.50    &   4.94 &    -   \\\cmidrule(lr){2-10}
%History Taking (\textit{n}=10)        & 0.59           & 3.16    & -  & 0.45        & 5.35        & -  & 0.24         & 6.67      &  -   \\
%SDS (\textit{n}=15)        & 0.69           & 2.47     & - & 0.46        & 5.23      &  -   & 0.36         & 5.49      &    - \\\cmidrule(lr){2-10}

%Retrieval (Wiki) \textcolor{red}{rerun}                   &            &    &  -  &         &         & -  &          &    &    -   \\
%MEDDxAgent         &                &            &             &                &              &               &              &             &                  \\
 MEDDx (\textit{iter}=1, \textit{n}=5)                       & 0.74           & 1.91     & ~~0.00 & 0.52        & 4.93      &  ~~0.00   & 0.51         & 4.37        &   ~~0.00\\
MEDDx (\textit{iter}=2, \textit{n}=10)                       & 0.78           & 1.56    & +0.32  & \textbf{0.54}        & \textbf{4.71}    &    +0.26   & \textbf{0.56}         & 4.10   &     +0.13   \\
MEDDx (\textit{iter}=3, \textit{n}=15)                       & \textbf{0.86}           & \textbf{1.29}    & +0.32  & \textbf{0.54}        & 4.80      & +0.17    & 0.50         & \textbf{4.09}       &   +0.16 \\\midrule
                               & \multicolumn{9}{c}{\textbf{Llama3.1-70B}}                                                           \\ \midrule
%\textcolor{cyan}{Zero-shot}                      &      0.54          &     3.53       &       -      &     0.40           &       4.87       &         -      &      0.39        &    4.05         &         -         \\
%\textcolor{cyan}{Zero-shot (CoT)}                      &     0.45          &    3.69       &      -      &       0.48         &     4.50         &         -       &       0.49       & 3.91            &   -              \\
%\textcolor{cyan}{Few-shot (Standard, Dyn\_BAII)} &                &            &      -       &                &              &          -      &              &             &        -         \\
%\textcolor{cyan}{Few-shot (CoT, Dyn\_BERT/Dyn\_BAII)}      &                &            &      -       &                &              &          -    &              &             &                -   \\
KR (\textit{n}=0)           &   0.19         &  7.58  &  -  &      0.13   &   8.19      & -  &    0.09      &  9.13  &    -   \\
DS (\textit{n}=0)    &        0.17        &       7.28     &       -      &      0.11          &      8.74        &          -      &       0.20       &      6.81       &           -      \\
%History Taking (\textit{n}=5)         & 0.45           & 4.15    &  - & 0.29        & 6.48   &     -   & 0.30         & 6.04      &   -  \\
KR (\textit{n}=5)  &      0.39      & 5.03   &  -  &  0.34  &  6.86       & -  &     0.29   &   5.86 &    -   \\
DS (\textit{n}=5)  &     0.50      &  2.89 &  -  & 0.24 &    7.33   & -  &   0.23    &  5.77  &    -   \\\cmidrule(lr){2-10}
%History Taking (\textit{n}=10)        & 0.58           & 3.12     & - & 0.33        & 5.82    &    -   & 0.36         & 4.51    &    -   \\
%History Taking (\textit{n}=15)        & 0.56           & 3.50      &- & 0.36        & \textbf{5.36}       &  -  & 0.31         & 4.80    &    -   \\\cmidrule(lr){2-10}
%Retrieval (Wiki) \textcolor{red}{rerun}                   &            &    &  -  &         &         & -  &          &    &    -   \\
%MEDDxAgent         &                &            &             &                &              &            &                &              &        \\
MEDDx (\textit{iter}=1, \textit{n}=5)                       & 0.61           & 2.91    & ~~0.00  & 0.29        & 7.05       & ~~0.00   & 0.39         & 5.05   &     ~~0.00   \\
MEDDx (\textit{iter}=2, \textit{n}=10)                       & \textbf{0.71}   & \textbf{2.20}      & +0.41 & 0.37        & \textbf{6.26}      & +0.07   & \textbf{0.48}         & 4.48  &    +0.75     \\
MEDDx (\textit{iter}=3, \textit{n}=15)                       & 0.68   & 2.30    & +0.17  & \textbf{0.42}        & 6.31     &   +0.26   & \textbf{0.48}         & \textbf{4.30}      &   +0.44  \\\midrule
                               & \multicolumn{9}{c}{\textbf{Llama3.1-8B}}                                                            \\\midrule
%\textcolor{cyan}{Zero-shot}                      &    0.45            &   9.00         &   -          &    0.27             &     7.02         &      -         &    0.33           &  5.45           &             -     \\
%\textcolor{cyan}{Zero-shot (CoT)}                      &    0.45            &   4.51         &   -          &    0.27             &     7.25         &      -         &    0.24           &  5.65           &             -     \\
%\textcolor{cyan}{Few-shot (Standard, Dyn\_BAII)} &                &            &     -        &                &              &        -      &              &             &             -      \\
%\textcolor{cyan}{Few-shot (CoT, Dyn\_BERT/Dyn\_BAII)}     &                &            &       -      &                &              &         -     &              &             &           -        \\
KR (\textit{n}=0)     &     0.20       &  7.49  &  -  &   0.11      &  \textbf{8.86}       & -  &     \textbf{0.11}     &  8.58  &    -   \\
DS (\textit{n}=0)  &       0.16         &       8.45     &       -      &      0.03          &      10.37        &          -      &         0.04     &       8.52      &           -      \\
%History Taking (\textit{n}=5)         & 0.23           & 6.85    & -  & 0.10        & 8.78        &  - & 0.05         & 8.38       &   - \\
KR (\textit{n}=5)  &      0.21      & 7.42   &  -  &  0.09  &  9.48       & -  &     0.04   &   9.69 &    -   \\
DS (\textit{n}=5)  &     0.23      &  5.77  &  -  &  0.03 &   10.08    & -  &   0.06    &  8.64  &    -   \\\cmidrule(lr){2-10}
%History Taking (\textit{n}=10)        & 0.35           & 5.46    &  - & 0.12        & 8.39     &   -   & \textbf{0.13}         & 8.25   &     -   \\
%History Taking (\textit{n}=15)        & 0.40           & 5.44   &  -  & 0.11        & \textbf{8.30}       &  -  & \textbf{0.11}        & 8.95       &  -  \\\cmidrule(lr){2-10}
%Retrieval (Wiki) \textcolor{red}{rerun}                   &            &    &  -  &         &         & -  &          &    &    -   \\
%MEDDxAgent         &                &            &             &                &              &               &                &              &     \\
MEDDx (\textit{iter}=1, \textit{n}=5)                       & 0.34           & 5.25   &   ~~0.00 & 0.11        & 9.38       &  ~~0.00  & 0.08         & 8.47    &    ~~0.00   \\
MEDDx (\textit{iter}=2, \textit{n}=10)                       & 0.56           & 3.59    & +1.73  & \textbf{0.14}        & 9.22       &  +0.22  & 0.09         & \textbf{8.11}      &  +0.44   \\
MEDDx (\textit{iter}=3, \textit{n}=15)                       & \textbf{0.58}           & \textbf{3.10}    &  +1.23 & 0.12        & 9.07     & +0.17     & 0.07         & 8.56    &  +0.38  \\  
\bottomrule
    \end{tabular}
    \vspace{-0.8em}
    \caption{Interactive experiment performance across 3 datasets without \textit{full} patient profile, with KR: knowledge retrieval agent; DS: diagnosis strategy agent; $n$ is the number of turns of the simulator; MEDDx uses KR+DS.
    %We compare the single-turn (\textit{upper}) with the proposed iterative setup for MEDDxAgent (\textit{bottom}). The selection of the agents and simulator are optimized (\autoref{subsec:optimize-agents}), unless controlled by the number of questions ($n$) asked from history taking simulator.\cc{May be we just need 3 entries of single turn for best agents and simulator compared to MEDDxAgent? For the others we leave it for ablation study?}} %\cl{why we don't have the baseline with diagnosis strategy only without full patient profile (e.g., few-shot CoT, Dyn\_BAII?)}
    }
    \label{tab:interactive_overall}
    \vspace{-1.8em}
\end{table*}

We experiment on two configurations: (1) optimizing individual agents (\autoref{subsec:optimize-agents}), by determining the best settings for knowledge retrieval and diagnosis strategy agents; and (2) interactive differential diagnosis (\autoref{subsec:iterative_learning}), where the optimized agents are used to assess MEDDxAgent's performance in the interactive DDx setup.

\subsection{Optimizing Individual Agents}
\label{subsec:optimize-agents}

We first explore the optimal single-turn configuration for the knowledge retrieval and diagnosis strategy agents, before integrating them into iterative setup. For this, we provide the full patient profile as in previous work~\cite{wu2024streambench,chen2024rarebench}, and present the results in~\autoref{tab:with_patient_profile}. For the knowledge retrieval agent, PubMed performs slightly better overall than Wikipedia, especially for Rarebench, which demands more complex disease information. For the diagnosis strategy agent, the best setting varies by dataset. 
Namely, dynamic few-shot with BAII embeddings performs the best on DDxPlus and RareBench, where relevant patient examples offer reliable contextual cues to likely diseases. 
In contrast, iCraft-MD benefits more from zero-shot CoT, which enables structured reasoning through complex clinical vignettes. Few-shot learning often decreases performance for iCraft-MD because each patient vignette is distinct, so additional examples can introduce noise.
Based on the above findings, we select the following configurations for the iterative scenario:\footnote{We do not run all possible settings in the interactive environment due to cost reasons.} PubMed for knowledge retrieval agent; few-shot (dynamic BAII) for DDxPlus and RareBench, and zero-shot (CoT) for iCraft-MD for diagnosis strategy agent.

\begin{figure*}[t]
    \centering
    \begin{subfigure}{0.48\textwidth}
    \includegraphics[trim={0.2cm 0cm 0cm 0cm },clip, width=\textwidth]{img/ddxplus_history.pdf}
    \vspace{-1.8em}
    \caption{}
    \end{subfigure}
    \begin{subfigure}{0.48\textwidth}
    \includegraphics[trim={0.2cm 0cm 0cm 0cm}, clip, width=\textwidth]{img/agent_iterations_plot_ddxplus.pdf}
    \vspace{-1.8em}
    \caption{}
    \end{subfigure}
    \vspace{-0.5em}
    \caption{Results of DDxPlus compared between (a) history taking simulator, and (b) MEDDxAgent, over the number of questions and iterations. For brevity, the results of iCraft-MD and RareBench are in~\autoref{subsec:comparison_history_taking_iterative}.}
    \label{fig:ddxplus_comparison}
    \vspace{-1.8em}
\end{figure*}

\subsection{Interactive Differential Diagnosis}
\label{subsec:interactive_differential_diagnosis}
We now evaluate the more challenging task of interactive DDx, where we begin with limited patient information and the history taking simulator enables the interactive environment~(\autoref{tab:interactive_overall}).
At $n=0$, the simulator has not yet learned any patient information, and performance drops significantly from observing the full patient profile (\autoref{tab:with_patient_profile}). 
For GPT-4o in RareBench, the knowledge retrieval agent (KR)'s GTPA@1 drops from 0.45  to 0.07. Similarly, the diagnosis strategy agent (DS) drops from 0.46 (zero-shot) to 0.11. This simple baseline showcases that previous evaluations do not hold well in the interactive setup with initially limited patient information. 
Already for $n=5$, we find a large boost in performance for both KR and DS. These findings reinforce the importance of history taking for diagnostic precision. 
We illustrate the trend for changing $n$ in~\autoref{fig:ddxplus_comparison} and find that gains also plateau around \textit{n}=10-15 questions, reinforcing the optimal balance between information gathering and diagnostic efficiency \cite{ely1999analysis}.

Finally, we run MEDDxAgent, which calls KR+DS in the \textit{fixed iteration} pipeline (\autoref{subsec:iterative_learning}). MEDDxAgent exhibits clear improvements over the KR and DS baselines for $n=5$, supporting our hypothesis that all three modules are important for interactive DDx. It also improves significantly over the history taking baselines, as we illustrate in \autoref{fig:ddxplus_comparison}. MEDDxAgent is also capable of improving upon the zero-shot setting with the full patient profile (\autoref{tab:with_patient_profile}). For DDxPlus, GTPA@1 for GPT-4o and Llama3.1-70B rise from 0.56 to 0.86 and from 0.46 to 0.71, respectively. For Llama3.1-8B, the trend continues for DDxPlus but inconsistently for iCraft-MD and RareBench, highlighting the importance of model scale. Notably, MEDDxAgent improves over successive iterations, though the optimal number of iterations (2, 3) depends on the dataset and LLM. The values of $\Delta$ are consistently positive, indicating that MEDDxAgent iteratively increases the rank of the ground-truth diagnosis over time. $\Delta$ Progress also varies by dataset and model, offering explainable insight to the diagnosistic improvement of MEDDxAgent. The overall results show that MEDDxAgent can operate well in the challenging, realistic setup of interactive DDx. Additionally, MEDDxAgent logs all intermediate reasoning, action, and observations, providing critical insight into its DDx process (\autoref{fig:Example}).
\vspace{-0.5em}


\section{Discussion and system limitations}

Our system employs compact nano-UAVs for both greenhouses and outdoor applications. 
Indoor environments provide ideal conditions with no weather constraints, while outdoor functionality is limited by weather conditions (e.g., wind speeds up to ~\SI{3.5}{\meter/\second}~\cite{9811834} and no rain). 
Though primarily demonstrated for pest detection, the system can also be applied to tasks like dry plant detection, crop monitoring, and counting. 
For Popillia japonica, optimal deployment conditions are sunny days with light winds and temperatures around \SI{29}{\celsius}~\cite{eppo-popillia} that fit the nano-UAVs' ideal operating conditions.

\begin{wrapfigure}{R}{0.48\textwidth}
\centering
\includegraphics[width=0.48\textwidth]{images/pop_sizes.png}
\caption{Example images depicting \textit{Popillia japonica} specimens at different scales, relative to image size. From left to right, bounding boxes occupy, on average, 0.9\%, 23.9\% and 41.9\% of the image's total pixels.}
\label{fig:popillia_sizes}
\end{wrapfigure}


Nano-UAVs are of particular interest for pest detection because they have a lower environmental impact than standard-sized drones that are currently used for this application.
In fact, nano-UAVs produce noise up to 40 dB~\cite{10.1145/3666025.3699337} while their bigger counterpart reaches up to 75 dB~\cite{ijerph18126202}, as such nano-UAVs provide an interesting solution that reduces the impact on the environment.

We test the system across varying hotspot densities (0 to 50) and three crop arrangements (environments 1, 2, and 3), highlighting its advantages over ground robots for early pest detection. 
Performance is influenced more by obstacle density than by the overall environment layout, thanks to the reliance on local path planning for obstacle avoidance.
We now analyze in detail the limitations of our insect detector and of the routing algorithm.


\subsection{Insect detector}

The dataset used in our work contains images gathered online rather than images taken directly from a nano-drone. 
This implies the presence of both pictures where the insects appear in close proximity, and pictures where the insects are much smaller relative to image size. 
On average, target insects cover 17.1\% of an image’s total pixels (8.9\% for \textit{Popillia japonica}, 19.8\% for \textit{Phyllopertha horticola}, 22.5\% for \textit{Cetonia aurata}). 
Figure~\ref{fig:popillia_sizes} provides a visual reference for this. From left to right, bounding boxes occupy, on average, 0.9\%, 23.9\% and 41.9\% of the image's total pixels.
We believe nano-drones can produce similar images, given their small size and capability of flying between vineyards' rows, close to vines. 
However, we point out that a real-world deployment would likely benefit from a model trained directly on images acquired by the drones during exploration.


\subsection{Routing}

\begin{figure}[tb!]
\centering
\includegraphics[width=1.0\columnwidth]{images/oscillation_ver_02.jpg}
\caption{An obstacle covers the entire depth sensor FoV, causing the blockage.}
\label{fig:deadlock}
\end{figure}


Blockages are a typical problem when relying on local planning strategies, they can occur when obstacles cover the entire FoV of the depth sensor, as reported in Figure~\ref{fig:deadlock}.
In fact, in this condition, the local planner provides a solution that passes through the uncertain region of the map.
However, when the UAV starts moving toward the uncertainty region, the depth sensor detects the presence of an obstacle that intersects the new locally planned path.
This causes a new iteration of the local planner, which provides a new solution that belongs to the uncertainty region that will result in a collision as soon as the UAV moves towards it.

The first solution involves maintaining a record of the surroundings based on the current depth measurement. 
This approach uses the same local planning algorithm proposed in this work, namely A*, applied to a local map that includes the current depth measurement along with all previous measurements within a 4$\times$\SI{4}{\meter} area used for local planning. 
This method gradually maps objects larger than the sensor’s FoV, which might otherwise obscure entirely the area ahead of the sensor and cause blockages. 
While this approach reduces the frequency of blockages, it does not eliminate them, as objects exceeding \SI{4}{\meter} in size can still lead to blockages in the current implementation.
Other local solutions to mitigate the blockages issue rely on reinforcement learning and swarm cooperation. 
Still, in our use case, that does not involve communication between UAVs and limits the knowledge of the map to a local instance of 4$\times$\SI{4}{\meter} obstacles that occlude the entire FoV may always result in blockages.
To avoid the blockage issue, a reliable solution is to perform obstacle avoidance on the global map, which cannot be done on our nano-UAVs due to the platform's computational constraints.





\section{Conclusions}

This work presents the building blocks for a novel, efficient transportation system applied to a pest detection and control scenario in vineyards, relying on flying and ground robots. 
We propose an implementation of the pest detection system that runs onboard the Crazyflie 2.1 nano-UAV on the ultra-low power multi-core GWT GAP9 SoC.
Due to the limited resources available on this platform, we explore and deploy a CNN-based detection system, i.e., the SSDLite-MobileNetV3 CNN (\SI{584}{\mega MAC/inference}), scoring an mAP of 0.79 with a throughput of~\SI{6.8}{frame/\second} at~\SI{33}{\milli\watt} on the GAP9 SoC.

We integrate the CNN-based insect detector with an obstacle avoidance algorithm running onboard the Crazyflie 2.1 nano-UAV to allow the autonomous exploration of vineyards.
Our local routing A*-based obstacle avoidance algorithm is able to reach up to 100\% of the planned waypoints, avoiding all the obstacles in two out of three environments with increasing complexity (from 0 to 10\% of the entire area covered with~\SI{1}{\meter\squared} obstacles).

If compared to the pre-planned path (B) and to the random (R) explorer baselines, our local routing algorithm (W) increases the number of waypoints visited within the battery life of the UAV between 16\% in the smallest environment, i.e., a~10$\times$\SI{10}{\meter} vineyard, to 90\% in our~40$\times$~\SI{40}{\meter} environment. 
The algorithm achieves real-time performance with planning requiring less than \SI{170}{\milli\second} on the STM32 MCU available on the nano-UAV while performing real-time flight control tasks.
Our multi-UAV system, using a swarm of 25 UAVs to explore a~200$\times$\SI{200}{\meter} vineyard, allows us to save up to~$\sim$\SI{20}{\hour}, i.e., 100\% of the time if no insects are detected, to perform pest control, paving the way to fully autonomous precise and effective pest control with an efficient transportation system applied to vineyards.


%%
%% The next two lines define the bibliography style to be used, and
%% the bibliography file.
\bibliographystyle{ACM-Reference-Format}
\bibliography{biblio}


%%
%% If your work has an appendix, this is the place to put it.
\appendix






\end{document}
\endinput
%%
%% End of file `sample-acmsmall.tex'.
