%%
%% This is file `sample-acmsmall.tex',
%% generated with the docstrip utility.
%%
%% The original source files were:
%%
%% samples.dtx  (with options: `all,journal,bibtex,acmsmall')
%% 
%% IMPORTANT NOTICE:
%% 
%% For the copyright see the source file.
%% 
%% Any modified versions of this file must be renamed
%% with new filenames distinct from sample-acmsmall.tex.
%% 
%% For distribution of the original source see the terms
%% for copying and modification in the file samples.dtx.
%% 
%% This generated file may be distributed as long as the
%% original source files, as listed above, are part of the
%% same distribution. (The sources need not necessarily be
%% in the same archive or directory.)
%%
%%
%% Commands for TeXCount
%TC:macro \cite [option:text,text]
%TC:macro \citep [option:text,text]
%TC:macro \citet [option:text,text]
%TC:envir table 0 1
%TC:envir table* 0 1
%TC:envir tabular [ignore] word
%TC:envir displaymath 0 word
%TC:envir math 0 word
%TC:envir comment 0 0
%%
%%
%% The first command in your LaTeX source must be the \documentclass
%% command.
%%
%% For submission and review of your manuscript please change the
%% command to \documentclass[manuscript, screen, review]{acmart}.
%%
%% When submitting camera ready or to TAPS, please change the command
%% to \documentclass[sigconf]{acmart} or whichever template is required
%% for your publication.
%%
%%
\documentclass[acmsmall]{acmart}

%%
%% \BibTeX command to typeset BibTeX logo in the docs
\AtBeginDocument{%
  \providecommand\BibTeX{{%
    Bib\TeX}}}
\usepackage{siunitx}
\usepackage{natbib}
\usepackage{multirow}
\usepackage{wrapfig}
\DeclareSIUnit{\nothing}{\relax}
\newcommand{\firstreviewer}[1]{\textit{\color{blue}~#1}}%LS
\newcommand{\secondreviewer}[1]{\noindent\textit{\color{brown}~#1}}%LS
\newcommand{\thirdreviewer}[1]{\noindent\textit{\color{red}~#1}}%LS
\sisetup{detect-all=true}    % In the abstract, makes numbers and SIUnit in SI{} bold
%% Rights management information.  This information is sent to you
%% when you complete the rights form.  These commands have SAMPLE
%% values in them; it is your responsibility as an author to replace
%% the commands and values with those provided to you when you
%% complete the rights form.
\setcopyright{acmlicensed}
\copyrightyear{2025}
\acmYear{2025}
\acmDOI{10.1145/3719210}


%%
%% These commands are for a JOURNAL article.
\acmJournal{JACM}
\acmVolume{1}
\acmNumber{1}
\acmArticle{1}
\acmMonth{1}

%%
%% Submission ID.
%% Use this when submitting an article to a sponsored event. You'll
%% receive a unique submission ID from the organizers
%% of the event, and this ID should be used as the parameter to this command.
%%\acmSubmissionID{123-A56-BU3}

%%
%% For managing citations, it is recommended to use bibliography
%% files in BibTeX format.
%%
%% You can then either use BibTeX with the ACM-Reference-Format style,
%% or BibLaTeX with the acmnumeric or acmauthoryear sytles, that include
%% support for advanced citation of software artefact from the
%% biblatex-software package, also separately available on CTAN.
%%
%% Look at the sample-*-biblatex.tex files for templates showcasing
%% the biblatex styles.
%%

%%
%% The majority of ACM publications use numbered citations and
%% references.  The command \citestyle{authoryear} switches to the
%% "author year" style.
%%
%% If you are preparing content for an event
%% sponsored by ACM SIGGRAPH, you must use the "author year" style of
%% citations and references.
%% Uncommenting
%% the next command will enable that style.
%%\citestyle{acmauthoryear}


%%
%% end of the preamble, start of the body of the document source.
\usepackage{mathtools}
\DeclarePairedDelimiter\floor{\lfloor}{\rfloor}
\DeclareMathOperator{\arctantwo}{arctan2}

%%% BEGIN ARXIV MARKER %%%
% Command to write a header to say "paper accepted at such conference"
\definecolor{somegray}{rgb}{0.5, 0.5, 0.5}
\newcommand{\darkgrayed}[1]{\textcolor{somegray}{#1}}
\makeatletter
\newcommand*\titleheader[1]{\gdef\@titleheader{#1}}
\AtBeginDocument{%
  \let\st@red@title\@title
  \def\@title{%
    \vskip-2.0em
    \bgroup\normalfont\large\centering\@titleheader\par\egroup
    \vskip0.0em\st@red@title}
}

\makeatother

% Here goes the MESSAGE THAT YOU WANT TO APPEAR above the paper title
\titleheader{\darkgrayed{This paper has been accepted for publication in the 2025 ACM Journal on Autonomous Transportation Systems\\\copyright 2025 ACM.}}
%%% END ARXIV MARKER %%%

\usepackage[firstpage]{draftwatermark} % Apply watermark only to the first page
\SetWatermarkText{This paper has been accepted for publication in the 2025 ACM Journal on Autonomous Transportation Systems\\\copyright 2025 ACM.} % The watermark text
\SetWatermarkScale{0.075} % Adjust size of the watermark
\SetWatermarkColor[gray]{0.85} % Light gray watermark
\SetWatermarkAngle{0} % Rotate the watermark for a diagonal appearance
\SetWatermarkVerCenter{0.1\paperheight} % Move watermark vertically (higher value moves it down)

\begin{document}



%%
%% The "title" command has an optional parameter,
%% allowing the author to define a "short title" to be used in page headers.
\title[An Efficient Ground-aerial Transportation System for Pest Control Enabled by Autonomous AI-based Nano-UAVs]{An Efficient Ground-aerial Transportation System for Pest Control Enabled by AI-based Autonomous Nano-UAVs}

% An Efficient Ground-aerial Transportation System Enabled by AI-based Autonomous Nano-drones for Pest Detection

% Seek and Transport: An Efficient Ground-aerial Transportation System for Ultra-low-power AI-based Pest Detection and Graph-based Obstacle Avoidance on Miniaturized Drones

% An AI-based Efficient Ground-aerial Transportation System for Pest Detection and Graph-based Obstacle Avoidance aboard Miniaturized Drones

% Onboard Graph-based Obstacle Avoidance for Autonomous Nano-drone Pest Detection in Vineyards

%%
%% The "author" command and its associated commands are used to define
%% the authors and their affiliations.
%% Of note is the shared affiliation of the first two authors, and the
%% "authornote" and "authornotemark" commands
%% used to denote shared contribution to the research.
\author{Luca Crupi}
\email{luca.crupi@supsi.ch}
\affiliation{%
  \institution{IDSIA, SUPSI}
  \city{Lugano}
  \country{Switzerland}
}
%\orcid{1234-5678-9012}
\author{Luca Butera}
\email{luca.butera@usi.ch}
\affiliation{%
  \institution{IDSIA, USI}
  \city{Lugano}
  \country{Switzerland}
}
\author{Alberto Ferrante}
\email{alberto.ferrante@usi.ch}
\affiliation{%
  \institution{IDSIA, USI}
  \city{Lugano}
  \country{Switzerland}
}
\author{Alessandro Giusti}
\email{alessandro.giusti@usi.ch}
\affiliation{%
  \institution{IDSIA, SUPSI}
  \city{Lugano}
  \country{Switzerland}
}
\author{Daniele Palossi}
\email{daniele.palossi@supsi.ch}
\affiliation{%
  \institution{IDSIA, SUPSI}
  \city{Lugano}
  \country{Switzerland}
}
\affiliation{%
  \institution{IIS, ETH Z\"urich}
  \city{Z\"urich}
  \country{Switzerland}
}


%%
%% By default, the full list of authors will be used in the page
%% headers. Often, this list is too long, and will overlap
%% other information printed in the page headers. This command allows
%% the author to define a more concise list
%% of authors' names for this purpose.

%%
%% The abstract is a short summary of the work to be presented in the
%% article.

\begin{abstract}


Efficient crop production requires early detection of pest outbreaks and timely treatments; we consider a solution based on a fleet of multiple autonomous miniaturized unmanned aerial vehicles (nano-UAVs) to visually detect pests and a single slower heavy vehicle that visits the detected outbreaks to deliver treatments.
To cope with the extreme limitations aboard nano-UAVs, e.g., low-resolution sensors and sub-\SI{100}{\milli\watt} computational power budget, we design, fine-tune, and optimize a tiny image-based convolutional neural network (CNN) for pest detection. 
Despite the small size of our CNN (i.e., \SI{0.58}{\giga Ops/inference}), on our dataset, it scores a mean average precision (mAP) of 0.79 in detecting harmful bugs, i.e., 14\% lower mAP but 32$\times$ fewer operations than the best-performing CNN in the literature. 
Our CNN runs in real-time at~\SI{6.8}{frame/\second}, requiring~\SI{33}{\milli\watt} on a GWT GAP9 System-on-Chip aboard a Crazyflie nano-UAV. 
Then, to cope with in-field unexpected obstacles, we leverage a global+local path planner based on the A* algorithm. 
The global path planner determines the best route for the nano-UAV to sweep the entire area, while the local one runs up to~\SI{50}{\hertz} aboard our nano-UAV and prevents collision by adjusting the short-distance path. 
Finally, we demonstrate with in-simulator experiments that once a 25 nano-UAVs fleet has combed a 200$\times$\SI{200}{\meter} vineyard, collected information can be used to plan the best path for the tractor, visiting all and only required hotspots. 
In this scenario, our efficient transportation system, compared to a traditional single-ground vehicle performing both inspection and treatment, can save up to~\SI{20}{\hour} working time.

%\firstreviewer{Timely collection of critical information to plan optimized strategies, such as the best path, best scheduling, or best production flow, is certainly key in increasing efficiency in civil and industrial transportation systems. In this context, we address, as an example and without loss of generality, an agribusiness use case.
%Prompt, precise, and efficient treatments are crucial in crop production to prevent pest outbreaks but require timely and fine-grained treatments. In this context, accurate pest detection and optimal planning of routes for slow machinery are of the essence, i.e., our transportation problem.}

%\firstreviewer{Our general solution presents a two-level autonomous transportation system.
%The forefront is represented by a fleet of agile miniaturized autonomous unmanned aerial vehicles (UAVs) as big as the palm of one hand (i.e., nano-UAVs), while a slow and bulky tractor acts as the backbone for the heavy-duty job.
% Thanks to their agility, nano-UAVs can quickly act as probes to comb vast cultivated areas, looking for the first signs of pest outbreaking, leaving the fine-grained treatment to the bulky and slow tractor. 
%Autonomous nano-UAVs can rely only on the limited resources aboard, which means low-resolution sensors and sub-\SI{100}{\milli\watt} power budget for processing.
%To cope with these limitations, we design, fine-tune, and optimize a tiny image-based SSDLite-MobileNetV3 convolutional neural network (CNN) for pest detection.
%Despite the small size of our CNN, on a custom testing dataset of 660 images, it scores a mean average precision of 0.79 in detecting harmful bugs (i.e., Popillia japonica), only 14\% less than the best-performing CNN in the literature that requires 32$\times$ more operations.
%Our CNN runs in real-time (i.e., \SI{6.8}{frame/\second}) and requires only \SI{33}{\milli\watt} on a Greenwaves Technologies GAP9 multi-core System-on-Chip, which we employ aboard a Crazyflie nano-UAV.
%To cope with the unexpected dynamic obstacles in the field, we leverage a global+local path planner based on the A* algorithm.
%The global path planner, executed ahead of the mission, determines the best route for the nano-UAV to sweep the entire area; the local one runs in real-time (up to \SI{50}{\hertz}) aboard our nano-UAV and prevents collision by adjusting the short-distance path.}

%\firstreviewer{We demonstrate with in-simulator experiments that once a 25 nano-UAVs fleet has combed all areas of a 200$\times$\SI{200}{\meter} vineyard, collected information can be used to plan the best path for the tractor, visiting all and only required hotspots.
%In this scenario, our efficient transportation system, compared to a traditional single-ground vehicle performing both inspection and treatment, can save up to \SI{20}{\hour} working time.}
\end{abstract}


%%
%% The code below is generated by the tool at http://dl.acm.org/ccs.cfm.
%% Please copy and paste the code instead of the example below.
%%
\begin{CCSXML}
<ccs2012>
<concept>
<concept_id>10010520.10010553.10010554.10010557</concept_id>
<concept_desc>Computer systems organization~Robotic autonomy</concept_desc>
<concept_significance>500</concept_significance>
</concept>
</ccs2012>
\end{CCSXML}

\ccsdesc[500]{Computer systems organization~Robotic autonomy}

%\ccsdesc[500]{Do Not Use This Code~Generate the Correct Terms for Your Paper}
%\ccsdesc[300]{Do Not Use This Code~Generate the Correct Terms for Your Paper}
%\ccsdesc{Do Not Use This Code~Generate the Correct Terms for Your Paper}
%\ccsdesc[100]{Do Not Use This Code~Generate the Correct Terms for Your Paper}

%%
%% Keywords. The author(s) should pick words that accurately describe
%% the work being presented. Separate the keywords with commas.
\keywords{routing, path planning, UAV, nano-UAV, drones, nano-drones, CNN, neural networks, pest detection, transportation}

%\received{xxx xxx xxx}
%\received[revised]{xxx xxx xxx}
%\received[accepted]{xxx xxx xxx}

%%
%% This command processes the author and affiliation and title
%% information and builds the first part of the formatted document.
\maketitle


\begin{acks}
We thank Hanna Müller, Victor Kartsch, and Luca Benini, authors of~\cite{müller2024gap9shield150gopsaicapableultralow} for providing us with a GAP9Shield prototype.
\end{acks}


The increasing reliance on LLMs for multimodal tasks across far-reaching sectors such as healthcare, finance, and manufacturing underscores the need to assess the accuracy and reliability of the information they generate. Vision-Language Models (VLM) have achieved state-of-the-art (SoTA) performance on Visual Question-Answering (VQA) benchmarks, and these models often utilize Retrieval-Augmented Generation (RAG) to maintain factual accuracy and relevance in a dynamic information environment. However, this has led to uncertainty in the information the LLM bases its answer on, as it may choose between parametric memory and retrieved sources. When models rely on memorized information instead of dynamically retrieving information, they may inadvertently propagate outdated or incorrect information, causing serious legal and ethical risks and undermining trust and reliability in AI systems \citep{huang2023survey}.
% The ability to strike a balance between generalization and specialization in AI systems is therefore crucial for ensuring the safe, reliable use of these technologies in real-world applications.

Despite these concerns, the way that Vision-Language models (VLMs) memorize and retrieve information, particularly in complex multimodal tasks, remains under-explored. Current research often focuses on either the general capabilities of large language models (LLMs) or the specialized retrieval mechanisms in retrieval augmented generation systems (RAG) \citep{incontext_rag,chen_murag_2022,liu_universal_2023}. Particularly in the context of multimodal retrieval and multihop reasoning, few studies analyze the tradeoff between finetuning for specialized tasks and zero-shot prompting for general-purpose vision-language capabilities. A lack of consensus on how to approach this tradeoff motivates the development of measures to quantify reliance on parametric memory, as well as metrics for quantifying the potential performance impact of extending LLMs with RAG systems.

To address this gap, we investigate how multimodal QA models balance accuracy with memorization on the WebQA benchmark. We compare finetuned multimodal systems against zero-shot VLMs, analyzing how retrieval performance influences QA accuracy. In particular, we focus on cases where retrieval fails, allowing us to measure reliance on parametric memory through two proposed metrics---the \ppr (\PPR) which quantifies how much model accuracy is influenced by retrieval quality, contrasting performance in best-case versus worst-case retrieval scenarios, and the \ucr (\UCR) which measures how often correct QA responses are generated when the retriever fails, providing a proxy for memorization.

To enable this analysis, we make several methodological contributions. For the finetuned QA models, we investigate Vision-Transformer (ViT) architectures, which allow for multihop reasoning over multiple sources. To investigate the impact of retrieval performance on trained LMs, we propose a variable-input Fusion-in-Decoder (FiD) model \cite{tanaka_slidevqa_2023, nlvr2}, building upon the VoLTA architecture \citep{pramanick_volta_2023}. For the zero-shot case, we build upon previous research on In-Context Retrieval \citep{incontext_rag} by demonstrating that LLMs such as GPT-4o are capable of performing the final ranking step of the retrieval process. In doing so, we find that GPT-4o, a general-purpose LLM, achieves SoTA performance on the WebQA task, outperforming existing finetuned RAG models by a significant margin (7\% higher accuracy). 

Crucially, our results reveal that while retrieval-augmented models reduce memorization, the training paradigm plays an important role. Finetuned models exhibit higher reliance on parametric memory, whereas zero-shot RAG approaches have lower memorization scores at the cost of accuracy. This suggests that while retrieval modules may mitigate the risks associated with outdated or incorrect information, SoTA performance requires that they be coupled with specialized QA models. Our memorization measures contribute to the development of transparent and reliable AI systems, particularly in applications where the sourcing of up-to-date, factual information is critical.



% We investigate the impact of question complexity on the ability of these models to integrate multiple data sources—such as images, text, and external retrievers—and produce coherent and accurate answers. We also explore whether in-context retrieval can be a viable alternative to traditional retrieval-augmented systems, offering a more streamlined approach to multimodal QA.

% To achieve this, we first compare zero-shot prompting multimodal LLMs with finetuned multimodal systems. We evaluate both types of models on the WebQA benchmark, a dataset designed for complex question answering that requires reasoning across both image and text sources. For the finetuned models, we use a Fusion-in-Decoder (FiD) architecture, which allows for multihop reasoning over multiple sources. Additionally, we introduce the concept of In-Context Retrieval Language Modeling (RLM), where the LLM itself performs retrieval tasks without the need for external retrievers. This method builds upon existing research in in-context learning  and aims to explore the viability of LLMs retrieving relevant sources and generating accurate answers directly from their context window.

% In order to investigate source utilization in finetuned multimodal models and LLMs, three lines of inquiry are established; 
% \begin{itemize}
%     \item Study 1: retrieval vs QA performance on webQA (motivating example, does QA answer correctly even with incorrect sources?)
%     \item Study 2: performance on adversarial examples where parametric knowledge would be incorrect by design
%     \item Study 3: improving performance on adversarial examples by fine-tuning (i.e model robustness)
% \end{itemize}

% Note, there is one weakness in this plan which is tying in the work we've already done. 
% If we added something from adversarial generation to the retrieval experiment (like a combination of study 1 + 3) it would be complete. So for instance we could try fine-tuning the retriever with adversarial examples (and not just the QA model)

% \begin{figure}
%     \centering
%     \includegraphics[width=0.95\linewidth]{figures/segmentation/webqa_segment_infill.png}
%     \caption{Example of the segmentation substitution pipeline from the WebQA task.}
%     % d5c76d760dba11ecb1e81171463288e9
%     \label{fig:seg_sub_pipeline}
% \end{figure}



% Retrieval augmented generation (RAG) with zero-shot prompting and fine-tuning Large Language Models (LLMs) have become the go-to methods for tasks relying on information retrieval and text generation. In many cases the LLMs parametric memory can sufficiently generalize to answer questions without being provided with retrieval mechanisms for out-of-domain knowledge. However, LLMs often hallucinate and provide wrong information in certain scenarios. This problem is amplified even further on open-domain Question Answering (QA) tasks involving multiple modalities. Grounded text generation using retrieved sources \citep{lewis2021retrievalaugmented} has been extensively studied for text-to-text QA tasks, but its application in multimodal settings has not been studied as much.


% Multimodal reasoning and question answering have gained prominence in recent research endeavors, with an increasing emphasis on handling various forms of data, particularly text and images. In this study, we address a specific gap in the existing literature by focusing on the development of a versatile multihop model capable of accommodating varying numbers of input images.

% Our motivation for this research lies in the growing complexity of answering questions using information on the web, where the challenge of navigating the open-domain setting is further complicated by the presence of multiple modalities and sometimes requires reasoning over multiple sources. WebQA is an ideal dataset on which to compare performance of finetuned RAG systems against general purpose LLMs; it is multimodal, with correct answers requiring reasoning over image and text sources. It is multihop, requiring a complex reasoning process over multiple sources. Finally, WebQA questions from different categories can be broken down into subdomains to analyze performance over domains of varying cardinality.

% Motivated by the real-world challenges of building retrieval and question answering (QA) systems, we design and finetune a closed domain, multimodal, multihop QA model, that is capable of reasoning over a varying number of sources taken as input from an external retriever module. This research contributes to the relatively underexplored domain of multihop reasoning across various input sources and modalities. Our goal is to explore the challenges posed by these scenarios and develop strategies that enable QA models to retrieve relevant information, conduct logical or numerical reasoning across diverse modalities, and generate coherent responses in natural language. To our knowledge, this is the first application of the Fusion-in-Decoder (FiD) architecture \cite{tanaka_slidevqa_2023, nlvr2} that is shown to work with a variable number of inputs, enabling multi-hop reasoning over sources.

% In-Context Learning refers to the ability of LLMs to perform any task by simply providing examples in the input prompt \citep{dong2022survey,min2022rethinking}. Inspired by this research, we propose a method to use the LLM itself as a multimodal retriever, potentially eschewing the requirement of a distinct retrieval module, thereby allowing the design of simpler retrieval-augmented QA systems. We dub this method In-Context Retrieval Language Modeling (RLM). To the best of the authors knowledge, In-Content RLM is disparate from other retrieval augmented approaches which utilize external retrieval modules \citep{incontext_rag,chen_murag_2022,liu_universal_2023}. Despite being a natural extension of In-Context learning, In-Context RLM has not yet been studied empirically.

% To expand on our contribution of In-Context Retrieval, this stems from the well-researched in-context learning of LLMs. In-context learning is the ability of a model to perform any task given a sufficient context window \citep{dong2022survey,min2022rethinking}. Such tasks could include retrieval and ranking, but typically, the go-to solution for tasks requiring retrieval has been RAG. To the best of the authors knowledge, In-Context Retrieval is distinct from In-Context Retrieval Augmented Language Modelling (RALM), and despite being a natural extension of In-Context learning, In-Context Retrieval has not yet been shown empirically.

% Finally, we explore the tradeoff between using zero-shot prompting LLMs and the fine-tuning approach. While we find that, overall, GPT-4o obtains SoTA performance on the WebQA task, outperforming the accuracy of existing finetuned RAG approaches by 7\%, finetuned approaches still perform better on more restricted subdomains\footnote{``In-Context RLM" @ \url{https://eval.ai/web/challenges/challenge-page/1255/leaderboard/3168}}. Finally, we validate that GPT-4o is relying on retrieval abilities to solve the task; we find that GPT-4o is capable of retrieving relevant sources in the presence of distractors and furthermore, when GPT-4o fails to retrieve correct sources, it answers incorrectly 75\% of the time, meaning that it is not relying on parametric memory for this task.

% \paragraph{Contributions}
% Based on our experimentation and analysis on the WebQA benchmark, we make the following contributions:
% \begin{itemize}
%     \item Propose a new architecture for multimodal multihop QA that takes variable number of input sources inspired by the Fusion-in-Decoder method.
%     \item Comparison of general purpose LLMs vs specialized models on the WebQA benchmark.
%     \item Observation of In-Context Multimodal Retrieval abilities of GPT-4o and that it does not rely on parametric memory for multimodal QA.
%     \item Analysis of relationship between retrieval and QA task performance.
%     \item Analysis of task and query complexity on the performance of retrieval and QA tasks.
% \end{itemize}
















% Throughout this paper, we will present our methodology, experiments, and findings, emphasizing our approach to multihop reasoning over varying numbers of input images. We believe that our work contributes to a deeper understanding of multimodal reasoning and has the potential to enhance the capabilities of question-answering systems in the intricate, multimodal landscape of web-based information.
\vspace{-0.02in}
We discuss three lines of related work: chart-to-code generation, multi-agent framework, and test-time scaling research.
\vspace{-0.01in}
\subsection{Chart Generation with VLMs}

Chart generation, or chart-to-code generation,  is an emerging task aimed at automatically translating visual representations of charts into corresponding visualization code \cite{shi2024chartmimic, wu2024plot2code}. This task is inherently challenging as it requires both visual understanding and precise code synthesis, often demanding complex reasoning over visual elements.

Recent advances in Vision-Language Models (VLMs) have expanded the capabilities of language models in tackling complex multimodal problem-solving tasks, such as visually-grounded code generation.
% \kw{It's a bit unclear to me what do you mean by multimodal reasoning and how it is related to code generation}
Leading proprietary models, such as GPT-4V~\cite{GPT4V}, Gemini~\cite{Gemini}, and Claude-3~\cite{Claude}, have demonstrated impressive capabilities in understanding complex visual patterns.
% \kw{what do you mean by structured outputs here?} 
The open-source community has contributed models like LLaVA~\cite{xu2024llava-uhd, li2024llavanext-strong}, Qwen-VL~\cite{Qwen-VL}, and DeepSeek-VL~\cite{lu2024deepseekvl}, which provide researchers with greater flexibility for specific applications like chart generation.

Despite these advancements, current VLMs often struggle with accurately interpreting chart structures and faithfully reproducing visualization code. 

\begin{figure*}[ht]
    \centering
    \includegraphics[width=1\linewidth]{figs/method.pdf}
    \caption{Overview of \model{}: A multi-agents system that consists of four specialized agents working in an iterative pipeline: (1) Generation Agent creates initial Python code to reproduce the reference chart, (2) Visual Critique Agent identifies visual discrepancies between the generated and reference charts, (3) Code Critique Agent analyzes the code and provides specific improvement guidelines, and (4) Revision Agent modifies the code based on the critiques. The process iterates until either reaching the verification score or maximum attempts limit.} %Each agent has a clearly defined objective and operates on specific inputs and outputs, enabling systematic improvement of the generated visualization.}
    \vspace{-0.1in}
    \label{fig:method}
\end{figure*}


\subsection{Multi-Agents Framework}

Many researchers have suggested a paradigm shift from single monolithic models to compound systems comprising multiple specialized components~\cite{compound-ai-blog, du2024compositional}. One prominent example is the multi-agent framework.

LLMs-driven multi-agent framework  has been widely explored in various domains, including narrative generation~\cite{huot2024agents}, financial trading~\cite{xiao2024tradingagents}, and cooperative problem-solving~\cite{du2023improving}.

Our work investigates the application of multi-agent framework to the visually-grounded code generation task.



\subsection{Test-Time Scaling} 
% \kw{I feel this subsection is a bit irrelevant }

Inference strategies have been a long-studied topic in the field of language processing. Traditional approaches include greedy decoding \cite{teller2000speech}, beam search \cite{graves2012sequence}, and Best-of-N.

Recent research has explored test-time scaling law for language model inference. For example, \citet{wu2024inference} empirically demonstrated that optimizing test-time compute allocation can significantly enhance problem-solving performance, while \citet{zhang2024scaling} and \citet{snell2024scaling} highlighted that dynamic adjustments in sample allocation can maximize efficiency under compute constraints. Although these studies collectively underscore the promise of test-time scaling for enhancing reasoning performance of LLMs, its existence in other contexts, such as different model types and application to cross-modal generation,  remains under-explored. 

\section{System}
\begin{figure*}[h]
    \centering
    \includegraphics[width=.85\textwidth]{fig/SYSTEM_IMAGE_TEST_FLIPPED.png}
    \caption{HumorSkills System Diagram. Given an image, the system first extracts visual details with a visual language model, then performs visual humor ideation to analyze the image and propose humorous angles. It then generates ten potential conflicts that could be used to extrapolate the image into a relatable experience. The system then generates humor with and without the narratives, for diversity. A separate instance of the LLM trained to rank gen-Z humor ranks all the captions and returns the top five.}
    \Description{HumorSkills System Diagram}
    \label{fig:system}
\end{figure*}

HumorSkills is a system that takes an input image and outputs 5 image captions. 
The architecture has three key steps that mimic human skills needed for humor. \textit{Visual Detail Extraction}, is a step that describes the image in depth in order to make non-obvious observations about it. \textit{Narrative and Conflict Extrapolation} is a step that finds narratives not in the image that could be related to it, to expand the topic of jokes to things that are not just in the image but also analogous to it.  \textit{Fine-tuning} the joke generator with examples of good Gen-Z humor helps the jokes be more relatable to the target audience by using references, slang, topics, and insecurities that resonate with this group.

% first, 
% a \textbf{divergent stage} where the image is analysed and multiple observations, angle, alternative narratives and humorous angles are generated. 
% Second, a \textbf{generation stage} where two types of captions are generated: 1) captions focusing on image content directly 2) captions that bring in an outside narrative to the image, often bringing in outside references. It generates 15 captions of each type. (Figure 1 has examples Of the Content Focused, and Narrative Expanded Captions). Finally, a \textbf{ranking stage} where a separate AI agent selects the top 5 captions

The system generates two types of captions: image-focused captions which common directly on the content in the image, and narrative-driven captions. Variety is important to humor. Humor relies on surprise, and jokes that are too similar start to become more predictable. Additionally, with an infinite set of input images with different subjects and situations, there are more strategies needed to find a humorous angle that fits the content. 

% With caption-based humor, often the humor can be focused on finding something in the image that is inherently interesting. 

% For example, the caption ”little man really thought he could escape bedtime” relies solely on information in the image. However, some images don’t have something funny in and of themselves, and it’s easier to make a joke by bringing in a new unrelated angle. For example, ``the police chasing me when I'm broke and in debt to the tune of \$100,000 for student loans''. Generally, Images with people doing interesting things lend themselves to visual humor because there are many stories one could tell about it. However, for images with only static objects, it's more difficult to tell a story on only the objects, so bringing a new story in is another way to find humor. 

\subsection{AI Humor Generation Walk Through}
Figure \ref{fig:system} contains a visual diagram and example of intermediate outputs when generating captions for an image. We describe each phase and implementation in detail.  
% The main contribution of this paper is the evaluation, rather than the system, but it is still it is important to understand the mechanism used to generate humor.
% Although the individual components of the system are not totally, the combination of features including the

\subsubsection{Visual Detail Extraction}

The first phase of the system’s workflow involves the Visual Detail Extraction component, which utilizes GPT-4o’s vision capabilities to analyze the input image. This system incorporates a prompt that asks for a detailed paragraph that explains the who/what/where of the image, distinguishing between identifying the subject of the image, the main action of the image if it exists, and the background elements of the image. This component is responsible for extracting key visual elements such as objects, human expressions, background settings, and any notable aspects that could serve as the foundation for humor.

For instance, in the demolition site example from the system diagram, the system identifies a large industrial demolition excavator and a person with a hose spraying the demolition site. 

\subsubsection{Visual Humor Ideation}

On top of the visual detail extraction, the system ideates on possible humorous elements from the visual of the image. This incorporates an additional prompt using GPT-4o that intakes the image and asks it to identify and ideate on potential humorous visual elements in the image, whether they are directly humorous elements, such as funny facial expressions, or more analogous elements. For example, for the system diagram image, the system noted the visual contrast of the excavator and person, reminiscent of a David versus Goliath scenario, which provides a foundational metaphor for generating humorous captions. 

\begin{figure*}[b]
    \centering
    \includegraphics[width=.95\textwidth]{fig/Workflow.png}
    \caption{A diagram for how narrative extrapolation works}
    % \Description{}
    \label{fig:systemLines}
\end{figure*}

\subsubsection{Narrative and Conflict Extrapolation}

In this next step, the system generates a narrative and conflict framework by drawing upon common and relatable Gen Z experiences such as work, school, family, social interactions, relationships, and more. 
The system chains together the results of the previous steps, into a new prompt sent to GPT-4o. 
% The system prompts GPT-4o to 
% GPT-4o is utilized by incorporating the text description of the image and potential humorous elements of the image, then being prompted to generate relatable scenarios applicable to the image description from a list of common Gen Z experiences. 
The prompt contains the visual details, the visual humor ideation, and a list of common Gen Z experiences,  and the instruction to "generate narratives that reflect the essence of the image that is set within the framework of the Gen Z experience."
This narrative generation adds depth to the humorous captions by applying relatable themes and conflicts to the visual elements identified earlier.

For instance, our system diagram generates narratives such as “Tackling student loans”, "Group Project Disaster", and “Relationship Issues” based on the image, both of which are common experiences among those who identify as Gen Z. These particular narratives are likely inspired by the imagery of a disaster site, referring to how the effort of paying off student loans, attempting to complete group projects during school, or addressing relationship -- all of which can feel like disaster clean up. These relatable conflicts can transform the visual of a demolition scene -- a setting that is not particularly relatable -- into a relatable scenario that has the potential for humor, thereby expanding the humorous possibilities by connecting the visual input with broader life experiences.



\subsubsection{Humorous Caption Generation}

Following the narrative and conflict extrapolation, the system generates humorous captions in the generation stage using a fine-tuned version of GPT-3.5 trained on humorous Instagram comments. This involves producing captions through two distinct strategies: one focused on the visual humor of the image, and the other by bringing in the previously generated external narratives. Caption generation is segmented into two separate prompts utilizing the fine-tuned GPT-3.5 model. For captions without generated external narratives, the prompt asks to generate 15 humorous captions in the style of Gen Z that bases the generation off the visual extraction and visual humor ideation of the input image. For captions with the external narratives, the prompt also asks to generate 15 humorous captions in the style of Gen Z that bases the generation off the visual extraction and visual humor ideation of the input image, but also asks the system to incorporate the list of generated narrative/conflict extrapolations to base the humorous captions off of.

Image-focused captions rely solely on the visual details in the image, such as “bro out here getting paid \textdollar8 an hour to spray some water on some bricks,” which references the direct visual elements in the scene in order to generate a caption. This particular caption directly references the humor of the image, poking fun at the minimal impact of the person spraying water on bricks while an excavator clearly has more impact on the demolition site. Narrative-driven captions, on the other hand, introduce external references to add humor. For instance, a caption like “The entitled bro you tried to make the group presentation with” introduces an outside, exaggerated, interpretation of the scene from earlier, "Group Project Disaster." This caption takes the group project narrative and pairs it with the visual of the image, analogizing the person spraying the hose with minimal impact on the demolition site to an entitled person who has not done much to complete the group project. 

This variety between visual humor and narrative-driven humor is crucial because jokes that are too similar become predictable, losing their element of surprise. Additionally, humor strategies need to adapt to the varying content in input images. Some images lend themselves to humor based on their inherent visual details, while others require bringing in outside references to create a joke. For instance, an image of static objects might not be inherently funny, such as the demolition image shown in the system diagram, but a caption introducing an unrelated, exaggerated narrative, such as “Eboy doin' his part to stop climate change” can inject humor and absurdity by making an unexpected connection.

\subsubsection{Caption Ranking using Gen Z Agent}

The final component of the system architecture is the Caption Ranking and Filtering Agent, a GPT-4o-based agent fine-tuned to evaluate humor from a Gen Z perspective. This agent receives the list of 30 total captions from the narrative and visual humor-based caption generations and ranks the captions generated in the previous stage based on humor, relatability, and alignment with the image and narrative.

As illustrated in our system diagram, this agent ranks captions such as “Me mopping up my last relationship” and “me pulling the emotional weight of the friend group” based on their relevance to Gen Z humor. Captions that fail to meet the humor threshold are filtered out, such as "Demolition worker really said 1v1 me bro," because although the phrase like "1v1 me bro" invokes Gen Z phrases, the content of the caption seems less relevant and relatable than a caption talking about school or relationships, ensuring that only the most effective and relatable captions are presented to the user.

\subsubsection{Fine-tuning}

To fine-tune a GPT-3.5 model, a dataset of 80 humorous comments were extracted from popular Instagram images. From three popular Instagram meme pages with over 400,000 followers, the top five comments of each image post were collected. All fit the style of Gen-Z humor. 
% The fine-tuning process ensured that the generated captions align with the humor style favored by Gen Z. 
Examples of the visual description of the images in addition to an explanation of potential humorous elements of the image were written in the fine-tuning prompts, then followed by the actual comment itself. This reflected the visual extraction and humor ideation being incorporated into the prompt of our current system.



\section{Accuracy in Distinguishing AI-generated Images from Real Photographs}\label{sec:results}

In the main phase of the experiment, we collected 539,749 responses on 599 images from 37,568 participants from February 5, 2024 to June 22, 2024. Sections~\ref{sec:acc-general} through \ref{sec:acc-model} focus on data from the main phase of the experiment. The second phase of the experiment started on June 22 and ended on August 30, with 83,577 responses on 482 images from 3,787 participants. Sections~\ref{sec:imagestimuli} and~\ref{sec:human-curation} describe the influence of human curation of the stimuli on how accurately participants identify the stimuli as AI-generated or real. 

The design of our experiment involves several important design choices. First, we selected the three models
of Midjourney, Firefly, and Stable Diffusion as the diffusion models. Second, we crafted prompts to produce realistic
outputs across various pose categories and content types. Third, we curated 450 images from over 3000 images generated
to use as image stimuli in the experiment. These images were selected to maximize realism while also representing
different visual artifacts and implausibilities. Inevitably, these design choices on models, prompts, and stimuli introduce some selection bias  into the experiment.

Additionally, we implemented two exclusion criteria that should be considered when interpreting our results. First, for all the analyses in Section ~\ref{sec:results}, we excluded observations where participants checked the box on the website ``I have seen this before''. These observations, which account for 2\% of the total observations, were excluded because of the strong possibility that participants who had previously seen the images were already aware of whether they were fake or real.
%the experiment aimed to evaluate whether participants could identify if an unfamiliar image was AI-generated.. 
For these observations marked as having been seen before, 38\% of these observations were on AI-generated stimuli and 62\% were on real images. The image most frequently reported as 'seen before' is a real portrait of Martin Luther King Jr, which was one of the few real images of a well-known celebrity included in the experiment. 

Second, in line with our goals of studying detection ability on images for which there was some ambiguity, we excluded all images where participants' accuracy suggested very little ambiguity. We operationalized this as accuracy above 90\%. 

These exclusion criteria remove all observations on 68 fake images and 4 real images, which represent 14\% of observations from the entire experiment. 

In the human-coded analysis of artifacts discussed in Section~\ref{sec:acc-presence-artifacts}, we apply an additional exclusion criterion to make the coding tractable. Specifically, we exclude all images accurately identified in more than 80\% of observations. This exclusion criterion focuses the analysis on the most challenging images by excluding the most egregious distortions that lead to low photorealism (i.e., high participant accuracy).
%and offers a lower bound on the full extent of the differences between image categories because .

\subsection{Overall Accuracy} \label{sec:acc-general}

In the main study, participants correctly identified AI-generated images and authentic photographs in 76\% and 74\% of observations, respectively. Accuracy varied substantially across images. Prior to implementing our accuracy-based exclusion described above, we found that for AI-generated images, accuracy ranged from 32\% to 99\%. Similarly, accuracy on real photographs ranged from 28\% to 92\%. Figure~\ref{fig:accuracy_real_fake} shows the distribution of accuracy in both AI-generated and real images with example images selected from the top, bottom, and middle deciles of each distribution. At the image level, the mean accuracy for identifying AI-generated and real images was 76\% (95\% CI:[74,77]) and 74\% (95\% CI:[72,76]), respectively. 

Despite our efforts to minimize obvious artifacts, some images - particularly non-portraits - were challenging to generate without noticeable artifacts. As a result, participants achieved nearly 100\% accuracy on a few AI-generated images with obvious features. We present examples of these images in Figure~\ref{fig:three-fake-images}.  %that further motivate the exclusion criteria that we apply to the rest of the results section. 
In contrast to AI-generated images, real photographs rarely contain definitive artifacts and visual cues often seen in AI-generated images, which limits participants from achieving near-perfect accuracy on real photographs.
\begin{figure}[H]
    \centering
    \includegraphics[width=\linewidth]{sections/images/general_accuracy.pdf}
    \caption{Distribution of accuracy scores for real and AI-generated images with example images representing different accuracy levels.}
    \label{fig:accuracy_real_fake}
    \Description{Histograms showing the distribution of accuracy scores for real and AI-generated images, accompanied by example images representing various accuracy levels.}
\end{figure}
\begin{figure}[H]
\centering
\captionsetup{justification=raggedright, singlelinecheck=false, skip=2pt, font=small}
\begin{subfigure}[t]{0.3\linewidth}
    \subcaption{}\vtop{\vskip0pt\hbox{\includegraphics[width=\linewidth]{sections/images/sd_portrait3_040.jpg}}}
\end{subfigure}
\hfill
\begin{subfigure}[t]{0.3\linewidth}
    \subcaption{}\vtop{\vskip0pt\hbox{\includegraphics[width=\linewidth]{sections/images/ff_pg3_010.jpeg}}}
\end{subfigure}
\hfill
\begin{subfigure}[t]{0.3\linewidth}
    \subcaption{}\vtop{\vskip0pt\hbox{\includegraphics[width=\linewidth]{sections/images/ff_fullbody3_007.jpeg}}}
\end{subfigure}
\caption{\mybold{Examples of obviously AI-generated images and their corresponding accuracy.} \normalfont{\textbf{A.} AI-generated portrait with 92\% accuracy. \textbf{B.} AI-generated posed group image with 95\% accuracy. \textbf{C.} AI-generated full-body image with 99\% accuracy.}}
\label{fig:three-fake-images}
\Description{Three examples of obviously AI-generated images with corresponding accuracy scores: A. Portrait with 92\% accuracy, B. Posed group image with 95\% accuracy, C. Full-body image with 99\% accuracy. }
\end{figure}

\subsection{Participant Level Accuracy}\label{sec:indiv-acc}

\begin{figure*}[h]
    \centering
    \captionsetup{justification=raggedright, singlelinecheck=false, skip=2pt, font=small}

    % Top Row - A and B
    \begin{subfigure}[t]{0.4\textwidth}  
        \centering
        \subcaption[]{}  
        \vspace{-3pt}  
        \includegraphics[width=\linewidth]{sections/images/scatter_plot_fake_images.jpg} 
        % \subcaption{}
    \end{subfigure}
    \hspace{0.05\textwidth} 
    \begin{subfigure}[t]{0.38\textwidth}  
        \centering
        \subcaption[]{}  
        \vspace{-3pt}
        \includegraphics[width=\linewidth]{sections/images/scatter_plot_real_images.jpg}
        % \subcaption{}
    \end{subfigure}

    % Bottom Row - C and D
    \begin{subfigure}[t]{0.36\textwidth}  
        \centering
        \subcaption[]{}  
        \vspace{-3pt}
        \includegraphics[width=\linewidth]{sections/images/accuracy_distribution.jpg}
        % \subcaption{}
    \end{subfigure}
    \hspace{0.05\textwidth}  
    \begin{subfigure}[t]{0.36\textwidth}  
        \centering
        \subcaption[]{}  
        \vspace{-3pt}
        \includegraphics[width=\linewidth]{sections/images/learning_curve_with_fake_real_bias.png}
        % \subcaption{}
    \end{subfigure}

    \caption{\textbf{Participant-level accuracy and learning trends.}  
    \normalfont{ \textbf{A.} Scatterplot of participant-level accuracy for AI-generated images. \textbf{B.} Scatterplot of participant-level accuracy for real images. \textbf{C.} Histogram showing the distribution of accuracy across the first ten images seen by participants who viewed at least 10 images. \textbf{D.} Learning curve illustrating accuracy trends and classification biases when detecting AI-generated and real images.}}
    
    \label{fig:combined-participant-accuracy}

    \Description{A composite figure showing participant accuracy trends:  
    A. Scatterplot displaying accuracy levels for detecting AI-generated images, with points representing individual participants.  
    B. Scatterplot for real images, structured similarly to A.  
    C. Histogram showing the distribution of accuracy for participants' first 10 images.  
    D. Learning curve tracking accuracy trends over time, highlighting biases in AI-generated and real image classification.}
\end{figure*}

Given the organic nature of participants' engagement with this experiment, we did not impose restrictions on the number of images a participant saw. Most participants in this study provided responses to at least seven images, but some participants only provided a single response, and one participant provided 502 responses. 

The vast majority of participants (75\%) saw 16 or fewer images. Figure~\ref{fig:combined-participant-accuracy}A and B present the distribution of participant--level accuracy by number of viewed images. 

In order to compare participant performance and avoid issues that arise with differential attrition, we focus on the first ten images seen by participants who saw at least 10 images, which includes 152,050 observations from 15,205 participants. First, we note that 34\% of these participants achieved 90\% accuracy or higher on the first ten images seen. If the AI-generated images were perfectly photorealistic such that the human ability to distinguish is no higher than random guessing, then we would have expected only 1\% of participants to achieve this threshold of accuracy (assuming random guessing at 50\% accuracy, with participants evaluating 10 images each, achieving at least 9 out of 10 correct responses would occur with a probability of approximately 1.07\%, based on the binomial probability distribution). Figure~\ref{fig:combined-participant-accuracy}C shows the distribution of accuracy across the first ten images seen by participants who saw at least 10 images.

In Figure~\ref{fig:combined-participant-accuracy}D, we present accuracy rates by the number of images seen. We find that on average, participants begin the experiment by disproportionally identifying images as fake in 63\% of observations. Notably, this bias is reduced after only a few images.


\subsection{Accuracy by Scene Complexity} \label{sec:acc-scene-complexity}

We find that on average, participants' accuracy increases as scene complexity increases. For example, we find that 16\% of portraits appear in the bottom decile of accuracy scores (representing the highest level of photorealism), whereas only 3\% of AI-generated posed group images appear in the bottom decile. Figure~\ref{fig:pose-complexity} presents the distribution of accuracy for each category, separately for real and AI-generated images. For AI-generated images, the mean accuracy was 72.7\% (95\% CI: [72.4, 72.9]) for portraits, 77.2\% (95\% CI: [76.8, 78.6]) for full body, 76.2\% (95\% CI: [75.8, 76.7]) for posed groups, and 73.4\% (95\% CI: [73, 73.8]) for candid groups. For real images, the accuracy was 71.1\% (95\% CI: [70, 71.4]) for portraits, 75.5\% (95\% CI: [75.1, 75.8]) for full body, 76.7\% (95\% CI: [76.3, 77]) for posed groups, and 74.8\% (95\% CI: [74.4, 75.1]) for candid groups. 

As exemplified in Figure~\ref{fig:pose-complexity}C, we note that portraits, relative to the other levels of scene complexity, typically have less detail, simpler and more standardized poses, more blurred backgrounds, and fewer available cues than full-body or group images. 
\begin{figure}[h]
    \captionsetup{justification=raggedright, singlelinecheck=false, skip=2pt, font=small}
    \centering

    % First subfigure - Reduce space
    \begin{subfigure}[t]{\linewidth}
        \subcaption{}
        \vspace{-12pt}  % Reduce vertical space
        \includegraphics[width=\linewidth]{sections/images/scene_complexity_filtered90_Real.png}
    \end{subfigure}
    
    % Second subfigure - Reduce space
    \begin{subfigure}[t]{\linewidth}
        \subcaption{}
        \vspace{-12pt}  % Reduce vertical space
        \includegraphics[width=\linewidth]{sections/images/scene_complexity_filtered90_AI-generated.png}
    \end{subfigure}
    
    % Third subfigure - Reduce space
    \begin{subfigure}[t]{\linewidth}
    \centering
        \subcaption{}
        \vspace{-12pt}  % Reduce vertical space
        \includegraphics[width=0.9\linewidth]{sections/images/scene_complexity_B.pdf}
    \end{subfigure}
    
    \caption{\textbf{Scene complexity}: Accuracy of real and AI-generated images by scene complexity levels. 
    \normalfont{Beeswarm plots of image-level accuracy for each dimension of scene complexity with bootstrapped 95\% confidence intervals. We exclude images identified with above 90\% accuracy in this analysis. \textbf{A.} Real images \textbf{B.} AI-generated images \textbf{C.} AI-generated images across scene complexities.}}
    
    \label{fig:pose-complexity}
    
    \Description{Beeswarm plots showing the accuracy of real and AI-generated images by pose complexity. Each dot represents an individual image, with error bars indicating the bootstrapped 95\% confidence interval around the mean.}
\end{figure}


\subsection{Accuracy by Presence of Artifacts}\label{sec:acc-presence-artifacts}
\begin{figure*}[h]
\centering
\captionsetup{justification=raggedright, singlelinecheck=false, skip=2pt, font=small}

% Full PDF Image
\includegraphics[width=\linewidth]{sections/images/combined_artifact_trends.pdf}

\caption{\textbf{Accuracy by artifact types and display times} \normalfont{\textbf{A. Mean accuracy for different artifact types.} Distribution of accuracy scores by artifact type
for images with at least one artifact. \textbf{B. Mean accuracy over display time.} Change in mean accuracy across different display time assignments (1 second, 5 seconds, 10 seconds, 20 seconds, and unlimited) with 95\% confidence intervals and bee swarm plots of image accuracy for AI-generated and real images. \textbf{C. Mean accuracy over time for different artifact types.} Change in mean accuracy across different time assignments (1 second, 5 seconds, 10 seconds, 20 seconds, and unlimited) with 95\% confidence intervals and bee swarm plots of image accuracy for
images with anatomical (pink), functional (green), and stylistic
(blue) artifacts. The x–axis shows the display time intervals, and the
y–axis shows accuracy.}}

\label{fig:combined-artifact-trends}

\Description{A composite figure showing accuracy-related analyses:  
(A) A beeswarm plot displaying the distribution of accuracy scores for images containing at least one artifact, categorized by artifact type.  
(B) A line plot illustrating mean accuracy across different display time conditions (1 second, 5 seconds, 10 seconds, 20 seconds, and unlimited), with 95\% confidence intervals. Overlaid bee swarm plots represent individual accuracy scores for AI-generated and real images.  
(C) A line plot showing mean accuracy over time for different artifact types (Anatomical, Functional, and Stylistic). Each artifact type is color-coded (pink for Anatomical, green for Functional, and blue for Stylistic). Bee swarm plots depict individual accuracy scores for images within each artifact category. The x-axis represents display time intervals, and the y-axis represents accuracy.}

\end{figure*}


In order to analyze accuracy by artifact type, we annotated images with diffusion model artifact categories from the taxonomy based on a three-step process. First, four co-authors independently annotated all 218 images with accuracy below 80\%, identifying artifacts and providing detailed explanations for their annotations. Second, each of these annotations was reviewed and edited by two additional co-authors. Third, a fifth co-author reviewed all annotations for consistency. Figures~\ref{fig:combined-varying-artifacts-visibility}A--C and \ref{fig:combined-varying-artifacts-visibility}D--F provide examples of how we annotated images, displaying the identified artifact categories, the reasoning behind their identification, and the associated detection accuracy for each image. During this process, we observed that the three main artifact types---anatomical implausibilities, stylistic artifacts, and functional artifacts---each appeared in nearly a third of the images we annotated. In contrast, violations of physics and sociocultural implausibilities were less common, appearing in only 20 and 12 images, respectively. In light of this distribution of artifacts, Figure~\ref{fig:combined-artifact-trends}A presents the distribution of accuracy scores across images containing at least the three listed artifact types.

Based on our annotations of artifacts in images, we find participants are less accurate on images with functional implausibilities than images with anatomical implausibilities or stylistic artifacts. The mean accuracy on images with at least one functional implausibility, one anatomical implausibility, and one stylistic artifact is 64.1\% (95\% CI: [63.8, 64.5]), 65\% (95\% CI: [64.6, 65.4]), and 64.9\% (95\% CI: [64.5, 65.3]), respectively. While the accuracy on images with functional implausibilities is lower than on images with other implausibilities and artifacts, the mean accuracy scores are similar. However, this similarity in means masks the differences in the distribution of accuracy scores, as shown in Figure~\ref{fig:combined-artifact-trends}A. We find that images with participant accuracy scores in the 40--60\% range (which represent images approaching indistinguishability between real and AI-generated) make up 32.8\% of images annotated with functional implausibilities compared to 21.4\% and 22.4\% of images annotated with anatomical implausibilities and stylistic artifacts, respectively. 


We find that images that we annotated as containing multiple artifacts can still appear photorealistic enough to make detection difficult for most people. Artifacts vary in levels of visibility, as shown in Figure~\ref{fig:combined-varying-artifacts-visibility}A--C. While Figure~\ref{fig:combined-varying-artifacts-visibility}A and C contain stylistic artifacts, they are far more apparent in Figure~\ref{fig:combined-varying-artifacts-visibility}B, which is reflected in its higher detection accuracy. Despite Figure~\ref{fig:combined-varying-artifacts-visibility}A and C containing multiple artifact categories, they had low detection accuracy, suggesting that the presence of multiple artifacts does not necessarily make images easier to identify and that artifact visibility is also a contributing factor. 


The visibility of artifacts is highly variable, and Figure~\ref{fig:combined-varying-artifacts-visibility}D--F present examples highlighting this variability. The anatomical implausibility in the fingers in image Figure~\ref{fig:combined-varying-artifacts-visibility}D is very noticeable, whereas the functional implausibilities in the tennis racket and shirt design of Figure~\ref{fig:combined-varying-artifacts-visibility}F are more subtle. The corresponding accuracy scores for these images--- 62\% for Figure~\ref{fig:combined-varying-artifacts-visibility}E and 54\% for Figure~\ref{fig:combined-varying-artifacts-visibility}F —reinforce the observation that anatomical artifacts tend to be more easily detected, while functional implausibilities often require closer attention and familiarity with depicted objects. The stylistic artifacts in the cinematization of Figure~\ref{fig:combined-varying-artifacts-visibility}E and plastic-like skin texture fall in between, further showing the spectrum of detectability across different artifact categories and visibility. 


\begin{figure*}[h!]
\centering
\captionsetup{justification=raggedright, singlelinecheck=false, skip=2pt, font=small}

% First Row of Images
\begin{subfigure}[t]{0.27\linewidth}
\centering
    \subcaption{}
    \includegraphics[width=\linewidth]{sections/images/8059c316907c586bdf33ad3cb9ca3f95.jpeg}
\end{subfigure}
\hspace{1cm}
\begin{subfigure}[t]{0.27\linewidth}
\centering
    \subcaption{}
    \includegraphics[width=\linewidth]{sections/images/616ba73f50088eb13244a807076248f7.jpeg}
\end{subfigure}
\hspace{1cm}
\begin{subfigure}[t]{0.27\linewidth}
\centering
    \subcaption{}
    \includegraphics[width=\linewidth]{sections/images/2c4c0b171577884f5c0991cacb5c5ebc.jpeg}
\end{subfigure}

\vskip 5mm % Adds vertical spacing between rows

% Second Row of Images
\begin{subfigure}[t]{0.27\linewidth}
\centering
    \subcaption{}
    \includegraphics[width=\linewidth]{sections/images/ff_pg3_009.jpeg}
\end{subfigure}
\hspace{1cm}
\begin{subfigure}[t]{0.27\linewidth}
\centering
    \subcaption{}
    \includegraphics[width=\linewidth]{sections/images/mj_portrait3_010.jpeg}
\end{subfigure}
\hspace{1cm}
\begin{subfigure}[t]{0.27\linewidth}
\centering
    \subcaption{}
    \includegraphics[width=\linewidth]{sections/images/ff_portrait3_004.jpeg}
\end{subfigure}

\caption{\textbf{Examples of images with varying artifact visibility.}  
\normalfont{\textbf{Top row (A--C):} Example images showcasing stylistic and functional artifacts with varying visibility.  
\textbf{A.} A subtle stylistic artifact in the soft and wispy textures of the woman's hair and a minor functional implausibility in the atypical design of her shirt collar (Accuracy: 47\%). \textbf{B.} An obvious stylistic artifact due to the overall cinematization of the image (Accuracy: 73\%). \textbf{C.} A combination of multiple artifacts, including anatomical implausibilities in the woman's hand, functional implausibilities in the table shape and wall panels, and a stylistic artifact in the soft texture of the woman's face (Accuracy: 38\%). \textbf{Bottom row (D--F):} Images with anatomical, stylistic, and functional artifacts of varying visibility. \textbf{D.} Anatomical implausibilities in the fingers of the three students (Accuracy: 84\%). \textbf{E.} A stylistic artifact in the cinematized look and plastic-like texture of the woman's skin (Accuracy: 62\%). \textbf{F.} No obvious anatomical or stylistic artifacts, but closer inspection reveals functional implausibilities: the tennis racket is asymmetrical, its strings are not taut, and the shirt has irregularly shaped designs with glitch-like inconsistencies (Accuracy: 54\%).}}

\label{fig:combined-varying-artifacts-visibility}

\Description{A composite figure showing six images with varying visibility of AI-generated artifacts.  
(A--C) The first row highlights stylistic and functional artifacts, including wispy hair, cinematized lighting, and a distorted table.  
(D--F) The second row focuses on anatomical, stylistic, and functional artifacts, including distorted fingers, plastic-like textures, and inconsistencies in objects like a tennis racket.}
\end{figure*}

\subsection{Accuracy by Randomized Display Time}\label{sec:acc-time}
\begin{figure*}[h]
\centering
\captionsetup{justification=raggedright, singlelinecheck=false, skip=2pt, font=small}
\begin{subfigure}[t]{0.27\linewidth}
\centering
    \subcaption{}
    \includegraphics[width=\linewidth]{sections/images/bbbfb2a12cd66783ce7e4015ec0084b9.jpg}
\end{subfigure}
\hspace{1cm}
\begin{subfigure}[t]{0.27\linewidth}
\centering
  \subcaption{}
    \includegraphics[width=\linewidth]{sections/images/5204545de13342cbefdc0e9022d821d2.jpg}
\end{subfigure}
\hspace{1cm}
\begin{subfigure}[t]{0.27\linewidth}
\centering
  \subcaption{}
    \includegraphics[width=\linewidth]{sections/images/ccc04b661d52c055a44fc01718c6a2bc.jpg}
\end{subfigure}
\caption{\textbf{Exemplar AI-generated images for which a closer look improves accuracy.} \normalfont{\textbf{A.} Accuracy: 38\% at 1 second display time to 65\% at 20 second display time. \textbf{B.} Accuracy: 44\% at 1 second display time to 82\% at 20 second display time. \textbf{C.} Accuracy: 27\% at 1 second display time to 70\% at 20 second display time.}}
\label{fig:displaytime}
\Description{Three images showing AI-generated images for which a closer look improves accuracy. (A) Image of woman generated by Stable Diffusion: Accuracy is 38\% at 1 second display time and it improves to 65\% at 20 second display time with  (B) Image of people generated by Stable Diffusion: Accuracy is 44\% at 1 second display time and it improves to 82\% at 20 second display time with (C) Image of people generated by Stable Diffusion: Accuracy is 27\% at 1 second display time and it improves to 70\% at 20 second display time with.}
\end{figure*}

By randomizing the display time of images in this experiment, our results support evaluating how viewing duration influences participants' accuracy. We find that longer viewing times improve performance. With just 1 second of display time, participants are  72\% accurate (95\% CI=[71.6, 72.5], 95\% CI=[71.3, 72.2]) on AI-generated and real images, respectively. With 5 seconds of display time, accuracy increases to 77\% (95\% CI=[77.0, 77.8], 95\% CI=[76.6, 77.4]) for both AI-generated and real images, respectively. While accuracy on real images appears to plateau by 5 seconds of display time, accuracy on AI-generated images increases up to 80\% (95\% CI=[79.6, 80.4]) at 10 seconds and 82\% (95\% CI=[81.2, 81.9]) at 20 seconds. Figure~\ref{fig:combined-artifact-trends}B presents the distribution of accuracy scores across display time conditions. Across the observations where display time was randomized, we find that the proportion of AI-generated images that are identified below random chance decreases from 43\% when participants only have 1 second to view the image to 30\%, 25\%, 17\%, and 17\% when participants have 5, 10, 20 seconds, and unlimited time to view the image.

In some images, AI artifacts can be noticed with a quick glance, but for others, careful attention to detail is necessary to spot the artifact. Figure~\ref{fig:displaytime} presents three images that require careful attention, as evidenced by the fact that most participants mark as real when they are limited to seeing the image for a second but
fake once they take into account the details of the scene.

Accuracy across all artifact types improved with increased display time. As shown in Figure~\ref{fig:combined-artifact-trends}C, participants showed higher accuracy when images were displayed for longer time (anatomical artifacts: 63\% at 5 seconds vs. 59\% at 1 second; stylistic artifacts: 63\% at 5 seconds vs. 60\% at 1 second; functional artifacts: 60\% at 5 seconds vs. 55\% at 1 second). For all artifacts, there is a significant improvement in detection accuracy when increasing display time from 5 seconds to unlimited. 

In Figure~\ref{fig:combined-artifact-trends}C, we observe that participants improved the most in identifying functional artifacts, with an 18\% improvement from 1 second to unlimited viewing time. In comparison, anatomical and stylistic artifacts showed smaller improvements of 11\% each over the same time interval. Unlike anatomical and stylistic implausibilities that can be identified at first glance, functional artifacts often require a closer look and familiarity with the elements in the image as they often appear in parts of the image that are not the main subject. 


\subsection{Qualitative Analysis of Participant Comments}\label{sec:qualitative-analysis}

We collected 34,675 comments from participants who filled out the optional text input box asking participants: ``If you think this is AI-generated, please explain why.'' In order to identify themes from these 34,675 comments, we prompted GPT-3.5 Turbo to identify 10 main themes across these comments. GPT-3.5 Turbo responded with the following ten themes, which we manually reviewed and refined to mitigate the ambiguities and generalization typical of large language models \cite{stephan2024rlvflearningverbalfeedback}: (1) Image quality focusing on the overall appearance, smoothness, and sometimes unrealistic perfection of image elements; (2) Facial and anatomical inconsistencies where participants pointed to irregularities in eyes, mouths, noses, skin texture, expressions, and general human anatomy; (3) Anatomical and functional anomalies such as deformities, misplaced body parts, and irregularities in objects or environments; (4) Lighting and environmental inconsistencies including unnatural lighting, inconsistent shadows, and reflections; (5) Digital manipulation indicators suggesting suspicions of AI-generation or digital alteration; (6) Biometric discrepancies particularly unnatural or imperfect body parts like hands and fingers; (7) Uncanny valley perceptions where images almost looked human but had subtle unnatural features that caused discomfort; (8) Contextual incongruities such as unrealistic scenarios and mismatched social elements; (9) Physical anomalies highlighting illogical physical interactions within the images; and (10) holistic authenticity assessment making overall judgments based on a combination of multiple cues and inconsistencies. Based on these ten main themes, we prompted GPT-3.5 to label each comment with one of the ten themes. Figure~\ref{fig:comments-all} illustrates examples of participant comments for four images and how they were categorized into themes. Figure~\ref{fig:themes} displays the distribution of themes across the comments and the related concept from our taxonomy in parentheses.

Based on GPT-3.5 Turbo, we find that 61\% of participants' comments mentioned relying on anatomical implausibilities. The next most common concept referred to is stylistic artifacts, which is mentioned in 30\% of comments. Participants mentioned functional implausibilities in 21\% of comments, violations of physics in 15\% of comments, and sociocultural implausibilities in only 4\% of comments. 

Based on the authors' annotations of artifacts, we find functional implausibilities to be the most prevalent, appearing in 58.7\% of images, followed by anatomical implausibilities in 51.4\% and stylistic artifacts in 39.0\% of images. We identify violations of physics and sociocultural implausibilities in only 9.17\% and 5.50\% of images, respectively. 
\begin{figure}[H]
\centering
\captionsetup{justification=raggedright, singlelinecheck=false, skip=2pt}
\begin{subfigure}[t]{0.9\linewidth}
    % \subcaption{}
    \includegraphics[width=\linewidth]{sections/images/theme_distribution_single_column.jpg}
\end{subfigure}
\caption{\mybold{Distribution of themes identified in participant comments.}}
\label{fig:themes}
\Description{A horizontal bar chart showing the distribution of themes identified in participant comments explaining their reasoning for AI image detection. The themes include Image Quality, Facial and Anatomical Inconsistencies, Anatomical and Functional Anomalies, Lighting and Environmental Inconsistencies, Digital Manipulation Indicators, Biometric Discrepancies, Uncanny Valley Perceptions, Contextual Incongruities, Physical Anomalies, and Holistic Authenticity Assessment.}
\end{figure}
While functional artifacts were the most prevalent in human researcher annotated images, they were less frequently mentioned in participant comments annotated by GPT--3.5. Conversely, anatomical artifacts were emphasized more in participant comments than in their prevalence in annotated images. 

\begin{figure}[htb]
\centering
\captionsetup{justification=raggedright, singlelinecheck=false, skip=2pt, font=small}
\begin{subfigure}[t]{0.22\textwidth}
    \subcaption{}\vtop{\vskip0pt\hbox{\includegraphics[width=\linewidth]{sections/images/mj_ng3_007.jpg}}}
\end{subfigure}
\hfill
\begin{subfigure}[t]{0.22\textwidth}
    \subcaption{}\vtop{\vskip0pt\hbox{\includegraphics[width=\linewidth]{sections/images/mj_portrait3_001.jpg}}}
\end{subfigure}
\hfill
\begin{subfigure}[t]{0.22\textwidth}
    \subcaption{}\vtop{\vskip0pt\hbox{\includegraphics[width=\linewidth]{sections/images/mj_fullbody3_003.jpg}}}
\end{subfigure}
% \hfill
% \begin{subfigure}[t]{0.19\textwidth}\subcaption{}\vtop{\vskip0pt\hbox{\includegraphics[width=\linewidth]{sections/images/fullbody3_023.jpeg}}}
% \end{subfigure}
\hfill
\begin{subfigure}[t]{0.22\textwidth}
    \subcaption{}\vtop{\vskip0pt\hbox{\includegraphics[width=\linewidth]{sections/images/mj_pg3_002.jpg}}}
\end{subfigure}
\caption{\mybold{Examples of participant comments mapped to themes.} \normalfont{\textbf{A.} ``Cosmetic style out of character with vintage setting": Contextual Incongruities. \textbf{B.} ``Skin too smooth, depth of field shallow.": Image Quality, Lighting Inconsistencies. \textbf{C.} ``If this is not AI then it is a staged photograph like a movie set because of the lighting and he is an actor.": Lighting inconsistencies, Contextual Incongruities.
\textbf{D.} ``Group looks pasted onto background.": Digital Manipulation Indicators.}}
\label{fig:comments-all}
\Description{Four images with participant comments mapped to themes.(A) AI-generated candid image with a comment on cosmetic style being out of character with a vintage setting.(B) AI image with smooth skin, with a comment on skin being too smooth and shallow depth of field.(C) AI-generated full body shot of a man, with a comment suggesting it resembles a staged photograph due to lighting.
(D) AI image of a group, with a comment on the group looking pasted onto the background.}
\end{figure}

\subsection{Accuracy by Models} \label{sec:acc-model}

In the process of generating the images for this experiment's stimuli set, we noticed that Midjourney, Firefly, and Stable Diffusion have different capabilities and limitations. For example, we noticed that Midjourney often produced images with persistent stylistic artifacts that were challenging to eliminate. Firefly, on the other hand, frequently exhibited a tendency toward synthetic emotional expressions, with subjects often appearing unnaturally and overly cheerful, necessitating multiple iterations to produce more realistic results. Stable Diffusion struggled significantly with generating group images, often introducing artifacts such as anatomical inconsistencies. In light of the limitations to generate non-portrait images with Stable Diffusion, 75\% of the Stable Diffusion-generated stimuli in this experiment were portraits. On the other hand, 30\% of Midjourney and Firefly-generated images in this experiment depict portraits. In order to compare the three models fairly, we focus our comparison on portrait images. Figure~\ref{fig:models} presents accuracy shown on portraits by each of the three models and reveals that participants' mean accuracy on Midjourney, Stable Diffusion, and Firefly were  76\% (95\% CI: [75.2, 75.8]), 74\% (95\% CI: [73.9, 74.8]), and 73\% (95\% CI: [72.7, 73.3]), respectively. 


\begin{figure}[H]
    \centering
    \includegraphics[width=0.8\linewidth]{sections/images/model_accuracy_combined.jpg} 
    \caption{\textbf{Accuracy across generative AI models} \normalfont{Each point represents an image. The black dots and error bars show the mean accuracy and 95\% bootstrapped confidence intervals for each model}}
    \label{fig:models}
    \Description{Bee swarm chart showing accuracy across different generative AI models. The chart compares the accuracy rates for identifying AI-generated content among various models along with bootstrapped 95\% confidence intervals}
\end{figure}


\subsection{Accuracy on Human Curated Images vs. Uncurated Images} \label{sec:human-curation}
\begin{figure*}[h]
\centering
\captionsetup{justification=raggedright, singlelinecheck=false, skip=2pt, font=small}
\begin{subfigure}[t]{0.24\linewidth}
    \subcaption{}\vtop{\vskip0pt\hbox{\includegraphics[width=\linewidth]{sections/images/sd_portrait3_003.jpg}}}
\end{subfigure}
\hfill
\begin{subfigure}[t]{0.24\linewidth}
    \subcaption{}\vtop{\vskip0pt\hbox{\includegraphics[width=\linewidth]{sections/images/0bce6c35c24ec5ce8ae8ad5bb4f67d59_r4.jpg}}}
\end{subfigure}
\hfill
\begin{subfigure}[t]{0.24\linewidth}
    \subcaption{}\vtop{\vskip0pt\hbox{\includegraphics[width=\linewidth]{sections/images/0bce6c35c24ec5ce8ae8ad5bb4f67d59_r11.jpg}}}
\end{subfigure}
\hfill
\begin{subfigure}[t]{0.24\linewidth}
    \subcaption{}\vtop{\vskip0pt\hbox{\includegraphics[width=\linewidth]{sections/images/0bce6c35c24ec5ce8ae8ad5bb4f67d59_r2.jpg}}}
\end{subfigure}
\caption{\mybold{Re-generated images from the same prompt.} \normalfont{ \textbf{A.} Stage 1 image generated by Stable Diffusion and curated by our team (37\% accuracy) \textbf{B.} Most photorealistic of 12 prompt-matched image generations by Stable Diffusion (42\% accuracy) \textbf{C.} Median photorealistic of 12 prompt-matched image generations by Stable Diffusion (59\% accuracy) \textbf{D.} Least photorealistic of 12 prompt-matched image generations by Stable Diffusion(83\% accuracy)}}
\label{fig:regeneration}
\Description{Four images showing re-generated outputs from the same prompt.\textbf{A.} Stage 1 image generated by Stable Diffusion and curated by our team (37\% accuracy) \textbf{B.} Most photorealistic of 12 prompt-matched image generations by Stable Diffusion (42\% accuracy) \textbf{C.} Median photorealistic of 12 prompt-matched image generations by Stable Diffusion (59\% accuracy) \textbf{D.} Least photorealistic of 12 prompt-matched image generations by Stable Diffusion(83\% accuracy)}
\end{figure*}

\begin{figure*}[h]
\centering
\captionsetup{justification=raggedright, singlelinecheck=false, skip=2pt}
\begin{subfigure}[t]{0.48\textwidth}  
\subcaption{}
\vtop{\vskip0pt\hbox{\includegraphics[width=\linewidth]{sections/images/curation_value_add_min.jpg}}}
\end{subfigure}
\hfill
\begin{subfigure}[t]{0.48\textwidth}
\subcaption{}
\vtop{\vskip0pt\hbox{\includegraphics[width=\linewidth]{sections/images/curation_value_add_mean.jpg}}}
\end{subfigure}
\caption{\mybold{Comparing accuracy scores on curated images and uncurated prompt-matched images.} \normalfont{\textbf{A.} Scatterplot showing human detection accuracy of the original curated image compared to human detection accuracy of its most photorealistic regeneration out of 11 to 24 prompt-matched images labeled as re-generations. \textbf{B.} Scatterplot showing human detection accuracy of the original curated image compared to human detection accuracy of its mean photorealistic regeneration out of 11 to 24 re-generations.}}
\label{fig:curation-value}
\Description{Two scatterplots showing curation value analysis.(A) Minimum curation value added.(B) Mean curation value added. Each chart illustrates the impact of curation on the overall value added to the dataset.}
\end{figure*}

Generating photorealistic AI-generated images involves three key ingredients: the diffusion model, the prompt, and human curation. In this section, we examine how human curation of diffusion model-generated images influences the aggregate accuracy scores of human participants. In order to show this influence, we compare diffusion model images from the main experiment, which were curated by our research team, with multiple diffusion model images generated from the same prompt as the curated images. This comparison reveals the increase in photorealism (as measured by the decrease in participants' accuracy) on the curated images relative to the prompt-matched images.

In this second phase of the experiment, we randomly sampled 39 AI-generated images from the main stimuli set, where the sample was stratified on 10 percentage point wide bins on human detection accuracy. For each of these 39 images, we generated at least 11 prompt-matched images using Midjourney, Firefly, and the same pipeline in Stable Diffusion. Figure~\ref{fig:regeneration} displays a Stable Diffusion-generated image from our original stimuli set and three of the twelve generations using the same prompt. We generated 482 total additional images, with at least 11 per prompt. These 482 images were included alongside the 149 real images on the experiment website.

In Figure~\ref{fig:curation-value}, we present scatterplots comparing human detection accuracy on the initial curated images and the best prompt-matched images in panel A, and mean prompt-matched images in panel B. We find the human-curated images have lower human detection accuracy than the best regenerated image in 18 of 39 instances and the mean re-generated image in 35 of 39 instances. In total, the human-curated images were perceived to be more photorealistic than 408 of the 482 (84\%) uncurated prompt-matched images. Specifically, we find the marginal value added by human curation for images that were initially detected in the range of 30\% to 50\% is 31 percentage points, 50 to 60\% is 23 percentage points, 60-70\% is 11 percentage points, 70-80\% is 8 percentage points, and 80+\% is 4 percentage points. Across the stimuli selected from Midjourney, Firefly, and Stable Diffusion, the marginal value of human curation is 7.8, 19.0, and 16.9 percentage points, respectively. 


The two panels in Figure~\ref{fig:curation-value} illustrate the positive correlation between accuracy on the human-curated image and accuracy on the regeneration. This reveals how the prompt influences photorealism. The Pearson Correlation Coefficient between accuracy on curated images and their best, mean, and worst re-generations are .58, .53, and .32, respectively. This positive correlation suggests the choice of a prompt plays a significant role in the photorealism of an image. Figure~\ref{fig:goodandbadprompt} displays two original curated images where A is generated by a prompt in which re-generations achieved low human detection accuracy (a `good' prompt), and B is generated by a prompt in which re-generations achieved a high human detection accuracy (a `bad' prompt). Prompts that consistently generate easily detectable images often have elements that are difficult to generate and result in artifacts. The prompt ``Persian woman astronaut in astronaut clothes, family photo with husband and two toddlers, high resolution, realistic" for Figure~\ref{fig:goodandbadprompt}B  generates a posed group image that tends to be easy to detect. On the other hand, the prompt ``American woman faculty portrait, not a close-up, blond" for Figure~\ref{fig:goodandbadprompt}A generates a portrait image that tends to be perceived as more photorealistic.


\section{Discussion and system limitations}

Our system employs compact nano-UAVs for both greenhouses and outdoor applications. 
Indoor environments provide ideal conditions with no weather constraints, while outdoor functionality is limited by weather conditions (e.g., wind speeds up to ~\SI{3.5}{\meter/\second}~\cite{9811834} and no rain). 
Though primarily demonstrated for pest detection, the system can also be applied to tasks like dry plant detection, crop monitoring, and counting. 
For Popillia japonica, optimal deployment conditions are sunny days with light winds and temperatures around \SI{29}{\celsius}~\cite{eppo-popillia} that fit the nano-UAVs' ideal operating conditions.

\begin{wrapfigure}{R}{0.48\textwidth}
\centering
\includegraphics[width=0.48\textwidth]{images/pop_sizes.png}
\caption{Example images depicting \textit{Popillia japonica} specimens at different scales, relative to image size. From left to right, bounding boxes occupy, on average, 0.9\%, 23.9\% and 41.9\% of the image's total pixels.}
\label{fig:popillia_sizes}
\end{wrapfigure}


Nano-UAVs are of particular interest for pest detection because they have a lower environmental impact than standard-sized drones that are currently used for this application.
In fact, nano-UAVs produce noise up to 40 dB~\cite{10.1145/3666025.3699337} while their bigger counterpart reaches up to 75 dB~\cite{ijerph18126202}, as such nano-UAVs provide an interesting solution that reduces the impact on the environment.

We test the system across varying hotspot densities (0 to 50) and three crop arrangements (environments 1, 2, and 3), highlighting its advantages over ground robots for early pest detection. 
Performance is influenced more by obstacle density than by the overall environment layout, thanks to the reliance on local path planning for obstacle avoidance.
We now analyze in detail the limitations of our insect detector and of the routing algorithm.


\subsection{Insect detector}

The dataset used in our work contains images gathered online rather than images taken directly from a nano-drone. 
This implies the presence of both pictures where the insects appear in close proximity, and pictures where the insects are much smaller relative to image size. 
On average, target insects cover 17.1\% of an image’s total pixels (8.9\% for \textit{Popillia japonica}, 19.8\% for \textit{Phyllopertha horticola}, 22.5\% for \textit{Cetonia aurata}). 
Figure~\ref{fig:popillia_sizes} provides a visual reference for this. From left to right, bounding boxes occupy, on average, 0.9\%, 23.9\% and 41.9\% of the image's total pixels.
We believe nano-drones can produce similar images, given their small size and capability of flying between vineyards' rows, close to vines. 
However, we point out that a real-world deployment would likely benefit from a model trained directly on images acquired by the drones during exploration.


\subsection{Routing}

\begin{figure}[tb!]
\centering
\includegraphics[width=1.0\columnwidth]{images/oscillation_ver_02.jpg}
\caption{An obstacle covers the entire depth sensor FoV, causing the blockage.}
\label{fig:deadlock}
\end{figure}


Blockages are a typical problem when relying on local planning strategies, they can occur when obstacles cover the entire FoV of the depth sensor, as reported in Figure~\ref{fig:deadlock}.
In fact, in this condition, the local planner provides a solution that passes through the uncertain region of the map.
However, when the UAV starts moving toward the uncertainty region, the depth sensor detects the presence of an obstacle that intersects the new locally planned path.
This causes a new iteration of the local planner, which provides a new solution that belongs to the uncertainty region that will result in a collision as soon as the UAV moves towards it.

The first solution involves maintaining a record of the surroundings based on the current depth measurement. 
This approach uses the same local planning algorithm proposed in this work, namely A*, applied to a local map that includes the current depth measurement along with all previous measurements within a 4$\times$\SI{4}{\meter} area used for local planning. 
This method gradually maps objects larger than the sensor’s FoV, which might otherwise obscure entirely the area ahead of the sensor and cause blockages. 
While this approach reduces the frequency of blockages, it does not eliminate them, as objects exceeding \SI{4}{\meter} in size can still lead to blockages in the current implementation.
Other local solutions to mitigate the blockages issue rely on reinforcement learning and swarm cooperation. 
Still, in our use case, that does not involve communication between UAVs and limits the knowledge of the map to a local instance of 4$\times$\SI{4}{\meter} obstacles that occlude the entire FoV may always result in blockages.
To avoid the blockage issue, a reliable solution is to perform obstacle avoidance on the global map, which cannot be done on our nano-UAVs due to the platform's computational constraints.





\section{Conclusions and Future Work}%
\label{sec:conclusions}

In this paper, we have investigated the role of truth assignment enumeration in OMT solving, and proposed some ways for exploiting partial truth assignments for improving the efficiency and effectiveness of the search. 
In particular, we have proposed a truth assignment reduction strategy that takes advantage of the properties of the optimization problem to accurately choose the atoms to remove from the truth assignment.

We have implemented the proposed strategies in the \optimathsat{} solver, and evaluated them on a set of \omlarat{} benchmarks.
Our experimental results show that the proposed strategies can significantly improve the performance of the solver, uniformly reducing the overall solving time for optimal solving, and finding much better solutions for anytime solving. 

The results also show that the proposed strategies are particularly effective in reducing the number of search iterations needed to find the optimum solution. 
This makes it very promising for their applicability in OMT problems where the incrementality of SMT calls is poorly effective, such as in the case of \laint{}.

In future work, we plan to investigate similar truth-assignment reduction techniques for other SMT theories, in particular \laint{}.

%%
%% The next two lines define the bibliography style to be used, and
%% the bibliography file.
\bibliographystyle{ACM-Reference-Format}
\bibliography{biblio}


%%
%% If your work has an appendix, this is the place to put it.
\appendix






\end{document}
\endinput
%%
%% End of file `sample-acmsmall.tex'.
