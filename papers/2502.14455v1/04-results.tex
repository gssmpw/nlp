
\section{Results}
\label{sec:results}
In this section, we report our system's results for both pest detection and the global+local planning task.
Furthermore, we provide an analysis of the benefits of employing a flying-ground cooperative system w.r.t. traditional ground pest treatment robots.


\subsection{Insect detector}

Following, we discuss the detection performance of our CNN for \textit{Popillia japoninca}, and we evaluate their performance on the selected embedded architecture.


\subsubsection{Object detection performance}
\label{subsec:object_detection_performance}

In this section, we assess the detection accuracy of each insect class, with a particular focus on identifying Popillia japonica. 
To evaluate the network's performance, we use the mAP, a standard object detection metric ranging from 0 to 1. 
A detection is considered correct if the Intersection over Union (IoU) between the predicted bounding box and the ground truth is at least 0.5, following the COCO standard~\cite{cocodataset}. 
The network is tested on a set of 660 samples with the most numerically accurate data type, i.e., \textit{float32}, and exploring variations in deployment \textit{float16} and \textit{int8} (quantized) formats.

\begin{wraptable}{R}{0.5\textwidth}
    \small\centering
    \caption{SSDLite-MobileNetV3 AP per each class of insect and per average bounding boxes dimension.}
    \label{tab:perclass_perbbox_mAP}
    \resizebox{0.48\textwidth}{!}{%
    \begin{tabular}{llll}
    \toprule
    \textbf{Configuration}& small & medium & large\\
    \midrule
    \textbf{Popillia japonica}&0.78&0.83&0.82\\
    \textbf{Cetonia aurata}&0.76&0.77&0.81\\
    \textbf{Phyllopertha horticola}&0.76&0.81&0.83\\
    \bottomrule
    \end{tabular}
    }
\end{wraptable}
When considering models in \textit{float32}, the SSDLite-MobileNetV3 (320$\times$240 input) reaches an mAP of 0.80. 
Additionally, when deploying the SSDLite-MobileNetV3 in \texttt{float16} to leverage the single-precision FPUs on the GAP9 SoC, we observe no significant mAP drop compared to its double-precision counterpart, i.e., mAP is equal to 0.80 in \texttt{float32} and \texttt{float16}. 
Lastly, when switching from \texttt{float32} to \texttt{int8} quantized version, the mAP reduction is less than 2\%. 



We consider this performance satisfactory for the detection of \textit{Popillia japonica} as the insects tend to aggregate in clusters on host plants~\citep{eppo-popillia}, hence misdetecting roughly $1$ in $5$ specimens has a small impact on hotspots detection.
Furthermore, the detections performed with nano-UAVs are used to trigger the intervention in the hotspot's position of a larger ground robot that is not constrained to microcontroller-unit class devices but can host heavy power-demanding GPUs onboard. 
As such, it can perform detections using computationally intensive, yet more accurate, models such as the FasterRCNN-MobileNetV3-FPN network reported in Table~\ref{tab:other_models_params}.

Furthermore, we report in Table~\ref{tab:perclass_perbbox_mAP} the performance of our SSDLite-MobileNetV3 network per each class of insect and size of bounding boxes w.r.t. the entire image, i.e, small (10\%), medium (between 10\% and 25\%), and large (more than 25\%).
The performance of the network are consistent across all the configuration reported in Table~\ref{tab:perclass_perbbox_mAP} with the minimum average precision (AP) obtained by the Phyllopertha horticola in the case of small insects while the highest AP is achieved by the Phyllopertha horticola in the case of large bounding boxes; the model provides slightly more accurate predictions in the case of larger bounding box, i.e., medium and large, w.r.t. small ones.




\subsubsection{Embedded system deployment performance}

\begin{wraptable}{R}{0.5\textwidth}
    \small\centering
    \caption{GAP9 SoC configurations.}
    \label{tab:configurationsGAP9}
    \resizebox{0.48\textwidth}{!}{%
    \begin{tabular}{llll}
    \toprule
    \textbf{Configuration}&\textbf{Voltage [\SI{}{\volt}]}&\textbf{\textbf{Freq CL [\SI{}{\mega\hertz}]}}&\textbf{\textbf{Freq FC [\SI{}{\mega\hertz}]}}\\
    \midrule
    \textbf{Min power}&0.65&150&150\\
    \textbf{Max efficiency}&0.65&240&240\\
    \textbf{Min latency}&0.80&370&370\\
    \bottomrule
    \end{tabular}
    }
\end{wraptable}


In this section, we conduct a comprehensive evaluation of the performance of all models deployed on the GAP9Shield. 
Table~\ref{tab:configurationsGAP9} details three GAP9 SoC configurations, with voltages ranging from~\SI{0.65}{\volt} to~\SI{0.8}{\volt}, enabling clock speeds of up to~\SI{370}{\mega\hertz}. 
For our experiments, we adopt the \textit{maximum efficiency} configuration, which provides optimal battery life while maintaining the target throughput.


Table~\ref{tab:results} presents a detailed analysis of each model’s memory requirements, inference latency per image, and total power consumption, including both compute unit and off-chip memory usage. 
The model is evaluated using different data types (\texttt{float16} and \texttt{int8}).
The peak memory usage is determined based on the network layer with the highest memory demand for input tensors, weights, and outputs. 
NN-Tool on the GAP9 can efficiently manage memory across the~\SI{1.6}{\mega\byte} on-chip L2 and up to~\SI{32}{\mega\byte} of off-chip memory.

We deploy the SSDLite-MobileNetV3 with a fixed input size of 320$\times$240$\times$3 (RGB) across three configurations: \textit{i}) \texttt{float16} running on general-purpose CL with FPU hardware support, \textit{ii}) \texttt{int8} quantized running on the CL, and \textit{iii}) \texttt{int8} using the NE16 convolutional accelerator. 
The \texttt{float16} version, while supported by the hardware FPU, requires the most memory to store all the weights ($\sim$\SI{7}{\mega\byte}) and the most peak memory during execution (at most $\sim$\SI{3.6}{\mega\byte}). Furthermore, it is the slowest, processing at~\SI{2.1}{frame/\second} with~\SI{40.6}{\milli\watt} of power consumption. 
By contrast, the \texttt{int8} version running on the NE16 accelerator requires only~\SI{3.4}{\mega\byte} of memory for all the weights while requires a peak ~\SI{1.8}{\mega\byte} memory for the execution, reaching~\SI{6.8}{frame/\second} while consuming~\SI{34}{\milli\watt}.
Table~\ref{tab:results} reports the peak memory required during the execution of the networks.
We do not deploy the \texttt{float32} version of the network since the GAP9 SoC does not have hardware support for this type of computations and, as such, this will result in a sub-optimal performance, in terms of throughput and power consumption, due to the soft-float emulation required which is not justified due to the iso-accuracy w.r.t. the \texttt{float16} version of the network.

\begin{table*}[t]
    \small
    \caption{Latency, power consumption, and mAP of all networks on the GAP9 SoC. %\firstreviewer{The \texttt{float32} network is not deployed onboard since the GAP9 does not have a \texttt{float32} FPU, and the iso-performance measured with the mAP w.r.t. the \texttt{float16} does not justify the overhead of the \texttt{float32} computations that leads to a decrease in throughput.}
    }
    \label{tab:results}
    \resizebox{\linewidth}{!}{%
    \begin{tabular}{llcccccc}
    \toprule
    \textbf{Network} & \textbf{Data type} & \textbf{Input size} & \textbf{MCU} & Memory [\SI{}{\kilo\byte}] * & \textbf{Latency [\SI{}{\milli\second}]} & \textbf{Total power [\SI{}{\milli\watt}]} & \textbf{mAP} \\
    \toprule
    %FOMO-MobileNetV2$^a$ & \texttt{float32} & 96$\times$96$\times$1 & STM32H74 & 1070 & N.D. & N.D. \\
    %FOMO-MobileNetV2 & \texttt{int8} & 96$\times$96$\times$1 & STM32H74 & 398 & 57 & 494 \\
    %FOMO-MobileNetV2$^a$ & \texttt{float32} & 96$\times$96$\times$3 & STM32H74 & 1045 & N.D. & N.D. \\
    %FOMO-MobileNetV2 & \texttt{int8} & 96$\times$96$\times$3 & STM32H74 & 416 & 62 & 498 \\
    %FOMO-MobileNetV2$^a$ & \texttt{float32} & 160$\times$160$\times$1 & STM32H74 & 2654 & N.D. & N.D. \\
    %FOMO-MobileNetV2 & \texttt{int8} & 160$\times$160$\times$1 & STM32H74 & 802 & 158 & 501 \\ 
    %FOMO-MobileNetV2$^a$ & \texttt{float32} & 160$\times$160$\times$3 & STM32H74 & 2862 & N.D. & N.D. \\
    %FOMO-MobileNetV2 & \texttt{int8} & 160$\times$160$\times$3 & STM32H74 & 854 & 169 & 499 \\
    %\midrule
    %\firstreviewer{SSDLite-MobileNetV3} & \firstreviewer{\texttt{float32}} & \firstreviewer{320$\times$240$\times$3} & \firstreviewer{-} & \firstreviewer{7244} & \firstreviewer{-} & \firstreviewer{-} & \firstreviewer{0.80}\\
    SSDLite-MobileNetV3 & \texttt{float16} & 320$\times$240$\times$3 & GAP9 (CL) & 3622 & 462 & 41 & 0.80\\
    SSDLite-MobileNetV3 & \texttt{int8} & 320$\times$240$\times$3 & GAP9 (CL) & 1811 & 249 & 31 & 0.79\\
    SSDLite-MobileNetV3 & \texttt{int8} & 320$\times$240$\times$3 & GAP9 (NE16) & 1811 & 147 & 34 & 0.79\\
    \bottomrule
    \vspace{0.01pt}
    \end{tabular}
    }
    \footnotesize{* The memory footprint includes the input and output tensors and weights of the largest network layer and the input image.}
\end{table*}

Figure~\ref{fig:powerGAP9} displays the power consumption waveforms for the GAP9 SoC under three conditions: \textit{i}) \texttt{float16}, \textit{ii}) \texttt{int8} on the general-purpose multicore cluster, and \textit{iii}) \texttt{int8} on the NE16 accelerator, shown in panels A, B, and C, respectively. 
All three configurations exhibit short non-overlapped memory transfers (up to~\SI{41}{\milli\second} in total), particularly at the start of execution when tensor movements between memory layers do not overlap with computation. 
These non-overlapped memory transfers decrease as tensor sizes shrink during network execution.

\begin{figure}{}
\centering
\includegraphics[width=0.5\textwidth]{images/idle_periods.png}
\caption{Power waveforms for the three deployed networks on the GAP9, i.e., A) float16, B) int8 running on the CL, and C) int8 running on the NE16 accelerator.}
\label{fig:powerGAP9}
\end{figure}


\subsection{Routing}

We evaluate our routing w.r.t. its effectiveness in visiting the predefined waypoints and from a latency standpoint.


\subsubsection{Waypoints visited}

\begin{table}[tb]
    \small\centering
    \caption{Obstacle avoidance performance of four different algorithms, i.e., blind (B), random (R), local routing weighted (W), and local routing shortest (S), in three different environments with changing number of obstacles. Performance measured in percentage of waypoints reached. R considers a position within the waypoint if the distance is less than~\SI{30}{\centi\meter}. }
    \label{tab:changing_obstacles}
    \resizebox{0.7\columnwidth}{!}{%
    \begin{tabular}{lcccccccccccc}
    \toprule
    \multirow{2}[3]{*}{\shortstack[c]{\textbf{Area}\\\textbf{occupied [\%]}}} & \multicolumn{4}{c}{\textbf{Env 1 - 10$\times$\SI{10}{\meter}}}&  \multicolumn{4}{c}{\textbf{Env 2 - 20$\times$\SI{20}{\meter}}} &  \multicolumn{4}{c}{\textbf{Env 3 - 40$\times$\SI{40}{\meter}}}\\
    \cmidrule(lr){2-5} \cmidrule(lr){6-9} \cmidrule(lr){10-13}
    & B & R & W & S & B & R & W & S & B & R & W & S\\
    \midrule\textbf{0}&100&84&\textbf{100}&100&100&44&\textbf{100}&100&100&13&\textbf{100}&100\\
\textbf{1}&88&83&\textbf{100}&53&56&43&\textbf{100}&65&30&10&\textbf{100}&32\\
    \textbf{2}&47&81&\textbf{100}&53&35&42&\textbf{100}&65&21&9&\textbf{80}&32\\
    \textbf{5}&37&71&\textbf{100}&20&36&41&\textbf{100}&54&20&9&\textbf{77}&32\\
    \textbf{10}&20&61&\textbf{100}&20&12&29&\textbf{100}&65&5&8&\textbf{50}&10\\
    \bottomrule
    \end{tabular}
    }
\end{table}

We evaluate our system using the three different maps described previously in Section~\ref{subsec:sim_env}: Env 1, Env 2, and Env 3.
We perform one run for each map represented in Figure~\ref{fig:webots_worlds}, gradually increasing the number of obstacles.
The number of obstacles is expressed as a percentage of the total area of the map, and we test five different configurations: 0\%, 1\%, 2\%, 5\%, and 10\%. 
The obstacles are cylindric barrels of~\SI{1}{\meter\squared} randomly positioned in the map without overlapping waypoints and checking that there is always at least one path available to reach the next waypoint of the path.
In this experiment, we compare all the routing methods described before each evaluated on every randomly generated configuration of obstacles.
The results of the experiment are reported in Table~\ref{tab:changing_obstacles}.
The algorithms reported are:
% 
\begin{itemize}
  \item Blind (B), i.e., the UAV follows the predefined path without considering the obstacle seen by the ToF sensor. 
  \item Random (R), the UAV performs a path composed of segments with uniformly sampled random orientation in [$-\pi, \pi$] and length between 0 and $min(dist,\SI{4}{\meter})$  where $dist$ represents the minimum distance retrieved at that instant by the ToF sensor.
  The random path is executed until the end of the battery, i.e.,~\SI{5}{\minute} and considering~$\sim$\SI{1}{\minute} to return to the starting point.
  \item Weighted (W) uses local routing with the weighted cost metric.
  \item Shortest (S) uses local routing, considering only the distance between visited cells as a cost metric.
\end{itemize}

As expected, the best-performing routing algorithms are the global ones in all environments; in fact, they are always able to find a solution for the considered path. 
The use of a global occupancy map allows the finding of globally optimal solutions for the paths and provides feasible path solutions even if local solutions are not available due to the presence of obstacles in the local occupancy map.
If we limit the knowledge to a local map, i.e., the map created from the ToF sensor data, the local routing with the W cost metrics achieves the best results in terms of the percentage of waypoints visited w.r.t. the total number of waypoints on the map.
It achieves up to 100\% of waypoints visited in Env1 and Env2 with all obstacle configurations, while in Env 3, it reaches the end of the path, i.e., 100\% of waypoints visited, only in the case of 0\% and 1\% of the total area occupied by obstacles.
Further increasing the percentage of occupied area leads to a decrease in the portion of path completed bottoming at 50\% in the case of 10\% of area occupied, i.e., 160 obstacles on the map.
Still, the routing method provides an improvement over the B and R baselines that reach up to 45\%.

The S cost metric underperforms in local routing compared to the weighted (W) cost metric. In particular, in our smallest environment, Env 1, the S cost metric yields the lowest performance among all methods. Paths generated with the S metric are more likely to encounter blockages when obstacles are nearby and partially obscured. Although the W cost metric can occasionally face similar issues, it does so less frequently under the same conditions (environment and obstacle positions). Local routing with the W metric demonstrates higher reliability, and blockages are encountered less often than with the S metric. This limitation can be effectively mitigated by introducing random yaw adjustments and forward movements toward a free direction when a blockage occurs.

Figure~\ref{fig:paths_obstacle} provides five examples of obstacle avoidance of irregularly shaped non-uniform obstacles reported within the local occupancy map computed onboard the UAV. 
These non-uniform obstacles may be present in a real-world vineyard; Figure~\ref {fig:paths_obstacle} highlights the capabilities of our algorithm to avoid nonconvex obstacles.
We provide results with our two cost metrics, i.e., the W and the S path cost metrics.
Figure~\ref{fig:paths_obstacle}-A and~\ref{fig:paths_obstacle}-B displays the difference between the two algorithms at the beginning of the path, near the source (white), where the local routing with the W cost metric provides a path that is equal to the pre-planned path and traverses the free area (green) which is desirable since the unknown cells may contain obstacles that are outside the FoV of the sensor.
Local routing with the S cost metric, instead, provides a solution that traverses the uncertain area with the risk of finding an obstacle outside the depth sensor's FoV.
Figure~\ref{fig:paths_obstacle}-C reports the same path for both cost metrics since traversing the free area does not lead to a feasible solution in the case of the W cost metric.
Figure~\ref{fig:paths_obstacle}-D and~\ref{fig:paths_obstacle}-E displays the same difference of~\ref{fig:paths_obstacle}-A and~\ref{fig:paths_obstacle}-B but at the end of the path, near the destination point (black).

The last row of Figure~\ref{fig:paths_obstacle} reports local planning performed with a baseline reactive local navigation algorithm~\cite{6630610}.
This algorithm chooses the heading and the speed of the robot depending on the obstacles in the environment.
The heading is chosen to minimize the distance from the endpoint of the current movement to the destination.
The baseline~\cite{6630610} struggles when the robot’s distance sensors encounter non-convex irregular obstacles that fully occupy their field of view (FoV). In such cases, the robot may become stuck in local minima and unable to avoid the obstacle. 
By contrast, our algorithm supports path planning beyond the FoV, even when obstacles entirely obstruct it.
To conclude, we provide a video\footnote{\href{https://github.com/idsia-robotics/A-star-path-planning-nano-drones}{https://github.com/idsia-robotics/A-star-path-planning-nano-drones}} with an experiment of obstacle avoidance done in simulation using our A*-based algorithm with the W cost.

\begin{figure}[tb!]
\centering
\includegraphics[width=1.0\columnwidth]{images/paths_obstacle_ver_09.jpg}
\caption{Local routing: plannings with five different obstacles. The cost metric for each replanning is reported on the left. A* refers to the path replanned with the A* algorithm.}
\label{fig:paths_obstacle}
\end{figure}


\subsubsection{Onboard computing time}

\begin{wrapfigure}{R}{0.5\textwidth}
  \centering
    \includegraphics[width=0.48\textwidth]{images/scatterplot_time_a_star.png}
  \caption{Time [\SI{}{\milli\second}] to compute A* path with the STM32 MCU available onboard the Crazyflie 2.1 with five different obstacles, depicted in Figure~\ref{fig:paths_obstacle}, using the local routing (4$\times$\SI{4}{\meter} maps) with two cost metrics weighted and shortest path.}
  \label{fig:time_a_star}
\end{wrapfigure}

Figure~\ref{fig:time_a_star} reports the onboard measurements of the time to compute a solution for the five different obstacles shown in Figure~\ref{fig:paths_obstacle}. 
In particular, the W cost metric requires up to $12\times$ the time required to find a solution with the S cost metric.
This is the case since the S cost metric allows the pruning of the paths that need to be saved because the cost depends only on the distance and thus has a branching factor b* that is lower than the W cost metric.
This results in an expansion phase with a lower number of nodes in the case of the S cost metric, which consequently leads to a proportionally lower amount of memory required w.r.t. the W cost metric. 

However, given the higher performance in the obstacle avoidance task achieved by the W cost metrics, reported in Table~\ref{tab:changing_obstacles}, and considering that the A* algorithm is used only in the case an obstacle is detected on the pre-planned path, thus achieving real-time control performance, the most promising approach to perform obstacle avoidance onboard the Crazyflie 2.1 is the local routing with W cost metric.
The onboard routing algorithm runs on the STM32 MCU available onboard, which has a peak power consumption of \SI{138}{\milli\watt}@\SI{168}{\mega\hertz}.


\subsection{Complete system performance}

In this section, we evaluate the performance of our flying-ground cooperative system in the treatment of pest insects in a 200$\times$\SI{200}{\meter} vineyard.
The vineyard is initially split into 25 areas of 40$\times$\SI{40}{\meter}, and one UAV per area executes an exploration looking for the Popillia japonica that takes only~$\sim$\SI{5}{\minute} considering an average speed of the UAV of~\SI{1}{\meter/\second}.
After the exploration phase is completed, a ground robot capable of performing pest control visits all the points where an insect has been detected.

\begin{wrapfigure}{R}{0.5\textwidth}
\centering
\includegraphics[width=0.48\textwidth]{images/exploration_time_ground_robot.png}
\caption{The time, in hours, a ground robot (speed \SI{0.2}{\meter/\second}~\cite{9177181}) needs to reach all the detected insects in a vineyard. The experiments are done knowing the exact location of $N$ hotspots of insects, thanks to the exploration done with UAVs and without prior information, i.e., no prior exploration line.}
\label{fig:insects_ground_robot}
\end{wrapfigure}

To evaluate the improvement in time achieved by the ground robot to perform pest control over the vineyard, we analyze two conditions: 
\begin{itemize}
    \item no prior exploration with nano-UAVs, i.e., the ground robot needs to explore the entire field;
    \item prior exploration with nano-UAVs with a variable number ($N \in [0, 50]$) of randomly positioned hotspots in the vineyard.
\end{itemize}
The prior exploration with the nano-UAVs creates a map of the hotspots in the field that the ground robot explores in a second step to perform pest control.
The improvement is evaluated by comparing the time required by the ground robot to perform pest control on the entire vineyard, i.e., no prior exploration, w.r.t. the case the ground robot performs the pest control only in the hotspots found by the UAVs.
We report the average time required by the ground robot to perform pest control over five runs per configuration in Figure~\ref{fig:insects_ground_robot}.

The shortest path that connects the affected points is computed using the global routing algorithm with A*, simplifying the complexity of the numerical problem thanks to the structure of the vineyard.
Without the exploration done by the UAVs, the total pest control time results in~$\sim$\SI{20}{\hour} independently from the number of hotspots in the vineyard.
With the exploration instead, the time depends on the number of hotspots and ranges between 0 and~\SI{17}{\hour} in the case of 50 clusters of insects in the vineyard.
Our system gives the highest benefit when the number of hotspots is low, which is the normal operating condition in a healthy vineyard since an increasing number of insect hotspots results in the ground robot having to travel the entire vineyard.
In the case of zero hotspots detected, the ground robot is not operated, resulting in the maximum working time saving of $\sim$\SI{20}{\hour}.

The applicability of the proposed solution is determined by the convergence of technical limitations, environmental conditions, and characteristics of the target insect(s). Different insect species fly in different conditions and at different times of the day. For \textit{Popillia japonica}, the greatest flight activity is reported to be on clear days and when the temperature is between 29°C and 35°C, relative humidity greater than 60\% and wind is lower than 20 km/h \cite{eppo-popillia}. These conditions perfectly fit with the technical limitations of the drones for what concerns constraints on flight conditions (e.g., max wind speed) and lightning for the camera. 