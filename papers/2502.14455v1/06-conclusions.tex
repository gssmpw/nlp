\section{Conclusions}

This work presents the building blocks for a novel, efficient transportation system applied to a pest detection and control scenario in vineyards, relying on flying and ground robots. 
We propose an implementation of the pest detection system that runs onboard the Crazyflie 2.1 nano-UAV on the ultra-low power multi-core GWT GAP9 SoC.
Due to the limited resources available on this platform, we explore and deploy a CNN-based detection system, i.e., the SSDLite-MobileNetV3 CNN (\SI{584}{\mega MAC/inference}), scoring an mAP of 0.79 with a throughput of~\SI{6.8}{frame/\second} at~\SI{33}{\milli\watt} on the GAP9 SoC.

We integrate the CNN-based insect detector with an obstacle avoidance algorithm running onboard the Crazyflie 2.1 nano-UAV to allow the autonomous exploration of vineyards.
Our local routing A*-based obstacle avoidance algorithm is able to reach up to 100\% of the planned waypoints, avoiding all the obstacles in two out of three environments with increasing complexity (from 0 to 10\% of the entire area covered with~\SI{1}{\meter\squared} obstacles).

If compared to the pre-planned path (B) and to the random (R) explorer baselines, our local routing algorithm (W) increases the number of waypoints visited within the battery life of the UAV between 16\% in the smallest environment, i.e., a~10$\times$\SI{10}{\meter} vineyard, to 90\% in our~40$\times$~\SI{40}{\meter} environment. 
The algorithm achieves real-time performance with planning requiring less than \SI{170}{\milli\second} on the STM32 MCU available on the nano-UAV while performing real-time flight control tasks.
Our multi-UAV system, using a swarm of 25 UAVs to explore a~200$\times$\SI{200}{\meter} vineyard, allows us to save up to~$\sim$\SI{20}{\hour}, i.e., 100\% of the time if no insects are detected, to perform pest control, paving the way to fully autonomous precise and effective pest control with an efficient transportation system applied to vineyards.
