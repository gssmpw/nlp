%%%%%%%%%%%%%%%%%%%%%%%%%%%%%%%%%%%%%%%%%%%%%%%%%%%%%%%%%%%%%%%%%%%%%%%%%%%%%%%%
\section{MICROSPINE ARRAY AND ROBOT DESIGNS}
There are several design parameters that impact the effectiveness of microspines. The critical ones include (1) adding compliance to individual microspines and  the angle at which the microspines interact with a surface, (2) the array configuration and (3) effective integration with the robot body. Even with an optimal design, the microspine array will not be effective on every surface. The aspects that are out of the hands of the designer, but also highly impact surface engagement, include surface roughness, distribution of asperities, and size of asperities. 


\subsection{Compliant Mechanism Microspine Design}
%\hl{Show CAD microspine design here.}
The single-material mechanism, shown in \Fig \ref{fig:comp}, allows compliance with an exposed joint while simplifying the fabrication process over previous microspine designs. The compliant mechanism is fabricated with an FDM 3D printer and TPU with 95A Shore hardness. Halfway through the additive manufacturing process, the print is paused. The microspine is inserted into a channel left in the middle of the mechanism, highlighted in \Fig \ref{fig:comp}c), and the print is resumed. Once finished, the angle of the bare microspine can be modified for different surface topologies while the body remains secure in the mechanism. The angle of surface interaction, $\alpha$, was fixed at roughly $45^{\circ}$ during testing.



\begin{figure}[h]
    \centering
    \includegraphics[width=\linewidth]{Figures/compliantV2.png}
    \caption{Compliant mechanism design. a) A hinge joint enables passive compliance.  b) Holes embedded on the righthand side of the mechanism allow anchoring into the silicone limb. c) A microspine is inserted in a center channel matching the spine topology set halfway into the mechanism.}
    \label{fig:comp}
\end{figure}


%\subsection{Microspine Interaction Angle}

\subsection{Microspine Array Configuration}
%\hl{Copy the previous mechatronics.png and possibly FBD.png in this section.}
The array configuration ensures multiple microspines remain active on various complex surfaces. We propose a two-row, stacked array configuration consisting of ten microspines with four on the top row and six on the bottom. The microspines on the bottom row are commonly active on more uniform terrain. The top row can become active on steep/highly irregular surfaces without hindering the movement of the bottom row of microspines. The critical parameter when designing the array configuration is ensuring adequate surface interaction and gripping regardless of topology. Crucially, not all microspines need to interact with a surface for the microspine array to be effective, shown in \Fig \ref{fig:eng}. This is a byproduct of the passive nature and built-in redundancy of the system.

\begin{figure}[h]
    \centering
    \includegraphics[width=\linewidth]{Figures/engageV2.png}
    \caption{Two-row, stacked array configuration. a) Close-up of the microspines gripping onto a non-uniform rock. b) This diagram highlights which microspines are interacting with the surface. On this rock, all 6 of the microspines on the bottom row are active. c) Close-up of the microspines gripping onto a steeper rock. d) On this rock, 2 of the bottom row and 3 of the top row microspines are active.}
    \label{fig:eng}
\end{figure}


\subsection{Effective Soft-Compliant Integration Through Anchoring}
%\hl{Show the mold here as well as discuss the robot design.}
The soft-compliant integration reduces design complexity by allowing each microspine to passively move independent of one another with a single actuator controlling the entire array configuration. To achieve this, a mold is created with channels for each microspine compliant mechanism to attach to the tip of a SoRo limb. The SoRo prototype used for experimentation is cast out of DragonSkin\texttrademark~ silicone rubber using a custom mold, shown in \Fig \ref{fig:totalMold}. Therefore, a modified limb mold is used for integrating the microspine array in a consistent, standardized manner. Half of the compliant mechanism contains holes that mechanically anchor it into the silicone limb, highlighted in \Fig \ref{fig:comp}b), preventing it from being freely pulled out of the limb during microspine gripping. This anchoring method is essential for ensuring the microspine does not come loose over time. The remaining, exposed half of the mechanism contains the microspine.

\begin{figure}[h]
    \centering
    \begin{subfigure}{0.4\linewidth}
         \centering
         \includegraphics[width=0.7\textwidth]{Figures/mold.PNG}
         \caption{Mold}
         \label{fig:base}\vspace{-20pt}
     \end{subfigure}
     \begin{subfigure}{0.55\linewidth}
         \centering
         \includegraphics[width=0.7\textwidth]{Figures/mold_ext.png}
         \caption{Modified mold}
         \label{fig:one}\vspace{-20pt}
     \end{subfigure}
     \begin{subfigure}{0.45\linewidth}
         \centering
         \includegraphics[width=0.7\textwidth]{Figures/mold_tip.png}
         \caption{Microspine mold tip}
         \label{fig:two}
     \end{subfigure}
     \begin{subfigure}{0.45\linewidth}
         \centering
         \includegraphics[width=0.7\textwidth]{Figures/mold_spines.png}
         \caption{Integrated microspines}
         \label{fig:two}
     \end{subfigure}
    \caption{Modular ends of the mold enable soft-rigid integration. a) A baseline robot mold. b) A modified mold that allows different configurations of microspine arrays per limb. c) Microspine compatible end mold with holes for a two-row stacked microspine array configuration. d) End mold with integrated microspine mechanisms and ready for casting.}
    \label{fig:totalMold}
\end{figure}



\subsection{Soft Robot Design}
A tetherless, three limb MTA SoRo with on-board power and processing with AprilTags on each limb is used as the experimental prototype. The components of the physical robot are shown in \Fig \ref{fig:mech}. %The topology design optimizes the locomotion and reconfiguration ability \cite{freeman_topology_2023}. Additionally, four of these robots are capable of reconfiguring into a sphere. Morphologically, 
Outward trapezoid cavities are introduced on the underside of each limb to provide optimal stiffness and curling ability. This allows the robot to lift the limb and electronic payload. The use of MTA for body deformation enables reliable and efficient limb actuation. The reader may refer to \cite{freeman_topology_2023} for more details.


\begin{figure}[!h]
    \centering
    \includegraphics[width=0.9\linewidth]{Figures/mechatronicsV2.PNG}
    \caption{The externally powered three-limb SoRo contains soft material limbs and a flexible, central hub that houses DC motors and a custom-designed PCB. The three AprilTags on the limbs help with pose tracking during experiments.}
    \label{fig:mech}
\end{figure}


% \subsection{Three-Limb Motor-Tendon Actuated (MTA) SoRo}
% An externally powered, three-limb MTA SoRo with markers is used as the test robot. The topology design optimizes the locomotion and reconfiguration ability \cite{freeman_topology_2023}. Additionally, four of these robots are capable of reconfiguring into a sphere. Morphologically, outward trapezoid cavities are introduced on the underside of the limb to provide optimal stiffness and curling ability. These design decisions allow the robot to lift the limb and electronic payload. The use of motor tendon actuation (MTA) for body deformation enables reliable and efficient limb actuation. The reader may refer to \cite{freeman_topology_2023} for more details.
% \begin{figure}
%     \centering
%     \includegraphics[width=0.9\linewidth]{Figures/mechatronics.png}
%     \caption{The externally powered three-limb SoRo comprises of soft material limbs and a flexible hub in the center that houses DC motors and a custom-designed PCB. The three markers on the limbs help with real-time pose tracking during experiments.}
%     \label{fig:mech}
% \end{figure}

% A robot is fabricated using a 3D printed cast mold and hub. The plastic mold is comprised of the negatives of the three limbs (referred to as limb chambers), while the central hub is designed to house the three motors (one per limb) and a custom-designed PCB. A Thermoplastic Polyurethane (TPU) hub is printed with shore hardness of 98A to provide flexibility in the center of the robot that houses rigid mechatronics. The fabrication process starts by placing the flexible hub at the center of the mold. Thin tubing is then routed through each of the limb chambers and into a side of the hub to create a tendon path. Equal parts of Dragon Skin\textsuperscript{TM} silicone rubber Part A and Part B are combined and cast into each of the limb chambers. Upon curing, the limbs are removed from the mold and the tendon path is threaded with fishing line. A fishing hook is embedded in the tip of each limb to act as an anchor point. Within the hub, the fishing line is wrapped around a spool attached to the corresponding limb motor. The three motors are secured to the hub base with a set of zipties that prevent the bodies from rotating as the spools along the shafts spin. The custom-designed PCB is placed in the center of the hub with a protective, flexible cap placed on top. The three-limb experiment SoRo is shown in \Fig~\ref{fig:mech}.

% A bench power supply feeds in 15V to the custom-made PCB. This powers the motors directly, and a step-down linear regulator supplies 5V to the Seeeduino Xiao microcontroller. The option of tetherless powering (LiPo battery) is also incorporated where safety circuitry is included to detect and indicate low cell voltage of 3.2V/cell. Three TB6643KQ full-bridge DC motor drivers are used to control three Maxon motors. Digital I/O pins from the microcontroller send PWM signals to actuate each of the motors in a controlled sequence. To ensure a perfect fit in the center of the hub, a triangular PCB with side length of $54mm$ is designed and fabricated. %This circuitry is scalable to allow for testing of SoRos with three, four and five limbs. %The flow of power and data throughout the schematic is shown in Figure~\ref{fig:circ}. 


% % \begin{figure}
% %     \centering
% %     \includegraphics[width=0.9\linewidth]{Figures/circuit.jpg}
% %     \caption{Circuit Design}
% %     \label{fig:circ}
% % \end{figure}

% %\subsection{Spool Design}
% %\hl{Add fig for spool actuation side by side}\\
% %Diameter of working spool: 4mm\\
% %Diameter of old spool: ?
% %Diameter of lone shaft: 1.4mm

% \subsection{Microspine Gripper Placement}

% The integration of the microspine gripper can be performed in a non-modular fashion where they are directly fabricated into the robot limb, or in a modular fashion where they are ``attached'' to the limb. The former has drawbacks relating to fabrication - a new robot would need to be fabricated for each configuration. The modular designs have challenges relating to how much they influence the limb material properties. For example, a `sock' design where the modular array slides onto the end of each limb impacts and elevates the resting position. However, an endcap design, shown in \Fig~\ref{fig:Isometric Endcap View}, is proposed where the array is located at the tip of the limb. The drawback for this design is that the effective length of the limb increases, as shown in \Fig~\ref{fig:Endcap on Bot}, changing the mechanical advantage from the motor. Both modular designs require a new method to secure the tendons to the limb for fast attachment and detachment. The endcap design is pursued in this research due to its efficiency in iteration and testing time. The engagement of the gripper with the surface upon limb actuation is visualized in \Fig~\ref{fig:FBD}.

% \begin{figure}
%      \centering
%      \begin{subfigure}{0.49\linewidth}
%          \centering
%          \includegraphics[width=\textwidth]{Figures/Single Endcap.jpg}
%          \caption{}
%          \label{fig:Isometric Endcap View}
%      \end{subfigure}
%      \begin{subfigure}{0.49\linewidth}
%          \centering
%          \includegraphics[width=\textwidth]{Figures/Endcap on Bot.jpg}
%          \caption{}
%          \label{fig:Endcap on Bot}
%      \end{subfigure}
%      \hspace{0.5cm}

%     \begin{subfigure}{0.49\linewidth}
%          \centering
%          \includegraphics[width=\textwidth]{Figures/Endcap Side View.jpg}
%          \caption{}
%          \label{fig:Endcap Side View}
%      \end{subfigure}
%      \begin{subfigure}{0.49\linewidth}
%          \centering
%          \includegraphics[width=\textwidth]{Figures/Endcap Side View on Bot.jpg}
%          \caption{}
%          \label{fig:Endcap Side View on Bot}
%      \end{subfigure}  
%      \hspace{0.5cm}
     
%      \begin{subfigure}{0.49\linewidth}
%          \centering
%          \includegraphics[width=\textwidth]{Figures/Empty Mold.jpg}
%          \caption{}
%          \label{fig:Empty Mold}
%      \end{subfigure}
%      \begin{subfigure}{0.49\linewidth}
%          \centering
%          \includegraphics[width=\textwidth]{Figures/Assembled_Filled_Mold.jpg}
%          \caption{}
%          \label{fig:Assembled Mold}
%      \end{subfigure}
        
%     \caption{Microspine gripper endcap design: (a) Isometric view  (b) Robot-endcap integrated isometric view  (c) Endcap side (d) Robot-endcap side view. Endcap casting mold: (e) Disassembled mold (f) Assembled mold}
%     \label{fig:Endcap FigureS}
% \end{figure}

% \begin{figure}
%     \centering
%     \includegraphics[width=0.75\linewidth]{Figures/FBD.png}
%     \caption{Visualization of how the microspine array gripper engages with surface asperities upon actuation.}
%     \label{fig:FBD}
% \end{figure}
% \subsection{Endcap Design and Fabrication}

% % \textbf{Paragraph about materials}
% To ensure effective integration and minimum stiffness mismatch, material selection for the endcap is comparable to that of the original robot. TPU 95A inlays are optimized to attach without deforming the original limb with minimal lengthening. TPU is also chosen to allow for quick iterations via 3D printing, while being firm enough not to deform during actuation. Endcaps made of DragonSkin\textsuperscript{TM} 10A and DragonSkin\textsuperscript{TM} 30A are fabricated to determine which best houses the microspine array. In order to keep the robot as homogeneous as possible, the 30A material is chosen to match the limbs. Additionally, the 30A material holds the spines more accurately and reliably when compared to the 10A material which becomes less effective after repetitive testing.

% % \textbf{Describe the endcap design}
% The endcap design consists of three major parts: a TPU inlay, microspines to grip the surface plane, and silicone to lock each part in place. First iterations are done to achieve the best fit for the TPU within the final segment of the limb. Loops are then added to the inlay to give the silicone a structure to wrap around. Finally, ten spines are patterned along the curved path at the end of each limb as seen in \Fig~\ref{fig:Isometric Endcap View}. After the first few attempts, it was quickly realized that the TPU inlay and DragonSkin assembly should maintain ground clearance such that only the spines interact with the surface plane as seen in \Fig~\ref{fig:Endcap FigureS}(c-d).

% % \textbf{Mold design}
% One major challenge in fabrication of the endcaps pertains to the multi-material design which requires several fabrication steps. To simplify the fabrication process, a three part mold is created and 3D printed. The first part acts as a holder for the spines such that a 45$\degree$ angle is maintained during casting. The second part of the mold secures the TPU inlay in place and makes half of the outline for the silicone. The final part solely acts as a dam and must be removable to allow for the insertion and removal of microspines. \Fig~\ref{fig:Endcap FigureS}(e-f) shows the individual parts and assembled mold. 

% %\subsection{Endcap Fabrication}
% First the TPU inlays are printed on an FDM machine. The three separate mold pieces are printed using PLA, allowing for insertion of heat set threaded inserts. After adding the TPU inlay and spines, the mold parts are screwed together. The mold is prepared with Ease Release 200 for quick and easy detachment of the silicone from the mold. Equal parts of DragonSkin\textsuperscript{TM} A and B are measured, vacuumed, combined, and poured into the mold to cure. After the silicone is fully set, the fully constructed endcap is removed and attached to the robot as seen in \Fig~\ref{fig:Endcap on Bot}.
%%%%%%%%%%%%%%%%%%%%%%%%%%%%%%%%%%%%%%%%%%%%%%%%%%%%%%%%%%%%%%%%%%%%%%%%%%%%%%%%