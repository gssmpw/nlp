%%%%%%%%%%%%%%%%%%%%%%%%%%%%%%%%%%%%%%%%%%%%%%%%%%%%%%%%%%%%%%%%%%%%%%%%%%%%%%%%
\section{CONCLUSION AND FUTURE WORK}
SoRos have shown immense potential with inherent conformability and adaptability to a multitude of surfaces, yet they previously lacked adequate grip stability to overcome non-uniform surfaces present outside of lab environments. Compliant microspines are one missing piece towards shrinking this real-life realization gap. We propose an elegant, compliant microspine design with a standardized soft-compliant integration technique. The stacked array configuration enables the SoRo to maintain surface interaction when extreme surface discrepancies are present. We provide results from a set of field experiments reflecting the improved performance of two microspine array configurations over a baseline SoRo on four different, ruggedized surfaces. %\hl{Our results indicate that the 1ML design is superior to 0ML with significantly increased planar movement on all tested surfaces, and the 2ML design results in improved performance on three of the surfaces with an emphasis on increased repeatability across trials.} 
Our results indicate that microspines are a vital technology for increasing terrain traversability in mobile SoRos. Future work includes optimizing microspine array configurations for different surfaces, performing additional field experiments, and exploring the generalizability of the design to different prototypes.





% %1. Designed microspine array grippers for soft robot.
% % 2. The modular design allowed for exploration of three spine configurations - xx,xx,xx
% % 3. The three test surfaces were yy, yy, yy
% % 4. The experimental setup allowed for real-time tracking of the robot pose. 
% % 180 experiments were conducted to investigate repeatability of locomotion.
% % A translation gait was used for these experiments which is an optimal translation gait for SoRo locomotion on rubber mat with no spines.
% % All three surfaces, there is increased surface interaction for both the spine configurations when compared with no spine configuration. 
% % This is visible in the increased coupled rotation and translation of the robot during the gaits.
% Modular soft microspine endcap grippers are designed for a three-limb SoRo to increase surface interaction regardless of surface topography. The modular design allows for different configurations to be explored, namely no spines, inward facing spines, and directional spines. Experiments are performed on three variable, modular surfaces: porous wood, rubber mat, and smooth whiteboard. Real-time tracking of the robot pose is used to record relative and cumulative displacement and rotation over a $49.5\sec$ trial. An optimal translation gait identified for the three-limb SoRo on a rubber mat without spines is used for all experiments. 20 experiments are performed for each surface and spine configuration combination for a total of 180 experiments to investigate locomotion repeatability. Increased surface interaction on every surface for inward facing and directional spines is observed when compared to no spines. Decreased variance in translation is seen in the smooth and porous surfaces, showing an increase in repeatability across trials when microspines are present.
% Additionally, the rotation and translation seen during the gaits are tightly coupled as there is never a significant increase in translation without a similar increase in rotation. In short, we find that microspine grippers provide a significant increase in the engagement between the robot and environment. %This is seen even more in preliminary testing of the SoRo on inclines as the spherically reconfigurable design orients itself towards the path of least resistance, introducing greater rotation on uneven surfaces.
% %During preliminary testing of the three-limbed SoRo on inclines, increased rotation was observed. The SoRo is spherically reconfigurable and wants to orient itself towards the path of least resistance which is down a given incline. 

% % Despite not being an optimal gait for all the other surfaces and spine configurations, an increased repeatability (decreased variance) for translation is observed.


% %Microspine grippers are small spines commonly found on insect legs that reinforce surface interaction by engaging with asperities to increase shear force and traction. An array of such microspines, when integrated into a robot can provide them with the ability to maneuver uneven terrains, inclines and even climb walls. The surface conformability and adaptability of soft robots (SoRos) makes them an ideal candidates for traversing complex terrains. %
% % Taking inspiration from nature, an array of these microspines can be attached to end effectors, enabling robots to traverse uneven terrain on inclines and climb vertical walls.
% %One way of Despite recent advancements, there remains a real-life realization gap for soft locomotors pertaining to their transition from controlled lab environment to the field.
% % in the field of soft robotics, the jump from controlled lab environments to real-world, unstructured test sites remains a core challenge. 

% % Integration of microspines into SoRos has the potential to shrink this gap by improving grip stability for traversability. %
% % % Integrating microspines into SoRos increases stability and traversal capabilities, effectively shrinking the reality gap.
% % In this paper, we propose a passive modular microspine array endcap for motor-tendon actuated (MTA) soft robots. Here, the direction of the microspine array in the endcap can be varied. 
% % % We use a spherically reconfigurable MTA three-limb SoRo design from a previous work as our experimental prototype. We propose a modular endcap design for MTA SoRos outfitted with different passive microspine configurations. 
% % Consequently, modularity enables exploration of nonidentical array configuration schemes to account for different surfaces. %
% % These microspine arrays are combined with a three-limb MTA SoRo for realizing  planar locomotion with real-time tracking. Experiments are conducted on smooth, rough, and porous surfaces with different array configurations. The directional configuration, where microspines are oriented in the robot forward direction, differs from the inward configuration where they are aligned in the same direction w.r.t. the individual limb.
% % %Planar locomotion tests are performed on flat ground with real-time tracking. We demonstrate experiments on smooth, rough, and porous surfaces to evaluate the impact of microspine placement on a variety of surfaces. 
% % Experiments are conducted using a push-pull translation gait for 180 trials where 20 trials were conducted for a particular configuration and surface. Each trial comprised of 45 gaits to examine repeatability. Results indicate that spine array integration increases displacement on all three surfaces and, on average, the directional configuration robot moves twice as much when compared with no spines. Additionally, for the given gait, they improve locomotion repeatability and reliability. The experiments show the microspine gripper consistently increases engagement of the robot with the surfaces.

% Future work will include experiments on more complex surfaces including inclines and dirt/grass (real-world). Controlled locomotion gait sequences will be explored to account for the changing inclines while minimizing rotation, but some amount of rotation will likely always be present. Increased surface area interaction is vital for the microspines to be effective. This can be achieved by attaching a larger array of spines to account for missed interactions. Consequently, we plan to investigate utilizing active microspine arrays. This will have the added capability of reconfiguring the SoRo to deploy microspines dependent on the surface topography as well as including microspines in only portions of gait sequences.
% %Different microspine materials will be tested as deterioration can occur over time on surfaces such as carpet where a large interaction takes place. 
% %This is clear in the literature where robots can have up to 250 spines/limb forcing a meaningful interaction occurs every time. 
% %Since the current robot is spherically reconfigurable, which innately introduces rotation into any gait and the system as a whole. This is a much larger problem on inclined surfaces as the robot wants to orient itself towards the path of least resistance, ultimately moving down the incline. Additionally, it is clear from the testing that the robot is not a perfectly stable system. Despite aligning the robot in the same orientations for each test case, the variance in the results is very noticeable. Simply stating that a symmetric robot would fix this problem is not the case. The researchers believe that stability is the more important factor than symmetry, hence designing and manufacturing a robot that is agnostic towards inclined surfaces through stability is a must for future works.
% %All mentioned approaches above will increase the overall performance, but without a holistic approach, the robot will not be able to be taken into the real world with meaningful testing cases. 
% % The addition of passive microspine endcaps increased the distance and speed traveled as well as the consistency of locomotion. The future improvements mentioned above will increase the overall performance, but a holistic approach must be taken as all parameters are linked. As progress continues, more surfaces and real world tests will be able to be conducted and applications can be built from this soft robotic platform.