%%%%%%%%%%%%%%%%%%%%%%%%%%%%%%%%%%%%%%%%%%%%%%%%%%%%%%%%%%%%%%%%%%%%%%%%%%%%%%%%
\section{MICROSPINE ORIENTATION AND SURFACE INTERACTION}
\label{Sec:Microspine}
The effectiveness of engaging with surface asperities is influenced by the array orientation, i.e., the angle at which the microspines are embedded. Preliminary tests are performed to identify the desirable angle of microspines to maximize shear force. Microspines angled at 45$\degree$, 60$\degree$, and 75$\degree$ are embedded in 30A Dragon Skin\textsuperscript{TM} silicone squares. These test blocks, weighing 51g,  are then connected to a fishing scale and pulled horizontally across a rough silicon mat, smooth whiteboard, and carpet square to provide variance in surface friction coefficient. For each surface, the maximum force observed before slipping occurred is recorded, shown in \Fig~\ref{fig:Pin Test}. The $45\degree$ angle is chosen as it displayed the highest resistive force across all the surfaces.

\begin{figure}[h]
    \centering
    \includegraphics[width=0.9\linewidth]{Figures/pin_test.png}
    \caption{The influence of the microspine angles on friction coefficient.}
    \label{fig:Pin Test}
\end{figure}

The microspines engaged into the surface asperities on softer surfaces, carpet and the silicon mat, requiring more pulling force. In contrast, on the rigid smooth surface, the spines do not display the increased interaction force. In fact, the engagement is so limited that the interaction force is lower than the no-spine silicone test square. Hence, this test is also able to show the extreme differences realized by the spine angles while obtaining an optimized angle for the maximum traction force.
%%%%%%%%%%%%%%%%%%%%%%%%%%%%%%%%%%%%%%%%%%%%%%%%%%%%%%%%%%%%%%%%%%%%%%%%%%%%%%%%