%%%%%%%%%%%%%%%%%%%%%%%%%%%%%%%%%%%%%%%%%%%%%%%%%%%%%%%%%%%%%%%%%%%%%%%%%%%%%%%%
\section{INTRODUCTION} 
In nature, animals resist slipping and falling by increasing interaction with surfaces in a number of ways. Snakes maintain a large surface area in contact with the ground to increase the amount of propulsive force aiding in locomotion. Snake inspired robots are developed with textured skins for increased traction \cite{liu_kirigami_2019, mckenna_toroidal_2008}. However, the efficacy is unknown for limbed robots equipped with these skins and will require additional coordination. Geckos utilize van der Waals interactions to provide an adhesive force against surfaces. Attaching gecko inspired, synthetic adhesives to extremities of robots is an attractive option due to the range of surfaces the robots can adhere to \cite{chen_testing_2022, hajj-ahmad_grasp_2023, han_climbing_2022, sangbae_kim_smooth_2008, sikdar_gecko-inspired_2022}. These directional adhesion mechanisms are sensitive to wear and tear and require periodic cleaning for reliable adhesion. Insects such as caterpillars and cockroaches are able to climb vertical surfaces with small spines attached to their legs. These microspine grippers increase shear force and traction by engaging with surface asperities. They do not penetrate surfaces, but rather reinforce surface interaction by gripping onto jagged, microscopic edges. Biological studies have looked at the relationship between microspine arrays and surface interaction, comparing the efficacy on different smooth and rough surfaces \cite{dai_roughness-dependent_2002}. The outcome from these studies encourage the creation of synthetic microspine arrays to enable maneuverability of uneven terrains, traversability of inclines, and climbing walls \cite{asbeck_scaling_2006, asbeck_designing_2012, iacoponi_simulation_2020, wang_design_2017, wang_palm_2016, jiang_stochastic_2018, sangbae_kim_spinybotii_2005, wang_spinyhand_2019, zi_mechanical_2023, asbeck_climbing_nodate, liu_novel_2019}. These arrays can become very complicated, and there exists a trade-off between design complexity and surface adaptability for many of these robots as shown in \Fig \ref{fig:adapt}.

Climbing and crawling rigid robots developed in the past decade employ similar methods to combat slipping along vertical walls \cite{asbeck_climbing_nodate, li_structure_2020, martone_enhancing_2019, ruotolo_load-sharing_2019, xu_rough_2016}. JPL's LEMUR IIB, a four-limbed robot, features multiple grippers, each equipped with at least 250 microspines capable of adhering to both convex and concave surfaces. Each microspine is suspended with a steel hook, allowing it to stretch and move relative to its neighbors to find suitable surfaces to grip. This enables the robot to climb vertically and traverse on cave ceilings. The shear forces from the microspines integrated in the gripper gives the ability to support heavy loads \cite{parness_gravity-independent_2013}. These grippers possess the necessary traction to navigate microgravity environments, such as asteroids, comets, and small moons, where conventional ground pressure is lacking \cite{parness_lemur_2017}. Microspines latch into small asperities on a rock’s surface such as pits, cracks, slopes, or any kind of topography that will snag a hook. Success of an individual spine catching is dependent on surface roughness, surface geometry, and asperity distribution \cite{parness_maturing_2017}. %Rotary Microspines, unlike linear ones, were used to build wheels that can climb curbs, ledges and vertical walls \cite{carpenter_rotary_2016}.

Integrating microspines into Soft Robots (SoRos) is an attractive option due to their adaptability and conformability to changing surface topologies, highlighted in \Fig \ref{fig:tree}. %SoRos use passive soft material properties for additional compliance, flexibility, unique actuation, and control. 
The continuum nature and impact resistance of soft materials passively allow SoRos additional flexibility and more effective interaction with complex and non-uniform surfaces. %However, experimenting in real world scenarios (unstructured environment e.g., grass/dirt) requires higher traction and gripping power to overcome unknown and complex obstacles. 
However, SoRos lack grip stability, contributing to them historically struggling with efficient locomotion as well as locomoting over unstructured terrain. Because of this, SoRo designs that can traverse outside and perform real tasks outside of a lab setting are under-researched. Utilizing microspines has the potential to improve traction, increase grip stability, and provide the ability to reliably maneuver real-world terrains. One of the main design challenges pertains to integrating a soft, low stiffness body with hard, high stiffness micropsines. A two segment, wriggling SoRo avoids this challenge by adhering an array of dual material, ``soft microspines'' made of rubber along the ventral of the body to successfully increase anisotropic friction \cite{ta_design_2018}. A starfish inspired SoRo implements a similar technique by including soft, tube feet reminiscent of microspines along the entire underside of the five-limbed robot. Magnetization of the tube feet allows omnidirectional movement and reduces the motion resistance from the ground, enhancing the adaptability of the SoRo on different surfaces \cite{yang_starfish_2021}. Embedding hard objects in soft materials through intelligent mechanical design is necessary to take advantage of the benefits of hard spines without hindering the soft deformable properties. A soft inchworm design attaches an array of microspines to either foot of the inchworm using adhesive bonding technology \cite{hu_inchworm-inspired_2019}. However, deeply irregular surfaces remain difficult if not impossible to overcome due to the uniform distribution of the microspines and integration technique that restricts the usage to surfaces with regular, fine asperities. This shows the need for compliance and independent movement per microspine to increase surface geometry traversability.


\begin{figure}[h]
    \centering
    \includegraphics[width=\linewidth]{Figures/adaptV2.png}
    \caption{Surface adaptability vs. design complexity with various microspine robots \cite{asbeck_scaling_2006, wang_palm_2016, parness_lemur_2017, liu_novel_2019, hu_inchworm-inspired_2019,Daltorio_Wei_Horchler_Southard_Wile_Quinn_Gorb_Ritzmann_2009, Spenko_Haynes_Saunders_Cutkosky_Rizzi_Full_Koditschek_2008} compared against one of the presented designs, 1ML.}
    \label{fig:adapt}
\end{figure}


\begin{figure}[!t]
    \centering
    \includegraphics[width=0.7\linewidth]{Figures/tree_root.jpg}
    \caption{Conformable nature of soft limbs enable traversal over large tree root present on forest floor surface.}
    \label{fig:tree}
\end{figure}


%Inspired by animals, microspines were first developed at Stanford University. When integrated, they provide robots with adhesive capabilities. JPL's LEMUR IIB, a four-limbed robot, features multiple grippers, each equipped with at least 250 microspines capable of adhering to both convex and concave surfaces. Each microspine has its suspension structure with a steel hook, allowing it to stretch and move relative to its neighbors to find suitable surfaces to grip. This enables the robot to climb vertically and traverse on cave ceilings. The shear forces from the microspines integrated in the gripper gives it the ability to support heavy loads. \cite{parness_gravity-independent_2013}
%These grippers possess the necessary traction to navigate microgravity environments, such as asteroids, comets, and small moons, where conventional ground pressure is lacking \cite{parness_lemur_2017}. Microspines latch into small asperities on a rock’s surface such as pits, cracks, grains, slopes or any kind of topography that a hook can snag. Grasping by spines is dependent on surface roughness, surface geometry, material \cite{parness_maturing_2017}. Rotary Microspines unlike linear ones were used to build wheels that can climb curbs, ledges and vertical walls \cite{carpenter_rotary_2016}.

%Discuss increase of shear force and traction as being main benefit of microspines (Mark + JPL papers). JPL papers focusing on climbing/gripping robots \cite{carpenter_rotary_2016, merriam_microspine_nodate, parness_lemur_2017, parness_maturing_2017, parness_gravity-independent_2013}. Robots with gecko-like adhesives \cite{chen_testing_2022, hajj-ahmad_grasp_2023, han_climbing_2022, sangbae_kim_smooth_2008, sikdar_gecko-inspired_2022}.

%Gripping robots that grab things? \cite{chen_reachbot_2022, hajj-ahmad_grasp_2023, li_underactuated_2023, merriam_microspine_nodate, xu_design_2018}

%There are many place in real life that microspines can be observed in a biological setting. Insects are the most common case with key examples such as caterpillars and cockroaches, but these spines can also be seen scaling up to terrestrial mammals, like what we see in cats. These spines are either passively deployed at all times, as in the case of insects, or are actively deployed, as in the case of cats. While these spines can greatly increase the locomotive abilities, allowing the animals to climb trees, walls, and stick on inclines, they can also serve as a hindrance causing the animal to get stuck and be unable to move without outside assistance. Since this robot is replicating a passive form of spines, and not active, the abilities of the spines must be evaluated such that the scenarios in which the robot's movement are hindered are known, and that in other use cases, the robot is able to outperform it's base state. 

%Next, bring SoRos into the picture: SoRos are really slow, so microspines can be added to increase speed as well as \textbf{where} they can traverse. Microspines help close reality gap. Discuss difficulties with attaching microspines to soft materials and what's already been done - soft on soft, soft on hard, etc. Soft robots with microspines \cite{hu_inchworm-inspired_2019, ta_design_2018, yang_starfish_2021, noauthor_monolith_nodate}.
%Examples of flying robots with microspine latching attachments \cite{mangan_autonomous_2017}. Examples of climbing robots with microspines on ends of legs or feet \cite{}. Robots using microspines with magnets \cite{lee_design_2011, yang_starfish_2021}. Robots using secondary skins for increasing surface interaction \cite{liu_kirigami_2019, mckenna_toroidal_2008}.

%Soft robots (SoRos) often try to integrate components designed for classic, rigid robots into their construction. This bridging of soft with hard generates design challenges as well as potential limitations of the unique attributes soft materials offer. On the other hand, designing robots exclusively with soft components can severely limit the robot capabilities. For this reason, the coexistence of embedding hard in soft through intelligent mechanical design is advantageous.

\textit{Contributions.} We aim to bridge the realization gap by attaching microspine technology onto the tips of soft Motor Tendon Actuated (MTA) limbs, vastly improving the grip stability and types of traversable terrain of mobile SoRos.  We propose an elegant, single-material design with intelligent soft-compliant integration that reduces complexity by using a single actuator to passively control an entire array fixed to the end of a soft limb. In this research, we 
(1) propose a compliant mechanism, two-row stacked microspine array design that improves grip stability and increases traversable surface topologies of mobile SoRos;
(2) identify critical design parameters that improve locomotion capabilities while reducing complexity by controlling an entire microspine array with only a single actuator through intelligent soft-compliant integration;
(3) investigate the grip stability and repeatability of a baseline SoRo compared against two different microspine array configurations on uniform concrete, partially uniform brick, granular compact sand, and non-uniform tree roots; and
(4) analyze tracking results that indicate the inclusion of compliant microspine arrays in SoRos increases planar displacement on all surfaces through enhanced surface engagement resulting in capabilities to traverse complex, unstructured environments.

\textit{Paper Organization.} The next section explores critical design parameters of the compliant mechanism microspine array and the SoRo prototype. The third section discusses the three experimental prototypes, experimental setup, and pose estimation algorithm. The fourth section highlights the results, validating the microspine array design. The fifth section concludes the paper and discusses the future work.
%%%%%%%%%%%%%%%%%%%%%%%%%%%%%%%%%%%%%%%%%%%%%%%%%%%%%%%%%%%%%%%%%%%%%%%%%%%%%%%%