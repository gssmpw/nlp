%%%%%%%%%%%%%%%%%%%%%%%%%%%%%%%%%%%%%%%%%%%%%%%%%%%%%%%%%%%%%%%%%%%%%%%%%%%%%%%%
\section{RESULTS}
%\subsection{Surface Results}
%\hl{Have more detailed discussion about specific surfaces - consider classifying as uniform, partially uniform, etc.}
The translation results of 0ML, 1ML, and 2ML are compared on 4 surfaces. %The overall trends of planar movement of the three prototypes on the four surfaces are briefly discussed. 
%We remind the reader that for each prototype on a given surface, three trials were performed. %The table below shows the total average displacement and rotation for all three runs of a SoRo on a given surface, with \textbf{bolded} numbers representing the greatest quantity for a given surface and \textit{italicized} numbers representing the lowest value.
%A short discussion is had on the effect of adding compliant microspines onto the tips of this specific MTA SoRo design.  Additionally, we discuss the coupling that occurs between translation and rotation with this given three-limb SoRo design.
\Fig \ref{fig:data} shows the average displacement and standard deviation of the 3 trials per prototype. 

The concrete surface is level with a uniform distribution of asperities. Here, both 1ML and 2ML outperform 0ML in terms of displacement. In fact, 1ML had an average displacement of 39.63cm which is over 30 times as much as the 1.32cm that 0ML traveled. 2ML traveled 14.13cm, nearly 11 times as much as 0ML. 2ML had higher consistency across trials, having a standard deviation of 0.76 whereas 0ML's standard deviation was 0.83.

The partially uniform brick contains gaps in between bricks that can cause microspines to become stuck. This was observed to happen randomly across all trials. However, in every instance where this occurred, the microspine that was caught on 1ML or 2ML was able to wiggle free within a few seconds and continue moving. Even with these obstacles, both 1ML traveling 15cm and 2ML traveling 10.52cm had far greater average displacement than 0ML, roughly 25 times and 18 times more, respectively. This was the only surface where 0ML had the lowest standard deviation, and this is due to the microspine limbs on the other two robots randomly getting stuck when crossing over brick perimeters. Additionally, this surface resulted in the lowest overall movement of 0ML, with an average displacement of 0.59cm. %The displacement was so consistent because the robot had difficulty overcoming the terrain at all.

The granular, compact sand was not entirely level and various pebbles, holes, insects, and small sticks were scattered around the testing area. This surface was difficult to overcome for all three robots as the compact sand was covered in a loose, granular top layer that was easy to become partially submerged in. Due to this, the average displacement is far lower on this surface. The two microspine limbs on 2ML seemed to dig itself deeper, resulting in average displacement of only 0.46cm and the lowest standard deviation. 1ML still outperformed 0ML with over 3 times increased displacement, traveling 2.53cm compared to 0.82cm.


\begin{figure}[!h]
    \centering
    \includegraphics[width=\linewidth]{Figures/micro_dataV3.png}
    \caption{The average displacement data per prototype and surface combination. The repeatability is shown in error bars.}
    \label{fig:data}
\end{figure}


\begin{figure}[!h]
    \centering
    \includegraphics[width=\linewidth]{Figures/del_x_del_yV4.png}
    \caption{Gait analysis. a) Every $\Delta$X and $\Delta$Y position per gait on concrete and brick. b) The average absolute $\Delta$X and $\Delta$Y displacement data per gait on concrete and brick.}
    \label{fig:delta}
\end{figure}

The forest floor surface was completely non-uniform with highly varying terrain. Specifically, the prototypes had to first overcome a large, 4" tall tree root and then move through leaves, acorns, and other tree debris. All SoRos were able to successfully traverse over the cylinder-like tree root due to the conformable nature of the soft limbs, highlighted in \Fig \ref{fig:tree}. However, only the 1ML and 2ML were able to navigate through the thick tree debris after making it over the large tree root; 0ML became stuck at the base of the tree root in each of its three trials. This is exhibited in the average displacement of 13.74cm for 0ML, 23.95cm for 1ML, and 21.32cm for 2ML. The forest floor was critical for testing as it was the most unstructured of the four experimental surfaces and showcases the benefits of the soft limbs paired with the added gripping stability of the microspine array.

For each of the 36 trials, the robots move for 60 gaits, resulting in a total of 2,160 gaits or 720 gaits per robot. Each gait results in relative change in position in the robot coordinate system $\Delta X, \Delta Y$. The locomotion consistency can be analyzed by examining the 180 poses on a given surface per prototype, such as those shown in \Fig \ref{fig:delta}a). The average displacement per gait ($\Delta X,\Delta Y)$  as well as the standard deviation over all gaits per prototype/surface combination is visualized in \Fig \ref{fig:delta}b). We only analyze the uniform/partially uniform surfaces as the gait-to-gait movement is much less consistent due to non-uniformity and randomness for the other two surfaces. On concrete, the average gait displacement per gait of both 1ML $(0.65cm,0.44cm)$ and 2ML $(0.20cm,0.15cm)$ is greater than 0ML $(0.07cm,0.06cm)$. On brick, both 1ML $(0.22cm,0.19cm)$ and 2ML $(0.13cm,0.15cm)$ outperform 0ML $(0.02cm,0.01cm)$. The relative standard deviation (standard deviation/mean) for 1ML and 2ML is lower than that for  0ML, indicating greater grip stability through improved repeatability.

Examples of a single trial of each prototype row on each surface column is visualized in \Fig \ref{fig:surfaces} with additional data in the accompanying video. The starting and end points are green and blue dots, and the path of traversal is a red, gradient line. On all surfaces, 1ML interacts with the environment significantly more than the baseline 0ML, resulting in greater overall movement. On every surface except for compact sand, 2ML outperforms 0ML. On concrete and compact sand, 2ML had the lowest standard deviation and on the forest floor, 1ML had the lowest standard deviation. This indicates the addition of microspine arrays also increasing the consistency and repeatability of planar locomotion with SoRos. 
%Line graphs plot the displacement of each run per SoRo on a given surface, resulting in four total graphs.  Discrepancies of consistency and improved(?) locomotion are shown.




\begin{figure*}[htbp]
    \centering
    \includegraphics[width=0.9\textwidth]{Figures/qual_results2V2.png}
    \caption{Experimental results for 0ML, 1ML, and 2ML on concrete, brick, compact sand, and a forest floor. Rows represent the different surfaces increasing in unstructured nature with the three different prototypes distinguished by columns.}
    \label{fig:surfaces}
\end{figure*}



% \begin{figure*}[h]
%     \centering
%     \begin{subfigure}{0.3\linewidth}
%          \centering
%          \includegraphics[width=\textwidth]{Figures/Sequence 02_crop.png}
%          %\caption{0ML Concrete}
%          \label{fig:base}\vspace{-9pt}
%      \end{subfigure}
%      \begin{subfigure}{0.3\linewidth}
%          \centering
%          \includegraphics[width=\textwidth]{Figures/Sequence 5_crop.png}
%          %\caption{1ML Concrete}
%          \label{fig:one}\vspace{-9pt}
%      \end{subfigure}
%      \begin{subfigure}{0.3\linewidth}
%          \centering
%          \includegraphics[width=\textwidth]{Figures/Sequence 9_crop.png}
%          %\caption{2ML Concrete}
%          \label{fig:two}\vspace{-9pt}
%      \end{subfigure}
%      \begin{subfigure}{0.3\linewidth}
%          \centering
%          \includegraphics[width=\textwidth]{Figures/Sequence 11_crop.png}
%          %\caption{0ML Brick}
%          \label{fig:base}\vspace{-9pt}
%      \end{subfigure}
%      \begin{subfigure}{0.3\linewidth}
%          \centering
%          \includegraphics[width=\textwidth]{Figures/Sequence 20_crop.png}
%          %\caption{1ML Brick}
%          \label{fig:one}\vspace{-9pt}
%      \end{subfigure}
%      \begin{subfigure}{0.3\linewidth}
%          \centering
%          \includegraphics[width=\textwidth]{Figures/Sequence 15_crop.png}
%          %\caption{2ML Brick}
%          \label{fig:two}\vspace{-9pt}
%      \end{subfigure}
%      \begin{subfigure}{0.3\linewidth}
%          \centering
%          \includegraphics[width=\textwidth]{Figures/Sequence 21_crop.png}
%          %\caption{0ML Sand}
%          \label{fig:base}\vspace{-9pt}
%      \end{subfigure}
%      \begin{subfigure}{0.3\linewidth}
%          \centering
%          \includegraphics[width=\textwidth]{Figures/Sequence 26_crop.png}
%          %\caption{1ML Sand}
%          \label{fig:one}\vspace{-9pt}
%      \end{subfigure}
%      \begin{subfigure}{0.3\linewidth}
%          \centering
%          \includegraphics[width=\textwidth]{Figures/Sequence 25_crop.png}
%          %\caption{2ML Sand}
%          \label{fig:two}\vspace{-9pt}
%      \end{subfigure}
%      \begin{subfigure}{0.3\linewidth}
%          \centering
%          \includegraphics[width=\textwidth]{Figures/Sequence 46_crop.png}
%          %\caption{0ML Forest}
%          \label{fig:base}
%      \end{subfigure}
%      \begin{subfigure}{0.3\linewidth}
%          \centering
%          \includegraphics[width=\textwidth]{Figures/Sequence 40b_crop.png}
%          %\caption{1ML Forest}
%          \label{fig:one}
%      \end{subfigure}
%      \begin{subfigure}{0.3\linewidth}
%          \centering
%          \includegraphics[width=\textwidth]{Figures/Sequence 44_crop.png}
%          %\caption{2ML Forest}
%          \label{fig:two}
%      \end{subfigure}
%     \caption{Experimental results for 0ML, 1ML, and 2ML on concrete, brick, compact sand, and a forest floor. Rows represent the different surfaces with the three different prototypes distinguished by columns.}
%     \label{fig:surfaces}
% \end{figure*}

% %\begin{table}
% %\caption{Test}
% \begin{center}
% \begin{tabular}{|c|c|c|c|} 
%  \hline
%  Robot & Terrain & $\Delta$ Displacement & $\Delta$ Rotation \\ %[0.5ex] 
%  \hline\hline
%  0ML & Concrete & \textit{1.32} & \textit{17.98} \\ 
%  \hline
%  1ML & Concrete & \textbf{39.63} & \textbf{131.30} \\
%  \hline
%  2ML & Concrete & 14.13 & 68.70 \\
%  \hline\hline
%  0ML & Brick & \textit{0.59} & \textit{2.32} \\ 
%  \hline
%  1ML & Brick & \textbf{15.00} & \textbf{77.25} \\
%  \hline
%  2ML & Brick & 10.52 & 37.16 \\
%  \hline\hline
%  0ML & Compact Sand & 0.82 & \textit{3.36} \\ 
%  \hline
%  1ML & Compact Sand & \textbf{2.53} & \textbf{49.88} \\
%  \hline
%  2ML & Compact Sand & \textit{0.46} & 4.82 \\
%  \hline\hline
%  0ML & Forest Floor & \textit{13.74} & 69.96 \\ 
%  \hline
%  1ML & Forest Floor & \textbf{23.95} & \textit{34.85} \\
%  \hline
%  2ML & Forest Floor & 21.32 & \textbf{158.77} \\
%  \hline %[1ex] 
% %\caption{Table Results}
% \end{tabular}
% \end{center}
% %\end{table}

% %\begin{center}
% %\begin{tabular}{|c|c|c|c|c|c|c|c|c|c|c|c|c|}
% %\hline
% %\multicolumn{13}{|c|}{Rotation Gait} \\% 
% %\hline
% %\multicolumn{1}{|c|} {Incline} & %
% %\multicolumn{3}{|c|} {Wood} & 
% %\multicolumn{3}{|c|} {White Board} & 
% %\multicolumn{3}{|c|} {Black Mat} & 
% %\multicolumn{3}{|c|} {Carpet} \\ \hline
% %$0^{\circ}$ & & & & \hspace{0.3cm} & \hspace{0.3cm} & \hspace{0.3cm} & \hspace{0.2cm} & %\hspace{0.2cm} & \hspace{0.2cm} & & &\\ \hline
% %$5^{\circ}$ & & & & & & & & & & & &\\ \hline
% %$10^{\circ}$ & & & & & & & & & & & &\\ \hline
% %$15^{\circ}$ & & & & & & & & & & & &\\ \hline
% %$20^{\circ}$ & & & & & & & & & & & &\\ \hline
% %$25^{\circ}$ & & & & & & & & & & & &\\ \hline
% %$30^{\circ}$ & & & & & & & & & & & &\\ \hline
% %\end{tabular}
% %\end{center}
% % \begin{table}
% % \begin{center}
% % \begin{tabular}{|c|c|c|c|c|c|c|c|c|c|c|c|c|}
% % \hline
% % \multicolumn{13}{|c|}{Translation Gait} \\% 
% % \hline
% % \multicolumn{1}{|c|} {Incline} & %
% % \multicolumn{3}{|c|} {Wood} & 
% % \multicolumn{3}{|c|} {White Board} & 
% % \multicolumn{3}{|c|} {Black Mat} & 
% % \multicolumn{3}{|c|} {Carpet}\\\hline
% % $0^{\circ}$ & & & & \hspace{0.3cm} & \hspace{0.3cm} & \hspace{0.3cm} & \hspace{0.2cm} & \hspace{0.2cm} & \hspace{0.2cm} & & &\\\hline
% % $3^{\circ}$ & & & & & & & & & & & &\\\hline
% % \end{tabular}
% % \caption{Average Displacement}
% % \label{table:1}
% % \end{center}
% % \end{table}

% %\pagebreak
% \begin{figure*}[h]
%      \centering
%      \begin{subfigure}{0.32\textwidth}
%          \centering
%          \includegraphics[width=\textwidth]{Figures/rubber_stddev.png}
%          %\caption{Black Mat}
%          \label{fig:black stddev}
%      \end{subfigure}
%      \begin{subfigure}{0.32\textwidth}
%          \centering
%          \includegraphics[width=\textwidth]{Figures/smooth_whiteboard_stddev.png}
%          %\caption{Whiteboard}
%          \label{fig:white stddev}
%      \end{subfigure}
%      \begin{subfigure}{0.32\textwidth}
%          \centering
%          \includegraphics[width=\textwidth]{Figures/porous_wood_stddev.png}
%          %\caption{Wood}
%          \label{fig:wood stddev}
%      \end{subfigure}
%           \begin{subfigure}{0.32\textwidth}
%          \centering
%          \includegraphics[width=\textwidth]{Figures/rubber_disp_rot.png}
%          \caption{Rubber Mat}
%          \label{fig:black disp vs rot}
%      \end{subfigure}
%      \begin{subfigure}{0.32\textwidth}
%          \centering
%          \includegraphics[width=\textwidth]{Figures/whiteboard_disp_rot.png}
%          \caption{Whiteboard}
%          \label{fig:white disp vs rot}
%      \end{subfigure}
%      \begin{subfigure}{0.32\textwidth}
%          \centering
%          \includegraphics[width=\textwidth]{Figures/wood_disp_rot.png}
%          \caption{Wood}
%          \label{fig:wood disp vs rot}
%      \end{subfigure}
%      \begin{subfigure}{0.4\textwidth}
%          \centering
%          \includegraphics[width=\textwidth]{Figures/graph_legend.png}
%      \end{subfigure}
%     \caption{The displacement and rotation results of the translation gait trials ($49.5\sec/\mathrm{trial}$) for the three surfaces and three spine configurations. The increased engagement with the surfaces is observed through coupled increase in translation-rotation quantities.}
%     \label{fig:disp vs rot}
% \end{figure*}
% %%%%%%%%%%%%%%%%%%%%%%%%%%%%%%%%%%%%%%%%%%%%%%%%%%%%%%%%%%%%%%%%%%%%%%%%%%%%%%%%
% The results for the planar locomotion tests on the three experimental surfaces - rubber mat, smooth whiteboard, and porous wood - are plotted as displacement vs. rotation in \Fig \ref{fig:disp vs rot}. For each surface, 20 trials per spine configuration are performed resulting in a total of 180 trials. Each trial comprised of 45 gaits with total duration of $1.1\sec\times45=49.5\sec$ .The reasoning behind the large number of tests is to investigate the consistency of movement of SoRos with and without microspines.

% The baseline SoRos without spine endcaps travel less distance than both spine configurations on all three surfaces. Additionally, there is a large variance in displacement on the smooth surface. The inward microspines increase overall displacement on all three surfaces, and display the least amount of rotation on average on the smooth and porous surfaces. Finally, the SoRos equipped with the directional microspine endcaps have the greatest displacement on the rubber and smooth surfaces.

% On rough surfaces containing asperities, microspines increase overall distance traversed, confirming what is seen in literature. On smooth surfaces where microspines historically have not been utilized, we find that they increase locomotion consistency. By running multiple tests for each set of parameters, we provide a more accurate picture of SoRo locomotion, microspine effectiveness, and the need for repeatability - an under-reported challenge in the field of soft robotics. In all tested scenarios, there is improvement in translation when microspine endcaps are present. For this specific translation gait, there is improved translation performance, but the cumulative rotation has high variance. This can be attributed to the fact that this particular gait is not optimal for all given surfaces. As mentioned earlier, the gait is optimal for a rubber mat with no spines, not for other surfaces and configurations. However, it can be concluded that there is a consistent increase in engagement of the robot with the surfaces by observing the increase in coupled translation-rotation quantities.