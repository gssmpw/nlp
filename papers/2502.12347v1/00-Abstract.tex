%%%%%%%%%%%%%%%%%%%%%%%%%%%%%%%%%%%%%%%%%%%%%%%%%%%%%%%%%%%%%%%%%%%%%%%%%%%%%%%%
\begin{abstract}
Microspine grippers are small spines commonly found on insect legs that reinforce surface interaction by engaging with asperities to increase shear force and traction. 
An array of such microspines, when integrated into the limbs or undercarriage of a robot, can provide the ability to maneuver uneven terrains, traverse inclines, and even climb walls. 
Conformability and adaptability of soft robots makes them ideal candidates for these applications involving traversal of complex, unstructured terrains. 
However, there remains a real-life realization gap for soft locomotors pertaining to their transition from controlled lab environment to the field by improving grip stability through effective integration of microspines.  
We propose a passive, compliant microspine stacked array design to enhance the locomotion capabilities of mobile soft robots, in our case, ones that are motor tendon actuated. 
We offer a standardized microspine array integration method with effective soft-compliant stiffness integration, and reduced complexity resulting from a single actuator passively controlling them.
The presented design utilizes a two-row, stacked microspine array configuration that offers additional gripping capabilities on extremely steep/irregular surfaces from the top row while not hindering the effectiveness of the more frequently active bottom row.
We explore different configurations of the microspine array to account for changing surface topologies and enable independent, adaptable gripping of asperities per microspine. 
Field test experiments are conducted on various rough surfaces including concrete, brick, compact sand, and tree roots with three robots consisting of a baseline without microspines compared against two robots with different combinations of microspine arrays. 
Tracking results indicate that the inclusion of microspine arrays increases planar displacement on average by 15 and 8 times over the baseline with the two microspine designs respectively, improves locomotion repeatability, and, critically, consistently increases terrain traversability.


% Microspine grippers are small spines commonly found on insect legs that reinforce surface interaction by engaging with asperities to increase shear force and traction. An array of such microspines, when integrated into a robot, can provide them with the ability to maneuver uneven terrains, traverse inclines, and even climb walls. The surface conformability and adaptability of soft robots (SoRos) makes them ideal candidates for traversing complex terrains.
% Despite recent advancements, there remains a real-life realization gap for soft locomotors pertaining to their transition from controlled lab environment to the field.
% Integration of microspines into SoRos has the potential to shrink this gap by improving grip stability for traversability.
% We propose a passive modular microspine array endcap for motor-tendon actuated (MTA) soft robots. Here, the direction of the microspine array in the endcap can be varied. 
% Consequently, modularity enables exploration of nonidentical array configuration schemes to account for different surfaces.
% These microspine arrays are attached to a three-limb MTA SoRo for improving planar locomotion with real-time tracking. Experiments are conducted on smooth, rough, and porous surfaces with different array configurations. The directional configuration, where microspines are oriented in the robot forward facing direction, differs from the inward configuration where they are aligned in the same direction with respect to the individual limb.
% Experiments are conducted using a push-pull translation gait for 180 trials where 20 trials (45 gaits per trial) are conducted for a particular configuration and surface. Results indicate that spine array integration increases displacement on all three surfaces and, on average, the directional configuration robot moves twice as much when compared with no spines. Additionally, for the given gait, they improve locomotion repeatability and reliability. Critically, the experiments show the microspine endcaps consistently increases engagement of the robot for all surfaces.
\end{abstract}
%%%%%%%%%%%%%%%%%%%%%%%%%%%%%%%%%%%%%%%%%%%%%%%%%%%%%%%%%%%%%%%%%%%%%%%%%%%%%%%%