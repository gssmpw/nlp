%%%%%%%%%%%%%%%%%%%%%%%%%%%%%%%%%%%%%%%%%%%%%%%%%%%%%%%%%%%%%%%%%%%%%%%%%%%%%%%%
%2345678901234567890123456789012345678901234567890123456789012345678901234567890
%        1         2         3         4         5         6         7         8

\documentclass[letterpaper, 10 pt, conference]{ieeeconf}  % Comment this line out if you need a4paper

%\documentclass[a4paper, 10pt, conference]{ieeeconf}      % Use this line for a4 paper

\IEEEoverridecommandlockouts                              % This command is only needed if 
% you want to use the \thanks command

\overrideIEEEmargins                                      % Needed to meet printer requirements.
%\bibliography{sphericon.bib}




%In case you encounter the following error:
%Error 1010 The PDF file may be corrupt (unable to open PDF file) OR
%Error 1000 An error occurred while parsing a contents stream. Unable to analyze the PDF file.
%This is a known problem with pdfLaTeX conversion filter. The file cannot be opened with acrobat reader
%Please use one of the alternatives below to circumvent this error by uncommenting one or the other
%\pdfobjcompresslevel=0
%\pdfminorversion=4

% See the \addtolength command later in the file to balance the column lengths
% on the last page of the document


% Use this sample document as your LaTeX source file to create your document. Save this file as {\bf root.tex}. You have to make sure to use the cls file that came with this distribution. If you use a different style file, you cannot expect to get required margins. Note also that when you are creating your out PDF file, the source file is only part of the equation. {\it Your \TeX\ $\rightarrow$ PDF filter determines the output file size. Even if you make all the specifications to output a letter file in the source - if your filter is set to produce A4, you will only get A4 output. }

% It is impossible to account for all possible situation, one would encounter using \TeX. If you are using multiple \TeX\ files you must make sure that the ``MAIN`` source file is called root.tex - this is particularly important if your conference is using PaperPlaza's built in \TeX\ to PDF conversion tool.

% The equations are an exception to the prescribed specifications of this template. You will need to determine whether or not your equation should be typed using either the Times New Roman or the Symbol font (please no other font). To create multileveled equations, it may be necessary to treat the equation as a graphic and insert it into the text after your paper is styled. Number equations consecutively. Equation numbers, within parentheses, are to position flush right, as in (1), using a right tab stop. To make your equations more compact, you may use the solidus ( / ), the exp function, or appropriate exponents. Italicize Roman symbols for quantities and variables, but not Greek symbols. Use a long dash rather than a hyphen for a minus sign. Punctuate equations with commas or periods when they are part of a sentence, as in

% $$
% \alpha + \beta = \chi \eqno{(1)}
% $$

% Note that the equation is centered using a center tab stop. Be sure that the symbols in your equation have been defined before or immediately following the equation. Use (1), not Eq. (1) or equation (1), except at the beginning of a sentence: Equation (1) is . . .

% Positioning Figures and Tables: Place figures and tables at the top and bottom of columns. Avoid placing them in the middle of columns. Large figures and tables may span across both columns. Figure captions should be below the figures; table heads should appear above the tables. Insert figures and tables after they are cited in the text. Use the abbreviation Fig. 1, even at the beginning of a sentence.

% \begin{table}[h]
% \caption{An Example of a Table}
% \label{table_example}
% \begin{center}
% \begin{tabular}{|c||c|}
% \hline
% One & Two\\
% \hline
% Three & Four\\
% \hline
% \end{tabular}
% \end{center}
% \end{table}


%    \begin{figure}[thpb]
%       \centering
%       \framebox{\parbox{3in}{We suggest that you use a text box to insert a graphic (which is ideally a 300 dpi TIFF or EPS file, with all fonts embedded) because, in an document, this method is somewhat more stable than directly inserting a picture.
% }}
%       %\includegraphics[scale=1.0]{figurefile}
%       \caption{Inductance of oscillation winding on amorphous
%        magnetic core versus DC bias magnetic field}
%       \label{figurelabel}
%    \end{figure}
   

% Figure Labels: Use 8 point Times New Roman for Figure labels. Use words rather than symbols or abbreviations when writing Figure axis labels to avoid confusing the reader. As an example, write the quantity Magnetization, or Magnetization, M, not just M. If including units in the label, present them within parentheses. Do not label axes only with units. In the example, write Magnetization (A/m) or Magnetization {A[m(1)]}, not just A/m. Do not label axes with a ratio of quantities and units. For example, write Temperature (K), not Temperature/K.



% \title{\LARGE \bf
% Locomotion Impact of Passive Modular Soft  Microspine Gippers on Motor Tendon Actuated (MTA) Soft Robots (SoRos)}

\title{\LARGE \bf
Improving Grip Stability Using Passive Compliant Microspine Arrays for Soft Robots in Unstructured Terrain}

\author{Lauren Ervin, Harish Bezawada, and Vishesh Vikas$^{1}$% <-this % stops a space
\thanks{*This work was supported in part by NSF {\#1830432}. The material contained in this document is based upon work supported in part by a National Aeronautics and Space Administration (NASA) grant or cooperative agreement. Any opinions, findings, conclusions, or recommendations expressed in this material are those of the authors and do not necessarily reflect the views of NASA. This work was supported through a NASA grant awarded to the Alabama/NASA Space Grant Consortium.}% <-this % stops a space
\thanks{$^{1}$Lauren Ervin, Harish Bezawada, and Vishesh Vikas are with the Agile Robotics Lab, University of Alabama, Tuscaloosa, AL 35487, USA
        {\tt\small \{lefaris, hbezawada\}@crimson.ua.edu, vvikas@ua.edu}}%
}
\newcommand{\reals}{\mathbb{R}}   

\newcommand{\pn}[1]{{#1}}
\newcommand{\nw}[1]{{#1}}
\newcommand{\ak}[1]{{#1}}%% color for changes made by Akhtar
\newcommand{\myac}[1]{#1}
\newcommand{\mynn}[1]{#1}
\newcommand{\mynnd}[1]{#1}
\newcommand{\mynne}[1]{#1}

\newcommand{\by}{\mynn{\delta_{y}}}

\newcommand{\clr}{car-like robot}
\newcommand{\agx}{\overline{x}}
% \newcommand{\reals}{\mathbb{R}}
\newcommand{\incfig}[2][1]{%
	\def\svgwidth{#1\columnwidth}
	\import{./Figures/pdfs/}{#2.pdf_tex}
}
\begin{document}



\maketitle
% \thispagestyle{empty}
% \pagestyle{empty}


\begin{abstract}
Role-Playing Agent (RPA) is an increasingly popular type of LLM Agent that simulates human-like behaviors in a variety of tasks. 
However, evaluating RPAs is challenging due to diverse task requirements and agent designs.
This paper proposes an evidence-based, actionable, and generalizable evaluation design guideline for LLM-based RPA by systematically reviewing $1,676$ papers published between Jan. 2021 and Dec. 2024.
Our analysis identifies six agent attributes, seven task attributes, and seven evaluation metrics from existing literature.
Based on these findings, we present an RPA evaluation design guideline to help researchers develop more systematic and consistent evaluation methods.

\end{abstract}


% to synthesize what agent attributes and task attributes prior literature have considered influence the selection of evaluation metrics, as well as the relationships between these factors.
% For each agent attribute and task category, we summarize its distinct associations with RPLA's evaluation metrics, providing practical guidance on comprehensive based on their RPLA's design. Additionally, we explore the  between agent attributes and downstream tasks to support researchers in refining RPLA design choices.
In recent years, there has been a notable increase in the development and research of tethered UAVs, reflecting a growing interest in their diverse applications. One of the main motivations is to carry out long-term missions with aerial vehicles, as these present significant challenges due to the limitations of current battery solutions \cite{robotics12040117}. A UAV tethered to a UGV is an interesting configuration, as the UGV can power the UAV through the tether for longer times given the higher payload of the former.  %According to this, an interesting configuration to allow long-duration flights of a UAV is a tethered robot configuration in which a UGV is tied to the UAV and powering it. 
This introduces a paradigm in robotic collaboration, offering distinct advantages over traditional standalone systems by combining the strengths of each of the robotic agents \cite{MooreIROS2018}. %As we venture UGV tied to UAV into scenarios requiring heightened enhanced situational awareness involving an extended operational endurance, the tethered approach proves invaluable, due to the capability to provide energy to the UAV, thus increasing fly time \cite{6961531}. 
When deploying a UGV tethered to a UAV in scenarios requiring increased situational awareness and extended operational endurance, the tethered configuration can become even more invaluable, not only providing the UAV with power to significantly extend its flight time \cite{6961531},  %In this way, the cable plays an important role in providing 
but also with safe high-bandwidth communications \cite{850822,9202196}. 

However, the tethering mechanism introduces several challenges, particularly in modeling the hanging tether state \cite{XiaoSSRR2018}. Unlike standalone systems, where each vehicle operates independently, the tether requires intricate and permanent coordination between the UGV and the UAV. Understanding and managing the state of the tether becomes a critical aspect, which requires sophisticated algorithms and real-time processing capabilities \cite{9561062}. 

\begin{figure}
  \includegraphics[width=0.2\textwidth]{Figures/setup1.png}
  \hfill
  \includegraphics[width=0.2\textwidth]{Figures/setup2.png}
  \caption{Simplified 2D sketch showing an example for motion planning of a tethered UAV-UGV with a hanging tether. (Left) Initial robots and tether configuration, and UAV goal (red circle). (Right) Sequence of robots positions and tether length to reach the given goal. Notice how the goal cannot be reached by means of a taut tether, a hanging tether must be considered in this case.}
  \label{fig:planning-setup}
\end{figure}

The state of the tether has traditionally been analyzed through parameterization, an approach that employs equations to represent its physical behavior, especially the catenary curve \cite{BOOKOFCURVES}. Numerous methodologies, with the aim of simplifying this process, approximate the tether as a straight line \cite{autonomousvisual}\cite{framworktether}\cite{uavfire}. This straight-line approximation is only suitable in scenarios where there is a direct line of sight between the tether endpoints, and thus it inherently restricts the exploratory range of the UAV.

In general, hanging-tether approaches allow UAVs to access a broader range of areas compared to straight tether setups; see Fig. \ref{fig:planning-setup} for an example. This concept has been explored by incorporating tether parameterization into localization or planning processes. For instance, Lima and Pereira \cite{9476778} use the catenary equation to determine the UAV's position.  % This concept has been explored by incorporating tether parameterization into the localization or planning processes, such as in the work conducted by Lima and Pereira \cite{9476778}, where using the catenary equation is feasible to find the UAV position. 
Similarly, in \cite{9364354}, the focus is on computing the state of a catenary tether to localize two UAVs attached at each end. This setup is specifically designed to suspend an object, providing a novel approach to object manipulation using UAVs while maintaining a constant tether length. Another interesting application of the catenary model is presented in \cite{LARANJEIRA2020107018} for underwater operations, where the catenary is used to monitor the status of a cable connected to an \emph{N}-number of ROVs (Remotely Operated Vehicles) performing exploration tasks, also with a constant tether length.

In \cite{8848946}, the parameterization of the tether is used in the localization and control stages to perform two autonomous motion primitives, reactive feedback-based position control and model-predictive feedforward velocity control, but is not used in the planning stage. An interesting approach is presented in \cite{drones7020073}, where a tied unmanned aerial vehicle (TUAV), named ``Oxpecke'', was designed for the inspection of stone-mine pillars. This system uses a sweeping (lawnmower) pattern path planning method intended to map and inspect an entire rectangular area, such as the surface of a pillar. However, the surface to inspect is simple (a rectangle), and the tether length is not directly included in the path planning.

%A general approach about the consideration of the tether in the planning stage is introduced in \cite{battocletti2024entanglementdefinitionstetheredrobots}, where the authors present the definition of tether entanglement problems. Specifically, it addresses the challenges posed by the presence of a tether, including the geometric constraints on the robot's motion due to the finite tether length. For that, different constraints are considered in the planning stage. However, the method is too general and mainly tested in ground points, so UAV implementation are not considering, and algo this method allow tether contact with the floor while entanglement desnt exit.

A comprehensive approach to incorporate a tether in the planning stage is presented in \cite{battocletti2024entanglementdefinitionstetheredrobots}, where the authors define the challenges associated with tether entanglement. Specifically, this work addresses the constraints imposed by the tether on the motion of the robot, particularly the limitations arising from the finite length of the tether. Various constraints are integrated into the planning stage to account for these challenges. However, the proposed method is limited and mainly focused on ground applications, 
thus limiting its applicability to UAVs. Additionally, the approach allows for tether contact with the ground, as long as it does not result in entanglement.

On the other hand, \cite{capitán2024efficientstrategypathplanning} focuses on the development of a path planning strategy for marsupial robotic systems composed of a UGV tethered to a UAV. The article introduces a sequential planning strategy called MASPA (Marsupial Sequential Path-Planning Approach), which allows calculating collision-free 3D trajectories for the tethered UAV-UGV system in complex scenarios, for which the UGV advances to a point where the UAV executes the take-off and then advances to a desired point. This method considers both the geometric limitations imposed by obstacles and the cable and the properties of the joint motion of both robots. A novel algorithm, the PVA (Polygonal Visibility Algorithm), is also presented to identify feasible take-off points and solve visibility problems for the UAV in a three-dimensional space. Despite the novelty of the approach, it is not able to consider coordinated planning of the UGV and the UAV at the same time.

In \cite{smartinezr2023}, the catenary approximation is used to parameterize the state of the tether and plan a collision-free trajectory, in which the UAV must achieve objectives using a hanging tether. However, using the catenary equation, the planning process becomes a time-consuming task, allowing only offline computations. %which makes the planning process to be carried out offline.

%This paper will focus on reducing the complexity associated with the calculation of the variable length hanging tether. We will propose an approach that efficiently calculates the tether state with a minimum representation error concerning the real state, and integrating it into a trajectory planning algorithm for a UGV-UAV tethered team. To this end, we test our approach in the motion planning method for a mobile UGV-UAV tethered system presented in \cite{smartinezr2023}, which is based on two stages. The first stage computes a free-collision path planning for UAV, UGV, and tether, using the RRT* algorithm. The second stage corresponds to a trajectory planning method based on nonlinear optimization that considers smoothness, speed, acceleration limitations of the UGV and UAV, and optimizes the tether configuration to maximize the distance from obstacles. %Unfortunately, considering the real catenary curve in the planner could make it computationally demanding, as shown in our previous work \cite{martinez2021optimization}. In it, we manage to design, implement and test in experiments a two-step optimized planner which considers the catenary shape. For this reason, we propose to approximate the shape of the tether as a parabola without affecting the safety of the planning system and making use of its simpler description to speed-up the computation of optimal paths.

%Our approach is based on the motion planning mentioned above due to the robustness of computed trajectories. Thus, we include in the first stage, a decision problem to set the initial tether length, to quickly obtain a collision-free state for the whole system. Furthermore, we propose a new planner-state parameterization and replace the use of the catenary equation with a parabola equation for estimating the shape of the tether. Thus, the main contributions of the article are:
%\begin{itemize}
%
%\item In Planner Stage: Solving the decision problem to find a collision-free parabola curve instead of the traditional catenary curve. This change allows the RRT* (Rapidly-exploring Random Tree) planner to calculate trajectories faster and more efficiently, since it avoids the computational complexity associated with the calculation of the catenary. The parabolic curve simplifies the collision decision process and increases the success rate in three-dimensional environments with obstacles.
%
%\item  In Optimizer stage: This stage introduces a direct parameterization of the tether in the trajectory state function, which includes the parameters of the curve (parabola or catenary) in the system state vector. This allows a more accurate evaluation of geometric constraints (such as distance to obstacles) and reduces the optimization time by up to an order of magnitude compared to previous methods, achieving safer and smoother trajectories for the UAV-UGV system.
%\end{itemize}

This paper focuses on reducing the complexity associated with the calculation of the variable length hanging tether. %The paper builds on the previous work of the authors \cite{smartinezr2023}, extending it with a new approach that efficiently calculates the tether state with a minimum representation error related to the actual state and a new parameterization of the tether curve in the trajectory optimizer for faster computation. Thus, 
The main contributions are listed below.

\begin{itemize}
    \item A new method for efficient computation of a collision-free catenary curve based on the parabola approximation. This paper proposes using the parabola curve to model the hanging tether curve, detailing the full pipeline, including the computation of the final catenary model. This method reduces the execution time of the path planner to great extent, since it avoids the computational complexity associated with the calculation of the catenary model for tether collision detection. This model also increases the feasibility of the trajectory planner approach, reaching an averaged 98\% of feasibility in the validation scenarios. 

    \item A direct parameterization of the tether in the trajectory state definition, which includes the parameters of the curve (parabola or catenary) in the system state vector. This allows a more accurate evaluation of geometric constraints (such as distance to obstacles) and reduces the optimization time \rev{by more than an order of magnitude} compared to previous methods, achieving safer and smoother trajectories for the UAV-UGV system. \rev{Such improvement opens the door to apply the proposed method to real-time local re-planning.}
\end{itemize}

%The experimental results will show how this new parameterization boosts the computation, while the parabola model will clearly improve the feasibility of the method over the catenary. 

%Thus, we include in the first stage, a decision problem to set the initial tether length, to quickly obtain a collision-free state for the whole system. Additionally, we replace the traditional catenary equation with a parabolic approximation to estimate the tether shape more efficiently. In the second stage of nonlinear optimization stage, we further simplify the process by parameterizing the tether instead of relying on the catenary model. This approach not only streamlines the representation of the curve but also facilitates more straightforward and efficient gradient calculations during optimization.

The paper is structured as follows. In Section \ref{sec:overview}, we show the general problem to be solved, whereas Section \ref{sec:approach} formalizes the solutions proposed. Section \ref{sec:path_planning} details the implementation of the solution within the planning stage. In Section \ref{sec:optimization_process}, we describe how curve parameterization is utilized to enhance the optimization process for trajectory computation. The experimental results are discussed in Section \ref{sec:experiments}. Finally, the paper is concluded in Section \ref{sec:conclusions}.

%%%%%%%%%%%%%%%%%%%%%%%%%%%%%%%%%%%%%%%%%%%%%%%%%%%%%%%%%%%%%%%%%%%%%%%%%%%%%%%%%
\section{MICROSPINE ORIENTATION AND SURFACE INTERACTION}
\label{Sec:Microspine}
The effectiveness of engaging with surface asperities is influenced by the array orientation, i.e., the angle at which the microspines are embedded. Preliminary tests are performed to identify the desirable angle of microspines to maximize shear force. Microspines angled at 45$\degree$, 60$\degree$, and 75$\degree$ are embedded in 30A Dragon Skin\textsuperscript{TM} silicone squares. These test blocks, weighing 51g,  are then connected to a fishing scale and pulled horizontally across a rough silicon mat, smooth whiteboard, and carpet square to provide variance in surface friction coefficient. For each surface, the maximum force observed before slipping occurred is recorded, shown in \Fig~\ref{fig:Pin Test}. The $45\degree$ angle is chosen as it displayed the highest resistive force across all the surfaces.

\begin{figure}[h]
    \centering
    \includegraphics[width=0.9\linewidth]{Figures/pin_test.png}
    \caption{The influence of the microspine angles on friction coefficient.}
    \label{fig:Pin Test}
\end{figure}

The microspines engaged into the surface asperities on softer surfaces, carpet and the silicon mat, requiring more pulling force. In contrast, on the rigid smooth surface, the spines do not display the increased interaction force. In fact, the engagement is so limited that the interaction force is lower than the no-spine silicone test square. Hence, this test is also able to show the extreme differences realized by the spine angles while obtaining an optimized angle for the maximum traction force.
%%%%%%%%%%%%%%%%%%%%%%%%%%%%%%%%%%%%%%%%%%%%%%%%%%%%%%%%%%%%%%%%%%%%%%%%%%%%%%%%
%%%%%%%%%%%%%%%%%%%%%%%%%%%%%%%%%%%%%%%%%%%%%%%%%%%%%%%%%%%%%%%%%%%%%%%%%%%%%%%%
\section{MICROSPINE ARRAY AND ROBOT DESIGNS}
There are several design parameters that impact the effectiveness of microspines. The critical ones include (1) adding compliance to individual microspines and  the angle at which the microspines interact with a surface, (2) the array configuration and (3) effective integration with the robot body. Even with an optimal design, the microspine array will not be effective on every surface. The aspects that are out of the hands of the designer, but also highly impact surface engagement, include surface roughness, distribution of asperities, and size of asperities. 


\subsection{Compliant Mechanism Microspine Design}
%\hl{Show CAD microspine design here.}
The single-material mechanism, shown in \Fig \ref{fig:comp}, allows compliance with an exposed joint while simplifying the fabrication process over previous microspine designs. The compliant mechanism is fabricated with an FDM 3D printer and TPU with 95A Shore hardness. Halfway through the additive manufacturing process, the print is paused. The microspine is inserted into a channel left in the middle of the mechanism, highlighted in \Fig \ref{fig:comp}c), and the print is resumed. Once finished, the angle of the bare microspine can be modified for different surface topologies while the body remains secure in the mechanism. The angle of surface interaction, $\alpha$, was fixed at roughly $45^{\circ}$ during testing.



\begin{figure}[h]
    \centering
    \includegraphics[width=\linewidth]{Figures/compliantV2.png}
    \caption{Compliant mechanism design. a) A hinge joint enables passive compliance.  b) Holes embedded on the righthand side of the mechanism allow anchoring into the silicone limb. c) A microspine is inserted in a center channel matching the spine topology set halfway into the mechanism.}
    \label{fig:comp}
\end{figure}


%\subsection{Microspine Interaction Angle}

\subsection{Microspine Array Configuration}
%\hl{Copy the previous mechatronics.png and possibly FBD.png in this section.}
The array configuration ensures multiple microspines remain active on various complex surfaces. We propose a two-row, stacked array configuration consisting of ten microspines with four on the top row and six on the bottom. The microspines on the bottom row are commonly active on more uniform terrain. The top row can become active on steep/highly irregular surfaces without hindering the movement of the bottom row of microspines. The critical parameter when designing the array configuration is ensuring adequate surface interaction and gripping regardless of topology. Crucially, not all microspines need to interact with a surface for the microspine array to be effective, shown in \Fig \ref{fig:eng}. This is a byproduct of the passive nature and built-in redundancy of the system.

\begin{figure}[h]
    \centering
    \includegraphics[width=\linewidth]{Figures/engageV2.png}
    \caption{Two-row, stacked array configuration. a) Close-up of the microspines gripping onto a non-uniform rock. b) This diagram highlights which microspines are interacting with the surface. On this rock, all 6 of the microspines on the bottom row are active. c) Close-up of the microspines gripping onto a steeper rock. d) On this rock, 2 of the bottom row and 3 of the top row microspines are active.}
    \label{fig:eng}
\end{figure}


\subsection{Effective Soft-Compliant Integration Through Anchoring}
%\hl{Show the mold here as well as discuss the robot design.}
The soft-compliant integration reduces design complexity by allowing each microspine to passively move independent of one another with a single actuator controlling the entire array configuration. To achieve this, a mold is created with channels for each microspine compliant mechanism to attach to the tip of a SoRo limb. The SoRo prototype used for experimentation is cast out of DragonSkin\texttrademark~ silicone rubber using a custom mold, shown in \Fig \ref{fig:totalMold}. Therefore, a modified limb mold is used for integrating the microspine array in a consistent, standardized manner. Half of the compliant mechanism contains holes that mechanically anchor it into the silicone limb, highlighted in \Fig \ref{fig:comp}b), preventing it from being freely pulled out of the limb during microspine gripping. This anchoring method is essential for ensuring the microspine does not come loose over time. The remaining, exposed half of the mechanism contains the microspine.

\begin{figure}[h]
    \centering
    \begin{subfigure}{0.4\linewidth}
         \centering
         \includegraphics[width=0.7\textwidth]{Figures/mold.PNG}
         \caption{Mold}
         \label{fig:base}\vspace{-20pt}
     \end{subfigure}
     \begin{subfigure}{0.55\linewidth}
         \centering
         \includegraphics[width=0.7\textwidth]{Figures/mold_ext.png}
         \caption{Modified mold}
         \label{fig:one}\vspace{-20pt}
     \end{subfigure}
     \begin{subfigure}{0.45\linewidth}
         \centering
         \includegraphics[width=0.7\textwidth]{Figures/mold_tip.png}
         \caption{Microspine mold tip}
         \label{fig:two}
     \end{subfigure}
     \begin{subfigure}{0.45\linewidth}
         \centering
         \includegraphics[width=0.7\textwidth]{Figures/mold_spines.png}
         \caption{Integrated microspines}
         \label{fig:two}
     \end{subfigure}
    \caption{Modular ends of the mold enable soft-rigid integration. a) A baseline robot mold. b) A modified mold that allows different configurations of microspine arrays per limb. c) Microspine compatible end mold with holes for a two-row stacked microspine array configuration. d) End mold with integrated microspine mechanisms and ready for casting.}
    \label{fig:totalMold}
\end{figure}



\subsection{Soft Robot Design}
A tetherless, three limb MTA SoRo with on-board power and processing with AprilTags on each limb is used as the experimental prototype. The components of the physical robot are shown in \Fig \ref{fig:mech}. %The topology design optimizes the locomotion and reconfiguration ability \cite{freeman_topology_2023}. Additionally, four of these robots are capable of reconfiguring into a sphere. Morphologically, 
Outward trapezoid cavities are introduced on the underside of each limb to provide optimal stiffness and curling ability. This allows the robot to lift the limb and electronic payload. The use of MTA for body deformation enables reliable and efficient limb actuation. The reader may refer to \cite{freeman_topology_2023} for more details.


\begin{figure}[!h]
    \centering
    \includegraphics[width=0.9\linewidth]{Figures/mechatronicsV2.PNG}
    \caption{The externally powered three-limb SoRo contains soft material limbs and a flexible, central hub that houses DC motors and a custom-designed PCB. The three AprilTags on the limbs help with pose tracking during experiments.}
    \label{fig:mech}
\end{figure}


% \subsection{Three-Limb Motor-Tendon Actuated (MTA) SoRo}
% An externally powered, three-limb MTA SoRo with markers is used as the test robot. The topology design optimizes the locomotion and reconfiguration ability \cite{freeman_topology_2023}. Additionally, four of these robots are capable of reconfiguring into a sphere. Morphologically, outward trapezoid cavities are introduced on the underside of the limb to provide optimal stiffness and curling ability. These design decisions allow the robot to lift the limb and electronic payload. The use of motor tendon actuation (MTA) for body deformation enables reliable and efficient limb actuation. The reader may refer to \cite{freeman_topology_2023} for more details.
% \begin{figure}
%     \centering
%     \includegraphics[width=0.9\linewidth]{Figures/mechatronics.png}
%     \caption{The externally powered three-limb SoRo comprises of soft material limbs and a flexible hub in the center that houses DC motors and a custom-designed PCB. The three markers on the limbs help with real-time pose tracking during experiments.}
%     \label{fig:mech}
% \end{figure}

% A robot is fabricated using a 3D printed cast mold and hub. The plastic mold is comprised of the negatives of the three limbs (referred to as limb chambers), while the central hub is designed to house the three motors (one per limb) and a custom-designed PCB. A Thermoplastic Polyurethane (TPU) hub is printed with shore hardness of 98A to provide flexibility in the center of the robot that houses rigid mechatronics. The fabrication process starts by placing the flexible hub at the center of the mold. Thin tubing is then routed through each of the limb chambers and into a side of the hub to create a tendon path. Equal parts of Dragon Skin\textsuperscript{TM} silicone rubber Part A and Part B are combined and cast into each of the limb chambers. Upon curing, the limbs are removed from the mold and the tendon path is threaded with fishing line. A fishing hook is embedded in the tip of each limb to act as an anchor point. Within the hub, the fishing line is wrapped around a spool attached to the corresponding limb motor. The three motors are secured to the hub base with a set of zipties that prevent the bodies from rotating as the spools along the shafts spin. The custom-designed PCB is placed in the center of the hub with a protective, flexible cap placed on top. The three-limb experiment SoRo is shown in \Fig~\ref{fig:mech}.

% A bench power supply feeds in 15V to the custom-made PCB. This powers the motors directly, and a step-down linear regulator supplies 5V to the Seeeduino Xiao microcontroller. The option of tetherless powering (LiPo battery) is also incorporated where safety circuitry is included to detect and indicate low cell voltage of 3.2V/cell. Three TB6643KQ full-bridge DC motor drivers are used to control three Maxon motors. Digital I/O pins from the microcontroller send PWM signals to actuate each of the motors in a controlled sequence. To ensure a perfect fit in the center of the hub, a triangular PCB with side length of $54mm$ is designed and fabricated. %This circuitry is scalable to allow for testing of SoRos with three, four and five limbs. %The flow of power and data throughout the schematic is shown in Figure~\ref{fig:circ}. 


% % \begin{figure}
% %     \centering
% %     \includegraphics[width=0.9\linewidth]{Figures/circuit.jpg}
% %     \caption{Circuit Design}
% %     \label{fig:circ}
% % \end{figure}

% %\subsection{Spool Design}
% %\hl{Add fig for spool actuation side by side}\\
% %Diameter of working spool: 4mm\\
% %Diameter of old spool: ?
% %Diameter of lone shaft: 1.4mm

% \subsection{Microspine Gripper Placement}

% The integration of the microspine gripper can be performed in a non-modular fashion where they are directly fabricated into the robot limb, or in a modular fashion where they are ``attached'' to the limb. The former has drawbacks relating to fabrication - a new robot would need to be fabricated for each configuration. The modular designs have challenges relating to how much they influence the limb material properties. For example, a `sock' design where the modular array slides onto the end of each limb impacts and elevates the resting position. However, an endcap design, shown in \Fig~\ref{fig:Isometric Endcap View}, is proposed where the array is located at the tip of the limb. The drawback for this design is that the effective length of the limb increases, as shown in \Fig~\ref{fig:Endcap on Bot}, changing the mechanical advantage from the motor. Both modular designs require a new method to secure the tendons to the limb for fast attachment and detachment. The endcap design is pursued in this research due to its efficiency in iteration and testing time. The engagement of the gripper with the surface upon limb actuation is visualized in \Fig~\ref{fig:FBD}.

% \begin{figure}
%      \centering
%      \begin{subfigure}{0.49\linewidth}
%          \centering
%          \includegraphics[width=\textwidth]{Figures/Single Endcap.jpg}
%          \caption{}
%          \label{fig:Isometric Endcap View}
%      \end{subfigure}
%      \begin{subfigure}{0.49\linewidth}
%          \centering
%          \includegraphics[width=\textwidth]{Figures/Endcap on Bot.jpg}
%          \caption{}
%          \label{fig:Endcap on Bot}
%      \end{subfigure}
%      \hspace{0.5cm}

%     \begin{subfigure}{0.49\linewidth}
%          \centering
%          \includegraphics[width=\textwidth]{Figures/Endcap Side View.jpg}
%          \caption{}
%          \label{fig:Endcap Side View}
%      \end{subfigure}
%      \begin{subfigure}{0.49\linewidth}
%          \centering
%          \includegraphics[width=\textwidth]{Figures/Endcap Side View on Bot.jpg}
%          \caption{}
%          \label{fig:Endcap Side View on Bot}
%      \end{subfigure}  
%      \hspace{0.5cm}
     
%      \begin{subfigure}{0.49\linewidth}
%          \centering
%          \includegraphics[width=\textwidth]{Figures/Empty Mold.jpg}
%          \caption{}
%          \label{fig:Empty Mold}
%      \end{subfigure}
%      \begin{subfigure}{0.49\linewidth}
%          \centering
%          \includegraphics[width=\textwidth]{Figures/Assembled_Filled_Mold.jpg}
%          \caption{}
%          \label{fig:Assembled Mold}
%      \end{subfigure}
        
%     \caption{Microspine gripper endcap design: (a) Isometric view  (b) Robot-endcap integrated isometric view  (c) Endcap side (d) Robot-endcap side view. Endcap casting mold: (e) Disassembled mold (f) Assembled mold}
%     \label{fig:Endcap FigureS}
% \end{figure}

% \begin{figure}
%     \centering
%     \includegraphics[width=0.75\linewidth]{Figures/FBD.png}
%     \caption{Visualization of how the microspine array gripper engages with surface asperities upon actuation.}
%     \label{fig:FBD}
% \end{figure}
% \subsection{Endcap Design and Fabrication}

% % \textbf{Paragraph about materials}
% To ensure effective integration and minimum stiffness mismatch, material selection for the endcap is comparable to that of the original robot. TPU 95A inlays are optimized to attach without deforming the original limb with minimal lengthening. TPU is also chosen to allow for quick iterations via 3D printing, while being firm enough not to deform during actuation. Endcaps made of DragonSkin\textsuperscript{TM} 10A and DragonSkin\textsuperscript{TM} 30A are fabricated to determine which best houses the microspine array. In order to keep the robot as homogeneous as possible, the 30A material is chosen to match the limbs. Additionally, the 30A material holds the spines more accurately and reliably when compared to the 10A material which becomes less effective after repetitive testing.

% % \textbf{Describe the endcap design}
% The endcap design consists of three major parts: a TPU inlay, microspines to grip the surface plane, and silicone to lock each part in place. First iterations are done to achieve the best fit for the TPU within the final segment of the limb. Loops are then added to the inlay to give the silicone a structure to wrap around. Finally, ten spines are patterned along the curved path at the end of each limb as seen in \Fig~\ref{fig:Isometric Endcap View}. After the first few attempts, it was quickly realized that the TPU inlay and DragonSkin assembly should maintain ground clearance such that only the spines interact with the surface plane as seen in \Fig~\ref{fig:Endcap FigureS}(c-d).

% % \textbf{Mold design}
% One major challenge in fabrication of the endcaps pertains to the multi-material design which requires several fabrication steps. To simplify the fabrication process, a three part mold is created and 3D printed. The first part acts as a holder for the spines such that a 45$\degree$ angle is maintained during casting. The second part of the mold secures the TPU inlay in place and makes half of the outline for the silicone. The final part solely acts as a dam and must be removable to allow for the insertion and removal of microspines. \Fig~\ref{fig:Endcap FigureS}(e-f) shows the individual parts and assembled mold. 

% %\subsection{Endcap Fabrication}
% First the TPU inlays are printed on an FDM machine. The three separate mold pieces are printed using PLA, allowing for insertion of heat set threaded inserts. After adding the TPU inlay and spines, the mold parts are screwed together. The mold is prepared with Ease Release 200 for quick and easy detachment of the silicone from the mold. Equal parts of DragonSkin\textsuperscript{TM} A and B are measured, vacuumed, combined, and poured into the mold to cure. After the silicone is fully set, the fully constructed endcap is removed and attached to the robot as seen in \Fig~\ref{fig:Endcap on Bot}.
%%%%%%%%%%%%%%%%%%%%%%%%%%%%%%%%%%%%%%%%%%%%%%%%%%%%%%%%%%%%%%%%%%%%%%%%%%%%%%%%
%%%%%%%%%%%%%%%%%%%%%%%%%%%%%%%%%%%%%%%%%%%%%%%%%%%%%%%%%%%%%%%%%%%%%%%%%%%%%%%%
\section{EXPERIMENTATION}

\subsection{Experimental Prototypes}
%\hl{Follow format of Directional.PNG, Inward.PNG, none.png, and TGait.png.}
The baseline experimental prototype with zero microspine limbs, 0ML, is a three-limb MTA SoRo coming from a previous work \cite{freeman2024environmentcentriclearningapproachgait}. We compare the baseline against the addition of two different microspine configurations. The first SoRo equipped with a microspine array, 1ML, has them affixed to one limb that acts as the leader. The last SoRo, 2ML, contains microspine arrays equipped on two limbs with these acting as dual leaders. In all microspine configurations, the microspines face opposite the intended direction of movement. CAD models of the three prototypes are shown in \Fig \ref{fig:SoRos}.

\begin{figure}[h]
    \centering
    \begin{subfigure}{0.32\linewidth}
         \centering
         \includegraphics[width=0.8\textwidth]{Figures/baseline.png}
         \caption{0ML}
         \label{fig:base}
     \end{subfigure}
     \begin{subfigure}{0.32\linewidth}
         \centering
         \includegraphics[width=0.8\textwidth]{Figures/one.png}
         \caption{1ML}
         \label{fig:one}
     \end{subfigure}
     \begin{subfigure}{0.32\linewidth}
         \centering
         \includegraphics[width=0.8\textwidth]{Figures/two.png}
         \caption{2ML}
         \label{fig:two}
     \end{subfigure}
    \caption{The three different microspine configurations explored in the experiments: a) a baseline aqua SoRo; b) a white SoRo with a total of 10 microspines configured in an array on one limb; c) a red SoRo with a total of 20 microspines configured in two arrays on two limbs.}
    \label{fig:SoRos}
\end{figure}


A push-pull translation gait, \Fig~\ref{fig:trans}, is used for the experiments. Here the gait sequence involves actuation of one limb followed by actuation of the two relaxed limbs. It is worth reminding the reader that this gait is not optimal for all four surfaces and three prototypes. In a previous work, translation and rotation gaits were discovered for four limb SoRos using an environment-centric framework \cite{freeman2024environmentcentriclearningapproachgait}. The same methodology was used to generate an optimal translation gait for the baseline three limb SoRo (0ML) on a rubber mat. This gait was used on all surfaces for 0ML, 1ML, and 2ML for a fair comparison, but it is expected that much better performance is possible with an optimized gait found for each prototype and surface combination.

\begin{figure}[h]
    \centering
    \includegraphics[width=0.55\columnwidth]{Figures/TGaitV2.png}
    \caption{Translation gait: one limb (in maroon) actuates while the other two relax, then the previously relaxed limbs actuate while the other relaxes. This is an optimal, 1.1s long gait for the baseline SoRo on a rubber mat \cite{freeman2024environmentcentriclearningapproachgait}.}
    \label{fig:trans}
\end{figure}


\subsection{Experimental Setup}
%\hl{Follow format of setup.JPG.} 
The experiments are carried out with an overhead camera attached to a tripod next to the test area, shown in \Fig \ref{fig:exp}. Field tests are performed on four different surfaces. The starting pose, position and orientation, is same for each SoRo to ensure consistency. Tracking is performed with an AprilTag fixed to the end of each limb serving as fiducial markers. The average displacement per gait as well as overall displacement for each trial run is recorded.

\begin{figure}[h]
    \centering
    \includegraphics[width=0.9\linewidth]{Figures/expV2.PNG}
    \caption{Example of experimental setup for outside field locations in a forest with a dirt patch. The overhead camera is orthogonal to the ground and approximately 4' above the surface. Experiments took place with the three prototypes discussed previously: 0ML, 1ML and 2ML.}
    \label{fig:exp}
\end{figure}


\subsection{Field Experiments}
%\hl{Follow format of yobot\_dir\_0\_black\_trans\_Test803\_imageU.png.}
The field experiments are tested on four rough surfaces around the University of Alabama campus. We look at uniform concrete, partially uniform brick, granular compact sand with pebbles, and a non-uniform forest floor containing leaf litter and large tree roots. %\hl{We show the average number and size of asperities on each surface as this is a critical parameter for the success of microspine gripping.} 
Tests are conducted using the push-pull translation gait shown in \Fig \ref{fig:trans}. Three trials (60 gaits per trial) are performed for a particular configuration and surface with three prototypes and four surfaces, resulting in a total of 36 trials. 

% -- HBezawada draft
% \\ Include why 3 tags were required: Occlusion: lighting or cutt off from camera's vision, replacing battery, accessing internal components

A 36h11 family AprilTag from the AprilTag visual fiducial system \cite{Olson_2011} is attached to each limb where they are readily visible while not hindering the movement. To compensate for occlusion and easy access to the hub, three tags were used to find the robot's position. Each limb of a robot has a different tag attached which represents a number from $0-8$. Before starting the tracking, the threshold parameters for different background environments are obtained. The tags are detected using an AprilTag detector library in Python. The tracking algorithm and data processing code is available at
\href{https://github.com/AgileRoboticsLab/SoftRobotics-Microspines}{github.com/AgileRoboticsLab/SoftRobotics-Microspines}.

% \begin{algorithm} 
%     \caption{Tracking using AprilTags}
%     \label{Alg - Marker Detection}
%     \KwData{Image from the camera}
%     \KwResult{AprilTag positions, traversed statistics}
%     Start the camera\;
%     \While{Camera is running, for each frame}{
%     %Change the image to HSV format\;
%     Image segmentation to detect AprilTags\;
%     Obtain the tags' center positions\;
%     Calculate the centroid of the three detected AprilTags $\rightarrow$ hub location\;
    
%     Calculate the Euclidean distance ($d_i$) traveled from the previous frame\;
%     }
%     Calculate the total distance ($D$)\;
%     Calculate the displacement traveled between starting ($i=0$) and final ($i=n$) positions \;
%     Calculate the rotation ($\theta$) between starting ($i=0$) and final ($i=n$) orientations
% \end{algorithm}
%At time $t$, let the AprilTag position on limb $i=1,2,3$ be defined as $\bm{m}_i(t)\in \Re^{2}$. The centroid position, relative displacement, rotation, and distance traveled are $\bm{c}(t), \bm{d}(t), \theta(t), D(t)$. AprilTag position from the centroid is $\displaystyle \bm{\widetilde{m}}_i=\bm{m}_i -\bm{c}(t)$ where $\bm{c}(t)=\frac{\sum_{i=1}^{3} \bm{m}_i}{3}$. Instantaneous rotation, relative displacement, and rotation are calculated as
% \begin{align}
%     \begin{gathered}
%     d_i(t) = ||\bm{c}(t)-\bm{c}(t-1)||, \qquad
%     % d_i = \sqrt{(x_{i-1}-x_{i})^2+(y_{i-1}-y_{i})^2}, \quad
%     D(t) = \sum_{j=1}^{t} d_j \\
%     \theta(t) = \arccos\left(\frac{\widetilde{\bm{m}}_k(0) \cdot \widetilde{\bm{m}}_k(t)}{{|\widetilde{\bm{m}}_k(0)| \cdot |\widetilde{\bm{m}}_k(t)|}}\right)        \qquad k=1,2,3
%     \end{gathered}
% \end{align}

% The rotation ($\theta$) is calculated based on this vector between start and final frame. %Distance and cumulative distance is calculated as shown in equation \ref{Eq: Euclidean Distance}. Displacement is the distance calculated between the centroid of the first and last frames. 
% Apart from rotation, which results in degrees, all other statistics are obtained in pixels and later scaled to centimeters. %An example of the tracking output with cumulative distance, displacement, and rotation is shown in \Fig~\ref{fig:track}.

% \begin{equation}
% \begin{rcases}
% \begin{aligned} \label{Eq: Euclidean Distance}
%     d_i &= \sqrt{(x_{i-1}-x_{i})^2+(y_{i-1}-y_{i})^2}\\
%     D &= \sum_{i=1}^{n} d_i \\
%     \theta &= \arccos\left(\frac{{V_{0} \cdot V_{n}}}{{|V_{0}| \cdot |V_{n}|}}\right)
% \end{aligned} 
% \qquad \text{\space}
% \end{rcases}
% \end{equation}
% %

% \begin{figure}[h]
%     \centering
%     \includegraphics[width=0.99\linewidth]{Figures/yobot_dir_0_black_trans_Test803_image.png}
%     \caption{\hl{Replace with new tracking}}
%     \label{fig:track}
% \end{figure}



% \subsection{Experimental Setup}
% The experimental setup, shown in \Fig~\ref{fig:test} comprises of an 8'x4' platform with a camera fixed parallel to the surface. A Raspberry Pi 4 equipped with a camera tracks the robot in real time to record the positional data of each limb. The platform is constructed so that smooth whiteboard, rubber black mat, and porous wood modular surfaces are quickly swapped and tested without disturbing the camera's field of view. The reader may be reminded that, unlike in Sec. \ref{Sec:Microspine}, a porous wood surface is used for the experiments in place of carpet. When the microspine endcaps attached to the SoRo grip onto carpet, the increased friction is so high that the motor tendon can snap, making the surface unsuitable for testing.
% \begin{figure}
%     \centering
%     \includegraphics[width=0.99\linewidth]{Figures/setup.JPG}
%     \caption{Experimental setup comprising of a modular platform equipped with camera that performs real-time tracking of the robot limb markers.}
%     \label{fig:test}
% \end{figure}

% \subsection{Microspine Configurations}
% Three microspine configurations are explored for testing - (1) spines facing in the direction of motion, (2) inwards towards the hub, and (3) no spines. The  configurations are illustrated in \Fig~\ref{fig:Spine_Config}.



% A push-pull translation gait, \Fig~\ref{fig:trans}, is used for the experiments. Here the gait sequence involves actuation of one limb followed by actuation of the two relaxed limbs. It is worth reminding the reader that this may not be optimal for all the three surfaces and spine configurations. In a previous work, translation and rotation gaits are discovered for four limb SoRos using an environment-centric framework \cite{mahendran_multi-gait_2023}. The same methodology is used to generate an optimal translation gait for the three limb SoRo with no spine configuration on a rubber mat. %It should be noted that this locomotion gait was found on a mat surface, and may not be optimal for translating on inclines.
% %All tests are repeated for both translation and rotation gaits.

% \begin{figure}
%     \centering    \includegraphics[width=0.75\columnwidth]{Figures/TGait.png}
%     % \begin{subfigure}{0.32\linewidth}
%     %      \centering
%     %      \includegraphics[width=\textwidth]{Figures/1limb.jpg}
%     %      \caption{Left}
%     %      \label{fig:limb1}
%     %  \end{subfigure}
%     %  \begin{subfigure}{0.32\linewidth}
%     %      \centering
%     %      \includegraphics[width=\textwidth]{Figures/2limb.jpg}
%     %      \caption{Top and Right}
%     %      \label{fig:limb2}
%     %  \end{subfigure}
%     \caption{Translation gait that involves actuation of one limb (in maroon) while other two are relaxed, followed by actuation of the previously relaxed limbs and relaxation of the previously actuated limb. This is an optimal gait for SoRo locomotion with no spines on a rubber mat, and has duration of $1.1\sec$.}
%     \label{fig:trans}
% \end{figure}

% %\begin{figure}
% %    \centering
% %    \begin{subfigure}{0.32\linewidth}
% %         \centering
% %         \includegraphics[width=\textwidth]{Figures/top1.jpg}
% %         \caption{Top}
% %         \label{fig:limb1}
% %     \end{subfigure}
% %     \begin{subfigure}{0.32\linewidth}
% %         \centering
% %         \includegraphics[width=\textwidth]{Figures/2left.jpg}
% %         \caption{Left and Top}
% %         \label{fig:limb2}
% %     \end{subfigure}
% %     \begin{subfigure}{0.32\linewidth}
% %         \centering
% %         \includegraphics[width=\textwidth]{Figures/None.jpg}
% %         \caption{None}
% %         \label{fig:blank}
% %     \end{subfigure}
% %    \caption{Rotation Gait}
% %    \label{fig:Spine_Config}
% %\end{figure}


% \subsection{Real-Time Pose Estimation} 
% Three markers of different colors are placed on each limb such that each of the colors are easily distinguished from the robot and background surfaces. Additionally, the markers are fixed so that they do not hinder the movement of limbs while easily being swapped out if necessary. 
% % -- HBezawada draft
% % \\
% The markers are detected using the OpenCV2 package and implemented in python on a Raspberry Pi. Before starting the experiment, the color threshold parameters for different colored markers based on different background surfaces are obtained. The real-time tracking algorithm is implemented as shown in Algorithm \ref{Alg - Marker Detection} and is available at 
% \hyperlink{https://github.com/AgileRoboticsLab/SoftRobotics-Microspines}{github.com/AgileRoboticsLab/SoftRobotics-Microspines}.

% \begin{algorithm} 
%     \caption{Real-Time Tracking}
%     \label{Alg - Marker Detection}
%     \KwData{Image from the camera}
%     \KwResult{Marker positions, traversed statistics}
%     Start the camera\;
%     \While{Camera is running, for each frame}{
%     Change the image to HSV format\;
%     Image segmentation to detect markers\;
%     Obtain the marker's center position\;
%     Calculate the centroid of the three detected markers $\rightarrow$ hub location\;
    
%     Calculate the Euclidean distance ($d_i$) traveled from the previous frame\;
%     }
%     Calculate the total distance ($D$), displacement traveled\;
%     Calculate the rotation ($\theta$) between starting ($i=0$) and final ($i=n$) orientation
% \end{algorithm}
% At time $t$, let the marker position on limb $i=1,2,3$ be defined as $\bm{m}_i(t)\in \Re^{2}$. The centroid position, relative displacement, rotation, and distance traveled are $\bm{c}(t), \bm{d}(t), \theta(t), D(t)$. Marker position from the centroid is $\displaystyle \bm{\widetilde{m}}_i=\bm{m}_i -\bm{c}(t)$ where $\bm{c}(t)=\frac{\sum_{i=1}^{3} \bm{m}_i}{3}$. Instantaneous rotation, relative displacement, and rotation are calculated as
% \begin{align}
%     \begin{gathered}
%     d_i(t) = ||\bm{c}(t)-\bm{c}(t-1)||, \qquad
%     % d_i = \sqrt{(x_{i-1}-x_{i})^2+(y_{i-1}-y_{i})^2}, \quad
%     D(t) = \sum_{j=1}^{t} d_j \\
%     \theta(t) = \arccos\left(\frac{\widetilde{\bm{m}}_k(0) \cdot \widetilde{\bm{m}}_k(t)}{{|\widetilde{\bm{m}}_k(0)| \cdot |\widetilde{\bm{m}}_k(t)|}}\right)        \qquad k=1,2,3
%     \end{gathered}
% \end{align}

% The rotation ($\theta$) is calculated based on this vector between start and final frame. %Distance and cumulative distance is calculated as shown in equation \ref{Eq: Euclidean Distance}. Displacement is the distance calculated between the centroid of the first and last frames. 
% Apart from rotation, which results in degrees, all other statistics are obtained in pixels and later scaled to inches. An example of the real-time tracking output with cumulative distance, displacement, and rotation is shown in \Fig~\ref{fig:track}.

% % \begin{equation}
% % \begin{rcases}
% % \begin{aligned} \label{Eq: Euclidean Distance}
% %     d_i &= \sqrt{(x_{i-1}-x_{i})^2+(y_{i-1}-y_{i})^2}\\
% %     D &= \sum_{i=1}^{n} d_i \\
% %     \theta &= \arccos\left(\frac{{V_{0} \cdot V_{n}}}{{|V_{0}| \cdot |V_{n}|}}\right)
% % \end{aligned} 
% % \qquad \text{\space}
% % \end{rcases}
% % \end{equation}
% % %

% \begin{figure}
%     \centering
%     \includegraphics[width=0.99\linewidth]{Figures/yobot_dir_0_black_trans_Test803_imageU.png}
%     \caption{Real-time tracking where distance, rotation, and displacement are relative to the previous time frame. The cumulative quantities are measured from time $t=0$.}
%     \label{fig:track}
% \end{figure}
%%%%%%%%%%%%%%%%%%%%%%%%%%%%%%%%%%%%%%%%%%%%%%%%%%%%%%%%%%%%%%%%%%%%%%%%%%%%%%%%
\section{RESULTS}
%\subsection{Surface Results}
%\hl{Have more detailed discussion about specific surfaces - consider classifying as uniform, partially uniform, etc.}
The translation results of 0ML, 1ML, and 2ML are compared on 4 surfaces. %The overall trends of planar movement of the three prototypes on the four surfaces are briefly discussed. 
%We remind the reader that for each prototype on a given surface, three trials were performed. %The table below shows the total average displacement and rotation for all three runs of a SoRo on a given surface, with \textbf{bolded} numbers representing the greatest quantity for a given surface and \textit{italicized} numbers representing the lowest value.
%A short discussion is had on the effect of adding compliant microspines onto the tips of this specific MTA SoRo design.  Additionally, we discuss the coupling that occurs between translation and rotation with this given three-limb SoRo design.
\Fig \ref{fig:data} shows the average displacement and standard deviation of the 3 trials per prototype. 

The concrete surface is level with a uniform distribution of asperities. Here, both 1ML and 2ML outperform 0ML in terms of displacement. In fact, 1ML had an average displacement of 39.63cm which is over 30 times as much as the 1.32cm that 0ML traveled. 2ML traveled 14.13cm, nearly 11 times as much as 0ML. 2ML had higher consistency across trials, having a standard deviation of 0.76 whereas 0ML's standard deviation was 0.83.

The partially uniform brick contains gaps in between bricks that can cause microspines to become stuck. This was observed to happen randomly across all trials. However, in every instance where this occurred, the microspine that was caught on 1ML or 2ML was able to wiggle free within a few seconds and continue moving. Even with these obstacles, both 1ML traveling 15cm and 2ML traveling 10.52cm had far greater average displacement than 0ML, roughly 25 times and 18 times more, respectively. This was the only surface where 0ML had the lowest standard deviation, and this is due to the microspine limbs on the other two robots randomly getting stuck when crossing over brick perimeters. Additionally, this surface resulted in the lowest overall movement of 0ML, with an average displacement of 0.59cm. %The displacement was so consistent because the robot had difficulty overcoming the terrain at all.

The granular, compact sand was not entirely level and various pebbles, holes, insects, and small sticks were scattered around the testing area. This surface was difficult to overcome for all three robots as the compact sand was covered in a loose, granular top layer that was easy to become partially submerged in. Due to this, the average displacement is far lower on this surface. The two microspine limbs on 2ML seemed to dig itself deeper, resulting in average displacement of only 0.46cm and the lowest standard deviation. 1ML still outperformed 0ML with over 3 times increased displacement, traveling 2.53cm compared to 0.82cm.


\begin{figure}[!h]
    \centering
    \includegraphics[width=\linewidth]{Figures/micro_dataV3.png}
    \caption{The average displacement data per prototype and surface combination. The repeatability is shown in error bars.}
    \label{fig:data}
\end{figure}


\begin{figure}[!h]
    \centering
    \includegraphics[width=\linewidth]{Figures/del_x_del_yV4.png}
    \caption{Gait analysis. a) Every $\Delta$X and $\Delta$Y position per gait on concrete and brick. b) The average absolute $\Delta$X and $\Delta$Y displacement data per gait on concrete and brick.}
    \label{fig:delta}
\end{figure}

The forest floor surface was completely non-uniform with highly varying terrain. Specifically, the prototypes had to first overcome a large, 4" tall tree root and then move through leaves, acorns, and other tree debris. All SoRos were able to successfully traverse over the cylinder-like tree root due to the conformable nature of the soft limbs, highlighted in \Fig \ref{fig:tree}. However, only the 1ML and 2ML were able to navigate through the thick tree debris after making it over the large tree root; 0ML became stuck at the base of the tree root in each of its three trials. This is exhibited in the average displacement of 13.74cm for 0ML, 23.95cm for 1ML, and 21.32cm for 2ML. The forest floor was critical for testing as it was the most unstructured of the four experimental surfaces and showcases the benefits of the soft limbs paired with the added gripping stability of the microspine array.

For each of the 36 trials, the robots move for 60 gaits, resulting in a total of 2,160 gaits or 720 gaits per robot. Each gait results in relative change in position in the robot coordinate system $\Delta X, \Delta Y$. The locomotion consistency can be analyzed by examining the 180 poses on a given surface per prototype, such as those shown in \Fig \ref{fig:delta}a). The average displacement per gait ($\Delta X,\Delta Y)$  as well as the standard deviation over all gaits per prototype/surface combination is visualized in \Fig \ref{fig:delta}b). We only analyze the uniform/partially uniform surfaces as the gait-to-gait movement is much less consistent due to non-uniformity and randomness for the other two surfaces. On concrete, the average gait displacement per gait of both 1ML $(0.65cm,0.44cm)$ and 2ML $(0.20cm,0.15cm)$ is greater than 0ML $(0.07cm,0.06cm)$. On brick, both 1ML $(0.22cm,0.19cm)$ and 2ML $(0.13cm,0.15cm)$ outperform 0ML $(0.02cm,0.01cm)$. The relative standard deviation (standard deviation/mean) for 1ML and 2ML is lower than that for  0ML, indicating greater grip stability through improved repeatability.

Examples of a single trial of each prototype row on each surface column is visualized in \Fig \ref{fig:surfaces} with additional data in the accompanying video. The starting and end points are green and blue dots, and the path of traversal is a red, gradient line. On all surfaces, 1ML interacts with the environment significantly more than the baseline 0ML, resulting in greater overall movement. On every surface except for compact sand, 2ML outperforms 0ML. On concrete and compact sand, 2ML had the lowest standard deviation and on the forest floor, 1ML had the lowest standard deviation. This indicates the addition of microspine arrays also increasing the consistency and repeatability of planar locomotion with SoRos. 
%Line graphs plot the displacement of each run per SoRo on a given surface, resulting in four total graphs.  Discrepancies of consistency and improved(?) locomotion are shown.




\begin{figure*}[htbp]
    \centering
    \includegraphics[width=0.9\textwidth]{Figures/qual_results2V2.png}
    \caption{Experimental results for 0ML, 1ML, and 2ML on concrete, brick, compact sand, and a forest floor. Rows represent the different surfaces increasing in unstructured nature with the three different prototypes distinguished by columns.}
    \label{fig:surfaces}
\end{figure*}



% \begin{figure*}[h]
%     \centering
%     \begin{subfigure}{0.3\linewidth}
%          \centering
%          \includegraphics[width=\textwidth]{Figures/Sequence 02_crop.png}
%          %\caption{0ML Concrete}
%          \label{fig:base}\vspace{-9pt}
%      \end{subfigure}
%      \begin{subfigure}{0.3\linewidth}
%          \centering
%          \includegraphics[width=\textwidth]{Figures/Sequence 5_crop.png}
%          %\caption{1ML Concrete}
%          \label{fig:one}\vspace{-9pt}
%      \end{subfigure}
%      \begin{subfigure}{0.3\linewidth}
%          \centering
%          \includegraphics[width=\textwidth]{Figures/Sequence 9_crop.png}
%          %\caption{2ML Concrete}
%          \label{fig:two}\vspace{-9pt}
%      \end{subfigure}
%      \begin{subfigure}{0.3\linewidth}
%          \centering
%          \includegraphics[width=\textwidth]{Figures/Sequence 11_crop.png}
%          %\caption{0ML Brick}
%          \label{fig:base}\vspace{-9pt}
%      \end{subfigure}
%      \begin{subfigure}{0.3\linewidth}
%          \centering
%          \includegraphics[width=\textwidth]{Figures/Sequence 20_crop.png}
%          %\caption{1ML Brick}
%          \label{fig:one}\vspace{-9pt}
%      \end{subfigure}
%      \begin{subfigure}{0.3\linewidth}
%          \centering
%          \includegraphics[width=\textwidth]{Figures/Sequence 15_crop.png}
%          %\caption{2ML Brick}
%          \label{fig:two}\vspace{-9pt}
%      \end{subfigure}
%      \begin{subfigure}{0.3\linewidth}
%          \centering
%          \includegraphics[width=\textwidth]{Figures/Sequence 21_crop.png}
%          %\caption{0ML Sand}
%          \label{fig:base}\vspace{-9pt}
%      \end{subfigure}
%      \begin{subfigure}{0.3\linewidth}
%          \centering
%          \includegraphics[width=\textwidth]{Figures/Sequence 26_crop.png}
%          %\caption{1ML Sand}
%          \label{fig:one}\vspace{-9pt}
%      \end{subfigure}
%      \begin{subfigure}{0.3\linewidth}
%          \centering
%          \includegraphics[width=\textwidth]{Figures/Sequence 25_crop.png}
%          %\caption{2ML Sand}
%          \label{fig:two}\vspace{-9pt}
%      \end{subfigure}
%      \begin{subfigure}{0.3\linewidth}
%          \centering
%          \includegraphics[width=\textwidth]{Figures/Sequence 46_crop.png}
%          %\caption{0ML Forest}
%          \label{fig:base}
%      \end{subfigure}
%      \begin{subfigure}{0.3\linewidth}
%          \centering
%          \includegraphics[width=\textwidth]{Figures/Sequence 40b_crop.png}
%          %\caption{1ML Forest}
%          \label{fig:one}
%      \end{subfigure}
%      \begin{subfigure}{0.3\linewidth}
%          \centering
%          \includegraphics[width=\textwidth]{Figures/Sequence 44_crop.png}
%          %\caption{2ML Forest}
%          \label{fig:two}
%      \end{subfigure}
%     \caption{Experimental results for 0ML, 1ML, and 2ML on concrete, brick, compact sand, and a forest floor. Rows represent the different surfaces with the three different prototypes distinguished by columns.}
%     \label{fig:surfaces}
% \end{figure*}

% %\begin{table}
% %\caption{Test}
% \begin{center}
% \begin{tabular}{|c|c|c|c|} 
%  \hline
%  Robot & Terrain & $\Delta$ Displacement & $\Delta$ Rotation \\ %[0.5ex] 
%  \hline\hline
%  0ML & Concrete & \textit{1.32} & \textit{17.98} \\ 
%  \hline
%  1ML & Concrete & \textbf{39.63} & \textbf{131.30} \\
%  \hline
%  2ML & Concrete & 14.13 & 68.70 \\
%  \hline\hline
%  0ML & Brick & \textit{0.59} & \textit{2.32} \\ 
%  \hline
%  1ML & Brick & \textbf{15.00} & \textbf{77.25} \\
%  \hline
%  2ML & Brick & 10.52 & 37.16 \\
%  \hline\hline
%  0ML & Compact Sand & 0.82 & \textit{3.36} \\ 
%  \hline
%  1ML & Compact Sand & \textbf{2.53} & \textbf{49.88} \\
%  \hline
%  2ML & Compact Sand & \textit{0.46} & 4.82 \\
%  \hline\hline
%  0ML & Forest Floor & \textit{13.74} & 69.96 \\ 
%  \hline
%  1ML & Forest Floor & \textbf{23.95} & \textit{34.85} \\
%  \hline
%  2ML & Forest Floor & 21.32 & \textbf{158.77} \\
%  \hline %[1ex] 
% %\caption{Table Results}
% \end{tabular}
% \end{center}
% %\end{table}

% %\begin{center}
% %\begin{tabular}{|c|c|c|c|c|c|c|c|c|c|c|c|c|}
% %\hline
% %\multicolumn{13}{|c|}{Rotation Gait} \\% 
% %\hline
% %\multicolumn{1}{|c|} {Incline} & %
% %\multicolumn{3}{|c|} {Wood} & 
% %\multicolumn{3}{|c|} {White Board} & 
% %\multicolumn{3}{|c|} {Black Mat} & 
% %\multicolumn{3}{|c|} {Carpet} \\ \hline
% %$0^{\circ}$ & & & & \hspace{0.3cm} & \hspace{0.3cm} & \hspace{0.3cm} & \hspace{0.2cm} & %\hspace{0.2cm} & \hspace{0.2cm} & & &\\ \hline
% %$5^{\circ}$ & & & & & & & & & & & &\\ \hline
% %$10^{\circ}$ & & & & & & & & & & & &\\ \hline
% %$15^{\circ}$ & & & & & & & & & & & &\\ \hline
% %$20^{\circ}$ & & & & & & & & & & & &\\ \hline
% %$25^{\circ}$ & & & & & & & & & & & &\\ \hline
% %$30^{\circ}$ & & & & & & & & & & & &\\ \hline
% %\end{tabular}
% %\end{center}
% % \begin{table}
% % \begin{center}
% % \begin{tabular}{|c|c|c|c|c|c|c|c|c|c|c|c|c|}
% % \hline
% % \multicolumn{13}{|c|}{Translation Gait} \\% 
% % \hline
% % \multicolumn{1}{|c|} {Incline} & %
% % \multicolumn{3}{|c|} {Wood} & 
% % \multicolumn{3}{|c|} {White Board} & 
% % \multicolumn{3}{|c|} {Black Mat} & 
% % \multicolumn{3}{|c|} {Carpet}\\\hline
% % $0^{\circ}$ & & & & \hspace{0.3cm} & \hspace{0.3cm} & \hspace{0.3cm} & \hspace{0.2cm} & \hspace{0.2cm} & \hspace{0.2cm} & & &\\\hline
% % $3^{\circ}$ & & & & & & & & & & & &\\\hline
% % \end{tabular}
% % \caption{Average Displacement}
% % \label{table:1}
% % \end{center}
% % \end{table}

% %\pagebreak
% \begin{figure*}[h]
%      \centering
%      \begin{subfigure}{0.32\textwidth}
%          \centering
%          \includegraphics[width=\textwidth]{Figures/rubber_stddev.png}
%          %\caption{Black Mat}
%          \label{fig:black stddev}
%      \end{subfigure}
%      \begin{subfigure}{0.32\textwidth}
%          \centering
%          \includegraphics[width=\textwidth]{Figures/smooth_whiteboard_stddev.png}
%          %\caption{Whiteboard}
%          \label{fig:white stddev}
%      \end{subfigure}
%      \begin{subfigure}{0.32\textwidth}
%          \centering
%          \includegraphics[width=\textwidth]{Figures/porous_wood_stddev.png}
%          %\caption{Wood}
%          \label{fig:wood stddev}
%      \end{subfigure}
%           \begin{subfigure}{0.32\textwidth}
%          \centering
%          \includegraphics[width=\textwidth]{Figures/rubber_disp_rot.png}
%          \caption{Rubber Mat}
%          \label{fig:black disp vs rot}
%      \end{subfigure}
%      \begin{subfigure}{0.32\textwidth}
%          \centering
%          \includegraphics[width=\textwidth]{Figures/whiteboard_disp_rot.png}
%          \caption{Whiteboard}
%          \label{fig:white disp vs rot}
%      \end{subfigure}
%      \begin{subfigure}{0.32\textwidth}
%          \centering
%          \includegraphics[width=\textwidth]{Figures/wood_disp_rot.png}
%          \caption{Wood}
%          \label{fig:wood disp vs rot}
%      \end{subfigure}
%      \begin{subfigure}{0.4\textwidth}
%          \centering
%          \includegraphics[width=\textwidth]{Figures/graph_legend.png}
%      \end{subfigure}
%     \caption{The displacement and rotation results of the translation gait trials ($49.5\sec/\mathrm{trial}$) for the three surfaces and three spine configurations. The increased engagement with the surfaces is observed through coupled increase in translation-rotation quantities.}
%     \label{fig:disp vs rot}
% \end{figure*}
% %%%%%%%%%%%%%%%%%%%%%%%%%%%%%%%%%%%%%%%%%%%%%%%%%%%%%%%%%%%%%%%%%%%%%%%%%%%%%%%%
% The results for the planar locomotion tests on the three experimental surfaces - rubber mat, smooth whiteboard, and porous wood - are plotted as displacement vs. rotation in \Fig \ref{fig:disp vs rot}. For each surface, 20 trials per spine configuration are performed resulting in a total of 180 trials. Each trial comprised of 45 gaits with total duration of $1.1\sec\times45=49.5\sec$ .The reasoning behind the large number of tests is to investigate the consistency of movement of SoRos with and without microspines.

% The baseline SoRos without spine endcaps travel less distance than both spine configurations on all three surfaces. Additionally, there is a large variance in displacement on the smooth surface. The inward microspines increase overall displacement on all three surfaces, and display the least amount of rotation on average on the smooth and porous surfaces. Finally, the SoRos equipped with the directional microspine endcaps have the greatest displacement on the rubber and smooth surfaces.

% On rough surfaces containing asperities, microspines increase overall distance traversed, confirming what is seen in literature. On smooth surfaces where microspines historically have not been utilized, we find that they increase locomotion consistency. By running multiple tests for each set of parameters, we provide a more accurate picture of SoRo locomotion, microspine effectiveness, and the need for repeatability - an under-reported challenge in the field of soft robotics. In all tested scenarios, there is improvement in translation when microspine endcaps are present. For this specific translation gait, there is improved translation performance, but the cumulative rotation has high variance. This can be attributed to the fact that this particular gait is not optimal for all given surfaces. As mentioned earlier, the gait is optimal for a rubber mat with no spines, not for other surfaces and configurations. However, it can be concluded that there is a consistent increase in engagement of the robot with the surfaces by observing the increase in coupled translation-rotation quantities.
\section{Conclusions}

We present an LLM-assisted hierarchical indoor scene synthesis approach to produce customized and diverse scenes. Our approach fully takes advantage of the three-level hierarchical structure, where the LLM generates the descriptions of hierarchical scenes, a hierarchy-aware network infers the fine-grained relative placements, and a divide-and-conquer optimization solves the feasible layout.

Our approach still holds some limitations. First, for the simplicity of optimization, we assume rectangular floors for the generated scene. It is possible to utilize the spatial-aware optimization with simulated annealing algorithms~\cite{Yu2011MakeIH} for irregular floors. Second, LLM sometimes generates infeasible configurations with too many or large objects and we randomly remove some areas or objects to address this, affecting scene quality. Third, since our hierarchical scene representation takes each object as its oriented bounding boxes without geometric details, our generated scenes are not sensitive to object shapes, such as L-shaped sofa. 




\section*{ACKNOWLEDGMENT}

We thank Bek Ervin for fabricating the 0ML prototype.

%%%%%%%%%%%%%%%%%%%%%%%%%%%%%%%%%%%%%%%%%%%%%%%%%%%%%%%%%%%%%%%%%%%%%%%%%%%%%%%%
%\section*{APPENDIX}

%Appendixes should appear before the acknowledgment.

\bibliographystyle{IEEEtran}
%\bibliography{IEEEabrv,MicroSpine1.bib}
\bibliography{IEEEabrv,MicroSpine1_NoLink.bib}
%%%%%%%%%%%%%%%%%%%%%%%%%%%%%%%%%%%%%%%%%%%%%%%%%%%%%%%%%%%%%%%%%%%%%%%%%%%%%%%%
\end{document}
