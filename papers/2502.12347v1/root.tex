%%%%%%%%%%%%%%%%%%%%%%%%%%%%%%%%%%%%%%%%%%%%%%%%%%%%%%%%%%%%%%%%%%%%%%%%%%%%%%%%
%2345678901234567890123456789012345678901234567890123456789012345678901234567890
%        1         2         3         4         5         6         7         8

\documentclass[letterpaper, 10 pt, conference]{ieeeconf}  % Comment this line out if you need a4paper

%\documentclass[a4paper, 10pt, conference]{ieeeconf}      % Use this line for a4 paper

\IEEEoverridecommandlockouts                              % This command is only needed if 
% you want to use the \thanks command

\overrideIEEEmargins                                      % Needed to meet printer requirements.
%\bibliography{sphericon.bib}




%In case you encounter the following error:
%Error 1010 The PDF file may be corrupt (unable to open PDF file) OR
%Error 1000 An error occurred while parsing a contents stream. Unable to analyze the PDF file.
%This is a known problem with pdfLaTeX conversion filter. The file cannot be opened with acrobat reader
%Please use one of the alternatives below to circumvent this error by uncommenting one or the other
%\pdfobjcompresslevel=0
%\pdfminorversion=4

% See the \addtolength command later in the file to balance the column lengths
% on the last page of the document


% Use this sample document as your LaTeX source file to create your document. Save this file as {\bf root.tex}. You have to make sure to use the cls file that came with this distribution. If you use a different style file, you cannot expect to get required margins. Note also that when you are creating your out PDF file, the source file is only part of the equation. {\it Your \TeX\ $\rightarrow$ PDF filter determines the output file size. Even if you make all the specifications to output a letter file in the source - if your filter is set to produce A4, you will only get A4 output. }

% It is impossible to account for all possible situation, one would encounter using \TeX. If you are using multiple \TeX\ files you must make sure that the ``MAIN`` source file is called root.tex - this is particularly important if your conference is using PaperPlaza's built in \TeX\ to PDF conversion tool.

% The equations are an exception to the prescribed specifications of this template. You will need to determine whether or not your equation should be typed using either the Times New Roman or the Symbol font (please no other font). To create multileveled equations, it may be necessary to treat the equation as a graphic and insert it into the text after your paper is styled. Number equations consecutively. Equation numbers, within parentheses, are to position flush right, as in (1), using a right tab stop. To make your equations more compact, you may use the solidus ( / ), the exp function, or appropriate exponents. Italicize Roman symbols for quantities and variables, but not Greek symbols. Use a long dash rather than a hyphen for a minus sign. Punctuate equations with commas or periods when they are part of a sentence, as in

% $$
% \alpha + \beta = \chi \eqno{(1)}
% $$

% Note that the equation is centered using a center tab stop. Be sure that the symbols in your equation have been defined before or immediately following the equation. Use (1), not Eq. (1) or equation (1), except at the beginning of a sentence: Equation (1) is . . .

% Positioning Figures and Tables: Place figures and tables at the top and bottom of columns. Avoid placing them in the middle of columns. Large figures and tables may span across both columns. Figure captions should be below the figures; table heads should appear above the tables. Insert figures and tables after they are cited in the text. Use the abbreviation Fig. 1, even at the beginning of a sentence.

% \begin{table}[h]
% \caption{An Example of a Table}
% \label{table_example}
% \begin{center}
% \begin{tabular}{|c||c|}
% \hline
% One & Two\\
% \hline
% Three & Four\\
% \hline
% \end{tabular}
% \end{center}
% \end{table}


%    \begin{figure}[thpb]
%       \centering
%       \framebox{\parbox{3in}{We suggest that you use a text box to insert a graphic (which is ideally a 300 dpi TIFF or EPS file, with all fonts embedded) because, in an document, this method is somewhat more stable than directly inserting a picture.
% }}
%       %\includegraphics[scale=1.0]{figurefile}
%       \caption{Inductance of oscillation winding on amorphous
%        magnetic core versus DC bias magnetic field}
%       \label{figurelabel}
%    \end{figure}
   

% Figure Labels: Use 8 point Times New Roman for Figure labels. Use words rather than symbols or abbreviations when writing Figure axis labels to avoid confusing the reader. As an example, write the quantity Magnetization, or Magnetization, M, not just M. If including units in the label, present them within parentheses. Do not label axes only with units. In the example, write Magnetization (A/m) or Magnetization {A[m(1)]}, not just A/m. Do not label axes with a ratio of quantities and units. For example, write Temperature (K), not Temperature/K.



% \title{\LARGE \bf
% Locomotion Impact of Passive Modular Soft  Microspine Gippers on Motor Tendon Actuated (MTA) Soft Robots (SoRos)}

\title{\LARGE \bf
Improving Grip Stability Using Passive Compliant Microspine Arrays for Soft Robots in Unstructured Terrain}

\author{Lauren Ervin, Harish Bezawada, and Vishesh Vikas$^{1}$% <-this % stops a space
\thanks{*This work was supported in part by NSF {\#1830432}. The material contained in this document is based upon work supported in part by a National Aeronautics and Space Administration (NASA) grant or cooperative agreement. Any opinions, findings, conclusions, or recommendations expressed in this material are those of the authors and do not necessarily reflect the views of NASA. This work was supported through a NASA grant awarded to the Alabama/NASA Space Grant Consortium.}% <-this % stops a space
\thanks{$^{1}$Lauren Ervin, Harish Bezawada, and Vishesh Vikas are with the Agile Robotics Lab, University of Alabama, Tuscaloosa, AL 35487, USA
        {\tt\small \{lefaris, hbezawada\}@crimson.ua.edu, vvikas@ua.edu}}%
}

% The following packages can be found on http:\\www.ctan.org
\usepackage{graphics} % for pdf, bitmapped graphics files
\usepackage{epsfig} % for postscript graphics files
\usepackage{mathptmx} % assumes new font selection scheme installed
\usepackage{times} % assumes new font selection scheme installed
\usepackage{amsmath} % assumes amsmath package installed
\usepackage{amssymb}  % assumes amsmath package installed
\usepackage{bm,soul,xcolor}
%\usepackage{natbib}
\usepackage{cite}
\usepackage{tikz}
\usetikzlibrary{arrows.meta, automata, positioning, quotes}
\usepackage{comment}
\usepackage[linesnumbered,ruled,vlined]{algorithm2e}
\usepackage{graphicx}
\usepackage{subcaption}
\usepackage{caption}
\usepackage{mathtools}
\usepackage{gensymb}
\usepackage[hidelinks]{hyperref}
%\usepackage{multicol}
\usepackage{stfloats}
%\usepackage{float}
\graphicspath{{Figures/}}
\newcommand{\Fig}{Fig. }
\newcommand{\bigso}{\mathfrak{SO}(3)}
\newcommand{\smallso}{\mathfrak{so}(3)}
\newcommand{\smallse}{\mathfrak{se}(3)}
\newcommand{\bigse}{{SE}(3)}
\newcommand{\twist}{\mathcal{V}}
\newcommand{\sphi}[1]{s_{\phi_{#1}}}
\newcommand{\cphi}[1]{c_{\phi_{#1}}}
\newcommand{\stheta}[1]{s_{\theta_{#1}}}
\newcommand{\ctheta}[1]{c_{\theta_{#1}}}
% \newcommand{\Fr}[#1]{F_{r #1}}
\usepackage{amssymb}
\renewcommand{\Re}{\mathbb{R}}
%\usepackage[showframe]{geometry}

\usepackage[export]{adjustbox}
\begin{document}



\maketitle
% \thispagestyle{empty}
% \pagestyle{empty}


\begin{abstract}
Role-Playing Agent (RPA) is an increasingly popular type of LLM Agent that simulates human-like behaviors in a variety of tasks. 
However, evaluating RPAs is challenging due to diverse task requirements and agent designs.
This paper proposes an evidence-based, actionable, and generalizable evaluation design guideline for LLM-based RPA by systematically reviewing $1,676$ papers published between Jan. 2021 and Dec. 2024.
Our analysis identifies six agent attributes, seven task attributes, and seven evaluation metrics from existing literature.
Based on these findings, we present an RPA evaluation design guideline to help researchers develop more systematic and consistent evaluation methods.

\end{abstract}


% to synthesize what agent attributes and task attributes prior literature have considered influence the selection of evaluation metrics, as well as the relationships between these factors.
% For each agent attribute and task category, we summarize its distinct associations with RPLA's evaluation metrics, providing practical guidance on comprehensive based on their RPLA's design. Additionally, we explore the  between agent attributes and downstream tasks to support researchers in refining RPLA design choices.
People engage in activities in online forums to exchange ideas and express diverse opinions. Such online activities can evolve and escalate into binary-style debates, pitting one person against another~\cite{sridhar_joint_2015}. Previous research has shown the potential benefits of debating in online forums such as enhancing deliberative democracy~\cite{habermas_theory_1984, semaan_designing_2015, baughan_someone_2021} and debaters' critical thinking skills~\cite{walton_dialogue_1989, tanprasert_debate_2024}. For example, people who hold conflicting stances can help each other rethink from a different perspective. However, research has also shown that such debates could result in people attacking each other using aggressive words, leading to depressive emotions~\cite{shuv-ami_new_2022}. Hatred could spread among various groups debating different topics~\cite{iandoli_impact_2021, nasim_investigating_2023, vasconcellos_analyzing_2023, qin_dismantling_2024}, such as politics, sports, and gender.

In recent years, people have integrated Generative AI (GenAI) into various writing tasks, such as summarizing~\cite{august_know_2024}, editing~\cite{li_value_2024}, creative writing~\cite{chakrabarty_help_2022, li_value_2024, yang_ai_2022, yuan_wordcraft_2022}, as well as constructing arguments~\cite{jakesch_co-writing_2023, li_value_2024} and assisting with online discussions~\cite{lin_case_2024}. This raises new concerns in online debates. For example, an internally synthesized algorithm of Large Language Models (LLMs) could produce hallucinations~\cite{fischer_generative_2023, razi_not_2024}, which may act as a catalyst for the spread of misinformation in online forums~\cite{fischer_generative_2023}. In addition, GenAI could introduce biased information to forum members~\cite{razi_not_2024}, which may intensify pre-existing debates. Moreover, integrating GenAI into various writing scenarios may also result in weak insights~\cite{hadan_great_2024}, raising concerns about the impact of GenAI on the ecology of online forum debates.

Given these concerns, this study aims to explore how people use GenAI to engage in debates in online forums. The integration of GenAI is not only reshaping everyday writing practices but also has the potential to redefine the online argument-making paradigm. Previous research has demonstrated the potential of co-writing with GenAI, focusing primarily on its influence on individual writing tasks~\cite{august_know_2024, chakrabarty_help_2022, jakesch_co-writing_2023, li_value_2024, yang_ai_2022}. However, the use of GenAI in the context of online debates, which combine elements of both confrontation and collaboration among remote members, remains underexplored. To explore it, we created an online forum for participants to engage in debates with the assistance of ChatGPT (GPT-4o) (\autoref{fig1}). This study enables us to closely observe how people make arguments and analyze their process data of using GenAI. We will examine three research questions to understand how the use of GenAI shapes debates in online forums: 

\begin{itemize}
\item{\textbf{RQ1}: How do people who participate in a debate on online forums collaborate with GenAI in making arguments?}

\item{\textbf{RQ2}: What patterns of arguments emerge when collaborating with GenAI to participate in a debate on online forums?}

\item{\textbf{RQ3}: How does the use of GenAI for making arguments change when a new member joins an existing debate in online forums?}
\end{itemize}

Given the universality and accessibility of debate topics, we chose one that is widely recognized and able to spark intense debates: soccer, which is regarded as the world's most popular sport~\cite{stolen_physiology_2005}. Building on this topic, we selected "Messi vs. Ronaldo: Who is better?" as the case for our study because it has been an enduring and heated debate among soccer fans. We created a small online forum as the platform for AI-mediated debates, particularly focusing on the debates among members and their interactions with ChatGPT. This approach enables more detailed observation and analysis of the entire process while fostering a nuanced understanding. The study consists of two parts: a one-on-one turn-based debate and a three-person free debate. In the first part, two participants, one supporting Messi and the other Ronaldo, took turns sharing their points of view to challenge each other through forum posts, mirroring the polarized debates that are omnipresent online. In the second part, a new participant joined the ongoing debate, and three participants were allowed to post freely without turn-based restriction, reflecting the spontaneous and unstructured nature of debates on social media. After the two-part study, semi-structured interviews were conducted to explore the participants' experiences. The researchers then applied content analysis and thematic analysis, triangulating the data from forum posts, ChatGPT records, and interview transcripts.

We found that participants prompted ChatGPT for aggressive responses, trying to tailor ChatGPT to fit the debate scenario. While ChatGPT provided participants with statistics and examples, it also led to the creation of similar posts. Furthermore, participants' posts contained logical fallacies such as hasty generalizations, straw man arguments, and ad hominem attacks. Participants reduced the use of ChatGPT to foster better human-human communication when a new member joined an ongoing debate midway. This work highlights the importance of examining how polarized forum members collaborated with GenAI to engage in online debates, aiming to inspire broader implications for socially oriented applications of GenAI.
%%%%%%%%%%%%%%%%%%%%%%%%%%%%%%%%%%%%%%%%%%%%%%%%%%%%%%%%%%%%%%%%%%%%%%%%%%%%%%%%%
\section{MICROSPINE ORIENTATION AND SURFACE INTERACTION}
\label{Sec:Microspine}
The effectiveness of engaging with surface asperities is influenced by the array orientation, i.e., the angle at which the microspines are embedded. Preliminary tests are performed to identify the desirable angle of microspines to maximize shear force. Microspines angled at 45$\degree$, 60$\degree$, and 75$\degree$ are embedded in 30A Dragon Skin\textsuperscript{TM} silicone squares. These test blocks, weighing 51g,  are then connected to a fishing scale and pulled horizontally across a rough silicon mat, smooth whiteboard, and carpet square to provide variance in surface friction coefficient. For each surface, the maximum force observed before slipping occurred is recorded, shown in \Fig~\ref{fig:Pin Test}. The $45\degree$ angle is chosen as it displayed the highest resistive force across all the surfaces.

\begin{figure}[h]
    \centering
    \includegraphics[width=0.9\linewidth]{Figures/pin_test.png}
    \caption{The influence of the microspine angles on friction coefficient.}
    \label{fig:Pin Test}
\end{figure}

The microspines engaged into the surface asperities on softer surfaces, carpet and the silicon mat, requiring more pulling force. In contrast, on the rigid smooth surface, the spines do not display the increased interaction force. In fact, the engagement is so limited that the interaction force is lower than the no-spine silicone test square. Hence, this test is also able to show the extreme differences realized by the spine angles while obtaining an optimized angle for the maximum traction force.
%%%%%%%%%%%%%%%%%%%%%%%%%%%%%%%%%%%%%%%%%%%%%%%%%%%%%%%%%%%%%%%%%%%%%%%%%%%%%%%%
%%%%%%%%%%%%%%%%%%%%%%%%%%%%%%%%%%%%%%%%%%%%%%%%%%%%%%%%%%%%%%%%%%%%%%%%%%%%%%%%
\section{MICROSPINE ARRAY AND ROBOT DESIGNS}
There are several design parameters that impact the effectiveness of microspines. The critical ones include (1) adding compliance to individual microspines and  the angle at which the microspines interact with a surface, (2) the array configuration and (3) effective integration with the robot body. Even with an optimal design, the microspine array will not be effective on every surface. The aspects that are out of the hands of the designer, but also highly impact surface engagement, include surface roughness, distribution of asperities, and size of asperities. 


\subsection{Compliant Mechanism Microspine Design}
%\hl{Show CAD microspine design here.}
The single-material mechanism, shown in \Fig \ref{fig:comp}, allows compliance with an exposed joint while simplifying the fabrication process over previous microspine designs. The compliant mechanism is fabricated with an FDM 3D printer and TPU with 95A Shore hardness. Halfway through the additive manufacturing process, the print is paused. The microspine is inserted into a channel left in the middle of the mechanism, highlighted in \Fig \ref{fig:comp}c), and the print is resumed. Once finished, the angle of the bare microspine can be modified for different surface topologies while the body remains secure in the mechanism. The angle of surface interaction, $\alpha$, was fixed at roughly $45^{\circ}$ during testing.



\begin{figure}[h]
    \centering
    \includegraphics[width=\linewidth]{Figures/compliantV2.png}
    \caption{Compliant mechanism design. a) A hinge joint enables passive compliance.  b) Holes embedded on the righthand side of the mechanism allow anchoring into the silicone limb. c) A microspine is inserted in a center channel matching the spine topology set halfway into the mechanism.}
    \label{fig:comp}
\end{figure}


%\subsection{Microspine Interaction Angle}

\subsection{Microspine Array Configuration}
%\hl{Copy the previous mechatronics.png and possibly FBD.png in this section.}
The array configuration ensures multiple microspines remain active on various complex surfaces. We propose a two-row, stacked array configuration consisting of ten microspines with four on the top row and six on the bottom. The microspines on the bottom row are commonly active on more uniform terrain. The top row can become active on steep/highly irregular surfaces without hindering the movement of the bottom row of microspines. The critical parameter when designing the array configuration is ensuring adequate surface interaction and gripping regardless of topology. Crucially, not all microspines need to interact with a surface for the microspine array to be effective, shown in \Fig \ref{fig:eng}. This is a byproduct of the passive nature and built-in redundancy of the system.

\begin{figure}[h]
    \centering
    \includegraphics[width=\linewidth]{Figures/engageV2.png}
    \caption{Two-row, stacked array configuration. a) Close-up of the microspines gripping onto a non-uniform rock. b) This diagram highlights which microspines are interacting with the surface. On this rock, all 6 of the microspines on the bottom row are active. c) Close-up of the microspines gripping onto a steeper rock. d) On this rock, 2 of the bottom row and 3 of the top row microspines are active.}
    \label{fig:eng}
\end{figure}


\subsection{Effective Soft-Compliant Integration Through Anchoring}
%\hl{Show the mold here as well as discuss the robot design.}
The soft-compliant integration reduces design complexity by allowing each microspine to passively move independent of one another with a single actuator controlling the entire array configuration. To achieve this, a mold is created with channels for each microspine compliant mechanism to attach to the tip of a SoRo limb. The SoRo prototype used for experimentation is cast out of DragonSkin\texttrademark~ silicone rubber using a custom mold, shown in \Fig \ref{fig:totalMold}. Therefore, a modified limb mold is used for integrating the microspine array in a consistent, standardized manner. Half of the compliant mechanism contains holes that mechanically anchor it into the silicone limb, highlighted in \Fig \ref{fig:comp}b), preventing it from being freely pulled out of the limb during microspine gripping. This anchoring method is essential for ensuring the microspine does not come loose over time. The remaining, exposed half of the mechanism contains the microspine.

\begin{figure}[h]
    \centering
    \begin{subfigure}{0.4\linewidth}
         \centering
         \includegraphics[width=0.7\textwidth]{Figures/mold.PNG}
         \caption{Mold}
         \label{fig:base}\vspace{-20pt}
     \end{subfigure}
     \begin{subfigure}{0.55\linewidth}
         \centering
         \includegraphics[width=0.7\textwidth]{Figures/mold_ext.png}
         \caption{Modified mold}
         \label{fig:one}\vspace{-20pt}
     \end{subfigure}
     \begin{subfigure}{0.45\linewidth}
         \centering
         \includegraphics[width=0.7\textwidth]{Figures/mold_tip.png}
         \caption{Microspine mold tip}
         \label{fig:two}
     \end{subfigure}
     \begin{subfigure}{0.45\linewidth}
         \centering
         \includegraphics[width=0.7\textwidth]{Figures/mold_spines.png}
         \caption{Integrated microspines}
         \label{fig:two}
     \end{subfigure}
    \caption{Modular ends of the mold enable soft-rigid integration. a) A baseline robot mold. b) A modified mold that allows different configurations of microspine arrays per limb. c) Microspine compatible end mold with holes for a two-row stacked microspine array configuration. d) End mold with integrated microspine mechanisms and ready for casting.}
    \label{fig:totalMold}
\end{figure}



\subsection{Soft Robot Design}
A tetherless, three limb MTA SoRo with on-board power and processing with AprilTags on each limb is used as the experimental prototype. The components of the physical robot are shown in \Fig \ref{fig:mech}. %The topology design optimizes the locomotion and reconfiguration ability \cite{freeman_topology_2023}. Additionally, four of these robots are capable of reconfiguring into a sphere. Morphologically, 
Outward trapezoid cavities are introduced on the underside of each limb to provide optimal stiffness and curling ability. This allows the robot to lift the limb and electronic payload. The use of MTA for body deformation enables reliable and efficient limb actuation. The reader may refer to \cite{freeman_topology_2023} for more details.


\begin{figure}[!h]
    \centering
    \includegraphics[width=0.9\linewidth]{Figures/mechatronicsV2.PNG}
    \caption{The externally powered three-limb SoRo contains soft material limbs and a flexible, central hub that houses DC motors and a custom-designed PCB. The three AprilTags on the limbs help with pose tracking during experiments.}
    \label{fig:mech}
\end{figure}


% \subsection{Three-Limb Motor-Tendon Actuated (MTA) SoRo}
% An externally powered, three-limb MTA SoRo with markers is used as the test robot. The topology design optimizes the locomotion and reconfiguration ability \cite{freeman_topology_2023}. Additionally, four of these robots are capable of reconfiguring into a sphere. Morphologically, outward trapezoid cavities are introduced on the underside of the limb to provide optimal stiffness and curling ability. These design decisions allow the robot to lift the limb and electronic payload. The use of motor tendon actuation (MTA) for body deformation enables reliable and efficient limb actuation. The reader may refer to \cite{freeman_topology_2023} for more details.
% \begin{figure}
%     \centering
%     \includegraphics[width=0.9\linewidth]{Figures/mechatronics.png}
%     \caption{The externally powered three-limb SoRo comprises of soft material limbs and a flexible hub in the center that houses DC motors and a custom-designed PCB. The three markers on the limbs help with real-time pose tracking during experiments.}
%     \label{fig:mech}
% \end{figure}

% A robot is fabricated using a 3D printed cast mold and hub. The plastic mold is comprised of the negatives of the three limbs (referred to as limb chambers), while the central hub is designed to house the three motors (one per limb) and a custom-designed PCB. A Thermoplastic Polyurethane (TPU) hub is printed with shore hardness of 98A to provide flexibility in the center of the robot that houses rigid mechatronics. The fabrication process starts by placing the flexible hub at the center of the mold. Thin tubing is then routed through each of the limb chambers and into a side of the hub to create a tendon path. Equal parts of Dragon Skin\textsuperscript{TM} silicone rubber Part A and Part B are combined and cast into each of the limb chambers. Upon curing, the limbs are removed from the mold and the tendon path is threaded with fishing line. A fishing hook is embedded in the tip of each limb to act as an anchor point. Within the hub, the fishing line is wrapped around a spool attached to the corresponding limb motor. The three motors are secured to the hub base with a set of zipties that prevent the bodies from rotating as the spools along the shafts spin. The custom-designed PCB is placed in the center of the hub with a protective, flexible cap placed on top. The three-limb experiment SoRo is shown in \Fig~\ref{fig:mech}.

% A bench power supply feeds in 15V to the custom-made PCB. This powers the motors directly, and a step-down linear regulator supplies 5V to the Seeeduino Xiao microcontroller. The option of tetherless powering (LiPo battery) is also incorporated where safety circuitry is included to detect and indicate low cell voltage of 3.2V/cell. Three TB6643KQ full-bridge DC motor drivers are used to control three Maxon motors. Digital I/O pins from the microcontroller send PWM signals to actuate each of the motors in a controlled sequence. To ensure a perfect fit in the center of the hub, a triangular PCB with side length of $54mm$ is designed and fabricated. %This circuitry is scalable to allow for testing of SoRos with three, four and five limbs. %The flow of power and data throughout the schematic is shown in Figure~\ref{fig:circ}. 


% % \begin{figure}
% %     \centering
% %     \includegraphics[width=0.9\linewidth]{Figures/circuit.jpg}
% %     \caption{Circuit Design}
% %     \label{fig:circ}
% % \end{figure}

% %\subsection{Spool Design}
% %\hl{Add fig for spool actuation side by side}\\
% %Diameter of working spool: 4mm\\
% %Diameter of old spool: ?
% %Diameter of lone shaft: 1.4mm

% \subsection{Microspine Gripper Placement}

% The integration of the microspine gripper can be performed in a non-modular fashion where they are directly fabricated into the robot limb, or in a modular fashion where they are ``attached'' to the limb. The former has drawbacks relating to fabrication - a new robot would need to be fabricated for each configuration. The modular designs have challenges relating to how much they influence the limb material properties. For example, a `sock' design where the modular array slides onto the end of each limb impacts and elevates the resting position. However, an endcap design, shown in \Fig~\ref{fig:Isometric Endcap View}, is proposed where the array is located at the tip of the limb. The drawback for this design is that the effective length of the limb increases, as shown in \Fig~\ref{fig:Endcap on Bot}, changing the mechanical advantage from the motor. Both modular designs require a new method to secure the tendons to the limb for fast attachment and detachment. The endcap design is pursued in this research due to its efficiency in iteration and testing time. The engagement of the gripper with the surface upon limb actuation is visualized in \Fig~\ref{fig:FBD}.

% \begin{figure}
%      \centering
%      \begin{subfigure}{0.49\linewidth}
%          \centering
%          \includegraphics[width=\textwidth]{Figures/Single Endcap.jpg}
%          \caption{}
%          \label{fig:Isometric Endcap View}
%      \end{subfigure}
%      \begin{subfigure}{0.49\linewidth}
%          \centering
%          \includegraphics[width=\textwidth]{Figures/Endcap on Bot.jpg}
%          \caption{}
%          \label{fig:Endcap on Bot}
%      \end{subfigure}
%      \hspace{0.5cm}

%     \begin{subfigure}{0.49\linewidth}
%          \centering
%          \includegraphics[width=\textwidth]{Figures/Endcap Side View.jpg}
%          \caption{}
%          \label{fig:Endcap Side View}
%      \end{subfigure}
%      \begin{subfigure}{0.49\linewidth}
%          \centering
%          \includegraphics[width=\textwidth]{Figures/Endcap Side View on Bot.jpg}
%          \caption{}
%          \label{fig:Endcap Side View on Bot}
%      \end{subfigure}  
%      \hspace{0.5cm}
     
%      \begin{subfigure}{0.49\linewidth}
%          \centering
%          \includegraphics[width=\textwidth]{Figures/Empty Mold.jpg}
%          \caption{}
%          \label{fig:Empty Mold}
%      \end{subfigure}
%      \begin{subfigure}{0.49\linewidth}
%          \centering
%          \includegraphics[width=\textwidth]{Figures/Assembled_Filled_Mold.jpg}
%          \caption{}
%          \label{fig:Assembled Mold}
%      \end{subfigure}
        
%     \caption{Microspine gripper endcap design: (a) Isometric view  (b) Robot-endcap integrated isometric view  (c) Endcap side (d) Robot-endcap side view. Endcap casting mold: (e) Disassembled mold (f) Assembled mold}
%     \label{fig:Endcap FigureS}
% \end{figure}

% \begin{figure}
%     \centering
%     \includegraphics[width=0.75\linewidth]{Figures/FBD.png}
%     \caption{Visualization of how the microspine array gripper engages with surface asperities upon actuation.}
%     \label{fig:FBD}
% \end{figure}
% \subsection{Endcap Design and Fabrication}

% % \textbf{Paragraph about materials}
% To ensure effective integration and minimum stiffness mismatch, material selection for the endcap is comparable to that of the original robot. TPU 95A inlays are optimized to attach without deforming the original limb with minimal lengthening. TPU is also chosen to allow for quick iterations via 3D printing, while being firm enough not to deform during actuation. Endcaps made of DragonSkin\textsuperscript{TM} 10A and DragonSkin\textsuperscript{TM} 30A are fabricated to determine which best houses the microspine array. In order to keep the robot as homogeneous as possible, the 30A material is chosen to match the limbs. Additionally, the 30A material holds the spines more accurately and reliably when compared to the 10A material which becomes less effective after repetitive testing.

% % \textbf{Describe the endcap design}
% The endcap design consists of three major parts: a TPU inlay, microspines to grip the surface plane, and silicone to lock each part in place. First iterations are done to achieve the best fit for the TPU within the final segment of the limb. Loops are then added to the inlay to give the silicone a structure to wrap around. Finally, ten spines are patterned along the curved path at the end of each limb as seen in \Fig~\ref{fig:Isometric Endcap View}. After the first few attempts, it was quickly realized that the TPU inlay and DragonSkin assembly should maintain ground clearance such that only the spines interact with the surface plane as seen in \Fig~\ref{fig:Endcap FigureS}(c-d).

% % \textbf{Mold design}
% One major challenge in fabrication of the endcaps pertains to the multi-material design which requires several fabrication steps. To simplify the fabrication process, a three part mold is created and 3D printed. The first part acts as a holder for the spines such that a 45$\degree$ angle is maintained during casting. The second part of the mold secures the TPU inlay in place and makes half of the outline for the silicone. The final part solely acts as a dam and must be removable to allow for the insertion and removal of microspines. \Fig~\ref{fig:Endcap FigureS}(e-f) shows the individual parts and assembled mold. 

% %\subsection{Endcap Fabrication}
% First the TPU inlays are printed on an FDM machine. The three separate mold pieces are printed using PLA, allowing for insertion of heat set threaded inserts. After adding the TPU inlay and spines, the mold parts are screwed together. The mold is prepared with Ease Release 200 for quick and easy detachment of the silicone from the mold. Equal parts of DragonSkin\textsuperscript{TM} A and B are measured, vacuumed, combined, and poured into the mold to cure. After the silicone is fully set, the fully constructed endcap is removed and attached to the robot as seen in \Fig~\ref{fig:Endcap on Bot}.
%%%%%%%%%%%%%%%%%%%%%%%%%%%%%%%%%%%%%%%%%%%%%%%%%%%%%%%%%%%%%%%%%%%%%%%%%%%%%%%%
%%%%%%%%%%%%%%%%%%%%%%%%%%%%%%%%%%%%%%%%%%%%%%%%%%%%%%%%%%%%%%%%%%%%%%%%%%%%%%%%
\section{EXPERIMENTATION}

\subsection{Experimental Prototypes}
%\hl{Follow format of Directional.PNG, Inward.PNG, none.png, and TGait.png.}
The baseline experimental prototype with zero microspine limbs, 0ML, is a three-limb MTA SoRo coming from a previous work \cite{freeman2024environmentcentriclearningapproachgait}. We compare the baseline against the addition of two different microspine configurations. The first SoRo equipped with a microspine array, 1ML, has them affixed to one limb that acts as the leader. The last SoRo, 2ML, contains microspine arrays equipped on two limbs with these acting as dual leaders. In all microspine configurations, the microspines face opposite the intended direction of movement. CAD models of the three prototypes are shown in \Fig \ref{fig:SoRos}.

\begin{figure}[h]
    \centering
    \begin{subfigure}{0.32\linewidth}
         \centering
         \includegraphics[width=0.8\textwidth]{Figures/baseline.png}
         \caption{0ML}
         \label{fig:base}
     \end{subfigure}
     \begin{subfigure}{0.32\linewidth}
         \centering
         \includegraphics[width=0.8\textwidth]{Figures/one.png}
         \caption{1ML}
         \label{fig:one}
     \end{subfigure}
     \begin{subfigure}{0.32\linewidth}
         \centering
         \includegraphics[width=0.8\textwidth]{Figures/two.png}
         \caption{2ML}
         \label{fig:two}
     \end{subfigure}
    \caption{The three different microspine configurations explored in the experiments: a) a baseline aqua SoRo; b) a white SoRo with a total of 10 microspines configured in an array on one limb; c) a red SoRo with a total of 20 microspines configured in two arrays on two limbs.}
    \label{fig:SoRos}
\end{figure}


A push-pull translation gait, \Fig~\ref{fig:trans}, is used for the experiments. Here the gait sequence involves actuation of one limb followed by actuation of the two relaxed limbs. It is worth reminding the reader that this gait is not optimal for all four surfaces and three prototypes. In a previous work, translation and rotation gaits were discovered for four limb SoRos using an environment-centric framework \cite{freeman2024environmentcentriclearningapproachgait}. The same methodology was used to generate an optimal translation gait for the baseline three limb SoRo (0ML) on a rubber mat. This gait was used on all surfaces for 0ML, 1ML, and 2ML for a fair comparison, but it is expected that much better performance is possible with an optimized gait found for each prototype and surface combination.

\begin{figure}[h]
    \centering
    \includegraphics[width=0.55\columnwidth]{Figures/TGaitV2.png}
    \caption{Translation gait: one limb (in maroon) actuates while the other two relax, then the previously relaxed limbs actuate while the other relaxes. This is an optimal, 1.1s long gait for the baseline SoRo on a rubber mat \cite{freeman2024environmentcentriclearningapproachgait}.}
    \label{fig:trans}
\end{figure}


\subsection{Experimental Setup}
%\hl{Follow format of setup.JPG.} 
The experiments are carried out with an overhead camera attached to a tripod next to the test area, shown in \Fig \ref{fig:exp}. Field tests are performed on four different surfaces. The starting pose, position and orientation, is same for each SoRo to ensure consistency. Tracking is performed with an AprilTag fixed to the end of each limb serving as fiducial markers. The average displacement per gait as well as overall displacement for each trial run is recorded.

\begin{figure}[h]
    \centering
    \includegraphics[width=0.9\linewidth]{Figures/expV2.PNG}
    \caption{Example of experimental setup for outside field locations in a forest with a dirt patch. The overhead camera is orthogonal to the ground and approximately 4' above the surface. Experiments took place with the three prototypes discussed previously: 0ML, 1ML and 2ML.}
    \label{fig:exp}
\end{figure}


\subsection{Field Experiments}
%\hl{Follow format of yobot\_dir\_0\_black\_trans\_Test803\_imageU.png.}
The field experiments are tested on four rough surfaces around the University of Alabama campus. We look at uniform concrete, partially uniform brick, granular compact sand with pebbles, and a non-uniform forest floor containing leaf litter and large tree roots. %\hl{We show the average number and size of asperities on each surface as this is a critical parameter for the success of microspine gripping.} 
Tests are conducted using the push-pull translation gait shown in \Fig \ref{fig:trans}. Three trials (60 gaits per trial) are performed for a particular configuration and surface with three prototypes and four surfaces, resulting in a total of 36 trials. 

% -- HBezawada draft
% \\ Include why 3 tags were required: Occlusion: lighting or cutt off from camera's vision, replacing battery, accessing internal components

A 36h11 family AprilTag from the AprilTag visual fiducial system \cite{Olson_2011} is attached to each limb where they are readily visible while not hindering the movement. To compensate for occlusion and easy access to the hub, three tags were used to find the robot's position. Each limb of a robot has a different tag attached which represents a number from $0-8$. Before starting the tracking, the threshold parameters for different background environments are obtained. The tags are detected using an AprilTag detector library in Python. The tracking algorithm and data processing code is available at
\href{https://github.com/AgileRoboticsLab/SoftRobotics-Microspines}{github.com/AgileRoboticsLab/SoftRobotics-Microspines}.

% \begin{algorithm} 
%     \caption{Tracking using AprilTags}
%     \label{Alg - Marker Detection}
%     \KwData{Image from the camera}
%     \KwResult{AprilTag positions, traversed statistics}
%     Start the camera\;
%     \While{Camera is running, for each frame}{
%     %Change the image to HSV format\;
%     Image segmentation to detect AprilTags\;
%     Obtain the tags' center positions\;
%     Calculate the centroid of the three detected AprilTags $\rightarrow$ hub location\;
    
%     Calculate the Euclidean distance ($d_i$) traveled from the previous frame\;
%     }
%     Calculate the total distance ($D$)\;
%     Calculate the displacement traveled between starting ($i=0$) and final ($i=n$) positions \;
%     Calculate the rotation ($\theta$) between starting ($i=0$) and final ($i=n$) orientations
% \end{algorithm}
%At time $t$, let the AprilTag position on limb $i=1,2,3$ be defined as $\bm{m}_i(t)\in \Re^{2}$. The centroid position, relative displacement, rotation, and distance traveled are $\bm{c}(t), \bm{d}(t), \theta(t), D(t)$. AprilTag position from the centroid is $\displaystyle \bm{\widetilde{m}}_i=\bm{m}_i -\bm{c}(t)$ where $\bm{c}(t)=\frac{\sum_{i=1}^{3} \bm{m}_i}{3}$. Instantaneous rotation, relative displacement, and rotation are calculated as
% \begin{align}
%     \begin{gathered}
%     d_i(t) = ||\bm{c}(t)-\bm{c}(t-1)||, \qquad
%     % d_i = \sqrt{(x_{i-1}-x_{i})^2+(y_{i-1}-y_{i})^2}, \quad
%     D(t) = \sum_{j=1}^{t} d_j \\
%     \theta(t) = \arccos\left(\frac{\widetilde{\bm{m}}_k(0) \cdot \widetilde{\bm{m}}_k(t)}{{|\widetilde{\bm{m}}_k(0)| \cdot |\widetilde{\bm{m}}_k(t)|}}\right)        \qquad k=1,2,3
%     \end{gathered}
% \end{align}

% The rotation ($\theta$) is calculated based on this vector between start and final frame. %Distance and cumulative distance is calculated as shown in equation \ref{Eq: Euclidean Distance}. Displacement is the distance calculated between the centroid of the first and last frames. 
% Apart from rotation, which results in degrees, all other statistics are obtained in pixels and later scaled to centimeters. %An example of the tracking output with cumulative distance, displacement, and rotation is shown in \Fig~\ref{fig:track}.

% \begin{equation}
% \begin{rcases}
% \begin{aligned} \label{Eq: Euclidean Distance}
%     d_i &= \sqrt{(x_{i-1}-x_{i})^2+(y_{i-1}-y_{i})^2}\\
%     D &= \sum_{i=1}^{n} d_i \\
%     \theta &= \arccos\left(\frac{{V_{0} \cdot V_{n}}}{{|V_{0}| \cdot |V_{n}|}}\right)
% \end{aligned} 
% \qquad \text{\space}
% \end{rcases}
% \end{equation}
% %

% \begin{figure}[h]
%     \centering
%     \includegraphics[width=0.99\linewidth]{Figures/yobot_dir_0_black_trans_Test803_image.png}
%     \caption{\hl{Replace with new tracking}}
%     \label{fig:track}
% \end{figure}



% \subsection{Experimental Setup}
% The experimental setup, shown in \Fig~\ref{fig:test} comprises of an 8'x4' platform with a camera fixed parallel to the surface. A Raspberry Pi 4 equipped with a camera tracks the robot in real time to record the positional data of each limb. The platform is constructed so that smooth whiteboard, rubber black mat, and porous wood modular surfaces are quickly swapped and tested without disturbing the camera's field of view. The reader may be reminded that, unlike in Sec. \ref{Sec:Microspine}, a porous wood surface is used for the experiments in place of carpet. When the microspine endcaps attached to the SoRo grip onto carpet, the increased friction is so high that the motor tendon can snap, making the surface unsuitable for testing.
% \begin{figure}
%     \centering
%     \includegraphics[width=0.99\linewidth]{Figures/setup.JPG}
%     \caption{Experimental setup comprising of a modular platform equipped with camera that performs real-time tracking of the robot limb markers.}
%     \label{fig:test}
% \end{figure}

% \subsection{Microspine Configurations}
% Three microspine configurations are explored for testing - (1) spines facing in the direction of motion, (2) inwards towards the hub, and (3) no spines. The  configurations are illustrated in \Fig~\ref{fig:Spine_Config}.



% A push-pull translation gait, \Fig~\ref{fig:trans}, is used for the experiments. Here the gait sequence involves actuation of one limb followed by actuation of the two relaxed limbs. It is worth reminding the reader that this may not be optimal for all the three surfaces and spine configurations. In a previous work, translation and rotation gaits are discovered for four limb SoRos using an environment-centric framework \cite{mahendran_multi-gait_2023}. The same methodology is used to generate an optimal translation gait for the three limb SoRo with no spine configuration on a rubber mat. %It should be noted that this locomotion gait was found on a mat surface, and may not be optimal for translating on inclines.
% %All tests are repeated for both translation and rotation gaits.

% \begin{figure}
%     \centering    \includegraphics[width=0.75\columnwidth]{Figures/TGait.png}
%     % \begin{subfigure}{0.32\linewidth}
%     %      \centering
%     %      \includegraphics[width=\textwidth]{Figures/1limb.jpg}
%     %      \caption{Left}
%     %      \label{fig:limb1}
%     %  \end{subfigure}
%     %  \begin{subfigure}{0.32\linewidth}
%     %      \centering
%     %      \includegraphics[width=\textwidth]{Figures/2limb.jpg}
%     %      \caption{Top and Right}
%     %      \label{fig:limb2}
%     %  \end{subfigure}
%     \caption{Translation gait that involves actuation of one limb (in maroon) while other two are relaxed, followed by actuation of the previously relaxed limbs and relaxation of the previously actuated limb. This is an optimal gait for SoRo locomotion with no spines on a rubber mat, and has duration of $1.1\sec$.}
%     \label{fig:trans}
% \end{figure}

% %\begin{figure}
% %    \centering
% %    \begin{subfigure}{0.32\linewidth}
% %         \centering
% %         \includegraphics[width=\textwidth]{Figures/top1.jpg}
% %         \caption{Top}
% %         \label{fig:limb1}
% %     \end{subfigure}
% %     \begin{subfigure}{0.32\linewidth}
% %         \centering
% %         \includegraphics[width=\textwidth]{Figures/2left.jpg}
% %         \caption{Left and Top}
% %         \label{fig:limb2}
% %     \end{subfigure}
% %     \begin{subfigure}{0.32\linewidth}
% %         \centering
% %         \includegraphics[width=\textwidth]{Figures/None.jpg}
% %         \caption{None}
% %         \label{fig:blank}
% %     \end{subfigure}
% %    \caption{Rotation Gait}
% %    \label{fig:Spine_Config}
% %\end{figure}


% \subsection{Real-Time Pose Estimation} 
% Three markers of different colors are placed on each limb such that each of the colors are easily distinguished from the robot and background surfaces. Additionally, the markers are fixed so that they do not hinder the movement of limbs while easily being swapped out if necessary. 
% % -- HBezawada draft
% % \\
% The markers are detected using the OpenCV2 package and implemented in python on a Raspberry Pi. Before starting the experiment, the color threshold parameters for different colored markers based on different background surfaces are obtained. The real-time tracking algorithm is implemented as shown in Algorithm \ref{Alg - Marker Detection} and is available at 
% \hyperlink{https://github.com/AgileRoboticsLab/SoftRobotics-Microspines}{github.com/AgileRoboticsLab/SoftRobotics-Microspines}.

% \begin{algorithm} 
%     \caption{Real-Time Tracking}
%     \label{Alg - Marker Detection}
%     \KwData{Image from the camera}
%     \KwResult{Marker positions, traversed statistics}
%     Start the camera\;
%     \While{Camera is running, for each frame}{
%     Change the image to HSV format\;
%     Image segmentation to detect markers\;
%     Obtain the marker's center position\;
%     Calculate the centroid of the three detected markers $\rightarrow$ hub location\;
    
%     Calculate the Euclidean distance ($d_i$) traveled from the previous frame\;
%     }
%     Calculate the total distance ($D$), displacement traveled\;
%     Calculate the rotation ($\theta$) between starting ($i=0$) and final ($i=n$) orientation
% \end{algorithm}
% At time $t$, let the marker position on limb $i=1,2,3$ be defined as $\bm{m}_i(t)\in \Re^{2}$. The centroid position, relative displacement, rotation, and distance traveled are $\bm{c}(t), \bm{d}(t), \theta(t), D(t)$. Marker position from the centroid is $\displaystyle \bm{\widetilde{m}}_i=\bm{m}_i -\bm{c}(t)$ where $\bm{c}(t)=\frac{\sum_{i=1}^{3} \bm{m}_i}{3}$. Instantaneous rotation, relative displacement, and rotation are calculated as
% \begin{align}
%     \begin{gathered}
%     d_i(t) = ||\bm{c}(t)-\bm{c}(t-1)||, \qquad
%     % d_i = \sqrt{(x_{i-1}-x_{i})^2+(y_{i-1}-y_{i})^2}, \quad
%     D(t) = \sum_{j=1}^{t} d_j \\
%     \theta(t) = \arccos\left(\frac{\widetilde{\bm{m}}_k(0) \cdot \widetilde{\bm{m}}_k(t)}{{|\widetilde{\bm{m}}_k(0)| \cdot |\widetilde{\bm{m}}_k(t)|}}\right)        \qquad k=1,2,3
%     \end{gathered}
% \end{align}

% The rotation ($\theta$) is calculated based on this vector between start and final frame. %Distance and cumulative distance is calculated as shown in equation \ref{Eq: Euclidean Distance}. Displacement is the distance calculated between the centroid of the first and last frames. 
% Apart from rotation, which results in degrees, all other statistics are obtained in pixels and later scaled to inches. An example of the real-time tracking output with cumulative distance, displacement, and rotation is shown in \Fig~\ref{fig:track}.

% % \begin{equation}
% % \begin{rcases}
% % \begin{aligned} \label{Eq: Euclidean Distance}
% %     d_i &= \sqrt{(x_{i-1}-x_{i})^2+(y_{i-1}-y_{i})^2}\\
% %     D &= \sum_{i=1}^{n} d_i \\
% %     \theta &= \arccos\left(\frac{{V_{0} \cdot V_{n}}}{{|V_{0}| \cdot |V_{n}|}}\right)
% % \end{aligned} 
% % \qquad \text{\space}
% % \end{rcases}
% % \end{equation}
% % %

% \begin{figure}
%     \centering
%     \includegraphics[width=0.99\linewidth]{Figures/yobot_dir_0_black_trans_Test803_imageU.png}
%     \caption{Real-time tracking where distance, rotation, and displacement are relative to the previous time frame. The cumulative quantities are measured from time $t=0$.}
%     \label{fig:track}
% \end{figure}
\section{Results and Evaluation}

In this section, we first present the results of each of the previous models using our dataset and compare the results that these models give using the updated lexicons (RQ1).
We also present the results of our hybrid approach that predicted whether a post contains hate speech or not using a modified version of BERT (RQ2). 

\subsection{Data Preprocessing}
To test our approach we use the publicly available datasets. To evaluate our models, we employ average model accuracy, F1-measure, as well as class-specific precision and recall.
While the accuracy and F1 values offer a broad overview of the model's performance, the precision and recall scores for each class provide more specific information.

\subsection{RQ1: Evaluating Our Adaptive Approach for Lexicon Improvement}
\textcolor{black}{To evaluate the effectiveness of our adaptive lexicon approach, we used the models provided by~\citeauthor{davidson2017automated}\cite{davidson2017automated}for several reasons. First, these models are well-established baselines in the field, frequently cited and used for benchmarking new approaches to hate speech detection. Second, they rely on traditional lexicon-based methods, making them ideal candidates to demonstrate the improvements achieved by our adaptive lexicon updates. Our goal for RQ1 is to \textit{validate} that our lexicon updating method enhances the performance of existing models by aligning them with evolving language trends.
We tested the models provided by Davidson et al.~\cite{davidson2017automated} using the new dataset from Founta et al.~\cite{founta2018large}. We found that the accuracy dropped from the originally reported ~90\% during training to 76\% in our tests. This indicates that language evolves over time and that toxic lexicons must be updated to remain effective for detecting toxic language. Next, we utilized the same models with newer datasets but incorporated updated lexicons to validate our approach. We implemented and evaluated the Support Vector Machine (SVM) and Random Forest (RF) classifiers provided by Davidson et al. to detect hate speech, using the 100,000 social media posts from Founta et al.~\cite{founta2018large} as training and testing datasets.}

\begin{table}
\caption{Model performance across different word embedding lexicons for traditional models.}
\label{tab:supervisedPerformance}
\begin{center}
\small
\begin{tabular}{|l|l|lllll|}
\hline
\textbf{Features} & \textbf{Lexicon Size} & \multicolumn{1}{l|}{\textbf{Class}} & \multicolumn{1}{l|}{\textbf{Prec.}} & \multicolumn{1}{l|}{\textbf{Rec.}} & \multicolumn{1}{l|}{\textbf{F1}} & \textbf{Accr.} \\ \hline
\textbf{Linear SVM} &  &  &  &  &  &  \\ \hline
\multirow{2}{*}{$S_{lexicons}$} & \multirow{2}{*}{749} & \multicolumn{1}{l|}{Hate} & \multicolumn{1}{l|}{0.69} & \multicolumn{1}{l|}{0.77} & \multicolumn{1}{l|}{0.73} & \multirow{2}{*}{0.76} \\ \cline{3-6}
 &  & \multicolumn{1}{l|}{Normal} & \multicolumn{1}{l|}{0.73} & \multicolumn{1}{l|}{0.65} & \multicolumn{1}{l|}{0.69} &  \\ \hline
\multirow{2}{*}{$U_{Word2Vec}$} & \multirow{2}{*}{1006} & \multicolumn{1}{l|}{Hate} & \multicolumn{1}{l|}{0.89} & \multicolumn{1}{l|}{0.68} & \multicolumn{1}{l|}{0.81} & \multirow{2}{*}{0.77} \\ \cline{3-6}
 &  & \multicolumn{1}{l|}{Normal} & \multicolumn{1}{l|}{0.77} & \multicolumn{1}{l|}{0.99} & \multicolumn{1}{l|}{0.87} &  \\ \hline
\multirow{2}{*}{$U_{GloVe}$} & \multirow{2}{*}{1010} & \multicolumn{1}{l|}{Hate} & \multicolumn{1}{l|}{0.83} & \multicolumn{1}{l|}{0.77} & \multicolumn{1}{l|}{0.80} & \multirow{2}{*}{0.82} \\ \cline{3-6}
 &  & \multicolumn{1}{l|}{Normal} & \multicolumn{1}{l|}{0.70} & \multicolumn{1}{l|}{0.76} & \multicolumn{1}{l|}{0.73} &  \\ \hline
\multirow{2}{*}{$U_{BERT}$} & \multirow{2}{*}{1433} & \multicolumn{1}{l|}{Hate} & \multicolumn{1}{l|}{0.90} & \multicolumn{1}{l|}{0.70} & \multicolumn{1}{l|}{0.79} & \multirow{2}{*}{0.82} \\ \cline{3-6}
 &  & \multicolumn{1}{l|}{Normal} & \multicolumn{1}{l|}{0.74} & \multicolumn{1}{l|}{0.92} & \multicolumn{1}{l|}{0.82} &  \\ \hline
\textbf{Random Forest} &  &  &  &  &  &  \\ \hline
\multirow{2}{*}{$S_{lexicons}$} & \multirow{2}{*}{749} & \multicolumn{1}{l|}{Hate} & \multicolumn{1}{l|}{0.74} & \multicolumn{1}{l|}{0.74} & \multicolumn{1}{l|}{0.74} & \multirow{2}{*}{0.79} \\ \cline{3-6}
 &  & \multicolumn{1}{l|}{Normal} & \multicolumn{1}{l|}{0.62} & \multicolumn{1}{l|}{0.62} & \multicolumn{1}{l|}{0.62} &  \\ \hline
\multirow{2}{*}{$U_{Word2Vec}$} & \multirow{2}{*}{1006} & \multicolumn{1}{l|}{Hate} & \multicolumn{1}{l|}{0.90} & \multicolumn{1}{l|}{0.70} & \multicolumn{1}{l|}{0.79} & \multirow{2}{*}{0.82} \\ \cline{3-6}
 &  & \multicolumn{1}{l|}{Normal} & \multicolumn{1}{l|}{0.74} & \multicolumn{1}{l|}{0.92} & \multicolumn{1}{l|}{0.82} &  \\ \hline
\multirow{2}{*}{$U_{GloVe}$} & \multirow{2}{*}{1010} & \multicolumn{1}{l|}{Hate} & \multicolumn{1}{l|}{0.94} & \multicolumn{1}{l|}{0.68} & \multicolumn{1}{l|}{0.79} & \multirow{2}{*}{0.83} \\ \cline{3-6}
 &  & \multicolumn{1}{l|}{Normal} & \multicolumn{1}{l|}{0.76} & \multicolumn{1}{l|}{0.96} & \multicolumn{1}{l|}{0.85} &  \\ \hline
\multirow{2}{*}{$U_{BERT}$} & \multirow{2}{*}{1433} & \multicolumn{1}{l|}{Hate} & \multicolumn{1}{l|}{0.86} & \multicolumn{1}{l|}{0.93} & \multicolumn{1}{l|}{0.89} & \multirow{2}{*}{\textbf{0.85}} \\ \cline{3-6}
 &  & \multicolumn{1}{l|}{Normal} & \multicolumn{1}{l|}{0.91} & \multicolumn{1}{l|}{0.84} & \multicolumn{1}{l|}{0.87} & \\ \hline
\end{tabular}%
\end{center}
\end{table}


\textcolor{black}{Table~\ref{tab:supervisedPerformance} presents the performance metrics of traditional machine learning models using different feature sets, which include lexicons derived from various word embedding models (Word2Vec, GloVe, and BERT). Overall, we find that the Random Forest model with lexicons updated through BERT achieves the highest accuracy at 0.85, outperforming other classifiers. When using only the seed lexicons $S_{lexicons}$, accuracy is lower compared to the updated lexicons generated by the word embedding models. Additionally, the model demonstrates strong class-specific precision and recall. For hate speech, recall (0.93) exceeds precision (0.86), while for normal content, precision (0.91) is higher than recall (0.84).}

\subsection{RQ2: Evaluating Our Hybrid Approach to Risk Detection}

In this section, we evaluate six different BERT-based models: BERT-base~\cite{devlin2018bert}, BERT-large~\cite{devlin2018bert}, RoBERTa~\cite{liu2019roberta}, and modified pre-trained BERT models for hate speech detection, including Detoxify~\cite{Detoxify}, BERT-HateXplain~\cite{Mathew_Saha_Yimam_Biemann_Goyal_Mukherjee_2021}, and HurtBERT~\cite{hurtbert2020}. These models are tested on six different test sets, as described in section~\ref{testset}.

For each BERT-based model, we evaluate performance across six different test sets. Table~\ref{tab:BERTPerformance} summarizes the performance metrics of these models using three feature sets: without lexicons ($W$), with seed lexicons ($S_{lexicons}$), and with the best-performing lexicons derived from BERT ($U_{BERT}$). Overall, we find that Detoxify and BERT-HateXplain outperform the other BERT models.

\begin{table*}[ht]
\centering
\caption{Performance of different BERT-based models for hate speech detection using different feature sets.}
\label{tab:BERTPerformance}
\small
\begin{tabular}{||c|c|c|c|c|c|c||c|c|c|c|c|c||}
\hline
\multirow{3}{*}{\textbf{TestSet}} & \multicolumn{6}{c||}{\textbf{BERT Base}} & \multicolumn{6}{c||}{\textbf{BERT Large}} \\ \cline{2-13} 
 & \multicolumn{2}{c|}{$W$} & \multicolumn{2}{c|}{$S_{lexicons}$} & \multicolumn{2}{c||}{$U_{BERT}$} & \multicolumn{2}{c|}{$W$} & \multicolumn{2}{c|}{$S_{lexicons}$} & \multicolumn{2}{c||}{$U_{BERT}$} \\ \cline{2-13}
 & \textbf{F1} & \textbf{Accr.} & \textbf{F1} & \textbf{Accr.} & \textbf{F1} & \textbf{Accr.} & \textbf{F1} & \textbf{Accr.} & \textbf{F1} & \textbf{Accr.} & \textbf{F1} & \textbf{Accr.} \\ \hline
 
\citeauthor{davidson2017automated}\cite{davidson2017automated} & 0.68 & 0.69 & 0.68 & 0.69 & 0.71 & 0.78 & 0.68 & 0.69 & 0.65 & 0.72 & 0.74 & 0.78 \\ \hline
\citeauthor{founta2018large}\cite{founta2018large} & 0.68 & 0.68 & 0.73 & 0.75 & 0.72 & 0.75 & 0.68 & 0.67 & 0.71 & 0.72 & 0.81 & 0.81 \\ \hline
Implicit Hate~\cite{elsherief-etal-2021-latent} & 0.67 & 0.72 & 0.79 & 0.70 & 0.79 & 0.70 & 0.67 & 0.72 & 0.79 & 0.70 & 0.79 & 0.71 \\ \hline
HateCheck~\cite{rottger-etal-2021-hatecheck} & 0.69 & 0.78 & 0.86 & 0.80 & 0.87 & 0.80 & 0.70 & 0.78 & 0.86 & 0.80 & 0.86 & 0.80 \\ \hline
ToxicSpan~\cite{pavlopoulos-etal-2022-acl} & 0.74 & 0.83 & 0.87 & 0.87 & 0.89 & 0.84 & 0.74 & 0.83 & 0.89 & 0.87 & 0.89 & 0.85 \\ \hline
ToxiGen~\cite{hartvigsen-etal-2022-toxigen} & 0.73 & 0.81 & 0.81 & 0.85 & 0.89 & 0.85 & 0.74 & 0.85 & 0.89 & 0.85 & 0.87 & 0.86 \\ \hline

\multicolumn{13}{c}{} \\ \hline
\multirow{3}{*}{\textbf{TestSet}} & \multicolumn{6}{c||}{\textbf{RoBERTa}} & \multicolumn{6}{c||}{\textbf{Detoxify}} \\ \cline{2-13} 
 & \multicolumn{2}{c|}{$W$} & \multicolumn{2}{c|}{$S_{lexicons}$} & \multicolumn{2}{c||}{$U_{BERT}$} & \multicolumn{2}{c|}{$W$} & \multicolumn{2}{c|}{$S_{lexicons}$} & \multicolumn{2}{c||}{$U_{BERT}$} \\ \cline{2-13}
 & \textbf{F1} & \textbf{Accr.} & \textbf{F1} & \textbf{Accr.} & \textbf{F1} & \textbf{Accr.} & \textbf{F1} & \textbf{Accr.} & \textbf{F1} & \textbf{Accr.} & \textbf{F1} & \textbf{Accr.} \\ \hline
 
\citeauthor{davidson2017automated}\cite{davidson2017automated} & 0.74 & 0.78 & 0.74 & 0.79 & 0.71 & 0.78 
                             & 0.81 & 0.79 & 0.85 & 0.86 & \textbf{0.84} & \textbf{0.88} \\ \hline
                             
\citeauthor{founta2018large}\cite{founta2018large} & 0.76 & 0.79 & 0.75 & 0.82 & 0.74 & 0.81
                       & 0.84 & 0.87 & 0.91 & 0.92 & \textbf{0.91} & \textbf{0.94} \\ \hline
                       
Implicit Hate~\cite{elsherief-etal-2021-latent} & 0.73 & 0.72 & 0.79 & 0.70 & 0.79 & 0.70 
                                                & 0.83 & 0.82 & 0.79 & 0.80 & \textbf{0.89} & \textbf{0.91} \\ \hline
                                                
HateCheck~\cite{rottger-etal-2021-hatecheck} & 0.73 & 0.76 & 0.79 & 0.79 & 0.79 & 0.80
                                             & 0.80 & 0.88 & 0.94 & 0.90 & 0.96 & 0.91 \\ \hline
                                             
ToxicSpan~\cite{pavlopoulos-etal-2022-acl} & 0.76 & 0.83 & 0.87 & 0.87 & 0.84 & 0.84 
                                           & 0.84 & 0.84 & 0.89 & 0.87 & 0.89 & 0.85 \\ \hline
                                           
ToxiGen~\cite{hartvigsen-etal-2022-toxigen} & 0.83 & 0.81 & 0.81 & 0.85 & 0.89 & 0.85 
                                            & 0.84 & 0.85 & 0.89 & 0.85 & 0.87 & 0.86 \\ \hline

\multicolumn{13}{c}{} \\ \hline
\multirow{3}{*}{\textbf{TestSet}} & \multicolumn{6}{c||}{\textbf{HurtBERT}} & \multicolumn{6}{c||}{\textbf{BERT-HateXplain}} \\ \cline{2-13} 
 & \multicolumn{2}{c|}{$W$} & \multicolumn{2}{c|}{$S_{lexicons}$} & \multicolumn{2}{c||}{$U_{BERT}$} & \multicolumn{2}{c|}{$W$} & \multicolumn{2}{c|}{$S_{lexicons}$} & \multicolumn{2}{c||}{$U_{BERT}$} \\ \cline{2-13}
 & \textbf{F1} & \textbf{Accr.} & \textbf{F1} & \textbf{Accr.} & \textbf{F1} & \textbf{Accr.} & \textbf{F1} & \textbf{Accr.} & \textbf{F1} & \textbf{Accr.} & \textbf{F1} & \textbf{Accr.} \\ \hline
 
\citeauthor{davidson2017automated}\cite{davidson2017automated} & 0.81 & 0.81 & 0.89 & 0.89 & 0.81 & 0.85 
                             & 0.85 & 0.82 & 0.86 & 0.89 & \textbf{0.84} & 0.83 \\ \hline
                             
\citeauthor{founta2018large}\cite{founta2018large} & 0.81 & 0.82 & 0.83 & 0.85 & 0.84 & 0.85 
                       & 0.84 & 0.84 & 0.84 & 0.87 & 0.82 & 0.85 \\ \hline
                       
Implicit Hate~\cite{elsherief-etal-2021-latent} & 0.73 & 0.76 & 0.79 & 0.79 & 0.79 & 0.81 & 0.73 & 0.77 & 0.80 & 0.82 & 0.80 & 0.84 \\ \hline

HateCheck~\cite{rottger-etal-2021-hatecheck} & 0.76 & 0.78 & 0.86 & 0.91 & 0.87 & 0.91 & 0.79 & 0.81 & 0.86 & 0.92 & \textbf{0.86} & \textbf{0.93} \\ \hline

ToxicSpan~\cite{pavlopoulos-etal-2022-acl} & 0.84 & 0.83 & 0.87 & 0.87 & 0.89 & 0.93 & 0.84 & 0.83 & 0.89 & 0.87 & \textbf{0.89} & \textbf{0.95} \\ \hline

ToxiGen~\cite{hartvigsen-etal-2022-toxigen} & 0.83 & 0.86 & 0.91 & 0.95 & 0.92 & 0.95 & 0.84 & 0.85 & 0.89 & 0.95 & \textbf{0.92} & \textbf{0.96} \\ \hline


\end{tabular}
\end{table*}




%%%%%%%%%%%%%%%%%%%%%%%%%%%%%%%%%%%%%%%%%%%%%%%%%%%%%%%%%%%%%%%%%%%%%%%%%%%%%%%%
\section{CONCLUSION AND FUTURE WORK}
SoRos have shown immense potential with inherent conformability and adaptability to a multitude of surfaces, yet they previously lacked adequate grip stability to overcome non-uniform surfaces present outside of lab environments. Compliant microspines are one missing piece towards shrinking this real-life realization gap. We propose an elegant, compliant microspine design with a standardized soft-compliant integration technique. The stacked array configuration enables the SoRo to maintain surface interaction when extreme surface discrepancies are present. We provide results from a set of field experiments reflecting the improved performance of two microspine array configurations over a baseline SoRo on four different, ruggedized surfaces. %\hl{Our results indicate that the 1ML design is superior to 0ML with significantly increased planar movement on all tested surfaces, and the 2ML design results in improved performance on three of the surfaces with an emphasis on increased repeatability across trials.} 
Our results indicate that microspines are a vital technology for increasing terrain traversability in mobile SoRos. Future work includes optimizing microspine array configurations for different surfaces, performing additional field experiments, and exploring the generalizability of the design to different prototypes.





% %1. Designed microspine array grippers for soft robot.
% % 2. The modular design allowed for exploration of three spine configurations - xx,xx,xx
% % 3. The three test surfaces were yy, yy, yy
% % 4. The experimental setup allowed for real-time tracking of the robot pose. 
% % 180 experiments were conducted to investigate repeatability of locomotion.
% % A translation gait was used for these experiments which is an optimal translation gait for SoRo locomotion on rubber mat with no spines.
% % All three surfaces, there is increased surface interaction for both the spine configurations when compared with no spine configuration. 
% % This is visible in the increased coupled rotation and translation of the robot during the gaits.
% Modular soft microspine endcap grippers are designed for a three-limb SoRo to increase surface interaction regardless of surface topography. The modular design allows for different configurations to be explored, namely no spines, inward facing spines, and directional spines. Experiments are performed on three variable, modular surfaces: porous wood, rubber mat, and smooth whiteboard. Real-time tracking of the robot pose is used to record relative and cumulative displacement and rotation over a $49.5\sec$ trial. An optimal translation gait identified for the three-limb SoRo on a rubber mat without spines is used for all experiments. 20 experiments are performed for each surface and spine configuration combination for a total of 180 experiments to investigate locomotion repeatability. Increased surface interaction on every surface for inward facing and directional spines is observed when compared to no spines. Decreased variance in translation is seen in the smooth and porous surfaces, showing an increase in repeatability across trials when microspines are present.
% Additionally, the rotation and translation seen during the gaits are tightly coupled as there is never a significant increase in translation without a similar increase in rotation. In short, we find that microspine grippers provide a significant increase in the engagement between the robot and environment. %This is seen even more in preliminary testing of the SoRo on inclines as the spherically reconfigurable design orients itself towards the path of least resistance, introducing greater rotation on uneven surfaces.
% %During preliminary testing of the three-limbed SoRo on inclines, increased rotation was observed. The SoRo is spherically reconfigurable and wants to orient itself towards the path of least resistance which is down a given incline. 

% % Despite not being an optimal gait for all the other surfaces and spine configurations, an increased repeatability (decreased variance) for translation is observed.


% %Microspine grippers are small spines commonly found on insect legs that reinforce surface interaction by engaging with asperities to increase shear force and traction. An array of such microspines, when integrated into a robot can provide them with the ability to maneuver uneven terrains, inclines and even climb walls. The surface conformability and adaptability of soft robots (SoRos) makes them an ideal candidates for traversing complex terrains. %
% % Taking inspiration from nature, an array of these microspines can be attached to end effectors, enabling robots to traverse uneven terrain on inclines and climb vertical walls.
% %One way of Despite recent advancements, there remains a real-life realization gap for soft locomotors pertaining to their transition from controlled lab environment to the field.
% % in the field of soft robotics, the jump from controlled lab environments to real-world, unstructured test sites remains a core challenge. 

% % Integration of microspines into SoRos has the potential to shrink this gap by improving grip stability for traversability. %
% % % Integrating microspines into SoRos increases stability and traversal capabilities, effectively shrinking the reality gap.
% % In this paper, we propose a passive modular microspine array endcap for motor-tendon actuated (MTA) soft robots. Here, the direction of the microspine array in the endcap can be varied. 
% % % We use a spherically reconfigurable MTA three-limb SoRo design from a previous work as our experimental prototype. We propose a modular endcap design for MTA SoRos outfitted with different passive microspine configurations. 
% % Consequently, modularity enables exploration of nonidentical array configuration schemes to account for different surfaces. %
% % These microspine arrays are combined with a three-limb MTA SoRo for realizing  planar locomotion with real-time tracking. Experiments are conducted on smooth, rough, and porous surfaces with different array configurations. The directional configuration, where microspines are oriented in the robot forward direction, differs from the inward configuration where they are aligned in the same direction w.r.t. the individual limb.
% % %Planar locomotion tests are performed on flat ground with real-time tracking. We demonstrate experiments on smooth, rough, and porous surfaces to evaluate the impact of microspine placement on a variety of surfaces. 
% % Experiments are conducted using a push-pull translation gait for 180 trials where 20 trials were conducted for a particular configuration and surface. Each trial comprised of 45 gaits to examine repeatability. Results indicate that spine array integration increases displacement on all three surfaces and, on average, the directional configuration robot moves twice as much when compared with no spines. Additionally, for the given gait, they improve locomotion repeatability and reliability. The experiments show the microspine gripper consistently increases engagement of the robot with the surfaces.

% Future work will include experiments on more complex surfaces including inclines and dirt/grass (real-world). Controlled locomotion gait sequences will be explored to account for the changing inclines while minimizing rotation, but some amount of rotation will likely always be present. Increased surface area interaction is vital for the microspines to be effective. This can be achieved by attaching a larger array of spines to account for missed interactions. Consequently, we plan to investigate utilizing active microspine arrays. This will have the added capability of reconfiguring the SoRo to deploy microspines dependent on the surface topography as well as including microspines in only portions of gait sequences.
% %Different microspine materials will be tested as deterioration can occur over time on surfaces such as carpet where a large interaction takes place. 
% %This is clear in the literature where robots can have up to 250 spines/limb forcing a meaningful interaction occurs every time. 
% %Since the current robot is spherically reconfigurable, which innately introduces rotation into any gait and the system as a whole. This is a much larger problem on inclined surfaces as the robot wants to orient itself towards the path of least resistance, ultimately moving down the incline. Additionally, it is clear from the testing that the robot is not a perfectly stable system. Despite aligning the robot in the same orientations for each test case, the variance in the results is very noticeable. Simply stating that a symmetric robot would fix this problem is not the case. The researchers believe that stability is the more important factor than symmetry, hence designing and manufacturing a robot that is agnostic towards inclined surfaces through stability is a must for future works.
% %All mentioned approaches above will increase the overall performance, but without a holistic approach, the robot will not be able to be taken into the real world with meaningful testing cases. 
% % The addition of passive microspine endcaps increased the distance and speed traveled as well as the consistency of locomotion. The future improvements mentioned above will increase the overall performance, but a holistic approach must be taken as all parameters are linked. As progress continues, more surfaces and real world tests will be able to be conducted and applications can be built from this soft robotic platform.




\section*{ACKNOWLEDGMENT}

We thank Bek Ervin for fabricating the 0ML prototype.

%%%%%%%%%%%%%%%%%%%%%%%%%%%%%%%%%%%%%%%%%%%%%%%%%%%%%%%%%%%%%%%%%%%%%%%%%%%%%%%%
%\section*{APPENDIX}

%Appendixes should appear before the acknowledgment.

\bibliographystyle{IEEEtran}
%\bibliography{IEEEabrv,MicroSpine1.bib}
\bibliography{IEEEabrv,MicroSpine1_NoLink.bib}
%%%%%%%%%%%%%%%%%%%%%%%%%%%%%%%%%%%%%%%%%%%%%%%%%%%%%%%%%%%%%%%%%%%%%%%%%%%%%%%%
\end{document}
