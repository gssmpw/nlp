\section{Related Works}
\label{sec:related}

This work investigates two distinct bifurcating phenomena in fluid flows in sudden-expansion channels: a symmetry-breaking bifurcation known as the Coandă effect \parencite{PichiDrivingBifurcatingParametrized2022a,Khamlich_2022,Pitton_2017}, and a Hopf bifurcation \parencite{QUAINI,Fortin_localization}. %
%
%
The Coandă effect arises from the tendency of the fluid jet to attach to nearby surfaces, driven by velocity fluctuations that generate transverse pressure gradients, which ultimately maintain flow asymmetry~\parencite{tritton1977physical, wille1965coanda}. The physical mechanism has been extensively studied \parencite{Cherdron_Durst_Whitelaw_1978, Sobey_Drazin_1986}, revealing that at low Reynolds numbers, viscous dissipation stabilizes the flow \parencite{HAWA_RUSAK_2001}. As Reynolds number increases, convective effects upstream of the expansion overcome this stabilization, leading to symmetric solution instability.

The seminal work by Sobey and Drazin \parencite{Sobey_Drazin_1986} combined asymptotic, numerical, and experimental approaches to identify a Hopf bifurcation, with the bifurcation point later refined by Quaini et al.\ \parencite{QUAINI}. The emergence of periodic solutions through Hopf bifurcations for different benchmarks is well-documented in \gls{cfd} literature \parencite{Sobey_Drazin_1986, Fortin_localization, ArioliKoch2021, Dušek_Gal_Fraunié_1994}.

%

Recent years have seen significant developments in \glspl{rom} for these systems, extending both the range of applications and the available methodologies. To reduce the gap with real-world problems, more advanced physical models have been considered, e.g.\ including fluid-structure interaction effects \parencite{Khamlich_2022}, optimization and control to drive bifurcating phenomena \parencite{PichiDrivingBifurcatingParametrized2022a,Boull2023}, interpretable \glspl{rom} for the fluidic pinball \parencite{Pastur_2019}, geometrically parametrized domains ____, and the compressible \gls{cfd} setting \parencite{Tonicello_2024}. As concerns the methodological advancements,  deflated approaches have been used to discover the multiple coexisting states \parencite{pintore_2021}, recently combined with greedy algorithms for efficiency, adaptivity and model certification \parencite{pichi2025deflationbasedcertifiedgreedyalgorithm}. A stochastic perturbation approach \parencite{gonnella2025stochasticperturbationapproachnonlinear} has been developed to obtain insights about the bifurcation points in the uncertainty quantification setting. Moreover, localized \gls{rom} strategies have been used to build specific reduced spaces depending on the fluid's regime \parencite{Hess_2019}.

%

From a different perspective, an interesting and potentially effective approach consists in tackling the problem by means of sparse identification, in particular employing \gls{sindy} \parencite{Brunton_2016}. Originally developed for learning dynamics from system snapshots, \gls{sindy} has evolved through various generalizations \parencite{SINDy_ensemble, SINDyCP, Messenger_2021} and found applications in fluid flow identification \parencite{Fukami_Murata_Zhang_Fukagata_2021, wang_zhang_2024} and convective phenomena. The integration of \gls{sindy} with \glspl{ae} for dimensionality reduction was pioneered by Champion et al.~\parencite{Champion_2019}, extended in~\parencite{DEEPDELAY}, and adapted for parameterized systems by~\parencite{Conti_2023}.

At the same time, neural networks and data-driven approaches have become increasingly prevalent in dynamical systems research, particularly for bifurcation analysis \parencite{Kalia2021LearningNF,PICHI2023105813} and low-dimensional flow representation \parencite{Della_Pia_2024, spider}. However, challenges persist regarding interpretability and overfitting tendencies. The \gls{nn}-\gls{sindy} approach was developed specifically to address these limitations.

Our work advances the state of the art by combining the SINDy-AE approach with nested \gls{pod} \parencite{nested_POD1, nested_POD2}, a dimensionality reduction technique well-suited for time-dependent parameterized PDEs. This integration improves computational efficiency during the offline phase, enabling the processing of larger snapshot sets with reduced computational resources. While \gls{sindy} has been applied to PDE identification \parencite{Zheng2024LESSINDyLS, Rudy_2017, Messenger_2021}, we maintain an ODE framework, consistent with recent developments \parencite{Champion_2019, Conti_2023, VVV}.

%

 %