\begin{abstract}
We study Variational Rectified Flow Matching, a framework that enhances classic rectified flow matching by modeling multi-modal velocity vector-fields. At inference time, classic rectified flow matching `moves' samples from a source distribution to the target distribution by solving an ordinary differential equation via integration along a velocity vector-field. At training time, the velocity vector-field is learnt by linearly interpolating between coupled samples one drawn from the source and one drawn from the target distribution randomly. This leads to ``ground-truth'' velocity vector-fields that point in different directions at the same location, i.e., the velocity vector-fields are multi-modal/ambiguous. However, since training uses a standard mean-squared-error loss, the learnt velocity vector-field averages ``ground-truth'' directions and isn't multi-modal. 
In contrast,  variational rectified flow matching learns and samples from multi-modal flow directions. We show on synthetic data, MNIST, CIFAR-10, and ImageNet that variational rectified flow matching leads to compelling results. 
\end{abstract}
