\section{Introduction}
\label{sec:intro}

\begin{figure*}[t]
    \centering
    \vspace{-1mm}
    \setlength{\tabcolsep}{1pt}
    \begin{tabular}{ccc}
    \includegraphics[width=0.275\linewidth]{figs/teaser/gt_adv_False_5.png} &
    \includegraphics[width=0.275\linewidth]{figs/teaser/baseline_itr_10000_3.png} &
    \includegraphics[width=0.275\linewidth]{figs/teaser/our_itr_6000_2.png} \\[-2                               mm]
    (a) Ground Truth &(b) Rectified FM (Baseline) &(c) Variational Rectified FM (Ours)
    \end{tabular}
    \vspace{-3mm}
    \caption{Intuition and motivation: Rectified flow matching randomly couples source data and target data samples, as illustrated in panel (a). This leads to velocity vector-fields with ambiguous directions. Panel (b) shows that the classic rectified flow matching averages ambiguous targets, which leads to curved flows. In contrast, the proposed variational rectified flow matching is able to successfully model ambiguity which leads to less curved flows as depicted in panel (c).}
    \vspace{-3mm}
    \label{fig:intuition}
\end{figure*}


Diffusion models~\citep{ho2020denoising,song2021denoising,SongICLR2021} and flow matching~\citep{liu2023flow,LipmanICLR2023,albergo2023building,albergo2023stochastic} have been remarkably successful in recent years. These techniques have been applied across domains from computer vision~\citep{ho2020denoising} and robotics~\citep{kapelyukh2023dall} to computational biology~\citep{guo2024diffusion} and medical imaging~\citep{song2022solving}.

Flow matching~\citep{LipmanICLR2023,liu2023flow,albergo2023building} can be viewed as a continuous time generalization of classic diffusion models~\citep{albergo2023stochastic,ma2024sit}. Those in turn can be viewed as 
a variant of a hierarchical variational auto-encoder~\citep{luo2022understanding}. At inference time, flow matching `moves' a sample from a source distribution  to the target distribution by solving an ordinary differential equation via integration along a velocity vector-field. To learn this velocity vector-field, at training time, flow matching regresses to a constructed vector-field/flow connecting any sample from the source distribution --- think of the data-domain positioned at time zero --- to any sample from the target distribution attained at time one. Notably, in a `rectified flow,' the samples from the source and target distribution are connected via a straight line as shown in \cref{fig:intuition}(a). Inevitably, this leads to multi-modality/ambiguity, i.e.,  flows pointing in different directions at the same location in the data-domain-time-domain, as illustrated for a one-dimensional data-domain in \cref{fig:intuition}(a). Since classic rectified flow matching employs a standard squared-norm loss to compare the predicted velocity vector-field to the constructed velocity vector-field, it does not capture this multi-modality. Hence, rectified flow matching aims to match the source and target distribution in alternative ways. %
This is illustrated in \cref{fig:intuition}(b).

To enable rectified flow matching to capture this multi-modality in the data-domain-time-domain, we study \emph{variational rectified flow matching}. Intuitively, variational rectified flow matching introduces a latent variable that permits to disentangle multi-modal/ambiguous flow directions at each location in the data-domain-time-domain. This approach follows the classic variational inference paradigm underlying expectation maximization or variational auto-encoders. Indeed, as shown in \cref{fig:intuition}(c),  variational rectified flow matching permits to model flow trajectories that intersect. 
This leads to learned trajectories that more closely align with the ground truth flow.
The latent variable can also be used to disentangle different directions.

Note that flow matching, diffusion models, and  variational auto-encoders are all able to capture multi-modality in the data-domain, as one expects from a generative model. Importantly, variational rectified flow matching differs in that \textit{it  also models multi-modality in the data-domain-time-domain}. This enables different flow directions at the same data-domain-time-domain point, allowing the resulting flows to intersect at that location.


We demonstrate the benefits of variational rectified flow matching across various datasets and model architectures. On synthetic data, our method more accurately models data distributions and better captures velocity ambiguity. On MNIST, it enables controllable image generation with improved quality. On CIFAR-10, our approach outperforms classic rectified flow matching across different integration steps. Lastly, on ImageNet, our method consistently improves the FID score of SiT-XL~\cite{ma2024sit}.

In summary, our contribution is as follows: we  study the properties of variational rectified flow matching, and, along the way, offer an alternative way to interpret the flow matching procedure. %
