\section{Related Works}
\label{sec:RelatedWork}
Motion planning for tractor-trailer robots can be mainly classified into geometry-based methods and optimization-based methods. Geometry-based methods only plan the geometric path for the robot and design the path tracking controller to track it. The geometric path is usually generated by search- or sampling-based methods. Search-based methods typically pre-construct a grid map or graph structure that represents the environment, and then perform graph searching within it to find paths~\cite{hybridAs}. For example, Liu et al.~\cite{geo_ga} utilized the conception of equivalent size for obstacles growing and robot shrinking, planning the path by applying a heuristic genetic algorithm. Sun et al.~\cite{geo_prm} built a global compound roadmap by combining the local regular roadmap with the universal probabilistic roadmap, lowering the complexity of the planning computation. To improve the efficiency and path quality, Ljungqvist et al.~\cite{geo_lat} generated a finite set of kinematically feasible motion primitives offline, searching solutions for a general 2-trailer system in a regular state lattice created by these motion primitives. They also proposed a novel parametrization of the reachable state space to make the graph-search problem tractable for real-time applications. Furthermore, by simplifying the design of motion primitives and utilizing reinforcement learning to obtain a well-defined heuristic function, Leu et al. presented an improved A-search guided tree\cite{geometry4} that allows quick off-lattice exploration to find a solution.

Sampling-based methods usually define a space for a path planning task. They maintain a tree structure and iteratively sample in this space, expanding tree nodes according to some criteria, ultimately extracting the trajectories from the tree containing the start and goal. By customizing the sampling space based on motion primitives instead of the space containing all $SE(2)$ states of vehicles in the tractor trailer robot, Cheng et al.~\cite{geometry1}, Lattarulo et al.~\cite{geo_rrt} used the Rapidly-exploring Random Tree\cite{rrt,rrts} (RRT) algorithm to efficiently plan a feasible path that satisfied with the nonholonomic and mechanical constraints. Moreover, Manav et al.~\cite{geo_crrt} proposed iterative analytical method, combining which with Closed-Loop Rapidly Exploring Random Tree (CL-RRT) approach in a cascade path planning, enabling the generation of both kinematically feasible and deterministic parking maneuvers with obstacle avoidance.

After obtaining the path by search- or sampling-based methods, the trajectory tracker will choose a piece of path as reference based on the current state of the robot and compute the control command sent to the robot\cite{geometry2}. For example, in work\cite{geometry1}, the path is followed with fuzzy control and Line-of-sight approach\cite{fuzzy}. While in work\cite{geometry3}, a semidefinite programming (SDP) problem is constructed to combine with motion primitive to get the command. Dahlmann et al.~\cite{mpc_opt,mpc_opt2} proposed model predictive controllers to perform local optimizations with Voronoi field\cite{volo} for suboptimal reference trajectories.

Although geometry-based methods can work in various scenarios, these approach always search in a discrete space, where the optimality of the obtained solution is positively correlated with the search duration and the granularity of the space. When information of higher dimensions in the state of the robot is considered to find a more optimal solution, these methods need to explore larger space, whereupon the problem of combinatorial explosion is magnified, severely affecting the efficiency. To find better solutions in such a large space, optimization-based approaches emerged. These methods often model motion planning as an optimization problem and use numerical solvers to solve it, whose initial values often derived from geometric paths obtained by search- or sampling-based methods. Trajectory optimization typically explores optimal solutions in a continuous space, which can take into account information about higher-order states with guarantees of local optimality. Muralidhara et al.~\cite{ocp_opt} modeled the motion planning problem of tractor-trailer robots as Optimal Control Problem (OCP), describing the trajectory by states in discrete timestamps, solving it by direct multiple shooting method. But in obstacle avoidance, this approach only considers not leaving the tracking line of the traffic lane, which cannot be directly applied to unstructured complex environments. Mohamed et al.~\cite{apf_opt} proposed an approach combining artificial potential fields and optimal control theory to achieve a more generalized obstacle avoidance for tractor-trailer robot. However, this method can easily fall into a local optimum when the environment is complex, resulting in the robot cannot reach the target region. Also using the potential field approach to represent the environment, Wang et al.~\cite{lat_opt2} adopt path integral policy improvement to optimize the paths from geometry based methods, solving the problem of poor performance of graph search methods in narrow scenes the limitations by resolution and search space. Nevertheless, this approach is time-consuming, making it difficult to apply to tasks that require online replanning, such as exploring unknown environments and navigating through dynamically changing scenarios.

The algorithm proposed in this paper is also an optimization-based methods. To plan efficiently, we do not adopt discrete states to represent trajectory and model the problem as an OCP, but propose a lightweight, compact and high-order smooth trajectory representation, reducing the dimension of optimization variables and simplifying the problem.

In works\cite{system_opt,system_opt2,lat_opt}, both obstacles and robots are represented by many circles , as this simple form can be easily modeled into OCP. However, real-world environments are complex that cannot be accurately modeled using only circles. 
% This would result in OCP having too many constraints to deal with, thus reducing the efficiency.
To represent the environment more concisely and accurately, some works~\cite{libai_nogen, libai_pcoc,apten,park_opt1,basedlibai_opt} use convex polygons to approximate obstacles. They also modeled the motion planning problem for tractor-trailer robots as OCP, considering safety constraints and duration in the optimization. Their methods require applying constraints to each obstacle and each vehicle of the robot at each moment. This feature tightly couples their computational complexity to the environment, limiting the efficiency of planning in obstacle-dense environments. To address this issue, some works\cite{libai_corridor,geometry5} shift to Safe Travel Corridors (STC) as obstacle avoidance constraints. The constraints imposed by this method are only related to the trajectory length and not to the density of obstacles. However, the classical safe corridors must be computed before trajectory optimization, since they are constructed based on the initial values of the optimization. For motion planning of a tractor-trailer robot, this will take a lot of time, especially if the number of trailers is large, since the safety corridors need to be built for each vehicle of the robot separately. The second disadvantage of the safe corridor is its high dependence on initial value, while a good one requires much more time due to the high state dimension and complex kinematics of the robot. Usually, the process of safe corridor generation selects some state points of a trajectory as seeds, which will affect the location and size of the generated safe corridors. Thus in some scenarios where it is more difficult to obtain a good initial value of the trajectory, e.g., the tractor-trailer robot needs to make a turn in a narrow environment, the feasible region constrained by the safe corridors may not be able to contain sufficiently good solutions. To reduce the dependency of the optimization on initial values, Li et al.\cite{libai_astar} proposed a multi-stage method that only uses paths from A* as the initial value for trajectory optimization of tractor-trailer robots traversing a curvy tunnel, improving some of the efficiency for this task.

In this paper, we fully leverage the collision-free regions of the environment, directly applying deformations to trajectories in continuous space, which can avoid the above-mentioned problems brought by safe corridors. In known scenarios, the process for environment can be pre-done to adapt our method, while in unknown environments we can also use separately opened thread for maintenance without occupying the time of trajectory planning. 
% Besides, a multi-terminal path search method is proposed in this work for obtaining the initial value of our optimization problem, achieving higher efficiency without losing much optimality.