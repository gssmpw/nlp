% autosam.tex
% Annotated sample file for the preparation of LaTeX files
% for the final versions of papers submitted to or accepted for 
% publication in AUTOMATICA.

% See also the Information for Authors.

% Make sure that the zip file that you send contains all the 
% files, including the files for the figures and the bib file.

% Output produced with the elsart style file does not imitate the
% AUTOMATICA style. The style file is generic for all Elsevier
% journals and the output is laid out for easy copy editing. The
% final document is produced from the source file in the
% AUTOMATICA style at Elsevier.

% You may use the style file autart.cls to obtain a two-column 
% document (see below) that more or less imitates the printed 
% Automatica style. This may helpful to improve the formatting 
% of the equations, tables and figures, and also serves to check 
% whether the paper satisfies the length requirements.

% Please note: Authors must not create their own macros.

% For further information regarding the preparation of LaTeX files 
% for Elsevier, please refer to the "Full Instructions to Authors" 
% from Elsevier's anonymous ftp server on ftp.elsevier.nl in the
% directory pub/styles, or from the internet (CTAN sites) on
% ftp.shsu.edu, ftp.dante.de and ftp.tex.ac.uk in the directory
% tex-archive/macros/latex/contrib/supported/elsevier.


%\documentclass{elsart}               % The use of LaTeX2e is preferred.

\documentclass[twocolumn]{autart}    % Enable this line and disable the 
                                     % preceding line to obtain a two-column 
                                     % document whose style resembles the
                                     % printed Automatica style.


\usepackage{graphicx}          % Include this line if your 
                               % document contains figures,
%\usepackage[dvips]{epsfig}    % or this line, depending on which
                               % you prefer.

\usepackage{xcolor}
\usepackage{amsmath}
\usepackage{tikzscale}
\usepackage{pgfplots}
\usepackage{tikz}
\usepackage{amssymb}
\newtheorem{assumption}{Assumption}
\newtheorem{proposition}{Proposition}
\newtheorem{lemma}{Lemma}
\newtheorem{corollary}{Corollary}
\newtheorem{theorem}{Theorem}
\newtheorem{property}{Property}
\newtheorem{remark}{Remark}
\newtheorem{example}{Example}

\def\cov{{\rm cov}}
\def\vec{{\rm vec}}
\def\E{\mathbb{E}}
\def\N{\mathbb{N}}
\def\R{\mathbb{R}}
\def\endproof{\begin{flushright} \vspace{-0.5cm} $\blacksquare$ \end{flushright}}
\def\A{\bar{A}}
\def\Ap{\A(p)}
\def\e{\epsilon}

\begin{document}

\begin{frontmatter}
%\runtitle{Insert a suggested running title}  % Running title for regular 
                                              % papers but only if the title  
                                              % is over 5 words. Running title 
                                              % is not shown in output.

\title{Exact Covariance Characterization for 
Controlled Linear Systems subject to Stochastic Parametric and Additive Uncertainties\thanksref{footnoteinfo}} % Title, preferably not more 
                                                % than 10 words.

\thanks[footnoteinfo]{This paper was not presented at any IFAC 
meeting. Corresponding author K.~Moussa. }

\author[UPHF,INSA]{Kaouther Moussa}\ead{kaouther.moussa@uphf.fr},    % Add the 
\author[UGA]{Mirko Fiacchini}\ead{mirko.fiacchini@gipsa-lab.fr}               % e-mail address 


\address[UPHF]{UPHF, CNRS, UMR 8201 - LAMIH, F-59313 Valenciennes, France}  % Please supply                                              
\address[INSA]{INSA Hauts-de-France, F-59313, Valenciennes, France}             % full addresses
\address[UGA]{Univ. Grenoble Alpes, CNRS, Grenoble INP, GIPSA-lab, 38000 Grenoble, France}        % here.

          
\begin{keyword}                           % Five to ten keywords,  
Uncertain systems, covariance characterization, invariance, stochastic MPC             % chosen from the IFAC 
\end{keyword}                             % keyword list or with the 
                                          % help of the Automatica 
                                          % keyword wizard


\begin{abstract}                          % Abstract of not more than 200 words.
This work addresses the exact characterization of the covariance dynamics related to  linear discrete-time systems subject to both additive and parametric stochastic uncertainties that are potentially unbounded. The derived exact representation allows to understand how the covariance of the multiplicative parametric uncertainties affects the stability of the state covariance dynamics through a transformation of the parameters covariance matrix, allowing therefore to address the problem of control design for state covariance dynamics in this context. Numerical results assess this new characterization by comparing it to the empirical covariance and illustrating the control design problem. 
\end{abstract}

\end{frontmatter}
 
\section{Introduction}
The covariance control problem, addressed in the literature since the 80s, see \cite{Collins1987,Hsieh1990}, aims at controlling the covariance matrix of a linear discrete-time system affected by additive stochastic noises. Also recent works addressed different types of stochastic systems, for example those subject to input constraints in \cite{Bakolas2018}, those considering chance constraints in \cite{Okamoto2018} and constant random parameters in \cite{Knaup2023}.  Furthermore, the stabilization of linear stochastic systems has also been addressed in \cite{HOSOE2019}, in which equivalent stability and synthesis conditions were provided for the case of independent and identically distributed (i.i.d.) additive and parametric uncertainties.

This paper addresses the problem of covariance control from the point of view of Stochastic Model Predictive Control (SMPC) approaches, for which the exact  characterization of covariance dynamics is useful to tighten time-varying constraints using concentration inequalities such as the Chebyshev's inequality, as used for example in \cite{FARINA2016} and \cite{Hewing2018}. Contrary to randomized methods relying on the generation of disturbance scenarios, for instance in \cite{Cannon2011,Lorenzen2016,Blackmore2010,calafiore2012}, concentration-inequalities based methods rely on an analytic formulation of the covariance dynamics. Exact characterization techniques for SMPC have mainly concerned linear discrete-time dynamical systems affected by additive stochastic uncertainties:
\begin{equation*}
    x_{k+1} = A x_k + B u_k + w_k,
\label{Eq:sys_dyn_Add}
\end{equation*} 
with $x_k, w_k \in \mathbb{R}^n, u_k \in \mathbb{R}^m$. 

One of the main approaches for uncertainties handling in Model Predictive Control (MPC) is the tube-based one \cite{Langson}. It consists in separating the state into a deterministic and an uncertain component and designing a prestabilizing feedback allowing to handle the uncertainties and their effects on chance constraints in the stochastic case. This is achieved by considering $e_k = x_k - z_k$, where $z_k \in \mathbb{R}^n$ represents the nominal deterministic component following the dynamics $z_{k+1} = A z_k + B v_k$, with $u_k = Ke_k + v_k$, in which $K \in \mathbb{R}^{m \times n}$ represents the prestabilizing feedback. The stochastic component $e_k$ follows, therefore, the dynamics $e_{k+1}=(A+BK)e_k+w_k$ and it can be directly noticed that if $e_0=0$, then, the expectation of $e_k$ is also null, which leads to the following covariance dynamics of $e_k$ under the assumption that $w_k$ is i.i.d. with respect to time $k$:
\begin{equation}
    \text{cov}(e_{k+1}) = (A+BK)\text{cov}(e_{k})(A+BK)^T + W,
    \label{Cov_Additive}
\end{equation}
with $W$ being the covariance of $w_k$, \textit{i.e.} $\text{cov}(w_k)=\mathbb{E}[w_k w_k^T]=W$, when $\mathbb{E}[w_k]=0$.  

We can notice that stabilizing the covariance dynamics in (\ref{Cov_Additive}) consists in designing the feedback $K$ such that $A+BK$ is Schur.  In the case where multiplicative parametric uncertainties are also involved, this stability condition does not hold anymore because of the presence of the uncertain parameters in the error dynamics. The error covariance dynamics in (\ref{Cov_Additive}) has been used, for instance, in \cite{KofmanAUT12,Fiacchini2021} for reachability analysis with correlated disturbances and in \cite{Arcari2023} for SMPC, in which the parametric uncertainties were considered to be bounded  with a polytopic description.  

\subsection*{Contribution}
The main contribution of this technical note is to derive a novel exact characterization of the error covariance dynamics when both multiplicative and additive uncertainties (of stochastic nature and potentially unbounded)  affect a discrete-time linear system. This characterization is derived on the vectorization of the error covariance dynamics, using a property linking the vectorization operator to the Kronecker product. To the best of our knowledge, this is the first time that such characterization is derived, allowing therefore to understand how the stochastic properties of the uncertain parameters affect the stability of the error covariance dynamics, via a specific matrix resulting from a transformation of the parameters covariance matrix. Furthermore, we derive a Linear Matrix Inequality (LMI)-based condition for covariance control design, the latter allowing to stabilize the covariance dynamics that contain quadratic terms of the prestabilizing feedback gain $K$ resulting from a Kronecker product property. 

\subsection*{Notation}
Denote with $\mathbb{R}$ and $\mathbb{N}$, respectively, the sets of real and integer numbers. The expectation of a random variable $x$ is denoted by $\mathbb{E}[x]$. Given a random vector $v$, $\cov(v)=\mathbb{E}[(v-\mathbb{E}[v])(v-\mathbb{E}[v])^T]$ stands for the covariance of $v$, if the latter has a zero mean ($\mathbb{E}[v]=0$), then the covariance of $v$ is simply $\cov(v)=\mathbb{E}[vv^T]$. The normal distribution of mean $\mu$ and covariance matrix $\Sigma$ is denoted $\mathcal{N} \left( \mu, \Sigma\right)$. The Kronecker product is denoted by $\otimes$, $\vec(\cdot)$ stands for the vectorization operator and $\vec(\cdot)^{-1}$ stands for the inverse of the vectorization operator. Given a square matrix $A \in \mathbb{R}^{n \times n },$ with $n \in \mathbb{N}$, $\rho(A)$ and $\sigma_{max}(A)$ stand, respectively, for the spectral radius and the maximal singular value of $A$. The multiset consisting of the eigenvalues of $A$ including their algebraic multiplicity is denoted by $\textnormal{mspec}(A)$. $\lambda_{max}(A)$ stands for the largest eigenvalue of $A$ having real eigenvalues. The zero  and identity matrices of appropriate dimensions are denoted, respectively, $0$ and $I$. Given a symmetric matrix $M$,  $M \succ 0$ means that $M$ is positive definite. \\


\section{Problem statement}

Consider the following discrete-time linear system:
\begin{equation}
    x_{k+1} = A(p_k) x_k + B u_k + w_k,
\label{Eq:sys_dyn}
\end{equation} 
where $x_k \in \mathbb{R}^{n}$ and $u_k \in \mathbb{R}^{m}$ represent, respectively, the state and the control input. The initial state $x_0$ is assumed to be deterministic. 

\begin{assumption}\label{ass:w}
The additive disturbance $w_k \in \mathbb{R}^{n}$ is an i.i.d. sequence of random variables with $\mathbb{E}[w_k]=0$ and covariance $\cov(w_k)=\mathbb{E}[(w_k-\mathbb{E}[w_k])(w_k-\mathbb{E}[w_k])^T]=\mathbb{E}[w_kw_k^T]=W$, with $W \succ 0$. 
\end{assumption}

We denote by $p_k \in \mathbb{R}^{l}$ an i.i.d. sequence of random variables representing the uncertain parameters, affecting the terms of the state matrix $A(p_k)$ in an affine way, and having as covariance $\Sigma \succ 0$.  Therefore, the state matrix $A(p_k)$ can be written as: 
\begin{equation*}
    A(p_k) = A_0 + \sum_{i = 1}^{l} A_i p_{ik}=A_0+\bar{A}(p_k),
\end{equation*}
where $A_0$ represents the known (or nominal) and deterministic component of  $A(p_k)$, whereas $\bar{A}(p_k)$ represents the stochastic time-varying component and $p_{ik}$ stands for the $i^{th}$ component of the random vector $p_k$.  

\begin{assumption}\label{ass:p}
The parameter vector $p_k \in \mathbb{R}^{l}$ is an i.i.d. sequence of random variables with $\mathbb{E}[p_k]=0$ and covariance $\cov(p_k) = \mathbb{E}[p_k p_k^T] = \Sigma$, with $\Sigma \succ 0$. 
\end{assumption}

Note that this assumption, from which $\mathbb{E}[\bar{A}(p_k)]=0$ follows, does not induce a loss of generality since the parameters means can always be accounted for by appropriately adding an offset to $A_0$. The pair $(A_0, B)$ is assumed stabilizable. Both Assumption \ref{ass:w} and \ref{ass:p} are supposed to hold in the rest of the paper.

    Furthermore, we assume that the elements of $\bar{A}(p_k)$ are mutually independent of the elements of $w_k$. Note that the latter assumption is not restrictive, it is only considered to simplify the exact characterization of the covariance dynamics, and additional terms resulting from its non-satisfaction (that can be easily considered) do not affect the stability analysis of the covariance dynamics. Since the sequences of $p_k$ and $w_k$ are i.i.d,  then the elements of $\bar{A}(p_k)$ and those of $w_k$ are also independent of  the state for the same time step $k$, meaning that $\mathbb{E}\left[ \bar{A}(p_k) x_k\right] = \mathbb{E}\left[\bar{A}(p_k)\right] \mathbb{E}\left[x_k\right]=0$  and $\mathbb{E}\left[x_kw_k^T\right]=\mathbb{E}\left[x_k\right]\mathbb{E}\left[w_k^T\right]=0$.

Given system~(\ref{Eq:sys_dyn}) and the assumptions formulated above, the problem that will be addressed in this paper is finding an exact expression of the state covariance dynamics related to system~(\ref{Eq:sys_dyn}), often useful in the context of tube-based stochastic MPC applications. Generally, in this context, the state  is expressed as a sum of a deterministic and a random component. This paper addresses the problem of finding the state covariance dynamics from the same point of view.

Moreover, in this paper, we are interested in studying the stability of the state covariance in order to derive a condition allowing to design a prestabilizing feedback gain  that guarantees the stability of the state covariance in the presence of stochastic parametric and additive uncertainties.



\section{Exact covariance characterization}
Consider system~(\ref{Eq:sys_dyn}), the state $x_k$ can be expressed as the sum of a deterministic component $z_k$ and a random component $e_k$ that is 
\begin{equation}
    x_k = z_k + e_k,
\end{equation}
such that
\begin{equation}\label{eq:z}
    z_{k+1} = A_0 z_k +B v_k,
\end{equation}
with $z_0 = x_0$ and then $e_0 = 0$. From $e_k = x_k-z_k$ and by considering $u_k = Ke_k + v_k$ we have:
\begin{align}
    e_{k+1} & = x_{k+1}-z_{k+1} = (A_0+BK)e_k+\bar{A}(p_k)x_k+w_k \nonumber\\
    &=(A(p_k)+BK)e_k+\bar{A}(p_k)z_k+w_k.\label{eq:e}
\end{align}
The following standard assumption is functional to the subsequent results and is not restrictive since $(A_0, B)$ is assumed to be stabilizable, which is commonly used in standard MPC methods. 
\begin{assumption}\label{Ass:exp_stab}
The system (\ref{eq:z}) is exponentially stabilized by the control $v_k$.
\end{assumption}
The following proposition shows that $\mathbb{E}[e_k]=0$ which helps in the exact characterization of the covariance dynamics presented subsequently. 
\begin{proposition}\label{Prop:1}
From $e_0 = 0$ it follows that $\mathbb{E}[e_k]=0$ for all time instants $k$.  
\end{proposition}
\paragraph*{Proof} 
The expectation of the error dynamics is  
\begin{align}
    \mathbb{E}[e_{k+1}]&=\mathbb{E}[(A(p_k)+BK)e_k+\bar{A}(p_k)z_k +w_k] \nonumber\\
    &=\mathbb{E}[(A(p_k)+BK)e_k]+\mathbb{E}[\bar{A}(p_k)z_k]+\mathbb{E}[w_k].\nonumber
\end{align}
Since $A(p_k)$ is independent of both $e_k$ and $z_k$  and the expectation of the product of two independent random variables is the product of their respective expectations \cite{Bertsekas2002} then it follows:  
\begin{equation}
    \mathbb{E}[e_{k+1}]=\mathbb{E}[(A(p_k)+BK)]\mathbb{E}[e_k]+\mathbb{E}[\bar{A}(p_k)]z_k+\mathbb{E}[w_k].
\end{equation}
Moreover, since $\mathbb{E}[\bar{A}(p_k)]=0$ and $\mathbb{E}[w_k]=0$, then
\begin{equation*}
\mathbb{E}[e_{k+1}]=\mathbb{E}[(A(p_k)+BK)]\mathbb{E}[e_k],
\end{equation*}
and therefore, since $e_0$ is deterministic and $e_0=0$, we have that $\mathbb{E}[e_k]=0$ for all time instants $k$. 
\endproof

A direct implication of Proposition \ref{Prop:1} is that $\cov(e_k) = \mathbb{E}[e_k e_k^T]$. The following property is used hereafter for the covariance exact characterization proof.  


\begin{property}\label{pr:BCA} \textnormal{(Proposition 7.1.9., page 401 in \cite{Bernstein2009})}\\
Let $A \in \mathbb{R}^{n \times m}, B \in \mathbb{R}^{m \times l}$ and $C \in \mathbb{R}^{l \times k}$, then:
\begin{equation*}
  \vec (ABC)= \left(C^T \otimes A\right) \vec(B). 
  \label{Eq:property_kron}
\end{equation*}
\end{property}

The main result on the characterization of the covariance matrix of the error is presented hereafter.
\begin{theorem} \label{th:1}
The dynamics of the error covariance related to system~(\ref{Eq:sys_dyn}) is given by the following equivalent expressions:  
\begin{align}
& \cov(e_{k+1})=(A_0+BK) \cov(e_k)(A_0+BK)^T  +W \nonumber\\
& + \vec^{-1} \left(\mathbb{E}[\bar{A}(p_k)\otimes\bar{A}(p_k)]\vec\left( \cov(e_k)+z_kz_k^T\right) \right),
\label{Eq:cov_dyn_w}
\end{align}
and 
\begin{equation}
    \epsilon_{k+1}= \Big( (A_0+BK) \otimes (A_0+BK) + C_p \Big) \epsilon_k + C_p \zeta_k + \omega_k ,
    \label{eq:err_cov_dynamics}
\end{equation}
where $\epsilon_k=\vec\left(\cov(e_k) \right)$, $\zeta_k = \vec \left( z_kz_k^T\right)$, $\omega_k = \vec \left( \cov(w_k)\right)$ and  $C_p=\mathbb{E}[\bar{A}(p_k)\otimes\bar{A}(p_k)]$.
\end{theorem}
\paragraph*{Proof} From (\ref{eq:e}) and Proposition \ref{Prop:1}, it follows
 \begin{align*}
\cov(e_{k+1}) & =\mathbb{E}\Big[\left( \left(A_0+BK\right)e_k+\bar{A}(p_k)x_k+w_k\right) \\ 
& \hspace{0.4cm} \cdot ( \left(A_0+BK\right)e_k+ \bar{A}(p_k)x_k +w_k)^T \Big]\\ 
& = \left(A_0+BK\right) \mathbb{E}[e_ke_k^T] \left(A_0+BK\right)^T\nonumber\\
& \hspace{0.4cm} +\mathbb{E}[\bar{A}(p_k)x_kx_k^T\bar{A}(p_k)^T] + \mathbb{E}[w_kw_k^T],
\end{align*}
since $\bar{A}(p_k)$ and $w_k$ are mutually independent from $x_k$ and $e_k$. The first term is dependent on the error covariance $\cov(e_k) = \mathbb{E}[e_ke_k^T]$, while the second one, resulting from the presence of uncertain parameters, is given by
\begin{align}
&\mathbb{E}[\bar{A}(p_k)x_kx_k^T \! \bar{A}(p_k)^T \! ] \! = \! \mathbb{E}[\bar{A}(p_k)(e_k \! + \! z_k \! )(e_k \! + \! z_k \! )^T \! \bar{A}(p_k)^T  \! ] \nonumber \\
& \hspace{0.25cm} = \mathbb{E}[\bar{A}(p_k)e_ke_k^T\bar{A}(p_k)^T]+\mathbb{E}[\bar{A}(p_k)z_kz_k^T\bar{A}(p_k)^T] \nonumber \\
& \hspace{0.25cm}  + \mathbb{E}[\bar{A}(p_k)e_kz_k^T\bar{A}(p_k)^T] +\mathbb{E}[\bar{A}(p_k)z_ke_k^T\bar{A}(p_k)^T]. \label{Eq:cov_p}
\end{align}
By using Property~\ref{Property_Kronecker} on the different terms of (\ref{Eq:cov_p}), from the linearity of the vectorization operator, and the fact that the expectation of a matrix is the matrix of expectations, implying that the vectorization operator and the expectation can commute, we obtain:
\begin{align*}
   &\vec \left( \mathbb{E}[\bar{A}(p_k)x_kx_k^T\bar{A}(p_k)^T]\right) = \vec \Bigl( \mathbb{E}[\bar{A}(p_k)e_ke_k^T\bar{A}(p_k)^T]  \nonumber\\
    &\! + \! \mathbb{E}[\bar{A}(p_k)z_kz_k^T \!\!\bar{A}(p_k)^T] \! + \! \mathbb{E}[\bar{A}(p_k)e_kz_k^T \!\! \bar{A}(p_k)^T ] \! \nonumber\\
    & \! + \! \mathbb{E}[\bar{A}(p_k)z_ke_k^T\bar{A}(p_k)^T] \!\Bigr) \nonumber\\
&  \!  =  \! \vec \! \left( \mathbb{E}[\bar{A}(p_k)e_ke_k^T\bar{A}(p_k)^T]\right) \!  +  \! \vec \! \left(\mathbb{E}[\bar{A}(p_k)z_kz_k^T\bar{A}(p_k)^T] \right) \nonumber\\ 
&  \! +  \! \vec  \! \left( \mathbb{E}[\bar{A}(p_k)e_kz_k^T\bar{A}(p_k)^T]\right)  \! +  \! \vec  \! \left(\mathbb{E}[\bar{A}(p_k)z_ke_k^T\bar{A}(p_k)^T] \right) \! . \nonumber
\end{align*}
From Proposition~\ref{Prop:1} and Property~\ref{Property_Kronecker} it follows 
\begin{align}
& \vec  \! \left(\mathbb{E}[\bar{A}(p_k)e_kz_k^T\bar{A}(p_k)^T] \right)  \! =  \! \mathbb{E}\left[\vec \! \left( \bar{A}(p_k)e_kz_k^T\bar{A}(p_k)^T\right)\right] \nonumber \\
& \! = \! \mathbb{E}[\left( \! \bar{A}(p_k) \! \otimes \! \bar{A}(p_k) \right) \!\vec \!\left(e_kz_k^T \right)] \!\nonumber\\
& = \! \mathbb{E}[\left( \! \bar{A}(p_k) \! \otimes \! \bar{A}(p_k) \right) ] \vec \! \left(\mathbb{E}[e_kz_k^T ]\right) \nonumber\\
& \!= \! \mathbb{E}[\left(\bar{A}(p_k) \otimes \bar{A}(p_k) \right)] \vec \left(\mathbb{E}[e_k]z_k^T\right) = 0.\label{Eq:Proof_kron}
\end{align}
Analogous results hold for the term $\mathbb{E}[\bar{A}(p_k)z_ke_k^T\bar{A}(p_k)^T]$, and hence, following the same steps as in (\ref{Eq:Proof_kron}), one has: 
\begin{align}
&  \vec \left( \mathbb{E}[\bar{A}(p_k)x_kx_k^T\bar{A}(p_k)^T]\right) = \nonumber \\ 
&\hspace{0.5cm} \mathbb{E}[\bar{A} (p_k)\otimes \bar{A} (p_k)] \vec \left( \mathbb{E}[e_ke_k^T]\right) \nonumber \\ 
&\hspace{0.5cm}+ \mathbb{E}[\bar{A} (p_k)\otimes \bar{A} (p_k) ]\vec \left(z_kz_k^T\right) ,       \label{Eq:penultimate_proof}
\end{align}
$z_k$ being deterministic. Finally, from  (\ref{Eq:penultimate_proof}) it follows equation (\ref{Eq:cov_dyn_w}). 

By defining $\epsilon_k = \vec(\cov(e_k)) \in \R^{n^2}\!\!, \ \zeta_k = \vec((z_k z_k^T)) \in \R^{n^2}\!\!$, $ \omega_k = \vec(\cov(w_k)) \in \R^{n^2}$ and $C_p = \mathbb{E}[\bar{A}(p_k)\otimes\bar{A}(p_k)]$, equation (\ref{eq:err_cov_dynamics}) follows directly.
\endproof

Theorem~\ref{th:1} is therefore a generalization of the covariance dynamics already presented in the literature, for example in \cite{KofmanAUT12,Fiacchini2021}, which considered only additive disturbances. It shows thereby that the covariance evolves like a linear controlled system whose dynamics is affected by uncertain parameters through the specific matrix $C_p = \mathbb{E}[\bar{A}(p_k)\otimes\bar{A}(p_k)]$.

\begin{remark}
    The matrix $C_p=\mathbb{E}[\bar{A}(p_k)\otimes\bar{A}(p_k)]$ is constant, since it contains the parameters variances $\mathbb{E}[p_{ik}^2], \quad i=1,\cdots,l$ as well as their mutual covariances $\mathbb{E}[p_{ik}p_{jk}], \quad  i,j=1,\cdots,l$ with $i \neq j $. Therefore, this matrix is a representation of the parameters covariance matrix with a different structure. 
\end{remark}
The following corollary provides the limit of the error covariance if system~(\ref{eq:err_cov_dynamics}) is asymptotically stable. 
\begin{corollary}\label{Cor:Stability}
Define $M$ as follows 
\begin{equation*}
M = \Big( (A_0+BK) \otimes (A_0+BK) + C_p \Big).
\end{equation*} 
assume that $K$ is such that $\rho\left( M \right) < 1$, and let Assumption~\ref{Ass:exp_stab} hold. Then the covariance matrix of $e_k$ corresponding to system~(\ref{Eq:sys_dyn}) converges to the matrix $\vec^{-1}\left(\left( I-M\right)^{-1} \vec\left(W\right)\right)$.
\end{corollary}

\section{Covariance control design}
The following properties will be used to derive an LMI condition for the design of the stabilizing gain $K$ for the matrix $M$.
\begin{property} \textnormal{(Fact 5.12.2., page 333 in \cite{Bernstein2009})} Given matrices $A,B \in \mathbb{R}^{n \times n }$:
\begin{equation*}
     \rho (A+B) \leq \sigma_{max}(A+B) \leq \sigma_{max}(A) + \sigma_{max}(B).   
\end{equation*}
    \label{Property_rho_sigma}
\end{property}
\begin{property} \textnormal{(Proposition 7.1.6., page 400 in \cite{Bernstein2009})}\\
    Let $A \in \mathbb{R}^{n \times m}, B \in \mathbb{R}^{l \times k }, C \in \mathbb{R}^{m \times q}$ and $ D \in \mathbb{R}^{k \times p}  $, then:
\begin{equation*}
    \left(A \otimes B \right)\left(C \otimes D\right) = AC \otimes BD.
\end{equation*}
\label{Property_Kronecker}
\end{property}
\begin{property} \textnormal{(Proposition 7.1.10., page 401 in \cite{Bernstein2009})}\\
    Let $A \in \mathbb{R}^{n \times n}$ and $B \in \mathbb{R}^{m \times m}$, then:
    \begin{equation*}
        \textnormal{mspec}(A \otimes B)= \{ \lambda \mu: \: \: \lambda \in \textnormal{mspec(A)}, \mu \in \textnormal{mspec}(B)\}_{\textnormal{ms}}.
    \end{equation*}
    \label{Property_mspec_Kronecker}
\end{property}
The following theorem presents a sufficient condition for the Schur stability of the matrix $(A_0+BK) \otimes (A_0+BK) + C_p$, ensuring the asymptotic stability of the covariance dynamics in presence of the stochastic parametric uncertainties, as mentioned in Corollary~\ref{Cor:Stability}, and allowing to design the stabilizing gain $K$.
\begin{theorem}\label{LMI_condition}
    Given $A_0 \in \mathbb{R}^{n \times n }$, $B\in \mathbb{R}^{n \times m}$ and $C_p \in \mathbb{R}^{n^2 \times n^2}$, if $K$ is such that as the following holds: 
\begin{equation*}
    \begin{bmatrix}
    (1 - \sigma_{max}(C_p)) I \ \  & (A_0+BK)^T\\
    (A_0+BK) & I 
    \end{bmatrix} \succ 0,
\end{equation*}
then $K$ is such that $\Big( (A_0+BK) \otimes (A_0+BK) + C_p \Big)$ is Schur stable.
\end{theorem}

\paragraph*{Proof}
Consider $A_K=A_0+BK$, using Property~\ref{Property_rho_sigma} on the matrix $A_K \otimes A_K + C_p $, we have:
\begin{equation*}
    \rho \left(A_K \otimes A_K + C_p  \right) \leq \sigma_{max} \left( A_K \otimes A_K\right) +\sigma_{max} \left(C_p\right).
\end{equation*}
Therefore, $\sigma_{max} \left( A_K \otimes A_K\right) +\sigma_{max} \left(C_p\right) < 1$ implies that $\rho \left(A_K \otimes A_K + C_p  \right) < 1$, and then, in order to impose that $A_K \otimes A_K + C_p $ is Schur stable, it is sufficient to impose that
\begin{equation}
    \sigma_{max} \left( A_K \otimes A_K\right) < 1 - \sigma_{max} \left(C_p\right),
    \label{eq:conservative_bound}
\end{equation}
which is equivalent to:
\begin{equation}
    \lambda_{max} \left( \left( A_K^T \otimes A_K^T\right) \left( A_K \otimes A_K\right)\right)  < \left(1 - \sigma_{max} \left(C_p\right)\right)^2.
    \label{Eq:Lambda_max_Condition}
\end{equation}
By using Property~\ref{Property_Kronecker} and considering $\beta = 1 - \sigma_{max} \left(C_p\right)$, (\ref{Eq:Lambda_max_Condition}) is equivalent to: 
\begin{equation*}
        \lambda_{max} \left( \left( A_K^T A_K\right)  \otimes \left(A_K^T   A_K\right)\right)  < \beta^2,
\end{equation*}
which is equivalent to $\lambda_{max}^2 \left( A_K^T A_K\right)  < \beta^2$ (by using Property~\ref{Property_mspec_Kronecker}), and to $\lambda_{max} \left( A_K^T A_K\right)  < \beta$, that leads to the following:
\begin{equation*}
A_K^T A_K \prec \beta I, 
\end{equation*}
which, by using the Schur complement, is equivalent to: 
\begin{equation*}
\begin{bmatrix} 
    \beta I  & (A_0+BK)^T\\
(A_0+BK) & I 
\end{bmatrix} \succ 0.
\end{equation*}
\endproof
Note that the condition provided by Theorem~\ref{LMI_condition} might be conservative because of the bound in (\ref{eq:conservative_bound}). The conditions provided in \cite{HOSOE2019} are necessary and sufficient for the control design related to system~(\ref{Eq:sys_dyn}), the dimension of these conditions is  $(n^2(n+m)+n )\times (n^2(n+m)+n)$. Although the condition provided in Theorem~\ref{LMI_condition} is sufficient and might be more conservative, it offers the possibility of having a lower dimensional condition ($2n \times 2n$) for a systematic design of $K$, in presence of unbounded stochastic parametric uncertainties, for the matrix $M$ involving quadratic terms of $K$. 

The next section presents a numerical example assessing the exact characterization of the error covariance dynamics and the design of the stabilizing gain $K$.
\section{Numerical example}
Consider the following dynamical system: 
\begin{equation}\label{Ex:sys1}
x_{k+1} = 
\begin{pmatrix}
1.2+p_{1k} & 0.1+p_{2k} \\
p_{3k} & 0.1+p_{4k}
\end{pmatrix}x_k+
\begin{pmatrix}
1 \\
1
\end{pmatrix}u_k+w_k,
\end{equation}
where the covariance of $w_k$ is $\mathbb{E}[w_kw_k^T]=I_{n}$. Note that the matrices $A_0$ and $\bar{A}(p_k)$, with $p_k=\left( p_{1k},p_{2k},p_{3k},p_{4k}\right)^T$ are defined as follows:
\begin{equation*}
    A_0= 
        \begin{pmatrix}
1.2 & 0.1 \\
0 & 0.1
\end{pmatrix},
\qquad
    \bar{A}(p_k)= 
        \begin{pmatrix}
p_{1k} & p_{2k} \\
p_{3k} & p_{4k}
\end{pmatrix}.
\end{equation*}
The parameter vector $p_k$ follows a multivariate normal distribution with zero mean and a covariance matrix $\Sigma$, i.e. $p_k \: \mathtt{\sim} \: \mathcal{N}\left(0, \Sigma \right)$, where
\begin{equation*}
\Sigma=
            \begin{pmatrix}
    7.88&    7.40&   7.43  &  8.17 \\
    7.40&    15.70&    13.91 &   14.24\\
    7.43&    13.91&    12.92 &   12.68\\
    8.17&  14.24  &    12.68 &   13.59
\end{pmatrix} \cdot 0.01,
\end{equation*}
resulting in the following matrix $C_p$: 
\begin{equation}\label{eq:cp1}
C_p =
\begin{pmatrix}
    7.88&    7.40&    7.40 &   15.70\\
    7.43&    8.17&    13.91 &   14.24\\
    7.43&    13.91&    8.17 &   14.24\\
    12.92&    12.68&    12.68 &   13.59
\end{pmatrix} \cdot 0.01.
\end{equation}
By solving the LMI condition in (\ref{LMI_condition}), we can obtain $K=(-0.6\  -0.1)^T$ stabilizing the matrix $M$. 

We compute the evolution of the vectorization of the error covariance using the difference equation in~(\ref{eq:err_cov_dynamics}), as well as the empirical covariance based on $N=1000$ trials. We denote by $\e_{ij}^{th}$ and $\e_{ij}^{em}$, respectively, the theoretical and the empirical  elements of the error covariance matrix, for $i,j\in \{1,2\}$. Fig.~\ref{fig:ex_sys1} shows that the empirical error covariance matches the theoretical one. Furthermore, they both converge to the following matrix:
\begin{equation*}
    \vec^{-1}\left(\left( I-M\right)^{-1} \vec \left(W\right)\right) =
    \begin{pmatrix}
2.33 & -0.42 \\
-0.42 & 2.35
\end{pmatrix}.
\label{Ex:cover_err}
\end{equation*}
Note that in this example, and for simulation purposes, the control $v_k$ is considered as a state feedback of the form $v_k = F z_k$ , where $F$ is designed to make $A_0 + B F$ Schur. In the case of a stochastic MPC implementation, $v_k$ should be designed by a deterministic MPC strategy. 
\begin{figure}
    \centering 
\includegraphics[width=1\linewidth]{Fig.tikz}
    \caption{Theoretical and empirical error covariance evolution related to system~(\ref{Ex:sys1}) with $C_p$ as in (\ref{eq:cp1}).}
    \label{fig:ex_sys1}
\end{figure}

\section{Conclusion}
In this paper, we provide an exact characterization of the dynamics of the error covariance for discrete-time linear systems under potentially unbounded additive and parametric uncertainties and present an LMI-based condition for the stability of these dynamics. The presented characterization is useful in the context of stochastic tube-based MPC approaches as well as in stochastic invariance problems. The proposed numerical example
shows that the theoretical and the empirical error covariance converge to the same matrix when the stability conditions are satisfied. Future works would focus on using this characterization to design stochastic invariant sets and SMPC strategies.
\begin{ack}                               % Place acknowledgements
This work was supported in part by the Clinical project, funded by the ANR under grant ANR-24-CE45-4255, in part by the FMJH Program Gaspard Monge for optimization and operations research and their interactions with data science and in part by the LabEx PERSYVAL-Lab funded by the French Program Investissement d’avenir under Grant ANR-11-LABX-0025-01
\end{ack}

\bibliographystyle{plain}        % Include this if you use bibtex 
\bibliography{Biblio}           % 

% \appendix
% \section{Appendix}    

\end{document}