\begin{figure*}[t]
    \centering
    \includegraphics[width=\linewidth]{figures/metric_intro.pdf}
    \caption{Illustration of the proposed categories of step-by-step reasoning evaluation criteria, \textit{i.e.} groundedness, validity, coherence, and utility. 
    The left shows an example of a query and a reasoning trace. The other four blocks demonstrate examples that fail to suffice the respective metric. Red filled rectangles indicate the error's location, and the outlined boxes and arrows show the cause of the error.}
    \label{fig:metric_intro}
\end{figure*}

\section{Taxonomy}
\label{sec:taxonomy}

% \newcommand{\ResultEightByEightCrossbarOverheadkGE}{13.1}
\newcommand{\ResultEightByEightCrossbarOverheadPercent}{9}
\newcommand{\ResultSixteenBySixteenCrossbarOverheadkGE}{45.4}
\newcommand{\ResultSixteenBySixteenCrossbarOverheadPercent}{12}
\newcommand{\ResultAsymptoticOverheadPercent}{21.6}
\newcommand{\ResultSixteenBySixteenCrossbarFrequencyOverheadPercent}{6}
\newcommand{\ResultThirtyTwoClusterEightKiBParallelFraction}{97}
\newcommand{\ResultThirtyTwoClusterTwoKiBSpeedup}{13.5}
\newcommand{\ResultThirtyTwoClusterThirtyTwoKiBSpeedup}{16.2}
\newcommand{\ResultThirtyTwoClusterGeometricMeanSpeedup}{5.6}
\newcommand{\ResultBaselineTileNOperationalIntensity}{1.9}
\newcommand{\ResultBaselineTileNPerformanceGFLOPS}{114.4}
\newcommand{\ResultBaselineTileNPerformancePercentage}{92}
\newcommand{\ResultHybridTileNOperationalIntensityIncrease}{3.7}
\newcommand{\ResultHybridTileNPerformanceIncrease}{2.6}
\newcommand{\ResultMulticastTileNOperationalIntensityIncrease}{16.5}
\newcommand{\ResultMulticastTileNPerformanceIncrease}{3.4}
\newcommand{\ResultMulticastTileNPerformanceIncreaseOverHybridPercentage}{29}
\newcommand{\ResultMulticastTileNPerformanceGFLOPS}{391.4}
 -> appendix

This section aims to provide a clear taxonomy of criteria for evaluating step-by-step reasoning. Existing criteria can be seen as falling into one of the four categories, namely \textbf{Groundedness}, \textbf{Validity}, \textbf{Coherence}, and \textbf{Utility}. These definitions are \textit{independent} (aim at different objectives -- Section \ref{sec:comparison-ours}), but \textit{not mutually exclusive} (a step can fail to suffice multiple criteria at once).

\subsection{Groundedness}

\textbf{Groundedness} evaluates if the \textit{step is factually true} according to the query \citep{NEURIPS2020_6b493230, gao2024retrievalaugmentedgenerationlargelanguage}. A step can be ungrounded to any part of the query, \textit{e.g.} the question (Figure \ref{fig:metric_intro}-Groundedness) or evidence (\textit{e.g.} falsely stating that \textit{Buddy Rich was born in Chicago}, where the retrieved document states that he was born in New York). 

\subsection{Validity}

\textbf{Validity} evaluates if a reasoning step contains no errors.

The validity of a reasoning step can be defined in terms of \textit{entailment} \citep{bowman-etal-2015-large}, which is widely accepted in factual/commonsense reasoning. Under this definition, a step is considered valid if it can be directly entailed from previous steps \citep{tafjord-etal-2021-proofwriter, dalvi-etal-2021-explaining, PrOntoQA} or at least does not contradict them \citep{DBLP:conf/iclr/GolovnevaCPCZFC23, prasad-etal-2023-receval, zhu2024deductivebeamsearchdecoding}.

The notion of validity often used in symbolic tasks is \textit{correctness}, \textit{e.g.} performing accurate calculations in math reasoning \citep{DBLP:conf/iclr/LightmanKBEBLLS24, jacovi-etal-2024-chain, zheng2024processbenchidentifyingprocesserrors} or inferring the correct logical conclusion based on the provided premises \citep{wu2024cofcastepwisecounterfactualmultihop, jacovi-etal-2024-chain, song2025prmbenchfinegrainedchallengingbenchmark}.

% While early works do not identify the type of fallacies by applying binary or ternary label, recent works tend to include fine-grained error types \citep{song2025prmbenchfinegrainedchallengingbenchmark} or human-written explanations \citep{jacovi-etal-2024-chain} to improve the explainability of the evaluation process.

% Tyen(2024) Direct mistake prompting

\subsection{Coherence}
\label{sec:coherence}

\textbf{Coherence} measures if a reasoning step's \textit{preconditions are satisfied} by the previous steps \citep{wang-etal-2023-towards}. For instance, if a trace includes the reasoning step \textit{"Next, we add 42 to 16."} but the origin of the value 42 was never explained in the previous steps, this step is considered incoherent. An intuitive way to obtain an incoherent trace is randomly shuffling a coherent trace \citep{wang-etal-2023-towards, nguyen-etal-2024-direct}, as the premise of some steps will not appear anywhere in the previous steps even though it can be eventually deduced (\textit{valid}).

Note that coherence judgment is inherently subjective and pragmatic compared to other criteria. For instance, seemingly trivial steps like \textit{"A part of something is present in that something"} in WorldTree V2 \citep{xie-etal-2020-worldtree} is annotated as necessary in \citet{dalvi-etal-2021-explaining} but not necessary in \citet{Ott_2023}.

% The same concept is also referred to as \textit{broad validity} (as opposed to strict validity) \citep{PrOntoQA} and \textit{prerequisite sensitivity} \citep{song2025prmbenchfinegrainedchallengingbenchmark}.

% \textbf{Symbolic solutions.} \hspace{0.1cm} Coherence can be clearly defined in \textit{symbolic} reasoning tasks, where the dependency between steps can be symbolically defined. \citep{PrOntoQA} defined steps that require applying two inference rules as valid but incoherent (\textit{broadly valid}). However, as they used a simple synthetic dataset that only includes \textit{Modus ponens}, observed instances of broadly valid steps can be well seen as coherent in common sense. \citet{nguyen-etal-2024-direct}, where a reasoning trace corresponds to a directed path in knowledge graphs (KGs), defines a coherent ERU as a directed KG edge where its source node was already introduced as a target node.

\subsection{Utility}
\label{sec:utility}

\textbf{Utility} measures whether a reasoning step contributes to getting the correct final answer (\textit{answer correctness}).

One interpretation of utility is \textit{progress}, or whether the step is correctly following the ground truth solution. For instance, in Game of 24 (making the number 24 using 4 natural numbers and basic arithmetic operations) \citep{NEURIPS2023_271db992}, a solution can be defined as a sequence of operations (\textit{e.g.} 5+7=12$\rightarrow$12-6=6$\rightarrow$6*4=24.). In this task, the utility of a step (making $5+7=12$ from $5$ and $7$) can be directly assessed by checking if it is a part of a correct solution.
% \footnote{Value function is also often described as \textit{progress} under the policy (\textit{i.e.} LLM) \citep{setlur2024rewardingprogressscalingautomated}. However, for clarity, we restrict the term to \textit{discrete} progress.}

Utility can also be interpreted as \textit{value function} (estimated reward), which is proportional to the probability of reaching the correct answer starting from the step \citep{hao-etal-2023-reasoning, wang-etal-2024-math, xie2024montecarlotreesearch, chen-etal-2023-rev}. This black-box interpretation of utility offers high scalability as it only requires the gold answer, without any human annotation or ground-truth solutions \citep{wang-etal-2024-math, lai2024stepdpostepwisepreferenceoptimization}.