\section{Related work}
\label{approaches}
A previous article \cite{46} discussed the role of use cases
in requirements and contrasted them with object-oriented requirements. The present paper extends that original discussion to a full-fledged requirements engineering method. 

A number of requirements approaches share at least some of the objectives of UOOR. The field is a very broad one, with hundreds of proposals. We identified 15 well-documented methods which lend themselves to a point-by-point comparison based on the criteria discussed in section~\ref{characteristics}. Table~\ref{tab:related_work} is an overview of the results.

%None of the reviewed approaches satisfies all of the criteria identified in section~\ref{characteristics}.


NL-based requirements \cite{wiegers, 17, RELAX} are requirements formulated in the form of unrestricted \acrshort{nl} text, or \acrshort{nl} text, restricted in a certain way. NL-based requirements are easy to learn and are supported by a wide variety of tools and education materials. Scenarios (use cases \cite{43}, user stories \cite{14}, and use cases 2.0 \cite{25, 26}) are a powerful requirements technique. Still, they cannot serve as a requirements methodology. NL-based requirements, including scenarios, are prone to ambiguity, which can be partially eliminated by constraining the natural language. Requirements traceability relies on manually created traceability links. Requirements are reused by copy-pasting.

Use cases are an important modeling tool in UML \cite{55}. UML makes it possible to treat use cases as objects, subject to specialization and decomposition. UML use cases can have pre- and postconditions. It is possible in UML to associate contracts with individual operations through  natural language or the OCL (Object Constraint Language) notation. SysML \cite{47}, an extended profile of UML, treats requirements as first-class entities, establishing direct links between requirements and other software artifacts (such as tests). \cite{58} illustrates the requirements specification process with SysML, and \cite{5, 52, 57} provide applications of SysML to all phases of software development. SysML does not provide semantics for requirements, although it is possible to associate contracts with individual operations through natural language or the OCL notation. SysML and UML are standardized notations and not methodologies. 

The Restricted Use Case Modeling approach \cite{62} relies on a use case template and a set of restriction rules to reduce the ambiguity of use case specification and facilitate the transition to analysis models, such as UML class diagram and sequence diagram. The aToucan tool automates the generation of UML class, sequence, and activity diagrams \cite{61}. The tool can generate traceability links from the textual use cases to the generated class diagram, but not to the source code. The approach does not advocate extracting abstract properties from use cases and domain knowledge, such as time-ordering and environmental constraints. 

A Use Case Map (UCM) \cite{3,12} depicts several scenarios simultaneously.
UCMs represent use cases as causal sequences of responsibilities, possibly over a set of abstract components. 
In UCMs, pre- and postconditions of use cases, as well as conditions at selection points, can be modeled with formal specification techniques such as ASM or LOTOS.  
UCMs specify properties of operations in relation to scenario sequences, rather than abstract properties of objects and operations. Use case maps do not provide a framework for requirements traceability and reuse.

Object-Oriented Analysis and Design (OOAD) \cite{8} is a unified methodology for use-case-driven analysis and design, supported by UML \cite{55} as a unified notation. OOAD applies OO techniques (class-based decomposition, OO modeling) to the initial requirements, produced at the earlier stages of the development process. OOAD does not provide a framework for requirements traceability and reuse.

The OO-Method \cite{80} combines conventional OO specification techniques \cite{8} with formal specification, relying on the OASIS object-oriented specification language \cite{lopez1995oasis}. 
The Integranova tool supports the specification process with an interactive interface and automatically generates the implementation code. However, this approach does not provide a framework for requirements traceability and reuse.

In goal-oriented requirements engineering  \cite{78}, \cite{60} requirements are obtained through a series of refinements of high-level goals. With the help of the Objectiver tool \cite{44},
requirements can be linked to other artifacts, such as goals, environment agents, or operations. However, traceability links to natural language requirements documents or implementation artifacts are out of the scope of the approach.

In test-driven development (TDD) \cite{66}, a software engineer writes unit tests before implementing the system’s functionality in small iterations. Unit tests can be viewed as the means of capturing requirements: tests serve as a guide to code writing. In behavior-driven development (BDD) \cite{75, 86}, requirements are formulated as user stories, following a specific template. Dedicated tools transform user stories into parameterized unit tests. TDD and BDD rely on scenarios, which are not abstract enough to be \textit{requirements}: if scenarios attempt to cover \textit{all} possible situations, their number explodes, which impedes requirements traceability. BDD and TDD do not provide mechanisms for requirements reusability and static verification.
   
In the ACL/VF framework \cite{6, 64}, use cases capture requirements, which are further formalized as grammars of responsibilities. Another Contract Language (ACL) contracts (pre- and postconditions and invariants) specify constraints, which scenarios' or responsibilities' execution poses on the system's state. In this approach, the requirements model is decoupled from the candidate implementation: a dedicated binding tool maps elements of the requirements model to the elements of candidate implementation. The approach requires a substantial background: familiarity with design by contract, ACL, and formal grammars. The approach is not seamless and does not provide a framework for requirements reuse.

The Multirequirements approach \cite{32} suggests using a single notation (Eiffel programming language) for requirements, design, and implementation. Requirements are formulated in 3 interconnected layers: natural language, software contracts in programming language, and diagrams. The approach does not provide a methodology and a framework for requirements traceability and reuse.

The PEGS approach attempts to provide a definition and taxonomy of requirements. According to this approach, requirements pertain to a Project intended, in a certain Environment, to achieve some Goals by building a System. Thus, requirements specification consists of four books: Project, Environment, Goals, and Systems, which correspond to each of these components \cite{Handbook}. The approach provides principles and techniques of requirements specification, including seamless OO specification, yet does not provide an explicit methodology. 

The SIRCOD approach \cite{galinier2021seamless} provides a pipeline for converting natural language requirements to programming language contracts. The approach relies on the domain specific language, RSML, for automating conversion from natural language to programming language. In the Seamless Object-Oriented Requirements approach (SOOR), requirements are documented as software classes, which makes them verifiable and reusable \cite{39}. Routines of those classes, called specification drivers, take objects to be specified as arguments and express the effect of operations on those objects with pre- and postconditions. The SIRCOD and SOOR approaches focus on translating existing requirements specifications to contracts expressed in a programming language, rather than extracting abstract requirements from scenarios.

The UOOR method relies on the advancements of the SIRCOD and SOOR approaches but focuses more on the approach's usability and requirements traceability management. 

%\begin{landscape}
\begingroup
%\renewcommand{\arraystretch}{1.3} % Default value: 1
\setlength\tabcolsep{0.08cm}.

\begin{table*}
\small 
%\resizebox{\textwidth}{!}
\centering
    \begin{tabular}[M]{|p{2.5cm} | p{1.7cm}| p{1.5cm}| p{1.0cm}| p{1.7cm}| p{1.7cm} |p{1.7cm}| p{1.5cm}| p{1.7cm}|}
    \hline
& Methodology & Required \newline background & Tool \newline support & Requirements reusability & Requirements  \newline verifiability & Requirements \newline unambiguity & Traceability \newline support & Seamlessness \\
\hline
NL-based & Yes & Some & Yes & No & No & Partial&No&No \\
UML and SysML & No & Substantial & Yes & No & Partial & Partial & Partial & No \\
Scenarios & No & Some & Yes & No & No & No & No & No \\
RUCM & Yes & Some & Yes & No & No & Partial & Partial & No \\
Use Case Maps & Yes & Substantial & Yes & No & Yes & Yes & No & No \\
OOAD & Yes & Substantial & Yes & No & Partial & Partial & No & No \\
OO-Method & Yes & Some & Yes & No & Yes & Partial & No & No \\
GORE & Yes & Some & Yes & No & No & Yes & Partial & No \\
TDD & Yes & Some & Yes & No & Yes & Yes & Yes & Yes \\ 
BDD & Yes & Some & Yes & No & Yes & Yes & Yes & Yes \\
ACL/VF & Yes & Substantial & No & No & Yes & Yes & Partial & No \\
Multirequirements & No & Some & Partial & No & Yes & Yes & No & Yes \\
SIRCOD & Partial & Some & Yes & No & Yes & Yes & Yes & Yes \\
SOOR & No & Some & Yes & Yes & Yes & Yes & No & Yes \\
PEGS & No & Some & Partial & No & Yes & Yes & No & Yes \\
\hline
    \end{tabular}
    \caption{Summary of related work.}
    \label{tab:related_work}
\end{table*}
\endgroup
%\end{landscape}