\makeatletter 
\@namedef{ver@breakurl.sty}{}
\makeatother

\documentclass[]{jot}
% Use the documentclass option 'lineno' to view line numbers

% Enter the JOT metadata in the following 

\usepackage{multirow}


\jotdetails{
    volume=vv,      % volume
    number=nn,       % number or issue
    articleno=aa,   % article number, eg a1 for research articles, e for editorials
    year=2024,      % year
    license=ccbyncnd    % choose from ccby, ccbynd, ccbyncnd
}

%\usepackage[utf8]{inputenc}
%\usepackage[T1]{fontenc}
\usepackage[english]{babel}
\usepackage{microtype} % optional, for aesthetics
\usepackage{tabularx} % nice to have
\usepackage{booktabs} % necessary for style

\newcommand{\e}[1]{\mbox{\lstinline[basicstyle=\normalsize,language=OOSC2Eiffel]|#1|}}
\newcommand{\es}[1]{\mbox{\lstinline[basicstyle=\scriptsize]|#1|}}
\newcommand{\et}[1]{\mbox{\lstinline[basicstyle=\tiny]|#1|}}
\newcommand{\etd}[1]{\mbox{\lstinline[basicstyle=\tiny]|#1|}}
\usepackage{boogie}
\usepackage{eiffel}
\usepackage{listings}
\usepackage{enumitem}
%\usepackage{hyperref}
\usepackage{tabto}
\usepackage{float}
\usepackage{graphicx}
\graphicspath{{./figures/}}
\usepackage{longtable}
\usepackage{tcolorbox}
\usepackage[framemethod=TikZ]{mdframed}
\usepackage{setspace}
%\usepackage{lipsum}
\usepackage{xurl}
\mdfdefinestyle{MyFrame}{%
        linecolor=grey!50!white,
        outerlinewidth=2pt,
        roundcorner=20pt,        innertopmargin=6pt,%\baselineskip   
        innerbottommargin=6pt,%\baselineskip
        innerrightmargin=20pt,
        innerleftmargin=20pt,
        backgroundcolor=white}
    
    %------------------- Comments -----------------------
\usepackage{color}      % for comments
\usepackage{amssymb}    % for comments
\newcommand{\mynote}[3][black]{\textcolor{#1}{\fbox{\bfseries\sffamily\scriptsize{#2}}
{\small$\blacktriangleright$\textsf{\emph{#3}}$\blacktriangleleft$}}}
\newcommand{\bm}[1]{\mynote[brown]{Bertrand}{#1}}
\newcommand{\jmb}[1]{\mynote[blue]{JMB}{#1}}
\newcommand{\an}[1]{\mynote[violet]{Sasha}{#1}}
\newcommand{\mb}[1]{\mynote[orange]{Maria}{#1}}
\newcommand{\se}[1]{\mynote[purple]{Sophie}{#1}}
\newcommand{\TODO}[1]{\mynote[red]{TODO}{#1}}
    \renewcommand{\TODO}[1]{} % to remove all the ToDos

\newcommand{\command}[1]{{\color{codepurple}\texttt{\textbackslash #1}}}
\newcommand{\param}[1]{{\color{blue}\texttt{#1}}}

% Trying to balance final page columns
\usepackage{flushend}


%%%%%%%%%%%%%%%%%%%packages from thesis
\usepackage{float}
\usepackage{flafter}
\usepackage{afterpage}
\renewcommand{\textfraction}{0.01}

\usepackage{comment}
\usepackage{array}
\usepackage{amsthm}

%\usepackage{multirow}
\usepackage{lscape}
% Urls
\usepackage{url}

% For figures
\usepackage{xcolor}
\usepackage{caption}
\usepackage{subcaption}
\captionsetup[lstlisting]{box=colorbox,boxcolor=white,font={color=black}}
%%% Really basic packages here to handle languages
\usepackage[english]{babel}
\usepackage[utf8]{inputenc}
\usepackage[T1]{fontenc}
\usepackage{stfloats} %for floating figures

%%%%%%%%%%%%%%%%%%%%%%%%%%



% Select the article type
\articletype{regular} 

    % {editorial} editorial 
    % {regular} regular contribution
    % {manual} manual
    % {column} column

\title{UOOR: Seamless and Traceable Requirements}

\author[$\ast$]{Maria Naumcheva}
\author[$\ast$]{Sophie Ebersold}
\author[$\ast$]{Jean-Michel Bruel}
\author[$\S$]{Bertrand Meyer}
%\author[$\S$]{Fourth Author}
%\author[$\ast\ast$,2,3]{Sixth Author}

\affil[$\ast$]{IRIT, University of Toulouse, France}
\affil[$\S$]{Constructor Institute of Technology, Schaffhausen, Switzerland and Constructor University Bremen, Germany}
%\affil[$\S$]{Author four affiliation}
%\affil[$\ast\ast$]{Author five affiliation}

\keywords{Software requirements, use cases, scenarios,  scenario-based testing, object-oriented requirements, traceability}

\runningtitle{UOOR: Towards Seamless and Traceable Requirements} % For use in the internal pages 

%% For the footnote.
%% Give the last name of the first author if only one author;
% \runningauthor{FirstAuthorLastname}
%% last names of both authors if there are two authors;
% \runningauthor{FirstAuthorLastname and SecondAuthorLastname}
%% last name of the first author followed by et al, if more than two authors.
\runningauthor{M. Naumcheva, S. Ebersold, J.-M. Bruel, B.
Meyer.}




\newenvironment{itemize*}%
  {
  \vspace{-6pt}
  \begin{itemize}%
    \setlength{\itemsep}{0pt}%
    \setlength{\parskip}{0pt}}%
  {\end{itemize}}

\usepackage[acronym]{glossaries}
%\makenoidxglossaries



\newacronym{nl}{NL}{Natural Language}
\newacronym{uoor}{UOOR}{Unified Object-Oriented Approach for Requirements}
\newacronym{uml}{UML}{Unified Modeling Language}
\newacronym{rsml}{RSML}{Requirements-Specific Modeling Language}
\newacronym{ireb}{IREB}{International Requirements Engineering Board}
\newacronym{babok}{BABOK}{Business Analysis Body of Knowledge}
%\newacronym{re}{RE}{}
\newacronym{srs}{SRS}{software Requirements Specification}
\newacronym{oo}{OO}{Object-Oriented}
\newacronym{sysml}{SysML}{Systems Modeling Language}
\newacronym{ears}{EARS}{Easy Approach to Requirements Syntax}
\newacronym{lms}{LMS}{Library Management System}
\newacronym{ocl}{OCL}{Object Constraint Language}
\newacronym{swebok}{SWEBOK}{Software Engineering Body of Knowledge}
\newacronym{rucm}{RUCM}{Restricted Use Case Modeling}
\newacronym{ucm}{UCM}{Use Case Maps}
\newacronym{asm}{ASM}{Abstract State Machines}
\newacronym{lotos}{LOTOS}{Language of Temporal Ordering Specification}
\newacronym{ooad}{OOAD}{Object-Oriented Analysis and Design}
\newacronym{gore}{GORE}{Goal-Oriented Requirements Engineering}
\newacronym{tdd}{TDD}{Test-Driven Software Development}
\newacronym{bdd}{BDD}{Behavior-Driven Development}
\newacronym{acl}{ACL}{Another Contract Language}
\newacronym{ide}{IDE}{Integrated Development Environment}
\newacronym{soor}{SOOR}{Seamless Object-Oriented Requirements}
\newacronym{soort}{SOORT}{Seamless Object-Oriented Requirement Templates }
\newacronym{eis}{EIS}{Eiffel Information System}
\newacronym{uri}{URI}{Uniform Resource Identifier}
\newacronym{gps}{GPS}{Global Positioning System}
\newacronym{imu}{IMU}{Inertial Measurement Unit}
\newacronym{gnss}{GNSS}{Global Navigation Satellite System}
\newacronym{ecu}{ECU}{Electronic Control Unit}
\newacronym{elem}{ELEM}{Project Element}
\newacronym{cstr}{CSTR}{Constraint}
\newacronym{fr}{FR}{Functional Requirement}
\newacronym{nlrq}{NLRQ}{Natural Language Requirement}
\newacronym{im}{IM}{Implementation Artifact}
\newacronym{sircod}{SIRCOD}{The Seamless Intergration of Requirements in Code}
\newacronym{doors}{DOORS}{Dynamic Object-Oriented Requirements System}
\newacronym{pegs}{PEGS}{Project, Environment, Goals, System}
%\newacronym{}{}{}
%\newacronym{}{}{}

%\newmdtheoremenv{def}{Definition}[chapter]
%https://tex.stackexchange.com/questions/36278/box-around-theorem-statement
%%%%%%%%%%%%%%%%% citations


% Glossary %%%%%%%%%%%%%%%%%%%%%%%%%%%%%%%%%%%%%%

% if you need a glossary
% \usepackage{glossaries}         
% \makeglossaries

% \newglossaryentry{ABF}{name=ABF, description={Adaptive Biasing Force}}

% Use \newglossaryentry{utc}{name=UTC, description={Coordinated Universal Time}} to add a glossary entry within the document
% Use gls{utc} to use that entry somewhere

%%%%%%%%%%%%%%%%%%%%%%%%%%%%%%%%%%%%%%%%%%%%%%%%%%%%%


%%%%%%%%%%%%%%%%%%%%%%%%%%%%%%%%%%%%%%%%%%%%%%%%%%%%%

%\begin{document}
%% Title page
%%% ----------------------------------------------------------------------------
% BIWI SA/MA thesis template
%
% Created 09/29/2006 by Andreas Ess
% Extended 13/02/2009 by Jan Lesniak - jlesniak@vision.ee.ethz.ch
% Updated 16/03/2023 by Danda Pani Paudel - paudel@vision.ee.ethz.ch
%% ----------------------------------------------------------------------------

\begin{titlepage}

\thispagestyle{empty}

\fancypagestyle{empty}{
\lhead{\includegraphics[height=1.5cm]{images/ethlogo_black}}
\renewcommand{\headrulewidth}{0.0pt}
\rhead{\vspace*{-0.2cm}\includegraphics[height=1.4cm]{images/cvl_logo}}
\fancyfoot{}
}



\vspace*{2cm}
\begin{center}
\LARGE{\textbf{NPSim: Nighttime Photorealistic Simulation From Daytime Images With Monocular Inverse Rendering and Ray Tracing
}\\}
% NPSim: Nighttime Photorealistic Simulation From Daytime Images With Monocular Inverse Rendering and Ray Tracing
% \LARGE{\textbf{Subtitle Subtitle Subtitle}\\[1cm]}
\vspace{5pt}
\large{Project Thesis\\[0.8cm]}
\LARGE{Shutong Zhang\\}
\normalsize{Department of Information Technology and Electrical Engineering}
\end{center}

\begin{center}
 


% \begin{center}
% \begin{tabular}{ll}
% \multirow{2}{*}{\includegraphics[height=1cm]{images/biwi_logo}} & Computer Vision Laboratory\\ 
% & ETH Zurich
% \end{tabular}
%  \end{center}

\end{center}


\vfill
\begin{center}
\begin{tabular}{ll}
\Large{\textbf Advisor:} & \Large{Dr.~Christos Sakaridis}\\
\Large{\textbf Supervisor:} & \Large{Prof.~Dr.~Luc Van Gool}\\
% 			    & \small{Computer Vision Laboratory}\\
% 			    & \small{Department of Information Technology and Electrical Engineering}\\
\end{tabular}
\end{center}

\begin{center}
% \today\\
August 18, 2023
\end{center}


\end{titlepage}



% %%% Abstract
%\begin{abstract}  
Test time scaling is currently one of the most active research areas that shows promise after training time scaling has reached its limits.
Deep-thinking (DT) models are a class of recurrent models that can perform easy-to-hard generalization by assigning more compute to harder test samples.
However, due to their inability to determine the complexity of a test sample, DT models have to use a large amount of computation for both easy and hard test samples.
Excessive test time computation is wasteful and can cause the ``overthinking'' problem where more test time computation leads to worse results.
In this paper, we introduce a test time training method for determining the optimal amount of computation needed for each sample during test time.
We also propose Conv-LiGRU, a novel recurrent architecture for efficient and robust visual reasoning. 
Extensive experiments demonstrate that Conv-LiGRU is more stable than DT, effectively mitigates the ``overthinking'' phenomenon, and achieves superior accuracy.
\end{abstract}  

%\input{chapters/remerciements.tex}



% MAIN TEXT %%%%%%%%%%%%%%%%%%%%%%%%%%%%%%%%%%%%%%%%%

%
\def\Lqtrain{\ensuremath{L_q^{\text{\tiny \raisebox{2pt}{train}}}}\xspace}
\def\Lxtrain{\ensuremath{L_x^{\text{\tiny \raisebox{2pt}{train}}}}\xspace}
\def\Lqtest{\ensuremath{L_q^{\text{\tiny \raisebox{2pt}{test}}}}\xspace}
\def\Lxtest{\ensuremath{L_x^{\text{\tiny \raisebox{2pt}{test}}}}\xspace}
\def\Lxasmk{\ensuremath{L^{\text{\tiny \raisebox{2pt}{asmk}}}_x}\xspace}

\newcommand{\ione}{i\hspace{-.05em}+\hspace{-.07em}1}

\newcommand{\mypartight}[1]{\noindent {\bf #1}}
\newcommand{\myparagraph}[1]{\vspace{3pt}\noindent\textbf{#1}\xspace}

\newcommand{\optional}[1]{{#1}}
\newcommand{\alert}[1]{{\color{red}{#1}}}
\newcommand{\gt}[1]{{\color{purple}{GT: #1}}}
\newcommand{\gtt}[1]{{\color{purple}{#1}}}
\newcommand{\gtr}[2]{{\color{purple}\st{#1} {#2}}}

\newcommand{\gkz}[1]{{\color{cyan}{GKZ: #1}}}
\newcommand{\gkzt}[1]{{\color{cyan}{#1}}}
\newcommand{\gkzr}[2]{{\color{cyan}\st{#1} {#2}}}

\newcommand{\ps}[1]{{\color{brown}{PS: #1}}}
\newcommand{\pst}[1]{{\color{brown}{#1}}}
\newcommand{\psr}[2]{{\color{brown}\st{#1} {#2}}}

\newcommand{\am}[1]{{\color{orange}{AM: #1}}}
\newcommand{\amt}[1]{{\color{orange}{#1}}}
\newcommand{\amr}[2]{{\color{orange}\st{#1} {#2}}}

\newcommand{\och}[1]{{\color{blue}{OCh: #1}}}
\newcommand{\ocht}[1]{{\color{blue}{#1}}}
\newcommand{\ochr}[2]{{\color{blue}\st{#1} {#2}}}

\newcommand{\gray}[1]{{\color{gray}{#1}}}

\definecolor{appleblue}{RGB}{0,122,255}

\newcolumntype{Y}{>{\centering\arraybackslash}p{4em}}

\def\roxf{$\mathcal{R}$Oxford\xspace}
\def\rox{$\mathcal{R}$Oxf\xspace}
\def\ro{$\mathcal{R}$O\xspace}
\def\rpar{$\mathcal{R}$Paris\xspace}
\def\rpa{$\mathcal{R}$Par\xspace}
\def\rp{$\mathcal{R}$P\xspace}
\def\rdis{$\mathcal{R}$1M\xspace}

\newcommand\resnet[3]{\ensuremath{\prescript{#2}{}{\mathtt{R}}{#1}_{\scriptscriptstyle #3}}\xspace}

\newcommand{\ours}{\mbox{ILIAS}\xspace} %
\newcommand{\miniours}{\textit{mini}-ILIAS\xspace} %

\newcommand{\stddev}[1]{\scriptsize{$\pm#1$}}

\newcommand{\diffup}[1]{{\color{OliveGreen}{($\uparrow$ #1)}}}
\newcommand{\diffdown}[1]{{\color{BrickRed}{($\downarrow$ #1)}}}

\def\nmsp{\hspace{-6pt}}
\def\nssp{\hspace{-3pt}}
\def\nxssp{\hspace{-1pt}}
\def\zsp{\hspace{0pt}}
\def\xssp{\hspace{1pt}}
\def\ssp{\hspace{3pt}}
\def\msp{\hspace{6pt}}
\def\mlsp{\hspace{9pt}}
\def\lsp{\hspace{12pt}}
\def\xlsp{\hspace{20pt}}

\newcommand{\head}[1]{{\smallskip\noindent\bf #1}}
\newcommand{\equ}[1]{(\ref{equ:#1})\xspace}

\newcommand{\nn}[1]{\ensuremath{\text{NN}_{#1}}\xspace}
\def\l1{\ensuremath{\ell_1}\xspace}
\def\l2{\ensuremath{\ell_2}\xspace}

\newcommand{\tran}{^\top}
\newcommand{\mtran}{^{-\top}}
\newcommand{\zcol}{\mathbf{0}}
\newcommand{\zrow}{\zcol\tran}

\newcommand{\ind}{\mathds{1}}
\newcommand{\expect}{\mathbb{E}}
\newcommand{\nat}{\mathbb{N}}
\newcommand{\zahl}{\mathbb{Z}}
\newcommand{\real}{\mathbb{R}}
\newcommand{\proj}{\mathbb{P}}
\newcommand{\prob}{\mathbf{Pr}}

\newcommand{\mif}{\textrm{if }}
\newcommand{\other}{\textrm{otherwise}}
\newcommand{\minimize}{\textrm{minimize }}
\newcommand{\maximize}{\textrm{maximize }}

\newcommand{\id}{\operatorname{id}}
\newcommand{\const}{\operatorname{const}}
\newcommand{\sgn}{\operatorname{sgn}}
\newcommand{\erf}{\operatorname{erf}}
\newcommand{\var}{\operatorname{Var}}
\newcommand{\mean}{\operatorname{mean}}
\newcommand{\trace}{\operatorname{tr}}
\newcommand{\diag}{\operatorname{diag}}
\newcommand{\vect}{\operatorname{vec}}
\newcommand{\cov}{\operatorname{cov}}

\newcommand{\softmax}{\operatorname{softmax}}
\newcommand{\clip}{\operatorname{clip}}

\newcommand{\defn}{\mathrel{:=}}
\newcommand{\peq}{\mathrel{+\!=}}
\newcommand{\meq}{\mathrel{-\!=}}

\newcommand{\floor}[1]{\left\lfloor{#1}\right\rfloor}
\newcommand{\ceil}[1]{\left\lceil{#1}\right\rceil}
\newcommand{\inner}[1]{\left\langle{#1}\right\rangle}
\newcommand{\norm}[1]{\left\|{#1}\right\|}
\newcommand{\frob}[1]{\norm{#1}_F}
\newcommand{\card}[1]{\left|{#1}\right|\xspace}
\newcommand{\diff}{\mathrm{d}}
\newcommand{\der}[3][]{\frac{d^{#1}#2}{d#3^{#1}}}
\newcommand{\pder}[3][]{\frac{\partial^{#1}{#2}}{\partial{#3^{#1}}}}
\newcommand{\ipder}[3][]{\partial^{#1}{#2}/\partial{#3^{#1}}}
\newcommand{\dder}[3]{\frac{\partial^2{#1}}{\partial{#2}\partial{#3}}}

\newcommand{\wb}[1]{\overline{#1}}
\newcommand{\wt}[1]{\widetilde{#1}}

\newcommand{\cA}{\mathcal{A}}
\newcommand{\cB}{\mathcal{B}}
\newcommand{\cC}{\mathcal{C}}
\newcommand{\cD}{\mathcal{D}}
\newcommand{\cE}{\mathcal{E}}
\newcommand{\cF}{\mathcal{F}}
\newcommand{\cG}{\mathcal{G}}
\newcommand{\cH}{\mathcal{H}}
\newcommand{\cI}{\mathcal{I}}
\newcommand{\cJ}{\mathcal{J}}
\newcommand{\cK}{\mathcal{K}}
\newcommand{\cL}{\mathcal{L}}
\newcommand{\cM}{\mathcal{M}}
\newcommand{\cN}{\mathcal{N}}
\newcommand{\cO}{\mathcal{O}}
\newcommand{\cP}{\mathcal{P}}
\newcommand{\cQ}{\mathcal{Q}}
\newcommand{\cR}{\mathcal{R}}
\newcommand{\cS}{\mathcal{S}}
\newcommand{\cT}{\mathcal{T}}
\newcommand{\cU}{\mathcal{U}}
\newcommand{\cV}{\mathcal{V}}
\newcommand{\cW}{\mathcal{W}}
\newcommand{\cX}{\mathcal{X}}
\newcommand{\cY}{\mathcal{Y}}
\newcommand{\cZ}{\mathcal{Z}}

\newcommand{\vA}{\mathbf{A}}
\newcommand{\vB}{\mathbf{B}}
\newcommand{\vC}{\mathbf{C}}
\newcommand{\vD}{\mathbf{D}}
\newcommand{\vE}{\mathbf{E}}
\newcommand{\vF}{\mathbf{F}}
\newcommand{\vG}{\mathbf{G}}
\newcommand{\vH}{\mathbf{H}}
\newcommand{\vI}{\mathbf{I}}
\newcommand{\vJ}{\mathbf{J}}
\newcommand{\vK}{\mathbf{K}}
\newcommand{\vL}{\mathbf{L}}
\newcommand{\vM}{\mathbf{M}}
\newcommand{\vN}{\mathbf{N}}
\newcommand{\vO}{\mathbf{O}}
\newcommand{\vP}{\mathbf{P}}
\newcommand{\vQ}{\mathbf{Q}}
\newcommand{\vR}{\mathbf{R}}
\newcommand{\vS}{\mathbf{S}}
\newcommand{\vT}{\mathbf{T}}
\newcommand{\vU}{\mathbf{U}}
\newcommand{\vV}{\mathbf{V}}
\newcommand{\vW}{\mathbf{W}}
\newcommand{\vX}{\mathbf{X}}
\newcommand{\vY}{\mathbf{Y}}
\newcommand{\vZ}{\mathbf{Z}}

\newcommand{\va}{\mathbf{a}}
\newcommand{\vb}{\mathbf{b}}
\newcommand{\vc}{\mathbf{c}}
\newcommand{\vd}{\mathbf{d}}
\newcommand{\ve}{\mathbf{e}}
\newcommand{\vf}{\mathbf{f}}
\newcommand{\vg}{\mathbf{g}}
\newcommand{\vh}{\mathbf{h}}
\newcommand{\vi}{\mathbf{i}}
\newcommand{\vj}{\mathbf{j}}
\newcommand{\vk}{\mathbf{k}}
\newcommand{\vl}{\mathbf{l}}
\newcommand{\vm}{\mathbf{m}}
\newcommand{\vn}{\mathbf{n}}
\newcommand{\vo}{\mathbf{o}}
\newcommand{\vp}{\mathbf{p}}
\newcommand{\vq}{\mathbf{q}}
\newcommand{\vr}{\mathbf{r}}
\newcommand{\Vs}{\mathbf{s}}
\newcommand{\vt}{\mathbf{t}}
\newcommand{\vu}{\mathbf{u}}
\newcommand{\vv}{\mathbf{v}}
\newcommand{\vw}{\mathbf{w}}
\newcommand{\vx}{\mathbf{x}}
\newcommand{\vy}{\mathbf{y}}
\newcommand{\vz}{\mathbf{z}}

\newcommand{\vone}{\mathbf{1}}
\newcommand{\vzero}{\mathbf{0}}

\newcommand{\valpha}{{\boldsymbol{\alpha}}}
\newcommand{\vbeta}{{\boldsymbol{\beta}}}
\newcommand{\vgamma}{{\boldsymbol{\gamma}}}
\newcommand{\vdelta}{{\boldsymbol{\delta}}}
\newcommand{\vepsilon}{{\boldsymbol{\epsilon}}}
\newcommand{\vzeta}{{\boldsymbol{\zeta}}}
\newcommand{\veta}{{\boldsymbol{\eta}}}
\newcommand{\vtheta}{{\boldsymbol{\theta}}}
\newcommand{\viota}{{\boldsymbol{\iota}}}
\newcommand{\vkappa}{{\boldsymbol{\kappa}}}
\newcommand{\vlambda}{{\boldsymbol{\lambda}}}
\newcommand{\vmu}{{\boldsymbol{\mu}}}
\newcommand{\vnu}{{\boldsymbol{\nu}}}
\newcommand{\vxi}{{\boldsymbol{\xi}}}
\newcommand{\vomikron}{{\boldsymbol{\omikron}}}
\newcommand{\vpi}{{\boldsymbol{\pi}}}
\newcommand{\vrho}{{\boldsymbol{\rho}}}
\newcommand{\vsigma}{{\boldsymbol{\sigma}}}
\newcommand{\vtau}{{\boldsymbol{\tau}}}
\newcommand{\vupsilon}{{\boldsymbol{\upsilon}}}
\newcommand{\vphi}{{\boldsymbol{\phi}}}
\newcommand{\vchi}{{\boldsymbol{\chi}}}
\newcommand{\vpsi}{{\boldsymbol{\psi}}}
\newcommand{\vomega}{{\boldsymbol{\omega}}}

\newcommand{\rLambda}{\mathrm{\Lambda}}
\newcommand{\rSigma}{\mathrm{\Sigma}}

\makeatletter
\DeclareRobustCommand\onedot{\futurelet\@let@token\@onedot}
\def\@onedot{\ifx\@let@token.\else.\null\fi\xspace}
\def\eg{\emph{e.g}\onedot} \def\Eg{\emph{E.g}\onedot}
\def\ie{\emph{i.e}\onedot} \def\Ie{\emph{I.e}\onedot}
\def\vs{\emph{vs\onedot}}
\def\cf{\emph{cf}\onedot} \def\Cf{\emph{C.f}\onedot}
\def\etc{\emph{etc}\onedot} \def\vs{\emph{vs}\onedot}
\def\wrt{w.r.t\onedot} \def\dof{d.o.f\onedot}
\def\etal{\emph{et al}\onedot}
\makeatother

\newcommand\rurl[1]{%
  \href{https://#1}{\nolinkurl{#1}}%
}


\newcommand{\bentarrow}[1][]{%
  \begin{tikzpicture}[#1]%
    \draw (0,0.7ex) -- (0,0) -- (0.75em,0);
    \draw (0.55em,0.2em) -- (0.75em,0) -- (0.55em,-0.2em);
  \end{tikzpicture}%
}

\definecolor{higha}{HTML}{009b10} % Green for high values
\definecolor{lowa}{HTML}{ec462e}  % Red for low values
\definecolor{mediuma}{HTML}{FFFFFF} % White for middle values

\newcommand*{\opacitya}{50} % Set opacity
\newcommand*{\minvalcolora}{2.5} % Define minimum value
\newcommand*{\midvalcolora}{11.6} % Define midpoint value for the white color
\newcommand*{\maxvalcolora}{37.3} % Define maximum value
\newcommand{\grca}[1]{
    \ifdim #1pt < \midvalcolora pt
        \pgfmathparse{(#1-\minvalcolora)/(\midvalcolora-\minvalcolora)}
        \let\normalizedval\pgfmathresult
    
        \pgfmathparse{100*(\normalizedval)^(2.0)} 
        \xdef\tempa{\pgfmathresult}
        \pgfmathparse{min(100,max(0,\tempa))}
        \xdef\tempa{\pgfmathresult}
    
        \cellcolor{mediuma!\tempa!lowa!\opacitya} #1
    \else
        \pgfmathparse{(#1-\midvalcolora)/(\maxvalcolora-\midvalcolora)}
        \let\normalizedval\pgfmathresult
    
        \pgfmathparse{100*(\normalizedval)^(2.0)}
        \xdef\tempa{\pgfmathresult}
        \pgfmathparse{min(100,max(0,\tempa))}
        \xdef\tempa{\pgfmathresult}
    
        \cellcolor{higha!\tempa!mediuma!\opacitya} #1
    \fi
}

\definecolor{highb}{HTML}{009b10} % Green for high values
\definecolor{lowb}{HTML}{ec462e}  % Red for low values
\definecolor{mediumb}{HTML}{FFFFFF} % White for middle values

\newcommand*{\opacityb}{50} % Set opacity
\newcommand*{\minvalcolorb}{1.8} % Define minimum value
\newcommand*{\midvalcolorb}{8.6} % Define midpoint value for the white color
\newcommand*{\maxvalcolorb}{31.3} % Define maximum value
\newcommand{\grcb}[1]{
    \ifdim #1pt < \midvalcolorb pt
        \pgfmathparse{(#1-\minvalcolorb)/(\midvalcolorb-\minvalcolorb)}
        \let\normalizedval\pgfmathresult
    
        \pgfmathparse{100*(\normalizedval)^(2.0)} 
        \xdef\tempa{\pgfmathresult}
        \pgfmathparse{min(100,max(0,\tempa))}
        \xdef\tempa{\pgfmathresult}
    
        \cellcolor{mediumb!\tempa!lowb!\opacityb} #1
    \else
        \pgfmathparse{(#1-\midvalcolorb)/(\maxvalcolorb-\midvalcolorb)}
        \let\normalizedval\pgfmathresult
    
        \pgfmathparse{100*(\normalizedval)^(2.0)}
        \xdef\tempa{\pgfmathresult}
        \pgfmathparse{min(100,max(0,\tempa))}
        \xdef\tempa{\pgfmathresult}
    
        \cellcolor{highb!\tempa!mediumb!\opacityb} #1
    \fi
}

\definecolor{highc}{HTML}{009b10} % Green for high values
\definecolor{lowc}{HTML}{ec462e}  % Red for low values
\definecolor{mediumc}{HTML}{FFFFFF} % White for middle values

\newcommand*{\opacityc}{50} % Set opacity
\newcommand*{\minvalcolorc}{1.7} % Define minimum value
\newcommand*{\midvalcolorc}{5.9} % Define midpoint value for the white color
\newcommand*{\maxvalcolorc}{20.8} % Define maximum value
\newcommand{\grcc}[1]{
    \ifdim #1pt < \midvalcolorc pt
        \pgfmathparse{(#1-\minvalcolorc)/(\midvalcolorc-\minvalcolorc)}
        \let\normalizedval\pgfmathresult
    
        \pgfmathparse{100*(\normalizedval)^(2.0)} 
        \xdef\tempa{\pgfmathresult}
        \pgfmathparse{min(100,max(0,\tempa))}
        \xdef\tempa{\pgfmathresult}
    
        \cellcolor{mediumc!\tempa!lowc!\opacityc} #1
    \else
        \pgfmathparse{(#1-\midvalcolorc)/(\maxvalcolorc-\midvalcolorc)}
        \let\normalizedval\pgfmathresult
    
        \pgfmathparse{100*(\normalizedval)^(2.0)}
        \xdef\tempa{\pgfmathresult}
        \pgfmathparse{min(100,max(0,\tempa))}
        \xdef\tempa{\pgfmathresult}
    
        \cellcolor{highc!\tempa!mediumc!\opacityc} #1
    \fi
}

\definecolor{highd}{HTML}{009b10} % Green for high values
\definecolor{lowd}{HTML}{ec462e}  % Red for low values
\definecolor{mediumd}{HTML}{FFFFFF} % White for middle values

\newcommand*{\opacityd}{50} % Set opacity
\newcommand*{\minvalcolord}{1.5} % Define minimum value
\newcommand*{\midvalcolord}{7.5} % Define midpoint value for the white color
\newcommand*{\maxvalcolord}{19.8} % Define maximum value
\newcommand{\grcd}[1]{
    \ifdim #1pt < \midvalcolord pt
        \pgfmathparse{(#1-\minvalcolord)/(\midvalcolord-\minvalcolord)}
        \let\normalizedval\pgfmathresult
    
        \pgfmathparse{100*(\normalizedval)^(2.0)} 
        \xdef\tempa{\pgfmathresult}
        \pgfmathparse{min(100,max(0,\tempa))}
        \xdef\tempa{\pgfmathresult}
    
        \cellcolor{mediumd!\tempa!lowd!\opacityd} #1
    \else
        \pgfmathparse{(#1-\midvalcolord)/(\maxvalcolord-\midvalcolord)}
        \let\normalizedval\pgfmathresult
    
        \pgfmathparse{100*(\normalizedval)^(2.0)}
        \xdef\tempa{\pgfmathresult}
        \pgfmathparse{min(100,max(0,\tempa))}
        \xdef\tempa{\pgfmathresult}
    
        \cellcolor{highd!\tempa!mediumd!\opacityd} #1
    \fi
}

\definecolor{highe}{HTML}{009b10} % Green for high values
\definecolor{lowe}{HTML}{ec462e}  % Red for low values
\definecolor{mediume}{HTML}{FFFFFF} % White for middle values

\newcommand*{\opacitye}{50} % Set opacity
\newcommand*{\minvalcolore}{2.3} % Define minimum value
\newcommand*{\midvalcolore}{10.6} % Define midpoint value for the white color
\newcommand*{\maxvalcolore}{24.7} % Define maximum value
\newcommand{\grce}[1]{
    \ifdim #1pt < \midvalcolore pt
        \pgfmathparse{(#1-\minvalcolore)/(\midvalcolore-\minvalcolore)}
        \let\normalizedval\pgfmathresult
    
        \pgfmathparse{100*(\normalizedval)^(2.0)} 
        \xdef\tempa{\pgfmathresult}
        \pgfmathparse{min(100,max(0,\tempa))}
        \xdef\tempa{\pgfmathresult}
    
        \cellcolor{mediume!\tempa!lowe!\opacitye} #1
    \else
        \pgfmathparse{(#1-\midvalcolore)/(\maxvalcolore-\midvalcolore)}
        \let\normalizedval\pgfmathresult
    
        \pgfmathparse{100*(\normalizedval)^(2.0)}
        \xdef\tempa{\pgfmathresult}
        \pgfmathparse{min(100,max(0,\tempa))}
        \xdef\tempa{\pgfmathresult}
    
        \cellcolor{highe!\tempa!mediume!\opacitye} #1
    \fi
}

%\clearpage

%\glsaddall

%\printnoidxglossary[type=\acronymtype]
%\clearpage

% normal numbering from here
%\pagenumbering{arabic}

%%%% Each chapter is in a different file
%% In the [], how the chapter will be called in the table of contents
%% In the {}, the name of the chapter in the text
%% The chaptermark is the name of the chapter in the footer
%%% You want a small enough chaptermark or it will overflow to the sides

%\part{Introduction}

%\part{Introduction to the problem}
\begin{abstract}
%In industrial practice, requirements are an indispensable element of any serious software project. Requirements can rarely be fixed at the beginning of the project since they evolve and change over the course of the project. 

%Addressing these goals, this article presents Unified Object Oriented Requirements, a methodology of developing OO requirements in a seamless way and presents its application to a significant case study – Roborace, focusing on developing software for autonomous racing cars.

In industrial practice, requirements are an indispensable element of any serious software project. In the academic study of software engineering, requirements are one of the heavily researched subjects. And yet requirements engineering, as practiced in industry, makes shockingly sparse use of the concepts propounded in the requirements literature. The present paper starts from an assumption about the \textit{causes} for this situation and proposes a \textit{remedy} to redress it. The posited explanation is that \textit{change} is the major factor affecting the practical application of even the best-intentioned requirements techniques. No sooner has the ink dried on the specifications than the system environment and stakeholders' views of the system begin to evolve. %Requirement methods that assume that requirements can be set once and for all to guide the development are doomed.
The proposed solution is a requirements engineering \textit{method}, called UOOR, which unifies many known requirements concepts and a few new ones in a framework entirely devised to accommodate and support seamless change throughout the project lifecycle. 
    

The Unified Object-Oriented Requirements (UOOR) method encompasses the commonly used requirements techniques, namely, scenarios, and integrates them into the seamless software development process. 
The work presented here introduces the notion of seamless requirements traceability, which relies on the propagation of traceability links, themselves based on formal properties of relations between project artifacts. As a proof of concept, the paper presents a traceability tool to be integrated into a general-purpose IDE that provides the ability to link requirements to other software project artifacts, display notifications of changes in requirements, and trace those changes to the related project elements. 

The UOOR approach is not just a theoretical proposal but has been designed for practical use and has been applied to a significant real-world case study—a Roborace that focuses on developing software for autonomous racing cars.


%In the academic study of software engineering, requirements are one of the heavily researched subjects. Yet, as practiced in industry, requirements engineering makes shockingly sparse use of the concepts propounded in the requirements literature. The present paper starts from an assumption about the \textit{causes} for this situation and proposes a \textit{remedy} to redress it.

%The posited explanation is that \textit{change} is the major factor affecting the practical application of even the best-intentioned requirements techniques. 
%Requirement methods that assume that requirements can be set once and for all to guide the development are doomed.

%The proposed remedy is a requirements engineering \textit{method}, called UOOR\footnote{Unified Object-Oriented Requirements, pronounced ??.}, which unifies many known requirements concepts and a few new ones in a framework entirely devised to accommodate and support seamless change throughout the project lifecycle. 

%The UOOR method encompasses the commonly used requirements techniques, namely, scenarios, and integrates them into the seamless software development process. 
%The paper introduces the notion of seamless requirements traceability, which relies on the propagation of traceability links, which are themselves based on the formal properties of relations between project artifacts. 

%The UOOR approach is not just a theoretical proposal but has been designed for practical use and applied to significant applications, including a full case study detailed in a forthcoming book \cite{companion2025}. 

\end{abstract}

\begin{document}
\maketitle

\section{Introduction}

Software handles an increasing number of tasks, from enterprise management and reporting to operating medical devices. 
Before we can rely on software, however, we must be sure that it does exactly what it is supposed to do. 
The process of description of what the software will do and how it will perform is known as software requirements specification. 
This process plays a major role in the software development lifecycle: according to an often-cited study \cite{hussain2016role}, requirements-related factors are among the leading reasons for software project failures. 

An earlier article \cite{46} analyzed some limitations of dominant practices of requirements engineering, particularly use cases. More generally, challenges in the current state of the art (see section \ref{approaches} for further analysis) include the following: 
\begin{itemize}
    \item Requirements are \textbf{subject to change}. Requirements are rarely definitive at the beginning of the project; they, most of the time, evolve and change.
    \item The ability to adapt to changes in requirements is tightly related to the \textbf{traceability} between requirements and other project artifacts. Due to the necessity to manually create and maintain traceability links, requirements traceability is often perceived as an inefficient practice and is applied only in 40\% of software projects.
    \item Producing “good” requirements requires substantial effort. \textbf{Balancing requirements efforts} and the \textbf{quality} of the obtained requirements is difficult: quality requirements require certain rigor and formality, which requires substantial training for requirements engineers. Conversely, affordable approaches are much easier to grasp but leave much room for deficient requirements.
\end{itemize}

%A well-known example of a requirements-related issue that caused several deaths is the Therac-25 incident. Therac-25 was a radiotherapy machine that caused a massive overdose of radiation due to software malfunctioning. One of the causes of failure was improper reuse of the  earlier machines' software and poor requirements traceability. The development was based on the wrong assumption that the software of the earlier machines worked correctly \cite{baase2008ethical}. 

%Software failures can also be attributed to missing or incorrect properties of the system's environment. In 2015, the Washington state prison case was revealed: due to improper calculations, more than 3200 inmates were released too early and went on to commit several homicides \cite{failures}.



%Seamless software development can satisfy the need to quickly adapt to changes in requirements, ensuring fast and smooth traceability between requirements and other software artifacts. In the past decade, a considerable effort was made to investigate how requirements can be integrated into a seamless software development process \cite{32, 39, galinier2021seamless}.

%Inspired by these works, the present article investigates the practical application of seamless requirements: what should be the process of seamless software development from requirements to code? What tool support is required to ensure traceability between the requirements and other project artifacts? 
%The article seeks to establish a proper balance between requirements effort and outcome by suggesting a usable and affordable approach that facilitates producing unambiguous, reusable, verifiable, and traceable requirements. 

To help address these issues, we have developed an approach, Unified Object-Oriented approach to Requirements (UOOR), which
includes the following concepts:
\begin{itemize}

    \item \textbf{Object-oriented decomposition techniques}. We demonstrate, that the concept of class is general enough to describe important artifacts such as scenarios and  test cases. 
    \item \textbf{Formulating requirements with} \textbf{contracts}. Contract-based specification is implementation-independent and ensures unambiguity, verifiability and reusability of requirements.
    \item \textbf{Seamless software development}. Software development, from requirements to implementation relies on a uniform process (based on refinement) and a uniform notation (Eiffel language). 
    %, that can satisfy the need to quickly adapt to changes in requirements, ensuring fast and smooth traceability between requirements and other software artifacts.
    \item \textbf{Seamless requirements traceability}. The burden of traceability links creation and management can be lowered by the  propagation of traceability links, themselves based on formal properties of relations between project artifacts.
\end{itemize}

Section~\ref{characteristics} discusses the desired characteristics of a requirements approach.
Section~\ref{UOOR_chapter} introduces the UOOR approach and provides guidance on producing UOOR specifications. 
Section~\ref{toolchapter} explores the notion of seamless requirements traceability. 
It devises the typed relations between project elements and the formal properties of such relations. 
Section~\ref{traceability_tool} presents a tool to be integrated into EiffelStudio, which facilitates managing requirements traceability links. 
Sections~\ref{Roborace} and~\ref{Experiment} evaluate the UOOR approach. 
Section~\ref{Roborace} presents the application of the UOOR approach to a significant project -- the Roborace. 
Section~\ref{Experiment} reveals the results of a controlled experiment conducted at the University of Toulouse and evaluates the perception of the approach. 
Section~\ref{approaches} presents related work.
Section~\ref{contributions} summarizes and evaluates the contributions, lists the limitations of the devised approach, and highlights the perspectives for future work.

\begin{comment}

%\section{Overview of the contributions}
The thesis introduces a Unified Object-Oriented approach to Requirements (UOOR), which unifies scenarios and object-oriented techniques. It demonstrates that the approach is affordable and practically usable and that it facilitates producing unambiguous, verifiable and traceable requirements. The specific advances are the following:
\begin{itemize*}
    \item A demonstration that  the concept of class is general enough to describe not only “objects” in a narrow sense but also scenarios such as use cases and user stories and other important artifacts such as test cases and oracles.
    \item A methodology that a requirements engineer may use as a guide for requirements specification process.
    \item A partial formal model of requirements engineering through the definition of relations and their formal mathematical properties.
    \item The notion of seamless requirements traceability, which relies on propagation of traceability links, themselves based on formal properties of relations between project artifacts.
    \item The Traceability tool, which supports requirements traceability by creating typed links between project elements, and tracking changes from requirements to the affected code elements.    
    \item An evaluation of the approach by applying it to a significant case study and conducting a controlled experiment.

\end{itemize*}





\end{comment}



\begin{comment}
    

%\chapter{Requirements engineering terminology}
\label{state_of_art}

\chaptermark{State of the art}

The thesis presents an approach to requirements specification. But what is the subject of such an approach? What are requirements? Is there a difference between requirements an specifications? And what particular aspects of requirements should be covered?

This chapter reviews requirements standards, main requirements engineering schools and their authors,  requirements structure and guidelines textbooks to establish common requirements terminology.
In particular, it addresses the following questions:
\begin{itemize*}
    \item What are requirements? (section~\ref{requirements}).
    \item What is the difference  between requirements and specifications? (section~\ref{specifications}).
    \item What are requirements dimensions and what are their respective roles in requirements? (section~\ref{dimensions}).
    \item What are the requirements types? (section~\ref{requirements_types}).
    \item What is the role of scenarios in requirements? (section~\ref{scenario_role}).
\end{itemize*}

The answers to these questions help to establish the subject of the requirements specification approach, i.e what a devised approach to requirements should address.

%\section{Studied approaches}


Considering main schools and their authors, the reviewed approaches are: 

\begin{itemize*}
    \item A. van Lamsweerde.  \textit{Requirements Engineering: From System Goals to UML Models to Software Specifications} \cite{lamsweerde}. 
    \item K. Wiegers and J. Beatty. \textit{Software Requirements} \cite{wiegers}. 
    \item M. Jackson and P. Zave. \cite{zave_four,zave,jackson3}. 
    \item P. A. Laplante. \textit{Requirements engineering for software and systems} \cite{laplante}. 
    \item D. Leffingwell. \textit{Agile software requirements} \cite{leffingwell}. 
    \item K. J. McDonald. \textit{Beyond requirements: analysis with an agile mindset} \cite{mcdonald}. 
    \item C. Larman. \textit{Applying UML and patterns: an introduction to object oriented analysis and design and iterative development} \cite{larman}.  
    \item A. Cockburn. \textit{Writing effective use cases} \cite{cockburn}. 
    \item J. Dick, E. Hull, K. Jackson. \textit{Requirements engineering}  \cite{dick}. 
    \item D. Bjørner.
    ``\textit{Domain engineering}'' \cite{bjorner2010domain}.
\end{itemize*}

The studied standards and related guides are: 
\begin{itemize*}
    \item \textit{ISO Standard 29148: 2011 Systems and Software Engineering} \cite{iso2011}. 
    \item \textit{ISO/IEC/IEEE Standard 29148: 2018 Systems and Software Engineering} \cite{iso2018}.
    \item \textit{ISO/IEC/IEEE 24765:2017 Systems and software engineering -- Vocabulary} \cite{iso2017}.
    \item \textit{IREB Glossary} \cite{ireb_glossary}, and a related textbook by K. Pohl \cite{pohl}.
    \item International Institute of Business Analysis (IIBA). \textit{A guide to the business analysis body of knowledge (BABOK)} \cite{BABOK}.
    \item IEEE Computer Engineering Society.
\textit{Guide to the Software Engineering Body of Knowledge (\acrshort{swebok})} \cite{SWEBOK}.
      
\end{itemize*}

Textbooks covering requirements structure and guidelines studied here are: 
\begin{itemize*}
    %\item EARS
    \item S. L. Pfleeger and J. M. Atlee. \textit{Software Engineering: Theory and practice} \cite{pfleeger} (chapter on requirements).
    \item I. Sommerville and P. Sawyer, \textit{Requirements Engineering: A Good Practice Guide} \cite{sommerville} \cite{kotonya}.
    \item B. Meyer \textit{Handbook of Requirements and Business Analysis} \cite{Handbook}.
    \item S. Robertson and J. Robertson \textit{Mastering the requirements process: Getting requirements right} \cite{volere}. 
\end{itemize*}



%\section{What are requirements?}
\label{requirements}

Even in standards, and even for the basic concepts -- such as  ``requirement'' -- some definitions  
are far from being ideal.
For example, IEEE standard \cite{ieee2005} defines ``requirement'' as

\textit{A statement that identifies a product or process operational, functional, or design characteristic or constraint, which is unambiguous, testable or measurable, and necessary for product or process acceptability (by consumers or internal quality assurance guidelines).
}

Not only the definition is long and ambiguous, it is prescriptive, whereas a good definition should be descriptive. The definition states that a requirement is \textit{``a statement ... which is unambiguous, testable or measurable ... ''} Does it mean that a statement that is ambiguous cannot be a requirement? In fact it is not true: natural language requirements are often ambiguous, yet they \textit{are} requirements (maybe not the perfect ones). Next, why only these three characteristics appear in the definition? Shouldn't a requirement be necessary and appropriate, as stated in IEEE standard \cite{iso2018}? In fact it is neither justified, nor appropriate to list quality factors in the definition: it distracts the reader from its essence and gives the impression that some quality factors (those mentioned in the definition) are more important than others (the omitted ones), which is incorrect. 

\vspace{12pt}
Other definitions of a ``requirement'' in the literature
differ considerably from each other:

\begin{enumerate}
    \item \textit{``A prescriptive statement to be enforced by the software-to-be, possibly in cooperation with other system components, and formulated in terms of environmental phenomena''} \cite{lamsweerde}.
    \item \textit{``A desired relationship among phenomena of the environment of a system, to be brought about by the hardware/software machine that will be constructed and installed in the environment''} \cite{jackson3}
    \item \textit{``A statement of a customer need or objective, or of a condition or capability that a product must possess to satisfy such a need or objective. ''} \cite{wiegers}
    \item \textit{``A usable representation of a need.'' }\cite{BABOK}
    \item \textit{``A statement which translates or expresses a need and its associated constraints and conditions''} \cite{iso2018}
    \item \textit{``A statement that specifies (in part) how a goal should be accomplished by a proposed system.''} \cite{laplante}
    \item \textit{``A statement of a system service or constraint''} \cite{kotonya}
    \item \textit{``A desired capability or behavior that a software system should possess''} \cite{leffingwell}
    \item \textit{``Capabilities and conditions to which the system -- and more broadly, the project -- must conform''} \cite{larman}
    \item \textit{``A capability or property that a system shall have'' }\cite{ireb_glossary}
    \item \textit{``A condition or capability that must be met or possessed by a system or system component to satisfy a contract, standard, specification, or other formally imposed documents.''} \cite{iso2017}
    \item \textit{``An expression of desired behavior''} \cite{pfleeger}
    \item \textit{``A description of how the system should behave, or of a system property or attribute'' }\cite{sommerville}
    \item \textit{``Something that the product must do, or a property that the product must have, that is needed or wanted by the stakeholders''} \cite{volere}
    \item \textit{``A property that must be exhibited by something in order to solve some problem in the real world.''} \cite{SWEBOK}
    \item \textit{``Descriptions of how the system should behave, application domain information, constraints on the system's operations, or specifications of a system property or attribute.''} \cite{kotonya}
\end{enumerate}


The first thing, which is common in most of these definitions is that a requirement is a statement/representation/expression/description.
In other words, requirements engineering concerns not the needs, capabilities and properties, but the way they are expressed in the requirements documentation.

Further analysis of the definitions gets more complicated, as some definitions are extremely vague. What exactly do requirements express \textit{``Desired behavior''}, as stated in definition (12)? Behavior of what/whom? A \textit{need}, as claims the definition (4)? Whose need? Besides, requirements do not necessarily express needs -- they can express constraints of limitations. \textit{``A property that must be exhibited by something in order to solve some problem in the real world''}? This definition is way too hesitant, as ``exhibited by something'' sounds too vague for an IT project. 

The definitions may be further grouped based on their similarity. Definitions (1) and (2) highlight the role of the environment in requirements specification: Requirements must be expressed in terms of environmental phenomena (not in terms of internal phenomena of the machine). These definitions distinguish requirements from specifications: requirements are formulated in term of environmental phenomena whereas specifications (or software requirements) are \textit{``enforced solely by the software-to-be and formulated only in terms of phenomena shared between the software and the environment''} \cite{lamsweerde}. 

Definitions (3), (4), (5), (6) claim that a requirement is an expression of a need or an objective. Whereas needs and objectives are the primary sources of requirements, some of the requirements come from constraints or regulations, rather than from the needs.

Definitions (7) -- (14), (16) state that requirements express desired capabilities or behavior.  Definition (12) \textit{``An expression of desired behavior''} \cite{pfleeger} could be a valid requirement definition if we considered a wide range of subjects (behavior of the system, development team, environment entities and so on) and if the term behavior was used in a very broad sense, covering not only functional, but also non-functional characteristics of behavior. However, this is not what first comes to mind when reading this definition, so it is at the very least ambiguous. 
If we read the definition as ``an expression of desired behavior of a system'', it turns just wrong, since requirements include many elements that do not relate to system's behavior, such as design constraints \cite{ieee1998}.  Similar reasoning applies to the definition (8).

Definitions (7), (10), (11), (13) can be summarized into ``system capabilities (behaviors) and constraints (properties)''. Such a definition indeed covers most of the requirements elements. 
Definitions (9) and (16) highlight that requirements should cover project constraints \cite{larman} and domain information \cite{kotonya}. We will discuss the role of project dimension in requirements in more details in section~\ref{Project}.

\vspace{12pt}
To summarize the above reasoning, we suggest the following definition of the term ``requirements'':
\begin{mdframed}
\begin{definition}
    \textbf{Requirements} are statements describing capabilities and constraints to which the system must conform.
\end{definition}  \end{mdframed}



%\section{Requirements vs specifications}
\label{specifications}
The term ``specification'' is often used as a synonym for the term ``requirements'' \cite{lamsweerde}, whereas some authors distinguish the two notions \cite{82, lexicon}.
According to Jackson \cite{lexicon}, requirements are formulated only in terms of environment phenomena, i.e., in terms of things, observable in the surrounding world; specifications are description of the behaviors that a machine should produce in order to meet the requirements, i.e., they are the interfaces between requirements and programs. Specifications are thus refinements of the requirements, yet not necessarily they can be derived directly from the requirements: some requirements are constrained by the domain knowledge, which has to be taken into account when producing specifications. Similarly, van Lamsweerde \cite{lamsweerde} distinguishes system requirements  (formulated only in terms of environment phenomena) and software requirements (formulated in terms of phenomena shared between the system and its environment). Software requirements are thus a subset of system requirements.

In the context of this thesis we do not distinguish requirements and specifications, so the term ``requirement'' encompasses both meanings. 





%\section{Requirements dimensions}
\label{dimensions}
Requirements engineering focuses on defining the objectives of a system, independently of their realization. Many approaches have been proposed to tackle the difficult tasks involved in this effort; among some of the best known are contributions by Michael Jackson and Pamela Zave \cite{zave_four, jackson3, zave}, Axel van Lamsweerde \cite{lamsweerde}, and others we  consider in this state of the art.

Generally speaking, requirements describe what a system should do (functional requirements) and how it should do it (non-functional requirements). However, this point of view is too simplistic. In their seminal work  Jackson and Zave established the key role that environment plays in formulating the requirements \cite{zave_four}, whereas Axel van Lamsweerde \cite{lamsweerde} emphasized the role of goals in requirements. Meyer et al. attempted to provide a taxonomy of requirements elements \cite{meyer2019towards}, which subsequently formed the basis of the PEGS approach to requirements. According to PEGS, 
the four requirements dimensions are Goals, Environment, System and Project. 

This section reviews requirements engineering literature (textbooks, research articles and standards) to study the role that each of the four dimensions plays in requirements.

%\subsection{Goals} \label{Goals}

The term ``goal'' refers to high-level needs or objectives. The term has slightly different meaning in different approaches. According to \cite{Handbook, laplante, leffingwell, volere, lamsweerde} goals are needs or objectives of the \textbf{organisation}, that the system must address. According to IREB \cite{ireb_glossary, pohl} and BABOK \cite{BABOK}, goals describe intentions of \textbf{stakeholders}. According to Pfleeger and Atlee \cite{pfleeger, SWEBOK}, goals express the purpose of the proposed software, from the \textbf{customer}'s perspective. 

\vspace{12pt}
The importance of the role that goals play in understanding requirements is highlighted by many authors:
\begin{itemize*}
    \item Goals, unlike other requirements elements, address the widest audience of the project (all stakeholders) \cite{Handbook}.
    \item Goal establish the key benefits that system implementation must bring and thus establish the priorities to consider when taking decisions in the course of the project \cite{Handbook}.
    \item Goals are high-level objectives, whereas requirements specify how the goals should be fulfilled by the system \cite{laplante}.
    \item System objectives are goals on the level of the entire system, which can be further refined to lower-level goals, software requirements and environment assumptions. A requirement is a fine-grained goal \textit{''that is under the responsibility of a single agent of the software-to-be}'' \cite{lamsweerde}.
    \item Goals are high-level requirements, which help to identify requirements that will provide the positive contribution towards the goals' achievement \cite{volere}.
    \item Goals serve as the basis for system requirements \cite{pohl}.
    \item Goals are used to validate that all requirements are necessary \cite{pfleeger}.
    \item Goals define the vision of the system and the scope of the project. Requirements must be aligned with the goals \cite{wiegers}.
\end{itemize*}




\vspace{12pt} 
To summarize the discussion, we suggest the following definition of the term ``goals'':
\begin{mdframed}     \begin{definition}
    \textbf{Goals} are high-level objectives of the organisation that the system must address.
\end{definition}  \end{mdframed}

In other words, we consider goals to be the \textit{source} of requirements, rather than \textit{requirements per se}.

%\subsection{Environment} \label{Environment}

In the late 1980 the desire to reuse the knowledge of the system's environment in production of new systems gave rise to domain engineering \cite{arango1988domain}. 
Dines Bjørner  \cite{bjorner2010domain} studied the application of domain engineering to requirements and emphasized that significant part of requirements can be derived from domain descriptions. 
The term ``domain'', although similar to the term ``environment'', has a slightly different meaning. 
\textit{Environment} refers to the world phenomena directly related to the particular system and may include phenomena outside the system's domain (the term ``application domain'', introduced by Michael Jackson in his late works \cite{lexicon} is also used in the same meaning). 
\textit{Domain} refers to a universe of a discourse (such as library management or traffic control) and encompasses phenomena that are related to many systems in that sphere \cite{bjorner2010domain, wiegers, lamsweerde, larman, BABOK, mcdonald}. 

\vspace{12pt}
To distil the meaning of the term environment, let us analyze the definitions of the term ``Environment'', provided in the studied literature:
 \begin{enumerate}
     \item \textit{``Other automated systems which are interfaced to [the system] and business processes which may use the system''} \cite{sommerville}.
     \item \textit{``External entities, such as users, hardware devices, and other systems''}\cite{wiegers}.
     \item \textit{``The part of the world with which the machine will interact, in which the effects of the machine will be observed and evaluated''} \cite{jackson1997}.
     \item Components pertaining to the machine's surrounding world, such as people or business units, physical devices, legacy or foreign software components with which the software-to-be needs to interact \cite{lamsweerde}.
     \item \textit{``Anything affecting a subject system or affected by a subject system through interactions with it, or anything sharing an interpretation of interactions with a subject system''} \cite{iso2010}.
     \item \textit{``The circumstances, objects, and conditions that will influence the completed system''} \cite{ieee1998}.
     \item \textit{``The context in which the system will function, outside the system itself''} \cite{loucopoulos1995system}.
     \item \textit{``The set of entities (such as people, organizations, devices and other material objects, regulations, and other systems) external to the project and system but with the potential to affect the goals, project or system or to be affected by them''} \cite{Handbook}.
    \item The surrounding world, its laws and regulations \cite{laplante}. \textit{``Non-functional requirements are requirements that are imposed by the environment in which the system is to operate''}\footnote{This last statement is very strong. In fact, not necessarily the non-functional requirements are imposed by the environment, and not necessarily only non-functional requirements can be imposed by the environment. For example, some non-functional requirements are derived from goals or stakeholders' decisions, rather than from environmental constraints. At the same time, some functional requirements, such as ``the system shall calculate the dosage based on the patient's weight'' can be imposed by the environment.}
 \end{enumerate}

\vspace{12pt}
From the above definitions (1)-(8), we can extract the key characteristics of the environment:
\begin{itemize*}
    \item It is part of the world.
    \item It is external to the system.
    \item It interacts with the system or affects it in some way.
\end{itemize*}

To summarize the discussion, we suggest the following definition of the term ``environment'':

\begin{mdframed}
    
   \begin{definition}
    \textbf{Environment} is the part of the world, external to the system, that affects or can be affected by the system.
\end{definition} 

\end{mdframed}

\vspace{12pt}
Environment properties relate to requirements in the following way:
\begin{itemize*}
    \item Requirements properties are not requirements themselves, but they serve as a rich source of requirements \cite{wiegers}.
    \item Domain properties and assumptions are required to properly translate system requirements (formulated in terms of environment phenomena) to software requirements (formulated in terms of phenomena shared between environment and the system) \cite{lamsweerde, jackson3}. Satisfaction of software requirements (specifications) denoted as S together with domain properties (denoted as E) must imply satisfaction of system requirements (denoted as R): $S, E \models R$. 
    \item Environment imposes non-functional requirements \cite{laplante}. 
    \item Domain rules express the constraints on how the domain works. They are not requirements, but can clarify incomplete or ambiguous requirements \cite{larman}.
\end{itemize*}


\vspace{12pt}
To summarize, we suggest the following definition of environment properties:
\begin{mdframed}
    \begin{definition}
\textbf{Environment properties} are statements related to the system's environment that affect the system in the following ways:
    \begin{itemize*}
        \item They express directly the properties of the components of the environment.
    \item They serve as a source of requirements.
    \item They serve as input in requirements refinement.
\end{itemize*}        
    \end{definition}
\end{mdframed}

The role of environment in requirements will be further explicated in section~\ref{model_environment}.


%\subsection{System}

\label{requirements_types}

Requirements pertaining to the system are at the heart of any requirements approach. What differs though is what types of system requirements the approach aims to cover. The present section establishes the classification of system requirements and defines the major categories of requirements.

\vspace{12pt}
There are several views on how one can categorize requirements:

\begin{itemize*}
    \item All requirements on the system, that are not functional, are non-functional \cite{lamsweerde, wiegers, mcdonald, larman, BABOK, SWEBOK}.
    \item Functional requirements, non-functional requirements, and domain requirements (requirements derived from the application domain) \cite{laplante}.
    \item Functional requirements, non-functional requirements, and design constraints, where design constraints are technical, business or contractual restrictions on system design or development process \cite{leffingwell, pohl, ireb_glossary}.
    %\item Functional requirements, quality requirements (or non-functional requirements), and constraints (restrictions on the system or development process that limit the solution space) \cite{pohl}
    \item Functional requirements, interface requirements, process requirements (requirements imposed by laws and regulations), quality (or non-functional) requirements, usability requirements, human factors requirements \cite{iso2018}. This standard also mentions design constraints, defined as \textit{ ``constraints on the system design imposed by external standards, regulatory requirements or project limitations''}.
    \item Functional requirements, quality (nonfunctional) requirements, design constraints (design decisions that restrict the set of solutions), process constraints (restrictions on the development process) \cite{pfleeger}. 
    
\end{itemize*}

The term ``functional requirements'' is the least controversial one: all of the approaches agree that a functional requirement specifies a function that a system shall perform. 
There is no such accordance towards non-functional requirements. 

Several authors highlight that the boundary between functional and non-functional requirements is not solid \cite{lamsweerde, wiegers}.
For example, safety and security requirements may fall into both categories simultaneously, as in the example \textit{``the safety injection signal shall be on whenever there is a loss of coolant except during normal start-up or cool down''} \cite{lamsweerde}.

In some approaches, the terms ``non-functional requirements'' and ``quality requirements'' are synonyms \cite{iso2018, pfleeger, pohl}. 
In other approaches, quality requirements (sometimes also called ``quality attributes'') are a subset of non-functional requirements \cite{wiegers, lamsweerde, laplante, larman}. 
Some approaches do not mention the term ``quality requirements'' (or ``quality attributes'') and thus do not allocate quality requirements to a separate category \cite{Handbook, mcdonald, volere}. 
Finally, some approaches, such as Cockburn's \cite{cockburn} or Jaskson's \cite{jackson3}, focus solely on functional requirements and thus do not discuss non-functional requirements.

If we take aside the terminology aspect, there is agreement between the majority of the authors that quality requirements are non-functional requirements (or some part of non-functional requirements).
What else should be specified as non-functional requirements is a more controversial ground.

According to Wiegers, non-functional requirements include quality attributes, external interfaces, and design and implementation constraints \cite{wiegers}. 
Laplante lists five categories of non-functional requirements: quality, design, economic (constraints on development cost), operating, and political/cultural (constraints from laws and standards) \cite{laplante}. 
Van Lamsweerde \cite{lamsweerde} mentions quality requirements, compliance requirements (constraints from laws and regulations), architectural requirements (distribution and installation constraints), and development requirements (constraints on the development process). In Volere \cite{volere} there are eight kinds of non-functional requirements: product’s appearance, usability, performance, the operating environment of the product, maintainability, security, cultural and political, and legal. 

Based on the analysis of the studied literature, we can identify three conceptually different types of requirements: functional requirements, non-functional requirements and constraints. 
The reason that constraints are allocated to a separate category is that they represent something in-between functional and non-functional requirements. On one hand, constraints restrict system functions, so they are somewhat functional. On the other hand, constraint describe the \textit{how}, not the \textit{what}, so in this way they are non-functional. 

\vspace{12pt}
We suggest the following definitions:
\begin{mdframed}
    \begin{definition}
    \textbf{Functional requirement} is a description of \textit{what} the system should do.
\end{definition}  

\end{mdframed}

\vspace{12pt}
The key property of a functional requirement is that it describes a function or a service that the system must provide.
\begin{mdframed}     \begin{definition}
    \textbf{Non-functional requirement} is a description of \textit{how} the system will perform.
\end{definition}  \end{mdframed}

Note that requirement is functional if it describes a function or a service that the system must provide, so the requirement \textit{``the safety injection signal shall be on whenever there is a loss of coolant except during normal start-up or cool down''} is considered to be functional even though it can also be considered a safety (non-functional) requirement.

Although some of the approaches do not allocate constraints to a separate category of requirements \cite{lamsweerde, laplante}, most of the authors admit the difference between constraints and functional requirements \cite{zave_four,zave,jackson3,wiegers, leffingwell, larman, pohl, dick, pfleeger, sommerville}. The difference between constraints and non-functional requirements is more subtle: some authors define constraints in a broad way, including restricting not only the system's functions behavior, but also more general properties \cite{lamsweerde, wiegers}. 

\vspace{12pt}
Nevertheless, it seems more natural to allocate such properties to non-functional requirements and define as constraints requirements that combine functional and non-functional properties:


\begin{mdframed}     \begin{definition}
    \textbf{Constraint} is a property that restricts the system function behavior.
\end{definition}  \end{mdframed}


%\subsection{Project} \label{Project}

There is no consensus between the authors whether project requirements should be included into requirements specification. Some of the authors do not discuss project requirements \cite{laplante, cockburn}. At the same time, many authors highlight project aspects that must be considered in requirements specification. 

According to Zave \cite{zave}, estimating costs, risks and schedules are the tasks of requirements engineers. 

Wiegers \cite{wiegers} provides a detailed list of project requirements, which concern the project itself rather than the product to be built. According to Wiegers, project's success depends on success of translating software requirements to project plans and software designs. 
In particular, estimates, project plans and schedules are based on requirements. 
Those estimates have to be updated when requirements change. 
Wiegers also distinguishes project-level constraints, such as schedule, staff and budget limitations. 
Nevertheless, project requirements, although a shared responsibility of the business analyst and a project manager, should not be included into a System Requirement Specification and best fit into the project management plan. 

Van Lamsweerde \cite{lamsweerde} highlights that the project type influences the requirements engineering process. 
Despite the fact that the author does not distinguish project requirements as a part of requirements document, he mentions some of project requirements, such as development requirements (constraints on the development process), as a part of non-functional requirements.

In the Volere approach \cite{volere}, project tasks and project risks are included into requirements document. 
The ``Task'' section of the requirements template is devoted to the planning of the project aiming at delivering the final product. 
It lists the tasks to be done and the estimates of required time and resources for each task. 
The ``Risks'' section is dedicated to identifying project risks based on the requirements knowledge as input.
Each risk is accompanied with the estimation of its probability. 

Pfleeger and Atlee \cite{pfleeger} identify \textbf{process constraints} as the restrictions on the techniques or resources that can be used for system construction.  
In particular, process constraints cover the characteristics of the required personnel and other resources, documentation requirements, and standards restricting system construction process.

Laplante \cite{laplante} attributes to nonfunctional requirements ``operating constraints''.
These include staff availability and skills, and the availability of the system for maintenance - which are essentially project constraints. 

According to the IEEE standard \cite{iso2018}, project constraints (which are \textit{``constraints to performing the project within cost and schedule''}) should be included to requirements documentation.

\vspace{12pt}
To summarize, we suggest the following definition of the term ``project requirements'':
\begin{mdframed}
    \begin{definition}
    \textbf{Project requirements} are requirements related to a \textit{process} of software development. 
    \end{definition}
\end{mdframed}

Project requirements, however, not necessarily should be a part of requirements specification and can be included instead into a project plan. 
%The UOOR approach, presented in this dissertation, does not address project-related requirements.



%\section{The role of scenarios} 
\label{scenario_role}

The role of scenarios in different requirements approaches differs significantly: whereas some authors argue that scenarios are examples of system behavior, others state that scenarios are requirements, not merely examples.

McDonald uses the term ``examples'' to denote 
\textit{``an approach to defining the behavior of a system using realistic examples instead of abstract statements''}\cite{mcdonald}. 
Similarly, van Lamsweerde describes scenarios as \textit{``concrete examples of how things are running in the system-as-is, or how they should be running in the system-to-be.''} These authors highlight that scenarios lack abstraction and thus can be considered as examples of system behavior, rather than its specification. 

Agile approaches, on the contrary, consider scenarios to be the requirements. Cockburn claims that scenarios are \textit{``not all of the requirements [...],  but
they are \textbf{all} of the behavioral requirements''.} \cite{cockburn}. Leffingwell is even more radical: he claims that user stories are ``\textit{the general-purpose agile substitute for what traditionally has been referred to as software requirements}'' \cite{leffingwell}.

\vspace{12pt}
To summarize, there is no agreement in the literature with respect to the role of scenario in requirements: some authors argue that scenarios are requirements, whereas others state that scenarios are solely examples of system's behavior. There is a way, however, to unify those views: in this unified approach scenarios serve as behavioral requirements, defining the admissible sequences of operations; at the same time, they serve as the source of more abstract functional requirements and constraints. 

The discussion on the role of scenarios will be continued in section~\ref{subsection4}.

%\section{Conclusion}

\vspace{12pt}
This chapter established the requirements engineering terminology relevant for the present thesis, based on critical review of the existing literature. It defined the notions of requirements and environment; explored the difference between requirements and specifications; explored requirements classification and established the notions of functional requirements, non-functional requirements and constraints; explored the four requirements dimensions: goals, environment, system and project -- and their role in requirements specification; discussed the controversial role of scenarios in system's requirements.

The chapter answers the question: ``What a devised approach to requirements should address?'' Summarizing the above discussion, we can formulate the following research questions (manuscript sections covering the respective questions appear in parentheses):
\begin{itemize*}
    \item How functional requirements should be specified? 
    (Section~\ref{subsection3})
    \item How constraints should be specified? (Section~\ref{subsection3})
    \item How environment properties should be specified? (Section~\ref{model_environment})
    \item What is the role of scenarios in requirements specification? (Section~\ref{subsection4})
\end{itemize*}

The next chapter~\ref{approaches} will continue exploration of the state of the art by defining the characteristics that a requirements approach should have and evaluating the existing approaches against these criteria.

\end{comment}

%\mb{add 5-paragraph summary of the approach}

\section{Desired characteristics of a requirements approach}
\label{characteristics}

This section explores the characteristics of a requirements engineering approach necessary to make it usable in the industry.

Section~\ref{quality} explores some of the qualities of ``good'' requirements. Section~\ref{usable} investigates the characteristics of the requirements approach that make it usable.  

\subsection{Qualities of ``good'' requirements}
\label{quality}
According to the IEEE standard \cite{iso2018} requirements should be necessary, appropriate,  unambiguous, complete, singular, feasible, verifiable, correct, conforming, and traceable. Nevertheless, it is hard to achieve some of these characteristics if the requirements are documented in informal natural language. In contrast, according to the industrial survey conducted by Fricker et al. \cite{fricker2015requirements}, only 6\% of projects utilize formal notations. %In particular, \mb{we highlight such requirements properties as} verifiability, unambiguity and traceability. 
Besides, verifiability, unambiguity, and traceability are requirements properties that appear fundamental in a requirements analysis approach. 

\subsubsection{Verifiability} 
A requirement is verifiable if and only if there exists a process that can prove that the system satisfies the specified requirement \cite{ieee1998}. Although the notion of verifiability is defined in IEEE standard, not much guidance is provided on how to achieve this property apart from the statement that \textit{``verifiability is enhanced when the requirement is measurable''} \cite{iso2018}. 
% \mb{See where to refer to the verification survey -> added to section \ref{verifiability}}

\subsubsection{Unambiguity} 
Ambiguity refers to possible different interpretations of one and the same requirement. Natural language, which is a predominant way of specifying requirements, is an innate source of ambiguity \cite{34}. The only way to avoid ambiguity is to introduce some level of formality: to constrain the natural language or to utilize formal or semi-formal methods and notations. At the same time, formal methods are not widely used. The key reasons for that are that, on the one hand, formal methods require learning related notation and modeling techniques, which often require a strong mathematical background, and on the other hand, formal specifications are not readable, which makes it difficult to share the information and ensure specification accuracy. In view of this, we can clearly identify the need for a method to produce unambiguous requirements and specifications. 

\subsubsection{Traceability} 
Requirements traceability is the ability to follow both the sources and consequences of requirements in the rest of the product \cite{Handbook}. Requirements traceability ensures that the impact of changes in requirements specification is easily localizable in the code, which significantly decreases the time and costs of assessing the impact of changes and implementing the changes. Traceability relations can also be used to verify that the system meets its requirements. 

Even in regulated or safety-critical domains, such as healthcare or the military, traceability links are often created at the very end of the process and contain many deficiencies, such as incomplete, missing, or erroneous traceability data \cite{cleland2014software}. In less formal projects, traceability is often perceived as a ``made up problem'' or an ``unnecessary evil'' \cite{cleland2014software}. Consequently, only 40\% of software projects practice requirements traceability \cite{fricker2015requirements}. 

At the same time, the projects with missing traceability links are vulnerable to change and evolution: it can be tremendously hard to identify all the system elements where the changes should be introduced.

\subsection{Characteristics of a usable approach to requirements}
\label{usable}
A requirements approach is a strategy for developing and managing project requirements together with supporting guides and tools. To be used in industry, an approach must meet certain criteria, which will be explored in further subsections.

\subsubsection{Documentation}

To compete with established, extensively documented approaches to requirements, a new candidate must provide potential users with a clear, actionable description of how to apply the methodology \cite{methodology_role}.


%There exist many approaches to requirements; the most mature  ones are covered in textbooks, video tutorials, and university curricula. In order to switch from such an approach to a new one, a person would expect to find detailed guidance in the supporting materials on how to apply the approach. It was demonstrated that the application of a new requirements methodology is more efficient when requirements engineers use the procedural knowledge of an approach (i.e., the examples of the application of the methodology) \cite{methodology_role}.

\subsubsection{Ease of learning}
In many cases, the primary competitor to a proposed requirements approach is natural language, which does not require any learning. While any more formal method will require learning new concepts and notations, the effort should remain commensurate with the expected benefits \cite{davis2013study}. 
%Requirements specification approaches place different demands on the qualification of requirements engineers. For example, natural language requirements do not require specific knowledge, whereas formal methods require significant training. Too high prerequisites on requirements engineers’ background may hinder the method's adoption \cite{davis2013study}.

\subsubsection{Tool support}
Relying on tool-supported requirements engineering facilitates capturing, tracing, analyzing, and managing changes to requirements \cite{ibm_rq}. According to state of practice surveys \cite{kassab2015, RE_in_china, franch2023state}, 40 to 67\% of projects use requirements specification and management tools, with a tendency that the access to such tools is better in large multi-national corporations \cite{RE_in_china}. 

Tool support is essential for ensuring requirements traceability and managing changes in requirements: manual performance of such tasks is tedious and decreases the project's agility.

\subsubsection{Requirement artifacts reusability} 
\label{reuse}

Reusability is the degree to which an asset can be used in more than one system \cite{iso2017}. Reusability is widely adopted in software engineering: libraries of reusable components have become an integral part of programming languages. Requirements reuse has not reached a comparable level.

The study of Irshad et al. \cite{reusability_survey} reviews approaches to requirements reuse. The majority of the approaches (57\%) suggest reusing requirements in a textual form, or the form of requirements reuse is undefined. Other forms of requirements reuse include templates, use cases, modeling language-based artifacts, formal models, and features. In practice, requirements reuse is mostly limited to copying and modifying natural language requirements from previous projects \cite{palomares2017requirements}. %Improper reuse or requirements is dangerous: simple copy-pasting requirements from preceding projects may lead to catastrophes. as was the case in Therac-25 project \cite{baase2008ethical}.

\subsubsection{Seamlessness}
\label{seamlessness}
Software development may rely on several different notations, such as natural language, modeling language, formal language, and programming language. The process of switching from one notation to the other, when not seamless, is prone to errors. Seamless software development uses a uniform method and notation throughout all activities, such as problem modeling and analysis, design, implementation, and maintenance \cite{33}. Seamless software development facilitates traceability between requirements and other software artifacts. 


\section{Unified Object-Oriented approach to Requirements}

\label{UOOR_chapter}

The idea of capturing software requirements with contracts has given  rise to several approaches \cite{32, 39, galinier2021seamless}. They address specific aspects of OO requirements and fall short of the goal of this article: offering a comprehensive requirements methodology.  
Starting from such ideas as Multirequirements \cite{32}, SOOR \cite{39}, SIRCOD \cite{galinier2021seamless} and PEGS \cite{Handbook}, the proposed methodology, Unified Object-Oriented approach to Requirements (UOOR) benefits from OO techniques, which have proven to be successful in software implementation, and scenarios,  commonly used for requirements. 

The  approach   unifies scenarios as a commonly used requirements technique with software contracts as a requirements formalization approach, which does not require a specific background in formal methods \cite{naumcheva2022object}. The reasons for combining these two techniques are the following:
\begin{itemize*}
    \item Scenarios are widely used in industry and are perceived as an excellent tool for eliciting requirements and communicating with project stakeholders.
    \item Scenarios, formulated in natural language, are ambiguous and lack abstraction.
    \item Software contracts provide means for requirements formalization.
    \item Software contracts do not rely on a specific mathematical notation and are as easy to formulate as ``if ... then ...'' instructions. 
    Capturing requirements with contracts makes them verifiable, reusable and traceable.
\end{itemize*}

%As discussed earlier in section~\ref{seamlessness}, 
In line with the seamlessness principle,  we rely, for the formal requirements notation, on a statically typed object-oriented programming language: Eiffel   \cite{Eiffel}, chosen for its readability and support for contracts. Examples in this article will be expressed in Eiffel and  the tool support will be based on the facilities of the EiffelStudio IDE  \cite{19}. Any other statically typed OO language supporting contracts could be used; for example, the RQCODE approach provides a framework for seamless expression of security requirements in Java \cite{ildar2, ildar3}. 

The UOOR approach builds on the general idea of object-oriented requirements and adds its own specifics. The following concepts are common to all OO requirements approaches:

\begin{itemize*}
    \item Object types, described through applicable operations -- queries (providing information) and commands (updating information).
    \item Software contracts, which capture the semantics of operations.
\end{itemize*}

To these, UOOR adds:
\begin{itemize*}
    \item Specification drivers \cite{38}, which capture the system's behaviors (section~\ref{subsection4}).
    \item Support for  traceability through seamlessness (section~\ref{toolchapter}).
\end{itemize*}
UOOR describes requirements specification across three dimensions: object model; functional specification; behavioral specification.  

\begin{comment}
    (see Fig.~\ref{UOOR_model}).

\vspace{-6pt}
\begin{figure}[htb!]
    \centering
    \includegraphics[width=0.6\linewidth]{UOOR_model.jpg}
    \caption{Three dimensions of UOOR requirements.}
    \label{UOOR_model}
\end{figure}

\end{comment}


An \textbf{object model} captures key abstractions in the application domain in the form of software classes. This static model specifies abstract data types for all relevant environmental phenomena.


A \textbf{functional specification} provides a specification of individual operations. Operations correspond to features of classes of the object model. Their abstract specification relies on in-class contracts. 


A \textbf{behavioral specification} defines permissible sequences of operations. Behavioral specification relies on the following mechanisms:

\begin{itemize*}
    \item Specification drivers capture scenarios as example sequences of operations and may serve testing purposes.
    \item Software contracts provide more abstract specifications of properties that would otherwise be expressed as time-ordering constraints.
\end{itemize*}

\begin{figure*}[!ht]
    \centering
    \includegraphics[width=\textwidth]{UOOR_flow4.jpg}
    \caption{Overview of the UOOR approach.}
    \label{UOOR_overview}
\end{figure*}

Object-oriented principles pervade the whole approach --- not as an afterthought (as in ``let's make these requirements OO now!''), but as a fundamental modeling discipline applied from beginning to end.
%In other words, the idea is not to produce some traditional kind of requirements and then make the result object-oriented; instead, OO principles help structure the entire effort by providing a unifying conceptual framework for describing the system and its environment.

The overall process iterates the following  basic steps, each with a specified resulting product:
\begin{enumerate}
    \item Eliciting and documenting requirements, resulting in natural-language requirements document. (section~\ref{subsection1}).
    \item Devising a OO model, which captures the key components of the system and its environment (section~\ref{model_environment}).
    \item Devising functional OO requirements, which enrich the OO model with features expressing functional requirements and their contracts, as well as  environment properties and their constraints (section~\ref{subsection3}). 
    \item Behavioral specification, which adds abstract properties of the system's functions (such as time-ordering constraints) extracted from scenarios, as well as concrete scenario examples expressed as specification drivers (contracted routines of scenario classes), also useful for verification purposes (section~\ref{subsection4}). 
    \item Refinement, leading to implementation classes, typically inheriting  from requirements classes and hence satisfying the corresponding requirements-level contracts (section~\ref{subsection_refinement}). 
\end{enumerate}


Fig.~\ref{UOOR_overview} shows the overall UOOR process. The following sections~\ref{subsection1}~-~\ref{subsection4} provide the details of each step.



\subsection{Eliciting and documenting requirements}
\label{subsection1}
The UOOR approach does not advocate any particular methodology for eliciting and documenting requirements. One possible starting point is the PEGS template \cite{Handbook}, which organizes requirements into four books: Project (focusing on project management aspects and not essential for the present discussion), Environment, Goals and System. It has room for all important requirements elements, for example:

\begin{itemize*}
    \item System and environment components (in the System book).
        \item System functions (also in the System book).
    \item Environment assumptions, constraints, and invariants (in the Environment book).
    \item Scenarios (in the Systems or Goals book depending on the nature of the scenarios)..
\end{itemize*}
%Whereas any other template may be used to document requirements in the UOOR approach, it is important that it includes the aforementioned elements.
Any template, whether PEGS or another one, should document each requirement in a separate paragraph.
%Otherwise, the approach leaves freedom in the process of requirements elicitation and provides specific guidance on how to proceed with the elicited data (i.e., how to analyze scenarios and environment properties).

%The approach will also tackle the important task of tracing requirements from requirements documents to other project artifacts and in the reverse direction (see sections~\ref{subsection3},~\ref{traceability}).

\subsection{Modeling components of the system and its environment}
\label{model_environment}

To produce an OO model, a requirements engineer should describe key abstractions in the application domain with classes. These classes should cover both the system and its environment. We will use the example of a simple Library Management System (LMS) to illustrate the approach.

Examples classes are: \e{LIBRARY} (system), \e{BOOK} and \e{PATRON} (both environment).  During its operation, the system manipulates objects - instances of these classes. In the LMS, such objects will be computer representations of such library objects --- physical or conceptual --- as books, patrons and catalog. 

The PEGS template includes sections for environment components (E.2 in the Environment book) and system components (S.1); for traceability, we can link their natural-language components with elements of the corresponding classes in the OO model (details appear in below in sections~\ref{traceability} and ~\ref{traceability_tool}).

Object technology supports modeling systems at various levels of abstraction. In particular, both a feature and a class may be either effective or deferred\footnote{The respective terms ``concrete'' and ``abstract'' are also used.}. ``Effective'' means fully implemented (and hence ready to execute as part of a program). A feature or class is “deferred” if its definition does not include  an implementation or (for a class) includes a partial implementation only; it may still, however, include an abstract specification of the properties of the feature or class, in the form of a ``contract'' as explained below. Deferred features and classes are particularly useful for requirements since requirements focus on specifying behavior rather than implementing it. For example, \e{LIBRARY_ITEM} may be a deferred class since it describes an abstract concept with several possible concrete realizations, such as \e{BOOK}. In some cases, requirements classes have to be effective, in particular, scenario classes and test classes.

Client and inheritance relations capture the relations between the environment and system components. For example, operations on a book  involve, among other objects, a patron and a library. The class representing a book will be a client of the
classes representing the other two concepts. 
A class inherits from another if it represents a specialized
or extended version of the other's concept. For example, books
and magazines both belong in the general category of library
items, which can be represented by a class \e{LIBRARY_ITEM}.
\e{BOOK} and \e{MAGAZINE} are descendants of \e{LIBRARY_ITEM}
through the inheritance relation. As another example, there could be two categories of patrons, regular and research patrons, with different restrictions such as the number of books on hold or the hold's duration. In that case, classes \e{REGULAR_PATRON} and \e{RESEARCH_PATRON} would
inherit from the class \e{PATRON}, which captures features and contracts applicable to all patrons.

Each class defines the properties of the corresponding objects through the operations applicable to them: queries and commands. Queries provide information about objects, whereas commands update the corresponding objects. 

A specification of the properties of systems and their objects,  through the list of associated classes and their features, only gives structural properties. To provide the actual semantics of these elements — other than through the implementation – we should also express their abstract properties. Contracts fulfill this need. They include:

\begin{itemize*}
    \item Preconditions of a feature, specifying the conditions under which it can be used.
    \item Postconditions of a feature, expressing properties resulting from its application.
    \item Class invariants, expressing consistency properties applicable to all instances of a class.
\end{itemize*}

Such a contract element consists of assertions, each of which is an individual boolean property applicable to the corresponding objects, such as (for a \e{BOOK} instance) \textit{``the book is currently on loan''}. We rely on contracts to express:
\begin{itemize*}
    \item Properties of the environment, such as \textit{``a patron can reserve no more than 5 books”} (discussed below in the present section). 
    \item Properties of the system's functions, such as \textit{``if a patron has less than five holds, placing a hold is successful and the
book becomes reserved''} (discussed in section 
\ref{subsection3}).
\end{itemize*}

The important part of the OO model is the properties of the system's environment. According to \cite{Handbook} those properties are of  three kinds:
\begin{itemize*}
    \item Assumptions - properties, that are satisfied by the environment, so the system takes them for granted (\textit{``The length of a book title does not exceed 100 characters''}).
    \item Constraints - obligations coming from the environment that the system must comply with (\textit{``a patron can reserve no more than 5 books''}).
    \item Invariants - environment properties that the system must maintain (\textit{ ``A book can have exactly one of the following statuses: available, on hold, borrowed, due''}).
\end{itemize*}

Various types of environment properties play different roles in software development. 

Assumptions facilitate the work of the developers,  by restricting the set of cases to be handled by the system, as in  \textit{``The length of a book title does not exceed 100 characters''}. They are engineering decisions and may need re-examination in later stages of the project; stating them explicitly is critical to assess their consequences and uncover possible violations, particularly at the time of system validation and verification. (The famous Ariane-5 software failure was due to an unspotted obsolete assumption.) 

%provides information on the size of manipulated data when performing operations with the book titles. At the same time, assumptions (and this one in particular) are falsifiable: they can be wrong. Expressing the assumptions explicitly with assertions helps in verifying their validity: if during system verification and validation, the assertion expressing the system assumption is violated, the respective error message will be displayed. The developers will then be able to assess and address the consequence of reconsidering the assumption. 

Constraints and invariants express the properties of the environment that the system must respect. Such constraints can be expressed in the associated class texts as class invariants. The  constraint \textit{``a patron can reserve no more than 5 books''} can be expressed as the following invariant clause in class \e{PATRON}: \textit{num\_reserved <= 5 }. Listing~\ref{book_class} provides another example.

\begin{comment}
\begin{lstlisting}[caption={Implementation of an invariant \textit{``A patron can reserve no more than 5 books''.}}, captionpos=b, label=5books]
deferred class 
    PATRON
feature
    num_reserved: INTEGER
       -- Number of books reseved by the patron
invariant
    num_reserved <= 5
end
\end{lstlisting}
\end{comment}

\begin{lstlisting}[caption={Expressing the invariant property \textit{``a book can have exactly one of the following statuses: available, on hold, borrowed, due.''}}, captionpos=b, label=book_class,language=OOSC2Eiffel]
deferred class BOOK feature
    is_available, is_on_hold, is_borrowed, is_due: BOOLEAN
invariant
    is_available implies not (is_on_hold or is_borrowed or is_due)
    is_on_hold implies not (is_available or is_borrowed or is_due)
    is_borrowed implies not (is_available or is_on_hold or is_due)
    is_due implies not (is_available or is_on_hold or is_borrowed)
end
\end{lstlisting}



%(listing~\ref{5books}):


\vspace{12pt}
We can use pre- and postconditions to express properties of operations on the components of the environmen, such as, in class \e{PATRON}, \textit{``a patron can place hold on a book only if the book is available.''}

This constraint is implemented as a precondition of the feature \e{place_hold} of the class \e{PATRON} (listing~\ref{precond_constraint}):

\begin{lstlisting}[caption={Implementation of the constraint \textit{``A patron can place hold on a book only if the book is available.''}}, captionpos=b, label=precond_constraint,language=OOSC2Eiffel]
deferred class PATRON feature
    num_reserved: INTEGER
        -- Number of books reserved by the patron       
    place_hold(b: BOOK)
        require 
            b.is_available
        deferred
        end
invariant 
    num_reserved <= 5
end
\end{lstlisting}

%The constraint, related to the feature ``place hold'' saying that a patron cannot reserve more than 5 books, is implicit. But what happens if he or she has 5 books reserved and attempts to reserve one more? One option is that the book reservation will be denied. Another option is to cancel the oldest hold and complete the last one. To avoid the ambiguity we add one more constraint (which is in fact an implicit business rule): \textit{``A patron can place hold on a book if she has less than 5 books on hold.''}

%To express this constraint, we add one more precondition to the feature \e{place_hold} %(listing~\ref{patron_precond}):

\begin{comment}
    
\begin{lstlisting}[caption={Implementation of the constraint \textit{``A patron can place hold on a bookif she has less than 5 books on hold.''}}, captionpos=b, label=patron_precond]
deferred class 
    PATRON
feature
    num_reserved: INTEGER
       -- Number of books reserved by the patron
       
    place_hold(b: BOOK)
        require
            b.is_available
            num_reserved < 5
        deferred
        end
invariant
    num_reserved <= 5
end
\end{lstlisting}
\end{comment}

\subsection{Producing an OO functional specification}
\label{subsection3}

In object-oriented requirements, elements of system functionality are expressed by features of classes. Contracts capture theif properties. 


Consider the following requirement: \textit{``The Library Management System shall provide the ability to place a hold on books''}. To express it,
%,requirement%in an object-oriented style
we add a feature \e{place_book_on_hold} to the \e{LIBRARY} class (listing~\ref{library_precond}). 


\begin{lstlisting}[caption={Implementation of the requirement \textit{``The Library Management System shall provide the ability to place hold on books.'' }}, captionpos=b, label=library_precond,language=OOSC2Eiffel]
deferred class 
    LIBRARY
feature       
    place_book_on_hold (b, p)
      -- Reserve a book b by patron p
        deferred
        end

\end{lstlisting}



Other system specifications follow from environment constraint, such as: \textit{A patron is limited to five holds at any given moment}. This constraint, already expressed in the OO model through the invariant of the environment class \e{PATRON}, yields a system property, governing the operation \e{place_book_on_hold}:
\begin{itemize*}
    \item If hold is allowed, placing a hold is successful and the book becomes reserved.
    \item If hold is not allowed, placing a hold is not successful, and the book remains available.
\end{itemize*}

Listing~\ref{place_hold_constraints} provides the OO implementation of these two constraints.

\vspace{-6pt}
\begin{lstlisting}[caption={Implementation of the constraints on the \e{place_book_on_hold} feature.}, captionpos=b, label=place_hold_constraints,language=OOSC2Eiffel]
    
deferred class LIBRARY feeature
   place_book_on_hold (b: BOOK; p:PATRON)
    -- Reserve a book b by patron p        
      deferred
      ensure
         old p.num_reserved < 5 implies 
           (b.is_on_hold and 
            b.patron.is_equal(p) and 
            p.num_reserved=old p.num_reserved+1)
         old p.num_reserved >= 5 implies 
           (b.is_available and 
            p.num_reserved = old p.num_reserved)            
      end

\end{lstlisting}


We should also link object-oriented requirements, expressed in an OO model, to  their counterparts in a natural-language requirement document. Hyperlinks between the two kinds of documents, discussed in section~\ref{traceability_tool}, will fulfill that purpose. 
%Clicking the hyperlink from the class code will open the requirements document at the bookmark, corresponding to the respective requirement. Clicking the hyperlink in the requirements document will open the source code at the feature that is specified by the given requirement. 

\subsection{Producing an OO behavioral specification} 
\label{subsection4}

%As noted in section~\ref{approaches}, it is common to rely on use cases or other forms of scenarios to model system behaviors. These techniques are not, however, the only way to describe behavior. In this section, we review how UOOR helps achieve this goal. 

The behavioral specification in UOOR relies on:
\begin{itemize}
    \item Expressing concrete scenario examples as specification drivers.
    \item Extracting abstract properties of operations from scenarios (such as time-ordering constraints).
\end{itemize}

A scenario, also known as a use case, is a pattern exercising the features (operations) of one or more classes. Use cases are procedural rather than object-oriented, but complement the OO model by describing practical examples of use of its abstractions and are particularly useful for both requirements elicitation and for system testing. In the  Use Case 2.0 approach\cite{26}, a successor to Ivar Jakobson's original use case, a \textit{use-case story} describes a possible path through a use case that is of value to a user or other stakeholder. 

UOOR expresses such a story as a routine, exercising the features of the target classes. The routines, specifying a set of related use case stories can be grouped in a separate class, providing ``specification drivers'' \cite{39}.
%, presented in section~\ref{contract_based}.
The following is an example, with specification drivers exercising features of classes \e{BOOK}, \e{PATRON} and \e{LIBRARY} (listing~\ref{use_case_stories_class}):

\begin{lstlisting}[caption={Use case stories class.}, captionpos=b, label=use_case_stories_class, language=OOSC2Eiffel]
class LIBRARY_BOOK_USAGE_STORIES feature  
    reserve_book_successfully (b: BOOK; lb: LIBRARY; p: PATRON) 
        require
            p.num_reserved < 5
            b.is_available
        do
            lb.place_book_on_hold (b, p) 
        ensure
            b.is_reserved 
            b.patron = p
        end             
    reserve_book_num_holds_exceeded (b: BOOK; lb: LIBRARY; p: PATRON)
        --See the implementation in a Github repo       
    -- Other use case stories
end

\end{lstlisting}

Expressed in this OO style, use case stories double down as test cases when provided with actual values for their arguments. If we call \e{reserve_book_successfully}, using actual instances of \e{BOOK}, \e{LIBRARY} and \e{PATRON} as arguments, we get a test case for that story. The arguments must satisfy the preconditions; for the routine to be correct, execution must satisfy the postcondition on exit.


OO techniques avoid \textit{premature} \textit{time-ordering decisions}. While it is possible for an OO specification to state a time-ordering constraint through a scenario, object technology also supports a more general and more abstract specification style based on contracts.
%Scenarios specify the order in which operations will be executed. Enforcing such an ordering specification at the level of requirements is often a premature decision. In reality, the order of the steps is not cast in stone. Using a preset ordering is convenient for describing desirable scenarios or, more generally, the expected ones. But what happens in life is not always what we hope for or expect. What if the customer returns a damaged book?
%Should the book not remain unavailable until it is repaired? Extensions can be used to specify scenarios that depart from the standard ones. However, this solution does not scale. Writing ever more use case extensions to cover all such situations leads to an explosion of special cases which soon become intractable. In practice, it is possible to write use cases to cover the most common scenarios, but they are only a small subset of the possible ones, in the same way that, in programming, tests can only cover a minute subset of possible inputs.
The idea is that instead of a use case strictly specifying the sole possible order $o_1$, $o_2$, $...$, we specify assertions $p_i$ and $q_i$ serving respectively as precondition and postcondition of $o_i$. To specify the indicated exact ordering, we just take $p_{i+1}$ to be the same as $q_i$; but by playing with the $p_i$ and $q_i$ we can much more flexibly specify a wide range of ordering constraints rather than prematurely prescribing just one obligatory order. 

%Class \e{BOOK} specifies these logical constraints in the form of contracts (listing~\ref{time_ordering_constraints}).  


\begin{lstlisting}[caption={Main scenario.}, captionpos=b, label=main_scenario, language=OOSC2Eiffel]
-- Main scenario of the use case ``borrow a book'':
    place_hold (patron: PATRON)
    checkout (patron: PATRON)
    return (patron: PATRON)

\end{lstlisting}

As an example, the specific sequence of actions described in the “Main scenario” use case (listing~\ref{main_scenario}) is compatible with the logical constraints specified in the class \e{BOOK}(list. \ref{time_ordering_constraints}), as postcondition of each step other than the first implies the precondition of the next one. It is an overspecification, however, prohibiting for example the inclusion of additional operations between \e{place_hold} and \e{check_out}. The logical properties expressed by the assertions of class \e{BOOK} relax this over-constraining order while preserving the necessary ordering constraints.


\begin{lstlisting}[caption={Illustration of logical constraints.}, captionpos=b, label=time_ordering_constraints, language=OOSC2Eiffel]
deferred class BOOK feature
    is_available, is_on_hold, is_checked_out, is_due: BOOLEAN     
    place_hold (p: PATRON)
	    -- Place a hold on a book. Set is_on_hold 
        require is_available 
        deferred 
        ensure
            is_on_hold 
            not is_available 
        end        
    checkout (p: PATRON)
        -- Check out the book 
        require is_on_hold 
        deferred 
        ensure is_checked_out 
        end        
    return 
        -- Return the book to the library 
        require is_checked_out or is_due 
        deferred 
        ensure is_available
        end 
end

\end{lstlisting}



\subsection{From requirements to code}
\label{subsection_refinement}

The seamless software development process works by iteratively refining requirements into executable code. Section~\ref{refinement} describes the details of that refinement process; ~\ref{traceability} explores how it supports requirements traceability. 


\subsubsection{Refinement}
\label{refinement}
In the UOOR approach, the development process consists of refinement steps: elicited requirements are refined into OO requirements, which are further refined to implementation code. The steps are the following:
\begin{itemize*}
    \item Refine natural-language component requirements into deferred Eiffel classes, which will constitute the object model of the system and its environment. %The hyperlink in a requirements document links a component requirement with the respective class. The EIS note links the class  with a component requirement in a requirements document.\se{paragraph to be discussed- EIS link not defined before, Object model?, the hyperlink? ... In fact, No reason to speak from traceability there.}
    A hyperlink in the requirements document links the considered component requirement with its respective class
    \item Formulate environment constraints as contracts in classes, implementing the environment components. %The EIS note links environment constraint with the respective feature in the object model. 
    A note (using the Eiffel ``note'' construct for adding structured annotations to the code) links an environment constraint with a feature in the implementation (class model).
    \item Refine constraints further into functional requirements and constraints, expressed both in the natural-language document and (in the respective form of features and contracts) to the OO model.
    \item Proceed to implementation by writing effective classes, which in many cases will inherit from (deferred) requirements classes, providing implementations of their deferred features. The language's rules on inheritance (invariant conservation, precondition conservation or weakening, postcondition conservation or strengthening) guarantee that they satisfy the contracts formulated in the ancestor requirements classes.
\end{itemize*}

\subsubsection{Traceability links}
\label{traceability}
To ensure two-way traceability, a requirements engineer should create traceability links of two types: from natural-language (NL) documents to code artifacts, and the other way around.

Links of the first kind will appear in the natural-language documents as hyperlinks, pointing:
\begin{itemize*}
    \item From an environment or system component to its counterpart (a class) in the OO model.
    \item From a functional requirement to its counterpart (a feature with its contract) in the OO model.
    \item From another environment property, such as an assumption or constraint, to its counterpart (such as a class invariant) in the OO model.
\end{itemize*}

Links in the reverse direction point from a software artifact, such as a class or one of its features, to some part of a natural-language document. A traceability tool developed for the UOOR approach supports such links in the form of ``note'' code annotations; see its presentation in section~\ref{traceability_tool} below.

\subsection{System verification and requirements reuse}
\label{section_verification-reuse}
The UOOR approach integrates requirements with the rest of the entire software development process, providing the means for verifying the solution against the requirements (section~\ref{verifiability}), and supports reuse of domain-specific component requirements (section~\ref{reusability}).

\subsubsection{System verification}
\label{verifiability}
UOOR requirements enable both static and dynamic verification of the implemented system.

Since OO requirements are code elements, an IDE provides basic consistency checks at compile time. When contract checking is enabled at runtime, the IDE monitors contract violations. Since every contract can have a unique tag, a developer can trace an exception to the violated contract.

Contract specifications serve as oracles for dynamic testing. In addition, scenarios, implemented as specification drivers, serve as tests when passed actual arguments.

A static verifier, such as Autoproof \cite{7}, can be used to ensure static verification of the system's functional correctness, which significatly reduces testing costs \cite{huang2023lessons}.
\textbf{Verifiability} of a requirement refers to the ease of checking that the constructed system meets the requirement. Since in UOOR requirements are captured by contracts, EiffelStudio provides the following mechanisms for requirements verification:

\begin{itemize*}
    \item When contract checking is enabled at runtime, EiffelStudio monitors contract violations. Since every assertion can have a unique tag, a developer can trace an exception to the violated assertion.
    \item Contracts serve as test oracles, which simplifies producing tests. 
    \item Since requirements are compilable software elements, the IDE provides basic consistency checks such as type checking.
    \item Autoproof \cite{7} provides static verification facilities for contracted Eiffel programs. This tool checks whether the implementation satisfies its contracts. If some of the contracts may be violated, Autoproof outputs a warning, pointing at the contract that may be violated (see Fig.~\ref{autoproof}). 
\end{itemize*}

\vspace{-6pt}
\begin{figure}[htb!]
    \centering
    \includegraphics[width=1.06\linewidth]{IDE4.jpg}
    \caption{Autoproof output for the incorrect implementation of the class \e{BOOK}.}
    \label{autoproof}
\end{figure}


\subsubsection{Requirement artifacts reusability}
\label{reusability}

Based on the capabilities provided by object-oriented technology, requirements obtained using the UOOR approach can be reused not by copy-pasting natural language texts but as libraries of domain-specific component specifications in the form of contracted deferred classes. Being implementation-independent, such specifications support different implementations.


UOOR organizes requirements around classes, which are abstractions of the objects in the application domain. Such abstractions can be organized into libraries and further shared between several projects. Thanks to genericity, requirements can be abstracted to generic modules, whose parameters represent types. The example of such a generic module is \e{CATALOG [CATALOG_ITEM]}, abstracted from a \e{BOOK_CATALOG} of the library case study, with such operations as ``has item?'', ``add item'', ``remove item'' (listing~\ref{reusable_class}). 

\begin{lstlisting}[caption={Example of a reusable requirements class.}, captionpos=b, label=reusable_class,language=OOSC2Eiffel]
deferred class CATALOG [CATALOG_ITEM] feature
    count: INTEGER
      -- Number of elements in catalog
    is_empty: BOOLEAN
      -- Is catalog empty?
        deferred
        ensure Result = (count = 0)
        end
    has (el: CATALOG_ITEM): BOOLEAN
      -- Is el an element of catalog?
        deferred
        ensure not_found_in_empty: Result implies not is_empty
        end
    add (element: CATALOG_ITEM)
      -- Add a new element to catalog
        deferred
        ensure count = 1 + old count
        end
    remove (element: CATALOG_ITEM)
      -- Remove element from catalog
        deferred
        ensure count = old count - 1
        end
end
\end{lstlisting}

Such a class defines basic operations applicable to objects of type catalog but stays free from implementation details. Due to its genericity, the class can be used for catalogs of different object types, not necessarily books or library items.

\section{Seamless requirements traceability}
\label{SeamlessTraceability}

\label{toolchapter}

\subsection{Seamless requirements traceability}

Requirements traceability is the ability  to follow both the sources and consequences of requirements \cite{Handbook}. Traceability involves \textit{traceability links}, connecting requirements with their sources and with related project elements.

%In practice, the navigation between the project elements, connected with a traceability link, can have a different nature. 

A basic technique consists of manually creating traceability matrices, by recording into a table individual links between requirements and test cases.
%Such matrices are created and updated manually, and requirements and test cases are added by copy-pasting from related documents.
Producing and maintaining such matrices requires substantial effort. They do not provide direct navigation from a requirement to a test case, but only establish a visual representation of the correspondence between requirements and related test cases. In their typical use they trace requirements not to general project artifacts but only to test cases. 

Dedicated requirements management tools (such as Polarion \cite{Polarion} and IBM DOORS \cite{69}) significantly decrease the requirements traceability effort. They allow the creation of clickable links between requirements and various project artifacts, for example by drag-and-drop. 
%In some cases it requires integration between several applications since project elements are not handled in a single application. 

\textit{Seamless requirements traceability} goes further. In addition to supporting direct links between requirements and other project artifacts, it relies on relation propagation, which is based on formal properties of relations. Three fundamental properties apply:
\begin{itemize*}
    \item Requirements are directly connected with other project artifacts with clickable links.
    \item A substantial number of links are propagated rather than created manually.
    \item Changes in requirements can be traced to the related project artifacts.
\end{itemize*}
%\cite{meyer1985software}


\subsection{UOOR traceability information model}

The traceability information model describes the relation between software project artifacts \cite{bunder2017model}. The basic traceability model has been established by the members of the Agile Project Management Forum \cite{cleland2011traceability} (see Fig.~\ref{agile_traceability_model}). This model reflects the common practice of agile projects by covering the most frequent tracing scenarios. 
\begin{figure}[h!]
    \centering
    \includegraphics[width=\linewidth]{agilemodel.jpg}
    \caption{Agile traceability information model.}
    \label{agile_traceability_model}
\end{figure}

The agile traceability information model includes three main blocks: requirements, tests (part of a test suite) and implementation. The relationship between requirements (user stories) and implementation is indirect: requirements are linked with test cases by external tools or by annotating test cases with user story ID. Between code and tests (and further to requirements), the relationship is implicit: we know that code implements the requirements if it passes the tests. 

Seamless development can extend this model by adding OO requirements, which link natural language requirements to tests and implementation code. In a coarse-grained view, the model looks as in Fig.~\ref{UOOR_traceability_model}.

\begin{figure}[h!t]
    \centering
    \includegraphics[width=1.1\linewidth]{UOOR_traceability_model4.jpg}
    \caption{UOOR traceability information model.}
    \label{UOOR_traceability_model}
\end{figure}

Requirements in the UOOR traceability information model  are linked directly to the implementation  and test artifacts. To provide a more fine-grained overview of project elements and relations between them, sections~\ref{project_elements}-\ref{relations} define the main types of project elements and relations between them. 

\subsection{Project elements}
\label{project_elements}

Project elements are the basic blocks of the model of requirements engineering activity. They are artifacts produced in the course of a software development project. It is important to identify and define project element types for the following reasons:
\begin{itemize*}
    \item To guide their possible implementations (for example, as a class or a section of a document).
    \item To make sure that requirements can be traced to certain project artifacts (such as implementation or test artifacts).
    \item To introduce type-dependent propagation rules.
\end{itemize*}
As presented in Fig.~\ref{UOOR_traceability_model}, project artifacts belong to one of four general types: natural language requirements, OO requirements, tests and implementation. Table~\ref{project_element_types} provides the definitions of these four types and of project elements belonging to each of these groups. 

\begin{table*}[htbp]
\centering
\begin{tabular}{|p{0.2\textwidth}|p{0.46\textwidth}|p{0.25\textwidth}|}
\hline
\textbf{Project element type} & \textbf{Definition}                                                                                                      & \textbf{Possible implementation}                          \\ \hline
NL requirement         & \multicolumn{2}{|l|} {A requirement expressed in a textual form}%                                            & Bookmarked   section of a document                          
\\ \hline



Component requirement          & A statement   that the system or its environment contains a certain part                                            & Bookmarked   section of a document                          \\ \hline
Functional requirement       & A statement   that the system shall provide a certain functionality                                                      & Bookmarked   section of a document                          \\ \hline
Constraint                     & A property   that restricts the behavior of a system function                                                                & Bookmarked   section of a document                          \\ \hline
Scenario                        & A description of the interaction of an actor with the system to reach a certain objective                                  & Bookmarked   section of a document                          \\ \hline
OO requirement    & \multicolumn{2}{|l|} {A requirement expressed in a programming language} %                                                     & Bookmarked   section of a document                          
\\ \hline
OO component                  & An abstract   representation of a component of a system or its environment in the form of a   class                      & Class                                                       \\ \hline
OO functional  requirement    & A feature of   a requirements class that realizes a functional requirement                                               & Feature                                                     \\ \hline
OO constraint                   & An assertion   specifying an abstract property of a feature or of a class                                                & One or more   pre- or postconditions or class invariants  \\ \hline
OO scenario                     & An abstract   representation of a scenario in the form of a routine, exercising the   routines of implementation classes & One or more   specification drivers                         \\ \hline
Test           & \multicolumn{2}{|p{14cm}|} {A programming language artifact used to test part of the functionality of a system}%                                                      & Bookmarked   section of a document                          
\\ \hline
Test case                     & A single   testing scenario                                                                                              & Feature                                                     \\ \hline
Test suite                    & A set of   tests covering one or more classes or features                                                                & Class or   cluster                                          \\ \hline
Implementation                        & \multicolumn{2}{|l|} {Code artifacts that are part of the developed system.} % & Bookmarked   section of a document                        }  
\\ \hline
Implementation   component     & A class or   cluster of the implemented system                                                                           & Class                                                       \\ \hline
Implementation   feature       & A feature of   the implemented system                                                                                    & Feature                                                     \\ \hline
\end{tabular}
\caption{Project element types.}
\label{project_element_types}
\end{table*}



\subsection{Relations between project elements}
\label{relations}

The second key element of the model of requirements engineering activity is the set of relations between project elements. Typed relations simplify traceability analysis: to assess the impact of change of a given requirement, selecting links of a certain type will yield the elements that have a certain relation to the requirement. Furthermore, it is possible to introduce formal properties of the relations, which can serve as a basis for relation propagation.

%This section does not attempt to cover all possible types of relations between project elements. Instead, it introduces some basic types and explores formal properties which can be expressed for these relations.

For each relation (except for the reflexive ones) an inverse relation will be defined. Inverse relations are particularly useful for propagating two-way traceability. The summary of project relations and their properties is presented in Table~\ref{Relations_table}.

\begin{table*}[htbp]
\centering
\renewcommand{\arraystretch}{1}
\begin{tabular}{|p{2.55cm}|p{2cm}|p{10.5cm}|}
    \hline
    \textbf{Relation} &
      \textbf{Element  types} &
      \textbf{Definition and formal properties} \\ \hline
    Repeats &
      \begin{tabular}[c]{@{}l@{}}A: ELEM\\ B: ELEM\end{tabular} &
      \begin{tabular}[c]{@{}l@{}}A repeats B if anything which is described by A is   also \\ described by B, and anything which is described by B \\ is also described by A.\\ A repeats B $\implies$ B repeats A\\ A repeats B; B $R_1$ C $\implies$ A $R_1$ C \\ A repeats B; A $R_2$ D $\implies$ B $R_2$ D \\ (where $R_1$, $R_2$ are some relations).\end{tabular} \\ \hline
    Complements &
      \begin{tabular}[c]{@{}l@{}}A: ELEM\\ B: like A\end{tabular} &
      \begin{tabular}[c]{@{}l@{}}A complements B if A and B cooperate towards \\ the   achievement of some higher aim.\\  A complements B $\implies$ B complements A\end{tabular} \\ \hline

    
    Constrains &
      \begin{tabular}[c]{@{}l@{}}A: CSTR\\ B: FR\end{tabular} &
      \begin{tabular}[c]{@{}l@{}}Requirement A constrains another requirement B if \\  it states a condition that B must satisfy.\\ constrains$^{-1}$ = is\_constrained\_by \\ A constrains B, C constrains B $\implies$ A complements C\end{tabular} \\ \hline
      
    Refines  &
      \begin{tabular}[c]{@{}l@{}}A: ELEM\\ B: ELEM\end{tabular} &
      \begin{tabular}[c]{@{}l@{}}A refines B if anything which is  described by A \\ is also described by B (but some things may be \\ described by B   which are not described by A).\\ (refines; refines) - id $\subseteq$ refines\\ inherits $\subseteq$ refines \\ refines$^{-1}$= generalizes; generalizes$^{-1}$= refines \end{tabular} \\  \hline
    Implements &
      \begin{tabular}[c]{@{}l@{}}A: NLRQ\\ B: IM\end{tabular} &
      \begin{tabular}[c]{@{}l@{}}An implementation feature or implementation \\ component A   implements a requirement B, if it provides \\ the   functionality or constraint   stated in B. \\  implements$^{-1}$ = specifies; specifies$^{-1}$ = implements \\ A refines B $\implies$ A implements B (where:\\ A is implementation feature or component;\\ B is requirement or constraint)\end{tabular} \\ \hline

      \begin{tabular}[c]{@{}l@{}}Contains \\ \end{tabular} &
      \begin{tabular}[c]{@{}l@{}}A: ELEM\\ B: ELEM\end{tabular} &
      \begin{tabular}[c]{@{}l@{}}A contains B if B is a constituent of   A.\\  contains$^{-1}$ = part\_of; part\_of$^{-1}$ = contains \\ (part-of; contains) - id $\subseteq$   complements;\\ (contains; contains) - id $\subseteq$    contains\end{tabular} \\ \hline

    Tests  &
      \begin{tabular}[c]{@{}l@{}}A: TEST\\ B: ELEM\end{tabular} &
      \begin{tabular}[c]{@{}l@{}}A tests B if passing A is required (but possibly not \\  sufficient) for B to be considered implemented correctly\\ tests$^{-1}$ = is\_tested\_by\\ A tests B, C tests B $\implies$ A complements C\end{tabular} \\ \hline    
    Validates  &
      \begin{tabular}[c]{@{}l@{}}A: TEST\\ B: NLRQ\end{tabular} &
      \begin{tabular}[c]{@{}l@{}}A validates B if passing A is required for B to be \\ considered implemented correctly\\validates$^{-1}$ = is\_validated\_by\\ A tests B, B refines C $\implies$ A validates C (where:\\ A is a test artifact; C is a NL requirement)\end{tabular} \\ \hline
    Refers to &
      \begin{tabular}[c]{@{}l@{}}A: ELEM\\ B: NLRQ\end{tabular} &
      \begin{tabular}[c]{@{}l@{}}A refers to B if A refers to B by its name.\\Is\_a\_client $\subseteq$  refers\_to\\ (refers\_to; refers\_to) - id $\subseteq$    refers\_to\end{tabular} \\ \hline

        \end{tabular}
    
    \caption{Relations between project elements and their properties.}
    \label{Relations_table}
\end{table*}


Figure~\ref{fig:project_elements_graph} presents project elements and some of the relations between them (some of the relations are not displayed for the sake of clarity).


\begin{figure*}[htb!]
    \centering
    \includegraphics[width=0.6\linewidth]{Projectelements3.jpg}
    \caption{Key project elements and relations.}
    \label{fig:project_elements_graph}
\end{figure*}

%Table~\ref{Relations_table} presents the summary of relations' definitions and formal properties. 


\subsection{Propagation of relations}
Requirements traceability is \textit{``the ability to describe
and follow the life of a requirement, in both forward and
backward directions (i.e., from its origins, through its
development and specification, to its subsequent deployment
and use, and through all periods of ongoing refinement and
iteration in any of these phases)''} \cite{gotel1994analysis}. The ability to trace a requirement to implementation and testing artifacts eases the process of validating delivered software against requirements and ensures the adaptability of software products to requirement changes. Manual establishment of traceability links, however, is tedious and time-consuming. Seamless software development may address this issue by providing link propagation so that the links are inferred from the initially created links, project element types, and formal properties of relations between project elements.

To illustrate the idea, let's consider the ``refine'' relation. In seamless software development, a requirement turns into an implementation as the result of a sequence of refinement steps. 
The first step -- linking a natural language requirement with its OO implementation -- is manual, yet such a link can be established in a few clicks, like in most requirements management tools in the field. 
An OO requirement is further linked with the implementation artifact via inheritance: a class implementing a requirement inherits from the requirements class. 
Not necessarily this relation is direct: in fact, there may exist several layers of inheritance. 
However, it is possible to extract the ancestors for implementation artifacts: for each class, it is possible to extract its ancestors and proper ancestors; for each feature, it is possible to identify in which ancestor class it was introduced. 
This way, it is possible to propagate the refinement links based on their formal properties (transitivity). 

When all the ``refine'' links are propagated for a given requirement, the ``implements'' relation can be inferred: if a natural language requirement and an implementation artifact are linked with the ``refines'' relation, they have to be linked with the ``implements'' relation.

%\section{Conclusion}
%This chapter introduced the notion of seamless requirements traceability. It demonstrated that by modeling requirements engineering activity as a set of project elements connected with typed links, it is possible to propagate traceability links based on formal properties of relations. As the same time, if it is possible to propagate traceability links between requirements and related implementation code, it is possible to track changes in requirements to the related code artifacts.

%Seamless requirements traceability, while a theoretical concept, requires  tool support: as it was established before, adequate tool support is one of the key issues in requirements traceability. Current functionality of EiffeStudio, however, does not support typed project elements and relations. The next chapter will present the Traceability tool, which serves as a proof of concept for seamless requirements traceability management.


\section{UOOR traceability tool}
\label{traceability_tool}

One of the key benefits of seamless software development is the ability to facilitate traceability between requirements and other project artifacts, such as implementation and tests. This ability, however, requires adequate tool support. A number of mature tools for traceability links creation and management exist in the market, although they fall short of providing full seamless traceability from requirements to code. 

\begin{comment}
    
To exploit all benefits of seamlessness, the following functionality is required: 
\begin{itemize*}
    \item To link code artifacts (features and classes) with external documents.
    \item To receive notifications of changes in requirements and to trace from requirements to related project artifacts (implementation and tests).
    \item To create typed relations between project elements.
    \item To infer traceability links base on the formal properties of the relations.
\end{itemize*}

\end{comment}
The recently developed UOOR Traceability tool \cite{zakaria} serves as a proof of concept for traceability links management in the UOOR approach. 
The tool is an addition to the EiffelStudio IDE, to which it adds the following mechanisms:
\begin{itemize*}
    \item Assigning types to project elements.
    \item Creating typed links between project elements.
    \item Displaying related links for a given project element.
    \item Displaying notifications of changes in requirements for a given project element.
\end{itemize*}

The Traceability tool is built on top of the Eiffel Information System (EIS), to which it adds four 4 buttons: ``Traceability links'', ``Add link'', ``Delete link'' and ``Track changes''.

\textbf{Traceability links management.} When adding a new link, a user assigns types to project elements, connected with the link. A link source is a code element (class or a feature). The link target can be a code element or an external document (a bookmarked section of a document). The tool automatically annotates an external document with bookmarks by assigning a bookmark to each paragraph of a text document. Once a traceability link has been created, we can visualize it directly within the code. The tool provides a dedicated interface to view and manage such links (Fig.~\ref{fig:link_management}).

%\mb{add figure with links table}

\begin{figure*}[t]
  \centering
    \includegraphics[width=\textwidth]{link_manegement.png}
  \caption{Link management in Traceability tool.}
  \label{fig:link_management}
\end{figure*}


\begin{comment}
    
\subsection{Traceability tool interface}
The Traceability tool is built on top of the Eiffel Information System (EIS), so it extends the EIS panel with several buttons: ``Traceability links'', ``Add link'', ``Delete link'' and ``Track changes''. The ''Traceability links'' button is the one a user has to press in order to access the Traceability tool: after pressing the button a user can see the traceability links table and manage the links (add, edit, remove, track changes). 
Figure~\ref{fig:traceability_grid} is a screenshot of the tool interface showing a traceability grid and the tool buttons.

\begin{figure}[!htb]
  \centering
%  \begin{adjustbox}{center=\linewidth-1cm}
    \includegraphics[width=14.5cm]{TLink_1.jpg}
%  \end{adjustbox}
  \caption{Traceability tool interface.}
  \label{fig:traceability_grid}
\end{figure}


\subsection{Link management}
\label{traceability_links_in_tool}

The tool provides the following link management functionalities:
\begin{itemize*}
    \item Creating a link.
    \item Viewing links.
    \item Editing a link.
    \item Deleting links.
\end{itemize*}


\textbf{Adding a link. }When we click to add a new traceability link, the interface allows us to select various elements for establishing the connection. Specifically, we can choose the source, the link type, the source \& target types and then define the target of the link.

For the target, there are two possible options: ``Select Class/Feature (for creating a link to internal project artifact) and ``Select Bookmark'' (for creating a link to an external do (Fig.~\ref{fig:option_selection}):
\begin{figure}[!htb]
  \centering
%  \begin{adjustbox}{center=\linewidth-1cm}
    \includegraphics[width=9.5cm]{Select_Option.png}
%  \end{adjustbox}
  \caption{Option Selection}
  \label{fig:option_selection}
\end{figure}

\begin{itemize*}
    \item \textbf{Internal Project Artifact:} This option allows us to select another component within the project. We can choose from all available features and classes inside the project(Fig.~\ref{fig:classSelector},~\ref{fig:feature_selector}).  
    
\begin{figure}[!htb]
  \centering
%  \begin{adjustbox}{center=\linewidth-1cm}
    \includegraphics[width=8.5cm]{class_selector2.png}
%  \end{adjustbox}
  \caption{Class selector}
  \label{fig:classSelector}
\end{figure}

\begin{figure}[!htb]
  \centering
%  \begin{adjustbox}{center=\linewidth-1cm}
    \includegraphics[width=12.5cm]{feature_selection.png}
%  \end{adjustbox}
  \caption{Feature selector.}
  \label{fig:feature_selector}
\end{figure}

\item \textbf{External Document or Specific Requirement:} Alternatively, it is possible to create a link to an external document or a specific requirement by selecting a bookmark within that document. 
    
\begin{figure}[!htb]
  \centering
%  \begin{adjustbox}{center=\linewidth-1cm}
    \includegraphics[width=8.5cm]{Choose_bookmark.png}
%  \end{adjustbox}
  \caption{Selecting Bookmark.}
  \label{fig:selecting_bookmark}
\end{figure}
\end{itemize*}

Once a traceability link is created, we can visualize it directly within the code. The tool provides a dedicated interface to view and manage these links.

\textbf{Viewing Links.} Within the class code each link appears as a note with several parameters, including the path to the linked element, types of project element and the link type (Fig.~\ref{fig:links_in_the_code}). Links can be modified within the class code, and after compilation the tool will display the modified links. Similarly, all changes made within the tool will be reflected in the note after recompilation. The path, displayed in the note and the tool is clickable, and clicking it will open the referenced project element. The note can appear at the level of a class (to link a class with another project element) or at the level of a feature (to link a feature with another project element)

 \begin{figure}[!htb]
  \centering
%  \begin{adjustbox}{center=\linewidth-1cm}
    \includegraphics[width=15cm]{EIS_note.png}
%  \end{adjustbox}
  \caption{Links in the code.}
  \label{fig:links_in_the_code}
\end{figure}

\textbf{Editing links. } A link can be modified inside the tool by clicking at a cell of a traceability table that has to be modified. In particular, the type of the link, project element types and the path to the linked element can be modified this way. Any link can be deleted by selecting it and pressing the ``Delete link'' button. In order to store the information, the project should be recompiled. 
\end{comment}


\begin{figure*}[b]
  \centering
    \includegraphics[width=\textwidth]{tracking_changes.png}
  \caption{Tracking changes in Traceability tool.}
  \label{fig:track_changes}
\end{figure*}

\textbf{Tracking changes.} The main functionality a project may expect from enforcing requirements traceability is the ability to identify where and how a change in requirements may affect the source code. The Traceability tool addresses this issue by linking requirements in a requirements document with the related code elements, and monitoring changes in text requirements.

The ``Track changes'' button makes it possible to find out changes in requirements and their consequences. If it finds relevant changes, the tool compares the content at each bookmark with the previous version, presenting the outcome in a table showing pairs of requirements and code elements, as illustrated in Fig.~\ref{fig:track_changes}. The table displays the path to the requirements document, modification date, affected project element, and modified requirement. Clicking on ``Click do display'' opens the new version of the modified requirement so that a developer can make relevant adjustments in the code. 



\begin{comment}
    
\subsection{Tool limitations and evaluation}
The Traceability tool, developed %as a proof of concept 
for the UOOR approach, has demonstrated that it is feasible:
\begin{itemize*}
    \item To established typed links between software project elements within the IDE.
    \item To track changes in requirements and receive notifications that a requirement, related to a given code element, has been modified.
\end{itemize*}

Being a proof of concept, this tool has a lot of room for improvement, in particular:
\begin{itemize*}
    \item Improving user interface by providing customizable dashboards at different levels of granularity.
    \item Implementing links propagation.
    \item Providing advanced notification system, integrated with communication tools (email, messengers).
\end{itemize*}

\end{comment}

\section{Roborace case study}
\label{Roborace}
Beyond the simple examples used above, an exploratory study has applied the UOOR approach to a practical project, using standard guidelines for empirical software engineering research \cite{casestudy_guidelines}. The example covers some of the requirements of the Roborace system for self-driven race cars. The case study was conducted in collaboration with the autonomous systems group of Constructor Tech (formerly SIT Autonomous) \cite{SIT_autonomous}.

The objective of the study is to explore the following research questions:

%\mb{add citation}
%Add bibliographic reference to https://institute.constructor.org/events/driving-the-future-with-ai

\begin{itemize*}
    \item Is the UOOR approach expressive enough to formulate requirements for a significant project?
    \item Does it facilitate requirements specification?
\end{itemize*}

Roborace is a global championship between autonomous cars. The hardware (the race cars) is the same for all participating teams; each gets access to an autonomous race car called Devbot 2.0 and develops software to drive it in races completely autonomously. Each season sees changes in the goals and rules and the introduction of new conditions.

Use cases helped provide a coarse-grained overview of the race-car functionality; Table~\ref{raceUC}  shows one example, \textit{``Race without obstacles''}, in the Cockburn's style of use-case specification \cite{cockburn}.


\begin{small}
    
\begin{table*}[htb!]
\centering

\begin{tabular}{ |p{0.17\textwidth}|p{0.77\textwidth}| } 

%\centering
%\begin{tabular}

 \hline
 Name & Race\_no\_obstacles \\[4pt]
  \hline
 Scope & System  \\ [4pt]
  \hline
 Level & Business summary  \\ [4pt]
 \hline
 Primary actor & Race car Operator \\[4pt]
 \hline
 Context of use & Race car has to obey an instruction \\ [4pt]
 \hline
 preconditions  & * Race car is on the racetrack grid.\\
                & * Race car is not moving. \\
                & * The global plan (trajectory and velocity profile) minimizing the race time is calculated. \\
                & * The green flag is shown. \\[4pt]
 \hline
 Trigger & The system receives a request from the race car operator to start the race \\[4pt]
 \hline
  Main success scenario & * The system calculates the local plan (path and velocity profile) during the race, trying to follow the global plan as closely as possible. \\
                        & * The race car follows the local plan.\\
                        & * After finishing the required number of laps the race car performs a safe stop.  \\[4pt]
\hline
Success guarantee & The race car has completed the required number of laps and stopped.\\[4pt]
\hline
Extensions
 & A. The red flag is received during the race \\
 & * The race car recalculates a global plan to perform an emergency stop. \\
 & * The race car performs an emergency stop.\\
 & B. The yellow flag is received during the race.\\
 & * The system sets the speed limit according to the received value.\\
 & * The race car finishes the race following the global trajectory and not exceeding the new speed limit.\\
 & C. The difference between the calculated (desired) location and the real (according to the sensors) location is more than a given threshold.\\
 & * The race car recalculates a global plan to perform an emergency stop. \\
 & * The race car performs an emergency stop.\\[4pt]
 \hline
Stakeholders 
& Race car Operator (requests the car to start the race). \\
and interests & Roborace Manager (sets the race goals and policies).\\
& Roborace Operator (shows the green, yellow, and red flags).\\[4pt]
 \hline
%\end{tabular}
\end{tabular}
\caption{A detailed description of the \textit{``Race without obstacles''} use case.}
\label{raceUC}
\end{table*}
\end{small}

\subsection{Modeling components of the system and its environment}
%In autonomous driving domain the system interacts with many environment components. A dedicated effort was made to make all assumptions and constraints explicit. The interviews were combined with domain exploration (by studying related papers and normative documents) to ensure that important environment properties are not overlooked. 

We identified the core components of the system at the start of the project, making it possible to assign functionality to particular modules; the rest of the case study focused on one of them, the planning module. The \e{RACECAR} class (listing~\ref{yellow_flag_constraint}) captures properties applicable to the system as a whole.


\begin{comment}
    
\begin{lstlisting}[caption={Example of a system class.}, captionpos=b, label=racecar_class]

class 
    RACECAR 
feature 
    control_module: CONTROL_MODULE 
    perception_module: PERCEPTION_MODULE 
    planning_module: PLANNING_MODULE 
    localization_and_mapping_module: LOCALIZATION_AND_MAPPING_MODULE
end
\end{lstlisting}
Listing~\ref{racetrack_class} provides an example of an environment class.

  
\begin{lstlisting}[caption={Example of an environment class.}, captionpos=b, label=racetrack_class]

class 
    RACETRACK 
feature 
    raceline: RACELINE
      -- Optimal raceline for the track 
    map: MAP
      -- Coordinates of the bounding lines 
end

\end{lstlisting}
\end{comment}

The complexity of the domain required splitting the environment into three clusters:
\begin{itemize*}
    \item Environment - elements of the system's environment, with such classes as \e{RACETRACK}, \e{MAP} and \e{OBSTACLE}.
    \item Interfaces - the race car's sensors and actuators, which are external to the developed software, but are part of the cyber-physical system and serve as interfaces between the system and its environment. They include such classes as \e{SENSOR} and \e{LIDAR}.
    \item Auxiliary classes - classes that define abstract concepts of the environment, that are not identified as the environment's elements. They include such classes as \e{LOCATION} and \e{ORIENTATION}.
\end{itemize*}

Inheritance makes it possible to organize the description of these concepts into hierarchies. For example, classes describing sensors, such as \e{LIDAR} and \e{CAMERA}, inherit from \e{SENSOR}, which captures the characteristics applicable to all sensors. In addition, each class describing a particular sensor type captures specific features of that sensor. 

\begin{comment}
    
\begin{lstlisting}[caption={Code of the \e{SENSOR} class}, captionpos=b, label=sensor_class]
deferred class
    SENSOR
feature
    position: LOCATION_3D
      --location in the world coordinates of the scene
    update_rate: REAL
      --sensor update rate
end

\end{lstlisting}  
%\end{figure}

The \e{SENSOR} class has such features as \e{position}, which captures the sensor location on a vehicle, and \e{update_rate} which reflects the frequency of sensor output.  The \e{LIDAR} class inherits from \e{SENSOR} and adds such features as \e{point_cloud} and \e{orientation}.

\begin{lstlisting}[caption={Code of the \e{LIDAR} class.}, captionpos=b, label=lidar_class]

deferred class
    LIDAR
inherit
    SENSOR
feature
    point_cloud: ARRAY2 [LOCATION]
		-- m by n matrix of detected points in lidar coordinate system

    object_points_distance: ARRAY2 [REAL]
		-- m by n matrix of distances to object points

    orientation: ORIENTATION
		-- Lidar orientation in the world coordinates of the scene	
end


\end{lstlisting} 

\end{comment}

General environment constraints, such as the constraint \textit{``If the yellow flag is, up race cars should limit their speed to a dedicated ``safe speed''''} are expressed through the class invariants, as in the following extract from the \e{RACECAR} class (listing~\ref{yellow_flag_constraint}). 



\begin{lstlisting}[caption={Implementation of the constraint \textit{``If the yellow flag is up race cars should limit their speed to a dedicated safe speed.''}}, captionpos=b, label=yellow_flag_constraint]
class RACECAR feature
    control_module: CONTROL_MODULE 
    perception_module: PERCEPTION_MODULE 
    planning_module: PLANNING_MODULE 
    localization_and_mapping_module: LOCALIZATION_AND_MAPPING_MODULE

    green_flag_is_up, yellow_flag_is_up, red_flag_is_up: BOOLEAN
    safe_stop_activated: BOOLEAN
    max_speed: REAL	
    current_max_speed: REAL		
        -- Current speed limit
    safe_speed: REAL
        -- Safe speed limit
invariant
    yellow_flag_is_up implies current_max_speed = safe_speed
    green_flag_is_up implies current_max_speed = max_speed
    red_flag_is_up implies safe_stop_activated
end
\end{lstlisting}

Environment assumptions are specified with pre- and postconditions, as the assumption related to the possible sequences of raising flags\footnote{A flag is a signal sent to a race car to indicate a change in racing conditions.} by the Roborace (listing~\ref{roborace_flags}). 

\begin{lstlisting}[caption={Illustration of an environment assunption.}, captionpos=b, label=roborace_flags]
    
class ROBORACE feature
    raise_yellow_flag 
        require green_flag.is_up 
        do 
        ensure 
            yellow_flag.is_up 
            not green_flag.is_up 
            not red_flag.is_up
        end 
    raise_red_flag 
        require green_flag.is_up or yellow_flag.is_up
        do 
        ensure 
            red_flag.is_up 
            not green_flag.is_up 
            not yellow_flag.is_up
        end
    end

\end{lstlisting}

\vspace{-6pt}
\subsection{Producing OO functional specification}
In the Roborace project, each mission focuses on the accomplishment of some scenarios, such as ``Race without obstacles''. Thus the elements of the system's functionality were extracted from scenarios. 

Initially, the system's functions appear simply as the features' names. 
The features are further enriched with contracts, formulating the requirements. Below is an implementation of the requirement ``At every position on a raceline the speed in the velocity profile shall not exceed the race car's maximum speed'' (listing~\ref{calculate_raceline2}).

\begin{comment}
    
(list.~\ref{calculate_raceline}): 

\begin{lstlisting}[caption={Initial implementation of a requirement ``Planning module shall calculate a raceline''.}, captionpos=b, label=calculate_raceline]
class PLANNING_MODULE feature
    calculate_race_trajectory (circuit_map: MAP; vehicle_param: VEHICLE_PARAMETERS; strategy: INTEGER)
      -- Calculate the optimal racing line for a circuit   
        do
        end
end
    
\end{lstlisting}

\end{comment}



\begin{lstlisting}[caption={Implementation of a requirement \textit{``At every position on a raceline the speed in the velocity profile shall not exceed the maximum race car's maximum speed''}.}, captionpos=b, label=calculate_raceline2]
class PLANNING_MODULE feature
    calculate_race_trajectory (circuit_map: MAP; vehicle_param: VEHICLE_PARAMETERS; strategy: INTEGER)
      -- Calculate the optimal racing line for a circuit   
        do
        ensure
            velocity_limit_obeyed: across raceline.velocity_profile as rl all rl.item < car.max_speed end
        end
end
    
\end{lstlisting}

\begin{comment}
    
Contracts also capture abstract properties derived from time-ordering constraints, such as that the global plan must be calculated before calculating the local plan (listing~\ref{local_plan}):

\begin{lstlisting}[caption={Capturing logical constraints.}, captionpos=b, label=local_plan]

class PLANNING_MODULE feature 
    calculate_local_plan: LOCAL_PLAN
      -- Calculate local path from current location 
        require car.global_plan_is_calculated
        do
        ensure car.local_plan_is_calculated
        end
\end{lstlisting}

\end{comment}

\subsection{Producing OO behavioral specification}
%\sectionmark{Roborace: integrating the use cases}
The \textit{``Race without obstacles''} use case, previously expressed in a tabular format, simply becomes a routine \e{race_no_obstacles} in the requirements class \e{ROBORACE_USE_CASES} sketched below (listing~\ref{race_no_obstacles_UC}). Where the use case has alternative flows, the routine  uses conditional expressions.


\begin{lstlisting}[caption={Implementation of the use case \textit{``Race without obstacles.''}}, captionpos=b, label=race_no_obstacles_UC]

race_no_obstacles   
    require
        not car.is_moving 
        car.global_plan_is_calculated
        car.green_flag_is_shown 
        car.is_on_starting_grid
    local local_plan: RACELINE
    do    --Sequence of system actions in use case main flow
        from
        until 
            car.race_is_finished or 
            car.red_flag_is_shown or 
            car.location_error_is_detected 
        loop
            if car.yellow_flag_is_shown then update_speed end
            local_plan := car.planning_module.calculate_local_plan
            car.control_module.move (local_plan.speed, local_plan.orientation)
        end
        if car.red_flag_is_shown or car.location_error_is_detected 
        then emergency_stop else safe_stop end
    ensure
        not car.is_moving 
        car.is_in_normal_mode implies car.race_is_finished
    end
\end{lstlisting}

\vspace{-6pt}
The \e{race_no_obstacles} routine implements the use case by calling the routines \e{update_speed}, \e{safe_stop}, and \e{emergency_stop}, which themselves implement finer-grain use cases. The ``calls'' relation between routines mirrors UML's  \texttt{<<include>>} and \texttt{<<extend>>} relations between use cases. 

The \e{ROBORACE_USE_CASES} class collects such routines reflecting use cases (listing~\ref{roborace_UC}). 

\begin{lstlisting}[caption={Roborace use-case class.}, captionpos=b, label=roborace_UC]
class ROBORACE_USE_CASES feature
    car: RACECAR
    safe_stop 
        require car.is_in_normal_mode
        do car.control_module.safe_stop
        ensure not car_is_moving
        end
    emergency_stop
        require 
            car.red_flag_is_shown or car.location_error_is_detected
        do car.control_module.emergency_stop
        ensure
            not car.is_in_normal_mode
            not car.is_moving
        end
    update_speed
        require car.yellow_flag_is_shown
        do car.update_max_speed (car.yellow_flag_speed)
        ensure car.max_speed = car.yellow_flag_speed
        end			
    race_no_obstacles 
            --implementation is listed above
    -- Other use cases 
end 
\end{lstlisting}

\subsubsection{Relation between use cases and test cases}
\label{use cases to test cases}

\textit{Use case stories} define test cases for use cases \cite{26}. The class
 \e{ROBORACE_USE_CASE_STORIES} inherits from \e{ROBORACE_USE_CASES} class. It includes a collection of routines corresponding to use case stories. 

When a use case takes the form of a routine with contracts, extracting use case stories from such a routine becomes a semi-automated task. 
For example, the \e{emergency_stop} use case accepts two options in its precondition --- (1) when the red flag is shown or (2) when a location error is detected.
These options map to the following use case stories written according to the UOOR approach (listing~\ref{uCstories}):

\begin{lstlisting}[caption={Use case stories extracted from the \textit{``emergency stop''} use case.}, captionpos=b, label=uCstories]
class ROBORACE_USE_CASE_STORIES inherit ROBORACE_USE_CASES feature
    emergency_stop_red_flag_story
        require car.red_flag_is_shown
        do emergency_stop 
        end
    emergency_stop_location_error_story
        require car.location_error_is_detected
        do emergency_stop 
        end
end
\end{lstlisting}

These routines represent the two different paths through the \e{emergency_stop} use case, characterized by their preconditions.
The connection with the parent use case is visible because the stories call the routine encoding the use case.
The two routines must be exercised at least once with test input that meets their preconditions.

A similar analysis makes it possible to extract 5 use case stories from the \textit{``Race without obstacles''} use case:

\begin{itemize*}
    \item 3 for each possible loop exit condition.
    \item 1 corresponding to the true antecedent of the implication in the second postcondition assertion.
    \item 1 corresponding to the true consequent and false antecedent of the said implication. 
\end{itemize*}

The full collection of the extracted use case stories may be found in a publicly available repository \cite{42}.

\subsection{Lessons learned from the case study}
Project stakeholders often fail to state environment-related information explicitly, as they take it for granted. The UOOR methodology can address the issue by serving as a structural basis for elicitation activities.
The approach requires explicitly specifying environmental components and properties such as assumptions, constraints, and invariants.

The \textit{``Race without obstacles''} use case provides a good illustration of the \textbf{difference between contract-based and scenario-based specification}. As a specification, this scenario expresses, among other properties, that the system calculates a local plan and then follows it. 
It states this property in the form of a strict sequence of operations which, however, only covers some of the many possible scenarios.

It does list extensions, but only three of them, and does not reflect the many ways in which they can overlap. For example:
\begin{itemize*}
    \item 

It can happen that the green flag is shown some time after the yellow flag, but the extensions do not even list it.
\item
In the same way, the red flag can be shown after a yellow flag. 

\end{itemize*}

\noindent An attempt to add extensions to cover all possibilities would have no end, as so many events may occur as to create a combinatorial explosion of possible sequencings.

One way out of this dead end would be to use temporal logic \cite{Pnueli}, which provides a finite way to describe a possibly infinite but constrained set of sequences of events or operations. UOOR relies on a different idea: use logical rather than sequential constraints. Sequential constraints become just a special case: we can express that \e{A} must come before \e{B} simply by defining a condition \e{C} as part of both the postcondition of \e{A} and the precondition of \e{B}. But the logic-based specification scheme covers many more possibilities than just this special case. The specification the example just mentioned is presented in listing~\ref{roborace_flags}. \\

The UOOR approach helped to \textbf{structure the entire process of requirements elicitation and analysis}:
\begin{itemize*}
    \item It required identifying and specifying the elements of the environment, separately from elements of the system.
    \item This process revealed the assumptions and constraints which could be overlooked otherwise.
    \item Analysis of use cases revealed more abstract properties, than time-ordering constraints.
    %\item Specifying use case stories with specification drivers defined the test cases for system's behavior.
\end{itemize*}

\subsection{Discussion}
This case study serves as a proof of concept on a significant ongoing project. It is not, however, a systematic empirical validation and as a consequence does not allow drawing firm conclusions such as a guarantee that the UOOR approach will increase productivity or decrease defects. It illustrates instead the observation that object-oriented technology with logical constraints is more general than scenario-based techniques, encompassing them as special cases. 


%\subsection{Conclusion}
%    \item Is UOOR approach expressive enough to formulate requirements for a significant project?
%    \item Does the UOOR methodology facilitate requirements specification?
 
%With respect to the research questions, we can make the following conclusions.

The case study did show that in a specific project the UOOR methodology helped a specific group (the authors) structure the process of requirements elicitation (by helping find the questions to be asked from stakeholders) and analysis (by helping to turn specialized scenarios into more general logical specification).

On the Roborace case study we have demonstrated that with the UOOR approach, one can:
\begin{enumerate}
    \item Express the fundamental abstractions in the form of requirements classes.
    \item Express the fundamental constraints in the form of invariants for these classes.
    \item Express typical usage scenarios with specification drivers. (Unlike the previous two, this task does not make any attempt at exhaustiveness, since examples can only cover a fragment of all possibilities; instead, it concentrates on the scenarios of most interest to stakeholders, and those most likely to cause potential issues or bugs.)
    \item As a consistency check, ascertain that the scenarios (item 3) preserve the invariants (item 2).
\end{enumerate}

More generally, the combination of an object-oriented approach to structure the requirements (1), equipped with invariants (2) as well as other forms of contracts (preconditions, postconditions), with use cases to illustrate the requirements through examples of direct interest to stakeholders (3) and shown to preserve the invariants (4) provides a promising method for obtaining correct and practically useful requirements.

\section{UOOR user study}
\label{Experiment}

A study conducted at the University of Toulouse \cite{experiment_paper} evaluates the perception of the UOOR approach and its potential to be adopted in industry. Since software engineers are already equipped with a set of well-known requirements techniques, their willingness to study and adopt a new approach is based on the following factors:
\begin{itemize*}
    \item Is this approach easy to learn? Will it require much time and efforts to study it?
    \item Does this approach provide an added value? Will it help me to improve requirements specifications?
\end{itemize*}

To address these concerns, the study formulates the following research questions: (i) is limited training sufficient for learning OO contract-based requirements? (ii) does learning contract-based OO requirements techniques help to produce better UML specifications? 

\subsection{Study design}
The study was conducted as a part of the ``OO Analysis and Design'' course at the University of Toulouse. This course, like its many counterparts in other universities \cite{20}, introduces UML as requirements modeling language. 

The study had two parts:
\begin{itemize*}
    \item A controlled experiment \cite{91} based on the ``Instantrame\footnote{Instantrame is a social network on smartphones allowing anyone with an account to share photos}'' case study, where students produced requirements applying two different approaches.
    \item A questionnaire, where students reflected on their experience.
\end{itemize*}

In total, 31 students participated in the study: bachelor's students in their third year and master's students in their first year. Course instructors provided a textual description of a case study to students, which they further used to elicit requirements and produce various requirements artifacts. 

The experiment was split into two parts. In the first part (1.5 hours) theory on UOOR requirements was presented to students. They were already familiar with UML and scenario modeling. Further, they had a task to describe two scenarios for each of the two given use cases according to a provided template. In the second part (4.5 hours), students were randomly split into two groups and had to complete two tasks:
\begin{enumerate}
    \item (2-3 hours) Students of Group 1 specified UOOR requirements for the first use case. Students of Group 2 produced a sequence diagram for the first use case.
    \item (2-3 hours) Students of Group 1 worked on a sequence diagram for the second use case. Students of Group 2 specified UOOR requirements for the second use case.
\end{enumerate}

After submitting the results of their work, students filled in an online questionnaire. The questionnaire included two types of questions. Single-choice questions were formulated as statements that participants had to evaluate based on a Likert scale (`Strongly disagree', `Disagree', `Agree', `Strongly agree', `Neutral' choices). We used those questions to collect quantitative feedback on the use of the UOOR approach. The questionnaire also included open questions to collect additional qualitative feedback, such as the perceived advantages of the approach, the difficulties that participants faced using the approach, what are the potential improvements of the UOOR, and how applying UOOR helps to improve UML-based specifications.

\subsection{Study results}
All of the 31 study participants have participated in a survey, yet one of them answered only part of the questions. Although the population size is not large enough to draw definite conclusions, the study provides a preliminary outlook on the usability of the UOOR approach.

%------------------------------------------------------------------
\begin{figure*}[!bhtp]
\includegraphics[scale=1]{Picture1.jpg}
\caption{Feedback on the use of UOOR.}
\label{UOOR_feedback}
\end{figure*}
%------------------------------------------------------------------


26\% of respondents declared that it was hard to understand and use software contracts for requirements specification, whereas 43\% had a positive experience and 30\% were neutral (Fig.~\ref{UOOR_feedback}). 

%\clearpage

%\newpage



All participants were able to list the advantages of formulating requirements in UOOR. The participants highlighted that the UOOR methodology is easy to understand and follow and that it facilitates producing detailed, specific, readable  requirements which are easy to validate 

\begin{comment}

(Table~\ref{tab:table4}). 

\begin{table}[h!]
    \begin{tabular}{|p{0.95\textwidth}|}
        \hline
        \textbf{What are the reasons to use UOOR?}\\[4pt]
        \hline
        Readability and clarity\\[4pt]
        \hline
        Brings more details at the specification level\\[4pt]
        \hline
        Modularity\\[4pt]
        \hline
        The methodology is easy to understand and follow\\[4pt]
        \hline
        Facilitates identification and analysis of use case scenarios\\[4pt]
        \hline
        Obliges to write preconditions and postconditions\\[4pt]
        \hline
        Well-detailed, very specific\\[4pt]
        \hline
        Easy reuse, which facilitates maintenance\\[4pt]
        \hline
        Better possibility of requirements’ validation\\[4pt]
        \hline
    \end{tabular}
    \caption{Summary of responses to the question “What are the reasons to use UOOR?”}
    \label{tab:table4}
\end{table}

\end{comment}

The particular difficulties stated by the experiment participants, were: not enough familiarity with contracts; not enough examples provided; not enough practice. 

\begin{comment}
    
(Table~\ref{tab:table3}).
 
\begin{table}[h!]
    \begin{tabular}{|p{0.95\textwidth}|}
    \hline
        \textbf{What difficulties have you faced when applying UOOR?}\\[4pt]
        \hline
       Not enough familiarity with Eiffel language and contracts \\[4pt]
        \hline
        It is difficult to formulate pre- and postconditions for some scenarios.\\[4pt]
        \hline
        Need more examples in order to adapt\\[4pt]
        \hline
        Need more practice\\[4pt]
        \hline
    \end{tabular}
    \caption{Summary of responses to the question “What difficulties have you faced when applying UOOR?”}
    \label{tab:table3}
\end{table}

\end{comment}

The questionnaire responses indicate that UOOR helped students to improve their UML specifications in the following ways: “\textit{to think of elements we had not thought of, for example, additional preconditions}”, “\textit{to discover details that need to be added to features}”, “\textit{to define and implement use cases better}”, “\textit{to identify alternative scenarios for the system’s use cases}”, “t\textit{o analyze better the requirements in a global way for a specification}” %(Table~\ref{tab:table5}). 
This is a significant result since people are more likely to adopt an approach which provides immediate benefits, such as improving requirements specifications produced with the currently used approach. 

\begin{comment}
    
\begin{table}[h!]
    \centering
    \begin{tabular}{|p{0.95\textwidth}|}
        \hline
        \textbf{How applying UOOR helped to improve UML specification?}\\[4pt]
        \hline
        Facilitates discovering details that need to be added to features\\[4pt]
        \hline
        Facilitates better analysis of the requirements in a global way\\[4pt]
        \hline
        Understanding of the purpose of the steps to be carried out \\ when writing specifications in UML\\[4pt]
        \hline
        Facilitates identifying alternative scenarios for the system’s use cases\\[4pt]
        \hline
        Facilitates better definition and implementation of use cases\\[4pt]
        \hline
    \end{tabular}
    \caption{Summary of responses to the question “How applying UOOR helped to improve UML specification?”}
    \label{tab:table5}
\end{table}

\end{comment}

The questionnaire also collected feedback from the participants on the usability of the UOOR approach. The responses indicate that a more detailed description of the approach with more examples would improve its usability, as would diagram and tool support. 

\begin{comment}
    
\ref{tab:table6}. 

 %\clearpage

\begin{table}[h!]
    \centering
    \begin{tabular}{|p{0.95\textwidth}|}
        \hline
        \textbf{What could make the UOOR approach more usable?}\\[4pt]
        \hline
        A tool allowing the construction of an architecture according to requirements\\[4pt]
        \hline
        Diagrams to make it more concrete and understandable\\[4pt]
        \hline
        A more detailed description of the approach\\[4pt]
        \hline
        Having more examples\\[4pt]
        \hline
    \end{tabular}
    \caption{Summary of responses to the question ``What could make the UOOR approach more usable?''}
    \label{tab:table6}
\end{table}
\end{comment}

\subsection{Discussion}
Since the experiment was limited to a single course at the University of Toulouse, it is not possible to guarantee that the results transpose to other environment. Another limitation is that the subjects of the experiment were students, rather than software engineering professionals. Note, however, that studies have shown  that there is little difference between the results of software engineering students and professionals for relatively small tasks \cite{host2000using}.

The difficulties reported by the participants, such as not being familiar enough with the Eiffel language and contracts, indicate that the course prerequisites might not have been explicit enough. Since the Eiffel language and Design by Contracts are part of the standard curriculum of the bachelor's software engineering program, we assumed that all study participants are familiar with these topics. Some deviations arise from the presence of guest master students from other programs. 

Several participants claimed that not enough explanations and examples were provided. The present article provides a detailed methodology with two illustrative examples. The approach will also be published in a textbook (``companion book'' \cite{COmpanion}) accompanied by a website and a GitHub repository, which will serve as a forum for discussions of the method. 

%\subsection{Conclusion}
%To be embraced by the industry practitioners, the requirements approach should satisfy two criteria: (i) it should not require substantial training, (ii) it should be perceived as useful for current practices. In the experiment, presented in this section, we evaluated the UOOR approach against these two criteria. With respect to the study participants, both of the criteria were satisfied: (i) most of the participants were able to learn the approach in a limited time, (ii) all participants were able to list the advantages of using the approach;  applying the approach helped to improve UML-based specifications. The results provide preliminary evidence that the UOOR method has a potential to be adopted by requirements engineering practitioners. 

%\section{Conclusion}
%This chapter provided exploratory empirical evaluation of the UOOR approach and assessed its expressiveness and usability. It demonstrated that the approach supports the expression of the fundamental abstractions in the form of requirements classes equipped with contracts, to express usage scenarios with specification drivers and to ascertain that the scenarios satisfy the contracts. It established that the UOOR approach does not require substantial training; applying the UOOR method helps to improve requirements specifications.

%The results provide preliminary evidence that the  UOOR method has a potential to be adopted by requirements engineering practitioners. 

%To draw definitive conclusions, the method should be evaluated in setting of larger scale, such as industrial projects and related user studies, which remains a prominent perspective for future work.

\section{Related work}
\label{approaches}
A previous article \cite{46} discussed the role of use cases
in requirements and contrasted them with object-oriented requirements. The present paper extends that original discussion to a full-fledged requirements engineering method. 

A number of requirements approaches share at least some of the objectives of UOOR. The field is a very broad one, with hundreds of proposals. We identified 15 well-documented methods which lend themselves to a point-by-point comparison based on the criteria discussed in section~\ref{characteristics}. Table~\ref{tab:related_work} is an overview of the results.

%None of the reviewed approaches satisfies all of the criteria identified in section~\ref{characteristics}.


NL-based requirements \cite{wiegers, 17, RELAX} are requirements formulated in the form of unrestricted \acrshort{nl} text, or \acrshort{nl} text, restricted in a certain way. NL-based requirements are easy to learn and are supported by a wide variety of tools and education materials. Scenarios (use cases \cite{43}, user stories \cite{14}, and use cases 2.0 \cite{25, 26}) are a powerful requirements technique. Still, they cannot serve as a requirements methodology. NL-based requirements, including scenarios, are prone to ambiguity, which can be partially eliminated by constraining the natural language. Requirements traceability relies on manually created traceability links. Requirements are reused by copy-pasting.

Use cases are an important modeling tool in UML \cite{55}. UML makes it possible to treat use cases as objects, subject to specialization and decomposition. UML use cases can have pre- and postconditions. It is possible in UML to associate contracts with individual operations through  natural language or the OCL (Object Constraint Language) notation. SysML \cite{47}, an extended profile of UML, treats requirements as first-class entities, establishing direct links between requirements and other software artifacts (such as tests). \cite{58} illustrates the requirements specification process with SysML, and \cite{5, 52, 57} provide applications of SysML to all phases of software development. SysML does not provide semantics for requirements, although it is possible to associate contracts with individual operations through natural language or the OCL notation. SysML and UML are standardized notations and not methodologies. 

The Restricted Use Case Modeling approach \cite{62} relies on a use case template and a set of restriction rules to reduce the ambiguity of use case specification and facilitate the transition to analysis models, such as UML class diagram and sequence diagram. The aToucan tool automates the generation of UML class, sequence, and activity diagrams \cite{61}. The tool can generate traceability links from the textual use cases to the generated class diagram, but not to the source code. The approach does not advocate extracting abstract properties from use cases and domain knowledge, such as time-ordering and environmental constraints. 

A Use Case Map (UCM) \cite{3,12} depicts several scenarios simultaneously.
UCMs represent use cases as causal sequences of responsibilities, possibly over a set of abstract components. 
In UCMs, pre- and postconditions of use cases, as well as conditions at selection points, can be modeled with formal specification techniques such as ASM or LOTOS.  
UCMs specify properties of operations in relation to scenario sequences, rather than abstract properties of objects and operations. Use case maps do not provide a framework for requirements traceability and reuse.

Object-Oriented Analysis and Design (OOAD) \cite{8} is a unified methodology for use-case-driven analysis and design, supported by UML \cite{55} as a unified notation. OOAD applies OO techniques (class-based decomposition, OO modeling) to the initial requirements, produced at the earlier stages of the development process. OOAD does not provide a framework for requirements traceability and reuse.

The OO-Method \cite{80} combines conventional OO specification techniques \cite{8} with formal specification, relying on the OASIS object-oriented specification language \cite{lopez1995oasis}. 
The Integranova tool supports the specification process with an interactive interface and automatically generates the implementation code. However, this approach does not provide a framework for requirements traceability and reuse.

In goal-oriented requirements engineering  \cite{78}, \cite{60} requirements are obtained through a series of refinements of high-level goals. With the help of the Objectiver tool \cite{44},
requirements can be linked to other artifacts, such as goals, environment agents, or operations. However, traceability links to natural language requirements documents or implementation artifacts are out of the scope of the approach.

In test-driven development (TDD) \cite{66}, a software engineer writes unit tests before implementing the system’s functionality in small iterations. Unit tests can be viewed as the means of capturing requirements: tests serve as a guide to code writing. In behavior-driven development (BDD) \cite{75, 86}, requirements are formulated as user stories, following a specific template. Dedicated tools transform user stories into parameterized unit tests. TDD and BDD rely on scenarios, which are not abstract enough to be \textit{requirements}: if scenarios attempt to cover \textit{all} possible situations, their number explodes, which impedes requirements traceability. BDD and TDD do not provide mechanisms for requirements reusability and static verification.
   
In the ACL/VF framework \cite{6, 64}, use cases capture requirements, which are further formalized as grammars of responsibilities. Another Contract Language (ACL) contracts (pre- and postconditions and invariants) specify constraints, which scenarios' or responsibilities' execution poses on the system's state. In this approach, the requirements model is decoupled from the candidate implementation: a dedicated binding tool maps elements of the requirements model to the elements of candidate implementation. The approach requires a substantial background: familiarity with design by contract, ACL, and formal grammars. The approach is not seamless and does not provide a framework for requirements reuse.

The Multirequirements approach \cite{32} suggests using a single notation (Eiffel programming language) for requirements, design, and implementation. Requirements are formulated in 3 interconnected layers: natural language, software contracts in programming language, and diagrams. The approach does not provide a methodology and a framework for requirements traceability and reuse.

The PEGS approach attempts to provide a definition and taxonomy of requirements. According to this approach, requirements pertain to a Project intended, in a certain Environment, to achieve some Goals by building a System. Thus, requirements specification consists of four books: Project, Environment, Goals, and Systems, which correspond to each of these components \cite{Handbook}. The approach provides principles and techniques of requirements specification, including seamless OO specification, yet does not provide an explicit methodology. 

The SIRCOD approach \cite{galinier2021seamless} provides a pipeline for converting natural language requirements to programming language contracts. The approach relies on the domain specific language, RSML, for automating conversion from natural language to programming language. In the Seamless Object-Oriented Requirements approach (SOOR), requirements are documented as software classes, which makes them verifiable and reusable \cite{39}. Routines of those classes, called specification drivers, take objects to be specified as arguments and express the effect of operations on those objects with pre- and postconditions. The SIRCOD and SOOR approaches focus on translating existing requirements specifications to contracts expressed in a programming language, rather than extracting abstract requirements from scenarios.

The UOOR method relies on the advancements of the SIRCOD and SOOR approaches but focuses more on the approach's usability and requirements traceability management. 

%\begin{landscape}
\begingroup
%\renewcommand{\arraystretch}{1.3} % Default value: 1
\setlength\tabcolsep{0.08cm}.

\begin{table*}
\small 
%\resizebox{\textwidth}{!}
\centering
    \begin{tabular}[M]{|p{2.5cm} | p{1.7cm}| p{1.5cm}| p{1.0cm}| p{1.7cm}| p{1.7cm} |p{1.7cm}| p{1.5cm}| p{1.7cm}|}
    \hline
& Methodology & Required \newline background & Tool \newline support & Requirements reusability & Requirements  \newline verifiability & Requirements \newline unambiguity & Traceability \newline support & Seamlessness \\
\hline
NL-based & Yes & Some & Yes & No & No & Partial&No&No \\
UML and SysML & No & Substantial & Yes & No & Partial & Partial & Partial & No \\
Scenarios & No & Some & Yes & No & No & No & No & No \\
RUCM & Yes & Some & Yes & No & No & Partial & Partial & No \\
Use Case Maps & Yes & Substantial & Yes & No & Yes & Yes & No & No \\
OOAD & Yes & Substantial & Yes & No & Partial & Partial & No & No \\
OO-Method & Yes & Some & Yes & No & Yes & Partial & No & No \\
GORE & Yes & Some & Yes & No & No & Yes & Partial & No \\
TDD & Yes & Some & Yes & No & Yes & Yes & Yes & Yes \\ 
BDD & Yes & Some & Yes & No & Yes & Yes & Yes & Yes \\
ACL/VF & Yes & Substantial & No & No & Yes & Yes & Partial & No \\
Multirequirements & No & Some & Partial & No & Yes & Yes & No & Yes \\
SIRCOD & Partial & Some & Yes & No & Yes & Yes & Yes & Yes \\
SOOR & No & Some & Yes & Yes & Yes & Yes & No & Yes \\
PEGS & No & Some & Partial & No & Yes & Yes & No & Yes \\
\hline
    \end{tabular}
    \caption{Summary of related work.}
    \label{tab:related_work}
\end{table*}
\endgroup
%\end{landscape}



\section{Conclusion and perspectives}
\label{contributions}

In this article, we explored the practical application of seamless requirements by addressing the following research questions:
\begin{itemize*}
    \item What should be the process of seamless software development from requirements to code?
    \item What tool support is required to ensure traceability between requirements and other project artifacts?
\end{itemize*}

In answer to these questions, we presented the Unified Object-Oriented approach to Requirements (UOOR) and the supporting Traceability tool. 
%It demonstrated that the approach is affordable and practically usable and that it facilitates producing unambiguous, verifiable, reusable and traceable requirements. 

The UOOR approach is: 
\begin{itemize*}
    \item \textbf{Explicit: }The approach provides a methodology and illustrative examples. The article serves as a guide for the process of producing requirements. The ongoing work on the book illustrating the approach \cite{COmpanion} and online portal with the book's supporting materials may serve as a discussion forum for the adopters of the approach.
    \item \textbf{Lightweight:} The approach does not require the knowledge of formal models or mathematical notations. It does require familiarity with a programming language and Design by Contract. The clarity of the Eiffel language, chosen as the requirements notation, significantly lowers this barrier: in order to be able to formulate contracts in Eiffel, a requirements engineer must learn only a few language constructs. The ability to rely on the mature IDE, EiffelStudio, makes the task even easier. Since OO requirements are compilable code elements, the IDE provides error codes if OO requirements are formulated with errors.      
    \item \textbf{Tool-supported:} The approach relies on the general-purpose IDE (EiffelStudio for the Eiffel language) and integrated Traceability tool. EiffelStudio provides facilities for static and dynamic verification of the developed system against the requirements, enables traceability links creation and management, and provides syntactic checks for OO requirements. The Traceability tool provides more advanced functionality for traceability links management: it allows assigning types to project elements, creating typed traceability links, and tracking changes from requirements to the related project elements. 
    \item \textbf{Seamless:} The UOOR approach provides a methodology for producing UOOR requirements rather than translating requirements documents to a programming language. Thus, UOOR relies on a uniform process (based on refinement) and a uniform notation (Eiffel language). 
\end{itemize*}

Requirements formulated with the UOOR approach are:
\begin{itemize*}
    \item \textbf{Reusable.} Based on the capabilities provided by the object-oriented technology, requirements in UOOR can be reused not by copy-pasting natural language texts but as libraries of domain-specific component specifications in the form of contracted deferred classes. Being implementation-independent, such specifications allow for different implementations.
    \item \textbf{Verifiable.} Since OO requirements are code elements, an IDE provides basic consistency checks at compile time. When contract checking is enabled at runtime, the IDE monitors contracts violations. Since every contract can have a unique tag, a developer can trace an exception to the violated contract. 
Contract specifications serve as oracles for dynamic testing. Moreover, scenarios implemented as specification drivers serve as tests when actual arguments are passed. 
A static verifier, such as Autoproof \cite{7}, can be used to ensure static verification of the system's functional correctness.
    \item \textbf{Unambiguous.} Ambiguity, innate to natural language texts, can be eliminated by introducing formal notation. In UOOR, contracts serve as the notation for requirements, which are understandable due to the readability and clarity of the Eiffel language; simultaneously, they remove the ambiguity inherent to natural language texts. 
    \item \textbf{Traceable.} The approach introduces the notion of seamless requirements traceability, which enables propagating traceability links based on formal properties of relations between project elements. A dedicated Traceability tool provides facilities for creating and managing traceability links and tracking changes from requirements to related code elements.
\end{itemize*}

\subsection*{Limitations}\label{limitations}
The approach has been so far used in connection with Eiffel, relying on the language's expressiveness, its native support of contracts, and the rich functionality of the supporting tool machinery: EiffelStudio, Autoproof, AutoTest. Applying UOOR in a different OO environment is possible but will lose some of the benefits of an integrated approach. The issue, however, is not all-or-nothing, since Eiffel and the tools have open architectures supporting the inclusion of various technology elements. An intermediate approach is possible in which a core system is built using the Eiffel-integrated approach, which can be integrated with numerous elements in other languages and used in conjunction with other tools. It is also possible to use Eiffel as an intermediate requirements, design and prototyping language and then move to another implementation language, although this approach obviously damages seamlessness. 

%The article explored the application of the approach to the greenfield (i.e., developed from scratch) projects. Its application to the brownfield projects may be tedious in case of poor documentation practices. 

The UOOR approach currently focuses on functional requirements. Its application to the Roborace case study shows that it covers a wide spectrum of requirements; including a systematic approach to non-functional requirements is one of the future work directions discussed next.

\subsection*{Perspectives}\label{perspectives}

Forthcoming steps in the development of the UOOR approach include applying and evaluating the approach on a larger scale, such as industrial projects and related user studies. The ongoing work on the ``Companion'' book illustrating the approach \cite{COmpanion} and a supporting website contributes to creating a community of the users and recruiting projects for  evaluation.  

Another direction of future work is the coverage of non-functional requirements. How can non-functional requirements be treated using the UOOR approach? Recent work on use-case modeling of crosscutting concerns \cite{yue2016practical} can be a starting point in such research. 

The work on seamless requirements traceability can be continued in the following directions:
\begin{itemize*}
    \item Further exploration of relation propagation rules.
    \item Extension of this modeling work to a full theory of relations in requirements engineering, with a set of axioms and a number of resulting theorems.
    \item Practical evaluation of seamless traceability on a significant project.
    \item Tool support.
\end{itemize*}

A recent development is  the release of a Traceability tool \cite{zakaria}, which enables creating typed links between project elements and tracking changes from requirements to the related project artifacts. At the time of writing the Traceability tool is in the process of being integrated with the next (mid-2025) version of EiffelStudio. It is itself subject to a number of planned enhancements, in particular a full implementation of link propagation, which will enable seamless traceability between software project artifacts.

Equipped with a traceability management and link propagation tool, UOOR is an effective and useful requirements analysis method for projects which, while not necessarily ready to adopt a fully formal requirements approach, need a rigorous, methodologically sound approach answering the major issues of requirements engineering and focused on producing requirements that are best poised to play their part in the search for very-high-quality software systems. 
%\mb{add closing sentence here}
%\se{TODO}

%%%% Bibliography before appendices
%%%% After the initial round, you need to fix it by hand depending on where you generate your .bib file from
%%%% Mendeley often messes with italics and accents for instance

%\bibliographystyle{abbrv}
%\printbibliography


\bibliography{all}
% Appendices
%\subsection{Lloyd-Max Algorithm}
\label{subsec:Lloyd-Max}
For a given quantization bitwidth $B$ and an operand $\bm{X}$, the Lloyd-Max algorithm finds $2^B$ quantization levels $\{\hat{x}_i\}_{i=1}^{2^B}$ such that quantizing $\bm{X}$ by rounding each scalar in $\bm{X}$ to the nearest quantization level minimizes the quantization MSE. 

The algorithm starts with an initial guess of quantization levels and then iteratively computes quantization thresholds $\{\tau_i\}_{i=1}^{2^B-1}$ and updates quantization levels $\{\hat{x}_i\}_{i=1}^{2^B}$. Specifically, at iteration $n$, thresholds are set to the midpoints of the previous iteration's levels:
\begin{align*}
    \tau_i^{(n)}=\frac{\hat{x}_i^{(n-1)}+\hat{x}_{i+1}^{(n-1)}}2 \text{ for } i=1\ldots 2^B-1
\end{align*}
Subsequently, the quantization levels are re-computed as conditional means of the data regions defined by the new thresholds:
\begin{align*}
    \hat{x}_i^{(n)}=\mathbb{E}\left[ \bm{X} \big| \bm{X}\in [\tau_{i-1}^{(n)},\tau_i^{(n)}] \right] \text{ for } i=1\ldots 2^B
\end{align*}
where to satisfy boundary conditions we have $\tau_0=-\infty$ and $\tau_{2^B}=\infty$. The algorithm iterates the above steps until convergence.

Figure \ref{fig:lm_quant} compares the quantization levels of a $7$-bit floating point (E3M3) quantizer (left) to a $7$-bit Lloyd-Max quantizer (right) when quantizing a layer of weights from the GPT3-126M model at a per-tensor granularity. As shown, the Lloyd-Max quantizer achieves substantially lower quantization MSE. Further, Table \ref{tab:FP7_vs_LM7} shows the superior perplexity achieved by Lloyd-Max quantizers for bitwidths of $7$, $6$ and $5$. The difference between the quantizers is clear at 5 bits, where per-tensor FP quantization incurs a drastic and unacceptable increase in perplexity, while Lloyd-Max quantization incurs a much smaller increase. Nevertheless, we note that even the optimal Lloyd-Max quantizer incurs a notable ($\sim 1.5$) increase in perplexity due to the coarse granularity of quantization. 

\begin{figure}[h]
  \centering
  \includegraphics[width=0.7\linewidth]{sections/figures/LM7_FP7.pdf}
  \caption{\small Quantization levels and the corresponding quantization MSE of Floating Point (left) vs Lloyd-Max (right) Quantizers for a layer of weights in the GPT3-126M model.}
  \label{fig:lm_quant}
\end{figure}

\begin{table}[h]\scriptsize
\begin{center}
\caption{\label{tab:FP7_vs_LM7} \small Comparing perplexity (lower is better) achieved by floating point quantizers and Lloyd-Max quantizers on a GPT3-126M model for the Wikitext-103 dataset.}
\begin{tabular}{c|cc|c}
\hline
 \multirow{2}{*}{\textbf{Bitwidth}} & \multicolumn{2}{|c|}{\textbf{Floating-Point Quantizer}} & \textbf{Lloyd-Max Quantizer} \\
 & Best Format & Wikitext-103 Perplexity & Wikitext-103 Perplexity \\
\hline
7 & E3M3 & 18.32 & 18.27 \\
6 & E3M2 & 19.07 & 18.51 \\
5 & E4M0 & 43.89 & 19.71 \\
\hline
\end{tabular}
\end{center}
\end{table}

\subsection{Proof of Local Optimality of LO-BCQ}
\label{subsec:lobcq_opt_proof}
For a given block $\bm{b}_j$, the quantization MSE during LO-BCQ can be empirically evaluated as $\frac{1}{L_b}\lVert \bm{b}_j- \bm{\hat{b}}_j\rVert^2_2$ where $\bm{\hat{b}}_j$ is computed from equation (\ref{eq:clustered_quantization_definition}) as $C_{f(\bm{b}_j)}(\bm{b}_j)$. Further, for a given block cluster $\mathcal{B}_i$, we compute the quantization MSE as $\frac{1}{|\mathcal{B}_{i}|}\sum_{\bm{b} \in \mathcal{B}_{i}} \frac{1}{L_b}\lVert \bm{b}- C_i^{(n)}(\bm{b})\rVert^2_2$. Therefore, at the end of iteration $n$, we evaluate the overall quantization MSE $J^{(n)}$ for a given operand $\bm{X}$ composed of $N_c$ block clusters as:
\begin{align*}
    \label{eq:mse_iter_n}
    J^{(n)} = \frac{1}{N_c} \sum_{i=1}^{N_c} \frac{1}{|\mathcal{B}_{i}^{(n)}|}\sum_{\bm{v} \in \mathcal{B}_{i}^{(n)}} \frac{1}{L_b}\lVert \bm{b}- B_i^{(n)}(\bm{b})\rVert^2_2
\end{align*}

At the end of iteration $n$, the codebooks are updated from $\mathcal{C}^{(n-1)}$ to $\mathcal{C}^{(n)}$. However, the mapping of a given vector $\bm{b}_j$ to quantizers $\mathcal{C}^{(n)}$ remains as  $f^{(n)}(\bm{b}_j)$. At the next iteration, during the vector clustering step, $f^{(n+1)}(\bm{b}_j)$ finds new mapping of $\bm{b}_j$ to updated codebooks $\mathcal{C}^{(n)}$ such that the quantization MSE over the candidate codebooks is minimized. Therefore, we obtain the following result for $\bm{b}_j$:
\begin{align*}
\frac{1}{L_b}\lVert \bm{b}_j - C_{f^{(n+1)}(\bm{b}_j)}^{(n)}(\bm{b}_j)\rVert^2_2 \le \frac{1}{L_b}\lVert \bm{b}_j - C_{f^{(n)}(\bm{b}_j)}^{(n)}(\bm{b}_j)\rVert^2_2
\end{align*}

That is, quantizing $\bm{b}_j$ at the end of the block clustering step of iteration $n+1$ results in lower quantization MSE compared to quantizing at the end of iteration $n$. Since this is true for all $\bm{b} \in \bm{X}$, we assert the following:
\begin{equation}
\begin{split}
\label{eq:mse_ineq_1}
    \tilde{J}^{(n+1)} &= \frac{1}{N_c} \sum_{i=1}^{N_c} \frac{1}{|\mathcal{B}_{i}^{(n+1)}|}\sum_{\bm{b} \in \mathcal{B}_{i}^{(n+1)}} \frac{1}{L_b}\lVert \bm{b} - C_i^{(n)}(b)\rVert^2_2 \le J^{(n)}
\end{split}
\end{equation}
where $\tilde{J}^{(n+1)}$ is the the quantization MSE after the vector clustering step at iteration $n+1$.

Next, during the codebook update step (\ref{eq:quantizers_update}) at iteration $n+1$, the per-cluster codebooks $\mathcal{C}^{(n)}$ are updated to $\mathcal{C}^{(n+1)}$ by invoking the Lloyd-Max algorithm \citep{Lloyd}. We know that for any given value distribution, the Lloyd-Max algorithm minimizes the quantization MSE. Therefore, for a given vector cluster $\mathcal{B}_i$ we obtain the following result:

\begin{equation}
    \frac{1}{|\mathcal{B}_{i}^{(n+1)}|}\sum_{\bm{b} \in \mathcal{B}_{i}^{(n+1)}} \frac{1}{L_b}\lVert \bm{b}- C_i^{(n+1)}(\bm{b})\rVert^2_2 \le \frac{1}{|\mathcal{B}_{i}^{(n+1)}|}\sum_{\bm{b} \in \mathcal{B}_{i}^{(n+1)}} \frac{1}{L_b}\lVert \bm{b}- C_i^{(n)}(\bm{b})\rVert^2_2
\end{equation}

The above equation states that quantizing the given block cluster $\mathcal{B}_i$ after updating the associated codebook from $C_i^{(n)}$ to $C_i^{(n+1)}$ results in lower quantization MSE. Since this is true for all the block clusters, we derive the following result: 
\begin{equation}
\begin{split}
\label{eq:mse_ineq_2}
     J^{(n+1)} &= \frac{1}{N_c} \sum_{i=1}^{N_c} \frac{1}{|\mathcal{B}_{i}^{(n+1)}|}\sum_{\bm{b} \in \mathcal{B}_{i}^{(n+1)}} \frac{1}{L_b}\lVert \bm{b}- C_i^{(n+1)}(\bm{b})\rVert^2_2  \le \tilde{J}^{(n+1)}   
\end{split}
\end{equation}

Following (\ref{eq:mse_ineq_1}) and (\ref{eq:mse_ineq_2}), we find that the quantization MSE is non-increasing for each iteration, that is, $J^{(1)} \ge J^{(2)} \ge J^{(3)} \ge \ldots \ge J^{(M)}$ where $M$ is the maximum number of iterations. 
%Therefore, we can say that if the algorithm converges, then it must be that it has converged to a local minimum. 
\hfill $\blacksquare$


\begin{figure}
    \begin{center}
    \includegraphics[width=0.5\textwidth]{sections//figures/mse_vs_iter.pdf}
    \end{center}
    \caption{\small NMSE vs iterations during LO-BCQ compared to other block quantization proposals}
    \label{fig:nmse_vs_iter}
\end{figure}

Figure \ref{fig:nmse_vs_iter} shows the empirical convergence of LO-BCQ across several block lengths and number of codebooks. Also, the MSE achieved by LO-BCQ is compared to baselines such as MXFP and VSQ. As shown, LO-BCQ converges to a lower MSE than the baselines. Further, we achieve better convergence for larger number of codebooks ($N_c$) and for a smaller block length ($L_b$), both of which increase the bitwidth of BCQ (see Eq \ref{eq:bitwidth_bcq}).


\subsection{Additional Accuracy Results}
%Table \ref{tab:lobcq_config} lists the various LOBCQ configurations and their corresponding bitwidths.
\begin{table}
\setlength{\tabcolsep}{4.75pt}
\begin{center}
\caption{\label{tab:lobcq_config} Various LO-BCQ configurations and their bitwidths.}
\begin{tabular}{|c||c|c|c|c||c|c||c|} 
\hline
 & \multicolumn{4}{|c||}{$L_b=8$} & \multicolumn{2}{|c||}{$L_b=4$} & $L_b=2$ \\
 \hline
 \backslashbox{$L_A$\kern-1em}{\kern-1em$N_c$} & 2 & 4 & 8 & 16 & 2 & 4 & 2 \\
 \hline
 64 & 4.25 & 4.375 & 4.5 & 4.625 & 4.375 & 4.625 & 4.625\\
 \hline
 32 & 4.375 & 4.5 & 4.625& 4.75 & 4.5 & 4.75 & 4.75 \\
 \hline
 16 & 4.625 & 4.75& 4.875 & 5 & 4.75 & 5 & 5 \\
 \hline
\end{tabular}
\end{center}
\end{table}

%\subsection{Perplexity achieved by various LO-BCQ configurations on Wikitext-103 dataset}

\begin{table} \centering
\begin{tabular}{|c||c|c|c|c||c|c||c|} 
\hline
 $L_b \rightarrow$& \multicolumn{4}{c||}{8} & \multicolumn{2}{c||}{4} & 2\\
 \hline
 \backslashbox{$L_A$\kern-1em}{\kern-1em$N_c$} & 2 & 4 & 8 & 16 & 2 & 4 & 2  \\
 %$N_c \rightarrow$ & 2 & 4 & 8 & 16 & 2 & 4 & 2 \\
 \hline
 \hline
 \multicolumn{8}{c}{GPT3-1.3B (FP32 PPL = 9.98)} \\ 
 \hline
 \hline
 64 & 10.40 & 10.23 & 10.17 & 10.15 &  10.28 & 10.18 & 10.19 \\
 \hline
 32 & 10.25 & 10.20 & 10.15 & 10.12 &  10.23 & 10.17 & 10.17 \\
 \hline
 16 & 10.22 & 10.16 & 10.10 & 10.09 &  10.21 & 10.14 & 10.16 \\
 \hline
  \hline
 \multicolumn{8}{c}{GPT3-8B (FP32 PPL = 7.38)} \\ 
 \hline
 \hline
 64 & 7.61 & 7.52 & 7.48 &  7.47 &  7.55 &  7.49 & 7.50 \\
 \hline
 32 & 7.52 & 7.50 & 7.46 &  7.45 &  7.52 &  7.48 & 7.48  \\
 \hline
 16 & 7.51 & 7.48 & 7.44 &  7.44 &  7.51 &  7.49 & 7.47  \\
 \hline
\end{tabular}
\caption{\label{tab:ppl_gpt3_abalation} Wikitext-103 perplexity across GPT3-1.3B and 8B models.}
\end{table}

\begin{table} \centering
\begin{tabular}{|c||c|c|c|c||} 
\hline
 $L_b \rightarrow$& \multicolumn{4}{c||}{8}\\
 \hline
 \backslashbox{$L_A$\kern-1em}{\kern-1em$N_c$} & 2 & 4 & 8 & 16 \\
 %$N_c \rightarrow$ & 2 & 4 & 8 & 16 & 2 & 4 & 2 \\
 \hline
 \hline
 \multicolumn{5}{|c|}{Llama2-7B (FP32 PPL = 5.06)} \\ 
 \hline
 \hline
 64 & 5.31 & 5.26 & 5.19 & 5.18  \\
 \hline
 32 & 5.23 & 5.25 & 5.18 & 5.15  \\
 \hline
 16 & 5.23 & 5.19 & 5.16 & 5.14  \\
 \hline
 \multicolumn{5}{|c|}{Nemotron4-15B (FP32 PPL = 5.87)} \\ 
 \hline
 \hline
 64  & 6.3 & 6.20 & 6.13 & 6.08  \\
 \hline
 32  & 6.24 & 6.12 & 6.07 & 6.03  \\
 \hline
 16  & 6.12 & 6.14 & 6.04 & 6.02  \\
 \hline
 \multicolumn{5}{|c|}{Nemotron4-340B (FP32 PPL = 3.48)} \\ 
 \hline
 \hline
 64 & 3.67 & 3.62 & 3.60 & 3.59 \\
 \hline
 32 & 3.63 & 3.61 & 3.59 & 3.56 \\
 \hline
 16 & 3.61 & 3.58 & 3.57 & 3.55 \\
 \hline
\end{tabular}
\caption{\label{tab:ppl_llama7B_nemo15B} Wikitext-103 perplexity compared to FP32 baseline in Llama2-7B and Nemotron4-15B, 340B models}
\end{table}

%\subsection{Perplexity achieved by various LO-BCQ configurations on MMLU dataset}


\begin{table} \centering
\begin{tabular}{|c||c|c|c|c||c|c|c|c|} 
\hline
 $L_b \rightarrow$& \multicolumn{4}{c||}{8} & \multicolumn{4}{c||}{8}\\
 \hline
 \backslashbox{$L_A$\kern-1em}{\kern-1em$N_c$} & 2 & 4 & 8 & 16 & 2 & 4 & 8 & 16  \\
 %$N_c \rightarrow$ & 2 & 4 & 8 & 16 & 2 & 4 & 2 \\
 \hline
 \hline
 \multicolumn{5}{|c|}{Llama2-7B (FP32 Accuracy = 45.8\%)} & \multicolumn{4}{|c|}{Llama2-70B (FP32 Accuracy = 69.12\%)} \\ 
 \hline
 \hline
 64 & 43.9 & 43.4 & 43.9 & 44.9 & 68.07 & 68.27 & 68.17 & 68.75 \\
 \hline
 32 & 44.5 & 43.8 & 44.9 & 44.5 & 68.37 & 68.51 & 68.35 & 68.27  \\
 \hline
 16 & 43.9 & 42.7 & 44.9 & 45 & 68.12 & 68.77 & 68.31 & 68.59  \\
 \hline
 \hline
 \multicolumn{5}{|c|}{GPT3-22B (FP32 Accuracy = 38.75\%)} & \multicolumn{4}{|c|}{Nemotron4-15B (FP32 Accuracy = 64.3\%)} \\ 
 \hline
 \hline
 64 & 36.71 & 38.85 & 38.13 & 38.92 & 63.17 & 62.36 & 63.72 & 64.09 \\
 \hline
 32 & 37.95 & 38.69 & 39.45 & 38.34 & 64.05 & 62.30 & 63.8 & 64.33  \\
 \hline
 16 & 38.88 & 38.80 & 38.31 & 38.92 & 63.22 & 63.51 & 63.93 & 64.43  \\
 \hline
\end{tabular}
\caption{\label{tab:mmlu_abalation} Accuracy on MMLU dataset across GPT3-22B, Llama2-7B, 70B and Nemotron4-15B models.}
\end{table}


%\subsection{Perplexity achieved by various LO-BCQ configurations on LM evaluation harness}

\begin{table} \centering
\begin{tabular}{|c||c|c|c|c||c|c|c|c|} 
\hline
 $L_b \rightarrow$& \multicolumn{4}{c||}{8} & \multicolumn{4}{c||}{8}\\
 \hline
 \backslashbox{$L_A$\kern-1em}{\kern-1em$N_c$} & 2 & 4 & 8 & 16 & 2 & 4 & 8 & 16  \\
 %$N_c \rightarrow$ & 2 & 4 & 8 & 16 & 2 & 4 & 2 \\
 \hline
 \hline
 \multicolumn{5}{|c|}{Race (FP32 Accuracy = 37.51\%)} & \multicolumn{4}{|c|}{Boolq (FP32 Accuracy = 64.62\%)} \\ 
 \hline
 \hline
 64 & 36.94 & 37.13 & 36.27 & 37.13 & 63.73 & 62.26 & 63.49 & 63.36 \\
 \hline
 32 & 37.03 & 36.36 & 36.08 & 37.03 & 62.54 & 63.51 & 63.49 & 63.55  \\
 \hline
 16 & 37.03 & 37.03 & 36.46 & 37.03 & 61.1 & 63.79 & 63.58 & 63.33  \\
 \hline
 \hline
 \multicolumn{5}{|c|}{Winogrande (FP32 Accuracy = 58.01\%)} & \multicolumn{4}{|c|}{Piqa (FP32 Accuracy = 74.21\%)} \\ 
 \hline
 \hline
 64 & 58.17 & 57.22 & 57.85 & 58.33 & 73.01 & 73.07 & 73.07 & 72.80 \\
 \hline
 32 & 59.12 & 58.09 & 57.85 & 58.41 & 73.01 & 73.94 & 72.74 & 73.18  \\
 \hline
 16 & 57.93 & 58.88 & 57.93 & 58.56 & 73.94 & 72.80 & 73.01 & 73.94  \\
 \hline
\end{tabular}
\caption{\label{tab:mmlu_abalation} Accuracy on LM evaluation harness tasks on GPT3-1.3B model.}
\end{table}

\begin{table} \centering
\begin{tabular}{|c||c|c|c|c||c|c|c|c|} 
\hline
 $L_b \rightarrow$& \multicolumn{4}{c||}{8} & \multicolumn{4}{c||}{8}\\
 \hline
 \backslashbox{$L_A$\kern-1em}{\kern-1em$N_c$} & 2 & 4 & 8 & 16 & 2 & 4 & 8 & 16  \\
 %$N_c \rightarrow$ & 2 & 4 & 8 & 16 & 2 & 4 & 2 \\
 \hline
 \hline
 \multicolumn{5}{|c|}{Race (FP32 Accuracy = 41.34\%)} & \multicolumn{4}{|c|}{Boolq (FP32 Accuracy = 68.32\%)} \\ 
 \hline
 \hline
 64 & 40.48 & 40.10 & 39.43 & 39.90 & 69.20 & 68.41 & 69.45 & 68.56 \\
 \hline
 32 & 39.52 & 39.52 & 40.77 & 39.62 & 68.32 & 67.43 & 68.17 & 69.30  \\
 \hline
 16 & 39.81 & 39.71 & 39.90 & 40.38 & 68.10 & 66.33 & 69.51 & 69.42  \\
 \hline
 \hline
 \multicolumn{5}{|c|}{Winogrande (FP32 Accuracy = 67.88\%)} & \multicolumn{4}{|c|}{Piqa (FP32 Accuracy = 78.78\%)} \\ 
 \hline
 \hline
 64 & 66.85 & 66.61 & 67.72 & 67.88 & 77.31 & 77.42 & 77.75 & 77.64 \\
 \hline
 32 & 67.25 & 67.72 & 67.72 & 67.00 & 77.31 & 77.04 & 77.80 & 77.37  \\
 \hline
 16 & 68.11 & 68.90 & 67.88 & 67.48 & 77.37 & 78.13 & 78.13 & 77.69  \\
 \hline
\end{tabular}
\caption{\label{tab:mmlu_abalation} Accuracy on LM evaluation harness tasks on GPT3-8B model.}
\end{table}

\begin{table} \centering
\begin{tabular}{|c||c|c|c|c||c|c|c|c|} 
\hline
 $L_b \rightarrow$& \multicolumn{4}{c||}{8} & \multicolumn{4}{c||}{8}\\
 \hline
 \backslashbox{$L_A$\kern-1em}{\kern-1em$N_c$} & 2 & 4 & 8 & 16 & 2 & 4 & 8 & 16  \\
 %$N_c \rightarrow$ & 2 & 4 & 8 & 16 & 2 & 4 & 2 \\
 \hline
 \hline
 \multicolumn{5}{|c|}{Race (FP32 Accuracy = 40.67\%)} & \multicolumn{4}{|c|}{Boolq (FP32 Accuracy = 76.54\%)} \\ 
 \hline
 \hline
 64 & 40.48 & 40.10 & 39.43 & 39.90 & 75.41 & 75.11 & 77.09 & 75.66 \\
 \hline
 32 & 39.52 & 39.52 & 40.77 & 39.62 & 76.02 & 76.02 & 75.96 & 75.35  \\
 \hline
 16 & 39.81 & 39.71 & 39.90 & 40.38 & 75.05 & 73.82 & 75.72 & 76.09  \\
 \hline
 \hline
 \multicolumn{5}{|c|}{Winogrande (FP32 Accuracy = 70.64\%)} & \multicolumn{4}{|c|}{Piqa (FP32 Accuracy = 79.16\%)} \\ 
 \hline
 \hline
 64 & 69.14 & 70.17 & 70.17 & 70.56 & 78.24 & 79.00 & 78.62 & 78.73 \\
 \hline
 32 & 70.96 & 69.69 & 71.27 & 69.30 & 78.56 & 79.49 & 79.16 & 78.89  \\
 \hline
 16 & 71.03 & 69.53 & 69.69 & 70.40 & 78.13 & 79.16 & 79.00 & 79.00  \\
 \hline
\end{tabular}
\caption{\label{tab:mmlu_abalation} Accuracy on LM evaluation harness tasks on GPT3-22B model.}
\end{table}

\begin{table} \centering
\begin{tabular}{|c||c|c|c|c||c|c|c|c|} 
\hline
 $L_b \rightarrow$& \multicolumn{4}{c||}{8} & \multicolumn{4}{c||}{8}\\
 \hline
 \backslashbox{$L_A$\kern-1em}{\kern-1em$N_c$} & 2 & 4 & 8 & 16 & 2 & 4 & 8 & 16  \\
 %$N_c \rightarrow$ & 2 & 4 & 8 & 16 & 2 & 4 & 2 \\
 \hline
 \hline
 \multicolumn{5}{|c|}{Race (FP32 Accuracy = 44.4\%)} & \multicolumn{4}{|c|}{Boolq (FP32 Accuracy = 79.29\%)} \\ 
 \hline
 \hline
 64 & 42.49 & 42.51 & 42.58 & 43.45 & 77.58 & 77.37 & 77.43 & 78.1 \\
 \hline
 32 & 43.35 & 42.49 & 43.64 & 43.73 & 77.86 & 75.32 & 77.28 & 77.86  \\
 \hline
 16 & 44.21 & 44.21 & 43.64 & 42.97 & 78.65 & 77 & 76.94 & 77.98  \\
 \hline
 \hline
 \multicolumn{5}{|c|}{Winogrande (FP32 Accuracy = 69.38\%)} & \multicolumn{4}{|c|}{Piqa (FP32 Accuracy = 78.07\%)} \\ 
 \hline
 \hline
 64 & 68.9 & 68.43 & 69.77 & 68.19 & 77.09 & 76.82 & 77.09 & 77.86 \\
 \hline
 32 & 69.38 & 68.51 & 68.82 & 68.90 & 78.07 & 76.71 & 78.07 & 77.86  \\
 \hline
 16 & 69.53 & 67.09 & 69.38 & 68.90 & 77.37 & 77.8 & 77.91 & 77.69  \\
 \hline
\end{tabular}
\caption{\label{tab:mmlu_abalation} Accuracy on LM evaluation harness tasks on Llama2-7B model.}
\end{table}

\begin{table} \centering
\begin{tabular}{|c||c|c|c|c||c|c|c|c|} 
\hline
 $L_b \rightarrow$& \multicolumn{4}{c||}{8} & \multicolumn{4}{c||}{8}\\
 \hline
 \backslashbox{$L_A$\kern-1em}{\kern-1em$N_c$} & 2 & 4 & 8 & 16 & 2 & 4 & 8 & 16  \\
 %$N_c \rightarrow$ & 2 & 4 & 8 & 16 & 2 & 4 & 2 \\
 \hline
 \hline
 \multicolumn{5}{|c|}{Race (FP32 Accuracy = 48.8\%)} & \multicolumn{4}{|c|}{Boolq (FP32 Accuracy = 85.23\%)} \\ 
 \hline
 \hline
 64 & 49.00 & 49.00 & 49.28 & 48.71 & 82.82 & 84.28 & 84.03 & 84.25 \\
 \hline
 32 & 49.57 & 48.52 & 48.33 & 49.28 & 83.85 & 84.46 & 84.31 & 84.93  \\
 \hline
 16 & 49.85 & 49.09 & 49.28 & 48.99 & 85.11 & 84.46 & 84.61 & 83.94  \\
 \hline
 \hline
 \multicolumn{5}{|c|}{Winogrande (FP32 Accuracy = 79.95\%)} & \multicolumn{4}{|c|}{Piqa (FP32 Accuracy = 81.56\%)} \\ 
 \hline
 \hline
 64 & 78.77 & 78.45 & 78.37 & 79.16 & 81.45 & 80.69 & 81.45 & 81.5 \\
 \hline
 32 & 78.45 & 79.01 & 78.69 & 80.66 & 81.56 & 80.58 & 81.18 & 81.34  \\
 \hline
 16 & 79.95 & 79.56 & 79.79 & 79.72 & 81.28 & 81.66 & 81.28 & 80.96  \\
 \hline
\end{tabular}
\caption{\label{tab:mmlu_abalation} Accuracy on LM evaluation harness tasks on Llama2-70B model.}
\end{table}

%\section{MSE Studies}
%\textcolor{red}{TODO}


\subsection{Number Formats and Quantization Method}
\label{subsec:numFormats_quantMethod}
\subsubsection{Integer Format}
An $n$-bit signed integer (INT) is typically represented with a 2s-complement format \citep{yao2022zeroquant,xiao2023smoothquant,dai2021vsq}, where the most significant bit denotes the sign.

\subsubsection{Floating Point Format}
An $n$-bit signed floating point (FP) number $x$ comprises of a 1-bit sign ($x_{\mathrm{sign}}$), $B_m$-bit mantissa ($x_{\mathrm{mant}}$) and $B_e$-bit exponent ($x_{\mathrm{exp}}$) such that $B_m+B_e=n-1$. The associated constant exponent bias ($E_{\mathrm{bias}}$) is computed as $(2^{{B_e}-1}-1)$. We denote this format as $E_{B_e}M_{B_m}$.  

\subsubsection{Quantization Scheme}
\label{subsec:quant_method}
A quantization scheme dictates how a given unquantized tensor is converted to its quantized representation. We consider FP formats for the purpose of illustration. Given an unquantized tensor $\bm{X}$ and an FP format $E_{B_e}M_{B_m}$, we first, we compute the quantization scale factor $s_X$ that maps the maximum absolute value of $\bm{X}$ to the maximum quantization level of the $E_{B_e}M_{B_m}$ format as follows:
\begin{align}
\label{eq:sf}
    s_X = \frac{\mathrm{max}(|\bm{X}|)}{\mathrm{max}(E_{B_e}M_{B_m})}
\end{align}
In the above equation, $|\cdot|$ denotes the absolute value function.

Next, we scale $\bm{X}$ by $s_X$ and quantize it to $\hat{\bm{X}}$ by rounding it to the nearest quantization level of $E_{B_e}M_{B_m}$ as:

\begin{align}
\label{eq:tensor_quant}
    \hat{\bm{X}} = \text{round-to-nearest}\left(\frac{\bm{X}}{s_X}, E_{B_e}M_{B_m}\right)
\end{align}

We perform dynamic max-scaled quantization \citep{wu2020integer}, where the scale factor $s$ for activations is dynamically computed during runtime.

\subsection{Vector Scaled Quantization}
\begin{wrapfigure}{r}{0.35\linewidth}
  \centering
  \includegraphics[width=\linewidth]{sections/figures/vsquant.jpg}
  \caption{\small Vectorwise decomposition for per-vector scaled quantization (VSQ \citep{dai2021vsq}).}
  \label{fig:vsquant}
\end{wrapfigure}
During VSQ \citep{dai2021vsq}, the operand tensors are decomposed into 1D vectors in a hardware friendly manner as shown in Figure \ref{fig:vsquant}. Since the decomposed tensors are used as operands in matrix multiplications during inference, it is beneficial to perform this decomposition along the reduction dimension of the multiplication. The vectorwise quantization is performed similar to tensorwise quantization described in Equations \ref{eq:sf} and \ref{eq:tensor_quant}, where a scale factor $s_v$ is required for each vector $\bm{v}$ that maps the maximum absolute value of that vector to the maximum quantization level. While smaller vector lengths can lead to larger accuracy gains, the associated memory and computational overheads due to the per-vector scale factors increases. To alleviate these overheads, VSQ \citep{dai2021vsq} proposed a second level quantization of the per-vector scale factors to unsigned integers, while MX \citep{rouhani2023shared} quantizes them to integer powers of 2 (denoted as $2^{INT}$).

\subsubsection{MX Format}
The MX format proposed in \citep{rouhani2023microscaling} introduces the concept of sub-block shifting. For every two scalar elements of $b$-bits each, there is a shared exponent bit. The value of this exponent bit is determined through an empirical analysis that targets minimizing quantization MSE. We note that the FP format $E_{1}M_{b}$ is strictly better than MX from an accuracy perspective since it allocates a dedicated exponent bit to each scalar as opposed to sharing it across two scalars. Therefore, we conservatively bound the accuracy of a $b+2$-bit signed MX format with that of a $E_{1}M_{b}$ format in our comparisons. For instance, we use E1M2 format as a proxy for MX4.

\begin{figure}
    \centering
    \includegraphics[width=1\linewidth]{sections//figures/BlockFormats.pdf}
    \caption{\small Comparing LO-BCQ to MX format.}
    \label{fig:block_formats}
\end{figure}

Figure \ref{fig:block_formats} compares our $4$-bit LO-BCQ block format to MX \citep{rouhani2023microscaling}. As shown, both LO-BCQ and MX decompose a given operand tensor into block arrays and each block array into blocks. Similar to MX, we find that per-block quantization ($L_b < L_A$) leads to better accuracy due to increased flexibility. While MX achieves this through per-block $1$-bit micro-scales, we associate a dedicated codebook to each block through a per-block codebook selector. Further, MX quantizes the per-block array scale-factor to E8M0 format without per-tensor scaling. In contrast during LO-BCQ, we find that per-tensor scaling combined with quantization of per-block array scale-factor to E4M3 format results in superior inference accuracy across models. 


\end{document}



%\textit{Acknowledgments}: We thank the JOT referees for many insightful comments on an earlier version of the article.


%%
%% The next two lines define the bibliography style to be used, and
%% the bibliography file.
%\bibliographystyle{abbrv}
\bibliography{main}
\section*{About the authors}
\shortbio{Maria Naumcheva}{%
is a PhD candidate at the University of Toulouse/IRIT. Her research interests include Requirements Engineering and Software Engineering Education.
\authorcontact {maria.naumcheva@irit.fr}}
\shortbio{Sophie Ebersold}{%
is full professor at the University of Toulouse. She is head of the SM@RT team of the IRIT CNRS laboratory. Her research areas include the modelling of complex systems, methods/model/language integration, with a focus on Requirements and Systems Engineering.
\authorcontact {sophie.ebersold@irit.fr}}
\shortbio{Alexandr Naumchev}{%
 holds a Ph.D. degree in Computer Science from the University of Toulouse and a Master of Science degree in Software Engineering from the Carnegie Mellon University. His doctoral thesis explores the idea of using an object-oriented language with contracts for requirements specification, verification, and reuse. Currently an independent researcher, Alexandr is focusing on static analysis of multithreaded programs and relational database queries.
\authorcontact {anaumchev@outlook.com}}
\shortbio{Jean-Michel Bruel}{%
is head of the Computer Science department of the Technical Institute of Blagnac and member of the Strategic Research Committee of the IRIT CNRS laboratory since 2021. His research areas include the development of software-intensive Cyber-Physical Systems, and methods/model/language integration, with a focus on Requirements and Model-Based Systems Engineering. 
\authorcontact {bruel@irit.fr}}
\shortbio{Florian Galinier}{%
 is co-founder and CEO of SPILEn, a company specialized in software and requirements engineering best practices. His research focuses on software engineering and model-driven engineering, with an emphasis on requirements engineering and domain specific modeling languages. He holds a PhD in Computer Science from the University of Toulouse (2021).
\authorcontact {fgalinier@spilen.fr}}
\shortbio{Bertrand Meyer}{%
is Professor of Software Engineering at Constructor Institute in Schaffhausen, Switzerland, and Chief Architect of Eiffel Software in Santa Barbara. He is interested in various aspects of software engineering including software verification and object technology.
\authorcontact {bm@sit.org}}
\end{document}