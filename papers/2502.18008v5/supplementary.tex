%%%% ijcai25.tex

\maketitle

\appendix

\section{MIDI Event Transformer}

MIDI Event Transformer (MET) is a pre-trained music generation model. It encodes MIDI events into token sequences and utilizes hierarchical transformer decoders for generation. 

\subsection{MIDI Encoding}

In the MIDI encoding process, each MIDI event is transformed into a token sequence $s_i$. This sequence begins with the token representing the event type $e_i$, followed by tokens corresponding to the event parameters $p_i^j$, as shown below:

\begin{equation}
s_i = \{e_i, p_i^1, p_i^2, ..., p_i^n\}
\end{equation}

The types of events and their corresponding parameters are listed in Table~\ref{tab:event list}. Among these, BOS (beginning-of-sequence) and EOS (end-of-sequence) events are used to mark the start and end of a musical piece, respectively. Parameters such as channel (16 values), pitch (128 values), velocity (128 values), controller (128 values), program (128 values), controller value (128 values) all comply with the MIDI standard.  The details of additional parameters are as follows:

\begin{itemize}

\item Time 1 and time 2: Represent the timing of an event. The absolute time position is calculated as the cumulative sum of ticks preceding the event, and the beat position is obtained by dividing this sum by the ticks per beat. Time 1 captures the beat difference between the current and previous events, with a vocabulary size of 128. To enable finer-grained positioning, a beat is subdivided into 16 parts, with time 2 indicating which subdivision the event falls into.

\item Track: Describes the track in which the event occurs, with a maximum of 128 possible tracks.

\item Duration: The note duration is calculated using a resolution of 64th notes, with an upper limit of 2048, which allows for the representation of notes up to 128 beats long.

\item BPM: Beats per minute, with a range from 1 to 384.

\item Numerator and denominator: The numerator can range from 1 to 16, while the denominator can be one of 2, 4, 8 or 16.

\item Key signature accidentals: Indicates the number of sharps or flats in the key signature. The value ranges from -7 (7 flats, C$\flat$) to 7 (7 sharps, C$\sharp$).

\item Mode: Specifies whether the key is major or minor, with values between 0 and 1.

\end{itemize}

Additionally, as a special token type for $p_i^j$, \texttt{<PAD>} is used to pad all sequences to a uniform length of 8. This results in a total vocabulary size of 3406 unique tokens.

During the fine-tuning and reinforcement learning stages of our experiments, we extend the original vocabulary by including 3 period IDs and 36 composer IDs. These are prepended as "period" and "composer" events—each represented by a sequence of 8 identical IDs—prior to the encoded MIDI sequences, serving as prompts.

\begin{table}[t]
    \centering
    \begin{tabular}{ll}
        \toprule
        Event  & Parameters \\
        \midrule
        \multirow{2}{*}{Note}           & time 1, time 2, track, channel, pitch,  \\
                                        & velocity, duration \\
        \multirow{2}{*}{Program Change} & time 1, time 2, track, channel, \\
                                        & program \\
        \multirow{2}{*}{Control Change} & time 1, time 2, track, channel,  \\
                                        & controller, controller value \\
        Set Tempo                       & time 1, time 2, track, bpm \\
        \multirow{2}{*}{Time Signature} & time 1, time 2, track, numerator,  \\
                                        & denominator \\
        \multirow{2}{*}{Key Signature}  & time 1, time 2, track, key signature \\
                                        & accidentals, mode \\
        BOS                             & \texttt{<BOS>} \\
        EOS                             & \texttt{<EOS>} \\
        \bottomrule
    \end{tabular}
    \caption{Event and parameter types in MIDI Event Transformer's encoding.}
    \label{tab:event list}
\end{table}


\subsection{Model Architecture}

The MIDI Event Transformer is comprised of two hierarchical decoders: an event-level decoder and a token-level decoder, both of which are based on the Llama architecture.

Initially, each event token sequence is processed through a token embedding layer, where individual token embeddings are aggregated to produce a dense vector representation. The event-level decoder focuses on modeling temporal dependencies across high-level events, thereby capturing their sequential relationships. The output hidden states from the event-level decoder are subsequently passed into the token-level decoder, which generates the detailed token sequence in an auto-regressive manner.

\begin{figure}[t] % "t" places the figure at the top of the page
\centering
\includegraphics[width=0.5\textwidth]{fig-MET.pdf} % Span across the entire page width
\caption{Illustration of the MIDI Event Transformer architecture, showcasing its two hierarchical decoders: the event-level decoder, which models temporal dependencies across high-level events, and the token-level decoder, which generates the detailed token sequence in an auto-regressive manner.}
\label{fig:example}
\end{figure}

\newpage

\section{Distribution of the Fine-tuning Dataset}

We utilized CLaMP 2 to extract the semantic features of all pieces, and the distribution of each label group is shown in Figure \ref{fig:Figure 3} (for composers, only eight composers with most pieces in the dataset are shown).



\begin{figure}[!b]
    \centering
    \begin{minipage}{0.41\textwidth}
        \centering
        \includegraphics[width=\textwidth]{imsleeping_v3_period.pdf}
        \caption*{(a) Distribution on periods.}
    \end{minipage} \\[0.5em]
    \begin{minipage}{0.41\textwidth}
        \centering
        \includegraphics[width=\textwidth]{imsleeping_v3_composer.pdf}
        \caption*{(b) Distribution on most frequent composers.}
    \end{minipage} \\[0.5em]
    \begin{minipage}{0.41\textwidth}
        \centering
        \includegraphics[width=\textwidth]{imsleeping_v3_inst.pdf}
        \caption*{(c) Distribution on instrumentations.}
    \end{minipage}
    \caption{\textit{t}-SNE visualizations for each label group on the fine-tuning dataset.}
    \label{fig:Figure 3}
\end{figure}




\section{Prompt Set in Experiments}

The full list of prompt set $P$ in the reinforcement learning stage of our experiments is listed in Table \ref{tab:promptset}.

\begin{table}[!htbp]
    \centering
    \begin{tabular}{@{}lll@{}}
        \toprule
        Period    & Composer                  & Instrumentation \\
        \midrule
        Baroque & Bach, Johann Sebastian & Chamber \\
        Baroque & Bach, Johann Sebastian & Choral \\
        Baroque & Bach, Johann Sebastian & Keyboard \\
        Baroque & Bach, Johann Sebastian & Orchestral \\
        Baroque & Bach, Johann Sebastian & Vocal-Orchestral \\
        Baroque & Corelli, Arcangelo & Chamber \\
        Baroque & Corelli, Arcangelo & Orchestral \\
        Baroque & Handel, George Frideric & Chamber \\
        Baroque & Handel, George Frideric & Keyboard \\
        Baroque & Handel, George Frideric & Orchestral \\
        Baroque & Handel, George Frideric & Vocal-Orchestral \\
        Baroque & Scarlatti, Domenico & Keyboard \\
        Baroque & Vivaldi, Antonio & Chamber \\
        Baroque & Vivaldi, Antonio & Orchestral \\
        Baroque & Vivaldi, Antonio & Vocal-Orchestral \\
        Classical & Beethoven, Ludwig van & Art Song \\
        Classical & Beethoven, Ludwig van & Chamber \\
        Classical & Beethoven, Ludwig van & Keyboard \\
        Classical & Beethoven, Ludwig van & Orchestral \\
        Classical & Haydn, Joseph & Chamber \\
        Classical & Haydn, Joseph & Keyboard \\
        Classical & Haydn, Joseph & Orchestral \\
        Classical & Haydn, Joseph & Vocal-Orchestral \\
        Classical & Mozart, Wolfgang Amadeus & Chamber \\
        Classical & Mozart, Wolfgang Amadeus & Choral \\
        Classical & Mozart, Wolfgang Amadeus & Keyboard \\
        Classical & Mozart, Wolfgang Amadeus & Orchestral \\
        Classical & Mozart, Wolfgang Amadeus & Vocal-Orchestral \\
        Classical & Paradis, Maria Theresia von & Art Song \\
        Classical & Reichardt, Louise & Art Song \\
        Classical & Saint-Georges, Joseph Bologne & Chamber \\
        Classical & Schroter, Corona & Art Song \\
        Romantic & Bartok, Bela & Keyboard \\
        Romantic & Berlioz, Hector & Choral \\
        Romantic & Bizet, Georges & Art Song \\
        Romantic & Boulanger, Lili & Art Song \\
        Romantic & Boulton, Harold & Art Song \\
        Romantic & Brahms, Johannes & Art Song \\
        Romantic & Brahms, Johannes & Chamber \\
        Romantic & Brahms, Johannes & Choral \\
        Romantic & Brahms, Johannes & Keyboard \\
        Romantic & Brahms, Johannes & Orchestral \\
        Romantic & Burgmuller, Friedrich & Keyboard \\
        Romantic & Butterworth, George & Art Song \\
        Romantic & Chaminade, Cecile & Art Song \\
        Romantic & Chausson, Ernest & Art Song \\
        Romantic & Chopin, Frederic & Art Song \\
        Romantic & Chopin, Frederic & Keyboard \\
        Romantic & Cornelius, Peter & Art Song \\
        Romantic & Debussy, Claude & Art Song \\
        Romantic & Debussy, Claude & Keyboard \\
        \bottomrule
        \multicolumn{3}{r}{Continued on next page}
    \end{tabular}
\end{table}

\begin{table}[!htbp]
    \centering
    \begin{tabular}{@{}lll@{}}
        \multicolumn{3}{r}{Table 2 -- Continued} \\
        \toprule
        Period    & Composer                  & Instrumentation \\
        \midrule
        Romantic & Dvorak, Antonin & Chamber \\
        Romantic & Dvorak, Antonin & Choral \\
        Romantic & Dvorak, Antonin & Keyboard \\
        Romantic & Dvorak, Antonin & Orchestral \\
        Romantic & Faisst, Clara & Art Song \\
        Romantic & Faure, Gabriel & Art Song \\
        Romantic & Faure, Gabriel & Chamber \\
        Romantic & Faure, Gabriel & Keyboard \\
        Romantic & Franz, Robert & Art Song \\
        Romantic & Gonzaga, Chiquinha & Art Song \\
        Romantic & Grandval, Clemence de & Art Song \\
        Romantic & Grieg, Edvard & Keyboard \\
        Romantic & Grieg, Edvard & Orchestral \\
        Romantic & Hensel, Fanny & Art Song \\
        Romantic & Holmes, Augusta Mary Anne & Art Song \\
        Romantic & Jaell, Marie & Art Song \\
        Romantic & Kinkel, Johanna & Art Song \\
        Romantic & Kralik, Mathilde & Art Song \\
        Romantic & Lang, Josephine & Art Song \\
        Romantic & Lehmann, Liza & Art Song \\
        Romantic & Liszt, Franz & Keyboard \\
        Romantic & Mayer, Emilie & Chamber \\
        Romantic & Medtner, Nikolay & Keyboard \\
        Romantic & Mendelssohn, Felix & Art Song \\
        Romantic & Mendelssohn, Felix & Chamber \\
        Romantic & Mendelssohn, Felix & Choral \\
        Romantic & Mendelssohn, Felix & Keyboard \\
        Romantic & Mendelssohn, Felix & Orchestral \\
        Romantic & Munktell, Helena & Art Song \\
        Romantic & Parratt, Walter & Choral \\
        Romantic & Prokofiev, Sergey & Keyboard \\
        Romantic & Rachmaninoff, Sergei & Choral \\
        Romantic & Rachmaninoff, Sergei & Keyboard \\
        Romantic & Ravel, Maurice & Art Song \\
        Romantic & Ravel, Maurice & Chamber \\
        Romantic & Ravel, Maurice & Keyboard \\
        Romantic & Saint-Saens, Camille & Chamber \\
        Romantic & Saint-Saens, Camille & Keyboard \\
        Romantic & Saint-Saens, Camille & Orchestral \\
        Romantic & Satie, Erik & Art Song \\
        Romantic & Satie, Erik & Keyboard \\
        Romantic & Schubert, Franz & Art Song \\
        Romantic & Schubert, Franz & Chamber \\
        Romantic & Schubert, Franz & Choral \\
        Romantic & Schubert, Franz & Keyboard \\
        Romantic & Schumann, Clara & Art Song \\
        Romantic & Schumann, Robert & Art Song \\
        Romantic & Schumann, Robert & Chamber \\
        Romantic & Schumann, Robert & Choral \\
        Romantic & Schumann, Robert & Keyboard \\
        Romantic & Scriabin, Aleksandr & Keyboard \\
        Romantic & Shostakovich, Dmitry & Chamber \\
        Romantic & Shostakovich, Dmitry & Keyboard \\
        Romantic & Sibelius, Jean & Keyboard \\
        Romantic & Smetana, Bedrich & Keyboard \\
        \bottomrule
        \multicolumn{3}{r}{Continued on next column}
    \end{tabular}
\end{table}


\begin{table}[!t]
    \centering
    \begin{tabular}{@{}lll@{}}
        \multicolumn{3}{r}{Table 2 -- Continued} \\
        \toprule
        Period    & Composer                  & Instrumentation \\
        \midrule        
        Romantic & Tchaikovsky, Pyotr & Keyboard \\
        Romantic & Tchaikovsky, Pyotr & Orchestral \\
        Romantic & Viardot, Pauline & Art Song \\
        Romantic & Warlock, Peter & Art Song \\
        Romantic & Wolf, Hugo & Art Song \\
        Romantic & Zumsteeg, Emilie & Art Song \\
        \bottomrule
    \end{tabular}
    \caption{The list of prompt set $P$ in reinforcement learning experiments.}
    \label{tab:promptset}
\end{table}



\end{document}

