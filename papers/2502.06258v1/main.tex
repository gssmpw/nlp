%%%%%%%% ICML 2025 EXAMPLE LATEX SUBMISSION FILE %%%%%%%%%%%%%%%%%

\documentclass{article}

% Recommended, but optional, packages for figures and better typesetting:
\usepackage{microtype}
\usepackage{graphicx}
% \usepackage{subfigure}
\usepackage{subcaption}
% \usepackage{booktabs} % for professional tables


% hyperref makes hyperlinks in the resulting PDF.
% If your build breaks (sometimes temporarily if a hyperlink spans a page)
% please comment out the following usepackage line and replace
% \usepackage{icml2025} with \usepackage[nohyperref]{icml2025} above.
\usepackage{hyperref}


% Attempt to make hyperref and algorithmic work together better:
\newcommand{\theHalgorithm}{\arabic{algorithm}}

% Use the following line for the initial blind version submitted for review:
% \usepackage{icml2025}
% \usepackage{etoolbox}
% \newtoggle{blind}
% \toggletrue{blind}

% Use the following line for the preprint (arxiv) version:
\usepackage[preprint]{icml2025}
\usepackage{etoolbox}
\newtoggle{blind}
\togglefalse{blind}

% If accepted, instead use the following line for the camera-ready submission:
% \usepackage[accepted]{icml2025}
% \usepackage{etoolbox}
% \newtoggle{blind}
% \togglefalse{blind}

% For theorems and such
\section{Summary of Mathematical Notations}
\label{sec:notations}

We summarize the main mathematical notations used in the main paper in Table \ref{table:notations}.

\begin{table*}[h]
	\centering
	\caption{Summary of main mathematical notations.}
	%\resizebox{1\columnwidth}{!}{
		\begin{tabular}{c|l}
			\toprule
			Notation  & Description  \\
			\hline
            $ \mathcal{G} $ & a document graph \\
			$ \mathcal{D} $ & a corpus of documents, $ \mathcal{D}=\{d_i\}_{i=1}^{N} $ \\
			$ N $ & number of documents in the corpus, $ N=|\mathcal{D}| $ \\
			$ d_i $ & document $ i $ containing a sequence of words, $ d_i=\{w_{i,v}\}_{v=1}^{|d_i|}\subset\mathcal{V} $ \\
			$ \mathcal{V} $ & vocabulary \\
			$ |d_i| $ & number of words in document $ i $ \\
			$ \mathcal{E} $ & a set of graph edges connecting documents, $ \mathcal{E}=\{e_{ij}\} $ \\
			$ \mathcal{N}(i) $ & the neighbor set of document $ i $ \\
            $ \Bbb H^{n,K} $ & Hyperboloid model with dimension $ n $ and curvature $ -1/K $ \\
			$ \mathcal{T}_{\textbf{x}}\Bbb H^{n,K} $ & tangent (Euclidean) space around hyperbolic vector $ x\in\Bbb H^{n,K} $ \\
			$ \exp_{\textbf{x}}^K(\textbf{v}) $ & exponential map, projecting tangent vector $ \textbf{v} $ to hyperbolic space \\
			$ \log_{\textbf{x}}^K(\textbf{y}) $ & logarithmic map, projecting hyperbolic vector $ \textbf{y} $ to $ \textbf{x} $'s tangent space \\
			$ d_{\mathcal{L}}^K(\textbf{x},\textbf{y}) $ & hyperbolic distance between hyperbolic vectors $ \textbf{x} $ and $ \textbf{y} $ \\
			$ \text{PT}_{\textbf{x}\rightarrow\textbf{y}}^K(\textbf{v}) $ & parallel transport, transporting $ \textbf{v} $ from $ \textbf{x} $'s tangent space to $ \textbf{y} $'s \\
			$ H $ & length of a path on topic tree \\
            $ \sigma(t,i) $ & similarity between topic $ t $ and document $ i $ \\
            $ \bm{\pi}_i $ & path distribution of document $ i $ over topic tree \\
            $ \textbf{z}_{t,p} $ & hyperbolic ancestral hidden state of topic $ t $ \\
            $ \textbf{z}_{t,s} $ & hyperbolic fraternal hidden state of topic $ t $ \\
			$ \textbf{z}_t $ & hyperbolic hidden state of topic $ t $ \\
            $ \sigma(h,i) $ & similarity between topic $ t $ and document $ i $ \\
            $ \textbf{z}_h $ & hyperbolic hidden state of level $ h $ \\
			$ \bm{\delta}_i $ & level distribution of document $ i $ over topic tree \\
            $ \bm{\theta}_i $ & topic distribution of document $ i $ over topic tree \\
            $ \textbf{e}_i $ & hierarchical tree embedding of document $ i $ \\
            $ T $ & number of topics on topic tree \\
            $ \textbf{g}_i $ & hierarchical graph embedding of document $ i $ \\
			$ \textbf{U} $ & a matrix of word embeddings, $ \textbf{U}\in\Bbb R^{|\mathcal{V}|\times(n+1)} $ \\
			$ \bm{\beta} $ & topic-word distribution $ \bm{\beta}\in\Bbb R^{T\times |\mathcal{V}|} $ \\
			\bottomrule
		\end{tabular}
	%}
	%\vspace{-0.2cm}
	\label{table:notations}
\end{table*}
% \usepackage{amsmath}
% \usepackage{amssymb}
% \usepackage{mathtools}
\usepackage{amsthm}
\usepackage{colortbl}
% if you use cleveref..
\usepackage[capitalize,noabbrev]{cleveref}

%%%%%%%%%%%%%%%%%%%%%%%%%%%%%%%%
% THEOREMS
%%%%%%%%%%%%%%%%%%%%%%%%%%%%%%%%
\theoremstyle{plain}
\newtheorem{theorem}{Theorem}[section]
\newtheorem{proposition}[theorem]{Proposition}
\newtheorem{lemma}[theorem]{Lemma}
\newtheorem{corollary}[theorem]{Corollary}
\theoremstyle{definition}
\newtheorem{definition}[theorem]{Definition}
\newtheorem{assumption}[theorem]{Assumption}
\theoremstyle{remark}
\newtheorem{remark}[theorem]{Remark}

% Todonotes is useful during development; simply uncomment the next line
%    and comment out the line below the next line to turn off comments
% \usepackage[disable,textsize=tiny]{todonotes}
% \usepackage[textsize=tiny]{todonotes} 


% The \icmltitle you define below is probably too long as a header.
% Therefore, a short form for the running title is supplied here:
\icmltitlerunning{Emergent Response Planning in LLM}

\begin{document}

\twocolumn[
\icmltitle{
% Beyond Next-Token Prediction: 
Emergent Response Planning in LLM
%through Hidden State Encoding
}

% It is OKAY to include author information, even for blind
% submissions: the style file will automatically remove it for you
% unless you've provided the [accepted] option to the icml2025
% package.

% List of affiliations: The first argument should be a (short)
% identifier you will use later to specify author affiliations
% Academic affiliations should list Department, University, City, Region, Country
% Industry affiliations should list Company, City, Region, Country

% You can specify symbols, otherwise they are numbered in order.
% Ideally, you should not use this facility. Affiliations will be numbered
% in order of appearance and this is the preferred way.
\icmlsetsymbol{equal}{*}

\begin{icmlauthorlist}
\icmlauthor{Zhichen Dong$^*$}{shailab}
\icmlauthor{Zhanhui Zhou$^*$}{shailab}
\icmlauthor{Zhixuan Liu}{shailab}
\icmlauthor{Chao Yang}{shailab}
\icmlauthor{Chaochao Lu}{shailab}
%\icmlauthor{}{sch}
%\icmlauthor{}{sch}
\end{icmlauthorlist}

% \icmlaffiliation{yyy}{Department of XXX, University of YYY, Location, Country}
% \icmlaffiliation{comp}{Company Name, Location, Country}
\icmlaffiliation{shailab}{Shanghai Artificial Intelligence Laboratory}

\icmlcorrespondingauthor{Zhichen Dong}{dongzhichen@pjlab.org.cn}
\icmlcorrespondingauthor{Zhanhui Zhou}{zhouzhanhui@pjlab.org.cn}
\icmlcorrespondingauthor{Chao yang}{yangchao@pjlab.org.cn}

% You may provide any keywords that you
% find helpful for describing your paper; these are used to populate
% the "keywords" metadata in the PDF but will not be shown in the document
\icmlkeywords{Large Language Models, Emergent Response Planning, Hidden Representation Probing}

\vskip 0.3in
]

% this must go after the closing bracket ] following \twocolumn[ ...

% This command actually creates the footnote in the first column
% listing the affiliations and the copyright notice.
% The command takes one argument, which is text to display at the start of the footnote.
% The \icmlEqualContribution command is standard text for equal contribution.
% Remove it (just {}) if you do not need this facility.

%\printAffiliationsAndNotice{}  % leave blank if no need to mention equal contribution
\printAffiliationsAndNotice{\icmlEqualContribution} % otherwise use the standard text.


\begin{abstract}


The choice of representation for geographic location significantly impacts the accuracy of models for a broad range of geospatial tasks, including fine-grained species classification, population density estimation, and biome classification. Recent works like SatCLIP and GeoCLIP learn such representations by contrastively aligning geolocation with co-located images. While these methods work exceptionally well, in this paper, we posit that the current training strategies fail to fully capture the important visual features. We provide an information theoretic perspective on why the resulting embeddings from these methods discard crucial visual information that is important for many downstream tasks. To solve this problem, we propose a novel retrieval-augmented strategy called RANGE. We build our method on the intuition that the visual features of a location can be estimated by combining the visual features from multiple similar-looking locations. We evaluate our method across a wide variety of tasks. Our results show that RANGE outperforms the existing state-of-the-art models with significant margins in most tasks. We show gains of up to 13.1\% on classification tasks and 0.145 $R^2$ on regression tasks. All our code and models will be made available at: \href{https://github.com/mvrl/RANGE}{https://github.com/mvrl/RANGE}.

\end{abstract}


\section{Introduction}

Video generation has garnered significant attention owing to its transformative potential across a wide range of applications, such media content creation~\citep{polyak2024movie}, advertising~\citep{zhang2024virbo,bacher2021advert}, video games~\citep{yang2024playable,valevski2024diffusion, oasis2024}, and world model simulators~\citep{ha2018world, videoworldsimulators2024, agarwal2025cosmos}. Benefiting from advanced generative algorithms~\citep{goodfellow2014generative, ho2020denoising, liu2023flow, lipman2023flow}, scalable model architectures~\citep{vaswani2017attention, peebles2023scalable}, vast amounts of internet-sourced data~\citep{chen2024panda, nan2024openvid, ju2024miradata}, and ongoing expansion of computing capabilities~\citep{nvidia2022h100, nvidia2023dgxgh200, nvidia2024h200nvl}, remarkable advancements have been achieved in the field of video generation~\citep{ho2022video, ho2022imagen, singer2023makeavideo, blattmann2023align, videoworldsimulators2024, kuaishou2024klingai, yang2024cogvideox, jin2024pyramidal, polyak2024movie, kong2024hunyuanvideo, ji2024prompt}.


In this work, we present \textbf{\ours}, a family of rectified flow~\citep{lipman2023flow, liu2023flow} transformer models designed for joint image and video generation, establishing a pathway toward industry-grade performance. This report centers on four key components: data curation, model architecture design, flow formulation, and training infrastructure optimization—each rigorously refined to meet the demands of high-quality, large-scale video generation.


\begin{figure}[ht]
    \centering
    \begin{subfigure}[b]{0.82\linewidth}
        \centering
        \includegraphics[width=\linewidth]{figures/t2i_1024.pdf}
        \caption{Text-to-Image Samples}\label{fig:main-demo-t2i}
    \end{subfigure}
    \vfill
    \begin{subfigure}[b]{0.82\linewidth}
        \centering
        \includegraphics[width=\linewidth]{figures/t2v_samples.pdf}
        \caption{Text-to-Video Samples}\label{fig:main-demo-t2v}
    \end{subfigure}
\caption{\textbf{Generated samples from \ours.} Key components are highlighted in \textcolor{red}{\textbf{RED}}.}\label{fig:main-demo}
\end{figure}


First, we present a comprehensive data processing pipeline designed to construct large-scale, high-quality image and video-text datasets. The pipeline integrates multiple advanced techniques, including video and image filtering based on aesthetic scores, OCR-driven content analysis, and subjective evaluations, to ensure exceptional visual and contextual quality. Furthermore, we employ multimodal large language models~(MLLMs)~\citep{yuan2025tarsier2} to generate dense and contextually aligned captions, which are subsequently refined using an additional large language model~(LLM)~\citep{yang2024qwen2} to enhance their accuracy, fluency, and descriptive richness. As a result, we have curated a robust training dataset comprising approximately 36M video-text pairs and 160M image-text pairs, which are proven sufficient for training industry-level generative models.

Secondly, we take a pioneering step by applying rectified flow formulation~\citep{lipman2023flow} for joint image and video generation, implemented through the \ours model family, which comprises Transformer architectures with 2B and 8B parameters. At its core, the \ours framework employs a 3D joint image-video variational autoencoder (VAE) to compress image and video inputs into a shared latent space, facilitating unified representation. This shared latent space is coupled with a full-attention~\citep{vaswani2017attention} mechanism, enabling seamless joint training of image and video. This architecture delivers high-quality, coherent outputs across both images and videos, establishing a unified framework for visual generation tasks.


Furthermore, to support the training of \ours at scale, we have developed a robust infrastructure tailored for large-scale model training. Our approach incorporates advanced parallelism strategies~\citep{jacobs2023deepspeed, pytorch_fsdp} to manage memory efficiently during long-context training. Additionally, we employ ByteCheckpoint~\citep{wan2024bytecheckpoint} for high-performance checkpointing and integrate fault-tolerant mechanisms from MegaScale~\citep{jiang2024megascale} to ensure stability and scalability across large GPU clusters. These optimizations enable \ours to handle the computational and data challenges of generative modeling with exceptional efficiency and reliability.


We evaluate \ours on both text-to-image and text-to-video benchmarks to highlight its competitive advantages. For text-to-image generation, \ours-T2I demonstrates strong performance across multiple benchmarks, including T2I-CompBench~\citep{huang2023t2i-compbench}, GenEval~\citep{ghosh2024geneval}, and DPG-Bench~\citep{hu2024ella_dbgbench}, excelling in both visual quality and text-image alignment. In text-to-video benchmarks, \ours-T2V achieves state-of-the-art performance on the UCF-101~\citep{ucf101} zero-shot generation task. Additionally, \ours-T2V attains an impressive score of \textbf{84.85} on VBench~\citep{huang2024vbench}, securing the top position on the leaderboard (as of 2025-01-25) and surpassing several leading commercial text-to-video models. Qualitative results, illustrated in \Cref{fig:main-demo}, further demonstrate the superior quality of the generated media samples. These findings underscore \ours's effectiveness in multi-modal generation and its potential as a high-performing solution for both research and commercial applications.
\begin{figure*}[t]
  \centering
    \includegraphics[width=1\linewidth]{visuals/final_registration.png}
    \caption{Target measurement process on low-cost scan data using ICP and Coloured ICP. (1) Initialisation: The source point cloud (checkerboard) is misaligned with the target point cloud. (2) Initial Registration using Point-to-Plane ICP: Standard ICP leads to suboptimal registration. (3) Final Registration using Coloured ICP: Colour information is incorporated after pre-processing with RANSAC and Binarisation with Otsu Thresholding for real data, resulting in improved alignment.}
    \label{fig:Registration_visualisation}
\end{figure*}

\subsection{Iterative Closest Point (ICP) Algorithm}
The Iterative Closest Point (ICP) algorithm has been a fundamental technique in 3D computer vision and robotics for point cloud. Originally proposed by \cite{besl_method_1992}, ICP aims to minimise the distance between two datasets, typically referred to as the source and the target. The algorithm operates in an iterative manner, identifying correspondences by matching each source point with its nearest target point \citep{survey_ICP}. It then computes the rigid transformation, usually involving both rotation and translation, to achieve the best alignment of these matched points \citep{survey_ICP}. This process is repeated until convergence, where the change in the alignment parameters or the overall alignment error becomes smaller than a predefined threshold.

One key advantage of the ICP framework lies in its simplicity: the algorithm is conceptually straightforward, and its basic version is relatively easy to implement. However, traditional ICP can be sensitive to local minima, often requiring a good initial alignment \citep{zhang2021fast}. Furthermore, outliers, noise, and partial overlaps between datasets can significantly degrade its performance \citep{zhang2021fast, bouaziz2013sparse}. Over the years, various modifications and improvements \citep{gelfand2005robust, rusu2009fast, aiger20084, gruen2005least, fitzgibbon2003robust} have been proposed to mitigate these issues. Among the most common strategies are robust cost functions \citep{fitzgibbon2003robust}, weighting schemes for correspondences \citep{rusu2009fast}, and more sophisticated techniques \citep{gelfand2005robust, bouaziz2013sparse} to reject outliers. 

In addition, there is significant interest in integrating additional information into the ICP pipeline. Instead of solely relying on geometric cues such as point coordinates or surface normals, recent approaches have proposed incorporating colour (RGB) or intensity data to enhance correspondence accuracy. These methods \citep{park_colored_2017, 5980407}, commonly known as "Colored ICP" employ differences in pixel intensities or colour values as additional constraints. This is particularly beneficial in situations where geometric attributes alone are inadequate for accurate alignment or where surfaces possess complex texture patterns that can assist in the matching process.

\subsection{Applications of Target Measurement}

One approach relies on the use of physical checkerboard targets for registration. \cite{fryskowska2019} analyse checkerboard target identification for terrestrial laser scanning. They propose a geometric method to determine the target centre with higher precision, demonstrating that their approach can reduce errors by up to 6 mm compared to conventional automatic methods.

\cite{becerik2011assessment} examines data acquisition errors in 3D laser scanning for construction by evaluating how different target types (paper, paddle, and sphere) and layouts impact registration accuracy in both indoor and outdoor environments and presents guidelines for optimal target configuration.

\citet{Liang2024} propose to use Coloured ICP to measure target centres for checkerboard targets, similar to our investigation. They use data from a survey-grade terrestrial laser scanner. Their intended application is structural bridge monitoring purposes. They report an average accuracy of the measurement below 1.3 millimetres.

Where targets cannot be placed in the scene, the intensity information form the scanner can still be used to identify distinctive points. For point cloud data that is captured with a regular pattern, standard image processing can be used in a similar way to target detection. For example, \citet{wendt_automation_2004} proposes to use the SUSAN operator on a co-registered image from a camera, \citet{bohm_automatic_2007} proposes to use the SIFT operator on the LIDAR reflectance directly and \citet{theiler_markerless_2013} propose to use a Difference-of-Gaussian approach on the reflectance information.
Most of these methods extract image features to find reliable 3D correspondences for the purpose of registration.

In the following we describe our approach to the measurement of the target centre. In contrast to most proposed methods above we focus on unordered point clouds, where raster-based methods are not available, and low-cost sensors, where increased measurement noise and outliers are expected. As we are not aware of a commercial reference solution to this problem, we start by conducting a series of synthetic experiments to explore the viability and accuracy potential of the approach.



%The reviewed studies primarily rely on physical targets or target-free methods and do not utilise 3D synthetic point cloud checkerboards. In contrast, our approach introduces synthetic point cloud checkerboards, which offer controlled and consistent target geometry and reduce variability caused by physical targets. This innovation has significant potential for commercialisation and industrial application.

% \section{Preliminaries}
% \textbf{Agentic LLM workflow.} Similar to how humans are more efficient in a well-coordinated group, language models also benefit from having a team of peer agents which contribute to the task and make the overall workflow more efficient. Some common practices involve decomposing the task into multiple subtasks, which are assigned to one or more agents for completion. While this makes the pipeline more involved, it vastly enhances the framework's efficiency. Creating agents specifically prompted to be efficient at that subtask is analogous to having multiple \emph{expert agents} who work in harmony, unlike a single generalized agent for the entire workflow. Moreover, LLM agents can also act as reviewer to process or evaluate responses and actions from other agents~\cite{zhuge2024agentasajudgeevaluateagentsagents}. This alleviates human evaluation in certain scenarios requiring significant time and compute. Agents can also be employed for guardrailing, preventing adversarial attacks on the framework in attempts to extract sensitive information. Section \ref{sec:5.3} highlights the efficacy of multi-agent frameworks against jailbreaking techniques.

% \textbf{LLM Unlearning.} Given a set $S = \{s_1, s_2, \cdots, s_N\}$ of $N$ unlearning targets and a user query $x \in \mathcal{X}$, the principle of an unlearning framework is to ensure that the unlearned model $\pi_{\theta_{\text{ul}}}$ generates responses $y$ which maximize unlearning efficacy and response utility. Hence, an ideal response must answer the user query effectively while obscuring references to the unlearning targets. We formalize this objective as follows 
% \begin{equation*}
% \begin{split}
% \pi^* = \underset{\pi_{\theta_{\text{ul}}}}{\operatorname{argmin}} \Biggl[ & \underbrace{\mathcal{D}_{\text{KL}}(\pi_{\theta_{\text{ul}}}(\cdot|x) || \pi_{\theta}(\cdot|x))}_{\text{Utility preservation}} \\
% & + \lambda \underbrace{\mathbb{E}_{y \sim \pi_{\theta_{\text{ul}}}(\cdot|x)} \left[\mathbbm{1}_{\{\exists s \in S : s \in y\}} \right]}_{\text{Unlearning penalty}} \Biggr]
% \end{split}
% \end{equation*}
% Here, $\mathcal{D}_{\text{KL}}(\cdots)$ measures the Kullback-Leibler divergence between the unlearned model  $\pi_{\theta_{\text{ul}}}$ and the original (non-unlearned) model $\pi_{\theta}$. Minimizing the KL-divergence between the two distributions allows the response from $\pi_{\theta_{\text{ul}}}$ to retain the utility of the response from $\pi_{\theta}$. $\lambda \ge 0$ is a hyperparameter that balances the utility and the unlearning strictness, increasing which will encourage the model to emphasize rigorous unlearning at the cost of response utility.

% $\mathcal{X}$ can contain prompts which are engineered to extract sensitive information from $\pi_{\theta_{\text{ul}}}$ \cite{zou2023universaltransferableadversarialattacks}, and we do not make assumptions about the intention of the user as done in \citet{thaker2024guardrail}. \citet{liu2024revisitingwhosharrypotter} points out that current post hoc unlearning methods are brittle to state-of-the-art adversarial attacks \cite{lynch2024eight, anil2024many}, preventing them from being deployed in practical settings. Section \ref{sec:5.4} highlights the robustness of \texttt{ALU} under adversarial attacks. 

\section{Experiment Results}

\begin{figure*}[t]
    \centering
    \includegraphics[width=\textwidth]{figure/QualitativeResults.png}
    \caption{Qualitative comparison of other methods and TTMG on Radiology and Ultrasound modalities.  (a) Input images with ground truth. (b) baseline (DeepLabV3+ with ResNet50, \cite{chen2018encoder}). (c) IN \cite{ulyanov2017improved}. (d) IW \cite{huang2018decorrelated}. (e) IBN \cite{pan2018two}. (f) RobustNet \cite{choi2021robustnet}. (g) SAN-SAW \cite{peng2022semantic}. (h) SPCNet \cite{huang2023style}. (i) BlindNet \cite{ahn2024style}. (j) \textbf{TTMG (Ours)}. In this figure, \textcolor{green}{\textbf{Green}} and \textcolor{red}{\textbf{Red}} lines denote the boundaries of the ground truth and prediction, respectively.}
    \label{fig:QualitativeResults}
\end{figure*}

\subsection{Experiment Settings}
\label{s41_experiment_settings}

In this paper, we use eleven datasets spanning four modalities (colonoscopy (C), ultrasound (U), dermoscopy (D), and radiology (R)), which have been used for training and evaluating various deep learning-based medical image segmentation models \cite{zhou2018unet++, chen2021transunet,  cao2022swin, wang2022uctransnet, zhao2023m, jiang2023vig, wang2024cfatransunet, nam2024modality}. For convenience, we denote the seen and unseen modalities as the test datasets, which are the same and different modalities as source training modality datasets, respectively. Due to the page limit, we present the detailed dataset description in the Appendix \ref{appendix_dataset_descriptions}. Since this paper consider the case of multi-source modality ($M > 1$), we conducted two experiments for $M = 3$ and $M = 2$, which are listed in Tables \ref{tab:comparison_three_modality} and \ref{tab:comparison_two_modality}, respectively. To evaluate the performance of each method, we selected two metrics, the Dice Score Coefficient (DSC) and mean Intersection over Union (mIoU), which are widely used in medical image segmentation. Additionally, the quantitative results with more various metrics \cite{margolin2014evaluate, fan2017structure, fan2018enhanced} and metrics descriptions are also available in the Appendix. 

We compared the proposed \textbf{TTMG} with seven representative domain generalization method, including IN \cite{ulyanov2017improved}, IW \cite{huang2018decorrelated}, IBN \cite{pan2018two}, RobustNet \cite{choi2021robustnet}, SAN-SAW \cite{peng2022semantic}, SPCNet \cite{huang2023style}, and BlindNet \cite{ahn2024style}. For all tables, \textcolor{red}{\textbf{\underline{Red}}} and \textcolor{blue}{\textbf{\textit{Blue}}} are the first and second-best performance results, respectively. And, $( \cdot )$ denotes the performance gap between the baseline and each method. Additionally, the last row indicates the performance gap between \textbf{TTMG} and other best methods.

\subsection{Implementation Details}
\label{s42_implementation_details}

\noindent\textbf{Training Settings.} We implemented TTMG on a single NVIDIA RTX 3090 Ti in Pytorch 1.8 \cite{NEURIPS2019_9015}. Following the other DGSS methods, we choose DeepLabV3+ \cite{chen2018encoder} with ResNet50 \cite{he2016deep} as baseline. Additionally, we also provide the experiment results with other backbones (MobileNetV2 \cite{sandler2018mobilenetv2} and ShuffleNetV2 \cite{ma2018shufflenet}) and TransUNet \cite{chen2021transunet}, which is the most representative Transformer-based model, in the Appendix (Table \ref{tab:backbone_type}) due to the page limit. Additionally, we employ an Adam optimizer \cite{KingmaB14} with an initial learning rate of $10^{-4}$ and reduced the parameters of each method to $10^{-6}$ using a cosine annealing learning rate scheduler \cite{loshchilov2016sgdr}. According to the most representative works for medical image segmentation, the required epochs for each modalities are 50, 100, 100, and 200 epochs. Since we only consider multi-modality ($M > 1$) training scheme in this paper, we choose the largest epoch when different epochs are used. For example, we trained the model using colonoscopy and radiology modalities, and then we optimized the model for 200 epochs since the radiology dataset requires more epochs. During training, we used horizontal/vertical flipping, with a probability of 50\%, and rotation between $-5^{\circ}$ and $5^{\circ}$, which is widely used in medical image segmentation \cite{fan2020pranet, zhao2021automatic, zhao2023m, nam2024modality, nam2025transgunet}. At this stage, because images in each dataset have different resolutions, all images were resized to $352 \times 352$.

\noindent\textbf{Hyperparameters of TTMG.} Key hyperparameters for TTMG on all datasets were set to $K_{m} = 8$ in MASP, and $k = 3$ and $s = 1$ in MSIW following RobustNet \cite{choi2021robustnet}. Additionally, $\beta_{m}$ is randomly initialized to start training. We apply MASP and MSIW in two shallow layers, that is, $L = \{ 1, 2 \}$ (Stage 1 and Stage 2). In the Appendix, we provide the experiment results on various hyperparameter settings. \vspace{-0.15cm}

\begin{table*} [t]
    \centering
    \small
    \setlength\tabcolsep{5.5pt} % default value: 6pt
    % \renewcommand{\arraystretch}{0.80} % Tighter
    \begin{tabular}{c|cc|cc|cc|cc|cc|cc} 
    \hline
    \multicolumn{1}{c|}{\multirow{3}{*}{Method}} & \multicolumn{4}{c|}{Training Modalities (C, D)} & \multicolumn{4}{c|}{Training Modalities (C, R)} & \multicolumn{4}{c}{Training Modalities (C, U)} \\\cline{2-13}
     & \multicolumn{2}{c|}{Seen (C, D)} & \multicolumn{2}{c|}{Unseen (U, R)} & \multicolumn{2}{c|}{Seen (C, R)} & \multicolumn{2}{c|}{Unseen (D, U)} & \multicolumn{2}{c|}{Seen (C, U)} & \multicolumn{2}{c}{Unseen (D, R)} \\\cline{2-13}
     & DSC & mIoU & DSC & mIoU & DSC & mIoU & DSC & mIoU & DSC & mIoU & DSC & mIoU \\
    \hline
    baseline \cite{chen2018encoder} & 81.62 & \textcolor{blue}{\textbf{\textit{75.69}}} & 14.81 & 9.00 & 73.51 & 66.57 & 4.82 & 3.40 & 80.62 & \textcolor{blue}{\textbf{\textit{73.45}}} & 31.78 & 25.68 \\
     \hline
     \multicolumn{1}{c|}{\multirow{2}{*}{IN \cite{ulyanov2017improved}}} & 81.19 & 74.55 & 21.86 & 14.59 & \textcolor{blue}{\textbf{\textit{74.82}}} & \textcolor{blue}{\textbf{\textit{67.44}}} & 2.23 & 1.49 & 80.04 & 72.57 & 34.65 & 27.97 \\
      & \textcolor{red}{\scriptsize{\textbf{(-0.43)}}}         & \textcolor{red}{\scriptsize{\textbf{(-1.14)}}} 
      & \textcolor{ForestGreen}{\scriptsize{\textbf{(+7.05)}}} & \textcolor{ForestGreen}{\scriptsize{\textbf{(+5.59)}}} 
      & \textcolor{ForestGreen}{\scriptsize{\textbf{(+1.31)}}} & \textcolor{ForestGreen}{\scriptsize{\textbf{(+0.87)}}} 
      & \textcolor{red}{\scriptsize{\textbf{(-2.59)}}}         & \textcolor{red}{\scriptsize{\textbf{(-1.91)}}} 
      & \textcolor{red}{\scriptsize{\textbf{(-0.58)}}}         & \textcolor{red}{\scriptsize{\textbf{(-0.88)}}} 
      & \textcolor{ForestGreen}{\scriptsize{\textbf{(+2.87)}}} & \textcolor{ForestGreen}{\scriptsize{\textbf{(+2.29)}}} \\
    \hline
     \multicolumn{1}{c|}{\multirow{2}{*}{IW \cite{huang2018decorrelated}}} & 70.82 & 62.99 & 18.41 & 12.37 & 53.71 & 46.37 & 4.52 & 2.67 & 66.70 & 57.98 & 32.87 & 25.58 \\
      & \textcolor{red}{\scriptsize{\textbf{(-10.80)}}}        & \textcolor{red}{\scriptsize{\textbf{(-12.70)}}} 
      & \textcolor{ForestGreen}{\scriptsize{\textbf{(+3.60)}}} & \textcolor{ForestGreen}{\scriptsize{\textbf{(+3.37)}}} 
      & \textcolor{red}{\scriptsize{\textbf{(-19.80)}}}        & \textcolor{red}{\scriptsize{\textbf{(-20.20)}}} 
      & \textcolor{red}{\scriptsize{(\textbf{-0.30)}}}         & \textcolor{red}{\scriptsize{\textbf{(-0.73)}}} 
      & \textcolor{red}{\scriptsize{\textbf{(-13.92)}}}        & \textcolor{red}{\scriptsize{\textbf{(-15.47)}}} 
      & \textcolor{ForestGreen}{\scriptsize{\textbf{(+1.09)}}} & \textcolor{red}{\scriptsize{\textbf{(-0.10)}}} \\
    \hline
     \multicolumn{1}{c|}{\multirow{2}{*}{IBN \cite{pan2018two}}} & \textcolor{red}{\textbf{\underline{82.30}}} & \textcolor{red}{\textbf{\underline{75.72}}} & 21.93 & 15.15 & 73.92 & 66.60 & 2.50 & 1.83 & 80.62 & 73.31 & \textcolor{blue}{\textbf{\textit{37.59}}} & \textcolor{red}{\textbf{\underline{31.41}}} \\
      & \textcolor{ForestGreen}{\scriptsize{\textbf{(+0.68)}}} & \textcolor{ForestGreen}{\scriptsize{\textbf{(+0.03)}}} 
      & \textcolor{ForestGreen}{\scriptsize{\textbf{(+7.12)}}} & \textcolor{ForestGreen}{\scriptsize{\textbf{(+6.15)}}} 
      & \textcolor{ForestGreen}{\scriptsize{\textbf{(+0.41)}}} & \textcolor{ForestGreen}{\scriptsize{\textbf{(+0.03)}}} 
      & \textcolor{red}{\scriptsize{\textbf{(-2.32)}}}         & \textcolor{red}{\scriptsize{\textbf{(-1.57)}}} 
      & \scriptsize{\textbf{(+0.00)}}                          & \textcolor{red}{\scriptsize{\textbf{(-0.14)}}} 
      & \textcolor{ForestGreen}{\scriptsize{\textbf{(+5.81)}}} & \textcolor{ForestGreen}{\scriptsize{\textbf{(+5.73)}}} \\
    \hline
     \multicolumn{1}{c|}{\multirow{2}{*}{RobustNet \cite{choi2021robustnet}}} & 79.69 & 72.54 & \textcolor{blue}{\textbf{\textit{25.49}}} & \textcolor{blue}{\textbf{\textit{19.29}}} & 67.18 & 60.01 & 7.13 & 4.63 & 75.83 & 68.01 & 31.92 & 25.96 \\
      & \textcolor{red}{\scriptsize{\textbf{(-1.93)}}}          & \textcolor{red}{\scriptsize{\textbf{(-3.15)}}} 
      & \textcolor{ForestGreen}{\scriptsize{\textbf{(+10.68)}}} & \textcolor{ForestGreen}{\scriptsize{\textbf{(+10.29)}}} 
      & \textcolor{red}{\scriptsize{\textbf{(-6.33)}}}          & \textcolor{red}{\scriptsize{\textbf{(-6.56)}}} 
      & \textcolor{ForestGreen}{\scriptsize{\textbf{(+2.31)}}}  & \textcolor{ForestGreen}{\scriptsize{\textbf{(+1.23)}}} 
      & \textcolor{red}{\scriptsize{\textbf{(-4.79)}}}          & \textcolor{red}{\scriptsize{\textbf{(-5.44)}}} 
      & \textcolor{ForestGreen}{\scriptsize{\textbf{(+0.14)}}}  & \textcolor{ForestGreen}{\scriptsize{\textbf{(+0.28)}}} \\
    \hline
     \multicolumn{1}{c|}{\multirow{2}{*}{SAN-SAW \cite{peng2022semantic}}} & 75.25 & 68.14 & 17.76 & 12.65 & 68.22 & 61.31 & 4.95 & 3.49 & 70.38 & 62.50 & 28.59 & 22.18 \\
      & \textcolor{red}{\scriptsize{\textbf{(-6.37)}}}         & \textcolor{red}{\scriptsize{\textbf{(-7.55)}}} 
      & \textcolor{ForestGreen}{\scriptsize{\textbf{(+2.95)}}} & \textcolor{ForestGreen}{\scriptsize{\textbf{(+3.65)}}} 
      & \textcolor{red}{\scriptsize{\textbf{(-5.20)}}}         & \textcolor{red}{\scriptsize{\textbf{(-5.26)}}} 
      & \textcolor{ForestGreen}{\scriptsize{\textbf{(+0.13)}}} & \textcolor{ForestGreen}{\scriptsize{\textbf{(+0.09)}}} 
      & \textcolor{red}{\scriptsize{\textbf{(-10.24)}}}        & \textcolor{red}{\scriptsize{\textbf{(-10.95)}}} 
      & \textcolor{red}{\scriptsize{\textbf{(-3.19)}}}         & \textcolor{red}{\scriptsize{\textbf{(-3.50)}}} \\
    \hline
     \multicolumn{1}{c|}{\multirow{2}{*}{SPCNet \cite{huang2023style}}} & \textcolor{blue}{\textbf{\textit{81.95}}} & 75.23 & 13.97 & 8.73 & 72.61 & 65.65 & \textcolor{blue}{\textbf{\textit{8.42}}} & \textcolor{blue}{\textbf{\textit{5.71}}} & \textcolor{blue}{\textbf{\textit{80.63}}} & 73.32 & 35.56 & 28.88 \\
      & \textcolor{ForestGreen}{\scriptsize{\textbf{(+0.33)}}} & \textcolor{red}{\scriptsize{\textbf{(-0.46)}}} 
      & \textcolor{red}{\scriptsize{\textbf{(-0.84)}}}         & \textcolor{red}{\scriptsize{\textbf{(-0.27)}}} 
      & \textcolor{red}{\scriptsize{\textbf{(-0.90)}}}         & \textcolor{red}{\scriptsize{\textbf{(-0.92)}}} 
      & \textcolor{ForestGreen}{\scriptsize{\textbf{(+3.60)}}} & \textcolor{ForestGreen}{\scriptsize{\textbf{(+2.31)}}} 
      & \textcolor{ForestGreen}{\scriptsize{\textbf{(+0.01)}}} & \textcolor{red}{\scriptsize{\textbf{(-0.13)}}} 
      & \textcolor{ForestGreen}{\scriptsize{\textbf{(+3.78)}}} & \textcolor{ForestGreen}{\scriptsize{\textbf{(+3.20)}}} \\
    \hline
     \multicolumn{1}{c|}{\multirow{2}{*}{BlindNet \cite{ahn2024style}}} & 80.58 & 73.76 & 24.51 & 18.32 & 73.02 & 65.75 & 7.06 & 4.81 & 78.07 & 70.57 & 28.96 & 23.27 \\
      & \textcolor{red}{\scriptsize{\textbf{(-1.04)}}}         & \textcolor{red}{\scriptsize{\textbf{(-1.93)}}} 
      & \textcolor{ForestGreen}{\scriptsize{\textbf{(+9.70)}}} & \textcolor{ForestGreen}{\scriptsize{\textbf{(+9.32)}}} 
      & \textcolor{red}{\scriptsize{\textbf{(-0.49)}}}         & \textcolor{red}{\scriptsize{\textbf{(-0.82)}}} 
      & \textcolor{ForestGreen}{\scriptsize{\textbf{(+2.24)}}} & \textcolor{ForestGreen}{\scriptsize{\textbf{(+1.41)}}} 
      & \textcolor{red}{\scriptsize{\textbf{(-2.55)}}}         & \textcolor{red}{\scriptsize{\textbf{(-2.88)}}} 
      & \textcolor{red}{\scriptsize{\textbf{(-2.82)}}}         & \textcolor{red}{\scriptsize{\textbf{(-2.41)}}} \\
     \hline
     \multicolumn{1}{c|}{\multirow{3}{*}{\textbf{TTMG (Ours)}}} & 77.85 & 70.53 & \textcolor{red}{\textbf{\underline{25.75}}} & \textcolor{red}{\textbf{\underline{20.10}}} & \textcolor{red}{\textbf{\underline{76.12}}} & \textcolor{red}{\textbf{\underline{68.77}}} & \textcolor{red}{\textbf{\underline{19.47}}} & \textcolor{red}{\textbf{\underline{13.68}}} & \textcolor{red}{\textbf{\underline{81.19}}} & \textcolor{red}{\textbf{\underline{73.66}}} & \textcolor{red}{\textbf{\underline{39.22}}} & \textcolor{blue}{\textbf{\textit{31.15}}} \\
      & \textcolor{red}{\scriptsize{\textbf{(-3.77)}}}          & \textcolor{red}{\scriptsize{\textbf{(-5.16)}}} 
      & \textcolor{ForestGreen}{\scriptsize{\textbf{(+10.94)}}} & \textcolor{ForestGreen}{\scriptsize{\textbf{(+11.10)}}} 
      & \textcolor{ForestGreen}{\scriptsize{\textbf{(+2.61)}}}  & \textcolor{ForestGreen}{\scriptsize{\textbf{(+2.20)}}} 
      & \textcolor{ForestGreen}{\scriptsize{\textbf{(+14.65)}}} & \textcolor{ForestGreen}{\scriptsize{\textbf{(+10.28)}}} 
      & \textcolor{ForestGreen}{\scriptsize{\textbf{(+0.57)}}}  & \textcolor{ForestGreen}{\scriptsize{\textbf{(+0.21)}}} 
      & \textcolor{ForestGreen}{\scriptsize{\textbf{(+7.44)}}}  & \textcolor{ForestGreen}{\scriptsize{\textbf{(+5.47)}}} \\\cline{2-13}
     & \textbf{-4.45} & \textbf{-5.19} & \textbf{+0.26}  & \textbf{+0.81} 
     & \textbf{+1.30} & \textbf{+1.33} & \textbf{+11.05} & \textbf{+7.97} 
     & \textbf{+0.56} & \textbf{+0.21} & \textbf{+1.63} & \textbf{-0.26} \\
     \hline
     \hline
    \multicolumn{1}{c|}{\multirow{3}{*}{Method}} & \multicolumn{4}{c|}{Training Modalities (D, U)} & \multicolumn{4}{c|}{Training Modalities (D, R)} & \multicolumn{4}{c}{Training Modalities (R, U)} \\\cline{2-13}
     & \multicolumn{2}{c|}{Seen (D, U)} & \multicolumn{2}{c|}{Unseen (C, R)} & \multicolumn{2}{c|}{Seen (D, R)} & \multicolumn{2}{c|}{Unseen (C, U)} & \multicolumn{2}{c|}{Seen (R, U)} & \multicolumn{2}{c}{Unseen (C, D)} \\\cline{2-13}
     & DSC & mIoU & DSC & mIoU & DSC & mIoU & DSC & mIoU & DSC & mIoU & DSC & mIoU \\
    \hline
    baseline \cite{chen2018encoder} & 84.60 & 76.56 & 16.97 & 10.70 & 75.42 & 67.39 & 18.76 & 12.25 & 70.77 & 61.59 & 3.92 & 2.33 \\
     \hline
     \multicolumn{1}{c|}{\multirow{2}{*}{IN \cite{ulyanov2017improved}}} & 85.70 & 77.85 & 21.95 & 14.77 & 57.06 & 49.18 & 20.66 & 13.26 & 70.49 & 60.96 & 19.01 & 15.83 \\
      & \textcolor{ForestGreen}{\scriptsize{\textbf{(+1.10)}}}  & \textcolor{ForestGreen}{\scriptsize{\textbf{(+1.29)}}} 
      & \textcolor{ForestGreen}{\scriptsize{\textbf{(+4.98)}}}  & \textcolor{ForestGreen}{\scriptsize{\textbf{(+4.07)}}} 
      & \textcolor{red}{\scriptsize{\textbf{(-18.36)}}}         & \textcolor{red}{\scriptsize{\textbf{(-18.21)}}} 
      & \textcolor{ForestGreen}{\scriptsize{\textbf{(+1.90)}}}  & \textcolor{ForestGreen}{\scriptsize{\textbf{(+1.01)}}} 
      & \textcolor{red}{\scriptsize{\textbf{(-0.28)}}}          & \textcolor{red}{\scriptsize{\textbf{(-0.63)}}} 
      & \textcolor{ForestGreen}{\scriptsize{\textbf{(+15.09)}}} & \textcolor{ForestGreen}{\scriptsize{\textbf{(+13.50)}}} \\
    \hline
     \multicolumn{1}{c|}{\multirow{2}{*}{IW \cite{huang2018decorrelated}}} & 81.15 & 72.05 & 14.09 & 8.84 & 62.49 & 56.17 & 22.83 & 15.33 & 56.05 & 48.01 & 17.64 & 13.43 \\
      & \textcolor{red}{\scriptsize{\textbf{(-3.45)}}}          & \textcolor{red}{\scriptsize{\textbf{(-4.51)}}} 
      & \textcolor{red}{\scriptsize{\textbf{(-2.88)}}}          & \textcolor{red}{\scriptsize{\textbf{(-1.86)}}} 
      & \textcolor{red}{\scriptsize{\textbf{(-12.93)}}}         &  \textcolor{red}{\scriptsize{\textbf{(-11.22)}}} 
      &  \textcolor{ForestGreen}{\scriptsize{\textbf{(+4.07)}}} &  \textcolor{ForestGreen}{\scriptsize{\textbf{(+3.08)}}} 
      & \textcolor{red}{\scriptsize{\textbf{(-14.72)}}}         & \textcolor{red}{\scriptsize{\textbf{(-13.58)}}} 
      & \textcolor{ForestGreen}{\scriptsize{\textbf{(+13.72)}}} & \textcolor{ForestGreen}{\scriptsize{\textbf{(+11.10)}}} \\
    \hline
     \multicolumn{1}{c|}{\multirow{2}{*}{IBN \cite{pan2018two}}} & \textcolor{red}{\textbf{\underline{86.38}}} & \textcolor{red}{\textbf{\underline{78.42}}} & \textcolor{blue}{\textbf{\textit{22.41}}} & 14.39 & \textcolor{blue}{\textbf{\textit{75.84}}} & \textcolor{blue}{\textbf{\textit{67.81}}} & 22.83 & 15.87 & \textcolor{blue}{\textbf{\textit{71.77}}} & \textcolor{blue}{\textbf{\textit{62.38}}} & 19.76 & 15.81 \\
      & \textcolor{ForestGreen}{\scriptsize{\textbf{(+1.78)}}}  & \textcolor{ForestGreen}{\scriptsize{\textbf{(+1.86)}}} 
      & \textcolor{ForestGreen}{\scriptsize{\textbf{(+5.44)}}}  & \textcolor{ForestGreen}{\scriptsize{\textbf{(+3.69)}}} 
      & \textcolor{ForestGreen}{\scriptsize{\textbf{(+0.42)}}}  & \textcolor{ForestGreen}{\scriptsize{\textbf{(+0.42)}}} 
      & \textcolor{ForestGreen}{\scriptsize{\textbf{(+4.07)}}}  & \textcolor{ForestGreen}{\scriptsize{\textbf{(+3.62)}}} 
      & \textcolor{ForestGreen}{\scriptsize{\textbf{(+1.00)}}}  & \textcolor{ForestGreen}{\scriptsize{\textbf{(+0.79)}}} 
      & \textcolor{ForestGreen}{\scriptsize{\textbf{(+15.84)}}} & \textcolor{ForestGreen}{\scriptsize{\textbf{(+13.48)}}} \\
    \hline
     \multicolumn{1}{c|}{\multirow{2}{*}{RobustNet \cite{choi2021robustnet}}} & 84.21 & 75.95 & 18.86 & 12.87 & 69.28 & 61.64 & 24.19 & 18.15 & 63.79 & 55.14 & 16.47 & 13.42 \\
      & \textcolor{red}{\scriptsize{\textbf{(-0.39)}}}          & \textcolor{red}{\scriptsize{\textbf{(-0.61)}}} 
      & \textcolor{ForestGreen}{\scriptsize{\textbf{(+1.89)}}}  & \textcolor{ForestGreen}{\scriptsize{\textbf{(+2.17)}}} 
      & \textcolor{red}{\scriptsize{\textbf{(-6.14)}}}          & \textcolor{red}{\scriptsize{\textbf{(-5.75)}}} 
      & \textcolor{ForestGreen}{\scriptsize{\textbf{(+5.43)}}}  & \textcolor{ForestGreen}{\scriptsize{\textbf{(+5.90)}}} 
      & \textcolor{red}{\scriptsize{\textbf{(-6.98)}}}          & \textcolor{red}{\scriptsize{\textbf{(-6.45)}}} 
      & \textcolor{ForestGreen}{\scriptsize{\textbf{(+12.55)}}} & \textcolor{ForestGreen}{\scriptsize{\textbf{(+11.09)}}} \\
    \hline
     \multicolumn{1}{c|}{\multirow{2}{*}{SAN-SAW \cite{peng2022semantic}}} & 81.65 & 72.24 & 19.36 & 12.06 & 67.62 & 60.41 & 23.20 & 15.60 & 66.64 & 57.97 & 5.33 & 3.83 \\
      & \textcolor{red}{\scriptsize{\textbf{(-2.95)}}}         & \textcolor{red}{\scriptsize{\textbf{(-4.32)}}} 
      & \textcolor{ForestGreen}{\scriptsize{\textbf{(+2.39)}}} & \textcolor{ForestGreen}{\scriptsize{\textbf{(+1.36)}}} 
      & \textcolor{red}{\scriptsize{\textbf{(-7.80)}}}         & \textcolor{red}{\scriptsize{\textbf{(-6.98)}}} 
      & \textcolor{ForestGreen}{\scriptsize{\textbf{(+4.44)}}} &  \textcolor{ForestGreen}{\scriptsize{\textbf{(+3.35)}}} 
      & \textcolor{red}{\scriptsize{\textbf{(-4.13)}}}         & \textcolor{red}{\scriptsize{\textbf{(-3.62)}}} 
      & \textcolor{ForestGreen}{\scriptsize{\textbf{(+1.41)}}} & \textcolor{ForestGreen}{\scriptsize{\textbf{(+1.50)}}} \\
    \hline
     \multicolumn{1}{c|}{\multirow{2}{*}{SPCNet \cite{huang2023style}}} & 85.99 & 78.23 & 21.67 & 14.71 & 75.75 & 67.59 & 20.34 & 13.66 & 70.72 & 61.27 & 16.69 & 13.15 \\
      & \textcolor{ForestGreen}{\scriptsize{\textbf{(+1.39)}}}  & \textcolor{ForestGreen}{\scriptsize{\textbf{(+1.67)}}} 
      & \textcolor{ForestGreen}{\scriptsize{\textbf{(+4.70)}}}  & \textcolor{ForestGreen}{\scriptsize{\textbf{(+4.01)}}} 
      & \textcolor{ForestGreen}{\scriptsize{\textbf{(+0.33)}}}  &  \textcolor{ForestGreen}{\scriptsize{\textbf{(+0.20)}}} 
      & \textcolor{ForestGreen}{\scriptsize{\textbf{(+1.58)}}}  &  \textcolor{ForestGreen}{\scriptsize{\textbf{(+1.41)}}} 
      & \textcolor{red}{\scriptsize{\textbf{(-0.05)}}}          & \textcolor{red}{\scriptsize{\textbf{(-0.32)}}} 
      & \textcolor{ForestGreen}{\scriptsize{\textbf{(+12.77)}}} & \textcolor{ForestGreen}{\scriptsize{\textbf{(+10.82)}}} \\
    \hline
     \multicolumn{1}{c|}{\multirow{2}{*}{BlindNet \cite{ahn2024style}}} & 84.86 & 77.02 & 21.30 & \textcolor{blue}{\textbf{\textit{15.28}}} & 71.85 & 63.93 & \textcolor{blue}{\textbf{\textit{25.96}}} & \textcolor{blue}{\textbf{\textit{18.73}}} & 66.07 & 57.35 & \textcolor{blue}{\textbf{\textit{20.12}}}  & \textcolor{blue}{\textbf{\textit{16.85}}} \\
      & \textcolor{ForestGreen}{\scriptsize{\textbf{(+0.26)}}}  & \textcolor{ForestGreen}{\scriptsize{\textbf{(+0.46)}}} 
      & \textcolor{ForestGreen}{\scriptsize{\textbf{(+4.33)}}}  & \textcolor{ForestGreen}{\scriptsize{\textbf{(+4.58)}}} 
      & \textcolor{red}{\scriptsize{\textbf{(-3.57)}}}          &  \textcolor{red}{\scriptsize{\textbf{(-3.46)}}} 
      & \textcolor{ForestGreen}{\scriptsize{\textbf{(+7.20)}}}  &  \textcolor{ForestGreen}{\scriptsize{\textbf{(+6.48)}}} 
      & \textcolor{red}{\scriptsize{\textbf{(-4.70)}}}          & \textcolor{red}{\scriptsize{\textbf{(-4.24)}}} 
      & \textcolor{ForestGreen}{\scriptsize{\textbf{(+16.20)}}} & \textcolor{ForestGreen}{\scriptsize{\textbf{(+14.52)}}} \\
     \hline
     \multicolumn{1}{c|}{\multirow{3}{*}{\textbf{TTMG (Ours)}}} & \textcolor{blue}{\textbf{\textit{86.26}}} & \textcolor{blue}{\textbf{\textit{78.39}}} & \textcolor{red}{\textbf{\underline{25.67}}} & \textcolor{red}{\textbf{\underline{17.90}}} & \textcolor{red}{\textbf{\underline{81.29}}} & \textcolor{red}{\textbf{\underline{73.34}}} & \textcolor{red}{\textbf{\underline{27.30}}} & \textcolor{red}{\textbf{\underline{18.98}}} &  \textcolor{red}{\textbf{\underline{73.62}}} &  \textcolor{red}{\textbf{\underline{65.22}}} &  \textcolor{red}{\textbf{\underline{20.96}}} &  \textcolor{red}{\textbf{\underline{16.98}}} \\
      & \textcolor{ForestGreen}{\scriptsize{\textbf{(+1.66)}}}  & \textcolor{ForestGreen}{\scriptsize{\textbf{(+1.83)}}} 
      & \textcolor{ForestGreen}{\scriptsize{\textbf{(+8.70)}}}  & \textcolor{ForestGreen}{\scriptsize{\textbf{(+7.20)}}} 
      & \textcolor{ForestGreen}{\scriptsize{\textbf{(+5.87)}}}  & \textcolor{ForestGreen}{\scriptsize{\textbf{(+5.95)}}} 
      & \textcolor{ForestGreen}{\scriptsize{\textbf{(+8.54)}}}  & \textcolor{ForestGreen}{\scriptsize{\textbf{(+6.73)}}} 
      & \textcolor{ForestGreen}{\scriptsize{\textbf{(+2.85)}}}  & \textcolor{ForestGreen}{\scriptsize{\textbf{(+3.63)}}} 
      & \textcolor{ForestGreen}{\scriptsize{\textbf{(+17.04)}}} & \textcolor{ForestGreen}{\scriptsize{\textbf{(+14.65)}}} \\\cline{2-13}
     & \textbf{-0.12} & \textbf{-0.03} & \textbf{+3.26} & \textbf{+2.62} 
     & \textbf{+5.45} & \textbf{+5.53} & \textbf{+1.34} & \textbf{+0.25} 
     & \textbf{+1.85} & \textbf{+2.84} & \textbf{+0.84} & \textbf{+0.13} \\
     \hline
    \end{tabular}
    \caption{Segmentation results on two modality training schemes (C, D), (C, R), (C, U), (D, U), (D, R), and (R, U).}
    \label{tab:comparison_two_modality}
\end{table*}

\subsection{Result Analysis}
\noindent \textbf{Quantitative Results.} We used the same model as that employed for the ’Seen’ modalities in Table \ref{tab:comparison_three_modality} and Table \ref{tab:comparison_two_modality} to evaluate the domain generalization performance for ’Unseen’ modalities for each table. We first trained each model on three modalities ((C, U, D), (C, U, R), (C, D, R), (U, D, R)) and then evaluated the performance on seen and unseen modality datasets. As shown in Table \ref{tab:comparison_three_modality}, TTMG has significantly improved the unseen modality generalization performance compared to the baseline. When compared to IN, IW, and IBN, TTMG exhibited DSC improvement of 7.63\%, 6.00\%, and 5.61\%, respectively, on average. Additionally, compared to SPCNet, which used the style projection approach based on prototype learning, TTMG demonstrated DSC improvement of 0.86\%, and 8.27\% on seen and unseen modality datasets on average. Furthermore, we trained each model on the more limited training scenario where the number of seen modalities was reduced to two ((C, D), (C, R), (C, U), (D, U), (D, R), (R, U)) and then evaluated them. As shown in Table \ref{tab:comparison_two_modality}, TTMG still outperforms various modalities settings even limited modality scenarios. Most notably, only TTMG and RobustNet improved generalization performance on the unseen modality dataset compared to the baseline in all scenarios. However, RobustNet shows severe performance degradation in all cases on the seen modality dataset. As shown in the Appendix (Table \ref{tab:efficiency_analysis}), TTMG performs without substantial computational cost and parameters. Additionally, TTMG still achieves higher performance in both seen and unseen modalities compared to existing DG methods even on different backbones (MobileNetV2 \cite{sandler2018mobilenetv2} and ShuffleNetV2 \cite{ma2018shufflenet}) and TransUNet \cite{chen2021transunet} in the Appendix (Table \ref{tab:backbone_type}). These results highlight that employing a prototype learning-based style-projection (MASP) with MSIW is versatile and crucial for efficiently generalizing unseen modalities in medical image segmentation.

\noindent \textbf{Qualitative Results.} Figure \ref{fig:QualitativeResults} illustrates qualitative results across different methods for radiology and ultrasound modalities. The baseline provides adequately seen modality datasets while producing highly noisy predictions on unseen modalities.  Although IN, IW, and IBN standardize feature map distributions between seen and unseen modalities, they still produce noisy predictions and demonstrate degraded performance on seen modalities. RobustNet, which removes domain-sensitive information through visual transformation, does not address modality-sensitive information, resulting in suboptimal performance on unseen modalities. Although SPCNet is similar to our approach (prototype learning-based style projection), semantic clustering in the decoder struggles to capture complex anatomical structures in medical images. In contrast, TTMG achieves robust predictions across all modalities, effectively handling noise, varying lesion sizes, and complex anatomical structures by leveraging the dual capabilities of MASP and MSIW. Due to the page limit, we also provide more qualitative results on the other various training scenarios in Appendix.

\noindent \textbf{Feature Visualization.} We employed T-SNE \cite{van2008visualizing} visualization in the (R, U) to (C, D) training scenario to illustrate how data achieves unseen modality generalization. Figure \ref{fig:FeatureProjection} shows the feature distributions across various modalities, where MASP effectively projects the feature representations around the style bases of seen modalities. This clustering effect compensates for the characteristic differences between modalities, facilitating more robust generalization to unseen modalities. Additionally, we observed that the baseline model exhibits high feature covariance, whereas TTMG selectively maintains lower feature covariance as shown in Figure \ref{fig:FeatureCovariance}. This effect aligns well with the objective of MSIW, effectively enhancing generalization.

\subsection{Ablation Study}

We want to clarify that the same settings that mentioned in Section \ref{s42_implementation_details} were applied across all ablation studies.

\begin{figure}[t]
    \centering
    \includegraphics[width=0.48\textwidth]{figure/FeatureProjection.png}
    \caption{T-SNE visualization \cite{van2008visualizing} of features for different modalities (a) before and (b) after test-time style projection from Stage1.}
    \label{fig:FeatureProjection}
\end{figure}

\begin{table}[t]
    \centering
    \footnotesize
    \setlength\tabcolsep{5.0pt} % default value: 6pt
    % \renewcommand{\arraystretch}{0.85} % Tighter
    \begin{tabular}{c|c|cc|cc}
    \hline
    \multicolumn{1}{c|}{\multirow{3}{*}{Setting}} & \multicolumn{1}{c|}{\multirow{3}{*}{Configuration}} & \multicolumn{4}{c}{Training Modalities (R, U)} \\ \cline{3-6}
     & &  \multicolumn{2}{c|}{Seen (R, U)} & \multicolumn{2}{c}{Unseen (C, D)} \\ \cline{3-6}
     & &  DSC & mIoU & DSC & mIoU \\
     \hline
     S0 & baseline & 70.77 & 61.59 & 3.92 & 2.33 \\
     \hline
     \multicolumn{1}{c|}{\multirow{2}{*}{S1}} &  \multicolumn{1}{c|}{\multirow{2}{*}{+ MASP \scriptsize{(w/o $\mathcal{L}_{\text{con}}$)}}} & 73.17 & 64.53 & 16.31 & 12.38 \\
      & & \textcolor{ForestGreen}{\scriptsize{\textbf{(+2.40)}}} & \textcolor{ForestGreen}{\scriptsize{\textbf{(+2.94)}}} & \textcolor{ForestGreen}{\scriptsize{\textbf{(+12.39)}}} & \textcolor{ForestGreen}{\scriptsize{\textbf{(+10.05)}}} \\
      \hline
     \multicolumn{1}{c|}{\multirow{2}{*}{S2}} &  \multicolumn{1}{c|}{\multirow{2}{*}{+ MASP \scriptsize{(w $\mathcal{L}_{\text{con}}$)}}} & \textcolor{red}{\textbf{\underline{73.87}}} & \textcolor{red}{\textbf{\underline{65.23}}} & 16.64 & 12.90 \\
      & & \textcolor{ForestGreen}{\scriptsize{\textbf{(+3.10)}}} & \textcolor{ForestGreen}{\scriptsize{\textbf{(+3.64)}}} & \textcolor{ForestGreen}{\scriptsize{\textbf{(+12.72)}}} & \textcolor{ForestGreen}{\scriptsize{\textbf{(+10.57)}}} \\
     \hline
     \multicolumn{1}{c|}{\multirow{2}{*}{S3}} &  \multicolumn{1}{c|}{\multirow{2}{*}{+ MSIW}} & 73.50 & 64.87 & \textcolor{blue}{\textbf{\textit{18.94}}} & \textcolor{blue}{\textbf{\textit{15.18}}} \\
      & & \textcolor{ForestGreen}{\scriptsize{\textbf{(+2.73)}}} & \textcolor{ForestGreen}{\scriptsize{\textbf{(+3.28)}}} & \textcolor{ForestGreen}{\scriptsize{\textbf{(+15.02)}}} & \textcolor{ForestGreen}{\scriptsize{\textbf{(+12.85)}}} \\
     \hline
     \multicolumn{1}{c|}{\multirow{3}{*}{S4}} & \multicolumn{1}{c|}{\multirow{3}{*}{\textbf{TTMG \scriptsize{(Ours)}}}} & \textcolor{blue}{\textbf{\textit{73.62}}} &  \textcolor{blue}{\textbf{\textit{65.22}}} &  \textcolor{red}{\textbf{\underline{20.96}}} &  \textcolor{red}{\textbf{\underline{16.98}}} \\
      & & \textcolor{ForestGreen}{\scriptsize{\textbf{(+2.85)}}} & \textcolor{ForestGreen}{\scriptsize{\textbf{(+3.63)}}} & \textcolor{ForestGreen}{\scriptsize{\textbf{(+17.04)}}} & \textcolor{ForestGreen}{\scriptsize{\textbf{(+14.66)}}} \\ \cline{3-6}
      & & \textbf{-0.25} & \textbf{-0.01} & \textbf{+2.02} & \textbf{+1.80} \\
     \hline
    \end{tabular}
    \caption{Ablation study of TTMG components (MASP and MSIW) on seen (R, U) and unseen (C, D) modalities. } 
    \label{tab:effectiveness_masp_msiw}
\end{table}

\noindent \textbf{Effectiveness of MASP and MSIW.} In this section, we conducted ablation studies on the (R, U) to (C, D) training scenario to demonstrate the effectiveness of MASP and MSIW, which are core components of the TTMG framework. As listed in Table \ref{tab:effectiveness_masp_msiw}, our approach (S4) exhibited the best performance on unseen modalities. The most notable result is that the application of MASP regardless of $\mathcal{L}_{\text{con}}$ provides the significant DSC improvement of 12.39\% and 12.72\% compared to baseline (S0). Compared to S1 and S2, applying $\mathcal{L}_{\text{con}}$ achieves performance improvement in both seen and unseen modalities because the content of the feature can be maintained after the projection. Additionally, applying MSIW alone (S3) resulted in a significant DSC and mIoU improvement of 15.02\% and 12.85\%, respectively, for unseen modalities, underscoring MSIW's role in reducing modality-sensitive information. Consequently, TTMG (S2 + S3), which employs the dual utilization of MASP and MSIW with $\mathcal{L}_{\text{con}}$, performs significantly better generalization ability on unseen modality datasets.

\begin{figure}[t]
    \centering
    \includegraphics[width=0.48\textwidth]{figure/Correlation_Matrix.png}
    \caption{Visualization of covariance matrix extracted from baseline and TTMG.}
    \label{fig:FeatureCovariance}
\end{figure}

\begin{table}[t]
    \centering
    \footnotesize
    \setlength\tabcolsep{7.0pt} % default value: 6pt
    % \renewcommand{\arraystretch}{0.85} % Tighter
    \begin{tabular}{c|cc|cc}
    \hline
    \multicolumn{1}{c|}{\multirow{3}{*}{Number of Style Bases $K_{m}$}} & \multicolumn{4}{c}{Training Modalities (R, U)} \\ \cline{2-5}
     & \multicolumn{2}{c|}{Seen (R, U)} & \multicolumn{2}{c}{Unseen (C, D)} \\ \cline{2-5}
     & DSC & mIoU & DSC & mIoU \\
     \hline
     $K_{m} = 4$ & 73.21 & 64.46 & \textcolor{blue}{\textbf{\textit{18.93}}} & \textcolor{blue}{\textbf{\textit{15.17}}} \\
     \textbf{$K_{m} = 8$ \scriptsize{(Ours)}} & \textcolor{blue}{\textbf{\textit{73.62}}} & \textcolor{blue}{\textbf{\textit{65.22}}} & \textcolor{red}{\textbf{\underline{20.96}}} & \textcolor{red}{\textbf{\underline{16.98}}} \\
     $K_{m} = 16$ & \textcolor{red}{\textbf{\underline{75.02}}} & \textcolor{red}{\textbf{\underline{66.34}}} & 18.04 & 14.27 \\
     $K_{m} = 32$ & 73.13 & 64.57 & 17.65 & 12.76 \\
     $K_{m} = 64$ & 72.21 & 63.43 & 14.77 & 70.72 \\
     \hline
    \end{tabular}
    \caption{Effect of the number of style bases \( K_{m} \) on TTMG’s generalization performance across seen (R, U) and unseen (C, D) modalities.}
    \label{tab:number_of_style_bases}
\end{table}

\noindent \textbf{Number of Style Bases.} In this section, we investigate the impact of the number of style bases $K_{m}$ on the generalization ability of TTMG in the (R, U) to (C, D) training scenario. As shown in Table \ref{tab:number_of_style_bases}, TTMG achieves the highest generalization performance on unseen modality datasets with $K_{m} = 8$, while also maintaining strong performance on seen modality datasets. With a lower number of style bases, such as $K_{m} = 4$, the limited bases fail to capture the complex style distribution of medical image datasets, resulting in lower performance. Conversely, as $K_{m}$ increases to 16, 32, and 64, performance on seen modalities improves; however, generalization on unseen modalities decreases, likely due to overfitting to the seen modality style bases. Based on these results, we set $K_{m} = 8$ for all experiments. \vspace{-0.25cm}
\begin{figure*}[tb!] 
    \centering
    \begin{subfigure}[b]{0.49\linewidth}
        \centering
        \includegraphics[width=\textwidth]{figures/exp_ScalingLaw_TinyStories_f1_score.pdf}
        \caption{Character choices prediction (TinyStories).}
    \end{subfigure}
    \hfill  % Creates horizontal spacing between subfigures
    \begin{subfigure}[b]{0.49\linewidth}
        \centering
        \includegraphics[width=\textwidth]{figures/exp_ScalingLaw_ultrachat_spearman.pdf}
        \caption{Response length prediction (Ultrachat).}
    \end{subfigure}
    \vspace{-5pt}
    \caption{Scaling effects on planning capabilities. Evaluated across four model families (LLaMA-2-chat, LLaMA-3-Instruct, Qwen-2-Instruct, Qwen-2.5-Instruct; 1.5B–72B) using UltraChat and TinyStories, structure and content attributes show family-specific scaling: larger models within each family improve planning.}
    \label{fig:exp_ablation_scaling}
    \vspace{-5pt}
\end{figure*}

\begin{figure*}[tb!] 
    \centering
    \begin{subfigure}[b]{0.48\linewidth}
        \centering
        \includegraphics[width=\textwidth]{figures/exp_TinyStories_stepwise_f1_dynamic_uniform.pdf}
        \caption{Character choice prediction (TinyStories).}
    \end{subfigure}
    \hfill  % Creates horizontal spacing between subfigures
    \begin{subfigure}[b]{0.48\linewidth}
        \centering
        \includegraphics[width=\textwidth]{figures/exp_medmcqa_stepwise_f1_dynamic_uniform.pdf}
        \caption{Answer confidence prediction (MedMCQA).}
    \end{subfigure}
    \vspace{-10pt}
    \caption{U-shaped planning dynamics during generation. Probing at equidistant positions (character choice, answer confidence) shows three-phase patterns: high accuracy in early segments (global planning intent), mid-segment decline (local token focus), and late-stage recovery (contextualized refinement). This suggests models first outline global attributes, then refine locally, before finalizing coherent plans.}
    \label{fig:exp_ablation_dynamics}
    \vspace{-10pt}
\end{figure*}

\begin{figure}[tb!]
    \centering
    \includegraphics[width=0.49\textwidth]{figures/exp_reflection_analysis_ultrachat_final.pdf}
    \vspace{-30pt}
    \caption{Implicit-explicit planning discrepancy. Models struggle to explicitly predict their own response lengths, with base models showing near-zero Spearman correlation and most fine-tuned models achieving only marginal gains. This gap implies limited introspective awareness despite underlying capability.}
    \label{fig:exp_ablation_selfAware_ultrachat}
    \vspace{-18pt}
\end{figure}

\section{Ablation}

\subsection{Planning Ability Scales with Model Size}
We analyze how emergent response planning scales across different model sizes using four model families: LLama-2-chat (7B, 13B, 70B), Llama-3-Instruct (8B, 70B), Qwen-2-Instruct (7B, 72B), and Qwen-2.5-Instruct (1.5B, 32B, 72B). Using grid search over layers and hidden sizes, we identify optimal configurations and evaluate models on UltraChat and TinyStories datasets, focusing on structure and content attributes.
We exclude base models as the relatively small models have short context which limit few-shot prompts, while the same prompts fail to effectively prompt larger base models to follow instructions. We omit the behavior attribute type as larger models tend to give correct answers consistently, making it difficult to obtain balanced data for analysis.

Results shown in Fig.~\ref{fig:exp_ablation_scaling} exhibit two key insights:  \textbf{(1)} within each model family, larger models demonstrate stronger planning capabilities, and \textbf{(2)} this scaling pattern does not generalize across different model families, suggesting that other factors like architectural differences also influence planning behavior.





\subsection{Evolution of Planning Representations During Response Generation}
We analyze how planning features evolve during generation by probing at different positions in the response sequence. For each response, we collect activations from the first token up to the token before attribute-revealing keywords (e.g., animal words in story character selection tasks) or throughout the entire sequence for tasks requiring external ground-truth labels (e.g., answer confidence tasks). We divide these positions into equal segments and apply  probes previously trained with in-dataset settings at each division point.
We conduct experiments on two datasets: TinyStories for character choice prediction and MedMCQA for answer confidence prediction. Results in Fig.~\ref{fig:exp_ablation_dynamics} reveal a distinctive pattern: \textbf{probing accuracy is high initially, decreases in the middle segments, and rises again toward the end}. This pattern suggests a three-phase planning process:
(1) initial phase with strong planning that provides an overview of the intended response;
(2) middle phase with weaker planning, characterized by more local, token-by-token generation;
(3) final phase with increased planning clarity as accumulated context makes the target attributes more apparent.



\subsection{Gap Between Probing and LLMs' Self-Predicted Results}

We investigate how the models' ability to predict their own response attributes compare to probe-based predictions with the UltraChat dataset on response length task. For fine-tuned models, we prompt: "Estimate your answer length in tokens using \texttt{[TOKENS]number[/TOKENS]}, then provide your answer." For base models, we provide few-shot examples with pre-calculated lengths. To evaluate self-prediction accuracy, we compare the predicted token count collected in a separated run against the length of the previous-collected model's greedy-decoded response.

As shown in Fig.~\ref{fig:exp_ablation_selfAware_ultrachat}, base models achieve near-zero Spearman correlation when predicting token lengths, even with examples. While fine-tuned models perform marginally better, there remains a substantial gap between models' direct predictions and probe-based predictions.
This gap suggests that \textbf{models encode more planning information in their hidden representations than they can explicitly access during token-by-token generation}, indicating a discrepancy between implicit planning capabilities and explicit self-awareness.

% While our results show emergent planning capabilities in LLMs, we also investigate whether models can explicitly predict attributes of their future responses. We test this using the response length prediction task on the UltraChat dataset, comparing direct model predictions against probe-based predictions.
% For fine-tuned models, we prompt: "Given this question, first estimate the number of tokens you would need for a complete answer using format as \texttt{[TOKENS]number[/TOKENS]}, then give your answer." For base models, we provide few-shot examples with pre-calculated token lengths.

% As shown in Fig.~\ref{fig:exp_ablation_selfAware_ultrachat}, we evaluate prediction accuracy using Spearman correlation. Base models struggle to predict their token lengths even with examples, while fine-tuned models perform better. However, there is an obvious gap between models' direct predictions with probe-based predictions, suggesting that probing better captures this capability.





% \subsection{Models across Different Architectures Show Similar Deep-in Planning Patterns}


% We investigate whether representations from one model can predict the behavior of another. Specifically, given a prompt $\mathbf{x}i$, can we use Model A's representations $\mathcal{H}_{i, A} = \{ \mathbf{H}^l{\mathbf{x}_{i, A}}\}^L_{l=1}$ to predict Model B's output label $\hat{g}_{i, B} = g(\mathbf{y}_{i, B})$? Success would suggest these representations capture intrinsic patterns beyond model-specific features.

% Results on the TinyStories dataset (Figure~\ref{fig:exp_cross_TinyStories}) compare models as predictors (horizontal axis) versus targets (vertical axis). We observe that: (1) diagonal terms (same predictor and target) show highest performance, indicating models best predict their own behavior; (2) strong non-diagonal performance suggests shared representations across different architectures, implying common underlying planning mechanisms.


% \begin{figure}[tb!]
%     \centering
%     % \vspace{-18pt}
%     \includegraphics[width=1.0\columnwidth]{figures/ablation_cross_model_correlations_f1_score_TinyStories_story_continuation.png}
%     \vspace{-4pt}
%     \caption{
%     Cross-model test for chat models on TinyStories dataset, with F1 score as metric.
%     }
%     \label{fig:exp_cross_TinyStories}
%     \vspace{-7pt}
% \end{figure} 

% \DZCtodo{First collect model A's activations on model B's responses, then use features of model A to predict label of model B's response;
% Results: 1) non-diagonal terms are much worse than diagonal terms, thus the planning features are model-specific, rather than a simple problem-label mapping; 2) some non-diagonal terms are non-zero, means different models show similar planning features to some degrees}


%% New Disucssion 
Our study reveals how heavy users integrate LLMs into their daily tasks through distinct patterns. Rather than simple tool usage, participants demonstrated sophisticated cognitive offloading strategies that transformed their decision-making processes. In our study, we observed participants delegating social and interpersonal reasoning to LLMs, suggesting ways users might leverage AI collaboration to support their social cognition processes.

Participants' mental models of LLMs directly influenced their cognitive strategies---those viewing LLMs as rational entities engaged in cognitive complementarity by leveraging LLM capabilities where they perceived personal limitations, while those viewing LLMs as average decision-makers used cognitive benchmarking, establishing baseline standards while reserving higher-order tasks for themselves.
% While delegating a broad range of decisions raised potential concerns about over-reliance and diminished critical thinking, our findings also highlight a nuanced form of human-AI collaboration where users and LLMs develop complementary relationships. Participants showed diverse usage strategies, treating LLMs as an emerging problem-solving tool and developing sophisticated prompting techniques. Most notably, participants frequently sought LLM guidance on social appropriateness and interpersonal situations. Although some users expressed concerns about potential skill degradation and a sense of unease, LLM consultations often led to a more thorough consideration of social factors and an enhanced understanding of different perspectives.

This raises questions for future research on redefining how we conceptualize and measure over-reliance on LLMs. Current metrics typically assess over-reliance through simplified quantitative measures in controlled settings, primarily focusing on users' acceptance rates of LLM outputs ~\cite{bo2024rely, kim2024rely}. However, our findings reveal more complex patterns of engagement. Participants did not blindly adopt LLM outputs, even in cases where they eventually accepted them. Instead, participants demonstrated thoughtful delegation strategies, using LLMs to validate existing decisions, automate routine tasks, or navigate unfamiliar situations. The critical concern was not users' acceptance of LLM outputs, but rather instances where users adopted LLM reasoning without exploring alternative perspectives. Future research should expand the definition of over-reliance beyond simple acceptance rates to examine how users critically engage with alternative lines of reasoning.

Another key direction for future research involves capturing diverse user contexts. Our participants valued the ability of LLMs to extract necessary contextual information when not initially provided. They appreciated that they could receive meaningful responses without extensively explaining background information, even for context-heavy topics like relationship advice. Future research should explore ways to incorporate multi-modal inputs beyond text-based interactions, allowing users to convey context through various channels. Additionally, LLMs' ability to elicit implicit user intentions without explicit prompting is crucial, as demonstrated by recent advances in reasoning-focused LLM architectures that can proactively identify and address underlying user needs.

The development of active usage patterns with LLMs appeared more prominent among younger users who had less experience managing tasks without these systems. Participants with extensive pre-LLM experience maintained clearer boundaries and showed greater awareness of system limitations. In contrast, users with less experience with LLMs demonstrated fewer reservations, viewing LLM interaction itself as a skill and actively developing their prompting strategies. Conducting design studies focused on younger generations, to better understand and support these emerging interaction patterns represents a crucial direction for future research.
\section{Discussion}\label{sec:discussion}



\subsection{From Interactive Prompting to Interactive Multi-modal Prompting}
The rapid advancements of large pre-trained generative models including large language models and text-to-image generation models, have inspired many HCI researchers to develop interactive tools to support users in crafting appropriate prompts.
% Studies on this topic in last two years' HCI conferences are predominantly focused on helping users refine single-modality textual prompts.
Many previous studies are focused on helping users refine single-modality textual prompts.
However, for many real-world applications concerning data beyond text modality, such as multi-modal AI and embodied intelligence, information from other modalities is essential in constructing sophisticated multi-modal prompts that fully convey users' instruction.
This demand inspires some researchers to develop multimodal prompting interactions to facilitate generation tasks ranging from visual modality image generation~\cite{wang2024promptcharm, promptpaint} to textual modality story generation~\cite{chung2022tale}.
% Some previous studies contributed relevant findings on this topic. 
Specifically, for the image generation task, recent studies have contributed some relevant findings on multi-modal prompting.
For example, PromptCharm~\cite{wang2024promptcharm} discovers the importance of multimodal feedback in refining initial text-based prompting in diffusion models.
However, the multi-modal interactions in PromptCharm are mainly focused on the feedback empowered the inpainting function, instead of supporting initial multimodal sketch-prompt control. 

\begin{figure*}[t]
    \centering
    \includegraphics[width=0.9\textwidth]{src/img/novice_expert.pdf}
    \vspace{-2mm}
    \caption{The comparison between novice and expert participants in painting reveals that experts produce more accurate and fine-grained sketches, resulting in closer alignment with reference images in close-ended tasks. Conversely, in open-ended tasks, expert fine-grained strokes fail to generate precise results due to \tool's lack of control at the thin stroke level.}
    \Description{The comparison between novice and expert participants in painting reveals that experts produce more accurate and fine-grained sketches, resulting in closer alignment with reference images in close-ended tasks. Novice users create rougher sketches with less accuracy in shape. Conversely, in open-ended tasks, expert fine-grained strokes fail to generate precise results due to \tool's lack of control at the thin stroke level, while novice users' broader strokes yield results more aligned with their sketches.}
    \label{fig:novice_expert}
    % \vspace{-3mm}
\end{figure*}


% In particular, in the initial control input, users are unable to explicitly specify multi-modal generation intents.
In another example, PromptPaint~\cite{promptpaint} stresses the importance of paint-medium-like interactions and introduces Prompt stencil functions that allow users to perform fine-grained controls with localized image generation. 
However, insufficient spatial control (\eg, PromptPaint only allows for single-object prompt stencil at a time) and unstable models can still leave some users feeling the uncertainty of AI and a varying degree of ownership of the generated artwork~\cite{promptpaint}.
% As a result, the gap between intuitive multi-modal or paint-medium-like control and the current prompting interface still exists, which requires further research on multi-modal prompting interactions.
From this perspective, our work seeks to further enhance multi-object spatial-semantic prompting control by users' natural sketching.
However, there are still some challenges to be resolved, such as consistent multi-object generation in multiple rounds to increase stability and improved understanding of user sketches.   


% \new{
% From this perspective, our work is a step forward in this direction by allowing multi-object spatial-semantic prompting control by users' natural sketching, which considers the interplay between multiple sketch regions.
% % To further advance the multi-modal prompting experience, there are some aspects we identify to be important.
% % One of the important aspects is enhancing the consistency and stability of multiple rounds of generation to reduce the uncertainty and loss of control on users' part.
% % For this purpose, we need to develop techniques to incorporate consistent generation~\cite{tewel2024training} into multi-modal prompting framework.}
% % Another important aspect is improving generative models' understanding of the implicit user intents \new{implied by the paint-medium-like or sketch-based input (\eg, sketch of two people with their hands slightly overlapping indicates holding hand without needing explicit prompt).
% % This can facilitate more natural control and alleviate users' effort in tuning the textual prompt.
% % In addition, it can increase users' sense of ownership as the generated results can be more aligned with their sketching intents.
% }
% For example, when users draw sketches of two people with their hands slightly overlapping, current region-based models cannot automatically infer users' implicit intention that the two people are holding hands.
% Instead, they still require users to explicitly specify in the prompt such relationship.
% \tool addresses this through sketch-aware prompt recommendation to fill in the necessary semantic information, alleviating users' workload.
% However, some users want the generative AI in the future to be able to directly infer this natural implicit intentions from the sketches without additional prompting since prompt recommendation can still be unstable sometimes.


% \new{
% Besides visual generation, 
% }
% For example, one of the important aspect is referring~\cite{he2024multi}, linking specific text semantics with specific spatial object, which is partly what we do in our sketch-aware prompt recommendation.
% Analogously, in natural communication between humans, text or audio alone often cannot suffice in expressing the speakers' intentions, and speakers often need to refer to an existing spatial object or draw out an illustration of her ideas for better explanation.
% Philosophically, we HCI researchers are mostly concerned about the human-end experience in human-AI communications.
% However, studies on prompting is unique in that we should not just care about the human-end interaction, but also make sure that AI can really get what the human means and produce intention-aligned output.
% Such consideration can drastically impact the design of prompting interactions in human-AI collaboration applications.
% On this note, although studies on multi-modal interactions is a well-established topic in HCI community, it remains a challenging problem what kind of multi-modal information is really effective in helping humans convey their ideas to current and next generation large AI models.




\subsection{Novice Performance vs. Expert Performance}\label{sec:nVe}
In this section we discuss the performance difference between novice and expert regarding experience in painting and prompting.
First, regarding painting skills, some participants with experience (4/12) preferred to draw accurate and fine-grained shapes at the beginning. 
All novice users (5/12) draw rough and less accurate shapes, while some participants with basic painting skills (3/12) also favored sketching rough areas of objects, as exemplified in Figure~\ref{fig:novice_expert}.
The experienced participants using fine-grained strokes (4/12, none of whom were experienced in prompting) achieved higher IoU scores (0.557) in the close-ended task (0.535) when using \tool. 
This is because their sketches were closer in shape and location to the reference, making the single object decomposition result more accurate.
Also, experienced participants are better at arranging spatial location and size of objects than novice participants.
However, some experienced participants (3/12) have mentioned that the fine-grained stroke sometimes makes them frustrated.
As P1's comment for his result in open-ended task: "\emph{It seems it cannot understand thin strokes; even if the shape is accurate, it can only generate content roughly around the area, especially when there is overlapping.}" 
This suggests that while \tool\ provides rough control to produce reasonably fine results from less accurate sketches for novice users, it may disappoint experienced users seeking more precise control through finer strokes. 
As shown in the last column in Figure~\ref{fig:novice_expert}, the dragon hovering in the sky was wrongly turned into a standing large dragon by \tool.

Second, regarding prompting skills, 3 out of 12 participants had one or more years of experience in T2I prompting. These participants used more modifiers than others during both T2I and R2I tasks.
Their performance in the T2I (0.335) and R2I (0.469) tasks showed higher scores than the average T2I (0.314) and R2I (0.418), but there was no performance improvement with \tool\ between their results (0.508) and the overall average score (0.528). 
This indicates that \tool\ can assist novice users in prompting, enabling them to produce satisfactory images similar to those created by users with prompting expertise.



\subsection{Applicability of \tool}
The feedback from user study highlighted several potential applications for our system. 
Three participants (P2, P6, P8) mentioned its possible use in commercial advertising design, emphasizing the importance of controllability for such work. 
They noted that the system's flexibility allows designers to quickly experiment with different settings.
Some participants (N = 3) also mentioned its potential for digital asset creation, particularly for game asset design. 
P7, a game mod developer, found the system highly useful for mod development. 
He explained: "\emph{Mods often require a series of images with a consistent theme and specific spatial requirements. 
For example, in a sacrifice scene, how the objects are arranged is closely tied to the mod's background. It would be difficult for a developer without professional skills, but with this system, it is possible to quickly construct such images}."
A few participants expressed similar thoughts regarding its use in scene construction, such as in film production. 
An interesting suggestion came from participant P4, who proposed its application in crime scene description. 
She pointed out that witnesses are often not skilled artists, and typically describe crime scenes verbally while someone else illustrates their account. 
With this system, witnesses could more easily express what they saw themselves, potentially producing depictions closer to the real events. "\emph{Details like object locations and distances from buildings can be easily conveyed using the system}," she added.

% \subsection{Model Understanding of Users' Implicit Intents}
% In region-sketch-based control of generative models, a significant gap between interaction design and actual implementation is the model's failure in understanding users' naturally expressed intentions.
% For example, when users draw sketches of two people with their hands slightly overlapping, current region-based models cannot automatically infer users' implicit intention that the two people are holding hands.
% Instead, they still require users to explicitly specify in the prompt such relationship.
% \tool addresses this through sketch-aware prompt recommendation to fill in the necessary semantic information, alleviating users' workload.
% However, some users want the generative AI in the future to be able to directly infer this natural implicit intentions from the sketches without additional prompting since prompt recommendation can still be unstable sometimes.
% This problem reflects a more general dilemma, which ubiquitously exists in all forms of conditioned control for generative models such as canny or scribble control.
% This is because all the control models are trained on pairs of explicit control signal and target image, which is lacking further interpretation or customization of the user intentions behind the seemingly straightforward input.
% For another example, the generative models cannot understand what abstraction level the user has in mind for her personal scribbles.
% Such problems leave more challenges to be addressed by future human-AI co-creation research.
% One possible direction is fine-tuning the conditioned models on individual user's conditioned control data to provide more customized interpretation. 

% \subsection{Balance between recommendation and autonomy}
% AIGC tools are a typical example of 
\subsection{Progressive Sketching}
Currently \tool is mainly aimed at novice users who are only capable of creating very rough sketches by themselves.
However, more accomplished painters or even professional artists typically have a coarse-to-fine creative process. 
Such a process is most evident in painting styles like traditional oil painting or digital impasto painting, where artists first quickly lay down large color patches to outline the most primitive proportion and structure of visual elements.
After that, the artists will progressively add layers of finer color strokes to the canvas to gradually refine the painting to an exquisite piece of artwork.
One participant in our user study (P1) , as a professional painter, has mentioned a similar point "\emph{
I think it is useful for laying out the big picture, give some inspirations for the initial drawing stage}."
Therefore, rough sketch also plays a part in the professional artists' creation process, yet it is more challenging to integrate AI into this more complex coarse-to-fine procedure.
Particularly, artists would like to preserve some of their finer strokes in later progression, not just the shape of the initial sketch.
In addition, instead of requiring the tool to generate a finished piece of artwork, some artists may prefer a model that can generate another more accurate sketch based on the initial one, and leave the final coloring and refining to the artists themselves.
To accommodate these diverse progressive sketching requirements, a more advanced sketch-based AI-assisted creation tool should be developed that can seamlessly enable artist intervention at any stage of the sketch and maximally preserve their creative intents to the finest level. 

\subsection{Ethical Issues}
Intellectual property and unethical misuse are two potential ethical concerns of AI-assisted creative tools, particularly those targeting novice users.
In terms of intellectual property, \tool hands over to novice users more control, giving them a higher sense of ownership of the creation.
However, the question still remains: how much contribution from the user's part constitutes full authorship of the artwork?
As \tool still relies on backbone generative models which may be trained on uncopyrighted data largely responsible for turning the sketch into finished artwork, we should design some mechanisms to circumvent this risk.
For example, we can allow artists to upload backbone models trained on their own artworks to integrate with our sketch control.
Regarding unethical misuse, \tool makes fine-grained spatial control more accessible to novice users, who may maliciously generate inappropriate content such as more realistic deepfake with specific postures they want or other explicit content.
To address this issue, we plan to incorporate a more sophisticated filtering mechanism that can detect and screen unethical content with more complex spatial-semantic conditions. 
% In the future, we plan to enable artists to upload their own style model

% \subsection{From interactive prompting to interactive spatial prompting}


\subsection{Limitations and Future work}

    \textbf{User Study Design}. Our open-ended task assesses the usability of \tool's system features in general use cases. To further examine aspects such as creativity and controllability across different methods, the open-ended task could be improved by incorporating baselines to provide more insightful comparative analysis. 
    Besides, in close-ended tasks, while the fixing order of tool usage prevents prior knowledge leakage, it might introduce learning effects. In our study, we include practice sessions for the three systems before the formal task to mitigate these effects. In the future, utilizing parallel tests (\textit{e.g.} different content with the same difficulty) or adding a control group could further reduce the learning effects.

    \textbf{Failure Cases}. There are certain failure cases with \tool that can limit its usability. 
    Firstly, when there are three or more objects with similar semantics, objects may still be missing despite prompt recommendations. 
    Secondly, if an object's stroke is thin, \tool may incorrectly interpret it as a full area, as demonstrated in the expert results of the open-ended task in Figure~\ref{fig:novice_expert}. 
    Finally, sometimes inclusion relationships (\textit{e.g.} inside) between objects cannot be generated correctly, partially due to biases in the base model that lack training samples with such relationship. 

    \textbf{More support for single object adjustment}.
    Participants (N=4) suggested that additional control features should be introduced, beyond just adjusting size and location. They noted that when objects overlap, they cannot freely control which object appears on top or which should be covered, and overlapping areas are currently not allowed.
    They proposed adding features such as layer control and depth control within the single-object mask manipulation. Currently, the system assigns layers based on color order, but future versions should allow users to adjust the layer of each object freely, while considering weighted prompts for overlapping areas.

    \textbf{More customized generation ability}.
    Our current system is built around a single model $ColorfulXL-Lightning$, which limits its ability to fully support the diverse creative needs of users. Feedback from participants has indicated a strong desire for more flexibility in style and personalization, such as integrating fine-tuned models that cater to specific artistic styles or individual preferences. 
    This limitation restricts the ability to adapt to varied creative intents across different users and contexts.
    In future iterations, we plan to address this by embedding a model selection feature, allowing users to choose from a variety of pre-trained or custom fine-tuned models that better align with their stylistic preferences. 
    
    \textbf{Integrate other model functions}.
    Our current system is compatible with many existing tools, such as Promptist~\cite{hao2024optimizing} and Magic Prompt, allowing users to iteratively generate prompts for single objects. However, the integration of these functions is somewhat limited in scope, and users may benefit from a broader range of interactive options, especially for more complex generation tasks. Additionally, for multimodal large models, users can currently explore using affordable or open-source models like Qwen2-VL~\cite{qwen} and InternVL2-Llama3~\cite{llama}, which have demonstrated solid inference performance in our tests. While GPT-4o remains a leading choice, alternative models also offer competitive results.
    Moving forward, we aim to integrate more multimodal large models into the system, giving users the flexibility to choose the models that best fit their needs. 
    


\section{Conclusion}\label{sec:conclusion}
In this paper, we present \tool, an interactive system designed to help novice users create high-quality, fine-grained images that align with their intentions based on rough sketches. 
The system first refines the user's initial prompt into a complete and coherent one that matches the rough sketch, ensuring the generated results are both stable, coherent and high quality.
To further support users in achieving fine-grained alignment between the generated image and their creative intent without requiring professional skills, we introduce a decompose-and-recompose strategy. 
This allows users to select desired, refined object shapes for individual decomposed objects and then recombine them, providing flexible mask manipulation for precise spatial control.
The framework operates through a coarse-to-fine process, enabling iterative and fine-grained control that is not possible with traditional end-to-end generation methods. 
Our user study demonstrates that \tool offers novice users enhanced flexibility in control and fine-grained alignment between their intentions and the generated images.



\section*{Impact Statement}
Our findings on LLM emergent planning raise specific considerations for model deployment. While these capabilities could enhance system reliability through better resource allocation and early warning mechanisms, they also present  concerns when handling sensitive data, as these probing methods reveal aspects of the model's internal thinking or decision-making process. We encourage careful evaluation of these trade-offs when implementing probing-based monitoring systems, particularly in applications involving sensitive information.


\section*{Acknowledgements}
% This work was supported in part by the National Key R\&D Program of China (NO.2022ZD0160201). 
We would like to thank Yuyu Fan, Jiachen Ma and anonymous reviewers for their valuable feedback and helpful discussions.

\iftoggle{blind}{
}
{
\section*{Author Contributions}
\textbf{Zhichen Dong} provided early inputs on the emergent response planning; proposed and ran the experimental tasks, and participated in writing all sections of the paper.

\textbf{Zhanhui Zhou} proposed the emergent response planning in discussion with \textbf{Zhichen Dong}; proposed experimental tasks; made substantial writing contributions to abstract, introduction, and Section 3.

\textbf{Zhixuan Liu} provided valuable feedback throughout the project; \textbf{Chao Yang} and \textbf{Chaochao Lu} supervised and managed the group.
}



\bibliography{
citation/hidden_representations, 
citation/prediction_or_planning,
citation/safety,
citation/others,
citation/datasets_and_models,
citation/general,
citation/prompt_engineer
}
\bibliographystyle{icml2025}


%%%%%%%%%%%%%%%%%%%%%%%%%%%%%%%%
% APPENDIX
\newpage
\appendix
\onecolumn
\section{Further Details on the Experimental Setup}
\subsection{Model Specification}
The following table lists the models and their corresponding links.
\label{appendix:model_specification_and_links}

\begin{longtable}{p{7cm}p{9cm}}
% \begin{longtable}{@{}p{7cm}p{9cm}@{}}
\toprule
\textbf{Models} & \textbf{Links} \\
\midrule
\texttt{Llama-2-7B}~\citep{touvron2023llama2} & \url{https://huggingface.co/meta-llama/Llama-2-7b-hf} \\
\texttt{Llama-2-7B-Chat}~\citep{touvron2023llama2} & \url{https://huggingface.co/meta-llama/Llama-2-7b-chat-hf} \\
\texttt{Llama-2-13B-Chat}~\citep{touvron2023llama2} & \url{https://huggingface.co/meta-llama/Llama-2-13b-chat-hf} \\
\texttt{Llama-2-70B-Chat}~\citep{touvron2023llama2} & \url{https://huggingface.co/meta-llama/Llama-2-70b-chat-hf} \\
\texttt{Llama-3-8B}~\citep{llama3modelcard} & \url{https://huggingface.co/meta-llama/Meta-Llama-3-8B} \\
\texttt{Llama-3-8B-Instruct}~\citep{llama3modelcard} & \url{https://huggingface.co/meta-llama/Llama-2-7b-hf} \\
\texttt{Llama-3-70B-Instruct}~\citep{llama3modelcard} & \url{https://huggingface.co/meta-llama/Meta-Llama-3-70B-Instruct} \\
\texttt{Mistral-7B}~\citep{jiang2023mistral7b} & \url{https://huggingface.co/mistralai/Mistral-7B-v0.1} \\
\texttt{Mistral-7B-Instruct}~\citep{jiang2023mistral7b} & \url{https://huggingface.co/mistralai/Mistral-7B-Instruct-v0.2} \\
\texttt{Qwen2-7B}~\citep{qwen2} & \url{https://huggingface.co/Qwen/Qwen2-7B} \\
\texttt{Qwen2-7B-Instruct}~\citep{qwen2} & \url{https://huggingface.co/Qwen/Qwen2-7B-Instruct} \\
\texttt{Qwen2-72B-Instruct}~\citep{qwen2} & \url{https://huggingface.co/Qwen/Qwen2-72B-Instruct} \\
\texttt{Qwen2.5-1.5B-Instruct}~\citep{qwen2.5} & \url{https://huggingface.co/Qwen/Qwen2.5-1.5B-Instruct} \\
\texttt{Qwen2.5-32B-Instruct}~\citep{qwen2.5} & \url{https://huggingface.co/Qwen/Qwen2.5-32B-Instruct} \\
\texttt{Qwen2.5-72B-Instruct}~\citep{qwen2.5} & \url{https://huggingface.co/Qwen/Qwen2.5-72B-Instruct} \\
\bottomrule
\end{longtable}

\subsection{Dataset Specification}
The following table lists the datasets and their corresponding links.
\begin{longtable}{@{}p{6cm}p{10cm}@{}}
\toprule
\textbf{Datasets} & \textbf{Links} \\
\midrule
\texttt{Ultrachat}~\citep{ding2023enhancing} & \url{https://huggingface.co/datasets/stingning/ultrachat} \\
\texttt{AlpacaEval}~\citep{alpaca} & \url{https://huggingface.co/datasets/tatsu-lab/alpaca} \\
\texttt{GSM8K}~\citep{cobbe2021gsm8k} & \url{https://huggingface.co/datasets/openai/gsm8k} \\
\texttt{MATH}~\citep{2019arXivMATH} & \url{https://huggingface.co/datasets/deepmind/math_dataset} \\
\texttt{TinyStories}~\citep{eldan2023tinystoriessmalllanguagemodels} & \url{https://huggingface.co/datasets/roneneldan/TinyStories} \\
\texttt{ROCStories}~\citep{mostafazadeh2016corpus} & \url{https://huggingface.co/datasets/Ximing/ROCStories} \\
\texttt{CommonsenseQA}~\citep{talmor-etal-2019-commonsenseqa} & \url{https://huggingface.co/datasets/tau/commonsense_qa} \\
\texttt{SocialIQA}~\citep{sap2019socialiqacommonsensereasoningsocial} & \url{https://huggingface.co/datasets/allenai/social_i_qa} \\
\texttt{MedMCQA}~\citep{pmlr-v174-pal22a} & \url{https://huggingface.co/datasets/openlifescienceai/medmcqa} \\
\texttt{ARC-Challenge}~\citep{allenai:arc} & \url{https://huggingface.co/datasets/allenai/ai2_arc} \\
\texttt{CREAK}~\citep{onoe2021creakdatasetcommonsensereasoning} & \url{https://huggingface.co/datasets/amydeng2000/CREAK} \\
\texttt{FEVER}~\citep{Thorne19FEVER2} & \url{https://huggingface.co/datasets/fever/fever} \\
\bottomrule
\end{longtable}


\subsection{Detailed Process of Response Collection and Labeling}
\label{appendix:setup_detailed_data_collection}
In this section, we detail the process of collecting a dataset $\mathcal{D} = \{\mathcal{H}_i, \hat{g}_i \}^{N}_{i=1}$ for each task $T = (p(\mathbf{x}), g(\mathbf{y}))$, pairing prompt representations with their corresponding attribute labels. First, we construct the prompt distribution $p(\mathbf{x})$ to elicit responses with target attributes from the models~(Sec.\ref{appendix:setup_prompts}). Second, we label these responses according to specific criteria $\hat{g}_i = g(\mathbf{y}_i)$ to capture their key attributes (Sec.\ref{appendix:setup_labeling}). Finally, we collect representations $\mathcal{H}_i = \{ \mathbf{H}^l_{\mathbf{x}_i}\}^L_{l=1}$ for each prompt (Sec.~\ref{appendix:setup_representation}).

\subsubsection{Prompt templates}
\label{appendix:setup_prompts}
To elicit responses with target attributes, we construct prompt distributions using carefully designed templates paired with datasets. We present the prompt templates for both chat and base models across all tasks, along with representative input-output examples.


% \texttt{Tokenizer.apply\_chat\_template}(\{\texttt{''data''}\}, \texttt{tokenize=False})\\

\begin{longtable}{p{14.5cm}}
\toprule
\rowcolor{blue!10}
\multicolumn{1}{c}{\texttt{Task 1: Response Length}} \\ 
\midrule

% Fine-tuned models section
\rowcolor{gray!10}
\texttt{Prompt for fine-tuned models} \\
\texttt{''} \\
\{\texttt{data}\} \\
\texttt{''} \\
($\rightarrow$ Gets formatted according to model's template) \\

% Example response section
\rowcolor{gray!10}
\texttt{Example Response} \\
\textbf{Data:} Why are oceans important to the global ecosystem? \\
\textbf{Output:} The oceans play a crucial role [...] \\ 
\midrule

% Base models section
\rowcolor{gray!10}
\texttt{Prompt for base models} \\
\texttt{''} \\
\texttt{Q: How can cross training benefit athletes?} \\
\texttt{A: Cross training offers various benefits [...] [END OF RESPONSE]} \\
\texttt{Q: What role does collaboration play in creativity?} \\
\texttt{A: Collaboration and originality complement each other [...] [END OF RESPONSE]} \\
\texttt{Q: \{data\}} \\
\texttt{A:} \\ 
\texttt{''} \\

% Example response section
\rowcolor{gray!10}
\texttt{Example Response} \\
\textbf{Data:} What are positive impacts of Reality TV? \\
\textbf{Output:} Reality TV provides entertainment and [...] [END OF RESPONSE] \\
\midrule



\rowcolor{blue!10}
\multicolumn{1}{c}{\texttt{Task 2: Reasoning Steps}} \\ \midrule

% Fine-tuned models section
\rowcolor{gray!10}
\texttt{Prompt for fine-tuned models} \\
\texttt{''} \\
\texttt{Provide step-by-step solution, starting with 'Step 1:'.} \\
\texttt{Problem:} \\
\texttt{\{data\}} \\
\texttt{''} \\
($\rightarrow$ Gets formatted according to model's template) \\
% Example response section
\rowcolor{gray!10}
\texttt{Example Response} \\
\textbf{Data:} Randy has 60 mango trees on his farm. He also has 5 less than half as many coconut trees as mango trees. How many trees does Randy have in all? \\
\textbf{Output:} Step 1: Write down the information [...] \\ 
\midrule
% Base models section
\rowcolor{gray!10}
\texttt{Prompt for base models} \\
\texttt{''} \\
\texttt{Solve this problem step-by-step, starting with 'Step 1:'.} \\
\texttt{Few-shot examples:} \\
\texttt{Problem: Let f(x)=\{ax+3 if x$>$2; x-5 if -2$\leq$x$\leq$2; 2x-b if x$<$-2\}. Find a+b if f is continuous.} \\
\texttt{Step 1: At x=2: a(2)+3=2-5 [...] [END OF RESPONSE]} \\
\texttt{Problem: If x=2 and y=5, find (x\^{}4+2y\^{}2)/6.} \\
\texttt{Step 1: Substitute: (2\^{}4+2(5\^{}2))/6 [...] [END OF RESPONSE]} \\
\texttt{Problem: \{data\}} \\\\
\texttt{''} \\
% Example response section
\rowcolor{gray!10}
\texttt{Example Response} \\
\textbf{Data:} Weng earns \$12 an hour for babysitting. Yesterday, she just did 50 minutes of babysitting. How much did she earn? \\
\textbf{Output:} Step 1: Substitute: 12(50/60) [...] [END OF RESPONSE] \\
\midrule


\rowcolor{blue!10}
\multicolumn{1}{c}{\texttt{Task 3: Character Choices}} \\ \midrule
% Fine-tuned models section
\rowcolor{gray!10}
\texttt{Prompt for fine-tuned models} \\
\texttt{''} \\
\texttt{Here's the first sentence of a story:} \texttt{\{data\}} \\
\texttt{Continue this story with one sentence that introduces a new animal character.} \\
\texttt{''} \\
($\rightarrow$ Gets formatted according to model's template) \\
% Example response section
\rowcolor{gray!10}
\texttt{Example Response} \\
\textbf{Data:} Once upon a time, there was a big car named Dependable. \\
\textbf{Output:} As Dependable was cruising down the highway, a chatty parrot [...] \\ 
\midrule
% Base models section
\rowcolor{gray!10}
\texttt{Prompt for base models} \\
\texttt{''} \\
\texttt{First sentence: Lily was a little mouse who liked to follow her big brother Leo.} \\
\texttt{Continuation: The garden was peaceful that morning until [...] [Animal: owl] [END OF RESPONSE]} \\
\texttt{First sentence: Lila and Ben were playing in the park with their toys.} \\
\texttt{Continuation: While building their epic sandcastle [...] [Animal: rabbit] [END OF RESPONSE]} \\
\texttt{First sentence: Sara was lonely.} \\
\texttt{Continuation: As she sat on the front steps drawing patterns [...] [Animal: puppy] [END OF RESPONSE]} \\
\texttt{First sentence: Lily and Ben were twins who liked to go on walks with their mom and dad.} \\
\texttt{Continuation: Their morning hike through the woods [...] [Animal: squirrel] [END OF RESPONSE]} \\
\texttt{First sentence: \{data\}} \\
\texttt{Continuation:} \\
\texttt{''} \\
% Example response section
\rowcolor{gray!10}
\texttt{Example Response} \\
\textbf{Data:} One day, a girl named Mia went for a walk. \\
\textbf{Output:} As she strolled through the park, she noticed a group of birds [...] [END OF RESPONSE] \\
\midrule


% Task 4
\rowcolor{blue!10}
\multicolumn{1}{c}{\texttt{Task 4: Multiple-Choice Answers}} \\ \midrule
% Fine-tuned models section
\rowcolor{gray!10}
\texttt{Prompt for fine-tuned models} \\
\texttt{''} \\
\texttt{Before choosing your answer, *briefly explain why in one short sentence*. Then select from the options:} \\
\texttt{\{data\}} \\
\texttt{''} \\
($\rightarrow$ Gets formatted according to model's template) \\
% Example response section
\rowcolor{gray!10}
\texttt{Example Response} \\
\textbf{Data:} Sammy wanted to go to where the people were. Where might he go? A) race track B) populated areas C) the desert D) apartment E) roadblock \\
\textbf{Output:} **He wants to be around people, so he would go to a populated area.**B) populated areas \\ 
\midrule
% Base models section
\rowcolor{gray!10}
\texttt{Prompt for base models} \\
\texttt{''} \\
\texttt{Select the correct answer. Choose the single best answer.} \\
\texttt{Q: The sanctions against the school were a punishing blow, and they seemed to what the efforts the school had made to change?} \\
\texttt{A)ignore B)enforce C)authoritarian D)yell at E)avoid} \\
\texttt{A: The sanctions ignore the school's reform efforts, contradicting their purpose. Therefore A.[END OF RESPONSE]} \\
\texttt{Q: Sammy wanted to go to where the people were. Where might he go?} \\
\texttt{A)race track B)populated areas C)the desert D)apartment E)roadblock} \\
\texttt{A: If Sammy wants to find people, he would logically go to populated areas where many people gather. Therefore B.[END OF RESPONSE]} \\
\texttt{Q: \{data\}} \\
\texttt{A:} \\
\texttt{''} \\
% Example response section
\rowcolor{gray!10}
\texttt{Example Response} \\
\textbf{Data:} Where do you put your grapes just before checking out? A) mouth B) grocery cart C) super market D) fruit basket E) fruit market \\
\textbf{Output:} The question asks where you put your grapes just before checking out. The answer is the grocery cart. Therefore B.[END OF RESPONSE] \\
\midrule


\rowcolor{blue!10}
\multicolumn{1}{c}{\texttt{Task 5: Answer Confidence}} \\ \midrule
% Fine-tuned models section
\rowcolor{gray!10}
\texttt{Prompt for fine-tuned models} \\
\texttt{''} \\
\texttt{Please first write analysis of the problem, then select the single correct answer to the following question from the options, and state your option explicitly:} \\
\texttt{\{data\}} \\
\texttt{''} \\
($\rightarrow$ Gets formatted according to model's template) \\
% Example response section
\rowcolor{gray!10}
\texttt{Example Response} \\
\textbf{Data:} Damage to median nerve produces - A) Claw hand B) Winging ofscapule C) Ape thumb D) Wrist drop \\
\textbf{Output:} Damage to the median nerve [...] The correct answer is: B) Winging of scapula \\ 
\midrule
% Base models section
\rowcolor{gray!10}
\texttt{Prompt for base models} \\
\texttt{''} \\
\texttt{Select the correct answer. Choose the single best answer.} \\
\texttt{Q: Which vitamin is supplied from only animal source:} \\
\texttt{A)Vitamin C B)Vitamin B7 C)Vitamin B12 D)Vitamin D} \\
\texttt{A: Vitamin B12 (cobalamin) is exclusively found in animal products as it is synthesized by bacteria in animals, and cannot be obtained from plant sources. Therefore C.[END OF RESPONSE]} \\
\texttt{Q: All of the following are surgical options for morbid obesity except -} \\
\texttt{A)Adjustable gastric banding B)Biliopancreatic diversion C)Duodenal Switch D)Roux en Y Duodenal By pass} \\
\texttt{A: While Roux-en-Y gastric bypass is a common bariatric procedure, there is no such thing as 'Roux en Y Duodenal Bypass'. The correct procedures include gastric banding, biliopancreatic diversion, and duodenal switch. Therefore D.[END OF RESPONSE]} \\
\texttt{Q: \{data\}} \\
\texttt{A:} \\
\texttt{''} \\
% Example response section
\rowcolor{gray!10}
\texttt{Example Response} \\
\textbf{Data:} Which of the following amino acid is active at neutral pH? A) Histidine B) Glycine C) Leucine D) Arginine \\
\textbf{Output:} Histidine is the only amino acid that is active at neutral pH. Therefore A.[END OF RESPONSE] \\
\midrule

\rowcolor{blue!10}
\multicolumn{1}{c}{\texttt{Task 6: Factual Consistency}} \\ \midrule
% Fine-tuned models section
\rowcolor{gray!10}
\texttt{Prompt for fine-tuned models} \\
\texttt{''} \\
\texttt{First discuss on this statement briefly in one short sentence, then explicitly state that you are in favour of the statement or explicitly state if you feel unsure instead:} \\
\texttt{\{data\}} \\
\texttt{''} \\
($\rightarrow$ Gets formatted according to model's template) \\
% Example response section
\rowcolor{gray!10}
\texttt{Example Response} \\
\textbf{Data:} American children's adventure film The Wizard of Oz was made by Warner Bros. \\
\textbf{Output:} The Wizard of Oz is a classic American children's adventure film [...] I am unsure about the statement as it appears to be incorrect. \\ 
\midrule
% Base models section
\rowcolor{gray!10}
\texttt{Prompt for base models} \\
\texttt{''} \\
\texttt{Q: Only people named Floyd wearing pink are allowed to attend Pink Floyd concerts.} \\
\texttt{A: This contradicts known concert admission practices and Pink Floyd's actual ticket policies. I cannot be confident about the accuracy of this statement.[END OF RESPONSE]} \\
\texttt{Q: Fax works without any internet connection.} \\
\texttt{A: Fax machines transmit data through telephone lines independently of internet infrastructure. I support this statement as accurate.[END OF RESPONSE]} \\
\texttt{Q: Larry King served tea during his show.} \\
\texttt{A: Larry King was known to have beverages during his long-running CNN show to maintain vocal comfort during interviews. I support this statement as accurate.[END OF RESPONSE]} \\
\texttt{Q: The band Lynyrd Skynyrd formed up in Beijing, China.} \\
\texttt{A: Available historical documentation about Lynyrd Skynyrd's origins contains conflicting or unclear information about their early formation. I cannot be confident about the accuracy of this statement.[END OF RESPONSE]} \\
\texttt{Q: \{data\}} \\
\texttt{A:} \\
\texttt{''} \\
% Example response section
\rowcolor{gray!10}
\texttt{Example Response} \\
\textbf{Data:} The crack in the Liberty Bell sets it apart from other famous bells. \\
\textbf{Output:} The Liberty Bell is a famous bell in the United States. I support this statement as accurate.[END OF RESPONSE] \\
\midrule



\end{longtable}


\subsubsection{Labeling}
\label{appendix:setup_labeling}

After collecting model responses, we first extract base model outputs using the \texttt{[END OF RESPONSE]} signal. Then, for both base and fine-tuned models, we label and filter responses using these criteria:
\begin{enumerate}
\item \textbf{Response length}: Calculate token count using the model's tokenizer, excluding special tokens. Exclude responses exceeding the 1000-token limit or those that are incomplete.
\item \textbf{Reasoning steps}: Count remaining steps by identifying step markers (e.g., ``Step 1:''). Exclude responses with more than 8 steps.
\item \textbf{Character choices}: Identify animal mentions in responses, excluding cases with no animals, multiple animals, or animals in the first two words. Select the top-4 most frequent animals per model and label them 0-3.
\item \textbf{Multiple-choice answers}: Extract answer selections (e.g., ``the answer is D'') using pattern matching. Exclude responses with zero or multiple answers, or answers at sentence start. Label options A-E as 0-4.
\item \textbf{Answer confidence}: Match the model's selected option against ground truth, excluding cases with multiple or no choices. Label correct answers as 1, incorrect as 0.
\item \textbf{Factual consistency}: Identify explicit agree/disagree statements and compare with ground truth, excluding cases without explicit agreement/disagreement. Label as 1 if the model agrees with true statements or disagrees with false ones, 0 otherwise.
\end{enumerate}

Then we perform data augmentation by: (1) removing responses shorter than 8 tokens and balancing class distributions across classification tasks while equalizing dataset sizes across models; (2) generating additional examples by randomly truncating responses several tokens before key information appears (e.g., end-of-response token, animal names in character choices, or option selections in multiple-choice answers), computing corresponding labels, and grouping original and augmented data to ensure they are assigned to the same data split (train/test/validation).



\subsubsection{Representation Collection}
\label{appendix:setup_representation}
For each truncated response, we concatenate the original LLM input with the truncated text and perform a forward pass to obtain representations from all layers at the truncation point. For answer-start representations, we directly use a forward pass on the original input. We then pair these collected representations with their corresponding labels to create the final dataset.






\section{Extended Experimental Results}
\subsection{Regression Fitting Performance}
We present complete regression fitting results for both in-dataset (Fig.~\ref{fig:exp_appendix_inDataset_regression}) and cross-dataset (Fig.~\ref{fig:exp_appendix_crossDataset_regression}) settings using hexbin density plots.

\begin{figure}[tb!]
\centering
\begin{subfigure}[b]{0.98\linewidth}
\centering
\includegraphics[width=0.98\textwidth]{figures/exp_basicResults_ultrachat_spearman.pdf}
\caption{Response length prediction on UltraChat dataset.}
\end{subfigure}
\
\begin{subfigure}[b]{0.98\linewidth}
\centering
\includegraphics[width=0.98\textwidth]{figures/exp_basicResults_gsm8k_spearman.pdf}
\caption{Reasoning steps prediction on GSM8K dataset.}
\end{subfigure}
\caption{\label{fig:exp_appendix_inDataset_regression}Hexbin plots showing in-dataset regression performance. Color intensity represents point density, with diagonal dashed lines indicating perfect predictions. The solid line in each subplot represents the linear regression fit applied to the predictions and the real labels.}
\vspace{-18pt}
\end{figure}

\begin{figure}[tb!]
\centering
\begin{subfigure}[b]{0.98\linewidth}
\centering
\includegraphics[width=0.98\textwidth]{figures/exp_generalizationResults_ultrachat_To_alpaca_eval_spearman.pdf}
\caption{UltraChat to AlpacaEval generalization for response length prediction.}
\end{subfigure}
\
\begin{subfigure}[b]{0.98\linewidth}
\centering
\includegraphics[width=0.98\textwidth]{figures/exp_generalizationResults_gsm8k_To_MATH_spearman.pdf}
\caption{GSM8K to MATH generalization for reasoning steps prediction.}
\end{subfigure}
\caption{\label{fig:exp_appendix_crossDataset_regression}Cross-dataset regression generalization visualized through hexbin plots. Color intensity represents point density, with diagonal dashed lines indicating perfect predictions. The solid line in each subplot represents the linear regression fit applied to the predictions and the real labels.}
\vspace{-18pt}
\end{figure}


\end{document}


% This document was modified from the file originally made available by
% Pat Langley and Andrea Danyluk for ICML-2K. This version was created
% by Iain Murray in 2018, and modified by Alexandre Bouchard in
% 2019 and 2021 and by Csaba Szepesvari, Gang Niu and Sivan Sabato in 2022.
% Modified again in 2023 and 2024 by Sivan Sabato and Jonathan Scarlett.
% Previous contributors include Dan Roy, Lise Getoor and Tobias
% Scheffer, which was slightly modified from the 2010 version by
% Thorsten Joachims & Johannes Fuernkranz, slightly modified from the
% 2009 version by Kiri Wagstaff and Sam Roweis's 2008 version, which is
% slightly modified from Prasad Tadepalli's 2007 version which is a
% lightly changed version of the previous year's version by Andrew
% Moore, which was in turn edited from those of Kristian Kersting and
% Codrina Lauth. Alex Smola contributed to the algorithmic style files.
