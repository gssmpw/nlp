\section{Discussion}
\subsection{Measuring Emergent Response Planning Under Sampling}
In this study, we focus on the greedy-decoded responses of LLMs, using greedy decoding to derive deterministic probe labels $\hat{g}_i = g(\mathbf{y}_i)$ for representations $\mathcal{H}_i = \{ \mathbf{H}^l_{\mathbf{x}_i}\}^L_{l=1}$.
But when generalizing to sampling settings, while greedy decoding simplifies sampling approximation by reflecting the LLM’s most probable output, this approach may not fully capture sampling nuances. We propose two potential ways for improvement:

\textbf{Averaging}: Replace greedy labels with attribute averages over multiple sampling trials (e.g., 10 samples) to approximate expected sampling behavior.

\textbf{Distributional probing}: Train probes to predict label distributions instead of single values, capturing uncertainty inherent to sampling. While greedy decoding reflects the LLM’s most probable output (approximating sampling averages), distribution-aware probing remains an open challenge, which we leave for future work.
% \begin{enumerate}
%     \item \textbf{}: 
%     \item \textbf{Distributional probing}: Train probes to predict label distributions instead of single values, capturing uncertainty inherent to sampling. While greedy decoding reflects the LLM’s most probable output (approximating sampling averages), distribution-aware probing remains an open challenge, which we leave for future work.
% \end{enumerate}

\subsection{Potential Applications of Emergent Response Planning in LLMs.}
Our findings on LLMs' emergent response planning suggest several practical applications. 
First, anticipating response length and complexity could enable dynamic resource allocation in LLM systems. Second, detecting low confidence or inaccuracies early during generation might trigger interventions like early stopping or retrieval-augmented refinement. Third, predictive capabilities could improve user interaction: predicting reasoning complexity might facilitate task decomposition in multi-step problems, while forecasting response characteristics could enable precise progress estimation during interactive tasks. These possibilities underscore the need to develop robust probing methods for deployed LLMs.
% For example, we could optimize resource allocation for LLM systems based on anticipated response length and complexity. During generation, early detection of low confidence or untruthfulness could enable proactive interventions like early stopping or retrieval augmentation.
% Also, these predictive capabilities could enhance user interaction—predicting reasoning complexity could guide task decomposition in multi-step problems, while anticipating response characteristics could provide accurate progress indicators in interactive settings.
% These applications highlight the value of developing robust real-time probing methods for deployed systems.

% Our findings on LLMs' ability to encode future response attributes in their hidden states suggest several practical applications. Systems could leverage these predictive signals to optimize resource allocation—for example, scaling computational resources based on anticipated response length and reasoning complexity. During generation, early detection of low confidence or potential untruthfulness could trigger interventions like early stopping or retrieval augmentation. In multi-step reasoning tasks, predicting reasoning complexity could help determine when to decompose problems into smaller steps. For interactive applications, anticipating response characteristics could help manage user expectations by providing accurate progress indicators or estimated completion times. These potential applications highlight the value of developing robust real-time probing methods for deployed LLM systems.



% \subsection{Broader Implications for LLM Understanding}
% The discovery of emergent planning in LLMs reveals an unexpected depth to these models' internal states. Despite being trained only on next-token prediction, LLMs develop representations that encode multiple aspects of their future outputs—from basic features like length to sophisticated attributes like truthfulness. This suggests that standard language modeling may naturally lead to the emergence of more complex representational capabilities than previously thought. 

\subsection{Future Research Directions}
Several key research directions emerge from our findings. Understanding the causal relationship between planning representations and generation outcomes stands as a primary challenge - while our probing results show these representations exist, determining whether and how they influence the generation process requires further investigation. Also, exploring whether similar planning phenomena emerge in multilingual or multimodal contexts could provide insights into how these capabilities develop across different domains and training objectives. Addressing these questions could require developing robust evaluation frameworks and more sophisticated probing techniques.