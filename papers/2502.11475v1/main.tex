% This must be in the first 5 lines to tell arXiv to use pdfLaTeX, which is strongly recommended.
%\pdfoutput=1
% In particular, the hyperref package requires pdfLaTeX in order to break URLs across lines.



\documentclass[11pt]{article}
%%%%% NEW MATH DEFINITIONS %%%%%

\usepackage{amsmath,amsfonts,bm}
\usepackage{derivative}
% Mark sections of captions for referring to divisions of figures
\newcommand{\figleft}{{\em (Left)}}
\newcommand{\figcenter}{{\em (Center)}}
\newcommand{\figright}{{\em (Right)}}
\newcommand{\figtop}{{\em (Top)}}
\newcommand{\figbottom}{{\em (Bottom)}}
\newcommand{\captiona}{{\em (a)}}
\newcommand{\captionb}{{\em (b)}}
\newcommand{\captionc}{{\em (c)}}
\newcommand{\captiond}{{\em (d)}}

% Highlight a newly defined term
\newcommand{\newterm}[1]{{\bf #1}}

% Derivative d 
\newcommand{\deriv}{{\mathrm{d}}}

% Figure reference, lower-case.
\def\figref#1{figure~\ref{#1}}
% Figure reference, capital. For start of sentence
\def\Figref#1{Figure~\ref{#1}}
\def\twofigref#1#2{figures \ref{#1} and \ref{#2}}
\def\quadfigref#1#2#3#4{figures \ref{#1}, \ref{#2}, \ref{#3} and \ref{#4}}
% Section reference, lower-case.
\def\secref#1{section~\ref{#1}}
% Section reference, capital.
\def\Secref#1{Section~\ref{#1}}
% Reference to two sections.
\def\twosecrefs#1#2{sections \ref{#1} and \ref{#2}}
% Reference to three sections.
\def\secrefs#1#2#3{sections \ref{#1}, \ref{#2} and \ref{#3}}
% Reference to an equation, lower-case.
\def\eqref#1{equation~\ref{#1}}
% Reference to an equation, upper case
\def\Eqref#1{Equation~\ref{#1}}
% A raw reference to an equation---avoid using if possible
\def\plaineqref#1{\ref{#1}}
% Reference to a chapter, lower-case.
\def\chapref#1{chapter~\ref{#1}}
% Reference to an equation, upper case.
\def\Chapref#1{Chapter~\ref{#1}}
% Reference to a range of chapters
\def\rangechapref#1#2{chapters\ref{#1}--\ref{#2}}
% Reference to an algorithm, lower-case.
\def\algref#1{algorithm~\ref{#1}}
% Reference to an algorithm, upper case.
\def\Algref#1{Algorithm~\ref{#1}}
\def\twoalgref#1#2{algorithms \ref{#1} and \ref{#2}}
\def\Twoalgref#1#2{Algorithms \ref{#1} and \ref{#2}}
% Reference to a part, lower case
\def\partref#1{part~\ref{#1}}
% Reference to a part, upper case
\def\Partref#1{Part~\ref{#1}}
\def\twopartref#1#2{parts \ref{#1} and \ref{#2}}

\def\ceil#1{\lceil #1 \rceil}
\def\floor#1{\lfloor #1 \rfloor}
\def\1{\bm{1}}
\newcommand{\train}{\mathcal{D}}
\newcommand{\valid}{\mathcal{D_{\mathrm{valid}}}}
\newcommand{\test}{\mathcal{D_{\mathrm{test}}}}

\def\eps{{\epsilon}}


% Random variables
\def\reta{{\textnormal{$\eta$}}}
\def\ra{{\textnormal{a}}}
\def\rb{{\textnormal{b}}}
\def\rc{{\textnormal{c}}}
\def\rd{{\textnormal{d}}}
\def\re{{\textnormal{e}}}
\def\rf{{\textnormal{f}}}
\def\rg{{\textnormal{g}}}
\def\rh{{\textnormal{h}}}
\def\ri{{\textnormal{i}}}
\def\rj{{\textnormal{j}}}
\def\rk{{\textnormal{k}}}
\def\rl{{\textnormal{l}}}
% rm is already a command, just don't name any random variables m
\def\rn{{\textnormal{n}}}
\def\ro{{\textnormal{o}}}
\def\rp{{\textnormal{p}}}
\def\rq{{\textnormal{q}}}
\def\rr{{\textnormal{r}}}
\def\rs{{\textnormal{s}}}
\def\rt{{\textnormal{t}}}
\def\ru{{\textnormal{u}}}
\def\rv{{\textnormal{v}}}
\def\rw{{\textnormal{w}}}
\def\rx{{\textnormal{x}}}
\def\ry{{\textnormal{y}}}
\def\rz{{\textnormal{z}}}

% Random vectors
\def\rvepsilon{{\mathbf{\epsilon}}}
\def\rvphi{{\mathbf{\phi}}}
\def\rvtheta{{\mathbf{\theta}}}
\def\rva{{\mathbf{a}}}
\def\rvb{{\mathbf{b}}}
\def\rvc{{\mathbf{c}}}
\def\rvd{{\mathbf{d}}}
\def\rve{{\mathbf{e}}}
\def\rvf{{\mathbf{f}}}
\def\rvg{{\mathbf{g}}}
\def\rvh{{\mathbf{h}}}
\def\rvu{{\mathbf{i}}}
\def\rvj{{\mathbf{j}}}
\def\rvk{{\mathbf{k}}}
\def\rvl{{\mathbf{l}}}
\def\rvm{{\mathbf{m}}}
\def\rvn{{\mathbf{n}}}
\def\rvo{{\mathbf{o}}}
\def\rvp{{\mathbf{p}}}
\def\rvq{{\mathbf{q}}}
\def\rvr{{\mathbf{r}}}
\def\rvs{{\mathbf{s}}}
\def\rvt{{\mathbf{t}}}
\def\rvu{{\mathbf{u}}}
\def\rvv{{\mathbf{v}}}
\def\rvw{{\mathbf{w}}}
\def\rvx{{\mathbf{x}}}
\def\rvy{{\mathbf{y}}}
\def\rvz{{\mathbf{z}}}

% Elements of random vectors
\def\erva{{\textnormal{a}}}
\def\ervb{{\textnormal{b}}}
\def\ervc{{\textnormal{c}}}
\def\ervd{{\textnormal{d}}}
\def\erve{{\textnormal{e}}}
\def\ervf{{\textnormal{f}}}
\def\ervg{{\textnormal{g}}}
\def\ervh{{\textnormal{h}}}
\def\ervi{{\textnormal{i}}}
\def\ervj{{\textnormal{j}}}
\def\ervk{{\textnormal{k}}}
\def\ervl{{\textnormal{l}}}
\def\ervm{{\textnormal{m}}}
\def\ervn{{\textnormal{n}}}
\def\ervo{{\textnormal{o}}}
\def\ervp{{\textnormal{p}}}
\def\ervq{{\textnormal{q}}}
\def\ervr{{\textnormal{r}}}
\def\ervs{{\textnormal{s}}}
\def\ervt{{\textnormal{t}}}
\def\ervu{{\textnormal{u}}}
\def\ervv{{\textnormal{v}}}
\def\ervw{{\textnormal{w}}}
\def\ervx{{\textnormal{x}}}
\def\ervy{{\textnormal{y}}}
\def\ervz{{\textnormal{z}}}

% Random matrices
\def\rmA{{\mathbf{A}}}
\def\rmB{{\mathbf{B}}}
\def\rmC{{\mathbf{C}}}
\def\rmD{{\mathbf{D}}}
\def\rmE{{\mathbf{E}}}
\def\rmF{{\mathbf{F}}}
\def\rmG{{\mathbf{G}}}
\def\rmH{{\mathbf{H}}}
\def\rmI{{\mathbf{I}}}
\def\rmJ{{\mathbf{J}}}
\def\rmK{{\mathbf{K}}}
\def\rmL{{\mathbf{L}}}
\def\rmM{{\mathbf{M}}}
\def\rmN{{\mathbf{N}}}
\def\rmO{{\mathbf{O}}}
\def\rmP{{\mathbf{P}}}
\def\rmQ{{\mathbf{Q}}}
\def\rmR{{\mathbf{R}}}
\def\rmS{{\mathbf{S}}}
\def\rmT{{\mathbf{T}}}
\def\rmU{{\mathbf{U}}}
\def\rmV{{\mathbf{V}}}
\def\rmW{{\mathbf{W}}}
\def\rmX{{\mathbf{X}}}
\def\rmY{{\mathbf{Y}}}
\def\rmZ{{\mathbf{Z}}}

% Elements of random matrices
\def\ermA{{\textnormal{A}}}
\def\ermB{{\textnormal{B}}}
\def\ermC{{\textnormal{C}}}
\def\ermD{{\textnormal{D}}}
\def\ermE{{\textnormal{E}}}
\def\ermF{{\textnormal{F}}}
\def\ermG{{\textnormal{G}}}
\def\ermH{{\textnormal{H}}}
\def\ermI{{\textnormal{I}}}
\def\ermJ{{\textnormal{J}}}
\def\ermK{{\textnormal{K}}}
\def\ermL{{\textnormal{L}}}
\def\ermM{{\textnormal{M}}}
\def\ermN{{\textnormal{N}}}
\def\ermO{{\textnormal{O}}}
\def\ermP{{\textnormal{P}}}
\def\ermQ{{\textnormal{Q}}}
\def\ermR{{\textnormal{R}}}
\def\ermS{{\textnormal{S}}}
\def\ermT{{\textnormal{T}}}
\def\ermU{{\textnormal{U}}}
\def\ermV{{\textnormal{V}}}
\def\ermW{{\textnormal{W}}}
\def\ermX{{\textnormal{X}}}
\def\ermY{{\textnormal{Y}}}
\def\ermZ{{\textnormal{Z}}}

% Vectors
\def\vzero{{\bm{0}}}
\def\vone{{\bm{1}}}
\def\vmu{{\bm{\mu}}}
\def\vtheta{{\bm{\theta}}}
\def\vphi{{\bm{\phi}}}
\def\va{{\bm{a}}}
\def\vb{{\bm{b}}}
\def\vc{{\bm{c}}}
\def\vd{{\bm{d}}}
\def\ve{{\bm{e}}}
\def\vf{{\bm{f}}}
\def\vg{{\bm{g}}}
\def\vh{{\bm{h}}}
\def\vi{{\bm{i}}}
\def\vj{{\bm{j}}}
\def\vk{{\bm{k}}}
\def\vl{{\bm{l}}}
\def\vm{{\bm{m}}}
\def\vn{{\bm{n}}}
\def\vo{{\bm{o}}}
\def\vp{{\bm{p}}}
\def\vq{{\bm{q}}}
\def\vr{{\bm{r}}}
\def\vs{{\bm{s}}}
\def\vt{{\bm{t}}}
\def\vu{{\bm{u}}}
\def\vv{{\bm{v}}}
\def\vw{{\bm{w}}}
\def\vx{{\bm{x}}}
\def\vy{{\bm{y}}}
\def\vz{{\bm{z}}}

% Elements of vectors
\def\evalpha{{\alpha}}
\def\evbeta{{\beta}}
\def\evepsilon{{\epsilon}}
\def\evlambda{{\lambda}}
\def\evomega{{\omega}}
\def\evmu{{\mu}}
\def\evpsi{{\psi}}
\def\evsigma{{\sigma}}
\def\evtheta{{\theta}}
\def\eva{{a}}
\def\evb{{b}}
\def\evc{{c}}
\def\evd{{d}}
\def\eve{{e}}
\def\evf{{f}}
\def\evg{{g}}
\def\evh{{h}}
\def\evi{{i}}
\def\evj{{j}}
\def\evk{{k}}
\def\evl{{l}}
\def\evm{{m}}
\def\evn{{n}}
\def\evo{{o}}
\def\evp{{p}}
\def\evq{{q}}
\def\evr{{r}}
\def\evs{{s}}
\def\evt{{t}}
\def\evu{{u}}
\def\evv{{v}}
\def\evw{{w}}
\def\evx{{x}}
\def\evy{{y}}
\def\evz{{z}}

% Matrix
\def\mA{{\bm{A}}}
\def\mB{{\bm{B}}}
\def\mC{{\bm{C}}}
\def\mD{{\bm{D}}}
\def\mE{{\bm{E}}}
\def\mF{{\bm{F}}}
\def\mG{{\bm{G}}}
\def\mH{{\bm{H}}}
\def\mI{{\bm{I}}}
\def\mJ{{\bm{J}}}
\def\mK{{\bm{K}}}
\def\mL{{\bm{L}}}
\def\mM{{\bm{M}}}
\def\mN{{\bm{N}}}
\def\mO{{\bm{O}}}
\def\mP{{\bm{P}}}
\def\mQ{{\bm{Q}}}
\def\mR{{\bm{R}}}
\def\mS{{\bm{S}}}
\def\mT{{\bm{T}}}
\def\mU{{\bm{U}}}
\def\mV{{\bm{V}}}
\def\mW{{\bm{W}}}
\def\mX{{\bm{X}}}
\def\mY{{\bm{Y}}}
\def\mZ{{\bm{Z}}}
\def\mBeta{{\bm{\beta}}}
\def\mPhi{{\bm{\Phi}}}
\def\mLambda{{\bm{\Lambda}}}
\def\mSigma{{\bm{\Sigma}}}

% Tensor
\DeclareMathAlphabet{\mathsfit}{\encodingdefault}{\sfdefault}{m}{sl}
\SetMathAlphabet{\mathsfit}{bold}{\encodingdefault}{\sfdefault}{bx}{n}
\newcommand{\tens}[1]{\bm{\mathsfit{#1}}}
\def\tA{{\tens{A}}}
\def\tB{{\tens{B}}}
\def\tC{{\tens{C}}}
\def\tD{{\tens{D}}}
\def\tE{{\tens{E}}}
\def\tF{{\tens{F}}}
\def\tG{{\tens{G}}}
\def\tH{{\tens{H}}}
\def\tI{{\tens{I}}}
\def\tJ{{\tens{J}}}
\def\tK{{\tens{K}}}
\def\tL{{\tens{L}}}
\def\tM{{\tens{M}}}
\def\tN{{\tens{N}}}
\def\tO{{\tens{O}}}
\def\tP{{\tens{P}}}
\def\tQ{{\tens{Q}}}
\def\tR{{\tens{R}}}
\def\tS{{\tens{S}}}
\def\tT{{\tens{T}}}
\def\tU{{\tens{U}}}
\def\tV{{\tens{V}}}
\def\tW{{\tens{W}}}
\def\tX{{\tens{X}}}
\def\tY{{\tens{Y}}}
\def\tZ{{\tens{Z}}}


% Graph
\def\gA{{\mathcal{A}}}
\def\gB{{\mathcal{B}}}
\def\gC{{\mathcal{C}}}
\def\gD{{\mathcal{D}}}
\def\gE{{\mathcal{E}}}
\def\gF{{\mathcal{F}}}
\def\gG{{\mathcal{G}}}
\def\gH{{\mathcal{H}}}
\def\gI{{\mathcal{I}}}
\def\gJ{{\mathcal{J}}}
\def\gK{{\mathcal{K}}}
\def\gL{{\mathcal{L}}}
\def\gM{{\mathcal{M}}}
\def\gN{{\mathcal{N}}}
\def\gO{{\mathcal{O}}}
\def\gP{{\mathcal{P}}}
\def\gQ{{\mathcal{Q}}}
\def\gR{{\mathcal{R}}}
\def\gS{{\mathcal{S}}}
\def\gT{{\mathcal{T}}}
\def\gU{{\mathcal{U}}}
\def\gV{{\mathcal{V}}}
\def\gW{{\mathcal{W}}}
\def\gX{{\mathcal{X}}}
\def\gY{{\mathcal{Y}}}
\def\gZ{{\mathcal{Z}}}

% Sets
\def\sA{{\mathbb{A}}}
\def\sB{{\mathbb{B}}}
\def\sC{{\mathbb{C}}}
\def\sD{{\mathbb{D}}}
% Don't use a set called E, because this would be the same as our symbol
% for expectation.
\def\sF{{\mathbb{F}}}
\def\sG{{\mathbb{G}}}
\def\sH{{\mathbb{H}}}
\def\sI{{\mathbb{I}}}
\def\sJ{{\mathbb{J}}}
\def\sK{{\mathbb{K}}}
\def\sL{{\mathbb{L}}}
\def\sM{{\mathbb{M}}}
\def\sN{{\mathbb{N}}}
\def\sO{{\mathbb{O}}}
\def\sP{{\mathbb{P}}}
\def\sQ{{\mathbb{Q}}}
\def\sR{{\mathbb{R}}}
\def\sS{{\mathbb{S}}}
\def\sT{{\mathbb{T}}}
\def\sU{{\mathbb{U}}}
\def\sV{{\mathbb{V}}}
\def\sW{{\mathbb{W}}}
\def\sX{{\mathbb{X}}}
\def\sY{{\mathbb{Y}}}
\def\sZ{{\mathbb{Z}}}

% Entries of a matrix
\def\emLambda{{\Lambda}}
\def\emA{{A}}
\def\emB{{B}}
\def\emC{{C}}
\def\emD{{D}}
\def\emE{{E}}
\def\emF{{F}}
\def\emG{{G}}
\def\emH{{H}}
\def\emI{{I}}
\def\emJ{{J}}
\def\emK{{K}}
\def\emL{{L}}
\def\emM{{M}}
\def\emN{{N}}
\def\emO{{O}}
\def\emP{{P}}
\def\emQ{{Q}}
\def\emR{{R}}
\def\emS{{S}}
\def\emT{{T}}
\def\emU{{U}}
\def\emV{{V}}
\def\emW{{W}}
\def\emX{{X}}
\def\emY{{Y}}
\def\emZ{{Z}}
\def\emSigma{{\Sigma}}

% entries of a tensor
% Same font as tensor, without \bm wrapper
\newcommand{\etens}[1]{\mathsfit{#1}}
\def\etLambda{{\etens{\Lambda}}}
\def\etA{{\etens{A}}}
\def\etB{{\etens{B}}}
\def\etC{{\etens{C}}}
\def\etD{{\etens{D}}}
\def\etE{{\etens{E}}}
\def\etF{{\etens{F}}}
\def\etG{{\etens{G}}}
\def\etH{{\etens{H}}}
\def\etI{{\etens{I}}}
\def\etJ{{\etens{J}}}
\def\etK{{\etens{K}}}
\def\etL{{\etens{L}}}
\def\etM{{\etens{M}}}
\def\etN{{\etens{N}}}
\def\etO{{\etens{O}}}
\def\etP{{\etens{P}}}
\def\etQ{{\etens{Q}}}
\def\etR{{\etens{R}}}
\def\etS{{\etens{S}}}
\def\etT{{\etens{T}}}
\def\etU{{\etens{U}}}
\def\etV{{\etens{V}}}
\def\etW{{\etens{W}}}
\def\etX{{\etens{X}}}
\def\etY{{\etens{Y}}}
\def\etZ{{\etens{Z}}}

% The true underlying data generating distribution
\newcommand{\pdata}{p_{\rm{data}}}
\newcommand{\ptarget}{p_{\rm{target}}}
\newcommand{\pprior}{p_{\rm{prior}}}
\newcommand{\pbase}{p_{\rm{base}}}
\newcommand{\pref}{p_{\rm{ref}}}

% The empirical distribution defined by the training set
\newcommand{\ptrain}{\hat{p}_{\rm{data}}}
\newcommand{\Ptrain}{\hat{P}_{\rm{data}}}
% The model distribution
\newcommand{\pmodel}{p_{\rm{model}}}
\newcommand{\Pmodel}{P_{\rm{model}}}
\newcommand{\ptildemodel}{\tilde{p}_{\rm{model}}}
% Stochastic autoencoder distributions
\newcommand{\pencode}{p_{\rm{encoder}}}
\newcommand{\pdecode}{p_{\rm{decoder}}}
\newcommand{\precons}{p_{\rm{reconstruct}}}

\newcommand{\laplace}{\mathrm{Laplace}} % Laplace distribution

\newcommand{\E}{\mathbb{E}}
\newcommand{\Ls}{\mathcal{L}}
\newcommand{\R}{\mathbb{R}}
\newcommand{\emp}{\tilde{p}}
\newcommand{\lr}{\alpha}
\newcommand{\reg}{\lambda}
\newcommand{\rect}{\mathrm{rectifier}}
\newcommand{\softmax}{\mathrm{softmax}}
\newcommand{\sigmoid}{\sigma}
\newcommand{\softplus}{\zeta}
\newcommand{\KL}{D_{\mathrm{KL}}}
\newcommand{\Var}{\mathrm{Var}}
\newcommand{\standarderror}{\mathrm{SE}}
\newcommand{\Cov}{\mathrm{Cov}}
% Wolfram Mathworld says $L^2$ is for function spaces and $\ell^2$ is for vectors
% But then they seem to use $L^2$ for vectors throughout the site, and so does
% wikipedia.
\newcommand{\normlzero}{L^0}
\newcommand{\normlone}{L^1}
\newcommand{\normltwo}{L^2}
\newcommand{\normlp}{L^p}
\newcommand{\normmax}{L^\infty}

\newcommand{\parents}{Pa} % See usage in notation.tex. Chosen to match Daphne's book.

\DeclareMathOperator*{\argmax}{arg\,max}
\DeclareMathOperator*{\argmin}{arg\,min}

\DeclareMathOperator{\sign}{sign}
\DeclareMathOperator{\Tr}{Tr}
\let\ab\allowbreak

% Remove the "review" option to generate the final version.
% \usepackage[review]{acl}
\usepackage[]{acl}


\usepackage{times}
\usepackage{latexsym}
\usepackage[T1]{fontenc}
\usepackage[utf8]{inputenc}
\usepackage{microtype}



\usepackage{url}
\usepackage{graphicx}  %Required
\usepackage{amsfonts}
\usepackage{CJK}
\usepackage{bm}
\usepackage{mathrsfs}
\usepackage{float}
\usepackage{xcolor,colortbl}
\usepackage{adjustbox}
\usepackage{subfigure}
\usepackage{booktabs,multirow}
\usepackage{bigstrut}
\usepackage{paralist}
\usepackage{bbm}
\usepackage{makecell}
\usepackage{nicefrac}       % compact symbols for 1/2, etc.
\usepackage{microtype} 
\usepackage{epsfig}
\usepackage{diagbox}
\usepackage{framed}
\usepackage{empheq}
\usepackage{array}
\usepackage{enumitem}
\usepackage{mdwlist}
\usepackage{xspace,mfirstuc,tabulary}
\usepackage{tcolorbox}
\usepackage{algorithm}
\usepackage{placeins}
\usepackage{algpseudocode}
\usepackage{microtype}


% % 表格用的
\usepackage[normalem]{ulem}
\useunder{\uline}{\ul}{}
\usepackage{threeparttable}
\usepackage{makecell}
\usepackage{booktabs}

\usepackage{xspace}
\usepackage{natbib}
\usepackage{hyperref}
\usepackage{url}

\usepackage{graphicx}
\usepackage{subfigure}
\usepackage{multirow}
\usepackage{multicol}
%\usepackage[breakable]{tcolorbox}
\usepackage{pifont}
\usepackage{enumitem}
\usepackage{amsmath}
\usepackage{appendix}
\usepackage{wasysym}


\usepackage{tikz}
\newcommand*{\circled}[1]{\lower.7ex\hbox{\tikz\draw (0pt, 0pt)%
    circle (.5em) node {\makebox[1em][c]{\small #1}};}}
%use method: \circled{1}  


% Algorithm style
\usepackage{amsmath}
\usepackage{amssymb}
\usepackage{mathtools}
\usepackage{amsthm}
\usepackage{listings} %
\lstset{language=Python,                %
        basicstyle=\fontsize{6.5pt}{6.5pt}\ttfamily\selectfont,
        keywordstyle=\color{blue},       %
        stringstyle=\color{purple},         %
        commentstyle=\color{blue},      %
        morecomment=[l][\color{magenta}]{\#},  %
        % frame=single,                    %
        breaklines=true,                 %
        showstringspaces=false           %
}


\newcommand{\ie}{{\emph{i.e.,}}\xspace}
\newcommand{\etc}{etc.}
\newcommand{\eg}{{\emph{e.g.,}}\xspace}
\newcommand{\tlc}[1]{\makecell[c]{#1}}

\newcommand{\increase}[1]{\textcolor{red}{\scriptsize #1}}
% \newcommand{\zkc}[1]{\textcolor{cyan}{[zkc: #1]}}
\newcommand{\lj}[1]{\textcolor{cyan}{[lj: #1]}}
\def\benchmark{{\sc CodeAgentBench}\xspace}
\def\textbfbenchmark{{\sc \textbf{CodeAgentBench}}\xspace}
\def\method{{\sc CodeAgent}\xspace}
\def\textbfmethod{{\sc \textbf{CodeAgent}}\xspace}
\def\textitmethod{{\sc \textit{\textbf{CodeAgent}}}\xspace}

\newcommand{\zkc}[1]{\textcolor{cyan}{[zkc: #1]}}
% \newcommand{\jjx}[1]{\textcolor{red}{[jjx: #1]}}
% \newcommand{\zj}[1]{\textcolor{pink}{\bf \small [#1 --zj]}}
% \newcommand{\zkc}[1]{\textcolor{cyan}{}}
\newcommand{\jjx}[1]{\textcolor{red}{}}
\newcommand{\zj}[1]{\textcolor{pink}{}}
\newcommand{\lijia}[1]{\textcolor{red}{[lijia: #1]}}

% 
% error-prone points

% \newcommand{\rev}[1]{\textcolor{blue}{#1}}
\newcommand{\rev}[1]{{#1}}

%\newtheorem{Def}{Definition}



% If the title and author information does not fit in the area allocated, uncomment the following
%
%\setlength\titlebox{<dim>}
%
% and set <dim> to something 5cm or larger.

\title{Focused-DPO: Enhancing Code Generation Through Focused Preference Optimization on Error-Prone Points}
% Focused-DPO: Optimizing Code Generation on Error-Prone Points



\author{Kechi Zhang, \ Ge Li\footnotemark[1], \ Jia Li \male, Yihong Dong, Jia Li \female,\ Zhi Jin\footnotemark[1] \\
Key Lab of High Confidence Software Technology (PKU), Ministry of Education \\
School of Computer Science, Peking University, China \\
\texttt{\{zhangkechi,lige,zhijin\}@pku.edu.cn}}

\begin{document}
\maketitle
\footnotetext[1]{Corresponding authors.}
\begin{abstract}


Code generation models have shown significant potential for automating programming tasks. However, the challenge of generating accurate and reliable code persists due to the highly complex and long-reasoning nature of the task.
% \lijia{Even state-of-the-art models often fail in code generation due to small errors, which can drastically affect the overall functionality of code.}
Even state-of-the-art models often fail in code generation due to small errors, which can drastically affect the overall functionality of code.
% Even state-of-the-art models struggle with key decision points, where small errors can drastically affect overall functionality. 
Our study identifies that current models tend to produce errors concentrated at specific error-prone points, which significantly impacts the accuracy of the generated code.
To address this issue, we introduce Focused-DPO, a framework that enhances code generation by directing preference optimization towards these critical error-prone areas. This approach builds on Direct Preference Optimization, emphasizing accuracy in parts prone to errors. Additionally, we develop a method called Error-Point Identification, which constructs a dataset that targets these problematic points without requiring costly human annotations.
Our experiments on benchmarks such as HumanEval(+), MBPP(+), and LiveCodeBench demonstrate that Focused-DPO significantly improves the precision and reliability of code generation, reducing common errors and enhancing overall code quality. By focusing on error-prone points, Focused-DPO advances the accuracy and functionality of model-generated code.


\end{abstract}

% \nocite{Ando2005,andrew2007scalable,rasooli-tetrault-2015}

\maketitle


\section{Introduction}


Code generation has emerged as a pivotal task in artificial intelligence, enabling models to automate essential software development tasks. 
Code Models \cite{GPT-4, guo2024deepseek,qwencoder} have demonstrated remarkable capabilities in code generation tasks.
These advancements have significantly improved developer's productivity, accelerating software delivery timelines.
% like code completion, bug fixing, and even generating entire programs from natural language descriptions. 

Despite their success, generating correct code remains a substantial challenge due to the complex and long-reasoning nature of the task. 
Writing code necessitates long reasoning, where numerous small decisions about syntax and logic must work together to produce a functional program. 
Even minor mistakes, such as an incorrect operator, can cause a program to fail.
Code generation, therefore, can be viewed as a multi-step long reasoning process. 
Ensuring the accuracy of every decision in this multi-step process collectively determines the correctness of the resulting output code.


\begin{figure}[t]
\centering
  \includegraphics[width=\columnwidth]{MotivatingExample.png}  
% \caption{Error-prone points in generated code from Qwen-2.5-Coder-Instruct-7B. We sample 20 outputs for this question. Outputs have common prefixes and suffixes, differing mainly at yellow-highlighted error points. If we continue generating based on different outputs at these points, the final code accuracy varies drastically (90.02\% vs. 3.17\% on our validation dataset). This disparity is not observed on non-highlighted parts of the code.}
\caption{Error-prone points in generated code from Qwen-2.5-Coder-Instruct-7B. We sample 20 outputs for this question. Outputs have common prefixes and suffixes, differing mainly at yellow-highlighted error points. Continuing generation at these points leads to drastically different accuracies (90.02\% vs. 3.17\%). This disparity is not seen in non-highlighted parts.}
\label{fig:motivationExample}
\vspace{-10pt}
\end{figure}

When examining the outputs of current code generation models, we find that errors are not evenly spread across the code. 
Large language models tend to produce errors concentrated in certain error-prone points, even when sampling multiple times with a high temperature.
We illustrate this phenomenon in Figure \ref{fig:motivationExample}, which shows error-prone points highlighted in yellow. 
Despite the overall code having similar prefixes and suffixes, differences at these highlighted error points significantly impact the final code accuracy. 
% 在这些Error-Prone Points上基于不同的片段继续生成,最终代码的正确性差异巨大。
Generating code from correct outputs at these error-prone points can achieve a final accuracy of up to 90.02\%, whereas starting from incorrect outputs reduces accuracy to 3.17\%.
% Our validation dataset reveals that if we continue code generation based on correct outputs at these points, the final accuracy can reach up to 90.02\%. 
% However, generating from incorrect outputs at these points leads to an accuracy drop to 3.17\%. 
Parts of the code, such as function headers (usually at the prefix) or return statements (usually at the suffix), often follow familiar patterns. However, some middle parts of the code, which involve more complex reasoning, are more prone to errors. 
% These parts often contain important keywords, operators, or function calls that significantly impact the correctness of the program. 
Errors in these parts can disrupt and affect the entire program's reliability.
% \lijia{I think that this paragraph is too long. The goals of this paragraph are: (i) introducing an important concept - \textit{error-prone points}; (ii) showing an example of error-prone points. Other details may distract readers.}

% One major challenge in generating correct code lies in how models handle these error-prone points.
% \lijia{Based on the above analyses, it is crucial to address these error-prone points for code generation. However, existing studies on code generation overlook this problem.}
It is crucial to address these error-prone points for code generation. However, existing studies on code generation overlook this problem.
While standard training approaches such as Supervised Fine-Tuning (SFT) \cite{wang2022self} help improve overall output quality, they do not specifically focus on the crucial parts necessary for correctness.
Methods like Direct Preference Optimization (DPO) \cite{rafailov2024direct} aim to align outputs with preferences (e.g., "chosen" vs. "rejected"), but often overlook fine-grained error-prone points of the code. 
As a result, these trained models might generate code that appears correct initially but contains critical issues at the error-prone points, ultimately affecting overall accuracy.

To tackle these issues, we introduce \textbf{Focused-DPO}, a framework designed to enhance code generation through focusing preference optimization on error-prone points. 
Focused-DPO builds on Direct Preference Optimization by emphasizing accuracy improvement in areas where errors are most likely to occur. 
Unlike traditional methods that treat all parts of the code equally, Focused-DPO specifically targets those error-prone points, which are essential for the overall correctness of the program.


Focused-DPO is a data-driven preference optimization method that relies on a specially constructed dataset with identified error-prone points. We propose a dataset construction method named \textbf{Error-Point Identification}, which includes an automated pipeline to construct paired code preference datasets. This method extracts concepts from real code repositories and synthesizes programming problems. By concurrently generating code and tests, and using a page-rank-inspired algorithm for ranking, we determine the relative performance of all generated code. Error-Point Identification employs common prefix and suffix matching to precisely locate error-prone points.
Additionally, our method automatically identifies error-prone code parts, eliminating the need for costly human input, making it scalable and efficient for a variety of programming tasks.


We evaluate Focused-DPO using standard benchmarks such as HumanEval(+) \cite{liu2024your}, MBPP(+), and LiveCodeBench \cite{jain2024livecodebench}, and observe significant improvements over existing methods. Even for models like \textit{Qwen2.5-Coder}, which already have undergone large-scale alignment training, Focused-DPO still achieves a 42.86\% relative improvement on extremely hard competition-level problems in LiveCodeBench. 
The results show notable increases in the generation quality on error-prone points, highlighting Focused-DPO's effectiveness in enhancing the accuracy of code generation.


Our contributions are summarized as follows:
\begin{itemize}
\item We propose Focused-DPO , a novel framework that enhances code generation by focusing preference optimization on error-prone points, resulting in more accurate codes.
\item We introduce a dataset construction method that automatically identifies error-prone points by generating both code and corresponding tests for fine-grained self-verification.
% Unlike traditional methods like SFT and DPO, Focused-DPO prioritizes resolving critical issues in high-impact code areas without relying on costly external human annotations.
% Unlike traditional training approaches such as SFT and DPO, Focused-DPO prioritizes resolving critical issues in high-impact code parts—essential for overall program correctness—without relying on costly external human annotations.
\item Experiments on widely-used benchmarks show that Focused-DPO improves the generation quality of code models, even for those that have already undergone extensive post-training on million-level datasets.
\end{itemize}




\section{Related Work}

% \subsection{Preference Optimization for Code Generation}
Large language models (LLMs) have made significant progress in generating code from natural language descriptions, showing great potential for automating software development tasks. Models\citep{GPT-4,li2023starcoder,qwencoder, guo2024deepseek, aixcoder} have demonstrated strong performance, thanks to extensive training on diverse datasets. To further enhance their capabilities, posting training methods like Supervised Fine-Tuning (SFT) \cite{luo2023wizardcoder, wei2023magicoder} and Direct Preference Optimization \cite{qwencoder, codedpo, stepcoder, codeoptimise, plum} are commonly applied. Preference optimization approaches focus on aligning model outputs with desired outcomes by prioritizing more favorable responses over less favorable ones. 
% Direct Preference Optimization (DPO) is one such method that has shown effectiveness in domains like mathematics, where it improves model performance through preference-based distinctions between outputs. DPO have achieves stable improvement in the code generation tasks .
However, existing DPO approaches fail to address one important issue: they do not directly target the most error-prone points in generated code. Errors in these high-impact parts can lead to significant quality and reliability issues in the final output. 
We aim to address this issue by focusing the preference optimization learning on these error-prone points in the generated code. 
% \lijia{Please consider citing our aiXcoder-7B paper, haha.}

% \subsection{Fine-grained Preference Optimization}

Some fine-grained preference optimization methods \cite{rafailov2024direct, lai2024step, stepctrldpo,tdpo, cdpo} have shown strong potential in domains like mathematics, which rely heavily on natural language reasoning. Step-DPO \cite{lai2024step} and Step-Controlled DPO \cite{stepctrldpo} propose generating step-wise preference datasets to enable optimization learning based on the standard DPO loss. TDPO \cite{tdpo} enhances the DPO loss by incorporating forward KL divergence constraints at the token level, achieving fine-grained alignment for each token. cDPO \cite{cdpo} proposes a tricky method to find the critical token in the thought chain that affects overall accuracy. However, the identified tokens are typical in natural language and the method does not apply to code, which features similar overall patterns but relies on specific key elements in long reasoning processes.
However, in the context of code generation, where a small error-prone point can lead to major functional errors, these exisiting methods often struggle to construct adequate datasets or fail to achieve ideal improvements due to weak fine-grained reward signals.
To address this, we propose Focused-DPO, a framework that improves code generation by focusing on optimizing these high-impact parts. 
Our dataset construction method employs a self-generation and validation process to construct datasets that explicitly identify error-prone points, ensuring the optimization learning process directly enhances the parts of the code that matter most for overall correctness.


\section{Focused-DPO}

\begin{figure}[t]
\centering
  \includegraphics[width=\columnwidth]{Method.png}  
\caption{Overview of the Focused-DPO framework. Focused-DPO consists of three key stages: \ding{182} Generating synthetic question prompts from real-world code repositories. \ding{183} Using a policy model to simultaneously generate code and test cases, applying a page-rank algorithm to identify correct and incorrect samples and locate error-prone points using common prefixes and suffixes. \ding{184} Applying Focused-DPO, which pays special attention on error-prone points as if applying a magnifying glass for focused optimization.}
\label{fig:Method}
\vspace{-10pt}
\end{figure}
% \mathcal{L}_{\tiny Focused-DPO} {\tiny } =  -\mathbb{E}\left[
% \log \sigma \left( \cdots\cdots +\cdots  -\cdots  \right)
% \right]

Our proposed Focused-DPO framework aims to enhance code generation by concentrating on error-prone points through focused preference optimization. 
Building on Direct Preference Optimization, our Focused-DPO specifically targets those high-impact parts of the source code, rather than treating all code parts equally.
% To identify these error-prone points, we use the policy model to simultaneously generate code and test cases and apply a page-rank verification process. 
% By comparing various generated code samples and analyzing persistent error locations, our method effectively pinpoints error-prone points that require more attention.
As illustrated in Figure \ref{fig:Method}, our method involves three main steps: 
\ding{182} \textbf{Synthetic Data Generation with Real-World Source Code} : We initiate by collecting a seed dataset from open-source code repositories and generate programming task prompts.
\ding{183} \textbf{Fine-Grained Verification to Identify Error-Prone Points} : We generate both code and tests simultaneously using a self-generation-and-validation loop. We apply a PageRank algorithm to iteratively update scores and rank the outputs, identifying correct and incorrect code samples. 
By distinguishing between similar versions of correct code and incorrect code, we locate significant parts that highly affect the final correctness and identify these parts as error-prone points, allowing for further fine-grained optimization learning.
\ding{184} \textbf{Focused Preference Optimization Learning} : We design a learning optimization algorithm specifically for these critical error-prone points. 
Using the constructed dataset, our novel training loss helps the model develop a preference for these focused parts within the code, thus optimizing performance more effectively.

\subsection{Synthetic Data Generation with Real-World Source Code}

The first step in our approach is the construction of a synthetic dataset. We collect a diverse set of programming snippets from open-source repositories to create a seed dataset. 
Similar to OSS-instruct \cite{selfoss}, we use the seed dataset to extract key programming concepts, such as algorithm design and data structure utilization. 
Then based on these concepts we generate the final prompts. 
This construction strategy allows the model to explore a broad range of scenarios. The generated question prompts are used in the following stages.


\subsection{Fine-Grained Verification to Identify Error-Prone Points}
\label{sec:dataconstruct}

To identify error-prone points, we propose a dataset construction method named \textbf{Error-Point Identification}.
Firstly, we use the policy model to simultaneously generate $k$ output codes and test cases based on the question prompts using a higher-temperature setting. In our experiment, we set $k = 10$. 
Using their execution relationships, we then adopt the ranking method from CodeDPO \cite{codedpo}, a page-rank algorithm to iteratively update scores and rank the outputs:

\begin{equation}
\resizebox{0.7\linewidth}{!}{$
\begin{split}
\text{Score}_t(c_i) &= (1 - d) \times \text{Score}_{t-1}(c_i) \\
&\quad + d \times \sum_{t_j} \text{Score}_{t-1}(t_j) \times \text{Link}(t_j, c_i)
% \end{split}
\\
% \begin{split}
\text{Score}_t(t_j) &= (1 - d) \times \text{Score}_{t-1}(t_j) \\
&\quad + d \times \sum_{c_i} \text{Score}_{t-1}(c_i) \times \text{Link}(c_i, t_j)
\end{split}
$}
\end{equation}
Where \( d \) is the damping factor, and \( \text{Link}(t_j, c_i) \) indicates whether a code snippet \( c_i \) passes the test case \( t_j \). 
The ranking score is updated iteratively until the ranking of the code stabilizes. 

We consider the test case that the highest-ranked code correctly passes as the ground truth test case for this question. Subsequently, we split all generated codes into two categories: correct code that passes all ground truth test cases and incorrect code that does not.
For each pair consisting of a correct code sample and an incorrect code sample, we match their common prefix and suffix to decompose each code snippet into three parts: \(\texttt{common\_prefix}\), \(\texttt{mid\_chosen}\) (or \(\texttt{mid\_rej}\)), and \(\texttt{common\_suffix}\). We then define a \(\text{\textit{Diff}}\) function as follows:


\begin{equation}
\resizebox{\linewidth}{!}{$
\begin{aligned}
\text{\textit{Rank}}(\text{mid}) = & \text{Score}(\text{common\_prefix}, \text{mid}, \text{common\_suffix}), \\
\text{\textit{Diff}} =& \text{Rank}(\text{mid\_chosen}) - \text{Rank}(\text{mid\_rej}) \\
&+ \lambda * (\text{length}(\text{common\_prefix}) +  \text{length}(\text{common\_suffix})).
\end{aligned}
$}
\end{equation}

Our constructed \(\text{\textit{Diff}}\) function includes two components: \ding{182} the difference in rank between the correct and incorrect code, and \ding{183} the sum of the lengths of the common prefix and suffix, which ensures that the error-prone points are more concentrated.
We maximize \(\text{\textit{Diff}}\) to choose the \(\texttt{mid\_chosen}\) and \(\texttt{mid\_rej}\) parts that significantly impact the code's correctness, and identify these as the error-prone points. 
% This construction process ensures that our constructed dataset focuses on the most crucial parts of the code, help us find the error-prone points.
By focusing on error-prone points, we create training samples that directly address the parts of the code that have significantly impact on correctness. 
For each policy model, we apply necessary filtering to the generated data, resulting in a final dataset containing 5,000 training samples and 1,000 validation samples. Table \ref{tab:training-dataset-statistics} presents an example of data statistics.



\subsection{Focused Preference Optimization Learning}
\label{sec:methodloss}

The core of our method lies in modifying the Direct Preference Optimization (DPO) framework to better enhance code generation by focusing on error-prone points of the code. 
Given a pairwise preference dataset \(\mathcal{D} = \{(x_i, y^{chosen}_i, y^{rej}_i)\}_{i=1}^M\), the standard DPO loss \cite{rafailov2024direct} is expressed as:
\begin{equation}
\resizebox{\linewidth}{!}{$
\ell_{\text{DPO}} = -\mathbb{E}_{(x, y^{chosen}, y^{rej}) \sim \mathcal{D}} \left[ \log \sigma \left(\phi(x, y^{chosen}) - \phi(x, y^{rej}) \right)\right],
$}
\end{equation}
where \(\phi(x, y)\) is an implicit reward function. The reward function is defined as:
\begin{equation}
\resizebox{\linewidth}{!}{$
% \phi(x, y) = \beta \cdot \log \frac{\pi_\theta(y|x)}{\pi_{\text{ref}}(y|x)} + \beta \cdot \log Z(x), 
\phi(x, y) = \beta \cdot \log \frac{\pi_\theta(y|x)}{\pi_{\text{ref}}(y|x)} + \underbrace{\beta \cdot \log Z(x)}_{\text{this term can ultimately be reduced}}
$}
\end{equation}
where \(\pi_\theta(y|x)\) represents the probability of a generated response \(y\) under the policy model, and \(\pi_{\text{ref}}(y|x)\) is the probability under a reference model, typically the SFT baseline. The goal of DPO loss is to maximize reward difference between the preferred and non-preferred samples.

\paragraph{Reward Function Modification}
In its original form, the DPO reward \(\phi(x, y)\) is calculated over the entirety of the sample \(y\), treating all parts of the code equally. 
However, in the context of code generation, not all parts of the code contribute equally to correctness. 
Building on our observation that the middle part (\(\texttt{mid}\)) of code—the error-prone point we identify in Section \ref{sec:dataconstruct}—should receive more attention, we restructure the reward to reflect the relative importance of different code parts.
The reward function is modified to weight the \(\texttt{mid}\) part more heavily, reflecting its critical contribution to the correctness of the code. For the preferred sample, the reward function becomes:

\begin{equation}
\resizebox{0.7\linewidth}{!}{$
\begin{split}
&\phi_{\text{chosen}}(x, y) = \beta  \cdot \Big( \log \frac{\pi_\theta(\texttt{prefix}|x)}{\pi_{\text{ref}}(\texttt{prefix}|x)} \\
& + w_{\text{focused}} \cdot \log \frac{\pi_\theta(\texttt{mid}|x, \texttt{prefix})}{\pi_{\text{ref}}(\texttt{mid}|x, \texttt{prefix})} \\
& + \log \frac{\pi_\theta(\texttt{suffix}|x, \texttt{prefix}, \texttt{mid})}{\pi_{\text{ref}}(\texttt{suffix}|x, \texttt{prefix}, \texttt{mid})} \Big)
\end{split}
$}
\end{equation}

Where \(w_{\text{focused}}\) is a weight that amplifies the importance of the \(\texttt{mid}\) part.

For the non-preferred sample, we adopt a similar structure but introduce an adjustment to further downweight the contribution of the \(\texttt{suffix}\). This adjustment is based on our observation that regardless of whether the \(\texttt{mid}\) part contains errors, the content of the \(\texttt{suffix}\) is often the same or similar. 
Our results in Section \ref{sec:experimentrq1} show that the correlation between the \(\texttt{suffix}\) and the overall accuracy of the final code is low, making it less significant in the reward calculation. The reward becomes:
\begin{equation}
\resizebox{0.7\linewidth}{!}{$
\begin{split}
&\phi_{\text{rej}}(x, y) = \gamma  \cdot \Big( \log \frac{\pi_\theta(\texttt{prefix}|x)}{\pi_{\text{ref}}(\texttt{prefix}|x)} \\
& + w_{\text{focused}} \cdot \log \frac{\pi_\theta(\texttt{mid}|x, \texttt{prefix})}{\pi_{\text{ref}}(\texttt{mid}|x, \texttt{prefix})} \Big)
\end{split}
$}
\end{equation}


\paragraph{Final Loss Function}

% 我们将定义好的positive reward 和 negative reward做差,进一步计算:
% \begin{equation}
% \resizebox{\linewidth}{!}{$
% \begin{split}
% &\phi_{\text{negative}}(x, y) = \gamma  \cdot \Big( \log \frac{\pi_\theta(\texttt{prefix}|x)}{\pi_{\text{ref}}(\texttt{prefix}|x)} \\
% & + w_{\text{focused}} \cdot \log \frac{\pi_\theta(\texttt{mid}|x, \texttt{prefix})}{\pi_{\text{ref}}(\texttt{mid}|x, \texttt{prefix})} \Big)
% \end{split}
% $}
% \end{equation}
% 从而定义出preference optimization的目标。

Substituting the modified rewards for the preferred (\(y^{chosen}\)) and non-preferred (\(y^{rej}\)) examples into the original DPO loss and simplifying by canceling common terms, we can obtain that:
\begin{equation}
\resizebox{\linewidth}{!}{$ % 注意,这里需要用 $ 来显式进入数学模式
\begin{aligned} % aligned 内可以使用对齐符号 & 进行对齐
\Delta_{\texttt{reward}} 
= & \phi_{\text{chosen}}(x, y^{chosen}) - \phi_{\text{rej}}(x, y^{rej}) \\
= & 
\underbrace{
\begin{aligned}
\sum_{j=k+1}^{m} \beta \cdot w_{\text{focused}} \cdot \log \frac{\pi_\theta(t_j^{\text{(mid\_chosen)}} | x, t_{0:k}^\text{(prefix)}, t_{k+1:j-1}^\text{(mid\_chosen)})}{\pi_{\text{SFT}}(t_j^{\text{(mid\_chosen)}} | x, t_{0:k}^\text{(prefix)}, t_{k+1:j-1}^\text{(mid\_chosen)})} \\
- \sum_{j=k+1}^{n} \beta \cdot w_{\text{focused}} \cdot \log \frac{\pi_\theta(t_j^{\text{(mid\_rej)}} | x, t_{0:k}^\text{(prefix)}, t_{k+1:j-1}^\text{(mid\_rej)})}{\pi_{\text{SFT}}(t_j^{\text{(mid\_rej)}} | x, t_{0:k}^\text{(prefix)}, t_{k+1:j-1}^\text{(mid\_rej)})}
\end{aligned}
}_{\Delta_{\texttt{mid}}} \\
& +
\underbrace{
\sum_{j=m+1}^{L_1} \beta \cdot \log \frac{\pi_\theta(t_j^{\text{(suffix)}} | x, t_{0:k}^\text{(prefix)}, t_{k+1:m}^\text{(mid\_chosen)}, t_{m+1:j-1}^\text{(suffix)})}{\pi_{\text{SFT}}(t_j^{\text{(suffix)}} | x, t_{0:k}^\text{(prefix)}, t_{k+1:m}^\text{(mid\_chosen)}, t_{m+1:j-1}^\text{(suffix)})}
}_{\Delta_{\texttt{suffix}}} \\
= & \Delta_{\texttt{mid}} + \Delta_{\texttt{suffix}}
\end{aligned}
$}
\end{equation}

So the final loss function for Focused-DPO is expressed as:
\begin{equation}
\resizebox{0.9\linewidth}{!}{$
\begin{split}
\mathcal{L}&_{\text{Focused-DPO}}(\pi_\theta; \pi_\text{SFT}) = \\
& -\mathbb{E}_{(x, y^{chosen}, y^{rej}) \sim \mathcal{D}} \left[
\log \sigma \left( \Delta_{\texttt{mid}} + \Delta_{\texttt{suffix}} \right)
\right],
\end{split}
$}
\end{equation}
% where:
% \begin{equation}
% \resizebox{\linewidth}{!}{$
% \begin{split}
% \Delta_{\texttt{mid}} = & \sum_{j=k+1}^{m} \beta \cdot w_{\text{focused}} \cdot \log \frac{\pi_\theta(t_j^{\text{(mid\_chosen)}} | x, t_{0:k}^\text{(prefix)}, t_{k+1:j-1}^\text{(mid\_chosen)})}{\pi_{\text{SFT}}(t_j^{\text{(mid\_chosen)}} | x, t_{0:k}^\text{(prefix)}, t_{k+1:j-1}^\text{(mid\_chosen)})} \\
% & -
% \sum_{j=k+1}^{n} \beta \cdot w_{\text{focused}} \cdot \log \frac{\pi_\theta(t_j^{\text{(mid\_rej)}} | x, t_{0:k}^\text{(prefix)}, t_{k+1:j-1}^\text{(mid\_rej)})}{\pi_{\text{SFT}}(t_j^{\text{(mid\_rej)}} | x, t_{0:k}^\text{(prefix)}, t_{k+1:j-1}^\text{(mid\_rej)})},
% \end{split}
% $}
% \end{equation}
% and:
% \begin{equation}
% \resizebox{\linewidth}{!}{$
% \begin{split}
% \Delta_{\texttt{suffix}} = & \sum_{j=m+1}^{L_1} \beta \cdot \log \frac{\pi_\theta(t_j^{\text{(chosen-suffix)}} | x, t_{0:k}^\text{(prefix)}, t_{k+1:m}^\text{(mid\_chosen)}, t_{m+1:j-1}^\text{(suffix)})}{\pi_{\text{SFT}}(t_j^{\text{(chosen-suffix)}} | x, t_{0:k}^\text{(prefix)}, t_{k+1:m}^\text{(mid\_chosen)}, t_{m+1:j-1}^\text{(suffix)})}.
% \end{split}
% $}
% \end{equation}

The terms \(\Delta_{\texttt{mid}}\) and \(\Delta_{\texttt{suffix}}\) capture the weighted differences in the probabilities of critical parts between the preferred and non-preferred samples, with greater emphasis focused on the \(\texttt{mid}\) parts, which is the error-prone point.

Through this modification, Focused-DPO shifts the focus of optimization toward the error-prone point in the code. 
By increasing the weight of these parts in the reward calculation, our framework ensures that the model prioritizes improvements where they matter most, leading to higher-quality and more reliable code generation.


\section{Experiment Setup}
\label{sec:experimentsetup}
% We conduct a series of experiments to evaluate the performance of our proposed Focused-DPO framework on various code generation tasks. 
We aim to answer the following research questions:

% \paragraph{RQ1: Are there critical regions in generated code that significantly affect the correctness of the output?}
% This question addresses the key motivation for Focused-DPO by empirically validating whether specific parts of the code have a substantial impact on the final correctness of generated code. 
% To explore this, we constructed a verification dataset using the data construction method described in Section \ref{sec:dataconstruct}. 
% This dataset consists of 1,000 programming prompts where we carefully identified the critical parts of the generated code using self-generated scores and validations.


% \paragraph{RQ2: 
% % Does Focused-DPO improve the correctness of generated code compared to baseline models on code generation benchmarks? 
% Can Focused-DPO improve the generation quality of code models, including those have already been heavily fine-tuned with alignment techniques?}
% To investigate this, we evaluate Focused-DPO on established code generation benchmarks, including HumanEval \citep{chen2021evaluating}, HumanEval+ \citep{liu2024your}, MBPP \citep{austin2021program}, MBPP+, and LiveCodeBench. The dataset statistics are shown in Table \ref{statistic}.
% These benchmarks cover a wide range of classical and challenging programming tasks, providing a comprehensive evaluation of code generation capabilities. We compare the accuracy of Focused-DPO with several advanced preference optimization baselines, including standard DPO, RPO, and Step-DPO. These methods serve as strong baselines for assessing the correctness improvements achieved by our framework. 

% 还要证明我们的数据标注方法是有效的,就是说我们的数据标注方法找error-prone points是高效的
\paragraph{RQ1: Are there error-prone points in generated code that significantly affect the correctness of the output?}
This question addresses the core motivation behind Focused-DPO. To investigate this, we construct the validation dataset following Section \ref{sec:dataconstruct}.
This setup provides empirical evidence supporting the theoretical underpinnings of our Focused-DPO.

\paragraph{RQ2: Can Focused-DPO improve the generation quality of code models, even those that have already been heavily post-trained with alignment techniques such as standard DPO?}
To explore this, we evaluate Focused-DPO on several widely-used code generation benchmarks, including HumanEval \citep{chen2021evaluating}, HumanEval+ \citep{liu2024your}, MBPP \citep{austin2021program}, MBPP+, and LiveCodeBench \cite{jain2024livecodebench}. 
% We compare with popular optimization methods, including standard DPO, Step-DPO, Token-DPO and SFT. 
% These benchmarks encompass a diverse range of well-known and challenging programming tasks, offering a comprehensive evaluation of code generation performance. The dataset statistics are summarized in Table \ref{statistic}.

% These methods provide strong benchmarks for evaluating the improvements in code correctness, particularly in scenarios where pre-trained models have already undergone significant fine-tuning and alignment.



% \paragraph{RQ2: Can Focused-DPO improve the generation quality of models that have already been heavily fine-tuned with alignment techniques?}
% To address this question, we analyze whether pre-trained and fine-tuned large code models, which are known to excel in general code generation tasks, still exhibit frequent errors in critical areas. Specifically, we aim to evaluate whether Focused-DPO can further reduce these logical errors and improve the overall reliability of generated outputs, even for models that have undergone extensive training.

% \paragraph{RQ3: How does Focused-DPO enhance the quality in critical code areas, and how do these enhancements affect overall program correctness?}
% Based on our validation dataset, which explicitly labels the critical regions in code samples, we measure the model's performance on these critical regions. 
% Specifically, we quantify how improvements in these critical regions contribute to the overall correctness of generated programs.

\paragraph{RQ3: How do different components of the Focused-DPO loss formulation affect model performance?}
% We perform ablation studies to analyze the impact of each design choice in our Focused-DPO. 
Ablation studies include evaluating our dataset construction method, as well as key components in our loss formulation.

% \paragraph{RQ5: How efficient is the fine-grained data construction strategy proposed in Focused-DPO?}
% To evaluate the efficiency of our data construction approach, we compare variations of our data generation process. This includes analyzing the trade-off between dataset quality and the computational overhead of generating fine-grained annotations. We aim to demonstrate that our method provides a high-quality dataset with minimal additional cost.

\subsection{Baselines}
\label{sec:setupLLM}

We evaluate several widely used large language models (LLMs) in the code generation domain.
For \textbf{\textit{base models}}, we apply Focused-DPO to \textbf{DeepSeekCoder-base-6.7B)} \citep{guo2024deepseek} and \textbf{Qwen2.5-Coder-7B} \citep{qwencoder}. 
For \textbf{\textit{instruct models}} , we evaluate on \textbf{Magicoder-S-DS-6.7B} \citep{wei2023magicoder} and \textbf{DeepSeekCoder-instruct-6.7B}, which are post-trained from \textit{DeepSeekCoder-base-6.7B} with large-scale SFT. We further evaluate \textbf{Qwen-2.5-Coder-Instruct-7B}, which is post-trained from \textit{Qwen2.5-Coder-7B} on million-level datasets with SFT and DPO.

We compare against several widely used training techniques, including: 
\textbf{SFT}, \textbf{standard DPO}, \textbf{Step-DPO} \cite{lai2024step}, \textbf{TDPO} \cite{tdpo}.  
SFT trains models only with positive samples, while the other methods utilize a pairwise dataset of preferred and rejected samples.

\subsection{Training and Inference Settings}

For each backbone LLM, we sample 10 code candidates and corresponding test cases for each problem prompt using \texttt{temperature=1.5}. An example of data statistics is in Table \ref{tab:training-dataset-statistics}. 
Our analysis shows this configuration results in a stable ranking score and ensures diversity. We focus on Python-based datasets given its widespread use.
For training, we train for 10 epochs on 8 NVIDIA V100 GPUs and select the best-performing checkpoint based on the lowest validation loss. We set $w_{focused} = 2$ in our experiments.
We use a learning rate of \(5 \times 10^{-6}\) with a linear scheduler and warm-up. 
We employ greedy search during inference. 

\section{Results and Analyses}

\subsection{Exploration of Error-Prone Points in Code (RQ1)}
\label{sec:experimentrq1}

% \lijia{Comments about this section have been sent to zkc via WeChat.}
We conduct experiments to validate our motivation:

\noindent \ding{182} Correlation analysis confirms that \textbf{error-prone points in the code significantly impact correctness}, whereas other code parts have minimal effect.

\noindent \ding{183} Generation experiments show that \textbf{continuing at these points with different content leads to significant differences in overall correctness}.

\noindent \ding{184} Observations reveal that \textbf{existing code models perform suboptimally at these points}.
% that there are error-prone points in generated code that significantly impact its correctness
% 前两个实验:Error-points确实存在,而且对代码正确性影响很大;
% 最后的一个实验是为了说明现有的LLMs在Error-points上表现不好。
\paragraph{Correlation Between Different Code Parts and Final Correctness}

% To address RQ1, we conduct experiments to confirm that there are error-prone points in generated code that significantly impact its correctness. 
Utilizing the dataset construction pipeline described in Section \ref{sec:dataconstruct}, we evaluate the validation dataset based on \textit{Qwen2.5-Coder-Instruct-7B}. 
We analyze the relationship between \(\texttt{prefix}\), \(\texttt{suffix}\), two types of \(\texttt{mid}\) parts, and the final code correctness, as presented in Table \ref{tab:segment-frequencies-and-phi}.


\begin{table}[h]
    \centering
    \resizebox{\linewidth}{!}{
    \begin{tabular}{lcc|c}
        \toprule
        \textbf{Segment} & \textbf{Correct} & \textbf{Incorrect} & \textbf{Phi Coefficient}  \\
        \midrule
        Common Prefix                    & 0.7907 & 0.7325 & \cellcolor{yellow!30}0.0683 \\
        Common Suffix                    & 0.8479 & 0.7864 & \cellcolor{yellow!30}0.0796  \\ \midrule
        Common Prefix + Chosen Mid       & 0.6367 & 0.0911 & \cellcolor{green!30}0.5651  \\
        Common Prefix + Reject Mid       & 0.0116 & 0.5575 & \cellcolor{red!30}-0.6085  \\
        \bottomrule
    \end{tabular}
    }
    \caption{Relationships between the \(\texttt{prefix}\), \(\texttt{suffix}\), and the two types of \(\texttt{mid}\) parts with the final code correctness. The table includes the frequency of each part in correct and incorrect code, as well as their correlation coefficients with overall code correctness.}
    % (Green indicates strong positive correlation, and red indicates strong negative correlation)
    \label{tab:segment-frequencies-and-phi}
    \vspace{-10pt}
\end{table}

Results in Table \ref{tab:segment-frequencies-and-phi} show that \texttt{common\_prefix + chosen\_mid} appears much more frequently in correct solutions, while \texttt{common\_prefix + rej\_mid} is prevalent in incorrect solutions. 
This confirms the critical influence of the \texttt{mid} part, with strong positive and negative correlations respectively, affirming the existence of error-prone points in generated code.
In contrast, we find that the prefix and suffix parts have little relation to the correctness of the final answer. It is important to note that in incorrect code, despite the errors in the \texttt{mid} section, the following suffix is not a significant cause of the errors. 
This observation justifies our decision to exclude the suffix in the reward modification in Section \ref{sec:methodloss}.
These findings provide empirical evidence supporting our hypothesis that focusing on these error-prone points is essential to enhance model performance, which is the core motivation behind our Focused-DPO framework.

\paragraph{Accuracy of Continuation at Error-Prone Points}

We further generate 20 code solutions based on different contents at error-prone points, to explore the correctness of the final code generated under different conditions in Table \ref{tab:pass-rates-mid}. 

\begin{table}[h]
    \centering
    \resizebox{\linewidth}{!}{
    \begin{tabular}{lcccc}
        \toprule
        \textbf{Based on Input} & \textbf{pass@1} & \textbf{pass@3} & \textbf{pass@5} & \textbf{pass@10} \\
        \midrule
        Common Prefix + Chosen Mid   & 0.9002 & 0.9532 & 0.9688 & 0.9871 \\
        Common Prefix + Reject Mid & 0.0317 & 0.0633 & 0.0810 & 0.1159 \\
        \bottomrule
    \end{tabular}
    }
    \caption{Pass rates based on different content at error-prone points.}
    \label{tab:pass-rates-mid}
    \vspace{-10pt}
\end{table}

The pass rates shown in Table \ref{tab:pass-rates-mid} highlight a striking contrast: using \texttt{chosen\_mid} at error-prone points results in significantly higher pass rates, reaching around 90\% at pass@1, compared to just over 3\% for the \texttt{rej\_mid} version. This demonstrates the critical importance of accurate content in the error-prone points for determining the correctness of the final generated code.


Based on the above results, we have noticed that \textbf{the generated content at the error-prone points significantly affects the final outcomes}. 
This leads to a question: \textit{how do current code generation models behave at these error-prone points?}

\paragraph{Generation Preferences at Error-Prone Points in Code Models}

% /mnt/bd/devdocs/output_for_skeleton_dpo_n10_mergeWithInput_filterCorrectSol/a_statistic_focusedDPO_passrate_with_mid_genprob.ipynb
\begin{figure}[h]
\centering
  \includegraphics[width=\columnwidth]{diff_dis_before.png}  
\caption{Generation probability difference \((p(\text{chosen\_mid}) - p(\text{rej\_mid}))\) with input.}
\label{fig:motivationprob}
\vspace{-10pt}
\end{figure}


We further analyze the \textit{Qwen-2.5-Coder-Instruct-7B}, which has been post-trained on million-level datasets using SFT and DPO. We examine the generation preferences of this heavily post-trained model at error-prone points. Specifically, we calculate the probability difference between generating \texttt{chosen\_mid} and \texttt{rej\_mid} when given the \texttt{common\_prefix} as input. The distribution of the difference is shown in Figure \ref{fig:motivationprob}. The model exhibits little to no clear preference, indicating that existing code generation models lack effective generation capability at these error-prone points.
Through this exploration, we confirm that focused preference optimization of error-prone points is crucial for improving the accuracy of code models, addressing RQ1.

\subsection{Main Results (RQ2)}

\paragraph{Results on benchmarks}
% We evaluate on five widely-used code generation benchmarks.
Tables \ref{tab:main-results-benchmark1} and \ref{tab:livecodebenchresults} summarize the performance of Focused-DPO compared to various baselines, including standard DPO, Step-DPO, TDPO, and SFT. 
Note that the formulas for standard DPO and Step-DPO are identical, making them equivalent.
% We also include results for \textbf{Qwen2.5-coder-instruct-7B} and \textbf{MagiCoder-S-DS-6.7B} as representative baseline models before applying our Focused-DPO framework. 
The relative improvements (\textit{Rel}) are reported for a clearer comparison.

\begin{table}[h]
    \centering

    \resizebox{\linewidth}{!}{
    
    \begin{tabular}{lcccc}
        \toprule
        \textbf{Model}                & \textbf{HumanEval} & \textbf{HumanEval+} & \textbf{MBPP} & \textbf{MBPP+} \\
        \bottomrule
        \textit{Instruct Model} & & & & \\
        \toprule
        \textbf{Qwen2.5-coder-instruct-7B}        & 0.915              & 0.841              & 0.828         & 0.714          \\
        + Our Focused-DPO                   & \textbf{0.927}     & \textbf{0.878}     & \textbf{0.847} & \textbf{0.762} \\
        \textit{Relative Improvement}    & 1.29\%             & 4.41\%             & 2.24\%        & 6.71\%         \\
        \midrule
        DPO / Step-DPO                & 0.921              & 0.854              & 0.841         & 0.743          \\
        Token-DPO                          & 0.927              & 0.872              & 0.833         & 0.751          \\
        SFT                           & 0.927              & 0.872              & 0.833         & 0.717          \\
        \midrule
                \textbf{DeepSeekCoder-instruct-6.7B}        & 0.774              & 0.701              & 0.751         & 0.659          \\
        + Our Focused-DPO                   & \textbf{0.823}     & \textbf{0.732}     & \textbf{0.765} & \textbf{0.669} \\
        \textit{Relative Improvement}    & 6.35\%             & 4.38\%             & 1.80\%        & 1.56\%         \\
        \midrule
        DPO / Step-DPO                & 0.787              & 0.713              & 0.751         & 0.661          \\
        Token-DPO                          & 0.799              & 0.726              & 0.751         & 0.661          \\
        SFT                           & 0.787              & 0.726              & 0.759         & 0.667          \\
        \midrule
                \textbf{MagiCoder-S-DS-6.7B}        & 0.732              & 0.683              & 0.767         & 0.667          \\
        + Our Focused-DPO                   & \textbf{0.823}     & \textbf{0.744}     & \textbf{0.794} & \textbf{0.698} \\
        \textit{Relative Improvement}    & 12.50\%            & 8.93\%             & 3.45\%        & 4.76\%         \\
        \midrule
        DPO / Step-DPO                & 0.762              & 0.701              & 0.772         & 0.675          \\
        Token-DPO                          & 0.811              & 0.732              & 0.780         & 0.680          \\
        SFT                           & 0.738              & 0.701              & 0.762         & 0.653          \\
        \bottomrule
        \textit{Base Model} & & & & \\
        \toprule
\textbf{Qwen2.5-coder-base}        & 0.835              & 0.787              & 0.794         & 0.683          \\
        + Our Focused-DPO                   & \textbf{0.884}     & \textbf{0.829}     & \textbf{0.817} & \textbf{0.704} \\
        \textit{Relative Improvement}    & 5.89\%             & 5.37\%             & 2.95\%        & 3.03\%         \\
        \midrule
        DPO / Step-DPO                & 0.848              & 0.799              & 0.802         & 0.688          \\
        Token-DPO                          & 0.866              & 0.799              & 0.815         & 0.690          \\
        SFT                           & 0.848              & 0.805              & 0.802         & 0.688          \\
        \midrule
                \textbf{DeepSeekCoder-base-6.7B}        & 0.476              & 0.396              & 0.702         & 0.566          \\
        + Our Focused-DPO                   & \textbf{0.518}     & \textbf{0.427}     & \textbf{0.717} & \textbf{0.574} \\
        \textit{Relative Improvement}    & 8.89\%             & 7.79\%             & 2.13\%        & 1.43\%         \\
        \midrule
        DPO / Step-DPO                & 0.488              & 0.396              & 0.709         & 0.569          \\
        Token-DPO                          & 0.500              & 0.421              & 0.717         & 0.574          \\
        SFT                           & 0.488              & 0.396              & 0.704         & 0.566          \\
        \bottomrule
    \end{tabular}
    }
        \caption{Pass Rate on HumanEval(+), MBPP(+)}
    \label{tab:main-results-benchmark1}
    \vspace{-10pt}
\end{table}

\begin{table}[h]
    \centering
    \resizebox{\linewidth}{!}{
    
    \begin{tabular}{lcccc}
        \toprule
        \textbf{Model}                & \textbf{Easy} & \textbf{Medium} & \textbf{Hard} & \textbf{Average} \\
        \bottomrule
        \textit{Instruct Model} & & & & \\
        \toprule
        \textbf{Qwen2.5-coder-instruct-7B} & 0.692       & 0.220         & 0.034         & 0.312        \\
        + Our Focused-DPO                   & \textbf{0.735} & \textbf{0.242} & \textbf{0.048} & \textbf{0.339} \\
        \textit{Relative Improvement}       & 6.22\%         & 10.04\%       & 42.86\%       & 8.44\%        \\
        \midrule
        DPO / Step-DPO                & 0.685       & 0.233         & 0.019         & 0.310        \\
        Token-DPO                          & 0.706       & 0.239         & 0.037         & 0.325        \\
        SFT                           & 0.670       & 0.208         & 0.015         & 0.295        \\
        \midrule
        \textbf{DeepSeekCoder-instruct-6.7B} & 0.453       & 0.091         & 0.009         & 0.181        \\
        + Our Focused-DPO                   & \textbf{0.477} & \textbf{0.106} & \textbf{0.019} & \textbf{0.197} \\
        \textit{Relative Improvement}       & 5.30\%         & 15.89\%       & 108.33\%      & 8.87\%        \\
        \midrule
        DPO / Step-DPO                & 0.462       & 0.094         & 0.007         & 0.184        \\
        Token-DPO                          & 0.470       & 0.100         & 0.019         & 0.192        \\
        SFT                           & 0.462       & 0.094         & 0.004         & 0.183        \\
        \midrule
        \textbf{MagiCoder-S-DS-6.7B} & 0.481       & 0.107         & 0.001         & 0.193        \\
        + Our Focused-DPO                   & \textbf{0.513} & \textbf{0.118} & \textbf{0.019} & \textbf{0.213} \\
        \textit{Relative Improvement}       & 6.56\%         & 10.12\%       & 1751.85\%     & 10.10\%       \\
        \midrule
        DPO / Step-DPO                & 0.491       & 0.109         & 0.004         & 0.198        \\
        Token-DPO                          & 0.505       & 0.118         & 0.015         & 0.209        \\
        SFT                           & 0.498       & 0.112         & 0.004         & 0.201        \\
        \bottomrule
        \textit{Base Model} & & & & \\
        \toprule
        \textbf{Qwen2.5-coder-base-7B}       & 0.567       & 0.150         & 0.017         & 0.241        \\
        + Our Focused-DPO                   & \textbf{0.595} & \textbf{0.175} & \textbf{0.030} & \textbf{0.264} \\
        \textit{Relative Improvement}       & 5.00\%         & 16.47\%       & 77.78\%       & 9.23\%        \\
        \midrule
        DPO / Step-DPO                & 0.577       & 0.151         & 0.015         & 0.244        \\
        Token-DPO                          & 0.584       & 0.163         & 0.022         & 0.253        \\
        SFT                           & 0.584       & 0.157         & 0.022         & 0.251        \\
        \midrule
        \textbf{DeepSeekCoder-base-6.7B} & 0.399       & 0.074         & 0.004         & 0.155        \\
        + Our Focused-DPO                   & \textbf{0.423} & \textbf{0.085} & \textbf{0.011} & \textbf{0.169} \\
        \textit{Relative Improvement}       & 6.00\%         & 14.31\%       & 177.78\%      & 9.24\%        \\
        \midrule
        DPO / Step-DPO                & 0.412       & 0.079         & 0.004         & 0.161        \\
        Token-DPO                          & 0.419       & 0.079         & 0.004         & 0.164        \\
        SFT                           & 0.419       & 0.082         & 0.007         & 0.166        \\
        \bottomrule
    \end{tabular}
    }
    \caption{Pass Rate on LiveCodeBench}
    \label{tab:livecodebenchresults}
    \vspace{-10pt}
\end{table}

As shown in Table \ref{tab:main-results-benchmark1}, Focused-DPO consistently outperforms the baseline models across all benchmarks. 
On the HumanEval(+) and MBPP(+) benchmarks, Focused-DPO improves relative accuracy by 4.79\% on average over the baseline.
We also evaluate on LiveCodeBench, a challenging benchmark that features iteratively updated, competition-level programming problems sourced from platforms such as LeetCode. The benchmark is divided into three levels of difficulty: Easy, Medium, and Hard. 
Focused-DPO achieves consistent improvements across all difficulty levels of LiveCodeBench. Notably, on the hardest category (\textit{Hard}), Focused-DPO can achieve huge relative performance.
Focused-DPO entirely outperforms other advanced preference optimization baselines such as Step-DPO and TDPO. These findings highlight the effectiveness of Focused-DPO in challenging code generation scenarios, where optimization on error-prone points of code plays a crucial role in determining final correctness.



\paragraph{Enhancing Heavily Post-trained Models}

Focused-DPO can significantly enhance the performance of code models that have already undergone extensive post-training. As demonstrated in Table \ref{tab:dsSFTDPO}, models like \textit{Qwen2.5-Coder-instruct}, which have been meticulously optimized using millions of data points from SFT and DPO processes, still exhibit substantial improvements with our Focused-DPO framework.
To further illustrate Focused-DPO's benefits on heavily post-trained models, we conducted an extensive initial DPO training phase. 
Following the methodology from CodeDPO, we used the model \textit{DeepSeekCoder-base-6.7} and a large-scale dataset with 93k samples for DPO training, continued until full convergence.
% Building on this transparent and solid foundation, w
We then apply Focused-DPO for further experiments. This allows us to explore the extent to which Focused-DPO could drive additional improvements, even in models already trained by intensive post-training processes. 
% The complete post-training procedure and results at each stage are detailed in Table \ref{tab:dsSFTDPO}.

\begin{table}[h]
\centering
\resizebox{\linewidth}{!}{
\begin{tabular}{l|c|c|c|c}
\toprule
\textbf{Model} & \textbf{HumanEval} & \textbf{HumanEval+} & \textbf{MBPP} & \textbf{MBPP+} \\
\midrule
DeepSeekCoder-base-6.7B & 0.4760 & 0.3960 & 0.7020 & 0.5660 \\
\midrule
+ SFT Stage & 0.7317 & 0.6829 & 0.7672 & 0.6667 \\
 \textit{(with MagiCoder-OSS-instruct)} & & & & \\
\midrule
+ First DPO Stage & 0.8354 & 0.7622 & 0.8070 & 0.7093 \\
 \textit{(with CodeDPO-OSS-instruct)} & & & & \\
\midrule
+ Focused-DPO & \textbf{0.8719} & \textbf{0.7926} & \textbf{0.8227} & \textbf{0.7275} \\
\bottomrule
\end{tabular}
}
\caption{Performance of DeepSeekCoder-6.7B at different training stages. The stages include base model, SFT with MagiCoder, first DPO with CodeDPO, and our Focused-DPO. Focused-DPO achieves additional improvements even after high-quality post-training.}
\label{tab:dsSFTDPO}
\vspace{-10pt}
\end{table}

As shown in Table \ref{tab:dsSFTDPO}, we start from the base model and progressively incorporate the SFT stage \cite{wei2023magicoder} and the first DPO stage \cite{codedpo}. 
Finally, applying our Focused-DPO leads to the highest pass rates achieved. 
These results demonstrate that Focused-DPO effectively boosts the performance of models that have already been extensively post-trained and optimized through previous stages. 
We further evaluate how Focused-DPO enhances the quality of error-prone points in Appendix \ref{sec:improveerrorpronepoints}.




\subsection{Ablation Study (RQ3)}

% To address RQ3, we focus on exploring the impact of different design settings in our methods, including the Error-Prone Points Dataset Construction strategy and the Focused-DPO loss function based on \textit{Qwen2.5-Coder-Instruct-7B}. 
% More experiments on other models are shown in Appendix.

\paragraph{Dataset Construction Ablations}

Focused-DPO includes an automated data construction and Error-Prone Identification process. 
We perform ablation experiments on the dataset construction methods in Table \ref{tab:dataset-construction-ablation}.
We design two alternative approaches:
\ding{182} The \textit{Step-DPO strategy} \cite{lai2024step} constructs datasets by considering only the common prefix parts, with the rest treated as Error-Prone Points for training.
\ding{183} Using a \textit{git-diff tool} \footnote{\url{https://git-scm.com/docs/git-diff}}, we construct datasets where the differences covered by the diff were treated as Error-Prone Points, with the parts following the final diff difference treated as the suffix.
Note that Step-DPO dataset construction method is closely tied to the formulation of the Step-DPO loss function, leading to consistent outcomes between the two. However, we observe that Step-DPO performs suboptimally on code generation tasks. 
In contrast, the current dataset construction method used in Focused-DPO, which employs a simple yet effective Error-Prone Identification strategy, achieves the best experimental results.


\paragraph{Loss Function Ablations}
Our Focused-DPO has made appropriate modifications to the calculation of positive and negative rewards. 
We carry out ablation experiments in Table \ref{tab:dataset-construction-ablation}, including trying different values of $w_{focused}$ and various treatments of the suffix in the reward function.
Our findings indicate that increasing or decreasing $w_{focused}$ leads to a decline in model performance, suggesting that the current value of $w_{focused}$ is optimal. 
Additionally, we observe that including the suffix part in the reward function results in degraded performance. Through detailed analysis in Section \ref{sec:experimentrq1}, the suffix in incorrect code does not exhibit strong correlations with the overall accuracy. 
These experiments validate the practical advantages of the design choices in our loss function.

\begin{table}[h!]
    \centering
    \resizebox{\linewidth}{!}{
    \begin{tabular}{lcc}
        \toprule
        \textbf{Dataset Construction} & \textbf{HumanEval / HumanEval+} & \textbf{MBPP / MBPP+} \\
        \midrule
        Focused-DPO    & & \\
        % ,\\ Using Common Prefix and Suffix
        \makecell[l]{Error Prone Identification}     & \textbf{0.9268 / 0.8780} & \textbf{0.8466 / 0.7619} \\
        \midrule
        Step-DPO Strategy                      & 0.9207 / 0.8537          & 0.8413 / 0.7434          \\
        Diff-based Strategy                    & 0.9268 / 0.8598          & 0.8439 / 0.7539          \\
        \bottomrule
        \toprule
        \textbf{Loss Function Setting} & \textbf{HumanEval / HumanEval+} & \textbf{MBPP / MBPP+} \\
        \midrule
        Focused-DPO    & & \\
        \makecell[l]{$w_{focused} = 2$,\\ No Suffix in Reject Reward}     & \textbf{0.9268 / 0.8780} & \textbf{0.8466 / 0.7619} \\
        \midrule
        \textit{Decrease Weight}    & & \\
        $w_{focused} = 1$         & 0.9268 / 0.8720          & 0.8386 / 0.7487          \\
        \midrule
        \textit{Increase Weight}    & & \\
        $w_{focused} = 3$             & 0.9268 / 0.8720          & 0.8439 / 0.7566          \\
        $w_{focused} = 5$              & 0.8963 / 0.7683          & 0.8201 / 0.6878          \\
        \midrule
        Suffix in Reject Reward & 0.9268 / 0.8659          & 0.8413 / 0.7487          \\
        \bottomrule
    \end{tabular}
    }
    \caption{Dataset Construction and Loss Function Ablation Results based on \textit{Qwen2.5-Coder-Instruct-7B}}
    \label{tab:dataset-construction-ablation}
    \vspace{-15pt}
\end{table}











% \subsection{Quality Analysis of Focused-DPO}


% \subsection{Robost Analysis of Focused-DPO}





\section{Conclusion}

% We propose Focused-DPO, a framework designed to enhance the accuracy of code generation by focusing on error-prone points in the code. 
% We find that these critical points significantly affect the functionality of generated programs. 
% Focused-DPO improves upon Direct Preference Optimization by prioritizing these areas, using our Error-Point Identification method to create datasets without costly human annotations.
% Our evaluations demonstrate that Focused-DPO reduces errors and enhances code quality, even in heavily post-trained models. 
% This research highlights the effectiveness of concentrating on error-prone points and sets a promising direction for conducting fine-grained preference optimization in AI-driven software development.

We propose Focused-DPO, a framework that improves code generation by focusing on error-prone points. 
These critical parts significantly impact overall program correctness. 
Focused-DPO improves Direct Preference Optimization by prioritizing these points, using our Error-Point Identification method to create datasets without costly human annotations.
Evaluations show Focused-DPO reduces errors and improves code quality, even in heavily post-trained models.  
This research highlights the benefits of focusing on fine-grained preference optimization in AI-driven software development.

\section{Acknowledgement}
This work is primarily based on our prior research and findings in CodeDPO \cite{codedpo}. We are deeply grateful to Jingjing Xu, Jing Su, and Jun Zhang for their valuable discussions regarding the preliminary experiments.




\section*{Limitations and Ethical Considerations}

\noindent\textbf{Limitations.} The primary limitation of our work is that it extends only the dataset provided by MUSE and employs DeepSeek-v3 for question generation. 
To mitigate this generalization risk, we have released our code and the generated audit suite, allowing researchers to utilize our framework to create additional audit datasets and evaluate their quality. Meanwhile, this is also our future work to extend our framework to other benchmarks.

\noindent\textbf{Ethical Considerations.} Machine unlearning can be employed to mitigate risks associated with LLMs in terms of privacy, security, bias, and copyright. Our work is dedicated to providing a comprehensive evaluation framework to help researchers better understand the unlearning effectiveness of LLMs, which we believe will have a positive impact on society.


% \bibliography{main}
% \bibliographystyle{acl_natbib}
% Entries for the entire Anthology, followed by custom entries
\bibliography{main}

% \clearpage
\clearpage
\newpage
\appendix

\subsection{Lloyd-Max Algorithm}
\label{subsec:Lloyd-Max}
For a given quantization bitwidth $B$ and an operand $\bm{X}$, the Lloyd-Max algorithm finds $2^B$ quantization levels $\{\hat{x}_i\}_{i=1}^{2^B}$ such that quantizing $\bm{X}$ by rounding each scalar in $\bm{X}$ to the nearest quantization level minimizes the quantization MSE. 

The algorithm starts with an initial guess of quantization levels and then iteratively computes quantization thresholds $\{\tau_i\}_{i=1}^{2^B-1}$ and updates quantization levels $\{\hat{x}_i\}_{i=1}^{2^B}$. Specifically, at iteration $n$, thresholds are set to the midpoints of the previous iteration's levels:
\begin{align*}
    \tau_i^{(n)}=\frac{\hat{x}_i^{(n-1)}+\hat{x}_{i+1}^{(n-1)}}2 \text{ for } i=1\ldots 2^B-1
\end{align*}
Subsequently, the quantization levels are re-computed as conditional means of the data regions defined by the new thresholds:
\begin{align*}
    \hat{x}_i^{(n)}=\mathbb{E}\left[ \bm{X} \big| \bm{X}\in [\tau_{i-1}^{(n)},\tau_i^{(n)}] \right] \text{ for } i=1\ldots 2^B
\end{align*}
where to satisfy boundary conditions we have $\tau_0=-\infty$ and $\tau_{2^B}=\infty$. The algorithm iterates the above steps until convergence.

Figure \ref{fig:lm_quant} compares the quantization levels of a $7$-bit floating point (E3M3) quantizer (left) to a $7$-bit Lloyd-Max quantizer (right) when quantizing a layer of weights from the GPT3-126M model at a per-tensor granularity. As shown, the Lloyd-Max quantizer achieves substantially lower quantization MSE. Further, Table \ref{tab:FP7_vs_LM7} shows the superior perplexity achieved by Lloyd-Max quantizers for bitwidths of $7$, $6$ and $5$. The difference between the quantizers is clear at 5 bits, where per-tensor FP quantization incurs a drastic and unacceptable increase in perplexity, while Lloyd-Max quantization incurs a much smaller increase. Nevertheless, we note that even the optimal Lloyd-Max quantizer incurs a notable ($\sim 1.5$) increase in perplexity due to the coarse granularity of quantization. 

\begin{figure}[h]
  \centering
  \includegraphics[width=0.7\linewidth]{sections/figures/LM7_FP7.pdf}
  \caption{\small Quantization levels and the corresponding quantization MSE of Floating Point (left) vs Lloyd-Max (right) Quantizers for a layer of weights in the GPT3-126M model.}
  \label{fig:lm_quant}
\end{figure}

\begin{table}[h]\scriptsize
\begin{center}
\caption{\label{tab:FP7_vs_LM7} \small Comparing perplexity (lower is better) achieved by floating point quantizers and Lloyd-Max quantizers on a GPT3-126M model for the Wikitext-103 dataset.}
\begin{tabular}{c|cc|c}
\hline
 \multirow{2}{*}{\textbf{Bitwidth}} & \multicolumn{2}{|c|}{\textbf{Floating-Point Quantizer}} & \textbf{Lloyd-Max Quantizer} \\
 & Best Format & Wikitext-103 Perplexity & Wikitext-103 Perplexity \\
\hline
7 & E3M3 & 18.32 & 18.27 \\
6 & E3M2 & 19.07 & 18.51 \\
5 & E4M0 & 43.89 & 19.71 \\
\hline
\end{tabular}
\end{center}
\end{table}

\subsection{Proof of Local Optimality of LO-BCQ}
\label{subsec:lobcq_opt_proof}
For a given block $\bm{b}_j$, the quantization MSE during LO-BCQ can be empirically evaluated as $\frac{1}{L_b}\lVert \bm{b}_j- \bm{\hat{b}}_j\rVert^2_2$ where $\bm{\hat{b}}_j$ is computed from equation (\ref{eq:clustered_quantization_definition}) as $C_{f(\bm{b}_j)}(\bm{b}_j)$. Further, for a given block cluster $\mathcal{B}_i$, we compute the quantization MSE as $\frac{1}{|\mathcal{B}_{i}|}\sum_{\bm{b} \in \mathcal{B}_{i}} \frac{1}{L_b}\lVert \bm{b}- C_i^{(n)}(\bm{b})\rVert^2_2$. Therefore, at the end of iteration $n$, we evaluate the overall quantization MSE $J^{(n)}$ for a given operand $\bm{X}$ composed of $N_c$ block clusters as:
\begin{align*}
    \label{eq:mse_iter_n}
    J^{(n)} = \frac{1}{N_c} \sum_{i=1}^{N_c} \frac{1}{|\mathcal{B}_{i}^{(n)}|}\sum_{\bm{v} \in \mathcal{B}_{i}^{(n)}} \frac{1}{L_b}\lVert \bm{b}- B_i^{(n)}(\bm{b})\rVert^2_2
\end{align*}

At the end of iteration $n$, the codebooks are updated from $\mathcal{C}^{(n-1)}$ to $\mathcal{C}^{(n)}$. However, the mapping of a given vector $\bm{b}_j$ to quantizers $\mathcal{C}^{(n)}$ remains as  $f^{(n)}(\bm{b}_j)$. At the next iteration, during the vector clustering step, $f^{(n+1)}(\bm{b}_j)$ finds new mapping of $\bm{b}_j$ to updated codebooks $\mathcal{C}^{(n)}$ such that the quantization MSE over the candidate codebooks is minimized. Therefore, we obtain the following result for $\bm{b}_j$:
\begin{align*}
\frac{1}{L_b}\lVert \bm{b}_j - C_{f^{(n+1)}(\bm{b}_j)}^{(n)}(\bm{b}_j)\rVert^2_2 \le \frac{1}{L_b}\lVert \bm{b}_j - C_{f^{(n)}(\bm{b}_j)}^{(n)}(\bm{b}_j)\rVert^2_2
\end{align*}

That is, quantizing $\bm{b}_j$ at the end of the block clustering step of iteration $n+1$ results in lower quantization MSE compared to quantizing at the end of iteration $n$. Since this is true for all $\bm{b} \in \bm{X}$, we assert the following:
\begin{equation}
\begin{split}
\label{eq:mse_ineq_1}
    \tilde{J}^{(n+1)} &= \frac{1}{N_c} \sum_{i=1}^{N_c} \frac{1}{|\mathcal{B}_{i}^{(n+1)}|}\sum_{\bm{b} \in \mathcal{B}_{i}^{(n+1)}} \frac{1}{L_b}\lVert \bm{b} - C_i^{(n)}(b)\rVert^2_2 \le J^{(n)}
\end{split}
\end{equation}
where $\tilde{J}^{(n+1)}$ is the the quantization MSE after the vector clustering step at iteration $n+1$.

Next, during the codebook update step (\ref{eq:quantizers_update}) at iteration $n+1$, the per-cluster codebooks $\mathcal{C}^{(n)}$ are updated to $\mathcal{C}^{(n+1)}$ by invoking the Lloyd-Max algorithm \citep{Lloyd}. We know that for any given value distribution, the Lloyd-Max algorithm minimizes the quantization MSE. Therefore, for a given vector cluster $\mathcal{B}_i$ we obtain the following result:

\begin{equation}
    \frac{1}{|\mathcal{B}_{i}^{(n+1)}|}\sum_{\bm{b} \in \mathcal{B}_{i}^{(n+1)}} \frac{1}{L_b}\lVert \bm{b}- C_i^{(n+1)}(\bm{b})\rVert^2_2 \le \frac{1}{|\mathcal{B}_{i}^{(n+1)}|}\sum_{\bm{b} \in \mathcal{B}_{i}^{(n+1)}} \frac{1}{L_b}\lVert \bm{b}- C_i^{(n)}(\bm{b})\rVert^2_2
\end{equation}

The above equation states that quantizing the given block cluster $\mathcal{B}_i$ after updating the associated codebook from $C_i^{(n)}$ to $C_i^{(n+1)}$ results in lower quantization MSE. Since this is true for all the block clusters, we derive the following result: 
\begin{equation}
\begin{split}
\label{eq:mse_ineq_2}
     J^{(n+1)} &= \frac{1}{N_c} \sum_{i=1}^{N_c} \frac{1}{|\mathcal{B}_{i}^{(n+1)}|}\sum_{\bm{b} \in \mathcal{B}_{i}^{(n+1)}} \frac{1}{L_b}\lVert \bm{b}- C_i^{(n+1)}(\bm{b})\rVert^2_2  \le \tilde{J}^{(n+1)}   
\end{split}
\end{equation}

Following (\ref{eq:mse_ineq_1}) and (\ref{eq:mse_ineq_2}), we find that the quantization MSE is non-increasing for each iteration, that is, $J^{(1)} \ge J^{(2)} \ge J^{(3)} \ge \ldots \ge J^{(M)}$ where $M$ is the maximum number of iterations. 
%Therefore, we can say that if the algorithm converges, then it must be that it has converged to a local minimum. 
\hfill $\blacksquare$


\begin{figure}
    \begin{center}
    \includegraphics[width=0.5\textwidth]{sections//figures/mse_vs_iter.pdf}
    \end{center}
    \caption{\small NMSE vs iterations during LO-BCQ compared to other block quantization proposals}
    \label{fig:nmse_vs_iter}
\end{figure}

Figure \ref{fig:nmse_vs_iter} shows the empirical convergence of LO-BCQ across several block lengths and number of codebooks. Also, the MSE achieved by LO-BCQ is compared to baselines such as MXFP and VSQ. As shown, LO-BCQ converges to a lower MSE than the baselines. Further, we achieve better convergence for larger number of codebooks ($N_c$) and for a smaller block length ($L_b$), both of which increase the bitwidth of BCQ (see Eq \ref{eq:bitwidth_bcq}).


\subsection{Additional Accuracy Results}
%Table \ref{tab:lobcq_config} lists the various LOBCQ configurations and their corresponding bitwidths.
\begin{table}
\setlength{\tabcolsep}{4.75pt}
\begin{center}
\caption{\label{tab:lobcq_config} Various LO-BCQ configurations and their bitwidths.}
\begin{tabular}{|c||c|c|c|c||c|c||c|} 
\hline
 & \multicolumn{4}{|c||}{$L_b=8$} & \multicolumn{2}{|c||}{$L_b=4$} & $L_b=2$ \\
 \hline
 \backslashbox{$L_A$\kern-1em}{\kern-1em$N_c$} & 2 & 4 & 8 & 16 & 2 & 4 & 2 \\
 \hline
 64 & 4.25 & 4.375 & 4.5 & 4.625 & 4.375 & 4.625 & 4.625\\
 \hline
 32 & 4.375 & 4.5 & 4.625& 4.75 & 4.5 & 4.75 & 4.75 \\
 \hline
 16 & 4.625 & 4.75& 4.875 & 5 & 4.75 & 5 & 5 \\
 \hline
\end{tabular}
\end{center}
\end{table}

%\subsection{Perplexity achieved by various LO-BCQ configurations on Wikitext-103 dataset}

\begin{table} \centering
\begin{tabular}{|c||c|c|c|c||c|c||c|} 
\hline
 $L_b \rightarrow$& \multicolumn{4}{c||}{8} & \multicolumn{2}{c||}{4} & 2\\
 \hline
 \backslashbox{$L_A$\kern-1em}{\kern-1em$N_c$} & 2 & 4 & 8 & 16 & 2 & 4 & 2  \\
 %$N_c \rightarrow$ & 2 & 4 & 8 & 16 & 2 & 4 & 2 \\
 \hline
 \hline
 \multicolumn{8}{c}{GPT3-1.3B (FP32 PPL = 9.98)} \\ 
 \hline
 \hline
 64 & 10.40 & 10.23 & 10.17 & 10.15 &  10.28 & 10.18 & 10.19 \\
 \hline
 32 & 10.25 & 10.20 & 10.15 & 10.12 &  10.23 & 10.17 & 10.17 \\
 \hline
 16 & 10.22 & 10.16 & 10.10 & 10.09 &  10.21 & 10.14 & 10.16 \\
 \hline
  \hline
 \multicolumn{8}{c}{GPT3-8B (FP32 PPL = 7.38)} \\ 
 \hline
 \hline
 64 & 7.61 & 7.52 & 7.48 &  7.47 &  7.55 &  7.49 & 7.50 \\
 \hline
 32 & 7.52 & 7.50 & 7.46 &  7.45 &  7.52 &  7.48 & 7.48  \\
 \hline
 16 & 7.51 & 7.48 & 7.44 &  7.44 &  7.51 &  7.49 & 7.47  \\
 \hline
\end{tabular}
\caption{\label{tab:ppl_gpt3_abalation} Wikitext-103 perplexity across GPT3-1.3B and 8B models.}
\end{table}

\begin{table} \centering
\begin{tabular}{|c||c|c|c|c||} 
\hline
 $L_b \rightarrow$& \multicolumn{4}{c||}{8}\\
 \hline
 \backslashbox{$L_A$\kern-1em}{\kern-1em$N_c$} & 2 & 4 & 8 & 16 \\
 %$N_c \rightarrow$ & 2 & 4 & 8 & 16 & 2 & 4 & 2 \\
 \hline
 \hline
 \multicolumn{5}{|c|}{Llama2-7B (FP32 PPL = 5.06)} \\ 
 \hline
 \hline
 64 & 5.31 & 5.26 & 5.19 & 5.18  \\
 \hline
 32 & 5.23 & 5.25 & 5.18 & 5.15  \\
 \hline
 16 & 5.23 & 5.19 & 5.16 & 5.14  \\
 \hline
 \multicolumn{5}{|c|}{Nemotron4-15B (FP32 PPL = 5.87)} \\ 
 \hline
 \hline
 64  & 6.3 & 6.20 & 6.13 & 6.08  \\
 \hline
 32  & 6.24 & 6.12 & 6.07 & 6.03  \\
 \hline
 16  & 6.12 & 6.14 & 6.04 & 6.02  \\
 \hline
 \multicolumn{5}{|c|}{Nemotron4-340B (FP32 PPL = 3.48)} \\ 
 \hline
 \hline
 64 & 3.67 & 3.62 & 3.60 & 3.59 \\
 \hline
 32 & 3.63 & 3.61 & 3.59 & 3.56 \\
 \hline
 16 & 3.61 & 3.58 & 3.57 & 3.55 \\
 \hline
\end{tabular}
\caption{\label{tab:ppl_llama7B_nemo15B} Wikitext-103 perplexity compared to FP32 baseline in Llama2-7B and Nemotron4-15B, 340B models}
\end{table}

%\subsection{Perplexity achieved by various LO-BCQ configurations on MMLU dataset}


\begin{table} \centering
\begin{tabular}{|c||c|c|c|c||c|c|c|c|} 
\hline
 $L_b \rightarrow$& \multicolumn{4}{c||}{8} & \multicolumn{4}{c||}{8}\\
 \hline
 \backslashbox{$L_A$\kern-1em}{\kern-1em$N_c$} & 2 & 4 & 8 & 16 & 2 & 4 & 8 & 16  \\
 %$N_c \rightarrow$ & 2 & 4 & 8 & 16 & 2 & 4 & 2 \\
 \hline
 \hline
 \multicolumn{5}{|c|}{Llama2-7B (FP32 Accuracy = 45.8\%)} & \multicolumn{4}{|c|}{Llama2-70B (FP32 Accuracy = 69.12\%)} \\ 
 \hline
 \hline
 64 & 43.9 & 43.4 & 43.9 & 44.9 & 68.07 & 68.27 & 68.17 & 68.75 \\
 \hline
 32 & 44.5 & 43.8 & 44.9 & 44.5 & 68.37 & 68.51 & 68.35 & 68.27  \\
 \hline
 16 & 43.9 & 42.7 & 44.9 & 45 & 68.12 & 68.77 & 68.31 & 68.59  \\
 \hline
 \hline
 \multicolumn{5}{|c|}{GPT3-22B (FP32 Accuracy = 38.75\%)} & \multicolumn{4}{|c|}{Nemotron4-15B (FP32 Accuracy = 64.3\%)} \\ 
 \hline
 \hline
 64 & 36.71 & 38.85 & 38.13 & 38.92 & 63.17 & 62.36 & 63.72 & 64.09 \\
 \hline
 32 & 37.95 & 38.69 & 39.45 & 38.34 & 64.05 & 62.30 & 63.8 & 64.33  \\
 \hline
 16 & 38.88 & 38.80 & 38.31 & 38.92 & 63.22 & 63.51 & 63.93 & 64.43  \\
 \hline
\end{tabular}
\caption{\label{tab:mmlu_abalation} Accuracy on MMLU dataset across GPT3-22B, Llama2-7B, 70B and Nemotron4-15B models.}
\end{table}


%\subsection{Perplexity achieved by various LO-BCQ configurations on LM evaluation harness}

\begin{table} \centering
\begin{tabular}{|c||c|c|c|c||c|c|c|c|} 
\hline
 $L_b \rightarrow$& \multicolumn{4}{c||}{8} & \multicolumn{4}{c||}{8}\\
 \hline
 \backslashbox{$L_A$\kern-1em}{\kern-1em$N_c$} & 2 & 4 & 8 & 16 & 2 & 4 & 8 & 16  \\
 %$N_c \rightarrow$ & 2 & 4 & 8 & 16 & 2 & 4 & 2 \\
 \hline
 \hline
 \multicolumn{5}{|c|}{Race (FP32 Accuracy = 37.51\%)} & \multicolumn{4}{|c|}{Boolq (FP32 Accuracy = 64.62\%)} \\ 
 \hline
 \hline
 64 & 36.94 & 37.13 & 36.27 & 37.13 & 63.73 & 62.26 & 63.49 & 63.36 \\
 \hline
 32 & 37.03 & 36.36 & 36.08 & 37.03 & 62.54 & 63.51 & 63.49 & 63.55  \\
 \hline
 16 & 37.03 & 37.03 & 36.46 & 37.03 & 61.1 & 63.79 & 63.58 & 63.33  \\
 \hline
 \hline
 \multicolumn{5}{|c|}{Winogrande (FP32 Accuracy = 58.01\%)} & \multicolumn{4}{|c|}{Piqa (FP32 Accuracy = 74.21\%)} \\ 
 \hline
 \hline
 64 & 58.17 & 57.22 & 57.85 & 58.33 & 73.01 & 73.07 & 73.07 & 72.80 \\
 \hline
 32 & 59.12 & 58.09 & 57.85 & 58.41 & 73.01 & 73.94 & 72.74 & 73.18  \\
 \hline
 16 & 57.93 & 58.88 & 57.93 & 58.56 & 73.94 & 72.80 & 73.01 & 73.94  \\
 \hline
\end{tabular}
\caption{\label{tab:mmlu_abalation} Accuracy on LM evaluation harness tasks on GPT3-1.3B model.}
\end{table}

\begin{table} \centering
\begin{tabular}{|c||c|c|c|c||c|c|c|c|} 
\hline
 $L_b \rightarrow$& \multicolumn{4}{c||}{8} & \multicolumn{4}{c||}{8}\\
 \hline
 \backslashbox{$L_A$\kern-1em}{\kern-1em$N_c$} & 2 & 4 & 8 & 16 & 2 & 4 & 8 & 16  \\
 %$N_c \rightarrow$ & 2 & 4 & 8 & 16 & 2 & 4 & 2 \\
 \hline
 \hline
 \multicolumn{5}{|c|}{Race (FP32 Accuracy = 41.34\%)} & \multicolumn{4}{|c|}{Boolq (FP32 Accuracy = 68.32\%)} \\ 
 \hline
 \hline
 64 & 40.48 & 40.10 & 39.43 & 39.90 & 69.20 & 68.41 & 69.45 & 68.56 \\
 \hline
 32 & 39.52 & 39.52 & 40.77 & 39.62 & 68.32 & 67.43 & 68.17 & 69.30  \\
 \hline
 16 & 39.81 & 39.71 & 39.90 & 40.38 & 68.10 & 66.33 & 69.51 & 69.42  \\
 \hline
 \hline
 \multicolumn{5}{|c|}{Winogrande (FP32 Accuracy = 67.88\%)} & \multicolumn{4}{|c|}{Piqa (FP32 Accuracy = 78.78\%)} \\ 
 \hline
 \hline
 64 & 66.85 & 66.61 & 67.72 & 67.88 & 77.31 & 77.42 & 77.75 & 77.64 \\
 \hline
 32 & 67.25 & 67.72 & 67.72 & 67.00 & 77.31 & 77.04 & 77.80 & 77.37  \\
 \hline
 16 & 68.11 & 68.90 & 67.88 & 67.48 & 77.37 & 78.13 & 78.13 & 77.69  \\
 \hline
\end{tabular}
\caption{\label{tab:mmlu_abalation} Accuracy on LM evaluation harness tasks on GPT3-8B model.}
\end{table}

\begin{table} \centering
\begin{tabular}{|c||c|c|c|c||c|c|c|c|} 
\hline
 $L_b \rightarrow$& \multicolumn{4}{c||}{8} & \multicolumn{4}{c||}{8}\\
 \hline
 \backslashbox{$L_A$\kern-1em}{\kern-1em$N_c$} & 2 & 4 & 8 & 16 & 2 & 4 & 8 & 16  \\
 %$N_c \rightarrow$ & 2 & 4 & 8 & 16 & 2 & 4 & 2 \\
 \hline
 \hline
 \multicolumn{5}{|c|}{Race (FP32 Accuracy = 40.67\%)} & \multicolumn{4}{|c|}{Boolq (FP32 Accuracy = 76.54\%)} \\ 
 \hline
 \hline
 64 & 40.48 & 40.10 & 39.43 & 39.90 & 75.41 & 75.11 & 77.09 & 75.66 \\
 \hline
 32 & 39.52 & 39.52 & 40.77 & 39.62 & 76.02 & 76.02 & 75.96 & 75.35  \\
 \hline
 16 & 39.81 & 39.71 & 39.90 & 40.38 & 75.05 & 73.82 & 75.72 & 76.09  \\
 \hline
 \hline
 \multicolumn{5}{|c|}{Winogrande (FP32 Accuracy = 70.64\%)} & \multicolumn{4}{|c|}{Piqa (FP32 Accuracy = 79.16\%)} \\ 
 \hline
 \hline
 64 & 69.14 & 70.17 & 70.17 & 70.56 & 78.24 & 79.00 & 78.62 & 78.73 \\
 \hline
 32 & 70.96 & 69.69 & 71.27 & 69.30 & 78.56 & 79.49 & 79.16 & 78.89  \\
 \hline
 16 & 71.03 & 69.53 & 69.69 & 70.40 & 78.13 & 79.16 & 79.00 & 79.00  \\
 \hline
\end{tabular}
\caption{\label{tab:mmlu_abalation} Accuracy on LM evaluation harness tasks on GPT3-22B model.}
\end{table}

\begin{table} \centering
\begin{tabular}{|c||c|c|c|c||c|c|c|c|} 
\hline
 $L_b \rightarrow$& \multicolumn{4}{c||}{8} & \multicolumn{4}{c||}{8}\\
 \hline
 \backslashbox{$L_A$\kern-1em}{\kern-1em$N_c$} & 2 & 4 & 8 & 16 & 2 & 4 & 8 & 16  \\
 %$N_c \rightarrow$ & 2 & 4 & 8 & 16 & 2 & 4 & 2 \\
 \hline
 \hline
 \multicolumn{5}{|c|}{Race (FP32 Accuracy = 44.4\%)} & \multicolumn{4}{|c|}{Boolq (FP32 Accuracy = 79.29\%)} \\ 
 \hline
 \hline
 64 & 42.49 & 42.51 & 42.58 & 43.45 & 77.58 & 77.37 & 77.43 & 78.1 \\
 \hline
 32 & 43.35 & 42.49 & 43.64 & 43.73 & 77.86 & 75.32 & 77.28 & 77.86  \\
 \hline
 16 & 44.21 & 44.21 & 43.64 & 42.97 & 78.65 & 77 & 76.94 & 77.98  \\
 \hline
 \hline
 \multicolumn{5}{|c|}{Winogrande (FP32 Accuracy = 69.38\%)} & \multicolumn{4}{|c|}{Piqa (FP32 Accuracy = 78.07\%)} \\ 
 \hline
 \hline
 64 & 68.9 & 68.43 & 69.77 & 68.19 & 77.09 & 76.82 & 77.09 & 77.86 \\
 \hline
 32 & 69.38 & 68.51 & 68.82 & 68.90 & 78.07 & 76.71 & 78.07 & 77.86  \\
 \hline
 16 & 69.53 & 67.09 & 69.38 & 68.90 & 77.37 & 77.8 & 77.91 & 77.69  \\
 \hline
\end{tabular}
\caption{\label{tab:mmlu_abalation} Accuracy on LM evaluation harness tasks on Llama2-7B model.}
\end{table}

\begin{table} \centering
\begin{tabular}{|c||c|c|c|c||c|c|c|c|} 
\hline
 $L_b \rightarrow$& \multicolumn{4}{c||}{8} & \multicolumn{4}{c||}{8}\\
 \hline
 \backslashbox{$L_A$\kern-1em}{\kern-1em$N_c$} & 2 & 4 & 8 & 16 & 2 & 4 & 8 & 16  \\
 %$N_c \rightarrow$ & 2 & 4 & 8 & 16 & 2 & 4 & 2 \\
 \hline
 \hline
 \multicolumn{5}{|c|}{Race (FP32 Accuracy = 48.8\%)} & \multicolumn{4}{|c|}{Boolq (FP32 Accuracy = 85.23\%)} \\ 
 \hline
 \hline
 64 & 49.00 & 49.00 & 49.28 & 48.71 & 82.82 & 84.28 & 84.03 & 84.25 \\
 \hline
 32 & 49.57 & 48.52 & 48.33 & 49.28 & 83.85 & 84.46 & 84.31 & 84.93  \\
 \hline
 16 & 49.85 & 49.09 & 49.28 & 48.99 & 85.11 & 84.46 & 84.61 & 83.94  \\
 \hline
 \hline
 \multicolumn{5}{|c|}{Winogrande (FP32 Accuracy = 79.95\%)} & \multicolumn{4}{|c|}{Piqa (FP32 Accuracy = 81.56\%)} \\ 
 \hline
 \hline
 64 & 78.77 & 78.45 & 78.37 & 79.16 & 81.45 & 80.69 & 81.45 & 81.5 \\
 \hline
 32 & 78.45 & 79.01 & 78.69 & 80.66 & 81.56 & 80.58 & 81.18 & 81.34  \\
 \hline
 16 & 79.95 & 79.56 & 79.79 & 79.72 & 81.28 & 81.66 & 81.28 & 80.96  \\
 \hline
\end{tabular}
\caption{\label{tab:mmlu_abalation} Accuracy on LM evaluation harness tasks on Llama2-70B model.}
\end{table}

%\section{MSE Studies}
%\textcolor{red}{TODO}


\subsection{Number Formats and Quantization Method}
\label{subsec:numFormats_quantMethod}
\subsubsection{Integer Format}
An $n$-bit signed integer (INT) is typically represented with a 2s-complement format \citep{yao2022zeroquant,xiao2023smoothquant,dai2021vsq}, where the most significant bit denotes the sign.

\subsubsection{Floating Point Format}
An $n$-bit signed floating point (FP) number $x$ comprises of a 1-bit sign ($x_{\mathrm{sign}}$), $B_m$-bit mantissa ($x_{\mathrm{mant}}$) and $B_e$-bit exponent ($x_{\mathrm{exp}}$) such that $B_m+B_e=n-1$. The associated constant exponent bias ($E_{\mathrm{bias}}$) is computed as $(2^{{B_e}-1}-1)$. We denote this format as $E_{B_e}M_{B_m}$.  

\subsubsection{Quantization Scheme}
\label{subsec:quant_method}
A quantization scheme dictates how a given unquantized tensor is converted to its quantized representation. We consider FP formats for the purpose of illustration. Given an unquantized tensor $\bm{X}$ and an FP format $E_{B_e}M_{B_m}$, we first, we compute the quantization scale factor $s_X$ that maps the maximum absolute value of $\bm{X}$ to the maximum quantization level of the $E_{B_e}M_{B_m}$ format as follows:
\begin{align}
\label{eq:sf}
    s_X = \frac{\mathrm{max}(|\bm{X}|)}{\mathrm{max}(E_{B_e}M_{B_m})}
\end{align}
In the above equation, $|\cdot|$ denotes the absolute value function.

Next, we scale $\bm{X}$ by $s_X$ and quantize it to $\hat{\bm{X}}$ by rounding it to the nearest quantization level of $E_{B_e}M_{B_m}$ as:

\begin{align}
\label{eq:tensor_quant}
    \hat{\bm{X}} = \text{round-to-nearest}\left(\frac{\bm{X}}{s_X}, E_{B_e}M_{B_m}\right)
\end{align}

We perform dynamic max-scaled quantization \citep{wu2020integer}, where the scale factor $s$ for activations is dynamically computed during runtime.

\subsection{Vector Scaled Quantization}
\begin{wrapfigure}{r}{0.35\linewidth}
  \centering
  \includegraphics[width=\linewidth]{sections/figures/vsquant.jpg}
  \caption{\small Vectorwise decomposition for per-vector scaled quantization (VSQ \citep{dai2021vsq}).}
  \label{fig:vsquant}
\end{wrapfigure}
During VSQ \citep{dai2021vsq}, the operand tensors are decomposed into 1D vectors in a hardware friendly manner as shown in Figure \ref{fig:vsquant}. Since the decomposed tensors are used as operands in matrix multiplications during inference, it is beneficial to perform this decomposition along the reduction dimension of the multiplication. The vectorwise quantization is performed similar to tensorwise quantization described in Equations \ref{eq:sf} and \ref{eq:tensor_quant}, where a scale factor $s_v$ is required for each vector $\bm{v}$ that maps the maximum absolute value of that vector to the maximum quantization level. While smaller vector lengths can lead to larger accuracy gains, the associated memory and computational overheads due to the per-vector scale factors increases. To alleviate these overheads, VSQ \citep{dai2021vsq} proposed a second level quantization of the per-vector scale factors to unsigned integers, while MX \citep{rouhani2023shared} quantizes them to integer powers of 2 (denoted as $2^{INT}$).

\subsubsection{MX Format}
The MX format proposed in \citep{rouhani2023microscaling} introduces the concept of sub-block shifting. For every two scalar elements of $b$-bits each, there is a shared exponent bit. The value of this exponent bit is determined through an empirical analysis that targets minimizing quantization MSE. We note that the FP format $E_{1}M_{b}$ is strictly better than MX from an accuracy perspective since it allocates a dedicated exponent bit to each scalar as opposed to sharing it across two scalars. Therefore, we conservatively bound the accuracy of a $b+2$-bit signed MX format with that of a $E_{1}M_{b}$ format in our comparisons. For instance, we use E1M2 format as a proxy for MX4.

\begin{figure}
    \centering
    \includegraphics[width=1\linewidth]{sections//figures/BlockFormats.pdf}
    \caption{\small Comparing LO-BCQ to MX format.}
    \label{fig:block_formats}
\end{figure}

Figure \ref{fig:block_formats} compares our $4$-bit LO-BCQ block format to MX \citep{rouhani2023microscaling}. As shown, both LO-BCQ and MX decompose a given operand tensor into block arrays and each block array into blocks. Similar to MX, we find that per-block quantization ($L_b < L_A$) leads to better accuracy due to increased flexibility. While MX achieves this through per-block $1$-bit micro-scales, we associate a dedicated codebook to each block through a per-block codebook selector. Further, MX quantizes the per-block array scale-factor to E8M0 format without per-tensor scaling. In contrast during LO-BCQ, we find that per-tensor scaling combined with quantization of per-block array scale-factor to E4M3 format results in superior inference accuracy across models. 



\end{document}
