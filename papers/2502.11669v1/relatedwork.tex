\section{Related Work}
\label{related}

This section commences with the deep learning-based methods for general 3D point clouds, subsequently delving into specialized techniques for 3D anomaly classification.

\subsection{Deep Learning for General 3D Point Clouds}

Unlike 2D images, 3D point cloud data are irregular, permutation invariant, and have an additional dimension, making them incompatible with conventional Convolutional Neural Networks. To deal with it, PointNet \citep{qi2017pointnet} adopted a shared MLP to encode points into latent features, further aggregated into a global one for classification. To incorporate the local semantics ignored by PointNet, the extended PointNet++ \citep{qi2017pointnet++} proposed a hierarchical framework to aggregate information from local regions. Furthermore, some approaches represented point clouds as graphs that were processed by graph neural networks \citep{te2018rgcnn,wang2019dynamic}. Among them, DGCNN \citep{wang2019dynamic} dynamically updated the graph structure during learning, enhancing the local information sharing and thus improving the performance of classification and semantic segmentation. Recently, Transformers have excelled in natural language processing through self-attention mechanisms that capture global data dependencies. This architecture, applied in 3D point cloud analysis, e.g., the Point Cloud Transformer (PCT) \citep{guo2021pct}, has also achieved impressive outcomes in multiple downstream tasks.

While these methods were originally developed for general point cloud classification and segmentation tasks, they are also applicable to anomaly classification tasks. However, they fail to effectively address the challenges posed by intra-class variation and inter-class similarity, which complicate the classification process. Additionally, they are unable to handle unseen types of anomalies.



\subsection{3D Anomaly Classification Methods}

Classical machine learning-based methods consist of feature extraction and further classification. The features can be defined by the discriminated statistics of sharp points \citep{du2022tensor}, spectral graph Laplacian eigenvalues \citep{samie2017classifying}, and the histogram of deviations between the skin model and CAD model \citep{yacob2019anomaly}. In addition, the anomalies were projected into 2D images such that the characteristics including moments and Fourier descriptors were extracted \citep{madrigal2017method}. Then, various machine learning classifiers can be applied for classification, including the sparse support vector machine \citep{du2022tensor}, sparse representation \citep{samie2017classifying}, and ensemble classifiers \citep{yacob2019anomaly}. The primary limitation of these methods is that the features may lack discriminative power for complex anomaly patterns, such as intra-class variation and inter-class similarity of anomalies.

Deep learning-based methods have emerged that enable more effective feature learning.
For example, leveraging the DGCNN model \citep{wang2019dynamic}, \cite{wang2023mvgcn} introduced a multi-view framework that utilized two sub-networks to explicitly consider both anomaly and reference points, emphasizing the significance of the potentially scarce anomaly points.  Similar to the PCT model \citep{guo2021pct}, \cite{zhou2022sewer} proposed a Transformer-based network that achieved satisfactory performance on 3D sewer anomaly classification. Moreover, \cite{bolourian2023point} utilized PointNet++ to detect the surface anomalies on concrete bridge surfaces. Generally speaking, PointNet, DGCNN, and their variants are potential choices for 3D anomaly classification applications. 

However, the above deep learning-based methods do not directly target the situation with intra-class variation and inter-class similarity, and cannot detect new types of anomaly. This situation is common and important in many realistic manufacturing applications.

%If your manuscript has supplementary content you can prepare this using the \verb"suppldata" document-class option, which will suppress the `article history' date. This option must \emph{not} be used on any primary content.