\section{RELATED WORK}
\begin{comment}
\subsection{Quadrotor Trajectory Planning}

The classical planners formulate trajectory planning as a nonlinear optimization problem and online optimizes a feasible trajectory, which trades off motion smoothness, dynamical feasibility, and collision violation. However, these planners requires large computational power to extract free space or environment representation methods, 

such as occupancy grid map\cite{zhou2020ego}, ESDF\cite{zhou2020robust}, SFC\cite{liu2017planning}. Observing this shortcoming, our community has also been exploring more lightweight environment representation methods. Mapless\cite{ji2021mapless} kd-tree . Meanwhile, the motion primitive planners\cite{ryll2019efficient,florence2020integrated} generate a bunch of trajectories in a sampling space, use the kd-tree to select safe trajectories, and then pick up a trajectory with the minimum user-defined cost to execute. Because the sampled trajectories are not optimized and the sampling resolution is limited, the quality of the generated trajectory cannot always be guaranteed.
In summary, the above methods cannot realize the wish that enjoys both efficiency and efficacy.
\end{comment}

\subsection{Motion Primitive Library}

Motion primitive library in autonomous navigation usually generates multiple paths/trajectories using different sampling methods.
Zhang et al. \cite{zhang2020falco} propose a motion primitive library for offline sampling fixed-length paths in position space. However, the paths lack dynamical information, making it difficult to exert the robot's mobility.
Ryll et al. \cite{ryll2019efficient} sample different local end states and generate multiple fixed duration min-jerk trajectories in combination with the start state.
Yang et al. \cite{yang2021intention} sample different local end states in the velocity space $a=\{v_x, v_z, \omega\}$ online, then combining the start state to generate multiple fixed-duration 8-th order polynomial trajectories and integrating them to provide position information.
Nevertheless, many primitives generated by these two methods are dynamical-invalid. They require sampling a large number of primitives to find a dynamical feasible and safe trajectory.
Collins et al. \cite{collins2020efficient} adaptively sample the end states based on the start speed, which enhances the trajectory library's reliability but requires more computational power to select a feasible trajectory.
Bucki et al. \cite{bucki2020rectangular} propose a new pyramid partitioning method, which speeds up collision checking and saves computational power, but leads to more conservative trajectories and poor performance in obstacle-dense scenarios.
Florence et al. \cite{florence2020integrated} sample a set of constant control variables online and integrate them forward with a fixed duration to form a trajectory library.
The above methods aim to find a dynamical feasible and safe primitive, but without considering the time optimality of the trajectories.


% Please add the following required packages to your document preamble:
% \usepackage{graphicx}

\begin{comment}
\begin{table}[h]
\centering
\caption{Evaluation of Motion Primitive Method}
\label{tab:primitive}
\renewcommand{\arraystretch}{1.35}
\resizebox{\columnwidth}{!}{%
\begin{tabular}{|c|c|c|c|c|c|}
\hline
\textbf{Method} &
\textbf{\begin{tabular}[c]{@{}c@{}}Offline\\ /Online\end{tabular}} &
\textbf{\begin{tabular}[c]{@{}c@{}}Kinematic\\  Information\end{tabular}} &
\textbf{\begin{tabular}[c]{@{}c@{}}Kinematic\\  Feasiblity\end{tabular}} &
\textbf{\begin{tabular}[c]{@{}c@{}}Time\\  Optimality\end{tabular}} &
\textbf{\begin{tabular}[c]{@{}c@{}}Collision\\  Check\end{tabular}} \\ \hline
\begin{tabular}[c]{@{}c@{}}Position\\  Space\end{tabular} & offline & no           & \textbackslash{} & no           & fast          \\ \hline
\begin{tabular}[c]{@{}c@{}}State\\  Space\end{tabular}    & online  & yes          & check            & no           & slow          \\ \hline
\begin{tabular}[c]{@{}c@{}}Velocity\\  Space\end{tabular} & online  & yes          & check            & no           & slow          \\ \hline
\begin{tabular}[c]{@{}c@{}}Control\\  Space\end{tabular}  & online  & yes          & satisfy          & no           & slow          \\ \hline
\textbf{Proposed}                                         & offline & \textbf{yes} & \textbf{satisfy} & \textbf{yes} & \textbf{fast} \\ \hline
\end{tabular}%
}
\end{table}


Motion primitive library in autonomous navigation usually generates multiple paths/trajectories using different sampling methods, which can find a feasible path/trajectory with low computing power in deterministic time.
Zhang et al. \cite{zhang2020falco} propose a motion primitive library for offline sampling fixed-length paths in position space. The collision checking is fast, but the paths lack dynamical information, making it difficult to exert the robot's mobility.
In the state space-based method, Muller et al. \cite{mueller2015computationally} online sample different local end states and generate multiple fixed duration min-jerk trajectories in combination with the start state. Afterwards, dynamical feasibility check is carried out to guarantee the robot’s dynamical constraints. 
%However, this method is used to carrying a flying ball, without considering any obstacles.
Ryll et al. \cite{ryll2019efficient} adopt Muller's method to quadrotor autonomous navigation, which can adjust the local end state of sampling based on obstacle density, but requires high-frequency replanning to maintain trajectory safety.
Bucki et al. \cite{bucki2020rectangular} propose a new pyramid partitioning method, which can speed up collision checking and save computing power, but leads to more conservative trajectories and poor performance in obstacle-dense scenarios.
In the velocity space-based method, Yang et al. \cite{yang2021intention} propose sampling different local end states in the velocity space $a=\{v_x, v_z, \omega\}$ online, then combining the start state to generate multiple fixed-duration 8-th order polynomial trajectories and integrating to provide position information.
Collins et al. \cite{collins2020efficient} propose adaptively sampling the end state based on the start speed, which enhances the trajectory library's reliability but requires more computing power to select the feasible trajectory.
In the control space-based method, Florence et al. \cite{florence2020integrated} propose to sample a set of constant control variables online and integrate them forward with a fixed duration to form a trajectory library. Trajectories satisfy dynamical constraints without extra feasiblity check.

The above methods don't consider the trajectory's time optimality. The state/velocity/control space-based methods usually need to sample all trajectories and 
query each possible collision, resulting in low efficiency.
In this work, the proposed method considers the trajectory's dynamical feasibility and time optimality, and introduces faster collision checking.
\end{comment}


\subsection{Time-optimal Path Parameterization (TOPP)}

TOPP is the problem of traversing a path in the fastest way possible while satisfying constraints. The numerical-based method \cite{pham2014general} directly solves the optimal control problem at each path position, which can be implemented quickly, but is less robust and difficult to deal with velocity bounds.
The optimization-based method \cite{verscheure2009time} robustly solves TOPP as a large convex optimization problem, which can satisfy the convex constraints, but with low solution efficiency.
TOPP-RA \cite{pham2018new} is a robust and efficient solver by solves some small LP problems.
TOPP-RA is mostly used for manipulators, and Ivanovic et al. \cite{ivanovic2022parabolic} use it for quadrotors, but it can only handle problems where the speed at the start and end of the path is zero. 

In this work, the proposed method can parameterize a non-zero start and end speed of the path based on TOPP-RA, which are realized on a quadrotor for the first time to the best of our knowledge.

%