\section{Introduction}
\label{sec:intro}

Predictions can significantly influence everyday life \citep{r18t}, an effect
known as performativity. For instance, traffic predictions can alter people's
daily routes, crime predictions can affect police resource allocation, and
stock price predictions can steer traders' decisions. These changes can lead to
shifts in the underlying data distribution, making the original predictions
less accurate.

To capture these effects, \citet{p20p} introduced the concept of
\emph{performative prediction}. In this framework, a deployed model $\prm$
induces a data distribution $D(\prm)$, which gives rise to a new learning objective, the
\emph{performative risk} $\E_{z \sim D(\prm)}(\ell(\prm, z))$, with $\ell$ being
a loss function and $z$ being a sample from the model-induced distribution.
Recent works have achieved significant progress in optimizing predictions for
performative accuracy \citep[see][for a recent survey]{hardt2023performative}.

While performative risk minimization (PRM) is preferable to standard risk minimization (RM) for the sake of test-time accuracy, the broader impact of PRM remain elusive. In particular, using PRM instead of RM leads to different predictions deployed by the learner and also changes the evolution of the data distribution, and these effects are compounded when deploying multiple models over time. This limited understanding of the impact of PRM on the predictions and distribution may be partially due to the mathematical challenges arizing from analyzing the long-term dynamics of the learned models, in the presense of intricate dependencies of the data distribution on all previous models.

%Understanding performative impact is challenging, as the performative distribution shift is hard to model for general learning problems in the first place. At the same time, such changes in
%distribution are vital to understand, as they may correspond to consequential societal adaptation. Additionally, the distribution shift may have non-trivial effects on other properties of the deployed model, beyond accuracy.

\myparagraph{Contributions} In this work, we initiate the analysis of the broader impact of PRM, by studying a sequential performative mean estimation problem for binary variables, in the presence of linear performative distribution shifts. The simplicity of the learning setup enables us to derive the long-term dynamics of PRM, despite the complicated downstream impact of each deployed prediction on the future data distributions. This in turn allows us to quantify the evolution of the predictions and the data distribution.

Within this model, we formulate two measures of impact. The first measure concerns the model predictions and corresponds to the usual statistical notion of a bias of an estimator. The second measure quantifies the shift in the mean of the binary random variable, relative to the mean in the case of lack of performative effects, and thus allows us to understand the evolution of the data distribution under PRM. 

We analyze PRM and the two measures in a one-period (single model deployment) and an infinite horizon (sequential model deployment) setting. In each case, we first study a full
information setting, where all problem parameters (e.g. strength of performativity and initial distribution) are known to the model provider, in
order to isolate the effects of performativity from exploration. We then analyze performativity and exploration jointly via theory and simulations.

Our results indicate that, compared to RM, PRM may select more biased estimators and/or ones that shift the mean to extreme values. This happens in particular because minimizing the PRM loss suggests trading-off the usual mean squared error (MSE) for reduced aleatoric uncertainty in the future data distribution. Such effects occur when the distribution responds positively to model
predictions or when the distribution responds negatively, but the model is
updated rapidly and the performativity is high.

Finally, we use two example scenarios to interpret our measures and technical results in a social context. 
