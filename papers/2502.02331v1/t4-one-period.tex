\section{One-Period Model}
\label{sec:one_period}

This section analyzes the case of $T=1$. First, we discuss the full information
case where $p_0$, $\alpha$, $\lambda$, and $\pi$ are known to the model
provider. This allows us to separate the effects of PRM
from the hardness of designing of algorithms that achieve PRM (due to exploration/finite-sample effects). Next, we assume that the initial mean $p_0$ is unknown and study how this
uncertainty affects our previous results. Finally, in an episodic RL
setting, we study via simulations the case where no information about the
parameters is available. 

Notice that the slow deployment case for $T=1$, which
is the main focus of this section, corresponds to the standard setting of
\cite{p20p}.

\subsection{Perfect Information}

\subsubsection{Slow Deployment}
\label{sec:one-slow}

\begin{proposition}[Proof in \cref{sec:proof-one-slow-sol}]
    \label{thm:one-slow-sol}
    The solution to the problem (\ref{eq:opt-cont-prob}) in the $T=1$ slow deployment case is
    \[
        \prm^*_0 =
        \begin{cases}
            \clip\Par[\big]{\frac{(1 - \abs{\alpha}) s_1}{1 - 2 \alpha},
            -\frac{1}{2}, \frac{1}{2}}, & 1 - 2 \alpha > 0,\\
            \sign(s_1) / 2, & 1 - 2 \alpha \le 0.
        \end{cases}
    \]
    Whenever $\abs{\prm^*_0} \neq 1 / 2$, we get
    \[
        p^*_1 = \frac{(1 - \alpha) (1 - \abs{\alpha})}{1 - 2 \alpha} s_1 =
        \frac{1 - \alpha}{1 - 2 \alpha} (\beta p_0 + (1 - \abs{\alpha} - \beta)
        \pi).
    \]
\end{proposition}

We visualize this solution in \cref{fig:one-slow-sol}. For the rest of the subsection we assume that $\prm^*_0 \neq 1/2$.

\begin{figure}[ht]
    \centering
    \input{fig1-one-slow.pgf}
    \caption{The dependence of $\prm^*_0$ (blue), $p^*_1$ (orange), and $s_1$
    (green) on $p_0$ for $\lambda = 0.8$ and $\pi = 0.2$ in slow $T=1$ case.
    Columns correspond to the different $\alpha$.}
    \label{fig:one-slow-sol}
\end{figure}

\myparagraph{Loss} We get that
\[
    \loss^*_0 =
    \begin{cases}
        \frac{1}{4} - \frac{(1 - \abs{\alpha})^2 s_1^2}{1 - 2 \alpha},
        & \abs{s_1} < \frac{\nicefrac{1}{2} - \alpha}{1 - \abs{\alpha}},\\
        \frac{1 - \alpha}{2} - (1 - \abs{\alpha}) \abs{s_1},
        & \abs{s_1} \ge \frac{\nicefrac{1}{2} - \alpha}{1 - \abs{\alpha}}.
    \end{cases}
\]

\myparagraph{Bias} We have that $$\bias^*_0 = \frac{\alpha (1 - \abs{\alpha}) s_1}{1 - 2 \alpha}.$$
Thus, the PRM path is generally biased. This bias does not arise from the usual
bias-variance trade-off in statistics. Instead, the performativity incentivizes
the model provider to reduce the uncertainty in the distribution. To see this, notice that the unbiased predictor, which minimizes the error term in MSE
(\ref{eq:mse}), exists: $\prm^u_0 = \frac{1 - \abs{\alpha}}{1 - \alpha} s_1$.
For positive feedback, the prediction is biased towards extreme values
(i.e., $-1/2$ or $1/2$). For negative, the prediction is biased towards
$0$. The absolute value of bias increases in $\abs{\alpha}$ if $\alpha >
-\frac{\sqrt{3} - 1}{2}$.

\myparagraph{Shift} Once $\prm^*_0$ is deployed, it induces shift $$\shift^*_1 = \frac{\alpha - \abs{\alpha} + \alpha \abs{\alpha}}{1 - 2\alpha} s_1.$$
The direction of the shift depends on $\sign(\alpha)$ in the same way as the
bias. The effect increases with $\abs{\alpha}$. While a no-shift prediction
exists, $\prm^0_0 = \sign(\alpha) s_1$, it differs from the unbiased prediction
in the negative feedback case. This discrepancy shows that unbiasedness and the
absence of shift can not be achieved simultaneously in the negative feedback
case.

\myparagraph{Discussion} We can see that, in general, the PRM prediction is biased (even though the
model provider has perfect information about the distribution), and its impact
on the mean of the distribution is not zero. In the positive feedback case, the
model provider benefits from shifting the mean to extreme values. Even though
this strategy increases the error term, it decreases the uncertainty. In the
negative feedback case, the performative response to the unbiased prediction
shifts the mean closer to $0$. So, the provider employs a biased prediction to
reduce this drop in the uncertainty.

\myparagraph{Comparison with Naive Path} Now, we consider the naive path, where, for the clarity of exposition, we
assume that the system was initially at equilibrium, i.e., $p_0 = \pi =
s_1$. We get
\[
    \begin{split}
        p^n_1 &= (1 + \alpha - \abs{\alpha}) s_1,\\
        \loss^n_0 &= 1/4 - (1 + 2 \alpha - 2 \abs{\alpha}) s_1^2,\\
        \shift^n_1 &= -\bias^n_1 = (\alpha - \abs{\alpha}) s_1.
    \end{split}
\]
If $\alpha > 0$, the bias and shift of the naive path is zero, which might be
more desirable compared to the PRM path. However, the naive loss is worse
than the PRM loss by $\frac{\alpha^2}{1 - 2 \alpha} s_1^2$.  At the same
time, if $\alpha \le 0$, the bias and shift of the naive path are higher in
absolute values than the bias and shift of the PRM path, i.e. RM increases our measures in the negative feedback case compared to PRM.
Moreover, in the negative case, the loss penalty increases to $\frac{9
\alpha^2}{1 - 2 \alpha} s_1^2$.

\begin{figure*}[ht]
    \includegraphics[width=\linewidth]{figures/bias_shift.png}
    \caption{The dependence of $\bias(\hat{\prm}^*_0)$ (left) and
    $\shift(\hat{\prm}^*_0)$ (right) and corresponding variances on $p_0$. The
    upper row corresponds to $\alpha=0.3$, the lower row corresponds to
    $\alpha=-0.4$. Columns correspond to the different $m$.}
    \label{fig: bias_shift}
\end{figure*}

\subsubsection{Rapid Deployment}

\myparagraph{One-Period Case} \cref{eq:opt-cont-prob} reduces to
\[
    \min_{\prm_0 \in [-\nicefrac{1}{2}, \nicefrac{1}{2}]} \prm_0^2 - 2 \prm_0
    p_0,
\]
which results in $\prm^*_0 = p_0$. (We visualize this solution in
\cref{fig:fin-sols}, top row, in Appendix.) Additionally, we get
\[
    \begin{split}
        p^*_1 &= (\alpha + \beta) p_0 + (1 - \abs{\alpha} - \beta) \pi,\\
        \bias^*_0 &= 0,\\
        \shift^*_1 &= (\alpha - \abs{\alpha} \lambda) p_0 - \abs{\alpha} (1 -
        \lambda) \pi.
    \end{split}
\]
If $\alpha > 0$, the PRM prediction shifts the mean closer to $p_0$
relative to $\pi$. If $\alpha < 0$, the effect is hard to interpret. We
only consider the case of $\pi = p_0$. In this case, $\sign(p_0) \neq
\sign(p^*_1 - s_1)$. The mean is shifted away from $p_0$ in the direction of
$0$. Also note that the absolute value of the shift increases in
$\abs{\alpha}$ under both negative and positive feedback.

\myparagraph{Comparison with Two-Period Case} To see whether the prediction remains unbiased once the distribution changes, we
compare the one- and two-period models. For $T=2$, we get the following two-period
problem:
\[
    \min_{\prm_0, \prm_1, p_1 \in [-\nicefrac{1}{2}, \nicefrac{1}{2}]}
    \sum_{t=0}^1 \gamma^t (\prm_t^2 - 2 \prm_t p_t)
\]
such that $p_1 = \alpha \prm_0 + \beta p_0 + (1 - \abs{\alpha}) (1 - \lambda)
\pi$.

\begin{proposition}[Proof in \cref{sec:proof-two-rapid-sol}]
    \label{thm:two-rapid-sol}
    The solution to the problem (\ref{eq:opt-cont-prob}) in the $T=2$ rapid deployment case is
    \[
        \begin{split}
            \prm^*_0 &= \clip\Par[\Big]{\frac{(1 + \gamma \alpha \beta) p_0 +
            \gamma \alpha (1 - \abs{\alpha} - \beta) \pi}{1 - \gamma \alpha^2},
            -\frac{1}{2}, \frac{1}{2}},\\
            \prm^*_1 &= p^*_1.
        \end{split}
    \]
    Whenever $\abs{\prm^*_0} \neq 1 / 2$, we get
    \[
        p^*_1 = \frac{(\alpha + \beta) p_0 + (1 - \abs{\alpha} - \beta) \pi}{1
        - \gamma \alpha^2}.
    \]
\end{proposition}

\cref{fig:fin-sols}, middle row, in Appendix visualizes $\prm^*_0$. If $\abs{\prm^*_0} < 1/2$, we get
\[
    \begin{split}
        \bias^*_0 & = \frac{\gamma \alpha (\alpha + \beta)
        p_0 + \gamma \alpha (1 - \abs{\alpha} - \beta) \pi}{1 - \gamma
        \alpha^2},\\
        \shift^*_1 & = \frac{(\alpha \! - \! \abs{\alpha} \lambda \! + \!
        \gamma \alpha^2 \lambda) p_0 - (\abs{\alpha} \! - \! \gamma \alpha^2)
        (1 \! - \! \lambda) \pi}{1 - \gamma \alpha^2},\\
        \bias^*_1 & = 0.
    \end{split}
\]
Compared to the case of $T=1$, the mean shifts to more extreme values due
to the denominator, and the first-period bias becomes non-zero. However, the
final prediction remains unbiased. The long-term loss incentivizes the
model provider to actively manipulate the mean, even if the short-term loss
suffers from such manipulation.

\myparagraph{Summary} Similarly to the slow case, the bias and shift of the PRM path are
generally not zero and increase in $\abs{\alpha}$. However, in contrast to the
slow case, in the rapid model, the model provider optimizes the uncertainty
term only because of the long-term effects.
