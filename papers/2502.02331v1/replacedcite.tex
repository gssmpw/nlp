\section{Related Work}
\myparagraph{Performative Prediction} In machine learning, performativity is studied within the framework of
\emph{performative prediction}, where the goal is to find a model with good
performance on the distribution that it induces. The setting was introduced by
____ and was inspired by earlier works on strategic classification
____. Numerous works study methods for finding performatively
optimal/stable models ____, see ____ for a recent overview.
____ extend this framework to stateful
environments, where previous model deployments impact the data distribution at
later stages.

In contrast to the works above, we focus on the impact of PRM on the data distribution and on the prediction made by these models. To our awareness, the only work that studies properties beyond accuracy in the context of performative prediction is that of ____, who, however, focus on the
fairness and polarization properties of PRM instead.

\myparagraph{Long-Term Fairness} The line of works on long-term fairness also studies the evolution of
distribution in social contexts. ____
focus on social feedback loops. ____
propose models for performative responses motivated by their learning context.
While these works model performativity, they focus on finding fair models. In
contrast, we focus on performatively optimal algorithms and analyze their
impact on the data distribution and predictions. 

\myparagraph{Instances of Performativity} Performativity arises in many social contexts. Economic agents respond to the
actions of central banks and the government ____. Performative policing
affects the distribution of observed crime rates ____. Traffic
predictions reroute drivers to new areas ____. Recommendation
systems affect the consumption of new content ____. Since performativity is so widespread, it is important to study optimization formulations in such settings and the effects of performatively-optimal solutions on their environment.