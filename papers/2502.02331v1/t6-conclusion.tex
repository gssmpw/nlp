\section{Discussion and Future Work}
\label{sec:conclusion}

%We analyzed how performative risk minimization affects the underlying
%distribution and the actions of the model provider. Generally, we found that
%the performatively optimal path is biased and introduces a non-zero mean shift.
%We linked the unbiasedness of the optimal path with the additional uncertainty
%term in the loss function of model provider, which incentivizes the model
%provider to decrease the aleatoric uncertainty of distribution. To achieve this
%goal, the model provider tries to shift of mean closer to extreme values.
%Moreover, we find that these incentives are especially strong in the case of
%big performativity.
%
%Additionally, we compare the optimal path with the naive path that does not
%account for distribution shifts. In the positive feedback case, we find that
%the bias and shift metrics are usually higher for the optimal path compared to
%the naive path. In the negative feedback case, the shift introduced by the
%naive path is typically higher than the shift introduced by the optimal path.
%However, the value of bias might be lower or higher depending on the setting.
%Overall, it suggest that the optimal path might be undesirable in the positive
%feedback situations where the model provider should give accurate predictions
%that do not change the distribution much.

%We analyzed how performative risk minimization affects the underlying
%distribution and the actions of the model provider, in the context of a binary mean estimation task and linear performative distribution shift.

Our results suggest that the performatively optimal (PRM) path is, in general, biased and introduces a non-zero mean shift. These effects are more expressed when the mean responds positively to model predictions or when it responds negatively, but the model is updated rapidly and the performativity is high. To understand the potential impact of such effects, we now provide two example scenarios and interpret our measures and technical results in a social context.

\myparagraph{Case study: drug efficacy estimation} Consider a scenario in which a company is trying to estimate the effectiveness of a drug they produce against a specific disease. We define our binary random variables as indicators that the drug cures a randomly sampled patient. To model the well-known placebo effect, under which beliefs about the effectiveness of a drug may further increase its positive impact, we assume a positive performative response ($\alpha > 0$). Consider the one-period positive feedback model in Section \ref{sec:one_period}. Then, a positive/negative bias indicates an exaggarated/understated prediction of the average drug efficacy respectively, which may make it harder to find the most effective drug on the market. At the same time, a positive shift indicates a higher drug efficacy due to the placebo effect, which is, of course, desirable for combating the disease.

Our results in Section 4 with $p_0 = \pi$ suggest that whenever $p_0 > 0$ (i.e. the drug is effective to begin with), PRM would lead to a positive bias, i.e. exaggerated prediction on the drug's effectiveness; as well as positive shift and thus increased drug effectiveness due to performativity.

\myparagraph{Case study: traffic prediction} Consider a model provider seeking to predict which of the two roads, A or B, is less busy. We model this by defining the binary random variable as an indicator for the event that road A is less busy. Consider our infinite horizon negative feedback model. Positive or negative bias of PRM corresponds to the model provider redirecting more traffic to road A or B respectively. At the same time, positive or negative shift indicates an increase in the usage of road B or A respectively. The bias is probably an undesirable property of the prediction as it makes some drivers choose a sub-optimal road. At the same time, the shift might be benign or adverse, depending on the context.

In the slow deployment case, the mean usage of roads becomes more equalized (Figure 5, top-right part) compared to the case when no performativity is present, which is intuitively desirable. In the rapid deployment case, if the strength of performativity is small, the usage becomes more equalized (Figure 5, bottom row, third plot). If the performativity is large, the usage oscillates between roads (Figure 5, bottom row, fourth plot), which may be undesirable.
% Thus, our results suggest that one should be careful when applying performative risk minimization in the case of strong performativity. Additionally, for the weak performativity case, we can see that the naive prediction path might be preferable in terms of shift because it makes the distribution more equalized.



\myparagraph{Limitations and future work} In this work, we focus on mean estimation of binary variables only and work under the linear response model (\ref{eqn:linear_response}). This makes the analysis of the long-term dynamics driven by (\ref{eq:opt-cont-prob}) tractable and allows for defining natural metrics of impact and interpreting them in context, such as those in the previous paragraphs. Despite its simplicity, we hope that our model can be qualitatively useful in broader settings. First, the linear response naturally arises as a first-order Taylor approximation for any performative response. Thus, our results may (at least qualitatively) transfer to situations of weak performative response. Second, as noted in the discussion of Lemma 3.2, the error-uncertainty decomposition holds for a broad class of distributions. Thus, we can expect PRM to generally prefer distributions with smaller aleatoric uncertainty. 

We hope that our work will encourage further analysis of the broader impact of PRM . In particular, it would be interesting to
analyze more complex distributions (e.g., in a regression setting) and models
of performative response.

%For example, in the case of multinomial distribution, the model provider would try to concentrate the probability mass on a small subset of outcomes.
