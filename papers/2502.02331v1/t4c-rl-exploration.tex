\subsubsection{RL Simulations}
\label{sec:one-rl}

In this section, we check whether our results in the perfect information case
transfer to the general performative prediction problem with information
restrictions. In this setting, we consider episodic exploration and
additionally assume that $\lambda = 0$ is known to the provider. In this case,
the samples from the second period after deployment allow the model provider to
estimate the performativity parameters.

We implement Algorithm 1 of \citet{l22w} with hyperparameter
$\beta=2^{-8}$ to find the optimal predictions. We visualize the prediction path
of the algorithm in \cref{fig:rl} (left). After some exploration episodes, the
predictions of the model provider and the means of the distribution quickly
converge to the theoretically predicted values, which validates our results in the perfect information case.

\begin{figure}[ht]
    \input{fig4-rl.pgf}
    \caption{The prediction path, $\prm_t$, (blue) the means, $p_t$, (orange)
    and the theoretical predictions of their equilibrium values (red and green,
    respectively) in RL setting over episodes (left) or time (right) for $\pi =
    0.2$, $\alpha = 0.15$, $\gamma = 0.9$, and $m=100$, where $m$ is the number
    of samples observed from test distribution at each step. The left plot
    corresponds to the $T=1$ slow episodic setting, the right plot corresponds
    to the $T=\infty$ slow setting.}
    \label{fig:rl}
\end{figure}
