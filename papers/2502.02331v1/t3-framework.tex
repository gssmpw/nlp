\section{Model}

We now discuss the sequential performative prediction framework we study, the specific
instance considered in our analysis and the impact metrics we focus on.

\subsection{Performative Prediction Framework}
\label{subsec:perf_pred_framework}

\myparagraph{Optimization Problem} This work analyzes how optimizing for performative accuracy influences the model provider actions and
the underlying probability distribution. To answer this question, we consider
the following sequential performative prediction problem inspired by
\citet{p20p,b22p,r22d}. Denote by $\Prm$ the space of models, by $\prm_t \in
\Prm$ the model parameters at time $t$, by $\mathcal{D}$ the space of all data
distributions on a data space $\mathcal{Z}$, by $D_t \in \mathcal{D}$ the data
distribution after the response to $\prm_{t-1}$, and by $\timeop\colon
\mathcal{D} \times \Prm \to \mathcal{D}$ the model of performative response,
such that $D_t = \timeop(D_{t-1}, \prm_{t-1})$. The model provider is
interested in minimizing a discounted loss
\begin{equation}
    \label{eq:opt-cont-prob}
    \min_{(\prm_t)_{t=0}^{T-1}} \E_{(\prm_t)_{t=0}^{T-1}}\Par*{\sum_{t=0}^{T-1}
    \gamma^t \E_{\z \sim D_t^\text{test}} (\ell(\prm_t, \z))},
\end{equation}
where $\ell$ is the loss function, $\gamma \in (0, 1)$, $T \in \mathbb{N} \cup
\{\infty\}$, and $\prm_t$ depends only on the information up to time $t$.
We denote the solution to this problem by
$(\prm^*_t)_{t=0}^{T-1}$ and refer to it as the \emph{PRM path}.

\begin{remark}
    Notice that our optimization problem falls within the framework of
    reinforcement learning under partial observability, where $\prm_t$
    corresponds to action and $D_t$ corresponds to state.
\end{remark}

\myparagraph{Test Distribution} Notice that the objective (\ref{eq:opt-cont-prob}) depends on the model of test
distributions $D_t^\text{test}$. In the standard performative setting \citep{p20p},
$D_t^\text{test} = D_{t+1}$, the model is tested in an environment adapted to
it. This property holds when the speed of model deployment is slower than that
of societal adaptation. Thus, we call this case the \emph{slow deployment}
case. For example, drug efficacy estimates can only be updated after
time-consuming clinical trials.

We also consider the case of $D_t^\text{test} = D_t$, when the environment
adapts to the predictions with delay. Such delays arise whenever models
are updated frequently. Therefore, we call this case the \emph{rapid
deployment} case. For example, the predictions of road congestion can be
updated ``on the fly'', so people may not be able to adapt to the latest
predictions.

\subsection{Instance of Performative Problem}
\label{subsec:perf_shift_model}

\myparagraph{Distribution} We assume that $D_t$ describes binary random variables $z \sim D_t$ with mean
$p_t$
\[
    z =
    \begin{cases}
        -1/2, & \text{w.p.} \: 1/2-p_t,\\
        1/2, & \text{w.p.} \: 1/2+p_t.
    \end{cases}
\]
Note that $z$ is a Bernoulli random variable shifted by $1/2$ for mathematical
convenience. For these variables, a positive outcome could mean that a drug is
effective for treating a patient or that a certain route is free from traffic.

\myparagraph{Loss} At time $t$, the model provider deploys $\prm_t \in [-\nicefrac{1}{2},
\nicefrac{1}{2}]$ to minimize mean squared error (MSE) $\ell(\prm_t, z) \defeq
(\prm_t - z)^2$. We denote the expected loss (w.r.t. all randomness) by
\[
    \loss_t \defeq \E(\E_{z \sim D^\text{test}_t}((\prm_t - z)^2)).
\]
We denote the means produced by the PRM path $(\prm^*_t)_{t=0}^{T}$ by $(p^*_t)_{t=0}^T$. We also denote by $p^\text{test}_t$ the mean of the distribution $D^\text{test}_t$. Note that $p^\text{test}_t$ is equal to $p_{t+1}$ and $p_t$ in the slow and
rapid deployment cases, respectively.

We have the following conditional loss decomposition.

\begin{lemma}[Error-Uncertainty Tradeoff]
    \label{lemma: bias-variance}
    The mean squared error of $\prm_t$ on $D^{\text{test}}_t$ is
    \begin{equation}
        \label{eq:mse}
        \E\Par{(\prm_t - z)^2 \given \prm_t, p^\text{test}_t} = (\prm_t -
        p^\text{test}_t)^2 + (\nicefrac{1}{4} - (p^\text{test}_t)^2).
    \end{equation}
\end{lemma}

Here, the first term corresponds to the standard mean squared error (MSE).
The second term corresponds to the aleatoric uncertainty of $D^\text{test}_t$
\citep[note that such decompositions are valid for a big class of
distributions,][]{g22e}. Thus, under performativity, the model provider is also incentivized to decrease the environment uncertainty, while in the
non-performative case they only minimize the MSE.

\myparagraph{Performative Response} Performativity might have different manifestations in different contexts. For example,
the estimates of route congestion might have \emph{negative feedback} on the
estimated variable, i.e., when the model predicts that one route is less
congested than others, people might use it more. On the
other hand, in the context of medical trials, predictions might have
\emph{positive feedback} due to the well-known placebo effect.

We capture these effects using a linear response model
\begin{equation}
\label{eqn:linear_response}
    p_{t+1} \defeq \alpha \prm_t + (1 - \abs{\alpha}) s_{t+1},
\end{equation}
where $s_{t+1} \defeq \lambda p_t + (1 - \lambda) \pi$, $\alpha \in (-1, 1)$,
$\lambda \in [0, 1)$, and $\pi \in [-1/2, 1/2]$. Here, $s_{t+1}$ is the next
period mean in the absence of performativity, $\alpha$ controls the strength
and direction of performativity, $\lambda$ controls the friction in the
distribution update, and $\pi$ is the equilibrium (long-term) mean in the
absence of model influence. Positive $\alpha$ describes positive feedback
situations. Negative $\alpha$ describes negative feedback situations. We also
use the notation $\beta \defeq (1 - \abs{\alpha}) \lambda$, under which $p_{t+1} = \alpha \prm_t + \beta p_t + (1 - \abs{\alpha} - \beta) \pi$.

\subsection{Measuring the impact of PRM}
\label{subsec:measures_of_impact}

In our model, the distribution is determined by its mean. This property allows
us to formulate two natural measures of ``impact'' of PRM, \textit{bias} and
\textit{mean shift}. The bias is a notion of \textit{impact of PRM on the predictions} made by the learner. The mean shift describes the
\textit{impact PRM on the underlying distribution}.

\myparagraph{Bias} Consider an arbitrary path (sequence of predictions) $(\theta_i)_{i=0}^T$. Inspired by the classic notion of bias, at each time $t$ we study the expected error in the
estimate of the mean
\begin{align}
\label{eqn:bias}
    \bias_t \defeq \E\Par{\prm_t - p^\text{test}_t}.
\end{align}
Intuitively, the bias captures how far (on average) are the predictions of the path from the true mean at a given time.

\myparagraph{Mean Shift} Here we compare the mean $p_t$ of the distribution under the path $(\theta_i)_{i=0}^T$ and corresponding mean in the absence of
performativity $p^0_t$ (i.e., when $\alpha = 0$). Formally,
\begin{equation}
    \label{eqn:shift}
    \shift_t \defeq \E\Par{p_t - p^0_t}.
\end{equation}
The mean shift measures the amount (and direction) of deviation of the mean of the distribution under the considered path $(p_i)_{i=0}^T$, compared to mean $p^0_t$ at time $t$ if the distribution was not affected by the predictions.

\myparagraph{Analyzing the impact of PRM} To understand the consequences of adaptating to performativity via PRM, we will study the bias and shift of the PRM path $(\prm^*_t)_{t=0}^{T-1}$, which we denote by $\bias^*_t$ and $\shift^*_t$ respectively. Additionally, we
compare the PRM path to a \emph{naive} path, which ignores the distribution
shifts when making predictions. Formally, the naive path $\prm^n_t$ is defined as the mean of the
previously observable distribution
\begin{equation}
    \label{eqn:naive_path}
    \prm^n_t \defeq p^\text{test}_{t-1}.
\end{equation}
This corresponds to the usual approach to prediction in which one minimizes the loss with respect to the currently observable distribution (akin to the usual ERM principle). In the rapid case, where no distribution is observed in the first period, we define $\prm^n_0 = 0$. We will denote the bias and shift under the naive path by $\bias^n_t$ and $\shift^n_t$ respectively. 

\myparagraph{Interpreting bias and shift} In Sections \ref{sec:one_period} and \ref{sec:infinite_horizon}, we will derive explicit formulas for our impact measures under the PRM and naive paths and hence reason about the quantitive behaviour of these two metrics due to performativity. In general, high bias can be interpreted as an undesirable property, even from a solely statistical standpoint \cite{young2005essentials}. However, as we will see, the PRM path may be biased due to the trade-off with distribution uncertainty (Lemma \ref{lemma: bias-variance}). In contrast, an interpretation of the mean shift is usually context-dependent.

To further illustrate these points, we provide two example scenarios where we interpret our measures and technical results, in Section \ref{sec:conclusion}.
