\section{Proofs}
\label{sec:proofs}

\subsection{Proof of Lemma \ref{lemma: bias-variance}}
\begin{proof}
    Using conditional expectation we have
    \begin{align*}
        \E[(\theta_t - z)^2 \mid \theta_t, p_t^{test}] &= \E[\theta_t^2 - 2
        \theta_t  z + z^2 \mid  \theta_t, p_t^{test}]\\
        &= \theta_t^2 - 2 \theta_t \E[z \mid \theta_t, p_t^{test}] + \E[z^2
        \mid \theta_t, p_t^{test}]\\
        &= \theta_t^2 - 2 \theta_t p_t^{test} + \frac14\\
        &= (\theta_t^2 - p_t^{test})^2 + \frac14 - (p_t^{test})^2
    \end{align*}
\end{proof}

\subsection{Proof of Proposition \ref{thm:one-slow-sol}}
\label{sec:proof-one-slow-sol}

By direct calculation,
\[
    \prm_0^2 - 2 \prm_0 (\alpha \prm_0 + (1 - \abs{\alpha}) s_1) = (1 - 2 \alpha) \prm_0^2
    - 2 (1 - \abs{\alpha}) s_1 \prm_0.
\]

If the parabola above opens downwards, $1 - 2 \alpha \le 0$, it achieves
minimum at the extreme point of the domain. By analyzing both extreme points,
we get $\prm^*_0 = \sign(s_1) / 2$.

If the parabola opens upwards, $1 - 2 \alpha > 0$, it achieves the minimum at
the point in our domain closest to the vertex point. Thus, $\prm^*_0 =
\clip\Par*{\frac{(1 - \abs{\alpha}) s_1}{1 - 2 \alpha}, -\frac{1}{2},
\frac{1}{2}}$.

\subsection{Proof of Proposition \ref{thm:two-rapid-sol}}
\label{sec:proof-two-rapid-sol}

Going backward, we get
\[
    \prm^*_1 = p_1,
\]
which results in the following problem
\[
    \min_{\prm_0, \prm_1, p_1} \prm_0^2 - 2 \prm_0 p_0 - \gamma p_1^2 \:
    \text{s.t.} \: p_1 = \alpha \prm_0 + (1 - \abs{\alpha}) (\lambda p_0 + (1 -
    \lambda) \pi), \prm_t \in [\nicefrac{-1}{2}, \nicefrac{1}{2}].
\]
Similarly to \cref{sec:proof-one-slow-sol}, we get
\[
    \prm^*_0 = \clip\Par*{\frac{(1 + \gamma \alpha \beta)
    p_0 + \gamma \alpha (1 - \abs{\alpha}) (1 - \lambda) \pi}{1 - \gamma
    \alpha^2}, -\frac{1}{2}, \frac{1}{2}}.
\]
(Notice that $1 - \gamma \alpha^2 > 0$, which reduces the number of cases.)




\subsection{Proof of Theorems \ref{theorem: expected_loss} and \ref{theorem:
expected_loss_full}}

Before computing the expected loss, we first show the following result
regarding the first two moments of the performative estimator.

\begin{lemma}[Moments of the Performative Estimator] \label{lemma: moments}
    For the performative estimator $\hat{\theta}_0^\ast$, we have that the
    first two moments are given by
    \begin{equation*}
        \E[\hat{\theta}_0^\ast] = 
        \begin{cases}
            \frac{(1 - |\alpha|) p_0}{1-2\alpha} & \alpha \in (-1, 0]\\
            \frac12 - F_{m, p_0 + \frac12} (\frac{m}{2}) & \alpha \in [0.5,
            1)\\
            \sum_{x \in I} \big(\frac{1 - \alpha}{1-2\alpha}\big)
            \big(\frac{x}{m} - \frac12 \big) p(x) & \\
            + \frac12  - \frac12 F_{m, p_0 + \frac12}\big( \frac{2 -
            3\alpha}{2-2\alpha}m \big) & \\
            - \frac12 F_{m, p_0 + \frac12}\big( \frac{\alpha }{2-2\alpha}m\big)
            & \alpha \in (0, 0.5)
        \end{cases}
    \end{equation*}
    and 
    \begin{equation*}
        \E[(\hat{\theta}_0^\ast)^2] = 
        \begin{cases}
            \big( \frac{1-|\alpha|}{1-2\alpha}\big)^2 \big( \frac{0.25 -
            p_0^2}{m} + p_0^2 \big) & \alpha \in (-1, 0]\\
            \frac14 & \alpha \in [0.5, 1)\\
            \sum_{x \in I} \big(\frac{1 - \alpha}{1-2\alpha}\big)^2
            \big(\frac{x}{m} - \frac12 \big)^2 p(x)  & \\
            + \frac14  - \frac14 F_{m, p_0 + \frac12}\big( \frac{2 -
            3\alpha}{2-2\alpha}m \big) & \\
            + \frac14 F_{m, p_0 + \frac12}\big( \frac{\alpha }{2-2\alpha}m\big)
            & \alpha \in (0, 0.5)
        \end{cases}
    \end{equation*}
    where $I$ is the set of integers in $\big(\frac{\alpha m}{2-2\alpha},
    \frac{(2-3\alpha)m}{2-2\alpha} \big]$, $F_{m, p_0 + \frac12} (x) :=
    \sum_{k=0}^{\lfloor x \rfloor} p(x),$ and
    \begin{equation*}
        p(x) := \binom{m}{x} \bigg(\frac12 + p_0\bigg)^x \bigg(\frac12 -
        p_0\bigg)^{m-x}
    \end{equation*}
\end{lemma}
\begin{proof}[Proof of Lemma \ref{lemma: moments}]
    Recall that $\hat{\theta}_0^\ast$ is given by
    \[
        \prm^*_0 =
        \begin{cases}
            \clip\Par[\big]{\frac{(1 - \abs{\alpha}) \bar{p}_0}{1 - 2
            \alpha},
            -\frac{1}{2}, \frac{1}{2}}, & 1 - 2 \alpha > 0,\\
            \sign(\bar{p}_0) / 2, & 1 - 2 \alpha \le 0.
        \end{cases}
    \]
    We consider three cases for the value of $\alpha$: 
    \begin{enumerate}[(i)]
        \item $\alpha \in (-1, 0]$

            In this case we have 
            \begin{equation*}
                \prm^\ast_0 = \frac{1 - |\alpha|}{1-2\alpha} \bar{p}_0
            \end{equation*}
            and therefore 
            \begin{equation*}
                \E[\prm^\ast_0] = \frac{1 - |\alpha|}{1-2\alpha}
                \bar{p}_0, \quad \E[(\prm^\ast_0)^2] = \bigg( \frac{1 -
                |\alpha|}{1-2\alpha}\bigg)^2\E[\bar{p}_0^2] = \bigg(
                \frac{1 - |\alpha|}{1-2\alpha}\bigg)^2\bigg( p_0^2 +
                \frac{\frac14 - p_0^2}{m}\bigg),
            \end{equation*}
            where we have used that $p_{0, i} \sim D_0$ for $i=1, \dots, m$,
            and thus $p_{0,i} + \frac12$ follows a Bernoulli distribution with
            parameter $p_0 + \frac12$.

        \item $\alpha \in [0.5, 1)$

            In this case, we have that 
            \begin{equation*}
                \prm^\ast_0 = 
                \begin{cases}
                    \frac12 & \bar{p}_0 \ge 0\\
                    -\frac12 & \bar{p}_0 <  0.
                \end{cases}
            \end{equation*}
            Since $\bar{p}_0 = \bar{q} - \frac12$, where
            $\bar{q} := \frac{1}{m}\sum_{i=1}^m q_i$ and $q_i := p_{0,
            i}$, so that $q_i \sim Bern(p_0 + \frac12)$, we know that the
            events can be written as
            \begin{align*}
                \{ \bar{p}_0 \ge 0 \} = \{ \bar{q} \ge 0.5 \}, \quad
                \{ \bar{p}_0 < 0 \} = \{ \bar{q} < 0.5 \}.
            \end{align*}
            Therefore, 
            \begin{align*}
                \prm^\ast_0 = \frac12 \chi_{\{ \bar{q} \ge 0.5 \}} -
                \frac12 \chi_{\{ \bar{q} < 0.5 \}}.
            \end{align*}
            Finally, using the law of total expectation, we get that
            \begin{align*}
                \E[\prm^\ast_0] &= \E[\prm^\ast_0 | \bar{q} \ge 0.5]
                \Pr[\bar{q} \ge 0.5] + \E[\prm^\ast_0 |\bar{q}
                < 0.5] \Pr[\bar{q} < 0.5]\\
                &= \frac12 \Pr[\bar{q} \ge 0.5] - \frac12
                \Pr[\bar{q} < 0.5]\\
                &= \frac12 - F_{m, p_0 + \frac12}(0.5m),
            \end{align*}
            where we have used that $m \bar{q} \sim Bin(m ,p_0 + 0.5)$.
            Similarly for the second moment
            \begin{align*}
                \E[(\prm^\ast_0)^2] &= \E[(\prm^\ast_0)^2 | \bar{q} \ge
                0.5] \Pr[\bar{q} \ge 0.5] + \E[(\prm^\ast_0)^2
                |\bar{q} < 0.5] \Pr[\bar{q} < 0.5]\\
                &= \frac14 \Pr[\bar{q} \ge 0.5] + \frac14 \Pr[\bar{q}
                < 0.5]\\
                &= \frac14.
            \end{align*}
        \item $\alpha \in (0, 0.5)$

            In this case we have 
            \begin{align*}
                \prm^\ast_0 &= 
                \begin{cases}
                    \frac{1 - \alpha}{1-2\alpha} \bar{p}_0, & \text{if }
                    \bar{p}_0 \in \big(- \frac{1-2\alpha}{2 - 2\alpha},
                    \frac{1-2\alpha}{2 - 2\alpha} \big] \eqdef A\\
                    \frac12 , & \text{if } \bar{p}_0 > \frac{1-2\alpha}{2
                    - 2\alpha} \eqdef B\\
                    -\frac12 , & \text{if } \bar{p}_0 \le -
                    \frac{1-2\alpha}{2 - 2\alpha} \eqdef C
                \end{cases}
            \end{align*}
            where we have denoted by $A,B,C$ the random events that we have not
            clipped the value of the performative estimator, that we have
            clipped it from above or that we have clipped in from below. Using
            the law of total expectation, we have
            \begin{align*}
                \E[\prm^\ast_0] &= \E[\prm^\ast_0 | A] \Pr[A] + \E[\prm^\ast_0
                | B] \Pr[B] + \E[\prm^\ast_0 | C] \Pr[C]\\
                &= \E[\prm^\ast_0 \chi_{A}] + \frac12 \Pr[B] - \frac12 \Pr[C]\\
                &= \E[\prm^\ast_0 \chi_{A}] + \frac12 \Pr\bigg[\bar{q} >
                \frac{2-3\alpha}{2-2\alpha}\bigg] - \frac12
                \Pr\bigg[\bar{q} \le \frac{\alpha}{2-2\alpha}\bigg]
            \end{align*}
            The first term can be computed as follows
            \begin{align*}
                \E[\prm^\ast_0 \chi_{A}] &= \sum_{x \in I} \frac{1 -
                \alpha}{1-2\alpha} \bigg( \frac{x}{m} - \frac12 \bigg) p(x),
            \end{align*}
            where we have used that $m\bar{p}_0 + m/2 \sim Bin(m, p_0 +
            \frac12)$ and have denoted by $p(x)$ the PMF of $Bin(m, p_0 +
            \frac12$. The last two terms are easily expressed via the CDF of
            the same distribution, giving us that
            \begin{align*}
                \E[\prm^\ast_0] = \sum_{x \in I} \bigg(\frac{1 -
                \alpha}{1-2\alpha}\bigg) \bigg(\frac{x}{m} - \frac12 \bigg)
                p(x)
                + \frac12  - \frac12 F_{m, p_0 + \frac12}\bigg( \frac{2 -
                3\alpha}{2-2\alpha}m \bigg)
                - \frac12 F_{m, p_0 + \frac12}\bigg( \frac{\alpha
                }{2-2\alpha}m\bigg).
            \end{align*}
            where $I$ is the set of integers in the interval $( \frac{\alpha
            }{2 - 2\alpha}m, \frac{2-3\alpha}{2 - 2\alpha} m]$. Similarly, for
            the second moment we have that
            \begin{align*}
                \E[(\prm^\ast_0)^2] &= \E[(\prm^\ast_0)^2 | A] \Pr[A] +
                \E[(\prm^\ast_0)^2 | B] \Pr[B] + \E[(\prm^\ast_0)^2 | C]
                \Pr[C]\\
                &= \E[(\prm^\ast_0)^2 \chi_{A}] + \frac14 \Pr[B] + \frac14
                \Pr[C]\\
                &= \E[(\prm^\ast_0)^2 \chi_{A}] + \frac14 \Pr\bigg[\bar{q}
                > \frac{2-3\alpha}{2-2\alpha}\bigg] - \frac12
                \Pr\bigg[\bar{q} \le \frac{\alpha}{2-2\alpha}\bigg]\\
                &= \sum_{x \in I} \bigg(\frac{1 - \alpha}{1-2\alpha}\bigg)^2
                \bigg(\frac{x}{m} - \frac12 \bigg)^2 p(x)
                + \frac14  - \frac14 F_{m, p_0 + \frac12}\bigg( \frac{2 -
                3\alpha}{2-2\alpha}m \bigg)
                + \frac14 F_{m, p_0 + \frac12}\bigg( \frac{\alpha
                }{2-2\alpha}m\bigg),
            \end{align*}
            which finishes the proof.
    \end{enumerate}
\end{proof}

Now, we are ready to present the full proof.

\begin{proof}[Proof of Theorems \ref{theorem: expected_loss} and \ref{theorem:
    expected_loss_full}]
    We begin by rewriting the expected loss as follows 
    \begin{align*}
        \E[(\theta_0 - z_0)^2] &= \E[\E[\theta_0^2 - 2\theta_0 z_0 + z_0^2 |
        \theta_0]]\\
        &= \E[\theta_0^2 - 2 \theta_0 \E[z_0 | \theta_0] + \E[z_0^2 |
        \theta_0]]\\
        &= \E\bigg[\theta_0^2 - 2 \theta_0 p_1(\theta_0) + \frac14\bigg]\\
        &=(1-2\alpha)\E[\theta_0^2] - 2 (1 - |\alpha|)p_0 \E[\theta_0] +
        \frac14
    \end{align*}
    where the expectation is only in terms of the randomness of the
    observations $\{p_{0, i} \}_{i=1}^m$.

    For the naive estimator, $\hat{\theta}_0^n$, we have that the first two
    moments are
    \begin{align*}
        \E[\hat{\theta}_0^n] & = p_0\\
        \E[(\hat{\theta}_0^n)^2] &= p_0^2 + \frac{(\frac12 - p_0)(\frac12 +
        p_0)}{m},
    \end{align*} 
    which follows since $p_{0,i} \sim D_0$ for $i = 1, \dots, m$. Therefore, we
    get
    \begin{align*}
        \E[(\hat{\theta}_0^n - z)^2] &=  
        (1-2\alpha)\E[\theta_0^2] - 2 (1 - |\alpha|)p_0 \E[\theta_0] +
        \frac14\\
        &= p_0^2 (2 |\alpha| - 2\alpha - 1) + \frac14 + \frac{(2\alpha - 1)(4
        p_0^2 - 1)}{4m}
    \end{align*}

    For the performative estimator, we use the first and second moments of
    $\hat{\theta}_0^\ast$ from \cref{lemma: moments} to obtain
    \begin{align*}
        \E&[(\hat{\theta}_0^\ast - z)^2] =\\
        &\begin{cases}
            \frac{(1 - |\alpha|)^2}{1-2\alpha} \bigg( \frac{\frac14 - p_0^2}{m}
            - p_0^2 \bigg) + \frac14 & \alpha \in (-1, 0]\\
            p_0 (1 - |\alpha|) \big(2F_{m, p_0 + \frac12}(\frac{m}{2}) - 1\big)
            + \frac{1-\alpha}{2} & \alpha \in [0.5, 1)\\
            \sum_{x \in I}((1 - 2\alpha)g(x)^2 - 2(1 - |\alpha|)p_0 g(x)) p(x) +
            (p_0(1-|\alpha|) - \frac{1-2\alpha}{4})F_{m, p_0 +
            \frac{1}{2}}\bigl(\frac{2 - 3\alpha}{2 - 2\alpha}m \bigr) & \\
            + (p_0(1-|\alpha|) + \frac{1-2\alpha}{4})F_{m, p_0 +
            \frac{1}{2}}\bigl( \frac{\alpha m}{2 - 2\alpha} \bigr) - p_0
            (1-|\alpha|) + \frac{1-\alpha}{2}, & \alpha \in (0, 0.5),
        \end{cases}
    \end{align*}
    where $g(x) \defeq (\frac{1-\alpha}{1-2\alpha})(\frac{x}{m} -
    \frac{1}{2})$.

    Asymptotically, as $m \to \infty$, we have that the moments of
    $\hat{\theta}_0^\ast$ for $\alpha \in (-1, 0]$ are given by
    \begin{align*}
        \E[\hat{\theta}_0^\ast] &= \frac{(1-|\alpha|)}{1-2\alpha} p_0 \to
        \frac{(1-|\alpha|)}{1-2\alpha} p_0\\
        \E[(\hat{\theta}_0^\ast)^2] &= \frac{(1-|\alpha|)^2}{(1-2\alpha)^2}
        \bigg( \frac{0.25 - p_0^2}{m} +  p_0^2\bigg) \to
        \frac{(1-|\alpha|)^2}{(1-2\alpha)^2} p_0^2
    \end{align*}
    Similarly, for $\alpha \in [0,5, 1)$, we have 
    \begin{align*}
        \E[\hat{\theta}_0^\ast] &= \frac12 - F_{m, p_0 +
        \frac12}\bigg(\frac{m}{2}\bigg) \to \frac{sign(p_0)}{2}\\
        \E[(\hat{\theta}_0^\ast)^2] &= \frac14 \to \frac14
    \end{align*}
    where we have used that the CDF function $F_{m, p_0 +
    \frac12}\bigg(\frac{m}{2}\bigg)$ converges to $1$ for non-negative $p_0$
    and to $0$ for negative $p_0$ as $m\to \infty$.

    Finally, for $\alpha \in (0, 0.5)$, we have that 
    \begin{align*}
        \E[\hat{\theta}_0^\ast] &= \E\bigg[\clip{\bigg( \frac{1 - |\alpha|
        p_0}{1-2\alpha}, -\frac12, \frac12 \bigg)}\bigg]\\
        &= \E\bigg[ \frac{(1-|\alpha|)\bar{p}_0}{1-2\alpha}
        \chi_{\{\bar{p}_0 \in A\}} \bigg] + \frac12 \Pr[\bar{p}_0 \in
        B  ] - \frac{1}{2} \Pr[\bar{p}_0 \in C ]\\
        &\to \frac{(1-|\alpha|){p_0}}{1-2\alpha}  \chi_{\{{p_0} \in A\}} +
        \frac12 \chi_{[{p_0} \in B  ]} - \frac{1}{2} \chi_{[{p_0} \in C ]}\\
        &= \E[{\theta}_0^\ast \mid \alpha \in (0, 0.5)].
    \end{align*}
    where $A$ denotes the region (a function of $\alpha$), where
    $\hat{\theta}_0^\ast$ has not been clipped, $B$ represents the region where
    it has been clipped from above, and $C$ is the region where it has been
    clipped from below. The third line follows from: (1) the law of large
    numbers, which ensures that $\bar{p}_0 \to p_0$ almost surely as $m
    \to \infty$, and (2) the dominated convergence theorem. The same argument
    applies for $\E[(\hat{\theta}_0^\ast)^2]$. Thus, combining this with the
    other two cases for $\alpha$, we get the following asymptotic results
    \begin{align*}
        \lim_{m \to \infty} \E[\hat{\theta}_0^\ast] = \theta_0^\ast, \quad
        \lim_{m \to \infty} \E[(\hat{\theta}_0^\ast)^2] = (\theta_0^\ast)^2.
    \end{align*}
    Therefore, we can conclude that as $m\to \infty$,
    \begin{equation*}
        \E[(\hat{\theta}_2^\ast - z)^2] \to \verb|loss|_0^\ast.
    \end{equation*}
\end{proof}



\subsection{Proof of Theorem \ref{thm:inf-slow-sol}}
\label{sec:proof-inf-slow-sol}

Consider Lagrangian function
\begin{multline*}
    L(\vec{w}, \vec{q}, \vec{\nu}, \vec{\mu}, \vec{\eta}) \defeq\\
    \sum_{t=0}^\infty \gamma^t (\prm_t^2 - 2 \prm_t p_{t+1}) - (\alpha \prm_t + \beta p_t
    + (1 - \abs{\alpha}) (1 - \lambda) \pi - p_{t+1}) \nu_t - (1 / 2 - \prm_t)
    \mu_t - (\prm_t + 1 / 2) \eta_t.
\end{multline*}
KKT conditions for this infinite-horizon problem \citep[see Section 4.5
of][]{s89r} give
\[
    \begin{aligned}
        0 &= 2 \gamma^t (\prm_t - p_{t+1}) - \alpha \nu_t + \mu_t - \eta_t,\\
        0 &= -2 \gamma^t \prm_t + \nu_t - \beta \nu_{t+1},\\
        0 &= (1 / 2 - \prm_t) \mu_t, \mu_t \ge 0,\\
        0 &= (\prm_t + 1 / 2) \eta_t, \eta_t \ge 0.
    \end{aligned}
\]
Thus, the solution for the case when the restrictions on $\prm_t$ are non-binding
satisfies
\[
    \begin{aligned}
        \prm_{t+1} &= \frac{(1 - 2 \alpha + \gamma \alpha \beta^2)}{\gamma (1 -
        \alpha) \beta} \prm_t - \frac{1 - \gamma \beta^2}{\gamma (1 - \alpha)
        \lambda} s_{t+1} + \frac{\beta (1 - \lambda)}{(1 - \alpha) \lambda}
        \pi,\\
        s_{t+2} &= \alpha \lambda \prm_t + \beta s_{t+1} + (1 - \lambda) \pi.
    \end{aligned}
\]

We get that the optimal path satisfies a first-order linear recurrence relation
for $\prm_t$ and $s_t$. Its characteristic equation follows
\[
    x^2 - \frac{1 - 2 \alpha + \gamma \beta^2}{\gamma (1 - \alpha) \beta} x +
    \frac{1}{\gamma} = 0.
\]
It gives the following eigenvalues
\[
    x_{0,1} = \frac{1 - 2 \alpha + \gamma \beta^2 \pm
    \sqrt{(1 - \gamma \beta^2) ((1 - 2 \alpha)^2 - \gamma \beta^2)}}{2 \gamma
    (1 - \alpha) \beta}.
\]
Notice that the product of these eigenvalues is $1/\gamma$. Thus, one of the
eigenvalues is necessarily bigger than $1$ in absolute value. Due to the
restrictions on $w$, the homogeneous solution corresponding to this eigenvalue
should be zero.

Consider the case of $1 - 2 \alpha \ge \sqrt{\gamma} \beta$, then, the smallest
eigenvalue, $\omega$, satisfies
\[
    \omega = \beta + \frac{(1 - 2 \alpha) (1 - \gamma \beta^2) - \sqrt{(1 -
    \gamma \beta^2) ((1 - 2 \alpha)^2 - \gamma \beta^2)}}{2 \gamma (1 - \alpha)
    \beta} = \beta + \frac{2 \alpha \beta}{1 - 2 \alpha + \xi}.
\]
Corresponding eigenvector gives the following homogeneous solution
\[
    s^h_{t+1} = s \omega^t, \:
    \prm^h_t = \frac{2 (1 - \abs{\alpha})}{1 - 2 \alpha + \xi} r \omega^t.
\]
One of inhomogeneous solutions satisfies
\[
    s^i_{t+1} = \frac{(1 - 2 \alpha - \gamma \beta (1 - \alpha)) (1 - \lambda)
    \pi}{1 - 2 \alpha - \beta + \alpha \beta - \gamma \beta (1 - \alpha -
    \beta)}, \:
    \prm^i_t = \frac{(1 - \gamma \beta) (1 - \abs{\alpha}) (1 - \lambda) \pi}{1
    - 2 \alpha - \beta + \alpha \beta - \gamma \beta (1 - \alpha - \beta)}.
\]
Using the initial conditions, we get the desired solution.



\subsection{Proof of Theorem \ref{thm:inf-rapid-sol}}
\label{sec:proof-inf-rapid-sol}

Consider Lagrangian function
\begin{multline*}
    L(\vec{w}, \vec{q}, \vec{\nu}, \vec{\mu}, \vec{\eta}) \defeq\\
    \sum_{t=0}^\infty \gamma^t (\prm_t^2 - 2 \prm_t p_t) - (\alpha \prm_t + \beta p_t
    + (1 - \abs{\alpha}) (1 - \lambda) \pi - p_{t+1}) \nu_t - (1 / 2 - \prm_t)
    \mu_t - (\prm_t + 1 / 2) \eta_t.
\end{multline*}
KKT conditions for this infinite-horizon problem \citep[see Section 4.5
of][]{s89r} give
\[
    \begin{aligned}
        0 &= 2 \gamma^t (\prm_t - p_t) - \alpha \nu_t + \mu_t - \eta_t,\\
        0 &= -2 \gamma^t \prm_t + \nu_{t-1} - \beta \nu_t,\\
        0 &= (1 / 2 - \prm_t) \mu_t, \mu_t \ge 0,\\
        0 &= (\prm_t + 1 / 2) \eta_t, \eta_t \ge 0.
    \end{aligned}
\]
Thus, the solution for the case when the restrictions on $\prm_t$ are non-binding
satisfies
\[
    \begin{aligned}
        \prm_{t+1} &= \frac{1 + \gamma \alpha \beta}{\gamma (\alpha + \beta)} \prm_t
        - \frac{1 - \gamma \beta^2}{\gamma (\alpha + \beta)} p_t + \frac{\beta
        (1 - \abs{\alpha}) (1 - \lambda)}{\alpha + \beta} \pi,\\
        p_{t+1} &= \alpha \prm_t + \beta p_t + (1 - \abs{\alpha}) (1 - \lambda)
        \pi.
    \end{aligned}
\]

We get that the optimal path satisfies a first-order linear recurrence relation
for $\prm_t$ and $p_t$. Its characteristic equation follows
\[
    x^2 - \frac{1 + \gamma \beta (2 \alpha + \beta)}{\gamma (\alpha + \beta)} x
    + \frac{1}{\gamma} = 0.
\]
It gives the following eigenvalues
\[
    x_{0,1} = \frac{1 + \gamma \beta (2 \alpha + \beta) \pm
    \sqrt{(1 - \gamma \beta^2) (1 - \gamma (2 \alpha + \beta)^2)}}{2 \gamma
    (\alpha + \beta)}.
\]
Notice that the product of these eigenvalues is $1/\gamma$. Thus, one of the
eigenvalues is necessarily bigger than $1$ in absolute value. Due to the
restrictions on $w$, the homogeneous solution corresponding to this eigenvalue
should be zero.

The smallest eigenvalue, $\kappa$, satisfies
\[
    \kappa = \frac{1 + \gamma \beta (2 \alpha + \beta) - \sqrt{(1 - \gamma
    \beta^2) (1 - \gamma (2 \alpha + \beta)^2)}}{2 \gamma (\alpha + \beta)} =
    \beta + \frac{2 \alpha}{1 + \chi}.
\]
Thus, homogeneous solution follows
\[
    q^h_t = q \kappa^t, \: \prm^h_t = \frac{2}{1 + \chi} q \kappa^t.
\]
One of inhomogeneous solutions satisfies
\[
    q^i_t = \frac{(1 - \gamma (\alpha + \beta)) (1 - \abs{\alpha}) (1 -
    \lambda) \pi}{1 - \alpha - \beta - \gamma (\alpha + \beta - \beta (2
    \alpha + \beta))},\:
    \prm^i_t = \frac{(1 - \gamma \beta) (1 - \abs{\alpha}) (1 - \lambda)
    \pi}{1 - \alpha - \beta - \gamma (\alpha + \beta - \beta (2 \alpha +
    \beta))}.
\]


\subsection{Proof of Proposition \ref{thm:two-slow-sol}}
\label{sec:proof-two-slow-sol}

Using the results of \cref{thm:one-slow-sol}, we get
\[
    \prm^*_1 =
    \begin{cases}
        \clip\Par*{\frac{(1 - \abs{\alpha}) s_2}{1 - 2 \alpha}, -\frac{1}{2},
        \frac{1}{2}}, & 1 - 2 \alpha > 0,\\
        \frac{\sign(s_2)}{2}, & 1 - 2 \alpha \le 0,
    \end{cases}
\]
which results in the following loss in the second period:
\[
    (\prm^*_1)^2 - 2 \prm^*_1 p^*_2 =
    \begin{cases}
        -\frac{(1 - \abs{\alpha})^2 s_2^2}{1 - 2 \alpha},
        & 2 (1 - \abs{\alpha}) \abs{s_2} < 1 - 2 \alpha,\\
        \frac{1 - 2 \alpha}{4} - (1 - \abs{\alpha}) \abs{s_2},
        & 2 (1 - \abs{\alpha}) \abs{s_2} \ge 1 - 2 \alpha.
    \end{cases}
\]
Notice that
\[
    (\prm^*_1)^2 - 2 \prm^*_1 p^*_2 \ge - \frac{(1 - \abs{\alpha})^2 s_2^2}{1 -
    2 \alpha}.
\]
Thus,
\[
    \sum_{t=0}^1 \gamma^t (\prm_t^2 - 2 \prm_t p_{t+1}) \ge \prm_0^2 - 2
    \prm_0 p_1 - \frac{\gamma (1 - \abs{\alpha})^2 s_2^2}{1 - 2 \alpha}.
\]
So, if the minimizer of the right hand side satisfies $2 (1 - \abs{\alpha})
\abs{s^{\text{rhs}, *}_2} \le 1 - 2 \alpha$, it will minimize the left-hand
side.

Similarly to \cref{sec:proof-one-slow-sol}, we have that the minimizer of the
right-hand side satisfies
\[
    \begin{split}
        & \prm^{\text{rhs}, *}_0 =\\
        &
        \begin{cases}
            \clip\Par[\big]{\frac{(1 - \abs{\alpha}) ((1 - 2 \alpha + \gamma
            \alpha \beta^2) s_1 + \gamma \alpha \beta (1 - \lambda) \pi)}{(1
            - 2 \alpha)^2 - \gamma \alpha^2 \beta^2}, -\frac{1}{2},
            \frac{1}{2}},
            & 1 - 2 \alpha > \sqrt{\gamma} \abs{\alpha} \beta,\\
            \frac{\sign((1 - 2 \alpha + \gamma \alpha \beta^2) s_1 + \gamma
            \alpha \beta (1 - \lambda) \pi)}{2},
            & 1 - 2 \alpha \le \sqrt{\gamma} \abs{\alpha} \beta.
        \end{cases}
    \end{split}
\]

Thus, when $1 - 2 \alpha > \sqrt{\gamma} \abs{\alpha} \beta$ and $2 (1 -
\abs{\alpha}) \abs{s^*_2} \le 1 - 2 \alpha$, we get
the desired solution to our problem.



\newpage
