\section{Related Work}
%\vspace{-0.5em}

\textbf{Problem Formulations.}
RAP has been modeled in various ways, such as classification task \cite{Zhao2018paperreviewerrecommnedation,Zhang2020multilabel}, topic coverage task \cite{kou2015weighted}, and recommendation \cite{DBLP:journals/corr/abs-0906-4044,peng2017time}. For instance, Zhao et al. \cite{Zhao2018paperreviewerrecommnedation} redefined RAP as a classification task, applying the word mover’s distance method for similarity calculations and the constructive covering algorithm to concurrently classify reviewers and manuscripts. Zhang et al. \cite{Zhang2020multilabel} approached the task as a multilabel classification problem, assigning reviewers based on multiple predicted labels. Conversely, some works have introduced the problem as a topic coverage task \cite{kou2015weighted}.
We suggest modeling RAP as a conventional IR task. In a typical IR task, the objective is to find pertinent documents from a vast collection based on a specific query. Here, the queries are the submitted papers for review, characterized by their metadata. %The collection consists of the previous publications of potential reviewers. %By treating the reviewers' past works as our document collection, we can utilize the similarities between the queries (submitted papers) and documents (reviewers' previous works) to make informed reviewer assignments. 
The task is essentially reformulated as finding authors whose earlier works closely match the content of the submitted papers.

\textbf{Solutions.}
A few efforts have already been undertaken to tackle the reviewer assignment problem. Those methods that have already been deployed in industrial settings are often proprietary in nature and hence lack transparency, making evaluation and comparison of their performance challenging. Other more transparent solutions for handling RAP typically fall into three categories: topic-based modeling techniques, lexical matching methods, and neural network-based embeddings.
Dumais and Nielsen \cite{Susan1992automatedrap} were the first to tackle RAP as an information retrieval challenge, using the latent semantic indexing (LSI) model to link reviewers with papers. With advancements in topic modeling, Mimno and McCallum \cite{Mimno2007ExpertiseMF} adopted the more sophisticated topic modeling algorithms to introduce the author-persona-topic model to more accurately capture the topics a reviewer might cover. Methods for topic modeling, such as Latent Dirichlet Allocation (LDA) \cite{Mimno2007ExpertiseMF,kou2015weighted}, have been extensively employed for reviewer assignment. These techniques aim to discern the hidden topics within a paper and align them with reviewers who possess knowledge in those domains. Kou et al. \cite{kou2015weighted} implemented topic-weighted coverage calculation using LDA features and introduced the branch-and-bound algorithm to identify reviewers rapidly.
Lexical matching strategies, like TF-IDF \cite{peng2017time}, emphasize extracting statistical characteristics from the texts of reviewers and papers to determine the best match. With the rise of neural-based embeddings, efforts have been made to address RAP using both static word2vec and contextualized embeddings \cite{ogunleye2017proposed,devlin2018bert,mikolov2013efficient}. % Although these models appear promising, they often show biased performance, being effective for prevalent topics but facing challenges with long-tail distributions where papers fall into specialized or emerging areas.
Since we are proposing to redefine RAP as an IR problem, in this paper, we focus on reporting IR-based approaches as benchmarks. %Although other approaches might be applied, further investigations of other methods on the \texttt{exHarmony} benchmark is left for future work.


\textbf{Datasets.}
%Data availability is a critical challenge in RAP. While some venues, such as NeurIPS, CIKM, SIGKDD, ICML, and ICLR, have publicly available datasets, these are often the only baselines against which models can be compared.
One of the major obstacles in RAP is the shortage of diverse and large datasets for both training and evaluating current solutions. This lack is partly due to the need to preserve the anonymity of reviewers. In addition, available datasets tend to have a limited range, not having high coverage on different subjects. Moreover, maintaining these datasets is crucial, as they need continuous updates to reflect the latest research developments and reviewer profiles. To the best of our knowledge, there is no dataset available that offers a varied and comprehensive ground truth for RAP. 
Furthermore, there were major problems with previous works \cite{karimzadehgan2008multi,karimzadehgan2009constrained,tang2010expmatching,kou2015weighted,kou2015topicbased,Xu2020Strategyproof,mirzaei2019multiaspect,Nguyen2018ADS}; they all used some conference assignments as the ground truth evaluation data and in fact, those assignments had been done by other RAP solutions whose performances are  unknown and potentially questionable. In other words, earlier works have been evaluated against some unverified anchor. Existing studies that align reviewers with accepted papers as gold standards are few and narrow in scope, often biased towards accepted papers \cite{Xu2020Strategyproof}. Therefore, in this work, we propose the \texttt{exHarmony} dataset, which overcomes the aforementioned challenges by not requiring the collection of reviewer data; instead, we only adopt meta-data from papers, e.g., having paper's title, abstract, their list of authors, and their citations. Since the only requirement for \texttt{exHarmony} is a collection of papers, maintaining it is very easy. It only requires updates with new papers, making it flexible to maintain and keep updated based on author profiles and new research topics.
This characteristic of \texttt{exHarmony} allows it to be large-scale at a very low cost for training and evaluation purposes. This data collection strategy allows us to have data on a diverse set of topics since the pipeline is not limited to data annotation and thus can be applied effortlessly on different topics.
    
%\vspace{-0.5em}