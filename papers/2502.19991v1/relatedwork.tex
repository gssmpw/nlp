\section{Related Work}
\label{sec:bg}
% Here we review previous studies on human-robot handover as a collaborative task and existing approaches to implement autonomous handovers.
As reviewed by Duan~et~al.~\cite{duan2024human}, recent handover research investigates how different methods for trajectory generation, motion planning, or timing control influence the objective handover performance (e.g., duration or success rate) and subjective human perceptions (e.g., trust or fluency). Existing studies focus on either human-to-robot (H2R) handovers (52.5\%) or robot-to-human (R2H) handovers (37.5\%), with limited research on bidirectional handovers (10\%), i.e., exchanging objects. In H2R handovers, previous work focused on predicting human behaviours or intentions under uncertainty with machine learning models (e.g.,~\cite{mavsar2022rovernet, yang2022model}). In R2H handovers, previous work focused on enhancing human comfort and increasing the movement efficiency of both parties~\cite{qin2022task, lagomarsino2023maximising}. 

Existing handover policies may adopt learning-based, control-based, or analysis-based approaches. For example, K{\"a}ppler~et~al.~\cite{kappler2023optimizing} developed an adaptive method for a table-mounted robot arm, which updated the object transfer location based on the location of a user's hand as observed from an RGB-D camera and proximity sensors on the gripper. This adaptive R2H handover model was compared with a non-adaptive model using a pre-defined object transfer location in a between-subject study, in which the robot handed a cup to a participant in four consecutive repetitions. They found that participants exhibited motor learning and adaptation to the robot's handover, which necessitated evaluating handover models in repeated episodes. 
% A custom questionnaire was used to collect participant's perception in addition to measures of handover time and location. The proposed method triggered handover initiation and completion at updated object transfer time and positions based on a participant's behaviours. However, comparison between the repeated episodes also showed that participants learned to adapt to the robot in both adaptive and non-adaptive handovers, with the adaptive method resulting in lower perception of trust, fluency and safety. 
% However, their user study did not incorporate collaborative task context (i.e., purposeful handovers) and the proposed adaptive handover model relied on rules based only on a participant's hand location.
In another example, Kedia~et~al.~\cite{kedia2024interact} developed a transformer-based human intent prediction model pre-trained on human-human collaborative manipulation data and fine-tuned on human-robot data with a teleoperated robot arm. This study incorporated task context in the handover episodes, such as a person and the fixed robot arm taking turns to each pick up one of two objects from a cart to place onto a shared table. However, this work focused on short interaction (3-15s) without direct object transfers between the human and the robot.
Zhuang~et~el.~\cite{zhuang2022goferbot} developed an R2H handover model in which a table-mounted robot arm delivered four legs of an IKEA table to a human to assist in assembly. The handover was either triggered by visual-based recognition of human actions, or by voice command (simulated by a supervisor pressing a button after a participant gave the voice command). They evaluated handover time, success rate, and action recognition performance, as well as participants' perception of the collaboration fluency. This work incorporated collaborative task context in repeated handovers. However, their approach focused on predicting the handover initiation timing, leaving out other timing and location coordination in the whole process.

As shown in the above examples and identified by Duan~et~al.~\cite{duan2024human}, research on complex long-sequence tasks with handover policies capable of spatial-temporal collaboration is extremely limited. Further, as identified by Ortenzi~et~al.~\cite{ortenzi2021object}, there is limited work on adaptive handovers incorporating human social and communicative cues beyond hand locations. Moreover, both reviews identified that pre- and post-handover tasks and HRC context were rarely considered in current handover research. Our previous work~\cite{tian2023crafting} investigated spatial-temporal adaptation in a mix of H2R, R2H, and bidirectional handovers contextualised in an hour-long HRC task of assembling and painting a wooden birdhouse. While the mobile manipulator robot was teleoperated by a hidden human operator (i.e., Wizard-of-Oz), the FACT HRC dataset collected facilitates the development of a collaborative handover policy by learning from the human operator's strategies.