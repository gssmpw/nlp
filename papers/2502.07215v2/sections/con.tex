\section{Conclusion}
\label{sec:con}
We introduce the Prompt Directional Vector (PDV), a simple yet effective approach for enhancing Zero-Shot Composed Image Retrieval. PDV captures semantic modifications induced by user prompts without requiring additional training or expensive data collection. Through extensive experiments across multiple benchmarks, we demonstrated three successful applications of PDV: dynamic text embedding synthesis, composed image embedding through semantic transfer, and effective multi-modal fusion. 

Our approach not only improves retrieval performance consistently, but also provides enhanced controllability through the use of scaling factors. PDV serves as a plug-and-play enhancement that can be readily integrated with existing ZS-CIR methods while incurring minimal computational overhead.

We note that PDV's effectiveness correlates strongly with the underlying method's ability to generate accurate compositional embeddings. This insight suggests promising future research directions, including developing more robust compositional embedding techniques and exploring adaptive scaling strategies for PDV. The simplicity and effectiveness of PDV also open possibilities for its application in multi-prompt composed image retrieval (\ie dialogue-based search) and other multi-modal tasks where semantic modifications play a crucial role.
