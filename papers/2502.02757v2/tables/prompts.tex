
\begin{table*}[!t]
    \centering
    \small 
    \caption{Prompt templates for noisy classification task.}
    \label{tab:prompt_templates}
    \begin{tabular}{p{\textwidth}} %p{10cm}
        \textbf{Prompt} $\|$ \textbf{Template} \\
        \toprule
         \textbf{$P_{\textsc{Definition}}$} $\|$ Your task, as an experienced code reviewer skilled in Java, Python, and C++, is to evaluate review comments generated by other developers on code changes (source or code diffs) submitted in the code review process. Your objective is to discern between ``noisy'' comments and those that are ``valid''. \\ewline 
        Valid comments are those that request code changes aimed at improving the source code. A comment is considered ``valid'' if given the input information (i.e., code diff for the code-to-comment task), it is possible for a human to understand the rationale for the related output. This means that a human evaluator is able to understand what the reviewer's comment refers to (i.e., what the problem in the submitted code diff spotted by the reviewer is). Furthermore, when an instance is valid, the comment usually specifies the types of changes, including: \\ewline 
        - Changes are needed to refactor the code to improve its quality; \\ewline 
        - Changes are needed since tests for this code must be written; \\ewline 
        - Changes are needed to better align this code to good object-oriented design principles; \\ewline 
        - Changes are needed to fix one or more bugs; \\ewline 
        - Changes are needed to improve the logging of its execution; \\ewline 
        - Changes are needed for other reasons not listed above. \\ewline 
        A comment is considered ``noisy'' if it is impossible even for a human to understand the reviewer's comment. This includes the following types of noisy comments: \\ewline 
        - Unclear comment: what the reviewer's comment is actually asking the developer is unclear. This could be because the comments are vague or irrelevant to code improvement. \\ewline 
        - No change asked: the reviewer’s comment does not request any change; \\ewline 
        - Noisy for other reasons not listed above. \\ewline
        Your evaluation should be guided by the following criteria: \\ewline 
        1. \textbf{Relevance to Code Change:} A valid comment should directly relate to the code diff submitted, specifically recommending precise alterations or improvements. You should classify the comment's suggested changes into the types listed above when a comment is identified as valid. Similarly, you should classify the discarded reason into types listed above when a comment is identified as discard. \\ewline 
        2. \textbf{Clarity and Constructiveness:} The comment should clearly articulate any identified issues and offer constructive suggestions for how to address them. \\ewline 
        3. \textbf{Focus on Improvement:} Comments should aim to enhance the code's quality by identifying defects, analyzing their potential impact, and suggesting actionable steps for remediation. \\ewline \\ewline
        This exercise involves a single-round code review process, where your primary goal is to improve the quality of feedback within the review ecosystem by filtering out noisy comments, thereby ensuring that only useful, clear, and constructive feedback is provided to developers for code enhancement. \\ 
        \midrule
        \textbf{$P_{\textsc{Guideline}}$ $\|$} As an experienced code reviewer skilled in multiple programming languages, your role during the code review process is to assess code review comments made by reviewers on code diffs, which are the code changes developers make. Review comments often highlight issues, suggest changes, and sometimes explain the reasoning behind suggested changes. Your primary objective is to determine whether each comment is ``valid'' or ``noisy''. \\ewline \\ewline
        A ``valid'' comment often explicitly requests changes to improve the code, clearly outlining necessary actions, such as refactoring the code to improve code quality (regarding documentation, style, programming conventions and more), writing tests, aligning with good object-oriented design principles, fixing bugs, enhancing logging, or addressing other specific needs. It should be straightforward to discern these actions from the comment, which often involves assessing whether the comment raises an issue or concern relevant to the code diff and identifying the specific actions being asked to improve the code. \\ewline \\ewline
        A comment is considered ``noisy'' if it fails to relate to the code diff, lacks direct and explicit requests for code improvement, or is unclear and nearly impossible to understand. These issues can arise from the comment's irrelevance, requests for mere clarifications, justifications of existing code diff, or other obstructive factors. \\ewline \\ewline
        Human experts classify comments as valid or noisy using the above high-level definition along with detailed guidelines to identify concrete aspects presented in comments and assess whether the comment is valid or noisy. Each aspect represents a potential component of a comment and includes a type name, definition, criteria for being valid, and criteria for being noisy. \\ 
        \\\\---\\ Type: Action Explicitness\\**Definition:** A criterion used to determine the explicitness of a comment regarding the need for a code change. You can extract the key verbs in comments to help you understand whether it is explicit or implicit.\\- **Valid:** A comment is considered valid if the action specified in the comment is direct and explicit, requesting a specific code change. Such comments provide clear suggestions or solutions on what to act on, leaving little room for interpretation. \\- **Noisy:** A comment is considered noisy if the action specified in the comment is indirect or implicit. These comments require further inference or speculation to discern the specific action being requested, leading to difficulty of mapping key information in comments back to code diff and a higher likelihood of misinterpretation.\\\\ Type: Questions in Comments\\**Definition:** This is a common type that frequently occurs in code review comments, where the comment is expressed as a question. Such comments might provide direct suggestions that prompt developers to take action or purely aim to seek information.\\- **Valid:**A comment is valid if it provides concrete suggestions or solutions on how to improve the relevant code. A considerable amount of questions are used to elicit an action from the developer who submitted the code diff. For example, many questions actually suggest alternative solutions. The use of a question rather than an affirmative sentence to recommend a different implementation strategy might indicate the reviewer’s attempt to be polite towards the change author. In some other cases, reviewers directly ask the change authors to perform a specific action using questions.\\- **Noisy:**  A comment expressed as a question is noisy if it doesn't lead to an action for code change or if the action is implicit, as explained below: (a) comments that request further details or clarification without directly prompting a specific course of action; (b) when the action is not explicitly stated and lacks clarity in what is being asked, requiring significant inference or speculation to deduce potential actions, which may vary widely. Evaluate this carefully as not all questions are noisy.\\\\ Type: Issue Scope\\**Definition:** This assesses the relevance regarding the scope of issue pointed out by a reviewer's comment.\\- **Valid:** A comment is valid if the issue scope is relevant to the code diff (code diff) and the action for refining the code is explicit and direct. \\- **Noisy:** A comment is noisy if the issues pointed out in the scope are irrelevant to the actual code changes, or do not provide actionable suggestions. \\\\ Type: Action Scope\\**Definition:** This defines the action or change scope asked in the comment.\\- **Valid:**  A comment is considered valid if the raised issue is relevant to the code diff, and the action being requested is direct and explicit, even if the action scope goes beyond the code diff scope.\\- **Noisy:** A comment is considered noisy if it raises issues outside the scope of the relevant code diff or if the action being requested is indirect and implicit.\\\\ Type: Programming or Package Knowledge  \\**Definition:** This type arises when the reviewer's comment requires programming language specific knowledge or package specific knowledge to understand and implement the requested changes.\\- **Valid:** A comment is valid if it provides direct, actionable suggestions and if the keywords in the comment are relevant to the code diff, file, or specific programming languages used in the code.\\- **Noisy:** A comment is noisy if no associations or connections to the domain-specific knowledge are made clear, making the feedback confusing and hard to act upon.\\\\ Type: Contributors Comments\\- **Definition:** This is a type of comment provided by contributors themselves, where they serve as reviewers.\\- **Valid:** A comment is valid if the change being requested is relevant to the code diff and is direct and explicit to apply, regardless of the source of the rationale or the existence of the rationale.\\- **Noisy:** A comment is noisy if it is a self-justification, i.e., purely clarifying what has already been done in the code and why it was done. You need to read the code and comment carefully, mapping the actions being requested in the comment to the changed lines in the code diff to understand whether it is a self-justification.\\\\ Type: Rationale Source\\- **Definition:** This is often part of a comment where the reasoning or rationale to support the suggestion comes from a scope beyond the code diff. This could be at a file level, from higher software development life cycles, or based on the reviewer's experience or conventions in a specific project or language.\\- **Valid:** A comment is valid if the change being requested is relevant to the code diff and is direct and explicit to apply, regardless of the source of the rationale or the existence of the rationale.\\- **Noisy:** A comment is noisy if it is not relevant to the code diff or if it asks for an implicit and unactionable code change.\\
        

        
        \bottomrule 
    \end{tabular}
    
\end{table*}

