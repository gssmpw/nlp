%% For submission and review of your manuscript please change the
%% command to \documentclass[manuscript, screen, review]{acmart}.
%%
%% When submitting camera ready or to TAPS, please change the command
%% to \documentclass[sigconf]{acmart} or whichever template is required
%% for your publication.

% \documentclass[manuscript,review,anonymous]{acmart} 
\documentclass[sigconf]{acmart} 

\AtBeginDocument{%
  \providecommand\BibTeX{{%
    Bib\TeX}}}


\setcopyright{acmlicensed}
\copyrightyear{2025}
\acmYear{2025}
\setcopyright{cc}
\setcctype{by}
\acmConference[CHI '25]{CHI Conference on Human Factors in Computing Systems}{April 26-May 1, 2025}{Yokohama, Japan}
\acmBooktitle{CHI Conference on Human Factors in Computing Systems (CHI '25), April 26-May 1, 2025, Yokohama, Japan}\acmDOI{10.1145/3706598.3713788}
\acmISBN{979-8-4007-1394-1/25/04}

\newcounter{answernum}
\setcounter{answernum}{0}
\newcommand{\newanswer}[1][]{\refstepcounter{answernum}{#1}\textbf{C{\theanswernum}}:}
\newcommand{\refanswer}[1][]{A{#1}}

\def\eg{{\it e.g.}}
\def\cf{{\it c.f.}}
\def\ie{{\it i.e.}}
\def\etal{{\it et al.}}
\def\etc{{\it etc}}

% \newcommand{\red}{\textcolor[rgb]{0.757,0.153,0.212}}
\newcommand{\red}{\textcolor[rgb]{0,0,0}}

% \newcommand{\rrred}{\textcolor[rgb]{0.757,0.153,0.212}}
\newcommand{\rrred}{\textcolor[rgb]{0,0,0}}

\newcommand{\rred}{\textcolor[rgb]{0.757,0.153,0.212}}
\newcommand{\blue}{\textcolor[rgb]{0.082, 0, 1}}
\newcommand{\green}{\textcolor[rgb]{0.180, 0.518, 0.349}}
\definecolor{lightgray}{rgb}{0.89, 0.89, 0.89}
\usepackage{newunicodechar}
\usepackage[utf8]{inputenc}
\usepackage[T1]{fontenc}    % use 8-bit T1 fonts
\usepackage{pifont}
\usepackage{newunicodechar}
\newunicodechar{✓}{\green{\ding{51}}}
\newunicodechar{✗}{\rred{\ding{55}}}

\usepackage{xcolor}
\usepackage{listings}
\usepackage{multirow}
\usepackage{booktabs}
\begin{document}


\title{WanderGuide: Indoor Map-less Robotic Guide\\for Exploration by Blind People}


\author{Masaki Kuribayashi}
\affiliation{
  \institution{Waseda University}
  \city{}
  \country{}}
\affiliation{
   \institution{Miraikan - The National Museum of Emerging Science and Innovation}
   \city{Tokyo}
\country{Japan}}

\author{Kohei Uehara}
\affiliation{
   \institution{Miraikan - The National Museum of Emerging Science and Innovation}
   \city{Tokyo}
\country{Japan}}

\author{Allan Wang}
\affiliation{
   \institution{Miraikan - The National Museum of Emerging Science and Innovation}
   \city{Tokyo}
\country{Japan}}

\author{Shigeo Morishima}
\affiliation{
  \institution{Waseda Research Institute for Science and Engineering}
  \city{Tokyo}
  \country{Japan}}

\author{Chieko Asakawa}
\affiliation{
   \institution{Miraikan - The National Museum of Emerging Science and Innovation}
   \city{Tokyo}
   \country{Japan}}

\renewcommand{\shortauthors}{Kuribayashi et al.}

% 150words
% Replying to workplace emails that are typically long and require politeness is time-consuming and cognitively demanding.
% Replying to lengthy and polite workplace emails is often time-consuming and cognitively demanding.
% takes time to understand and reply
\red{Replying to formal emails is time-consuming and cognitively demanding, as it requires crafting polite phrasing and providing an adequate response to the sender's demands.}
% \red{Replying to formal emails, which often takes time to understand and require polite phrasing, is time-consuming and cognitively demanding.}
Although systems with Large Language Models (LLM) were designed to simplify the email replying process, users still need to provide detailed prompts to obtain the expected output.
Therefore, we proposed and evaluated an \red{LLM-powered question-and-answer (QA)-based approach} for users to reply to emails by answering a set of simple and short questions generated from the incoming email.
We developed a prototype system, \textit{ResQ}, and conducted controlled and field experiments with 12 and \red{8} participants.
Our results demonstrated that \red{the QA-based approach} improves the efficiency of replying to emails and reduces workload while maintaining email quality, compared to a conventional prompt-based approach that requires users to craft appropriate prompts to obtain email drafts.
We discuss how \red{the QA-based approach} influences the email reply process and interpersonal relationship dynamics, as well as the opportunities and challenges associated with using a QA-based approach in AI-mediated communication.

% original
% Replying to lengthy and polite workplace emails is often time-consuming and cognitively demanding.
% Although systems with Large Language Models were designed to simplify the email replying process, users still needed to provide detailed prompts to obtain the expected output.
% Therefore, we proposed and evaluated a question-and-answer-based approach for users to reply to emails by answering a set of simple and short questions generated from the incoming email.
% We developed a prototype system, \textit{ResQ}, and conducted both controlled and field experiments with 12 and 9 participants.
% Our results demonstrated that ResQ improves the efficiency of replying to emails and reduces workload while maintaining email quality compared to a conventional prompt-based approach that requires users to craft appropriate prompts to obtain email drafts.
% We discuss how ResQ influences the email reply process and interpersonal relationship dynamics, as well as the opportunities and challenges associated with using a QA-based approach in AI-mediated communication.

\begin{CCSXML}
<ccs2012>
   <concept>
       <concept_id>10003120.10011738.10011776</concept_id>
       <concept_desc>Human-centered computing~Accessibility systems and tools</concept_desc>
       <concept_significance>500</concept_significance>
       </concept>
   <concept>
       <concept_id>10003456.10010927.10003616</concept_id>
       <concept_desc>Social and professional topics~People with disabilities</concept_desc>
       <concept_significance>500</concept_significance>
       </concept>
 </ccs2012>
\end{CCSXML}

\ccsdesc[500]{Human-centered computing~Accessibility systems and tools}
\ccsdesc[500]{Social and professional topics~People with disabilities}

\begin{teaserfigure}
  \includegraphics[width=\textwidth]{figure/teaser.png}
  \caption{Five core functionalities of WanderGuide. The system assists users in recreational exploration by explaining the surrounding environment through images obtained from the robot's camera. Users can adjust the level of detail and ask questions about their surroundings. Additionally, the system can guide users to locations they have visited before.}
  \Description{
    The figure shows the five core functionalities of WanderGuide. These functionalities include: Map-less Navigation and Obstacle Avoidance: WanderGuide selects potential waypoint candidates using visual information from the environment. It helps users navigate around obstacles and choose a path to their destination without depending on a pre-existing map. The figure illustrates the robot predicting several possible waypoints, selecting one, and navigating while avoiding a person. Describe Surrounding Environment: WanderGuide communicates details about the environment's layout. In the figure, the system is describing, "This is a long corridor with restaurants on the right. A Chinese restaurant offers dumplings, and further ahead is a Japanese restaurant. There are seats available in the Chinese restaurant, and tall seating on your left. I can also see two chefs preparing dishes." Adjust Detail of Descriptions: Users can adjust the level of detail in descriptions. A concise description might say, "There are Chinese and Japanese restaurants," while a more detailed version could include, "This is a long corridor with restaurants on the right. A Chinese restaurant offers dumplings, and ..." Question and Answering Interaction: WanderGuide allows users to ask questions about their surroundings. In the figure, the user is asking, "What does the Chinese restaurant have?" and WanderGuide could respond, "I can see they have dumplings, but please ask for further menus." "Take-Me-There" Functionality: WanderGuide allows users to request to be guided to a specific location. In the figure, the user is saying, "Take me to the Chinese restaurant," and WanderGuide guides the user to that location. 
  }
  \label{fig:teaser}
\end{teaserfigure}

\keywords{visual impairment, map-less navigation, recreational exploration}

\maketitle

\section{Introduction}

\begin{figure*}
    \centering
    \includegraphics[width=\textwidth]{figures/Introduction.pdf}
    \caption{Showing the novel problem statement applied to traffic prediction use case. Multiple unstructured observations from the past are used to reconstruct a hidden traffic state from which a full traffic state is forecast with a set of query locations. }
    \label{fig:intro}
\end{figure*}

% Was sagen denn die anderen warum Traffic Prediction gut ist? 
Forecasting the traffic in the near future is an important task for city management.
Data from the near past is used to predict future traffic states with spatio-temporal Graph Neural Networks \cite{bui22}.
Accurate prediction provides the opportunity to optimize traffic flow, reduce traffic jams and increase air quality \cite{Po19}.

% Wieso ist Sparsity in allen Dimensionen wichtig.
While traffic prediction relies on the availability of data from traffic sensors, there exists a plethora of reasons why sensors may stop working temporarily, such as simple errors, energy saving, or overloaded communication systems.
Considering small- or medium-sized cities, the coverage of sensors may be low because the sensors are too expensive or not available.
Also, the sensors are typically static and do not adapt to changes in the traffic flow (e.g. caused by a construction site), which motivates moving sensors that for example could be mounted on cars. 
However, both missing and moving sensors introduce sparsity, since measurements may not be available for all locations at all times.
This sparsity must be explicitly addressed in traffic prediction for a realistic application scenario, which is illustrated in figure \ref{fig:intro}.
From one hour of data on Sunday morning, only few observations of the traffic state are available at each timestep.
The number of observations may differ throughout the observed time and the observation itself can be distributed arbitrarily in the city. 
We assume a relatively low number of sensors to account for resource saving and sensor failure in our proposed framework SUSTeR.
The task is to predict the dense traffic state one timestep after the observations at all possible sensor locations.
We study this problem on the traffic dataset Metr-LA and PEMS-BAY to test our assumption that only a fraction of the sensor values would be enough for good predictions.
By modifying an existing traffic dataset, we are able to compare our results from very sparse observations to the bottom line with all information available.
A successful study will provide insights in how sensors in new cities can be reduced before installing them and further mobile sensors would save more resources and are able to adapt to new traffic situations.
We argue that in order to be adaptable to other cities and changes in traffic flows, prior information like the road network should be neglected and just the sparse observations considered.
This comes with the added benefit of making our solution applicable in regions where no openly available road network is maintained or pathways change frequently (e.g. flood areas, animal observations). 


The aforementioned problem is novel and more challenging than the commonly considered traffic prediction problem, since there exist very few observations in each input sample.
Current works for the traffic prediction problem do not consider any missing values. \cite{Li2021, Shao22}
A common method among state of the art approaches is the usage of Graph Neural Networks on graphs that model the sensor network \cite{bui22}.
The values of a sensor are applied to the same graph node for each timestep which prohibits any non-stationary sensors . 
With fixed sensor locations, the resulting sensor network is highly correlated with the road network.
Streets connecting two intersections with sensors should be also an interesting point for correlations in the sensor network.
However, variable observations and high temporal sparsity rates can not be modeled adequately in a static network.
We show in our experiments that the road network has only a small influence on the traffic predictions.

Besides the traffic prediction for future timesteps, some works explore the field of traffic speed imputation \cite{Cini22, Cuza22} where missing sensor values are predicted.
But the amount of missing values is assumed to be at most 80\%, which on average are still over 40 given sensors in each timestep in the Metr-LA dataset with a total of 207 sensors.
We consider up to 99.9\% missing values which are on average 2.4 observations in each timestep that are used as input.
Such high sparsity rates drastically decrease the chance that multiple values are present in one input sample from the same sensor location, which makes it challenging to recognize and learn temporal correlations for each location on its own.

High sparsity rates (>95\%) result in few sensor values, but if a reconstruction of the traffic state would be possible, we question if spatio-temporal graphs require nodes for each sensor.
In SUSTeR we utilize only a small amount of graph nodes for the encoding of information and do not relate such nodes to the sensor network.
We call this the hidden graph (see figure \ref{fig:intro}), which is still able to reconstruct the complete traffic state.
Due to the reduced number of nodes SUSTeR achieves faster runtimes, as shown in the experiments.
This hidden graph is not embedded directly in the spatial domain, which is why the assignment of observations, as well as the querying of the future traffic, is done with an encoder and a decoder, implemented as neural networks.
The decoding from the hidden graph to future values depends on a set of query locations.
Figure \ref{fig:intro} shows the query locations as given from outside and in combination with the reconstructed traffic state the future values are predicted.

To construct the hidden graph we encode observations from each timestep into from multiple graphs, one for each timestep. 
The graphs are created in a residual style and information is added to the node embeddings from the previous timesteps.
We choose this method to incorporate all timesteps equally into the hidden state because the redundant information along the past is non-existing for high sparsity rates.
From the sequence of graphs where our framework inserted the observations step by step we apply STGCN \cite{Yu18}, an algorithm for traffic prediction to find and learn the spatio-temporal correlations on our small number of graph nodes.
The first future timestep of the STGCN is our hidden graph in which the traffic state is reconstructed. 

% Recent work has an implicit embedding of the graph nodes into the spatial domain as the assignment from the sensor to graph node is fixed one by one.
% Because the graph has the same structure as the road network spatio-temporal correlations can be learned between those sensors.
% We reduce the number of nodes and use a non-linear assignment learned data-driven from the observations.

We find in the experiments that SUSTeR outperforms the plain STGCN and modern traffic prediction frameworks like D2STGNN for high sparsity rates $(\geq 99\%)$.
This is equivalent to only $0.2$ to $2.4$ observation for each timestep on average.
SUSTeR uses fewer parameters than the baselines and can train faster and with less training data.
Our main contributions can be summarized as follows:
\begin{itemize}
    \item We introduce a sparse and unstructured variant of the traffic prediction problem with sparsity in all dimensions. The sensors report only a fraction of their values and are arbitrarily distributed in the spatial domain.
    \item We propose SUSTeR, a framework around the STGCN architecture, which maps sparse observations onto a dense hidden graph to reconstruct the complete traffic state.
    Our code is available at github.\footnote{https://github.com/ywoelker/SUSTeR}
    \item We conducts experiments that show that SUSTeR outperforms the baselines in very sparse situations ($\geq 95\%$) and has a competitive performance in low sparsity rates.
    % \item SUSTeR trains a third faster than the next competitor.
\end{itemize}

\section{Related Work}
\label{sec:related_work}

\subsection{Robustness of Audio-Visual Speech Recognition} 

The robustness of AVSR systems has significantly advanced by integrating auditory and visual cues to improve speech recognition, especially in noisy environments. Conventional ASR methods have evolved from relying solely on audio signals \cite{schneider2019wav2vec, gulati2020conformer, baevski2020wav2vec, hsu2021hubert, chen2022wavlm, chiu2022self, radford2023robust} to incorporating visual data from speech videos \citep{makino2019recurrent}.
The multimodal AVSR methods \citep{pan2022leveraging, shi2022learning, seo2023avformer, ma2023auto} have enhanced robustness under audio-corrupted conditions, leveraging visual details like speaker's face or lip movements as well as acoustic features of speech. These advancements have been driven by various approaches, including end-to-end learning frameworks \citep{dupont2000audio, ma2021end, hong2022visual, burchi2023audio} and self-supervised pretraining \citep{ma2021lira, qu2022lipsound2, seo2023avformer, zhu2023vatlm, kim2025multitask}, which focus on audio-visual alignment and the joint training of modalities~\citep{zhang2023self, lian2023av, haliassos2022jointly, haliassos2024braven}.


Furthermore, recent advancements in AVSR highlight the importance of visual understanding alongside audio \citep{dai2024study, kim2024learning}. While initial research primarily targeted audio disturbances \citep{shi2022robust, hu2023hearing, hu2023cross, chen2023leveraging}, latest studies increasingly focus on the visual robustness to address challenges such as real-world audio-visual corruptions~\citep{hong2023watch, wang2024restoring, kim2025multitask} or modality asynchrony~\citep{zhang2024visual, fu2024boosting, li2024unified}. These efforts remark a shift towards a more balanced use of audio and visual modalities. Yet, there has been limited exploration in scaling model capacity or introducing innovative architectural designs, leaving room for further developments in AVSR system that can meticulously balance audio and visual modalities.



\subsection{MoE for Language, Vision, and Speech Models}

Mixture-of-Experts (MoE), first introduced by \citet{jacobs1991adaptive}, is a hybrid structure incorporating multiple sub-models, \ie experts, within a unified framework. The essence of sparsely-gated MoE \cite{shazeer2017outrageously, lepikhin2021gshard, dai2022stablemoe} lies in its routing mechanism where a learned router activates only a subset of experts for processing each token, significantly enhancing computational efficiency. Initially applied within LLMs using Transformer blocks, this structure has enabled unprecedented scalability \cite{fedus2022switch, zoph2022st, jiang2024mixtral, guo2025deepseek} and has been progressively adopted in multimodal models, especially in large vision-language models (LVLMs) \cite{mustafa2022multimodal, lin2024moellava, mckinzie2025mm1}.
Among these multimodal MoEs, \citet{zhu2022uni, shen2023scaling, li2023pace, li2024uni} and \citet{lee2025moai} share the similar philosophy to ours, assigning specific roles to each expert and decoupling them based on distinct modalities or tasks. These models design an expert to focus on specialized segments of input and enhance the targeted processing.

Beyond its applications in LLMs and LVLMs, the MoE framework has also been applied for speech processing \cite{you2021speechmoe, you2022speechmoe2, hu2023mixture, wang2023language}, where it has shown remarkable effectiveness in multilingual and code-switching ASR tasks. In addition, MoE has been employed in audio-visual models \cite{cheng2024mixtures, wu2024robust}, although they primarily focus on general video processing and not specifically on human speech videos. These approaches leverage MoE to model interactions between audio and visual tokens without directly processing multimodal tokens.
Our research advances the application of the MoE framework to AVSR by designing a modality-aware hierarchical gating mechanism, which categorizes experts into audio and visual groups and effectively dispatches multimodal tokens to each expert group. 
This tailored design enhances the adaptability in managing audio-visual speech inputs, which often vary in complexity due to diverse noise conditions.

\section{CavePI System Design}
The CavePI AUV design includes three major subsystems: sensory bay, computational bay, and locomotion bay. Our proposed system and its components are shown in Fig.~\ref{fig:system_design}.  
% \vspace{-1 mm}


\subsection{Sensor Bay: Acoustic-Optic Perception Subsystem}
The CavePI platform includes visual and acoustic sensors: a front-facing fisheye camera, a downward-facing low-light camera, and a Ping2 active sonar. The fisheye camera, housed within a transparent dome at the \textit{head} of the AUV, captures forward-facing visuals with a $160^\circ$ field-of-view (FOV) and outputs a video feed at $1920\times1080$ resolution. It is worth noting that the cylindrical enclosure is a $6''$ tube while the dome has a $4''$ diameter. A custom interface is built to connect the two; see section \ref{sub:Stability} for more details. The low-light camera, mounted inside the computational enclosure, captures downward-facing visuals with an $80^\circ \times 64^\circ$ FOV, also at the same resolution. Additionally, a Ping2 sonar altimeter-echosounder from Blue Robotics\texttrademark{} is mounted on the underside of the robot; the sonar has a range of $100$ meters, a depth rating of $300$ meters, and a resolution of $0.5\%$ of the range, allowing it to detect obstacles in the surrounding environment beneath CavePI. These sensory components collectively provide robust environmental awareness for autonomous navigation in challenging underwater environments.
%This camera provides critical imagery for tracking cavelines, enabling autonomous operation. 
% \vspace{-1mm}




\subsection{Computational Bay}
\vspace{-1 mm}
As illustrated in Fig.~\ref{fig:system_design}, the computational and electronic components of CavePI are housed within an acrylic cylindrical enclosure. This enclosure, with a thickness of $6.35$\,mm and a depth rating of 65 meters, forms the \textit{main body} of the robot, providing mechanical stability, buoyancy, and waterproof protection for the electronics. The computational elements include a Raspberry Pi-5, a Nvidia\texttrademark{} Jetson Nano, and a Pixhawk\texttrademark{} flight controller. The Jetson Nano is dedicated to processing visual data from the cameras, performing image processing tasks critical for scene perception and state estimation. The Raspberry Pi-5 manages planning and control modules, ensuring real-time underwater navigation. The Pixhawk flight controller acts as a bridge between hardware and software, receiving actuation commands from the Raspberry Pi-5 and transmitting them to the thrusters and lights via the MAVLink communication protocol. Additionally, the Pixhawk integrates a 9-DOF IMU, offering 3-axis gyroscope, accelerometer, and magnetometer measurements, which are used to calculate the attitude of CavePI during underwater operations.

The enclosure also contains the battery compartment, voltage regulators, electronic speed controllers (ESCs), and a Bar-30 pressure sensor. The battery compartment holds a $14.8$\,V ($18$\,Ah) battery pack, regulated to power internal components (\eg, cameras, computers) and external components (\eg, thrusters, sonar). Each thruster is controlled by an ESC, which drives the three-phase brushless motor using PWM signals from the Pixhawk. The Bar-30 sensor provides high-precision pressure readings with a resolution of $0.2$\,mbar and an accuracy of $2$\,mm, with a working depth of up to $300$ meters. This pressure data is processed to determine CavePI’s underwater depth, ensuring reliable and accurate interoceptive perception during operations.

%Furthermore, it houses the battery compartment, voltage regulators, electronic speed controllers (ESCs), and a Bar-30 sensor. The battery compartment features a $14.8$V, $18$Ah pack -- which is regulated to individual components inside (e.g. cameras, computers) and outside (e.g. thrusters, sonar) of the enclosure. The ESCs, one for each thruster, controls a three-phase brushless motor inside a thruster by receiving an input PWM signal from Pixhawk. The Bar-30 sensor provides pressure readings with a resolution of $0.2$\,mbar and $2$\,mm accuracy up to a working depth of $300$\,m.  This pressure data is then processed to compute the underwater depth of the CavePI ensuring robust and accurate interoceptive perception.



% \begin{figure*}[t]
%      \centering
%      \begin{subfigure}[]{0.49\textwidth}
%          \centering
%          \includegraphics[width=\linewidth]{figures/Fig4_Electronics.png}%
%          \vspace{-1.5 mm}
%          \caption{Major electronics and sensor-actuator connections.}
%          \label{fig:electronics}
%      \end{subfigure}~     
%      \begin{subfigure}[]{0.47\textwidth}
%          \centering
%          \includegraphics[width=\linewidth]{figures/Fig4_ROS.png}
%          \vspace{-1.5 mm}
%          \caption{Data flow of major computational modules in the form of \textit{ROS topics}: red and blue arrows represent \textit{subscribed} and \textit{published} topics, respectively.}
%          \label{fig:ROS}
%      \end{subfigure}
%      \vspace{-1 mm}
%         \caption{Simplified outlines of the end-to-end hardware, software, and ROS2 middleware integration of NemoGator are shown.}%
%      \label{fig:hw_mw_sw}
% \vspace{-4 mm}
% \end{figure*}




% \subsection{Locomotion Subsystem}
% \vspace{-1 mm}
% As opposed to traditional thruster-based AUV systems (eg, CUREE~\cite{girdhar2023curee}, ReefGlider~\cite{macauley2024reefglider}, LoCO~\cite{edge2020design}), NemoGator employs a bio-inspired propulsion system~\cite{zhang2010biologically} driven by three servo motors that actuate a \textit{caudal} tail (BCF) and two \textit{pectoral} fins (MPF). This design is inspired by the carangiform design~\cite{macias2024numerical,raj2016fish,costa2018design} (see Sec~\ref{sec:background}), combining the benefits of BCF and MPF for propulsion and maneuverability control, respectively~\cite{zhang2021development,marchese2013towards}. Specifically, the tail generates forward thrust and contributes to yaw motions, powered by a single servo motor; the pectoral fins ensures the stability of the system, regulating pitch within the water column, and assisting in roll and yaw motions~\cite{zhang2021design}. They are powered by respective servo motors, ensuring low-power operation with only three $35$\,Kg-cm torque ($7.4$V) motors. 

% When the fins are angled upward, the resulting hydrodynamic lift combined with tail propulsion, enabling ascent. Conversely, downward angling of the fins induces controlled descent, allowing depth modulation in the water column. We make sure that these fins oscillate synchronously for forward propulsion, while independent actuation of a single fin modulates yaw, enabling precise directional control. 

% The caudal fin of the NemoGator is directly connected to a servo motor to control the oscillations of the caudal fin. Forward movement is achieved through symmetric oscillations of the caudal fin around the NemoGator's longitudinal axis, while the yaw control is managed through asymmetric oscillations.
% cite:Towards a Self-contained Soft Robotic Fish: On-Board Pressure Generation and Embedded Electro-permanent Magnet Valves 

% The two side fins are engineered to replicate the functional dynamics of pectoral fins in fish, contributing to the stability of the system, regulating pitch within the water column, and assisting in roll and yaw motions. The pectoral fins of the NemoGator are directly connected to their respective servo motors, which are mounted on the outer body.%cite:Design and Locomotion Control of a Dactylopteridae-Inspired Biomimetic Underwater Vehicle With Hybrid Propulsion

% Specifically, oscillating only the left fin generates a yaw moment, inducing a left turn, whereas oscillating the right fin similarly facilitates a right turn.
% \vspace{-1mm}


 

% CavePI AUV is intended for low-power autonomous operation with portable ROS support for application-specific perception, planner, and control modules. As shown in Fig.~\ref{fig:ROS}, we incorporate SVIn (sonar-visual-inertial navigation)~\cite{rahman2022svin2} packages for general-purpose state estimation and waypoint-based trajectory planning. NemoGator can also be used as an underwater ROV with an optional tether-based TeleOp module. Depending on \textit{ROV mode} or \textit{autonomous mode}, a unified AutoPilot package is designed to control the servo motor commands for navigation.

\begin{figure}[t]
% \vspace{-1 mm}
    \centering
    \includegraphics[width=0.98\linewidth]{figures/Fig4_Electronics.png}%
    \vspace{-1mm}
    \caption{Major electronics and sensor-actuator connections of CavePI.}
    \label{fig:electronics}
    \vspace{-3mm}
\end{figure}


\subsection{Locomotion Bay: Middleware Integration}
\vspace{-1 mm}
The end-to-end integration of CavePI ensures that each computational component operates in sync, tied to a ROS2 Humble-based middleware backbone. The modular design also allows for future upgrades, ensuring that the CavePI can be tailored to meet evolving research in marine ecosystem exploration and monitoring. The sensor-actuator signal communication graph is illustrated in Fig.~\ref{fig:electronics}.


The CavePI autonomous underwater vehicle (AUV) is designed for low-power operation and integrates a portable ROS2 framework to support application-specific perception, planning, and control modules. As depicted in Fig.~\ref{fig:ROS}, the \textit{detector} node acquires visual data from the two cameras to identify the caveline for navigation. In the absence of GPS underwater, the system employs SVIn (sonar-visual-inertial navigation)~\cite{rahman2022svin2} packages to estimate the AUV’s position relative to the detected caveline. The \textit{mission planner} node then integrates the caveline information with the estimated position data to generate subsequent waypoints for the mission. Finally, the \textit{autopilot controller} node utilizes these waypoints, along with the detected caveline, positional data, and depth readings from the Bar-30 sensor, to generate precise actuation signals for the thrusters, enabling accurate movements and depth control. Additionally, CavePI can function as an ROV through an optional tether-based teleoperation module. This module transmits user commands from a joystick to the onboard Raspberry Pi-5, which processes the inputs and relays them to the thrusters for manual control.

\begin{figure}[t]
    \centering
    \includegraphics[width=\linewidth]{figures/Fig4_ROS.png}%
    \vspace{-1mm}
    \caption{Data flow among major computational modules of CavePI is shown in the form of \texttt{ROS Topics}: red and blue arrows represent \textit{subscribed} and \textit{published} topics in the ROS graph, respectively.}
    \label{fig:ROS}
    \vspace{-4mm}
\end{figure}


\subsection{CavePI Digital Twin}
%\vspace{-1 mm}
% The digital twin of CavePI is created in ROS, contains a similar structure as presented in Fig.~\ref{fig:ROS}. 
We develop a digital twin (DT) model of CavePI by using the Unified Robot Description Format (URDF), with links and joints carefully assigned to represent the various CAD components designed in SolidWorks. To replicate the sensor suite of the physical CavePI, Gazebo plugins are integrated to simulate the front-facing camera, down-facing camera, IMU, pressure sensor, and sonar. Additional plugins are employed to simulate environmental forces, including buoyancy, thrust, and hydrodynamic drag, thereby enhancing the physical realism.

A controlled open-water scenario is created in Gazebo to simulate realistic missions, featuring a thin line arranged in a rectangular loop to mimic a caveline. Since the simulated environment lacks real-world perception challenges such as low light or turbid water conditions, the perception subsystem remains simplified. Instead of deploying computationally intensive deep visual learning models, simpler edge detection and contour extraction techniques~\cite{SUZUKI198532} are used to identify the caveline from the down-facing camera feed. The remaining navigation and control subsystems mirror the real-world implementation and operate via two ROS nodes. The first node processes the extracted contours to make navigation decisions and publishes high-level control commands (\eg, yaw angle). The second node subscribes to these commands, computes the required thrust and hydrodynamic drag forces, and publishes them as ROS topics to control the simulated robot model.

Beyond replicating caveline following experiments, we utilize the DT system for preliminary testing and fine-tuning of new control algorithms. It also enables the simulation of complex cave scenes, such as narrow passages, dead ends, and sharp turns. Conducting repeated real-world experiments in such scenarios to improve the control system can be logistically demanding where the simulation offers an efficient alternative for extensive evaluation and fine-tuning.  

% These force commands are then subscribed to by the robot model in Gazebo, enabling it to execute the desired motion accurately. Finally, the URDF and Gazebo world files, along with the three control nodes, are integrated into a ROS launch file. Executing this launch file initiates a simulation of CavePI's digital twin in Gazebo, demonstrating its ability to follow the caveline within an underwater environment.

% To control the digital twin to follow the cave line in the underwater world, three ROS nodes are created for different applications. In the first ROS node, subscribed frames from the downward-facing camera are utilized by a caveline detection algorithm~\cite{SUZUKI198532} that publishes contours along the caveline. In the second node, control decisions are taken based on the detected contours along the caveline and high-level control signals are published. In the third node, these high-level control signals are subscribed, and thrust and hydrodynamic drag forces on the robot model are calculated. These forces are then published to ROS topics and later subscribed by the robot model in Gazebo to achieve the desired motion.

% \subsubsection{Integration}
% Finally, the URDF and Gazebo world files, along with the three control nodes, are integrated into a ROS launch file. Executing this launch file initiates a simulation of CavePI's digital twin in Gazebo, demonstrating its ability to follow the caveline within an underwater environment.
\vspace{-2mm}
\section{Formative Study}
\label{sec:study1}
We first conducted a formative study to investigate the requirements of the system, such as how the system should explain its surroundings and what potential interactions may happen between the robot and the user.
To conduct the study, we recruited ten participants through our existing email list.
Interestingly, our recruitment emails were shared among blind people, eventually reaching people not on our emailing list. 
In the recruitment email, we specified that those who are unfamiliar with the experimental location, \ie, even if they have had previous visiting experience, they do not have a clear understanding of the building or know what is there, would be eligible to participate.
Tab.~\ref{tab:demographics1} shows the demographics of the participants. 
All studies in this paper have been approved by our institution's review board.
Informed consent was read out to all participants in this paper and obtained from them. 
The study took approximately two hours, and the participants were compensated \$20 per hour and reimbursed for their transit costs.
Only one participant was present for each session. 

\begin{figure*}
    \centering
    \includegraphics[width=1\linewidth]{figure/routes.png}
    \caption{Floor maps of the location of the study. The left panel shows the two floors of the science museum, \rrred{the fifth floor of Miraikan}, which feature exhibits on various topics, such as environmental issues and space exploration. On the right panel is a floor plan of a shopping mall, \rrred{the fourth floor of Toranomon Hills Station Tower}, which includes a variety of restaurants offering different cuisines, including French, Japanese, Chinese, and cafes.}
    \Description{
    The image contains three floor plans, each marked with dotted red lines and stars, indicating different routes and destinations. On the left side are two floors of a science museum, while the right side displays a single-floor plan of a shopping mall. The first map of the science museum highlights exhibits in light blue, connected by hallways in light gray. The floor has a rectangular shape with a main pathway running through the center. There are three round exhibits—two in the middle and one on the right—with additional exhibits placed along the left and right sides of the main walkway. A red dotted line shows the route followed in the first study, starting from the left side, circling the floor, and returning to the starting point. The starting point for the main study is located at the same spot as in the formative study. The second map represents the museum's second floor, also rectangular in shape and similarly structured to the previous floor but without the round exhibits. Again, a red dotted line marks the path followed in the formative study, beginning on the left side, looping around the floor, and returning to the starting point. However, the main study on this floor begins at a different location, on the right side of the floor. The last map on the right shows part of a shopping mall, featuring shops in blue and common areas in gray. The floor has a maze-like layout with several intersections and includes 21 restaurants. The route for the formative study begins in the center of the floor, loops around each restaurant, and returns to the starting point.
    }
    \label{fig:route}
\end{figure*}

\subsection{Prototype System}
We developed our prototype robot system according to Sec.~\ref{sec:system_design}. 
It was based on an open-source robot platform\footnote{https://github.com/CMU-cabot/cabot} and could guide users while explaining the surrounding environment. 
To ensure that the participants experienced the same level of autonomy, we used teleoperation, a Wizrd-of-Oz-based approach~\cite{riek2012wizard-of-oz}, to force the robot to be in full-automatic mode when guiding the participants.
We adopted a suitcase-shaped wheeled robot for this study.
The suitcase's appearance allows blind users to seamlessly blend into their environment, leading to higher social acceptance from users, surrounding pedestrians, and facility managers~\cite{kayukawa2022HowUsers}. 
As shown in Fig.~\ref{fig:device&ui}--A-1, the robot has a handle embedded with five buttons, a touch sensor beneath the handle, a 360$^\circ$ \red{Velodyne VLP-16 LiDAR sensor~\cite{Velodyne}} sensor, three RGBD cameras with resolutions of 640×360, \red{one RealSense D455 camera~\cite{RealSenseD455} mounted at the front, two RealSense D435 cameras~\cite{RealSenseD435} on the left and right}, and a pair of motorized wheels in differential drive configuration.
\red{
Inside the suitcase, it has Ruby R8 powered by an AMD Ryzen R7-4800U CPU~\cite{NUC}, and a Jetson Mate featuring multiple Jetson Xavier NX GPUs~\cite{JetsonMate}.
}
The RGBD cameras were attached 0.51 meters above the ground.
The touch sensor detects whether or not the user is holding the handle and moves only when it is being held by the user. 
The cameras combined have a horizontal field of view of approximately 180$^\circ$.
The weight of the robot is approximately 15kg.
We set the default speed of the robot to 0.5 meters per second to maintain a balance between a comfortable walking speed and a speed that allows sufficient time to absorb the scene description audios.
A smartphone is attached to the suitcase to provide audio feedback through a neck speaker worn by users, connected via Bluetooth.

To convey the surrounding information to the participants, we used GPT-4o~\cite{GPT4o}, a popular MLLM model.
We inputted the images from the three RGBD cameras into the MLLM model and asked the model to generate descriptions of the surrounding environment.
The robot was designed to describe surrounding information 5-10 seconds after the end of the previous description every time. 
We engineered the prompts to ask the MLLM model to first provide a general overview of the scene, followed by specific details on the left, front, and right. 
We asked the descriptions to include as many objects as possible and incorporate layout information, such as navigable directions and the presence of walls~\cite{jain2023want}. 
\red{The processing time and cost to generate a description was 6.087 seconds and \$0.00740 on average.}
We attach the full prompts in Appendix Sec.~\ref{appendix:prompt_formative}.

\begin{table*}[]
\caption{Demographics of participants who attended the formative study. The table reports their gender, age, navigation aid, which they frequently use, frequency of exploration done either independently or with sighted people per year, their experimental location, number of previous visits to the experimental location, and analyzed preference. }
\Description{The table presents demographic data for participants in the formative study. The information includes gender, age, type of mobility aid used, the age at which they started using the aid, frequency of exploration per year, the location of the experiment, the number of previous visits, and their preference analysis. The following describes each participant. P01 is a 64-year-old female who uses a cane, started using it at the age of 44, explores 48 times per year, participated at the Science Museum with 1 previous visit, and is exploration-inclined. P02 is a 53-year-old male who uses a cane, started using it at the age of 13, explores 36 times per year, participated at the Science Museum with 0 previous visits, and is destination-oriented. P03 is a 74-year-old male who uses a cane, started using it at the age of 0, explores 1 time per year, participated at the Science Museum with 0 previous visits, and is destination-oriented. P04 is a 54-year-old female who uses a cane, started using it at the age of 0, explores 12 times per year, participated at the Science Museum with 0 previous visits, and is exploration-inclined. P05 is a 56-year-old male who uses a cane, started using it at the age of 52, explores 2 times per year, participated at the Science Museum with 0 previous visits, and is intermediate in preference. P06 is a 32-year-old male who uses a cane, started using it at the age of 0, explores 12 times per year, participated at a shopping mall with 0 previous visits, and is intermediate in preference. P07 is a 55-year-old female who uses a cane, started using it at the age of 52, explores 0 times per year, participated at a shopping mall with 1 previous visit, and is exploration-inclined. P08 is a 63-year-old male who uses a cane, started using it at the age of 22, explores 12 times per year, participated at a shopping mall with 0 previous visits, and is intermediate in preference. P09 is a 78-year-old female who uses a guide dog, started using it at the age of 22, explores 12 times per year, participated at a shopping mall with 0 previous visits, and is destination-oriented. P10 is a 49-year-old female who uses a cane, started using it at the age of 3, explores 1 time per year, participated at a shopping mall with 0 previous visits, and is exploration-inclined.}
\label{tab:demographics1}
\resizebox{\textwidth}{!}{%
\begin{tabular}{ccccccccc}
\toprule
    & Gender & Age & Aid       & \begin{tabular}[c]{@{}c@{}}Age \\  of Onset\end{tabular} & \begin{tabular}[c]{@{}c@{}}Frequency of\\  Exploration per Year\end{tabular} & \begin{tabular}[c]{@{}c@{}}Experiment \\  Location\end{tabular} & \begin{tabular}[c]{@{}c@{}}Number of \\  Previous Visits\end{tabular} & Preference Analysis     \\
    \midrule
P01 & F      & 64  & Cane      & 44                                                       & 48                                                                           & Science Museum                                                  & 1                                                                     & Exploration-Inclined \\
P02 & M      & 53  & Cane      & 13                                                       & 36                                                                           & Science Museum                                                  & 0                                                                     & Destination-Oriented \\
P03 & M      & 74  & Cane      & 0                                                        & 1                                                                            & Science Museum                                                  & 0                                                                     & Destination-Oriented \\
P04 & F      & 54  & Cane      & 0                                                        & 12                                                                           & Science Museum                                                  & 0                                                                     & Exploration-Inclined \\
P05 & M      & 56  & Cane      & 52                                                       & 2                                                                            & Science Museum                                                  & 0                                                                     & Intermediate         \\
P06 & M      & 32  & Cane      & 0                                                        & 12                                                                           & Shopping Mall                                                   & 0                                                                     & Intermediate         \\
P07 & F      & 55  & Cane      & 52                                                       & 0                                                                            & Shopping Mall                                                   & 1                                                                     & Exploration-Inclined \\
P08 & M      & 63  & Cane      & 22                                                       & 12                                                                           & Shopping Mall                                                   & 0                                                                     & Intermediate         \\
P09 & F      & 78  & Guide dog & 22                                                       & 12                                                                           & Shopping Mall                                                   & 0                                                                     & Destination-Oriented \\
P10 & F      & 49  & Cane      & 3                                                        & 1                                                                            & Shopping Mall                                                   & 0                                                                     & Exploration-Inclined \\
\bottomrule
\end{tabular}
}
\end{table*}

\subsection{Experimental Location}
To ensure the diversity of the findings we would obtain from this study, we conducted the study in two different locations. 
\red{We chose to conduct our studies in a science museum and a shopping mall, as these are locations where people typically engage in exploration, and they have been utilized in previous research~\cite{asakawa2019independent,asakawa2018present,Kaniwa2024ChitChatGuide}. 
A museum is generally a place for learning about exhibits, while a shopping mall often requires exploration both before and during visits to stores.
\rrred{Specifically, we used the fifth floor of Miraikan\footnote{\url{https://www.miraikan.jst.go.jp/en/}} for the science museum and the fourth floor of Toranomon Hills Station Tower\footnote{\url{https://www.toranomonhills.com/}} for the shopping mall.}
The floor map of the science museum is illustrated in the left panel of Fig.~\ref{fig:route}, which contains two floors, both primarily featuring science exhibits. 
For the studies, the order of the two floors was counterbalanced.
The study in the museum was conducted after business hours, during which customers were absent, but staff were present for their duties.
The floor map of the shopping mall is illustrated in the right panel of Fig.~\ref{fig:route}, a floor that contains several restaurants from various countries. 
The study in the shopping mall was conducted during regular business hours.}
As shown in Tab.~\ref{tab:demographics1}, the study with P01--P05 took place in the science museum, and the study with P06--P10 took place in the shopping mall.

\subsection{Procedure}
For each participant, we first conducted a pre-study interview to learn about their experience in exploring buildings, followed by an explanation that the study aimed to gather their opinions on a guide system designed to assist with exploration.
Then, participants were given a task to navigate the predetermined route (red arrow of Fig.~\ref{fig:route}) guided by the robot.
Adopting a Wizrd-of-Oz-based approach, an experimenter controlled the robot to navigate along the route and stop when there were nearby pedestrians. 
During exploration, the robot periodically generated descriptions of the scenes. 
We show an example of the generated description in Fig.~\ref{fig:study1example}.
After the exploration, we asked the participants if there were any additional things they wanted to do to partially simulate the potential interaction, such as going to additional places or going around the floor again for more exploration.
Finally, we conducted a post-interview session to gather their feedback on the system. 

\subsection{Result}
\subsubsection{Interests to Exploration}
All participants stated that totally independent exploration is challenging, but they expressed a desire for exploration if a guide system can help them do so. For example:
\newanswer[\label{P02Conditioned}]\textit{``I don't really explore much. I go out with a specific purpose in mind [...] The reason is that it's just too bothersome. But I do think it would be fun if I did [...]  I'm more of an old-timer, so exploration never really caught my interest. It's not that I didn't care at all, but perhaps I've been living this way (not to explore).}\footnote{The comments were obtained in the native language where the study was conducted. We translate the comments into English using publicly available LLM to ensure reproducibility. We show the full prompt used for translation in Appendix Sec.~\ref{appendix:translate}.} (P02)

\subsubsection{Positive Feedback and Appreciated Information}
Seven participants (P01, P04--P08, and P10) expressed their enjoyment while navigating with the robot, particularly with the provided surrounding descriptions, as described in the following comment:
\newanswer[\label{P07Enjoy}]\textit{``My first impression was that it was a lot of fun. The reason is, as you just mentioned, unlike the person I usually walk with, the system provided detailed explanations about things like the color of the walls and the signs we saw and even described how the chef was preparing the food. Normally, you might get some of this information from others, but it's rare to get such thorough details. I found myself thinking, ``Oh, I see, that's how it looks to sighted people,'' and I felt there was a lot of new information. In that sense, I really enjoyed it.''} (P07)

Participants appreciated a variety of real-time details about their surroundings, notable examples include patterns on the walls, lighting conditions, subjective descriptors such as ``beautiful,'' the presence and actions of nearby people, the existence of signboards, the layout of the environment, and the visibility of a chef in an open kitchen. 
Additionally, P10, who requested to walk around the floor again, noted that receiving different descriptions of the same location was beneficial, as it gave them a sense of presence:
\newanswer[\label{P10VariousAndDifferentInformation}]\textit{``The system mentioned those things, as well as details about the plants and wall decorations. It's like, you talked about so many different things that it feels like I was actually looking around myself. Honestly, most of the time, I get so occupied with just reaching my destination that I don't notice things around me. [...] The system also mentioned things in the second round of explanations that weren't covered in the first round, which was nice. It conveyed a sense of the ongoing atmosphere and gave a good understanding of the situation at the time.''} (P10)

\subsubsection{Information Needs}
\label{sec:info_needs}
Participants hoped for further polishing of the delivered information about the scenes. Six participants (P01--P03, P06, and P09--P10) felt the information conveyed about the surroundings was too abstract, indicating the need for more specific information:
\newanswer[\label{P01NeedMoreConcreteInformation}]\textit{``The system talked about there are just exhibits, or there's information on panels, but I think it would be nice if the system talked about specific titles. There are places where the system talked about them, but there are also places where it did not, so I found myself wondering about that.''} (P01)
In particular, three participants (P02, P03, and P09) commented that the descriptions neither helped them learn the environment nor make decisions such as determining which shops or exhibits to enjoy:
\newanswer[\label{P09NegativeImpression}]\textit{``I expected it to at least tell me the name of the store, but it was disappointing to find out that it didn't do that at all. I really wish there was a system that could provide pinpointed information about what I want to know. Especially in an unfamiliar restaurant area, for example, if I come alone and use the device to enter the premises, it starts running, and then when I think, ``Oh, should I have Japanese food today, or maybe tonkatsu?'', without such information, I end up just walking around aimlessly.''} (P09)

Participants also described specifics about what types of information would be beneficial to include, such as the position of objects given in meters and clock directions, the availability of seats, people on collision paths, identities of surrounding individuals (\eg staff), and specific names of objects. 
In science museums, participants also wanted to know whether exhibits are touchable. 
In shopping malls, participants also wanted to learn the store menus and whether there is a spacious area for a guide dog to rest while the user is eating.
However, three other participants (P02, P03, and P09) found certain information, such as details about lighting, surrounding people, and wall design, unnecessary. 

\begin{figure*}
    \centering
    \includegraphics[width=1\linewidth]{figure/examples.png}
    \caption{Examples of descriptions described in the formative study. Panel A shows an example of a description generated at the science museum, and Panel B shows the one generated at a shopping mall.}
    \Description{Examples of descriptions described in the formative study. Panel A shows an example of a description generated at the science museum, saying "This is a futuristic exhibition hall that has vibrant displays. To your left, there is a uniquely shaped wooden table and archway. Ahead, you can see a curved blue sofa and a white sign that reads "Entrance." On your right, large colorful panels line the wall, displaying information about the future and health." and Panel B shows the one generated at a shopping mall, saying "This is a bright, modern corner of a commercial facility. On the left side, there are tall-backed chairs made of black metal lined up, and beyond them, round tables are arranged. Ahead, a man in a suit is standing, and in the background, there's an electronic menu board, suggesting the presence of a restaurant. To the right, there's an eatery enclosed by warm-colored walls in shades of red and orange, with many metallic chairs and tables, and menu boards are set up."}
    \label{fig:study1example}
\end{figure*}

\subsection{Design Considerations}
The results of the study affirmed that there are certain appreciations and room for improvement for the exploration robot for blind people.
Based on the above results, we derived several requirements for the system, as listed below.

\subsubsection{Vary Detail of Descriptions Based on Preferences and Contexts}
\label{sec:implication_varydetail}
We observed three types of preferences: one that enjoyed all the descriptions provided by the system (\textit{Exploration-Inclined}), another that enjoyed the descriptions but preferred to limit certain information (\textit{Intermediate}), and a third group that only wanted information useful for determining where to go (\textit{Destination-Oriented}). 
In Tab.~\ref{tab:demographics1}, we show the description preference of each participant.
To classify the preferences, we first classified three participants who did not enjoy the description of the system as \textit{Destination-Oriented}.
Then, based on the discussion between the authors, we classified the rest as \textit{Intermediate} or \textit{Exploration-Inclined}.
Furthermore, the type of information needed varied slightly depending on the experimental location. 
For instance, participants sought seating information for guide dogs in shopping areas, whereas in the science museum, they were more interested in whether the exhibits were touchable.
Given these three types of preferences and context-dependent information needs, we modified the system so that it could adjust the amount and types of information conveyed to each participant.

\subsubsection{Add Question and Answer Functionality}
\label{sec:implication_Q&A}
There was a clear need for question-and-answer (Q\&A) interaction, as seven participants (P02--P05 and P08--P10) noted that they would like the option to ask more detailed questions through conversation. 
Participants expressed interest in this functionality when they were curious about the system's descriptions. This would allow them to ask more detailed questions about the objects of interest.

\subsubsection{Add ``Take-Me-There'' Functionality} 
\label{sec:implication_takemethere}
Four participants (P02, P04, P06, and P10) mentioned that they would like to revisit locations they found interesting after walking around the floor. 
Example situations include deciding to visit a shop, engaging with touchable exhibits, or returning to chairs discovered during the exploration. 
In unfamiliar locations, where users may lose their sense of direction, participants also expressed the need for a feature that guides them back to their initial location~\cite{kuribayashi2023pathfinder}.

\subsubsection{Vary Speed and Be Able to Stop the Robot}
\label{sec:implication_speed}
While the majority found the default speed appropriate for listening and understanding the described information, there were requests for customizable speed settings. 
Eight participants stated that the robot's speed was appropriate for exploring. 
Two participants (P04 and P06) expressed a preference for a faster speed.
P01 additionally wanted to stop when the robot read out the descriptions of interest.
In conclusion, users who are \textit{Destination-Oriented} or have already determined the destination through exploration may want to increase the speed, while users who prefer to take time exploring might wish to slow down or stop the robot entirely. 

\subsubsection{Add Direction Specifying Functionality}
\label{sec:implication_directionspecification}
Participants expressed a desire for more active engagement by specifying the movement direction themselves. 
Four participants (P02--P05) mentioned that they wanted more active control over the movement direction based on their interests.
Additionally, we extrapolated that instead of simply following the robot, some users may prefer to interactively choose the direction based on the audio description of the surroundings.
This could lead to greater autonomy because it would enrich the exploratory experience by aligning the robot's movement with the users' real-time curiosity and needs, creating a more personalized and engaging exploration experience.

\section{Implementation}
We train the user and \interfaceagent's policies simultaneously in a shared environment (the AUI). All policies receive an independent reward, and the actions of the policies influence a shared environment. We execute actions in the following order: (1) the \interfaceagent's action, (2) the \useragent's high-level action, followed by (3) the \useragent's low-level motor action. The reward for the two learned policies is computed after the low-level motor action has been executed. The episode is terminated when the \useragent has either completed the task or exceeded a time limit.

We implement our framework in Python 3.8 using RLLIB \cite{liang2018rllib} and Gym \cite{brockman2016openai}. We use PPO \add{\cite{schulman2017proximal, yu2022surprising}} to train our policies. We use 3 cores on an Intel(R) Xeon(R) CPU @ 2.60GHz during training \add{and an NVIDIA TITAN Xp GPU}. Training takes $\sim$36 hours. \del{We utilize an NVIDIA TITAN Xp GPU for training.} The \useragent's high-level decision-making policy $\policy_d$ is a 3-layer MLP with 512 neurons per layer and ReLU activation functions. The  \interfaceagent's policy  $\policy_I$ is a two-layer network with 256 neurons per layer and ReLU activation functions. \addiui{We sample the full state initialization (including goal) from a uniform distribution. We use stochastic sampling for our exploration-exploitation trade-off.} 

We use curriculum learning to increase the task difficulty and improve learnability. Specifically, we adjust the difficulty level every time a criteria has been met by increasing the mean number of initial attribute differences. More initial attribute differences result in longer action sequences and are therefore more complex to learn. We increase the mean by 0.01 every time the successful completion rate is above 90\% and the last level up was at least 10 epochs away.

We randomly sample the number of attribute differences from a normal distribution with standard deviation $1$, normalize the sampled number into the range $[1, n_a]$ and round it to the nearest integer, where $n_a$ is the number of attributes of a setting (in the case of game character $n_a=5$).

\del{The difference between agents of different applications is their respective state- and action spaces.}
\section{Main User Study}
\label{sec:study2}
This study was conducted to validate WanderGuide and explore further design space.
Participants were recruited and compensated similarly to those in the formative study.
Similar to the formative study, in the recruitment email, we specified that participants unfamiliar with the experimental location would be eligible to participate.
We conducted this study on the same two floors of the science museum. 
Tab.~\ref{tab:demographics2} shows the demographics of the participants.
\red{
None of the participants from the formative study participated in this study.
Similar to the formative study, this study was conducted after business hours.
}

\subsection{Task and Procedure}
For each participant, we first conducted a pre-study interview similar to the formative study.
Then, the participant joined a 30-minute training session to get familiar with the robot system before the main tasks.
For the main tasks, they were asked to freely explore the floor for 20 minutes using the system from a fixed starting location, as illustrated in Fig.~\ref{fig:route}.
The ordering of the floors was counterbalanced to mitigate the order effect.
After the main tasks, we conducted a post-study interview to ask several seven-point Likert scale questions (1: Strongly Disagree, 4: Neutral, and 7: Strongly Agree) that measure their self-evaluated exploration performance, Raw Task Load Index (TLX)~\cite{byers1989traditional} to measure the task workload, and system usability scale (SUS)~\cite{brooke1996sus} to evaluate the usability of the system.
Finally, we asked open-ended questions to gather comments on the system.
Below, we report the results of the study.

\begin{table}[]
\caption{The statistics of duration time and the count of interactions for each mode (Auto, Conversation, and Manual Control).
The ratio of the duration time is calculated based on the total duration time of the experiment per participant.}
\Description{
The table provides an analysis of the duration ratio (percentage of total time) and the count of interactions for three different modes: Auto, Conversation, and Manual Control. The statistics are presented for five participants: P11, P12, P13, P14, and P15. P11 spent 59.77\% of the time in Auto mode with 25 interactions, spent 37.52\% of the time in Conversation mode with 21 interactions, and spent 2.70\% of the time in Manual Control mode with 4 interactions. P12 spent 91.66\% of the time in Auto mode with 13 interactions, spent 8.16\% of the time in Conversation mode with 9 interactions, and spent 0.18\% of the time in Manual Control mode with 1 interaction. P13 spent 67.88\% of the time in Auto mode with 12 interactions, spent 30.56\% of the time in Conversation mode with 9 interactions, and spent 1.56\% of the time in Manual Control mode with 1 interaction. P14 spent 64.86\% of the time in Auto mode with 17 interactions, spent 33.94\% of the time in Conversation mode with 15 interactions, and spent 1.20\% of the time in Manual Control mode with 1 interaction. P15 spent 58.53\% of the time in Auto mode with 28 interactions, spent 20.03\% of the time in Conversation mode with 19 interactions, and spent 21.44\% of the time in Manual Control mode with 10 interactions.
}
\label{tab:activity_breakdown}
\begin{tabular}{@{}lcccccc@{}}
\toprule
    & \multicolumn{2}{c}{Auto} & \multicolumn{2}{c}{Conversation} & \multicolumn{2}{c}{Manual Control} \\
    & Ratio(\%)          & Count         & Ratio(\%)          & Count         & Ratio(\%)           & Count          \\ \midrule
P11 & 59.77         & 25            & 37.52         & 21            & 2.70           & 4              \\
P12 & 91.66         & 13            & 8.16          & 9             & 0.18           & 1              \\
P13 & 67.88         & 12            & 30.56         & 9             & 1.56           & 1              \\
P14 & 64.86         & 17            & 33.94         & 15            & 1.20           & 1              \\
P15 & 58.53         & 28            & 20.03         & 19            & 21.44          & 10             \\ \bottomrule
\end{tabular}
\end{table}
\begin{table}[]
\caption{Analysis of participants' requests to the system during conversation mode. We defined three types of queries, General Query, Specific Query, and Command Query, and classified each participant request into one of them. The ratios here are simply the count percentages over total counts.}
\Description{
The table provides a analysis of participants' requests made to the system during conversation mode. Requests are categorized into three types: General Query, Specific Query, and Command Query. The ratio (percentage) and count of each type of request are shown for each participant, along with the total number of requests. The following summarizes the data for each participant: P11 made 11.43\% of their requests as General Queries (4 requests), 45.71\% as Specific Queries (16 requests), and 42.86\% as Command Queries (15 requests), with a total of 35 requests. P12 made 30.00\% of their requests as General Queries (3 requests), 10.00\% as Specific Queries (1 request), and 60.00\% as Command Queries (6 requests), with a total of 10 requests. P13 made 61.54\% of their requests as General Queries (8 requests), 38.46\% as Specific Queries (5 requests), and 0.00\% as Command Queries (0 requests), with a total of 13 requests. P14 made 14.29\% of their requests as General Queries (4 requests), 46.43\% as Specific Queries (13 requests), and 39.29\% as Command Queries (11 requests), with a total of 28 requests. P15 made 8.33\% of their requests as General Queries (2 requests), 25.00\% as Specific Queries (6 requests), and 66.67\% as Command Queries (16 requests), with a total of 24 requests.
}
\label{tab:request_breakdown}
\resizebox{\columnwidth}{!}{%
\begin{tabular}{@{}cccccccc@{}}
\toprule
                     & \multicolumn{2}{c}{General Query} & \multicolumn{2}{c}{Specific Query} & \multicolumn{2}{c}{Command Query} & \multirow{2}{*}{Total}\\
                     & Ratio (\%)       & Count      & Ratio (\%)       & Count      & Ratio (\%)        & Count       \\ \midrule
P11  & 11.43            & 4         & 45.71            & 16         & 42.86              & 15           &  35\\ 
P12  & 30.00            & 3         & 10.00            & 1          & 60.00              & 6           &  10\\ 
P13  & 61.54            & 8         & 38.46            & 5          & 0.00              & 0           &  13\\ 
P14  & 14.29            & 4         & 46.43            & 13         & 39.29              & 11           &  28\\ 
P15  & 8.33            & 2         & 25.00            & 6         & 66.67              & 16           &  24\\ \bottomrule
\end{tabular}
}
\end{table}


\subsection{Analysis of Participants Activity During The Task}
\label{sec:activity_breakdown}
We report the statistics of each participant's activity during the task by referring to the system's log and the video captured during the tasks. 
Tab.~\ref{tab:activity_breakdown} shows the analysis of their time spent on the three modes as specified in Sec.~\ref{sec:implementation_button}.
We noticed that the activation quantity and duration of each mode varied significantly among participants.
P11, P13, P14, and P15 frequently used the conversation mode.
Notably, P11 spent nearly 40\% of the total time engaging in conversation with the robot.
In contrast, P12 barely used the conversation mode and relied on the auto mode for 90\% of the total time.
%Secondly, there was also a difference in the usage of the Auto and Manual control modes. 
P15 was the only participant who actively used the manual control mode.


\begin{table*}[]
\caption{Error analysis of outputs from MLLM.  We classified the errors into five categories and counted the number of them. Note that a single response could contain multiple errors, so the sum of errors does not match the total output of MLLM.}
\Description{
The table presents an error analysis of outputs from an MLLM (Multimodal Large Language Model). The errors are categorized into five types: Wrong Character Recognition, Wrong Object Recognition, Nonexistent Objects and Texts, Misunderstanding User Input, and Inaccurate User Input. It also includes a count of outputs with no errors. The total number of outputs is also provided, and a single response can contain multiple errors. The following describes the findings. For Scene Description, there were 31 instances of Wrong Character Recognition, 6 instances of Wrong Object Recognition, 11 instances of Nonexistent Objects and Texts, no occurrences of Misunderstanding or Inaccurate User Input, 117 outputs with No Error, and a total of 164 outputs. For Q\&A Response, there were 9 instances of Wrong Character Recognition, 6 instances of Wrong Object Recognition, 15 instances of Nonexistent Objects and Texts, 5 instances of Misunderstanding User Input, 1 instance of Inaccurate User Input, 21 outputs with No Error, and a total of 53 outputs.
}
\label{tab:hallucinations}

\begin{tabular}{@{}cccccccc@{}}
\toprule
                  & \begin{tabular}[c]{@{}c@{}}Wrong\\ Character \\ Recognition\end{tabular} & \begin{tabular}[c]{@{}c@{}}Wrong\\ Object\\ Recognition\end{tabular} & \begin{tabular}[c]{@{}c@{}}Nonexistent\\ Objects and \\Texts\end{tabular} & \begin{tabular}[c]{@{}c@{}}Misunderstanding\\ User\\ Input\end{tabular} & \begin{tabular}[c]{@{}c@{}}Inaccurate\\ User\\ Input\end{tabular} & \begin{tabular}[c]{@{}c@{}}No\\ Error\end{tabular} & \begin{tabular}[c]{@{}c@{}}Total \\ output\end{tabular} \\ \midrule
Scene Description & 31                                                  & 6                                                                    & 11                                                                   & -                                                                     & -                                                          & 117                                                   & 164                                                          \\
Q\&A Response     & 9                                                   & 6                                                                    & 15                                                                   & 5                                                                     & 1                                                          & 21                                                    & 53                                                           \\ \bottomrule
\end{tabular}%
\end{table*}
\begin{table}[]
\caption{The statistics of the usage of each description level.
Usage statistics for each description level are calculated by normalizing the duration of each level with respect to the total duration of the experiment.
}
\Description{
The table presents the usage statistics for different description levels—Concise, Balanced-Length, and Descriptive—by normalizing the duration of each level with respect to the total duration of the experiment for five participants (P11 to P15). P11 used Concise descriptions 0.10\% of the time, Balanced-Length descriptions 76.46\% of the time, and Descriptive descriptions 23.43\% of the time. P12 used Concise descriptions 15.22\% of the time, Balanced-Length descriptions 29.88\% of the time, and Descriptive descriptions 54.91\% of the time. P13 used Concise descriptions 0.19\% of the time, Balanced-Length descriptions 49.14\% of the time, and Descriptive descriptions 50.66\% of the time. P14 used Concise descriptions 0.15\% of the time, Balanced-Length descriptions 87.94\% of the time, and Descriptive descriptions 11.91\% of the time. P15 used Concise descriptions 0.14\% of the time, Balanced-Length descriptions 99.86\% of the time, and Descriptive descriptions 0.00\% of the time.
}
\label{tab:description_level}
\begin{tabular}{@{}lccc@{}}
\toprule
    & Concise & Balanced-Length & Detailed \\ \midrule
P11 & 0.10\%  & 76.46\%         & 23.43\%     \\
P12 & 15.22\% & 29.88\%         & 54.91\%     \\
P13 & 0.19\%  & 49.14\%         & 50.66\%     \\
P14 & 0.15\%  & 87.94\%         & 11.91\%     \\
P15 & 0.14\%  & 99.86\%         & 0.00\%      \\ \bottomrule
\end{tabular}
\end{table}

\subsection{Analysis of Requests from Participants During Within The Conversation Mode}
\label{sec:request_breakdown}
In Tab.~\ref{tab:request_breakdown}, we further report the statistics of requests from participants within the conversation mode. 
Note that the total count of conversations in Tab.~\ref{tab:request_breakdown} is bigger than the conversation mode counts in Tab.~\ref{tab:activity_breakdown}, as multiple turns of conversation could happen in one conversation mode interaction.
We classify each verbal request into three categories. 
\begin{description}
    \item[General Query] Request general information in the surrounding area or in a particular direction.
    \item[Specific Query] Request detailed information about a specific object in the environment.
    \item[Command Query] Issue command to guide to destination, triggering ``Take-Me-There'' functionality or direction specification via conversation.
\end{description}

Overall, we discovered that although our system constantly provided environmental descriptions in auto mode, users still preferred to ask for general information about their surroundings or in a specific direction in conversation mode. 
For example, P13 predominantly made General Queries (61.54\%). 
Users also had diverse preferences when using our system. 
Some users such as P11 (45.71\%), P13 (38.46\%) and P14 (46.43\%) were interested in learning the specifics of POIs, reflecting the takeaways obtained in Sec.~\ref{sec:info_needs}. 
Some users such as P11 (42.86\%), P12 (60.00\%), P14 (39.29\%), and P15 (66.67\%) favored using conversation mode to instruct the robot to guide them to their destinations. 
In particular, by referencing Tab.~\ref{tab:activity_breakdown}, we can see that P11, P12, and P14 preferred conversation mode over manual control mode to issue commands. 
This validates the extrapolated idea in Sec.~\ref{sec:implication_directionspecification}.

\subsection{Error Analysis of Scene Description and Q\&A Responses}
\label{sec:error_analysis}
In Tab.~\ref{tab:hallucinations}, we report the accuracy of MLLM responses both during auto and conversation modes.
We manually analyzed the text output generated by MLLM and compared it with the logs of the images saved.
We classified and counted the errors made by MLLM into six categories.
\begin{description}
    \item[Wrong Character Recognition] Misrecognition of text, such as misreading signs.
    \item[Wrong Object Recognition] Misidentification of objects in the scene.
    \item[Nonexistent Objects and Texts] Mistakenly recognizing objects or text that are not present. Note that this differs from the previous two categories, where some similar objects or text were actually present.
    \item[Misunderstanding User Input] Misinterpreting a user’s question in conversation mode, such as providing an environmental description when asked to read text from a panel.
    \item[Inaccurate User Input] Errors made when the user asked about objects or text that were not present.
    \item[No Error] Accurate responses with no errors.
\end{description}
\red{
When multiple errors occur in a single sentence, errors of the same type are grouped together and counted as one. 
Errors of different types are counted separately. 
For instance, if there are multiple text recognition errors in a single sentence, they are counted as one text recognition error. 
If a sentence contains both text recognition errors and object recognition errors, each is counted separately as one text recognition error and one object recognition error.
Thus, note that the total number of errors may not match the total number of outputs.}

The results showed that 28.6\% of the outputs contained some form of error during scene descriptions whereas 60.3\% of conversation mode outputs had errors.
This difference is likely because users in conversation mode often asked for more detailed explanations, which led MLLM to attempt more complex responses and, as a result, made more mistakes.
This was particularly evident in the \textit{Nonexistent Objects and Texts} category, which accounted for only 0.07\% of errors during scene descriptions but significantly higher at 28.3\% in conversation mode.
This means that MLLM often generated descriptions of objects or text that did not exist in the environment when asked for more detailed information.
Character recognition errors were common in both modes, likely due to MLLM’s limitation in reading distant text. 
In a general sense, instead of complete failures, MLLM often partially misread the text or misidentified objects with similar-looking ones (\eg, mistaking a tall table for a reception desk).
Nevertheless, over 70\% of responses in the auto mode were accurate, demonstrating the overall usefulness of the system.

\subsection{Analysis of Usage of Each Description Level}
\label{sec:description_level_analysis}
In Tab.~\ref{tab:description_level}, we report the statistics of how much time participants spend their time using each description level.
The result shows that there were three types of usage during the study. 
P15 only used Balanced-Length mode, P11 and P14 used Balanced-Length mode most of the time while sometimes using Detailed mode, and P12 and P13 used Detailed mode most of the time. 

\begin{figure*}
    \centering
    \includegraphics[width=0.8\linewidth]{figure/boxplot.png}
    \caption{Box plot of evaluation with human experts in seven-point Likert points.}
    \Description{The figure displays box plots for four questions evaluating the generated descriptions: Q1. I think the generated descriptions are natural: The median score is 5, with a minimum of 2, a first quartile of 4, a third quartile of 6, and a maximum of 7. Q2. I think the generated descriptions are precise: The median score is 5, with a minimum of 1, a first quartile of 3, a third quartile of 6, and a maximum of 7. Q3. I think the generated descriptions are appropriate as descriptions provided by a navigation robot to blind people onsite: The median score is 4, with a minimum of 2, a first quartile of 3, a third quartile of 5, and a maximum of 7. Q4. I think the generated descriptions are appropriate as descriptions provided when I am there to explain to blind people onsite: The median score is 3, with a minimum of 1, a first quartile of 2, a third quartile of 4, and a maximum of 7.    
    }
    \label{fig:boxplot}
\end{figure*}


% \usepackage{graphicx}
\begin{table*}[]
\caption{Rating to seven-point Likert score questions (1: strongly disagree; 4: neutral; 7: strongly agree).}
\Description{The table presents ratings from five participants (P11, P12, P13, P14, and P15) on a set of questions based on a seven-point Likert scale, where 1 indicates "strongly disagree," 4 is "neutral," and 7 is "strongly agree." The median rating for each question is also provided. The following summarizes the responses for each question. For Q1, "I was able to explore the facility," P11 and P13 rated 4, while P12, P14, and P15 rated 6, resulting in a median score of 6. For Q2, "I was able to enjoy the exploration," P11 and P13 rated 4, while P12, P14, and P15 rated 6 or 7, with a median score of 6. For Q3, "I was able to gain an interest in the things around me," P11 rated 4, P13 rated 6, and P12, P14, and P15 rated 6, leading to a median score of 6. For Q4, "The interface of the system was easy to understand," P11, P14, and P15 rated 5, while P12 and P13 rated 6, resulting in a median score of 5. For Q5, "I want to explore where I am familiar with this system," P11, P12, and P14 rated 7, while P13 and P15 rated 6, leading to a median score of 7. For Q6, "I want to explore where I am unfamiliar with this system," P11, P14, and P15 rated 7 or 6, resulting in a median score of 6.}
\label{tab:likert}

\begin{tabular}{l|ccccc|c}
\toprule
\multicolumn{1}{c|}{}                                         & P11 & P12 & P13 & P14 & P15 & Median \\ \hline
Q1. I was able to explore the facility.                       & 4   & 6   & 4   & 6   & 6   & 6      \\
Q2. I was able to enjoy the exploration.                      & 4   & 7   & 4   & 6   & 6   & 6      \\
Q3. I was able to gain an interest in the things around me.   & 4   & 6   & 6   & 6   & 6   & 6      \\
Q4. The interface of the system was easy to understand.       & 5   & 6   & 6   & 5   & 5   & 5      \\
Q5. I want to explore where I am familiar with this system.   & 7   & 7   & 6   & 7   & 6   & 7      \\
Q6. I want to explore where I am unfamiliar with this system. & 7   & 6   & 6   & 7   & 6   & 6      \\ \bottomrule
\end{tabular}%
\end{table*}
\begin{table}[]
\caption{Scores for the Raw TLX provided by each participant. Lower total scores indicate a lower workload. Each item is scored on a scale from 1 to 10, where 1 represents a lower level, and 10 represents a higher level of Mental Demand, Physical Demand, Temporal Demand, Effort, and Frustration. For Performance, 1 indicates good performance, and 10 indicates poor performance.}
\Description{
The table presents Raw TLX scores from five participants (P11, P12, P13, P14, and P15), measuring subjective workload across six dimensions: Mental Demand, Physical Demand, Temporal Demand, Performance, Effort, and Frustration. Each dimension is rated on a scale from 1 to 10, where higher scores generally indicate higher levels of demand, effort, or frustration, except for Performance, where a higher score indicates poorer performance (1: Good, 10: Poor). Lower total scores signify a lower overall workload.
Participant P11 reported low levels of Mental Demand (2), Physical Demand (2), Temporal Demand (2), and Effort (3). However, they scored higher in Performance (7) and Frustration (8), leading to a total score of 24. Participant P12 had low scores across all dimensions: Mental Demand (2), Physical Demand (2), Temporal Demand (3), Performance (2), Effort (2), and Frustration (4), resulting in the lowest total score of 15. Participant P13 scored higher in Mental Demand (5), Effort (5), and Performance (5), moderate in Physical Demand (3), and low in Temporal Demand (1) and Frustration (1), totaling a score of 20. Participant P14 reported low Mental Demand (3) and Physical Demand (2), but higher Temporal Demand (5), Performance (7), Effort (6), and Frustration (3), culminating in a total score of 26. Participant P15 had low Mental Demand (2) and Temporal Demand (2), but higher Physical Demand (6), Performance (5), Effort (6), and Frustration (7), leading to the highest total score of 28.
}
\label{tab:tlx}
\begin{tabular}{l|ccccc|c}
\toprule
                & P11 & P12 & P13 & P14 & P15 & Median \\ \hline
Mental Demand   & 2   & 2   & 5   & 3   & 2   & 2      \\
Physical Demand & 2   & 2   & 3   & 2   & 6   & 2      \\
Temporal Demand & 2   & 3   & 1   & 5   & 2   & 2      \\
Performance     & 7   & 2   & 5   & 7   & 5   & 5      \\
Effort          & 3   & 2   & 5   & 6   & 6   & 5      \\
Frustration     & 8   & 4   & 1   & 3   & 7   & 4      \\ \hline
Total Score     & 24  & 15  & 20  & 26  & 28  &        \\ \bottomrule
\end{tabular}
\end{table}

\subsection{Scene Description Quality Evaluation}
\label{sec:quality_eval}
Finally, to analyze the quality of the MLLM-generated scene descriptions from the human expert perspective, we conducted a survey with human museum guides and asked them to evaluate using a seven-point Likert scale. 
The participants were presented with images captured by the robot, each accompanied by its corresponding generated description, and were asked to evaluate the descriptions in a survey, as shown in Fig.~\ref{fig:boxplot}. 
The survey was conducted in a counterbalanced manner to mitigate potential biases.
During the main study, 164 descriptions were \rrred{generated}, and we randomly sampled half (82) of the total descriptions for evaluation.
\rrred{
The randomly sampled descriptions contain mixed levels of detail.
}
Each description is evaluated by three to four participants.
In total, 56 museum guides participated in the evaluation, with each randomly assessing five descriptions. 
There were 32 males and 20 females, and four participants did not report their gender.
Their average age was 39.6 years, with an average of 5.9 years of experience as a museum guide. 
On seven-point Likert scale items, the median self-reported familiarity with museums was 5.0, and the familiarity with LLMs was 4.0 (1: very unfamiliar, 4: neutral, and 7: very familiar).
Our analysis revealed that the experts generally perceived the generated descriptions as somewhat natural (Q1) and precise in describing an image (Q2) as shown by their median of five.
Meanwhile, they found the generated descriptions less suitable as image descriptions for blind people (Q3) and as onsite descriptions provided by experts for blind people (Q4).



\subsection{Usability and Workload Evaluation}
In Tab.~\ref{tab:likert}, we report the results of seven-point Likert items. 
For Likert items, a median score of five or higher indicates that participants generally responded positively.
The total SUS for P11 to P15 were 72.5, 80, 90, 82.5, and 77.5, respectively, showing acceptable usability of all being above 70~\cite{bangor2009determining}. 
The total Raw TLX scores for P11 to P15 were 24, 15, 20, 26, and 28, respectively. 
We show the distribution of Raw TLX scores in Tab.~\ref{tab:tlx}.
Raw TLX~\cite{byers1989traditional}, a simplified version of NASA TLX~\cite{hart2006nasa}, is known to have a high correlation with NASA TLX, and the total NASA-TLX scores for people with special needs typically ranged from 26 to 48 in previous research~\cite{hertzum2021reference}. 
Overall, our total Raw TLX scores may suggest that participants did not experience a significant load during the task.
We also observed that the median value for mental, physical, and temporal demand was relatively lower, scoring 2. 
This is likely due to the robot navigating them, allowing participants to explore without being burdened by these demands. 
Nonetheless, a relatively higher median value was observed for Performance, Effort, and Frustration, indicating that some users experienced a lack of satisfaction with the exploration experience provided by the system. 

\subsection{Qualitative Analysis}
\subsubsection{Positive Feedback}
All participants expressed their appreciation for the experience of wandering around a building to explore without specific destinations in mind with the help of our system:
\newanswer[\label{P12IWantThis}]\textit{``
When the camera explains things it recognizes, like how bright the room is or what the floor looks like, or what objects are placed where, I found myself nodding in agreement multiple times, like, ``Oh, so this is how it looks.'' 
I remember when I first held the suitcase robot, I deeply empathized with guide dog users. I thought, ``Oh, so this is what it's like to have a guide dog.'' However, since I can't take care of a guide dog, I’ve given up on that option. 
And now, with this navigation system that explains various situations, it's exactly what I need. It’s not just about setting a destination and getting there but feeling the freedom to explore spontaneously. For example, the ability to roam a large shopping mall freely and explore on a whim feels like true freedom to me. Instead of pre-planning every move or relying on a guide, I could simply grab my suitcase and decide to venture out spontaneously.''} (P12)

\red{
The same participant, P12, who had been to the facility previously, noted that they still had new discoveries with the system:
\newanswer[\label{P12IHaveBeen}]\textit{``
I've been to this museum before, but when the guide explained things to me back then, it was more like a general explanation about the atmosphere and such. 
But earlier with the system, there was a very detailed explanation that came out of the suitcase. Like, about how bright sunlight comes [...] 
There were things I didn’t know that made me learn new stuff, even though I thought I knew about the facility.''} (P12)
Also, P12 and P14 noted the feeling of relief not relying on sighted assistance:
\newanswer[\label{P14ThereHaveBeenNoSystem}]\textit{``
I don’t think there has ever been a system that explains your surroundings while walking. [...]
When walking with other people, I often find myself feeling a sense of obligation. I worry that they’re putting in extra effort to describe things because I can’t see. And then I feel like I have to respond to them since they’re trying so hard—which can be exhausting. But with this system, I feel I can go strolling by myself.''} (P14)
}


Participants also noted the functionality to go to an aforementioned destination and Q\&A functionality particularly useful:
\newanswer[\label{P11TakeMeBack}]\textit{``(The ``Take-Me-There'' functionality is) I think it's wonderful. After all, spatial awareness is difficult, so going back to landmarks is very important. If it is accurate, I think it's great because it can be extremely helpful for spatial cognition.''} (P11) and
\newanswer[\label{P14Q&A}]\textit{``When engaging in a conversation, not knowing what kind of response you'll get, the feeling of unease and excitement that's both a plus and a minus, I think. But I found it really great that you can still ask questions. So even if the response you get doesn't answer your question, or even if it's just ``I don't know,'' the fact that you can at least ask is important.''} (P14) 

\subsubsection{Adjusting Detail of Description}
When we discussed their preference in the level of detail of descriptions, all participants described that it would rather depend on the scenario they are in:
\newanswer[\label{P12NormalMode}]\textit{``It might depend on the location, but I know I can get detailed information in Q\&A functionality. So, for familiar places, the Balanced-Length mode might be fine. However, there are parts where I'd want the Detailed Description mode for unfamiliar places. For example, switching between modes could be useful, like having Detailed Description mode first for explanations about the room's brightness and how easy it is to walk around. ''} (P12)

\subsubsection{Comments to Improve the System}
\label{sec:improve}
Participants suggested various improvements to the system.
One particular suggestion was to incorporate functionality for the robot to understand sounds. 
As the experiment location was a science museum, various exhibits emitted sounds.
P13 noted that they would like to inquire about the sound sources, which were not supported by the system:
\newanswer[\label{P13Sound}]\textit{``We are extremely sensitive to sounds, and it becomes a point of interest. At a place like the exhibition hall we're visiting this time, various sounds are coming from all directions. This prompts questions like, ''What's happening at that sound over there?'' Therefore, it would be advantageous if we could ask specific questions like, ``What's that sound coming from the right?'' ''} (P13)

Also, four participants (P11-P13 and P15) found the descriptions from the system still insufficient to explore, as described in the following comments:
\newanswer[\label{P15Insufficient}]\textit{``The place we did the task this time was quite out of the ordinary. Even if you were walking around with my family, I think they would also have difficulty explaining it. Therefore, I felt it might still be somewhat challenging for machines to handle this kind of thing. However, I did feel it was good that I got a sense of what was there. But when it comes to the actual detailed explanations, it was not there [...]''} (P15)

\subsubsection{Specification of Proceeding Direction}
While we introduced all functionality to participants within the training session, we observed that only P15 used the functionality to specify which way to proceed via a button or conversation.
P15 tended to use the functionality when P15 was interested in a specific object:
\newanswer[\label{P15DirectionSpecification}]\textit{``It seems that when I was told, ``There's something on the right,'' I tried to approach toward it because I wanted to get closer when I used something like that.''} (P15)
\section{Limitations \& Challenges}
%From the initial CAD design to real-world deployment of CavePI, we identify several engineering challenges and practical limitations. Key observations and potential improvements in the mechanical design, perception, planning, and control subsystems are outlined below.

\subsection{Design Aspects}
CavePI’s onboard components are packed within a single pressure-sealed tube, leaving minimal space for additional hardware. Notably, major housing volume is occupied by the LiPo battery, which can be replaced with a more compact alternative to create space for future add-ons. We also plan to upgrade the perception SBC from the existing Jetson Nano to a Jetson Orin Nano that offers $4\times$ memory with more advanced GPU resources. The redesigned system will also incorporate a magnetic switching mechanism for seamless power control. Outside of the enclosure, the 4 thrusters are arranged to allow control over surge, heave, roll, and yaw motions, however, their adjacent placement induces mutual turbulence and reduces thrust efficiency. To address this, we are investigating alternative thruster placements that will enhance motion dynamics without affecting the robot's buoyancy properties and dynamic stability.

\begin{figure}[t]
    \centering
    % \vspace{2mm}
    \includegraphics[width=\columnwidth]{figures/PerceptionFailures.png}%
    %\vspace{-1mm}
    \caption{A few perception failure modes are shown. The down-facing camera falsely detects (a) the edge of the laboratory tank; (b) tree roots as the caveline, resulting in incorrect tracking.
    }%
    \vspace{-2mm}
    \label{fig:perception_failure}
\end{figure}

% \begin{figure*}[ht]
% \centering
% \begin{subfigure}[]{0.24\linewidth}
% \includegraphics[width=\linewidth]{figures/Incorrect_Detection.jpg}%
% % \vspace{-1mm}
% \caption{}
% \label{fig:Incorrect_Detection}
% % \vspace{-2mm}
% \end{subfigure}
% \begin{subfigure}[]{0.24\linewidth}
% \includegraphics[width=\linewidth]{figures/Drifting.jpg}%
% % \vspace{-1mm}
% \caption{}
% \label{fig:Drifting}
% % \vspace{-2mm}
% \end{subfigure}
% \begin{subfigure}[]{0.24\linewidth}
% \includegraphics[width=\linewidth]{figures/Pitching.jpg}%
% % \vspace{-1mm}
% \caption{}
% \label{fig:Pitching}
% % \vspace{-2mm}
% \end{subfigure}
% \begin{subfigure}[]{0.24\linewidth}
% \includegraphics[width=\linewidth]{figures/Overshooting.jpg}%
% % \vspace{-1mm}
% \caption{}
% \label{fig:Overshooting}
% % \vspace{-2mm}
% \end{subfigure}
% \caption{A few tracking failure modes, observed during the field experiment, are illustrated. CavePI is (a) Falsely detecting a tree's root as the caveline; (b) Experiencing significant lateral movement under strong water currents; (c) Pitching upward in the presence of water currents, attributed to its slightly back-heavy design; (d) Overshooting its intended trajectory while moving downstream in water currents.
% }
% \label{fig:failures}
% \vspace{-3mm}
% \end{figure*}


% Talk about: limited space for sensor add-ons and a larger more powerful computer, more visual commands (gesture) for diver-robot co-op, buoyancy adjustment and better thruster config, 

\subsection{Perception Challenges}
The current onboard sensor suite offers a limited understanding of the surrounding 3D environment. For instance, the Ping2 sonar provides only 1D depth measurements, which we will replace with an advanced $360^\circ$ scanning sonar system for enhanced spatial awareness. The scanning sonar, combined with other state estimation sensors, will map the environment and effectively avoid obstacles during navigation. Moreover, the two cameras currently operate independently for different purposes without synchronization. In the next iteration, we propose incorporating a $45^\circ$ slanted camera and combining all the visual feeds into a mosaic vision. This advanced vision system will offer wider FOV with more peripheral information and improve visual servoing performance. Additionally, the current caveline detection model occasionally misidentifies objects such as submerged tree roots as part of the caveline as shown in Fig.~\ref{fig:perception_failure}b, and it sometimes fails to detect the line under low-light conditions. These issues lead to significant tracking inaccuracies. Although more computationally intensive models might address these shortcomings, they are currently infeasible due to hardware limitations. With the proposed design modifications for the next iteration, we plan to adopt a more robust detection model to improve CavePI's tracking performance. Furthermore, hand gesture recognition~\cite{xu2008natural} will be integrated to enable seamless cooperation between divers and CavePI during underwater cave operations. With this advanced sensor setup, we will deploy advanced SLAM algorithms, such as SVIn2~\cite{rahman2022svin2} for robust navigation in GPS-denied underwater cave environments. 

%\JI{I assumed there will be some challenging cases for detection/segmentation - shown here}


\begin{figure}[h]
    \centering
    % \vspace{2mm}
    \includegraphics[width=\columnwidth]{figures/Tracking_Failure.png}%
    %\vspace{-1mm}
    \caption{A few tracking failure modes are shown: (a) CavePI is correctly following the line; (b) lateral drift under strong currents; (c) pitching upward due to currents, attributed to its back-heavy design; (d) overshooting its intended trajectory while moving downstream.
    }%
    \vspace{-2mm}
    \label{fig:tracking_failure}
\end{figure}


\subsection{Smooth 6-DOF Control}

Although CavePI is a 6-DOF AUV capable of maneuvering in a 3D environment, it currently offers active control over only four DOFs -- surge, heave, roll, and yaw. In future iterations, we intend to reposition its thrusters to enable control over the remaining two DOFs -- pitch and sway -- thereby enhancing the robot’s maneuverability in complex underwater environments. Furthermore, CavePI's autonomous control primarily utilizes a proportional-derivative (PD) controller, which performs effectively under conditions with minimal environmental disturbances. However, this straightforward control strategy becomes unstable in more complex scenarios, such as when water currents are present. In such conditions, lower proportional gains are insufficient for adjusting CavePI’s yaw to align with the cave line, while higher proportional gains cause overshooting in the robot's trajectory during sharp turns (see Fig.~\ref{fig:tracking_failure}\,d), attributed to both perception latency and the limitations of the current control method. Additionally, nonlinear (primarily quadratic) drag forces significantly impact the robot's stability and must be incorporated into the control system design. To overcome these challenges, we are developing a more robust nonlinear adaptive control system to enhance stability. Concurrently, we are optimizing the communication between the perception and control modules to reduce latency and improve overall responsiveness.




%\JI{No images showing the effects of current or difficulty in failure cases? If you are showing some failure cases in video - perhaps use a frame or two to show here}

% CavePI's autonomous control primarily relies on a proportional-derivative (PD) controller, which is effective for planar motions such as forward movement, yaw, and depth holding. However, this simple control strategy proves unstable for more complex maneuvers such as roll and pitch. In narrow underwater caves, caveline is often placed along the sidewall rather than the floor, which requires roll adjustments to maintain visibility in the down-facing camera’s field of view. Additionally, during sharp turns, we observe frequent overshoots in the robot's trajectory (see Fig.~\ref{fig:heading_control}\,d), attributed to both perception latency and the limitations of the current control method. To address these challenges, we are developing a more robust PID control system for better stability. Simultaneously, we are optimizing communication between the perception and control bays to reduce latency and improve overall responsiveness.


\section{Conclusion}
Towards realizing a scalable map-less guide system that assists blind people in exploring, we developed WanderGuide, a robotic guide  system designed to provide real-time descriptions of surroundings and to offer conversation functionalities that allow users to specify their destinations or ask questions.
The formative study with ten blind participants revealed that there are three types of preferences over the levels of details of the descriptions generated by the system.
In a subsequent main study with five blind participants, all of them expressed appreciation for the experience of wandering freely without a fixed destination, as well as a desire to use the system for exploring both familiar and unfamiliar areas. 
The study further revealed that including audio recognition would be the immediate next step for developing our system. 
It also revealed that customizing to diverse user preferences is important and that MLLM is the key bottleneck of the technology development of our system.
We hope this research contributes to the potential deployment of robotic guide systems in general use cases, enabling blind users to explore independently.

\begin{acks}
We would like to thank all the participants in our user study.
We are also deeply thankful to Mori Building Co., Ltd. for providing the experimental location.
Finally, we thank all members of Miraikan, including Hironobu Takagi and Hiromi Kurokawa,  and
the Consortium for Advanced Assistive Mobility Platform for their support.
This work was supported by JSPS KAKENHI (JP23KJ2048).
\end{acks}


\bibliographystyle{ACM-Reference-Format}
\bibliography{main}

\appendix

\lstset{
  backgroundcolor=\color{gray!20}, % 20% gray background
  basicstyle=\ttfamily\footnotesize, % Monospaced font with smaller size
  breaklines=true,                 % Automatically break long lines
  frame=single,                    % Frame the listing
  framerule=0pt,                   % No frame border
  xleftmargin=5pt, xrightmargin=5pt % Add some margin around the text
}


\section{Appendix: Prompts to MLLM}
In this section, we list full prompts to MLLM and LLM, which were used in this paper.


\subsection{Prompt Used For Translating Native Language to English}
\label{appendix:translate}
As the research was conducted in a country where English is not spoken, we used the below prompt to translate any data obtained in the native language throughout the paper. 
Note that the authors manually refined the output to keep the nuances of the original language.
This prompt was also used to translate the prompt engineered in the native language, which was fed into the MLLM for generating scene descriptions.
\begin{lstlisting}
Please translate the given <Native Language> to English. Make sure to keep the nuances and context of the original text.
<Native Language>:  Text written in native Language
English:
\end{lstlisting}


\subsection{Prompt Used In The Formative Study}
\label{appendix:prompt_formative}
Below is the prompt used to generate descriptions in the formative study.
\begin{lstlisting}
# Instructions  
Please describe the image.  
The text you generate will be read directly to visually impaired individuals. Make sure your description is engaging so that visually impaired individuals can enjoy listening to it.  
To describe the image, you must follow the rules outlined below.  

## Rules you must strictly follow to comply with the instructions  

### Rules on what you should do  
1. Since visually impaired individuals will listen while walking, provide a description in one cohesive sentence. Please describe as many objects and their details as possible.  
2. Generate the description in 1 to 4 sentences in total.  
3. If necessary, first describe the overall layout or the general view of the location.  
4. After that, identify the objects located on the left, in front, and on the right of the image, and explain the information required to understand the scene.  
5. Always describe the scene in the following order: overall view, left side, front, right side.  
6. When describing, use a tone similar to a guide for the visually impaired, such as "On the right, there is..."  
7. If there is a store, make sure to include information about what the store offers (for example, the type of cuisine if it is a restaurant). Also, include a description of the store's atmosphere (e.g., bright, calm).  
8. Only describe objects that are clearly visible. Include descriptions of distinctive objects.  
9. Create a description that is enjoyable to listen to and allows the listener to learn about their surroundings.  

### Rules on what you should not do  
10. Avoid unnatural words for the listener, such as "the image" or "viewpoint."  
11. You do not need to include common and unremarkable objects (e.g., tables and chairs in a restaurant) in the description.  
12. If there is nothing to describe in a particular direction (e.g., there is nothing on the right), you do not need to mention that direction.  
13. Do not describe the floor, ceiling, shadows, distant unclear objects, or the brightness or darkness of the lighting.  

## Response Format  
If you generate a good description that follows the above rules, you will receive a tip.  
Please respond in JSON format.  
Include the image description under the "description" key.  
Start your response with ```json\n{ to indicate the beginning of the JSON.  

Here is an example of a response:  
```json  
{  
"description": "<description of the surroundings>",  
}  
```
\end{lstlisting}


\subsection{Prompts Used In The Formative Study}
\label{appendix:prompt_main}
This section provides prompts used in the main study.

\subsubsection{The Prompt for Generating Detailed Description}
Below is the prompt used to generate a detailed description.
\begin{lstlisting}
# Instructions  
Please describe the image.  
You are given three images that provide a view of your left, right, and front, as well as a view from a fisheye camera that captures the overall view from a high point of view.
The text you generate will be read directly to visually impaired individuals.  
When writing the description, please aim to make it appealing so that it creates an enjoyable experience for the listener.  
The most important thing is to provide detailed and specific information so that the listener can feel as if they are actually at the scene.  
Being specific means describing the category or name of objects, their condition, and the role they play.  
For example, a description like "circular wooden object" is not specific, but "a circular wooden table with YYY written on the nearby guide" is specific.  
Similarly, "iron exhibit" is vague, while "a tall, iron exhibit, possibly XXX" is specific.  
When describing the image, you must follow the rules below.

## Rules that must be followed to comply with the instructions  
1. The description must be something that a visually impaired person can listen to while walking. Provide a coherent description in one block of text. Explain as many objects and their details as possible.  
2. Keep the description to 3-4 sentences at most (120-240 characters).  
3. Use polite language (honorifics).  
4. Identify and describe objects located in the overall scene, to the left, front, and right that are necessary to understand the scene.  
5. Only describe clearly visible objects. Include distinctive objects in your description.  
6. Always describe the overall scene first, followed by objects on the left, in front, and then on the right.  
7. Strive to include the following information:
    - Details about the building's interior and decoration.
    - Information about the layout of the building (such as whether the front is open, where walls are, and the directions one can go).
    - Information on the surrounding brightness and the amount of light coming through windows.
    - Information about people in the surroundings, their actions, clothing colors, and whether they are staff or customers.
    - If it's a store, provide information on whether the entrance is open like a terrace and whether guide dogs can wait there.
    - Include information on visible stores or exhibits. Be sure to mention their category (e.g., the type of food if it's a restaurant, or what kind of place the exhibit is). If possible, include the name of the place. For exhibits, state whether they are interactive or for viewing only.
    - When describing objects, be specific (mention the category or name). For example, if there's a counter, specify if it looks like a cafe counter.
    - Mention people walking toward the front if there's a risk of collision.
    - Use numbers when explaining object positions (e.g., "5 meters to the right").
    - If there is a sign or guidepost, describe what it is and read out the text written on it.
    - Read out visible text.
    - Use adjectives like futuristic, stylish, modern, or classic to make the exploration more enjoyable and to help the listener visualize the scene.
8. Do not use unnatural words for the listener like "image," "viewpoint," or "overall."
9. If there's nothing to describe in a certain direction (e.g., nothing on the right side), do not describe that direction.  
10. Do not summarize or conclude with a description of the overall direction or scene when finishing the explanation.  
11. Do not describe anything not visible in the image. Do not lie or hallucinate details.

## Response Format  
If you follow the rules above, you will receive a tip.  
If you ignore the rules, you will be penalized and have to pay a fine.  
Please do your best to comply with these instructions.

Respond in JSON format.  
First, include the initial description of the image under the "initial_description" key.  
Next, include points for improvement under the "improve_thoughts" key.  
Finally, include the revised image description under the "description" key.  
Start your response with `json\n{`.  
Here is an example response:

```json
{
"initial_description": "<initial description>",
"improve_thoughts": "<points for improvement>",
"description": "<revised description>",
}
```

\end{lstlisting}

\subsubsection{The Prompt for Generating Balanced-Length Description}
Below is the prompt used to generate a balanced-length description.
\begin{lstlisting}
# Instructions  
Please describe the image.  
You are given three images that provide a view of your left, right, and front, as well as a view from a fisheye camera that captures the overall view from a high point of view.
The text you generate will be read directly to visually impaired individuals.  
Keep the description concise, but aim to make it appealing and enjoyable for the listener.  
The most important thing is to provide detailed and specific information so that the listener can feel as if they are actually at the scene.  
Being specific means describing the category or name of objects, their condition, and the role they play.  
For example, a description like "circular wooden object" is not specific, but "a circular wooden table with YYY written on the nearby guide" is specific.  
Similarly, "iron exhibit" is vague, while "a tall, iron exhibit, possibly XXX" is specific.  
When describing the image, you must follow the rules below.

## Rules that must be followed to comply with the instructions  
1. The description must be something that a visually impaired person can listen to while walking. Provide a coherent description in one block of text. Explain as many objects and their details as possible.  
2. Keep the description to 2-3 sentences at most (60-120 characters).  
3. Use polite language (honorifics).  
4. Identify and describe objects located to the left, front, and right that are necessary to understand the scene.  
5. Only describe clearly visible objects. Include distinctive objects in your description.  
6. Always describe objects in the following order: left, front, and right.  
7. Strive to include the following information:
    - If it's a store, provide information on whether the entrance is open like a terrace and whether guide dogs can wait there.
    - Include information on visible stores or exhibits. Be sure to mention their category (e.g., the type of food if it's a restaurant, or what kind of place the exhibit is). If possible, include the name of the place. For exhibits, state whether they are interactive or for viewing only.
    - When describing objects, be specific (mention the category or name). For example, if there's a counter, specify if it looks like a cafe counter.
    - Mention people walking toward the front if there's a risk of collision.
    - Use numbers when explaining object positions (e.g., "5 meters to the right...").
    - If there is a sign or guidepost, describe what it is and read out the text written on it.
    - Read out visible text.
8. Do not use unnatural words for the listener like "image," "viewpoint," or "overall."
9. If there's nothing to describe in a certain direction (e.g., nothing on the right side), do not describe that direction.  
10. Do not summarize or conclude with a description of the overall direction or scene when finishing the explanation.  
11. Do not describe objects if you cannot provide specific information about them.  
12. Do not include information about people in the surroundings unless there is a risk of collision.  
13. Do not include information about the amount of light or brightness in the surroundings.  
15. Do not use subjective adjectives like futuristic, stylish, modern, or classic.  
16. Do not describe anything not visible in the image. Do not lie or hallucinate details.

## Response Format  
If you follow the rules above, you will receive a tip.  
If you ignore the rules, you will be penalized and have to pay a fine.  
Please do your best to comply with these instructions.

Respond in JSON format.  
First, include the initial description of the image under the "initial_description" key.  
Next, include points for improvement under the "improve_thoughts" key.  
Finally, include the revised image description under the "description" key.  
Start your response with `json\n{`.  
Here is an example response:

```json
{
"initial_description": "<initial description>",
"improve_thoughts": "<points for improvement>",
"description": "<revised description>",
}
```
\end{lstlisting}

\subsubsection{The Prompt for Generating Concise Description}
Below is the prompt used to generate a concise description.
\begin{lstlisting}
# Instructions  
Please describe the image.  
You are given three images that provide a view of your left, right, and front, as well as a view from a fisheye camera that captures the overall view from a high point of view.
The text you generate will be read directly to visually impaired individuals.  
The description should be concise and minimal, allowing the listener to quickly understand their surroundings.  
Visually impaired individuals are listening to the image description to locate their destination.  
The most important thing is to provide detailed and specific information so that the listener can feel as if they are actually at the scene.  
Being specific means describing the category or name of objects, their condition, and the role they play.  
For example, a description like "circular wooden object" is not specific, but "a circular wooden table with YYY written on the nearby guide" is specific.  
Similarly, "iron exhibit" is vague, while "a tall, iron exhibit, possibly XXX" is specific.  
When describing the image, you must follow the rules below.

## Rules that must be followed to comply with the instructions  
1. The description must be something that a visually impaired person can listen to while walking. Provide a coherent description in one block of text.  
2. Keep the description to 1-2 sentences at most (0-60 characters).  
3. Use polite language (honorifics).  
4. Identify and describe objects located to the left, front, and right that are necessary to understand the scene.  
5. Only describe clearly visible objects. Include distinctive objects in your description.  
6. Always describe objects in the following order: left, front, and right.  
7. Strive to include the following information:
    - If it's a store, provide information on whether the entrance is open like a terrace and whether guide dogs can wait there.
    - Include information on visible stores or exhibits. Be sure to mention their category (e.g., the type of food if it's a restaurant, or what kind of place the exhibit is). If possible, include the name of the place. For exhibits, state whether they are interactive or for viewing only.
    - Use numbers when explaining object positions (e.g., "5 meters to the right...").
    - If there is a sign or guidepost, describe what it is and read out the text written on it.
    - Read out visible text.
8. Only convey specific information.
9. Keep the description short, direct, and concise.  
10. Do not use unnatural words for the listener like "image," "viewpoint," or "overall."
11. If there's nothing to describe in a certain direction (e.g., nothing on the right side), do not describe that direction.  
12. Do not summarize or conclude with a description of the overall direction or scene at the beginning or end of the explanation.  
13. Do not describe decorations. Simply convey what is there and provide specific information.  
14. To keep the length minimal, do not include subjective adjectives.  
15. Do not include unnecessary information that does not help the listener locate their destination (e.g., details about furniture such as chairs or tables).  
16. Do not describe objects if you cannot provide specific information about them.  
17. Do not include information about people in the surroundings unless there is a risk of collision.  
18. Do not include information about the amount of light or brightness in the surroundings.  
19. Do not describe anything not visible in the image. Do not lie or hallucinate details.

## Response Format  
If you follow the rules above, you will receive a tip.  
If you ignore the rules, you will be penalized and have to pay a fine.  
Please do your best to comply with these instructions.

Respond in JSON format.  
First, include the initial description of the image under the "initial_description" key.  
Next, include points for improvement under the "improve_thoughts" key.  
Finally, include the revised image description under the "description" key.  
Start your response with `json\n{`.  
Here is an example response:

```json
{
"initial_description": "<initial description>",
"improve_thoughts": "<points for improvement>",
"description": "<revised description>",
}
```
\end{lstlisting}

\end{document}
