%% For submission and review of your manuscript please change the
%% command to \documentclass[manuscript, screen, review]{acmart}.
%%
%% When submitting camera ready or to TAPS, please change the command
%% to \documentclass[sigconf]{acmart} or whichever template is required
%% for your publication.

% \documentclass[manuscript,review,anonymous]{acmart} 
\documentclass[sigconf]{acmart} 

\AtBeginDocument{%
  \providecommand\BibTeX{{%
    Bib\TeX}}}


\setcopyright{acmlicensed}
\copyrightyear{2025}
\acmYear{2025}
\setcopyright{cc}
\setcctype{by}
\acmConference[CHI '25]{CHI Conference on Human Factors in Computing Systems}{April 26-May 1, 2025}{Yokohama, Japan}
\acmBooktitle{CHI Conference on Human Factors in Computing Systems (CHI '25), April 26-May 1, 2025, Yokohama, Japan}\acmDOI{10.1145/3706598.3713788}
\acmISBN{979-8-4007-1394-1/25/04}

\newcounter{answernum}
\setcounter{answernum}{0}
\newcommand{\newanswer}[1][]{\refstepcounter{answernum}{#1}\textbf{C{\theanswernum}}:}
\newcommand{\refanswer}[1][]{A{#1}}

\def\eg{{\it e.g.}}
\def\cf{{\it c.f.}}
\def\ie{{\it i.e.}}
\def\etal{{\it et al.}}
\def\etc{{\it etc}}

% \newcommand{\red}{\textcolor[rgb]{0.757,0.153,0.212}}
\newcommand{\red}{\textcolor[rgb]{0,0,0}}

% \newcommand{\rrred}{\textcolor[rgb]{0.757,0.153,0.212}}
\newcommand{\rrred}{\textcolor[rgb]{0,0,0}}

\newcommand{\rred}{\textcolor[rgb]{0.757,0.153,0.212}}
\newcommand{\blue}{\textcolor[rgb]{0.082, 0, 1}}
\newcommand{\green}{\textcolor[rgb]{0.180, 0.518, 0.349}}
\definecolor{lightgray}{rgb}{0.89, 0.89, 0.89}
\usepackage{newunicodechar}
\usepackage[utf8]{inputenc}
\usepackage[T1]{fontenc}    % use 8-bit T1 fonts
\usepackage{pifont}
\usepackage{newunicodechar}
\newunicodechar{✓}{\green{\ding{51}}}
\newunicodechar{✗}{\rred{\ding{55}}}

\usepackage{xcolor}
\usepackage{listings}
\usepackage{multirow}
\usepackage{booktabs}
\begin{document}


\title{WanderGuide: Indoor Map-less Robotic Guide\\for Exploration by Blind People}


\author{Masaki Kuribayashi}
\affiliation{
  \institution{Waseda University}
  \city{}
  \country{}}
\affiliation{
   \institution{Miraikan - The National Museum of Emerging Science and Innovation}
   \city{Tokyo}
\country{Japan}}

\author{Kohei Uehara}
\affiliation{
   \institution{Miraikan - The National Museum of Emerging Science and Innovation}
   \city{Tokyo}
\country{Japan}}

\author{Allan Wang}
\affiliation{
   \institution{Miraikan - The National Museum of Emerging Science and Innovation}
   \city{Tokyo}
\country{Japan}}

\author{Shigeo Morishima}
\affiliation{
  \institution{Waseda Research Institute for Science and Engineering}
  \city{Tokyo}
  \country{Japan}}

\author{Chieko Asakawa}
\affiliation{
   \institution{Miraikan - The National Museum of Emerging Science and Innovation}
   \city{Tokyo}
   \country{Japan}}

\renewcommand{\shortauthors}{Kuribayashi et al.}

\begin{abstract}\label{00_Abstract}
Research in the field of automated vehicles, or more generally cognitive cyber-physical systems that operate in the real world, is leading to increasingly complex systems. Among other things, artificial intelligence enables an ever-increasing degree of autonomy. In this context, the V-model, which has served for decades as a process reference model of the system development lifecycle is reaching its limits. To the contrary, innovative processes and frameworks have been developed that take into account the characteristics of emerging autonomous systems. To bridge the gap and merge the different methodologies, we present an extension of the V-model for iterative data-based development processes that harmonizes and formalizes the existing methods towards a generic framework. The iterative approach allows for seamless integration of continuous system refinement. While the data-based approach constitutes the consideration of data-based development processes and formalizes the use of synthetic and real world data. In this way, formalizing the process of development, verification, validation, and continuous integration contributes to ensuring the safety of emerging complex systems that incorporate AI. 
\end{abstract}


\begin{IEEEkeywords}
	Process Reference Model, V-Model, Continuous Integration, AI Systems, Autonomy Technology, Safety Assurance
\end{IEEEkeywords}


\begin{CCSXML}
<ccs2012>
   <concept>
       <concept_id>10003120.10011738.10011776</concept_id>
       <concept_desc>Human-centered computing~Accessibility systems and tools</concept_desc>
       <concept_significance>500</concept_significance>
       </concept>
   <concept>
       <concept_id>10003456.10010927.10003616</concept_id>
       <concept_desc>Social and professional topics~People with disabilities</concept_desc>
       <concept_significance>500</concept_significance>
       </concept>
 </ccs2012>
\end{CCSXML}

\ccsdesc[500]{Human-centered computing~Accessibility systems and tools}
\ccsdesc[500]{Social and professional topics~People with disabilities}

\begin{teaserfigure}
  \includegraphics[width=\textwidth]{figure/teaser.png}
  \caption{Five core functionalities of WanderGuide. The system assists users in recreational exploration by explaining the surrounding environment through images obtained from the robot's camera. Users can adjust the level of detail and ask questions about their surroundings. Additionally, the system can guide users to locations they have visited before.}
  \Description{
    The figure shows the five core functionalities of WanderGuide. These functionalities include: Map-less Navigation and Obstacle Avoidance: WanderGuide selects potential waypoint candidates using visual information from the environment. It helps users navigate around obstacles and choose a path to their destination without depending on a pre-existing map. The figure illustrates the robot predicting several possible waypoints, selecting one, and navigating while avoiding a person. Describe Surrounding Environment: WanderGuide communicates details about the environment's layout. In the figure, the system is describing, "This is a long corridor with restaurants on the right. A Chinese restaurant offers dumplings, and further ahead is a Japanese restaurant. There are seats available in the Chinese restaurant, and tall seating on your left. I can also see two chefs preparing dishes." Adjust Detail of Descriptions: Users can adjust the level of detail in descriptions. A concise description might say, "There are Chinese and Japanese restaurants," while a more detailed version could include, "This is a long corridor with restaurants on the right. A Chinese restaurant offers dumplings, and ..." Question and Answering Interaction: WanderGuide allows users to ask questions about their surroundings. In the figure, the user is asking, "What does the Chinese restaurant have?" and WanderGuide could respond, "I can see they have dumplings, but please ask for further menus." "Take-Me-There" Functionality: WanderGuide allows users to request to be guided to a specific location. In the figure, the user is saying, "Take me to the Chinese restaurant," and WanderGuide guides the user to that location. 
  }
  \label{fig:teaser}
\end{teaserfigure}

\keywords{visual impairment, map-less navigation, recreational exploration}

\maketitle
\section{Introduction}

% \textcolor{red}{Still on working}

% \textcolor{red}{add label for each section}


Robot learning relies on diverse and high-quality data to learn complex behaviors \cite{aldaco2024aloha, wang2024dexcap}.
Recent studies highlight that models trained on datasets with greater complexity and variation in the domain tend to generalize more effectively across broader scenarios \cite{mann2020language, radford2021learning, gao2024efficient}.
% However, creating such diverse datasets in the real world presents significant challenges.
% Modifying physical environments and adjusting robot hardware settings require considerable time, effort, and financial resources.
% In contrast, simulation environments offer a flexible and efficient alternative.
% Simulations allow for the creation and modification of digital environments with a wide range of object shapes, weights, materials, lighting, textures, friction coefficients, and so on to incorporate domain randomization,
% which helps improve the robustness of models when deployed in real-world conditions.
% These environments can be easily adjusted and reset, enabling faster iterations and data collection.
% Additionally, simulations provide the ability to consistently reproduce scenarios, which is essential for benchmarking and model evaluation.
% Another advantage of simulations is their flexibility in sensor integration. Sensors such as cameras, LiDARs, and tactile sensors can be added or repositioned without the physical limitations present in real-world setups. Simulations also eliminate the risk of damaging expensive hardware during edge-case experiments, making them an ideal platform for testing rare or dangerous scenarios that are impractical to explore in real life.
By leveraging immersive perspectives and interactions, Extended Reality\footnote{Extended Reality is an umbrella term to refer to Augmented Reality, Mixed Reality, and Virtual Reality \cite{wikipediaExtendedReality}}
(XR)
is a promising candidate for efficient and intuitive large scale data collection \cite{jiang2024comprehensive, arcade}
% With the demand for collecting data, XR provides a promising approach for humans to teach robots by offering users an immersive experience.
in simulation \cite{jiang2024comprehensive, arcade, dexhub-park} and real-world scenarios \cite{openteach, opentelevision}.
However, reusing and reproducing current XR approaches for robot data collection for new settings and scenarios is complicated and requires significant effort.
% are difficult to reuse and reproduce system makes it hard to reuse and reproduce in another data collection pipeline.
This bottleneck arises from three main limitations of current XR data collection and interaction frameworks: \textit{asset limitation}, \textit{simulator limitation}, and \textit{device limitation}.
% \textcolor{red}{ASSIGN THESE CITATION PROPERLY:}
% \textcolor{red}{list them by time order???}
% of collecting data by using XR have three main limitations.
Current approaches suffering from \textit{asset limitation} \cite{arclfd, jiang2024comprehensive, arcade, george2025openvr, vicarios}
% Firstly, recent works \cite{jiang2024comprehensive, arcade, dexhub-park}
can only use predefined robot models and task scenes. Configuring new tasks requires significant effort, since each new object or model must be specifically integrated into the XR application.
% and it takes too much effort to configure new tasks in their systems since they cannot spawn arbitrary models in the XR application.
The vast majority of application are developed for specific simulators or real-world scenarios. This \textit{simulator limitation} \cite{mosbach2022accelerating, lipton2017baxter, dexhub-park, arcade}
% Secondly, existing systems are limited to a single simulation platform or real-world scenarios.
significantly reduces reusability and makes adaptation to new simulation platforms challenging.
Additionally, most current XR frameworks are designed for a specific version of a single XR headset, leading to a \textit{device limitation} 
\cite{lipton2017baxter, armada, openteach, meng2023virtual}.
% and there is no work working on the extendability of transferring to a new headsets as far as we know.
To the best of our knowledge, no existing work has explored the extensibility or transferability of their framework to different headsets.
These limitations hamper reproducibility and broader contributions of XR based data collection and interaction to the research community.
% as each research group typically has its own data collection pipeline.
% In addition to these main limitations, existing XR systems are not well suited for managing multiple robot systems,
% as they are often designed for single-operator use.

In addition to these main limitations, existing XR systems are often designed for single-operator use, prohibiting collaborative data collection.
At the same time, controlling multiple robots at once can be very difficult for a single operator,
making data collection in multi-robot scenarios particularly challenging \cite{orun2019effect}.
Although there are some works using collaborative data collection in the context of tele-operation \cite{tung2021learning, Qin2023AnyTeleopAG},
there is no XR-based data collection system supporting collaborative data collection.
This limitation highlights the need for more advanced XR solutions that can better support multi-robot and multi-user scenarios.
% \textcolor{red}{more papers about collaborative data collection}

To address all of these issues, we propose \textbf{IRIS},
an \textbf{I}mmersive \textbf{R}obot \textbf{I}nteraction \textbf{S}ystem.
This general system supports various simulators, benchmarks and real-world scenarios.
It is easily extensible to new simulators and XR headsets.
IRIS achieves generalization across six dimensions:
% \begin{itemize}
%     \item \textit{Cross-scene} : diverse object models;
%     \item \textit{Cross-embodiment}: diverse robot models;
%     \item \textit{Cross-simulator}: 
%     \item \textit{Cross-reality}: fd
%     \item \textit{Cross-platform}: fd
%     \item \textit{Cross-users}: fd
% \end{itemize}
\textbf{Cross-Scene}, \textbf{Cross-Embodiment}, \textbf{Cross-Simulator}, \textbf{Cross-Reality}, \textbf{Cross-Platform}, and \textbf{Cross-User}.

\textbf{Cross-Scene} and \textbf{Cross-Embodiment} allow the system to handle arbitrary objects and robots in the simulation,
eliminating restrictions about predefined models in XR applications.
IRIS achieves these generalizations by introducing a unified scene specification, representing all objects,
including robots, as data structures with meshes, materials, and textures.
The unified scene specification is transmitted to the XR application to create and visualize an identical scene.
By treating robots as standard objects, the system simplifies XR integration,
allowing researchers to work with various robots without special robot-specific configurations.
\textbf{Cross-Simulator} ensures compatibility with various simulation engines.
IRIS simplifies adaptation by parsing simulated scenes into the unified scene specification, eliminating the need for XR application modifications when switching simulators.
New simulators can be integrated by creating a parser to convert their scenes into the unified format.
This flexibility is demonstrated by IRIS’ support for Mujoco \cite{todorov2012mujoco}, IsaacSim \cite{mittal2023orbit}, CoppeliaSim \cite{coppeliaSim}, and even the recent Genesis \cite{Genesis} simulator.
\textbf{Cross-Reality} enables the system to function seamlessly in both virtual simulations and real-world applications.
IRIS enables real-world data collection through camera-based point cloud visualization.
\textbf{Cross-Platform} allows for compatibility across various XR devices.
Since XR device APIs differ significantly, making a single codebase impractical, IRIS XR application decouples its modules to maximize code reuse.
This application, developed by Unity \cite{unity3dUnityManual}, separates scene visualization and interaction, allowing developers to integrate new headsets by reusing the visualization code and only implementing input handling for hand, head, and motion controller tracking.
IRIS provides an implementation of the XR application in the Unity framework, allowing for a straightforward deployment to any device that supports Unity. 
So far, IRIS was successfully deployed to the Meta Quest 3 and HoloLens 2.
Finally, the \textbf{Cross-User} ability allows multiple users to interact within a shared scene.
IRIS achieves this ability by introducing a protocol to establish the communication between multiple XR headsets and the simulation or real-world scenarios.
Additionally, IRIS leverages spatial anchors to support the alignment of virtual scenes from all deployed XR headsets.
% To make an seamless user experience for robot learning data collection,
% IRIS also tested in three different robot control interface
% Furthermore, to demonstrate the extensibility of our approach, we have implemented a robot-world pipeline for real robot data collection, ensuring that the system can be used in both simulated and real-world environments.
The Immersive Robot Interaction System makes the following contributions\\
\textbf{(1) A unified scene specification} that is compatible with multiple robot simulators. It enables various XR headsets to visualize and interact with simulated objects and robots, providing an immersive experience while ensuring straightforward reusability and reproducibility.\\
\textbf{(2) A collaborative data collection framework} designed for XR environments. The framework facilitates enhanced robot data acquisition.\\
\textbf{(3) A user study} demonstrating that IRIS significantly improves data collection efficiency and intuitiveness compared to the LIBERO baseline.

% \begin{table*}[t]
%     \centering
%     \begin{tabular}{lccccccc}
%         \toprule
%         & \makecell{Physical\\Interaction}
%         & \makecell{XR\\Enabled}
%         & \makecell{Free\\View}
%         & \makecell{Multiple\\Robots}
%         & \makecell{Robot\\Control}
%         % Force Feedback???
%         & \makecell{Soft Object\\Supported}
%         & \makecell{Collaborative\\Data} \\
%         \midrule
%         ARC-LfD \cite{arclfd}                              & Real        & \cmark & \xmark & \xmark & Joint              & \xmark & \xmark \\
%         DART \cite{dexhub-park}                            & Sim         & \cmark & \cmark & \cmark & Cartesian          & \xmark & \xmark \\
%         \citet{jiang2024comprehensive}                     & Sim         & \cmark & \xmark & \xmark & Joint \& Cartesian & \xmark & \xmark \\
%         \citet{mosbach2022accelerating}                    & Sim         & \cmark & \cmark & \xmark & Cartesian          & \xmark & \xmark \\
%         ARCADE \cite{arcade}                               & Real        & \cmark & \cmark & \xmark & Cartesian          & \xmark & \xmark \\
%         Holo-Dex \cite{holodex}                            & Real        & \cmark & \xmark & \cmark & Cartesian          & \cmark & \xmark \\
%         ARMADA \cite{armada}                               & Real        & \cmark & \xmark & \cmark & Cartesian          & \cmark & \xmark \\
%         Open-TeleVision \cite{opentelevision}              & Real        & \cmark & \cmark & \cmark & Cartesian          & \cmark & \xmark \\
%         OPEN TEACH \cite{openteach}                        & Real        & \cmark & \xmark & \cmark & Cartesian          & \cmark & \cmark \\
%         GELLO \cite{wu2023gello}                           & Real        & \xmark & \cmark & \cmark & Joint              & \cmark & \xmark \\
%         DexCap \cite{wang2024dexcap}                       & Real        & \xmark & \cmark & \xmark & Cartesian          & \cmark & \xmark \\
%         AnyTeleop \cite{Qin2023AnyTeleopAG}                & Real        & \xmark & \xmark & \cmark & Cartesian          & \cmark & \cmark \\
%         Vicarios \cite{vicarios}                           & Real        & \cmark & \xmark & \xmark & Cartesian          & \cmark & \xmark \\     
%         Augmented Visual Cues \cite{augmentedvisualcues}   & Real        & \cmark & \cmark & \xmark & Cartesian          & \xmark & \xmark \\ 
%         \citet{wang2024robotic}                            & Real        & \cmark & \cmark & \xmark & Cartesian          & \cmark & \xmark \\
%         Bunny-VisionPro \cite{bunnyvisionpro}              & Real        & \cmark & \cmark & \cmark & Cartesian          & \cmark & \xmark \\
%         IMMERTWIN \cite{immertwin}                         & Real        & \cmark & \cmark & \cmark & Cartesian          & \xmark & \xmark \\
%         \citet{meng2023virtual}                            & Sim \& Real & \cmark & \cmark & \xmark & Cartesian          & \xmark & \xmark \\
%         Shared Control Framework \cite{sharedctlframework} & Real        & \cmark & \cmark & \cmark & Cartesian          & \xmark & \xmark \\
%         OpenVR \cite{openvr}                               & Real        & \cmark & \cmark & \xmark & Cartesian          & \xmark & \xmark \\
%         \citet{digitaltwinmr}                              & Real        & \cmark & \cmark & \xmark & Cartesian          & \cmark & \xmark \\
        
%         \midrule
%         \textbf{Ours} & Sim \& Real & \cmark & \cmark & \cmark & Joint \& Cartesian  & \cmark & \cmark \\
%         \bottomrule
%     \end{tabular}
%     \caption{This is a cross-column table with automatic line breaking.}
%     \label{tab:cross-column}
% \end{table*}

% \begin{table*}[t]
%     \centering
%     \begin{tabular}{lccccccc}
%         \toprule
%         & \makecell{Cross-Embodiment}
%         & \makecell{Cross-Scene}
%         & \makecell{Cross-Simulator}
%         & \makecell{Cross-Reality}
%         & \makecell{Cross-Platform}
%         & \makecell{Cross-User} \\
%         \midrule
%         ARC-LfD \cite{arclfd}                              & \xmark & \xmark & \xmark & \xmark & \xmark & \xmark \\
%         DART \cite{dexhub-park}                            & \cmark & \cmark & \xmark & \xmark & \xmark & \xmark \\
%         \citet{jiang2024comprehensive}                     & \xmark & \cmark & \xmark & \xmark & \xmark & \xmark \\
%         \citet{mosbach2022accelerating}                    & \xmark & \cmark & \xmark & \xmark & \xmark & \xmark \\
%         ARCADE \cite{arcade}                               & \xmark & \xmark & \xmark & \xmark & \xmark & \xmark \\
%         Holo-Dex \cite{holodex}                            & \cmark & \xmark & \xmark & \xmark & \xmark & \xmark \\
%         ARMADA \cite{armada}                               & \cmark & \xmark & \xmark & \xmark & \xmark & \xmark \\
%         Open-TeleVision \cite{opentelevision}              & \cmark & \xmark & \xmark & \xmark & \cmark & \xmark \\
%         OPEN TEACH \cite{openteach}                        & \cmark & \xmark & \xmark & \xmark & \xmark & \cmark \\
%         GELLO \cite{wu2023gello}                           & \cmark & \xmark & \xmark & \xmark & \xmark & \xmark \\
%         DexCap \cite{wang2024dexcap}                       & \xmark & \xmark & \xmark & \xmark & \xmark & \xmark \\
%         AnyTeleop \cite{Qin2023AnyTeleopAG}                & \cmark & \cmark & \cmark & \cmark & \xmark & \cmark \\
%         Vicarios \cite{vicarios}                           & \xmark & \xmark & \xmark & \xmark & \xmark & \xmark \\     
%         Augmented Visual Cues \cite{augmentedvisualcues}   & \xmark & \xmark & \xmark & \xmark & \xmark & \xmark \\ 
%         \citet{wang2024robotic}                            & \xmark & \xmark & \xmark & \xmark & \xmark & \xmark \\
%         Bunny-VisionPro \cite{bunnyvisionpro}              & \cmark & \xmark & \xmark & \xmark & \xmark & \xmark \\
%         IMMERTWIN \cite{immertwin}                         & \cmark & \xmark & \xmark & \xmark & \xmark & \xmark \\
%         \citet{meng2023virtual}                            & \xmark & \cmark & \xmark & \cmark & \xmark & \xmark \\
%         \citet{sharedctlframework}                         & \cmark & \xmark & \xmark & \xmark & \xmark & \xmark \\
%         OpenVR \cite{george2025openvr}                               & \xmark & \xmark & \xmark & \xmark & \xmark & \xmark \\
%         \citet{digitaltwinmr}                              & \xmark & \xmark & \xmark & \xmark & \xmark & \xmark \\
        
%         \midrule
%         \textbf{Ours} & \cmark & \cmark & \cmark & \cmark & \cmark & \cmark \\
%         \bottomrule
%     \end{tabular}
%     \caption{This is a cross-column table with automatic line breaking.}
% \end{table*}

% \begin{table*}[t]
%     \centering
%     \begin{tabular}{lccccccc}
%         \toprule
%         & \makecell{Cross-Scene}
%         & \makecell{Cross-Embodiment}
%         & \makecell{Cross-Simulator}
%         & \makecell{Cross-Reality}
%         & \makecell{Cross-Platform}
%         & \makecell{Cross-User}
%         & \makecell{Control Space} \\
%         \midrule
%         % Vicarios \cite{vicarios}                           & \xmark & \xmark & \xmark & \xmark & \xmark & \xmark \\     
%         % Augmented Visual Cues \cite{augmentedvisualcues}   & \xmark & \xmark & \xmark & \xmark & \xmark & \xmark \\ 
%         % OpenVR \cite{george2025openvr}                     & \xmark & \xmark & \xmark & \xmark & \xmark & \xmark \\
%         \citet{digitaltwinmr}                              & \xmark & \xmark & \xmark & \xmark & \xmark & \xmark &  \\
%         ARC-LfD \cite{arclfd}                              & \xmark & \xmark & \xmark & \xmark & \xmark & \xmark &  \\
%         \citet{sharedctlframework}                         & \cmark & \xmark & \xmark & \xmark & \xmark & \xmark &  \\
%         \citet{jiang2024comprehensive}                     & \cmark & \xmark & \xmark & \xmark & \xmark & \xmark &  \\
%         \citet{mosbach2022accelerating}                    & \cmark & \xmark & \xmark & \xmark & \xmark & \xmark & \\
%         Holo-Dex \cite{holodex}                            & \cmark & \xmark & \xmark & \xmark & \xmark & \xmark & \\
%         ARCADE \cite{arcade}                               & \cmark & \cmark & \xmark & \xmark & \xmark & \xmark & \\
%         DART \cite{dexhub-park}                            & Limited & Limited & Mujoco & Sim & Vision Pro & \xmark &  Cartesian\\
%         ARMADA \cite{armada}                               & \cmark & \cmark & \xmark & \xmark & \xmark & \xmark & \\
%         \citet{meng2023virtual}                            & \cmark & \cmark & \xmark & \cmark & \xmark & \xmark & \\
%         % GELLO \cite{wu2023gello}                           & \cmark & \xmark & \xmark & \xmark & \xmark & \xmark \\
%         % DexCap \cite{wang2024dexcap}                       & \xmark & \xmark & \xmark & \xmark & \xmark & \xmark \\
%         % AnyTeleop \cite{Qin2023AnyTeleopAG}                & \cmark & \cmark & \cmark & \cmark & \xmark & \cmark \\
%         % \citet{wang2024robotic}                            & \xmark & \xmark & \xmark & \xmark & \xmark & \xmark \\
%         Bunny-VisionPro \cite{bunnyvisionpro}              & \cmark & \cmark & \xmark & \xmark & \xmark & \xmark & \\
%         IMMERTWIN \cite{immertwin}                         & \cmark & \cmark & \xmark & \xmark & \xmark & \xmark & \\
%         Open-TeleVision \cite{opentelevision}              & \cmark & \cmark & \xmark & \xmark & \cmark & \xmark & \\
%         \citet{szczurek2023multimodal}                     & \xmark & \xmark & \xmark & Real & \xmark & \cmark & \\
%         OPEN TEACH \cite{openteach}                        & \cmark & \cmark & \xmark & \xmark & \xmark & \cmark & \\
%         \midrule
%         \textbf{Ours} & \cmark & \cmark & \cmark & \cmark & \cmark & \cmark \\
%         \bottomrule
%     \end{tabular}
%     \caption{TODO, Bruce: this table can be further optimized.}
% \end{table*}

\definecolor{goodgreen}{HTML}{228833}
\definecolor{goodred}{HTML}{EE6677}
\definecolor{goodgray}{HTML}{BBBBBB}

\begin{table*}[t]
    \centering
    \begin{adjustbox}{max width=\textwidth}
    \renewcommand{\arraystretch}{1.2}    
    \begin{tabular}{lccccccc}
        \toprule
        & \makecell{Cross-Scene}
        & \makecell{Cross-Embodiment}
        & \makecell{Cross-Simulator}
        & \makecell{Cross-Reality}
        & \makecell{Cross-Platform}
        & \makecell{Cross-User}
        & \makecell{Control Space} \\
        \midrule
        % Vicarios \cite{vicarios}                           & \xmark & \xmark & \xmark & \xmark & \xmark & \xmark \\     
        % Augmented Visual Cues \cite{augmentedvisualcues}   & \xmark & \xmark & \xmark & \xmark & \xmark & \xmark \\ 
        % OpenVR \cite{george2025openvr}                     & \xmark & \xmark & \xmark & \xmark & \xmark & \xmark \\
        \citet{digitaltwinmr}                              & \textcolor{goodred}{Limited}     & \textcolor{goodred}{Single Robot} & \textcolor{goodred}{Unity}    & \textcolor{goodred}{Real}          & \textcolor{goodred}{Meta Quest 2} & \textcolor{goodgray}{N/A} & \textcolor{goodred}{Cartesian} \\
        ARC-LfD \cite{arclfd}                              & \textcolor{goodgray}{N/A}        & \textcolor{goodred}{Single Robot} & \textcolor{goodgray}{N/A}     & \textcolor{goodred}{Real}          & \textcolor{goodred}{HoloLens}     & \textcolor{goodgray}{N/A} & \textcolor{goodred}{Cartesian} \\
        \citet{sharedctlframework}                         & \textcolor{goodred}{Limited}     & \textcolor{goodred}{Single Robot} & \textcolor{goodgray}{N/A}     & \textcolor{goodred}{Real}          & \textcolor{goodred}{HTC Vive Pro} & \textcolor{goodgray}{N/A} & \textcolor{goodred}{Cartesian} \\
        \citet{jiang2024comprehensive}                     & \textcolor{goodred}{Limited}     & \textcolor{goodred}{Single Robot} & \textcolor{goodgray}{N/A}     & \textcolor{goodred}{Real}          & \textcolor{goodred}{HoloLens 2}   & \textcolor{goodgray}{N/A} & \textcolor{goodgreen}{Joint \& Cartesian} \\
        \citet{mosbach2022accelerating}                    & \textcolor{goodgreen}{Available} & \textcolor{goodred}{Single Robot} & \textcolor{goodred}{IsaacGym} & \textcolor{goodred}{Sim}           & \textcolor{goodred}{Vive}         & \textcolor{goodgray}{N/A} & \textcolor{goodgreen}{Joint \& Cartesian} \\
        Holo-Dex \cite{holodex}                            & \textcolor{goodgray}{N/A}        & \textcolor{goodred}{Single Robot} & \textcolor{goodgray}{N/A}     & \textcolor{goodred}{Real}          & \textcolor{goodred}{Meta Quest 2} & \textcolor{goodgray}{N/A} & \textcolor{goodred}{Joint} \\
        ARCADE \cite{arcade}                               & \textcolor{goodgray}{N/A}        & \textcolor{goodred}{Single Robot} & \textcolor{goodgray}{N/A}     & \textcolor{goodred}{Real}          & \textcolor{goodred}{HoloLens 2}   & \textcolor{goodgray}{N/A} & \textcolor{goodred}{Cartesian} \\
        DART \cite{dexhub-park}                            & \textcolor{goodred}{Limited}     & \textcolor{goodred}{Limited}      & \textcolor{goodred}{Mujoco}   & \textcolor{goodred}{Sim}           & \textcolor{goodred}{Vision Pro}   & \textcolor{goodgray}{N/A} & \textcolor{goodred}{Cartesian} \\
        ARMADA \cite{armada}                               & \textcolor{goodgray}{N/A}        & \textcolor{goodred}{Limited}      & \textcolor{goodgray}{N/A}     & \textcolor{goodred}{Real}          & \textcolor{goodred}{Vision Pro}   & \textcolor{goodgray}{N/A} & \textcolor{goodred}{Cartesian} \\
        \citet{meng2023virtual}                            & \textcolor{goodred}{Limited}     & \textcolor{goodred}{Single Robot} & \textcolor{goodred}{PhysX}   & \textcolor{goodgreen}{Sim \& Real} & \textcolor{goodred}{HoloLens 2}   & \textcolor{goodgray}{N/A} & \textcolor{goodred}{Cartesian} \\
        % GELLO \cite{wu2023gello}                           & \cmark & \xmark & \xmark & \xmark & \xmark & \xmark \\
        % DexCap \cite{wang2024dexcap}                       & \xmark & \xmark & \xmark & \xmark & \xmark & \xmark \\
        % AnyTeleop \cite{Qin2023AnyTeleopAG}                & \cmark & \cmark & \cmark & \cmark & \xmark & \cmark \\
        % \citet{wang2024robotic}                            & \xmark & \xmark & \xmark & \xmark & \xmark & \xmark \\
        Bunny-VisionPro \cite{bunnyvisionpro}              & \textcolor{goodgray}{N/A}        & \textcolor{goodred}{Single Robot} & \textcolor{goodgray}{N/A}     & \textcolor{goodred}{Real}          & \textcolor{goodred}{Vision Pro}   & \textcolor{goodgray}{N/A} & \textcolor{goodred}{Cartesian} \\
        IMMERTWIN \cite{immertwin}                         & \textcolor{goodgray}{N/A}        & \textcolor{goodred}{Limited}      & \textcolor{goodgray}{N/A}     & \textcolor{goodred}{Real}          & \textcolor{goodred}{HTC Vive}     & \textcolor{goodgray}{N/A} & \textcolor{goodred}{Cartesian} \\
        Open-TeleVision \cite{opentelevision}              & \textcolor{goodgray}{N/A}        & \textcolor{goodred}{Limited}      & \textcolor{goodgray}{N/A}     & \textcolor{goodred}{Real}          & \textcolor{goodgreen}{Meta Quest, Vision Pro} & \textcolor{goodgray}{N/A} & \textcolor{goodred}{Cartesian} \\
        \citet{szczurek2023multimodal}                     & \textcolor{goodgray}{N/A}        & \textcolor{goodred}{Limited}      & \textcolor{goodgray}{N/A}     & \textcolor{goodred}{Real}          & \textcolor{goodred}{HoloLens 2}   & \textcolor{goodgreen}{Available} & \textcolor{goodred}{Joint \& Cartesian} \\
        OPEN TEACH \cite{openteach}                        & \textcolor{goodgray}{N/A}        & \textcolor{goodgreen}{Available}  & \textcolor{goodgray}{N/A}     & \textcolor{goodred}{Real}          & \textcolor{goodred}{Meta Quest 3} & \textcolor{goodred}{N/A} & \textcolor{goodgreen}{Joint \& Cartesian} \\
        \midrule
        \textbf{Ours}                                      & \textcolor{goodgreen}{Available} & \textcolor{goodgreen}{Available}  & \textcolor{goodgreen}{Mujoco, CoppeliaSim, IsaacSim} & \textcolor{goodgreen}{Sim \& Real} & \textcolor{goodgreen}{Meta Quest 3, HoloLens 2} & \textcolor{goodgreen}{Available} & \textcolor{goodgreen}{Joint \& Cartesian} \\
        \bottomrule
        \end{tabular}
    \end{adjustbox}
    \caption{Comparison of XR-based system for robots. IRIS is compared with related works in different dimensions.}
\end{table*}


\section{Related Work}

In this section, we review research related to the importance and barriers to parental involvement; parental use of learning technologies; and the use of generative AI and robot in educational and parenting scenarios.

\subsection{Importance and Barriers to Parental Involvement}\label{sec-rw-2.1}

% 79 words
Early childhood is a critical period for predicting future success and well-being, with early education investments resulting in higher returns than later interventions \cite{duncan2007school, doyle2009investing}. Effective parental involvement fosters cognitive and social skills, especially in younger children \cite{blevins2016early, peck1992parent}. Parents are encouraged to prioritize home-based involvement to maximize their influence \cite{ma2016meta}, as their involvement has a greater impact on children's learning outcomes \cite{hoffner2002parents, fehrmann1987home, hill2004parent} within the family setting than partnerships with schools or communities \cite{ma2016meta, harris2008parents, fantuzzo2004multiple, sui1996effects}.

However, parents' involvement in their children's education is often constrained by practical challenges related to parents' \textit{skills}, \textit{time}, and \textit{energy}. The Hoover-Dempsey and Sandler (HDS) framework \cite{green2007parents} and the CAM framework \cite{ho2024s} both highlight these factors-- parents' perceived \textit{skills and knowledge} (capability), \textit{time} (availability), and \textit{motivation} (energy)--influence the extent of their engagement. For instance, a parent confident in math may choose to engage more in math-related tasks, while those facing inflexible schedules may participate less \cite{green2007parents}. Unlike teachers, parents often lack formal pedagogical training and may underestimate their role in supporting children's learning, particularly as young children struggle to articulate their needs \cite{hara1998parent}. The CAM framework similarly suggests parents delegate tasks to a robot when they feel less capable, have limited time, or are unmotivated. These factors reflect parents' life contexts, shaped by demographic backgrounds, occupations, and parenting responsibilities \cite{grolnick1997predictors}, highlighting the need to help parents overcome barriers to effective involvement in early education within their life contexts.

\subsection{Parental Use of Learning Technologies}\label{sec-rw-2.2}
% 207 words
Technology encourages parental involvement by facilitating parent-child engagement in learning activities while introducing risks that require active parental mediation \cite{gonzalez2022parental}. On the positive side, technology offers novel opportunities for parental engagement and enhances children's learning outcomes. For example, e-books promote interactive behaviors between parents and children better than print books \cite{korat2010new}. In addition, having access to computers at home significantly boost academic achievement of young children when parents actively mediate their use \cite{hofferth2010home, espinosa2006technology}. However, the effectiveness of these tools often depends on parents' familiarity with and attitude toward technology. Mobile applications, for instance, can improve learning outcomes but require parents to possess sufficient technology efficacy to guide their use \cite{papadakis2019parental}.

On the negative side, technology introduces risks such as excessive screen time, exposure to inappropriate content, and misinformation, which necessitate parental intervention \cite{oswald2020psychological, howard2021digital}. According to parental mediation theory, parents mitigate these risks through restrictive mediation (e.g., setting limits), active mediation (e.g., discussing content), and co-use (e.g., shared use of technology) \cite{valkenburg1999developing}. Modern technologies like video games, location-based games (\textit{e.g.,} Pokemon Go), and conversational agents (\textit{e.g.,} Alexa) also require parents to adapt their mediation strategies to ensure responsible use \cite{valkenburg1999developing, nikken2006parental, sobel2017wasn, beneteau2020parenting, yu2024parent}. Overall, parents seek to leverage technology to support their children's learning due to ite effectivenss but are also mindful of its risks. Their involvement is therefore driven by both opportunities and concerns, highlighting the need to design tools that effectively involve parents to balance benefits and risks.

\subsection{Generative AI and Companion Robots for Parenting and Education}
Generative AI and companion robots offer human-like affordances, with AI simulating human intelligence and robots providing physical human-like features. Compared to conventional models (\textit{e.g.,} machine learning) and devices (\textit{e.g.,} laptops), these emerging technologies enable natural and social interactions, creating opportunities for novel paradigms to enhance parental involvement and children's learning while introducing their unique challenges.

\subsubsection{Generative AI}
GAI offers promising support for parents by enhancing their ability to educate and engage with their children. Prior work suggested that AI-driven systems can support parenting education \cite{petsolari2024socio} and provide evidence-based advice through applications and chatbots, delivering micro-interventions such as teaching parents how to praise their children effectively \cite{davis2017parent, entenberg2023user} or offering strategies to teach complex concepts \cite{mogavi2024chatgpt, su2023unlocking}. Many parents also prefer using GAI to create educational materials tailored to their children's needs, rather than granting children direct access to these tools \cite{han2023design}. Beyond educational support, AI-based storytelling tools address practical challenges (\textit{e.g.,} time constraints) by alleviating physical labor while fostering parent-child interactions \cite{sun2024exploring}. Furthermore, GAI offers advantages to children's learning directly. It can help create personalized learning experiences by providing timely feedback and tailoring content \cite{su2023unlocking, mogavi2024chatgpt, han2024teachers}, enhancing positive learning experiences \cite{jauhiainen2023generative}. For example, a LLM-driven conversational system can teach children mathematical concepts through co-creative storytelling, achieving learning outcomes similar to human-led instructions \cite{zhang2024mathemyths}.

Despite these benefits, several concerns persist regarding the use of GAI in education. Prior work highlighted the limitations of GAI, such as its limited effectiveness in more complex learning tasks,the limited quality of the training data, and its inability to offer comprehensive educational support \cite{su2023unlocking}. There is also a significant risk of GAI producing inaccurate or biased information, discouraging independent thought among children, and threatening user privacy \cite{su2023unlocking, han2023design, han2024teachers}. Many parents are skeptical about the role of AI in their children's academic processes, concerned about the accuracy of AI-generated content, and worry that over-reliance on AI could stifle independent thinking \cite{han2023design}.

%\todo{might need to add some structural transition here}
\subsubsection{Social companion robots}
Social companion robots have proven potential to assist parents in home education settings through studies in \textit{parent-child-robot} interactions. \citet{gvirsman2020patricc} showed that the robotic system, \texttt{Patricc}, fostered more triadic interaction between parents and toddlers than a tablet, and \citet{gvirsman2024effect} found that, in a parent-toddler-robot interaction, parents tend to decrease their scaffolding affectively when the robot increases its scaffolding behavior. Similarly, \citet{chen2022designing} found that social robots enhanced parent-child co-reading activities, while \citet{chan2017wakey} demonstrated that the WAKEY robot improved morning routines and reduced parental frustration. Beyond educational support, \citet{ho2024s} uncovered that parents envisioned robots as their \textit{collaborators} to support their children's learning at home and that their collaboration patterns can be determined by the parents' capability, availability, and motivation. Although parents generally have positive attitudes toward incorporating robots into their children's learning, they remain concerned about the risk of disrupting school-based learning and potential teacher replacement \cite{tolksdorf2020parents, lee2008elementary, louie2021desire}.

In addition to parental support, social companion robots also support children in education directly through \textit{child-robot interactions}. Physically embodied robots provide adaptive assistance and verbal interaction similar to virtual or conversational agents \cite{ramachandran2019personalized, leyzberg2014personalizing, schodde2017adaptive, brown2014positive}, yet they foster greater engagement with the physical environment and encourage more advanced social behaviors during learning \cite{belpaeme2018social}, leading to improved learning outcomes \cite{leyzberg2012physical}. Prior work demonstrated that companion robots can effectively support both school-based learning (\textit{e.g.,} math \cite{lopez2018robotic}, literacy \cite{kennedy2016social, gordon2016affective}, and science \cite{davison2020working}) and home-based learning activities (\textit{e.g.,} reading \cite{michaelis2018reading, michaelis2019supporting}, number board games \cite{ho2021robomath}, and math-oriented conversations with parents \cite{ho2023designing}). For example, \citet{kennedy2016social} suggested that children can learn elements of a second language from a robot in short-term interactions, and \citet{tanaka2009use} found that children who took on the role of teaching the robot gained confidence and improved learning outcomes.

%\todo{may need to make this a separate section and explain why we propose AI-assisted robots}

% \subsubsection{Research Gap}
Parental involvement in early education is crucial and AI-assisted robots can offer promising support by helping parents overcome practical barriers (\textit{i.e.,} time, energy, and skills) and addressing concerns about technological risks. Yet, limited research has examined how technology design can simultaneously alleviate these barriers and concerns. Though \citet{zhang2022storybuddy} emphasized the importance of flexible parental involvement during reading through a system called \textit{Storybuddy}, yet they focused on a virtual chatbot rather than a physical robot, and how the flexible modes may be used in different scenarios remain unknown. Similarly, \textit{ContextQ} \cite{dietz2024contextq} presented auto-generated dialogic questions to caregivers for dialogic reading, but primarily considered situations where parents are actively involved, not scenarios where parents cannot participate fully.

In this work, we address these gaps by exploring parental involvement contexts, understanding parents' perceptions of AI-generated content, and examining how parents collaborate with AI and robots under different scenarios. In the following sections, we describe our development of  \texttt{SET}, a card-based activity, to understand parental involvement contexts (Section~\ref{sec-card}), the design of the \texttt{PAiREd} system to enable parents to co-create learning activities with an LLM (Section~\ref{sec-system}), and user study aimed to discover use patterns and understand user perceptions of the system (Section~\ref{sec-study}).
\section{System Design Focus}
\label{sec:system_design}
Our goal is to finalize a system that assists blind people in exploring an indoor environment independently.
In this section, we describe the key design elements of the system.

\begin{figure*}
    \centering
    \includegraphics[width=1\linewidth]{figure/device.png}
    \caption{Image of the robot and handle interface used in the study. Panel A-1 shows the robot used in the formative study, while Panel A-2 presents the robot used in the main study. Panels B-1 and B-2 illustrate the mapping of the handle interface buttons' functions, depending on the selected navigation mode.}
    \label{fig:device&ui}
    \Description{The image consists of four panels labeled A-1, A-2, B-1, and B-2. Panels A-1 and A-2 display a suitcase-shaped device used in the robot, while Panels B-1 and B-2 illustrate the button mapping of the handle interface for navigation. Panel A-1 depicts the initial robot used in the formative study. The robot is red and resembles a suitcase. The panel outlines six key components: The handle interface is located where a typical suitcase handle would be and includes five buttons. A smartphone is mounted on the back of the handle using a mounting device. A touch sensor, positioned under the handle, detects when users are touching it. Three RGBD sensors, situated on the front of the robot, are used for depth and color sensing to assist in obstacle detection and navigation. A 360-degree LiDAR sensor is mounted on top of the robot, on top of the three cameras. Motorized wheels at the base provide mobility, allowing the robot to move autonomously or in response to user input. Panel A-2 shows the updated robot used in the main study. The most significant change is the addition of a 1080p-resolution fisheye camera, positioned near the handle to capture images from a higher point of view. Panel C-1 illustrates the button mapping when the robot is in Automated Navigation Mode: The up button increases speed, and the right button decreases it. The left and right buttons adjust the level of detail. A long press on the middle button triggers conversation mode, while a single press switches to manual control mode. Panel B-2 displays the button mapping when the robot is in Manual Control Mode: The up button moves the robot forward, the back button moves it backward, the left button turns it left, and the right button turns it right. A long press on the middle button triggers conversation mode, while a single press returns the robot to automated navigation mode.
    }
\end{figure*}

\subsection{Device}
Assistance systems for blind people have been proposed in various devices, such as smartphones~\cite{presti2019watchout}, \red{handheld haptic devices~\cite{spiers2016outdoor,choiniere2016development,liu2021tactile},} wearable devices~\cite{li2016isana}, cane-like devices~\cite{ranganeni2023exploring} and robots~\cite{liu2024dragon}.
Each type of device offers unique advantages - Smartphones \red{and handheld haptic devices are portable; Smartphones} are also widely used by blind people~\cite{morris2014blind,martiniello2022exploring}; Wearable devices free the user's hands~\cite{lee2014wearable}; Cane-like devices resemble traditional canes~\cite{ranganeni2023exploring}; And robots are able to autonomously guide users~\cite{guerreiro2019cabot}.
\red{While handheld devices~\cite{presti2019watchout,spiers2016outdoor,choiniere2016development,liu2021tactile,ranganeni2023exploring} have often been used due to their portability, in exploration scenarios, they require users to point the devices in their directions of interest while navigating around unfamiliar locations and obstacles, which involve high cognitive load.
Thus}, we chose robots because of their autonomous navigation and obstacle avoidance capabilities. 
This allows users to concentrate on learning the environment~\cite{cai2024navigating,zhang2023follower,jain2023want}. 
In particular, we adopted a wheeled robot~\cite{guerreiro2019cabot,zhang2023follower,wang2022can}.
While wheeled robots are unable to navigate stairs like quadruped robots~\cite{cai2024navigating}, blind users often find wheeled robots more suitable due to their silence and stability~\cite{wang2022can}.
Our assumption is that the devices should ensure the users' safety during navigation and allow users to focus on exploration. 
As a result, the findings in our study can be extended to any similar devices other than wheeled robots. 

\subsection{Describing Scenes}
Previous navigation systems relied on hardcoded information~\cite{sato2019navcog3,Kaniwa2024ChitChatGuide} or simple image captioning models~\cite{saha2019closing} to provide scene descriptions. 
They only convey information related to navigating to destinations. 
In exploratory tasks, any information and details could be relevant. 
Therefore, we decided to use MLLM, a foundational model capable of recognizing a variety of objects and describing them in natural language. 
We injected MLLM into the system to periodically provide comprehensive information about the surroundings to inform blind users during exploration. 
In this paper, we investigate the appropriate presentation format, such as content types and lengths, and the quality of the responses from MLLMs through our user studies.

\subsection{Interaction} % This section is so hard to write... Plz gimme idea if there is any better way
The ability for users to select destinations and routes according to their interests, often referred to as autonomy, is particularly important for exploration~\cite{Kaniwa2024ChitChatGuide,kayukawa2022HowUsers}. 
In our system, to what extent users prefer to take control over the robot (\ie, interaction) remains unknown.
Based on the scene descriptions given by the system~\cite{Kaniwa2024ChitChatGuide}, some blind users may fully embrace letting the robot guide them automatically, while others may prefer to decide which way to go on their own.
Additionally, this preference may also be influenced by the robot's descriptions of the scenes. 
Given that the extent of user preference for autonomy remains unclear, we first conducted the formative study (Sec.~\ref{sec:study1}) to explore the requirement of autonomy based on interaction needs. 
Then, we conducted a full study (Sec.~\ref{sec:study2}) to evaluate the users' opinions on autonomy in our improved system, which integrated the feedback from the formative study.

\vspace{-2mm}
\section{Formative Study}
\label{sec:study1}
We first conducted a formative study to investigate the requirements of the system, such as how the system should explain its surroundings and what potential interactions may happen between the robot and the user.
To conduct the study, we recruited ten participants through our existing email list.
Interestingly, our recruitment emails were shared among blind people, eventually reaching people not on our emailing list. 
In the recruitment email, we specified that those who are unfamiliar with the experimental location, \ie, even if they have had previous visiting experience, they do not have a clear understanding of the building or know what is there, would be eligible to participate.
Tab.~\ref{tab:demographics1} shows the demographics of the participants. 
All studies in this paper have been approved by our institution's review board.
Informed consent was read out to all participants in this paper and obtained from them. 
The study took approximately two hours, and the participants were compensated \$20 per hour and reimbursed for their transit costs.
Only one participant was present for each session. 

\begin{figure*}
    \centering
    \includegraphics[width=1\linewidth]{figure/routes.png}
    \caption{Floor maps of the location of the study. The left panel shows the two floors of the science museum, \rrred{the fifth floor of Miraikan}, which feature exhibits on various topics, such as environmental issues and space exploration. On the right panel is a floor plan of a shopping mall, \rrred{the fourth floor of Toranomon Hills Station Tower}, which includes a variety of restaurants offering different cuisines, including French, Japanese, Chinese, and cafes.}
    \Description{
    The image contains three floor plans, each marked with dotted red lines and stars, indicating different routes and destinations. On the left side are two floors of a science museum, while the right side displays a single-floor plan of a shopping mall. The first map of the science museum highlights exhibits in light blue, connected by hallways in light gray. The floor has a rectangular shape with a main pathway running through the center. There are three round exhibits—two in the middle and one on the right—with additional exhibits placed along the left and right sides of the main walkway. A red dotted line shows the route followed in the first study, starting from the left side, circling the floor, and returning to the starting point. The starting point for the main study is located at the same spot as in the formative study. The second map represents the museum's second floor, also rectangular in shape and similarly structured to the previous floor but without the round exhibits. Again, a red dotted line marks the path followed in the formative study, beginning on the left side, looping around the floor, and returning to the starting point. However, the main study on this floor begins at a different location, on the right side of the floor. The last map on the right shows part of a shopping mall, featuring shops in blue and common areas in gray. The floor has a maze-like layout with several intersections and includes 21 restaurants. The route for the formative study begins in the center of the floor, loops around each restaurant, and returns to the starting point.
    }
    \label{fig:route}
\end{figure*}

\subsection{Prototype System}
We developed our prototype robot system according to Sec.~\ref{sec:system_design}. 
It was based on an open-source robot platform\footnote{https://github.com/CMU-cabot/cabot} and could guide users while explaining the surrounding environment. 
To ensure that the participants experienced the same level of autonomy, we used teleoperation, a Wizrd-of-Oz-based approach~\cite{riek2012wizard-of-oz}, to force the robot to be in full-automatic mode when guiding the participants.
We adopted a suitcase-shaped wheeled robot for this study.
The suitcase's appearance allows blind users to seamlessly blend into their environment, leading to higher social acceptance from users, surrounding pedestrians, and facility managers~\cite{kayukawa2022HowUsers}. 
As shown in Fig.~\ref{fig:device&ui}--A-1, the robot has a handle embedded with five buttons, a touch sensor beneath the handle, a 360$^\circ$ \red{Velodyne VLP-16 LiDAR sensor~\cite{Velodyne}} sensor, three RGBD cameras with resolutions of 640×360, \red{one RealSense D455 camera~\cite{RealSenseD455} mounted at the front, two RealSense D435 cameras~\cite{RealSenseD435} on the left and right}, and a pair of motorized wheels in differential drive configuration.
\red{
Inside the suitcase, it has Ruby R8 powered by an AMD Ryzen R7-4800U CPU~\cite{NUC}, and a Jetson Mate featuring multiple Jetson Xavier NX GPUs~\cite{JetsonMate}.
}
The RGBD cameras were attached 0.51 meters above the ground.
The touch sensor detects whether or not the user is holding the handle and moves only when it is being held by the user. 
The cameras combined have a horizontal field of view of approximately 180$^\circ$.
The weight of the robot is approximately 15kg.
We set the default speed of the robot to 0.5 meters per second to maintain a balance between a comfortable walking speed and a speed that allows sufficient time to absorb the scene description audios.
A smartphone is attached to the suitcase to provide audio feedback through a neck speaker worn by users, connected via Bluetooth.

To convey the surrounding information to the participants, we used GPT-4o~\cite{GPT4o}, a popular MLLM model.
We inputted the images from the three RGBD cameras into the MLLM model and asked the model to generate descriptions of the surrounding environment.
The robot was designed to describe surrounding information 5-10 seconds after the end of the previous description every time. 
We engineered the prompts to ask the MLLM model to first provide a general overview of the scene, followed by specific details on the left, front, and right. 
We asked the descriptions to include as many objects as possible and incorporate layout information, such as navigable directions and the presence of walls~\cite{jain2023want}. 
\red{The processing time and cost to generate a description was 6.087 seconds and \$0.00740 on average.}
We attach the full prompts in Appendix Sec.~\ref{appendix:prompt_formative}.

\begin{table*}[]
\caption{Demographics of participants who attended the formative study. The table reports their gender, age, navigation aid, which they frequently use, frequency of exploration done either independently or with sighted people per year, their experimental location, number of previous visits to the experimental location, and analyzed preference. }
\Description{The table presents demographic data for participants in the formative study. The information includes gender, age, type of mobility aid used, the age at which they started using the aid, frequency of exploration per year, the location of the experiment, the number of previous visits, and their preference analysis. The following describes each participant. P01 is a 64-year-old female who uses a cane, started using it at the age of 44, explores 48 times per year, participated at the Science Museum with 1 previous visit, and is exploration-inclined. P02 is a 53-year-old male who uses a cane, started using it at the age of 13, explores 36 times per year, participated at the Science Museum with 0 previous visits, and is destination-oriented. P03 is a 74-year-old male who uses a cane, started using it at the age of 0, explores 1 time per year, participated at the Science Museum with 0 previous visits, and is destination-oriented. P04 is a 54-year-old female who uses a cane, started using it at the age of 0, explores 12 times per year, participated at the Science Museum with 0 previous visits, and is exploration-inclined. P05 is a 56-year-old male who uses a cane, started using it at the age of 52, explores 2 times per year, participated at the Science Museum with 0 previous visits, and is intermediate in preference. P06 is a 32-year-old male who uses a cane, started using it at the age of 0, explores 12 times per year, participated at a shopping mall with 0 previous visits, and is intermediate in preference. P07 is a 55-year-old female who uses a cane, started using it at the age of 52, explores 0 times per year, participated at a shopping mall with 1 previous visit, and is exploration-inclined. P08 is a 63-year-old male who uses a cane, started using it at the age of 22, explores 12 times per year, participated at a shopping mall with 0 previous visits, and is intermediate in preference. P09 is a 78-year-old female who uses a guide dog, started using it at the age of 22, explores 12 times per year, participated at a shopping mall with 0 previous visits, and is destination-oriented. P10 is a 49-year-old female who uses a cane, started using it at the age of 3, explores 1 time per year, participated at a shopping mall with 0 previous visits, and is exploration-inclined.}
\label{tab:demographics1}
\resizebox{\textwidth}{!}{%
\begin{tabular}{ccccccccc}
\toprule
    & Gender & Age & Aid       & \begin{tabular}[c]{@{}c@{}}Age \\  of Onset\end{tabular} & \begin{tabular}[c]{@{}c@{}}Frequency of\\  Exploration per Year\end{tabular} & \begin{tabular}[c]{@{}c@{}}Experiment \\  Location\end{tabular} & \begin{tabular}[c]{@{}c@{}}Number of \\  Previous Visits\end{tabular} & Preference Analysis     \\
    \midrule
P01 & F      & 64  & Cane      & 44                                                       & 48                                                                           & Science Museum                                                  & 1                                                                     & Exploration-Inclined \\
P02 & M      & 53  & Cane      & 13                                                       & 36                                                                           & Science Museum                                                  & 0                                                                     & Destination-Oriented \\
P03 & M      & 74  & Cane      & 0                                                        & 1                                                                            & Science Museum                                                  & 0                                                                     & Destination-Oriented \\
P04 & F      & 54  & Cane      & 0                                                        & 12                                                                           & Science Museum                                                  & 0                                                                     & Exploration-Inclined \\
P05 & M      & 56  & Cane      & 52                                                       & 2                                                                            & Science Museum                                                  & 0                                                                     & Intermediate         \\
P06 & M      & 32  & Cane      & 0                                                        & 12                                                                           & Shopping Mall                                                   & 0                                                                     & Intermediate         \\
P07 & F      & 55  & Cane      & 52                                                       & 0                                                                            & Shopping Mall                                                   & 1                                                                     & Exploration-Inclined \\
P08 & M      & 63  & Cane      & 22                                                       & 12                                                                           & Shopping Mall                                                   & 0                                                                     & Intermediate         \\
P09 & F      & 78  & Guide dog & 22                                                       & 12                                                                           & Shopping Mall                                                   & 0                                                                     & Destination-Oriented \\
P10 & F      & 49  & Cane      & 3                                                        & 1                                                                            & Shopping Mall                                                   & 0                                                                     & Exploration-Inclined \\
\bottomrule
\end{tabular}
}
\end{table*}

\subsection{Experimental Location}
To ensure the diversity of the findings we would obtain from this study, we conducted the study in two different locations. 
\red{We chose to conduct our studies in a science museum and a shopping mall, as these are locations where people typically engage in exploration, and they have been utilized in previous research~\cite{asakawa2019independent,asakawa2018present,Kaniwa2024ChitChatGuide}. 
A museum is generally a place for learning about exhibits, while a shopping mall often requires exploration both before and during visits to stores.
\rrred{Specifically, we used the fifth floor of Miraikan\footnote{\url{https://www.miraikan.jst.go.jp/en/}} for the science museum and the fourth floor of Toranomon Hills Station Tower\footnote{\url{https://www.toranomonhills.com/}} for the shopping mall.}
The floor map of the science museum is illustrated in the left panel of Fig.~\ref{fig:route}, which contains two floors, both primarily featuring science exhibits. 
For the studies, the order of the two floors was counterbalanced.
The study in the museum was conducted after business hours, during which customers were absent, but staff were present for their duties.
The floor map of the shopping mall is illustrated in the right panel of Fig.~\ref{fig:route}, a floor that contains several restaurants from various countries. 
The study in the shopping mall was conducted during regular business hours.}
As shown in Tab.~\ref{tab:demographics1}, the study with P01--P05 took place in the science museum, and the study with P06--P10 took place in the shopping mall.

\subsection{Procedure}
For each participant, we first conducted a pre-study interview to learn about their experience in exploring buildings, followed by an explanation that the study aimed to gather their opinions on a guide system designed to assist with exploration.
Then, participants were given a task to navigate the predetermined route (red arrow of Fig.~\ref{fig:route}) guided by the robot.
Adopting a Wizrd-of-Oz-based approach, an experimenter controlled the robot to navigate along the route and stop when there were nearby pedestrians. 
During exploration, the robot periodically generated descriptions of the scenes. 
We show an example of the generated description in Fig.~\ref{fig:study1example}.
After the exploration, we asked the participants if there were any additional things they wanted to do to partially simulate the potential interaction, such as going to additional places or going around the floor again for more exploration.
Finally, we conducted a post-interview session to gather their feedback on the system. 

\subsection{Result}
\subsubsection{Interests to Exploration}
All participants stated that totally independent exploration is challenging, but they expressed a desire for exploration if a guide system can help them do so. For example:
\newanswer[\label{P02Conditioned}]\textit{``I don't really explore much. I go out with a specific purpose in mind [...] The reason is that it's just too bothersome. But I do think it would be fun if I did [...]  I'm more of an old-timer, so exploration never really caught my interest. It's not that I didn't care at all, but perhaps I've been living this way (not to explore).}\footnote{The comments were obtained in the native language where the study was conducted. We translate the comments into English using publicly available LLM to ensure reproducibility. We show the full prompt used for translation in Appendix Sec.~\ref{appendix:translate}.} (P02)

\subsubsection{Positive Feedback and Appreciated Information}
Seven participants (P01, P04--P08, and P10) expressed their enjoyment while navigating with the robot, particularly with the provided surrounding descriptions, as described in the following comment:
\newanswer[\label{P07Enjoy}]\textit{``My first impression was that it was a lot of fun. The reason is, as you just mentioned, unlike the person I usually walk with, the system provided detailed explanations about things like the color of the walls and the signs we saw and even described how the chef was preparing the food. Normally, you might get some of this information from others, but it's rare to get such thorough details. I found myself thinking, ``Oh, I see, that's how it looks to sighted people,'' and I felt there was a lot of new information. In that sense, I really enjoyed it.''} (P07)

Participants appreciated a variety of real-time details about their surroundings, notable examples include patterns on the walls, lighting conditions, subjective descriptors such as ``beautiful,'' the presence and actions of nearby people, the existence of signboards, the layout of the environment, and the visibility of a chef in an open kitchen. 
Additionally, P10, who requested to walk around the floor again, noted that receiving different descriptions of the same location was beneficial, as it gave them a sense of presence:
\newanswer[\label{P10VariousAndDifferentInformation}]\textit{``The system mentioned those things, as well as details about the plants and wall decorations. It's like, you talked about so many different things that it feels like I was actually looking around myself. Honestly, most of the time, I get so occupied with just reaching my destination that I don't notice things around me. [...] The system also mentioned things in the second round of explanations that weren't covered in the first round, which was nice. It conveyed a sense of the ongoing atmosphere and gave a good understanding of the situation at the time.''} (P10)

\subsubsection{Information Needs}
\label{sec:info_needs}
Participants hoped for further polishing of the delivered information about the scenes. Six participants (P01--P03, P06, and P09--P10) felt the information conveyed about the surroundings was too abstract, indicating the need for more specific information:
\newanswer[\label{P01NeedMoreConcreteInformation}]\textit{``The system talked about there are just exhibits, or there's information on panels, but I think it would be nice if the system talked about specific titles. There are places where the system talked about them, but there are also places where it did not, so I found myself wondering about that.''} (P01)
In particular, three participants (P02, P03, and P09) commented that the descriptions neither helped them learn the environment nor make decisions such as determining which shops or exhibits to enjoy:
\newanswer[\label{P09NegativeImpression}]\textit{``I expected it to at least tell me the name of the store, but it was disappointing to find out that it didn't do that at all. I really wish there was a system that could provide pinpointed information about what I want to know. Especially in an unfamiliar restaurant area, for example, if I come alone and use the device to enter the premises, it starts running, and then when I think, ``Oh, should I have Japanese food today, or maybe tonkatsu?'', without such information, I end up just walking around aimlessly.''} (P09)

Participants also described specifics about what types of information would be beneficial to include, such as the position of objects given in meters and clock directions, the availability of seats, people on collision paths, identities of surrounding individuals (\eg staff), and specific names of objects. 
In science museums, participants also wanted to know whether exhibits are touchable. 
In shopping malls, participants also wanted to learn the store menus and whether there is a spacious area for a guide dog to rest while the user is eating.
However, three other participants (P02, P03, and P09) found certain information, such as details about lighting, surrounding people, and wall design, unnecessary. 

\begin{figure*}
    \centering
    \includegraphics[width=1\linewidth]{figure/examples.png}
    \caption{Examples of descriptions described in the formative study. Panel A shows an example of a description generated at the science museum, and Panel B shows the one generated at a shopping mall.}
    \Description{Examples of descriptions described in the formative study. Panel A shows an example of a description generated at the science museum, saying "This is a futuristic exhibition hall that has vibrant displays. To your left, there is a uniquely shaped wooden table and archway. Ahead, you can see a curved blue sofa and a white sign that reads "Entrance." On your right, large colorful panels line the wall, displaying information about the future and health." and Panel B shows the one generated at a shopping mall, saying "This is a bright, modern corner of a commercial facility. On the left side, there are tall-backed chairs made of black metal lined up, and beyond them, round tables are arranged. Ahead, a man in a suit is standing, and in the background, there's an electronic menu board, suggesting the presence of a restaurant. To the right, there's an eatery enclosed by warm-colored walls in shades of red and orange, with many metallic chairs and tables, and menu boards are set up."}
    \label{fig:study1example}
\end{figure*}

\subsection{Design Considerations}
The results of the study affirmed that there are certain appreciations and room for improvement for the exploration robot for blind people.
Based on the above results, we derived several requirements for the system, as listed below.

\subsubsection{Vary Detail of Descriptions Based on Preferences and Contexts}
\label{sec:implication_varydetail}
We observed three types of preferences: one that enjoyed all the descriptions provided by the system (\textit{Exploration-Inclined}), another that enjoyed the descriptions but preferred to limit certain information (\textit{Intermediate}), and a third group that only wanted information useful for determining where to go (\textit{Destination-Oriented}). 
In Tab.~\ref{tab:demographics1}, we show the description preference of each participant.
To classify the preferences, we first classified three participants who did not enjoy the description of the system as \textit{Destination-Oriented}.
Then, based on the discussion between the authors, we classified the rest as \textit{Intermediate} or \textit{Exploration-Inclined}.
Furthermore, the type of information needed varied slightly depending on the experimental location. 
For instance, participants sought seating information for guide dogs in shopping areas, whereas in the science museum, they were more interested in whether the exhibits were touchable.
Given these three types of preferences and context-dependent information needs, we modified the system so that it could adjust the amount and types of information conveyed to each participant.

\subsubsection{Add Question and Answer Functionality}
\label{sec:implication_Q&A}
There was a clear need for question-and-answer (Q\&A) interaction, as seven participants (P02--P05 and P08--P10) noted that they would like the option to ask more detailed questions through conversation. 
Participants expressed interest in this functionality when they were curious about the system's descriptions. This would allow them to ask more detailed questions about the objects of interest.

\subsubsection{Add ``Take-Me-There'' Functionality} 
\label{sec:implication_takemethere}
Four participants (P02, P04, P06, and P10) mentioned that they would like to revisit locations they found interesting after walking around the floor. 
Example situations include deciding to visit a shop, engaging with touchable exhibits, or returning to chairs discovered during the exploration. 
In unfamiliar locations, where users may lose their sense of direction, participants also expressed the need for a feature that guides them back to their initial location~\cite{kuribayashi2023pathfinder}.

\subsubsection{Vary Speed and Be Able to Stop the Robot}
\label{sec:implication_speed}
While the majority found the default speed appropriate for listening and understanding the described information, there were requests for customizable speed settings. 
Eight participants stated that the robot's speed was appropriate for exploring. 
Two participants (P04 and P06) expressed a preference for a faster speed.
P01 additionally wanted to stop when the robot read out the descriptions of interest.
In conclusion, users who are \textit{Destination-Oriented} or have already determined the destination through exploration may want to increase the speed, while users who prefer to take time exploring might wish to slow down or stop the robot entirely. 

\subsubsection{Add Direction Specifying Functionality}
\label{sec:implication_directionspecification}
Participants expressed a desire for more active engagement by specifying the movement direction themselves. 
Four participants (P02--P05) mentioned that they wanted more active control over the movement direction based on their interests.
Additionally, we extrapolated that instead of simply following the robot, some users may prefer to interactively choose the direction based on the audio description of the surroundings.
This could lead to greater autonomy because it would enrich the exploratory experience by aligning the robot's movement with the users' real-time curiosity and needs, creating a more personalized and engaging exploration experience.

\section{Implementation}
We train the user and \interfaceagent's policies simultaneously in a shared environment (the AUI). All policies receive an independent reward, and the actions of the policies influence a shared environment. We execute actions in the following order: (1) the \interfaceagent's action, (2) the \useragent's high-level action, followed by (3) the \useragent's low-level motor action. The reward for the two learned policies is computed after the low-level motor action has been executed. The episode is terminated when the \useragent has either completed the task or exceeded a time limit.

We implement our framework in Python 3.8 using RLLIB \cite{liang2018rllib} and Gym \cite{brockman2016openai}. We use PPO \add{\cite{schulman2017proximal, yu2022surprising}} to train our policies. We use 3 cores on an Intel(R) Xeon(R) CPU @ 2.60GHz during training \add{and an NVIDIA TITAN Xp GPU}. Training takes $\sim$36 hours. \del{We utilize an NVIDIA TITAN Xp GPU for training.} The \useragent's high-level decision-making policy $\policy_d$ is a 3-layer MLP with 512 neurons per layer and ReLU activation functions. The  \interfaceagent's policy  $\policy_I$ is a two-layer network with 256 neurons per layer and ReLU activation functions. \addiui{We sample the full state initialization (including goal) from a uniform distribution. We use stochastic sampling for our exploration-exploitation trade-off.} 

We use curriculum learning to increase the task difficulty and improve learnability. Specifically, we adjust the difficulty level every time a criteria has been met by increasing the mean number of initial attribute differences. More initial attribute differences result in longer action sequences and are therefore more complex to learn. We increase the mean by 0.01 every time the successful completion rate is above 90\% and the last level up was at least 10 epochs away.

We randomly sample the number of attribute differences from a normal distribution with standard deviation $1$, normalize the sampled number into the range $[1, n_a]$ and round it to the nearest integer, where $n_a$ is the number of attributes of a setting (in the case of game character $n_a=5$).

\del{The difference between agents of different applications is their respective state- and action spaces.}
\section{Main User Study}
\label{sec:study2}
This study was conducted to validate WanderGuide and explore further design space.
Participants were recruited and compensated similarly to those in the formative study.
Similar to the formative study, in the recruitment email, we specified that participants unfamiliar with the experimental location would be eligible to participate.
We conducted this study on the same two floors of the science museum. 
Tab.~\ref{tab:demographics2} shows the demographics of the participants.
\red{
None of the participants from the formative study participated in this study.
Similar to the formative study, this study was conducted after business hours.
}

\subsection{Task and Procedure}
For each participant, we first conducted a pre-study interview similar to the formative study.
Then, the participant joined a 30-minute training session to get familiar with the robot system before the main tasks.
For the main tasks, they were asked to freely explore the floor for 20 minutes using the system from a fixed starting location, as illustrated in Fig.~\ref{fig:route}.
The ordering of the floors was counterbalanced to mitigate the order effect.
After the main tasks, we conducted a post-study interview to ask several seven-point Likert scale questions (1: Strongly Disagree, 4: Neutral, and 7: Strongly Agree) that measure their self-evaluated exploration performance, Raw Task Load Index (TLX)~\cite{byers1989traditional} to measure the task workload, and system usability scale (SUS)~\cite{brooke1996sus} to evaluate the usability of the system.
Finally, we asked open-ended questions to gather comments on the system.
Below, we report the results of the study.

\begin{table}[]
\caption{The statistics of duration time and the count of interactions for each mode (Auto, Conversation, and Manual Control).
The ratio of the duration time is calculated based on the total duration time of the experiment per participant.}
\Description{
The table provides an analysis of the duration ratio (percentage of total time) and the count of interactions for three different modes: Auto, Conversation, and Manual Control. The statistics are presented for five participants: P11, P12, P13, P14, and P15. P11 spent 59.77\% of the time in Auto mode with 25 interactions, spent 37.52\% of the time in Conversation mode with 21 interactions, and spent 2.70\% of the time in Manual Control mode with 4 interactions. P12 spent 91.66\% of the time in Auto mode with 13 interactions, spent 8.16\% of the time in Conversation mode with 9 interactions, and spent 0.18\% of the time in Manual Control mode with 1 interaction. P13 spent 67.88\% of the time in Auto mode with 12 interactions, spent 30.56\% of the time in Conversation mode with 9 interactions, and spent 1.56\% of the time in Manual Control mode with 1 interaction. P14 spent 64.86\% of the time in Auto mode with 17 interactions, spent 33.94\% of the time in Conversation mode with 15 interactions, and spent 1.20\% of the time in Manual Control mode with 1 interaction. P15 spent 58.53\% of the time in Auto mode with 28 interactions, spent 20.03\% of the time in Conversation mode with 19 interactions, and spent 21.44\% of the time in Manual Control mode with 10 interactions.
}
\label{tab:activity_breakdown}
\begin{tabular}{@{}lcccccc@{}}
\toprule
    & \multicolumn{2}{c}{Auto} & \multicolumn{2}{c}{Conversation} & \multicolumn{2}{c}{Manual Control} \\
    & Ratio(\%)          & Count         & Ratio(\%)          & Count         & Ratio(\%)           & Count          \\ \midrule
P11 & 59.77         & 25            & 37.52         & 21            & 2.70           & 4              \\
P12 & 91.66         & 13            & 8.16          & 9             & 0.18           & 1              \\
P13 & 67.88         & 12            & 30.56         & 9             & 1.56           & 1              \\
P14 & 64.86         & 17            & 33.94         & 15            & 1.20           & 1              \\
P15 & 58.53         & 28            & 20.03         & 19            & 21.44          & 10             \\ \bottomrule
\end{tabular}
\end{table}
\begin{table}[]
\caption{Analysis of participants' requests to the system during conversation mode. We defined three types of queries, General Query, Specific Query, and Command Query, and classified each participant request into one of them. The ratios here are simply the count percentages over total counts.}
\Description{
The table provides a analysis of participants' requests made to the system during conversation mode. Requests are categorized into three types: General Query, Specific Query, and Command Query. The ratio (percentage) and count of each type of request are shown for each participant, along with the total number of requests. The following summarizes the data for each participant: P11 made 11.43\% of their requests as General Queries (4 requests), 45.71\% as Specific Queries (16 requests), and 42.86\% as Command Queries (15 requests), with a total of 35 requests. P12 made 30.00\% of their requests as General Queries (3 requests), 10.00\% as Specific Queries (1 request), and 60.00\% as Command Queries (6 requests), with a total of 10 requests. P13 made 61.54\% of their requests as General Queries (8 requests), 38.46\% as Specific Queries (5 requests), and 0.00\% as Command Queries (0 requests), with a total of 13 requests. P14 made 14.29\% of their requests as General Queries (4 requests), 46.43\% as Specific Queries (13 requests), and 39.29\% as Command Queries (11 requests), with a total of 28 requests. P15 made 8.33\% of their requests as General Queries (2 requests), 25.00\% as Specific Queries (6 requests), and 66.67\% as Command Queries (16 requests), with a total of 24 requests.
}
\label{tab:request_breakdown}
\resizebox{\columnwidth}{!}{%
\begin{tabular}{@{}cccccccc@{}}
\toprule
                     & \multicolumn{2}{c}{General Query} & \multicolumn{2}{c}{Specific Query} & \multicolumn{2}{c}{Command Query} & \multirow{2}{*}{Total}\\
                     & Ratio (\%)       & Count      & Ratio (\%)       & Count      & Ratio (\%)        & Count       \\ \midrule
P11  & 11.43            & 4         & 45.71            & 16         & 42.86              & 15           &  35\\ 
P12  & 30.00            & 3         & 10.00            & 1          & 60.00              & 6           &  10\\ 
P13  & 61.54            & 8         & 38.46            & 5          & 0.00              & 0           &  13\\ 
P14  & 14.29            & 4         & 46.43            & 13         & 39.29              & 11           &  28\\ 
P15  & 8.33            & 2         & 25.00            & 6         & 66.67              & 16           &  24\\ \bottomrule
\end{tabular}
}
\end{table}


\subsection{Analysis of Participants Activity During The Task}
\label{sec:activity_breakdown}
We report the statistics of each participant's activity during the task by referring to the system's log and the video captured during the tasks. 
Tab.~\ref{tab:activity_breakdown} shows the analysis of their time spent on the three modes as specified in Sec.~\ref{sec:implementation_button}.
We noticed that the activation quantity and duration of each mode varied significantly among participants.
P11, P13, P14, and P15 frequently used the conversation mode.
Notably, P11 spent nearly 40\% of the total time engaging in conversation with the robot.
In contrast, P12 barely used the conversation mode and relied on the auto mode for 90\% of the total time.
%Secondly, there was also a difference in the usage of the Auto and Manual control modes. 
P15 was the only participant who actively used the manual control mode.


\begin{figure*}[t] \centering
    % \vspace{15pt}
    \includegraphics[width=\textwidth]{figures/images/error_analysis.pdf}
    \caption{Error Analysis of Proprietary Models and Open-source Models. We count the proportion of each error type in the code generated by proprietary and open-source Models.} \label{fig:error_analysis}
\end{figure*}
\begin{table}[]
\caption{The statistics of the usage of each description level.
Usage statistics for each description level are calculated by normalizing the duration of each level with respect to the total duration of the experiment.
}
\Description{
The table presents the usage statistics for different description levels—Concise, Balanced-Length, and Descriptive—by normalizing the duration of each level with respect to the total duration of the experiment for five participants (P11 to P15). P11 used Concise descriptions 0.10\% of the time, Balanced-Length descriptions 76.46\% of the time, and Descriptive descriptions 23.43\% of the time. P12 used Concise descriptions 15.22\% of the time, Balanced-Length descriptions 29.88\% of the time, and Descriptive descriptions 54.91\% of the time. P13 used Concise descriptions 0.19\% of the time, Balanced-Length descriptions 49.14\% of the time, and Descriptive descriptions 50.66\% of the time. P14 used Concise descriptions 0.15\% of the time, Balanced-Length descriptions 87.94\% of the time, and Descriptive descriptions 11.91\% of the time. P15 used Concise descriptions 0.14\% of the time, Balanced-Length descriptions 99.86\% of the time, and Descriptive descriptions 0.00\% of the time.
}
\label{tab:description_level}
\begin{tabular}{@{}lccc@{}}
\toprule
    & Concise & Balanced-Length & Detailed \\ \midrule
P11 & 0.10\%  & 76.46\%         & 23.43\%     \\
P12 & 15.22\% & 29.88\%         & 54.91\%     \\
P13 & 0.19\%  & 49.14\%         & 50.66\%     \\
P14 & 0.15\%  & 87.94\%         & 11.91\%     \\
P15 & 0.14\%  & 99.86\%         & 0.00\%      \\ \bottomrule
\end{tabular}
\end{table}

\subsection{Analysis of Requests from Participants During Within The Conversation Mode}
\label{sec:request_breakdown}
In Tab.~\ref{tab:request_breakdown}, we further report the statistics of requests from participants within the conversation mode. 
Note that the total count of conversations in Tab.~\ref{tab:request_breakdown} is bigger than the conversation mode counts in Tab.~\ref{tab:activity_breakdown}, as multiple turns of conversation could happen in one conversation mode interaction.
We classify each verbal request into three categories. 
\begin{description}
    \item[General Query] Request general information in the surrounding area or in a particular direction.
    \item[Specific Query] Request detailed information about a specific object in the environment.
    \item[Command Query] Issue command to guide to destination, triggering ``Take-Me-There'' functionality or direction specification via conversation.
\end{description}

Overall, we discovered that although our system constantly provided environmental descriptions in auto mode, users still preferred to ask for general information about their surroundings or in a specific direction in conversation mode. 
For example, P13 predominantly made General Queries (61.54\%). 
Users also had diverse preferences when using our system. 
Some users such as P11 (45.71\%), P13 (38.46\%) and P14 (46.43\%) were interested in learning the specifics of POIs, reflecting the takeaways obtained in Sec.~\ref{sec:info_needs}. 
Some users such as P11 (42.86\%), P12 (60.00\%), P14 (39.29\%), and P15 (66.67\%) favored using conversation mode to instruct the robot to guide them to their destinations. 
In particular, by referencing Tab.~\ref{tab:activity_breakdown}, we can see that P11, P12, and P14 preferred conversation mode over manual control mode to issue commands. 
This validates the extrapolated idea in Sec.~\ref{sec:implication_directionspecification}.

\subsection{Error Analysis of Scene Description and Q\&A Responses}
\label{sec:error_analysis}
In Tab.~\ref{tab:hallucinations}, we report the accuracy of MLLM responses both during auto and conversation modes.
We manually analyzed the text output generated by MLLM and compared it with the logs of the images saved.
We classified and counted the errors made by MLLM into six categories.
\begin{description}
    \item[Wrong Character Recognition] Misrecognition of text, such as misreading signs.
    \item[Wrong Object Recognition] Misidentification of objects in the scene.
    \item[Nonexistent Objects and Texts] Mistakenly recognizing objects or text that are not present. Note that this differs from the previous two categories, where some similar objects or text were actually present.
    \item[Misunderstanding User Input] Misinterpreting a user’s question in conversation mode, such as providing an environmental description when asked to read text from a panel.
    \item[Inaccurate User Input] Errors made when the user asked about objects or text that were not present.
    \item[No Error] Accurate responses with no errors.
\end{description}
\red{
When multiple errors occur in a single sentence, errors of the same type are grouped together and counted as one. 
Errors of different types are counted separately. 
For instance, if there are multiple text recognition errors in a single sentence, they are counted as one text recognition error. 
If a sentence contains both text recognition errors and object recognition errors, each is counted separately as one text recognition error and one object recognition error.
Thus, note that the total number of errors may not match the total number of outputs.}

The results showed that 28.6\% of the outputs contained some form of error during scene descriptions whereas 60.3\% of conversation mode outputs had errors.
This difference is likely because users in conversation mode often asked for more detailed explanations, which led MLLM to attempt more complex responses and, as a result, made more mistakes.
This was particularly evident in the \textit{Nonexistent Objects and Texts} category, which accounted for only 0.07\% of errors during scene descriptions but significantly higher at 28.3\% in conversation mode.
This means that MLLM often generated descriptions of objects or text that did not exist in the environment when asked for more detailed information.
Character recognition errors were common in both modes, likely due to MLLM’s limitation in reading distant text. 
In a general sense, instead of complete failures, MLLM often partially misread the text or misidentified objects with similar-looking ones (\eg, mistaking a tall table for a reception desk).
Nevertheless, over 70\% of responses in the auto mode were accurate, demonstrating the overall usefulness of the system.

\subsection{Analysis of Usage of Each Description Level}
\label{sec:description_level_analysis}
In Tab.~\ref{tab:description_level}, we report the statistics of how much time participants spend their time using each description level.
The result shows that there were three types of usage during the study. 
P15 only used Balanced-Length mode, P11 and P14 used Balanced-Length mode most of the time while sometimes using Detailed mode, and P12 and P13 used Detailed mode most of the time. 

\begin{figure*}
    \centering
    \includegraphics[width=0.8\linewidth]{figure/boxplot.png}
    \caption{Box plot of evaluation with human experts in seven-point Likert points.}
    \Description{The figure displays box plots for four questions evaluating the generated descriptions: Q1. I think the generated descriptions are natural: The median score is 5, with a minimum of 2, a first quartile of 4, a third quartile of 6, and a maximum of 7. Q2. I think the generated descriptions are precise: The median score is 5, with a minimum of 1, a first quartile of 3, a third quartile of 6, and a maximum of 7. Q3. I think the generated descriptions are appropriate as descriptions provided by a navigation robot to blind people onsite: The median score is 4, with a minimum of 2, a first quartile of 3, a third quartile of 5, and a maximum of 7. Q4. I think the generated descriptions are appropriate as descriptions provided when I am there to explain to blind people onsite: The median score is 3, with a minimum of 1, a first quartile of 2, a third quartile of 4, and a maximum of 7.    
    }
    \label{fig:boxplot}
\end{figure*}


% \usepackage{graphicx}
\begin{table*}[]
\caption{Rating to seven-point Likert score questions (1: strongly disagree; 4: neutral; 7: strongly agree).}
\Description{The table presents ratings from five participants (P11, P12, P13, P14, and P15) on a set of questions based on a seven-point Likert scale, where 1 indicates "strongly disagree," 4 is "neutral," and 7 is "strongly agree." The median rating for each question is also provided. The following summarizes the responses for each question. For Q1, "I was able to explore the facility," P11 and P13 rated 4, while P12, P14, and P15 rated 6, resulting in a median score of 6. For Q2, "I was able to enjoy the exploration," P11 and P13 rated 4, while P12, P14, and P15 rated 6 or 7, with a median score of 6. For Q3, "I was able to gain an interest in the things around me," P11 rated 4, P13 rated 6, and P12, P14, and P15 rated 6, leading to a median score of 6. For Q4, "The interface of the system was easy to understand," P11, P14, and P15 rated 5, while P12 and P13 rated 6, resulting in a median score of 5. For Q5, "I want to explore where I am familiar with this system," P11, P12, and P14 rated 7, while P13 and P15 rated 6, leading to a median score of 7. For Q6, "I want to explore where I am unfamiliar with this system," P11, P14, and P15 rated 7 or 6, resulting in a median score of 6.}
\label{tab:likert}

\begin{tabular}{l|ccccc|c}
\toprule
\multicolumn{1}{c|}{}                                         & P11 & P12 & P13 & P14 & P15 & Median \\ \hline
Q1. I was able to explore the facility.                       & 4   & 6   & 4   & 6   & 6   & 6      \\
Q2. I was able to enjoy the exploration.                      & 4   & 7   & 4   & 6   & 6   & 6      \\
Q3. I was able to gain an interest in the things around me.   & 4   & 6   & 6   & 6   & 6   & 6      \\
Q4. The interface of the system was easy to understand.       & 5   & 6   & 6   & 5   & 5   & 5      \\
Q5. I want to explore where I am familiar with this system.   & 7   & 7   & 6   & 7   & 6   & 7      \\
Q6. I want to explore where I am unfamiliar with this system. & 7   & 6   & 6   & 7   & 6   & 6      \\ \bottomrule
\end{tabular}%
\end{table*}
\begin{table}[]
\caption{Scores for the Raw TLX provided by each participant. Lower total scores indicate a lower workload. Each item is scored on a scale from 1 to 10, where 1 represents a lower level, and 10 represents a higher level of Mental Demand, Physical Demand, Temporal Demand, Effort, and Frustration. For Performance, 1 indicates good performance, and 10 indicates poor performance.}
\Description{
The table presents Raw TLX scores from five participants (P11, P12, P13, P14, and P15), measuring subjective workload across six dimensions: Mental Demand, Physical Demand, Temporal Demand, Performance, Effort, and Frustration. Each dimension is rated on a scale from 1 to 10, where higher scores generally indicate higher levels of demand, effort, or frustration, except for Performance, where a higher score indicates poorer performance (1: Good, 10: Poor). Lower total scores signify a lower overall workload.
Participant P11 reported low levels of Mental Demand (2), Physical Demand (2), Temporal Demand (2), and Effort (3). However, they scored higher in Performance (7) and Frustration (8), leading to a total score of 24. Participant P12 had low scores across all dimensions: Mental Demand (2), Physical Demand (2), Temporal Demand (3), Performance (2), Effort (2), and Frustration (4), resulting in the lowest total score of 15. Participant P13 scored higher in Mental Demand (5), Effort (5), and Performance (5), moderate in Physical Demand (3), and low in Temporal Demand (1) and Frustration (1), totaling a score of 20. Participant P14 reported low Mental Demand (3) and Physical Demand (2), but higher Temporal Demand (5), Performance (7), Effort (6), and Frustration (3), culminating in a total score of 26. Participant P15 had low Mental Demand (2) and Temporal Demand (2), but higher Physical Demand (6), Performance (5), Effort (6), and Frustration (7), leading to the highest total score of 28.
}
\label{tab:tlx}
\begin{tabular}{l|ccccc|c}
\toprule
                & P11 & P12 & P13 & P14 & P15 & Median \\ \hline
Mental Demand   & 2   & 2   & 5   & 3   & 2   & 2      \\
Physical Demand & 2   & 2   & 3   & 2   & 6   & 2      \\
Temporal Demand & 2   & 3   & 1   & 5   & 2   & 2      \\
Performance     & 7   & 2   & 5   & 7   & 5   & 5      \\
Effort          & 3   & 2   & 5   & 6   & 6   & 5      \\
Frustration     & 8   & 4   & 1   & 3   & 7   & 4      \\ \hline
Total Score     & 24  & 15  & 20  & 26  & 28  &        \\ \bottomrule
\end{tabular}
\end{table}

\subsection{Scene Description Quality Evaluation}
\label{sec:quality_eval}
Finally, to analyze the quality of the MLLM-generated scene descriptions from the human expert perspective, we conducted a survey with human museum guides and asked them to evaluate using a seven-point Likert scale. 
The participants were presented with images captured by the robot, each accompanied by its corresponding generated description, and were asked to evaluate the descriptions in a survey, as shown in Fig.~\ref{fig:boxplot}. 
The survey was conducted in a counterbalanced manner to mitigate potential biases.
During the main study, 164 descriptions were \rrred{generated}, and we randomly sampled half (82) of the total descriptions for evaluation.
\rrred{
The randomly sampled descriptions contain mixed levels of detail.
}
Each description is evaluated by three to four participants.
In total, 56 museum guides participated in the evaluation, with each randomly assessing five descriptions. 
There were 32 males and 20 females, and four participants did not report their gender.
Their average age was 39.6 years, with an average of 5.9 years of experience as a museum guide. 
On seven-point Likert scale items, the median self-reported familiarity with museums was 5.0, and the familiarity with LLMs was 4.0 (1: very unfamiliar, 4: neutral, and 7: very familiar).
Our analysis revealed that the experts generally perceived the generated descriptions as somewhat natural (Q1) and precise in describing an image (Q2) as shown by their median of five.
Meanwhile, they found the generated descriptions less suitable as image descriptions for blind people (Q3) and as onsite descriptions provided by experts for blind people (Q4).



\subsection{Usability and Workload Evaluation}
In Tab.~\ref{tab:likert}, we report the results of seven-point Likert items. 
For Likert items, a median score of five or higher indicates that participants generally responded positively.
The total SUS for P11 to P15 were 72.5, 80, 90, 82.5, and 77.5, respectively, showing acceptable usability of all being above 70~\cite{bangor2009determining}. 
The total Raw TLX scores for P11 to P15 were 24, 15, 20, 26, and 28, respectively. 
We show the distribution of Raw TLX scores in Tab.~\ref{tab:tlx}.
Raw TLX~\cite{byers1989traditional}, a simplified version of NASA TLX~\cite{hart2006nasa}, is known to have a high correlation with NASA TLX, and the total NASA-TLX scores for people with special needs typically ranged from 26 to 48 in previous research~\cite{hertzum2021reference}. 
Overall, our total Raw TLX scores may suggest that participants did not experience a significant load during the task.
We also observed that the median value for mental, physical, and temporal demand was relatively lower, scoring 2. 
This is likely due to the robot navigating them, allowing participants to explore without being burdened by these demands. 
Nonetheless, a relatively higher median value was observed for Performance, Effort, and Frustration, indicating that some users experienced a lack of satisfaction with the exploration experience provided by the system. 

\subsection{Qualitative Analysis}
\subsubsection{Positive Feedback}
All participants expressed their appreciation for the experience of wandering around a building to explore without specific destinations in mind with the help of our system:
\newanswer[\label{P12IWantThis}]\textit{``
When the camera explains things it recognizes, like how bright the room is or what the floor looks like, or what objects are placed where, I found myself nodding in agreement multiple times, like, ``Oh, so this is how it looks.'' 
I remember when I first held the suitcase robot, I deeply empathized with guide dog users. I thought, ``Oh, so this is what it's like to have a guide dog.'' However, since I can't take care of a guide dog, I’ve given up on that option. 
And now, with this navigation system that explains various situations, it's exactly what I need. It’s not just about setting a destination and getting there but feeling the freedom to explore spontaneously. For example, the ability to roam a large shopping mall freely and explore on a whim feels like true freedom to me. Instead of pre-planning every move or relying on a guide, I could simply grab my suitcase and decide to venture out spontaneously.''} (P12)

\red{
The same participant, P12, who had been to the facility previously, noted that they still had new discoveries with the system:
\newanswer[\label{P12IHaveBeen}]\textit{``
I've been to this museum before, but when the guide explained things to me back then, it was more like a general explanation about the atmosphere and such. 
But earlier with the system, there was a very detailed explanation that came out of the suitcase. Like, about how bright sunlight comes [...] 
There were things I didn’t know that made me learn new stuff, even though I thought I knew about the facility.''} (P12)
Also, P12 and P14 noted the feeling of relief not relying on sighted assistance:
\newanswer[\label{P14ThereHaveBeenNoSystem}]\textit{``
I don’t think there has ever been a system that explains your surroundings while walking. [...]
When walking with other people, I often find myself feeling a sense of obligation. I worry that they’re putting in extra effort to describe things because I can’t see. And then I feel like I have to respond to them since they’re trying so hard—which can be exhausting. But with this system, I feel I can go strolling by myself.''} (P14)
}


Participants also noted the functionality to go to an aforementioned destination and Q\&A functionality particularly useful:
\newanswer[\label{P11TakeMeBack}]\textit{``(The ``Take-Me-There'' functionality is) I think it's wonderful. After all, spatial awareness is difficult, so going back to landmarks is very important. If it is accurate, I think it's great because it can be extremely helpful for spatial cognition.''} (P11) and
\newanswer[\label{P14Q&A}]\textit{``When engaging in a conversation, not knowing what kind of response you'll get, the feeling of unease and excitement that's both a plus and a minus, I think. But I found it really great that you can still ask questions. So even if the response you get doesn't answer your question, or even if it's just ``I don't know,'' the fact that you can at least ask is important.''} (P14) 

\subsubsection{Adjusting Detail of Description}
When we discussed their preference in the level of detail of descriptions, all participants described that it would rather depend on the scenario they are in:
\newanswer[\label{P12NormalMode}]\textit{``It might depend on the location, but I know I can get detailed information in Q\&A functionality. So, for familiar places, the Balanced-Length mode might be fine. However, there are parts where I'd want the Detailed Description mode for unfamiliar places. For example, switching between modes could be useful, like having Detailed Description mode first for explanations about the room's brightness and how easy it is to walk around. ''} (P12)

\subsubsection{Comments to Improve the System}
\label{sec:improve}
Participants suggested various improvements to the system.
One particular suggestion was to incorporate functionality for the robot to understand sounds. 
As the experiment location was a science museum, various exhibits emitted sounds.
P13 noted that they would like to inquire about the sound sources, which were not supported by the system:
\newanswer[\label{P13Sound}]\textit{``We are extremely sensitive to sounds, and it becomes a point of interest. At a place like the exhibition hall we're visiting this time, various sounds are coming from all directions. This prompts questions like, ''What's happening at that sound over there?'' Therefore, it would be advantageous if we could ask specific questions like, ``What's that sound coming from the right?'' ''} (P13)

Also, four participants (P11-P13 and P15) found the descriptions from the system still insufficient to explore, as described in the following comments:
\newanswer[\label{P15Insufficient}]\textit{``The place we did the task this time was quite out of the ordinary. Even if you were walking around with my family, I think they would also have difficulty explaining it. Therefore, I felt it might still be somewhat challenging for machines to handle this kind of thing. However, I did feel it was good that I got a sense of what was there. But when it comes to the actual detailed explanations, it was not there [...]''} (P15)

\subsubsection{Specification of Proceeding Direction}
While we introduced all functionality to participants within the training session, we observed that only P15 used the functionality to specify which way to proceed via a button or conversation.
P15 tended to use the functionality when P15 was interested in a specific object:
\newanswer[\label{P15DirectionSpecification}]\textit{``It seems that when I was told, ``There's something on the right,'' I tried to approach toward it because I wanted to get closer when I used something like that.''} (P15)
\section{Discussion}
\subsection{Experience of Using WanderGuide}
WanderGuide provided participants with the experience of exploring unfamiliar indoor environments without a specific destination in their minds, mimicking the spontaneous wandering experience of sighted people (C\ref{P12IWantThis}). 
Participants expressed a sense of confidence when using the system, noting that it allowed them to navigate independently without relying on traditional tools like white canes (C\ref{P12IWantThis}). 
As described by C\ref{P15Insufficient} and the ratings of 4 from P11 and P13 to Q1 and Q2 in Tab.~\ref{tab:likert}, there still exists the limitation of being unable to describe specific information.
Thus, there is a need for further research on how to appropriately convey surrounding information to blind people.
Still, the system’s ability to deliver real-time descriptions of objects, walls, and spatial layouts enabled participants to form an imagination (C\ref{P12IWantThis}) of their surroundings, sparking their desire to use the system in familiar and unfamiliar environments (Tab.~\ref{tab:likert} Q5 and Q6).
In short, WanderGuide has the potential to provide users with an experience similar to that of navigating with sighted assistants to explore the environment, but the users can explore independently.
We believe this research opens a new frontier to the concept of \textit{map-less exploration} guide system for blind people.

\red{
\subsection{Scene Description by MLLM}
\label{sec:scene_description}
Our survey in Sec.~\ref{sec:quality_eval} revealed that descriptions by MLLM were rated high for their naturalness and suitability for general image description but were not for actual descriptions to be provided to blind people by sighted experts.
This may be because the style and content of the generated descriptions differ from those typically provided to blind people during live interactions. 
For example, museum guides often focus on explaining notable objects or visible exhibits, complementing their descriptions with additional knowledge about the exhibits.
In contrast, the generated description often lacked concrete explanation about exhibits and shops, such as their names (C\ref{P01NeedMoreConcreteInformation}, C\ref{P09NegativeImpression}, and C\ref{P15Insufficient}).
This problem may be more prominent because the study was conducted in a science museum, where each exhibit contains detailed information that is not visually apparent but needs to be explained. 
On the other hand, from participant feedback, participants noted that MLLM-generated descriptions are comprehensive (C\ref{P07Enjoy}, C\ref{P10VariousAndDifferentInformation}, C\ref{P12IHaveBeen}, and C\ref{P14ThereHaveBeenNoSystem}), and provide them with enjoyment (C\ref{P07Enjoy}) and imagination of vision perception (C\ref{P10VariousAndDifferentInformation}).
They noted that MLLM provided them with information that they usually do not get from sighted assistants, leading to new discoveries (C\ref{P12IHaveBeen}). The descriptions provided by MLLM additionally allow blind people to tune in without hesitation and the need to rely on sighted people (C\ref{P14ThereHaveBeenNoSystem}).
These results indicate that evaluation from sighted experts may be stricter than that from blind people.
Nonetheless, these results suggest that MLLM for blind people's exploration could be further enhanced by providing more specific information about surrounding shops or exhibits, potentially inferring details when necessary. 
}

\subsection{Personal Preferences}
The studies revealed distinct preferences among participants regarding the levels of detail in the descriptions (Sec.~\ref{sec:Implementation_description_mode}) and interaction modes (Sec.~\ref{sec:implementation_button}). 
From the formative study, participants were divided into three preference groups, highlighting users' diverse information needs regarding exploration and goal-oriented navigation (Sec.~\ref{sec:implication_varydetail}).
Differences in preferences were mainly attributed to personality traits, because participants who were ``Destination-Oriented'' (Tab.~\ref{tab:demographics1}), or were mostly concerned with reaching destinations, mentioned they did not enjoy the detailed explanation of the system and preferred short, concise information. 
For example, one early blinded participant mentioned that exploration did not interest him, as he had barely done it in his daily life (C\ref{P02Conditioned}). 
On the other hand, some participants enjoyed imagining the scenes conveyed by the system. 
Congenital users commented that the descriptions felt as if they were actually seeing the surroundings, while acquired users likened it to their recalled experiences when they could still see. 
Interestingly, those who particularly enjoyed the system and were ``Exploration-Inclined'' were all female, while the Intermediate group, who enjoyed exploration but wanted more control over the information provided, consisted mainly of male participants.
We note that ``Destination-Oriented'' users expressed dissatisfaction with the system because they felt the scene description capabilities of the MLLM did not meet their expectations for exploration. Therefore, if the system was improved and was able to convey more concrete information, they might express different opinions. 

In the main study, further differences regarding how users interacted with the system were observed. 
Firstly, we observed that participants adjusted the system's levels of description, demonstrating our design aligns with their needs, which were based on three types of preferences identified in the formative study (Sec~\ref{sec:description_level_analysis}). 
The variation in the portions used for each mode further underscores the need for configurable descriptions. 
Also, how they used the conversation mode varied. 
Three participants frequently asked questions to the system to gather information about their surroundings (Sec.~\ref{sec:activity_breakdown}), while P15 preferred having more manual control over the robot’s navigation. 
Meanwhile, P12 favored the auto mode, where the robot guided them with minimal intervention. 
These observations highlight the need to consider customizing to various dimensions of personal preference, from description details to user autonomy, for future development.


\subsection{Design Implications and Future Development Directions} 
\label{sec:design_implication}
Two key design implications were observed in our studies. 
First, allowing the users to control the level of detail in the scene descriptions emerged as one of the most important design requirements. 
The system may benefit from further \textit{personalization} by users verbally describing their personal information needs as in previous research~\cite{Kaniwa2024ChitChatGuide}. 
Second, participants expressed the need for audio-based recognition capabilities, especially in environments where sound is an integral part of the experience, such as museums (C\ref{P13Sound}). 
The ability to answer questions about sounds and potentially guide users to the sounds' sources would enhance their exploration experience.

On the development side, the primary challenge encountered throughout the two studies was the system’s inability to provide detailed information that participants required, particularly regarding the identification of POI-related objects, as described in \red{Sec.~\ref{sec:scene_description}.}
We attempted to address this by upgrading the robot's hardware, \ie, adding a 1080p resolution fisheye camera to a much higher position. 
Still, participants found the descriptions lacking in detail and conveyed information somewhat vague, as partially shown by the ratings of 4 from P11 and P13 to Q1 Tab.~\ref{tab:likert}. 
We deduce that this was because the captured images sometimes did not contain useful information, such as the names of certain objects, or because the MLLM failed to accurately identify the useful information.
As a possible improvement, the robot could utilize history images by selecting the image with the best view to generate descriptions. 
Also, the robot could utilize, other modalities, such as colored point clouds by fusing camera images with the LiDAR sensor to provide three-dimentional sensor details to MLLM~\cite{liu2024uni3d,xu2023pointllm}. 
In conclusion, the MLLM module is the bottleneck of our system's technological development.
Similar system development efforts in the future should allocate the most resources to tackling this technological challenge.
Still, the issue may be gradually solved as MLLM is the current core area actively developed by researchers.

Another significant challenge in development we encountered is the challenge of running map-less navigation algorithms in diverse novel environments, which requires extensive development. 
Incorporating vision modalities~\cite{chang2024goat}, which we did not use in this study, could potentially enhance the robot's navigation capabilities. 
Achieving this, however, demands human-level object and layout recognition and real-time processing speed, where further research is required.


\subsection{Limitation and Future Work}
We were unable to examine user preferences over the long term, as participants in our study interacted with the system only for a short duration (20-40 minutes in the formative study and 70 minutes in the main study). 
Only a small portion of the reliance on concise descriptions may be due to the study's design limiting participants' time to explore.
The time constraints may have led users to act on the cost-effective information acquisition.
However, if the system is used regularly, users may encounter more situations where they prefer to use the concise mode, as indicated by C\ref{P12NormalMode}. 
Also, their preferences might change as they become more adept at utilizing it as a tool to query information, which the MLLM is particularly proficient at.
Thus, future research should investigate the effects of long-term use of the system.

We conducted two studies in two indoor locations. 
To capture more diverse needs, future studies should also explore the system’s performance in more diverse environments. 
This may reveal various additional information needs. 
The usage of the wheeled robot, while beneficial in guiding blind users because it is silent~\cite {wang2022can}, remains a constraint when navigating stairs or uneven terrain. 
This limitation, however, could be alleviated through user collaboration, such as assisting the robot in getting onto elevators or slightly lifting the robot over small steps.
Thus, future research should investigate the system devices's capabilities in different environments, as well as how these robots can address physical limitations by interacting with users.
\red{
Finally, for the main study, we were unable to conduct it in crowd environments with bystanders potentially obstructing the cameras, because the primary study was conducted in the science museum outside of regular operational hours. 
Handling crowded environments with robots, even when prebuilt maps are used, remains a significant challenge in the field of robotics~\cite{wang2022group}. 
Therefore, in future work, we aim to address the usability limitations of our system in such scenarios by integrating novel algorithms designed to manage crowded environments~\cite{wang2022group}.
}

The MLLM often made mistakes or referred to non-existent objects, with these errors being particularly noticeable in its responses within the Q\&A functionality (Sec.~\ref{sec:error_analysis}). 
The most common misrecognitions involved either partially reading the text or confusing objects with similar-looking ones. 
However, the performance of the MLLM is not the primary focus of our research. 
To ensure users receive the most accurate information possible, we will continue updating the MLLM used in the system.
\red{Also, some of the image inputs provided to the MLLM may have been affected by motion blur, potentially leading to a degradation in the quality of the generated descriptions. 
This issue could be addressed by using cameras that are more resistant to motion blur or by implementing algorithms that detect motion blur and select alternative frames for processing.}

Recruitment was conducted through our institution's email list, which includes many participants from previous studies. 
We acknowledge that these participants may have exhibited a positive bias toward our study, as they had expectations regarding the development of the robot system.
\red{
Furthermore, we obtained valuable insights from five participants, and involving more participants might have provided additional perspectives. 
% However, the number of participants in each study, which was five, was informed by a previous study~\cite{nielsen1993mathematical} that models the relationship between the number of participants and the percentage of usability problems identified. 
% This study found that, in many cases, five participants would identify more than 75\% of usability problems. 
Given the difficulty of recruiting many blind participants, we chose to iterate the study with five participants in each study, rather than conducting a single study with a larger group.
}



\vspace{-0.7em}

\section{Conclusions}
In this paper, we propose a new distribution-aware divergence-based metric, DistFaiR, for amortized fairness measurement. We identify metrics under DistFaiR with the useful property that group unfairness is upper bounded by individual unfairness. We show that we can reduce individual and group unfairness under DistFaiR for different choices of divergence measures. We emphasize query polarity as a crucial yet overlooked aspect in fair-ranking literature, noting that neglecting polarity can result in fairwashing. We also empirically demonstrate fairwashing effects due to a lack of query polarity consideration and propose/evaluate a method to mitigate this effect. 

Our work has some limitations. For example, we assume a position bias model of attention. However, we note that this assumption can be relaxed to consider more complex user attention patterns under our framework, with some modifications made to the cumulative attention formulation. We also make normative assumptions that the distribution of attention should be close to that of relevance. However, a different link function may be more appropriate~\cite{saito2022fair}. Additionally, scores allotted to minority groups may be under-estimates of their true value~\cite{pierson2021algorithmic,krieg2022perceived} and may need to be pre-processed ~\cite{liao2023social}.  Importantly, there may not be purely technical fixes for operationalizing real-world fair ranking~\cite{gichoya2021equity}. Our approach, we believe, is a step towards reducing the scale of such issues.








\begin{acks}
We would like to thank all the participants in our user study.
We are also deeply thankful to Mori Building Co., Ltd. for providing the experimental location.
Finally, we thank all members of Miraikan, including Hironobu Takagi and Hiromi Kurokawa,  and
the Consortium for Advanced Assistive Mobility Platform for their support.
This work was supported by JSPS KAKENHI (JP23KJ2048).
\end{acks}


\bibliographystyle{ACM-Reference-Format}
\bibliography{main}

\appendix

\lstset{
  backgroundcolor=\color{gray!20}, % 20% gray background
  basicstyle=\ttfamily\footnotesize, % Monospaced font with smaller size
  breaklines=true,                 % Automatically break long lines
  frame=single,                    % Frame the listing
  framerule=0pt,                   % No frame border
  xleftmargin=5pt, xrightmargin=5pt % Add some margin around the text
}


\section{Appendix: Prompts to MLLM}
In this section, we list full prompts to MLLM and LLM, which were used in this paper.


\subsection{Prompt Used For Translating Native Language to English}
\label{appendix:translate}
As the research was conducted in a country where English is not spoken, we used the below prompt to translate any data obtained in the native language throughout the paper. 
Note that the authors manually refined the output to keep the nuances of the original language.
This prompt was also used to translate the prompt engineered in the native language, which was fed into the MLLM for generating scene descriptions.
\begin{lstlisting}
Please translate the given <Native Language> to English. Make sure to keep the nuances and context of the original text.
<Native Language>:  Text written in native Language
English:
\end{lstlisting}


\subsection{Prompt Used In The Formative Study}
\label{appendix:prompt_formative}
Below is the prompt used to generate descriptions in the formative study.
\begin{lstlisting}
# Instructions  
Please describe the image.  
The text you generate will be read directly to visually impaired individuals. Make sure your description is engaging so that visually impaired individuals can enjoy listening to it.  
To describe the image, you must follow the rules outlined below.  

## Rules you must strictly follow to comply with the instructions  

### Rules on what you should do  
1. Since visually impaired individuals will listen while walking, provide a description in one cohesive sentence. Please describe as many objects and their details as possible.  
2. Generate the description in 1 to 4 sentences in total.  
3. If necessary, first describe the overall layout or the general view of the location.  
4. After that, identify the objects located on the left, in front, and on the right of the image, and explain the information required to understand the scene.  
5. Always describe the scene in the following order: overall view, left side, front, right side.  
6. When describing, use a tone similar to a guide for the visually impaired, such as "On the right, there is..."  
7. If there is a store, make sure to include information about what the store offers (for example, the type of cuisine if it is a restaurant). Also, include a description of the store's atmosphere (e.g., bright, calm).  
8. Only describe objects that are clearly visible. Include descriptions of distinctive objects.  
9. Create a description that is enjoyable to listen to and allows the listener to learn about their surroundings.  

### Rules on what you should not do  
10. Avoid unnatural words for the listener, such as "the image" or "viewpoint."  
11. You do not need to include common and unremarkable objects (e.g., tables and chairs in a restaurant) in the description.  
12. If there is nothing to describe in a particular direction (e.g., there is nothing on the right), you do not need to mention that direction.  
13. Do not describe the floor, ceiling, shadows, distant unclear objects, or the brightness or darkness of the lighting.  

## Response Format  
If you generate a good description that follows the above rules, you will receive a tip.  
Please respond in JSON format.  
Include the image description under the "description" key.  
Start your response with ```json\n{ to indicate the beginning of the JSON.  

Here is an example of a response:  
```json  
{  
"description": "<description of the surroundings>",  
}  
```
\end{lstlisting}


\subsection{Prompts Used In The Formative Study}
\label{appendix:prompt_main}
This section provides prompts used in the main study.

\subsubsection{The Prompt for Generating Detailed Description}
Below is the prompt used to generate a detailed description.
\begin{lstlisting}
# Instructions  
Please describe the image.  
You are given three images that provide a view of your left, right, and front, as well as a view from a fisheye camera that captures the overall view from a high point of view.
The text you generate will be read directly to visually impaired individuals.  
When writing the description, please aim to make it appealing so that it creates an enjoyable experience for the listener.  
The most important thing is to provide detailed and specific information so that the listener can feel as if they are actually at the scene.  
Being specific means describing the category or name of objects, their condition, and the role they play.  
For example, a description like "circular wooden object" is not specific, but "a circular wooden table with YYY written on the nearby guide" is specific.  
Similarly, "iron exhibit" is vague, while "a tall, iron exhibit, possibly XXX" is specific.  
When describing the image, you must follow the rules below.

## Rules that must be followed to comply with the instructions  
1. The description must be something that a visually impaired person can listen to while walking. Provide a coherent description in one block of text. Explain as many objects and their details as possible.  
2. Keep the description to 3-4 sentences at most (120-240 characters).  
3. Use polite language (honorifics).  
4. Identify and describe objects located in the overall scene, to the left, front, and right that are necessary to understand the scene.  
5. Only describe clearly visible objects. Include distinctive objects in your description.  
6. Always describe the overall scene first, followed by objects on the left, in front, and then on the right.  
7. Strive to include the following information:
    - Details about the building's interior and decoration.
    - Information about the layout of the building (such as whether the front is open, where walls are, and the directions one can go).
    - Information on the surrounding brightness and the amount of light coming through windows.
    - Information about people in the surroundings, their actions, clothing colors, and whether they are staff or customers.
    - If it's a store, provide information on whether the entrance is open like a terrace and whether guide dogs can wait there.
    - Include information on visible stores or exhibits. Be sure to mention their category (e.g., the type of food if it's a restaurant, or what kind of place the exhibit is). If possible, include the name of the place. For exhibits, state whether they are interactive or for viewing only.
    - When describing objects, be specific (mention the category or name). For example, if there's a counter, specify if it looks like a cafe counter.
    - Mention people walking toward the front if there's a risk of collision.
    - Use numbers when explaining object positions (e.g., "5 meters to the right").
    - If there is a sign or guidepost, describe what it is and read out the text written on it.
    - Read out visible text.
    - Use adjectives like futuristic, stylish, modern, or classic to make the exploration more enjoyable and to help the listener visualize the scene.
8. Do not use unnatural words for the listener like "image," "viewpoint," or "overall."
9. If there's nothing to describe in a certain direction (e.g., nothing on the right side), do not describe that direction.  
10. Do not summarize or conclude with a description of the overall direction or scene when finishing the explanation.  
11. Do not describe anything not visible in the image. Do not lie or hallucinate details.

## Response Format  
If you follow the rules above, you will receive a tip.  
If you ignore the rules, you will be penalized and have to pay a fine.  
Please do your best to comply with these instructions.

Respond in JSON format.  
First, include the initial description of the image under the "initial_description" key.  
Next, include points for improvement under the "improve_thoughts" key.  
Finally, include the revised image description under the "description" key.  
Start your response with `json\n{`.  
Here is an example response:

```json
{
"initial_description": "<initial description>",
"improve_thoughts": "<points for improvement>",
"description": "<revised description>",
}
```

\end{lstlisting}

\subsubsection{The Prompt for Generating Balanced-Length Description}
Below is the prompt used to generate a balanced-length description.
\begin{lstlisting}
# Instructions  
Please describe the image.  
You are given three images that provide a view of your left, right, and front, as well as a view from a fisheye camera that captures the overall view from a high point of view.
The text you generate will be read directly to visually impaired individuals.  
Keep the description concise, but aim to make it appealing and enjoyable for the listener.  
The most important thing is to provide detailed and specific information so that the listener can feel as if they are actually at the scene.  
Being specific means describing the category or name of objects, their condition, and the role they play.  
For example, a description like "circular wooden object" is not specific, but "a circular wooden table with YYY written on the nearby guide" is specific.  
Similarly, "iron exhibit" is vague, while "a tall, iron exhibit, possibly XXX" is specific.  
When describing the image, you must follow the rules below.

## Rules that must be followed to comply with the instructions  
1. The description must be something that a visually impaired person can listen to while walking. Provide a coherent description in one block of text. Explain as many objects and their details as possible.  
2. Keep the description to 2-3 sentences at most (60-120 characters).  
3. Use polite language (honorifics).  
4. Identify and describe objects located to the left, front, and right that are necessary to understand the scene.  
5. Only describe clearly visible objects. Include distinctive objects in your description.  
6. Always describe objects in the following order: left, front, and right.  
7. Strive to include the following information:
    - If it's a store, provide information on whether the entrance is open like a terrace and whether guide dogs can wait there.
    - Include information on visible stores or exhibits. Be sure to mention their category (e.g., the type of food if it's a restaurant, or what kind of place the exhibit is). If possible, include the name of the place. For exhibits, state whether they are interactive or for viewing only.
    - When describing objects, be specific (mention the category or name). For example, if there's a counter, specify if it looks like a cafe counter.
    - Mention people walking toward the front if there's a risk of collision.
    - Use numbers when explaining object positions (e.g., "5 meters to the right...").
    - If there is a sign or guidepost, describe what it is and read out the text written on it.
    - Read out visible text.
8. Do not use unnatural words for the listener like "image," "viewpoint," or "overall."
9. If there's nothing to describe in a certain direction (e.g., nothing on the right side), do not describe that direction.  
10. Do not summarize or conclude with a description of the overall direction or scene when finishing the explanation.  
11. Do not describe objects if you cannot provide specific information about them.  
12. Do not include information about people in the surroundings unless there is a risk of collision.  
13. Do not include information about the amount of light or brightness in the surroundings.  
15. Do not use subjective adjectives like futuristic, stylish, modern, or classic.  
16. Do not describe anything not visible in the image. Do not lie or hallucinate details.

## Response Format  
If you follow the rules above, you will receive a tip.  
If you ignore the rules, you will be penalized and have to pay a fine.  
Please do your best to comply with these instructions.

Respond in JSON format.  
First, include the initial description of the image under the "initial_description" key.  
Next, include points for improvement under the "improve_thoughts" key.  
Finally, include the revised image description under the "description" key.  
Start your response with `json\n{`.  
Here is an example response:

```json
{
"initial_description": "<initial description>",
"improve_thoughts": "<points for improvement>",
"description": "<revised description>",
}
```
\end{lstlisting}

\subsubsection{The Prompt for Generating Concise Description}
Below is the prompt used to generate a concise description.
\begin{lstlisting}
# Instructions  
Please describe the image.  
You are given three images that provide a view of your left, right, and front, as well as a view from a fisheye camera that captures the overall view from a high point of view.
The text you generate will be read directly to visually impaired individuals.  
The description should be concise and minimal, allowing the listener to quickly understand their surroundings.  
Visually impaired individuals are listening to the image description to locate their destination.  
The most important thing is to provide detailed and specific information so that the listener can feel as if they are actually at the scene.  
Being specific means describing the category or name of objects, their condition, and the role they play.  
For example, a description like "circular wooden object" is not specific, but "a circular wooden table with YYY written on the nearby guide" is specific.  
Similarly, "iron exhibit" is vague, while "a tall, iron exhibit, possibly XXX" is specific.  
When describing the image, you must follow the rules below.

## Rules that must be followed to comply with the instructions  
1. The description must be something that a visually impaired person can listen to while walking. Provide a coherent description in one block of text.  
2. Keep the description to 1-2 sentences at most (0-60 characters).  
3. Use polite language (honorifics).  
4. Identify and describe objects located to the left, front, and right that are necessary to understand the scene.  
5. Only describe clearly visible objects. Include distinctive objects in your description.  
6. Always describe objects in the following order: left, front, and right.  
7. Strive to include the following information:
    - If it's a store, provide information on whether the entrance is open like a terrace and whether guide dogs can wait there.
    - Include information on visible stores or exhibits. Be sure to mention their category (e.g., the type of food if it's a restaurant, or what kind of place the exhibit is). If possible, include the name of the place. For exhibits, state whether they are interactive or for viewing only.
    - Use numbers when explaining object positions (e.g., "5 meters to the right...").
    - If there is a sign or guidepost, describe what it is and read out the text written on it.
    - Read out visible text.
8. Only convey specific information.
9. Keep the description short, direct, and concise.  
10. Do not use unnatural words for the listener like "image," "viewpoint," or "overall."
11. If there's nothing to describe in a certain direction (e.g., nothing on the right side), do not describe that direction.  
12. Do not summarize or conclude with a description of the overall direction or scene at the beginning or end of the explanation.  
13. Do not describe decorations. Simply convey what is there and provide specific information.  
14. To keep the length minimal, do not include subjective adjectives.  
15. Do not include unnecessary information that does not help the listener locate their destination (e.g., details about furniture such as chairs or tables).  
16. Do not describe objects if you cannot provide specific information about them.  
17. Do not include information about people in the surroundings unless there is a risk of collision.  
18. Do not include information about the amount of light or brightness in the surroundings.  
19. Do not describe anything not visible in the image. Do not lie or hallucinate details.

## Response Format  
If you follow the rules above, you will receive a tip.  
If you ignore the rules, you will be penalized and have to pay a fine.  
Please do your best to comply with these instructions.

Respond in JSON format.  
First, include the initial description of the image under the "initial_description" key.  
Next, include points for improvement under the "improve_thoughts" key.  
Finally, include the revised image description under the "description" key.  
Start your response with `json\n{`.  
Here is an example response:

```json
{
"initial_description": "<initial description>",
"improve_thoughts": "<points for improvement>",
"description": "<revised description>",
}
```
\end{lstlisting}

\end{document}
