

\section{RELATED WORK}
\label{sec:relatedwork}
In this section, we describe the previous works related to our proposal, which are divided into two parts. In Section~\ref{sec:relatedwork_exoplanet}, we present a review of approaches based on machine learning techniques for the detection of planetary transit signals. Section~\ref{sec:relatedwork_attention} provides an account of the approaches based on attention mechanisms applied in Astronomy.\par

\subsection{Exoplanet detection}
\label{sec:relatedwork_exoplanet}
Machine learning methods have achieved great performance for the automatic selection of exoplanet transit signals. One of the earliest applications of machine learning is a model named Autovetter \citep{MCcauliff}, which is a random forest (RF) model based on characteristics derived from Kepler pipeline statistics to classify exoplanet and false positive signals. Then, other studies emerged that also used supervised learning. \cite{mislis2016sidra} also used a RF, but unlike the work by \citet{MCcauliff}, they used simulated light curves and a box least square \citep[BLS;][]{kovacs2002box}-based periodogram to search for transiting exoplanets. \citet{thompson2015machine} proposed a k-nearest neighbors model for Kepler data to determine if a given signal has similarity to known transits. Unsupervised learning techniques were also applied, such as self-organizing maps (SOM), proposed \citet{armstrong2016transit}; which implements an architecture to segment similar light curves. In the same way, \citet{armstrong2018automatic} developed a combination of supervised and unsupervised learning, including RF and SOM models. In general, these approaches require a previous phase of feature engineering for each light curve. \par

%DL is a modern data-driven technology that automatically extracts characteristics, and that has been successful in classification problems from a variety of application domains. The architecture relies on several layers of NNs of simple interconnected units and uses layers to build increasingly complex and useful features by means of linear and non-linear transformation. This family of models is capable of generating increasingly high-level representations \citep{lecun2015deep}.

The application of DL for exoplanetary signal detection has evolved rapidly in recent years and has become very popular in planetary science.  \citet{pearson2018} and \citet{zucker2018shallow} developed CNN-based algorithms that learn from synthetic data to search for exoplanets. Perhaps one of the most successful applications of the DL models in transit detection was that of \citet{Shallue_2018}; who, in collaboration with Google, proposed a CNN named AstroNet that recognizes exoplanet signals in real data from Kepler. AstroNet uses the training set of labelled TCEs from the Autovetter planet candidate catalog of Q1–Q17 data release 24 (DR24) of the Kepler mission \citep{catanzarite2015autovetter}. AstroNet analyses the data in two views: a ``global view'', and ``local view'' \citep{Shallue_2018}. \par


% The global view shows the characteristics of the light curve over an orbital period, and a local view shows the moment at occurring the transit in detail

%different = space-based

Based on AstroNet, researchers have modified the original AstroNet model to rank candidates from different surveys, specifically for Kepler and TESS missions. \citet{ansdell2018scientific} developed a CNN trained on Kepler data, and included for the first time the information on the centroids, showing that the model improves performance considerably. Then, \citet{osborn2020rapid} and \citet{yu2019identifying} also included the centroids information, but in addition, \citet{osborn2020rapid} included information of the stellar and transit parameters. Finally, \citet{rao2021nigraha} proposed a pipeline that includes a new ``half-phase'' view of the transit signal. This half-phase view represents a transit view with a different time and phase. The purpose of this view is to recover any possible secondary eclipse (the object hiding behind the disk of the primary star).


%last pipeline applies a procedure after the prediction of the model to obtain new candidates, this process is carried out through a series of steps that include the evaluation with Discovery and Validation of Exoplanets (DAVE) \citet{kostov2019discovery} that was adapted for the TESS telescope.\par
%



\subsection{Attention mechanisms in astronomy}
\label{sec:relatedwork_attention}
Despite the remarkable success of attention mechanisms in sequential data, few papers have exploited their advantages in astronomy. In particular, there are no models based on attention mechanisms for detecting planets. Below we present a summary of the main applications of this modeling approach to astronomy, based on two points of view; performance and interpretability of the model.\par
%Attention mechanisms have not yet been explored in all sub-areas of astronomy. However, recent works show a successful application of the mechanism.
%performance

The application of attention mechanisms has shown improvements in the performance of some regression and classification tasks compared to previous approaches. One of the first implementations of the attention mechanism was to find gravitational lenses proposed by \citet{thuruthipilly2021finding}. They designed 21 self-attention-based encoder models, where each model was trained separately with 18,000 simulated images, demonstrating that the model based on the Transformer has a better performance and uses fewer trainable parameters compared to CNN. A novel application was proposed by \citet{lin2021galaxy} for the morphological classification of galaxies, who used an architecture derived from the Transformer, named Vision Transformer (VIT) \citep{dosovitskiy2020image}. \citet{lin2021galaxy} demonstrated competitive results compared to CNNs. Another application with successful results was proposed by \citet{zerveas2021transformer}; which first proposed a transformer-based framework for learning unsupervised representations of multivariate time series. Their methodology takes advantage of unlabeled data to train an encoder and extract dense vector representations of time series. Subsequently, they evaluate the model for regression and classification tasks, demonstrating better performance than other state-of-the-art supervised methods, even with data sets with limited samples.

%interpretation
Regarding the interpretability of the model, a recent contribution that analyses the attention maps was presented by \citet{bowles20212}, which explored the use of group-equivariant self-attention for radio astronomy classification. Compared to other approaches, this model analysed the attention maps of the predictions and showed that the mechanism extracts the brightest spots and jets of the radio source more clearly. This indicates that attention maps for prediction interpretation could help experts see patterns that the human eye often misses. \par

In the field of variable stars, \citet{allam2021paying} employed the mechanism for classifying multivariate time series in variable stars. And additionally, \citet{allam2021paying} showed that the activation weights are accommodated according to the variation in brightness of the star, achieving a more interpretable model. And finally, related to the TESS telescope, \citet{morvan2022don} proposed a model that removes the noise from the light curves through the distribution of attention weights. \citet{morvan2022don} showed that the use of the attention mechanism is excellent for removing noise and outliers in time series datasets compared with other approaches. In addition, the use of attention maps allowed them to show the representations learned from the model. \par

Recent attention mechanism approaches in astronomy demonstrate comparable results with earlier approaches, such as CNNs. At the same time, they offer interpretability of their results, which allows a post-prediction analysis. \par


\section{Introduction}
Exploration is a fundamental skill that allows one to gain familiarity with novel environments that blind people do not know. 
Sighted people explore by visually perceiving points of interest (POI) and navigating to desirable destinations. 
However, blind people face significant challenges in independently exploring new environments~\cite{Engel2020travelling,muller2022traveling}. 
They typically rely on sighted assistants, such as friends or family members, to help them navigate and describe their surroundings.
Unfortunately, these assistants are not always readily available, resulting in limited opportunities for blind people to explore independently.

\red{
Over the past recent years, various guide systems~\cite{bineeth2020blindsurvey, sulaiman2021analysis, manjari2020survey}, that are aimed for navigation~\cite{sato2019navcog3,li2016isana} or exploration~\cite{Kaniwa2024ChitChatGuide,kayukawa2023enhancing}}, have been developed to guide blind people and provide details about surrounding POIs in the environment. 
These systems typically rely on prebuilt maps and localization infrastructure 
\red{(\eg, Bluetooth Low Energy (BLE) beacons~\cite{sato2019navcog3,murata2018smartphone,kim2016navigating,chen2015blindnavi,InclusiveNavi} and ultrawide-bandwidth beacons~\cite{lu2021assistive})}
that are highly customized to the environments to continually update their current locations and offer turn-by-turn navigation guidance.
\red{
As access to the prebuilt maps also allows these systems to convey information about nearby POIs while navigating, some systems are specialized in assisting exploration activity~\cite{Kaniwa2024ChitChatGuide,kayukawa2023enhancing}.
}
However, only a limited number of \red{guide systems (\eg, InclusiveNavi~\cite{InclusiveNavi} and BlindSquare~\cite{BlindSquare})} are publicly \red{deployed} because configuring and maintaining prebuilt maps and localization infrastructure is expensive, and it is infeasible for them to be deployed in unseen environments. 
\red{
Several systems that do not require maps, as well as remote sighted assistance (RSA)~\cite{kamikubo2020support,Aira,BeMyEyes}, have been developed to guide blind people in various locations~\cite{kuribayashi2023pathfinder,Kuribayashi2022CorridorWalker,fallah2012user,lacey2000context}. 
However, these systems primarily focus on navigation to target destinations, not exploration, thus providing only navigation-related information to users (\eg,  intersections~\cite{Kuribayashi2022CorridorWalker,kuribayashi2023pathfinder} and signs~\cite{kuribayashi2023pathfinder}). 
These systems are also not independent from human assistance.
}
To promote social inclusion and equality for blind people, there is a need to develop a \textit{map-less} guide system that assists blind people in exploring diverse novel locations without relying on prebuilt maps or infrastructure.

\red{
To bridge the gaps and address the shortcomings of existing systems, we developed a system with the following characteristics as shown in Table~\ref{tab:relatedwork}: 1. Our system does not rely on prebuilt maps or preinstalled infrastructure. 2. Our system focuses on exploration. 3. Our system does not require supplementary assistance from humans. 4. Our system can automatically guide users physically during exploration. None of the prior systems possess the combination of all these characteristics.
Given that the design space for a map-less exploration guide robot remains underexplored, this work aims to investigate and establish the key components of such a system.
We begin by selecting a wheeled robot platform as the device. 
The decision to use a wheeled robot is based on its ability to autonomously guide blind users. 
It alleviates the challenge of navigation, which is cognitively demanding while learning about the surrounding environment.
Additionally, we equip the robot system with the ability to convey real-time information about the surrounding environment to users using natural language, accomplished through a multimodal large language model (MLLM~\cite{GPT4o}). 
}

Using our prototype system, we employed an iterative process with the direct involvement of target users to develop our system. 
In the formative study, the participants were asked to follow the robot, which was controlled in a Wizard-of-Oz fashion~\cite{riek2012wizard-of-oz}, along predetermined routes while listening to the environment descriptions.
The study revealed three groups of user preferences in the system's descriptions with respect to varying levels of details in the descriptive information received.
It also revealed requirements in certain functionalities, such as revisiting locations where the system had mentioned, specifying directions to proceed, and obtaining in-depth information through question-and-answering (Q\&A) functionality.

In the second stage, taking the lessons learned from the first study, we present \textit{WanderGuide}, a map-less exploration system for blind people (Fig.~\ref{fig:teaser}).
Taking into consideration the previously discovered three groups of user preferences, the system offers three modes for describing the surroundings: (1) Detailed description --- in-depth information with high granularity, (2) Balanced-Length description --- balanced level of information, and (3) Concise description --- minimal but essential details for obtaining quick awareness. 
We also implemented various new features based on the feedback received from the first study, which includes adopting a high-resolution fisheye camera for better perception of the surrounding environment, allowing users to verbally interact to query about the environment and set explored POIs to be navigation destinations, and allowing users to use directional buttons to control the robot for navigation towards the direction of interest. 
Our system is also fully integrated with the automatic mapping, localization, map-less navigation, and obstacle avoidance functions of the wheeled mobile robot.

Finally, we conducted a main user study with five blind participants, who were asked to freely explore two floors of the science museum.
All participants appreciated the experience of wandering freely without a fixed destination, and they expressed their desire to use the system to explore both familiar and unfamiliar areas. 
Participants also highlighted the need to incorporate recognition of auditory cues from the environment.
Additionally, differences in how they interacted with the system were observed: one frequently used buttons to guide the robot towards their areas of interest, one passively followed the robot, and others often asked questions. 
We also identified a limitation in the system's MLLM when conveying detailed information about the surroundings, such as identifying specific names of objects, which suggests the need for further development in how we input information into the MLLM for exploration purposes.

\red{
To the best of our knowledge, our work is the first to investigate the design space of a map-less system for blind people to explore independently.
To this end, we made the following contributions.}
\begin{itemize}
    \item \red{We formulated the requirements for the system through a formative study, such as the ability to adjust the level of description based on user preferences and to guide users to previously visited locations of interest, thereby enhancing the exploration experience.}
    \item \red{
     We developed a full stack map-less exploration system that consists of a waypoint detection algorithm and an MLLM-based perception interaction system on top of an existing navigation guide robot. Additionally, we integrated several functionalities based on the formative study that facilitates the exploration experience.
    }
    \item \red{We confirmed key design requirements, such as varying the level of descriptions based on user preferences through a usability study. We also gained further insights into users' interaction preferences and into design implications for improving the system, including better recognition of audio cues.}
\end{itemize}

\rrred{
The codes of the system are publicly available in the following link: \url{https://github.com/chestnutforestlabo/WanderGuide}.
}