\section{Conclusion}
Towards realizing a scalable map-less guide system that assists blind people in exploring, we developed WanderGuide, a robotic guide  system designed to provide real-time descriptions of surroundings and to offer conversation functionalities that allow users to specify their destinations or ask questions.
The formative study with ten blind participants revealed that there are three types of preferences over the levels of details of the descriptions generated by the system.
In a subsequent main study with five blind participants, all of them expressed appreciation for the experience of wandering freely without a fixed destination, as well as a desire to use the system for exploring both familiar and unfamiliar areas. 
The study further revealed that including audio recognition would be the immediate next step for developing our system. 
It also revealed that customizing to diverse user preferences is important and that MLLM is the key bottleneck of the technology development of our system.
We hope this research contributes to the potential deployment of robotic guide systems in general use cases, enabling blind users to explore independently.