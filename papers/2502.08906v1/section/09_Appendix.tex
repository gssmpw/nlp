\appendix

\lstset{
  backgroundcolor=\color{gray!20}, % 20% gray background
  basicstyle=\ttfamily\footnotesize, % Monospaced font with smaller size
  breaklines=true,                 % Automatically break long lines
  frame=single,                    % Frame the listing
  framerule=0pt,                   % No frame border
  xleftmargin=5pt, xrightmargin=5pt % Add some margin around the text
}


\section{Appendix: Prompts to MLLM}
In this section, we list full prompts to MLLM and LLM, which were used in this paper.


\subsection{Prompt Used For Translating Native Language to English}
\label{appendix:translate}
As the research was conducted in a country where English is not spoken, we used the below prompt to translate any data obtained in the native language throughout the paper. 
Note that the authors manually refined the output to keep the nuances of the original language.
This prompt was also used to translate the prompt engineered in the native language, which was fed into the MLLM for generating scene descriptions.
\begin{lstlisting}
Please translate the given <Native Language> to English. Make sure to keep the nuances and context of the original text.
<Native Language>:  Text written in native Language
English:
\end{lstlisting}


\subsection{Prompt Used In The Formative Study}
\label{appendix:prompt_formative}
Below is the prompt used to generate descriptions in the formative study.
\begin{lstlisting}
# Instructions  
Please describe the image.  
The text you generate will be read directly to visually impaired individuals. Make sure your description is engaging so that visually impaired individuals can enjoy listening to it.  
To describe the image, you must follow the rules outlined below.  

## Rules you must strictly follow to comply with the instructions  

### Rules on what you should do  
1. Since visually impaired individuals will listen while walking, provide a description in one cohesive sentence. Please describe as many objects and their details as possible.  
2. Generate the description in 1 to 4 sentences in total.  
3. If necessary, first describe the overall layout or the general view of the location.  
4. After that, identify the objects located on the left, in front, and on the right of the image, and explain the information required to understand the scene.  
5. Always describe the scene in the following order: overall view, left side, front, right side.  
6. When describing, use a tone similar to a guide for the visually impaired, such as "On the right, there is..."  
7. If there is a store, make sure to include information about what the store offers (for example, the type of cuisine if it is a restaurant). Also, include a description of the store's atmosphere (e.g., bright, calm).  
8. Only describe objects that are clearly visible. Include descriptions of distinctive objects.  
9. Create a description that is enjoyable to listen to and allows the listener to learn about their surroundings.  

### Rules on what you should not do  
10. Avoid unnatural words for the listener, such as "the image" or "viewpoint."  
11. You do not need to include common and unremarkable objects (e.g., tables and chairs in a restaurant) in the description.  
12. If there is nothing to describe in a particular direction (e.g., there is nothing on the right), you do not need to mention that direction.  
13. Do not describe the floor, ceiling, shadows, distant unclear objects, or the brightness or darkness of the lighting.  

## Response Format  
If you generate a good description that follows the above rules, you will receive a tip.  
Please respond in JSON format.  
Include the image description under the "description" key.  
Start your response with ```json\n{ to indicate the beginning of the JSON.  

Here is an example of a response:  
```json  
{  
"description": "<description of the surroundings>",  
}  
```
\end{lstlisting}


\subsection{Prompts Used In The Formative Study}
\label{appendix:prompt_main}
This section provides prompts used in the main study.

\subsubsection{The Prompt for Generating Detailed Description}
Below is the prompt used to generate a detailed description.
\begin{lstlisting}
# Instructions  
Please describe the image.  
You are given three images that provide a view of your left, right, and front, as well as a view from a fisheye camera that captures the overall view from a high point of view.
The text you generate will be read directly to visually impaired individuals.  
When writing the description, please aim to make it appealing so that it creates an enjoyable experience for the listener.  
The most important thing is to provide detailed and specific information so that the listener can feel as if they are actually at the scene.  
Being specific means describing the category or name of objects, their condition, and the role they play.  
For example, a description like "circular wooden object" is not specific, but "a circular wooden table with YYY written on the nearby guide" is specific.  
Similarly, "iron exhibit" is vague, while "a tall, iron exhibit, possibly XXX" is specific.  
When describing the image, you must follow the rules below.

## Rules that must be followed to comply with the instructions  
1. The description must be something that a visually impaired person can listen to while walking. Provide a coherent description in one block of text. Explain as many objects and their details as possible.  
2. Keep the description to 3-4 sentences at most (120-240 characters).  
3. Use polite language (honorifics).  
4. Identify and describe objects located in the overall scene, to the left, front, and right that are necessary to understand the scene.  
5. Only describe clearly visible objects. Include distinctive objects in your description.  
6. Always describe the overall scene first, followed by objects on the left, in front, and then on the right.  
7. Strive to include the following information:
    - Details about the building's interior and decoration.
    - Information about the layout of the building (such as whether the front is open, where walls are, and the directions one can go).
    - Information on the surrounding brightness and the amount of light coming through windows.
    - Information about people in the surroundings, their actions, clothing colors, and whether they are staff or customers.
    - If it's a store, provide information on whether the entrance is open like a terrace and whether guide dogs can wait there.
    - Include information on visible stores or exhibits. Be sure to mention their category (e.g., the type of food if it's a restaurant, or what kind of place the exhibit is). If possible, include the name of the place. For exhibits, state whether they are interactive or for viewing only.
    - When describing objects, be specific (mention the category or name). For example, if there's a counter, specify if it looks like a cafe counter.
    - Mention people walking toward the front if there's a risk of collision.
    - Use numbers when explaining object positions (e.g., "5 meters to the right").
    - If there is a sign or guidepost, describe what it is and read out the text written on it.
    - Read out visible text.
    - Use adjectives like futuristic, stylish, modern, or classic to make the exploration more enjoyable and to help the listener visualize the scene.
8. Do not use unnatural words for the listener like "image," "viewpoint," or "overall."
9. If there's nothing to describe in a certain direction (e.g., nothing on the right side), do not describe that direction.  
10. Do not summarize or conclude with a description of the overall direction or scene when finishing the explanation.  
11. Do not describe anything not visible in the image. Do not lie or hallucinate details.

## Response Format  
If you follow the rules above, you will receive a tip.  
If you ignore the rules, you will be penalized and have to pay a fine.  
Please do your best to comply with these instructions.

Respond in JSON format.  
First, include the initial description of the image under the "initial_description" key.  
Next, include points for improvement under the "improve_thoughts" key.  
Finally, include the revised image description under the "description" key.  
Start your response with `json\n{`.  
Here is an example response:

```json
{
"initial_description": "<initial description>",
"improve_thoughts": "<points for improvement>",
"description": "<revised description>",
}
```

\end{lstlisting}

\subsubsection{The Prompt for Generating Balanced-Length Description}
Below is the prompt used to generate a balanced-length description.
\begin{lstlisting}
# Instructions  
Please describe the image.  
You are given three images that provide a view of your left, right, and front, as well as a view from a fisheye camera that captures the overall view from a high point of view.
The text you generate will be read directly to visually impaired individuals.  
Keep the description concise, but aim to make it appealing and enjoyable for the listener.  
The most important thing is to provide detailed and specific information so that the listener can feel as if they are actually at the scene.  
Being specific means describing the category or name of objects, their condition, and the role they play.  
For example, a description like "circular wooden object" is not specific, but "a circular wooden table with YYY written on the nearby guide" is specific.  
Similarly, "iron exhibit" is vague, while "a tall, iron exhibit, possibly XXX" is specific.  
When describing the image, you must follow the rules below.

## Rules that must be followed to comply with the instructions  
1. The description must be something that a visually impaired person can listen to while walking. Provide a coherent description in one block of text. Explain as many objects and their details as possible.  
2. Keep the description to 2-3 sentences at most (60-120 characters).  
3. Use polite language (honorifics).  
4. Identify and describe objects located to the left, front, and right that are necessary to understand the scene.  
5. Only describe clearly visible objects. Include distinctive objects in your description.  
6. Always describe objects in the following order: left, front, and right.  
7. Strive to include the following information:
    - If it's a store, provide information on whether the entrance is open like a terrace and whether guide dogs can wait there.
    - Include information on visible stores or exhibits. Be sure to mention their category (e.g., the type of food if it's a restaurant, or what kind of place the exhibit is). If possible, include the name of the place. For exhibits, state whether they are interactive or for viewing only.
    - When describing objects, be specific (mention the category or name). For example, if there's a counter, specify if it looks like a cafe counter.
    - Mention people walking toward the front if there's a risk of collision.
    - Use numbers when explaining object positions (e.g., "5 meters to the right...").
    - If there is a sign or guidepost, describe what it is and read out the text written on it.
    - Read out visible text.
8. Do not use unnatural words for the listener like "image," "viewpoint," or "overall."
9. If there's nothing to describe in a certain direction (e.g., nothing on the right side), do not describe that direction.  
10. Do not summarize or conclude with a description of the overall direction or scene when finishing the explanation.  
11. Do not describe objects if you cannot provide specific information about them.  
12. Do not include information about people in the surroundings unless there is a risk of collision.  
13. Do not include information about the amount of light or brightness in the surroundings.  
15. Do not use subjective adjectives like futuristic, stylish, modern, or classic.  
16. Do not describe anything not visible in the image. Do not lie or hallucinate details.

## Response Format  
If you follow the rules above, you will receive a tip.  
If you ignore the rules, you will be penalized and have to pay a fine.  
Please do your best to comply with these instructions.

Respond in JSON format.  
First, include the initial description of the image under the "initial_description" key.  
Next, include points for improvement under the "improve_thoughts" key.  
Finally, include the revised image description under the "description" key.  
Start your response with `json\n{`.  
Here is an example response:

```json
{
"initial_description": "<initial description>",
"improve_thoughts": "<points for improvement>",
"description": "<revised description>",
}
```
\end{lstlisting}

\subsubsection{The Prompt for Generating Concise Description}
Below is the prompt used to generate a concise description.
\begin{lstlisting}
# Instructions  
Please describe the image.  
You are given three images that provide a view of your left, right, and front, as well as a view from a fisheye camera that captures the overall view from a high point of view.
The text you generate will be read directly to visually impaired individuals.  
The description should be concise and minimal, allowing the listener to quickly understand their surroundings.  
Visually impaired individuals are listening to the image description to locate their destination.  
The most important thing is to provide detailed and specific information so that the listener can feel as if they are actually at the scene.  
Being specific means describing the category or name of objects, their condition, and the role they play.  
For example, a description like "circular wooden object" is not specific, but "a circular wooden table with YYY written on the nearby guide" is specific.  
Similarly, "iron exhibit" is vague, while "a tall, iron exhibit, possibly XXX" is specific.  
When describing the image, you must follow the rules below.

## Rules that must be followed to comply with the instructions  
1. The description must be something that a visually impaired person can listen to while walking. Provide a coherent description in one block of text.  
2. Keep the description to 1-2 sentences at most (0-60 characters).  
3. Use polite language (honorifics).  
4. Identify and describe objects located to the left, front, and right that are necessary to understand the scene.  
5. Only describe clearly visible objects. Include distinctive objects in your description.  
6. Always describe objects in the following order: left, front, and right.  
7. Strive to include the following information:
    - If it's a store, provide information on whether the entrance is open like a terrace and whether guide dogs can wait there.
    - Include information on visible stores or exhibits. Be sure to mention their category (e.g., the type of food if it's a restaurant, or what kind of place the exhibit is). If possible, include the name of the place. For exhibits, state whether they are interactive or for viewing only.
    - Use numbers when explaining object positions (e.g., "5 meters to the right...").
    - If there is a sign or guidepost, describe what it is and read out the text written on it.
    - Read out visible text.
8. Only convey specific information.
9. Keep the description short, direct, and concise.  
10. Do not use unnatural words for the listener like "image," "viewpoint," or "overall."
11. If there's nothing to describe in a certain direction (e.g., nothing on the right side), do not describe that direction.  
12. Do not summarize or conclude with a description of the overall direction or scene at the beginning or end of the explanation.  
13. Do not describe decorations. Simply convey what is there and provide specific information.  
14. To keep the length minimal, do not include subjective adjectives.  
15. Do not include unnecessary information that does not help the listener locate their destination (e.g., details about furniture such as chairs or tables).  
16. Do not describe objects if you cannot provide specific information about them.  
17. Do not include information about people in the surroundings unless there is a risk of collision.  
18. Do not include information about the amount of light or brightness in the surroundings.  
19. Do not describe anything not visible in the image. Do not lie or hallucinate details.

## Response Format  
If you follow the rules above, you will receive a tip.  
If you ignore the rules, you will be penalized and have to pay a fine.  
Please do your best to comply with these instructions.

Respond in JSON format.  
First, include the initial description of the image under the "initial_description" key.  
Next, include points for improvement under the "improve_thoughts" key.  
Finally, include the revised image description under the "description" key.  
Start your response with `json\n{`.  
Here is an example response:

```json
{
"initial_description": "<initial description>",
"improve_thoughts": "<points for improvement>",
"description": "<revised description>",
}
```
\end{lstlisting}