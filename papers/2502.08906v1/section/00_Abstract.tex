\begin{abstract}
Blind people have limited opportunities to explore an environment based on their interests. While existing navigation systems could provide them with surrounding information while navigating, they have limited scalability as they require preparing prebuilt maps. Thus, to develop a map-less robot that assists blind people in exploring, we first conducted a study with ten blind participants at a shopping mall and science museum to investigate the requirements of the system, which revealed the need for three levels of detail to describe the surroundings based on users' preferences. Then, we developed WanderGuide, with functionalities that allow users to adjust the level of detail in descriptions and verbally interact with the system to ask questions about the environment or to go to points of interest. The study with five blind participants revealed that WanderGuide could provide blind people with the enjoyable experience of wandering around without a specific destination in their minds.
\end{abstract}