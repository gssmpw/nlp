\section{Related Work}
\label{relatedWork}

\subsection{Collaborative writing studies}

Collaborative writing has been a topic of interest since the 1980s~\cite{engelbart1984collaboration, fish1988quilt, baecker1993user}. Early research focused on awareness and coordination during collaborative writing~\cite{dourish1992awareness}, common writing tasks, and the number of collaborators~\cite{kim2001reviewing}. The rise of online collaborative writing tools like Google Docs, Microsoft Word, and Overleaf~\cite{overleaf} has made collaborative writing a common practice. For instance, Olson et al.~\cite{olson2017people} analyzed collaborative writing patterns in 96 college assignments in Google Docs and found that balanced participation and leadership would result in higher writing quality~\cite{olson2017people}. Researchers also explored various aspects of collaboration, such as impression management~\cite{birnholtz2013write}, reluctance to write closely~\cite{wang2017users},
preference over edits with explanations~\cite{park2023importance}, differences of tasks across writing stages~\cite{sarrafzadeh2021characterizing}, and territorial behaviors~\cite{larsen2019territorial}.

Few studies focus on authors' off-document writing-related activities during collaborative writing, for example, navigating multiple applications (e.g., Google Docs and Adobe InDesign)~\cite{larsen2020collaborative} and coordinating writing tasks on Wikipedia discussion pages~\cite{schneider2011understanding}. 
Collaborated with communication researchers, we focus on writing-related behaviors, including off-document behaviors like using a translator and browsing the internet, aiming to provide a new perspective on collaborative writing analysis. 

%Olson et al.~\cite{olson2017people} analyzed collaborative writing patterns in 96 college class assignments on Google Docs and found that balanced participation and the presence of leadership will result in higher writing quality. Birnholtz et al.~\cite{birnholtz2013write, birnholtz2012tracking} studied group maintenance, impression management, and relationship-focused behaviors. Park et al.~\cite{park2023importance} conducted a collaborative writing experiment of 20 pairs of authors and found that participants prefer co-writers who provide rationales for edits. Sarrafzadeh et al. ~\cite{sarrafzadeh2021characterizing} analyzed user interaction logs (e.g., interacting with paragraphs, find-and-replace, etc.) on a collaborative writing platform at different temporal stages. Larsen-Ledet et al. studied territorial functioning in collaborative writing and emphasized the importance of understanding interpersonal dynamics and the temporal aspects of writing, such as timing and turn-taking. Larsen-Ledet et al.~\cite{larsen2019territorial} studied how territorial functioning in collaborative writing, and emphasized the importance of understanding interpersonal dynamics and the temporal aspects of writing, such as timing and turn-taking.

\subsection{Text visual analytics}
Several visual analytics approaches have been designed to analyze the evolution of documents in collaborative writing. Itero~\cite{turkay2018itero} is a revision history analytics tool based on Google Docs that visualizes character insertion patterns and user contributions. History Flow~\cite{viegas2004studying} and DocuViz~\cite{wang2015docuviz} encode each author's contribution as a colored vertical line, with the height of the line proportional to the content length. The flow-like visualization reveals the cooperation and conflict among co-authors by connecting the same line across different versions. Graphs are also widely used, where authors are represented as nodes, and edges could be disagreement~\cite{flock2015whovis} or revert actions~\cite{kittur2007he}. Time Curves~\cite{bach2015time} is a timeline visualization based on points' similarity, which could visualize different document versions. Other visualizations include branch-based visualizations~\cite{perez2018organic}, revision maps~\cite{southavilay2013analysis} and color-coded words by authorship~\cite{flock2015towards, torres2019visualizing}.
Compared to text visual analytics approaches, \name{} focuses on sequences of writing-related behaviors.


\subsection{Non-native speakers vs. native speakers}
Compared to native speakers (NS), non-native speakers (NNS) usually produce shorter and less complicated content and have difficulty transferring writing strategies from their mother tongue~\cite{chenoweth2001fluency, wolfersberger2003l1, uzawa1989writing}. Though NNS need more help in the expression aspects~\cite{severino2009comparison}, they may still contribute to the ideation aspect~\cite{goffman1981forms}. Cheng et al.~\cite{cheng2013non} found in a case study that NS students had more power in collaborative writing at the beginning, but the NNS student developed academic literacy along the way, and overall the group writing experience has improved. NNS' writing could also be improved by receiving direct edit feedback at early versions~\cite{karim2020revision, yang2021supervisor}, or exposure to well-written model text by NS~\cite{kang2020using}. Compared to these previous case studies, we have a larger collaborative writing dataset of NS-NNS with video-recorded author behaviors, poised to reveal more patterns beyond anecdotal evidence.

\subsection{Event sequence analysis}
There are numerous methods to analyze event sequences. Besides \textit{visualizing} sequences, we categorize analysis methods based on tasks: \textit{comparing}, \textit{clustering} and \textit{summarizing}.

\bpstart{Visualizing event sequences}
The most straightforward visualization design for event sequences is to arrange the events on a timeline~\cite{plaisant1996lifelines, krstajic2011cloudlines}. When the number of sequences is large, flow-based visualizations could show the trend of bundled sequences. For example, Sankey diagrams represent each event as a node, the length of the node and the thickness of the links between nodes encode event frequencies~\cite{wongsuphasawat2011outflow, perer2014frequence}. Tree-based visualizations encode the frequency of events as the thickness of edges~\cite{hu2016visualizing, liu2017coreflow}. Like tree-based visualizations, icicle plots encode events as stacked rectangles, ordered from top to bottom, usually colored by event categories~\cite{wongsuphasawat2011lifeflow, liu2016patterns}. When subsequences are highly repetitive, matrix-based visualizations could show the transition trend clearly~\cite{perer2012matrixflow, zhao2015matrixwave}.

\bpstart{Comparing event sequences}
Multiple tools focus on comparison. CoCo compares two patient cohorts via statistical analysis with built-in metrics~\cite{malik2015cohort} distilled from domain expertise~\cite{monroe2012exploring}. TipoVis compares event sequences of social and communicative behaviors by overlaying two sequences~\cite{han2015visual}. COQUITO~\cite{krause2015supporting} assists users in defining cohorts with temporal constraints and comparing sequences by overlapped branches.
Directly linking event sub-sequences for comparison is also common~\cite{qi2019stbins, meyer2009mizbee, zhao2023contextwing}.  

\bpstart{Clustering event sequences}
Several interactive tools are designed for clustering sequences. For example, Wang et al.~\cite{wang2016unsupervised} built an unsupervised interactive clustering system to analyze large-scale clickstream data. EventThread~\cite{guo2017eventthread} clusters event sequences by latent stage categories. Gotz et al.~\cite{gotz2019visual} group event sequences by dynamic hierarchical dimension aggregation. Sequen-C~\cite{magallanes2021sequen} adopts an align-score-simplify strategy to cluster sequences. VASABI~\cite{nguyen2019vasabi} clusters user profiles by topic modeling and uses multi-dimensional distributions to characterize each cluster.

\bpstart{Summarizing event sequences}
Numerous methods have been developed to find an overview of a cluster of sequences. Sequence Synopsis~\cite{chen2017sequence} constructs a high-level overview of sequences by balancing the minimum description length (MDL) principle and the information loss. CoreFlow~\cite{liu2017coreflow} extracts branching patterns in temporal event sequences. Frequence~\cite{perer2014frequence} is a visual analytics tool built on a frequent pattern mining algorithm that handles multiple levels of details and concurrency. SentenTree~\cite{hu2016visualizing} summarizes unstructured social media text in a tree structure. 

\name{} is also equipped with visualizing, comparing, clustering, and summarizing features. Compared to existing approaches, we focus on interpretation and trust by including multiple clustering methods and displaying uncertainties, supporting multi-level granularity of sequences, and leveraging large language models to generate more intuitive descriptions~\cite{brown2020language}. 



%%%% 2-related-work.tex ends here %%%%






%%%% 3-background.tex starts here %%%%