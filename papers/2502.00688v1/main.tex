\def\isarxiv{1} %%% for icml submission version, we comment this line

\ifdefined\isarxiv

\documentclass[11pt]{article}
\usepackage[numbers]{natbib}

\else

%\documentclass[nohyperref]{article}
\documentclass{article}
%\usepackage{icml2022}
% \usepackage{neurips_2022}
\fi



\ifdefined\isarxiv

\usepackage{amsmath}
\usepackage{amsthm}
\usepackage{amssymb}
\usepackage{algorithm}
\usepackage{subfig}
\usepackage{algpseudocode}
\usepackage{graphicx}
\usepackage{grffile}
\usepackage{wrapfig,epsfig}
\usepackage{url}
\usepackage{xcolor}
\usepackage{epstopdf}


\usepackage{bbm}
\usepackage{dsfont}

\usepackage{booktabs} % for professional tables

 
\allowdisplaybreaks

\else

% \usepackage{amsmath}
% \usepackage{amsthm}
% \usepackage{amssymb}
\usepackage{algorithm}
\usepackage{subfig}
\usepackage{algpseudocode}
% \usepackage{graphicx}
\usepackage{grffile}
\usepackage{wrapfig,epsfig}
\usepackage{url}
\usepackage{xcolor}
\usepackage{epstopdf}


\usepackage{bbm}
\usepackage{dsfont}


% Recommended, but optional, packages for figures and better typesetting:
\usepackage{microtype}
\usepackage{graphicx}
% \usepackage{subfigure}
\usepackage{booktabs} % for professional tables

% hyperref makes hyperlinks in the resulting PDF.
% If your build breaks (sometimes temporarily if a hyperlink spans a page)
% please comment out the following usepackage line and replace
% \usepackage{icml2025} with \usepackage[nohyperref]{icml2025} above.
\usepackage{hyperref}


% Attempt to make hyperref and algorithmic work together better:
\newcommand{\theHalgorithm}{\arabic{algorithm}}

% Use the following line for the initial blind version submitted for review:
\usepackage{icml2025}

% If accepted, instead use the following line for the camera-ready submission:
% \usepackage{icml2025}

% \usepackage{hyperref}

% For theorems and such
\usepackage{amsmath}
\usepackage{amssymb}
\usepackage{mathtools}
\usepackage{amsthm}


\fi


 

\ifdefined\isarxiv

\let\C\relax
\usepackage{tikz}
\usepackage{hyperref}  %%% arxiv don't allow this.
\hypersetup{colorlinks=true,citecolor=blue,linkcolor=blue} %%% Zhao : maybe we should comment this in submission.
\usetikzlibrary{arrows}
\usepackage[margin=1in]{geometry}

\else

\usepackage{microtype}
\usepackage{hyperref}
\definecolor{mydarkblue}{rgb}{0,0.08,0.45}
\hypersetup{colorlinks=true, citecolor=mydarkblue,linkcolor=mydarkblue}
 

\fi
 
\graphicspath{{./figs/}}

\theoremstyle{plain}
\newtheorem{theorem}{Theorem}[section]
\newtheorem{lemma}[theorem]{Lemma}
\newtheorem{definition}[theorem]{Definition}
\newtheorem{notation}[theorem]{Notation}
%\newtheorem{proof}[theorem]{Proof}
\newtheorem{proposition}[theorem]{Proposition}
\newtheorem{corollary}[theorem]{Corollary}
\newtheorem{conjecture}[theorem]{Conjecture}
\newtheorem{assumption}[theorem]{Assumption}
\newtheorem{observation}[theorem]{Observation}
\newtheorem{fact}[theorem]{Fact}
\newtheorem{remark}[theorem]{Remark}
\newtheorem{claim}[theorem]{Claim}
\newtheorem{example}[theorem]{Example}
\newtheorem{problem}[theorem]{Problem}
\newtheorem{open}[theorem]{Open Problem}
\newtheorem{property}[theorem]{Property}
\newtheorem{hypothesis}[theorem]{Hypothesis}

\newcommand{\wh}{\widehat}
\newcommand{\wt}{\widetilde}
\newcommand{\ov}{\overline}
\newcommand{\N}{\mathcal{N}}
\newcommand{\R}{\mathbb{R}}
\newcommand{\F}{\mathcal{F}}
\newcommand{\G}{\mathcal{G}}
\newcommand{\True}{\mathrm{true}}
\newcommand{\Noisy}{\mathrm{noisy}}
\newcommand{\Clean}{\mathrm{clean}}
\newcommand{\Est}{\mathrm{est}}
\newcommand{\RHS}{\mathrm{RHS}}
\newcommand{\LHS}{\mathrm{LHS}}
\renewcommand{\d}{\mathrm{d}}
\renewcommand{\i}{\mathbf{i}}
\renewcommand{\tilde}{\wt}
\renewcommand{\hat}{\wh}
\newcommand{\Tmat}{{\cal T}_{\mathrm{mat}}}

\DeclareMathOperator*{\E}{{\mathbb{E}}}
\DeclareMathOperator*{\var}{\mathrm{Var}}
\DeclareMathOperator*{\Z}{\mathbb{Z}}
\DeclareMathOperator*{\C}{\mathbb{C}}
\DeclareMathOperator*{\D}{\mathcal{D}}
\DeclareMathOperator*{\median}{median}
\DeclareMathOperator*{\mean}{mean}
\DeclareMathOperator{\OPT}{OPT}
\DeclareMathOperator{\supp}{supp}
\DeclareMathOperator{\poly}{poly}

\DeclareMathOperator{\nnz}{nnz}
\DeclareMathOperator{\sparsity}{sparsity}
\DeclareMathOperator{\rank}{rank}
\DeclareMathOperator{\diag}{diag}
\DeclareMathOperator{\dist}{dist}
\DeclareMathOperator{\cost}{cost}
\DeclareMathOperator{\vect}{vec}
\DeclareMathOperator{\tr}{tr}
\DeclareMathOperator{\dis}{dis}
\DeclareMathOperator{\cts}{cts}



\makeatletter
\newcommand*{\RN}[1]{\expandafter\@slowromancap\romannumeral #1@}
\makeatother
% \newcommand{\Zhao}[1]{{\color{red}[Zhao: #1]}}
% \newcommand{\Mingda}[1]{{\color{purple}[Mingda: #1]}}
% \newcommand{\Zhizhou}[1]{{\color{violet}[Zhizhou: #1]}}
% \newcommand{\Bo}[1]{{\color{blue}[Bo: #1]}} 
% \newcommand{\Xiaoyu}[1]{{\color{orange}[Xiaoyu: #1]}} 
% \newcommand{\Zhenmei}[1]{{\color{blue}[Zhenmei: #1]}} 
% \newcommand{\InernNameB}[1]{{\color{blue}[InternNameB: #1]}} %%%Change to intern name


\usepackage{lineno}
\def\linenumberfont{\normalfont\small}



\begin{document}

\ifdefined\isarxiv

\date{}


\title{High-Order Matching for One-Step Shortcut Diffusion Models}
\author{
Bo Chen\thanks{\texttt{
bc7b@mtmail.mtsu.edu}. Middle Tennessee State University.}
\and
Chengyue Gong\thanks{\texttt{ cygong17@utexas.edu}. The University of Texas at Austin.}
\and
Xiaoyu Li\thanks{\texttt{
xiaoyu.li2@student.unsw.edu.au}. University of New South Wales.}
\and
Yingyu Liang\thanks{\texttt{
yingyul@hku.hk}. The University of Hong Kong. \texttt{
yliang@cs.wisc.edu}. University of Wisconsin-Madison.} 
\and
Zhizhou Sha\thanks{\texttt{
shazz20@mails.tsinghua.edu.cn}. Tsinghua University.}
\and
Zhenmei Shi\thanks{\texttt{
zhmeishi@cs.wisc.edu}. University of Wisconsin-Madison.}
\and
Zhao Song\thanks{\texttt{ magic.linuxkde@gmail.com}. The Simons Institute for the Theory of Computing at UC Berkeley.}
\and
Mingda Wan\thanks{\texttt{
dylan.r.mathison@gmail.com}. Anhui University.}
}




\else

% \title{Intern Project} 
% \maketitle 
% \iffalse
% \icmltitlerunning{Second-order Shortcut Models}
%\linenumbers

\twocolumn[

\icmltitle{High-Order Matching for One-Step Shortcut Diffusion Models}
% It is OKAY to include author information, even for blind
% submissions: the style file will automatically remove it for you
% unless you've provided the [accepted] option to the icml2019
% package.

% List of affiliations: The first argument should be a (short)
% identifier you will use later to specify author affiliations
% Academic affiliations should list Department, University, City, Region, Country
% Industry affiliations should list Company, City, Region, Country

% You can specify symbols, otherwise they are numbered in order.
% Ideally, you should not use this facility. Affiliations will be numbered
% in order of appearance and this is the preferred way.
\icmlsetsymbol{equal}{*}

\begin{icmlauthorlist}
\icmlauthor{Aeiau Zzzz}{equal,to}
\icmlauthor{Bauiu C.~Yyyy}{equal,to,goo}
\icmlauthor{Cieua Vvvvv}{goo}
\icmlauthor{Iaesut Saoeu}{ed}
\icmlauthor{Fiuea Rrrr}{to}
\icmlauthor{Tateu H.~Yasehe}{ed,to,goo}
\icmlauthor{Aaoeu Iasoh}{goo}
\icmlauthor{Buiui Eueu}{ed}
\icmlauthor{Aeuia Zzzz}{ed}
\icmlauthor{Bieea C.~Yyyy}{to,goo}
\icmlauthor{Teoau Xxxx}{ed}\label{eq:335_2}
\icmlauthor{Eee Pppp}{ed}
\end{icmlauthorlist}

\icmlaffiliation{to}{Department of Computation, University of Torontoland, Torontoland, Canada}
\icmlaffiliation{goo}{Googol ShallowMind, New London, Michigan, USA}
\icmlaffiliation{ed}{School of Computation, University of Edenborrow, Edenborrow, United Kingdom}

\icmlcorrespondingauthor{Cieua Vvvvv}{c.vvvvv@googol.com}
\icmlcorrespondingauthor{Eee Pppp}{ep@eden.co.uk}

% You may provide any keywords that you
% find helpful for describing your paper; these are used to populate
% the "keywords" metadata in the PDF but will not be shown in the document
\icmlkeywords{Machine Learning, ICML}

\vskip 0.3in
]

\printAffiliationsAndNotice{\icmlEqualContribution} 
% \fi
\fi





\ifdefined\isarxiv
\begin{titlepage}
  \maketitle
  \begin{abstract}
\begin{abstract}


The choice of representation for geographic location significantly impacts the accuracy of models for a broad range of geospatial tasks, including fine-grained species classification, population density estimation, and biome classification. Recent works like SatCLIP and GeoCLIP learn such representations by contrastively aligning geolocation with co-located images. While these methods work exceptionally well, in this paper, we posit that the current training strategies fail to fully capture the important visual features. We provide an information theoretic perspective on why the resulting embeddings from these methods discard crucial visual information that is important for many downstream tasks. To solve this problem, we propose a novel retrieval-augmented strategy called RANGE. We build our method on the intuition that the visual features of a location can be estimated by combining the visual features from multiple similar-looking locations. We evaluate our method across a wide variety of tasks. Our results show that RANGE outperforms the existing state-of-the-art models with significant margins in most tasks. We show gains of up to 13.1\% on classification tasks and 0.145 $R^2$ on regression tasks. All our code and models will be made available at: \href{https://github.com/mvrl/RANGE}{https://github.com/mvrl/RANGE}.

\end{abstract}



  \end{abstract}
  \thispagestyle{empty}
\end{titlepage}

{\hypersetup{linkcolor=black}
\tableofcontents
}
\newpage

\else

\begin{abstract}
\begin{abstract}


The choice of representation for geographic location significantly impacts the accuracy of models for a broad range of geospatial tasks, including fine-grained species classification, population density estimation, and biome classification. Recent works like SatCLIP and GeoCLIP learn such representations by contrastively aligning geolocation with co-located images. While these methods work exceptionally well, in this paper, we posit that the current training strategies fail to fully capture the important visual features. We provide an information theoretic perspective on why the resulting embeddings from these methods discard crucial visual information that is important for many downstream tasks. To solve this problem, we propose a novel retrieval-augmented strategy called RANGE. We build our method on the intuition that the visual features of a location can be estimated by combining the visual features from multiple similar-looking locations. We evaluate our method across a wide variety of tasks. Our results show that RANGE outperforms the existing state-of-the-art models with significant margins in most tasks. We show gains of up to 13.1\% on classification tasks and 0.145 $R^2$ on regression tasks. All our code and models will be made available at: \href{https://github.com/mvrl/RANGE}{https://github.com/mvrl/RANGE}.

\end{abstract}


\end{abstract}

\fi


\section{Introduction}
Backdoor attacks pose a concealed yet profound security risk to machine learning (ML) models, for which the adversaries can inject a stealth backdoor into the model during training, enabling them to illicitly control the model's output upon encountering predefined inputs. These attacks can even occur without the knowledge of developers or end-users, thereby undermining the trust in ML systems. As ML becomes more deeply embedded in critical sectors like finance, healthcare, and autonomous driving \citep{he2016deep, liu2020computing, tournier2019mrtrix3, adjabi2020past}, the potential damage from backdoor attacks grows, underscoring the emergency for developing robust defense mechanisms against backdoor attacks.

To address the threat of backdoor attacks, researchers have developed a variety of strategies \cite{liu2018fine,wu2021adversarial,wang2019neural,zeng2022adversarial,zhu2023neural,Zhu_2023_ICCV, wei2024shared,wei2024d3}, aimed at purifying backdoors within victim models. These methods are designed to integrate with current deployment workflows seamlessly and have demonstrated significant success in mitigating the effects of backdoor triggers \cite{wubackdoorbench, wu2023defenses, wu2024backdoorbench,dunnett2024countering}.  However, most state-of-the-art (SOTA) backdoor purification methods operate under the assumption that a small clean dataset, often referred to as \textbf{auxiliary dataset}, is available for purification. Such an assumption poses practical challenges, especially in scenarios where data is scarce. To tackle this challenge, efforts have been made to reduce the size of the required auxiliary dataset~\cite{chai2022oneshot,li2023reconstructive, Zhu_2023_ICCV} and even explore dataset-free purification techniques~\cite{zheng2022data,hong2023revisiting,lin2024fusing}. Although these approaches offer some improvements, recent evaluations \cite{dunnett2024countering, wu2024backdoorbench} continue to highlight the importance of sufficient auxiliary data for achieving robust defenses against backdoor attacks.

While significant progress has been made in reducing the size of auxiliary datasets, an equally critical yet underexplored question remains: \emph{how does the nature of the auxiliary dataset affect purification effectiveness?} In  real-world  applications, auxiliary datasets can vary widely, encompassing in-distribution data, synthetic data, or external data from different sources. Understanding how each type of auxiliary dataset influences the purification effectiveness is vital for selecting or constructing the most suitable auxiliary dataset and the corresponding technique. For instance, when multiple datasets are available, understanding how different datasets contribute to purification can guide defenders in selecting or crafting the most appropriate dataset. Conversely, when only limited auxiliary data is accessible, knowing which purification technique works best under those constraints is critical. Therefore, there is an urgent need for a thorough investigation into the impact of auxiliary datasets on purification effectiveness to guide defenders in  enhancing the security of ML systems. 

In this paper, we systematically investigate the critical role of auxiliary datasets in backdoor purification, aiming to bridge the gap between idealized and practical purification scenarios.  Specifically, we first construct a diverse set of auxiliary datasets to emulate real-world conditions, as summarized in Table~\ref{overall}. These datasets include in-distribution data, synthetic data, and external data from other sources. Through an evaluation of SOTA backdoor purification methods across these datasets, we uncover several critical insights: \textbf{1)} In-distribution datasets, particularly those carefully filtered from the original training data of the victim model, effectively preserve the model’s utility for its intended tasks but may fall short in eliminating backdoors. \textbf{2)} Incorporating OOD datasets can help the model forget backdoors but also bring the risk of forgetting critical learned knowledge, significantly degrading its overall performance. Building on these findings, we propose Guided Input Calibration (GIC), a novel technique that enhances backdoor purification by adaptively transforming auxiliary data to better align with the victim model’s learned representations. By leveraging the victim model itself to guide this transformation, GIC optimizes the purification process, striking a balance between preserving model utility and mitigating backdoor threats. Extensive experiments demonstrate that GIC significantly improves the effectiveness of backdoor purification across diverse auxiliary datasets, providing a practical and robust defense solution.

Our main contributions are threefold:
\textbf{1) Impact analysis of auxiliary datasets:} We take the \textbf{first step}  in systematically investigating how different types of auxiliary datasets influence backdoor purification effectiveness. Our findings provide novel insights and serve as a foundation for future research on optimizing dataset selection and construction for enhanced backdoor defense.
%
\textbf{2) Compilation and evaluation of diverse auxiliary datasets:}  We have compiled and rigorously evaluated a diverse set of auxiliary datasets using SOTA purification methods, making our datasets and code publicly available to facilitate and support future research on practical backdoor defense strategies.
%
\textbf{3) Introduction of GIC:} We introduce GIC, the \textbf{first} dedicated solution designed to align auxiliary datasets with the model’s learned representations, significantly enhancing backdoor mitigation across various dataset types. Our approach sets a new benchmark for practical and effective backdoor defense.


 %%% Section 1. Introduction
\section{Related Work}

\subsection{Large 3D Reconstruction Models}
Recently, generalized feed-forward models for 3D reconstruction from sparse input views have garnered considerable attention due to their applicability in heavily under-constrained scenarios. The Large Reconstruction Model (LRM)~\cite{hong2023lrm} uses a transformer-based encoder-decoder pipeline to infer a NeRF reconstruction from just a single image. Newer iterations have shifted the focus towards generating 3D Gaussian representations from four input images~\cite{tang2025lgm, xu2024grm, zhang2025gslrm, charatan2024pixelsplat, chen2025mvsplat, liu2025mvsgaussian}, showing remarkable novel view synthesis results. The paradigm of transformer-based sparse 3D reconstruction has also successfully been applied to lifting monocular videos to 4D~\cite{ren2024l4gm}. \\
Yet, none of the existing works in the domain have studied the use-case of inferring \textit{animatable} 3D representations from sparse input images, which is the focus of our work. To this end, we build on top of the Large Gaussian Reconstruction Model (GRM)~\cite{xu2024grm}.

\subsection{3D-aware Portrait Animation}
A different line of work focuses on animating portraits in a 3D-aware manner.
MegaPortraits~\cite{drobyshev2022megaportraits} builds a 3D Volume given a source and driving image, and renders the animated source actor via orthographic projection with subsequent 2D neural rendering.
3D morphable models (3DMMs)~\cite{blanz19993dmm} are extensively used to obtain more interpretable control over the portrait animation. For example, StyleRig~\cite{tewari2020stylerig} demonstrates how a 3DMM can be used to control the data generated from a pre-trained StyleGAN~\cite{karras2019stylegan} network. ROME~\cite{khakhulin2022rome} predicts vertex offsets and texture of a FLAME~\cite{li2017flame} mesh from the input image.
A TriPlane representation is inferred and animated via FLAME~\cite{li2017flame} in multiple methods like Portrait4D~\cite{deng2024portrait4d}, Portrait4D-v2~\cite{deng2024portrait4dv2}, and GPAvatar~\cite{chu2024gpavatar}.
Others, such as VOODOO 3D~\cite{tran2024voodoo3d} and VOODOO XP~\cite{tran2024voodooxp}, learn their own expression encoder to drive the source person in a more detailed manner. \\
All of the aforementioned methods require nothing more than a single image of a person to animate it. This allows them to train on large monocular video datasets to infer a very generic motion prior that even translates to paintings or cartoon characters. However, due to their task formulation, these methods mostly focus on image synthesis from a frontal camera, often trading 3D consistency for better image quality by using 2D screen-space neural renderers. In contrast, our work aims to produce a truthful and complete 3D avatar representation from the input images that can be viewed from any angle.  

\subsection{Photo-realistic 3D Face Models}
The increasing availability of large-scale multi-view face datasets~\cite{kirschstein2023nersemble, ava256, pan2024renderme360, yang2020facescape} has enabled building photo-realistic 3D face models that learn a detailed prior over both geometry and appearance of human faces. HeadNeRF~\cite{hong2022headnerf} conditions a Neural Radiance Field (NeRF)~\cite{mildenhall2021nerf} on identity, expression, albedo, and illumination codes. VRMM~\cite{yang2024vrmm} builds a high-quality and relightable 3D face model using volumetric primitives~\cite{lombardi2021mvp}. One2Avatar~\cite{yu2024one2avatar} extends a 3DMM by anchoring a radiance field to its surface. More recently, GPHM~\cite{xu2025gphm} and HeadGAP~\cite{zheng2024headgap} have adopted 3D Gaussians to build a photo-realistic 3D face model. \\
Photo-realistic 3D face models learn a powerful prior over human facial appearance and geometry, which can be fitted to a single or multiple images of a person, effectively inferring a 3D head avatar. However, the fitting procedure itself is non-trivial and often requires expensive test-time optimization, impeding casual use-cases on consumer-grade devices. While this limitation may be circumvented by learning a generalized encoder that maps images into the 3D face model's latent space, another fundamental limitation remains. Even with more multi-view face datasets being published, the number of available training subjects rarely exceeds the thousands, making it hard to truly learn the full distibution of human facial appearance. Instead, our approach avoids generalizing over the identity axis by conditioning on some images of a person, and only generalizes over the expression axis for which plenty of data is available. 

A similar motivation has inspired recent work on codec avatars where a generalized network infers an animatable 3D representation given a registered mesh of a person~\cite{cao2022authentic, li2024uravatar}.
The resulting avatars exhibit excellent quality at the cost of several minutes of video capture per subject and expensive test-time optimization.
For example, URAvatar~\cite{li2024uravatar} finetunes their network on the given video recording for 3 hours on 8 A100 GPUs, making inference on consumer-grade devices impossible. In contrast, our approach directly regresses the final 3D head avatar from just four input images without the need for expensive test-time fine-tuning.


\section{Preliminary}

This section establishes the notations and theoretical foundations for the subsequent analysis. 
Section~\ref{sec:notation} provides a comprehensive list of the primary notations adopted in this work. Section~\ref{sec:pre_flow_matching} elaborates on the flow-matching framework, extending it to the second-order case, with critical definitions underscored. 

\subsection{Notations}\label{sec:notation}

We use $\Pr[\cdot]$ to denote the probability. We use $\E[\cdot]$ to denote the expectation. We use $\var[\cdot]$ to denote the variance.
We use $\|x\|_p$ to denote the $\ell_p$ norm of a vector $x \in \R^n$, i.e. $\|x\|_1 := \sum_{i=1}^n |x_i|$, $\|x\|_2 := (\sum_{i=1}^n x_i^2)^{1/2}$, and $\|x\|_{\infty} := \max_{i \in [n]} |x_i|$. 
We use $f(x) = O(g(x))$ or $f(x) \lesssim g(x)$ to denote that $f(x) \leq C\cdot g(x)$ for some constant $C>0$.
We use $\N(0,I)$ to denote the standard Gaussian distribution.

\subsection{Shortcut model}\label{sec:pre_flow_matching}

Next, we describe the general framework of flow matching and its second-order rectification. These concepts form the basis for our proposed method, as they integrate first and second-order information for trajectory estimation.

\begin{fact}\label{fac:one_second_order}
Let a field $x_t$ be defined as 
\begin{align*}
x_t = \alpha_t x_0 + \beta_t x_1,
\end{align*}
where $\alpha_t$ and $\beta_t$ are functions of $t$, and $x_0, x_1$ are constants. Then, the first-order gradient $\dot{x_t}$ and the second-order gradient $\ddot{x_t}$ can be manually calculated as
\begin{align*}
\dot{x}_t &= \dot{\alpha_t} x_0 + \dot{\beta_t} x_1, \\
\ddot{x}_t &= \ddot{\alpha_t} x_0 + \ddot{\beta_t} x_1.
\end{align*}
\end{fact}

In practice, one often samples $(x_0, x_1)$ from $(\mu_0, \pi_0)$ and parameterizes $x_t$ (e.g., interpolation) at intermediate times to build a training objective that matches the velocity field to the true time derivative $\dot{x}_t$.

\begin{definition}[Shortcut models, implicit definition from page 3 on~\cite{fhla24}]
\label{def:shortcut_models}

Let $\Delta t = 1 / 128$. 
Let $x_t$ be current field. 
Let $t \in \mathbb{N}$ denote time step. 
Let $u_1( x_t, t, d )$ be the network to be trained. 
Let $d \in ( 1 / 128, 1 / 64,\dots, 1 / 2, 1 )$ denote step size. 
Then, we define Shortcut model compute next field $x_{t + d}$ as follow: 
\begin{align*}
x_{t + d} = 
\begin{cases}
x_t + u_1( x_t, t, d ) d & \mathrm{if } d \geq 1 / 128, \\
x_t + u_1( x_t, t, 0 ) \Delta t & \mathrm{if } d < 1 / 128.
\end{cases}
\end{align*}
\end{definition}


\section{Methodology} \label{sec:methodology}

When training a flow-based model, such as Shortcut model, using only the first-order term as the training loss has several limitations compared to incorporating high-order losses. (1) Firstly, relying solely on the first-order term results in a less accurate approximation of the true dynamics, as it captures only the linear component and misses important nonlinear aspects that higher-order terms can represent. This can lead to slower convergence, as the model must implicitly learn complex dynamics without explicit guidance from higher-order terms. (2) Additionally, while the first-order approach reduces model complexity and the risk of overfitting, it may also limit the model's ability to generalize effectively to unseen data, particularly when the underlying dynamics are highly nonlinear. (3) In contrast, including higher-order terms enhances the model's capacity to capture intricate patterns, improving both accuracy and generalization, albeit at the cost of increased computational complexity and potential overfitting risks.

Then, we introducing our HOMO (High-Order Matching for One-step Shortcut diffusion model). The intuition behind this design is to leverage high-order dynamics to achieve a more accurate and stable approximation of the field evolution. By incorporating higher-order losses, we aim to capture the nonlinearities and complex interactions that are often present in real-world systems. This approach not only improves the fidelity of the model but also enhances its ability to generalize across different scenarios.

\begin{definition}[HOMO Inference]
\label{def:HOMO_inference}
Let $\Delta t = 1 / 128$. 
Let $x_t$ be the current field. 
Let $t \in \mathbb{N}$ denote the time step. 
Let $u_{1,\theta_1}( \cdot )$ and $u_{2,\theta_2}( \cdot )$ denote the HOMO models to be trained. 
Let $d \in (0, 1 / 128, 1 / 64,\dots, 1 / 2, 1 )$ denote the step size. 
Then, we define the HOMO computation of the next field $x_{t + d}$ as follows: 
\begin{align*}
x_{t + d} = 
\begin{cases}
x_t + d \cdot u_1( x_t, t, d ) 
+ \frac{d^2}{2} \\
\qquad \cdot u_2(u_1 ( x_t, t, d), x_t, t, d ) & \text{if } d \geq 1 / 128, \\
x_t + \Delta t \cdot u_1( x_t, t, 0 )
+ \frac{(\Delta t)^2}{2} \\
\qquad \cdot u_2(u_1 ( x_t, t, 0), x_t, t, 0 ) & \text{if } d < 1 / 128.
\end{cases}
\end{align*}
\end{definition}

The self-consistency target is to ensure that the model's predictions are consistent across different time steps. This is crucial for maintaining the stability and accuracy of the model over long-term predictions. 

\begin{definition}[HOMO Self-Consistency Target]
\label{def:2nd_self_consistency_target}
Let $u_{1,\theta_1}$ be the networks to be trained.
Let $x_t$ be the current field and $x_{t+d}$ be defined in Definition~\ref{def:HOMO_inference}.
Let $t \in \mathbb{N}$ denote the time step. 
Let $d \in (0, 1 / 128, 1 / 64,\dots, 1 / 2, 1 )$ denote the step size.
Then, we define the Self-Consistency target as follows: 
\begin{align*}
    \dot{x}_t^{\mathrm{target}} = & ~ u_{1,\theta_1} ( x_t, t, d ) / 2 + u_{1,\theta_1}( x_{t + d}, t, d ) / 2 
\end{align*}
\end{definition}

The second-order HOMO loss is designed to optimize the model by minimizing the discrepancy between the predicted and true velocities and accelerations. This loss function ensures that the model not only captures the immediate dynamics but also the underlying trends and changes in the system. 

\begin{definition}[Second-order HOMO Loss] 
\label{def:HOMO_loss}
Let $x_t$ be the current field. 
Let $t \in \mathbb{N}$ denote the time step. 
Let $\dot{x}_t^{\mathrm{target}}$ be defined by Definition~\ref{def:2nd_self_consistency_target}.
Let $u_{1,\theta_1}( \cdot )$ and $u_{2,\theta_2}( \cdot )$ denote the HOMO models to be trained. 
Let $d \in (0, 1 / 128, 1 / 64,\dots, 1 / 2, 1 )$ denote the step size. 
Let $\dot{x}_t^\True$ and $\ddot{x}_t^\True$ be the observed (or numerically approximated) true velocity and acceleration. 
Let $\dot{x}_t^{\mathrm{pred}} := u_{1,\theta_1}(x_t, t, 2 d)$ denote the model prediction of the first-order term.
Then, we define the HOMO Loss as follows: 
\begin{align*}
L_{(\theta_1,\theta_2)} = & ~ \E[\ell_{2,1,\theta_1}(x_t, \dot{x}_t^{\True})] + \E[\ell_{2,2,\theta_2, \theta_1}(x_t, \ddot{x}_t^{\True})] \\
& ~ +
\E[ \| u_{1,\theta_1}(x_t, t, 2 d)- \dot{x}_t^{\mathrm{target}}\|^2]
\end{align*}

We define 
\begin{align*}
    \ell_{2,1,\theta_1}(x_t, \dot{x}_t^{\True}) := &~ \|  u_{1,\theta_1}(x_t, t, 2 d) - \dot{x}_t^{\True} \|^2, \\
    \ell_{2,2,\theta_2, \theta_1}(x_t, \ddot{x}_t^{\True}) := &~ \| u_{2,\theta_2}  (\dot{x}_t^{\mathrm{pred}}, x_t, t, 2 d) - \ddot{x}_t^{\True} \|^2 \\
    \ell_{\mathrm{selfc}}(x_t, \dot{x}_t^{\mathrm{target}}) := &~ \| u_{1,\theta_1}(x_t, t, 2 d)- \dot{x}_t^{\mathrm{target}}\|^2
\end{align*}
and 
\begin{align*}
    \ell_{(\theta_1, \theta_2)}(x_t, x_t^{\True}) := &~ \ell_{2,1,\theta_1}(x_t, \dot{x}_t^{\True})+\ell_{2,2,\theta_2, \theta_1}(x_t, \ddot{x}_t^{\True}) \\ + &~ \ell_{\mathrm{selfc}}(x_t, \dot{x}_t^{\mathrm{target}}).
\end{align*}
\end{definition}

\begin{remark} [Simple notations] \label{rem:simplicity_notations}
For simplicity, we denote first-order matching as M1, which implies that HOMO is optimized solely by the first-order loss $\ell_{2,1,\theta_1}(x_t, \dot{x}_t^{\True})$. Second-order matching is denoted as M2, where HOMO is optimized only by the second-order loss $\ell_{2,2,\theta_2, \theta_1}(x_t, \ddot{x}_t^{\True})$. We refer to HOMO optimized solely by the self-consistency loss as SC, denoted by $\ell_{\mathrm{selfc}}(x_t, \dot{x}_t^{\mathrm{target}})$. Combinations of M1, M2, and SC are used to indicate HOMO optimized by corresponding combinations of loss terms. For example, (M1 + M2) denotes HOMO optimized by both first-order and second-order terms, while (M1 + M2 + SC) represents HOMO optimized by the first-order, second-order, and self-consistency terms.
\end{remark}






\section{Main Results}\label{sec:main_result}


In Section~\ref{sec:mr_privacy}, we will provide the privacy of our algorithm. Then, we will examine the utility implications of our algorithm applying a random response mechanism. 
In Section~\ref{sec:main_result:utility}, we introduce the utility guarantees of our algorithm.
In Section~\ref{sec:main_result:running_complexity}, we demonstrate that DPBloomfilter does not import the running complexity burden to the standard Bloom filter.

\subsection{Privacy for DPBloomfilter}\label{sec:mr_privacy}

Algorithm~\ref{alg:init} illustrates the application of the random response mechanism to the standard Bloom filter, thereby accomplishing differential privacy. In detail, once the Bloom filter is initialized, each bit in the $m$-bit array is independently toggled with a probability of $\frac{1}{\epsilon_0 + 1}$. Our algorithm will ensure that modifications to any element within the dataset are protected to a degree, as the DPBloomfilter maintains the privacy of the altered element. Then, we present the Theorem demonstrating that our algorithm is $(\epsilon,\delta)$-DP.

\begin{theorem}[Privacy for Query, informal version of Theorem~\ref{thm:query_privacy:formal}]\label{thm:query_privacy:informal}
Let $N := F_W^{-1}(1 - \delta)$ and $\epsilon_0 = \epsilon / N$.
Then, we can show,
the output of \textsc{Query} procedure of Algorithm~\ref{alg:init} achieves $(\epsilon, \delta)$-DP. 
\end{theorem}

Theorem~\ref{thm:query_privacy:informal} shows that our DPBloomfilter in Algorithm~\ref{alg:init} is $(\epsilon, \delta)$-DP. 
Our main technique leverages the single-bit random response technique to enhance the privacy properties of the traditional Bloom filter by composition rule (Lemma~\ref{lem:pre_com_lem}). 

\subsection{Utility for DPBloomfilter}\label{sec:main_result:utility}

Despite the introduction of privacy-preserving mechanisms, our algorithm still ensures that the utility of the Bloom Filter remains acceptable. This is achieved through careful calibration of the Random Response technique parameters, balancing the need for privacy with the requirement for accurate set membership queries. 
Here, we present the theorem for the entire utility loss between the output of our algorithm and ground truth.

\begin{theorem}[Accuracy (compare DPBloom with true-answer) for Query, informal version of Theorem~\ref{thm:dpbloom_true_accuracy:formal}]\label{thm:dpbloom_true_accuracy:informal}
If the following conditions hold
\begin{itemize}
    \item Let $z \in \{0,1\}$ denote the true answer for whether $x \in A$. 
    \item Let $\wh{z} \in \{0,1\}$ denote the answer for whether $x \in A$ output by Bloom Filter.
    \item Let $\alpha: = \Pr[ z = 0 ] \in [0,1]$, $t := e^{\epsilon_0} / (e^{\epsilon_0} + 1)$, and  $\delta_{\mathrm{err}} > 0$.
\end{itemize}
Then, we can show 
\begin{align*}
\Pr[ \wt{z} = z ] \geq \delta_{\mathrm{err}}\cdot\alpha\cdot(1-t-t^{k}) + \alpha \cdot t.
\end{align*}
\end{theorem}

Theorem~\ref{thm:dpbloom_true_accuracy:informal} shows that when most queries are not in $A$, the above theorem can ensure that the utility of DPBloomfilter has a good guarantee. Namely, in such cases, answers from DPBloomfilter are correct with high probability. 


\subsection{Running Complexity of DPBloomfilter}\label{sec:main_result:running_complexity}

Now, we introduce the running complexity for the DPBloomfilter in the following theorem. 

\begin{theorem} [Running complexity of DPBloomfilter] \label{thm:running_complexity}
Let $\mathcal{T}_h$ denote the time of evaluation of function $h$ at any point. 
Then, for the DPBloomfilter (Algorithm~\ref{alg:init}) we have
\begin{itemize}
    \item The running complexity for the initialization procedure is $O(|A| \cdot k \cdot \mathcal{T}_h + m)$.
    \item The running complexity $O(k\cdot \mathcal{T}_h)$ for a single query. 
\end{itemize}
\end{theorem}

\begin{proof}
It can be proved by combining Lemma~\ref{lem:init_time} and \ref{lem:query_time}. 
\end{proof}

Our Theorem~\ref{thm:running_complexity} shows that DPBloomfilter not only addresses the critical need to protect the privacy of elements stored with Bloom filter but also ensures that the data structure's utility remains acceptable, with minimal impact on its computational efficiency.
By keeping the running time within the same order of magnitude as the standard Bloom filter, our approach is practical for real-world applications requiring fast and scalable set operations. 


% \input{5_technical_overview}
\section{Evaluations}
\label{sec:experiment}

In this section, we demonstrate that \sassha can indeed improve upon existing second-order methods available for standard deep learning tasks.
We also show that \sassha performs competitively to the first-order baseline methods.
Specifically, \sassha is compared to AdaHessian \citep{adahessian}, Sophia-H \citep{sophia}, Shampoo \cite{gupta2018shampoo}, SGD, AdamW \citep{loshchilov2018decoupled}, and SAM \citep{sam} on a diverse set of both vision and language tasks.
We emphasize that we perform an \emph{extensive} hyperparameter search to rigorously tune all optimizers and ensure fair comparisons.
We provide the details of experiment settings to reproduce our results in \cref{app:hypersearch}.
The code to reproduce all results reported in this work is made available for download at \url{https://github.com/LOG-postech/Sassha}.

\subsection{Image Classification}
\begin{table*}[t!]
    \vspace{-0.5em}
    \centering
    \caption{Image classification results of various optimization methods in terms of final validation accuracy (mean$\pm$std).
    \sassha consistently outperforms the other methods for all workloads.
    * means \emph{omitted} due to excessive computational requirements.}
    
    \vskip 0.1in
    \resizebox{0.8\linewidth}{!}{
        \begin{tabular}{clcccccc}
        \toprule
         & 
         & \multicolumn{2}{c}{CIFAR-10} 
         & \multicolumn{2}{c}{CIFAR-100} 
         & \multicolumn{2}{c}{ImageNet} \\
         \cmidrule(l{3pt}r{3pt}){3-4} \cmidrule(l{3pt}r{3pt}){5-6} \cmidrule(l{3pt}r{3pt}){7-8}
         \multicolumn{1}{c}{ Category }
         & \multicolumn{1}{c}{ Method }
         & \multicolumn{1}{c}{ ResNet-20 } 
         & \multicolumn{1}{c}{ ResNet-32 } 
         & \multicolumn{1}{c}{ ResNet-32 }  
         & \multicolumn{1}{c}{ WRN-28-10} 
         & \multicolumn{1}{c}{ ResNet-50 } 
         & \multicolumn{1}{c}{ ViT-s-32} \\ \midrule

        
       \multirow{4}{*}{First-order}  
       & SGD       & 
         $ 92.03 _{ \textcolor{black!60}{\pm 0.32} } $    &
         $ 92.69 _{\textcolor{black!60}{\pm 0.06} }  $    &
         $ 69.32 _{\textcolor{black!60}{\pm 0.19} }  $    &
         $ 80.06 _{\textcolor{black!60}{\pm 0.15} }  $    &
         $ 75.58 _{\textcolor{black!60}{\pm 0.05} }  $    &
         $ 62.90 _{\textcolor{black!60}{\pm 0.36} }  $   \\

        & AdamW      & 
        $ 92.04 _{\textcolor{black!60}{\pm 0.11} }  $     &
        $ 92.42 _{\textcolor{black!60}{\pm 0.13} }  $     &
        $ 68.78 _{\textcolor{black!60}{\pm 0.22} }  $     &
        $ 79.09 _{\textcolor{black!60}{\pm 0.35} }  $     &
        $ 75.38 _{\textcolor{black!60}{\pm 0.08} }  $     &
        $ 66.46 _{\textcolor{black!60}{\pm 0.15} }  $    \\
        
        & SAM $_{\text{SGD}}$  &
        $ 92.85 _{\textcolor{black!60}{\pm 0.07} }  $    &
        $ 93.89 _{\textcolor{black!60}{\pm 0.13} }  $    &
        $ 71.99 _{\textcolor{black!60}{\pm 0.20} }  $    &
        $ 83.14 _{\textcolor{black!60}{\pm 0.13} }  $    &
        $ 76.36 _{\textcolor{black!60}{\pm 0.16} }  $    &
        $ 64.54 _{\textcolor{black!60}{\pm 0.63} }  $    \\
        
        & SAM $_{\text{AdamW}}$  &
        $ 92.77 _{\textcolor{black!60}{\pm 0.29} }  $    &
        $ 93.45 _{\textcolor{black!60}{\pm 0.24} }  $    &
        $ 71.15 _{\textcolor{black!60}{\pm 0.37} }  $    &
        $ 82.88 _{\textcolor{black!60}{\pm 0.31} }  $    &
        $ 76.35 _{\textcolor{black!60}{\pm 0.16} }  $    &
        $ 68.31 _{\textcolor{black!60}{\pm 0.17} }  $    \\

        \midrule
        
        \multirow{4}{*}{Second-order} &
        AdaHessian &
        $ 92.00 _{\textcolor{black!60}{\pm 0.17} } $  &
        $ 92.48 _{\textcolor{black!60}{\pm 0.15} } $  &
        $ 68.06 _{\textcolor{black!60}{\pm 0.22} } $  &
        $ 76.92 _{\textcolor{black!60}{\pm 0.26} } $  &
        $ 73.64 _{\textcolor{black!60}{\pm 0.16} } $  &
        $ 66.42 _{\textcolor{black!60}{\pm 0.23} } $  \\
        
        & Sophia-H   & 
        $ 91.81 _{\textcolor{black!60}{\pm 0.27} } $  &
        $ 91.99 _{\textcolor{black!60}{\pm 0.08} } $  &
        $ 67.76 _{\textcolor{black!60}{\pm 0.37} } $  & 
        $ 79.35 _{\textcolor{black!60}{\pm 0.24} } $  & 
        $ 72.06 _{\textcolor{black!60}{\pm 0.49} } $  &
        $ 62.44 _{\textcolor{black!60}{\pm 0.36} } $  \\
        
        & Shampoo    & 
        $ 88.55 _ {\textcolor{black!60}{\pm 0.83}}$  &
        $ 90.23 _{\textcolor{black!60}{\pm 0.24}} $  &
        $ 64.08 _{\textcolor{black!60}{\pm 0.46}} $  &
        $ 74.06 _{\textcolor{black!60}{\pm 1.28}} $  &
        $*$                                          &
        $*$  \\
        
        \cmidrule(l{3pt}r{3pt}){2-8}
        
        \rowcolor{green!20} &
        \sassha    &
        $ \textbf{92.98} _{\textcolor{black!60}{\pm 0.05} }  $ &
        $ \textbf{94.09} _{\textcolor{black!60}{\pm 0.24} }  $ &
        $ \textbf{72.14} _{\textcolor{black!60}{\pm 0.16} }  $ & 
        $ \textbf{83.54} _{\textcolor{black!60}{\pm 0.08} }  $ &
        $ \textbf{76.43} _{\textcolor{black!60}{\pm 0.18} }  $ &
        $ \textbf{69.20} _{\textcolor{black!60}{\pm 0.30} }  $ \\
        
        \bottomrule
        \end{tabular}
    }
    \vskip 0.1in
    \label{tab:im_cls_results}
\end{table*}

\begin{figure*}[t!]
    \vspace{-0.5em}
    \centering
    \resizebox{0.8\linewidth}{!}{
    \includegraphics[width=0.325\linewidth]{figures/validation/Res32-CIFAR10-Acc.pdf}
    \includegraphics[width=0.325\linewidth]{figures/validation/WRN28-CIFAR100-Acc.pdf}
    \includegraphics[width=0.325\linewidth]{figures/validation/Res50-ImageNet-Acc.pdf}
    }
    \vspace{-0.5em}
    \caption{
    Validation accuracy curves along the training trajectory.
    We also provide loss curves in \cref{app:valloss}.
    }
    \label{fig:im_cls_results}
    \vspace{-0.7em}
\end{figure*}

\begin{table*}[ht!]
    \centering
    \caption{
    Language finetuning and pertraining results for various optimizers. For finetuning, \sassha achieves better results than AdamW and AdaHessian and compares competitively with Sophia-H. For pretraining, \sassha achieves the lowest perplexity among all optimizers.
    }
    \vskip 0.1in
    \resizebox{\linewidth}{!}{
        \begin{tabular}{lc}
            \toprule
             & \multicolumn{1}{c}{$\textbf{Pretrain} / $ GPT1-mini} \\
             \cmidrule(l{3pt}r{3pt}){2-2}
             & Wikitext-2 \\
             & \texttt{Perplexity}\\
            \midrule
            
            AdamW & $ 175.06 $ \\
            SAM $_{\text{AdamW}}$ & $ 158.06 $ \\
            AdaHessian & $ 407.69 $ \\
            Sophia-H & $ 157.60 $ \\
            
            \midrule 
            
            \rowcolor{green!20}
            \sassha &
            $ \textbf{122.40} $ \\
            
            \bottomrule
        \end{tabular}
        
        \begin{tabular}{|ccccccc}
            \toprule
                         \multicolumn{7}{|c}{ \textbf{Finetune} /  SqeezeBERT } \\
                         \cmidrule(l{3pt}r{3pt}){1-7}
                         SST-2 &  MRPC & STS-B & QQP & MNLI & QNLI & RTE \\
             \texttt{Acc} &  \texttt{Acc / F1}  & \texttt{S/P corr.} & \texttt{F1 / Acc} & \texttt{mat/m.mat} &  \texttt{Acc} &  \texttt{Acc} \\
            \midrule
            
            %AdamW         & 
            $ 90.29 _{\textcolor{black!60}{\pm 0.52}} $ 
            & $ 84.56 _{ \textcolor{black!60}{\pm 0.25} } $ / $ 88.99 _{\textcolor{black!60}{\pm 0.11}} $ 
            & $ 88.34 _{\textcolor{black!60}{\pm 0.15}} $ / $ 88.48 _{\textcolor{black!60}{\pm 0.20}} $ 
            & $ 89.92 _{\textcolor{black!60}{\pm 0.05}} $ / $ 86.58 _{\textcolor{black!60}{\pm 0.11}} $ 
            & $ 81.22 _{\textcolor{black!60}{\pm 0.07}} $ / $ 82.26 _{\textcolor{black!60}{\pm 0.05}} $ 
            & $ 89.93 _{\textcolor{black!60}{\pm 0.14}} $ 
            & $ 68.95 _{\textcolor{black!60}{\pm 0.72}} $  \\
    
            %SAM _{\text{AdamW}}   &
            $ \textbf{90.52} _{\textcolor{black!60}{\pm 0.27}} $ 
            & $ 83.25 _{\textcolor{black!60}{\pm 2.79}} $ / $ 87.90 _{\textcolor{black!60}{\pm 2.21}} $ 
            & $ 88.38 _{\textcolor{black!60}{\pm 0.01}} $ / $ 88.79 _{\textcolor{black!60}{\pm 0.99}} $ 
            & $ 90.26 _{\textcolor{black!60}{\pm 0.28}} $ / $ 86.99 _{\textcolor{black!60}{\pm 0.31}} $ 
            & $ 81.56 _{\textcolor{black!60}{\pm 0.18}} $ / $ \textbf{82.46} _{\textcolor{black!60}{\pm 0.19}} $ 
            & $ \textbf{90.38} _{\textcolor{black!60}{\pm 0.05}} $ 
            & $ 68.83 _{\textcolor{black!60}{\pm 1.46}} $  \\
    
            %AdaHessian    & 
            $ 89.64 _{\textcolor{black!60}{\pm 0.13}} $ 
            & $ 79.74 _{\textcolor{black!60}{\pm 4.00}} $ / $ 85.26 _{\textcolor{black!60}{\pm 3.50}} $ 
            & $ 86.08 _{\textcolor{black!60}{\pm 4.04}} $ / $ 86.46 _{\textcolor{black!60}{\pm 4.06}} $ 
            & $ 90.37 _{\textcolor{black!60}{\pm 0.05}} $ / $ 87.07 _{\textcolor{black!60}{\pm 0.05}} $ 
            & $ 81.33 _{\textcolor{black!60}{\pm 0.17}} $ / $ 82.08 _{\textcolor{black!60}{\pm 0.02}} $ 
            & $ 89.94 _{\textcolor{black!60}{\pm 0.12}} $ 
            & $ 71.00 _{\textcolor{black!60}{\pm 1.04}} $ \\
            
            % Sophia-H  &
            $ 90.44 _{\textcolor{black!60}{\pm 0.46}} $ 
            & $ 85.78 _{\textcolor{black!60}{\pm 1.07}} $ / $ 89.90 _{\textcolor{black!60}{\pm 0.82}} $ 
            & $ 88.17 _{\textcolor{black!60}{\pm 1.07}} $ / $ 88.53 _{\textcolor{black!60}{\pm 1.13}} $ 
            & $ 90.70 _{\textcolor{black!60}{\pm 0.04}} $ / $ 87.60 _{\textcolor{black!60}{\pm 0.06}} $ 
            & $ \textbf{81.77} _{\textcolor{black!60}{\pm 0.18}} $ / $ 82.36 _{\textcolor{black!60}{\pm 0.22}} $ 
            & $ 90.12_{\textcolor{black!60}{\pm 0.14}} $ 
            & $ 70.76 _{\textcolor{black!60}{\pm 1.44}} $  \\
            
            \midrule
            
            \rowcolor{green!20} 
            $ 90.44 _{\textcolor{black!60}{\pm 0.98}} $    &
            $ \textbf{86.28} _{\textcolor{black!60}{\pm 0.28}} $ / $ \textbf{90.13} _{\textcolor{black!60}{\pm 0.161}} $     &
            $ \textbf{88.72} _{\textcolor{black!60}{\pm 0.75}} $ / $ \textbf{89.10} _{\textcolor{black!60}{\pm 0.70}}  $     &
            $ \textbf{90.91} _{\textcolor{black!60}{\pm 0.06}} $ / $ \textbf{87.85}  _{\textcolor{black!60}{\pm 0.09}} $     &
            $ 81.61 _{\textcolor{black!60}{\pm 0.25}} $ / $ 81.71 _{\textcolor{black!60}{\pm 0.11}} $     &
            $ 89.85_{\textcolor{black!60}{\pm 0.20}} $    &
            $ \textbf{72.08} _{\textcolor{black!60}{\pm 0.55}} $  \\
            
            \bottomrule
        \end{tabular}
    }
    \vspace{-0.5em}
    \label{tab:language}
\end{table*}

We first evaluate \sassha for image classification on CIFAR-10, CIFAR-100, and ImageNet.
We train various models of the ResNet family \citep{he2016deep,zagoruyko2016wide} and an efficient variant of Vision Transformer \citep{beyer2022better}.
We adhere to standard inception-style data augmentations during training instead of making use of advanced data augmentation techniques \citep{devries2017improved} or regularization methods \citep{gastaldi2017shake}.
Results are presented in \cref{tab:im_cls_results} and \cref{fig:im_cls_results}.

We begin by comparing the generalization performance of adaptive second-order methods to that of first-order methods.
Across all settings, adaptive second-order methods consistently exhibit lower accuracy than their first-order counterparts.
This observation aligns with previous studies indicating that second-order optimization often result in poorer generalization compared to first-order approaches.
In contrast, \sassha, benefiting from sharpness minimization, consistently demonstrates superior generalization performance, outperforming both first-order and second-order methods in every setting.
Particularly, \sassha is up to 4\% more effective than the best-performing adaptive or second-order methods (\eg, WRN-28-10, ViT-s-32).
Moreover, \sassha continually surpasses SGD and AdamW, even when they are trained for twice as many epochs, achieving a performance margin of about 0.3\% to 3\%. 
Further details are provided in \cref{app:comp_fo_fair}.

Interestingly, \sassha also outperforms SAM.
Since first-order methods typically exhibit superior generalization performance compared to second-order methods, it might be intuitive to expect SAM to surpass \sassha if the two are viewed merely as the outcomes of applying sharpness minimization to first-order and second-order methods, respectively.
However, the results conflict with this intuition.
We attribute this to the careful design choices made in \sassha, stabilizing Hessian approximation under sharpness minimization, so as to unleash the potential of the second-order method, leading to its outstanding performance.
As a support, we show that naively incorporating SAM into other second-order methods does not yield these favorable results in \cref{app:samsophia}.
We also make more comparisons with SAM in \cref{sec:sassha_vs_sam}.

\subsection{Language Modeling}

Recent studies have shown the potential of second-order methods for pretraining language models.
Here, we first evaluate how \sassha performs on this task.
Specifically, we train GPT1-mini, a scaled-down variant of GPT1 \citep{radford2019language}, on Wikitext-2 dataset \citep{merity2022pointer} using various methods including \sassha and compare their results (see the left of \cref{tab:language}).
Our results show that \sassha achieves the lowest perplexity among all methods including Sophia-H \citep{sophia}, a recent method that is designed specifically for language modeling tasks and sets state of the art, which highlights generality in addition to the numerical advantage of \sassha.

We also extend our evaluation to finetuning tasks.
Specifically, we finetune SqueezeBERT \citep{iandola2020squeezebert} for diverse tasks in the GLUE benchmark \citep{wang2018glue}.
The results are on the right side of \cref{tab:language}.
It shows that \sassha compares competitively to other second-order methods.
Notably, it also outperforms AdamW---often the method of choice for training language models---on nearly all tasks.

\subsection{Comparison to SAM}\label{sec:sassha_vs_sam}

So far, we have seen that \sassha outperforms second-order methods quite consistently on both vision and language tasks.
Interestingly, we also find that \sassha often improves upon SAM.
In particular, it appears that the gain is larger for the Transformer-based architectures, \ie, ViT results in \cref{tab:im_cls_results} or GPT/BERT results in \cref{tab:language}.

We posit that this is potentially due to the robustness of \sassha to the block heterogeneity inherent in Transformer architectures, where the Hessian spectrum varies significantly across different blocks.
This characteristic is known to make SGD perform worse than adaptive methods like Adam on Transformer-based models \citep{zhang2024why}.
Since \sassha leverages second-order information via preconditioning gradients, it has the potential to address the ill-conditioned nature of Transformers more effectively than SAM with first-order methods.

To push further, we conducted additional experiments.
First, we allocate more training budgets to SAM to see whether it compares to \sassha.
% additionally compare \sassha to SAM with more training budgets.
The results are presented in \cref{tab:sam}.
We find that SAM still underperforms \sassha, even though it is given more budgets of training iterations over data or wall-clock time.
Furthermore, we also compare \sassha to more advanced variants of SAM including ASAM \citep{asam} and GSAM \citep{gsam}, showing that \sassha performs competitively even to these methods (\cref{app:samvariants_vs_sassha}).
Notably, however, these variants of SAM require a lot more hyperparameter tuning to be compared.


\section{Conclusion \& Future Work}\label{conclusion}
This work presents XAMBA, the first framework optimizing SSMs on COTS NPUs, removing the need for specialized accelerators. XAMBA mitigates key bottlenecks in SSMs like CumSum, ReduceSum, and activations using ActiBA, CumBA, and ReduBA, transforming sequential operations into parallel computations. These optimizations improve latency, throughput (Tokens/s), and memory efficiency. Future work will extend XAMBA to other models, explore compression, and develop dynamic optimizations for broader hardware platforms.



% This work introduces XAMBA, the first framework to optimize SSMs on COTS NPUs, eliminating the need for specialized hardware accelerators. XAMBA addresses key bottlenecks in SSM execution, including CumSum, ReduceSum, and activation functions, through techniques like ActiBA, CumBA, and ReduBA, which restructure sequential operations into parallel matrix computations. These optimizations reduce latency, enhance throughput, and improve memory efficiency. 
% Experimental results show up to 2.6$\times$ performance improvement on Intel\textregistered\ Core\texttrademark\ Ultra Series 2 AI PC. 
% Future work will extend XAMBA to other models, incorporate compression techniques, and explore dynamic optimization strategies for broader hardware platforms.


% This work presents XAMBA, an optimization framework that enhances the performance of SSMs on NPUs. Unlike transformers, SSMs rely on structured state transitions and implicit recurrence, which introduce sequential dependencies that challenge efficient hardware execution. XAMBA addresses these inefficiencies by introducing CumBA, ReduBA, and ActiBA, which optimize cumulative summation, ReduceSum, and activation functions, respectively, significantly reducing latency and improving throughput. By restructuring sequential computations into parallelizable matrix operations and leveraging specialized hardware acceleration, XAMBA enables efficient execution of SSMs on NPUs. Future work will extend XAMBA to other state-space models, integrate advanced compression techniques like pruning and quantization, and explore dynamic optimization strategies to further enhance performance across various hardware platforms and frameworks.
% This work presents XAMBA, an optimization framework that enhances the performance of SSMs on NPUs. Key techniques, including CumBA, ReduBA, and ActiBA, achieve significant latency reductions by optimizing operations like cumulative summation, ReduceSum, and activation functions. Future work will focus on extending XAMBA to other state-space models, integrating advanced compression techniques, and exploring dynamic optimization strategies to further improve performance across various hardware platforms and frameworks.

% This work introduces XAMBA, an optimization framework for improving the performance of Mamba-2 and Mamba models on NPUs. XAMBA includes three key techniques: CumBA, ReduBA, and ActiBA. CumBA reduces latency by transforming cumulative summation operations into matrix multiplication using precomputed masks. ReduBA optimizes the ReduceSum operation through matrix-vector multiplication, reducing execution time. ActiBA accelerates activation functions like Swish and Softplus by mapping them to specialized hardware during the DPU’s drain phase, avoiding sequential execution bottlenecks. Additionally, XAMBA enhances memory efficiency by reducing SRAM access, increasing data reuse, and utilizing Zero Value Compression (ZVC) for masks. The framework provides significant latency reductions, with CumBA, ReduBA, and ActiBA achieving up to 1.8X, 1.1X, and 2.6X reductions, respectively, compared to the baseline.
% Future work includes extending XAMBA to other state-space models (SSMs) and exploring further hardware optimizations for emerging NPUs. Additionally, integrating advanced compression techniques like pruning and quantization, and developing adaptive strategies for dynamic optimization, could enhance performance. Expanding XAMBA's compatibility with other frameworks and deployment environments will ensure broader adoption across various hardware platforms.

\ifdefined\isarxiv
%\section*{Acknowledgments}
\bibliographystyle{alpha}
\bibliography{ref} 
\else
\bibliography{ref}
\bibliographystyle{icml2025}
% \bibliographystyle{alpha}

\fi



\newpage
\onecolumn
\appendix

\begin{center}
    \textbf{\LARGE Appendix}
\end{center}

% {\hypersetup{linkcolor=black}
% \tableofcontents
% }




\newpage
\appendix
\section{Applicability of SparseTransX for dense graphs} 
\label{A:density}
Even for fully dense graphs, our KGE computations remain highly sparse. This is because our SpMM leverages the incidence matrix for triplets, rather than the graph's adjacency matrix. In the paper, the sparse matrix $A \in \{-1,0,1\}^{M \times (N+R)}$ represents the triplets, where $N$ is the number of entities, $R$ is the number of relations, and $M$ is the number of triplets. This representation remains extremely sparse, as each row contains exactly three non-zero values (or two in the case of the "ht" representation). Hence, the sparsity of this formulation is independent of the graph's structure, ensuring computational efficiency even for dense graphs.

\section{Computational Complexity}
\label{A:complexity}
 For a sparse matrix $A$ with $m \times k$ having $nnz(A)=$ number of non zeros and dense matrix $X$ with $k \times n$ dimension, the computational complexity of the SpMM is $O(nnz(A) \cdot n)$ since there are a total of $nnz(A)$ number of dot products each involving $n$ components. Since our sparse matrix contains exactly three non-zeros in each row, $nnz(A) = 3m$. Therefore, the complexity of SpMM is $O(3m \cdot n)$ or $O(m \cdot n)$, meaning the complexity increases when triplet counts or embedding dimension is increased. Memory access pattern will change when the number of entities is increased and it will affect the runtime, but the algorithmic complexity will not be affected by the number of entities/relations.

\section{Applicability to Non-translational Models}
\label{A:non_trans}
Our paper focused on translational models using sparse operations, but the concept extends broadly to various other knowledge graph embedding (KGE) methods. Neural network-based models, which are inherently matrix-multiplication-based, can be seamlessly integrated into this framework. Additionally, models such as DistMult, ComplEx, and RotatE can be implemented with simple modifications to the SpMM operations. Implementing these KGE models requires modifying the addition and multiplication operators in SpMM, effectively changing the semiring that governs the multiplication.   

In the paper, the sparse matrix $A \in \{-1,0,1\}^{M \times (N+R)}$ represents the triplets, and the dense matrix $E \in \mathbb{R}^{(N+R) \times d}$ represents the embedding matrix, where $N$ is the number of entities, $R$ is the number of relations, and $M$ is the number of triplets. TransE’s score function, defined as $h + r - t$, is computed by multiplying $A$ and $E$ using an SpMM followed by the L2 norm. This operation can be generalized using a semiring-based SpMM model: $Z_{ij} = \bigoplus_{k=1}^{n} (A_{ik} \otimes E_{kj})$

Here, $\oplus$ represents the semiring addition operator, and $\otimes$ represents the semiring multiplication operator. For TransE, these operators correspond to standard arithmetic addition and multiplication, respectively.

\subsection*{DistMult} 
DistMult’s score function has the expression $h \odot r \odot t$. To adapt SpMM for this model, two key adjustments are required: The sparse matrix $A$ stores $+1$ at the positions corresponding to $h_{\text{idx}}$, $t_{\text{idx}}$, and $r_{\text{idx}}$. Both the semiring addition and multiplication operators are set to arithmetic multiplication. These changes enable the use of SpMM for the DistMult score function.

\subsection*{ComplEx} 
ComplEx’s score function has $h \odot r \odot \bar{t}$, where embeddings are stored as complex numbers (e.g., using PyTorch). In this case, the semiring operations are similar to DistMult, but with complex number multiplication replacing real number multiplication.

\subsection*{RotatE} 
RotatE’s score function has $h \odot r - t$. For this model, the semiring requires both arithmetic multiplication and subtraction for $\oplus$. With minor modifications to our SpMM implementation, the semiring addition operator can be adapted to compute $h \odot r - t$.

\subsection*{Support from other libraries}
Many existing libraries, such as GraphBLAS (Kimmerer, Raye, et al., 2024), Ginkgo (Anzt, Hartwig, et al., 2022), and Gunrock (Wang, Yangzihao, et al., 2017), already support custom semirings in SpMM. We can leverage C++ templates to extend support for KGE models with minimal effort.


\begin{figure*}[t]
\centering     %%% not \center
\includegraphics[width=\textwidth]{figures/all-eval.pdf}
\caption{Loss curve for sparse and non-sparse approach. Sparse approach eventually reaches the same loss value with similar Hits@10 test accuracy.}
\label{fig:loss_curve}
\end{figure*}

\section{Model Performance Evaluation and Convergence}
\label{A:eval}
SpTransX follows a slightly different loss curve (see Figure \ref{fig:loss_curve}) and eventually converges with the same loss as other non-sparse implementations such as TorchKGE. We test SpTransX with the WN18 dataset having embedding size 512 (128 for TransR and TransH due to memory limitation) and run 200-1000 epochs. We compute average Hits@10 of 9 runs with different initial seeds and a learning rate scheduler. The results are shown below. We find that Hits@10 is generally comparable to or better than the Hits@10 achieved by TorchKGE.

\begin{table}[h]
\centering
\caption{Average of 9 Hits@10 Accuracy for WN18 dataset}
\begin{tabular}{|c|c|c|}
\hline
\textbf{Model} & \textbf{TorchKGE} & \textbf{SpTransX} \\ \hline
TransE         & 0.79 ± 0.001700   & 0.79 ± 0.002667   \\ \hline
TransR         & 0.29 ± 0.005735   & 0.33 ± 0.006154   \\ \hline
TransH         & 0.76 ± 0.012285   & 0.79 ± 0.001832   \\ \hline
TorusE         & 0.73 ± 0.003258   & 0.73 ± 0.002780   \\ \hline
\end{tabular}
\label{table:perf_eval}
\end{table}

% We also plot the loss curve for different models in Figure \ref{fig:loss_curve}. We observe that the sparse approach follows a similar loss curve and eventually converges to the same final loss.

\section{Distributed SpTransX and Its Applicability to Large KGs}
\label{A:dist}
SpTransX framework includes several features to support distributed KGE training across multi-CPU, multi-GPU, and multi-node setups. Additionally, it incorporates modules for model and dataset streaming to handle massive datasets efficiently. 

Distributed SpTransX relies on PyTorch Distributed Data Parallel (DDP) and Fully Sharded Data Parallel (FSDP) support to distribute sparse computations across multiple GPUs. 

\begin{table}[h]
\centering
\caption{Average Time of 15 Epochs (seconds). Training time of TransE model with Freebase dataset (250M triplets, 77M entities. 74K relations, batch size 393K)  on 32 NVIDIA A100 GPUs. FSDP enables model training with larger embedding when DDP fails.}
\begin{tabular}{|p{2cm}|p{2.5cm}|p{2.5cm}|}
\hline
\textbf{Embedding Size} & \textbf{DDP (Distributed Data Parallel)} & \textbf{FSDP (Fully Sharded Data Parallel)} \\ \hline
16                      & 65.07 ± 1.641                            & 63.35 ± 1.258                               \\ \hline
20                      & Out of Memory                            & 96.44 ± 1.490                               \\ \hline
\end{tabular}
\end{table}

We run an experiment with a large-scale KG to showcase the performance of distributed SpTransX. Freebase (250M triplets, 77M entities. 74K relations, batch size 393K) dataset is trained using the TransE model on 32 NVIDIA A100 GPUs of NERSC using various distributed settings. SpTransX’s Streaming dataset module allows fetching only the necessary batch from the dataset and enables memory-efficient training. FSDP enables model training with larger embedding when DDP fails.

\section{Scaling and Communication Bottlenecks for Large KG Training}
\label{A:scaling}
Communication can be a significant bottleneck in distributed KGE training when using SpMM. However, by leveraging Distributed Data-Parallel (DDP) in PyTorch, we successfully scale distributed SpTransX to 64 NVIDIA A100 GPUs with reasonable efficiency. The training time for the COVID-19 dataset with 60,820 entities, 62 relations, and 1,032,939 triplets is in Table \ref{table:scaling}. 
% \vspace{-.3cm}
\begin{table}[h]
\centering
\caption{Scaling TransE model on COVID-19 dataset}
\begin{tabular}{|c|c|}
\hline
\textbf{Number of GPUs} & \textbf{500 epoch time (seconds)} \\ \hline
4                       & 706.38                            \\ \hline
8                       & 586.03                            \\ \hline
16                      & 340.00                               \\ \hline
32                      & 246.02                            \\ \hline
64                      & 179.95                            \\ \hline
\end{tabular}
\label{table:scaling}
\end{table}
% \vspace{-.2cm}
It indicates that communication is not a bottleneck up to 64 GPUs. If communication becomes a performance bottleneck at larger scales, we plan to explore alternative communication-reducing algorithms, including 2D and 3D matrix distribution techniques, which are known to minimize communication overhead at extreme scales. Additionally, we will incorporate model parallelism alongside data parallelism for large-scale knowledge graphs.

\section{Backpropagation of SpMM}
\label{A:backprop}
 Our main computational kernel is the sparse-dense matrix multiplication (SpMM). The computation of backpropagation of an SpMM w.r.t. the dense matrix is also another SpMM. To see how, let's consider the sparse-dense matrix multiplication $AX = C$ which is part of the training process. As long as the computational graph reduces to a single scaler loss $\mathfrak{L}$, it can be shown that $\frac{\partial C}{\partial X} = A^T$. Here, $X$ is the learnable parameter (embeddings), and $A$ is the sparse matrix. Since $A^T$ is also a sparse matrix and $\frac{\partial \mathfrak{L}}{\partial C}$ is a dense matrix, the computation $\frac{\partial \mathfrak{L}}{\partial X} = \frac{\partial C}{\partial X} \times \frac{\partial \mathfrak{L}}{\partial C} = A^T \times \frac{\partial \mathfrak{L}}{\partial C} $ is an SpMM. This means that both forward and backward propagation of our approach benefit from the efficiency of a high-performance SpMM.

\subsection*{Proof that $\frac{\partial C}{\partial X} = A^T$}
 To see why $\frac{\partial C}{\partial X} = A^T$ is used in the gradient calculation, we can consider the following small matrix multiplication without loss of generality.
\begin{align*}
A &= \begin{bmatrix}
a_1 & a_2 \\
a_3 & a_4
\end{bmatrix} \\ 
 X &= \begin{bmatrix}
x_1 & x_2 \\
x_3 & x_4
\end{bmatrix} \\
 C &=  \begin{bmatrix}
c_1 & c_2 \\
c_3 & c_4
\end{bmatrix}
\end{align*}
Where $C=AX$, thus-
\begin{align*}
c_1&=f(x_1, x_3) \\
c_2&=f(x_2, x_4) \\
c_3&=f(x_1, x_3) \\
c_4&=f(x_2, x_4) \\
\end{align*}
Therefore-
\begin{align*}
\frac{\partial \mathfrak{L}}{\partial x_1} &= \frac{\partial \mathfrak{L}}{\partial c_1} \times \frac{\partial c_1}{\partial x_1} + \frac{\partial \mathfrak{L}}{\partial c_2} \times \frac{\partial c_2}{\partial x_1} + \frac{\partial \mathfrak{L}}{\partial c_3} \times \frac{\partial c_3}{\partial x_1} + \frac{\partial \mathfrak{L}}{\partial c_4} \times \frac{\partial c_4}{\partial x_1}\\
&= \frac{\partial \mathfrak{L}}{\partial c_1} \times \frac{\partial \mathfrak{c_1}}{\partial x_1} + 0 + \frac{\partial \mathfrak{L}}{\partial c_3} \times \frac{\partial \mathfrak{c_3}}{\partial x_1} + 0\\
&= a_1 \times \frac{\partial \mathfrak{L}}{\partial c_1} + a_3 \times \frac{\partial \mathfrak{L}}{\partial c_3}\\
\end{align*}

Similarly-
\begin{align*}
\frac{\partial \mathfrak{L}}{\partial x_2}
&= a_1 \times \frac{\partial \mathfrak{L}}{\partial c_2} + a_3 \times \frac{\partial \mathfrak{L}}{\partial c_4}\\
\frac{\partial \mathfrak{L}}{\partial x_3}
&= a_2 \times \frac{\partial \mathfrak{L}}{\partial c_1} + a_4 \times \frac{\partial \mathfrak{L}}{\partial c_3}\\
\frac{\partial \mathfrak{L}}{\partial x_4}
&= a_2 \times \frac{\partial \mathfrak{L}}{\partial c_2} + a_4 \times \frac{\partial \mathfrak{L}}{\partial c_4}\\
\end{align*}
This can be expressed as a matrix equation in the following manner-
\begin{align*}
\frac{\partial \mathfrak{L}}{\partial X} &= \frac{\partial C}{\partial X} \times \frac{\partial \mathfrak{L}}{\partial C}\\
\implies \begin{bmatrix}
\frac{\partial \mathfrak{L}}{\partial x_1} & \frac{\partial \mathfrak{L}}{\partial x_2} \\
\frac{\partial \mathfrak{L}}{\partial x_3} & \frac{\partial \mathfrak{L}}{\partial x_4}
\end{bmatrix} &= \frac{\partial C}{\partial X} \times \begin{bmatrix}
\frac{\partial \mathfrak{L}}{\partial c_1} & \frac{\partial \mathfrak{L}}{\partial c_2} \\
\frac{\partial \mathfrak{L}}{\partial c_3} & \frac{\partial \mathfrak{L}}{\partial c_4}
\end{bmatrix}
\end{align*}
By comparing the individual partial derivatives computed earlier, we can say-

\begin{align*}
\begin{bmatrix}
\frac{\partial \mathfrak{L}}{\partial x_1} & \frac{\partial \mathfrak{L}}{\partial x_2} \\
\frac{\partial \mathfrak{L}}{\partial x_3} & \frac{\partial \mathfrak{L}}{\partial x_4}
\end{bmatrix} &= \begin{bmatrix}
a_1 & a_3 \\
a_2 & a_4
\end{bmatrix} \times \begin{bmatrix}
\frac{\partial \mathfrak{L}}{\partial c_1} & \frac{\partial \mathfrak{L}}{\partial c_2} \\
\frac{\partial \mathfrak{L}}{\partial c_3} & \frac{\partial \mathfrak{L}}{\partial c_4}
\end{bmatrix}\\
\implies \begin{bmatrix}
\frac{\partial \mathfrak{L}}{\partial x_1} & \frac{\partial \mathfrak{L}}{\partial x_2} \\
\frac{\partial \mathfrak{L}}{\partial x_3} & \frac{\partial \mathfrak{L}}{\partial x_4}
\end{bmatrix} &= A^T \times \begin{bmatrix}
\frac{\partial \mathfrak{L}}{\partial c_1} & \frac{\partial \mathfrak{L}}{\partial c_2} \\
\frac{\partial \mathfrak{L}}{\partial c_3} & \frac{\partial \mathfrak{L}}{\partial c_4}
\end{bmatrix}\\
\implies \frac{\partial \mathfrak{L}}{\partial X} &= A^T \times \frac{\partial \mathfrak{L}}{\partial C}\\
\therefore \frac{\partial C}{\partial X} &= A^T \qed
\end{align*}

\section{Tools from Previous Works}\label{sec:app:main_theorems}

We state the tools in \cite{fsi+24} that we will use to prove our main results.

\subsection{Definitions of Besov Space}

\begin{definition}[Modulus of Smoothness]
\label{def:modulus_smoothness}
Let $\Omega$ be a domain in $\R^d$. For a function 
$f \in L^{p'}(\Omega)$ with $p' \in (0,\infty]$, 
the $r$-th modulus of smoothness of $f$ is defined by
\begin{align*}
    w_{r, p'}(f,t) = \sup_{\|h\|_2 \leq t} \|\Delta_h^r (f) \|_{p'},
\end{align*}
where the finite difference operator $\Delta_h^r (f)(x)$ is given by
\begin{align*}
    \Delta_h^r (f)(x) =
\begin{cases}
\sum_{j=0}^{r} \binom{r}{j} (-1)^{r-j} f(x + jh), & \mathrm{if } x + jh \in \Omega \mathrm{ for all } j, \\
0, & \mathrm{otherwise}.
\end{cases}
\end{align*}
\end{definition}

\begin{definition}[Besov Seminorm]
\label{def:besov_seminorm}
Let $0 < p', q' \leq \infty$, $s > 0$, and set $r := | s | + 1$. The Besov seminorm of $f \in L^{p'}(\Omega)$ is defined as
\begin{align*}
| f |_{B^{s}_{p',q'}} :=
\begin{cases}
\left( \int_0^{\infty} (t^{-s} w_{r, p'}(f,t))^{q'} \frac{dt}{t} \right)^{\frac{1}{q'}}, & q' < \infty, \\
\sup_{t>0} t^{-s} w_{r, p'}(f,t), & q' = \infty.
\end{cases}
\end{align*}
\end{definition}

\begin{definition}[Besov Space]
\label{def:besov_space}
The Besov space $B^{s}_{p',q'}(\Omega)$ is the function space equipped with the norm
\begin{align*}
    \| f \|_{B^{s}_{p',q'}} := \| f \|_{p'} + | f |_{B^{s}_{p',q'}},
\end{align*}
It consists of all functions $f \in L^{p'}(\Omega)$ such that
\begin{align*}
    B^{s}_{p',q'}(\Omega) := \{ f \in L^{p'}(\Omega) \mid \| f \|_{B^{s}_{p',q'}} < \infty \}.
\end{align*}
\end{definition}

\begin{remark}
The parameter $s$ governs the degree of smoothness of functions in $B^{s}_{p',q'}(\Omega)$. In particular, when $p' = q'$ and $s$ is an integer, the Besov space $B^{s}_{p',q'}(\Omega)$ coincides with the standard Sobolev space of order $s$. For further details on the properties and applications of Besov spaces, see \cite{t92}.
\end{remark}

\subsection{B-spline}

\begin{definition}[Indicator Function]\label{def:indicator_func}
Let ${\cal N}(x)$ be the characteristic function defined by
\begin{align*}
    {\cal N}(x) &=
    \begin{cases}
        1, & x \in [0,1],\\
        0, & \mathrm{otherwise}.
    \end{cases}
\end{align*}
\end{definition}

\begin{definition}[Cardinal B-Spline]\label{def:cardinal_b_spline}
For $\ell \in \N$, the cardinal B-spline of order $\ell$ is defined by
\begin{align*}
    {\cal N}_\ell(x) 
    := & 
    \underbrace{{\cal N} * {\cal N} * \cdots * {\cal N}}_{\ell+1 \mathrm{ times}}(x),
\end{align*}
where $*$ denotes the convolution operation. Explicitly, the convolution of two functions $f,g: \R \to \R$ is given by
\begin{align*}
    (f * g)(x) =& \int_{\R} f(x-y)  g(y)  \d y.
\end{align*}
Thus, ${\cal N}_\ell(x)$ is obtained by convolving ${\cal N}$ with itself $(\ell+1)$ times.
\end{definition}

\begin{definition}[Tensor Product B-Spline Basis]\label{def:tensor_product_b_spline}
For a multi-index $k \in \N^d$ and $j \in  \Z^d$, the tensor product B-spline basis in $ \R^d$ of order $\ell$ is defined as
\begin{align*}
    M_{k,j}^d(x) :=&~ \prod_{i=1}^d {\cal N}_\ell (2^{k_i} x_i - j_i).
\end{align*}
This basis is constructed as the product of univariate B-splines, scaled and translated according to the parameters $k$ and $j$.
\end{definition}

\begin{definition}[B-Spline Approximation in Besov Spaces in \cite{s19, oas23}]\label{def:approx_b_spline}
A function $f$ in the Besov space can be approximated using a superposition of tensor product B-splines as
\begin{align*}
    f_N(x) =&~ \sum_{(k,j)} \alpha_{k,j}  M_{k,j}^d(x),
\end{align*}
where the summation is taken over appropriate index sets $(k,j)$, and the coefficients $\alpha_{k,j}$ are real numbers that determine the contribution of each basis function.
\end{definition}

\subsection{Class of Neural Networks}
\begin{definition}[Neural Network Class in \cite{fsi+24}]\label{def:sparse_nn_class}
\label{def:nn_class}
Let $L \in \mathbb{N}$ denote the depth (number of layers), $W = (W_1, W_2, \dots, W_{L+1}) \in \mathbb{N}^{L+1}$ the width configuration of the network, $S \in \mathbb{N}$ a sparsity constraint, and $B > 0$ a norm bound. The class of neural networks ${\cal M}(L,W,S,B)$ is defined as
\begin{align*}
    {\cal M}(L,W,S,B) := 
     \{ &~
    \psi_{A^{(L)},b^{(L)}} 
    \circ \cdots \circ 
    \psi_{A^{(2)},b^{(2)}} (A^{(1)}x + b^{(1)} ) 
    m|  
    A^{(i)} \in \R^{W_{i+1} \times W_i}, 
    b^{(i)} \in \R^{W_{i+1}}, \\ 
    &~
    \sum_{i=1}^{L}  (\|A^{(i)}\|_0 + \|b^{(i)}\|_0 ) \leq S, 
    \quad 
    \max_{1 \leq i \leq L}  \{\|A^{(i)}\|_\infty \vee \|b^{(i)}\|_\infty \} \leq B
     \}.
\end{align*}
Here, the function $\psi_{A,b}: \R^{W_i} \to \R^{W_{i+1}}$ represents the affine transformation with ReLU activation, given by
\begin{align*}
    \psi_{A,b}(z) = A \cdot \mathsf{ReLU}(z) + b, \quad \mathrm{where} \quad \mathsf{ReLU}(z) = \max\{0, z\}.
\end{align*}
The sparsity constraint ensures that the total number of nonzero entries in all weight matrices and bias vectors does not exceed $S$, while the norm constraint limits their maximum absolute values to $B$.
\end{definition}

\subsection{Assumptions}

\begin{remark}\label{rmk:setting_on_assumption}
We introduce a small positive constant $\delta>0$ and denote by $N$ the number of basis functions in the B-spline used to approximate $p_t(x)$. The value of $N$ is determined by the sample size $n$, specifically following the relation $N = n^{\frac{d}{2s+d}}$, which balances the approximation error and the complexity of both the B-spline and the neural network. \end{remark}

\begin{definition}[Stopping Time]\label{def:stop_time}
  As we introduce in Remark~\ref{rmk:setting_on_assumption}, we define the stopping time as $T_0 = N^{-R_0}$, where $R_0$ is a parameter to be specified later, and consider solving the ODE backward in time from $t=1$ down to $t=T_0$.   
\end{definition}

\begin{definition}[Reduced Cube]
Let $I^d = [-1,1]^d$ denote the $d$-dimensional cube. To mitigate boundary effects when $N$ is large, we define the reduced cube as
\begin{align*}
I^d_N := [-1 + N^{-(1-\kappa\delta)}, 1 - N^{-(1-\kappa\delta)}]^d,  
\end{align*}
where the parameter $\kappa > 0$ will be specified later in Assumption~\ref{ass:A3}.
\end{definition}

\begin{assumption}[Smoothness and support of $p_0$]\label{ass:A1}
The target probability $P_0$ has support contained in $I^d$, and its probability density function $p_0$ satisfies
\begin{align*}
    p_0 \in B^s_{p',q'}(I^d)
  \quad\mathrm{and}\quad
  p_0 \in B^{\wt{s}}_{p',q'}(I^d \setminus I^d_N)
  \quad\mathrm{with}\quad
  \wt{s} \geq \max\{6s-1, 1\}.
\end{align*}
\end{assumption}


\begin{assumption}[Boundedness away from $0$ and above]\label{ass:A2}
There exists a constant $C_0>0$ such that
\begin{align*}
    C_0^{-1} \leq  p_0(x) \leq  C_0 \quad\mathrm{for all}\quad x \in I^d.  
\end{align*}
\end{assumption}

\begin{assumption}[Form of $(\alpha_t,\beta_t)$ and their bounds]\label{ass:A3}
There are constants $\kappa \geq \frac{1}{2}$, $b_0>0$, $\wt{\kappa}>0$, and $\wt{b}_0>0$ such that, for sufficiently small $t \geq T_0$,
\begin{align*}
  \alpha_t  =  b_0, t^{\kappa},
  \quad\mathrm{and}\quad
  1 - \beta_t  =  \wt{b}_0, t^{\wt{\kappa}}.
\end{align*}
Moreover, there exist $D_0 > 0$ and $K_0 > 0$ such that $\forall t \in [T_0,1]$, we have
\begin{align*}
  D_0^{-1}  \leq  \alpha_t^2 + \beta_t^2   \leq   D_0,
  \quad
   |\dot{\alpha}_t | +  |\dot{\beta}_t|  \leq   N^{K_0}.
\end{align*}
\end{assumption}

\begin{assumption}[Additional bound in the critical case $\kappa = \frac{1}{2}$]\label{ass:A4}
If $\kappa = \frac{1}{2}$, then there exist $b_1>0$ and $D_1>0$ such that, for all $0 \leq \gamma < R_0$,
\begin{align*}
\int_{T_0}^{N^{-\gamma}} \{ (\dot{\alpha}_t )^2 +  (\dot{\beta}_t )^2 \} \d t \leq  
D_1  (\log N )^{b_1}.
\end{align*}
\end{assumption}

\begin{assumption}[Lipschitz bound on the first moment]\label{ass:A5}
There is a constant $C_L > 0$ such that, for all $t \in [T_0, 1]$,
\begin{align*}
\| \frac{\partial}{\partial x} \int y p_t(y | x) \d y \|_{\mathrm{op}} \leq  C_L.
\end{align*}
\end{assumption}

\subsection{Approximation error for small \texorpdfstring{$t$}{}}
\begin{lemma}[Theorem 7 in \cite{fsi+24}]\label{lem:error_approx_small_t}
    Under Assumptions~\ref{ass:A1}~\ref{ass:A2}~\ref{ass:A3}~\ref{ass:A4} and \ref{ass:A5}, and if the following holds 
    \begin{itemize}
        \item $L = O(\log^4 N )$.
        \item $\|W\|_{\infty} = O(N \log^{6} N)$
        \item $S = O(N \log^{8} N)$
        \item $B = \exp(O (\log N \log \log N ) ).$
    \end{itemize}
    Then there exists a neural network $\phi  \in  {\cal M}(L,W,S,B)$ such that, for sufficiently large $N$, we have
    \begin{align*}
        \int  \|\phi(x,t) - \dot{x}_t^\mathrm{true} \|^{2}_2 p_{t}(x) \d x \lesssim  (\dot{\alpha}_t^{2} \log N  +  \dot{\beta}_t^{2} ) N^{-\frac{2s}{d}},
    \end{align*}
    holds for any $t \in [T_{0}, 3T_{*}]$.
    In addition, $\phi$ can be taken so we have
    \begin{align*}
         \|\phi(\cdot,t) \|_\infty = O(  |\dot{\alpha}_t | \sqrt{\log n} +  |\dot{\beta}_t |) .
    \end{align*}
\end{lemma}

\subsection{Approximation error for large \texorpdfstring{$t$}{}}

\begin{lemma}[Theorem 7 in \cite{fsi+24}]\label{lem:error_approx_large_t}
    Fix $t_{*} \in [T_{*},1]$ and let $\eta>0$ be arbitrary, under Assumptions~\ref{ass:A1}~\ref{ass:A2}~\ref{ass:A3}~\ref{ass:A4} and \ref{ass:A5}, and if the following holds 
    \begin{itemize}
        \item $L = O(\log^4 N )$.
        \item $\|W\|_{\infty} = O(N)$
        \item $S = O(t_{*}^{- d\kappa} N^{\delta\kappa})$
        \item $B = \exp(O (\log N \log \log N ) ).$
    \end{itemize}
  Then there exist a neural network $\phi  \in  {\cal M}(L,W,S,B)$ such that
\begin{align*}
  \int  \| \phi(x,t) - \dot{x}^\mathrm{true}_t \|^2 p_{t} (x) \d x \lesssim (\dot{\alpha}_t^{2} \log N  +   \dot{\beta}_t^{2} ) N^{-\eta}.
 \end{align*}
 holds for any $t \in [2t_{*}, 1]$. In addition, $\phi$ can be taken so we have
 \begin{align*}
       \|\phi(\cdot,t) \|_{\infty} =  O( |\dot{\alpha}_t | \log N +  |\dot{\beta}_t | ).
 \end{align*}
\end{lemma}

\section{Theory of Higher Order Flow Matching}\label{sec:app:higher_order_flow_matching}

We use $\frac{\d^k}{\d t^k} x_t^\mathrm{true}$ to denote the $k$-th order derivative of $x_t^\mathrm{true}$ with respect to $t$. Note that $\dot{x}^\mathrm{true}_t := \frac{\d}{\d t} x^\mathrm{true}_t$, and $\ddot{x}^\mathrm{true}_t := \frac{\d^2}{\d t^2} x^\mathrm{true}_t$.

\subsection{Approximation Error of Second Order Flow Matching for Small \texorpdfstring{$t$}{}}
\begin{theorem}[Approximation error of second order flow matching for small $t$, formal version of Theorem~\ref{thm:secon_order_small_t:informal}]\label{thm:secon_order_small_t:formal}
    Under Assumptions~\ref{ass:A1}~\ref{ass:A2}~\ref{ass:A3}~\ref{ass:A4} and \ref{ass:A5}, and if the following holds 
    \begin{itemize}
        \item $L = O(\log^4 N )$.
        \item $\|W\|_{\infty} = O(N \log^{6} N)$
        \item $S = O(N \log^{8} N)$
        \item $B = \exp(O (\log N \log \log N ) ).$
    \end{itemize}
    Then there exists neural networks $\phi_{1},\phi_2  \in  {\cal M}(L,W,S,B)$ such that, for sufficiently large $N$, we have
\begin{align*}
    &~ \int (\|\phi_1(x, t) - \dot{x}_t^\mathrm{true}\|_2^2 + \|\phi_2(x, t) - \ddot{x}_t^\mathrm{true}\|_2^2) p_t(x) \d x \\ \lesssim &~ (\dot{\alpha}_t^2 \log N + \dot{\beta}_t^2 ) N^{- \frac{2s}{d}} +
    \E_{x \sim P_t}[\|\dot{x}^\mathrm{true}_t - \ddot{x}^\mathrm{true}_t\|_2^2]
\end{align*}
    holds for any $t \in [T_{0}, 3T_{*}]$. In addition, $\phi_1, \phi_2$ can be taken so we have
    \begin{align*}
         \|\phi_1(\cdot,t) \|_\infty = O(  |\dot{\alpha}_t | \sqrt{\log n} +  |\dot{\beta}_t |) \mathrm{~~~and~~~} \|\phi_2(\cdot,t) \|_\infty = O(  |\dot{\alpha}_t | \sqrt{\log n} +  |\dot{\beta}_t |).
    \end{align*}
\end{theorem}
\begin{proof}
    Suppose that $t \in [T_0, 3T_*]$.
    By Lemma~\ref{lem:error_approx_small_t}, there is $\phi_1 \in {\cal M}(L,W,S,B)$ such that
    \begin{align}
        \label{eq:tmp_1}
        \int (\|\phi_1(x, t) - \dot{x}_t^\mathrm{true}\|_2^2 p_t(x) \lesssim  (\dot{\alpha}_t^2 \log N + \dot{\beta}_t^2 ) N^{- \frac{2s}{d}}.
    \end{align}
    
    Next, we can show that there exists some $\phi_2 \in {\cal M}(L,W,S,B)$ such that
    \begin{align}
         \int \|\phi_2(x, t) - \ddot{x}_t^\mathrm{true}\|_2^2 p_t(x) \d x 
         = &~\int \|\phi_2(x, t) - \dot{x}_t^\mathrm{true} +\dot{x}_t^\mathrm{true} - \ddot{x}_t^\mathrm{true}\|_2^2 p_t(x) \d x \notag \\
         \leq &~ \int (\|\phi_2(x, t) - \dot{x}_t^\mathrm{true}\|_2 + \| \dot{x}_t^\mathrm{true}-\ddot{x}_t^\mathrm{true}\|_2)^2 p_t(x) \d x \notag \\
         \leq &~ \int 2(\|\phi_2(x, t) - \dot{x}_t^\mathrm{true}\|_2^2 + \| \dot{x}_t^\mathrm{true}-\ddot{x}_t^\mathrm{true}\|_2^2) p_t(x) \d x \notag \\
         = &~ 2\int \|\phi_2(x, t) - \dot{x}_t^\mathrm{true}\|_2^2 p_t(x) \d x + 2\int \| \dot{x}_t^\mathrm{true}-\ddot{x}_t^\mathrm{true}\|_2^2 p_t(x) \d x \notag \\
         = &~ 2\int \|\phi_2(x, t) - \dot{x}_t^\mathrm{true}\|_2^2p_t(x)\d x + 2\E_{x \sim P_t}[\|\dot{x}^\mathrm{true}_t - \ddot{x}^\mathrm{true}_t\|_2^2 \notag \\
         \lesssim &~ (\dot{\alpha}_t^2 \log N + \dot{\beta}_t^2 ) N^{- \frac{2s}{d}} + \E_{x \sim P_t}[\|\dot{x}^\mathrm{true}_t - \ddot{x}^\mathrm{true}_t\|_2^2] \label{eq:tmp_2}
    \end{align}
    where the first step follows from the basic algebra, the second step follows from the triangle inequality, the third step follows from $(a+b)^2 \leq 2a^2 + 2b^2$, the fourth step follows from basic algebra, the fifth step follows from the definition of expectation, and the last step follows from Lemma~\ref{lem:error_approx_small_t}.

    Finally, by Eq.~\eqref{eq:tmp_1} and Eq.~\eqref{eq:tmp_2}, for any $t \in [T_0, 3T_*]$, we have
    \begin{align*}
    &~ \int (\|\phi_1(x, t) - \dot{x}_t^\mathrm{true}\|_2^2 + \|\phi_2(x, t) - \ddot{x}_t^\mathrm{true}\|_2^2) p_t(x) \d x \\ \lesssim &~ (\dot{\alpha}_t^2 \log N + \dot{\beta}_t^2 ) N^{- \frac{2s}{d}} +
    \E_{x \sim P_t}[\|\dot{x}^\mathrm{true}_t - \ddot{x}^\mathrm{true}_t\|_2^2].
    \end{align*}

    Moreover, by Lemma~\ref{lem:error_approx_small_t}, $\phi_1, \phi_2$ can be taken so we have
    \begin{align*}
         \|\phi_1(\cdot,t) \|_\infty = O(  |\dot{\alpha}_t | \sqrt{\log n} +  |\dot{\beta}_t |) \mathrm{~~~and~~~} \|\phi_2(\cdot,t) \|_\infty = O(  |\dot{\alpha}_t | \sqrt{\log n} +  |\dot{\beta}_t |).
    \end{align*}
    Thus, the proof is complete.
\end{proof}

\subsection{Approximation Error of Higher Order Flow Matching for Small \texorpdfstring{$t$}{}}
\begin{theorem}[Approximation error of higher order flow matching for small $t$]\label{thm:higher_order_small_t:formal}
    Under Assumptions~\ref{ass:A1}~\ref{ass:A2}~\ref{ass:A3}~\ref{ass:A4} and \ref{ass:A5}, and if the following holds 
    \begin{itemize}
        \item $L = O(\log^4 N )$.
        \item $\|W\|_{\infty} = O(N \log^{6} N)$
        \item $S = O(N \log^{8} N)$
        \item $B = \exp(O (\log N \log \log N ) )$
        \item $K = O(1)$
    \end{itemize}
    Then there exists neural networks $\phi_{1},\phi_2, \ldots, \phi_K \in {\cal M}(L,W,S,B)$ such that, for sufficiently large $N$, we have
\begin{align*}
    &~ \int (\sum_{k=1}^K\|\phi_k(x, t) - \frac{\d^k}{\d t^k}x_t^\mathrm{true}\|_2^2) p_t(x) \d x \\ \lesssim &~ (\dot{\alpha}_t^2 \log N + \dot{\beta}_t^2 ) N^{- \frac{2s}{d}} +
    \sum_{k=1}^{K-1}\E_{x \sim P_t}[\|\frac{\d^k}{\d t^k}x_t^\mathrm{true} - \frac{\d^{k+1}}{\d t^{k+1}}x_t^\mathrm{true}\|_2^2] 
\end{align*}
    holds for any $t \in [T_{0}, 3T_{*}]$. In addition, for any $k \in [K]$, $\phi_k$ can be taken so we have
    \begin{align*}
         \|\phi_k(\cdot,t) \|_\infty = O(  |\dot{\alpha}_t | \sqrt{\log n} +  |\dot{\beta}_t |).
    \end{align*}
\end{theorem}
\begin{proof}

    We first show that for any $k \geq 2$, for any $t \in [T_0, 3T_*]$, there exists $\phi \in {\cal M}(L,W,S,B)$ such that
    \begin{align}
    \label{eq:tmp_3}
    &~ \int \|\phi(x, t) - \frac{\d^k}{\d t^k}{x}_t^\mathrm{true}\|^2_2 p_t(x) \d x \notag \\
    \lesssim &~ 
    (\dot{\alpha}_t^2 \log N + \dot{\beta}_t^2 ) N^{- \frac{2s}{d}} +
    \sum_{j=1}^{k}\E_{x \sim P_t}[\|\frac{\d^{j}}{\d t^{j}}x_t^\mathrm{true} - \frac{\d^{j+1}}{\d t^{j+1}}x_t^\mathrm{true}\|_2^2].
    \end{align}

    We prove this by mathematical induction.

    \textbf{Base case.} The statements hold when $k = 2$ because of Lemma~\ref{thm:secon_order_small_t:formal}.

    \textbf{Induction step.} We assume that the statement hold for $k \geq 2$. We would like to show that it holds for $k+1$. We can show that, for any $t \in [T_0, 3T_*]$, there exists $\phi \in {\cal M}(L,S,W, B)$ such that
    \begin{align}
    &~ \int \|\phi(x, t) - \frac{\d^{k+1}}{\d t^{k+1}}{x}_t^\mathrm{true}\|^2_2 p_t(x) \d x \notag
    \\ = &~
    \int \|\phi(x, t) - \frac{\d^{k}}{\d t^{k}}{x}_t^\mathrm{true} + \frac{\d^{k}}{\d t^{k}}{x}_t^\mathrm{true} - \frac{\d^{k+1}}{\d t^{k+1}}{x}_t^\mathrm{true}\|^2_2 p_t(x) \d x \notag \\
    \leq &~ \int (\|\phi(x, t) - \frac{\d^{k}}{\d t^{k}}{x}_t^\mathrm{true}\|_2 + \| \frac{\d^{k}}{\d t^{k}}{x}_t^\mathrm{true} - \frac{\d^{k+1}}{\d t^{k+1}}{x}_t^\mathrm{true}\|_2 )^2 p_t(x) \d x \notag \\
    \leq &~ \int 2(\|\phi(x, t) - \frac{\d^{k}}{\d t^{k}}{x}_t^\mathrm{true}\|_2^2 + \| \frac{\d^{k}}{\d t^{k}}{x}_t^\mathrm{true} - \frac{\d^{k+1}}{\d t^{k+1}}{x}_t^\mathrm{true}\|_2^2) p_t(x) \d x \notag \\
    = &~ 2\int \|\phi(x, t) - \frac{\d^{k}}{\d t^{k}}{x}_t^\mathrm{true}\|_2^2p_t(x) \d x + 2 \int \| \frac{\d^{k}}{\d t^{k}}{x}_t^\mathrm{true} - \frac{\d^{k+1}}{\d t^{k+1}}{x}_t^\mathrm{true}\|_2^2 p_t(x) \d x \notag \\
    = &~ 2\int \|\phi(x, t) - \frac{\d^{k}}{\d t^{k}}{x}_t^\mathrm{true}\|_2^2p_t(x) \d x + 2 \E_{x \sim P_t} [ \| \frac{\d^{k}}{\d t^{k}}{x}_t^\mathrm{true} - \frac{\d^{k+1}}{\d t^{k+1}}{x}_t^\mathrm{true}\|_2^2]\notag \\
    \lesssim &~ (\dot{\alpha}_t^2 \log N + \dot{\beta}_t^2 ) N^{- \frac{2s}{d}} +
    \sum_{j=1}^{k}\E_{x \sim P_t}[\|\frac{\d^{j}}{\d t^{j}}x_t^\mathrm{true} - \frac{\d^{j+1}}{\d t^{j+1}}x_t^\mathrm{true}\|_2^2] + \E_{x \sim P_t} [ \| \frac{\d^{k}}{\d t^{k}}{x}_t^\mathrm{true} - \frac{\d^{k+1}}{\d t^{k+1}}{x}_t^\mathrm{true}\|_2^2]\notag \\
    = &~ (\dot{\alpha}_t^2 \log N + \dot{\beta}_t^2 ) N^{- \frac{2s}{d}} +
    \sum_{j=1}^{k+1}\E_{x \sim P_t}[\|\frac{\d^{j}}{\d t^{j}}x_t^\mathrm{true} - \frac{\d^{j+1}}{\d t^{j+1}}x_t^\mathrm{true}\|_2^2], \label{eq:tmp_4}
    \end{align}
    where the first step follows from basic algebra, the second step follows from triangle inequality, the third step follows from the Cauchy-Schwarz inequality, the fourth step follows from basic algebra, the fifth step follows from the definition of expectation, the six step follows from Eq.~\eqref{eq:tmp_3}.
    
     Hence, there exists $\phi_1, \phi_2, \ldots, \phi_K \in {\cal M}(L,W,S,B)$ such that for $k \in [K]$, for any $t \in [T_0, 3T_*]$, we have
    \begin{align}
    &~ \int \|\phi_k(x, t) - \frac{\d^k}{\d t^k}{x}_t^\mathrm{true}\|^2_2 p_t(x) \d x \notag \\
    \lesssim &~ 
    (\dot{\alpha}_t^2 \log N + \dot{\beta}_t^2 ) N^{- \frac{2s}{d}} +
    \sum_{j=1}^{k}\E_{x \sim P_t}[\|\frac{\d^{j}}{\d t^{j}}x_t^\mathrm{true} - \frac{\d^{j+1}}{\d t^{j+1}}x_t^\mathrm{true}\|_2^2]. \label{eq:tmp_5}
    \end{align}

    Taking the summation over $k \in [K]$, we have for any $t \in [T_0, 3T_*]$,
    \begin{align*}
        &~ \int \sum_{k=1}^K \|\phi_k(x, t) - \frac{\d^k}{\d t^k}{x}_t^\mathrm{true}\|^2_2 p_t(x) \d x \notag \\
    \lesssim &~ 
    K \cdot (\dot{\alpha}_t^2 \log N + \dot{\beta}_t^2 ) N^{- \frac{2s}{d}} +
    \sum_{k=1}^{K} (k \cdot \E_{x \sim P_t}[\|\frac{\d^{j}}{\d t^{j}}x_t^\mathrm{true} - \frac{\d^{j+1}}{\d t^{j+1}}x_t^\mathrm{true}\|_2^2]) \\
    \lesssim &~ ((\dot{\alpha}_t)^2 \log N + (\dot{\beta}_t)^2 ) N^{- \frac{2s}{d}} +
    \sum_{k=1}^{K} \E_{x \sim P_t}[\|\frac{\d^{j}}{\d t^{j}}x_t^\mathrm{true} - \frac{\d^{j+1}}{\d t^{j+1}}x_t^\mathrm{true}\|_2^2]
    \end{align*}
    where the first step follows from Eq.~\eqref{eq:tmp_5}, and the second step uses $K=O(1)$.
    
    Moreover, by Lemma~\ref{lem:error_approx_large_t}, $\phi_1, \phi_2, \ldots, \phi_K$ can be taken so we have for $k \in [K]$,
    \begin{align*}
         \|\phi_k(\cdot,t) \|_\infty = O(  |\dot{\alpha}_t | \log \sqrt{n} +  |\dot{\beta}_t |).
    \end{align*}
    Thus, the proof is complete.
\end{proof}

\subsection{Approximation Error of Second Order Flow Matching for Large \texorpdfstring{$t$}{}}
\begin{theorem}[Approximation error of second order flow matching for large $t$, formal version of Theorem~\ref{thm:secon_order_large_t:informal}]\label{thm:secon_order_large_t:formal}
    Fix $t_{*} \in [T_{*},1]$ and let $\eta>0$ be arbitrary, under Assumptions~\ref{ass:A1}~\ref{ass:A2}~\ref{ass:A3}~\ref{ass:A4} and \ref{ass:A5}, and if the following holds 
    \begin{itemize}
        \item $L = O(\log^4 N )$.
        \item $\|W\|_{\infty} = O(N)$
        \item $S = O(t_{*}^{- d\kappa} N^{\delta\kappa})$
        \item $B = \exp(O (\log N \log \log N ) ).$
    \end{itemize}
  Then there exist neural networks $\phi_{1},\phi_2  \in  {\cal M}(L,W,S,B)$ such that
\begin{align*}
    &~ \int (\|\phi_1(x, t) - \dot{x}_t^\mathrm{true}\|_2^2 + \|\phi_2(x, t) - \ddot{x}_t^\mathrm{true}\|_2^2) p_t(x) \d x \\ \lesssim &~ (\dot{\alpha}_t^{2} \log N  +   \dot{\beta}_t^{2} ) N^{-\eta} +
    \E_{x \sim P_t}[\|\dot{x}^\mathrm{true}_t - \ddot{x}^\mathrm{true}_t\|_2^2]
\end{align*}
    holds for any $t \in [2t_*, 1]$. In addition, $\phi_1, \phi_2$ can be taken so we have
    \begin{align*}
         \|\phi_1(\cdot,t) \|_\infty = O(  |\dot{\alpha}_t | \log N +  |\dot{\beta}_t |) \mathrm{~~~and~~~} \|\phi_2(\cdot,t) \|_\infty = O(  |\dot{\alpha}_t | \log N +  |\dot{\beta}_t |).
    \end{align*}
\end{theorem}
\begin{proof}
    Suppose that $t \in [2t_*, 1]$.
    By Lemma~\ref{lem:error_approx_large_t}, there is $\phi_1 \in {\cal M}(L,W,S,B)$ such that
    \begin{align}
        \label{eq:tmp_1_large}
        \int (\|\phi_1(x, t) - \dot{x}_t^\mathrm{true}\|_2^2p_t(x)\d x \lesssim  (\dot{\alpha}_t^{2} \log N  +   \dot{\beta}_t^{2} ) N^{-\eta}.
    \end{align}
    
    Next, we can show that there exists some $\phi_2 \in {\cal M}(L,W,S,B)$ such that
    \begin{align}
         \int \|\phi_2(x, t) - \ddot{x}_t^\mathrm{true}\|_2^2 p_t(x) \d x 
         = &~\int \|\phi_2(x, t) - \dot{x}_t^\mathrm{true} +\dot{x}_t^\mathrm{true} - \ddot{x}_t^\mathrm{true}\|_2^2 p_t(x) \d x \notag \\
         \leq &~ \int (\|\phi_2(x, t) - \dot{x}_t^\mathrm{true}\|_2 + \| \dot{x}_t^\mathrm{true}-\ddot{x}_t^\mathrm{true}\|_2)^2 p_t(x) \d x \notag \\
         \leq &~ \int 2(\|\phi_2(x, t) - \dot{x}_t^\mathrm{true}\|_2^2 + \| \dot{x}_t^\mathrm{true}-\ddot{x}_t^\mathrm{true}\|_2^2) p_t(x) \d x \notag \\
         = &~ 2\int \|\phi_2(x, t) - \dot{x}_t^\mathrm{true}\|_2^2 p_t(x) \d x + 2\int \|_2 \dot{x}_t^\mathrm{true}-\ddot{x}_t^\mathrm{true}\|_2^2 p_t(x) \d x \notag \\
         = &~ 2\int \|\phi_2(x, t) - \dot{x}_t^\mathrm{true}\|_2^2p_t(x)\d x + 2\E_{x \sim P_t}[\|\dot{x}^\mathrm{true}_t - \ddot{x}^\mathrm{true}_t\|_2^2 \notag \\
         \lesssim &~ (\dot{\alpha}_t^{2} \log N  +   \dot{\beta}_t^{2} ) N^{-\eta} + \E_{x \sim P_t}[\|\dot{x}^\mathrm{true}_t - \ddot{x}^\mathrm{true}_t\|_2^2] \label{eq:tmp_2_large}
    \end{align}
    where the first step follows from the basic algebra, the second step follows from the triangle inequality, the third step follows from $(a+b)^2 \leq 2a^2 + 2b^2$, the fourth step follows from basic algebra, the fifth step follows from the definition of expectation, and the last step follows from Lemma~\ref{lem:error_approx_large_t}.

    Finally, by Eq.~\eqref{eq:tmp_1_large} and Eq.~\eqref{eq:tmp_2_large}, we have
    \begin{align*}
    &~ \int (\|\phi_1(x, t) - \dot{x}_t^\mathrm{true}\|^2 + \|\phi_2(x, t) - \ddot{x}_t^\mathrm{true}\|^2) p_t(x) \d x \\ \lesssim &~ (\dot{\alpha}_t^{2} \log N  +   \dot{\beta}_t^{2} ) N^{-\eta} +
    \E_{x \sim P_t}[\|\dot{x}^\mathrm{true}_t - \ddot{x}^\mathrm{true}_t\|^2].
    \end{align*}

    Moreover, by Lemma~\ref{lem:error_approx_large_t}, $\phi_1, \phi_2$ can be taken so we have
    \begin{align*}
         \|\phi_1(\cdot,t) \|_\infty = O(  |\dot{\alpha}_t | \log N +  |\dot{\beta}_t |) \mathrm{~~~and~~~} \|\phi_2(\cdot,t) \|_\infty = O(  |\dot{\alpha}_t |\log N +  |\dot{\beta}_t |).
    \end{align*}
    Thus, the proof is complete.
\end{proof}

\subsection{Approximation Error of Higher Order Flow Matching for Large \texorpdfstring{$t$}{}}
\begin{theorem}[Approximation error of higher order flow matching for large $t$]\label{thm:higher_order_large_t:formal}
    Fix $t_{*} \in [T_{*},1]$ and let $\eta>0$ be arbitrary, under Assumptions~\ref{ass:A1}~\ref{ass:A2}~\ref{ass:A3}~\ref{ass:A4} and \ref{ass:A5}, and if the following holds 
    \begin{itemize}
        \item $L = O(\log^4 N )$.
        \item $\|W\|_{\infty} = O(N)$
        \item $S = O(t_{*}^{- d\kappa} N^{\delta\kappa})$
        \item $B = \exp(O (\log N \log \log N ) )$ 
        \item $K = O(1)$
    \end{itemize}
  Then there exist neural networks $\phi_{1},\phi_2, \ldots, \phi_K \in {\cal M}(L,W,S,B)$ such that,
\begin{align*}
    &~ \int (\sum_{k=1}^K\|\phi_k(x, t) - \frac{\d^k}{\d t^k}x_t^\mathrm{true}\|^2) p_t(x) \d x \\ \lesssim &~ (\dot{\alpha}_t^{2} \log N  +   \dot{\beta}_t^{2} ) N^{-\eta} +
    \sum_{k=1}^{K-1}\E_{x \sim P_t}[\|\frac{\d^k}{\d t^k}x_t^\mathrm{true} - \frac{\d^{k+1}}{\d t^{k+1}}x_t^\mathrm{true}\|^2] 
\end{align*}
    holds for any $t \in [2t_*, 1]$. In addition, for any $k \in [K]$, $\phi_k$ can be taken so we have
    \begin{align*}
         \|\phi_k(\cdot,t) \|_\infty = O(  |\dot{\alpha}_t | \log N +  |\dot{\beta}_t |).
    \end{align*}
\end{theorem}
\begin{proof}
    We first show that for any $k \geq 2$, for any $t \in [2t_*,1]$, there exists $\phi \in {\cal M}(L,W,S,B)$ such that
    \begin{align}
    \label{eq:tmp_3_large}
    &~ \int \|\phi(x, t) - \frac{\d^k}{\d t^k}{x}_t^\mathrm{true}\|^2_2 p_t(x) \d x \notag \\
    \lesssim &~ 
    (\dot{\alpha}_t^{2} \log N  +   \dot{\beta}_t^{2} ) N^{-\eta} +
    \sum_{j=1}^{k}\E_{x \sim P_t}[\|\frac{\d^{j}}{\d t^{j}}x_t^\mathrm{true} - \frac{\d^{j+1}}{\d t^{j+1}}x_t^\mathrm{true}\|^2].
    \end{align}

    We prove this by mathematical induction.

    \textbf{Base case.} The statements hold when $k = 2$ because of Lemma~\ref{thm:secon_order_large_t:formal}.

    \textbf{Induction step.} We assume that the statement hold for $k \geq 2$. We would like to show that it holds for $k+1$. We can show that, for any $t \in [2t_*, 1]$, there exists $\phi \in {\cal M}(L,S,W, B)$ such that
    \begin{align}
    &~ \int \|\phi(x, t) - \frac{\d^{k+1}}{\d t^{k+1}}{x}_t^\mathrm{true}\|^2_2 p_t(x) \d x \notag
    \\ = &~
    \int \|\phi(x, t) - \frac{\d^{k}}{\d t^{k}}{x}_t^\mathrm{true} + \frac{\d^{k}}{\d t^{k}}{x}_t^\mathrm{true} - \frac{\d^{k+1}}{\d t^{k+1}}{x}_t^\mathrm{true}\|^2_2 p_t(x) \d x \notag \\
    \leq &~ \int (\|\phi(x, t) - \frac{\d^{k}}{\d t^{k}}{x}_t^\mathrm{true}\|_2 + \| \frac{\d^{k}}{\d t^{k}}{x}_t^\mathrm{true} - \frac{\d^{k+1}}{\d t^{k+1}}{x}_t^\mathrm{true}\|_2 )^2 p_t(x) \d x \notag \\
    \leq &~ \int 2(\|\phi(x, t) - \frac{\d^{k}}{\d t^{k}}{x}_t^\mathrm{true}\|_2^2 + \| \frac{\d^{k}}{\d t^{k}}{x}_t^\mathrm{true} - \frac{\d^{k+1}}{\d t^{k+1}}{x}_t^\mathrm{true}\|_2^2) p_t(x) \d x \notag \\
    = &~ 2\int \|\phi(x, t) - \frac{\d^{k}}{\d t^{k}}{x}_t^\mathrm{true}\|_2^2p_t(x) \d x + 2 \int \| \frac{\d^{k}}{\d t^{k}}{x}_t^\mathrm{true} - \frac{\d^{k+1}}{\d t^{k+1}}{x}_t^\mathrm{true}\|_2^2 p_t(x) \d x \notag \\
    = &~ 2\int \|\phi(x, t) - \frac{\d^{k}}{\d t^{k}}{x}_t^\mathrm{true}\|_2^2p_t(x) \d x + 2 \E_{x \sim P_t} [ \| \frac{\d^{k}}{\d t^{k}}{x}_t^\mathrm{true} - \frac{\d^{k+1}}{\d t^{k+1}}{x}_t^\mathrm{true}\|_2^2]\notag \\
    \lesssim &~(\dot{\alpha}_t^{2} \log N  +   \dot{\beta}_t^{2} ) N^{-\eta} +
    \sum_{j=1}^{k}\E_{x \sim P_t}[\|\frac{\d^{j}}{\d t^{j}}x_t^\mathrm{true} - \frac{\d^{j+1}}{\d t^{j+1}}x_t^\mathrm{true}\|_2^2] + \E_{x \sim P_t} [ \| \frac{\d^{k}}{\d t^{k}}{x}_t^\mathrm{true} - \frac{\d^{k+1}}{\d t^{k+1}}{x}_t^\mathrm{true}\|_2^2]\notag \\
    = &~ (\dot{\alpha}_t^{2} \log N  +   \dot{\beta}_t^{2} ) N^{-\eta}+
    \sum_{j=1}^{k+1}\E_{x \sim P_t}[\|\frac{\d^{j}}{\d t^{j}}x_t^\mathrm{true} - \frac{\d^{j+1}}{\d t^{j+1}}x_t^\mathrm{true}\|_2^2], \label{eq:tmp_4_large}
    \end{align}
    where the first step follows from basic algebra, the second step follows from triangle inequality, the third step follows from the Cauchy-Schwarz inequality, the fourth step follows from basic algebra, the fifth step follows from the definition of expectation, the six step follows from Eq.~\eqref{eq:tmp_3_large}.
    
     Hence, there exists $\phi_1, \phi_2, \ldots, \phi_K \in {\cal M}(L,W,S,B)$ such that for $k \in [K]$, for any $t \in [2t_*, 1]$, we have
    \begin{align}
    &~ \int \|\phi_k(x, t) - \frac{\d^k}{\d t^k}{x}_t^\mathrm{true}\|^2_2 p_t(x) \d x \notag \\
    \lesssim &~ 
    (\dot{\alpha}_t^{2} \log N  +   \dot{\beta}_t^{2} ) N^{-\eta} +
    \sum_{j=1}^{k}\E_{x \sim P_t}[\|\frac{\d^{j}}{\d t^{j}}x_t^\mathrm{true} - \frac{\d^{j+1}}{\d t^{j+1}}x_t^\mathrm{true}\|_2^2]. \label{eq:tmp_5_large}
    \end{align}

    Taking the summation over $k \in [K]$, we have for any $t \in [2 t_*, 1]$,
    \begin{align*}
        &~ \int \sum_{k=1}^K \|\phi_k(x, t) - \frac{\d^k}{\d t^k}{x}_t^\mathrm{true}\|^2_2 p_t(x) \d x \notag \\
    \lesssim &~ 
    ((\dot{\alpha}_t )^{2} \log N  +   (\dot{\beta}_t )^{2} ) N^{-\eta} +
    \sum_{k=1}^{K} (k \cdot \E_{x \sim P_t}[\|\frac{\d^{j}}{\d t^{j}}x_t^\mathrm{true} - \frac{\d^{j+1}}{\d t^{j+1}}x_t^\mathrm{true}\|_2^2]) \\
    \lesssim &~ (\dot{\alpha}_t^{2} \log N  +   \dot{\beta}_t^{2} ) N^{-\eta} +
    \sum_{k=1}^{K} \E_{x \sim P_t}[\|\frac{\d^{j}}{\d t^{j}}x_t^\mathrm{true} - \frac{\d^{j+1}}{\d t^{j+1}}x_t^\mathrm{true}\|_2^2]
    \end{align*}
    where the first step follows from Eq.~\eqref{eq:tmp_5_large}, and the second step uses $K=O(1)$.
   Moreover, by Lemma~\ref{lem:error_approx_large_t}, $\phi_1, \phi_2, \ldots, \phi_K$ can be taken so we have for $k \in [K]$,
    \begin{align*}
         \|\phi_k(\cdot,t) \|_\infty = O(  |\dot{\alpha}_t | \log N +  |\dot{\beta}_t |).
    \end{align*}
    Thus, the proof is complete.
\end{proof}
\section{Empirical Ablation Study} \label{sec:app:empirical_ablation_study}
In Section~\ref{sec:app:dataset}, we introduce the three Gaussian mixture distribution datasets—four-mode, five-mode, and eight-mode—used in our empirical ablation study, along with their configurations for source and target modes. The subsequent subsections analyze the impact of different optimization terms. Section~\ref{sec:app:rectified_flow} evaluates the performance of HOMO optimized solely with the first-order term. Section~\ref{sec:app:second_order} examines the effect of using only the second-order term. Section~\ref{sec:app:self_consistency} assesses results when optimization is guided by the self-consistency term. Section~\ref{sec:app:first_order_plus_second_order} explores the combined effect of first- and second-order terms, while Section~\ref{sec:app:second_order_plus_self_target} investigates the combination of second-order and self-consistency terms. Through these analyses, we aim to dissect the contributions of individual and combined loss terms in achieving effective transport trajectories.
\subsection{Dataset}\label{sec:app:dataset}

Here we introduce three datasets we use: four-mode, five-mode, and eight-mode Gaussian mixture distribution datasets; each Gaussian component has a variance of $0.3$. In the four-mode Gaussian mixture distribution, four source mode(\textbf{brown}) positioned at a distance $D_0 = 5$ from the origin, and four target mode(\textbf{indigo}) positioned at a distance $D_0 = 14$ from the origin, each mode sample 200 points. In five-mode Gaussian mixture distribution, five source mode(\textbf{brown}) positioned at a distance $D_0 = 6$ from the origin, and five target mode(\textbf{indigo}) positioned at a distance $D_0 = 13$ from the origin, each mode sample 200 points. And in eight-mode Gaussian mixture distribution, eight source mode(\textbf{brown}) positioned at a distance $D_0 = 6$ from the origin, and eight target mode(\textbf{indigo}) positioned at a distance $D_0 = 13$ from the origin, each mode sample 100 points. 

\begin{figure}[!ht] 
\centering
\includegraphics[width=0.25\textwidth]{4_dataset.pdf}
\includegraphics[width=0.25\textwidth]{5_dataset.pdf}
\includegraphics[width=0.25\textwidth]{8_dataset.pdf}
\caption{
The four-mode Gaussian mixture distribution (\textbf{Left}), five-mode Gaussian mixture distribution (\textbf{Middle}), and eight-mode Gaussian mixture distribution (\textbf{Right}). Our goal is to make HOMO learn a transport trajectory from distribution $\pi_0$ ({\textbf{brown}}) to distribution $\pi_1$ ({\textbf{indigo}}). 
}
\label{fig:three_normal_dataset}
\end{figure}

\subsection{Only First Order Term}\label{sec:app:rectified_flow}

We optimize models by the sum of squared error(SSE). The source distribution and target distribution are all Gaussian distributions. For the target transport trajectory setting, we follow the VP ODE framework from~\cite{rectified_flow}, which is $x_t = \alpha_t x_0 + \beta_t x_1$. We choose $\alpha_t = \exp(-\frac{1}{4} a(1-t)^2 - \frac{1}{2} b(1-t))$ and $\beta_t = \sqrt{1 - \alpha_t^2}$, with hyperparameters $a = 19.9$ and $b = 0.1$. In the four-mode dataset, five-mode dataset, and eight-mode dataset, we all sample 100 points in each source mode and target mode. And in four-mode dataset training, we use an ODE solver and Adam optimizer, with 2 hidden layer MLP, 100 hidden dimensions, $800$ batch size, $0.005$ learning rate, and $1000$ training steps. In five-mode dataset training, we also use an ODE solver and Adam optimizer, with 2 hidden layer MLP, 100 hidden dimensions, $1000$ batch size, $0.005$ learning rate, and $1000$ training steps. And in eight-mode dataset training, we use an ODE solver and Adam optimizer, with 2 hidden layer MLP, 100 hidden dimensions, $1600$ batch size, $0.005$ learning rate, and $1000$ training steps. 


\begin{figure}[!ht]
\centering
\includegraphics[width=0.25\textwidth]{4_1_output.pdf}
\includegraphics[width=0.25\textwidth]{5_1_output.pdf}
\includegraphics[width=0.25\textwidth]{8_1_output.pdf}
\caption{
(A) The distributions generated by HOMO are only optimized by first-order term in four-mode dataset (\textbf{Left}), five-mode dataset (\textbf{Middle}), and eight-mode dataset (\textbf{Right}). 
The source distribution, $\pi_0$ ({\textbf{brown}}), and the target distribution, $\pi_1$ ({\textbf{indigo}}), are shown, along with the generated distribution ({\textbf{pink}}). 
}
\label{fig:1_distribution}
\end{figure}


\subsection{Only Second Order Term}\label{sec:app:second_order}
We optimize models by the sum of squared error(SSE). The source distribution and target distribution are all Gaussian distributions. For the target transport trajectory setting, we follow the VP ODE framework from~\cite{rectified_flow}, which is $x_t = \alpha_t x_0 + \beta_t x_1$. We choose $\alpha_t = \exp(-\frac{1}{4} a(1-t)^2 - \frac{1}{2} b(1-t))$ and $\beta_t = \sqrt{1 - \alpha_t^2}$, with hyperparameters $a = 19.9$ and $b = 0.1$. In the four-mode dataset, five-mode dataset, and eight-mode dataset, we all sample 100 points in each source mode and target mode. And in four-mode dataset training, we use an ODE solver and Adam optimizer, with 2 hidden layer MLP, 100 hidden dimensions, $800$ batch size, $0.005$ learning rate, and $100$ training steps. In five-mode dataset training, we also use an ODE solver and Adam optimizer, with 2 hidden layer MLP, 100 hidden dimensions, $1000$ batch size, $0.005$ learning rate, and $100$ training steps. And in eight-mode dataset training, we use an ODE solver and Adam optimizer, with 2 hidden layer MLP, 100 hidden dimensions, $1600$ batch size, $0.005$ learning rate, and $100$ training steps. 


\begin{figure}[!ht]
\centering
\includegraphics[width=0.25\textwidth]{4_2_output.pdf}
\includegraphics[width=0.25\textwidth]{5_2_output.pdf}
\includegraphics[width=0.25\textwidth]{8_2_output.pdf}
\caption{
(B) The distributions generated by HOMO are only optimized by second-order term in the four-mode dataset (\textbf{Left}), five-mode dataset (\textbf{Middle}), and eight-mode dataset (\textbf{Right}). 
The source distribution, $\pi_0$ ({\textbf{brown}}), and the target distribution, $\pi_1$ ({\textbf{indigo}}), are shown, along with the generated distribution ({\textbf{pink}}). 
}
\label{fig:2_distribution}
\end{figure}

\subsection{Only Self-Consistency Term}\label{sec:app:self_consistency}
We optimize models by the sum of squared error(SSE). The source distribution and target distribution are all Gaussian distributions. For the target transport trajectory setting, we follow the VP ODE framework from~\cite{rectified_flow}, which is $x_t = \alpha_t x_0 + \beta_t x_1$. We choose $\alpha_t = \exp(-\frac{1}{4} a(1-t)^2 - \frac{1}{2} b(1-t))$ and $\beta_t = \sqrt{1 - \alpha_t^2}$, with hyperparameters $a = 19.9$ and $b = 0.1$. In the four-mode dataset, five-mode dataset, and eight-mode dataset, we all sample 100 points in each source mode and target mode. And in four-mode dataset training, we use an ODE solver and Adam optimizer, with 2 hidden layer MLP, 100 hidden dimensions, $800$ batch size, $0.005$ learning rate, and $50$ training steps. In five-mode dataset training, we also use an ODE solver and Adam optimizer, with 2 hidden layer MLP, 100 hidden dimensions, $1000$ batch size, $0.005$ learning rate, and $50$ training steps. And in eight-mode dataset training, we use an ODE solver and Adam optimizer, with 2 hidden layer MLP, 100 hidden dimensions, $1600$ batch size, $0.005$ learning rate, and $50$ training steps. 
\begin{figure}[!ht]
\centering
\includegraphics[width=0.25\textwidth]{4_3_output.pdf}
\includegraphics[width=0.25\textwidth]{5_3_output.pdf}
\includegraphics[width=0.25\textwidth]{8_3_output.pdf}
\caption{
(C) The distributions generated by HOMO are only optimized by self-consistency term in the four-mode dataset (\textbf{Left}), five-mode dataset (\textbf{Middle}), and eight-mode dataset (\textbf{Right}). 
The source distribution, $\pi_0$ ({\textbf{brown}}), and the target distribution, $\pi_1$ ({\textbf{indigo}}), are shown, along with the generated distribution ({\textbf{pink}}). 
}
\label{fig:3_distribution}
\end{figure}

\subsection{First Order Plus Second Order}\label{sec:app:first_order_plus_second_order}
We optimize models by the sum of squared error(SSE). The source distribution and target distribution are all Gaussian distributions. For the target transport trajectory setting, we follow the VP ODE framework from~\cite{rectified_flow}, which is $x_t = \alpha_t x_0 + \beta_t x_1$. We choose $\alpha_t = \exp(-\frac{1}{4} a(1-t)^2 - \frac{1}{2} b(1-t))$ and $\beta_t = \sqrt{1 - \alpha_t^2}$, with hyperparameters $a = 19.9$ and $b = 0.1$. In the four-mode dataset, five-mode dataset, and eight-mode dataset, we all sample 100 points in each source mode and target mode. And in four-mode dataset training, we use an ODE solver and Adam optimizer, with 2 hidden layer MLP, 100 hidden dimensions, $800$ batch size, $0.005$ learning rate, and $1000$ training steps. In five-mode dataset training, we also use an ODE solver and Adam optimizer, with 2 hidden layer MLP, 100 hidden dimensions, $1000$ batch size, $0.005$ learning rate, and $2000$ training steps. And in eight-mode dataset training, we use an ODE solver and Adam optimizer, with 2 hidden layer MLP, 100 hidden dimensions, $1600$ batch size, $0.005$ learning rate, and $2000$ training steps. 
\begin{figure}[!ht]
\centering
\includegraphics[width=0.25\textwidth]{4_12_output.pdf}
\includegraphics[width=0.25\textwidth]{5_12_output.pdf}
\includegraphics[width=0.25\textwidth]{8_12_output.pdf}
\caption{
(A + B) The distributions generated by HOMO, optimized by first-order term and second-order term in four-mode dataset (\textbf{Left}), five-mode dataset (\textbf{Middle}), and eight-mode dataset (\textbf{Right}). 
The source distribution, $\pi_0$ ({\textbf{brown}}), and the target distribution, $\pi_1$ ({\textbf{indigo}}), are shown, along with the generated distribution ({\textbf{pink}}). 
}
\label{fig:12_distribution}
\end{figure}


\subsection{Second Order Plus Self-Target}\label{sec:app:second_order_plus_self_target}
We optimize models by the sum of squared error(SSE). The source distribution and target distribution are all Gaussian distributions. For the target transport trajectory setting, we follow the VP ODE framework from~\cite{rectified_flow}, which is $x_t = \alpha_t x_0 + \beta_t x_1$. We choose $\alpha_t = \exp(-\frac{1}{4} a(1-t)^2 - \frac{1}{2} b(1-t))$ and $\beta_t = \sqrt{1 - \alpha_t^2}$, with hyperparameters $a = 19.9$ and $b = 0.1$. In the four-mode dataset, five-mode dataset, and eight-mode dataset, we all sample 100 points in each source mode and target mode. And in four-mode dataset training, we use an ODE solver and Adam optimizer, with 2 hidden layer MLP, 100 hidden dimensions, $800$ batch size, $0.005$ learning rate, and $100$ training steps. In five-mode dataset training, we also use an ODE solver and Adam optimizer, with 2 hidden layer MLP, 100 hidden dimensions, $1000$ batch size, $0.005$ learning rate, and $100$ training steps. And in eight-mode dataset training, we use an ODE solver and Adam optimizer, with 2 hidden layer MLP, 100 hidden dimensions, $1600$ batch size, $0.005$ learning rate, and $100$ training steps. 
\begin{figure}[!ht]
\centering
\includegraphics[width=0.25\textwidth]{4_23_output.pdf}
\includegraphics[width=0.25\textwidth]{5_23_output.pdf}
\includegraphics[width=0.25\textwidth]{8_23_output.pdf}
\caption{
(B + C) The distributions generated by HOMO, optimized by second-order term and self-consistency term in four-mode dataset (\textbf{Left}), five-mode dataset (\textbf{Middle}), and eight-mode dataset (\textbf{Right}). 
The source distribution, $\pi_0$ ({\textbf{brown}}), and the target distribution, $\pi_1$ ({\textbf{indigo}}), are shown, along with the generated distribution ({\textbf{pink}}). 
}
\label{fig:23_distribution}
\end{figure}

\subsection{First Order Plus Self-Target}\label{sec:app:first_order_plus_self_target}
We optimize models by the sum of squared error(SSE). The source distribution and target distribution are all Gaussian distributions. For the target transport trajectory setting, we follow the VP ODE framework from~\cite{rectified_flow}, which is $x_t = \alpha_t x_0 + \beta_t x_1$. We choose $\alpha_t = \exp(-\frac{1}{4} a(1-t)^2 - \frac{1}{2} b(1-t))$ and $\beta_t = \sqrt{1 - \alpha_t^2}$, with hyperparameters $a = 19.9$ and $b = 0.1$. In the four-mode dataset, five-mode dataset, and eight-mode dataset, we all sample 100 points in each source mode and target mode. And in four-mode dataset training, we use an ODE solver and Adam optimizer, with 2 hidden layer MLP, 100 hidden dimensions, $800$ batch size, $0.005$ learning rate, and $1000$ training steps. In five-mode dataset training, we also use an ODE solver and Adam optimizer, with 2 hidden layer MLP, 100 hidden dimensions, $1000$ batch size, $0.005$ learning rate, and $1000$ training steps. And in eight-mode dataset training, we use an ODE solver and Adam optimizer, with 2 hidden layer MLP, 100 hidden dimensions, $1600$ batch size, $0.005$ learning rate, and $1000$ training steps. 
\begin{figure}[!ht]
\centering
\includegraphics[width=0.25\textwidth]{4_13_output.pdf}
\includegraphics[width=0.25\textwidth]{5_13_output.pdf}
\includegraphics[width=0.25\textwidth]{8_13_output.pdf}
\caption{
(A + C) The distributions generated by HOMO, optimized by first-order term and self-consistency term in four-mode dataset (\textbf{Left}), five-mode dataset (\textbf{Middle}), and eight-mode dataset (\textbf{Right}). 
The source distribution, $\pi_0$ ({\textbf{brown}}), and the target distribution, $\pi_1$ ({\textbf{indigo}}), are shown, along with the generated distribution ({\textbf{pink}}). 
}
\label{fig:13_distribution}
\end{figure}

\subsection{HOMO}\label{sec:app:homo}
We optimize models by the sum of squared error(SSE). The source distribution and target distribution are all Gaussian distributions. For the target transport trajectory setting, we follow the VP ODE framework from~\cite{rectified_flow}, which is $x_t = \alpha_t x_0 + \beta_t x_1$. We choose $\alpha_t = \exp(-\frac{1}{4} a(1-t)^2 - \frac{1}{2} b(1-t))$ and $\beta_t = \sqrt{1 - \alpha_t^2}$, with hyperparameters $a = 19.9$ and $b = 0.1$. In the four-mode dataset, five-mode dataset, and eight-mode dataset, we all sample 100 points in each source mode and target mode. And in four-mode dataset training, we use an ODE solver and Adam optimizer, with 2 hidden layer MLP, 100 hidden dimensions, $800$ batch size, $0.005$ learning rate, and $1000$ training steps. In five-mode dataset training, we also use an ODE solver and Adam optimizer, with 2 hidden layer MLP, 100 hidden dimensions, $1000$ batch size, $0.005$ learning rate, and $1000$ training steps. And in eight-mode dataset training, we use an ODE solver and Adam optimizer, with 2 hidden layer MLP, 100 hidden dimensions, $1600$ batch size, $0.005$ learning rate, and $1000$ training steps. 
\begin{figure}[!ht]
\centering
\includegraphics[width=0.25\textwidth]{4_123_output.pdf}
\includegraphics[width=0.25\textwidth]{5_123_output.pdf}
\includegraphics[width=0.25\textwidth]{8_123_output.pdf}
\caption{
(A + B + C) The distributions generated by HOMO in four-mode dataset (\textbf{Left}), five-mode dataset (\textbf{Middle}), and eight-mode dataset (\textbf{Right}). 
The source distribution, $\pi_0$ ({\textbf{brown}}), and the target distribution, $\pi_1$ ({\textbf{indigo}}), are shown, along with the generated distribution ({\textbf{pink}}). 
}
\label{fig:123_distribution}
\end{figure}


\section{Complex Distribution Experiment}\label{sec:app:complex_distribution_experiment}
In Section~\ref{sec:app:datasets2}, we introduce the datasets used in our experiments. The analysis of results with first-order and second-order terms in Section~\ref{sec:app:first_second}, and we evaluate the performance with first-order and self-consistency terms in Section~\ref{sec:app:first_third}, assess the impact of second-order and self-consistency terms in Section~\ref{sec:app:second_third}. Finally, we present the overall results of HOMO with all loss terms combined in Section~\ref{sec:app:homo2}.

\subsection{Datasets}\label{sec:app:datasets2}
Here, we introduce four datasets we proposed: circle dataset, irregular ring dataset, spiral line dataset, and spin dataset. In the circle dataset, we sample 600 points from Gaussian distribution with $0.3$ variance for both source distribution and target distribution. In the irregular ring dataset, we sample 600 points from Gaussian distribution with $0.3$ variance for both source distribution and target distribution. In the spiral line dataset, we sample 600 points from Gaussian distribution with $0.3$ variance for both source distribution and target distribution. In the spin dataset, we sample 600 points from the Gaussian distribution with $0.3$ variance for both source distribution and target distribution. 
\begin{figure}[!ht]
\centering
\includegraphics[width=0.2\textwidth]{center_to_circle_dataset.pdf}
\includegraphics[width=0.2\textwidth]{center_to_irregular_ring_dataset.pdf}
\includegraphics[width=0.2\textwidth]{spiral_dataset.pdf}
\includegraphics[width=0.2\textwidth]{new_spiral_dataset.pdf}
\caption{
The circle dataset(\textbf{Left most}), irregular ring dataset (\textbf{Middle left}), spiral line dataset (\textbf{Middle right}), and spin dataset (\textbf{Right most}). 
Our goal is to make HOMO to learn a transport trajectory from distribution $\pi_0$ ({\textbf{brown}}) to distribution $\pi_1$ ({\textbf{indigo}}). 
}
\label{fig:datasets}
\end{figure}

\subsection{First Order Plus Second Order}\label{sec:app:first_second}
We optimize models by the sum of squared error(SSE). The source distribution and target distribution are all Gaussian distributions. For the target transport trajectory setting, we follow the VP ODE framework from~\cite{rectified_flow}, which is $x_t = \alpha_t x_0 + \beta_t x_1$. We choose $\alpha_t = \exp(-\frac{1}{4} a(1-t)^2 - \frac{1}{2} b(1-t))$ and $\beta_t = \sqrt{1 - \alpha_t^2}$, with hyperparameters $a = 19.9$ and $b = 0.1$. In the circle dataset, we all sample 400 points, both source distribution and target distribution. In the irregular ring dataset, we all sample 600 points, both source distribution and target distribution. In the spiral line dataset, we all sample 300 points, both source distribution and target distribution. In circle dataset training, we use an ODE solver and Adam optimizer, with 2 hidden layer MLP, 100 hidden dimensions, $800$ batch size, $0.005$ learning rate, and $1000$ training steps. In irregular ring dataset training, we also use an ODE solver and Adam optimizer, with 2 hidden layer MLP, 100 hidden dimensions, $1000$ batch size, $0.005$ learning rate, and $1000$ training steps. In spiral line dataset training, we use an ODE solver and Adam optimizer, with 2 hidden layer MLP, 100 hidden dimensions, $1600$ batch size, $0.005$ learning rate, and $1000$ training steps. In spiral line dataset training, we use an ODE solver and Adam optimizer, with 2 hidden layer MLP, 100 hidden dimensions, $1600$ batch size, $0.005$ learning rate, and $1000$ training steps. 
\begin{figure}[!ht]
\centering
\includegraphics[width=0.2\textwidth]{12_circle_output.pdf}
\includegraphics[width=0.2\textwidth]{12_irr_circle_output.pdf}
\includegraphics[width=0.2\textwidth]{12_spiral_output.pdf}
\includegraphics[width=0.2\textwidth]{12_new_spiral_output.pdf}
\caption{
(M1+M2) \textbf{HOMO results on complex datasets with two kinds of loss: first-order and second-order terms.} The distributions generated by HOMO,
in circle dataset(\textbf{Left most}), irregular ring dataset (\textbf{Middle left}), spiral line dataset (\textbf{Middle right}) and spin dataset (\textbf{Right most}).  
The source distribution, $\pi_0$ ({\textbf{brown}}), and the target distribution, $\pi_1$ ({\textbf{indigo}}), are shown, along with the generated distribution ({\textbf{pink}}). }
\label{fig:m1_m2_appendix}
\end{figure}

\subsection{First Order Plus Self-Consistency Term}\label{sec:app:first_third}
We optimize models by the sum of squared error(SSE). The source distribution and target distribution are all Gaussian distributions. For the target transport trajectory setting, we follow the VP ODE framework from~\cite{rectified_flow}, which is $x_t = \alpha_t x_0 + \beta_t x_1$. We choose $\alpha_t = \exp(-\frac{1}{4} a(1-t)^2 - \frac{1}{2} b(1-t))$ and $\beta_t = \sqrt{1 - \alpha_t^2}$, with hyperparameters $a = 19.9$ and $b = 0.1$. In the circle dataset, we all sample 400 points, both source distribution and target distribution. In the irregular ring dataset, we all sample 600 points, both source distribution and target distribution. In the spiral line dataset, we all sample 300 points, both source distribution and target distribution. In circle dataset training, we use an ODE solver and Adam optimizer, with 2 hidden layer MLP, 100 hidden dimensions, $800$ batch size, $0.005$ learning rate, and $1000$ training steps. In irregular ring dataset training, we also use an ODE solver and Adam optimizer, with 2 hidden layer MLP, 100 hidden dimensions, $1000$ batch size, $0.005$ learning rate, and $1000$ training steps. In spiral line dataset training, we use an ODE solver and Adam optimizer, with 2 hidden layer MLP, 100 hidden dimensions, $1600$ batch size, $0.005$ learning rate, and $1000$ training steps. In spiral line dataset training, we use an ODE solver and Adam optimizer, with 2 hidden layer MLP, 100 hidden dimensions, $1600$ batch size, $0.005$ learning rate, and $1000$ training steps. 
\begin{figure}[!ht]
\centering
\includegraphics[width=0.2\textwidth]{13_circle_output.pdf}
\includegraphics[width=0.2\textwidth]{13_irr_circle_output.pdf}
\includegraphics[width=0.2\textwidth]{13_spiral_output.pdf}
\includegraphics[width=0.2\textwidth]{13_new_spiral_output.pdf}
\caption{
(M1+SC) \textbf{HOMO results on complex datasets with two kinds of loss: first-order and self-consistency terms.} The distributions generated by HOMO,
in circle dataset(\textbf{Left most}), irregular ring dataset (\textbf{Middle left}), spiral line dataset (\textbf{Middle right}) and spin dataset (\textbf{Right most}). 
The source distribution, $\pi_0$ ({\textbf{brown}}), and the target distribution, $\pi_1$ ({\textbf{indigo}}), are shown, along with the generated distribution ({\textbf{pink}}). }
\label{fig:m1_sc_appendix}
\end{figure}

\subsection{Second Order Plus Self-Consistency Term}\label{sec:app:second_third}
We optimize models by the sum of squared error(SSE). The source distribution and target distribution are all Gaussian distributions. For the target transport trajectory setting, we follow the VP ODE framework from~\cite{rectified_flow}, which is $x_t = \alpha_t x_0 + \beta_t x_1$. We choose $\alpha_t = \exp(-\frac{1}{4} a(1-t)^2 - \frac{1}{2} b(1-t))$ and $\beta_t = \sqrt{1 - \alpha_t^2}$, with hyperparameters $a = 19.9$ and $b = 0.1$. In the circle dataset, we all sample 400 points, both source distribution and target distribution. In the irregular ring dataset, we all sample 600 points, both source distribution and target distribution. In the spiral line dataset, we all sample 300 points, both source distribution and target distribution. And in circle dataset training, we use an ODE solver and Adam optimizer, with 2 hidden layer MLP, 100 hidden dimensions, $800$ batch size, $0.005$ learning rate, and $100$ training steps. In irregular ring dataset training, we also use an ODE solver and Adam optimizer, with 2 hidden layer MLP, 100 hidden dimensions, $100$ batch size, $0.005$ learning rate, and $1000$ training steps. In spiral line dataset training, we use an ODE solver and Adam optimizer, with 2 hidden layer MLP, 100 hidden dimensions, $1600$ batch size, $0.005$ learning rate, and $100$ training steps. In spiral line dataset training, we use an ODE solver and Adam optimizer, with 2 hidden layer MLP, 100 hidden dimensions, $1600$ batch size, $0.005$ learning rate, and $1000$ training steps. 
\begin{figure}[!ht]
\centering
\includegraphics[width=0.2\textwidth]{23_circle_output.pdf}
\includegraphics[width=0.2\textwidth]{23_irr_circle_output.pdf}
\includegraphics[width=0.2\textwidth]{23_spiral_output.pdf}
\includegraphics[width=0.2\textwidth]{23_new_spiral_output.pdf}
\caption{
(M2+SC) \textbf{HOMO results on complex datasets with two kinds of loss: second-order and self-consistency terms.} The distributions generated by HOMO,
in circle dataset(\textbf{Left most}), irregular ring dataset (\textbf{Middle left}), spiral line dataset (\textbf{Middle right}) and spin dataset (\textbf{Right most}). 
The source distribution, $\pi_0$ ({\textbf{brown}}), and the target distribution, $\pi_1$ ({\textbf{indigo}}), are shown, along with the generated distribution ({\textbf{pink}}). }
\label{fig:m2_sc_appendix}
\end{figure}

\subsection{HOMO}\label{sec:app:homo2}
We optimize models by the sum of squared error(SSE). The source distribution and target distribution are all Gaussian distributions. For the target transport trajectory setting, we follow the VP ODE framework from~\cite{rectified_flow}, which is $x_t = \alpha_t x_0 + \beta_t x_1$. We choose $\alpha_t = \exp(-\frac{1}{4} a(1-t)^2 - \frac{1}{2} b(1-t))$ and $\beta_t = \sqrt{1 - \alpha_t^2}$, with hyperparameters $a = 19.9$ and $b = 0.1$. In the circle dataset, we all sample 400 points, both source distribution and target distribution. In the irregular ring dataset, we all sample 600 points, both source distribution and target distribution. In the spiral line dataset, we all sample 300 points, both source distribution and target distribution. And in circle dataset training, we use an ODE solver and Adam optimizer, with 2 hidden layer MLP, 100 hidden dimensions, $800$ batch size, $0.005$ learning rate, and $1000$ training steps. In irregular ring dataset training, we also use an ODE solver and Adam optimizer, with 2 hidden layer MLP, 100 hidden dimensions, $1000$ batch size, $0.005$ learning rate, and $1000$ training steps. In spiral line dataset training, we use an ODE solver and Adam optimizer, with 2 hidden layer MLP, 100 hidden dimensions, $1600$ batch size, $0.005$ learning rate, and $1000$ training steps. In spiral line dataset training, we use an ODE solver and Adam optimizer, with 2 hidden layer MLP, 100 hidden dimensions, $1600$ batch size, $0.005$ learning rate, and $1000$ training steps. 
\begin{figure}[!ht]
\centering
\includegraphics[width=0.2\textwidth]{123_circle_output.pdf}
\includegraphics[width=0.2\textwidth]{123_irr_circle_output.pdf}
\includegraphics[width=0.2\textwidth]{123_spiral_output.pdf}
\includegraphics[width=0.2\textwidth]{123_new_spiral_output.pdf}
\caption{
(M1+M2+SC) \textbf{HOMO results on complex datasets with three kinds of loss: first-order, second-order, and self-consistency terms.} The distributions generated by HOMO in circle dataset(\textbf{Left most}), irregular ring dataset (\textbf{Middle left}), spiral line dataset (\textbf{Middle right}), and spin dataset (\textbf{Right most}). 
The source distribution, $\pi_0$ ({\textbf{brown}}), and the target distribution, $\pi_1$ ({\textbf{indigo}}), are shown, along with the generated distribution ({\textbf{pink}}). }
\label{fig:m1_m2_sc_appendix}
\end{figure}

\section{Third-Order HOMO}\label{sec:app:3rd_homo}
This section extends HOMO to third-order dynamics and analyzes its performance on complex synthetic tasks. Section~\ref{sec:app:3rd_homo_algorithm} introduces the training and sampling algorithms incorporating third-order dynamics. Section~\ref{sec:app:trajectory_setting} compares two trajectory parameterization strategies for high-order systems. Section~\ref{sec:app:complex_dataset} describes the 2 Round Spin, 3 Round Spin, and Dot-Circle datasets designed to test complex mode transitions. Section~\ref{sec:app:euclidean_distance_loss} provides quantitative analysis through Euclidean distance metrics between generated and target distributions. Section~\ref{sec:app:3rd_self_consistency} evaluates the isolated impact of self-consistency constraints. Section~\ref{sec:app:3rd_first_order_plus_self_consistency} examines first-order dynamics coupled with self-consistency regularization. Section~\ref{sec:app:3rd_first_order_plus_second_order_plus_self_consistency} studies the combined effect of first-, second-order dynamics and self-consistency. Finally, Section~\ref{sec:app:3rd_order_homo} demonstrates full third-order HOMO with all optimization terms, analyzing trajectory linearity and mode fidelity under different trajectory settings.
\subsection{Algorithm}\label{sec:app:3rd_homo_algorithm}
Here we first introduce the training algorithm of our third-order HOMO: 
\begin{algorithm}[!ht]\caption{ Third-Order HOMO Training}
\begin{algorithmic}[1]
\While{not converged}
\State $x_0 \sim \N (0, I), x_1 \sim D, (d, t) \sim p(d, t)$
\State $\beta_t \gets \sqrt{1-\alpha_t^2}$
\State $x_t \gets \alpha_t \cdot x_0 + \beta_t \cdot x_1$ \Comment{Noise data point}
\For{first $k$ batch elements}
\State $\dot s_t^{\True} \gets \dot{\alpha_t} x_0 + \dot{\beta_t} x_1$ \Comment{First-order target}
\State $\ddot s_t^{\True} \gets \ddot{\alpha_t} x_0 + \ddot{\beta_t} x_1$ \Comment{Second-order target}
\State $\dddot s_t^{\True} \gets \dddot{\alpha_t} x_0 + \dddot{\beta_t} x_1$ \Comment{Third-order target}
\State $d \gets 0$
\EndFor
\For{other batch elements}
\State $s_t \gets u_1 ( x_t, t, d)$ \Comment{First small step of first order}
\State $\dot s_t \gets u_2 (u_1 ( x_t, t, d), x_t, t, d)$ \Comment{First small step of second order}
\State $\ddot s_t \gets u_3 (u_2(u_1(x_t, t, d), x_t, t, d), u_1(x_t, t, d), x_t, t, d)$ \Comment{First small step of third order}
\State $x_{t + d} \gets x_t + d \cdot s_t + \frac{d^2}{2} \dot s_t + \frac{d^6}{3} \ddot s_t $ \Comment{Follow ODE}
\State $s_{t + d} \gets u_1 ( x_{t + d}, t + d, d )$ \Comment{Second small step of first order}
\State $\dot s_t^{\mathrm{target}} \gets$ stopgrad $(s_t + s_{t+d}) / 2$ \Comment{Self-consistency target of first order }
\EndFor
\State $\theta \gets \nabla_\theta ( \| u_1 ( x_t, t, 2d ) - \dot s_t^{\True} \|^2$
\Statex \hspace{4.2em} $ + \| u_2 (u_1 (x_t, t, 2d), x_t, t, 2d) - \ddot s_t^{\True} \|^2$
\Statex \hspace{4.2em} $ + \| u_3 (u_2(u_1(x, t, d), x, t, d), u_1(x, t, d), x, t, d) - \dddot{s}_t^{\True} \|^2 $
\Statex \hspace{4.2em} $ + \| u_{1}(x_t, t, 2 d) - \dot{s}_t^{\mathrm{target}}\|^2$
\EndWhile
\end{algorithmic}
\end{algorithm}

Then we will discuss the sampling algorithm in third-order HOMO: 
\begin{algorithm}[!ht]\caption{Third-Order HOMO Sampling}
\begin{algorithmic}[1]
\State $x \sim \N (0, I)$
\State $d \gets 1 / M$
\State $t \gets 0$
\For{$n \in [0, \dots, M - 1]$}
\State $x \gets x + d \cdot u_1(x, t, d) + \frac{d^2}{2} \cdot u_2(u_1(x, t, d), x, t, d) + \frac{d^3}{6}\cdot u_3 (u_2(u_1(x, t, d), x, t, d), u_1(x, t, d), x, t, d)$
\State $t \gets t + d$
\EndFor
\State \textbf{return} $x$
\end{algorithmic}
\end{algorithm}

\subsection{Trajectory setting}\label{sec:app:trajectory_setting}
We have trajectory as:
\begin{align*}
    z_t = \alpha_t z_0 + \beta_t z_1
\end{align*}
In original trajectory, we choose $\alpha_t = \exp(-\frac{1}{4} a(1-t)^2 - \frac{1}{2} b(1-t))$ and $\beta_t = \sqrt{1 - \alpha_t^2}$, with hyperparameters $a = 19.9$ and $b = 0.1$. And new trajectory as $\alpha_t = 1 - ( 3t^2 - 2t^3 )$ and $\beta_t = 3t^2 - 2t^3$. 

\subsection{Dataset}\label{sec:app:complex_dataset}
Here, we introduce three datasets we use: 2 Round spin, 3 Round spin, and Dot-Circle datasets. In 2 Round spin dataset and 3 Round spin dataset, we both sample 600 points from Gaussian distribution with $0.3$ variance for both source distribution and target distribution. In Dot-Circle datasets, we sample 300 points from the center dot and 300 points from the outermost circle, combine them as source distribution, and then sample 600 points from 2 round spin distribution. 
\begin{figure}[!ht] 
\centering
\includegraphics[width=0.25\textwidth]{new_spiral_dataset.pdf}
\includegraphics[width=0.25\textwidth]{3round_spiral_dataset.pdf}
\includegraphics[width=0.25\textwidth]{dotpluscircle_dataset.pdf}
\caption{
The 2 Round spin dataset(\textbf{Left}), 3 Round spin dataset(\textbf{Middle}), and Dot-Circle datasets(\textbf{Right}). Our goal is to make HOMO learn a transport trajectory from distribution $\pi_0$ ({\textbf{brown}}) to distribution $\pi_1$ ({\textbf{indigo}}). 
}
\label{fig:three_complex_dataset}
\end{figure}


\subsection{Euclidean distance loss}\label{sec:app:euclidean_distance_loss}
Here, we present the Euclidean distance loss performance of four different loss terms combined under the original trajectory setting and the new trajectory setting. 


\begin{table}[!ht] 
\centering
\caption{
\textbf{Euclidean distance loss of three complex distribution datasets under new trajectory setting.} Lower values indicate more accurate distribution transfer results. Optimal values are highlighted in \textbf{Bold}. And \underline{Underlined} numbers represent the second best (second lowest) loss value for each dataset (row). 
For the qualitative results of a mixture of Gaussian experiments, please refer to Figure~\ref{fig:mixture_of_gaussian_experiment}.
}
\label{tab:euclidean_distance_complex_datasets_new}
\begin{tabular}{l|c|c|c}
\toprule
& \textbf{2 Round} & \textbf{3 Round} & \textbf{Dot-} \\
\textbf{Loss terms}  & \textbf{spin} & \textbf{spin} & \textbf{Circle} \\
\midrule
SC              & 41.265 & 48.201 & 87.407 \\
M1 + SC          & 14.926 & 18.376 & 30.027 \\
M1 + M2 + SC      & \underline{11.435} & \underline{12.422} & \underline{24.712} \\
M1 + M2 + SC + M3		& \textbf{4.701} & \textbf{9.261} & \textbf{21.968} \\
\bottomrule
\end{tabular}
\end{table}

\subsection{Only Self-Consistency Term}\label{sec:app:3rd_self_consistency}
We optimize models by the sum of squared error(SSE). The source distribution and target distribution are all Gaussian distributions. For the first line, we use the original transport trajectory setting, followed by the VP ODE framework from~\cite{rectified_flow}, which is $x_t = \alpha_t x_0 + \beta_t x_1$. We choose $\alpha_t = \exp(-\frac{1}{4} a(1-t)^2 - \frac{1}{2} b(1-t))$ and $\beta_t = \sqrt{1 - \alpha_t^2}$, with hyperparameters $a = 19.9$ and $b = 0.1$. In 2 Round spin datasets and 3 Round spin datasets, we sample 400 points, both source distribution and target distribution. In the Dot-Circle dataset, we sample 600 points from both source distribution and target distribution, 300 points of source points from the circle, and another 300 from the center dot. In 2-round dataset training, we use an ODE solver and Adam optimizer, with 2 hidden layer MLP, 100 hidden dimensions, $800$ batch size, $0.005$ learning rate, and $180$ training steps. In 3-round spin dataset training, we also use an ODE solver and Adam optimizer, with 2 hidden layer MLP, 100 hidden dimensions, $1000$ batch size, $0.005$ learning rate, and $180$ training steps. In Dot-Circle dataset training, we use an ODE solver and Adam optimizer, with 2 hidden layer MLP, 100 hidden dimensions, $1600$ batch size, $0.005$ learning rate, and $180$ training steps. 
\begin{figure}[!ht]
\centering
\subfloat[(original)SC / 2 Round]
{\includegraphics[width=0.25\textwidth]{1_3_2round_spiral_output.pdf}}
\subfloat[(original)SC / 3 Round]
{\includegraphics[width=0.25\textwidth]{1_3_3round_spiral_output.pdf}}
\subfloat[(original)SC / DotPlusCircle]
{\includegraphics[width=0.25\textwidth]{1_3_dotpluscircle_output.pdf}} \\
\subfloat[(new)SC / 2 Round]
{\includegraphics[width=0.25\textwidth]{2_3_2round_spiral_output.pdf}}
\subfloat[(new)SC / 3 Round]
{\includegraphics[width=0.25\textwidth]{2_3_3round_spiral_output.pdf}}
\subfloat[(new)SC / DotPlusCircle]
{\includegraphics[width=0.25\textwidth]{2_3_dotpluscircle_output.pdf}}
\caption{
(SC )
The distributions generated by HOMO are only optimized by self-consistency loss. 
\textbf{Upper row(original trajectory setting):}
Figure (a), in 2 Round spin dataset. Figure (b), in 3 Round spin dataset. 
Figure (c), in Dot-Circle dataset. 
\textbf{Lower row(new trajectory setting):}
Figure (d), in 2 Round spin dataset. 
Figure (e), in 3 Round spin dataset. 
Figure (f), in Dot-Circle dataset. 
The source distribution, $\pi_0$ ({\textbf{brown}}), and the target distribution, $\pi_1$ ({\textbf{indigo}}), are shown, along with the generated distribution ({\textbf{pink}}). 
}
\label{fig:3rd_self_consistency}
\end{figure}

\subsection{First Order Plus Self-Consistency}\label{sec:app:3rd_first_order_plus_self_consistency}
We optimize models by the sum of squared error(SSE). The source distribution and target distribution are all Gaussian distributions. For the first line, we use the original transport trajectory setting, followed by the VP ODE framework from~\cite{rectified_flow}, which is $x_t = \alpha_t x_0 + \beta_t x_1$. We choose $\alpha_t = \exp(-\frac{1}{4} a(1-t)^2 - \frac{1}{2} b(1-t))$ and $\beta_t = \sqrt{1 - \alpha_t^2}$, with hyperparameters $a = 19.9$ and $b = 0.1$. In 2 Round spin datasets and 3 Round spin datasets, we sample $400$ points in both source distribution and target distribution. And in the Dot-Circle dataset, we sample $600$ points from both source distribution and target distribution, $300$ points of sources points from the circle, and another $300$ from the center dot. In 2 Round dataset training, we use ODE solver and Adam optimizer, with 2 hidden layer MLP, 100 hidden dimension, $800$ batch size, $0.005$ learning rate, and $1000$ training steps. In 3 Round spin dataset training, we also use ODE solver and Adam optimizer, with 2 hidden layer MLP, 100 hidden dimension, $1000$ batch size, $0.005$ learning rate, and $2000$ training steps. And in Dot-Circle dataset training, we use an ODE solver and Adam optimizer, with 2 hidden layer MLP, 100 hidden dimensions, $1600$ batch size, $0.005$ learning rate, and $10000$ training steps. 
\begin{figure}[!ht]
\centering
\subfloat[(original)(M1+SC) / 2 Round]
{\includegraphics[width=0.25\textwidth]{1_13_2round_spiral_output.pdf}}
\subfloat[(original)(M1+SC) / 3 Round]
{\includegraphics[width=0.25\textwidth]{1_13_3round_spiral_output.pdf}}
\subfloat[(original)(M1+SC) / Dot-Circle]
{\includegraphics[width=0.25\textwidth]{1_13_dotpluscircle_output.pdf}} \\
\subfloat[(new)(M1 + SC) / 2 Round]
{\includegraphics[width=0.25\textwidth]{2_13_2round_spiral_output.pdf}}
\subfloat[(new)(M1 + SC) / 3 Round]
{\includegraphics[width=0.25\textwidth]{2_13_3round_spiral_output.pdf}}
\subfloat[(new)(M1+SC) / Dot-Circle]
{\includegraphics[width=0.25\textwidth]{2_13_dotpluscircle_output.pdf}}
\caption{
(M1+SC)
The distributions generated by HOMO are only optimized by first-order loss and self-consistency loss. 
\textbf{Upper row(original trajectory setting):}
Figure (a), in 2 Round spin dataset. Figure (b), in 3 Round spin dataset. 
Figure (c), in Dot-Circle dataset. 
\textbf{Lower row(new trajectory setting):}
Figure (d), in 2 Round spin dataset. 
Figure (e), in 3 Round spin dataset. 
Figure (f), in Dot-Circle dataset. 
The source distribution, $\pi_0$ ({\textbf{brown}}), and the target distribution, $\pi_1$ ({\textbf{indigo}}), are shown, along with the generated distribution ({\textbf{pink}}). 
}
\label{fig:3rd_first_order_plus_self_consistency}
\end{figure}

\subsection{First Order Plus Second Order Plus Self-Consistency}\label{sec:app:3rd_first_order_plus_second_order_plus_self_consistency}
We optimize models by the sum of squared error(SSE). The source distribution and target distribution are all Gaussian distributions. For the first line, we use the original transport trajectory setting, followed by the VP ODE framework from~\cite{rectified_flow}, which is $x_t = \alpha_t x_0 + \beta_t x_1$. We choose $\alpha_t = \exp(-\frac{1}{4} a(1-t)^2 - \frac{1}{2} b(1-t))$ and $\beta_t = \sqrt{1 - \alpha_t^2}$, with hyperparameters $a = 19.9$ and $b = 0.1$. In 2 Round spin datasets and 3 Round spin datasets, we sample 400 points, both source distribution and target distribution. And in the Dot-Circle dataset, we sample 600 points from both source distribution and target distribution, 300 points of source points from the circle, and another 300 from the center dot. In 2 Round dataset training, we use ODE solver and Adam optimizer, with 2 hidden layer MLP, 100 hidden dimension, $800$ batch size, $0.005$ learning rate, and $1000$ training steps. In 3 Round spin dataset training, we also use ODE solver and Adam optimizer, with 2 hidden layer MLP, 100 hidden dimension, $1000$ batch size, $0.005$ learning rate, and $2000$ training steps. And in Dot-Circle dataset training, we use an ODE solver and Adam optimizer, with 2 hidden layer MLP, 100 hidden dimensions, $1600$ batch size, $0.005$ learning rate, and $10000$ training steps. 
\begin{figure}[!ht]
\centering
\subfloat[(original) (M1+M2+SC) / 2 Round]
{\includegraphics[width=0.25\textwidth]{1_123_2round_spiral_output.pdf}}
\subfloat[(original) (M1+M2+SC) / 3 Round]
{\includegraphics[width=0.25\textwidth]{1_123_3round_spiral_output.pdf}}
\subfloat[(original) (M1+M2+SC) / Dot-Circle]
{\includegraphics[width=0.25\textwidth]{1_123_dotpluscircle_output.pdf}} \\
\subfloat[(new) (M1+M2+SC) / 2 Round]
{\includegraphics[width=0.25\textwidth]{2_123_2round_spiral_output.pdf}}
\subfloat[(new) (M1+M2+SC) / 3 Round]
{\includegraphics[width=0.25\textwidth]{2_123_3round_spiral_output.pdf}}
\subfloat[(new) (M1+M2+SC) / Dot-Circle]
{\includegraphics[width=0.25\textwidth]{2_123_dotpluscircle_output.pdf}}
\caption{
(M1+M2+SC)
The distributions generated by HOMO are only optimized by first-order loss and second order loss, and self-consistency loss. 
\textbf{Upper row(original trajectory setting):}
Figure (a), in 2 Round spin dataset. Figure (b), in 3 Round spin dataset. 
Figure (c), in Dot-Circle dataset. 
\textbf{Lower row(new trajectory setting):}
Figure (d), in 2 Round spin dataset. 
Figure (e), in 3 Round spin dataset. 
Figure (f), in Dot-Circle dataset. 
The source distribution, $\pi_0$ ({\textbf{brown}}), and the target distribution, $\pi_1$ ({\textbf{indigo}}), are shown, along with the generated distribution ({\textbf{pink}}). 
}
\label{fig:3rd_frist_order_plus_second_order_plus_self_consistency}
\end{figure}

\subsection{Third-Order HOMO}\label{sec:app:3rd_order_homo}
We optimize models by the sum of squared error(SSE). The source distribution and target distribution are all Gaussian distributions. For the first line, we use the original transport trajectory setting, followed by the VP ODE framework from~\cite{rectified_flow}, which is $x_t = \alpha_t x_0 + \beta_t x_1$. We choose $\alpha_t = \exp(-\frac{1}{4} a(1-t)^2 - \frac{1}{2} b(1-t))$ and $\beta_t = \sqrt{1 - \alpha_t^2}$, with hyperparameters $a = 19.9$ and $b = 0.1$. In 2 Round spin datasets and 3 Round spin datasets, we sample 400 points, both source distribution and target distribution. In the Dot-Circle dataset, we sample 600 points, both source distribution and target distribution, 300 points of source points from the circle, and another 300 from the center dot. In 2-round dataset training, we use an ODE solver and Adam optimizer, with 2 hidden layer MLP, 100 hidden dimensions, $800$ batch size, $0.005$ learning rate, and $1000$ training steps. In 3-round spin dataset training, we also use an ODE solver and Adam optimizer, with 2 hidden layer MLP, 100 hidden dimensions, $1000$ batch size, $0.005$ learning rate, and $2000$ training steps. In Dot-Circle dataset training, we use an ODE solver and Adam optimizer, with 2 hidden layer MLP, 100 hidden dimensions, $1600$ batch size, $0.005$ learning rate, and $10000$ training steps. 

\begin{figure}[!ht]
\centering
\subfloat[(original) (M1+M2+M3+SC) / 2 Round]
{\includegraphics[width=0.25\textwidth]{1_1234_2round_spiral_output.pdf}}
\subfloat[(original) (M1+M2+M3+SC) / 3 Round]
{\includegraphics[width=0.25\textwidth]{1_1234_3round_spiral_output.pdf}}
\subfloat[(original) (M1+M2+M3+SC) / Dot-Circle]
{\includegraphics[width=0.25\textwidth]{1_1234_dotpluscircle_output.pdf}}\\
\subfloat[(new) (M1+M2+M3+SC) / 2 Round]
{\includegraphics[width=0.25\textwidth]{2_1234_2round_spiral_output.pdf}}
\subfloat[(new) (M1+M2+M3+SC) / 3 Round]
{\includegraphics[width=0.25\textwidth]{2_1234_3round_spiral_output.pdf}}
\subfloat[(new) (M1+M2+M3+SC) / Dot-Circle]
{\includegraphics[width=0.25\textwidth]{2_1234_dotpluscircle_output.pdf}}
\caption{
(M1+M2+M3+SC) The distributions generated by Third-Order HOMO, optimized by first-order loss and second-order loss, third-order loss and self-consistency loss. 
\textbf{Upper row(original trajectory setting):}
Figure (a), in 2 Round spin dataset. Figure (b), in 3 Round spin dataset. 
Figure (c), in Dot-Circle dataset. 
\textbf{Lower row(new trajectory setting):}
Figure (d), in 2 Round spin dataset. 
Figure (e), in 3 Round spin dataset. 
Figure (f), in Dot-Circle dataset. 
The source distribution, $\pi_0$ ({\textbf{brown}}), and the target distribution, $\pi_1$ ({\textbf{indigo}}), are shown, along with the generated distribution ({\textbf{pink}}). 
}
\label{fig:3rd_order_homo}
\end{figure}

\section{Computational Cost and Optimization Cost} \label{sec:app:computational_cost}
We profile computational efficiency on the Apple MacBook Air (M1 8GB) with an 8-core CPU. Through systematic analysis, we observe three critical tradeoffs: (1) The M2 configuration demonstrates an 8.15$\times$ FLOPs increase over M1 while achieving 4.07$\times$ parameter expansion, revealing the fundamental FLOPs-parameters scaling relationship. (2) The self-consistency (SC) term introduces minimal computational overhead, with the M2+SC configuration maintaining 144.73 it/s versus vanilla M2's 146.34 it/s (1.1\% throughput reduction). (3) Architectural innovations yield substantial gains - the Shortcut Model (M1+SC) achieves 33.6\% faster iterations than vanilla M1 (283.20 vs 477.03 it/s) with comparable parameter counts. Table~\ref{tab:computational_cost} quantifies these effects through comprehensive benchmarking:

\begin{table}[!ht]
\centering
\caption{Computational Cost Analysis of Different Configurations}
\label{tab:computational_cost}
\begin{tabular}{lccc}
\toprule
\textbf{Configuration} & \textbf{FLOPs (M)} & \textbf{Params (K)} & \textbf{Training Speed (it/s)} \\
\midrule
M1 & 8.400 & 10.702 & 477.03 \\
M2 & 68.480 & 43.608 & 146.34 \\ 
M3 & 8.400 & 10.702 & 357.45 \\
M1 + M2 & 16.960 & 21.604 & 248.15 \\
M2 + SC & 68.480 & 43.608 & 144.73 \\
(Shortcut Model) M1 + SC & 8.480 & 10.802 & 283.20 \\
M1 + M2 + SC & 68.480 & 43.608 & 136.46 \\
M1 + M2 + M3 + SC & 103.680 & 66.012 & 122.18 \\
\bottomrule
\end{tabular}
\end{table}

Notably, our architecture maintains practical viability even for high-order extensions - the third-order HOMO configuration (M1+M2+M3+SC) sustains 122.18 it/s despite requiring 12.34$\times$ more FLOPs than the base M1 model. This demonstrates our method's ability to balance computational complexity with real-time performance requirements. 

%%%% Cut-line between first 10 pages and appendix







%%% some writing rules

%% Writing rule for creating tags.
%% Tags :
%% Theorem    \ref{thm:bla_bla}
%% Lemma      \ref{lem:bla_bla}
%% Claim      \ref{cla:bla_bla}
%% Corollary  \ref{cor:bla_bla}
%% Fact       \ref{fac:bla_bla}
%% Definition \ref{def:bla_bla}
%% Section    \ref{sec:bla_bla}
%% Subsection \ref{sub:bla_bla}
%% Equation   \ref{eq:bla_bla}



\end{document}



%%%%%%%%%%%%%%%%%%%%%%%%%%%%%%%%%%%%%%%%%%%%%%%%%%%%%%%%%%%%%%%%%%%%%%%%%%%%%%%%%%%%%%%%%%%%%%%%%%%%%%%%%%%%%%%%%%%%%%%%%%%%%%%%%%%%%%%%%%%%%%%%%%%%%%%%%%%%%%%%%%%%%%%%%%%%%%%%%%%%%%%%%%%%%%%%%%%%%%%%%%%%%%%%%%%%%%%%%%%%%%%%%%%%%%%%%%%%%%%%%%%%%%%%%%%%%%%%%%%%%%%%%%%%%%%%%%%%%%%%%%%%%%%%%%%%%%%%%%%%%%%%%%%%%%%%%%%%%%%%%%%%%%%%%%%%%%%%%%%%%%%%%%%%%%%%%%%%%%%%%%%%%%%%%%%%%%%%%%%%%%%%%%%%%%%%%%%%%%%%%%%%%%%%%%%%%%%%%%%%%%%%%%%%%%%%%%%%%%%%%%%%%%%%%%%%%%%%%%%%%%
