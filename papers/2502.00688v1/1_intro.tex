\section{Introduction}


In recent years, deep generative models have exhibited extraordinary promise across various types of data modalities. Techniques such as Generative Adversarial Networks (GANs) \cite{gpm+14}, autoregressive models \cite{v17}, normalizing flows \cite{lcb+22}, and diffusion models \cite{hja20} have achieved outstanding results in tasks related to image, audio, and video generation \cite{kes+18, brl+23}. These models have attracted considerable interest owing to their capacity to create invertible and highly expressive mappings, transforming simple prior distributions into complex target data distributions. This fundamental characteristic is the key reason they are capable of modeling any data distribution. Particularly, \cite{lcb+22, lgl22} have effectively unified conventional normalizing flows with score-based diffusion methods. These techniques produce a continuous trajectory, often referred to as a ``flow'', which transitions samples from the prior distribution to the target data distribution. By adjusting parameterized velocity fields to align with the time derivatives of the transformation, flow matching achieves not only significant experimental gains but also retains a strong theoretical foundation.

Despite the remarkable progress in flow-based generative models, such as the Shortcut model \cite{fhla24}, these approaches still face challenges in accurately modeling complex data distributions, particularly in regions of high curvature or intricate geometric structure \cite{wet+24, hwa+24}. This limitation stems from the reliance on first-order techniques, which primarily focus on aligning instantaneous velocities while neglecting the influence of higher-order dynamics on the overall flow geometry. Recent research in diffusion-based modeling \cite{c23, hg24, llly24} has highlighted the importance of capturing higher-order information to improve the fidelity of learned trajectories. However, a systematic framework for incorporating such higher-order dynamics into flow matching, especially within Shortcut models, remains an open problem.

In this work, we propose HOMO (High-Order Matching for One-Step Shortcut Diffusion), a revolutionary leap beyond the limitations of the original Shortcut model \cite{fhla24}. While Shortcut models rely on simplistic first-order dynamics, often empirically struggling to capture complex data distributions and producing erratic trajectories in high-curvature regions, HOMO shatters these barriers by introducing high-order supervision. By incorporating acceleration, jerk, and beyond, HOMO not only addresses the empirical shortcomings of the Shortcut model but also achieves unparalleled geometric precision and stability. Where the Shortcut model falters—yielding suboptimal trajectories and poor distributional alignment—HOMO thrives, delivering smoother, more accurate, and fundamentally superior results.


Our primary contribution is a rigorous theoretical and empirical framework that showcases the dominance of HOMO. We prove that HOMO's high-order supervision drastically reduces approximation errors, ensuring precise trajectory alignment from the earliest stages to long-term evolution. Empirically, we demonstrate that the Shortcut model's first-order dynamics fall short in complex settings, while HOMO consistently outperforms it, achieving faster convergence, better sample quality, and unmatched robustness.

The contributions of our work is summarized as follows:
\begin{itemize}
    \item We introduce high-order supervision into the Shortcut model, resulting in the HOMO framework, which includes novel training and sampling algorithms.
    \item We provide rigorous theoretical guarantees for the approximation error of high-order flow matching, demonstrating its effectiveness in both the early and late stages of the generative process.
    \item We demonstrate that HOMO achieves superior empirical performance in complex settings, especially in intricate distributional landscapes, beyond the capabilities of the original Shortcut model \cite{fhla24}.
\end{itemize}
