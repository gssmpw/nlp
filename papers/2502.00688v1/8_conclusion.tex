\section{Conclusion} \label{sec:conclusion}

In this work, we introduced HOMO (High-Order Matching for One-Step Shortcut Diffusion), a novel framework that incorporates high-order dynamics into the training and sampling processes of Shortcut models. By leveraging high-order supervision, our method significantly enhances the geometric consistency and precision of learned trajectories.
Theoretical analyses demonstrate that high-order supervision ensures stability and generalization across different phases of the generative process. These findings are supported by extensive experiments, where HOMO outperforms original Shortcut models \cite{fhla24}, achieving more accurate distributional alignment and fewer suboptimal trajectories.
The integration of high-order terms establishes a new style for geometrically-aware generative modeling, highlighting the importance of capturing higher-order dynamics for accurate transport learning. Our results suggest that high-order supervision is a powerful tool for improving the fidelity and robustness of flow-based generative models. 

\ifdefined\isarxiv
\else
\section*{Impact Statement}
This paper presents work whose goal is to advance the field of Machine Learning. There are many potential societal consequences of our work, none of which we feel must be specifically highlighted here.
\fi
