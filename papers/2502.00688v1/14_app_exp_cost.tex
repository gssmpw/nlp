\section{Computational Cost and Optimization Cost} \label{sec:app:computational_cost}
We profile computational efficiency on the Apple MacBook Air (M1 8GB) with an 8-core CPU. Through systematic analysis, we observe three critical tradeoffs: (1) The M2 configuration demonstrates an 8.15$\times$ FLOPs increase over M1 while achieving 4.07$\times$ parameter expansion, revealing the fundamental FLOPs-parameters scaling relationship. (2) The self-consistency (SC) term introduces minimal computational overhead, with the M2+SC configuration maintaining 144.73 it/s versus vanilla M2's 146.34 it/s (1.1\% throughput reduction). (3) Architectural innovations yield substantial gains - the Shortcut Model (M1+SC) achieves 33.6\% faster iterations than vanilla M1 (283.20 vs 477.03 it/s) with comparable parameter counts. Table~\ref{tab:computational_cost} quantifies these effects through comprehensive benchmarking:

\begin{table}[!ht]
\centering
\caption{Computational Cost Analysis of Different Configurations}
\label{tab:computational_cost}
\begin{tabular}{lccc}
\toprule
\textbf{Configuration} & \textbf{FLOPs (M)} & \textbf{Params (K)} & \textbf{Training Speed (it/s)} \\
\midrule
M1 & 8.400 & 10.702 & 477.03 \\
M2 & 68.480 & 43.608 & 146.34 \\ 
M3 & 8.400 & 10.702 & 357.45 \\
M1 + M2 & 16.960 & 21.604 & 248.15 \\
M2 + SC & 68.480 & 43.608 & 144.73 \\
(Shortcut Model) M1 + SC & 8.480 & 10.802 & 283.20 \\
M1 + M2 + SC & 68.480 & 43.608 & 136.46 \\
M1 + M2 + M3 + SC & 103.680 & 66.012 & 122.18 \\
\bottomrule
\end{tabular}
\end{table}

Notably, our architecture maintains practical viability even for high-order extensions - the third-order HOMO configuration (M1+M2+M3+SC) sustains 122.18 it/s despite requiring 12.34$\times$ more FLOPs than the base M1 model. This demonstrates our method's ability to balance computational complexity with real-time performance requirements. 