%\clearpage

% \begin{figure*}[!t]
%     \centering
%     \includegraphics[width=\textwidth]{figures/dse/acc_vs_width.pdf}
%     \caption{Impact of width multiplier on accuracy. Rows represent the quantization graph (Academic vs. Hardware), columns show the models (ResNet18 vs. MobileNetV2), and the color and symbol denote the precision level.}
%     \label{fig_app:width_vs_acc}
% \end{figure*}

% \begin{figure*}[!t]
%     \centering
%     \includegraphics[width=\textwidth]{figures/dse/e_lbl_vs_width.pdf}
%     \caption{Impact of width multiplier on energy. Rows represent the quantization graph (Academic vs. Hardware), columns show the models (ResNet18 vs. MobileNetV2), and the color and symbol denote the precision level.}
%     \label{fig_app:width_vs_e}
% \end{figure*}
% \null
% \newpage
% \addtocounter{page}{-1}
\null
\newpage
\begin{center}
\textbf{\large Supplementary information for STEMS: \\ Spatial-Temporal Mapping Tool For Spiking Neural Networks}
\end{center}

\begin{figure}[t]
    \centering
    \includegraphics[width=.7\columnwidth]
    %[width=3.3in]
    {Figures/Appendix/Meta_prototype_like_single_core-RED_LIF_T-DSEwBarr-DRAM-per_inf.pdf}
    \caption{RED-LIF hybrid schedule exploration with 1 MB on-chip memory; DRAM energy per inference.}
    \label{fig:redlif-1m}
\end{figure}

\begin{figure}[t]
    \centering
    \includegraphics[width=.7\columnwidth]
    %[width=3.3in]
    {Figures/Appendix/Meta_prototype_like_single_core_2MB-RED_LIF_T-DSEwBarr-DRAM-per_inf.pdf}
    \caption{RED-LIF hybrid schedule exploration with 2 MB on-chip memory; DRAM energy per inference.}
    \label{fig:redlif-2m}
\end{figure}



\section*{Hybrid Schedule Exploration} \label{app:width_vs_accuracy}

We present here the remaining results of our hybrid schedule exploration. We perform exploratory studies for RED-LIF, SEW-7, and SEW-5 with different on-chip memory constraints. We present here the DRAM energy of all hybrid schedules. 

For RED-LIF, in addition to the exploration at 512 KB on-chip capacity (Figure \ref{fig:redlif-512}), we performed explorations at 1 MB (Figure \ref{fig:redlif-1m}) and 2 MB (Figure \ref{fig:redlif-2m}) on-chip capacity. For SEW-7 and SEW-5, we performed explorations at 128KB (Figures \ref{fig:sew7-128} and \ref{fig:sew5-128} respectively), 256 KB (Figures \ref{fig:sew7-256} and \ref{fig:sew5-256} respectively), and 512 KB (Figures \ref{fig:sew7-512} and \ref{fig:sew5-512} respectively) on-chip capacity.

%For SEW-7, we performed explorations at 1 MB, 512 KB, 256 KB, and 128 KB on-chip memory. While for SEW-5, we performed explorations at only 512 KB, 256 KB, and 128 KB of on-chip energy.

Our results show that, under more relaxed on-chip memory, several schedules perform well. With enough on-chip memory, all schedules would perform equally well, with optimal DRAM traffic (1 mJ for RED-LIF, 0.1 mJ for SEW-ResNet). However, under tight memory constraints, only few schedules outperform the others. Such schedules are in line with our conclusions. Notice how the optimal schedule may vary depending on on-chip capacity.
%better performance is achieved and multiple schedules can achieve decent performance. 
\begin{figure}[t]
    \centering
    \includegraphics[width=.7\columnwidth]
{Figures/Appendix/Meta_prototype_like_single_core_128KB-7PLIF-DSEwBarr-DRAM-per_inf.pdf}
    \caption{SEW-7 hybrid schedule exploration with 128 KB on-chip memory; DRAM energy per inference.}
    \label{fig:sew7-128}
\end{figure}
\begin{figure}[t]
    \centering
    \includegraphics[width=.7\columnwidth]
    %[width=3.3in]
    {Figures/Appendix/Meta_prototype_like_single_core_128KB-5PLIF-DSEwBarr-DRAM-per_inf.pdf}
    \caption{SEW-5 hybrid schedule exploration with 128 KB on-chip memory; DRAM energy per inference.}
    \label{fig:sew5-128}
\end{figure}


\begin{figure}[t]
    \centering
    \includegraphics[width=.7\columnwidth]
    %[width=3.3in]
    {Figures/Appendix/Meta_prototype_like_single_core_256KB-7PLIF-DSEwBarr-DRAM-per_inf.pdf}
    \caption{SEW-7 hybrid schedule exploration with 256 KB on-chip memory; DRAM energy per inference.}
    \label{fig:sew7-256}
\end{figure}

\begin{figure}[t]
    \centering
    \includegraphics[width=.7\columnwidth]
    %[width=3.3in]
    {Figures/Appendix/Meta_prototype_like_single_core_256KB-5PLIF-DSEwBarr-DRAM-per_inf.pdf}
    \caption{SEW-5 hybrid schedule exploration with 256 KB on-chip memory; DRAM energy per inference.}
    \label{fig:sew5-256}
\end{figure}

\begin{figure}[t]
    \centering
    \includegraphics[width=.7\columnwidth]{Figures/Appendix/Meta_prototype_like_single_core_512KB-7PLIF-DSE-DRAM-per_inf.pdf}
    \caption{SEW-7 hybrid schedule exploration with 512 KB on-chip memory; DRAM energy per inference.}
    \label{fig:sew7-512}
\end{figure}



\begin{figure}[t]
    \centering
    \includegraphics[width=.7\columnwidth]{Figures/Appendix/Meta_prototype_like_single_core_512KB-5PLIF-DSE-DRAM-per_inf.pdf}
    \caption{SEW-5 hybrid schedule exploration with 512 KB on-chip memory; DRAM energy per inference.}
    \label{fig:sew5-512}
\end{figure}


\section{SEW-ResNet-152 ImageNet model}
\label{sec:apndx}

To demonstrate the scalability of STEMS, we explored the mapping space of a deeper model, SEW-ResNet-152. In \cite{sewresnet}, this model was used for ImageNet classification, where images were rate-encoded into 4 timesteps resulting in an input spike map of size 4x224x224. We assume the same input spike map in this experiment.

\subsection{Model structure}

The SEW-ResNet-152 model consists of 50 residual blocks organized by channel depth. It consists of 3 blocks with channel depth 64, 8 blocks with channel depth 128, 36 blocks with channel depth 256, and 3 blocks with channel depth 512. 

To reduce the number of experiments and have more readable exploration results, we coarsen the model into 7 hyperblocks as follows. The first hyperblock consists of the first 3 blocks with channel depth 64. Then, the second hyperblock consists of the 8 blocks with channel depth 128. Then, the 36 blocks with channel depth 256 are partitioned into 4 hyperblocks. Finally, the last 3 blocks with channel depth 512 represent the last hyperblock.



\subsection{Schedule exploration results}

We present the results of hybrid schedule exploration for SEW-ResNet-152 with 1 MB and 2 MB on-chip global buffer. We explore both time batching and layer fusion starting from the input block, as the earlier blocks contain large amounts of features and neuron states.

Figures \ref{fig:sew152-1} and \ref{fig:sew152-2} show the DRAM energy consumed per inference (4 timesteps) for all 64 possible schedules mapped on the hardware architecture with 1 MB and 2 MB on-chip memory respectively, where the horizontal axis represents the number of blocks that are time-batched (from the input side) and the vertical axis represents the number of blocks that are layer-fused (from the input side). The optimal schedule deploys a time-batched layer-fused (TB-LF) schedule for the first hyperblock, and a time-batched layer-by-layer (TB-LBL) schedule for the rest of the network, for both hardware architectures. Such results are similar to the results of SEW-ResNet-18 hybrid schedule exploration (SEW-7), where earlier blocks favor TB-LF schedule and later blocks favor TB-LBL schedule. Applying this schedule results in 10x reduction in DRAM data traffic, compared to the baseline schedule, for both 1 MB and 2 MB on-chip memory architectures.

\begin{figure}[t]
    \centering
    \includegraphics[width=.7\columnwidth]{Figures/Appendix/Meta_prototype_like_single_core-SEW-ImageNet-DSEwBarr-DRAM-per_inf--cropped.pdf}
    \caption{SEW-ResNet-152 hybrid schedule exploration with 1 MB on-chip memory; DRAM energy per inference.}
    \label{fig:sew152-1}
\end{figure}



\begin{figure}[t]
    \centering
    \includegraphics[width=.7\columnwidth]{Figures/Appendix/Meta_prototype_like_single_core_2MB-SEW-ImageNet-DSEwBarr-DRAM-per_inf--cropped.pdf}
    \caption{SEW-ResNet-152 hybrid schedule exploration with 2 MB on-chip memory; DRAM energy per inference.}
    \label{fig:sew152-2}
\end{figure}

%agree with the claim that blocks with more features favor schedules that minimize intermediate features, while blocks with more memory favor schedules that maximize memory re-use. The optimal hybrid schedule is illustrated in Figure \ref{fig:red-best}.


% \begin{figure}[t]
%     \centering
%     \includegraphics[width=3.5in]{Figures/Appendix/Meta_prototype_like_single_core-7PLIF-DSE-DRAM-per_inf.pdf}
%     \caption{1 MB  on-chip memory SEW-7 hybrid schedule exploration.}
%     \label{fig:sew5-1}
% \end{figure}



















