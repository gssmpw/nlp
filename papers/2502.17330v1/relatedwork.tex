\section{Related Work}
In the physical security domain, the adoption of Neural Networks (NNs) has marked a transformative phase, particularly in enhancing Side-Channel Attack (SCA) strategies. These NN-empowered SCAs have surpassed traditional approaches by yielding more potent results with reduced observational demands \cite{maghrebi2016breaking,picek2017side,wang2014learning,wu2021best,perin2021keep,nascimento2017applying,weissbart2019one,cagli2017convolutional,zaid2020methodology,carbone2019deep,picek2021sok}. Research initiatives \cite{maghrebi2019deep,ramezanpour2020scaul,benadjila2020deep}, have taken on sophisticated Deep Learning techniques to exploit side channel traces in examining symmetric algorithms. Focusing on the ECC Double-And-Add-Always algorithm implemented on FPGA platforms, Mukhtar et al.\cite{mukhtar2018machine} applied classification methods to reveal secret key bits of the ECC. In a similar vein, Weissbart et al.\cite{weissbart2019one} orchestrated a power analysis attack on the Edwards-curve Digital Signature Algorithm (EdDSA)\cite{bernstein2012high}, revealing the superior capabilities of CNN over classical side-channel techniques like Template Attacks~\cite{chari2002template}. Weissbart et al.\cite{weissbart2020systematic} further expanded their investigation, evaluating additional protected targets and highlighting the efficacy of Deep Learning, particularly CNNs, in breaching protected implementations of scalar multiplication on Curve25519\cite{bernstein2006curve25519}.

Perin et al.\cite{perin2021keep} proposed a groundbreaking Deep Learning-based iterative framework for unsupervised horizontal attacks, aimed at refining the accuracy of single-trace attacks and reducing errors in the decryption of private keys, particularly in protected ECC implementations. This effort, similar to Nascimento et al.\cite{nascimento2017applying}, exploited vulnerabilities within the $\mu$NaCl library's \texttt{cswap} function~\cite{NACLlib}, showcasing the ongoing evolution of NN-enabled SCA methodologies in enhancing cryptographic security. In a recent development, Staib et al.\cite{staib2023deep} sought to advance collision side channel attacks through deep learning, demonstrating a neural network's superiority in collision detection over traditional methods on a public dataset\cite{luo2018effective,clavier2011improved,bauer2015horizontal}. 

Our research uses long- and short-term memory (LSTM) networks, which are renowned for their effectiveness in tasks that resemble recognition of human activity. This innovative approach not only goes beyond the traditional scope of side-channel attacks (SCA) but reframes the problem as an "operation recognition" task. By adopting methodologies similar to those in human activity detection, we enable sophisticated pattern recognition within cryptographic operations. This unique framework is pivotal for uncovering hidden vulnerabilities, highlighting the intricate interplay between cryptographic processes and exploitable weaknesses.