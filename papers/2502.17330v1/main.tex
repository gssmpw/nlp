\documentclass[runningheads]{llncs}
\usepackage[T1]{fontenc}
\def\Snospace~{\S{}}
\usepackage{nicefrac}
\usepackage{siunitx}
\usepackage{listings}
\usepackage{array,framed}
\usepackage{
  color,
  float,
  epsfig,
  wrapfig,
  graphics,
  graphicx
}
\usepackage{textcomp,amssymb}
\usepackage{setspace}
\usepackage{amsfonts}
\usepackage{enumerate}
\usepackage{enumitem}
\usepackage[compatible]{algpseudocode}
\usepackage{graphics}
\usepackage{subfig}
\usepackage{xparse} 
\usepackage{xspace}
\usepackage{multirow}
\usepackage{hyperref}
\usepackage{xfrac}
\usepackage{tabularx}
\usepackage{flushend}
\usepackage{mathptmx,avant}
 \usepackage{
  tikz,
  pgfplots,
  pgfplotstable
}
\usepackage{hyperref}

\usetikzlibrary{
  shapes.geometric,
  arrows,
  external,
  pgfplots.groupplots,
  matrix
}
\usepackage{amsmath}

\renewcommand{\sectionautorefname}{Section}
\renewcommand{\subsectionautorefname}{Subsection}

\newcommand{\Mod}[1]{\ (\mathrm{mod}\ #1)}

\pgfplotsset{compat=1.9}

\newcommand{\kms}{km\,s$^{-1}$}
\newcommand{\msun}{$M_\odot}
\newcommand{\sys}{{\sc VS-TEE}\xspace}
\newcommand{\circled[1]}{\tikz[baseline=(char.base)]{\node[font=\sffamily,
shape=circle,draw,inner sep=0.5pt,color=white,fill=black] (char) {#1};}}

\DeclareGraphicsExtensions{
    .png,.PNG,
    .pdf,.PDF,
    .jpg,.mps,.jpeg,.jbig2,.jb2,.JPG,.JPEG,.JBIG2,.JB2}

\usepackage{Settings/my_commands}
\usepackage{Settings/listing_algorithm}
\usepackage{caption}

\captionsetup[table]{position=bottom}

\usepackage{svg}
\usepackage{soul}
\setstcolor{red}
\usepackage{colortbl}
\usepackage{multicol}
\usepackage{arydshln}
\usepackage{verbatim}
\usepackage{fancyvrb}
\usepackage{fancyhdr}

\newcommand{\andrea}[1]{{\bf \textcolor{green}{Andrea: #1}}}
\newcommand{\davide}[1]{}
\usepackage{bbm}
\usepackage{graphicx}
\usepackage{amsmath,amssymb,amsthm,amsfonts}

\usepackage{paralist}
\usepackage{bm}
\usepackage{xspace}
\usepackage{url}
\usepackage{prettyref}
\usepackage{boxedminipage}
\usepackage{wrapfig}
\usepackage{ifthen}
\usepackage{color}
\usepackage{xspace}

\newcommand{\ii}{{\sc Indicator-Instance}\xspace}
\newcommand{\midd}{{\sf mid}}


\usepackage{amsmath,amsthm,amsfonts,amssymb}
\usepackage{mathtools}
\usepackage{graphicx}


% \usepackage{fullpage}

\usepackage{nicefrac}

\newtheorem{inftheorem}{Informal Theorem}
\newtheorem{claim}{Claim}
\newtheorem*{definition*}{Definition}
\newtheorem{example}{Example}

\DeclareMathOperator*{\argmax}{arg\,max}
\DeclareMathOperator*{\argmin}{arg\,min}
\usepackage{subcaption}

\newtheorem{problem}{Problem}
\usepackage[utf8]{inputenc}
\newcommand{\rank}{\mathsf{rank}}
\newcommand{\tr}{\mathsf{Tr}}
\newcommand{\tv}{\mathsf{TV}}
\newcommand{\opt}{\mathsf{OPT}}
\newcommand{\rr}{\textsc{R}\space}
\newcommand{\alg}{\textsf{Alg}\space}
\newcommand{\sd}{\textsf{sd}_\lambda}
\newcommand{\lblq}{\mathfrak{lq} (X_1)}
\newcommand{\diag}{\textsf{diag}}
\newcommand{\sign}{\textsf{sgn}}
\newcommand{\BC}{\texttt{BC} }
\newcommand{\MM}{\texttt{MM} }
\newcommand{\Nexp}{N_{\mathrm{exp}}}
\newcommand{\Nrep}{N_{\mathrm{replay}}}
\newcommand{\Drep}{D_{\mathrm{replay}}}
\newcommand{\Nsim}{N_{\mathrm{sim}}}
\newcommand{\piBC}{\pi^{\texttt{BC}}}
\newcommand{\piRE}{\pi^{\texttt{RE}}}
\newcommand{\piEMM}{\pi^{\texttt{MM}}}
\newcommand{\mmd}{\texttt{Mimic-MD} }
\newcommand{\RE}{\texttt{RE} }
\newcommand{\dem}{\pi^E}
\newcommand{\Rlint}{\mathcal{R}_{\mathrm{lin,t}}}
\newcommand{\Rlipt}{\mathcal{R}_{\mathrm{lip,t}}}
\newcommand{\Rlin}{\mathcal{R}_{\mathrm{lin}}}
\newcommand{\Rlip}{\mathcal{R}_{\mathrm{lip}}}
\newcommand{\Rmax}{R_{\mathrm{max}}}
\newcommand{\Rall}{\mathcal{R}_{\mathrm{all}}}
\newcommand{\Rdet}{\mathcal{R}_{\mathrm{det}}}
\newcommand{\Fmax}{F_{\mathrm{max}}}
\newcommand{\Nmax}{\mathcal{N}_{\mathrm{max}}}
\newcommand{\piref}{\pi^{\mathrm{ref}}}
\newcommand{\green}{\text{\color{green!75!black} green}\;}
\newcommand{\thetaBC}{\widehat{\theta}^{\textsf{BC}}}
\newcommand{\ent}{\mathcal{E}_{\Theta,n,\delta}}
\newcommand{\eNt}{\mathcal{E}_{\Theta_t,\Nexp,\delta}}
\newcommand{\eNtH}{\mathcal{E}_{\Theta_t,\Nexp,\delta/H}}

\newcommand{\eref}[1]{(\ref{#1})}
\newcommand{\sref}[1]{Sec. \ref{#1}}
\newcommand{\dr}{\widehat{d}_{\mathrm{replay}}}
\newcommand{\figref}[1]{Fig. \ref{#1}}

\usepackage{xcolor}
\definecolor{expert}{HTML}{008000}
\definecolor{error}{HTML}{f96565}
\newcommand{\GKS}[1]{{\textcolor{violet}{\textbf{GKS: #1}}}}
\newcommand{\Q}[1]{{\textcolor{red}{\textbf{Question #1}}}}
\newcommand{\ZSW}[1]{{\textcolor{orange}{\textbf{ZSW: #1}}}}
\newcommand{\JAB}[1]{{\textcolor{teal}{\textbf{JAB: #1}}}}
\newcommand{\jab}[1]{{\textcolor{teal}{\textbf{JAB: #1}}}}
\newcommand{\SAN}[1]{{\textcolor{blue}{\textbf{SC: #1}}}}
\newcommand{\scnote}[1]{\SAN{#1}}
\newcommand{\norm}[1]{\left\lVert #1 \right\rVert}

\usepackage{color-edits}
\addauthor{sw}{blue}

\usepackage{thmtools}
\usepackage{thm-restate}

\usepackage{tikz}
\usetikzlibrary{arrows,calc} 
\newcommand{\tikzAngleOfLine}{\tikz@AngleOfLine}
\def\tikz@AngleOfLine(#1)(#2)#3{%
\pgfmathanglebetweenpoints{%
\pgfpointanchor{#1}{center}}{%
\pgfpointanchor{#2}{center}}
\pgfmathsetmacro{#3}{\pgfmathresult}%
}

\declaretheoremstyle[
    headfont=\normalfont\bfseries, 
    bodyfont = \normalfont\itshape]{mystyle} 
\declaretheorem[name=Theorem,style=mystyle,numberwithin=section]{thm}

% \usepackage{algorithm}
% \usepackage{algorithmic}
\usepackage[linesnumbered,algoruled,boxed,lined,noend]{algorithm2e}

\usepackage{listings}
\usepackage{amsmath}
\usepackage{amsthm}
\usepackage{tikz}
\usepackage{caption}
\usepackage{mdwmath}
\usepackage{multirow}
\usepackage{mdwtab}
\usepackage{eqparbox}
\usepackage{multicol}
\usepackage{amsfonts}
\usepackage{tikz}
\usepackage{multirow,bigstrut,threeparttable}
\usepackage{amsthm}
\usepackage{bbm}
\usepackage{epstopdf}
\usepackage{mdwmath}
\usepackage{mdwtab}
\usepackage{eqparbox}
\usetikzlibrary{topaths,calc}
\usepackage{latexsym}
\usepackage{cite}
\usepackage{amssymb}
\usepackage{bm}
\usepackage{amssymb}
\usepackage{graphicx}
\usepackage{mathrsfs}
\usepackage{epsfig}
\usepackage{psfrag}
\usepackage{setspace}
\usepackage[%dvips,
            CJKbookmarks=true,
            bookmarksnumbered=true,
            bookmarksopen=true,
%						bookmarks=false,
            colorlinks=true,
            citecolor=red,
            linkcolor=blue,
            anchorcolor=red,
            urlcolor=blue
            ]{hyperref}
%\usepackage{algorithm}
\usepackage[linesnumbered,algoruled,boxed,lined]{algorithm2e}
\usepackage{algpseudocode}
\usepackage{stfloats}
\RequirePackage[numbers]{natbib}

\usepackage{comment}
\usepackage{mathtools}
\usepackage{blkarray}
\usepackage{multirow,bigdelim,dcolumn,booktabs}

\usepackage{xparse}
\usepackage{tikz}
\usetikzlibrary{calc}
\usetikzlibrary{decorations.pathreplacing,matrix,positioning}

\usepackage[T1]{fontenc}
\usepackage[utf8]{inputenc}
\usepackage{mathtools}
\usepackage{blkarray, bigstrut}
\usepackage{gauss}

\newenvironment{mygmatrix}{\def\mathstrut{\vphantom{\big(}}\gmatrix}{\endgmatrix}

\newcommand{\tikzmark}[1]{\tikz[overlay,remember picture] \node (#1) {};}

%% Adapted form https://tex.stackexchange.com/questions/206898/braces-for-cases-in-tabular-environment/207704#207704
\newcommand*{\BraceAmplitude}{0.4em}%
\newcommand*{\VerticalOffset}{0.5ex}%  
\newcommand*{\HorizontalOffset}{0.0em}% 
\newcommand*{\blocktextwid}{3.0cm}%
\NewDocumentCommand{\InsertLeftBrace}{%
	O{} % #1 = draw options
	O{\HorizontalOffset,\VerticalOffset} % #2 = optional brace shift options
	O{\blocktextwid} % #3 = optional text width
	m   % #4 = top tikzmark
	m   % #5 = bottom tikzmark
	m   % #6 = node text
}{%
	\begin{tikzpicture}[overlay,remember picture]
	\coordinate (Brace Top)    at ($(#4.north) + (#2)$);
	\coordinate (Brace Bottom) at ($(#5.south) + (#2)$);
	\draw [decoration={brace, amplitude=\BraceAmplitude}, decorate, thick, draw=black, #1]
	(Brace Bottom) -- (Brace Top) 
	node [pos=0.5, anchor=east, align=left, text width=#3, color=black, xshift=\BraceAmplitude] {#6};
	\end{tikzpicture}%
}%
\NewDocumentCommand{\InsertRightBrace}{%
	O{} % #1 = draw options
	O{\HorizontalOffset,\VerticalOffset} % #2 = optional brace shift options
	O{\blocktextwid} % #3 = optional text width
	m   % #4 = top tikzmark
	m   % #5 = bottom tikzmark
	m   % #6 = node text
}{%
	\begin{tikzpicture}[overlay,remember picture]
	\coordinate (Brace Top)    at ($(#4.north) + (#2)$);
	\coordinate (Brace Bottom) at ($(#5.south) + (#2)$);
	\draw [decoration={brace, amplitude=\BraceAmplitude}, decorate, thick, draw=black, #1]
	(Brace Top) -- (Brace Bottom) 
	node [pos=0.5, anchor=west, align=left, text width=#3, color=black, xshift=\BraceAmplitude] {#6};
	\end{tikzpicture}%
}%
\NewDocumentCommand{\InsertTopBrace}{%
	O{} % #1 = draw options
	O{\HorizontalOffset,\VerticalOffset} % #2 = optional brace shift options
	O{\blocktextwid} % #3 = optional text width
	m   % #4 = top tikzmark
	m   % #5 = bottom tikzmark
	m   % #6 = node text
}{%
	\begin{tikzpicture}[overlay,remember picture]
	\coordinate (Brace Top)    at ($(#4.west) + (#2)$);
	\coordinate (Brace Bottom) at ($(#5.east) + (#2)$);
	\draw [decoration={brace, amplitude=\BraceAmplitude}, decorate, thick, draw=black, #1]
	(Brace Top) -- (Brace Bottom) 
	node [pos=0.5, anchor=south, align=left, text width=#3, color=black, xshift=\BraceAmplitude] {#6};
	\end{tikzpicture}%
}%

\usetikzlibrary{patterns}

\definecolor{cof}{RGB}{219,144,71}
\definecolor{pur}{RGB}{186,146,162}
\definecolor{greeo}{RGB}{91,173,69}
\definecolor{greet}{RGB}{52,111,72}

% provide arXiv number if available:
% \arxiv{cs.IT/1502.00326}

% put your definitions there:

%\newtheorem{remark}{Remark} \def\remref#1{Remark~\ref{#1}}
%\newtheorem{conjecture}{Conjecture} \def\remref#1{Remark~\ref{#1}}
%\newtheorem{example}{Example}

%\theorembodyfont{\itshape}
%\newtheorem{theorem}{Theorem}
%\newtheorem{proposition}{Proposition}
%\newtheorem{lemma}{Lemma} \def\lemref#1{Lemma~\ref{#1}}
%\newtheorem{corollary}{Corollary}


%\theorembodyfont{\rmfamily}
%\newtheorem{definition}{Definition}
%\numberwithin{equation}{section}
% \theoremstyle{plain}
% \newtheorem{theorem}{Theorem}
% \newtheorem{Example}{Example}
% \newtheorem{lemma}{Lemma}
% \newtheorem{remark}{Remark}
% \newtheorem{corollary}{Corollary}
% \newtheorem{definition}{Definition}
% \newtheorem{conjecture}{Conjecture}
% \newtheorem{question}{Question}
% \newtheorem*{induction}{Induction Hypothesis}
% \newtheorem*{folklore}{Folklore}
% \newtheorem{assumption}{Assumption}

\def \by {\bar{y}}
\def \bx {\bar{x}}
\def \bh {\bar{h}}
\def \bz {\bar{z}}
\def \cF {\mathcal{F}}
\def \bP {\mathbb{P}}
\def \bE {\mathbb{E}}
\def \bR {\mathbb{R}}
\def \bF {\mathbb{F}}
\def \cG {\mathcal{G}}
\def \cM {\mathcal{M}}
\def \cB {\mathcal{B}}
\def \cN {\mathcal{N}}
\def \var {\mathsf{Var}}
\def\1{\mathbbm{1}}
\def \FF {\mathbb{F}}


\newenvironment{keywords}
{\bgroup\leftskip 20pt\rightskip 20pt \small\noindent{\bfseries
Keywords:} \ignorespaces}%
{\par\egroup\vskip 0.25ex}
\newlength\aftertitskip     \newlength\beforetitskip
\newlength\interauthorskip  \newlength\aftermaketitskip















%%%%%%%%%%%%%%%%%%%%%%%%%%%% by Wu %%%%%%%%%%%%%%%%%%%%%%%%%%%%
\usepackage{xspace}

\newcommand{\Lip}{\mathrm{Lip}}
\newcommand{\stepa}[1]{\overset{\rm (a)}{#1}}
\newcommand{\stepb}[1]{\overset{\rm (b)}{#1}}
\newcommand{\stepc}[1]{\overset{\rm (c)}{#1}}
\newcommand{\stepd}[1]{\overset{\rm (d)}{#1}}
\newcommand{\stepe}[1]{\overset{\rm (e)}{#1}}
\newcommand{\stepf}[1]{\overset{\rm (f)}{#1}}


\newcommand{\floor}[1]{{\left\lfloor {#1} \right \rfloor}}
\newcommand{\ceil}[1]{{\left\lceil {#1} \right \rceil}}

\newcommand{\blambda}{\bar{\lambda}}
\newcommand{\reals}{\mathbb{R}}
\newcommand{\naturals}{\mathbb{N}}
\newcommand{\integers}{\mathbb{Z}}
\newcommand{\Expect}{\mathbb{E}}
\newcommand{\expect}[1]{\mathbb{E}\left[#1\right]}
\newcommand{\Prob}{\mathbb{P}}
\newcommand{\prob}[1]{\mathbb{P}\left[#1\right]}
\newcommand{\pprob}[1]{\mathbb{P}[#1]}
\newcommand{\intd}{{\rm d}}
\newcommand{\TV}{{\sf TV}}
\newcommand{\LC}{{\sf LC}}
\newcommand{\PW}{{\sf PW}}
\newcommand{\htheta}{\hat{\theta}}
\newcommand{\eexp}{{\rm e}}
\newcommand{\expects}[2]{\mathbb{E}_{#2}\left[ #1 \right]}
\newcommand{\diff}{{\rm d}}
\newcommand{\eg}{e.g.\xspace}
\newcommand{\ie}{i.e.\xspace}
\newcommand{\iid}{i.i.d.\xspace}
\newcommand{\fracp}[2]{\frac{\partial #1}{\partial #2}}
\newcommand{\fracpk}[3]{\frac{\partial^{#3} #1}{\partial #2^{#3}}}
\newcommand{\fracd}[2]{\frac{\diff #1}{\diff #2}}
\newcommand{\fracdk}[3]{\frac{\diff^{#3} #1}{\diff #2^{#3}}}
\newcommand{\renyi}{R\'enyi\xspace}
\newcommand{\lpnorm}[1]{\left\|{#1} \right\|_{p}}
\newcommand{\linf}[1]{\left\|{#1} \right\|_{\infty}}
\newcommand{\lnorm}[2]{\left\|{#1} \right\|_{{#2}}}
\newcommand{\Lploc}[1]{L^{#1}_{\rm loc}}
\newcommand{\hellinger}{d_{\rm H}}
\newcommand{\Fnorm}[1]{\lnorm{#1}{\rm F}}
%% parenthesis
\newcommand{\pth}[1]{\left( #1 \right)}
\newcommand{\qth}[1]{\left[ #1 \right]}
\newcommand{\sth}[1]{\left\{ #1 \right\}}
\newcommand{\bpth}[1]{\Bigg( #1 \Bigg)}
\newcommand{\bqth}[1]{\Bigg[ #1 \Bigg]}
\newcommand{\bsth}[1]{\Bigg\{ #1 \Bigg\}}
\newcommand{\xxx}{\textbf{xxx}\xspace}
\newcommand{\toprob}{{\xrightarrow{\Prob}}}
\newcommand{\tolp}[1]{{\xrightarrow{L^{#1}}}}
\newcommand{\toas}{{\xrightarrow{{\rm a.s.}}}}
\newcommand{\toae}{{\xrightarrow{{\rm a.e.}}}}
\newcommand{\todistr}{{\xrightarrow{{\rm D}}}}
\newcommand{\eqdistr}{{\stackrel{\rm D}{=}}}
\newcommand{\iiddistr}{{\stackrel{\text{\iid}}{\sim}}}
%\newcommand{\var}{\mathsf{var}}
\newcommand\indep{\protect\mathpalette{\protect\independenT}{\perp}}
\def\independenT#1#2{\mathrel{\rlap{$#1#2$}\mkern2mu{#1#2}}}
\newcommand{\Bern}{\text{Bern}}
\newcommand{\Poi}{\mathsf{Poi}}
\newcommand{\iprod}[2]{\left \langle #1, #2 \right\rangle}
\newcommand{\Iprod}[2]{\langle #1, #2 \rangle}
\newcommand{\indc}[1]{{\mathbf{1}_{\left\{{#1}\right\}}}}
\newcommand{\Indc}{\mathbf{1}}
\newcommand{\regoff}[1]{\textsf{Reg}_{\mathcal{F}}^{\text{off}} (#1)}
\newcommand{\regon}[1]{\textsf{Reg}_{\mathcal{F}}^{\text{on}} (#1)}

\definecolor{myblue}{rgb}{.8, .8, 1}
\definecolor{mathblue}{rgb}{0.2472, 0.24, 0.6} % mathematica's Color[1, 1--3]
\definecolor{mathred}{rgb}{0.6, 0.24, 0.442893}
\definecolor{mathyellow}{rgb}{0.6, 0.547014, 0.24}


\newcommand{\red}{\color{red}}
\newcommand{\blue}{\color{blue}}
\newcommand{\nb}[1]{{\sf\blue[#1]}}
\newcommand{\nbr}[1]{{\sf\red[#1]}}

\newcommand{\tmu}{{\tilde{\mu}}}
\newcommand{\tf}{{\tilde{f}}}
\newcommand{\tp}{\tilde{p}}
\newcommand{\tilh}{{\tilde{h}}}
\newcommand{\tu}{{\tilde{u}}}
\newcommand{\tx}{{\tilde{x}}}
\newcommand{\ty}{{\tilde{y}}}
\newcommand{\tz}{{\tilde{z}}}
\newcommand{\tA}{{\tilde{A}}}
\newcommand{\tB}{{\tilde{B}}}
\newcommand{\tC}{{\tilde{C}}}
\newcommand{\tD}{{\tilde{D}}}
\newcommand{\tE}{{\tilde{E}}}
\newcommand{\tF}{{\tilde{F}}}
\newcommand{\tG}{{\tilde{G}}}
\newcommand{\tH}{{\tilde{H}}}
\newcommand{\tI}{{\tilde{I}}}
\newcommand{\tJ}{{\tilde{J}}}
\newcommand{\tK}{{\tilde{K}}}
\newcommand{\tL}{{\tilde{L}}}
\newcommand{\tM}{{\tilde{M}}}
\newcommand{\tN}{{\tilde{N}}}
\newcommand{\tO}{{\tilde{O}}}
\newcommand{\tP}{{\tilde{P}}}
\newcommand{\tQ}{{\tilde{Q}}}
\newcommand{\tR}{{\tilde{R}}}
\newcommand{\tS}{{\tilde{S}}}
\newcommand{\tT}{{\tilde{T}}}
\newcommand{\tU}{{\tilde{U}}}
\newcommand{\tV}{{\tilde{V}}}
\newcommand{\tW}{{\tilde{W}}}
\newcommand{\tX}{{\tilde{X}}}
\newcommand{\tY}{{\tilde{Y}}}
\newcommand{\tZ}{{\tilde{Z}}}

\newcommand{\sfa}{{\mathsf{a}}}
\newcommand{\sfb}{{\mathsf{b}}}
\newcommand{\sfc}{{\mathsf{c}}}
\newcommand{\sfd}{{\mathsf{d}}}
\newcommand{\sfe}{{\mathsf{e}}}
\newcommand{\sff}{{\mathsf{f}}}
\newcommand{\sfg}{{\mathsf{g}}}
\newcommand{\sfh}{{\mathsf{h}}}
\newcommand{\sfi}{{\mathsf{i}}}
\newcommand{\sfj}{{\mathsf{j}}}
\newcommand{\sfk}{{\mathsf{k}}}
\newcommand{\sfl}{{\mathsf{l}}}
\newcommand{\sfm}{{\mathsf{m}}}
\newcommand{\sfn}{{\mathsf{n}}}
\newcommand{\sfo}{{\mathsf{o}}}
\newcommand{\sfp}{{\mathsf{p}}}
\newcommand{\sfq}{{\mathsf{q}}}
\newcommand{\sfr}{{\mathsf{r}}}
\newcommand{\sfs}{{\mathsf{s}}}
\newcommand{\sft}{{\mathsf{t}}}
\newcommand{\sfu}{{\mathsf{u}}}
\newcommand{\sfv}{{\mathsf{v}}}
\newcommand{\sfw}{{\mathsf{w}}}
\newcommand{\sfx}{{\mathsf{x}}}
\newcommand{\sfy}{{\mathsf{y}}}
\newcommand{\sfz}{{\mathsf{z}}}
\newcommand{\sfA}{{\mathsf{A}}}
\newcommand{\sfB}{{\mathsf{B}}}
\newcommand{\sfC}{{\mathsf{C}}}
\newcommand{\sfD}{{\mathsf{D}}}
\newcommand{\sfE}{{\mathsf{E}}}
\newcommand{\sfF}{{\mathsf{F}}}
\newcommand{\sfG}{{\mathsf{G}}}
\newcommand{\sfH}{{\mathsf{H}}}
\newcommand{\sfI}{{\mathsf{I}}}
\newcommand{\sfJ}{{\mathsf{J}}}
\newcommand{\sfK}{{\mathsf{K}}}
\newcommand{\sfL}{{\mathsf{L}}}
\newcommand{\sfM}{{\mathsf{M}}}
\newcommand{\sfN}{{\mathsf{N}}}
\newcommand{\sfO}{{\mathsf{O}}}
\newcommand{\sfP}{{\mathsf{P}}}
\newcommand{\sfQ}{{\mathsf{Q}}}
\newcommand{\sfR}{{\mathsf{R}}}
\newcommand{\sfS}{{\mathsf{S}}}
\newcommand{\sfT}{{\mathsf{T}}}
\newcommand{\sfU}{{\mathsf{U}}}
\newcommand{\sfV}{{\mathsf{V}}}
\newcommand{\sfW}{{\mathsf{W}}}
\newcommand{\sfX}{{\mathsf{X}}}
\newcommand{\sfY}{{\mathsf{Y}}}
\newcommand{\sfZ}{{\mathsf{Z}}}


\newcommand{\calA}{{\mathcal{A}}}
\newcommand{\calB}{{\mathcal{B}}}
\newcommand{\calC}{{\mathcal{C}}}
\newcommand{\calD}{{\mathcal{D}}}
\newcommand{\calE}{{\mathcal{E}}}
\newcommand{\calF}{{\mathcal{F}}}
\newcommand{\calG}{{\mathcal{G}}}
\newcommand{\calH}{{\mathcal{H}}}
\newcommand{\calI}{{\mathcal{I}}}
\newcommand{\calJ}{{\mathcal{J}}}
\newcommand{\calK}{{\mathcal{K}}}
\newcommand{\calL}{{\mathcal{L}}}
\newcommand{\calM}{{\mathcal{M}}}
\newcommand{\calN}{{\mathcal{N}}}
\newcommand{\calO}{{\mathcal{O}}}
\newcommand{\calP}{{\mathcal{P}}}
\newcommand{\calQ}{{\mathcal{Q}}}
\newcommand{\calR}{{\mathcal{R}}}
\newcommand{\calS}{{\mathcal{S}}}
\newcommand{\calT}{{\mathcal{T}}}
\newcommand{\calU}{{\mathcal{U}}}
\newcommand{\calV}{{\mathcal{V}}}
\newcommand{\calW}{{\mathcal{W}}}
\newcommand{\calX}{{\mathcal{X}}}
\newcommand{\calY}{{\mathcal{Y}}}
\newcommand{\calZ}{{\mathcal{Z}}}

\newcommand{\bara}{{\bar{a}}}
\newcommand{\barb}{{\bar{b}}}
\newcommand{\barc}{{\bar{c}}}
\newcommand{\bard}{{\bar{d}}}
\newcommand{\bare}{{\bar{e}}}
\newcommand{\barf}{{\bar{f}}}
\newcommand{\barg}{{\bar{g}}}
\newcommand{\barh}{{\bar{h}}}
\newcommand{\bari}{{\bar{i}}}
\newcommand{\barj}{{\bar{j}}}
\newcommand{\bark}{{\bar{k}}}
\newcommand{\barl}{{\bar{l}}}
\newcommand{\barm}{{\bar{m}}}
\newcommand{\barn}{{\bar{n}}}
\newcommand{\baro}{{\bar{o}}}
\newcommand{\barp}{{\bar{p}}}
\newcommand{\barq}{{\bar{q}}}
\newcommand{\barr}{{\bar{r}}}
\newcommand{\bars}{{\bar{s}}}
\newcommand{\bart}{{\bar{t}}}
\newcommand{\baru}{{\bar{u}}}
\newcommand{\barv}{{\bar{v}}}
\newcommand{\barw}{{\bar{w}}}
\newcommand{\barx}{{\bar{x}}}
\newcommand{\bary}{{\bar{y}}}
\newcommand{\barz}{{\bar{z}}}
\newcommand{\barA}{{\bar{A}}}
\newcommand{\barB}{{\bar{B}}}
\newcommand{\barC}{{\bar{C}}}
\newcommand{\barD}{{\bar{D}}}
\newcommand{\barE}{{\bar{E}}}
\newcommand{\barF}{{\bar{F}}}
\newcommand{\barG}{{\bar{G}}}
\newcommand{\barH}{{\bar{H}}}
\newcommand{\barI}{{\bar{I}}}
\newcommand{\barJ}{{\bar{J}}}
\newcommand{\barK}{{\bar{K}}}
\newcommand{\barL}{{\bar{L}}}
\newcommand{\barM}{{\bar{M}}}
\newcommand{\barN}{{\bar{N}}}
\newcommand{\barO}{{\bar{O}}}
\newcommand{\barP}{{\bar{P}}}
\newcommand{\barQ}{{\bar{Q}}}
\newcommand{\barR}{{\bar{R}}}
\newcommand{\barS}{{\bar{S}}}
\newcommand{\barT}{{\bar{T}}}
\newcommand{\barU}{{\bar{U}}}
\newcommand{\barV}{{\bar{V}}}
\newcommand{\barW}{{\bar{W}}}
\newcommand{\barX}{{\bar{X}}}
\newcommand{\barY}{{\bar{Y}}}
\newcommand{\barZ}{{\bar{Z}}}

\newcommand{\hX}{\hat{X}}
\newcommand{\Ent}{\mathsf{Ent}}
\newcommand{\awarm}{{A_{\text{warm}}}}
\newcommand{\thetaLS}{{\widehat{\theta}^{\text{\rm LS}}}}

\newcommand{\jiao}[1]{\langle{#1}\rangle}
\newcommand{\gaht}{\textsc{GoodActionHypTest}\;}
\newcommand{\iaht}{\textsc{InitialActionHypTest}\;}
\newcommand{\true}{\textsf{True}\;}
\newcommand{\false}{\textsf{False}\;}

% \usepackage[capitalize,noabbrev]{cleveref}
% \crefname{lemma}{Lemma}{Lemmas}
% \Crefname{lemma}{Lemma}{Lemmas}
% \crefname{thm}{Theorem}{Theorems}
% \Crefname{thm}{Theorem}{Theorems}
% \Crefname{assumption}{Assumption}{Assumptions}
% \Crefname{inftheorem}{Informal Theorem}{Informal Theorems}
% \crefformat{equation}{(#2#1#3)}

% % if you use cleveref..
% \usepackage[capitalize,noabbrev]{cleveref}
% \crefname{lemma}{Lemma}{Lemmas}
% \crefname{proposition}{Proposition}{Propositions}
% \crefname{remark}{Remark}{Remarks}
% \crefname{corollary}{Corollary}{Corollaries}
% \crefname{definition}{Definition}{Definitions}
% \crefname{conjecture}{Conjecture}{Conjectures}
% \crefname{figure}{Fig.}{Figures}

\usepackage{Settings/my_commands}
\usepackage{Settings/listing_algorithm}
\usepackage{caption}
\captionsetup[table]{position=bottom}

\begin{document}

\title{Unveiling ECC Vulnerabilities: LSTM Networks for Operation Recognition in Side-Channel Attacks}

\author{Alberto Battistello\inst{1} \and Guido Bertoni\inst{1} \and Michele Corrias\inst{1} \and Lorenzo Nava\inst{1} \and Davide Rusconi\inst{2} \and Matteo Zoia\inst{2} \and Fabio Pierazzi\inst{3} \and Andrea Lanzi\inst{2}}

\authorrunning{Battistello et al.}

\institute{Security Pattern, Milan, Italy \and
University of Milan, Milan, Italy \and
King's College London, London, United Kingdom}

\maketitle 

\begin{abstract}
We propose a novel approach for performing side-channel attacks on elliptic curve cryptography. Unlike previous approaches and inspired by the ``activity detection'' literature, we adopt a long-short-term memory (LSTM) neural network to analyze a power trace and identify patterns of operation in the scalar multiplication algorithm performed during an ECDSA signature,  that allows us to recover bits of the ephemeral key, and thus retrieve the signer's private key. Our approach is based on the fact that modular reductions are conditionally performed by micro-ecc and depend on key bits. 

We evaluated the feasibility and reproducibility of our attack through experiments in both simulated and real implementations. We demonstrate the effectiveness of our attack by implementing it on a real target device, an STM32F415 with the micro-ecc library, and successfully compromise it. 
Furthermore, we show that current countermeasures, specifically the coordinate randomization technique, are not sufficient to protect against side channels. Finally, we suggest other approaches that may be implemented to thwart our attack. 

\keywords{Hardware security \and Side-channel attacks \and Elliptic curve cryptography \and Key recovery \and Deep learning}
\end{abstract}

\section{Introduction}

Within the security domain, Machine Learning (ML) methods have been utilized to address issues such as email spam filtering \cite{tretyakov2004machine,crawford2015survey}, intrusion detection systems \cite{tsai2009intrusion,buczak2015survey}, and facial recognition~\cite{taigman2014deepface,oravec2014feature}. Although these uses highlight the effectiveness of ML in protective security strategies, its influence also spans offensive actions like side-channel attacks (SCAs). In SCAs, attackers examine physical attributes of devices, such as timing variations, power usage, and electromagnetic signals during computations, to uncover confidential information, emphasizing the dual use nature of ML in the security field \cite{kocher1996timing,kocher1999differential,quisquater2001electromagnetic}.

This duality is particularly prominent in cryptographic applications, such as the Elliptic Curve Digital Signature Algorithm (ECDSA), where physical attributes can reveal critical secret information, presenting a significant vulnerability. Our research focuses on side channel attacks (SCA) against ECDSA implementations, particularly those \textit{ protected with the coordinate randomization mechanism}, as discussed in~\cite{rivain2011fast}. Based on this foundation, our study proposes an innovative technique using a long- and short-term memory (LSTM) network architecture~\cite{yu2019review} to identify the execution patterns of operations in the scalar multiplication process, an essential part of ECDSA. By examining these operational patterns, our method enables the extraction of ephemeral key bits, potentially leading to the compromise of the signer's full private key. Our study explores the real-world implementation of these insights in elliptic curve cryptography, specifically using the micro-ecc library~\cite{microecc}. This is a widely used open source library, especially common in Internet of Things (IoT) applications.

Luo et al.~\cite{luo2018effective} illustrated the possibility of attacking the ECDSA by using collision attacks, which leverage the detection of specific operational patterns in the traces of power consumption. In these attacks, "collision" signifies the recurrence of an identical value at various stages of the algorithm, identifiable through pattern analysis and correlation methods. Our research improves the attack strategy discussed by~\cite{luo2018effective}, presenting a more efficient SCA approach to exploit vulnerabilities in micro-ecc. By employing a machine learning-based approach and lattice techniques, our method concepts the detection of key operations and recovery of the signer's key, even with a few known ephemeral key bits from signature processes. Additionally, we show that our technique does not depend on electromagnetic (EM) analysis by effectively extracting key information from the noisy power consumption traces of advanced hardware such as the ARM Cortex-M4 processor in the STM32F4 series. This discovery represents a considerable advancement in side-channel attack strategies. Our results also question the efficacy of existing countermeasures against collision attacks, prompting us to suggest more robust alternatives that offer better protection against these and similar vulnerabilities. 

More in detail our framework stands out by employing Long-Short-Term Memory (LSTM) networks, which are well regarded for their proficiency in tasks analogous to human activity recognition. In particular, this approach enhances our analysis from basic side-channel attack execution to "operation recognition," akin to the methods used in detecting human activities. Using this strategy, we can analyze patterns in cryptographic operations in a detailed way, which is essential to detect hidden vulnerabilities. The LSTM network was specifically chosen for its aptitude in deciphering the sequential and temporal dynamics inherent in our attack's operational recognition phase. This decision was informed by the nature of our problem, which aligns more with operational recognition, a concept akin to human activity recognition, than with classic data classification. This delineation underlines the unsuitability of Convolutional Neural Networks (CNNs) for our analysis, as they fall short in capturing the essential temporal context. The LSTM's design, known for its proficiency in processing time-dependent data, perfectly aligns with the requirements of identifying and analyzing execution operational patterns, reinforcing our methodological choices with the goals of our side-channel attack strategy.

To evaluate the practicality of our attack technique, we employed the secp160r1 curve in a simulated setup, executing the attack using an LSTM-based neural network to determine the possibility of extracting the signer's private key. We conducted a successful experiment on the STM32F415 microcontroller~\cite{stmicroelectronics}, selected for its incorporation of the micro-ecc library. This library implements co-Z algorithms and includes a \textit{coordinate randomization countermeasure} to defend against known side-channel attacks (SCAs). To obtain the necessary side-channel traces for our investigation, we utilized the ChipWhisperer~\cite{chipwhisperer} platform, which allowed us to collect power consumption data as leakage vector. Using the same LSTM-based Neural Network (NN) model that was effective in our simulated settings, we modified and trained it specifically for this microcontroller. This was a crucial step in demonstrating that our attack strategy, which was refined through simulations and theoretical models, is not only conceptually sound but also practically applicable to actual hardware. This paper offers the following contributions:

\begin{itemize}

\item Developed an LSTM-based methodology for identifying operations execution patterns in scalar multiplication algorithms, enabling the extraction of ephemeral key bits in ECC.

\item Demonstrated the practicality of this approach through real-world testing on the STM32F415 device, utilizing the micro-ecc library and the secp160r1 curve with the \textit{protected ccoordinate randomization mechanism}, highlighting its adaptability to various types of curves.

\item Illustrated the limitations of existing countermeasures against side-channel attacks while providing an in-depth analysis of potential improvements in security protocols.

\end{itemize}

\section{Background}
\label{sec:background}

This Section provides an introduction to the basic cryptographic concepts necessary to understand the rest of this work. The following Section explores the specific cryptosystem solutions for elliptic curves that are implemented in the target library.
Consider an elliptic curve $\curve_\field$ defined over a field $\field$ of characteristic $\neq 2,3$ according to the short Weierstrass equation as:
\begin{equation}
\label{eq:ec_weierstrass}
    \mathcal{E}: y^2 = x^3 + ax + b \quad a, b \in \mathbb{K}
\end{equation}
with $a,b \in \field$ such that $4a^3+27b^2 \neq 0$. Given the curve, the set of points that satisfy its equation $(x, y) \in \mathbb{K}^2$, augmented with a particular point $\infinitePoint$ called \emph{point at infinity}, forms a \emph{additive abelian group} \cite{de1997elliptic}. 

\subsubsection{Scalar Multiplication}
\label{scalar}
A crucial operation on which protocols based on ECC are based is \emph{scalar multiplication} \cite{lopez2000overview}. Given an integer $k \in \field$ and a point $P$ on $\mathcal{E}$, the scalar multiplication $k$ with $P$ is defined as:
\begin{equation}
    \label{eq:scalarMul}
    kP = \underbrace{P + P + \dots + P}_{k \textsc{ times}}
\end{equation}
In this work we will focus on the scalar multiplication algorithm implemented in micro-ecc, the Montgomery Ladder
implementation with Jacobian co-Z coordinates [10], with coordinates randomization.

\subsubsection{Jacobian co-\textit{Z} representation}
\label{coordinates}
The Jacobian co-\textit{Z} representation
is a projective representation of the points of an elliptic curve that uses three coordinates.

In this system the input points of the operations are represented by triplets sharing the same Z coordinate. This representation was first suggested by Meloni~\cite{meloni2007new}, and then adapted with different trade-offs by Rivain~\cite{rivain2011fast}, and implemented in micro-ecc~\cite{microecc}.

\begin{algorithm}[caption={Montgomery ladder with \textit{(X, Y)}-only co-\textit{Z} addition}, label={alg:montgomeryLadderCoZ}]
input: $P \in \curve(\finiteFieldP)$, $k = (k_{n-1},\dots,k_1,k_0)_2 \in \naturals$/+\text{ with }+/$k_{n-1} = 1$
output: $kP$
begin
    $(R_1,R_0) \gets \texttt{XYCZ-IDBL}(P)$
    for $i \gets n - 2$ to $1$ do
        $b \gets k_i$
        $(R_{1-b},R_b) \gets \texttt{XYCZ-ADDC}(R_b, R_{1-b})$
        $(R_b,R_{1-b}) \gets \texttt{XYCZ-ADD}(R_{1-b}, R_b)$
    end
    $b \gets k_0$
    $(R_{1-b},R_b) \gets \texttt{XYCZ-ADDC}(R_b, R_{1-b})$
    $\lambda \gets \texttt{FinalInvZ}(R_0, R_1, P, b)$
    $(R_b,R_{1-b}) \gets \texttt{XYCZ-ADD}(R_{1-b}, R_b)$
    return $(X_0\lambda^2, Y_0\lambda^3)$
end
\end{algorithm}

\subsubsection{Montgomery ladder with \textit{(X, Y)}-only co-\textit{Z} addition}
\label{para:mlcoz}
The \emph{Montgomery ladder with (X, Y)-only co-Z addition}~\cite{rivain2011fast} \autoref{alg:montgomeryLadderCoZ} is an algorithm employing on Jacobian co-\textit{Z} coordinates 
used in the target ECC implementation~\cite{microecc}
to perform the scalar multiplication required for ECDSA. It performs an initial doubling (in micro-ecc it uses the Jacobian Doubling algorithm), then starts a loop on the bits of the scalar, where for each loop cycle a XYcZ-ADDC  followed by a XYcZ-ADD are executed.\\\\

\begin{algorithm}[caption={XYcZ-ADDC}, label={alg:XYCZ-ADDC}]
input: $(X_1,Y_1)$ and $(X_2,Y_2)$ s.t. $P \equiv (X_1 : Y_1 : Z)$ and $Q \equiv (X_2 : Y_2 : Z)$ for some $Z \in \F_q, P,Q \in \mathcal{E}(\F_q)$
output: $(X_3, Y_3)$ and $(X_3',Y_3')$ s.t. $P + Q \equiv (X_3 : Y_3 : Z_3)$ and $P - Q \equiv (X_3' : Y_3' : Z_3)$ for some $Z_3 \in \F_q$
begin
    $A \gets (X_2 - X_1)^2$
    $B \gets X_1A$
    $C \gets X_2A$
    $D \gets (Y_2 - Y_1)^2$
    $F \gets (Y_1 + Y_2)^2$
    $E \gets Y_1(C - B)$
    $X_3 \gets D - (B + C)$
    $Y_3 \gets (Y_2 - Y_1)(B - X_3) - E$
    $X_3' \gets F - (B + C)$
    $Y_3' \gets (Y_1 + Y_2)(X_3'-B) - E$
    return $((X_3,Y_3),(X_3',Y_3'))$
end
\end{algorithm}
\clearpage
\begin{algorithm}[caption={XYcZ-ADD}, label={alg:XYCZ-ADD}]
input: $(X_1,Y_1)$ and $(X_2,Y_2)$ s.t. $P \equiv (X_1 : Y_1 : Z)$ and $Q \equiv (X_2 : Y_2 : Z)$ for some $Z \in \F_q, P,Q \in \mathcal{E}(\F_q)$
output: $(X_3, Y_3)$ and $(X_1',Y_1')$ s.t. $P \equiv (X_1' : Y_1' : Z_3)$ and $P + Q \equiv (X_3 : Y_3 : Z_3)$ for some $Z_3 \in \F_q$
begin
    $A \gets (X_2 - X_1)^2$
    $B \gets X_1A$
    $C \gets X_2A$
    $D \gets (Y_2 - Y_1)^2$
    $E \gets Y_1(C - B)$
    $X_3 \gets D - (B + C)$
    $Y_3 \gets (Y_2 - Y_1)(B - X_3) - E$
    $X_1' \gets B$
    $Y_1' \gets E$
    return $((X_3,Y_3),(X_1',Y_1'))$
end
\end{algorithm}




\section{Related Work}

In the physical security domain, the adoption of Neural Networks (NNs) has marked a transformative phase, particularly in enhancing Side-Channel Attack (SCA) strategies. These NN-empowered SCAs have surpassed traditional approaches by yielding more potent results with reduced observational demands \cite{maghrebi2016breaking,picek2017side,wang2014learning,wu2021best,perin2021keep,nascimento2017applying,weissbart2019one,cagli2017convolutional,zaid2020methodology,carbone2019deep,picek2021sok}. Research initiatives \cite{maghrebi2019deep,ramezanpour2020scaul,benadjila2020deep}, have taken on sophisticated Deep Learning techniques to exploit side channel traces in examining symmetric algorithms. Focusing on the ECC Double-And-Add-Always algorithm implemented on FPGA platforms, Mukhtar et al.\cite{mukhtar2018machine} applied classification methods to reveal secret key bits of the ECC. In a similar vein, Weissbart et al.\cite{weissbart2019one} orchestrated a power analysis attack on the Edwards-curve Digital Signature Algorithm (EdDSA)\cite{bernstein2012high}, revealing the superior capabilities of CNN over classical side-channel techniques like Template Attacks~\cite{chari2002template}. Weissbart et al.\cite{weissbart2020systematic} further expanded their investigation, evaluating additional protected targets and highlighting the efficacy of Deep Learning, particularly CNNs, in breaching protected implementations of scalar multiplication on Curve25519\cite{bernstein2006curve25519}.

Perin et al.\cite{perin2021keep} proposed a groundbreaking Deep Learning-based iterative framework for unsupervised horizontal attacks, aimed at refining the accuracy of single-trace attacks and reducing errors in the decryption of private keys, particularly in protected ECC implementations. This effort, similar to Nascimento et al.\cite{nascimento2017applying}, exploited vulnerabilities within the $\mu$NaCl library's \texttt{cswap} function~\cite{NACLlib}, showcasing the ongoing evolution of NN-enabled SCA methodologies in enhancing cryptographic security. In a recent development, Staib et al.\cite{staib2023deep} sought to advance collision side channel attacks through deep learning, demonstrating a neural network's superiority in collision detection over traditional methods on a public dataset\cite{luo2018effective,clavier2011improved,bauer2015horizontal}. 

Our research uses long- and short-term memory (LSTM) networks, which are renowned for their effectiveness in tasks that resemble recognition of human activity. This innovative approach not only goes beyond the traditional scope of side-channel attacks (SCA) but reframes the problem as an "operation recognition" task. By adopting methodologies similar to those in human activity detection, we enable sophisticated pattern recognition within cryptographic operations. This unique framework is pivotal for uncovering hidden vulnerabilities, highlighting the intricate interplay between cryptographic processes and exploitable weaknesses.

\section{Attack Preconditions}
\label{sec:attackz}

Our objective for this attack is the \textit{micro-ecc} library, which is an open-source implementation of ECDH and ECDSA tailored for 8/32/64 bit processors \cite{microecc}. Although this library supports multiple elliptic curves, our study specifically targets \textit{secp160r1}. This choice is driven by the widespread use of the curve in resource-limited environments such as IoT, where both security and efficiency are paramount, making it an important focus of our investigation. Our target utilizes the Jacobian Co-\textit{Z} representation along with the Montgomery ladder algorithm (\autoref{alg:montgomeryLadderCoZ}), for cryptographic processes. Furthermore, the library incorporates a countermeasure of coordinate randomization, which produces a random $z \in \finiteFieldQ$ before each execution. This random $z$ is then applied to the representation of points before entering the Montgomery ladder. This initial randomization ensures that the values computed during the ladder are not correlated with the original point, making modular reductions unpredictable based on guesses of bits from the ephemeral key. However, we have discovered several issues that make this implementation susceptible to side-channel attacks, which will be discussed in the following.

\paragraph{Consistent Timing of the Implementation.}
\label{subsec:ct}
The initial problem arises because the micro-ecc code performs a conditional \emph{modular reduction}, which is triggered by an over / overflow occurring after an addition or subtraction operation. This implies that the algorithm does not run in constant time, thereby leaking information during execution. However, exploiting this vulnerability (e.g., intercepting the over/underflow operations) is made difficult by the randomization of the coordinates. This countermeasure effectively nullifies any predictive attempts without prior knowledge of the randomness used in the countermeasure, thus preventing exploitation of the issue \textit{ as is}.

\paragraph{Repeated Operations Depending on Key Bit.}The second vulnerability in the micro-ecc implementation arises from the Montgomery ladder (\autoref{alg:montgomeryLadderCoZ}) performing certain calculations twice, based on the ephemeral key bit. Specifically, during the $(n-1)$\textsuperscript{th} iteration of the Montgomery ladder, the value $B - X_3$ is calculated. Subsequently, both $B$ and $X_3$ are returned as the x-coordinates of the two output points ($X_1'$ and $X_3$, respectively). In the next $n$\textsuperscript{th} iteration of the Montgomery ladder, depending on the $n$\textsuperscript{th} bit of the scalar, the implementation either recalculates $B - X_3$ or computes $X_3 - B$. This means that whenever two consecutive bits of the scalar are identical, the $B - X_3$ operation is performed twice. 

\paragraph{Attacking the First Bit.} The final issue identified in the target system is a leak resulting from the interaction between the Jacobian doubling operation, performed before the Montgomery ladder, and the addition operation executed during it. Specifically, the value of the first key bit determines whether the same subtraction is performed or its inverse is calculated. Consequently, detecting a collision between the subtraction within the Jacobian doubling and the first iteration of the Montgomery ladder reveals the first bit of the ephemeral key.
\\\\
Our attack exploits all three identified issues to leak the initial bits of the ephemeral key and subsequently infer the private key. Specifically, we used a timing side channel to detect whether an overflow occurred during the addition operation, leveraging the first issue. Then, by distinguishing between overflowing and non-overflowing subtractions, we directly leak the first bit using the third issue. Additionally, we exploit the second vulnerability to infer the equality of the first bits. This information enables us to mount a lattice reduction attack, effectively extracting the private key used in the signatures, thereby compromising the system's security. To achieve this, we need to design a neural network capable of identifying the operating patterns based on the power trace.

\section{System Overview}
\label{sec:sys_ov}
To exploit the vulnerability described in Section \ref{sec:attackz}, we have developed a system consisting of five main components: the acquisition unit responsible for acquiring the power traces from the target device, the windowing algorithm is primarily designed to divide the power traces into smaller segments (e.g., windows), a machine learning model that categorizes the acquired traces, a post-processing unit that analyzes the output of the ML model to identify the presence or absence of modular operations within the traces, and a final component that deduces the scalar bits as discussed in Section \ref{sec:attackz} and extracts the key.

As illustrated in Figure~\ref{fig:sys_arch}, the process begins with recording raw power traces from the targeted device through an acquisition unit. These raw traces are then preprocessed and divided into multiple windows of uniform size (e.g, windowing algorithm). Each window is subsequently fed into the neural network, where our model determines whether the operation within the window is a short operation (SO), such as addition or subtraction, or a long operation (LO), such as multiplication and division. Based on the neural network's classification, the window vector is transformed into a binary vector, enabling us to identify the type of operation executed. For short operations, we also check for overflow. This generates a list of operations and details concerning modular reductions, which can be utilized, following the leaks in \autoref{sec:attackz}, to determine the values and relationships of the bits of the ephemeral key used in the attacked ECDSA round. With sufficient leakage information on the ephemeral key, we can proceed to recover the private key using a known lattice reduction technique, thus compromising the target and completing the attack.


\section{System Architecture}
\label{sec:sys}
\begin{figure*}[h]
    \centering\includegraphics[width=0.95\textwidth]{images/ecc-ml-schema.pdf}
    \caption{Architectural Overview}  \label{fig:sys_arch}
\end{figure*}

We will now describe the design of each component of our framework, except for the unit responsible for obtaining power consumption traces, as its design depends on the target device. Each component represents a part of our contribution and has been designed to target the cryptographic algorithm.

\subsection{Pre-Processing}

As described in \autoref{sec:sys_ov}, we employ a sliding window algorithm to segment the entire trace into smaller, fixed-size sequences. This method offers several advantages for our analysis and is particularly beneficial when used in conjunction with long-short-term memory (LSTM) networks, known for their effectiveness in tasks similar to recognizing human activities.

Firstly, the use of a sliding window enables us to focus on manageable chunks of data, which simplifies the computational process. By tuning the window size and the offset between adjacent windows, both of which are hyperparameters, we can optimize our analysis for both performance and accuracy. After rigorous testing and evaluation, we determined that a window size of 500 samples and an offset of 10 samples provide the best results. The chosen window size of 500 samples produces an optimal balance by ensuring that each segment contains sufficient detail to accurately classify the operations within it, while also maintaining computational efficiency. A larger window might capture more detail but would significantly increase computational overhead, potentially slowing down the analysis. In contrast, a smaller window might be more efficient, but could miss critical information necessary for accurate classification. The 10-sample offset ensures that we achieve a high-resolution analysis by overlapping windows, allowing for a more granular examination of the data. This overlap helps to ensure that no important transitions or details are missed between windows, enhancing the overall accuracy of our classification. Integrating the sliding-window approach with LSTM networks offers significant benefits. LSTMs are designed to capture temporal dependencies and trends within sequential data, which is crucial for accurate classification.

From a machine learning perspective, splitting the data into windows can significantly improve the efficacy of our LSTM model. By creating a larger number of training samples from the original dataset, we expose the LSTM model to a greater variety of patterns and transitions within the data. This increased dataset size allows the model to learn more effectively and generalize better. Moreover, by tuning the window size and offset, we can control the granularity of the input data, allowing the model to focus on the most relevant features and patterns. 

\subsection{Neural Network Architecture \& Parameters}\label{sec:nn}

In this study, we created a neural network framework aimed at effectively handling binary classification tasks, integrating both convolutional and long-short-term memory (LSTM) layers. This framework is intended to recognize intricate patterns and temporal relationships within the data, thus enhancing the model's predictive accuracy. Our framework is designed to balance computational efficiency with the capacity to extract and utilize significant features, leading to a reliable and robust model. The neural network architecture selected for our framework is a sequential model consisting of the following layers.

\begin{itemize}[noitemsep]
\item \textbf{Convolutional Layer:} This layer performs convolution operations on the input data to decrease its dimensions and transform its structure, thus increasing computational efficiency. The convolutional layers enhance the resilience to noise and scale changes in the input traces. In particular, this layer contains 64 feature maps, each with a 3x3 kernel, and applies the ReLU activation function to incorporate nonlinearity, allowing the network to capture intricate patterns.

\item \textbf{Pooling Layer:} Following the convolutional layer, this layer performs average pooling, reducing the dimensionality of the feature maps produced by the convolutional operations. The pooling layer uses a pooling size of 10, which helps in further down-sampling the input and retaining the most salient features.

\item \textbf{LSTM Layer:} This layer forms the core of our neural network and consists of 1000 internal units. LSTM (Long Short-Term Memory) networks are adept at capturing long-term dependencies and temporal correlations within sequential data. Using previous sample information, the LSTM layer is able to make context-sensitive evaluations, crucial for tasks involving temporal sequences.

\item \textbf{Dropout Layer:} To prevent overfitting and enhance the generalization capabilities of the network, this layer randomly drops a subset of its nodes during training. The dropout rate is set to 0.5, which means that 50\% of the nodes are temporarily ignored during each training iteration. This technique helps the network develop more robust features by mitigating the dependence on any specific subset of neurons.

\item \textbf{Dense Layer:} This fully connected layer comprises 1000 neurons and employs the ReLU activation function. The dense layer integrates information from the preceding layers and contributes to the network's ability to perform high-level abstractions and complex decision making.

\item \textbf{Output Layer:} The final layer of the network is responsible for the binary classification task. It consists of dense nodes that utilize the softmax activation function to produce probabilities for the binary classes, enabling the network to produce the final classification decision.

\end{itemize}

The LSTM network utilizes the binary cross-entropy loss function for training, with optimization done through stochastic gradient descent (SGD). Inspired by real-world applications in the literature \cite{zhang2022multi}, we chose a classic architecture that is typically used in recognition of human activity. In total, the network comprises 5,263,258 trainable parameters, allocated as follows: \textit{Convolutional Layer:} 256 parameters \textit{LSTM Layer:} 4,260,000 parameters (the main component of our model) \textit{Dense Layer:} 1,001,000 parameters \textit{Output Layer:} 2,002 parameters.

\subsection{Post-Processing to Identify Operations}

In order to transform the model output into a series of operations, a two-step method is utilized. The initial step involves aligning the classified operations within the algorithm. The model output already identifies whether the operations within the windows are short operations (SO) or not, setting the groundwork for further analysis. The second step differentiates between short operations with overflows and those without, which is critical for our attack. This differentiation is based on the observation that overflows cause a modular reduction, leading to an increased duration. Thus, by examining a continuous sequence of windows identified as short operations, we can estimate the duration of each individual operation in the sequence by analyzing the starting points of SOs and the subsequent operations.  Specifically, the procedure involves the following steps:

\begin{enumerate}
    \item \textbf{Classification Matching:} Use the output of the model to determine the particular operations that occur within the algorithm, depending on whether each window is classified as containing a short operation or not.
    \item \textbf{Duration Analysis:} For groups of consecutive windows marked as short operations, determine the estimated duration of each operation. An extended duration can suggest the occurrence of a modular reduction, which indicates an overflow. This distinction uses the pre-determined window length and step size to precisely measure operation durations.
\end{enumerate}

It is crucial to understand that the time taken for a subtraction operation involving modular reduction can differ between various target devices. Therefore, when implementing this post-processing step on a new target device, modifications must be made to consider the unique physical properties of that device. This adaptation guarantees a precise differentiation between operations in diverse hardware contexts. This dual-phase approach improves the ability to correctly detect and distinguish between brief operations with and without overflows, thereby enhancing the attack's overall efficiency.

\subsection{Collision Template \& Leaking the Ephemeral Key}
Following the identification operations process, it is necessary to deploy a collision attack template to retrieve the bits of the ephemeral key. This template can indicate the locations where collisions occur in the algorithm's operations, and from these collisions one can infer the positions and values of the ephemeral key bits. This phase relies on the specific implementation of the cipher algorithm and must be adapted if the algorithm is modified. To construct the collision template model and extract the ephemeral key bits, we started with a simulation of the target algorithm, as outlined in Section \ref{sec:eval}. This simulation gives us insight into the intermediate values during the computation and the sequence of executed instructions, noting the occurrences of overflows and underflows. Using these insights and the knowledge from \autoref{sec:attackz}, we identified specific operations that reveal ephemeral key bits, even with coordinate randomization countermeasures in effect.

In~\autoref{tab:key_op01}, we present examples of the algorithm simulation that selected operations across different initial bits and multiple keys within the coordinate randomization defense framework. These operations include the Initial Double and the first iteration of the Montgomery loop, with ``\texttt{||}'' indicating the start of the loop. The operations are labeled as ``\texttt{A}'' for ``\emph{Add}, \texttt{S}'' for \emph{Sub}, and ``\texttt{M}'' for both \emph{Mul} and \emph{Sqr}, with a ``\texttt{+}'' next to an operation representing a modular reduction. 
A \emph{collision} is described as the occurrence of identical underflow behavior in two operations, underscoring their importance in our analysis.

% \begin{table*}
% \setlength{\tabcolsep}{1.8mm}
% \centering
% \begin{tabular}{clcccccc} 
% \toprule
% \multirow{2}{*}{\textbf{Category}}     & \multicolumn{1}{c}{\multirow{2}{*}{\textbf{Method}}} & \multicolumn{3}{c}{\textbf{AAPM}}                    & \multicolumn{3}{c}{\textbf{COVID-19}}     \\ 
% \cmidrule(lr){3-5}\cmidrule(lr){6-8}
%                               & \multicolumn{1}{c}{}                        & 30 Views             & 60 Views & 90 Views & 30 Views & 60 Views & 90 Views  \\ 
% \midrule
% Model-based                 & FBP                                         &   19.46/0.3139                  & 23.92/0.4769          &   27.06/0.6075           &   18.57/0.3273        &  22.96/0.4578          &  25.77/0.5574            \\ 
% \cmidrule(lr){1-8}
% \multirow{2}{*}{Supervised}   & FBPConvNet       & 29.51/0.9162 &       32.88/0.9532   &      35.04/0.9669       &         &          &            \\
%                               & RegFormer         &     36.38/0.9257    &         41.38/0.9695  &      43.72/0.9798    &    29.11/0.8207      &    36.28/0.9424      &  39.24/0.9613           \\ 
% \cmidrule(lr){1-8}
% \multirow{3}{*}{Unsupervised} & CoIL     &      28.15/0.7872        &    33.67/0.8965       &   36.58/0.9370        &          &          &            \\
%                               & SCOPE            &    30.34/0.8256       & 36.03/0.9248         &  39.36/0.9587        &     28.66/0.8072     &   34.84/0.9250       &   38.54/0.9611         \\
%                               & En-INR      & 30.74/0.8303                     & 37.27/0.9430        & \textbf{41.09}/\textbf{0.9704}           &          &      35.04/0.9372    &    40.65/0.9759       \\
% \bottomrule
% \end{tabular}
% \caption{Quantitative results of compared methods on AAPM and COVID-19 dataset.}
% \end{table*}

% \begin{table*}

% \centering
% \begin{tabular}{clcccc} 
% \toprule
% \multirow{2.5}{*}{\textbf{Category}} & \multicolumn{1}{c}{\multirow{2.5}{*}{\textbf{Method}}} & \multicolumn{2}{c}{\textbf{AAPM}}             & \multicolumn{2}{c}{\textbf{COVID-19}}  \\ 
% \cmidrule(lr){3-4} \cmidrule(lr){5-6}
%                                    & \multicolumn{1}{c}{}                                 & \textbf{60 Views}     & \textbf{90 Views}                       &  \textbf{60 Views}     & \textbf{90 Views}                \\ 
% \midrule
% \texttt{Analytical}                        & FBP                                                  & 23.61/0.4786 & 26.46/0.6050                   & 22.96/0.4578 & 25.77/0.5574            \\ 
% \cmidrule(lr){1-6}
% \multirow{2}{*}{\texttt{Supervised}}        & FBPConvNet                                           & 29.82/0.8588 & 31.05/0.8864                  &   28.20/0.7717           &       29.27/0.8080                  \\
%                                    & RegFormer                                            & \underline{33.78}/\textbf{0.9399} & \underline{34.61}/\textbf{0.9564}                   & \underline{30.25}/0.8436 & 31.02/0.8810            \\ 
% \cmidrule(lr){1-6}
% \multirow{3}{*}{\texttt{Unsupervised}}      & CoIL                                                 & 30.18/0.8604 &     31.84/0.9032               &  27.77/0.7156            &   29.72/0.7819                      \\
%                                    & SCOPE                                                 & 32.40/0.8939 & 34.31/0.9322   & 30.21/\underline{0.8504} &  \underline{32.63}/\underline{0.8997}           \\
%                                    & Spener (Ours)                                              & \textbf{34.47}/\underline{0.9163}
%  & \textbf{37.16}/\underline{0.9430} & \textbf{31.29}/\textbf{0.8709}  & \textbf{33.73}/\textbf{0.9181}            \\
% \bottomrule
% \end{tabular}
% \caption{Quantitative results of the compared methods on AAPM and COVID-19 datasets. The best and second performances are highlighted in \textbf{blod} and \underline{underline}, respectively.}
% \label{table1}
% \end{table*}

% \usepackage[normalem]{ulem}
% \usepackage{multirow}
% \usepackage{booktabs}


\begin{table*}[!h]
\setlength{\tabcolsep}{1.3mm}
\centering

\begin{tabular}{clcccccc} 
\toprule
\multirow{2.5}{*}{\textbf{Category}} & \multicolumn{1}{c}{\multirow{2.5}{*}{\textbf{Method}}} & \multicolumn{2}{c}{\textbf{AAPM}}                             & \multicolumn{2}{c}{\textbf{COVID-19}}                           & \multicolumn{2}{c}{\textbf{CMB-CRC Head}}                        \\ 
\cmidrule(lr){3-4}\cmidrule(lr){5-6}\cmidrule(lr){7-8}
                              & \multicolumn{1}{c}{}                & \textbf{60 Views}             & \textbf{90 Views}             & \textbf{60 Views}              & \textbf{90 Views}              & \textbf{\textbf{60 Views}} & \textbf{\textbf{90 Views}}  \\ 
\midrule
\texttt{Analytical}                    & FBP                                 & 23.61/0.4786                  & 26.46/0.6050                  & 22.96/0.4578                   & 25.77/0.5574                   & 24.05/0.4116               & 27.37/0.5086                \\ 
\cmidrule(l){1-5}\cmidrule{6-8}
\multirow{2}{*}{\texttt{Supervised}}   & FBPConvNet                          & 29.82/0.8588                  & 31.05/0.8864                  & 28.20/0.7717                   & 29.27/0.8080                   & 27.22/0.5415               & 29.74/0.6324                \\
                              & RegFormer                           & \underline{33.78}/\textbf{0.9399} & \underline{34.61}/\textbf{0.9564} & \underline{30.25}/0.8436           & 31.02/0.8810                   & 34.15/0.9264               & 36.11/0.9102                \\ 
\cmidrule(l){1-5}\cmidrule{6-8}
\multirow{3}{*}{\texttt{Unsupervised}} & CoIL                                & 30.18/0.8604                  & 31.84/0.9032                  & 27.77/0.7156                   & 29.72/0.7819                   &      28.14/0.7867                  & 31.01/0.8763                           \\
                              & SCOPE                               & 32.40/0.8939                  & 34.31/0.9322                  & 30.21/\underline{0.8504}           & \underline{32.63}/\underline{0.8997}   & \underline{36.76}/\underline{0.9740}               & \underline{38.76}/\underline{0.9857}                \\
                              & Spener (Ours)                       & \textbf{34.47}/\underline{0.9163} & \textbf{37.16}/\underline{0.9430} & \textbf{31.29}/\textbf{0.8709} & \textbf{33.73}/\textbf{0.9181} & \textbf{37.56}/\textbf{0.9809}               & \textbf{40.27}/\textbf{0.9900}                \\
\bottomrule
\end{tabular}
\caption{Quantitative results of the compared methods on AAPM, COVID-19 and CMB-CRC Head datasets. The best and second performances are highlighted in \textbf{bold} and \underline{underline}, respectively.}
\label{table1}
\end{table*}

% \begin{table*}
% % \setlength{\tabcolsep}{1.8mm}
% \centering
% \begin{tabular}{clcccccc} 
% \toprule
% \multirow{2}{*}{\textbf{Category}}     & \multicolumn{1}{c}{\multirow{2}{*}{\textbf{Method}}} & \multicolumn{3}{c}{\textbf{Normal Dose} $(\text{I}_0 = 10^6)$}                    & \multicolumn{3}{c}{\textbf{Low Dose} $(\text{I}_0 = 5\times10^5)$}     \\ 
% \cmidrule(lr){3-5}\cmidrule(lr){6-8}
%                               & \multicolumn{1}{c}{}                        & 30 Views             & 60 Views & 90 Views & 30 Views & 60 Views & 90 Views  \\ 
% \midrule
% Model-based                 & FBP                                         &    -                  & -          &   -           &   -        & -          &  -         \\ 
% \midrule
% \multirow{2}{*}{Supervised}   & FBPConvNet                                   & 29.51/0.916 &       32.88/0.953   &      35.04/0.967       &         &          &            \\
%                               & RegFormer                                   & \multicolumn{1}{l}{} &          &           &          &          &            \\ 
% \midrule
% \multirow{3}{*}{Unsupervised} & CoIL                                        &                      &          &           &          &          &            \\
%                               & SCOPE                                       &                      &          &           &          &          &            \\
%                               & En-INR                                      &                      &          &           &          &          &            \\
% \bottomrule
% \end{tabular}
% \caption{Quantitative results of compared methods on AAPM dataset under different dose setting.}
% \end{table*}


% \usepackage{multirow}
% \usepackage{booktabs}


\begin{table*}
\centering

\begin{tabular}{clcccc} 
\toprule
\multirow{2.5}{*}{\textbf{Category}} & \multicolumn{1}{c}{\multirow{2.5}{*}{\textbf{Method}}} & \multicolumn{2}{c}{\textbf{Normal Dose} $(I_0 = 10^6)$} & \multicolumn{2}{c}{\textbf{Low Dose} $(I_0 = 5\times10^5)$}  \\ 
\cmidrule(lr){3-4} \cmidrule(lr){5-6}
                                   & \multicolumn{1}{c}{}                                 & \textbf{60 Views}    & \textbf{90 Views}                                         & \textbf{60 Views} & \textbf{90 Views}                                                \\ 
\midrule
\texttt{Analytical}                      & FBP                                                  &  23.25/0.4502           & 26.01/0.5708                                                & 22.92/0.4272        & 25.59/0.5416                                                       \\ 
\midrule
\multirow{2}{*}{\texttt{Supervised}}        & FBPConvNet                                           & 29.67/0.8527 & 30.85/0.8787                                      &  29.53/0.8473        &   30.67/0.8721                                                       \\
                                   & RegFormer                                            &      \underline{33.11}/\textbf{0.9173}       &    \underline{33.77}/\underline{0.9256}                                              &   \underline{32.40}/\underline{0.8865}       &    32.92/0.8901                                                      \\ 
\midrule
\multirow{3}{*}{\texttt{Unsupervised}}      & CoIL                                                 &    30.04/0.8555         &     31.84/0.9001                                            &   30.03/0.8557        &        31.74/0.8966                                                  \\
                                   & SCOPE                                                & 32.00/0.8782             & 33.67/0.9139                                                 &    31.68/0.8657      &  \underline{33.29}/\underline{0.9007}                                                        \\
                                   & Spener (Ours)                                               &   \textbf{34.33}/\underline{0.9170}          &      \textbf{36.61}/\textbf{0.9424}                                            &  \textbf{34.03}/\textbf{0.9110}        &    \textbf{36.17}/\textbf{0.9349}                                                      \\
\bottomrule
\end{tabular}
\caption{Quantitative results of compared methods on AAPM dataset under different dose setting. The best performance is highlighted in \textbf{bold}, and the second best is \underline{underlined}.}
\label{table2}
\end{table*}

In~\autoref{tab:key_op01}, we note that operations 18 and 26 (marked in red) experience \emph{collision} when the second bit of the scalar is ``1'', regardless of the random ``z'' value.
  Similarly, we can deduce the equality of the 2\textsuperscript{nd} and 3\textsuperscript{rd} bits by detecting collisions between operations 55 and 59, as detailed in~\autoref{tab:key_op55_59}. These collisions occur whether both operations undergo a modular reduction as discussed in~\autoref{sec:attackz}. This pattern of identifying bit-equality through operational collisions can be extended iteratively. For example, \autoref{tab:collisions_indexes} lists operation indexes useful for extracting information about the first 6 bits of the ephemeral key. Upon completion of this phase, the component reveals details regarding the collision model as depicted in \autoref{tab:collisions_indexes}, and these indexes will be utilized to identify the bits of the ephemeral key.

\begin{table}[!ht]
    \caption{Correspondence between the index of operations to search for collision (2\textsuperscript{nd} column), the bit of the key retrieved (1\textsuperscript{st} column) and the information provided by a collision (3\textsuperscript{rd} column).}\label{tab:collisions_indexes}
\centering
\scalebox{0.6}{
\resizebox{\columnwidth}{!}{
\begin{tabular}{r|l|l}
\emph{Bit/s} & \emph{Subs indexes} & \emph{Information}\\
\hline
2 & (18,26) &  \mbox{``1'' if collision, ``0'' otherwise}\\
2,3 & (55,59) & \mbox{bits equal if no collision, different otherwise}\\
3,4 & (88,92) & \mbox{bits equal if no collision, different otherwise}\\
4,5 & (121,125) & \mbox{bits equal if no collision, different otherwise}\\
5,6 & (154,158)  &\mbox{bits equal if no collision, different otherwise}\\

\end{tabular}%
}
}
\end{table}

\subsection{Extracting the Private Key}
The final stage in the process involves extracting the private key by performing an LLL attack. Using the earlier-mentioned recognition method, we can obtain segments of the ephemeral key used in ECDSA scalar multiplication. Subsequently, we apply an LLL attack to derive the signer's secret key from the gathered data. Various versions and implementations of the LLL attack are available, and we specifically opted for the FPlll~\cite{fplll} lattice implementation. This choice is motivated by its Python compatibility, support for various LLL reduction techniques, and its accessibility and efficiency. The basis used to represent the extracted values for each ephemeral key in the LLL algorithm aligns with that of~\cite{jancar2020minerva}, followed by Babai's nearest plane reduction. With this setup, we managed to recover the signer's key using as few as 60 signatures, where each signature included an ephemeral key with at least 5 known bits. Additional details on LLL attacks can be found in~\cite{jancar2020minerva}. 

\section{Evaluation Methodology}
\label{sec:eval}

Our evaluation methodology uses a two-phase approach to determine the efficiency of our system. The initial phase utilizes a simulated setting to adjust the neural network (NN) hyperparameters. The subsequent phase assesses the network performance using power traces obtained from a target device, aiming to evaluate the practicality of the proposed attack.

\subsection{Simulated Environment}
The simulated environment aims to improve the neural network's performance on real power consumption data. Therefore, the artificial dataset is designed to accurately represent real-world situations.

To create the synthetic power traces, we started by re-implementing our chosen Montgomery ladder algorithm. We then ran it with varying values and linked each instruction of the algorithm to a sample in the trace. The value of each sample is based on leakage measurements from thousands of previously analyzed real-world traces. This method ensures that the synthetic traces accurately represent the data we can gather in actual scenarios, maintaining coherence in the training and tuning of the model, although it requires retraining when the target device is changed.

To further improve the dataset's realism, we apply a few extra modifications to the traces: to accurately represent modular reductions, we prolong the leakage trace of a subset of the SOs by adding a segment of its base pattern to simulate additional computation time. Additionally, we introduce vertical random noise to each operation type's base leakage pattern, with a mean of 0 and a standard deviation of 1.5. Lastly, we incorporate horizontal random noise to alter the trace's length, which reflects the operational jitter present in the actual target.

Through the aforementioned steps, we obtain a final set of 12 traces. This quantity has been shown to be adequate, as discussed in Section \ref{sec:nn}, because our model processes not the entire traces at once, but the overlapping sliding windows of samples extracted from the power traces. Consequently, this allows us to maintain a manageable trace count while still deriving enough input data from these traces to fully train and test our ML model.
In particular, our dataset comprises 240,800 windows in total, with 160,533 allocated for NN training and 80,267 designated for testing. This represents an approximate 70/30 split of the entire dataset. To ensure that the network does not detect patterns unrelated to the recognition task, we balanced the ratio of SOs and LOs equally across both sets.

\subsection{Real World Target}
\label{sec:exp_real}
In order to assess our system in a real world scenario, we employed the following configuration. The target device utilized is an STM32F415RG microcontroller integrated within the ChipWhisperer platform. Its core comprises an ARM 32-bit Cortex-M4 CPU operating at speeds of up to 168 MHz. For measuring power consumption, we used the ChipWhisperer acquisition device. This equipment features a 10-bit ADC with a maximum sample rate of 105 MS/s, an AC-Coupled analog input, an adjustable gain up to +55 dB, and it can generate clocks ranging from 5 MHz to 200 MHz. The acquisition of traces was carried out in streaming mode, capturing one sample per clock cycle, at a frequency of 7.37 MHz. This sampling rate was chosen based on the acquisition device's limitations, which supports a maximum of 10 MS/s in streaming mode because of its buffer capacity. To assess the classification results, the test bench also supplied additional metadata for each acquired trace. Specifically, for each segment within the power trace, the operation that produced it is collected, as demonstrated in Figure~\ref{fig:real_power_consumption_trace}.

\begin{figure}[h]
    \centering\includegraphics[width=\columnwidth]{images/real_power_consumption_trace.pdf}
    \caption{The first 24 thousand samples of a real power consumption trace (translated on the y-axis by 20 points for typesetting reasons).}  \label{fig:real_power_consumption_trace}
\end{figure}

We obtained a total of five power traces, using four for model training and one for testing. As mentioned earlier, this quantity was more than adequate for training and evaluating the model due to their extensive data. Table \ref{tab:details_realPwerConsumptionTraces} provides a detailed breakdown of these traces, demonstrating that the number of inputs extracted is sufficient for a thorough evaluation of our system.

\begin{table}[h!]
    \centering
    \caption[Real power consumption traces sizes]{Sizes of four real power consumption traces, acquired for testing, with derived datasets. For each power consumption trace (1\textsuperscript{st} column), we provide its samples size (2\textsuperscript{nd} column), the total number of short operation (SO) and long operation (LO) samples (3\textsuperscript{rd} column), the size of the testing dataset (4\textsuperscript{th} column) and the number of \texttt{true}-tagged and \texttt{false}-tagged windows (last column).
    }
    \scalebox{0.6}{
    \label{tab:details_realPwerConsumptionTraces}
    \resizebox{\columnwidth}{!}
    {%
        \begin{tabular}{c||c|cc||cl|cc}
            
            
            \textit{Trace} &
            
            \textit{Trace Size} &
            \multicolumn{2}{c||}{
            \begin{tabular}[c]{@{}c@{}}\textit{Trace Samples} \\\textit{\small SO samples} $\;$ \textit{\small LO samples}\end{tabular}} &
            \multicolumn{2}{c|}{
            \textit{Dataset size}} &
            \multicolumn{2}{c}{
            \begin{tabular}[c]{@{}c@{}}\textit{Dataset Windows}\\\texttt{\small true}\textit{\small -tag} $\;$ \texttt{\small false}\textit{\small -tag}\end{tabular}} \\\hline
            
            \texttt{T1} &
            $7,032,251$ &
            \multicolumn{1}{c:}{$\hspace{0.6em} 524,697 \hspace{0.6em}$} &
            $6,507,554$ &
            \multicolumn{2}{c|}{\begin{tabular}[c]{@{}c@{}}$703,176$\end{tabular}} &
            \multicolumn{1}{c:}{$\hspace{0.4em} 86,948 \hspace{0.4em}$} &
            $616,228$ \\ \hline
            
            \texttt{T2} &
            $7,002,209$ &
            \multicolumn{1}{c:}{$\hspace{0.6em} 522,537 \hspace{0.6em}$} &
            $6,479,672$ &
            \multicolumn{2}{c|}{\begin{tabular}[c]{@{}c@{}}$700,171$\end{tabular}} &
            \multicolumn{1}{c:}{$\hspace{0.4em} 86,739 \hspace{0.4em}$} &
            $613,432$ \\ \hline
            
            \texttt{T3} &
            $7,005,521$ &
            \multicolumn{1}{c:}{$\hspace{0.6em} 525,957 \hspace{0.6em}$} &
            $6,479,564$ &
            \multicolumn{2}{c|}{\begin{tabular}[c]{@{}c@{}}$700,503$\end{tabular}} &
            \multicolumn{1}{c:}{$\hspace{0.4em} 87,086 \hspace{0.4em}$} &
            $613,417$ \\ \hline
            
            \texttt{T4} &
            $7,027,186$ &
            \multicolumn{1}{c:}{$\hspace{0.6em} 524,247 \hspace{0.6em}$} &
            $6,502,939$ &
            \multicolumn{2}{c|}{\begin{tabular}[c]{@{}c@{}}$702,669$\end{tabular}} &
            \multicolumn{1}{c:}{$\hspace{0.4em} 86,893 \hspace{0.4em}$} &
            $615,776$ \\
            
        \end{tabular}
    }
    }
\end{table}

\section{Results}
\label{sec:res}
As mentioned in Section \ref{sec:sys}, in order for our system to work, the network must properly classify the windows given in the input, since misclassifying an input can hinder the estimation of the duration of operations.
After training and testing on both datasets detailed in Section \ref{sec:eval} our NN, the core part of our system, proved to be extremely effective both in the simulated environment and, most importantly, in the real world target.
Specifically in the simulated environment, we reached an accuracy of approximately 99.9\%.

\begin{figure}[h!]
\centering
\scalebox{0.6}{
\includegraphics[width=\linewidth]{images/epoch_accuracy_paper_2.pdf}
}
\caption{Epoch accuracy plot for 100 epochs, batch size of 64 and validation split of $20\%$: train in orange, validation in blue.}
\label{fig:epoch_accuracy}
\end{figure}

\begin{figure}[h!]
\centering
\scalebox{0.6}{
\includegraphics[width=\linewidth]{images/epoch_loss_paper_2.pdf}
}
\caption{Epoch loss plot for 100 epochs, batch size of 64 and validation split of $20\%$: train in orange, validation in blue.}
\label{fig:epoch_loss}
\end{figure}
Our network performed extremely well, as shown in Table \ref{tab:lstm_vs_cnn}, also on real-world traces, reaching an accuracy of around 97\% for each trace. 
In~\autoref{fig:accuracy_plot}, we plot the accuracy results for the initial portion of a real power consumption trace used during the testing phase.

From~\autoref{fig:accuracy_plot}, it is clear that the network was correct in most cases
(the green part of the plot). 

In the field of side-channel attack research, CNNs have been the predominant choice for analytical tasks. To demonstrate the superiority of the LSTM for this specific task, we performed an ablation study in which we removed the LSTM layer, thus converting the architecture into a CNN and using the same network parameters. In \ref{tab:lstm_vs_cnn}, we present the accuracy and loss results of our LSTM model when tested on real power consumption traces, and compare these with the results of a trained CNN model obtained by excluding the LSTM layer from the architecture mentioned in Section \ref{sec:sys}. This comparison validates our hypothesis that, given the similarity to the HAR problem, the LSTM is more suitable than the traditional CNN.

\begin{figure}[h]
    \centering
    \scalebox{0.85}{
    \includegraphics[width=\columnwidth]{images/bokeh_accuracy_plot_T2_raw_slicesize10.pdf}}
    \caption{Accuracy results for the initial portion of a real power consumption trace. In green the correct predictions, in red the failed ones.}
    \label{fig:accuracy_plot}
\end{figure}

\begin{table}[!ht]
    \caption{Accuracy and loss of LSTM and CNN collected for testing real power consumption traces.}\label{tab:lstm_vs_cnn}
\centering
\scalebox{0.60}{
\resizebox{\columnwidth}{!}{
\begin{tabular}{r||l|l||l|l}
\emph{Trace} & \emph{LSTM Accuracy}& \emph{CNN Accuracy} & \emph{LSTM Loss} & \emph{CNN Loss} \\
\hline
\texttt{T1} & 0.9758 & 0.9437 & 0.0621 & 0.1763 \\
\texttt{T2} & 0.9766 & 0.9463 & 0.0603 & 0.1692 \\
\texttt{T3} & 0.9756 & 0.9499 & 0.0646 & 0.1746 \\
\texttt{T4} & 0.9728 & 0.9344 & 0.0702 & 0.2039 \\

\end{tabular}%
}
}
\end{table}

\subsection{End-To-End Attack}
An important issue encountered during the development of our system is the improper classification of successive operations of the same type (false positive). This issue arises from the neural network's lack of training to recognize the start and end of an operation. It has only been trained to distinguish between short- and long-term operations. As a result, when the model is faced with a sequence of \emph{Sub} operations, it is unable to discern which are short and which are long.

To overcome these limitations, we have adjusted our post-processing approach to operate on sequences rather than individual operations. We will refer to these sequences as \emph{clusters}, using them to identify the key bits. Specifically, we consider a trace to be valid if and only if all operations within it exhibit the same behavior (we define these as homogeneous clusters). This characteristic can be readily determined by comparing the cluster's length with the durations of short and long operations.

A significant disadvantage of this approach is the increased number of traces required because we are limited to analyzing traces that display homogeneous clusters. The decrease in the number of useful traces for any individual cluster is directly proportional to the cluster's length. Specifically, for a cluster of length $n$, the likelihood of obtaining a useful trace is $1/n$. Consequently, for $\Gamma$, the set of groups examined, the probability of finding a single trace where all clusters are homogeneous is $\prod_{\gamma \in \Gamma} 1/len(\gamma)$.

In order to quantify the increase in the number of traces needed for a successful attack, we recall ~\autoref{tab:key_op01}. We can clearly observe that operation 18 is part of a cluster of three \emph{Subs}, those at positions 17, 18, and 19. Instead, operation 26 is part of a cluster of 1. Then, operations 55, 88, 121, 154 and so on (each is 33 operations away),  are always in a cluster of 1, while operations 59, 92, 125, 158, etc. (each is 33 operations away) are always in a cluster of two (the \emph{Sub} before the change of the bit and the \emph{Sub} afterwards). This means that the first bit depends on the correct identification of the 3\textsuperscript{rd} and 5\textsuperscript{th} clusters. Also, since the 4\textsuperscript{th} cluster is a cluster of 3 operations, while the 5\textsuperscript{th} is a cluster of one, the attacker has a chance of 1/4 to get a useful prediction for a given signature trace. Instead, subsequent collisions depend on a cluster of 1 (operations 55, 88, 121, 154, etc.\ldots) and a cluster of two (operations 59, 92, 125, 158, etc\ldots). So, in this case, the predictions have probability one-half to be useful.
We also remark that from a useful collision we are able to correctly identify the exact value of the 2\textsuperscript{nd} bit of the ephemeral key, while for the next bit a useful collision only provides equality with the previous bit value. Thus, to obtain the value of the 3\textsuperscript{rd} bit of the ephemeral key, the attacker needs to get a useful collision on the 2\textsuperscript{nd} bit and a useful collision on the 3\textsuperscript{rd} bit. Similarly, to reveal the 4\textsuperscript{th} bit, three consecutive useful collisions on the same signature are needed.
\autoref{tab:bits_proba} summarizes these results. In particular, we observe that to obtain one signature where the attacker is able to retrieve 5 bits, she needs, on average, 64 signatures. This in turn means that to obtain 100 such signatures in order to perform the lattice reduction $6,400$ executions, it should be needed. In our scenario, execution of $6,400$ takes almost 10 days using the crypto algorithm on the Teledyne Lecroix 3104z oscilloscope model. Albeit this is an increase in the number of traces that should be acquired, this number is far from unfeasible in the context of a side-channel attack meaning that the solution does not limit the exploitation capability of the designed system.

\begin{table}[!ht]
    \caption{Probability to obtain \emph{n} bits for each signature, and estimation on the number of signatures required to obtain them.}\label{tab:bits_proba}
\centering
\scalebox{0.6}{
\resizebox{\columnwidth}{!}{
\begin{tabular}{r|l|l}
\#\emph{Bits} & \emph{Probability} & \emph{Estimated sign. required}\\
\hline
1 & $\sfrac{1}{4}$ & 4 \\
2 & $\sfrac{1}{4} \cdot \sfrac{1}{2} = \sfrac{1}{8}$ & 8\\
3 & $\sfrac{1}{4} \cdot \sfrac{1}{2} \cdot \sfrac{1}{2} = \sfrac{1}{16}$ & 16\\
4 & $\sfrac{1}{4} \cdot \sfrac{1}{2} \cdot \sfrac{1}{2} \cdot \sfrac{1}{2} = \sfrac{1}{32}$ & 32\\
5 & $\sfrac{1}{4} \cdot \sfrac{1}{2} \cdot \sfrac{1}{2} \cdot \sfrac{1}{2} \cdot \sfrac{1}{2} = \sfrac{1}{64}$ & 64\\

\end{tabular}%
}
}
\end{table}

\section{Countermeasures}
\label{sec:countermeasures}

In this Section, we discuss a range of countermeasures suitable for various types of implementation. These countermeasures vary in their requirements for random generation and differ in efficiency levels.

\subsection{Incomplete Countermeasures}
\paragraph{Real effectiveness of the~\cite{luo2018effective} countermeasure.}
If the implementation can afford using one additional register, it is possible to adopt the countermeasure suggested by Luo~\textit{et al.}~\cite{luo2018effective}. 
Since the weakness arises from the subtraction computation in XYCZ-ADDC, and since the result is squared,
it does not matter if it is $X_0-X_1$ or $X_1-X_0$. So, the authors suggest saving the subtraction result performed during XYCZ-ADD, and reuse it in XYCZ-ADDC for the next bit. 
We remark that this countermeasure is very efficient and also saves one subtraction. However, XYCZ-ADDC subtraction is not the only collision exploitable during execution. For example, we observe that the order of use of $X_1$ and $X_2$ in iteration $i$ in XYCZ-ADDC depends on the value of the key bits $k_{b_{i-1}}$ and $k_{b_i}$. So, for example, if $k_{b_{i-1}}$ equals $k_{b_i}$, the $X$ outputs of XYCZ-ADD will be used in the same order as they have been computed; otherwise, their order will be reversed. Using collisions, an attacker can detect if the same value is used during the first or second computation (that is, when computing $B$ or $C$), and infer the values of the key bits. 

\paragraph{Modular Reduction.}
A classic countermeasure that can be applied to the implementation is to always perform modular reductions, regardless of whether the computation overflows or underflows. This approach was previously suggested in \cite{ryan2019return} and is probably the easiest solution to this problem. However, this countermeasure would not prevent a side-channel attacker from detecting that the same value has been computed twice (that is, a collision on the values) through correlation techniques on the power or EM traces. This is the kind of weakness exploited in~\cite{luo2018effective} and, as observed in Section~\ref{sec:attackz}, it would again make it possible for an attacker to obtain sensitive information. 

\subsection{Effective Countermeasures}
\paragraph{Masking Technique.}
A more robust approach to counteract the attack is to mask the values (that is, the coordinates) manipulated during the execution of the Montgomery ladder. This method is more effective, but also less efficient. To avoid collisions, masks must be changed for each iteration of the Montgomery ladder. Otherwise, an attacker could detect the occurrence of a collision regardless of the presence of the masks. 

\paragraph{Coordinates Re-Randomization.}
Another solution to this problem is to repeat randomization of the coordinates after each iteration of the Montgomery loop. This approach involves generating a new \emph{z} for each iteration and reprojecting the curve to the new coordinates. However, it could be more cost-effective than a fully masked implementation and may also prevent collision attacks. The process of re-randomizing the coordinates essentially ensures that there is no correlation between the previous and current computations, and thus eliminates the possibility of a collision attack. 

\section{Conclusions}
Our study rigorously evaluated the implementation of ECDSA on a commercial standard device using the micro-ecc library, focusing on the secp160r1 curve. We identified vulnerabilities, particularly in non-constant-time execution, due to selective modular reductions. By improving existing collision attacks and applying them to real-world hardware, we exposed more flaws. Using simulations and an LSTM neural network, we detected operational patterns indicating modular reductions, enabling us to derive ephemeral key bits and recover the signing key through lattice attacks. Testing on a real STM32F415RG device with the ChipWhisperer confirmed our method's high accuracy and effectiveness in revealing significant security risks in prevalent cryptographic systems.

%\bibliographystyle{plain} 
%\bibliography{bib_IEEE,bib}
% This must be in the first 5 lines to tell arXiv to use pdfLaTeX, which is strongly recommended.
\pdfoutput=1
% In particular, the hyperref package requires pdfLaTeX in order to break URLs across lines.

\documentclass[11pt]{article}

% Change "review" to "final" to generate the final (sometimes called camera-ready) version.
% Change to "preprint" to generate a non-anonymous version with page numbers.
\usepackage{acl}

% Standard package includes
\usepackage{times}
\usepackage{latexsym}

% Draw tables
\usepackage{booktabs}
\usepackage{multirow}
\usepackage{xcolor}
\usepackage{colortbl}
\usepackage{array} 
\usepackage{amsmath}

\newcolumntype{C}{>{\centering\arraybackslash}p{0.07\textwidth}}
% For proper rendering and hyphenation of words containing Latin characters (including in bib files)
\usepackage[T1]{fontenc}
% For Vietnamese characters
% \usepackage[T5]{fontenc}
% See https://www.latex-project.org/help/documentation/encguide.pdf for other character sets
% This assumes your files are encoded as UTF8
\usepackage[utf8]{inputenc}

% This is not strictly necessary, and may be commented out,
% but it will improve the layout of the manuscript,
% and will typically save some space.
\usepackage{microtype}
\DeclareMathOperator*{\argmax}{arg\,max}
% This is also not strictly necessary, and may be commented out.
% However, it will improve the aesthetics of text in
% the typewriter font.
\usepackage{inconsolata}

%Including images in your LaTeX document requires adding
%additional package(s)
\usepackage{graphicx}
% If the title and author information does not fit in the area allocated, uncomment the following
%
%\setlength\titlebox{<dim>}
%
% and set <dim> to something 5cm or larger.

\title{Wi-Chat: Large Language Model Powered Wi-Fi Sensing}

% Author information can be set in various styles:
% For several authors from the same institution:
% \author{Author 1 \and ... \and Author n \\
%         Address line \\ ... \\ Address line}
% if the names do not fit well on one line use
%         Author 1 \\ {\bf Author 2} \\ ... \\ {\bf Author n} \\
% For authors from different institutions:
% \author{Author 1 \\ Address line \\  ... \\ Address line
%         \And  ... \And
%         Author n \\ Address line \\ ... \\ Address line}
% To start a separate ``row'' of authors use \AND, as in
% \author{Author 1 \\ Address line \\  ... \\ Address line
%         \AND
%         Author 2 \\ Address line \\ ... \\ Address line \And
%         Author 3 \\ Address line \\ ... \\ Address line}

% \author{First Author \\
%   Affiliation / Address line 1 \\
%   Affiliation / Address line 2 \\
%   Affiliation / Address line 3 \\
%   \texttt{email@domain} \\\And
%   Second Author \\
%   Affiliation / Address line 1 \\
%   Affiliation / Address line 2 \\
%   Affiliation / Address line 3 \\
%   \texttt{email@domain} \\}
% \author{Haohan Yuan \qquad Haopeng Zhang\thanks{corresponding author} \\ 
%   ALOHA Lab, University of Hawaii at Manoa \\
%   % Affiliation / Address line 2 \\
%   % Affiliation / Address line 3 \\
%   \texttt{\{haohany,haopengz\}@hawaii.edu}}
  
\author{
{Haopeng Zhang$\dag$\thanks{These authors contributed equally to this work.}, Yili Ren$\ddagger$\footnotemark[1], Haohan Yuan$\dag$, Jingzhe Zhang$\ddagger$, Yitong Shen$\ddagger$} \\
ALOHA Lab, University of Hawaii at Manoa$\dag$, University of South Florida$\ddagger$ \\
\{haopengz, haohany\}@hawaii.edu\\
\{yiliren, jingzhe, shen202\}@usf.edu\\}



  
%\author{
%  \textbf{First Author\textsuperscript{1}},
%  \textbf{Second Author\textsuperscript{1,2}},
%  \textbf{Third T. Author\textsuperscript{1}},
%  \textbf{Fourth Author\textsuperscript{1}},
%\\
%  \textbf{Fifth Author\textsuperscript{1,2}},
%  \textbf{Sixth Author\textsuperscript{1}},
%  \textbf{Seventh Author\textsuperscript{1}},
%  \textbf{Eighth Author \textsuperscript{1,2,3,4}},
%\\
%  \textbf{Ninth Author\textsuperscript{1}},
%  \textbf{Tenth Author\textsuperscript{1}},
%  \textbf{Eleventh E. Author\textsuperscript{1,2,3,4,5}},
%  \textbf{Twelfth Author\textsuperscript{1}},
%\\
%  \textbf{Thirteenth Author\textsuperscript{3}},
%  \textbf{Fourteenth F. Author\textsuperscript{2,4}},
%  \textbf{Fifteenth Author\textsuperscript{1}},
%  \textbf{Sixteenth Author\textsuperscript{1}},
%\\
%  \textbf{Seventeenth S. Author\textsuperscript{4,5}},
%  \textbf{Eighteenth Author\textsuperscript{3,4}},
%  \textbf{Nineteenth N. Author\textsuperscript{2,5}},
%  \textbf{Twentieth Author\textsuperscript{1}}
%\\
%\\
%  \textsuperscript{1}Affiliation 1,
%  \textsuperscript{2}Affiliation 2,
%  \textsuperscript{3}Affiliation 3,
%  \textsuperscript{4}Affiliation 4,
%  \textsuperscript{5}Affiliation 5
%\\
%  \small{
%    \textbf{Correspondence:} \href{mailto:email@domain}{email@domain}
%  }
%}

\begin{document}
\maketitle
\begin{abstract}
Recent advancements in Large Language Models (LLMs) have demonstrated remarkable capabilities across diverse tasks. However, their potential to integrate physical model knowledge for real-world signal interpretation remains largely unexplored. In this work, we introduce Wi-Chat, the first LLM-powered Wi-Fi-based human activity recognition system. We demonstrate that LLMs can process raw Wi-Fi signals and infer human activities by incorporating Wi-Fi sensing principles into prompts. Our approach leverages physical model insights to guide LLMs in interpreting Channel State Information (CSI) data without traditional signal processing techniques. Through experiments on real-world Wi-Fi datasets, we show that LLMs exhibit strong reasoning capabilities, achieving zero-shot activity recognition. These findings highlight a new paradigm for Wi-Fi sensing, expanding LLM applications beyond conventional language tasks and enhancing the accessibility of wireless sensing for real-world deployments.
\end{abstract}

\section{Introduction}

In today’s rapidly evolving digital landscape, the transformative power of web technologies has redefined not only how services are delivered but also how complex tasks are approached. Web-based systems have become increasingly prevalent in risk control across various domains. This widespread adoption is due their accessibility, scalability, and ability to remotely connect various types of users. For example, these systems are used for process safety management in industry~\cite{kannan2016web}, safety risk early warning in urban construction~\cite{ding2013development}, and safe monitoring of infrastructural systems~\cite{repetto2018web}. Within these web-based risk management systems, the source search problem presents a huge challenge. Source search refers to the task of identifying the origin of a risky event, such as a gas leak and the emission point of toxic substances. This source search capability is crucial for effective risk management and decision-making.

Traditional approaches to implementing source search capabilities into the web systems often rely on solely algorithmic solutions~\cite{ristic2016study}. These methods, while relatively straightforward to implement, often struggle to achieve acceptable performances due to algorithmic local optima and complex unknown environments~\cite{zhao2020searching}. More recently, web crowdsourcing has emerged as a promising alternative for tackling the source search problem by incorporating human efforts in these web systems on-the-fly~\cite{zhao2024user}. This approach outsources the task of addressing issues encountered during the source search process to human workers, leveraging their capabilities to enhance system performance.

These solutions often employ a human-AI collaborative way~\cite{zhao2023leveraging} where algorithms handle exploration-exploitation and report the encountered problems while human workers resolve complex decision-making bottlenecks to help the algorithms getting rid of local deadlocks~\cite{zhao2022crowd}. Although effective, this paradigm suffers from two inherent limitations: increased operational costs from continuous human intervention, and slow response times of human workers due to sequential decision-making. These challenges motivate our investigation into developing autonomous systems that preserve human-like reasoning capabilities while reducing dependency on massive crowdsourced labor.

Furthermore, recent advancements in large language models (LLMs)~\cite{chang2024survey} and multi-modal LLMs (MLLMs)~\cite{huang2023chatgpt} have unveiled promising avenues for addressing these challenges. One clear opportunity involves the seamless integration of visual understanding and linguistic reasoning for robust decision-making in search tasks. However, whether large models-assisted source search is really effective and efficient for improving the current source search algorithms~\cite{ji2022source} remains unknown. \textit{To address the research gap, we are particularly interested in answering the following two research questions in this work:}

\textbf{\textit{RQ1: }}How can source search capabilities be integrated into web-based systems to support decision-making in time-sensitive risk management scenarios? 
% \sq{I mention ``time-sensitive'' here because I feel like we shall say something about the response time -- LLM has to be faster than humans}

\textbf{\textit{RQ2: }}How can MLLMs and LLMs enhance the effectiveness and efficiency of existing source search algorithms? 

% \textit{\textbf{RQ2:}} To what extent does the performance of large models-assisted search align with or approach the effectiveness of human-AI collaborative search? 

To answer the research questions, we propose a novel framework called Auto-\
S$^2$earch (\textbf{Auto}nomous \textbf{S}ource \textbf{Search}) and implement a prototype system that leverages advanced web technologies to simulate real-world conditions for zero-shot source search. Unlike traditional methods that rely on pre-defined heuristics or extensive human intervention, AutoS$^2$earch employs a carefully designed prompt that encapsulates human rationales, thereby guiding the MLLM to generate coherent and accurate scene descriptions from visual inputs about four directional choices. Based on these language-based descriptions, the LLM is enabled to determine the optimal directional choice through chain-of-thought (CoT) reasoning. Comprehensive empirical validation demonstrates that AutoS$^2$-\ 
earch achieves a success rate of 95–98\%, closely approaching the performance of human-AI collaborative search across 20 benchmark scenarios~\cite{zhao2023leveraging}. 

Our work indicates that the role of humans in future web crowdsourcing tasks may evolve from executors to validators or supervisors. Furthermore, incorporating explanations of LLM decisions into web-based system interfaces has the potential to help humans enhance task performance in risk control.






\section{Related Work}
\label{sec:relatedworks}

% \begin{table*}[t]
% \centering 
% \renewcommand\arraystretch{0.98}
% \fontsize{8}{10}\selectfont \setlength{\tabcolsep}{0.4em}
% \begin{tabular}{@{}lc|cc|cc|cc@{}}
% \toprule
% \textbf{Methods}           & \begin{tabular}[c]{@{}c@{}}\textbf{Training}\\ \textbf{Paradigm}\end{tabular} & \begin{tabular}[c]{@{}c@{}}\textbf{$\#$ PT Data}\\ \textbf{(Tokens)}\end{tabular} & \begin{tabular}[c]{@{}c@{}}\textbf{$\#$ IFT Data}\\ \textbf{(Samples)}\end{tabular} & \textbf{Code}  & \begin{tabular}[c]{@{}c@{}}\textbf{Natural}\\ \textbf{Language}\end{tabular} & \begin{tabular}[c]{@{}c@{}}\textbf{Action}\\ \textbf{Trajectories}\end{tabular} & \begin{tabular}[c]{@{}c@{}}\textbf{API}\\ \textbf{Documentation}\end{tabular}\\ \midrule 
% NexusRaven~\citep{srinivasan2023nexusraven} & IFT & - & - & \textcolor{green}{\CheckmarkBold} & \textcolor{green}{\CheckmarkBold} &\textcolor{red}{\XSolidBrush}&\textcolor{red}{\XSolidBrush}\\
% AgentInstruct~\citep{zeng2023agenttuning} & IFT & - & 2k & \textcolor{green}{\CheckmarkBold} & \textcolor{green}{\CheckmarkBold} &\textcolor{red}{\XSolidBrush}&\textcolor{red}{\XSolidBrush} \\
% AgentEvol~\citep{xi2024agentgym} & IFT & - & 14.5k & \textcolor{green}{\CheckmarkBold} & \textcolor{green}{\CheckmarkBold} &\textcolor{green}{\CheckmarkBold}&\textcolor{red}{\XSolidBrush} \\
% Gorilla~\citep{patil2023gorilla}& IFT & - & 16k & \textcolor{green}{\CheckmarkBold} & \textcolor{green}{\CheckmarkBold} &\textcolor{red}{\XSolidBrush}&\textcolor{green}{\CheckmarkBold}\\
% OpenFunctions-v2~\citep{patil2023gorilla} & IFT & - & 65k & \textcolor{green}{\CheckmarkBold} & \textcolor{green}{\CheckmarkBold} &\textcolor{red}{\XSolidBrush}&\textcolor{green}{\CheckmarkBold}\\
% LAM~\citep{zhang2024agentohana} & IFT & - & 42.6k & \textcolor{green}{\CheckmarkBold} & \textcolor{green}{\CheckmarkBold} &\textcolor{green}{\CheckmarkBold}&\textcolor{red}{\XSolidBrush} \\
% xLAM~\citep{liu2024apigen} & IFT & - & 60k & \textcolor{green}{\CheckmarkBold} & \textcolor{green}{\CheckmarkBold} &\textcolor{green}{\CheckmarkBold}&\textcolor{red}{\XSolidBrush} \\\midrule
% LEMUR~\citep{xu2024lemur} & PT & 90B & 300k & \textcolor{green}{\CheckmarkBold} & \textcolor{green}{\CheckmarkBold} &\textcolor{green}{\CheckmarkBold}&\textcolor{red}{\XSolidBrush}\\
% \rowcolor{teal!12} \method & PT & 103B & 95k & \textcolor{green}{\CheckmarkBold} & \textcolor{green}{\CheckmarkBold} & \textcolor{green}{\CheckmarkBold} & \textcolor{green}{\CheckmarkBold} \\
% \bottomrule
% \end{tabular}
% \caption{Summary of existing tuning- and pretraining-based LLM agents with their training sample sizes. "PT" and "IFT" denote "Pre-Training" and "Instruction Fine-Tuning", respectively. }
% \label{tab:related}
% \end{table*}

\begin{table*}[ht]
\begin{threeparttable}
\centering 
\renewcommand\arraystretch{0.98}
\fontsize{7}{9}\selectfont \setlength{\tabcolsep}{0.2em}
\begin{tabular}{@{}l|c|c|ccc|cc|cc|cccc@{}}
\toprule
\textbf{Methods} & \textbf{Datasets}           & \begin{tabular}[c]{@{}c@{}}\textbf{Training}\\ \textbf{Paradigm}\end{tabular} & \begin{tabular}[c]{@{}c@{}}\textbf{\# PT Data}\\ \textbf{(Tokens)}\end{tabular} & \begin{tabular}[c]{@{}c@{}}\textbf{\# IFT Data}\\ \textbf{(Samples)}\end{tabular} & \textbf{\# APIs} & \textbf{Code}  & \begin{tabular}[c]{@{}c@{}}\textbf{Nat.}\\ \textbf{Lang.}\end{tabular} & \begin{tabular}[c]{@{}c@{}}\textbf{Action}\\ \textbf{Traj.}\end{tabular} & \begin{tabular}[c]{@{}c@{}}\textbf{API}\\ \textbf{Doc.}\end{tabular} & \begin{tabular}[c]{@{}c@{}}\textbf{Func.}\\ \textbf{Call}\end{tabular} & \begin{tabular}[c]{@{}c@{}}\textbf{Multi.}\\ \textbf{Step}\end{tabular}  & \begin{tabular}[c]{@{}c@{}}\textbf{Plan}\\ \textbf{Refine}\end{tabular}  & \begin{tabular}[c]{@{}c@{}}\textbf{Multi.}\\ \textbf{Turn}\end{tabular}\\ \midrule 
\multicolumn{13}{l}{\emph{Instruction Finetuning-based LLM Agents for Intrinsic Reasoning}}  \\ \midrule
FireAct~\cite{chen2023fireact} & FireAct & IFT & - & 2.1K & 10 & \textcolor{red}{\XSolidBrush} &\textcolor{green}{\CheckmarkBold} &\textcolor{green}{\CheckmarkBold}  & \textcolor{red}{\XSolidBrush} &\textcolor{green}{\CheckmarkBold} & \textcolor{red}{\XSolidBrush} &\textcolor{green}{\CheckmarkBold} & \textcolor{red}{\XSolidBrush} \\
ToolAlpaca~\cite{tang2023toolalpaca} & ToolAlpaca & IFT & - & 4.0K & 400 & \textcolor{red}{\XSolidBrush} &\textcolor{green}{\CheckmarkBold} &\textcolor{green}{\CheckmarkBold} & \textcolor{red}{\XSolidBrush} &\textcolor{green}{\CheckmarkBold} & \textcolor{red}{\XSolidBrush}  &\textcolor{green}{\CheckmarkBold} & \textcolor{red}{\XSolidBrush}  \\
ToolLLaMA~\cite{qin2023toolllm} & ToolBench & IFT & - & 12.7K & 16,464 & \textcolor{red}{\XSolidBrush} &\textcolor{green}{\CheckmarkBold} &\textcolor{green}{\CheckmarkBold} &\textcolor{red}{\XSolidBrush} &\textcolor{green}{\CheckmarkBold}&\textcolor{green}{\CheckmarkBold}&\textcolor{green}{\CheckmarkBold} &\textcolor{green}{\CheckmarkBold}\\
AgentEvol~\citep{xi2024agentgym} & AgentTraj-L & IFT & - & 14.5K & 24 &\textcolor{red}{\XSolidBrush} & \textcolor{green}{\CheckmarkBold} &\textcolor{green}{\CheckmarkBold}&\textcolor{red}{\XSolidBrush} &\textcolor{green}{\CheckmarkBold}&\textcolor{red}{\XSolidBrush} &\textcolor{red}{\XSolidBrush} &\textcolor{green}{\CheckmarkBold}\\
Lumos~\cite{yin2024agent} & Lumos & IFT  & - & 20.0K & 16 &\textcolor{red}{\XSolidBrush} & \textcolor{green}{\CheckmarkBold} & \textcolor{green}{\CheckmarkBold} &\textcolor{red}{\XSolidBrush} & \textcolor{green}{\CheckmarkBold} & \textcolor{green}{\CheckmarkBold} &\textcolor{red}{\XSolidBrush} & \textcolor{green}{\CheckmarkBold}\\
Agent-FLAN~\cite{chen2024agent} & Agent-FLAN & IFT & - & 24.7K & 20 &\textcolor{red}{\XSolidBrush} & \textcolor{green}{\CheckmarkBold} & \textcolor{green}{\CheckmarkBold} &\textcolor{red}{\XSolidBrush} & \textcolor{green}{\CheckmarkBold}& \textcolor{green}{\CheckmarkBold}&\textcolor{red}{\XSolidBrush} & \textcolor{green}{\CheckmarkBold}\\
AgentTuning~\citep{zeng2023agenttuning} & AgentInstruct & IFT & - & 35.0K & - &\textcolor{red}{\XSolidBrush} & \textcolor{green}{\CheckmarkBold} & \textcolor{green}{\CheckmarkBold} &\textcolor{red}{\XSolidBrush} & \textcolor{green}{\CheckmarkBold} &\textcolor{red}{\XSolidBrush} &\textcolor{red}{\XSolidBrush} & \textcolor{green}{\CheckmarkBold}\\\midrule
\multicolumn{13}{l}{\emph{Instruction Finetuning-based LLM Agents for Function Calling}} \\\midrule
NexusRaven~\citep{srinivasan2023nexusraven} & NexusRaven & IFT & - & - & 116 & \textcolor{green}{\CheckmarkBold} & \textcolor{green}{\CheckmarkBold}  & \textcolor{green}{\CheckmarkBold} &\textcolor{red}{\XSolidBrush} & \textcolor{green}{\CheckmarkBold} &\textcolor{red}{\XSolidBrush} &\textcolor{red}{\XSolidBrush}&\textcolor{red}{\XSolidBrush}\\
Gorilla~\citep{patil2023gorilla} & Gorilla & IFT & - & 16.0K & 1,645 & \textcolor{green}{\CheckmarkBold} &\textcolor{red}{\XSolidBrush} &\textcolor{red}{\XSolidBrush}&\textcolor{green}{\CheckmarkBold} &\textcolor{green}{\CheckmarkBold} &\textcolor{red}{\XSolidBrush} &\textcolor{red}{\XSolidBrush} &\textcolor{red}{\XSolidBrush}\\
OpenFunctions-v2~\citep{patil2023gorilla} & OpenFunctions-v2 & IFT & - & 65.0K & - & \textcolor{green}{\CheckmarkBold} & \textcolor{green}{\CheckmarkBold} &\textcolor{red}{\XSolidBrush} &\textcolor{green}{\CheckmarkBold} &\textcolor{green}{\CheckmarkBold} &\textcolor{red}{\XSolidBrush} &\textcolor{red}{\XSolidBrush} &\textcolor{red}{\XSolidBrush}\\
API Pack~\cite{guo2024api} & API Pack & IFT & - & 1.1M & 11,213 &\textcolor{green}{\CheckmarkBold} &\textcolor{red}{\XSolidBrush} &\textcolor{green}{\CheckmarkBold} &\textcolor{red}{\XSolidBrush} &\textcolor{green}{\CheckmarkBold} &\textcolor{red}{\XSolidBrush}&\textcolor{red}{\XSolidBrush}&\textcolor{red}{\XSolidBrush}\\ 
LAM~\citep{zhang2024agentohana} & AgentOhana & IFT & - & 42.6K & - & \textcolor{green}{\CheckmarkBold} & \textcolor{green}{\CheckmarkBold} &\textcolor{green}{\CheckmarkBold}&\textcolor{red}{\XSolidBrush} &\textcolor{green}{\CheckmarkBold}&\textcolor{red}{\XSolidBrush}&\textcolor{green}{\CheckmarkBold}&\textcolor{green}{\CheckmarkBold}\\
xLAM~\citep{liu2024apigen} & APIGen & IFT & - & 60.0K & 3,673 & \textcolor{green}{\CheckmarkBold} & \textcolor{green}{\CheckmarkBold} &\textcolor{green}{\CheckmarkBold}&\textcolor{red}{\XSolidBrush} &\textcolor{green}{\CheckmarkBold}&\textcolor{red}{\XSolidBrush}&\textcolor{green}{\CheckmarkBold}&\textcolor{green}{\CheckmarkBold}\\\midrule
\multicolumn{13}{l}{\emph{Pretraining-based LLM Agents}}  \\\midrule
% LEMUR~\citep{xu2024lemur} & PT & 90B & 300.0K & - & \textcolor{green}{\CheckmarkBold} & \textcolor{green}{\CheckmarkBold} &\textcolor{green}{\CheckmarkBold}&\textcolor{red}{\XSolidBrush} & \textcolor{red}{\XSolidBrush} &\textcolor{green}{\CheckmarkBold} &\textcolor{red}{\XSolidBrush}&\textcolor{red}{\XSolidBrush}\\
\rowcolor{teal!12} \method & \dataset & PT & 103B & 95.0K  & 76,537  & \textcolor{green}{\CheckmarkBold} & \textcolor{green}{\CheckmarkBold} & \textcolor{green}{\CheckmarkBold} & \textcolor{green}{\CheckmarkBold} & \textcolor{green}{\CheckmarkBold} & \textcolor{green}{\CheckmarkBold} & \textcolor{green}{\CheckmarkBold} & \textcolor{green}{\CheckmarkBold}\\
\bottomrule
\end{tabular}
% \begin{tablenotes}
%     \item $^*$ In addition, the StarCoder-API can offer 4.77M more APIs.
% \end{tablenotes}
\caption{Summary of existing instruction finetuning-based LLM agents for intrinsic reasoning and function calling, along with their training resources and sample sizes. "PT" and "IFT" denote "Pre-Training" and "Instruction Fine-Tuning", respectively.}
\vspace{-2ex}
\label{tab:related}
\end{threeparttable}
\end{table*}

\noindent \textbf{Prompting-based LLM Agents.} Due to the lack of agent-specific pre-training corpus, existing LLM agents rely on either prompt engineering~\cite{hsieh2023tool,lu2024chameleon,yao2022react,wang2023voyager} or instruction fine-tuning~\cite{chen2023fireact,zeng2023agenttuning} to understand human instructions, decompose high-level tasks, generate grounded plans, and execute multi-step actions. 
However, prompting-based methods mainly depend on the capabilities of backbone LLMs (usually commercial LLMs), failing to introduce new knowledge and struggling to generalize to unseen tasks~\cite{sun2024adaplanner,zhuang2023toolchain}. 

\noindent \textbf{Instruction Finetuning-based LLM Agents.} Considering the extensive diversity of APIs and the complexity of multi-tool instructions, tool learning inherently presents greater challenges than natural language tasks, such as text generation~\cite{qin2023toolllm}.
Post-training techniques focus more on instruction following and aligning output with specific formats~\cite{patil2023gorilla,hao2024toolkengpt,qin2023toolllm,schick2024toolformer}, rather than fundamentally improving model knowledge or capabilities. 
Moreover, heavy fine-tuning can hinder generalization or even degrade performance in non-agent use cases, potentially suppressing the original base model capabilities~\cite{ghosh2024a}.

\noindent \textbf{Pretraining-based LLM Agents.} While pre-training serves as an essential alternative, prior works~\cite{nijkamp2023codegen,roziere2023code,xu2024lemur,patil2023gorilla} have primarily focused on improving task-specific capabilities (\eg, code generation) instead of general-domain LLM agents, due to single-source, uni-type, small-scale, and poor-quality pre-training data. 
Existing tool documentation data for agent training either lacks diverse real-world APIs~\cite{patil2023gorilla, tang2023toolalpaca} or is constrained to single-tool or single-round tool execution. 
Furthermore, trajectory data mostly imitate expert behavior or follow function-calling rules with inferior planning and reasoning, failing to fully elicit LLMs' capabilities and handle complex instructions~\cite{qin2023toolllm}. 
Given a wide range of candidate API functions, each comprising various function names and parameters available at every planning step, identifying globally optimal solutions and generalizing across tasks remains highly challenging.



\section{Preliminaries}
\label{Preliminaries}
\begin{figure*}[t]
    \centering
    \includegraphics[width=0.95\linewidth]{fig/HealthGPT_Framework.png}
    \caption{The \ourmethod{} architecture integrates hierarchical visual perception and H-LoRA, employing a task-specific hard router to select visual features and H-LoRA plugins, ultimately generating outputs with an autoregressive manner.}
    \label{fig:architecture}
\end{figure*}
\noindent\textbf{Large Vision-Language Models.} 
The input to a LVLM typically consists of an image $x^{\text{img}}$ and a discrete text sequence $x^{\text{txt}}$. The visual encoder $\mathcal{E}^{\text{img}}$ converts the input image $x^{\text{img}}$ into a sequence of visual tokens $\mathcal{V} = [v_i]_{i=1}^{N_v}$, while the text sequence $x^{\text{txt}}$ is mapped into a sequence of text tokens $\mathcal{T} = [t_i]_{i=1}^{N_t}$ using an embedding function $\mathcal{E}^{\text{txt}}$. The LLM $\mathcal{M_\text{LLM}}(\cdot|\theta)$ models the joint probability of the token sequence $\mathcal{U} = \{\mathcal{V},\mathcal{T}\}$, which is expressed as:
\begin{equation}
    P_\theta(R | \mathcal{U}) = \prod_{i=1}^{N_r} P_\theta(r_i | \{\mathcal{U}, r_{<i}\}),
\end{equation}
where $R = [r_i]_{i=1}^{N_r}$ is the text response sequence. The LVLM iteratively generates the next token $r_i$ based on $r_{<i}$. The optimization objective is to minimize the cross-entropy loss of the response $\mathcal{R}$.
% \begin{equation}
%     \mathcal{L}_{\text{VLM}} = \mathbb{E}_{R|\mathcal{U}}\left[-\log P_\theta(R | \mathcal{U})\right]
% \end{equation}
It is worth noting that most LVLMs adopt a design paradigm based on ViT, alignment adapters, and pre-trained LLMs\cite{liu2023llava,liu2024improved}, enabling quick adaptation to downstream tasks.


\noindent\textbf{VQGAN.}
VQGAN~\cite{esser2021taming} employs latent space compression and indexing mechanisms to effectively learn a complete discrete representation of images. VQGAN first maps the input image $x^{\text{img}}$ to a latent representation $z = \mathcal{E}(x)$ through a encoder $\mathcal{E}$. Then, the latent representation is quantized using a codebook $\mathcal{Z} = \{z_k\}_{k=1}^K$, generating a discrete index sequence $\mathcal{I} = [i_m]_{m=1}^N$, where $i_m \in \mathcal{Z}$ represents the quantized code index:
\begin{equation}
    \mathcal{I} = \text{Quantize}(z|\mathcal{Z}) = \arg\min_{z_k \in \mathcal{Z}} \| z - z_k \|_2.
\end{equation}
In our approach, the discrete index sequence $\mathcal{I}$ serves as a supervisory signal for the generation task, enabling the model to predict the index sequence $\hat{\mathcal{I}}$ from input conditions such as text or other modality signals.  
Finally, the predicted index sequence $\hat{\mathcal{I}}$ is upsampled by the VQGAN decoder $G$, generating the high-quality image $\hat{x}^\text{img} = G(\hat{\mathcal{I}})$.



\noindent\textbf{Low Rank Adaptation.} 
LoRA\cite{hu2021lora} effectively captures the characteristics of downstream tasks by introducing low-rank adapters. The core idea is to decompose the bypass weight matrix $\Delta W\in\mathbb{R}^{d^{\text{in}} \times d^{\text{out}}}$ into two low-rank matrices $ \{A \in \mathbb{R}^{d^{\text{in}} \times r}, B \in \mathbb{R}^{r \times d^{\text{out}}} \}$, where $ r \ll \min\{d^{\text{in}}, d^{\text{out}}\} $, significantly reducing learnable parameters. The output with the LoRA adapter for the input $x$ is then given by:
\begin{equation}
    h = x W_0 + \alpha x \Delta W/r = x W_0 + \alpha xAB/r,
\end{equation}
where matrix $ A $ is initialized with a Gaussian distribution, while the matrix $ B $ is initialized as a zero matrix. The scaling factor $ \alpha/r $ controls the impact of $ \Delta W $ on the model.

\section{HealthGPT}
\label{Method}


\subsection{Unified Autoregressive Generation.}  
% As shown in Figure~\ref{fig:architecture}, 
\ourmethod{} (Figure~\ref{fig:architecture}) utilizes a discrete token representation that covers both text and visual outputs, unifying visual comprehension and generation as an autoregressive task. 
For comprehension, $\mathcal{M}_\text{llm}$ receives the input joint sequence $\mathcal{U}$ and outputs a series of text token $\mathcal{R} = [r_1, r_2, \dots, r_{N_r}]$, where $r_i \in \mathcal{V}_{\text{txt}}$, and $\mathcal{V}_{\text{txt}}$ represents the LLM's vocabulary:
\begin{equation}
    P_\theta(\mathcal{R} \mid \mathcal{U}) = \prod_{i=1}^{N_r} P_\theta(r_i \mid \mathcal{U}, r_{<i}).
\end{equation}
For generation, $\mathcal{M}_\text{llm}$ first receives a special start token $\langle \text{START\_IMG} \rangle$, then generates a series of tokens corresponding to the VQGAN indices $\mathcal{I} = [i_1, i_2, \dots, i_{N_i}]$, where $i_j \in \mathcal{V}_{\text{vq}}$, and $\mathcal{V}_{\text{vq}}$ represents the index range of VQGAN. Upon completion of generation, the LLM outputs an end token $\langle \text{END\_IMG} \rangle$:
\begin{equation}
    P_\theta(\mathcal{I} \mid \mathcal{U}) = \prod_{j=1}^{N_i} P_\theta(i_j \mid \mathcal{U}, i_{<j}).
\end{equation}
Finally, the generated index sequence $\mathcal{I}$ is fed into the decoder $G$, which reconstructs the target image $\hat{x}^{\text{img}} = G(\mathcal{I})$.

\subsection{Hierarchical Visual Perception}  
Given the differences in visual perception between comprehension and generation tasks—where the former focuses on abstract semantics and the latter emphasizes complete semantics—we employ ViT to compress the image into discrete visual tokens at multiple hierarchical levels.
Specifically, the image is converted into a series of features $\{f_1, f_2, \dots, f_L\}$ as it passes through $L$ ViT blocks.

To address the needs of various tasks, the hidden states are divided into two types: (i) \textit{Concrete-grained features} $\mathcal{F}^{\text{Con}} = \{f_1, f_2, \dots, f_k\}, k < L$, derived from the shallower layers of ViT, containing sufficient global features, suitable for generation tasks; 
(ii) \textit{Abstract-grained features} $\mathcal{F}^{\text{Abs}} = \{f_{k+1}, f_{k+2}, \dots, f_L\}$, derived from the deeper layers of ViT, which contain abstract semantic information closer to the text space, suitable for comprehension tasks.

The task type $T$ (comprehension or generation) determines which set of features is selected as the input for the downstream large language model:
\begin{equation}
    \mathcal{F}^{\text{img}}_T =
    \begin{cases}
        \mathcal{F}^{\text{Con}}, & \text{if } T = \text{generation task} \\
        \mathcal{F}^{\text{Abs}}, & \text{if } T = \text{comprehension task}
    \end{cases}
\end{equation}
We integrate the image features $\mathcal{F}^{\text{img}}_T$ and text features $\mathcal{T}$ into a joint sequence through simple concatenation, which is then fed into the LLM $\mathcal{M}_{\text{llm}}$ for autoregressive generation.
% :
% \begin{equation}
%     \mathcal{R} = \mathcal{M}_{\text{llm}}(\mathcal{U}|\theta), \quad \mathcal{U} = [\mathcal{F}^{\text{img}}_T; \mathcal{T}]
% \end{equation}
\subsection{Heterogeneous Knowledge Adaptation}
We devise H-LoRA, which stores heterogeneous knowledge from comprehension and generation tasks in separate modules and dynamically routes to extract task-relevant knowledge from these modules. 
At the task level, for each task type $ T $, we dynamically assign a dedicated H-LoRA submodule $ \theta^T $, which is expressed as:
\begin{equation}
    \mathcal{R} = \mathcal{M}_\text{LLM}(\mathcal{U}|\theta, \theta^T), \quad \theta^T = \{A^T, B^T, \mathcal{R}^T_\text{outer}\}.
\end{equation}
At the feature level for a single task, H-LoRA integrates the idea of Mixture of Experts (MoE)~\cite{masoudnia2014mixture} and designs an efficient matrix merging and routing weight allocation mechanism, thus avoiding the significant computational delay introduced by matrix splitting in existing MoELoRA~\cite{luo2024moelora}. Specifically, we first merge the low-rank matrices (rank = r) of $ k $ LoRA experts into a unified matrix:
\begin{equation}
    \mathbf{A}^{\text{merged}}, \mathbf{B}^{\text{merged}} = \text{Concat}(\{A_i\}_1^k), \text{Concat}(\{B_i\}_1^k),
\end{equation}
where $ \mathbf{A}^{\text{merged}} \in \mathbb{R}^{d^\text{in} \times rk} $ and $ \mathbf{B}^{\text{merged}} \in \mathbb{R}^{rk \times d^\text{out}} $. The $k$-dimension routing layer generates expert weights $ \mathcal{W} \in \mathbb{R}^{\text{token\_num} \times k} $ based on the input hidden state $ x $, and these are expanded to $ \mathbb{R}^{\text{token\_num} \times rk} $ as follows:
\begin{equation}
    \mathcal{W}^\text{expanded} = \alpha k \mathcal{W} / r \otimes \mathbf{1}_r,
\end{equation}
where $ \otimes $ denotes the replication operation.
The overall output of H-LoRA is computed as:
\begin{equation}
    \mathcal{O}^\text{H-LoRA} = (x \mathbf{A}^{\text{merged}} \odot \mathcal{W}^\text{expanded}) \mathbf{B}^{\text{merged}},
\end{equation}
where $ \odot $ represents element-wise multiplication. Finally, the output of H-LoRA is added to the frozen pre-trained weights to produce the final output:
\begin{equation}
    \mathcal{O} = x W_0 + \mathcal{O}^\text{H-LoRA}.
\end{equation}
% In summary, H-LoRA is a task-based dynamic PEFT method that achieves high efficiency in single-task fine-tuning.

\subsection{Training Pipeline}

\begin{figure}[t]
    \centering
    \hspace{-4mm}
    \includegraphics[width=0.94\linewidth]{fig/data.pdf}
    \caption{Data statistics of \texttt{VL-Health}. }
    \label{fig:data}
\end{figure}
\noindent \textbf{1st Stage: Multi-modal Alignment.} 
In the first stage, we design separate visual adapters and H-LoRA submodules for medical unified tasks. For the medical comprehension task, we train abstract-grained visual adapters using high-quality image-text pairs to align visual embeddings with textual embeddings, thereby enabling the model to accurately describe medical visual content. During this process, the pre-trained LLM and its corresponding H-LoRA submodules remain frozen. In contrast, the medical generation task requires training concrete-grained adapters and H-LoRA submodules while keeping the LLM frozen. Meanwhile, we extend the textual vocabulary to include multimodal tokens, enabling the support of additional VQGAN vector quantization indices. The model trains on image-VQ pairs, endowing the pre-trained LLM with the capability for image reconstruction. This design ensures pixel-level consistency of pre- and post-LVLM. The processes establish the initial alignment between the LLM’s outputs and the visual inputs.

\noindent \textbf{2nd Stage: Heterogeneous H-LoRA Plugin Adaptation.}  
The submodules of H-LoRA share the word embedding layer and output head but may encounter issues such as bias and scale inconsistencies during training across different tasks. To ensure that the multiple H-LoRA plugins seamlessly interface with the LLMs and form a unified base, we fine-tune the word embedding layer and output head using a small amount of mixed data to maintain consistency in the model weights. Specifically, during this stage, all H-LoRA submodules for different tasks are kept frozen, with only the word embedding layer and output head being optimized. Through this stage, the model accumulates foundational knowledge for unified tasks by adapting H-LoRA plugins.

\begin{table*}[!t]
\centering
\caption{Comparison of \ourmethod{} with other LVLMs and unified multi-modal models on medical visual comprehension tasks. \textbf{Bold} and \underline{underlined} text indicates the best performance and second-best performance, respectively.}
\resizebox{\textwidth}{!}{
\begin{tabular}{c|lcc|cccccccc|c}
\toprule
\rowcolor[HTML]{E9F3FE} &  &  &  & \multicolumn{2}{c}{\textbf{VQA-RAD \textuparrow}} & \multicolumn{2}{c}{\textbf{SLAKE \textuparrow}} & \multicolumn{2}{c}{\textbf{PathVQA \textuparrow}} &  &  &  \\ 
\cline{5-10}
\rowcolor[HTML]{E9F3FE}\multirow{-2}{*}{\textbf{Type}} & \multirow{-2}{*}{\textbf{Model}} & \multirow{-2}{*}{\textbf{\# Params}} & \multirow{-2}{*}{\makecell{\textbf{Medical} \\ \textbf{LVLM}}} & \textbf{close} & \textbf{all} & \textbf{close} & \textbf{all} & \textbf{close} & \textbf{all} & \multirow{-2}{*}{\makecell{\textbf{MMMU} \\ \textbf{-Med}}\textuparrow} & \multirow{-2}{*}{\textbf{OMVQA}\textuparrow} & \multirow{-2}{*}{\textbf{Avg. \textuparrow}} \\ 
\midrule \midrule
\multirow{9}{*}{\textbf{Comp. Only}} 
& Med-Flamingo & 8.3B & \Large \ding{51} & 58.6 & 43.0 & 47.0 & 25.5 & 61.9 & 31.3 & 28.7 & 34.9 & 41.4 \\
& LLaVA-Med & 7B & \Large \ding{51} & 60.2 & 48.1 & 58.4 & 44.8 & 62.3 & 35.7 & 30.0 & 41.3 & 47.6 \\
& HuatuoGPT-Vision & 7B & \Large \ding{51} & 66.9 & 53.0 & 59.8 & 49.1 & 52.9 & 32.0 & 42.0 & 50.0 & 50.7 \\
& BLIP-2 & 6.7B & \Large \ding{55} & 43.4 & 36.8 & 41.6 & 35.3 & 48.5 & 28.8 & 27.3 & 26.9 & 36.1 \\
& LLaVA-v1.5 & 7B & \Large \ding{55} & 51.8 & 42.8 & 37.1 & 37.7 & 53.5 & 31.4 & 32.7 & 44.7 & 41.5 \\
& InstructBLIP & 7B & \Large \ding{55} & 61.0 & 44.8 & 66.8 & 43.3 & 56.0 & 32.3 & 25.3 & 29.0 & 44.8 \\
& Yi-VL & 6B & \Large \ding{55} & 52.6 & 42.1 & 52.4 & 38.4 & 54.9 & 30.9 & 38.0 & 50.2 & 44.9 \\
& InternVL2 & 8B & \Large \ding{55} & 64.9 & 49.0 & 66.6 & 50.1 & 60.0 & 31.9 & \underline{43.3} & 54.5 & 52.5\\
& Llama-3.2 & 11B & \Large \ding{55} & 68.9 & 45.5 & 72.4 & 52.1 & 62.8 & 33.6 & 39.3 & 63.2 & 54.7 \\
\midrule
\multirow{5}{*}{\textbf{Comp. \& Gen.}} 
& Show-o & 1.3B & \Large \ding{55} & 50.6 & 33.9 & 31.5 & 17.9 & 52.9 & 28.2 & 22.7 & 45.7 & 42.6 \\
& Unified-IO 2 & 7B & \Large \ding{55} & 46.2 & 32.6 & 35.9 & 21.9 & 52.5 & 27.0 & 25.3 & 33.0 & 33.8 \\
& Janus & 1.3B & \Large \ding{55} & 70.9 & 52.8 & 34.7 & 26.9 & 51.9 & 27.9 & 30.0 & 26.8 & 33.5 \\
& \cellcolor[HTML]{DAE0FB}HealthGPT-M3 & \cellcolor[HTML]{DAE0FB}3.8B & \cellcolor[HTML]{DAE0FB}\Large \ding{51} & \cellcolor[HTML]{DAE0FB}\underline{73.7} & \cellcolor[HTML]{DAE0FB}\underline{55.9} & \cellcolor[HTML]{DAE0FB}\underline{74.6} & \cellcolor[HTML]{DAE0FB}\underline{56.4} & \cellcolor[HTML]{DAE0FB}\underline{78.7} & \cellcolor[HTML]{DAE0FB}\underline{39.7} & \cellcolor[HTML]{DAE0FB}\underline{43.3} & \cellcolor[HTML]{DAE0FB}\underline{68.5} & \cellcolor[HTML]{DAE0FB}\underline{61.3} \\
& \cellcolor[HTML]{DAE0FB}HealthGPT-L14 & \cellcolor[HTML]{DAE0FB}14B & \cellcolor[HTML]{DAE0FB}\Large \ding{51} & \cellcolor[HTML]{DAE0FB}\textbf{77.7} & \cellcolor[HTML]{DAE0FB}\textbf{58.3} & \cellcolor[HTML]{DAE0FB}\textbf{76.4} & \cellcolor[HTML]{DAE0FB}\textbf{64.5} & \cellcolor[HTML]{DAE0FB}\textbf{85.9} & \cellcolor[HTML]{DAE0FB}\textbf{44.4} & \cellcolor[HTML]{DAE0FB}\textbf{49.2} & \cellcolor[HTML]{DAE0FB}\textbf{74.4} & \cellcolor[HTML]{DAE0FB}\textbf{66.4} \\
\bottomrule
\end{tabular}
}
\label{tab:results}
\end{table*}
\begin{table*}[ht]
    \centering
    \caption{The experimental results for the four modality conversion tasks.}
    \resizebox{\textwidth}{!}{
    \begin{tabular}{l|ccc|ccc|ccc|ccc}
        \toprule
        \rowcolor[HTML]{E9F3FE} & \multicolumn{3}{c}{\textbf{CT to MRI (Brain)}} & \multicolumn{3}{c}{\textbf{CT to MRI (Pelvis)}} & \multicolumn{3}{c}{\textbf{MRI to CT (Brain)}} & \multicolumn{3}{c}{\textbf{MRI to CT (Pelvis)}} \\
        \cline{2-13}
        \rowcolor[HTML]{E9F3FE}\multirow{-2}{*}{\textbf{Model}}& \textbf{SSIM $\uparrow$} & \textbf{PSNR $\uparrow$} & \textbf{MSE $\downarrow$} & \textbf{SSIM $\uparrow$} & \textbf{PSNR $\uparrow$} & \textbf{MSE $\downarrow$} & \textbf{SSIM $\uparrow$} & \textbf{PSNR $\uparrow$} & \textbf{MSE $\downarrow$} & \textbf{SSIM $\uparrow$} & \textbf{PSNR $\uparrow$} & \textbf{MSE $\downarrow$} \\
        \midrule \midrule
        pix2pix & 71.09 & 32.65 & 36.85 & 59.17 & 31.02 & 51.91 & 78.79 & 33.85 & 28.33 & 72.31 & 32.98 & 36.19 \\
        CycleGAN & 54.76 & 32.23 & 40.56 & 54.54 & 30.77 & 55.00 & 63.75 & 31.02 & 52.78 & 50.54 & 29.89 & 67.78 \\
        BBDM & {71.69} & {32.91} & {34.44} & 57.37 & 31.37 & 48.06 & \textbf{86.40} & 34.12 & 26.61 & {79.26} & 33.15 & 33.60 \\
        Vmanba & 69.54 & 32.67 & 36.42 & {63.01} & {31.47} & {46.99} & 79.63 & 34.12 & 26.49 & 77.45 & 33.53 & 31.85 \\
        DiffMa & 71.47 & 32.74 & 35.77 & 62.56 & 31.43 & 47.38 & 79.00 & {34.13} & {26.45} & 78.53 & {33.68} & {30.51} \\
        \rowcolor[HTML]{DAE0FB}HealthGPT-M3 & \underline{79.38} & \underline{33.03} & \underline{33.48} & \underline{71.81} & \underline{31.83} & \underline{43.45} & {85.06} & \textbf{34.40} & \textbf{25.49} & \underline{84.23} & \textbf{34.29} & \textbf{27.99} \\
        \rowcolor[HTML]{DAE0FB}HealthGPT-L14 & \textbf{79.73} & \textbf{33.10} & \textbf{32.96} & \textbf{71.92} & \textbf{31.87} & \textbf{43.09} & \underline{85.31} & \underline{34.29} & \underline{26.20} & \textbf{84.96} & \underline{34.14} & \underline{28.13} \\
        \bottomrule
    \end{tabular}
    }
    \label{tab:conversion}
\end{table*}

\noindent \textbf{3rd Stage: Visual Instruction Fine-Tuning.}  
In the third stage, we introduce additional task-specific data to further optimize the model and enhance its adaptability to downstream tasks such as medical visual comprehension (e.g., medical QA, medical dialogues, and report generation) or generation tasks (e.g., super-resolution, denoising, and modality conversion). Notably, by this stage, the word embedding layer and output head have been fine-tuned, only the H-LoRA modules and adapter modules need to be trained. This strategy significantly improves the model's adaptability and flexibility across different tasks.


\section{Experiment}
\label{s:experiment}

\subsection{Data Description}
We evaluate our method on FI~\cite{you2016building}, Twitter\_LDL~\cite{yang2017learning} and Artphoto~\cite{machajdik2010affective}.
FI is a public dataset built from Flickr and Instagram, with 23,308 images and eight emotion categories, namely \textit{amusement}, \textit{anger}, \textit{awe},  \textit{contentment}, \textit{disgust}, \textit{excitement},  \textit{fear}, and \textit{sadness}. 
% Since images in FI are all copyrighted by law, some images are corrupted now, so we remove these samples and retain 21,828 images.
% T4SA contains images from Twitter, which are classified into three categories: \textit{positive}, \textit{neutral}, and \textit{negative}. In this paper, we adopt the base version of B-T4SA, which contains 470,586 images and provides text descriptions of the corresponding tweets.
Twitter\_LDL contains 10,045 images from Twitter, with the same eight categories as the FI dataset.
% 。
For these two datasets, they are randomly split into 80\%
training and 20\% testing set.
Artphoto contains 806 artistic photos from the DeviantArt website, which we use to further evaluate the zero-shot capability of our model.
% on the small-scale dataset.
% We construct and publicly release the first image sentiment analysis dataset containing metadata.
% 。

% Based on these datasets, we are the first to construct and publicly release metadata-enhanced image sentiment analysis datasets. These datasets include scenes, tags, descriptions, and corresponding confidence scores, and are available at this link for future research purposes.


% 
\begin{table}[t]
\centering
% \begin{center}
\caption{Overall performance of different models on FI and Twitter\_LDL datasets.}
\label{tab:cap1}
% \resizebox{\linewidth}{!}
{
\begin{tabular}{l|c|c|c|c}
\hline
\multirow{2}{*}{\textbf{Model}} & \multicolumn{2}{c|}{\textbf{FI}}  & \multicolumn{2}{c}{\textbf{Twitter\_LDL}} \\ \cline{2-5} 
  & \textbf{Accuracy} & \textbf{F1} & \textbf{Accuracy} & \textbf{F1}  \\ \hline
% (\rownumber)~AlexNet~\cite{krizhevsky2017imagenet}  & 58.13\% & 56.35\%  & 56.24\%& 55.02\%  \\ 
% (\rownumber)~VGG16~\cite{simonyan2014very}  & 63.75\%& 63.08\%  & 59.34\%& 59.02\%  \\ 
(\rownumber)~ResNet101~\cite{he2016deep} & 66.16\%& 65.56\%  & 62.02\% & 61.34\%  \\ 
(\rownumber)~CDA~\cite{han2023boosting} & 66.71\%& 65.37\%  & 64.14\% & 62.85\%  \\ 
(\rownumber)~CECCN~\cite{ruan2024color} & 67.96\%& 66.74\%  & 64.59\%& 64.72\% \\ 
(\rownumber)~EmoVIT~\cite{xie2024emovit} & 68.09\%& 67.45\%  & 63.12\% & 61.97\%  \\ 
(\rownumber)~ComLDL~\cite{zhang2022compound} & 68.83\%& 67.28\%  & 65.29\% & 63.12\%  \\ 
(\rownumber)~WSDEN~\cite{li2023weakly} & 69.78\%& 69.61\%  & 67.04\% & 65.49\% \\ 
(\rownumber)~ECWA~\cite{deng2021emotion} & 70.87\%& 69.08\%  & 67.81\% & 66.87\%  \\ 
(\rownumber)~EECon~\cite{yang2023exploiting} & 71.13\%& 68.34\%  & 64.27\%& 63.16\%  \\ 
(\rownumber)~MAM~\cite{zhang2024affective} & 71.44\%  & 70.83\% & 67.18\%  & 65.01\%\\ 
(\rownumber)~TGCA-PVT~\cite{chen2024tgca}   & 73.05\%  & 71.46\% & 69.87\%  & 68.32\% \\ 
(\rownumber)~OEAN~\cite{zhang2024object}   & 73.40\%  & 72.63\% & 70.52\%  & 69.47\% \\ \hline
(\rownumber)~\shortname  & \textbf{79.48\%} & \textbf{79.22\%} & \textbf{74.12\%} & \textbf{73.09\%} \\ \hline
\end{tabular}
}
\vspace{-6mm}
% \end{center}
\end{table}
% 

\subsection{Experiment Setting}
% \subsubsection{Model Setting.}
% 
\textbf{Model Setting:}
For feature representation, we set $k=10$ to select object tags, and adopt clip-vit-base-patch32 as the pre-trained model for unified feature representation.
Moreover, we empirically set $(d_e, d_h, d_k, d_s) = (512, 128, 16, 64)$, and set the classification class $L$ to 8.

% 

\textbf{Training Setting:}
To initialize the model, we set all weights such as $\boldsymbol{W}$ following the truncated normal distribution, and use AdamW optimizer with the learning rate of $1 \times 10^{-4}$.
% warmup scheduler of cosine, warmup steps of 2000.
Furthermore, we set the batch size to 32 and the epoch of the training process to 200.
During the implementation, we utilize \textit{PyTorch} to build our entire model.
% , and our project codes are publicly available at https://github.com/zzmyrep/MESN.
% Our project codes as well as data are all publicly available on GitHub\footnote{https://github.com/zzmyrep/KBCEN}.
% Code is available at \href{https://github.com/zzmyrep/KBCEN}{https://github.com/zzmyrep/KBCEN}.

\textbf{Evaluation Metrics:}
Following~\cite{zhang2024affective, chen2024tgca, zhang2024object}, we adopt \textit{accuracy} and \textit{F1} as our evaluation metrics to measure the performance of different methods for image sentiment analysis. 



\subsection{Experiment Result}
% We compare our model against the following baselines: AlexNet~\cite{krizhevsky2017imagenet}, VGG16~\cite{simonyan2014very}, ResNet101~\cite{he2016deep}, CECCN~\cite{ruan2024color}, EmoVIT~\cite{xie2024emovit}, WSCNet~\cite{yang2018weakly}, ECWA~\cite{deng2021emotion}, EECon~\cite{yang2023exploiting}, MAM~\cite{zhang2024affective} and TGCA-PVT~\cite{chen2024tgca}, and the overall results are summarized in Table~\ref{tab:cap1}.
We compare our model against several baselines, and the overall results are summarized in Table~\ref{tab:cap1}.
We observe that our model achieves the best performance in both accuracy and F1 metrics, significantly outperforming the previous models. 
This superior performance is mainly attributed to our effective utilization of metadata to enhance image sentiment analysis, as well as the exceptional capability of the unified sentiment transformer framework we developed. These results strongly demonstrate that our proposed method can bring encouraging performance for image sentiment analysis.

\setcounter{magicrownumbers}{0} 
\begin{table}[t]
\begin{center}
\caption{Ablation study of~\shortname~on FI dataset.} 
% \vspace{1mm}
\label{tab:cap2}
\resizebox{.9\linewidth}{!}
{
\begin{tabular}{lcc}
  \hline
  \textbf{Model} & \textbf{Accuracy} & \textbf{F1} \\
  \hline
  (\rownumber)~Ours (w/o vision) & 65.72\% & 64.54\% \\
  (\rownumber)~Ours (w/o text description) & 74.05\% & 72.58\% \\
  (\rownumber)~Ours (w/o object tag) & 77.45\% & 76.84\% \\
  (\rownumber)~Ours (w/o scene tag) & 78.47\% & 78.21\% \\
  \hline
  (\rownumber)~Ours (w/o unified embedding) & 76.41\% & 76.23\% \\
  (\rownumber)~Ours (w/o adaptive learning) & 76.83\% & 76.56\% \\
  (\rownumber)~Ours (w/o cross-modal fusion) & 76.85\% & 76.49\% \\
  \hline
  (\rownumber)~Ours  & \textbf{79.48\%} & \textbf{79.22\%} \\
  \hline
\end{tabular}
}
\end{center}
\vspace{-5mm}
\end{table}


\begin{figure}[t]
\centering
% \vspace{-2mm}
\includegraphics[width=0.42\textwidth]{fig/2dvisual-linux4-paper2.pdf}
\caption{Visualization of feature distribution on eight categories before (left) and after (right) model processing.}
% 
\label{fig:visualization}
\vspace{-5mm}
\end{figure}

\subsection{Ablation Performance}
In this subsection, we conduct an ablation study to examine which component is really important for performance improvement. The results are reported in Table~\ref{tab:cap2}.

For information utilization, we observe a significant decline in model performance when visual features are removed. Additionally, the performance of \shortname~decreases when different metadata are removed separately, which means that text description, object tag, and scene tag are all critical for image sentiment analysis.
Recalling the model architecture, we separately remove transformer layers of the unified representation module, the adaptive learning module, and the cross-modal fusion module, replacing them with MLPs of the same parameter scale.
In this way, we can observe varying degrees of decline in model performance, indicating that these modules are indispensable for our model to achieve better performance.

\subsection{Visualization}
% 


% % 开始使用minipage进行左右排列
% \begin{minipage}[t]{0.45\textwidth}  % 子图1宽度为45%
%     \centering
%     \includegraphics[width=\textwidth]{2dvisual.pdf}  % 插入图片
%     \captionof{figure}{Visualization of feature distribution.}  % 使用captionof添加图片标题
%     \label{fig:visualization}
% \end{minipage}


% \begin{figure}[t]
% \centering
% \vspace{-2mm}
% \includegraphics[width=0.45\textwidth]{fig/2dvisual.pdf}
% \caption{Visualization of feature distribution.}
% \label{fig:visualization}
% % \vspace{-4mm}
% \end{figure}

% \begin{figure}[t]
% \centering
% \vspace{-2mm}
% \includegraphics[width=0.45\textwidth]{fig/2dvisual-linux3-paper.pdf}
% \caption{Visualization of feature distribution.}
% \label{fig:visualization}
% % \vspace{-4mm}
% \end{figure}



\begin{figure}[tbp]   
\vspace{-4mm}
  \centering            
  \subfloat[Depth of adaptive learning layers]   
  {
    \label{fig:subfig1}\includegraphics[width=0.22\textwidth]{fig/fig_sensitivity-a5}
  }
  \subfloat[Depth of fusion layers]
  {
    % \label{fig:subfig2}\includegraphics[width=0.22\textwidth]{fig/fig_sensitivity-b2}
    \label{fig:subfig2}\includegraphics[width=0.22\textwidth]{fig/fig_sensitivity-b2-num.pdf}
  }
  \caption{Sensitivity study of \shortname~on different depth. }   
  \label{fig:fig_sensitivity}  
\vspace{-2mm}
\end{figure}

% \begin{figure}[htbp]
% \centerline{\includegraphics{2dvisual.pdf}}
% \caption{Visualization of feature distribution.}
% \label{fig:visualization}
% \end{figure}

% In Fig.~\ref{fig:visualization}, we use t-SNE~\cite{van2008visualizing} to reduce the dimension of data features for visualization, Figure in left represents the metadata features before model processing, the features are obtained by embedding through the CLIP model, and figure in right shows the features of the data after model processing, it can be observed that after the model processing, the data with different label categories fall in different regions in the space, therefore, we can conclude that the Therefore, we can conclude that the model can effectively utilize the information contained in the metadata and use it to guide the model for classification.

In Fig.~\ref{fig:visualization}, we use t-SNE~\cite{van2008visualizing} to reduce the dimension of data features for visualization.
The left figure shows metadata features before being processed by our model (\textit{i.e.}, embedded by CLIP), while the right shows the distribution of features after being processed by our model.
We can observe that after the model processing, data with the same label are closer to each other, while others are farther away.
Therefore, it shows that the model can effectively utilize the information contained in the metadata and use it to guide the classification process.

\subsection{Sensitivity Analysis}
% 
In this subsection, we conduct a sensitivity analysis to figure out the effect of different depth settings of adaptive learning layers and fusion layers. 
% In this subsection, we conduct a sensitivity analysis to figure out the effect of different depth settings on the model. 
% Fig.~\ref{fig:fig_sensitivity} presents the effect of different depth settings of adaptive learning layers and fusion layers. 
Taking Fig.~\ref{fig:fig_sensitivity} (a) as an example, the model performance improves with increasing depth, reaching the best performance at a depth of 4.
% Taking Fig.~\ref{fig:fig_sensitivity} (a) as an example, the performance of \shortname~improves with the increase of depth at first, reaching the best performance at a depth of 4.
When the depth continues to increase, the accuracy decreases to varying degrees.
Similar results can be observed in Fig.~\ref{fig:fig_sensitivity} (b).
Therefore, we set their depths to 4 and 6 respectively to achieve the best results.

% Through our experiments, we can observe that the effect of modifying these hyperparameters on the results of the experiments is very weak, and the surface model is not sensitive to the hyperparameters.


\subsection{Zero-shot Capability}
% 

% (1)~GCH~\cite{2010Analyzing} & 21.78\% & (5)~RA-DLNet~\cite{2020A} & 34.01\% \\ \hline
% (2)~WSCNet~\cite{2019WSCNet}  & 30.25\% & (6)~CECCN~\cite{ruan2024color} & 43.83\% \\ \hline
% (3)~PCNN~\cite{2015Robust} & 31.68\%  & (7)~EmoVIT~\cite{xie2024emovit} & 44.90\% \\ \hline
% (4)~AR~\cite{2018Visual} & 32.67\% & (8)~Ours (Zero-shot) & 47.83\% \\ \hline


\begin{table}[t]
\centering
\caption{Zero-shot capability of \shortname.}
\label{tab:cap3}
\resizebox{1\linewidth}{!}
{
\begin{tabular}{lc|lc}
\hline
\textbf{Model} & \textbf{Accuracy} & \textbf{Model} & \textbf{Accuracy} \\ \hline
(1)~WSCNet~\cite{2019WSCNet}  & 30.25\% & (5)~MAM~\cite{zhang2024affective} & 39.56\%  \\ \hline
(2)~AR~\cite{2018Visual} & 32.67\% & (6)~CECCN~\cite{ruan2024color} & 43.83\% \\ \hline
(3)~RA-DLNet~\cite{2020A} & 34.01\%  & (7)~EmoVIT~\cite{xie2024emovit} & 44.90\% \\ \hline
(4)~CDA~\cite{han2023boosting} & 38.64\% & (8)~Ours (Zero-shot) & 47.83\% \\ \hline
\end{tabular}
}
\vspace{-5mm}
\end{table}

% We use the model trained on the FI dataset to test on the artphoto dataset to verify the model's generalization ability as well as robustness to other distributed datasets.
% We can observe that the MESN model shows strong competitiveness in terms of accuracy when compared to other trained models, which suggests that the model has a good generalization ability in the OOD task.

To validate the model's generalization ability and robustness to other distributed datasets, we directly test the model trained on the FI dataset, without training on Artphoto. 
% As observed in Table 3, compared to other models trained on Artphoto, we achieve highly competitive zero-shot performance, indicating that the model has good generalization ability in out-of-distribution tasks.
From Table~\ref{tab:cap3}, we can observe that compared with other models trained on Artphoto, we achieve competitive zero-shot performance, which shows that the model has good generalization ability in out-of-distribution tasks.


%%%%%%%%%%%%
%  E2E     %
%%%%%%%%%%%%


\section{Conclusion}
In this paper, we introduced Wi-Chat, the first LLM-powered Wi-Fi-based human activity recognition system that integrates the reasoning capabilities of large language models with the sensing potential of wireless signals. Our experimental results on a self-collected Wi-Fi CSI dataset demonstrate the promising potential of LLMs in enabling zero-shot Wi-Fi sensing. These findings suggest a new paradigm for human activity recognition that does not rely on extensive labeled data. We hope future research will build upon this direction, further exploring the applications of LLMs in signal processing domains such as IoT, mobile sensing, and radar-based systems.

\section*{Limitations}
While our work represents the first attempt to leverage LLMs for processing Wi-Fi signals, it is a preliminary study focused on a relatively simple task: Wi-Fi-based human activity recognition. This choice allows us to explore the feasibility of LLMs in wireless sensing but also comes with certain limitations.

Our approach primarily evaluates zero-shot performance, which, while promising, may still lag behind traditional supervised learning methods in highly complex or fine-grained recognition tasks. Besides, our study is limited to a controlled environment with a self-collected dataset, and the generalizability of LLMs to diverse real-world scenarios with varying Wi-Fi conditions, environmental interference, and device heterogeneity remains an open question.

Additionally, we have yet to explore the full potential of LLMs in more advanced Wi-Fi sensing applications, such as fine-grained gesture recognition, occupancy detection, and passive health monitoring. Future work should investigate the scalability of LLM-based approaches, their robustness to domain shifts, and their integration with multimodal sensing techniques in broader IoT applications.


% Bibliography entries for the entire Anthology, followed by custom entries
%\bibliography{anthology,custom}
% Custom bibliography entries only
\bibliography{main}
\newpage
\appendix

\section{Experiment prompts}
\label{sec:prompt}
The prompts used in the LLM experiments are shown in the following Table~\ref{tab:prompts}.

\definecolor{titlecolor}{rgb}{0.9, 0.5, 0.1}
\definecolor{anscolor}{rgb}{0.2, 0.5, 0.8}
\definecolor{labelcolor}{HTML}{48a07e}
\begin{table*}[h]
	\centering
	
 % \vspace{-0.2cm}
	
	\begin{center}
		\begin{tikzpicture}[
				chatbox_inner/.style={rectangle, rounded corners, opacity=0, text opacity=1, font=\sffamily\scriptsize, text width=5in, text height=9pt, inner xsep=6pt, inner ysep=6pt},
				chatbox_prompt_inner/.style={chatbox_inner, align=flush left, xshift=0pt, text height=11pt},
				chatbox_user_inner/.style={chatbox_inner, align=flush left, xshift=0pt},
				chatbox_gpt_inner/.style={chatbox_inner, align=flush left, xshift=0pt},
				chatbox/.style={chatbox_inner, draw=black!25, fill=gray!7, opacity=1, text opacity=0},
				chatbox_prompt/.style={chatbox, align=flush left, fill=gray!1.5, draw=black!30, text height=10pt},
				chatbox_user/.style={chatbox, align=flush left},
				chatbox_gpt/.style={chatbox, align=flush left},
				chatbox2/.style={chatbox_gpt, fill=green!25},
				chatbox3/.style={chatbox_gpt, fill=red!20, draw=black!20},
				chatbox4/.style={chatbox_gpt, fill=yellow!30},
				labelbox/.style={rectangle, rounded corners, draw=black!50, font=\sffamily\scriptsize\bfseries, fill=gray!5, inner sep=3pt},
			]
											
			\node[chatbox_user] (q1) {
				\textbf{System prompt}
				\newline
				\newline
				You are a helpful and precise assistant for segmenting and labeling sentences. We would like to request your help on curating a dataset for entity-level hallucination detection.
				\newline \newline
                We will give you a machine generated biography and a list of checked facts about the biography. Each fact consists of a sentence and a label (True/False). Please do the following process. First, breaking down the biography into words. Second, by referring to the provided list of facts, merging some broken down words in the previous step to form meaningful entities. For example, ``strategic thinking'' should be one entity instead of two. Third, according to the labels in the list of facts, labeling each entity as True or False. Specifically, for facts that share a similar sentence structure (\eg, \textit{``He was born on Mach 9, 1941.''} (\texttt{True}) and \textit{``He was born in Ramos Mejia.''} (\texttt{False})), please first assign labels to entities that differ across atomic facts. For example, first labeling ``Mach 9, 1941'' (\texttt{True}) and ``Ramos Mejia'' (\texttt{False}) in the above case. For those entities that are the same across atomic facts (\eg, ``was born'') or are neutral (\eg, ``he,'' ``in,'' and ``on''), please label them as \texttt{True}. For the cases that there is no atomic fact that shares the same sentence structure, please identify the most informative entities in the sentence and label them with the same label as the atomic fact while treating the rest of the entities as \texttt{True}. In the end, output the entities and labels in the following format:
                \begin{itemize}[nosep]
                    \item Entity 1 (Label 1)
                    \item Entity 2 (Label 2)
                    \item ...
                    \item Entity N (Label N)
                \end{itemize}
                % \newline \newline
                Here are two examples:
                \newline\newline
                \textbf{[Example 1]}
                \newline
                [The start of the biography]
                \newline
                \textcolor{titlecolor}{Marianne McAndrew is an American actress and singer, born on November 21, 1942, in Cleveland, Ohio. She began her acting career in the late 1960s, appearing in various television shows and films.}
                \newline
                [The end of the biography]
                \newline \newline
                [The start of the list of checked facts]
                \newline
                \textcolor{anscolor}{[Marianne McAndrew is an American. (False); Marianne McAndrew is an actress. (True); Marianne McAndrew is a singer. (False); Marianne McAndrew was born on November 21, 1942. (False); Marianne McAndrew was born in Cleveland, Ohio. (False); She began her acting career in the late 1960s. (True); She has appeared in various television shows. (True); She has appeared in various films. (True)]}
                \newline
                [The end of the list of checked facts]
                \newline \newline
                [The start of the ideal output]
                \newline
                \textcolor{labelcolor}{[Marianne McAndrew (True); is (True); an (True); American (False); actress (True); and (True); singer (False); , (True); born (True); on (True); November 21, 1942 (False); , (True); in (True); Cleveland, Ohio (False); . (True); She (True); began (True); her (True); acting career (True); in (True); the late 1960s (True); , (True); appearing (True); in (True); various (True); television shows (True); and (True); films (True); . (True)]}
                \newline
                [The end of the ideal output]
				\newline \newline
                \textbf{[Example 2]}
                \newline
                [The start of the biography]
                \newline
                \textcolor{titlecolor}{Doug Sheehan is an American actor who was born on April 27, 1949, in Santa Monica, California. He is best known for his roles in soap operas, including his portrayal of Joe Kelly on ``General Hospital'' and Ben Gibson on ``Knots Landing.''}
                \newline
                [The end of the biography]
                \newline \newline
                [The start of the list of checked facts]
                \newline
                \textcolor{anscolor}{[Doug Sheehan is an American. (True); Doug Sheehan is an actor. (True); Doug Sheehan was born on April 27, 1949. (True); Doug Sheehan was born in Santa Monica, California. (False); He is best known for his roles in soap operas. (True); He portrayed Joe Kelly. (True); Joe Kelly was in General Hospital. (True); General Hospital is a soap opera. (True); He portrayed Ben Gibson. (True); Ben Gibson was in Knots Landing. (True); Knots Landing is a soap opera. (True)]}
                \newline
                [The end of the list of checked facts]
                \newline \newline
                [The start of the ideal output]
                \newline
                \textcolor{labelcolor}{[Doug Sheehan (True); is (True); an (True); American (True); actor (True); who (True); was born (True); on (True); April 27, 1949 (True); in (True); Santa Monica, California (False); . (True); He (True); is (True); best known (True); for (True); his roles in soap operas (True); , (True); including (True); in (True); his portrayal (True); of (True); Joe Kelly (True); on (True); ``General Hospital'' (True); and (True); Ben Gibson (True); on (True); ``Knots Landing.'' (True)]}
                \newline
                [The end of the ideal output]
				\newline \newline
				\textbf{User prompt}
				\newline
				\newline
				[The start of the biography]
				\newline
				\textcolor{magenta}{\texttt{\{BIOGRAPHY\}}}
				\newline
				[The ebd of the biography]
				\newline \newline
				[The start of the list of checked facts]
				\newline
				\textcolor{magenta}{\texttt{\{LIST OF CHECKED FACTS\}}}
				\newline
				[The end of the list of checked facts]
			};
			\node[chatbox_user_inner] (q1_text) at (q1) {
				\textbf{System prompt}
				\newline
				\newline
				You are a helpful and precise assistant for segmenting and labeling sentences. We would like to request your help on curating a dataset for entity-level hallucination detection.
				\newline \newline
                We will give you a machine generated biography and a list of checked facts about the biography. Each fact consists of a sentence and a label (True/False). Please do the following process. First, breaking down the biography into words. Second, by referring to the provided list of facts, merging some broken down words in the previous step to form meaningful entities. For example, ``strategic thinking'' should be one entity instead of two. Third, according to the labels in the list of facts, labeling each entity as True or False. Specifically, for facts that share a similar sentence structure (\eg, \textit{``He was born on Mach 9, 1941.''} (\texttt{True}) and \textit{``He was born in Ramos Mejia.''} (\texttt{False})), please first assign labels to entities that differ across atomic facts. For example, first labeling ``Mach 9, 1941'' (\texttt{True}) and ``Ramos Mejia'' (\texttt{False}) in the above case. For those entities that are the same across atomic facts (\eg, ``was born'') or are neutral (\eg, ``he,'' ``in,'' and ``on''), please label them as \texttt{True}. For the cases that there is no atomic fact that shares the same sentence structure, please identify the most informative entities in the sentence and label them with the same label as the atomic fact while treating the rest of the entities as \texttt{True}. In the end, output the entities and labels in the following format:
                \begin{itemize}[nosep]
                    \item Entity 1 (Label 1)
                    \item Entity 2 (Label 2)
                    \item ...
                    \item Entity N (Label N)
                \end{itemize}
                % \newline \newline
                Here are two examples:
                \newline\newline
                \textbf{[Example 1]}
                \newline
                [The start of the biography]
                \newline
                \textcolor{titlecolor}{Marianne McAndrew is an American actress and singer, born on November 21, 1942, in Cleveland, Ohio. She began her acting career in the late 1960s, appearing in various television shows and films.}
                \newline
                [The end of the biography]
                \newline \newline
                [The start of the list of checked facts]
                \newline
                \textcolor{anscolor}{[Marianne McAndrew is an American. (False); Marianne McAndrew is an actress. (True); Marianne McAndrew is a singer. (False); Marianne McAndrew was born on November 21, 1942. (False); Marianne McAndrew was born in Cleveland, Ohio. (False); She began her acting career in the late 1960s. (True); She has appeared in various television shows. (True); She has appeared in various films. (True)]}
                \newline
                [The end of the list of checked facts]
                \newline \newline
                [The start of the ideal output]
                \newline
                \textcolor{labelcolor}{[Marianne McAndrew (True); is (True); an (True); American (False); actress (True); and (True); singer (False); , (True); born (True); on (True); November 21, 1942 (False); , (True); in (True); Cleveland, Ohio (False); . (True); She (True); began (True); her (True); acting career (True); in (True); the late 1960s (True); , (True); appearing (True); in (True); various (True); television shows (True); and (True); films (True); . (True)]}
                \newline
                [The end of the ideal output]
				\newline \newline
                \textbf{[Example 2]}
                \newline
                [The start of the biography]
                \newline
                \textcolor{titlecolor}{Doug Sheehan is an American actor who was born on April 27, 1949, in Santa Monica, California. He is best known for his roles in soap operas, including his portrayal of Joe Kelly on ``General Hospital'' and Ben Gibson on ``Knots Landing.''}
                \newline
                [The end of the biography]
                \newline \newline
                [The start of the list of checked facts]
                \newline
                \textcolor{anscolor}{[Doug Sheehan is an American. (True); Doug Sheehan is an actor. (True); Doug Sheehan was born on April 27, 1949. (True); Doug Sheehan was born in Santa Monica, California. (False); He is best known for his roles in soap operas. (True); He portrayed Joe Kelly. (True); Joe Kelly was in General Hospital. (True); General Hospital is a soap opera. (True); He portrayed Ben Gibson. (True); Ben Gibson was in Knots Landing. (True); Knots Landing is a soap opera. (True)]}
                \newline
                [The end of the list of checked facts]
                \newline \newline
                [The start of the ideal output]
                \newline
                \textcolor{labelcolor}{[Doug Sheehan (True); is (True); an (True); American (True); actor (True); who (True); was born (True); on (True); April 27, 1949 (True); in (True); Santa Monica, California (False); . (True); He (True); is (True); best known (True); for (True); his roles in soap operas (True); , (True); including (True); in (True); his portrayal (True); of (True); Joe Kelly (True); on (True); ``General Hospital'' (True); and (True); Ben Gibson (True); on (True); ``Knots Landing.'' (True)]}
                \newline
                [The end of the ideal output]
				\newline \newline
				\textbf{User prompt}
				\newline
				\newline
				[The start of the biography]
				\newline
				\textcolor{magenta}{\texttt{\{BIOGRAPHY\}}}
				\newline
				[The ebd of the biography]
				\newline \newline
				[The start of the list of checked facts]
				\newline
				\textcolor{magenta}{\texttt{\{LIST OF CHECKED FACTS\}}}
				\newline
				[The end of the list of checked facts]
			};
		\end{tikzpicture}
        \caption{GPT-4o prompt for labeling hallucinated entities.}\label{tb:gpt-4-prompt}
	\end{center}
\vspace{-0cm}
\end{table*}
% \section{Full Experiment Results}
% \begin{table*}[th]
    \centering
    \small
    \caption{Classification Results}
    \begin{tabular}{lcccc}
        \toprule
        \textbf{Method} & \textbf{Accuracy} & \textbf{Precision} & \textbf{Recall} & \textbf{F1-score} \\
        \midrule
        \multicolumn{5}{c}{\textbf{Zero Shot}} \\
                Zero-shot E-eyes & 0.26 & 0.26 & 0.27 & 0.26 \\
        Zero-shot CARM & 0.24 & 0.24 & 0.24 & 0.24 \\
                Zero-shot SVM & 0.27 & 0.28 & 0.28 & 0.27 \\
        Zero-shot CNN & 0.23 & 0.24 & 0.23 & 0.23 \\
        Zero-shot RNN & 0.26 & 0.26 & 0.26 & 0.26 \\
DeepSeek-0shot & 0.54 & 0.61 & 0.54 & 0.52 \\
DeepSeek-0shot-COT & 0.33 & 0.24 & 0.33 & 0.23 \\
DeepSeek-0shot-Knowledge & 0.45 & 0.46 & 0.45 & 0.44 \\
Gemma2-0shot & 0.35 & 0.22 & 0.38 & 0.27 \\
Gemma2-0shot-COT & 0.36 & 0.22 & 0.36 & 0.27 \\
Gemma2-0shot-Knowledge & 0.32 & 0.18 & 0.34 & 0.20 \\
GPT-4o-mini-0shot & 0.48 & 0.53 & 0.48 & 0.41 \\
GPT-4o-mini-0shot-COT & 0.33 & 0.50 & 0.33 & 0.38 \\
GPT-4o-mini-0shot-Knowledge & 0.49 & 0.31 & 0.49 & 0.36 \\
GPT-4o-0shot & 0.62 & 0.62 & 0.47 & 0.42 \\
GPT-4o-0shot-COT & 0.29 & 0.45 & 0.29 & 0.21 \\
GPT-4o-0shot-Knowledge & 0.44 & 0.52 & 0.44 & 0.39 \\
LLaMA-0shot & 0.32 & 0.25 & 0.32 & 0.24 \\
LLaMA-0shot-COT & 0.12 & 0.25 & 0.12 & 0.09 \\
LLaMA-0shot-Knowledge & 0.32 & 0.25 & 0.32 & 0.28 \\
Mistral-0shot & 0.19 & 0.23 & 0.19 & 0.10 \\
Mistral-0shot-Knowledge & 0.21 & 0.40 & 0.21 & 0.11 \\
        \midrule
        \multicolumn{5}{c}{\textbf{4 Shot}} \\
GPT-4o-mini-4shot & 0.58 & 0.59 & 0.58 & 0.53 \\
GPT-4o-mini-4shot-COT & 0.57 & 0.53 & 0.57 & 0.50 \\
GPT-4o-mini-4shot-Knowledge & 0.56 & 0.51 & 0.56 & 0.47 \\
GPT-4o-4shot & 0.77 & 0.84 & 0.77 & 0.73 \\
GPT-4o-4shot-COT & 0.63 & 0.76 & 0.63 & 0.53 \\
GPT-4o-4shot-Knowledge & 0.72 & 0.82 & 0.71 & 0.66 \\
LLaMA-4shot & 0.29 & 0.24 & 0.29 & 0.21 \\
LLaMA-4shot-COT & 0.20 & 0.30 & 0.20 & 0.13 \\
LLaMA-4shot-Knowledge & 0.15 & 0.23 & 0.13 & 0.13 \\
Mistral-4shot & 0.02 & 0.02 & 0.02 & 0.02 \\
Mistral-4shot-Knowledge & 0.21 & 0.27 & 0.21 & 0.20 \\
        \midrule
        
        \multicolumn{5}{c}{\textbf{Suprevised}} \\
        SVM & 0.94 & 0.92 & 0.91 & 0.91 \\
        CNN & 0.98 & 0.98 & 0.97 & 0.97 \\
        RNN & 0.99 & 0.99 & 0.99 & 0.99 \\
        % \midrule
        % \multicolumn{5}{c}{\textbf{Conventional Wi-Fi-based Human Activity Recognition Systems}} \\
        E-eyes & 1.00 & 1.00 & 1.00 & 1.00 \\
        CARM & 0.98 & 0.98 & 0.98 & 0.98 \\
\midrule
 \multicolumn{5}{c}{\textbf{Vision Models}} \\
           Zero-shot SVM & 0.26 & 0.25 & 0.25 & 0.25 \\
        Zero-shot CNN & 0.26 & 0.25 & 0.26 & 0.26 \\
        Zero-shot RNN & 0.28 & 0.28 & 0.29 & 0.28 \\
        SVM & 0.99 & 0.99 & 0.99 & 0.99 \\
        CNN & 0.98 & 0.99 & 0.98 & 0.98 \\
        RNN & 0.98 & 0.99 & 0.98 & 0.98 \\
GPT-4o-mini-Vision & 0.84 & 0.85 & 0.84 & 0.84 \\
GPT-4o-mini-Vision-COT & 0.90 & 0.91 & 0.90 & 0.90 \\
GPT-4o-Vision & 0.74 & 0.82 & 0.74 & 0.73 \\
GPT-4o-Vision-COT & 0.70 & 0.83 & 0.70 & 0.68 \\
LLaMA-Vision & 0.20 & 0.23 & 0.20 & 0.09 \\
LLaMA-Vision-Knowledge & 0.22 & 0.05 & 0.22 & 0.08 \\

        \bottomrule
    \end{tabular}
    \label{full}
\end{table*}




\end{document}


\end{document}
