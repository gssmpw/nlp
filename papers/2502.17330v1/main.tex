\documentclass[runningheads]{llncs}
\usepackage[T1]{fontenc}
\def\Snospace~{\S{}}
\usepackage{nicefrac}
\usepackage{siunitx}
\usepackage{listings}
\usepackage{array,framed}
\usepackage{
  color,
  float,
  epsfig,
  wrapfig,
  graphics,
  graphicx
}
\usepackage{textcomp,amssymb}
\usepackage{setspace}
\usepackage{amsfonts}
\usepackage{enumerate}
\usepackage{enumitem}
\usepackage[compatible]{algpseudocode}
\usepackage{graphics}
\usepackage{subfig}
\usepackage{xparse} 
\usepackage{xspace}
\usepackage{multirow}
\usepackage{hyperref}
\usepackage{xfrac}
\usepackage{tabularx}
\usepackage{flushend}
\usepackage{mathptmx,avant}
 \usepackage{
  tikz,
  pgfplots,
  pgfplotstable
}
\usepackage{hyperref}

\usetikzlibrary{
  shapes.geometric,
  arrows,
  external,
  pgfplots.groupplots,
  matrix
}
\usepackage{amsmath}

\renewcommand{\sectionautorefname}{Section}
\renewcommand{\subsectionautorefname}{Subsection}

\newcommand{\Mod}[1]{\ (\mathrm{mod}\ #1)}

\pgfplotsset{compat=1.9}

\newcommand{\kms}{km\,s$^{-1}$}
\newcommand{\msun}{$M_\odot}
\newcommand{\sys}{{\sc VS-TEE}\xspace}
\newcommand{\circled[1]}{\tikz[baseline=(char.base)]{\node[font=\sffamily,
shape=circle,draw,inner sep=0.5pt,color=white,fill=black] (char) {#1};}}

\DeclareGraphicsExtensions{
    .png,.PNG,
    .pdf,.PDF,
    .jpg,.mps,.jpeg,.jbig2,.jb2,.JPG,.JPEG,.JBIG2,.JB2}

\usepackage{Settings/my_commands}
\usepackage{Settings/listing_algorithm}
\usepackage{caption}

\captionsetup[table]{position=bottom}

\usepackage{svg}
\usepackage{soul}
\setstcolor{red}
\usepackage{colortbl}
\usepackage{multicol}
\usepackage{arydshln}
\usepackage{verbatim}
\usepackage{fancyvrb}
\usepackage{fancyhdr}

\newcommand{\andrea}[1]{{\bf \textcolor{green}{Andrea: #1}}}
\newcommand{\davide}[1]{}
\newcommand{\ones}{\mathbf 1}
\newcommand{\reals}{{\mbox{\bf R}}}
\newcommand{\integers}{{\mbox{\bf Z}}}
\newcommand{\symm}{{\mbox{\bf S}}}  % symmetric matrices

\newcommand{\nullspace}{{\mathcal N}}
\newcommand{\range}{{\mathcal R}}
\newcommand{\Rank}{\mathop{\bf Rank}}
%\newcommand{\Tr}{\mathop{\bf Tr}}
\newcommand{\diag}{\mathop{\bf diag}}
\newcommand{\card}{\mathop{\bf card}}
\newcommand{\rank}{\mathop{\bf rank}}
\newcommand{\conv}{\mathop{\bf conv}}
\newcommand{\prox}{\mathbf{prox}}

\newcommand{\Expect}{\mathop{\bf E{}}}
\newcommand{\var}{\mathop{\bf var{}}}
\newcommand{\Prob}{\mathop{\bf Prob}}
\newcommand{\Co}{{\mathop {\bf Co}}} % convex hull
\newcommand{\dist}{\mathop{\bf dist{}}}
%\newcommand{\argmin}{\mathop{\rm argmin}}
%\newcommand{\argmax}{\mathop{\rm argmax}}
\newcommand{\epi}{\mathop{\bf epi}} % epigraph
\newcommand{\Vol}{\mathop{\bf vol}}
\newcommand{\dom}{\mathop{\bf dom}} % domain
\newcommand{\intr}{\mathop{\bf int}}
%\newcommand{\sign}{\mathop{\bf sign}}

\newcommand{\cf}{{\it cf.}}
\newcommand{\eg}{{\it e.g.}}
\newcommand{\ie}{{\it i.e.}}
\newcommand{\etc}{{\it etc.}}

\newcommand{\todo}{{\bf TODO}}

\newcommand{\bone}{\boldsymbol{1}}
\newcommand{\balpha}{\boldsymbol{\alpha}}
\newcommand{\bbeta}{\boldsymbol{\beta}}
\newcommand{\bdelta}{\boldsymbol{\delta}}
\newcommand{\bepsilon}{\boldsymbol{\epsilon}}
\newcommand{\blambda}{\boldsymbol{\lambda}}
\newcommand{\bomega}{\boldsymbol{\omega}}
\newcommand{\bpi}{\boldsymbol{\pi}}
\newcommand{\bnu}{\boldsymbol{\nu}}
\newcommand{\bphi}{\boldsymbol{\phi}}
\newcommand{\bvphi}{\boldsymbol{\varphi}}
\newcommand{\bpsi}{\boldsymbol{\psi}}
\newcommand{\bsigma}{\boldsymbol{\sigma}}
\newcommand{\btheta}{\boldsymbol{\theta}}
\newcommand{\bzeta}{\boldsymbol{\zeta}}
\newcommand{\bxi}{\boldsymbol{\xi}}
\newcommand{\ba}{\boldsymbol{a}}
\newcommand{\bb}{\boldsymbol{b}}
\newcommand{\bc}{\boldsymbol{c}}
\newcommand{\bd}{\boldsymbol{d}}
\newcommand{\be}{\boldsymbol{e}}
\newcommand{\boldf}{\boldsymbol{f}}
\newcommand{\bg}{\boldsymbol{g}}
\newcommand{\bh}{\boldsymbol{h}}
\newcommand{\bi}{\boldsymbol{i}}
\newcommand{\bj}{\boldsymbol{j}}
\newcommand{\bk}{\boldsymbol{k}}
\newcommand{\bell}{\boldsymbol{\ell}}
\newcommand{\bp}{\boldsymbol{p}}
\newcommand{\br}{\boldsymbol{r}}
\newcommand{\bs}{\boldsymbol{s}}
\newcommand{\bt}{\boldsymbol{t}}
\newcommand{\bu}{\boldsymbol{u}}
\newcommand{\bv}{\boldsymbol{v}}
\newcommand{\bw}{\boldsymbol{w}}
\newcommand{\bx}{{\boldsymbol{x}}}
\newcommand{\by}{\boldsymbol{y}}
\newcommand{\bz}{\boldsymbol{z}}
\newcommand{\bA}{\boldsymbol{A}}
\newcommand{\bB}{\boldsymbol{B}}
\newcommand{\bC}{\boldsymbol{C}}
\newcommand{\bD}{\boldsymbol{D}}
\newcommand{\bE}{\boldsymbol{E}}
\newcommand{\bF}{\boldsymbol{F}}
\newcommand{\bG}{\boldsymbol{G}}
\newcommand{\bH}{\boldsymbol{H}}
\newcommand{\bI}{\boldsymbol{I}}
\newcommand{\bJ}{\boldsymbol{J}}
\newcommand{\bL}{\boldsymbol{L}}
\newcommand{\bM}{\boldsymbol{M}}
\newcommand{\bP}{\boldsymbol{P}}
\newcommand{\bQ}{\boldsymbol{Q}}
\newcommand{\bR}{\boldsymbol{R}}
\newcommand{\bS}{\boldsymbol{S}}
\newcommand{\bT}{\boldsymbol{T}}
\newcommand{\bU}{\boldsymbol{U}}
\newcommand{\bV}{\boldsymbol{V}}
\newcommand{\bW}{\boldsymbol{W}}
\newcommand{\bX}{\boldsymbol{X}}
\newcommand{\bY}{\boldsymbol{Y}}
\newcommand{\bZ}{\boldsymbol{Z}}

% new theorems
% \newtheorem{theorem}{Theorem}
%\newtheorem*{proof}{Proof}

% usepackages
\usepackage{amsmath}
\usepackage{amsfonts}
\usepackage{textcomp} % for \textlangle and \textrangle macros
\newcommand{\qdist}[1]{\ifmmode\langle#1\rangle\else\textlangle#1\textrangle\fi}
\usepackage{xcolor}
\usepackage{algorithm} % for algorithms
\usepackage{algpseudocode} % for pseudocode
\usepackage{comment} % for large comments
\usepackage{bbm}
\usepackage{dsfont}
\usepackage{subfigure}
\usepackage{bm}
\usepackage{booktabs} % For better table lines
\usepackage{array} % For better column formatting
%\usepackage{appendix}
%\usepackage[english]{babel}
%\usepackage{amsthm}
\usepackage{graphicx} % for graphs




\usepackage{Settings/my_commands}
\usepackage{Settings/listing_algorithm}
\usepackage{caption}
\captionsetup[table]{position=bottom}

\begin{document}

\title{Unveiling ECC Vulnerabilities: LSTM Networks for Operation Recognition in Side-Channel Attacks}

\author{Alberto Battistello\inst{1} \and Guido Bertoni\inst{1} \and Michele Corrias\inst{1} \and Lorenzo Nava\inst{1} \and Davide Rusconi\inst{2} \and Matteo Zoia\inst{2} \and Fabio Pierazzi\inst{3} \and Andrea Lanzi\inst{2}}

\authorrunning{Battistello et al.}

\institute{Security Pattern, Milan, Italy \and
University of Milan, Milan, Italy \and
King's College London, London, United Kingdom}

\maketitle 

\begin{abstract}
We propose a novel approach for performing side-channel attacks on elliptic curve cryptography. Unlike previous approaches and inspired by the ``activity detection'' literature, we adopt a long-short-term memory (LSTM) neural network to analyze a power trace and identify patterns of operation in the scalar multiplication algorithm performed during an ECDSA signature,  that allows us to recover bits of the ephemeral key, and thus retrieve the signer's private key. Our approach is based on the fact that modular reductions are conditionally performed by micro-ecc and depend on key bits. 

We evaluated the feasibility and reproducibility of our attack through experiments in both simulated and real implementations. We demonstrate the effectiveness of our attack by implementing it on a real target device, an STM32F415 with the micro-ecc library, and successfully compromise it. 
Furthermore, we show that current countermeasures, specifically the coordinate randomization technique, are not sufficient to protect against side channels. Finally, we suggest other approaches that may be implemented to thwart our attack. 

\keywords{Hardware security \and Side-channel attacks \and Elliptic curve cryptography \and Key recovery \and Deep learning}
\end{abstract}

\section{Introduction}

Within the security domain, Machine Learning (ML) methods have been utilized to address issues such as email spam filtering \cite{tretyakov2004machine,crawford2015survey}, intrusion detection systems \cite{tsai2009intrusion,buczak2015survey}, and facial recognition~\cite{taigman2014deepface,oravec2014feature}. Although these uses highlight the effectiveness of ML in protective security strategies, its influence also spans offensive actions like side-channel attacks (SCAs). In SCAs, attackers examine physical attributes of devices, such as timing variations, power usage, and electromagnetic signals during computations, to uncover confidential information, emphasizing the dual use nature of ML in the security field \cite{kocher1996timing,kocher1999differential,quisquater2001electromagnetic}.

This duality is particularly prominent in cryptographic applications, such as the Elliptic Curve Digital Signature Algorithm (ECDSA), where physical attributes can reveal critical secret information, presenting a significant vulnerability. Our research focuses on side channel attacks (SCA) against ECDSA implementations, particularly those \textit{ protected with the coordinate randomization mechanism}, as discussed in~\cite{rivain2011fast}. Based on this foundation, our study proposes an innovative technique using a long- and short-term memory (LSTM) network architecture~\cite{yu2019review} to identify the execution patterns of operations in the scalar multiplication process, an essential part of ECDSA. By examining these operational patterns, our method enables the extraction of ephemeral key bits, potentially leading to the compromise of the signer's full private key. Our study explores the real-world implementation of these insights in elliptic curve cryptography, specifically using the micro-ecc library~\cite{microecc}. This is a widely used open source library, especially common in Internet of Things (IoT) applications.

Luo et al.~\cite{luo2018effective} illustrated the possibility of attacking the ECDSA by using collision attacks, which leverage the detection of specific operational patterns in the traces of power consumption. In these attacks, "collision" signifies the recurrence of an identical value at various stages of the algorithm, identifiable through pattern analysis and correlation methods. Our research improves the attack strategy discussed by~\cite{luo2018effective}, presenting a more efficient SCA approach to exploit vulnerabilities in micro-ecc. By employing a machine learning-based approach and lattice techniques, our method concepts the detection of key operations and recovery of the signer's key, even with a few known ephemeral key bits from signature processes. Additionally, we show that our technique does not depend on electromagnetic (EM) analysis by effectively extracting key information from the noisy power consumption traces of advanced hardware such as the ARM Cortex-M4 processor in the STM32F4 series. This discovery represents a considerable advancement in side-channel attack strategies. Our results also question the efficacy of existing countermeasures against collision attacks, prompting us to suggest more robust alternatives that offer better protection against these and similar vulnerabilities. 

More in detail our framework stands out by employing Long-Short-Term Memory (LSTM) networks, which are well regarded for their proficiency in tasks analogous to human activity recognition. In particular, this approach enhances our analysis from basic side-channel attack execution to "operation recognition," akin to the methods used in detecting human activities. Using this strategy, we can analyze patterns in cryptographic operations in a detailed way, which is essential to detect hidden vulnerabilities. The LSTM network was specifically chosen for its aptitude in deciphering the sequential and temporal dynamics inherent in our attack's operational recognition phase. This decision was informed by the nature of our problem, which aligns more with operational recognition, a concept akin to human activity recognition, than with classic data classification. This delineation underlines the unsuitability of Convolutional Neural Networks (CNNs) for our analysis, as they fall short in capturing the essential temporal context. The LSTM's design, known for its proficiency in processing time-dependent data, perfectly aligns with the requirements of identifying and analyzing execution operational patterns, reinforcing our methodological choices with the goals of our side-channel attack strategy.

To evaluate the practicality of our attack technique, we employed the secp160r1 curve in a simulated setup, executing the attack using an LSTM-based neural network to determine the possibility of extracting the signer's private key. We conducted a successful experiment on the STM32F415 microcontroller~\cite{stmicroelectronics}, selected for its incorporation of the micro-ecc library. This library implements co-Z algorithms and includes a \textit{coordinate randomization countermeasure} to defend against known side-channel attacks (SCAs). To obtain the necessary side-channel traces for our investigation, we utilized the ChipWhisperer~\cite{chipwhisperer} platform, which allowed us to collect power consumption data as leakage vector. Using the same LSTM-based Neural Network (NN) model that was effective in our simulated settings, we modified and trained it specifically for this microcontroller. This was a crucial step in demonstrating that our attack strategy, which was refined through simulations and theoretical models, is not only conceptually sound but also practically applicable to actual hardware. This paper offers the following contributions:

\begin{itemize}

\item Developed an LSTM-based methodology for identifying operations execution patterns in scalar multiplication algorithms, enabling the extraction of ephemeral key bits in ECC.

\item Demonstrated the practicality of this approach through real-world testing on the STM32F415 device, utilizing the micro-ecc library and the secp160r1 curve with the \textit{protected ccoordinate randomization mechanism}, highlighting its adaptability to various types of curves.

\item Illustrated the limitations of existing countermeasures against side-channel attacks while providing an in-depth analysis of potential improvements in security protocols.

\end{itemize}

\section{Background}
\label{sec:background}

This Section provides an introduction to the basic cryptographic concepts necessary to understand the rest of this work. The following Section explores the specific cryptosystem solutions for elliptic curves that are implemented in the target library.
Consider an elliptic curve $\curve_\field$ defined over a field $\field$ of characteristic $\neq 2,3$ according to the short Weierstrass equation as:
\begin{equation}
\label{eq:ec_weierstrass}
    \mathcal{E}: y^2 = x^3 + ax + b \quad a, b \in \mathbb{K}
\end{equation}
with $a,b \in \field$ such that $4a^3+27b^2 \neq 0$. Given the curve, the set of points that satisfy its equation $(x, y) \in \mathbb{K}^2$, augmented with a particular point $\infinitePoint$ called \emph{point at infinity}, forms a \emph{additive abelian group} \cite{de1997elliptic}. 

\subsubsection{Scalar Multiplication}
\label{scalar}
A crucial operation on which protocols based on ECC are based is \emph{scalar multiplication} \cite{lopez2000overview}. Given an integer $k \in \field$ and a point $P$ on $\mathcal{E}$, the scalar multiplication $k$ with $P$ is defined as:
\begin{equation}
    \label{eq:scalarMul}
    kP = \underbrace{P + P + \dots + P}_{k \textsc{ times}}
\end{equation}
In this work we will focus on the scalar multiplication algorithm implemented in micro-ecc, the Montgomery Ladder
implementation with Jacobian co-Z coordinates [10], with coordinates randomization.

\subsubsection{Jacobian co-\textit{Z} representation}
\label{coordinates}
The Jacobian co-\textit{Z} representation
is a projective representation of the points of an elliptic curve that uses three coordinates.

In this system the input points of the operations are represented by triplets sharing the same Z coordinate. This representation was first suggested by Meloni~\cite{meloni2007new}, and then adapted with different trade-offs by Rivain~\cite{rivain2011fast}, and implemented in micro-ecc~\cite{microecc}.

\begin{algorithm}[caption={Montgomery ladder with \textit{(X, Y)}-only co-\textit{Z} addition}, label={alg:montgomeryLadderCoZ}]
input: $P \in \curve(\finiteFieldP)$, $k = (k_{n-1},\dots,k_1,k_0)_2 \in \naturals$/+\text{ with }+/$k_{n-1} = 1$
output: $kP$
begin
    $(R_1,R_0) \gets \texttt{XYCZ-IDBL}(P)$
    for $i \gets n - 2$ to $1$ do
        $b \gets k_i$
        $(R_{1-b},R_b) \gets \texttt{XYCZ-ADDC}(R_b, R_{1-b})$
        $(R_b,R_{1-b}) \gets \texttt{XYCZ-ADD}(R_{1-b}, R_b)$
    end
    $b \gets k_0$
    $(R_{1-b},R_b) \gets \texttt{XYCZ-ADDC}(R_b, R_{1-b})$
    $\lambda \gets \texttt{FinalInvZ}(R_0, R_1, P, b)$
    $(R_b,R_{1-b}) \gets \texttt{XYCZ-ADD}(R_{1-b}, R_b)$
    return $(X_0\lambda^2, Y_0\lambda^3)$
end
\end{algorithm}

\subsubsection{Montgomery ladder with \textit{(X, Y)}-only co-\textit{Z} addition}
\label{para:mlcoz}
The \emph{Montgomery ladder with (X, Y)-only co-Z addition}~\cite{rivain2011fast} \autoref{alg:montgomeryLadderCoZ} is an algorithm employing on Jacobian co-\textit{Z} coordinates 
used in the target ECC implementation~\cite{microecc}
to perform the scalar multiplication required for ECDSA. It performs an initial doubling (in micro-ecc it uses the Jacobian Doubling algorithm), then starts a loop on the bits of the scalar, where for each loop cycle a XYcZ-ADDC  followed by a XYcZ-ADD are executed.\\\\

\begin{algorithm}[caption={XYcZ-ADDC}, label={alg:XYCZ-ADDC}]
input: $(X_1,Y_1)$ and $(X_2,Y_2)$ s.t. $P \equiv (X_1 : Y_1 : Z)$ and $Q \equiv (X_2 : Y_2 : Z)$ for some $Z \in \F_q, P,Q \in \mathcal{E}(\F_q)$
output: $(X_3, Y_3)$ and $(X_3',Y_3')$ s.t. $P + Q \equiv (X_3 : Y_3 : Z_3)$ and $P - Q \equiv (X_3' : Y_3' : Z_3)$ for some $Z_3 \in \F_q$
begin
    $A \gets (X_2 - X_1)^2$
    $B \gets X_1A$
    $C \gets X_2A$
    $D \gets (Y_2 - Y_1)^2$
    $F \gets (Y_1 + Y_2)^2$
    $E \gets Y_1(C - B)$
    $X_3 \gets D - (B + C)$
    $Y_3 \gets (Y_2 - Y_1)(B - X_3) - E$
    $X_3' \gets F - (B + C)$
    $Y_3' \gets (Y_1 + Y_2)(X_3'-B) - E$
    return $((X_3,Y_3),(X_3',Y_3'))$
end
\end{algorithm}
\clearpage
\begin{algorithm}[caption={XYcZ-ADD}, label={alg:XYCZ-ADD}]
input: $(X_1,Y_1)$ and $(X_2,Y_2)$ s.t. $P \equiv (X_1 : Y_1 : Z)$ and $Q \equiv (X_2 : Y_2 : Z)$ for some $Z \in \F_q, P,Q \in \mathcal{E}(\F_q)$
output: $(X_3, Y_3)$ and $(X_1',Y_1')$ s.t. $P \equiv (X_1' : Y_1' : Z_3)$ and $P + Q \equiv (X_3 : Y_3 : Z_3)$ for some $Z_3 \in \F_q$
begin
    $A \gets (X_2 - X_1)^2$
    $B \gets X_1A$
    $C \gets X_2A$
    $D \gets (Y_2 - Y_1)^2$
    $E \gets Y_1(C - B)$
    $X_3 \gets D - (B + C)$
    $Y_3 \gets (Y_2 - Y_1)(B - X_3) - E$
    $X_1' \gets B$
    $Y_1' \gets E$
    return $((X_3,Y_3),(X_1',Y_1'))$
end
\end{algorithm}




\section{Related Work}

In the physical security domain, the adoption of Neural Networks (NNs) has marked a transformative phase, particularly in enhancing Side-Channel Attack (SCA) strategies. These NN-empowered SCAs have surpassed traditional approaches by yielding more potent results with reduced observational demands \cite{maghrebi2016breaking,picek2017side,wang2014learning,wu2021best,perin2021keep,nascimento2017applying,weissbart2019one,cagli2017convolutional,zaid2020methodology,carbone2019deep,picek2021sok}. Research initiatives \cite{maghrebi2019deep,ramezanpour2020scaul,benadjila2020deep}, have taken on sophisticated Deep Learning techniques to exploit side channel traces in examining symmetric algorithms. Focusing on the ECC Double-And-Add-Always algorithm implemented on FPGA platforms, Mukhtar et al.\cite{mukhtar2018machine} applied classification methods to reveal secret key bits of the ECC. In a similar vein, Weissbart et al.\cite{weissbart2019one} orchestrated a power analysis attack on the Edwards-curve Digital Signature Algorithm (EdDSA)\cite{bernstein2012high}, revealing the superior capabilities of CNN over classical side-channel techniques like Template Attacks~\cite{chari2002template}. Weissbart et al.\cite{weissbart2020systematic} further expanded their investigation, evaluating additional protected targets and highlighting the efficacy of Deep Learning, particularly CNNs, in breaching protected implementations of scalar multiplication on Curve25519\cite{bernstein2006curve25519}.

Perin et al.\cite{perin2021keep} proposed a groundbreaking Deep Learning-based iterative framework for unsupervised horizontal attacks, aimed at refining the accuracy of single-trace attacks and reducing errors in the decryption of private keys, particularly in protected ECC implementations. This effort, similar to Nascimento et al.\cite{nascimento2017applying}, exploited vulnerabilities within the $\mu$NaCl library's \texttt{cswap} function~\cite{NACLlib}, showcasing the ongoing evolution of NN-enabled SCA methodologies in enhancing cryptographic security. In a recent development, Staib et al.\cite{staib2023deep} sought to advance collision side channel attacks through deep learning, demonstrating a neural network's superiority in collision detection over traditional methods on a public dataset\cite{luo2018effective,clavier2011improved,bauer2015horizontal}. 

Our research uses long- and short-term memory (LSTM) networks, which are renowned for their effectiveness in tasks that resemble recognition of human activity. This innovative approach not only goes beyond the traditional scope of side-channel attacks (SCA) but reframes the problem as an "operation recognition" task. By adopting methodologies similar to those in human activity detection, we enable sophisticated pattern recognition within cryptographic operations. This unique framework is pivotal for uncovering hidden vulnerabilities, highlighting the intricate interplay between cryptographic processes and exploitable weaknesses.

\section{Attack Preconditions}
\label{sec:attackz}

Our objective for this attack is the \textit{micro-ecc} library, which is an open-source implementation of ECDH and ECDSA tailored for 8/32/64 bit processors \cite{microecc}. Although this library supports multiple elliptic curves, our study specifically targets \textit{secp160r1}. This choice is driven by the widespread use of the curve in resource-limited environments such as IoT, where both security and efficiency are paramount, making it an important focus of our investigation. Our target utilizes the Jacobian Co-\textit{Z} representation along with the Montgomery ladder algorithm (\autoref{alg:montgomeryLadderCoZ}), for cryptographic processes. Furthermore, the library incorporates a countermeasure of coordinate randomization, which produces a random $z \in \finiteFieldQ$ before each execution. This random $z$ is then applied to the representation of points before entering the Montgomery ladder. This initial randomization ensures that the values computed during the ladder are not correlated with the original point, making modular reductions unpredictable based on guesses of bits from the ephemeral key. However, we have discovered several issues that make this implementation susceptible to side-channel attacks, which will be discussed in the following.

\paragraph{Consistent Timing of the Implementation.}
\label{subsec:ct}
The initial problem arises because the micro-ecc code performs a conditional \emph{modular reduction}, which is triggered by an over / overflow occurring after an addition or subtraction operation. This implies that the algorithm does not run in constant time, thereby leaking information during execution. However, exploiting this vulnerability (e.g., intercepting the over/underflow operations) is made difficult by the randomization of the coordinates. This countermeasure effectively nullifies any predictive attempts without prior knowledge of the randomness used in the countermeasure, thus preventing exploitation of the issue \textit{ as is}.

\paragraph{Repeated Operations Depending on Key Bit.}The second vulnerability in the micro-ecc implementation arises from the Montgomery ladder (\autoref{alg:montgomeryLadderCoZ}) performing certain calculations twice, based on the ephemeral key bit. Specifically, during the $(n-1)$\textsuperscript{th} iteration of the Montgomery ladder, the value $B - X_3$ is calculated. Subsequently, both $B$ and $X_3$ are returned as the x-coordinates of the two output points ($X_1'$ and $X_3$, respectively). In the next $n$\textsuperscript{th} iteration of the Montgomery ladder, depending on the $n$\textsuperscript{th} bit of the scalar, the implementation either recalculates $B - X_3$ or computes $X_3 - B$. This means that whenever two consecutive bits of the scalar are identical, the $B - X_3$ operation is performed twice. 

\paragraph{Attacking the First Bit.} The final issue identified in the target system is a leak resulting from the interaction between the Jacobian doubling operation, performed before the Montgomery ladder, and the addition operation executed during it. Specifically, the value of the first key bit determines whether the same subtraction is performed or its inverse is calculated. Consequently, detecting a collision between the subtraction within the Jacobian doubling and the first iteration of the Montgomery ladder reveals the first bit of the ephemeral key.
\\\\
Our attack exploits all three identified issues to leak the initial bits of the ephemeral key and subsequently infer the private key. Specifically, we used a timing side channel to detect whether an overflow occurred during the addition operation, leveraging the first issue. Then, by distinguishing between overflowing and non-overflowing subtractions, we directly leak the first bit using the third issue. Additionally, we exploit the second vulnerability to infer the equality of the first bits. This information enables us to mount a lattice reduction attack, effectively extracting the private key used in the signatures, thereby compromising the system's security. To achieve this, we need to design a neural network capable of identifying the operating patterns based on the power trace.

\section{System Overview}
\label{sec:sys_ov}
To exploit the vulnerability described in Section \ref{sec:attackz}, we have developed a system consisting of five main components: the acquisition unit responsible for acquiring the power traces from the target device, the windowing algorithm is primarily designed to divide the power traces into smaller segments (e.g., windows), a machine learning model that categorizes the acquired traces, a post-processing unit that analyzes the output of the ML model to identify the presence or absence of modular operations within the traces, and a final component that deduces the scalar bits as discussed in Section \ref{sec:attackz} and extracts the key.

As illustrated in Figure~\ref{fig:sys_arch}, the process begins with recording raw power traces from the targeted device through an acquisition unit. These raw traces are then preprocessed and divided into multiple windows of uniform size (e.g, windowing algorithm). Each window is subsequently fed into the neural network, where our model determines whether the operation within the window is a short operation (SO), such as addition or subtraction, or a long operation (LO), such as multiplication and division. Based on the neural network's classification, the window vector is transformed into a binary vector, enabling us to identify the type of operation executed. For short operations, we also check for overflow. This generates a list of operations and details concerning modular reductions, which can be utilized, following the leaks in \autoref{sec:attackz}, to determine the values and relationships of the bits of the ephemeral key used in the attacked ECDSA round. With sufficient leakage information on the ephemeral key, we can proceed to recover the private key using a known lattice reduction technique, thus compromising the target and completing the attack.


\section{System Architecture}
\label{sec:sys}
\begin{figure*}[h]
    \centering\includegraphics[width=0.95\textwidth]{images/ecc-ml-schema.pdf}
    \caption{Architectural Overview}  \label{fig:sys_arch}
\end{figure*}

We will now describe the design of each component of our framework, except for the unit responsible for obtaining power consumption traces, as its design depends on the target device. Each component represents a part of our contribution and has been designed to target the cryptographic algorithm.

\subsection{Pre-Processing}

As described in \autoref{sec:sys_ov}, we employ a sliding window algorithm to segment the entire trace into smaller, fixed-size sequences. This method offers several advantages for our analysis and is particularly beneficial when used in conjunction with long-short-term memory (LSTM) networks, known for their effectiveness in tasks similar to recognizing human activities.

Firstly, the use of a sliding window enables us to focus on manageable chunks of data, which simplifies the computational process. By tuning the window size and the offset between adjacent windows, both of which are hyperparameters, we can optimize our analysis for both performance and accuracy. After rigorous testing and evaluation, we determined that a window size of 500 samples and an offset of 10 samples provide the best results. The chosen window size of 500 samples produces an optimal balance by ensuring that each segment contains sufficient detail to accurately classify the operations within it, while also maintaining computational efficiency. A larger window might capture more detail but would significantly increase computational overhead, potentially slowing down the analysis. In contrast, a smaller window might be more efficient, but could miss critical information necessary for accurate classification. The 10-sample offset ensures that we achieve a high-resolution analysis by overlapping windows, allowing for a more granular examination of the data. This overlap helps to ensure that no important transitions or details are missed between windows, enhancing the overall accuracy of our classification. Integrating the sliding-window approach with LSTM networks offers significant benefits. LSTMs are designed to capture temporal dependencies and trends within sequential data, which is crucial for accurate classification.

From a machine learning perspective, splitting the data into windows can significantly improve the efficacy of our LSTM model. By creating a larger number of training samples from the original dataset, we expose the LSTM model to a greater variety of patterns and transitions within the data. This increased dataset size allows the model to learn more effectively and generalize better. Moreover, by tuning the window size and offset, we can control the granularity of the input data, allowing the model to focus on the most relevant features and patterns. 

\subsection{Neural Network Architecture \& Parameters}\label{sec:nn}

In this study, we created a neural network framework aimed at effectively handling binary classification tasks, integrating both convolutional and long-short-term memory (LSTM) layers. This framework is intended to recognize intricate patterns and temporal relationships within the data, thus enhancing the model's predictive accuracy. Our framework is designed to balance computational efficiency with the capacity to extract and utilize significant features, leading to a reliable and robust model. The neural network architecture selected for our framework is a sequential model consisting of the following layers.

\begin{itemize}[noitemsep]
\item \textbf{Convolutional Layer:} This layer performs convolution operations on the input data to decrease its dimensions and transform its structure, thus increasing computational efficiency. The convolutional layers enhance the resilience to noise and scale changes in the input traces. In particular, this layer contains 64 feature maps, each with a 3x3 kernel, and applies the ReLU activation function to incorporate nonlinearity, allowing the network to capture intricate patterns.

\item \textbf{Pooling Layer:} Following the convolutional layer, this layer performs average pooling, reducing the dimensionality of the feature maps produced by the convolutional operations. The pooling layer uses a pooling size of 10, which helps in further down-sampling the input and retaining the most salient features.

\item \textbf{LSTM Layer:} This layer forms the core of our neural network and consists of 1000 internal units. LSTM (Long Short-Term Memory) networks are adept at capturing long-term dependencies and temporal correlations within sequential data. Using previous sample information, the LSTM layer is able to make context-sensitive evaluations, crucial for tasks involving temporal sequences.

\item \textbf{Dropout Layer:} To prevent overfitting and enhance the generalization capabilities of the network, this layer randomly drops a subset of its nodes during training. The dropout rate is set to 0.5, which means that 50\% of the nodes are temporarily ignored during each training iteration. This technique helps the network develop more robust features by mitigating the dependence on any specific subset of neurons.

\item \textbf{Dense Layer:} This fully connected layer comprises 1000 neurons and employs the ReLU activation function. The dense layer integrates information from the preceding layers and contributes to the network's ability to perform high-level abstractions and complex decision making.

\item \textbf{Output Layer:} The final layer of the network is responsible for the binary classification task. It consists of dense nodes that utilize the softmax activation function to produce probabilities for the binary classes, enabling the network to produce the final classification decision.

\end{itemize}

The LSTM network utilizes the binary cross-entropy loss function for training, with optimization done through stochastic gradient descent (SGD). Inspired by real-world applications in the literature \cite{zhang2022multi}, we chose a classic architecture that is typically used in recognition of human activity. In total, the network comprises 5,263,258 trainable parameters, allocated as follows: \textit{Convolutional Layer:} 256 parameters \textit{LSTM Layer:} 4,260,000 parameters (the main component of our model) \textit{Dense Layer:} 1,001,000 parameters \textit{Output Layer:} 2,002 parameters.

\subsection{Post-Processing to Identify Operations}

In order to transform the model output into a series of operations, a two-step method is utilized. The initial step involves aligning the classified operations within the algorithm. The model output already identifies whether the operations within the windows are short operations (SO) or not, setting the groundwork for further analysis. The second step differentiates between short operations with overflows and those without, which is critical for our attack. This differentiation is based on the observation that overflows cause a modular reduction, leading to an increased duration. Thus, by examining a continuous sequence of windows identified as short operations, we can estimate the duration of each individual operation in the sequence by analyzing the starting points of SOs and the subsequent operations.  Specifically, the procedure involves the following steps:

\begin{enumerate}
    \item \textbf{Classification Matching:} Use the output of the model to determine the particular operations that occur within the algorithm, depending on whether each window is classified as containing a short operation or not.
    \item \textbf{Duration Analysis:} For groups of consecutive windows marked as short operations, determine the estimated duration of each operation. An extended duration can suggest the occurrence of a modular reduction, which indicates an overflow. This distinction uses the pre-determined window length and step size to precisely measure operation durations.
\end{enumerate}

It is crucial to understand that the time taken for a subtraction operation involving modular reduction can differ between various target devices. Therefore, when implementing this post-processing step on a new target device, modifications must be made to consider the unique physical properties of that device. This adaptation guarantees a precise differentiation between operations in diverse hardware contexts. This dual-phase approach improves the ability to correctly detect and distinguish between brief operations with and without overflows, thereby enhancing the attack's overall efficiency.

\subsection{Collision Template \& Leaking the Ephemeral Key}
Following the identification operations process, it is necessary to deploy a collision attack template to retrieve the bits of the ephemeral key. This template can indicate the locations where collisions occur in the algorithm's operations, and from these collisions one can infer the positions and values of the ephemeral key bits. This phase relies on the specific implementation of the cipher algorithm and must be adapted if the algorithm is modified. To construct the collision template model and extract the ephemeral key bits, we started with a simulation of the target algorithm, as outlined in Section \ref{sec:eval}. This simulation gives us insight into the intermediate values during the computation and the sequence of executed instructions, noting the occurrences of overflows and underflows. Using these insights and the knowledge from \autoref{sec:attackz}, we identified specific operations that reveal ephemeral key bits, even with coordinate randomization countermeasures in effect.

In~\autoref{tab:key_op01}, we present examples of the algorithm simulation that selected operations across different initial bits and multiple keys within the coordinate randomization defense framework. These operations include the Initial Double and the first iteration of the Montgomery loop, with ``\texttt{||}'' indicating the start of the loop. The operations are labeled as ``\texttt{A}'' for ``\emph{Add}, \texttt{S}'' for \emph{Sub}, and ``\texttt{M}'' for both \emph{Mul} and \emph{Sqr}, with a ``\texttt{+}'' next to an operation representing a modular reduction. 
A \emph{collision} is described as the occurrence of identical underflow behavior in two operations, underscoring their importance in our analysis.

\begin{table}[]
\small
\caption{Operations for the \texttt{initial\_double} and the first iteration of the Montgomery loop, for values of the 3 initial bits of the ephemeral key. Randomization countermeasure is active (i.e., \emph{z} is random). In red are highlighted operation 18 and 26, and the bit of the key that correlates with their collision.}\label{tab:key_op01}
\centering
\scalebox{0.7}{
\resizebox{\columnwidth}{!}{%
\begin{tabular}{r|l}
\emph{Key bits} & \emph{Operations} \\
\hline
% \rule{0pt}{3ex} 
position & \texttt{0 1 2 3 4 5 6 7 8 9 10\! 11\! 12\!\! 13\!\! 14\! 15\!\! 16\!\! 17\!\! {\color{red}18}\!\! 19\!\! 20\!\! 21\!\! 22\!\! 23\!\!\! 24\!\! 25\!\!\! {\color{red}26}\!\! 27\!\! 28\!\! 29 } \\
0b1{\color{red}1}0 & \texttt{M  M  M  M  M  M  M  M  M A }\space \texttt{A+}       \texttt{S }\space \texttt{M A }\space \texttt{A+}       \texttt{M S+}       \texttt{S+}       \texttt{\textcolor{red}{S+}}       \texttt{M S }\space \texttt{M  M  M  M} \texttt{||} \texttt{\textcolor{red}{S+}}       \texttt{M  M  M} \\
0b1{\color{red}0}0 & \texttt{M  M  M  M  M  M  M  M  M A }\space \texttt{A+}       \texttt{S }\space \texttt{M A+}       \texttt{A+}       \texttt{M S+}       \texttt{S }\space \texttt{\textcolor{red}{S }}\space \texttt{M S+}       \texttt{M  M  M  M} \texttt{||} \texttt{\textcolor{red}{S+}}       \texttt{M  M  M} \\
0b1{\color{red}0}0 & \texttt{M  M  M  M  M  M  M  M  M A+}       \texttt{A+}       \texttt{S+}       \texttt{M A }\space \texttt{A+}       \texttt{M S }\space \texttt{S+}       \texttt{\textcolor{red}{S }}\space \texttt{M S+}       \texttt{M  M  M  M} \texttt{||} \texttt{\textcolor{red}{S+}}       \texttt{M  M  M} \\
0b1{\color{red}0}0 & \texttt{M  M  M  M  M  M  M  M  M A }\space \texttt{A }\space \texttt{S }\space \texttt{M A }\space \texttt{A }\space \texttt{M S }\space \texttt{S }\space \texttt{\textcolor{red}{S+}}       \texttt{M S+}       \texttt{M  M  M  M} \texttt{||} \texttt{\textcolor{red}{S }}\space \texttt{M  M  M} \\
0b1{\color{red}0}1 & \texttt{M  M  M  M  M  M  M  M  M A+}       \texttt{A+}       \texttt{S+}       \texttt{M A }\space \texttt{A+}       \texttt{M S }\space \texttt{S+}       \texttt{\textcolor{red}{S+}}       \texttt{M S }\space \texttt{M  M  M  M} \texttt{||} \texttt{\textcolor{red}{S }}\space \texttt{M  M  M} \\
0b1{\color{red}1}1 & \texttt{M  M  M  M  M  M  M  M  M A }\space \texttt{A }\space \texttt{S+}       \texttt{M A+}       \texttt{A+}       \texttt{M S }\space \texttt{S+}       \texttt{\textcolor{red}{S }}\space \texttt{M S+}       \texttt{M  M  M  M} \texttt{||} \texttt{\textcolor{red}{S }}\space \texttt{M  M  M} \\
0b1{\color{red}0}1 & \texttt{M  M  M  M  M  M  M  M  M A+}       \texttt{A+}       \texttt{S+}       \texttt{M A }\space \texttt{A+}       \texttt{M S }\space \texttt{S+}       \texttt{\textcolor{red}{S }}\space \texttt{M S }\space \texttt{M  M  M  M} \texttt{||} \texttt{\textcolor{red}{S+}}       \texttt{M  M  M} \\
0b1{\color{red}0}1 & \texttt{M  M  M  M  M  M  M  M  M A }\space \texttt{A+}       \texttt{S }\space \texttt{M A }\space \texttt{A }\space \texttt{M S+}       \texttt{S }\space \texttt{\textcolor{red}{S }}\space \texttt{M S+}       \texttt{M  M  M  M} \texttt{||} \texttt{\textcolor{red}{S+}}       \texttt{M  M  M} \\
0b1{\color{red}1}0 & \texttt{M  M  M  M  M  M  M  M  M A+}       \texttt{A+}       \texttt{S }\space \texttt{M A+}       \texttt{A }\space \texttt{M S+}       \texttt{S+}       \texttt{\textcolor{red}{S }}\space \texttt{M S+}       \texttt{M  M  M  M} \texttt{||} \texttt{\textcolor{red}{S }}\space \texttt{M  M  M} \\
\end{tabular}
}
}
\end{table}

% \begin{table*}
% % \setlength{\tabcolsep}{1.8mm}
% \centering
% \begin{tabular}{clcccccc} 
% \toprule
% \multirow{2}{*}{\textbf{Category}}     & \multicolumn{1}{c}{\multirow{2}{*}{\textbf{Method}}} & \multicolumn{3}{c}{\textbf{Normal Dose} $(\text{I}_0 = 10^6)$}                    & \multicolumn{3}{c}{\textbf{Low Dose} $(\text{I}_0 = 5\times10^5)$}     \\ 
% \cmidrule(lr){3-5}\cmidrule(lr){6-8}
%                               & \multicolumn{1}{c}{}                        & 30 Views             & 60 Views & 90 Views & 30 Views & 60 Views & 90 Views  \\ 
% \midrule
% Model-based                 & FBP                                         &    -                  & -          &   -           &   -        & -          &  -         \\ 
% \midrule
% \multirow{2}{*}{Supervised}   & FBPConvNet                                   & 29.51/0.916 &       32.88/0.953   &      35.04/0.967       &         &          &            \\
%                               & RegFormer                                   & \multicolumn{1}{l}{} &          &           &          &          &            \\ 
% \midrule
% \multirow{3}{*}{Unsupervised} & CoIL                                        &                      &          &           &          &          &            \\
%                               & SCOPE                                       &                      &          &           &          &          &            \\
%                               & En-INR                                      &                      &          &           &          &          &            \\
% \bottomrule
% \end{tabular}
% \caption{Quantitative results of compared methods on AAPM dataset under different dose setting.}
% \end{table*}


% \usepackage{multirow}
% \usepackage{booktabs}


\begin{table*}
\centering

\begin{tabular}{clcccc} 
\toprule
\multirow{2.5}{*}{\textbf{Category}} & \multicolumn{1}{c}{\multirow{2.5}{*}{\textbf{Method}}} & \multicolumn{2}{c}{\textbf{Normal Dose} $(I_0 = 10^6)$} & \multicolumn{2}{c}{\textbf{Low Dose} $(I_0 = 5\times10^5)$}  \\ 
\cmidrule(lr){3-4} \cmidrule(lr){5-6}
                                   & \multicolumn{1}{c}{}                                 & \textbf{60 Views}    & \textbf{90 Views}                                         & \textbf{60 Views} & \textbf{90 Views}                                                \\ 
\midrule
\texttt{Analytical}                      & FBP                                                  &  23.25/0.4502           & 26.01/0.5708                                                & 22.92/0.4272        & 25.59/0.5416                                                       \\ 
\midrule
\multirow{2}{*}{\texttt{Supervised}}        & FBPConvNet                                           & 29.67/0.8527 & 30.85/0.8787                                      &  29.53/0.8473        &   30.67/0.8721                                                       \\
                                   & RegFormer                                            &      \underline{33.11}/\textbf{0.9173}       &    \underline{33.77}/\underline{0.9256}                                              &   \underline{32.40}/\underline{0.8865}       &    32.92/0.8901                                                      \\ 
\midrule
\multirow{3}{*}{\texttt{Unsupervised}}      & CoIL                                                 &    30.04/0.8555         &     31.84/0.9001                                            &   30.03/0.8557        &        31.74/0.8966                                                  \\
                                   & SCOPE                                                & 32.00/0.8782             & 33.67/0.9139                                                 &    31.68/0.8657      &  \underline{33.29}/\underline{0.9007}                                                        \\
                                   & Spener (Ours)                                               &   \textbf{34.33}/\underline{0.9170}          &      \textbf{36.61}/\textbf{0.9424}                                            &  \textbf{34.03}/\textbf{0.9110}        &    \textbf{36.17}/\textbf{0.9349}                                                      \\
\bottomrule
\end{tabular}
\caption{Quantitative results of compared methods on AAPM dataset under different dose setting. The best performance is highlighted in \textbf{bold}, and the second best is \underline{underlined}.}
\label{table2}
\end{table*}

In~\autoref{tab:key_op01}, we note that operations 18 and 26 (marked in red) experience \emph{collision} when the second bit of the scalar is ``1'', regardless of the random ``z'' value.
  Similarly, we can deduce the equality of the 2\textsuperscript{nd} and 3\textsuperscript{rd} bits by detecting collisions between operations 55 and 59, as detailed in~\autoref{tab:key_op55_59}. These collisions occur whether both operations undergo a modular reduction as discussed in~\autoref{sec:attackz}. This pattern of identifying bit-equality through operational collisions can be extended iteratively. For example, \autoref{tab:collisions_indexes} lists operation indexes useful for extracting information about the first 6 bits of the ephemeral key. Upon completion of this phase, the component reveals details regarding the collision model as depicted in \autoref{tab:collisions_indexes}, and these indexes will be utilized to identify the bits of the ephemeral key.

\begin{table}[!ht]
    \caption{Correspondence between the index of operations to search for collision (2\textsuperscript{nd} column), the bit of the key retrieved (1\textsuperscript{st} column) and the information provided by a collision (3\textsuperscript{rd} column).}\label{tab:collisions_indexes}
\centering
\scalebox{0.6}{
\resizebox{\columnwidth}{!}{
\begin{tabular}{r|l|l}
\emph{Bit/s} & \emph{Subs indexes} & \emph{Information}\\
\hline
2 & (18,26) &  \mbox{``1'' if collision, ``0'' otherwise}\\
2,3 & (55,59) & \mbox{bits equal if no collision, different otherwise}\\
3,4 & (88,92) & \mbox{bits equal if no collision, different otherwise}\\
4,5 & (121,125) & \mbox{bits equal if no collision, different otherwise}\\
5,6 & (154,158)  &\mbox{bits equal if no collision, different otherwise}\\

\end{tabular}%
}
}
\end{table}

\subsection{Extracting the Private Key}
The final stage in the process involves extracting the private key by performing an LLL attack. Using the earlier-mentioned recognition method, we can obtain segments of the ephemeral key used in ECDSA scalar multiplication. Subsequently, we apply an LLL attack to derive the signer's secret key from the gathered data. Various versions and implementations of the LLL attack are available, and we specifically opted for the FPlll~\cite{fplll} lattice implementation. This choice is motivated by its Python compatibility, support for various LLL reduction techniques, and its accessibility and efficiency. The basis used to represent the extracted values for each ephemeral key in the LLL algorithm aligns with that of~\cite{jancar2020minerva}, followed by Babai's nearest plane reduction. With this setup, we managed to recover the signer's key using as few as 60 signatures, where each signature included an ephemeral key with at least 5 known bits. Additional details on LLL attacks can be found in~\cite{jancar2020minerva}. 

\section{Evaluation Methodology}
\label{sec:eval}

Our evaluation methodology uses a two-phase approach to determine the efficiency of our system. The initial phase utilizes a simulated setting to adjust the neural network (NN) hyperparameters. The subsequent phase assesses the network performance using power traces obtained from a target device, aiming to evaluate the practicality of the proposed attack.

\subsection{Simulated Environment}
The simulated environment aims to improve the neural network's performance on real power consumption data. Therefore, the artificial dataset is designed to accurately represent real-world situations.

To create the synthetic power traces, we started by re-implementing our chosen Montgomery ladder algorithm. We then ran it with varying values and linked each instruction of the algorithm to a sample in the trace. The value of each sample is based on leakage measurements from thousands of previously analyzed real-world traces. This method ensures that the synthetic traces accurately represent the data we can gather in actual scenarios, maintaining coherence in the training and tuning of the model, although it requires retraining when the target device is changed.

To further improve the dataset's realism, we apply a few extra modifications to the traces: to accurately represent modular reductions, we prolong the leakage trace of a subset of the SOs by adding a segment of its base pattern to simulate additional computation time. Additionally, we introduce vertical random noise to each operation type's base leakage pattern, with a mean of 0 and a standard deviation of 1.5. Lastly, we incorporate horizontal random noise to alter the trace's length, which reflects the operational jitter present in the actual target.

Through the aforementioned steps, we obtain a final set of 12 traces. This quantity has been shown to be adequate, as discussed in Section \ref{sec:nn}, because our model processes not the entire traces at once, but the overlapping sliding windows of samples extracted from the power traces. Consequently, this allows us to maintain a manageable trace count while still deriving enough input data from these traces to fully train and test our ML model.
In particular, our dataset comprises 240,800 windows in total, with 160,533 allocated for NN training and 80,267 designated for testing. This represents an approximate 70/30 split of the entire dataset. To ensure that the network does not detect patterns unrelated to the recognition task, we balanced the ratio of SOs and LOs equally across both sets.

\subsection{Real World Target}
\label{sec:exp_real}
In order to assess our system in a real world scenario, we employed the following configuration. The target device utilized is an STM32F415RG microcontroller integrated within the ChipWhisperer platform. Its core comprises an ARM 32-bit Cortex-M4 CPU operating at speeds of up to 168 MHz. For measuring power consumption, we used the ChipWhisperer acquisition device. This equipment features a 10-bit ADC with a maximum sample rate of 105 MS/s, an AC-Coupled analog input, an adjustable gain up to +55 dB, and it can generate clocks ranging from 5 MHz to 200 MHz. The acquisition of traces was carried out in streaming mode, capturing one sample per clock cycle, at a frequency of 7.37 MHz. This sampling rate was chosen based on the acquisition device's limitations, which supports a maximum of 10 MS/s in streaming mode because of its buffer capacity. To assess the classification results, the test bench also supplied additional metadata for each acquired trace. Specifically, for each segment within the power trace, the operation that produced it is collected, as demonstrated in Figure~\ref{fig:real_power_consumption_trace}.

\begin{figure}[h]
    \centering\includegraphics[width=\columnwidth]{images/real_power_consumption_trace.pdf}
    \caption{The first 24 thousand samples of a real power consumption trace (translated on the y-axis by 20 points for typesetting reasons).}  \label{fig:real_power_consumption_trace}
\end{figure}

We obtained a total of five power traces, using four for model training and one for testing. As mentioned earlier, this quantity was more than adequate for training and evaluating the model due to their extensive data. Table \ref{tab:details_realPwerConsumptionTraces} provides a detailed breakdown of these traces, demonstrating that the number of inputs extracted is sufficient for a thorough evaluation of our system.

\begin{table}[h!]
    \centering
    \caption[Real power consumption traces sizes]{Sizes of four real power consumption traces, acquired for testing, with derived datasets. For each power consumption trace (1\textsuperscript{st} column), we provide its samples size (2\textsuperscript{nd} column), the total number of short operation (SO) and long operation (LO) samples (3\textsuperscript{rd} column), the size of the testing dataset (4\textsuperscript{th} column) and the number of \texttt{true}-tagged and \texttt{false}-tagged windows (last column).
    }
    \scalebox{0.6}{
    \label{tab:details_realPwerConsumptionTraces}
    \resizebox{\columnwidth}{!}
    {%
        \begin{tabular}{c||c|cc||cl|cc}
            
            
            \textit{Trace} &
            
            \textit{Trace Size} &
            \multicolumn{2}{c||}{
            \begin{tabular}[c]{@{}c@{}}\textit{Trace Samples} \\\textit{\small SO samples} $\;$ \textit{\small LO samples}\end{tabular}} &
            \multicolumn{2}{c|}{
            \textit{Dataset size}} &
            \multicolumn{2}{c}{
            \begin{tabular}[c]{@{}c@{}}\textit{Dataset Windows}\\\texttt{\small true}\textit{\small -tag} $\;$ \texttt{\small false}\textit{\small -tag}\end{tabular}} \\\hline
            
            \texttt{T1} &
            $7,032,251$ &
            \multicolumn{1}{c:}{$\hspace{0.6em} 524,697 \hspace{0.6em}$} &
            $6,507,554$ &
            \multicolumn{2}{c|}{\begin{tabular}[c]{@{}c@{}}$703,176$\end{tabular}} &
            \multicolumn{1}{c:}{$\hspace{0.4em} 86,948 \hspace{0.4em}$} &
            $616,228$ \\ \hline
            
            \texttt{T2} &
            $7,002,209$ &
            \multicolumn{1}{c:}{$\hspace{0.6em} 522,537 \hspace{0.6em}$} &
            $6,479,672$ &
            \multicolumn{2}{c|}{\begin{tabular}[c]{@{}c@{}}$700,171$\end{tabular}} &
            \multicolumn{1}{c:}{$\hspace{0.4em} 86,739 \hspace{0.4em}$} &
            $613,432$ \\ \hline
            
            \texttt{T3} &
            $7,005,521$ &
            \multicolumn{1}{c:}{$\hspace{0.6em} 525,957 \hspace{0.6em}$} &
            $6,479,564$ &
            \multicolumn{2}{c|}{\begin{tabular}[c]{@{}c@{}}$700,503$\end{tabular}} &
            \multicolumn{1}{c:}{$\hspace{0.4em} 87,086 \hspace{0.4em}$} &
            $613,417$ \\ \hline
            
            \texttt{T4} &
            $7,027,186$ &
            \multicolumn{1}{c:}{$\hspace{0.6em} 524,247 \hspace{0.6em}$} &
            $6,502,939$ &
            \multicolumn{2}{c|}{\begin{tabular}[c]{@{}c@{}}$702,669$\end{tabular}} &
            \multicolumn{1}{c:}{$\hspace{0.4em} 86,893 \hspace{0.4em}$} &
            $615,776$ \\
            
        \end{tabular}
    }
    }
\end{table}

\section{Results}
\label{sec:res}
As mentioned in Section \ref{sec:sys}, in order for our system to work, the network must properly classify the windows given in the input, since misclassifying an input can hinder the estimation of the duration of operations.
After training and testing on both datasets detailed in Section \ref{sec:eval} our NN, the core part of our system, proved to be extremely effective both in the simulated environment and, most importantly, in the real world target.
Specifically in the simulated environment, we reached an accuracy of approximately 99.9\%.

\begin{figure}[h!]
\centering
\scalebox{0.6}{
\includegraphics[width=\linewidth]{images/epoch_accuracy_paper_2.pdf}
}
\caption{Epoch accuracy plot for 100 epochs, batch size of 64 and validation split of $20\%$: train in orange, validation in blue.}
\label{fig:epoch_accuracy}
\end{figure}

\begin{figure}[h!]
\centering
\scalebox{0.6}{
\includegraphics[width=\linewidth]{images/epoch_loss_paper_2.pdf}
}
\caption{Epoch loss plot for 100 epochs, batch size of 64 and validation split of $20\%$: train in orange, validation in blue.}
\label{fig:epoch_loss}
\end{figure}
Our network performed extremely well, as shown in Table \ref{tab:lstm_vs_cnn}, also on real-world traces, reaching an accuracy of around 97\% for each trace. 
In~\autoref{fig:accuracy_plot}, we plot the accuracy results for the initial portion of a real power consumption trace used during the testing phase.

From~\autoref{fig:accuracy_plot}, it is clear that the network was correct in most cases
(the green part of the plot). 

In the field of side-channel attack research, CNNs have been the predominant choice for analytical tasks. To demonstrate the superiority of the LSTM for this specific task, we performed an ablation study in which we removed the LSTM layer, thus converting the architecture into a CNN and using the same network parameters. In \ref{tab:lstm_vs_cnn}, we present the accuracy and loss results of our LSTM model when tested on real power consumption traces, and compare these with the results of a trained CNN model obtained by excluding the LSTM layer from the architecture mentioned in Section \ref{sec:sys}. This comparison validates our hypothesis that, given the similarity to the HAR problem, the LSTM is more suitable than the traditional CNN.

\begin{figure}[h]
    \centering
    \scalebox{0.85}{
    \includegraphics[width=\columnwidth]{images/bokeh_accuracy_plot_T2_raw_slicesize10.pdf}}
    \caption{Accuracy results for the initial portion of a real power consumption trace. In green the correct predictions, in red the failed ones.}
    \label{fig:accuracy_plot}
\end{figure}

\begin{table}[!ht]
    \caption{Accuracy and loss of LSTM and CNN collected for testing real power consumption traces.}\label{tab:lstm_vs_cnn}
\centering
\scalebox{0.60}{
\resizebox{\columnwidth}{!}{
\begin{tabular}{r||l|l||l|l}
\emph{Trace} & \emph{LSTM Accuracy}& \emph{CNN Accuracy} & \emph{LSTM Loss} & \emph{CNN Loss} \\
\hline
\texttt{T1} & 0.9758 & 0.9437 & 0.0621 & 0.1763 \\
\texttt{T2} & 0.9766 & 0.9463 & 0.0603 & 0.1692 \\
\texttt{T3} & 0.9756 & 0.9499 & 0.0646 & 0.1746 \\
\texttt{T4} & 0.9728 & 0.9344 & 0.0702 & 0.2039 \\

\end{tabular}%
}
}
\end{table}

\subsection{End-To-End Attack}
An important issue encountered during the development of our system is the improper classification of successive operations of the same type (false positive). This issue arises from the neural network's lack of training to recognize the start and end of an operation. It has only been trained to distinguish between short- and long-term operations. As a result, when the model is faced with a sequence of \emph{Sub} operations, it is unable to discern which are short and which are long.

To overcome these limitations, we have adjusted our post-processing approach to operate on sequences rather than individual operations. We will refer to these sequences as \emph{clusters}, using them to identify the key bits. Specifically, we consider a trace to be valid if and only if all operations within it exhibit the same behavior (we define these as homogeneous clusters). This characteristic can be readily determined by comparing the cluster's length with the durations of short and long operations.

A significant disadvantage of this approach is the increased number of traces required because we are limited to analyzing traces that display homogeneous clusters. The decrease in the number of useful traces for any individual cluster is directly proportional to the cluster's length. Specifically, for a cluster of length $n$, the likelihood of obtaining a useful trace is $1/n$. Consequently, for $\Gamma$, the set of groups examined, the probability of finding a single trace where all clusters are homogeneous is $\prod_{\gamma \in \Gamma} 1/len(\gamma)$.

In order to quantify the increase in the number of traces needed for a successful attack, we recall ~\autoref{tab:key_op01}. We can clearly observe that operation 18 is part of a cluster of three \emph{Subs}, those at positions 17, 18, and 19. Instead, operation 26 is part of a cluster of 1. Then, operations 55, 88, 121, 154 and so on (each is 33 operations away),  are always in a cluster of 1, while operations 59, 92, 125, 158, etc. (each is 33 operations away) are always in a cluster of two (the \emph{Sub} before the change of the bit and the \emph{Sub} afterwards). This means that the first bit depends on the correct identification of the 3\textsuperscript{rd} and 5\textsuperscript{th} clusters. Also, since the 4\textsuperscript{th} cluster is a cluster of 3 operations, while the 5\textsuperscript{th} is a cluster of one, the attacker has a chance of 1/4 to get a useful prediction for a given signature trace. Instead, subsequent collisions depend on a cluster of 1 (operations 55, 88, 121, 154, etc.\ldots) and a cluster of two (operations 59, 92, 125, 158, etc\ldots). So, in this case, the predictions have probability one-half to be useful.
We also remark that from a useful collision we are able to correctly identify the exact value of the 2\textsuperscript{nd} bit of the ephemeral key, while for the next bit a useful collision only provides equality with the previous bit value. Thus, to obtain the value of the 3\textsuperscript{rd} bit of the ephemeral key, the attacker needs to get a useful collision on the 2\textsuperscript{nd} bit and a useful collision on the 3\textsuperscript{rd} bit. Similarly, to reveal the 4\textsuperscript{th} bit, three consecutive useful collisions on the same signature are needed.
\autoref{tab:bits_proba} summarizes these results. In particular, we observe that to obtain one signature where the attacker is able to retrieve 5 bits, she needs, on average, 64 signatures. This in turn means that to obtain 100 such signatures in order to perform the lattice reduction $6,400$ executions, it should be needed. In our scenario, execution of $6,400$ takes almost 10 days using the crypto algorithm on the Teledyne Lecroix 3104z oscilloscope model. Albeit this is an increase in the number of traces that should be acquired, this number is far from unfeasible in the context of a side-channel attack meaning that the solution does not limit the exploitation capability of the designed system.

\begin{table}[!ht]
    \caption{Probability to obtain \emph{n} bits for each signature, and estimation on the number of signatures required to obtain them.}\label{tab:bits_proba}
\centering
\scalebox{0.6}{
\resizebox{\columnwidth}{!}{
\begin{tabular}{r|l|l}
\#\emph{Bits} & \emph{Probability} & \emph{Estimated sign. required}\\
\hline
1 & $\sfrac{1}{4}$ & 4 \\
2 & $\sfrac{1}{4} \cdot \sfrac{1}{2} = \sfrac{1}{8}$ & 8\\
3 & $\sfrac{1}{4} \cdot \sfrac{1}{2} \cdot \sfrac{1}{2} = \sfrac{1}{16}$ & 16\\
4 & $\sfrac{1}{4} \cdot \sfrac{1}{2} \cdot \sfrac{1}{2} \cdot \sfrac{1}{2} = \sfrac{1}{32}$ & 32\\
5 & $\sfrac{1}{4} \cdot \sfrac{1}{2} \cdot \sfrac{1}{2} \cdot \sfrac{1}{2} \cdot \sfrac{1}{2} = \sfrac{1}{64}$ & 64\\

\end{tabular}%
}
}
\end{table}

\section{Countermeasures}
\label{sec:countermeasures}

In this Section, we discuss a range of countermeasures suitable for various types of implementation. These countermeasures vary in their requirements for random generation and differ in efficiency levels.

\subsection{Incomplete Countermeasures}
\paragraph{Real effectiveness of the~\cite{luo2018effective} countermeasure.}
If the implementation can afford using one additional register, it is possible to adopt the countermeasure suggested by Luo~\textit{et al.}~\cite{luo2018effective}. 
Since the weakness arises from the subtraction computation in XYCZ-ADDC, and since the result is squared,
it does not matter if it is $X_0-X_1$ or $X_1-X_0$. So, the authors suggest saving the subtraction result performed during XYCZ-ADD, and reuse it in XYCZ-ADDC for the next bit. 
We remark that this countermeasure is very efficient and also saves one subtraction. However, XYCZ-ADDC subtraction is not the only collision exploitable during execution. For example, we observe that the order of use of $X_1$ and $X_2$ in iteration $i$ in XYCZ-ADDC depends on the value of the key bits $k_{b_{i-1}}$ and $k_{b_i}$. So, for example, if $k_{b_{i-1}}$ equals $k_{b_i}$, the $X$ outputs of XYCZ-ADD will be used in the same order as they have been computed; otherwise, their order will be reversed. Using collisions, an attacker can detect if the same value is used during the first or second computation (that is, when computing $B$ or $C$), and infer the values of the key bits. 

\paragraph{Modular Reduction.}
A classic countermeasure that can be applied to the implementation is to always perform modular reductions, regardless of whether the computation overflows or underflows. This approach was previously suggested in \cite{ryan2019return} and is probably the easiest solution to this problem. However, this countermeasure would not prevent a side-channel attacker from detecting that the same value has been computed twice (that is, a collision on the values) through correlation techniques on the power or EM traces. This is the kind of weakness exploited in~\cite{luo2018effective} and, as observed in Section~\ref{sec:attackz}, it would again make it possible for an attacker to obtain sensitive information. 

\subsection{Effective Countermeasures}
\paragraph{Masking Technique.}
A more robust approach to counteract the attack is to mask the values (that is, the coordinates) manipulated during the execution of the Montgomery ladder. This method is more effective, but also less efficient. To avoid collisions, masks must be changed for each iteration of the Montgomery ladder. Otherwise, an attacker could detect the occurrence of a collision regardless of the presence of the masks. 

\paragraph{Coordinates Re-Randomization.}
Another solution to this problem is to repeat randomization of the coordinates after each iteration of the Montgomery loop. This approach involves generating a new \emph{z} for each iteration and reprojecting the curve to the new coordinates. However, it could be more cost-effective than a fully masked implementation and may also prevent collision attacks. The process of re-randomizing the coordinates essentially ensures that there is no correlation between the previous and current computations, and thus eliminates the possibility of a collision attack. 

\section{Conclusions}
Our study rigorously evaluated the implementation of ECDSA on a commercial standard device using the micro-ecc library, focusing on the secp160r1 curve. We identified vulnerabilities, particularly in non-constant-time execution, due to selective modular reductions. By improving existing collision attacks and applying them to real-world hardware, we exposed more flaws. Using simulations and an LSTM neural network, we detected operational patterns indicating modular reductions, enabling us to derive ephemeral key bits and recover the signing key through lattice attacks. Testing on a real STM32F415RG device with the ChipWhisperer confirmed our method's high accuracy and effectiveness in revealing significant security risks in prevalent cryptographic systems.

%\bibliographystyle{plain} 
%\bibliography{bib_IEEE,bib}
\documentclass{MITstyle}

%\usepackage[table]{xcolor}
\usepackage{chngcntr}
\usepackage{hyperref}
\usepackage{microtype}

\title{A Lightweight and Extensible Cell Segmentation and Classification Model for Whole Slide Images}

\author{Nikita Shvetsov~$^{1, }$\footnote{Correspondence e-mail: nikita.shvetsov@uit.no}, Thomas K. Kilvaer~$^{2, 3}$, Masoud Tafavvoghi~$^{4}$, Anders Sildnes~$^{1}$, \\ Kajsa Møllersen~$^{4}$, Lill-Tove Rasmussen Busund~$^{5, 6}$, Lars Ailo Bongo~$^{1}$ \\
%
\vspace{1em} % Space between authors and afilliations
%
\normalfont{\small $^{1}$Department of Computer Science, UiT The Arctic University of Norway}\\
\normalfont{\small $^{2}$Department of Oncology, University Hospital of North Norway}\\
\normalfont{\small $^{3}$Department of Clinical Medicine, UiT The Arctic University of Norway}\\
\normalfont{\small $^{4}$Department of Community Medicine, UiT The Arctic University of Norway}\\
\normalfont{\small $^{5}$Department of Medical Biology, UiT The Arctic University of Norway} \\
\normalfont{\small $^{6}$Department of Clinical Pathology, University Hospital of North Norway} %\vspace{2em}
}

\begin{document}
\maketitle

\section*{Abstract}

% \begin{abstract}
% Developing clinically useful cell-level analysis tools in digital pathology remains challenging due to limitations in dataset granularity, inconsistent annotations, computational demands of advanced models, and difficulties in integrating new technologies into clinical workflows. To address these challenges, we propose a multi-faceted solution that enhances data quality, model performance, and usability to create a lightweight and extensible cell segmentation and classification model.

% First, we update data labels by employing a cross-relabeling process that refines the labels of two existing datasets, PanNuke and MoNuSAC, to create a new unified dataset with enhanced granularity, encompassing seven distinct cell types. Second, we leverage the H-Optimus foundation model as a fixed encoder to improve feature representation for simultaneous cell segmentation and classification tasks. Third, to address the computational demands of foundation models, we employ knowledge distillation to reduce model size and complexity while maintaining comparable performance. Finally, to facilitate integration into clinical workflows, we integrate the distilled model into the QuPath software, a widely used open-source platform in digital pathology.

% Our results demonstrate improvements in cell segmentation and classification performance using the H‑Optimus-based model compared to a CNN-based model. Specifically, the average $R^2$ improved from 0.575 to 0.871, and the average $PQ$ score improved from 0.450 to 0.492, indicating better alignment with actual cell counts and enhanced segmentation and classification quality. Furthermore, the distilled student model maintains performance comparable to the larger foundation model while reducing the parameter count by a factor of 48.
% Overall, by reducing computational complexity and integrating it into existing workflows, the proposed approach may significantly impact diagnostic processes, reduce the workload of pathologists, and contribute to improved patient outcomes. Though our approach shows potential enhancements in efficiency and usability of cell segmentation and classification models in digital pathology, extensive validation is needed to deploy these models in clinical practice.
% \end{abstract}

%%% shortened abstract
\begin{abstract}
Developing clinically useful cell-level analysis tools in digital pathology remains challenging due to limitations in dataset granularity, inconsistent annotations, high computational demands, and difficulties integrating new technologies into workflows. To address these issues, we propose a solution that enhances data quality, model performance, and usability by creating a lightweight, extensible cell segmentation and classification model. 

First, we update data labels through cross-relabeling to refine annotations of PanNuke and MoNuSAC, producing a unified dataset with seven distinct cell types. Second, we leverage the H-Optimus foundation model as a fixed encoder to improve feature representation for simultaneous segmentation and classification tasks. Third, to address foundation models' computational demands, we distill knowledge to reduce model size and complexity while maintaining comparable performance. Finally, we integrate the distilled model into QuPath, a widely used open-source digital pathology platform. 

Results demonstrate improved segmentation and classification performance using the H-Optimus-based model compared to a CNN-based model. Specifically, average $R^2$ improved from 0.575 to 0.871, and average $PQ$ score improved from 0.450 to 0.492, indicating better alignment with actual cell counts and enhanced segmentation quality. The distilled model maintains comparable performance while reducing parameter count by a factor of 48. By reducing computational complexity and integrating into workflows, this approach may significantly impact diagnostics, reduce pathologist workload, and improve outcomes. Although the method shows promise, extensive validation is necessary prior to clinical deployment.
\end{abstract}
\clearpage

\section{Introduction}
In digital pathology, accurate segmentation and classification of cells are crucial for many diagnostic, prognostic, and predictive analyses \cite{Jaber_Beziaeva_etal._2019,Lin_Pan_etal._2022,Park_Ock_etal._2022,Shen_Choi_etal._2024}. Nowadays, developments in computational pathology offer multiple solutions \cite{H._Qu_P._Wu_etal._2020,Javed_Mahmood_etal._2020} to utilize cell-level datasets to train machine learning models that solve these problems. The quality and specificity of training datasets are critical for robust and accurate models. Adhering to the principle of "garbage in, garbage out", it is essential to ensure that these datasets are extensively and accurately labeled with distinct classes that reflect the diverse biological characteristics of different cell types. Unfortunately, the number of open-source datasets comprising such high-quality annotations is limited. Existing cell segmentation datasets \cite{Gamper_Koohbanani_etal._2019,Graham_Vu_etal._2019,Verma_Kumar_etal._2021} may offer extensive annotations for certain cell types while providing more general labels for others. For example, in PanNuke, which is one of the largest open-source datasets comprising labeled cells, various types of morphologically and functionally different inflammatory cells like macrophages and lymphocytes are clustered in a broad "inflammatory" class. Consequently, these classes are frequently omitted from analyses or aggregated into broader meta-classes \cite{Gamper_Koohbanani_etal._2020} and likely interfere with other cell classes included in the dataset. This and similar inconsistencies in annotation granularity limit the ability of machine learning models to learn the comprehensive and nuanced features necessary for accurate cell segmentation and classification. To address these challenges, methods for refining and standardizing dataset annotations are essential to enhance the quality of training data.

A complementary approach to mitigate the absence of high-quality training data is the use of foundation models. Foundation models as encoders are defined as large-scale, versatile networks pre-trained on vast, diverse datasets using self-supervised learning, contrasting with convolutional neural network (CNN) pre-trained encoders that rely on supervised learning with labeled data. In practice, foundation models leverage enormous amounts of weakly or unlabeled data from millions of whole slide images (WSIs) and employ self-attention mechanisms to capture long-range dependencies and global context \cite{Chen_Ding_etal._2024,Saillard_Jenatton_etal._2024,Vorontsov_Bozkurt_etal._2024,Xu_Usuyama_etal._2024}. As a consequence, foundation models are able to produce transferable feature representations across different cell types and tissue environments. The feature representations can be leveraged by decoder networks to produce segmentation masks and pixel-level classifications. Because foundation models have comprehensive feature representations, they can be effectively fine-tuned using much smaller amounts of cell-level data compared to the large datasets needed to train models from scratch. Furthermore, foundation models incorporate adversarial training elements or contrastive learning \cite{Chen_Ding_etal._2024,Xu_Usuyama_etal._2024}, enhancing their resilience and adaptability by exposing them to challenging and varied scenarios during training. This may result in more generalizable models, often making them well-suited for diverse and complex tasks in digital pathology.

Despite the inherent advantages of foundation models, their deployment for practical use faces its own obstacles. In particular, they require substantial computational power, financial investments and rigorous testing to ensure reliability and efficacy for a given task \cite{Akkus_Dangott_etal._2022,Dragomir_Cocuz_etal._2022,Go_2022,Jafri_Farooqui_etal._2024}. Moreover, while foundation models enhance feature representation and performance, they depend on the quality of available annotations for decoder fine-tuning and, like any other model, cannot resolve existing inconsistencies or ambiguities in data labels. Therefore, there remains a critical need for solutions that address both data quality and practical deployment considerations.
Further, integrating new technologies into existing clinical workflows often encounters resistance, as it necessitates adjustments to established diagnostic processes. So, there is a need to develop solutions that could be integrated into current practices, minimizing the burden on medical professionals to adopt new tools \cite{King_Williams_etal._2023}.

Existing solutions \cite{Goldsborough_Philps_etal._2024,Hörst_Rempe_etal._2024}, while addressing some aspects of these challenges, fall short in providing a comprehensive approach. To address the data quality and clinical deployment issues, we propose a multi-faceted solution that encompasses data refinement, model optimization, and integration with existing pathology tools (\hyperref[fig:fig1]{Figure 1}). The outcome is a lightweight cell segmentation and classification model that can be integrated into digital pathology workflows for practical clinical use.

\begin{figure}[h!]
    \centering
    \includegraphics[width=\textwidth, height=0.82\textheight, keepaspectratio]{images/Figure_1.pdf}
    \caption{Overview of the proposed solution, including 1) Data refinement using cross-relabeling, 2) Teacher model development and fine tuning, 3) Student model optimization with knowledge distillation and 4) Student model and QuPath integration}
    \label{fig:fig1}
\end{figure}
\clearpage

Our approach begins with preparing the data for the fine-tuning and training of the machine learning models. We create a refined dataset, acquired via cross-relabeling two cell-level datasets, enhancing annotation specificity and consistency of the labeled data. Subsequently, we create a cell segmentation and classification model based on the foundation model. We leverage the foundation model as a fixed encoder and fine-tune a decoder using the refined dataset to improve generalization across diverse tissue- and cell types.
To ensure that the model remains lightweight and deployable in a possibly resource-constrained environment, we employ knowledge distillation to approximate the functionality of the foundation model. Finally, to facilitate the practical application of our model in digital pathology workflows, we integrate it with the QuPath \cite{Bankhead_Loughrey_etal._2017} application. Each methodological component contributes to the overarching goal of enhancing model performance, generalizability, and usability in clinical settings.

The primary contributions of this paper are:
\begin{enumerate}
    \item \textit{Data labels refinement through cross-relabeling:}
    
    We propose a new method for refining labels of cell-level datasets through cross-relabeling. This method employs classification models to re-label broad and ambiguous instances, resulting in a more diverse dataset. Our evaluation demonstrates that these classification models achieve high accuracy on test subsets, indicating the reliability of the method for label refinement.

    \item \textit{Enhanced model performance via foundation models:}
    
    We employ a foundation model as a feature extractor for the cell segmentation and classification task. In comparison with training a CNN model from scratch, the foundation model backbone only needs fine-tuning, which significantly reduces training time, computational resources and data requirements. We show that using a foundation model encoder leads to better performance in cell segmentation and classification networks than using a CNN-based encoder. This improvement may enable the model to generalize more effectively across various tissue types and imaging methods.
    
    \item \textit{Model optimization through knowledge distillation:}
    
    We show that a smaller student model trained using knowledge distillation on the refined dataset obtained via our cross-relabeling approach from a foundation model achieves comparable performance in cell segmentation and quantification tasks. As a result, this model is more suitable for deployment in environments without high-performance computing resources.
    
    \item \textit{Integration with QuPath:}
    
    We integrate the distilled cell segmentation and classification model into QuPath, a widely used open-source digital pathology platform, to accelerate clinical adaptation by enabling pathologists to more easily incorporate advanced computational tools into their existing workflows.
\end{enumerate}

Through these methodological steps, we aim to bridge the gap between advanced machine learning techniques and practical clinical applications, making accurate and efficient digital pathology accessible in a broader range of healthcare settings.

\section{Refining Existing Datasets Using Cross-Relabeling}
To address the limitations of sparse and ambiguous labeling of cell-level datasets, we propose a generalizable cross-relabeling strategy that can be applied to any dataset containing broadly categorized or imprecisely labeled cell types. This approach involves training and subsequently leveraging classification models to refine broad categories into more specific or biologically relevant classes.
When applied to cell-level data, the methodology includes extracting individual cell images from the dataset patches, preprocessing these images to standardize the size and accommodate partial cells, and then training deep learning classifiers capable of distinguishing between the finer cell subtypes within the coarser categories. 
To illustrate our approach, we focus on the PanNuke \cite{Gamper_Koohbanani_etal._2020, Gamper_Koohbanani_etal._2019} and MoNuSAC \cite{Verma_Kumar_etal._2021} datasets that we have used to train models for cell quantification in our previous works \cite{Shvetsov_Grønnesby_etal._2022,Shvetsov_Sildnes_etal._2024}. We find that for better cell differentiation we have to introduce more granular labels. PanNuke includes a broad classification of "inflammatory" cells, encompassing lymphocytes, macrophages, and neutrophils. Each cell type differs significantly in structure, function, and clinical relevance. Conversely, MoNuSAC uses the label "epithelial" for a class that comprises both benign epithelial cells and malignant neoplastic cells. This practice makes it challenging to differentiate between benign and malignant epithelial cells in the dataset, which is a critical distinction when identifying tumor areas within tissue samples. To address these issues, we implement a cross-relabeling strategy as shown in \hyperref[fig:fig2]{Figure 2}. The key components are two classification models: one is trained on singular cell images from PanNuke data to classify the epithelial meta-class into epithelial and neoplastic classes. The other is trained on MoNuSAC to refine the inflammatory class into lymphocytes, neutrophils, and macrophages.

\begin{figure}[h!]
    \centering
    \includegraphics[width=\textwidth]{images/Figure_2.pdf}
    \caption{Refined dataset generation via cross relabeling}
    \label{fig:fig2}
\end{figure}

The refining approach consists of three consecutive steps. The first is the preprocessing step, in which we extract individual cells from both datasets (\hyperref[fig:fig3]{Figure 3}). The specifics of PanNuke and MoNuSAC patch preparation before cell preprocessing are provided in \hyperref[chap:S1]{Appendix S1}.

\begin{figure}[h!]
    \centering
    \includegraphics[width=\textwidth]{images/Figure_3.pdf}
    \caption{Cell instances preprocessing including (1) cell map extraction, (2) bounding box delineation, (3) adjusting cell boxes and (4) cropping and resizing of cell images}
    \label{fig:fig3}
\end{figure}

During preprocessing, we extract cell type maps from the ground truth label mask and calculate bounding boxes around each cell instance. To accommodate partial cells at patch borders, a common issue in cropped patch images, we employ mirror padding and extend the field of view of the cell label by 15 pixels to capture adjacent cells. We then crop and resize the identified regions to $64 \times 64$ pixels using bicubic interpolation.

The preprocessed PanNuke dataset comprises 68,031 neoplastic and 23,207 epithelial cell images, while MoNuSAC comprises  33,104 lymphocytes, 1,252 neutrophils, and 1,695 macrophages, which we subsequently use in training cell classification models and classifying the cell image data \hyperref[fig:S2]{Appendix Figure S2 (1)}. 

The next step is to train two distinct ResNet50-based classifiers tailored to address the specific labeling challenges inherent in each dataset. We use ResNet50 for classification models due to its proven effectiveness for image classification tasks in histopathology \cite{pan2022reviewmachinelearningapproaches}, and its compatibility with small images. For the PanNuke dataset, we design the classifier, trained on MoNuSAC data, to disaggregate the heterogeneous "inflammatory" cell category into distinct subtypes: lymphocytes, macrophages, and neutrophils. Similarly, for the MoNuSAC dataset, the classifier is trained on PanNuke data and distinguishes between benign and malignant epithelial cells within the overarching "epithelial" label. By applying these targeted classifiers to their respective datasets, we assign more specific labels to individual cell instances, thus enabling us to create a unified dataset.
To ensure a balanced representation of classes, we train both models on datasets that had been equalized to match the size of the least represented class. Thus, we obtain datasets comprising 23,207 samples per class for PanNuke and 1,252 samples per class for MoNuSAC data. Next, we partition both of them into training (70\%), validation (20\%), and testing (10\%) subsets. To mitigate the risk of overfitting, we use a single dropout layer with a rate of p=0.5 in both models and data augmentation using randomized color perturbations, rotation, and horizontal and vertical flipping. We employ AdamW optimizer and the cross-entropy loss function for the training criterion.

To evaluate the two trained models, we measure the classification accuracy on the respective test subsets. The accuracies on the test subset for both classifiers are presented in \hyperref[tab:1]{Table 1}. The PanNuke model achieves an average accuracy of 93.57\%, with higher accuracy for neoplastic cells (96.06\%) compared to epithelial cells (86.26\%). The confusion matrix in Figure A3.1 shows that the model predominantly distinguishes accurately between epithelial and neoplastic tissues, with a substantial number of correct classifications and relatively few misclassifications. The MoNuSAC model demonstrates an average accuracy of 98.92\%, excelling in classifying lymphocytes (99.67\%) and macrophages (94.12\%), with lower performance for neutrophils (85.71\%). The confusion matrix in Figure A3.2 shows that the model identifies lymphocytes and performs reasonably well with macrophages and neutrophils.

\begin{table}[h!]
\renewcommand{\arraystretch}{1.5}
  \centering
  \caption{Cell classification results for PanNuke and MoNuSAC trained models (CI 95\%).}
  \label{tab:1}
  \begin{tabular}{|l|c|c|}
   \hline
   %\rowcolor{gray!30}
    Accuracy               & PanNuke model              & MoNuSAC model              \\
    \hline
    Average      & 0.936 (0.931--0.941)         & 0.989 (0.986--0.993)        \\
    \hline
    Neoplastic   & 0.961 (0.956--0.965)         & -                          \\
    \hline
    Epithelial   & 0.863 (0.849--0.877)         & -                          \\
    \hline
    Lymphocytes  & -                          & 0.997 (0.995--0.999)        \\
    \hline
    Neutrophils  & -                          & 0.857 (0.796--0.918)        \\
    \hline
    Macrophages  & -                          & 0.941 (0.906--0.976)        \\
    \hline
  \end{tabular}
\end{table}

Finally, during the last step, we use the model trained on PanNuke data for epithelial cells in MoNuSAC and the model trained on MoNuSAC for the inflammatory cells class in PanNuke. Specifically, we use classifier models to relabel epithelial cells in MoNuSAC and inflammatory cells in PanNuke data. Then we combine cells with refined labels and the rest of the cells in both datasets to create a refined dataset (\hyperref[fig:S2]{Appendix Figure S2 (2)}). The process of relabeling cells and visualizing them on a patch is shown in \hyperref[fig:fig4]{Figure 4}. The cell counts in the refined dataset are provided in \hyperref[tab:S4]{Appendix Table S4}.

\begin{figure}[h!]
    \centering
    \includegraphics[width=\textwidth, height=0.42\textheight, keepaspectratio]{images/Figure_4.pdf}
    \caption{Cell relabeling procedure for epithelial and inflammatory cell classes}
    \label{fig:fig4}
\end{figure}

%\hfill

Relabeling and combining datasets have been explored in a prior study \cite{Parulekar_Kanwat_etal._2023}, where consecutive fine-tuning on multiple datasets was employed to account for hierarchical class label structures. While the method presented in \cite{Parulekar_Kanwat_etal._2023} is intuitive, it often lacks consistency and requires multiple fine-tuning runs, which can be cumbersome and time-consuming. 
In contrast, cross-relabeling simplifies this process by using specialized classification models tailored to each dataset's specific labeling challenges. This approach provides better transparency and produces a unified dataset encompassing seven distinct cell types across multiple tissue samples, enhancing data diversity for further model training or fine-tuning.

Despite these improvements, cross-relabeling does not entirely resolve issues related to poor labeling quality or the amount of labeled data. Specifically, our results show lower accuracies persist for underrepresented classes, such as macrophages, which may stem from a limited sample availability and intrinsic challenges in distinguishing these cells based solely on H\&E staining. Furthermore, while our method enhances label specificity, it relies on the initial quality of the broad labels; thus, any fundamental inaccuracies in the original annotations can propagate through the relabeling process. Addressing the overall problem of limited data labels may require integrating additional data sources or utilizing complementary immunohistochemical staining methods.
Although the reported performance metrics are obtained from evaluations on the native test sets of each dataset, it is important to note that the primary application of these classifiers is to perform cross-relabeling, where a model trained on one dataset (e.g., PanNuke) is applied to another (e.g., MoNuSAC) and vice versa. We acknowledge that a more systematic evaluation of cross-dataset generalization is needed and could be performed in future work.

Overall, the refined dataset produced by our approach can enhance the supervised training or fine-tuning of cell segmentation and classification models, especially those that utilize pre-trained foundation models to improve feature extraction robustness. In addition, these models can detect nuanced classes that enable researchers to conduct more detailed analyses of biological processes in computational pathology.

\section{Foundation models for robust cell segmentation and classification}

Accurate cell segmentation and classification in digital pathology are hindered by limited labeled data and the fact that conventional CNNs are unable to capture global contextual information due to their local receptive field constraints \cite{Gheflati_Rivaz_2022,Yang_Marcus_etal.}. Traditional approaches in cell quantification have predominantly relied on CNN encoders, such as ResNet50, given their proven effectiveness in semantic segmentation tasks \cite{Deshmane_2023,Graham_Vu_etal._2019,Mukasheva_Koishiyeva_etal._2024,Stringer_Wang_etal._2021}. However, approaches that include fine-tuning of pretrained CNNs, data augmentation, and stain normalization to partially increase data variability and address staining differences often fail to achieve the necessary generalization and robustness across diverse tissue types and staining conditions \cite{G._Wang_W._Li_etal._2018,Gao_Bagci_etal._2018,Karim_El_Khoury_Martin_Fockedey_etal._2021}.

To overcome these challenges, we leverage an encoder-decoder network that uses a foundation model as the encoder and a CNN upsampling decoder (\hyperref[fig:fig5]{Figure 5}) for simultaneous cell segmentation and classification in 2D patches extracted from WSIs. Foundation models with transformer-based architectures are viable alternatives to CNN-based encoders \cite{Shamshad_Khan_etal._2023,Sourget_2023}. They enable the creation of more advanced architectures that can decode or transform learned features more effectively \cite{Chen_Duan_etal._2023,Cheng_Misra_etal._2022,Xie_Wang_etal._2021}.

\begin{figure}[h!]
    \centering
    \includegraphics[width=\textwidth]{images/Figure_5.pdf}
    \caption{UNETR-like model with foundational model as backbone}
    \label{fig:fig5}
\end{figure}

By utilizing a transformer-based encoder, we incorporate global contextual information into the feature extraction process, which is a key advantage of such architectures \cite{Chen_Lu_etal._2021}. This foundation model integration facilitates accurate pixel-wise segmentation and classification without the need for extensive encoder training, thereby potentially improving generalization across varied cellular structures and tissue types.
In our implementation, we employ a modified UNETR \cite{Hatamizadeh_Tang_etal._2021} architecture that combines a vision transformer (ViT) \cite{Dosovitskiy_Beyer_etal._2021} encoder with a CNN-based decoder. The encoder utilizes the pretrained H-Optimus foundation model, which contains 1.1 billion parameters and is trained on over 500,000 H\&E stained WSIs \cite{Saillard_Jenatton_etal._2024}. We extract outputs from four evenly spaced transformer blocks $Z_i$, where $i \in [1, 14, 26, 38]$, to serve as residual connections for the CNN decoder. We select these blocks based on our observation that features from non-adjacent levels of the encoder lead to better overall performance on the test subset.

The CNN decoder upsamples the feature representations, acquired from the transformer blocks, to generate an intermediate vector that is handled by two task-specific layers that generate cell segmentation and classification masks. The first task-specific layer is the ‘Cellpose head’,  which is used to delineate cell instances. The layer generates horizontal and vertical gradient maps to form vector fields that are refined through gradient tracking in a post-processing step using the Cellpose algorithm \cite{Stringer_Wang_etal._2021}, known for its efficacy in cell segmentation tasks and generalizability across multiple domains \cite{Pachitariu_Stringer_2022,Stringer_Pachitariu_2024}. The second task-specific layer is the "Cell type head", which assigns labels to individual pixels. In the post-processing step, we determine the output classification label of each segmented cell instance by majority voting over the labeled pixels that comprise the cell in the segmentation map.

To evaluate model performance and measure the impact of adding a foundation model as backbone, we compare it to a ResNet50-based model. ResNet50 is a widely used solution for encoders in segmentation architectures in the medical domain \cite{Deshmane_2023,Graham_Vu_etal._2019,Mukasheva_Koishiyeva_etal._2024,Stringer_Wang_etal._2021}. For the H-Optimus-based model, we utilize frozen weights for the encoder and only fine-tune the decoder to take advantage of the extensive pre-training of the foundation model. For the ResNet50-based model we start with ImageNet \cite{Deng_Dong_etal.} weights and train both encoder and decoder parts. Hyperparameters for the training step are set to be identical, where possible, for comparable evaluation. 
For this evaluation, we deliberately use the PanNuke dataset to provide a standardized and controlled comparison between the H‑Optimus and ResNet50-based models (\hyperref[fig:S2]{Appendix Figure S2 (3)}). Specifically, we use two of the default PanNuke dataset splits (66\%) for training and validation, and reserve the third split (33\%) for testing.

To address the challenge of cell class imbalance in the PanNuke dataset, which is a common characteristic in most cell-level H\&E patch datasets, both models’ training processes employ a weighted loss function comprising cross-entropy and focal loss \cite{Lin_Goyal_etal._2018}. The focal loss component is adjusted with coefficients derived from each cell class' instance frequency, emphasizing learning from underrepresented classes and enhancing the model's sensitivity to rare but significant cellular patterns. The cross-entropy loss is augmented with spectral decoupling regularization \cite{Pezeshki_Kaba_etal._2021,Pohjonen_Stürenberg_etal._2022} and spatially varying label smoothing \cite{Islam_Glocker_2021}, which potentially stabilizes training and improves generalization in case of complex tissue morphologies. For optimization, we employ the \textit{AdamW} \cite{Loshchilov_Hutter_2019} to counter unbalanced class scenarios, with cosine annealing learning rate scheduler.

We utilize the scikit-learn library \cite{Van_der_Walt_Schönberger_etal._2014} and HoVer-Net \cite{Graham_Vu_etal._2019} implementations of $R^2$ (the coefficient of determination) and $PQ$ (panoptic quality) to evaluate our experiments. Complete mathematical formulations and detailed explanations of these metrics are provided in \hyperref[chap:S5]{Appendix S5}. To compute confidence intervals, we use nonparametric bootstrapping, where after calculating the metric on the full sample, we generated 1000 bootstrap replicates by resampling with replacement and then determined the 95\% confidence intervals as the 2.5th and 97.5th percentiles of the resulting empirical distribution.

%\hfill

The model comparisons are summarized in \hyperref[tab:2]{Table 2}. The H‑Optimus-based model achieves higher $R^2$ across all cell classes compared to the ResNet50-based model, which means that its predictions are more closely aligned with the PanNuke cell counts, indicating a stronger correlation with the observed data. Notably, the improvement of $R^2_{dead}$ may be an indicator of better global contextual representations provided by the foundation model backbone. In terms of segmentation and classification quality combined, measured by the PQ score, the H‑Optimus-based model demonstrates notable improvements across most cell classes. Overall, the average $R^2$ improved from 0.575 to 0.871, while the average $PQ$ score improved from 0.450 to 0.492, demonstrating better performance of the H-Optimus-based model.

\begin{table}[h!]
\renewcommand{\arraystretch}{1.5}
  \centering
  \caption{Cell quantification metrics for baseline and proposed models (CI 95\%).}
  \label{tab:2}
  \begin{tabular}{|l|c|c|}
    \hline
    %\rowcolor{gray!30}
    Metric             & Resnet50-based            & H-optimus-based              \\
    \hline
    $R^2_{neoplastic}$    & 0.681 (0.576--0.769)       & \textbf{0.941 (0.917--0.960)} \\
    \hline
    $R^2_{inflammatory}$  & 0.863 (0.778--0.903)       & \textbf{0.949 (0.918--0.966)} \\
    \hline
    $R^2_{connective}$    & 0.600 (0.488--0.698)       & 0.609 (0.436--0.772)          \\
    \hline
    $R^2_{dead}$          & 0.097 (-11.389--0.669)     & 0.925 (0.404--0.982)          \\
    \hline
    $R^2_{epithelial}$    & 0.635 (0.490--0.747)       & \textbf{0.930 (0.886--0.964)} \\
    \hline
    $PQ_{neoplastic}$       & 0.517 (0.499--0.535)       & \textbf{0.589 (0.575--0.604)} \\
    \hline
    $PQ_{inflammatory}$     & 0.455 (0.429--0.482)       & \textbf{0.528 (0.507--0.549)} \\
    \hline
    $PQ_{connective}$       & 0.416 (0.400--0.431)       & \textbf{0.451 (0.436--0.465)} \\
    \hline
    $PQ_{dead}$             & 0.374 (0.342--0.408)       & 0.292 (0.209--0.365)          \\
    \hline
    $PQ_{epithelial}$       & 0.488 (0.460--0.519)       & \textbf{0.599 (0.579--0.618)} \\
    \hline
  \end{tabular}
\end{table}

Our results  show that integrating the H‑Optimus foundation model within the UNETR architecture enhances the model's ability to segment and classify cells across diverse tissues from PanNuke data. The pretrained transformer encoder provides robust feature representations, resulting in higher average $R^2$ and $PQ$ scores compared to the CNN-based model. This leads to more reliable cell quantification and more accurate downstream analysis. Additionally, the streamlined fine-tuning process reduces computational overhead and training time, making the model more adaptable for new data.

Despite these advancements, the foundation model-based approach does not fully resolve all challenges related to cell segmentation and classification. We observe lower metric scores for underrepresented classes in the training data. Furthermore, foundation models typically encompass billions of parameters, resulting in substantial computational and memory requirements. It therefore poses challenges for deployment in resource-constrained environments, limiting their practical applicability in certain clinical settings.

\section{Model optimization via Knowledge Distillation}

To address the limitations posed by the extensive size of foundation models, we implement knowledge distillation — a model compression technique that leverages the teacher-student paradigm \cite{Hinton_Vinyals_etal._2015}. By training a smaller, more efficient student model to replicate the output of a larger, pre-trained teacher model, we retain performance while significantly reducing the model's complexity and resource requirements (\hyperref[fig:fig6]{Figure 6}).

\begin{figure}[h!]
    \centering
    \includegraphics[width=\textwidth, height=0.45\textheight, keepaspectratio]{images/Figure_6.pdf}
    \caption{Knowledge distillation framework for training a student model using a pre-trained teacher}
    \label{fig:fig6}
\end{figure}

We employ knowledge distillation to compress the H‑Optimus-based teacher model into a more efficient student model. The teacher model is the modified UNETR architecture with the H‑Optimus foundation model described in the previous chapter. The student model is based on a UNet architecture augmented with residual connections and incorporates a smaller ViT encoder with 9 million parameters \cite{Steiner_Kolesnikov_etal._2022,Wightman_2019}. 

First, we fine-tune the teacher model using the refined dataset from the cross-relabeling procedure (Section 2). Initially we train the decoder of the teacher model while keeping the encoder weights frozen. We split the refined dataset into train (70\%), validation (20\%) and test (10\%) subsets (\hyperref[fig:S2]{Appendix Figure S2 (4)}). During fine-tuning, we use the train and validation subsets, while leaving the test subset for model evaluation. We set the training procedure and model hyperparameters to be identical to those that were used to demonstrate the utility of foundation models for the simultaneous cell segmentation and classification task.

Next, we perform knowledge distillation from teacher to student using the refined dataset used to fine-tune the teacher model. The student model is trained to replicate the teacher model's outputs. We utilize a specialized loss function that aligns the student's predicted probability distribution with the teacher's, incorporating the teacher's class probability distribution derived from the output. Following the methodology of Hinton et al. \cite{Hinton_Vinyals_etal._2015}, we experiment with various hyperparameter settings for the temperature ($T$) and the balancing coefficients ($\alpha$ and $\beta$) in the loss function. We vary $T$ from 1 to 20 and adjust $\alpha$ and $\beta$ to balance the distillation and student losses. Through iterative tuning and evaluation, we identify that setting $T=14$, $\alpha=0.3$, and $\beta=0.7$ yields a configuration that converges and closely approximates the teacher model's performance during training.

Finally, we assess the performance of both models using the $R^2$ and $PQ$ (defined in \hyperref[chap:S5]{Appendix S5}) on the test set of the refined dataset (\hyperref[tab:3]{Table 3}). We observe that the 95\% confidence intervals overlap for most cell types, so we cannot claim statistically significant performance differences between the teacher and student models. One exception appears in the neoplastic class. The teacher model produces an $R^2$ of 0.919, while the student model shows an $R^2$ of 0.852. In addition, the student model achieves higher $PQ$ values for the neoplastic and connective classes, though the confidence intervals show overlap.

\begin{table}[h!]
\renewcommand{\arraystretch}{1.5}
  \centering
  \caption{Cell quantification metrics for teacher and distilled student models (CI 95\%).}
  \label{tab:3}
  \begin{tabular}{|l|c|c|}
    \hline
    %\rowcolor{gray!30}
    Metric & Teacher & Student \\
    \hline
    $R^2_{neoplastic}$    & \textbf{0.919} (0.898--0.939) & 0.852 (0.800--0.891) \\
    \hline
    $R^2_{lymphocyte}$    & 0.969 (0.956--0.977)         & 0.969 (0.956--0.978) \\
    \hline
    $R^2_{connective}$    & 0.694 (0.548--0.809)         & 0.618 (0.469--0.741) \\
    \hline
    $R^2_{dead}$          & 0.755 (0.400--0.908)         & 0.424 (0.100--0.731) \\
    \hline
    $R^2_{epithelial}$    & 0.922 (0.870--0.958)         & 0.843 (0.738--0.917) \\
    \hline
    $R^2_{macrophage}$    & 0.384 (-0.369--0.724)        & 0.704 (0.352--0.859) \\
    \hline
    $R^2_{neutrofil}$     & 0.854 (0.578--0.929)         & 0.833 (0.502--0.925) \\
    \hline
    $PQ_{neoplastic}$       & 0.581 (0.569--0.593)         & 0.601 (0.588--0.613) \\
    \hline
    $PQ_{lymphocyte}$       & 0.536 (0.520--0.553)         & 0.563 (0.544--0.579) \\
    \hline
    $PQ_{connective}$       & 0.436 (0.421--0.451)         & 0.457 (0.441--0.474) \\
    \hline
    $PQ_{dead}$             & 0.272 (0.235--0.315)         & 0.279 (0.201--0.369) \\
    \hline
    $PQ_{epithelial}$       & 0.522 (0.500--0.545)         & 0.530 (0.506--0.555) \\
    \hline
    $PQ_{macrophage}$       & 0.524 (0.459--0.588)         & 0.474 (0.405--0.543) \\
    \hline
    $PQ_{neutrofil}$        & 0.541 (0.490--0.592)         & 0.565 (0.522--0.607) \\
    \hline
  \end{tabular}
\end{table}


We further decompose the $PQ$ metric into its $SQ$ and $DQ$ components (\hyperref[tab:S6]{Appendix Table S6}). Both models produce nearly identical $SQ$ values, which indicates that they predict instance boundaries with similar precision. Although the student model shows some improvement in $DQ$ scores for certain classes, the confidence intervals overlap and do not confirm a statistically significant difference.

We observe that the student and teacher models yield comparable detection performance despite the student model using a much smaller and simpler architecture. A model with fewer parameters reduces the risk of overfitting when training data are scarce relative to the model’s complexity \cite{Farias_Ludermir_etal._2022}. The knowledge distillation process also encourages the student model to focus on the most generalizable detection features learned from the teacher. These factors enable the student model to achieve similar detection performance across different cell types.

Additionally, considering the model sizes reported in \hyperref[tab:4]{Table 4}, the distilled model achieves a significant reduction compared to the teacher model, with a 48-fold decrease in parameter count and a 5.5-fold reduction in on-disk size. In inference mode, the teacher model requires 16 GB of VRAM for a batch size of 32, while the distilled model only needs 3 GB of VRAM for the same batch size. These reductions make the distilled model significantly more practical for fine-tuning and deployment in resource-constrained environments.

\begin{table}[h!]
\renewcommand{\arraystretch}{1.5}
  \centering
  \caption{Parameter counts and size of teacher and distilled model}
  \label{tab:4}
  \adjustbox{max width=\textwidth}{%
  \begin{tabular}{|l|c|c|c|}
    \hline
    %\rowcolor{gray!30}
    Metric & H-optimus-based (Teacher) & mobileViT-based (Student) & Magnitude of difference \\
    \hline
    Parameters count       & 1,158,917,906   & \textbf{24,093,393}   & \textbf{48x}  \\
    \hline
    Estimated Total Size (MB) & 87,912       & \textbf{15,935}    & \textbf{5.5x} \\
    \hline
  \end{tabular}%
}
\end{table}

%\hfill

With recent advancements in complex network architectures and the use of pretrained encoders to achieve state-of-the-art performance \cite{Baumann_Dislich_etal._2024,Hörst_Rempe_etal._2024} in cell segmentation and classification tasks, model size, computational complexity, and processing times have increased. This limits the scalability and accessibility of these models. As we demonstrate, this may be mitigated using knowledge distillation. Studies in the field of natural language processing have demonstrated the efficacy of knowledge distillation in retaining the capabilities of the teacher model while achieving significant reductions in size and complexity \cite{Huangpu_Gao_2024,Sun_Yu_etal.}. 

We demonstrate the feasibility of knowledge distillation in digital pathology, specifically for cell segmentation and classification tasks. Moreover, we achieve this performance while also significantly reducing the parameter count. In addressing the challenge of knowledge transfer, we found that distillation from a transformer-based model to a smaller transformer is more straightforward than attempting to map transformer features to CNN blocks. In our experiments, using a CNN-based network as a student results in worse cell quantification performance due to the structural constraints of CNN feature space dimensions. 

Although our primary approach relies on a transformer-based student model that performs well, it can be further optimized to incorporate advantages from CNN architectures. For example, employing alternative techniques such as using ViT adapters \cite{Chen_Duan_etal._2023} or $1 \times 1$ convolutions to adjust feature map sizes may be beneficial for harnessing CNN advantages like enhanced local feature extraction. Moreover, if additional performance improvements are desired, the process can be further enhanced by applying supplementary knowledge distillation techniques, such as self-distillation \cite{Zhang_Song_etal._2019} or online distillation \cite{Houyon_Cioppa_etal._2023}.

Despite these promising results, further validation on independent datasets is necessary to fully understand the model's limitations. Underrepresented classes may pose challenges when addressing complex cases. Pathologists need to validate these models to adopt them in clinical settings. While the distilled models are smaller and more deployable, a technological gap persists because pathologists traditionally rely on established methods for inspecting WSIs and diagnosing diseases. Addressing the complexities involved in deploying models for inference and supporting pathologists in adopting new tools is essential for integrating these models into clinical workflows.

\section{Model integration with QuPath}
Digital pathology tools with graphical user interfaces are essential for visualizing and analyzing WSIs. To make our student model useful in clinical pathology workflows, it needs to be integrated into a tool that enables inspecting regions, creating annotations, and providing quantitative analyses of biomarkers. Therefore, we integrate the trained student model from the previous chapter into the QuPath open‑source platform \cite{Bankhead_Loughrey_etal._2017}. QuPath provides the required annotation, visualization, and analysis tools to interpret complex histological data, including workflows for cell segmentation, classification, and quantification (\hyperref[fig:fig7]{Figure 7}). 

\begin{figure}[h!]
    \centering
    \includegraphics[width=\textwidth]{images/Figure_7.pdf}
    \caption{Visualization of model-generated cell quantification annotations (left) and the corresponding unannotated slide (right) in QuPath}
    \label{fig:fig7}
\end{figure}

To identify the regions in a WSI critical for prognosticating tumor development, such as specific tumor areas or border regions without overlapping healthy tissue, the pathologist uses QuPath to outline these regions. Then, the pathologist initiates a cell segmentation and classification script through the QuPath interface for the selected regions. The resulting annotations and quantified cell information are then directly overlaid onto the WSI in the QuPath interface. Additional design and implementation details are in \hyperref[chap:S7]{Appendix S7}. 

Two common approaches for integrating deep learning models into QuPath are Java‑based native QuPath extensions \cite{Goldsborough_Philps_etal._2024} and the execution of RESTful API requests to a model server coupled with handling the response via an extension, as demonstrated in the application of cell segmentation models applied to immunofluorescence images \cite{Sugawara_2023}. While the community is actively working on these integration strategies, there is currently no universal solution that fully addresses all integration and performance requirements.

Extensions may offer better integration with QuPath, allowing slightly improved performance and more widespread usage of the built-in QuPath models, but they lack the flexibility to customize models and modify their behavior. For example, the newest version of QuPath includes models such as StarDist \cite{Weigert_Schmidt} and InstanSeg \cite{Goldsborough_Philps_etal._2024} that can perform cell segmentation. Both models pose limitations when applied to simultaneous cell segmentation and classification. StarDist performs well only on convex, round shapes by design, whereas some neoplastic, inflammatory, and connective cells exhibit complex and non-convex shapes. InstanSeg provides only semantic segmentation without assigning classes to the segmented cells.

%\hfill

In contrast, our approach offers an alternative integration strategy. It utilizes the paquo library to directly interact with QuPath’s internal application programming interface from within Python. This enables data exchange and processing without the need for intermediate conversion steps and provides greater control over model customization, retraining, and the incorporation of custom processing steps.

The integration of our custom model with QuPath underscores its potential to significantly enhance the diagnostic process by reducing the time burden on pathologists and enabling them to focus on more complex interpretative tasks using familiar software. Leveraging a tool that is already well-established among pathologists increases the likelihood of its adoption into daily clinical workflows. The quantitative data generated through the automated workflow is critical for both clinical decision-making and research, facilitating more accurate biomarker analysis, enabling robust statistical evaluations, and supporting hypothesis generation and testing. Additionally, by streamlining cell segmentation and classification, the tool enhances the scalability and reproducibility of pathological assessments, ultimately contributing to improved diagnostic accuracy and patient outcomes.

\section{Conclusion and future work}

In this study, we address critical challenges in digital pathology and tackle the usability and deployment issues of the developed models in standard computing environments without the need for high-performance computing systems. Our multi-faceted approach encompasses data refinement through cross-relabeling, leveraging foundation models for robust cell segmentation and classification, optimizing model performance via knowledge distillation, and integrating the optimized model into the QuPath software for practical application. This approach is used to construct a capable, versatile, and adjustable model for cell segmentation and classification, with enhanced performance and usability.

\begin{sloppypar}
While our approach shows potential in the field of computational pathology, certain limitations persist. 
For example, our implementation currently exhibits lower performance in detecting macrophages. 
This serves as an instance of the broader challenge of accurately identifying complex cell types. In order to address this issue, extending our approach to incorporate additional data sources, exploring alternative modeling approaches, and integrating other imaging modalities such as immunohistochemical staining may help improve detection accuracy. Moreover, although the distilled model reduces computational demands, integrating advanced deep learning models into clinical practice requires addressing technological gaps and potential resistance to adopting new tools within established diagnostic processes.
\end{sloppypar}

Future work could focus on several key areas to refine the proposed approach and facilitate its adoption in clinical environments. Enhancing the cell-relabeling process with additional datasets \cite{Graham_Jahanifar_etal._2021} could improve the representation of underrepresented cell types and enhance overall model performance. Also, incorporating additional data sources, such as multi-modal imaging or complementary staining methods, may address limitations related to cell type differentiation and class imbalance. Exploring other foundation models \cite{Vorontsov_Bozkurt_etal._2024,Zimmermann_Vorontsov_etal._2024} or introducing additional modalities \cite{Ding_Wagner_etal._2024,Vaidya_Zhang_etal._2025} may provide alternative architectures better suited to specific tasks or offer improved efficiency. Implementing more complex knowledge distillation techniques \cite{Houyon_Cioppa_etal._2023,Zhang_Song_etal._2019} could further optimize the model's performance and adaptability. Additionally, deeper integration with QuPath or other digital pathology software could provide pathologists more control over cell quantification analysis directly within the QuPath interface, thereby increasing accessibility and usability. Such enhancements would not only refine model performance but also ensure greater adaptability and scalability within various clinical environments. Finally, extensive validation of the model by pathologists and benchmarking against independent datasets are essential steps toward establishing the model's reliability and fostering confidence in its clinical utility.

\section*{Acknowledgments} 
This work was funded in part by the Research Council of Norway grant no. 309439 SFI Visual Intelligence, and the North Norwegian Health Authority grant no. HNF1521-20.

\bibliographystyle{IEEEtran}
\begin{sloppypar}
\begin{thebibliography}{99}

\bibitem{chaplot2020neural} Chaplot, Devendra Singh, et al. "Neural topological slam for visual navigation." Proceedings of the IEEE/CVF conference on computer vision and pattern recognition. 2020.

\bibitem{maksymets2021thda} Maksymets, Oleksandr, et al. "Thda: Treasure hunt data augmentation for semantic navigation." Proceedings of the IEEE/CVF International Conference on Computer Vision. 2021.

\bibitem{mezghan2022memory} Mezghan, Lina, et al. "Memory-augmented reinforcement learning for image-goal navigation." 2022 IEEE/RSJ International Conference on Intelligent Robots and Systems (IROS). IEEE, 2022.

\bibitem{al2022zero} Al-Halah, Ziad, Santhosh Kumar Ramakrishnan, and Kristen Grauman. "Zero experience required: Plug \& play modular transfer learning for semantic visual navigation." Proceedings of the IEEE/CVF Conference on Computer Vision and Pattern Recognition. 2022.

\bibitem{ye2021auxiliary} Ye, Joel, et al. "Auxiliary tasks and exploration enable objectgoal navigation." Proceedings of the IEEE/CVF international conference on computer vision. 2021.

\bibitem{chaplot2020object} Chaplot, Devendra Singh, et al. "Object goal navigation using goal-oriented semantic exploration." Advances in Neural Information Processing Systems 33 (2020)

\bibitem{ramakrishnan2022poni} Ramakrishnan, Santhosh Kumar, et al. "Poni: Potential functions for objectgoal navigation with interaction-free learning." Proceedings of the IEEE/CVF Conference on Computer Vision and Pattern Recognition. 2022.

\bibitem{ramrakhya2022habitat} Ramrakhya, Ram, et al. "Habitat-web: Learning embodied object-search strategies from human demonstrations at scale." Proceedings of the IEEE/CVF Conference on Computer Vision and Pattern Recognition. 2022.

\bibitem{mousavian2019visual} Mousavian, Arsalan, et al. "Visual representations for semantic target driven navigation." 2019 International Conference on Robotics and Automation (ICRA). IEEE, 2019.

\bibitem{dhariwal2021diffusion} Dhariwal, Prafulla, and Alexander Nichol. "Diffusion models beat gans on image synthesis." Advances in neural information processing systems 34 (2021)

\bibitem{ho2022classifier} Ho, Jonathan, and Tim Salimans. "Classifier-free diffusion guidance." arXiv preprint arXiv:2207.12598 (2022).

\bibitem{nichol2021glide} Nichol, Alex, et al. "Glide: Towards photorealistic image generation and editing with text-guided diffusion models." arXiv preprint arXiv:2112.10741 (2021)

\bibitem{brooks2023instructpix2pix} Brooks, Tim, Aleksander Holynski, and Alexei A. Efros. "Instructpix2pix: Learning to follow image editing instructions." Proceedings of the IEEE/CVF Conference on Computer Vision and Pattern Recognition. 2023.

\bibitem{fu2023guiding} Fu, Tsu-Jui, et al. "Guiding instruction-based image editing via multimodal large language models." arXiv preprint arXiv:2309.17102 (2023).

\bibitem{geng2024instructdiffusion} Geng, Zigang, et al. "Instructdiffusion: A generalist modeling interface for vision tasks." Proceedings of the IEEE/CVF Conference on Computer Vision and Pattern Recognition. 2024.

\bibitem{zhou2024minedreamer} Zhou, Enshen, et al. "Minedreamer: Learning to follow instructions via chain-of-imagination for simulated-world control." arXiv preprint arXiv:2403.12037 (2024).

\bibitem{zhou2023esc} Zhou, Kaiwen, et al. "Esc: Exploration with soft commonsense constraints for zero-shot object navigation." International Conference on Machine Learning. PMLR, 2023.

\bibitem{yu2023l3mvn} Yu, Bangguo, Hamidreza Kasaei, and Ming Cao. "L3mvn: Leveraging large language models for visual target navigation." 2023 IEEE/RSJ International Conference on Intelligent Robots and Systems (IROS). IEEE, 2023.

\bibitem{gadre2023cows} Gadre, Samir Yitzhak, et al. "Cows on pasture: Baselines and benchmarks for language-driven zero-shot object navigation." Proceedings of the IEEE/CVF Conference on Computer Vision and Pattern Recognition. 2023.

\bibitem{shah2023navigation} Shah, Dhruv, et al. "Navigation with large language models: Semantic guesswork as a heuristic for planning." Conference on Robot Learning. PMLR, 2023.

\bibitem{cai2024bridging} Cai, Wenzhe, et al. "Bridging zero-shot object navigation and foundation models through pixel-guided navigation skill." 2024 IEEE International Conference on Robotics and Automation (ICRA). IEEE, 2024.

\bibitem{yu2023co} Yu, Bangguo, Hamidreza Kasaei, and Ming Cao. "Co-NavGPT: Multi-robot cooperative visual semantic navigation using large language models." arXiv preprint arXiv:2310.07937 (2023).

\bibitem{wu2024voronav} Wu, Pengying, et al. "Voronav: Voronoi-based zero-shot object navigation with large language model." arXiv preprint arXiv:2401.02695 (2024).

\bibitem{qin2023mp5} Qin, Yiran, et al. "Mp5: A multi-modal open-ended embodied system in minecraft via active perception." arXiv preprint arXiv:2312.07472 (2023).

\bibitem{du2024learning} Du, Yilun, et al. "Learning universal policies via text-guided video generation." Advances in Neural Information Processing Systems 36 (2024).

\bibitem{ajay2024compositional} Ajay, Anurag, et al. "Compositional foundation models for hierarchical planning." Advances in Neural Information Processing Systems 36 (2024).

\bibitem{liang2024skilldiffuser} Liang, Zhixuan, et al. "Skilldiffuser: Interpretable hierarchical planning via skill abstractions in diffusion-based task execution." Proceedings of the IEEE/CVF Conference on Computer Vision and Pattern Recognition. 2024.

\bibitem{heusel2017gans} Heusel, Martin, et al. "Gans trained by a two time-scale update rule converge to a local nash equilibrium." Advances in neural information processing systems 30 (2017).

\bibitem{zhang2018unreasonable} Zhang, Richard, et al. "The unreasonable effectiveness of deep features as a perceptual metric." Proceedings of the IEEE conference on computer vision and pattern recognition. 2018.

\bibitem{brown2020language} Brown, Tom B. "Language models are few-shot learners." arXiv preprint arXiv:2005.14165 (2020).

\bibitem{podell2023sdxl} Podell, Dustin, et al. "Sdxl: Improving latent diffusion models for high-resolution image synthesis." arXiv preprint arXiv:2307.01952 (2023).

\bibitem{brohan2022rt} Brohan, Anthony, et al. "Rt-1: Robotics transformer for real-world control at scale." arXiv preprint arXiv:2212.06817 (2022).

\bibitem{brohan2023rt} Brohan, Anthony, et al. "Rt-2: Vision-language-action models transfer web knowledge to robotic control." arXiv preprint arXiv:2307.15818 (2023).

\bibitem{li2024manipllm} Li, Xiaoqi, et al. "Manipllm: Embodied multimodal large language model for object-centric robotic manipulation." Proceedings of the IEEE/CVF Conference on Computer Vision and Pattern Recognition. 2024.

\bibitem{shah2023vint} Shah, Dhruv, et al. "ViNT: A foundation model for visual navigation." arXiv preprint arXiv:2306.14846 (2023).

\bibitem{liu2024visual} Liu, Haotian, et al. "Visual instruction tuning." Advances in neural information processing systems 36 (2024).

\bibitem{hu2021lora} Hu, Edward J., et al. "Lora: Low-rank adaptation of large language models." arXiv preprint arXiv:2106.09685 (2021).

\bibitem{qin2023supfusion} Qin, Yiran, et al. "SupFusion: Supervised LiDAR-camera fusion for 3D object detection." Proceedings of the IEEE/CVF International Conference on Computer Vision. 2023.

\bibitem{qin2024worldsimbench} Qin, Yiran, et al. "Worldsimbench: Towards video generation models as world simulators." arXiv preprint arXiv:2410.18072 (2024).

\bibitem{yu2025gamefactory} Yu, Jiwen, et al. "GameFactory: Creating New Games with Generative Interactive Videos." arXiv preprint arXiv:2501.08325 (2025).

\bibitem{zhou2024code} Zhou, Enshen, et al. "Code-as-Monitor: Constraint-aware Visual Programming for Reactive and Proactive Robotic Failure Detection." arXiv preprint arXiv:2412.04455 (2024).

\bibitem{zhang2024ad} Zhang, Zaibin, et al. "AD-H: Autonomous Driving with Hierarchical Agents." arXiv preprint arXiv:2406.03474 (2024).

\bibitem{wang2024toward} Wang, Chaoqun, et al. "Toward Accurate Camera-based 3D Object Detection via Cascade Depth Estimation and Calibration." arXiv preprint arXiv:2402.04883 (2024).

\bibitem{huang2024story3d} Huang, Yuzhou, et al. "Story3d-agent: Exploring 3d storytelling visualization with large language models." arXiv preprint arXiv:2408.11801 (2024).

\bibitem{savinov2018semi} Savinov, Nikolay, Alexey Dosovitskiy, and Vladlen Koltun. "Semi-parametric topological memory for navigation." arXiv preprint arXiv:1803.00653 (2018).

\bibitem{majumdar2022zson} Majumdar, Arjun, et al. "Zson: Zero-shot object-goal navigation using multimodal goal embeddings." Advances in Neural Information Processing Systems 35 (2022): 32340-32352.

\bibitem{yadav2023offline} Yadav, Karmesh, et al. "Offline visual representation learning for embodied navigation." Workshop on Reincarnating Reinforcement Learning at ICLR 2023. 2023.

\bibitem{yadav2023ovrl} Yadav, Karmesh, et al. "Ovrl-v2: A simple state-of-art baseline for imagenav and objectnav." arXiv preprint arXiv:2303.07798 (2023).

\bibitem{sun2024fgprompt} Sun, Xinyu, et al. "FGPrompt: fine-grained goal prompting for image-goal navigation." Advances in Neural Information Processing Systems 36 (2024).

\bibitem{zhu2017target} Zhu, Yuke, et al. "Target-driven visual navigation in indoor scenes using deep reinforcement learning." 2017 IEEE international conference on robotics and automation (ICRA). IEEE, 2017.

\bibitem{koh2024generating} Koh, Jing Yu, Daniel Fried, and Russ R. Salakhutdinov. "Generating images with multimodal language models." Advances in Neural Information Processing Systems 36 (2024).

\bibitem{krantz2022instance} Krantz, Jacob, et al. "Instance-specific image goal navigation: Training embodied agents to find object instances." arXiv preprint arXiv:2211.15876 (2022).

\bibitem{schulman2017proximal} Schulman, John, et al. "Proximal policy optimization algorithms." arXiv preprint arXiv:1707.06347 (2017).

\bibitem{anderson2018evaluation} Anderson, Peter, et al. "On evaluation of embodied navigation agents." arXiv preprint arXiv:1807.06757 (2018).

\bibitem{lin2024navcot} Lin, Bingqian, et al. "NavCoT: Boosting LLM-Based Vision-and-Language Navigation via Learning Disentangled Reasoning." arXiv preprint arXiv:2403.07376 (2024).

\bibitem{NavGPT} Zhou, Gengze, Yicong Hong, and Qi Wu. "Navgpt: Explicit reasoning in vision-and-language navigation with large language models." Proceedings of the AAAI Conference on Artificial Intelligence.

\bibitem{hahn2021no} Hahn, Meera, et al. "No rl, no simulation: Learning to navigate without navigating." Advances in Neural Information Processing Systems 34 (2021): 26661-26673.

\bibitem{li2025t2isafety} Li, Lijun, et al. "T2ISafety: Benchmark for Assessing Fairness, Toxicity, and Privacy in Image Generation." arXiv preprint arXiv:2501.12612 (2025).

\bibitem{an2024agfsync} An, Jingkun, et al. "AGFSync: Leveraging AI-Generated Feedback for Preference Optimization in Text-to-Image Generation." arXiv preprint arXiv:2403.13352 (2024).


\end{thebibliography}
\end{sloppypar}

\clearpage
\beginsupplement
\section*{Appendix}
\renewcommand{\thesubsection}{S\arabic{subsection}}

\subsection{\label{chap:S1}PanNuke and MoNuSAC preprocessing}
The PanNuke dataset comprises a set of 7,901 RGB patches, each with dimensions of $256 \times 256$ pixels, which we set as the standard patch size for our analysis. In contrast, the MoNuSAC dataset encompasses 294 images of heterogeneous dimensions. To standardize the MoNuSAC images with our experiments, we implement a standardization protocol. Specifically, for images exceeding the dimensions of $256 \times 256$ pixels, we segment them into equal-sized patches and apply mirror padding to the remaining portions to avoid information loss at the peripherals. Patches with dimensions less than $128 \times 128$ pixels are excluded from the dataset due to the insufficient resolution to capture relevant cellular details. For patches where either dimension falls between 128 and 256 pixels, we employ upsampling to achieve the standard patch size. As a result, we obtain a total of 2,823 RGB patches derived from the MoNuSAC dataset for subsequent analysis. For additional details on the MoNuSAC data preparation process, refer to the source code \cite{Shvetsov_2025a}.
\clearpage

\subsection{\label{chap:S2}Data usage for the methodology}

\counterwithin{figure}{subsection}
\renewcommand{\thefigure}{S\arabic{subsection}}

\begin{figure}[h!]
    \centering
    \includegraphics[width=\textwidth, height=0.85\textheight, keepaspectratio]{images/A2.pdf}
    \caption{Overview of the methodology for cross-labeling, dataset refinement, and model comparison. (1) Cross-relabeling - training and testing cell classification models, (2) Cross-relabeling - using cell classification models to create refined dataset, (3) Fine-tuning and training models for comparison, (4) Student knowledge distillation with refined dataset}
    \label{fig:S2}
\end{figure}
\clearpage

\subsection{\label{chap:S3}Confusion matrices for classification models}
\counterwithin{figure}{subsection}
\renewcommand{\thefigure}{S\arabic{subsection}.\arabic{figure}}

\begin{figure}[h!]
    \centering
    \includegraphics[width=\textwidth, height=0.4\textheight, keepaspectratio]{images/A3_1.pdf}
    \caption{Confusion matrix for PanNuke trained model}
    \label{fig:S3.1}
\end{figure}

\begin{figure}[h!]
    \centering
    \includegraphics[width=\textwidth, height=0.4\textheight, keepaspectratio]{images/A3_2.pdf}
    \caption{Confusion matrix for MoNuSAC trained model}
    \label{fig:S3.2}
\end{figure}

\clearpage

\subsection{\label{chap:S4}Datasets cell counts}

\counterwithin{table}{subsection}
\renewcommand{\thetable}{S\arabic{subsection}}

\begin{table}[h!]
\renewcommand{\arraystretch}{2.0}
\centering
\caption{\label{tab:S4}Cell counts for PanNuke, MoNuSAC and refined datasets. Numbers in parentheses indicate preprocessed cell counts for cell classifier models training and testing.}
%\adjustbox{max width=\textwidth}{%
\begin{tabular}{|l|c|c|c|}
\hline
%\rowcolor{gray!30}
Cell type & PanNuke & MoNuSAC & Refined \\
\hline
Neoplastic & 77,403 (68,031) & - & 105,451 \\
\hline
Epithelial & 26,572 (23,207) & - & 29,926 \\
\hline
Epithelial (benign and malignant) & - & 31,402 & - \\
\hline
Inflammatory & 32,276 & - & - \\
\hline
Lymphocytes & - & 37,045 (33,104) & 65,275 \\
\hline
Neutrophils & - & 1,355 (1,252) & 3,833 \\
\hline
Macrophage & - & 1,842 (1,695) & 3,410 \\
\hline
Dead & 2,908 & - & 2,908 \\
\hline
Connective & 50,585 & - & 50,585 \\
\hline
\end{tabular}
%
%}
\end{table}



\clearpage

\subsection{\label{chap:S5}Definition of validation metrics}
\counterwithin{equation}{subsection}
\renewcommand{\theequation}{\arabic{equation}}

\subsubsection{\label{chap:S5.1}R\textsuperscript{2}}
The coefficient of determination, denoted as $R^2$, is a statistical measure that represents the proportion of variance in the dependent variable that is predictable from the independent variables. In the context of cell quantification in pathology, $R^2$ is used to assess how well the predicted quantities of different cell types in a patch align with the actual quantities observed in the ground truth data, with higher values representing more accurate quantification. $R^2$ is defined as
\begin{equation*}
R^2 = 1 - \frac{\sum_{i=1}^n (y_i - \hat{y}_i)^2}{\sum_{i=1}^n (y_i - \bar{y})^2},
\end{equation*}
where $y_i$ represents the actual number of cells of a specific type in the $i$-th image, $\hat{y}_i$ represents the predicted number of cells of that type in the $i$-th image, $\bar{y}$ is the mean of the actual numbers across all images, and $n$ is the total number of images in the dataset.

The $R^2$ metric has a range of $(-\infty, 1]$. An $R^2$ of 1 indicates perfect prediction, where all predicted values exactly match the actual values. An $R^2$ of 0 suggests that the model explains none of the variability of the response data around its mean. If $R^2$ is negative, it indicates that the model performs worse than a model that simply predicts the mean of the actual values for all observations.

\subsubsection{\label{chap:S5.2}PQ}
Panoptic Quality ($PQ$) is a comprehensive metric used to evaluate the performance of segmentation models in tasks that require both instance segmentation and classification. $PQ$ provides a single score that encapsulates both the detection accuracy (i.e., how many objects were correctly identified) and the segmentation quality (i.e., how accurately the objects' boundaries were delineated). This metric is particularly useful in multiclass scenarios where each pixel is classified into distinct categories, such as different cell types in pathology images.

$PQ$ is calculated as the product of two terms: Detection Quality ($DQ$) and Segmentation Quality ($SQ$). It can be expressed as
\begin{equation*}
PQ = DQ \cdot SQ,
\end{equation*}
where
\begin{equation*}
DQ = \frac{TP}{TP + 0.5\, FP + 0.5\, FN},
\end{equation*}
\begin{equation*}
SQ = \frac{\sum_{(p, g) \in \mathcal{M}} IoU(p, g)}{TP}.
\end{equation*}
In these formulas, $TP$ denotes the number of correctly matched instances between ground truth and prediction, $FP$ denotes the predicted instances that have no corresponding ground truth, $FN$ denotes the ground truth instances that were not detected, $IoU(p, g)$ is the Intersection over Union for a pair of matched instances $p$ (prediction) and $g$ (ground truth), and $\mathcal{M}$ is the set of matched pairs.

The $PQ$ metric is calculated for each class and is averaged across classes to provide a global performance measure.

The $PQ$ score has a range of $[0, 1.0]$, where a higher score indicates better performance in both detecting and segmenting the instances correctly. A $PQ$ of 1 signifies perfect identification and segmentation of all instances, whereas a $PQ$ of 0 indicates that no instances were correctly identified and segmented.

\clearpage

\subsection{\label{chap:S6}Segmentation and Detection quality metrics for teacher and student models}

\begin{table}[h!]
\renewcommand{\arraystretch}{2.0}
\centering
\caption{Segmentation and detection quality for student and teacher models (CI 95\%)}
\label{tab:S6}
%\adjustbox{max width=\textwidth}{%
\begin{tabular}{|l|c|c|}
\hline
%\rowcolor{gray!30}
Metric & Teacher & Student \\
\hline
$SQ_{neoplastic}$ & 0.819 (0.815--0.823) & 0.824 (0.819--0.828) \\
\hline
$SQ_{lymphocyte}$ & 0.795 (0.788--0.802) & 0.790 (0.783--0.796) \\
\hline
$SQ_{connective}$ & 0.770 (0.762--0.776) & 0.780 (0.772--0.786) \\
\hline
$SQ_{dead}$ & 0.659 (0.623--0.688) & 0.657 (0.624--0.695) \\
\hline
$SQ_{epithelial}$ & 0.780 (0.770--0.790) & 0.788 (0.779--0.797) \\
\hline
$SQ_{macrophage}$ & 0.788 (0.760--0.810) & 0.757 (0.730--0.783) \\
\hline
$SQ_{neutrofil}$ & 0.782 (0.761--0.801) & 0.775 (0.759--0.792) \\
\hline
$DQ_{neoplastic}$ & 0.706 (0.692--0.719) & 0.727 (0.712--0.741) \\
\hline
$DQ_{lymphocyte}$ & 0.675 (0.656--0.698) & 0.713 (0.691--0.734) \\
\hline
$DQ_{connective}$ & 0.566 (0.546--0.584) & 0.583 (0.565--0.602) \\
\hline
$DQ_{dead}$ & 0.410 (0.361--0.465) & 0.435 (0.306--0.561) \\
\hline
$DQ_{epithelial}$ & 0.668 (0.639--0.694) & 0.673 (0.644--0.702) \\
\hline
$DQ_{macrophage}$ & 0.657 (0.583--0.727) & 0.615 (0.531--0.703) \\
\hline
$DQ_{neutrofil}$ & 0.691 (0.625--0.753) & 0.729 (0.679--0.778) \\
\hline
\end{tabular}
%
%}
\end{table}

\clearpage

\subsection{\label{chap:S7}QuPath integration method}
We adopt an integration strategy leveraging the paquo \cite{Bayer_AG} library, a Python package that enables direct interaction with QuPath’s internal API, thereby facilitating seamless data exchange without intermediate conversion steps. The data processing pipeline (\hyperref[fig:S7]{Appendix Figure S7}) begins with the acquisition of WSIs and their associated annotations from QuPath, which are represented as Shapely \cite{Gillies_Wel_etal._2024} polygons. Utilizing paquo, we directly read, create, and modify these annotations and detections within a QuPath project in the Python environment. Images are then cropped using these polygons and processed by cell segmentation and classification models employing standard vision processing toolkits such as OpenCV, pyvips, and PyTorch. Additionally, QuPath employs Groovy scripts to initiate a Python process that starts the entire pipeline from QuPath graphical interface: fetching polygons, extracting images from them, and running deep learning model inference on the cropped images. 
The results are returned to QuPath, leveraging paquo's Python bindings to manipulate QuPath data while minimizing the computational overhead typically associated with cross-environment communication.

\counterwithin{figure}{subsection}
\renewcommand{\thefigure}{S\arabic{subsection}}

\begin{figure}[h!]
    \centering
    \includegraphics[width=\textwidth]{images/A7.pdf}
    \caption{QuPath integration workflow using Python environment}
    \label{fig:S7}
\end{figure}

Compared to traditional workflows that involve exporting annotations as GeoJSON, classifying them in Python, and reimporting them into QuPath, our approach offers several advantages. We eliminate the need to switch between programming languages, providing a cohesive and streamlined development process entirely within QuPath software and removing the necessity to use other tools. Meanwhile, we avoid storing annotations as intermediate JSON files unless required for external use or archiving. By conducting the entire inference and post-processing workflow within the Python environment, we leverage the power and flexibility of Python libraries for image processing and machine learning. This approach also enables adjustments to any set of labels and models, thereby improving its applicability.

%\hfill

The distilled model and QuPath integration code are packaged into a Docker container, enabling streamlined execution with the Docker engine. Detailed integration code and deployment instructions can be found in the GitHub repository \cite{Shvetsov_2025b}.

Despite these benefits, we acknowledge that the paquo library is a proof‑of‑concept project in its early development stage and has not been tested across all versions of QuPath.

\clearpage

\subsection{\label{chap:S8}Data and code availability statement}
All datasets, models, and code used in this study are publicly available and can be obtained from the repositories listed below. 
The PanNuke \cite{Gamper_Koohbanani_etal._2019} and MoNuSAC \cite{Verma_Kumar_etal._2021} datasets are publicly accessible, and download information along with detailed descriptions can be found in their respective articles. Preprocessing scripts for PanNuke and MoNuSAC data, as well as individual cell extraction scripts, are available on GitHub \cite{Shvetsov_2025a}. The H-Optimus foundation model used in our experiments can be downloaded from the HuggingFace repository \cite{hoptimus2024}, and model information is available on GitHub \cite{Saillard_Jenatton_etal._2024}. In addition, the integration code for QuPath and the distilled model packaged in a Docker container are provided in the repository \cite{Shvetsov_2025b}, and paquo Python library is available from the authors GitHub repository \cite{Bayer_AG}.
\clearpage

\end{document}


\end{document}
