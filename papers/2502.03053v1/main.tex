
\documentclass[11pt]{article}
\usepackage[final]{acl}
\usepackage{times}
\usepackage{latexsym}
\usepackage{tabularx}
\usepackage[T1]{fontenc}
\usepackage[utf8]{inputenc}
\usepackage{microtype}
\usepackage{inconsolata}
\usepackage{graphicx}
\usepackage{enumitem} 
\graphicspath{ {./images/} }
\usepackage{float}
% \usepackage{emoji}
\usepackage{arydshln} \usepackage{stfloats} 
\usepackage{times} 

\newcommand{\nb}[3]{{\colorbox{#2}{\bfseries\sffamily\scriptsize\textcolor{white}{#1}~}}
{\textcolor{#2}{\sf\small\textit{#3}}}}


\title{DOLFIN - \underline{Do}cument-\underline{L}evel \underline{Fin}ancial test set for Machine Translation}


\author{Mariam Nakhlé$^{1,2}$, Marco Dinarelli$^1$, \textbf{Raheel Qader}$^2$, \\ 
\textbf{Emmanuelle Esperança-Rodier}$^1$,  \textbf{Hervé Blanchon}$^1$ \\ 
  (1) Univ. Grenoble Alpes, CNRS, Grenoble INP \footnote{Institute of Engineering Univ. Grenoble Alpes}, LIG, 38000 Grenoble, France \\
  (2) Lingua Custodia, 75008, Paris, France \\
  \texttt{contact: mariam.nakhle@univ-grenoble-alpes.fr} }


\begin{document}
\maketitle
\begin{abstract}


Despite the strong research interest in document-level Machine Translation (MT), the test sets dedicated to this task are still scarce. The existing test sets mainly cover topics from the general domain and fall short on specialised domains, such as legal and financial. Also, in spite of their document-level aspect, they still follow a sentence-level logic that does not allow for including certain linguistic phenomena such as information reorganisation.
In this work, we aim to fill this gap by proposing a novel test set: DOLFIN. The dataset is built from specialised financial documents, and it makes a step towards true document-level MT by abandoning the paradigm of perfectly aligned sentences, presenting data in units of sections rather than sentences.
The test set consists of an average of 1950 aligned sections for five language pairs. We present a detailed data collection pipeline that can serve as inspiration for aligning new document-level datasets.
We demonstrate the usefulness and quality of this test set by evaluating a number of models. 
Our results show that the test set is able to discriminate between context-sensitive and context-agnostic models and shows the weaknesses when models fail to accurately translate financial texts.
The test set is made public for the community\footnote{\url{https://huggingface.co/datasets/LinguaCustodia/dolfin}}.
\end{abstract}


\section{Introduction}


\begin{table}
\centering
\begin{tabular}{llll}
\hline
\textbf{Language} & \textbf{\# Seg.}   & \textbf{\# Avg sent.}  \\
\hline
En-Fr &    2081  & 12.1  \\
En-De &    2026  & 11.4  \\
Fr-Es &    1983  & 15.2  \\
En-Es &    1932  & 14.8  \\
En-It &    1737  & 10.8  \\
\hline
Average &    1951  & 12.8  \\
\hline
\end{tabular}
\caption{\label{table:stats_langs}Number of segments and sentences per language pair. \# Seg. is the number of segments per language in the dataset and \# Avg sent. is the average number of sentences in a segment.
}
\end{table}


Document-level translation has received a lot of attention in the Machine Translation (MT) community in the past years \cite{toral2018attaining, laubli2020set}. However, document-level testing data is still scarce, covering mainly the news domain and only a few specialised domains, which hinders progress \cite{anastasopoulos-etal-2020-tico, goyal2022flores, federmann-etal-2022-ntrex}. 

In this work, we focus on the financial domain which is characterised by a highly specialised language and a high demand for fast and reliable translations. The translations of some of the financial documents are required by law, which creates a significant demand for translation services with a high degree of in-domain expertise. A salient example of this is the French word \textit{couverture} that translates as \textit{blanket} in a general domain text, but as \textit{hedge} in a financial text.
This makes out of financial translation an interesting use-case for MT. In order to facilitate a robust context-sensitive evaluation within this domain, we propose \emph{DOLFIN}, a new document-level test set that focuses on specialised language from the financial domain. Beside the novelty of the domain, this test set also employs a novel approach to defining translation units: rather than considering aligned sentences, it considers aligned sections, thereby enabling higher-level information reorganisation, such as sentence reordering within the section. A further innovation is that this test set was built with a focus on context-sensitive phenomena, which makes it an approach in-between a regular test set and a test-suite.

As was shown in \citet{toral2018attaining} and \citet{laubli2020set}, sentence-level MT suffers from errors that are only identifiable when taking the context into consideration. This is true for the financial domain as well. In specialised domains, terminology plays an important role and the terms used must be consistent throughout the document, therefore one can only assess the correctness of a term when having access to its previous and following translations. This is especially true for legal financial documents, where the beginning of a document contains an explicit definition of terms used for the mentioned entities that must be respected throughout the document. Another example would be the consistency of numerical formats. Financial documents contain many tables stating monetary amounts and using different norms of formatting within the same table would be unacceptable (for example \textit{5 million USD}, \textit{5M USD} or \textit{\$5M}). These phenomena are why a document-level evaluation is especially relevant for financial MT.

We use the Fundinfo website\footnote{\url{https://www.fundinfo.com}} as a source of data because it offers a vast choice of parallel financial documents in numerous European languages. The \emph{Terms of Use} of this website allow us to crawl the documents where a download-link is provided\footnote{\url{https://api.fundinfo.com/static/legal/1/en}, last consulted on 09/09/2024.}. The documents on Fundinfo  gather all the official translations for the given regions and languages.

In this work we are contributing to the goal of a robust document-level MT evaluation in the specialised financial domain by collecting a test set covering 5 language pairs with 1951 segments on average per language pair, as shown in Table \ref{table:stats_langs}. The contributions of this work can be summarised as following: \\
(1) we create and publicly release DOLFIN, a novel test set for document-level financial MT, \\
(2) we describe a detailed pipeline to extract and align sections from long documents, \\
(3) we propose an approach to measure the impact of longer context on translation quality by comparing the scores obtained in a sentence-level and in a document-level fashion, \\ 
(4) we present an experiment showing the usefulness of this test set and analyse the capabilities of LLMs to deal with longer contexts and specialised financial language.


\section{Related work}

\label{sec:related_work}



\textbf{Document-level test sets}

Most resources in MT research are sentence-level, but there is a number of test sets that include not only individual sentences but also longer texts. These are usually made without trying to target any context-sensitive phenomenon and typically the evaluation is performed at sentence-level, i.e., by feeding the sentences separately to the evaluation metrics \cite{anastasopoulos-etal-2020-tico, goyal2022flores, federmann-etal-2022-ntrex,deutsch-etal-2023-training}.
These works usually take the form of a source file, target file and a file with document ids. The order of lines creates a mapping between the source and target sentences as well as the document ids, therefore all three files must contain the same number of lines.

Our test set diverges by taking a novel approach and abandoning the idea of a perfect sentence-to-sentence alignment. We consider that it constrains the translation process and strips the texts of high-level phenomena such as information reorganisation, sentence splitting and merging. 
Section \ref{sec:imperf-alignment} details the reasons behind this choice.

There are also a number of document-level training datasets \cite{koehn-2005-europarl, cettolo2012wit3, tiedemann-2012-parallel, lison-tiedemann-2016-opensubtitles2016} and recently there have been efforts to recover the contexts of large sentence-level datasets \cite{wicks2024recovering}. Since these are for training purposes, they are out of the scope of this work that focuses on test data.

As for datasets specialised on financial language, these are very scarce; we can mention the European Central Bank corpus that is included in OPUS \cite{tiedemann_opus} and a diachronic corpus based on a banking magazine \cite{volk_banking}. However, these are resources are the sentence level.

\textbf{Targeted context-sensitive evaluation}

An important line of work in the document-level MT evaluation are test-suites. These are manually crafted test sets that target specific context-sensitive phenomena and the evaluation focuses on errors when translating them. There are two approaches: 1) evaluation based on references and 2) evaluation based on contrastive pairs.

The first type is inspired by the traditional approach of comparing the generated translation to a reference \cite{guillou-etal-2016-findings, wong-kit-2012-extending, hardmeier-federico-2010-modelling}.
The use of references has the disadvantage that it only measures the agreement between the reference and hypothesis and not the internal agreement across generated sentences, i.e., whether the translation respects agreement with its previous and following sentences.

To overcome this, the contrastive pairs were proposed. For every source sentence, multiple context sentences are provided, along with a correct translation, as well as an incorrect one (or multiple incorrect ones) \cite{bawden-etal-2018-evaluating, muller-etal-2018-large, voita-etal-2019-good}. The evaluation measures if the model gives a higher probability to the correct translation without actually requiring the model to generate it. The disadvantage of these test-suites is the need to access models' probabilities, which is impossible for most commercial systems, and another flaw is that even if the model gives a higher probability to the correct answer, it doesn't guarantee that the generated translation would be correct. 
To build new targeted test-suites more easily, automatic means to identify sentences that contain context-sensitive phenomena were proposed  \cite{fernandes-etal-2023-translation, wicks-post-2023-identifying}.

The development of new test-suites usually requires a high level of linguistic expertise, knowledge about the studied context-sensitive phenomena and/or tools designed for the task. Therefore, test-suites are only available for some languages and well studied phenomena, which makes the evaluation limited. This makes the test-suites difficult to use as a universal method for document-level MT evaluation. This issue is usually bridged by combining a test-suite evaluation with a general quality test set evaluation. However, a more ideal approach would be being able to capture the models' capabilities to deal with context, as well as its overall translation quality. 
In the following, we present the DOLFIN test set that aims to evaluate both of these aspects.


\section{Dataset collection}
\label{sec:dataset_collection}

\subsection{Format and structure}
\label{sec:format_structure}


This test set is built from PDF documents that undergo a processing pipeline with the first step consisting of PDF to text extraction. This step can introduce errors that sometimes affect parts of documents, and rarely even full documents.
This is why the test set is built of sections, rather than full-length documents.
Also, in the financial domain, documents range from a few pages to more than hundreds of pages, it would be unfeasible to include such long documents as single segments.
Since most MT test sets annotated with human judgements are also used to train or meta-evaluate automatic metrics, it is preferable to have a single score per section instead of per document, which would drastically decrease the number of scores (and data available for training and meta-evaluation). 


When performing the extraction, we chose markdown (MD) as the format for our segments. It is richer than plain text and conserves more information about formatting, like tables, titles and subsections that are common in the PDF documents.


\subsection{Imperfect sentence to sentence alignment}
\label{sec:imperf-alignment}

As mentioned by~\citet{deutsch-etal-2023-training}, there are practically no test sets with real-life translations without a perfect sentence to sentence alignment. The majority of test sets available in the community, such as those from the annual WMT conference \cite{kocmi-etal-2023-findings}, were crafted for the purpose of evaluating sentence-level MT systems and they maintain a perfect sentence-to-sentence alignment. Even though there isn't a clear consensus in the translators' community on whether the original number of sentences must be kept \cite{baker2018other}, it is clear from the analysis of~\citet{merkel-2001-comparing}, that the number of sentences in the human translations differs from the source. This means that the test sets used in the MT community do not contain some of the more challenging linguistic phenomena such as information and sentence reordering, merging or splitting.
In order to allow these phenomena to happen, we don't enforce a perfect sentence-level alignment, even though it makes the calculation of automatic metrics more challenging. We consider that a truly document-level MT system treats the segment as a whole and won't necessarily generate the same amount of sentences as in the source, therefore we abandon the paradigm of perfectly aligned parallel sentences in the rest of this work.

\subsection{Processing Pipeline}
\label{sec:pipeline}

The only source for this test set is the website Fundinfo, a provider of data about investment funds for different stakeholders of the investment industry. It gathers a large amount of financial documents available in multiple languages and document types. Below, we detail the pipeline designed to process the documents.

\textbf{PDF document alignment.} The PDF files were paired into bilingual documents based on the ISIN code (unique identifier), document type, emission date and language code. The website does not provide any information about which language is the original and which one the translation, for practical reasons we refer to the English document as the source and the other as the target, except for French-Spanish document pairs, where we consider the French as the source.

\textbf{PDF to text extraction.} We used the tool \emph{Apryse}\footnote{\url{https://apryse.com/}, a paid software} to extract text from the PDFs. We chose it because it yielded the best results on our kind of documents in our preliminary tests, compared to the PyMuPDF library. More details about these tests can be found in Appendix \ref{annex:pdf_extractor}.

\textbf{Filtering of problematic documents.} Some documents were flawed and removed as a whole at this stage using the following filters: language identification using XLM-Roberta-base \cite{xlm-roberta}\footnote{papluca/xlm-roberta-base-language-detection} and PDF extraction error identification (texts without any space, texts with a space between every character, gibberish, etc.)

\textbf{Section identification.} In the documents, there is no explicit indication of the beginning and ending of sections. We identified the sections using the markdown structure, such as titles, headings and tables, using regular expressions.

\textbf{Section alignment.} The identified sections were aligned using the LASER contextual sentence embeddings \cite{schwenk-douze-2017-learning} and the \textit{easylaser} library\footnote{\url{https://pypi.org/project/easylaser/}} by calculating the cosine similarity between sentences. Since this model was not trained on long texts, we don't embed full texts but only sentences and align the sections using different heuristics. One of the approaches is to leverage the consecutive titles. If two pairs of consecutive titles have a high laser score, we consider that the section between the two titles is also aligned. When aligning sections that don't start with a title, we use the same logic, but instead of using the titles, we verify if the first and last three sentences are aligned. In both cases we define a threshold of 0.75 for the score, above which we consider that the sentences are aligned. In order to align tables, we only consider documents where the number of tables in the two languages is identical and they have the exact same shape (number of lines and columns). We consider them aligned in the order in which they appear in the document.
This way of aligning the data makes this test set intrinsically document-level, with the main unit being the section rather that the individual sentence. This differs from classical alignment approaches, where only sentences are aligned and those that do not meet a minimum threshold are discarded, thus losing the continuity of the document.


\textbf{Deduplication.} Since the financial documents contain many sections dictated by law, they tend to be repetitive. Therefore we apply harsh deduplication using the minhash algorithm and remove near duplicates. Inspired by~\citet{penedo_dataset_falcon}, we used the library \textit{text-dedup}\footnote{\url{https://github.com/ChenghaoMou/text-dedup}} and this step removed approximately 80\% of the sections. However, it keeps repeated sentences if the segment as a whole is not a duplicate. 

\textbf{Data filtering.} We filtered out noisy segments using the same filters as the ones applied to whole documents. We redo this cleaning because in many cases only some parts of the document are impacted. We also remove segments with a big length difference between source and target language, segments with empty tables, segments where the number of headings in the two languages differ, as well as too short (two sentences or less) and too long segments (more than 4000 characters).

\textbf{Quality Estimation filtering.} 
In order to further filter the samples, we used quality estimation to identify the highest quality segments.
We used the Comet-kiwi model~\cite{rei-etal-2022-Cometkiwi}\footnote{Unbabel/wmt22-Cometkiwi-da}, which is trained to evaluate sentences, but not paragraphs, and has a limited maximum context length. 
In order to score the full segments, we used the SLIDE approach~\cite{raunak-etal-2023-evaluating} with a window size of 3 and stride of 1. This approach would ideally need a perfect sentence to sentence alignment, but we overcome this by using the window size of three, which helps to buffer the differences in number of sentences. Segments shorter than the window size are fed to Comet-kiwi as a whole. We looked for a quality threshold separately for every language pair and selected between 45\

\textbf{Identification of interesting segments.} 
In order to build a test set rich in context-sensitive phenomena to challenge the MT models, we identified such phenomena using two approaches.
First, using the \emph{CTXPRO} tool~\cite{wicks-post-2023-identifying} which identifies gender, formality, and animacy for pronouns, plus verb phrase ellipsis and ambiguous noun inflections.
A limitation of this tool is that it is intended for perfect sentence-to-sentence alignment. We approximated such constraint for segments where the number of sentences in source and target language wasn't the same. We simply concatenated strictly the minimum number of sentences in order to obtain the same number of sentences on both sides.
For the second approach, we prompted a Large Language Model (LLM), namely Llama3-70b~\cite{dubey2024Llama}, to annotate segments that contain elements that need context in order to be correctly translated and evaluated. The full prompt used for the annotation can be found in Annex~\ref{annex:prompts}.

\textbf{Manual filtering.} Based on the annotations from the previous step, we created a selection of candidate segments for every language. As a finishing step, these segments were manually verified and we removed some of the repetitive or less interesting segments.
Not every segment contains a context-sensitive phenomenon, because they don't naturally occur in every text. The segments that don't contain any specific phenomenon are kept because they serve the purpose of assessing the overall translation quality. 




\section{The DOLFIN test set}
\label{sec:dolfin}

Table~\ref{table:stats_langs} shows the main statistics of the final test set. There are on average 1950 segments per language pair and each segment contains on average 12 sentences. The number of sentences was calculated using the PySBD segmenter \cite{sadvilkar-neumann-2020-pysbd}. A total of 257706 PDF documents were processed, resulting in a final test set containing 7812274 tokens, with an average of 400 tokens per segment, or 800 tokens source and target text combined. The minimum token length of a segment is 26 (in English) and maximum is 6963 (in Italian).

In finance, there are multiple sub-domains that correspond to different document types. In this test set, 15 document types are represented and Annex~\ref{annex:sub-domains} gives the statistics per document type.

The test set contains the following metadata: 
\texttt{src\_lang}: source language;
\texttt{trg\_lang}: target language;
\texttt{sub\_domain}: sub-domain of finance;
\texttt{date}:  date of publication;
\texttt{Annotation}: annotations of context-sensitive phenomena (obtained by CTXPRO and Llama-3-70b);

Table~\ref{table-examples} shows a few examples of context-sensitive phenomena present in the test set, where the context is needed to correctly disambiguate the sentence to translate, or to maintain a consistent translation along the document.



\begin{table*}[!ht]
\centering
\small
\begin{tabularx}{\textwidth}{XX}
\hline
\underline{\textbf{English text}} \\
The ratings of fixed income securities by credit  rating agencies are a generally accepted  barometer  of credit risk. They are, however, subject to certain  limitations from  an investor’s standpoint. \\
\underline{\textbf{French text}} \\
Les \textbf{notes} accordées aux titres à taux fixe par les  agences de notation sont généralement acceptées  comme le « baromètre » du risque de crédit qu’ils  représentent. Toutefois, \textbf{elles} sont quelque peu \textbf{limitées} du point de vue de l’investisseur. \\
\underline{\textbf{Observation:}} the translation of the anaphoric pronoun ``They'' must respect the  agreement with its antecedent ``ratings'', which is in French a feminine noun ``notes''. \\
\hline 
\underline{\textbf{English text}} \\
| Balance, beginning of period |  | \\
| --- | --- | \\
| - Share class A - USD | 25,000.000 | \\
{| - Share class S - CHF | 85,811.152 |}\\

\underline{\textbf{German text}} \\
| Stand zu Beginn der Berichtsperiode |  \\
| --- | --- | \\
| - Anteilsklasse A - USD | \textbf{25'000.000} | \\
| - Anteilsklasse S - CHF | \textbf{85'811.152} | \\
\underline{\textbf{Observation:}} the formatting of numbers must stay consistent across the document to avoid confusions. In this case, the symbol apostrophe is used as a thousand separator, which is the norm in Swiss German.  \\ 
\hline 
\underline{\textbf{English text}} \\

This product is suitable for investors who ... \\
... have at least a basic knowledge of the financial instruments contained in the fund; \\
... have at least a medium-term investment horizon; \\
... would, in a worst-case scenario, be able to withstand the loss of the entire invested capital. \\
\underline{\textbf{Italian text}} \\
Questo prodotto è adatto agli investitori che ... \\
... \textbf{sono} in possesso di conoscenze almeno elementari in merito agli strumenti finanziari detenuti dal fondo; \\
... \textbf{hanno} almeno un orizzonte d'investimento a medio termine; \\
... \textbf{sono} in grado di sopportare, nel peggiore dei casi, anche la perdita dell'intero capitale investito.\\
\underline{\textbf{Observation:}} the document is written in a condensed writing style, which separates the verbs from their subject, nevertheless the verb-subject agreement must be respected. \\ 
\hline
\underline{\textbf{French text}} \\
À titre indicatif, la performance du Fonds est comparée à l’indice Bloomberg Barclays U.S. Government/Credit Bond Index (Total Return)  (ci-après, « l’Indice de Référence »). Le Fonds a sous-performé son Indice  de Référence au cours du semestre clos le 30 juin 2022. \\

\underline{\textbf{Spanish text}} \\
A título meramente indicativo, la rentabilidad del Fondo se  compara con la del índice Bloomberg Barclays U.S. Government/Credit Bond Index (Total Return) (el «\textbf{Índice de referencia}»). El Fondo tuvo un rendimiento  inferior al de su \textbf{Índice de referencia} en el semestre cerrado a 30 de junio de 2022. \\

\underline{\textbf{Observation:}} the term ``Índice de referencia'' is chosen as a replacement of the full bond name it must be used consistently throughout the rest of the document. \\ 
\hline
\end{tabularx}
\caption{\label{table-examples}
Examples of context-sensitive phenomena in the test set. The words in bold are context-dependent and need extra-sentential information to be correctly translated.
}
\end{table*}



\section{Experiments}
\label{sec:experiment}

\subsection{Experimental setting}

In order to demonstrate the relevance and usefulness of our proposed test set, we use it to evaluate a selection of models, namely the LLMs Llama-3-70b, Llama-3.1-8b, GPT-4o and GPT-4o-mini. 
We do not compare the performance to traditional MT systems because of the limited context size. Most open-source models support a context size of maximum 512 tokens, which would exclude a large proportion of our segments. And as for commercial MT systems such as Google Translate API or DeepL API, the implementation details of whether the systems encode the inputs as a whole are unknown.
That is why we chose LLMs. They have larger context-size limits and we can control whether we want the model to treat the segments as a whole or sentence by sentence. We selected these models in pairs (taking a big and a small one from the same family of models) and we try to answer the following research questions : 
\\
1) \textit{Can the test set show the differences in how models deal with  context?} 
\\
2) \textit{Can the test set show if the models are able to adequately translate specialised financial texts?}

We considered English as the source language for all language pairs, except for French-Spanish where French is considered the source. We translated the test set in two contrastive ways: (1) translating each sentence of a segment one by one, called \textit{``Per sent''} setting and (2) translating each segment as a whole, called the \textit{``Full seg''} setting, hence imitating a sentence-level and document-level translation respectively. To obtain the translations, we used the APIs Groq\footnote{\url{https://wow.groq.com/} to run \texttt{llama3-70b-8192}}, Together\footnote{\url{https://api.together.xyz/} to run \texttt{meta-llama/Meta-Llama-3.1-8B-Instruct-Turbo}} and OpenAI\footnote{\url{https://platform.openai.com/} to run \texttt{GPT-4o} and \texttt{GPT-4o-mini}}. We used a simple prompt to obtain the translations: ``\texttt{Translate the following text in \{src\_lang\} into \{trg\_lang\}. Only provide the translation without any other text. The text to translate:\textbackslash n\{segment\}}''.


We evaluated the translations using Comet-da-22\footnote{Unbabel/wmt22-comet-da} (model that uses the reference as well), with SLIDE approach and a stride of one and window size of three. Since the test set doesn't respect a perfect sentence-to-sentence alignment, we had to approximate the alignment by merging some sentences, in order to be able to use the sliding method. 



\subsection{Results}

 \begin{table*}
 \centering
    \begin{tabular}{c|ccc:ccc}      
     \hline 
        Language & Per sent & Full seg & diff & Per sent & Full seg & diff  \\ 
         \hline 
         & Llama-3.1-8b &  &  & Llama-3-70b &  &   \\ 
         \hdashline
        En-Es & 74.14 & 75.22 & +1.08 & 77.71 & 81.51 & +3.79 \\ 
        Fr-Es & 74.48 & 74.90 & +0.42 & 76.46 & 82.03 & +5.57 \\ 
        En-Fr & 74.00 & 73.55 & -0.45 & 77.96 & 82.08 & +4.12 \\ 
        En-It & 74.16 & 73.77 & -0.39 & 78.75 & 82.93 & +4.18 \\ 
        En-De & 69.57 & 61.52 & -8.05 & 79.75 & 81.75 & +2.00 \\ 
        
        \hline 
         & GPT-4o-mini &  &  & GPT-4o &  &   \\ 
         \hdashline
        En-Es & 79.28 & 82.55 & +3.27 & 79.61 & 83.02 & +3.41 \\ 
        Fr-Es & 79.22 & 82.82 & +3.61 & 79.27 & 82.92 & +3.65 \\ 
        En-Fr & 80.72 & 82.87 & +2.15 & 81.01 & 83.37 & +2.36 \\ 
        En-It & 81.57 & 84.54 & +2.97 & 81.86 & 84.87 & +3.01 \\ 
        En-De & 81.39 & 83.75 & +2.36 & 81.78 & 84.24 & +2.46 \\ 
        
        \end{tabular}
    \caption{\label{table:results}Comet-slide scores on the DOLFIN test set.\\
        }
\end{table*}

\textbf{Context-sensitive aspect.} The results of translations of DOLFIN using the Comet-slide metric are shown in Table~\ref{table:results}. 
Since this test set is intrinsically document-level, we hypothesise that the models will perform best when given the full segment, if capable of dealing with long contexts.


We can see that among the four tested models, Llama-3.1-8b is the only one that doesn't improve with full context and even degrades for three language pairs, while its bigger counterpart, Llama-3-70b consistently improves. When analysing the translations of the small model, we observe that for some segments, the generation enters a downhill and with every token the model's predictions get worse and the translation ends in gibberish. This might be explained by the fact that although the model's maximum context size is 128k tokens, this model was fine-tuned from a model with a smaller maximum context size in order to extend it and this didn't give it enough capacity to model long context.

On the other hand, the other bigger models consistently improve when provided with more context. Even in the \textit{Per sent} setting, these models achieve higher scores, which shows their overall higher quality. This gets strengthened when provided with the full context, which shows that they can effectively model context and take advantage of it when generating the full translation. 

The fact that the scores do indeed get higher with more context proves that this test set is appropriate when evaluating document-level MT since it contains context-sensitive phenomena that need additional context to be correctly translated and to obtain higher automatic score.


Apart from the automatic score, the test set allows qualitative manual analysis. To illustrate such an analysis, we performed a simple evaluation (with one evaluator), focusing on segments containing a CTXPRO annotation. This showed many cases where additional context guided the model to generate a more appropriate translation.  As an example, the improvement in translation can be observed in the following sentence: ``These key issues are defined by sector and are regularly reviewed. They are, however, by definition not exhaustive.'' The ``key issues'' were translated in French by Llama-3-70b as ``questions clés''  (feminine noun) and therefore must be referred to by the pronoun ``elles''. The \emph{Per sent} translation contained the masculine ``ils'', which was corrected when provided with the full context. 
Another example of an issue in the \emph{Per sent} setting is the inconsistent level of formality when translating the pronoun ``you''. Within the same section, the model GPT-4o-mini translated it into Spanish by a mix of the informal pronoun ``tú'' and the formal ``usted'', which was also solved in the \textit{Full seg} setting.

To verify the general quality of the test set, we compared the scores obtained on DOLFIN to those on NTREX, a high-quality document-level general domain test set, and they are of similar magnitudes. Due to budget constraints, we computed the translations only with the Llama-3-70b model and we obtained 77.64, 75.93, 77.26 for En-Fr, En-De and En-Es respectively in the \textit{Per sent} setting, which is comparable to the scores on DOLFIN. Since this evaluation method leverages the reference, this shows that the alignment of our sections is correct. If the reference weren't aligned to the source, the scores would be much lower because the translation wouldn't be in line with the reference. We show this by running an evaluation with Comet-slide and using always the same reference to make sure the scores are not high just because the metric relies on the source text. Under these conditions indeed the scores dropped to approximately 44, supporting our hypothesis. We are aware that scores obtained across different test sets are not directly comparable, but we provide this as an additional guarantee of good alignments.




\textbf{Financial aspect.} The test set also allows to evaluate the ability of models to translate financial texts with the required degree of specialisation.
Indeed, the tested models do have issues related to the specialised aspect of the documents. Even when including the full context, some polysemous words are translated incorrectly by the Llama-3-70b model. The word ``Charges'' was translated as ``criminal charges'' instead of ``price'' which is the word's most common usage in finance, giving ``Anklagen'' and ``Anschuldigungen'' in German.
The tested models fail to adhere to a correct formatting style for currency values in the target language and they don't maintain the same format when generating the sentences separately. The translation usually keeps the format as in the source language, English, which is incorrect in most other European languages (like using ``.'' as a decimal separator and placing the ISO currency code before the number).

In terms of table formatting, we find that some models struggle with the markdown syntax. A common problem is a misinterpretation of the horizontal lines (denoted by ``\texttt{---}''), which are misplaced in the translation, resulting in an unreadable table. The content of the table itself is not affected, however in terms of post-editing time and effort, this type of problem degrades the quality of the translation.



\section{Discussion}
\label{sec:discussion}



\textbf{Document-level MT evaluation metrics}

The SLIDE technique was proposed to overcome the limit of maximum context-length of modern evaluation metrics. However, this solution is far from ideal, since it expects a perfect sentence-to-sentence alignment, which doesn't necessarily occur in translations. The evaluation beyond the sentence level remains thus an open issue.

\textbf{Preference of Comet for shorter segments}

When using \emph{Comet-slide} to score the segments, there is a clear tendency of the scores getting lower with increased sentence length. We calculated the correlation between the score and the source/target length and it is of -0.281/-0.283 respectively, which shows a weak negative correlation. This means that as segments get longer, the scores tend to get lower. This must be taken into consideration when using this metric to filter datasets as such a preference for short segments might unnaturally skew the data by stripping the dataset of long sequences.


\textbf{Rich visual formatting in MT}

The formatting remains unsolved in MT, where research mainly focuses on plain text. In this work, we processed PDF files to obtain MD files. While this helps to preserve some information expressed through formatting, it also introduces noise. The final sections can contain formatting unpleasant to the eye and can be made only of a sequence of numbers or a few words, probably extracted from charts and graphs, which makes any linguistic analysis difficult.
The PDF extraction step is a bottleneck. In the case of financial documents with very rich formatting and multi-modal content (charts, graphs, info-graphics, etc.), one would ideally make use of the visual aspects when translating; a multi-modal approach combining vision and language would be interesting to test.

\textbf{Low-resource languages}

We experimented with using Fundinfo to obtain data in a low-resource language, namely Slovak. There were only 73 pairs of documents available, all issued from the same company, which made the texts repetitive and the result was only 8 unique parallel segments. This is the case for most low-resource languages, since only five languages (English, French, German, Spanish, Italian) make up 83\

\section{Future work}


In the future, we aim to provide further manual annotations of the test set with regard to the context-sensitive phenomena to offer the possibility of a targeted automatic evaluation of these aspects.
We will also translate the test set using different MT models (context-aware and context-agnostic) and perform a human evaluation. This evaluation will be used to meta-evaluate existing metrics, to see if 1) they are capable of penalising context-related errors and 2) correctly assess translations in the financial domain. 
The human judgements thus obtained can also be used for training a new metric that would be specialised for financial texts and capable of correctly assessing the models' capability to take context into account.

\section{Conclusions}

We presented DOLFIN, a novel test set for document-level evaluation of MT in the financial domain. It allows to assess the models' capability to translate longer texts while taking into account the context within the financial domain. We described the pipeline created to build it and we release it publicly. In order to illustrate the usefulness of this resource, we used this test set to evaluate a series of models and found that although some of the bigger models benefit from longer contexts, the context might negatively affect the overall quality when the model can't handle it correctly.


\section*{Limitations}
One limitation of our work is that we lack information on the original language of a document. We translate the segments in a ``English-to-any'' fashion (except for French-Spanish), which might cause some of the segments to be less challenging for an MT system if the English text is in fact a translation itself \cite{federmann-etal-2022-ntrex}.

Another identified limitation is that we used the Llama-3 model in order to select the most challenging segments. As any language model, Llama-3 is prone to error, therefore this way of selecting segments is subject to imperfections coming from the model. However, all the segments were also manually verified. Another limitation associated with the use of LLMs is the uncertainty as to whether the texts in the test set have been seen by the models. But the data is present on the web in its PDF format and not in its text and aligned form. Moreover, our study compares the model's performance in the sentence and document-level scenarios, so both scenarios would have the same advantage and therefore the gains or drops in translation quality are still meaningful.



\bibliographystyle{acl_natbib}
\documentclass{MITstyle}

%\usepackage[table]{xcolor}
\usepackage{chngcntr}
\usepackage{hyperref}
\usepackage{microtype}

\title{A Lightweight and Extensible Cell Segmentation and Classification Model for Whole Slide Images}

\author{Nikita Shvetsov~$^{1, }$\footnote{Correspondence e-mail: nikita.shvetsov@uit.no}, Thomas K. Kilvaer~$^{2, 3}$, Masoud Tafavvoghi~$^{4}$, Anders Sildnes~$^{1}$, \\ Kajsa Møllersen~$^{4}$, Lill-Tove Rasmussen Busund~$^{5, 6}$, Lars Ailo Bongo~$^{1}$ \\
%
\vspace{1em} % Space between authors and afilliations
%
\normalfont{\small $^{1}$Department of Computer Science, UiT The Arctic University of Norway}\\
\normalfont{\small $^{2}$Department of Oncology, University Hospital of North Norway}\\
\normalfont{\small $^{3}$Department of Clinical Medicine, UiT The Arctic University of Norway}\\
\normalfont{\small $^{4}$Department of Community Medicine, UiT The Arctic University of Norway}\\
\normalfont{\small $^{5}$Department of Medical Biology, UiT The Arctic University of Norway} \\
\normalfont{\small $^{6}$Department of Clinical Pathology, University Hospital of North Norway} %\vspace{2em}
}

\begin{document}
\maketitle

\section*{Abstract}

% \begin{abstract}
% Developing clinically useful cell-level analysis tools in digital pathology remains challenging due to limitations in dataset granularity, inconsistent annotations, computational demands of advanced models, and difficulties in integrating new technologies into clinical workflows. To address these challenges, we propose a multi-faceted solution that enhances data quality, model performance, and usability to create a lightweight and extensible cell segmentation and classification model.

% First, we update data labels by employing a cross-relabeling process that refines the labels of two existing datasets, PanNuke and MoNuSAC, to create a new unified dataset with enhanced granularity, encompassing seven distinct cell types. Second, we leverage the H-Optimus foundation model as a fixed encoder to improve feature representation for simultaneous cell segmentation and classification tasks. Third, to address the computational demands of foundation models, we employ knowledge distillation to reduce model size and complexity while maintaining comparable performance. Finally, to facilitate integration into clinical workflows, we integrate the distilled model into the QuPath software, a widely used open-source platform in digital pathology.

% Our results demonstrate improvements in cell segmentation and classification performance using the H‑Optimus-based model compared to a CNN-based model. Specifically, the average $R^2$ improved from 0.575 to 0.871, and the average $PQ$ score improved from 0.450 to 0.492, indicating better alignment with actual cell counts and enhanced segmentation and classification quality. Furthermore, the distilled student model maintains performance comparable to the larger foundation model while reducing the parameter count by a factor of 48.
% Overall, by reducing computational complexity and integrating it into existing workflows, the proposed approach may significantly impact diagnostic processes, reduce the workload of pathologists, and contribute to improved patient outcomes. Though our approach shows potential enhancements in efficiency and usability of cell segmentation and classification models in digital pathology, extensive validation is needed to deploy these models in clinical practice.
% \end{abstract}

%%% shortened abstract
\begin{abstract}
Developing clinically useful cell-level analysis tools in digital pathology remains challenging due to limitations in dataset granularity, inconsistent annotations, high computational demands, and difficulties integrating new technologies into workflows. To address these issues, we propose a solution that enhances data quality, model performance, and usability by creating a lightweight, extensible cell segmentation and classification model. 

First, we update data labels through cross-relabeling to refine annotations of PanNuke and MoNuSAC, producing a unified dataset with seven distinct cell types. Second, we leverage the H-Optimus foundation model as a fixed encoder to improve feature representation for simultaneous segmentation and classification tasks. Third, to address foundation models' computational demands, we distill knowledge to reduce model size and complexity while maintaining comparable performance. Finally, we integrate the distilled model into QuPath, a widely used open-source digital pathology platform. 

Results demonstrate improved segmentation and classification performance using the H-Optimus-based model compared to a CNN-based model. Specifically, average $R^2$ improved from 0.575 to 0.871, and average $PQ$ score improved from 0.450 to 0.492, indicating better alignment with actual cell counts and enhanced segmentation quality. The distilled model maintains comparable performance while reducing parameter count by a factor of 48. By reducing computational complexity and integrating into workflows, this approach may significantly impact diagnostics, reduce pathologist workload, and improve outcomes. Although the method shows promise, extensive validation is necessary prior to clinical deployment.
\end{abstract}
\clearpage

\section{Introduction}
In digital pathology, accurate segmentation and classification of cells are crucial for many diagnostic, prognostic, and predictive analyses \cite{Jaber_Beziaeva_etal._2019,Lin_Pan_etal._2022,Park_Ock_etal._2022,Shen_Choi_etal._2024}. Nowadays, developments in computational pathology offer multiple solutions \cite{H._Qu_P._Wu_etal._2020,Javed_Mahmood_etal._2020} to utilize cell-level datasets to train machine learning models that solve these problems. The quality and specificity of training datasets are critical for robust and accurate models. Adhering to the principle of "garbage in, garbage out", it is essential to ensure that these datasets are extensively and accurately labeled with distinct classes that reflect the diverse biological characteristics of different cell types. Unfortunately, the number of open-source datasets comprising such high-quality annotations is limited. Existing cell segmentation datasets \cite{Gamper_Koohbanani_etal._2019,Graham_Vu_etal._2019,Verma_Kumar_etal._2021} may offer extensive annotations for certain cell types while providing more general labels for others. For example, in PanNuke, which is one of the largest open-source datasets comprising labeled cells, various types of morphologically and functionally different inflammatory cells like macrophages and lymphocytes are clustered in a broad "inflammatory" class. Consequently, these classes are frequently omitted from analyses or aggregated into broader meta-classes \cite{Gamper_Koohbanani_etal._2020} and likely interfere with other cell classes included in the dataset. This and similar inconsistencies in annotation granularity limit the ability of machine learning models to learn the comprehensive and nuanced features necessary for accurate cell segmentation and classification. To address these challenges, methods for refining and standardizing dataset annotations are essential to enhance the quality of training data.

A complementary approach to mitigate the absence of high-quality training data is the use of foundation models. Foundation models as encoders are defined as large-scale, versatile networks pre-trained on vast, diverse datasets using self-supervised learning, contrasting with convolutional neural network (CNN) pre-trained encoders that rely on supervised learning with labeled data. In practice, foundation models leverage enormous amounts of weakly or unlabeled data from millions of whole slide images (WSIs) and employ self-attention mechanisms to capture long-range dependencies and global context \cite{Chen_Ding_etal._2024,Saillard_Jenatton_etal._2024,Vorontsov_Bozkurt_etal._2024,Xu_Usuyama_etal._2024}. As a consequence, foundation models are able to produce transferable feature representations across different cell types and tissue environments. The feature representations can be leveraged by decoder networks to produce segmentation masks and pixel-level classifications. Because foundation models have comprehensive feature representations, they can be effectively fine-tuned using much smaller amounts of cell-level data compared to the large datasets needed to train models from scratch. Furthermore, foundation models incorporate adversarial training elements or contrastive learning \cite{Chen_Ding_etal._2024,Xu_Usuyama_etal._2024}, enhancing their resilience and adaptability by exposing them to challenging and varied scenarios during training. This may result in more generalizable models, often making them well-suited for diverse and complex tasks in digital pathology.

Despite the inherent advantages of foundation models, their deployment for practical use faces its own obstacles. In particular, they require substantial computational power, financial investments and rigorous testing to ensure reliability and efficacy for a given task \cite{Akkus_Dangott_etal._2022,Dragomir_Cocuz_etal._2022,Go_2022,Jafri_Farooqui_etal._2024}. Moreover, while foundation models enhance feature representation and performance, they depend on the quality of available annotations for decoder fine-tuning and, like any other model, cannot resolve existing inconsistencies or ambiguities in data labels. Therefore, there remains a critical need for solutions that address both data quality and practical deployment considerations.
Further, integrating new technologies into existing clinical workflows often encounters resistance, as it necessitates adjustments to established diagnostic processes. So, there is a need to develop solutions that could be integrated into current practices, minimizing the burden on medical professionals to adopt new tools \cite{King_Williams_etal._2023}.

Existing solutions \cite{Goldsborough_Philps_etal._2024,Hörst_Rempe_etal._2024}, while addressing some aspects of these challenges, fall short in providing a comprehensive approach. To address the data quality and clinical deployment issues, we propose a multi-faceted solution that encompasses data refinement, model optimization, and integration with existing pathology tools (\hyperref[fig:fig1]{Figure 1}). The outcome is a lightweight cell segmentation and classification model that can be integrated into digital pathology workflows for practical clinical use.

\begin{figure}[h!]
    \centering
    \includegraphics[width=\textwidth, height=0.82\textheight, keepaspectratio]{images/Figure_1.pdf}
    \caption{Overview of the proposed solution, including 1) Data refinement using cross-relabeling, 2) Teacher model development and fine tuning, 3) Student model optimization with knowledge distillation and 4) Student model and QuPath integration}
    \label{fig:fig1}
\end{figure}
\clearpage

Our approach begins with preparing the data for the fine-tuning and training of the machine learning models. We create a refined dataset, acquired via cross-relabeling two cell-level datasets, enhancing annotation specificity and consistency of the labeled data. Subsequently, we create a cell segmentation and classification model based on the foundation model. We leverage the foundation model as a fixed encoder and fine-tune a decoder using the refined dataset to improve generalization across diverse tissue- and cell types.
To ensure that the model remains lightweight and deployable in a possibly resource-constrained environment, we employ knowledge distillation to approximate the functionality of the foundation model. Finally, to facilitate the practical application of our model in digital pathology workflows, we integrate it with the QuPath \cite{Bankhead_Loughrey_etal._2017} application. Each methodological component contributes to the overarching goal of enhancing model performance, generalizability, and usability in clinical settings.

The primary contributions of this paper are:
\begin{enumerate}
    \item \textit{Data labels refinement through cross-relabeling:}
    
    We propose a new method for refining labels of cell-level datasets through cross-relabeling. This method employs classification models to re-label broad and ambiguous instances, resulting in a more diverse dataset. Our evaluation demonstrates that these classification models achieve high accuracy on test subsets, indicating the reliability of the method for label refinement.

    \item \textit{Enhanced model performance via foundation models:}
    
    We employ a foundation model as a feature extractor for the cell segmentation and classification task. In comparison with training a CNN model from scratch, the foundation model backbone only needs fine-tuning, which significantly reduces training time, computational resources and data requirements. We show that using a foundation model encoder leads to better performance in cell segmentation and classification networks than using a CNN-based encoder. This improvement may enable the model to generalize more effectively across various tissue types and imaging methods.
    
    \item \textit{Model optimization through knowledge distillation:}
    
    We show that a smaller student model trained using knowledge distillation on the refined dataset obtained via our cross-relabeling approach from a foundation model achieves comparable performance in cell segmentation and quantification tasks. As a result, this model is more suitable for deployment in environments without high-performance computing resources.
    
    \item \textit{Integration with QuPath:}
    
    We integrate the distilled cell segmentation and classification model into QuPath, a widely used open-source digital pathology platform, to accelerate clinical adaptation by enabling pathologists to more easily incorporate advanced computational tools into their existing workflows.
\end{enumerate}

Through these methodological steps, we aim to bridge the gap between advanced machine learning techniques and practical clinical applications, making accurate and efficient digital pathology accessible in a broader range of healthcare settings.

\section{Refining Existing Datasets Using Cross-Relabeling}
To address the limitations of sparse and ambiguous labeling of cell-level datasets, we propose a generalizable cross-relabeling strategy that can be applied to any dataset containing broadly categorized or imprecisely labeled cell types. This approach involves training and subsequently leveraging classification models to refine broad categories into more specific or biologically relevant classes.
When applied to cell-level data, the methodology includes extracting individual cell images from the dataset patches, preprocessing these images to standardize the size and accommodate partial cells, and then training deep learning classifiers capable of distinguishing between the finer cell subtypes within the coarser categories. 
To illustrate our approach, we focus on the PanNuke \cite{Gamper_Koohbanani_etal._2020, Gamper_Koohbanani_etal._2019} and MoNuSAC \cite{Verma_Kumar_etal._2021} datasets that we have used to train models for cell quantification in our previous works \cite{Shvetsov_Grønnesby_etal._2022,Shvetsov_Sildnes_etal._2024}. We find that for better cell differentiation we have to introduce more granular labels. PanNuke includes a broad classification of "inflammatory" cells, encompassing lymphocytes, macrophages, and neutrophils. Each cell type differs significantly in structure, function, and clinical relevance. Conversely, MoNuSAC uses the label "epithelial" for a class that comprises both benign epithelial cells and malignant neoplastic cells. This practice makes it challenging to differentiate between benign and malignant epithelial cells in the dataset, which is a critical distinction when identifying tumor areas within tissue samples. To address these issues, we implement a cross-relabeling strategy as shown in \hyperref[fig:fig2]{Figure 2}. The key components are two classification models: one is trained on singular cell images from PanNuke data to classify the epithelial meta-class into epithelial and neoplastic classes. The other is trained on MoNuSAC to refine the inflammatory class into lymphocytes, neutrophils, and macrophages.

\begin{figure}[h!]
    \centering
    \includegraphics[width=\textwidth]{images/Figure_2.pdf}
    \caption{Refined dataset generation via cross relabeling}
    \label{fig:fig2}
\end{figure}

The refining approach consists of three consecutive steps. The first is the preprocessing step, in which we extract individual cells from both datasets (\hyperref[fig:fig3]{Figure 3}). The specifics of PanNuke and MoNuSAC patch preparation before cell preprocessing are provided in \hyperref[chap:S1]{Appendix S1}.

\begin{figure}[h!]
    \centering
    \includegraphics[width=\textwidth]{images/Figure_3.pdf}
    \caption{Cell instances preprocessing including (1) cell map extraction, (2) bounding box delineation, (3) adjusting cell boxes and (4) cropping and resizing of cell images}
    \label{fig:fig3}
\end{figure}

During preprocessing, we extract cell type maps from the ground truth label mask and calculate bounding boxes around each cell instance. To accommodate partial cells at patch borders, a common issue in cropped patch images, we employ mirror padding and extend the field of view of the cell label by 15 pixels to capture adjacent cells. We then crop and resize the identified regions to $64 \times 64$ pixels using bicubic interpolation.

The preprocessed PanNuke dataset comprises 68,031 neoplastic and 23,207 epithelial cell images, while MoNuSAC comprises  33,104 lymphocytes, 1,252 neutrophils, and 1,695 macrophages, which we subsequently use in training cell classification models and classifying the cell image data \hyperref[fig:S2]{Appendix Figure S2 (1)}. 

The next step is to train two distinct ResNet50-based classifiers tailored to address the specific labeling challenges inherent in each dataset. We use ResNet50 for classification models due to its proven effectiveness for image classification tasks in histopathology \cite{pan2022reviewmachinelearningapproaches}, and its compatibility with small images. For the PanNuke dataset, we design the classifier, trained on MoNuSAC data, to disaggregate the heterogeneous "inflammatory" cell category into distinct subtypes: lymphocytes, macrophages, and neutrophils. Similarly, for the MoNuSAC dataset, the classifier is trained on PanNuke data and distinguishes between benign and malignant epithelial cells within the overarching "epithelial" label. By applying these targeted classifiers to their respective datasets, we assign more specific labels to individual cell instances, thus enabling us to create a unified dataset.
To ensure a balanced representation of classes, we train both models on datasets that had been equalized to match the size of the least represented class. Thus, we obtain datasets comprising 23,207 samples per class for PanNuke and 1,252 samples per class for MoNuSAC data. Next, we partition both of them into training (70\%), validation (20\%), and testing (10\%) subsets. To mitigate the risk of overfitting, we use a single dropout layer with a rate of p=0.5 in both models and data augmentation using randomized color perturbations, rotation, and horizontal and vertical flipping. We employ AdamW optimizer and the cross-entropy loss function for the training criterion.

To evaluate the two trained models, we measure the classification accuracy on the respective test subsets. The accuracies on the test subset for both classifiers are presented in \hyperref[tab:1]{Table 1}. The PanNuke model achieves an average accuracy of 93.57\%, with higher accuracy for neoplastic cells (96.06\%) compared to epithelial cells (86.26\%). The confusion matrix in Figure A3.1 shows that the model predominantly distinguishes accurately between epithelial and neoplastic tissues, with a substantial number of correct classifications and relatively few misclassifications. The MoNuSAC model demonstrates an average accuracy of 98.92\%, excelling in classifying lymphocytes (99.67\%) and macrophages (94.12\%), with lower performance for neutrophils (85.71\%). The confusion matrix in Figure A3.2 shows that the model identifies lymphocytes and performs reasonably well with macrophages and neutrophils.

\begin{table}[h!]
\renewcommand{\arraystretch}{1.5}
  \centering
  \caption{Cell classification results for PanNuke and MoNuSAC trained models (CI 95\%).}
  \label{tab:1}
  \begin{tabular}{|l|c|c|}
   \hline
   %\rowcolor{gray!30}
    Accuracy               & PanNuke model              & MoNuSAC model              \\
    \hline
    Average      & 0.936 (0.931--0.941)         & 0.989 (0.986--0.993)        \\
    \hline
    Neoplastic   & 0.961 (0.956--0.965)         & -                          \\
    \hline
    Epithelial   & 0.863 (0.849--0.877)         & -                          \\
    \hline
    Lymphocytes  & -                          & 0.997 (0.995--0.999)        \\
    \hline
    Neutrophils  & -                          & 0.857 (0.796--0.918)        \\
    \hline
    Macrophages  & -                          & 0.941 (0.906--0.976)        \\
    \hline
  \end{tabular}
\end{table}

Finally, during the last step, we use the model trained on PanNuke data for epithelial cells in MoNuSAC and the model trained on MoNuSAC for the inflammatory cells class in PanNuke. Specifically, we use classifier models to relabel epithelial cells in MoNuSAC and inflammatory cells in PanNuke data. Then we combine cells with refined labels and the rest of the cells in both datasets to create a refined dataset (\hyperref[fig:S2]{Appendix Figure S2 (2)}). The process of relabeling cells and visualizing them on a patch is shown in \hyperref[fig:fig4]{Figure 4}. The cell counts in the refined dataset are provided in \hyperref[tab:S4]{Appendix Table S4}.

\begin{figure}[h!]
    \centering
    \includegraphics[width=\textwidth, height=0.42\textheight, keepaspectratio]{images/Figure_4.pdf}
    \caption{Cell relabeling procedure for epithelial and inflammatory cell classes}
    \label{fig:fig4}
\end{figure}

%\hfill

Relabeling and combining datasets have been explored in a prior study \cite{Parulekar_Kanwat_etal._2023}, where consecutive fine-tuning on multiple datasets was employed to account for hierarchical class label structures. While the method presented in \cite{Parulekar_Kanwat_etal._2023} is intuitive, it often lacks consistency and requires multiple fine-tuning runs, which can be cumbersome and time-consuming. 
In contrast, cross-relabeling simplifies this process by using specialized classification models tailored to each dataset's specific labeling challenges. This approach provides better transparency and produces a unified dataset encompassing seven distinct cell types across multiple tissue samples, enhancing data diversity for further model training or fine-tuning.

Despite these improvements, cross-relabeling does not entirely resolve issues related to poor labeling quality or the amount of labeled data. Specifically, our results show lower accuracies persist for underrepresented classes, such as macrophages, which may stem from a limited sample availability and intrinsic challenges in distinguishing these cells based solely on H\&E staining. Furthermore, while our method enhances label specificity, it relies on the initial quality of the broad labels; thus, any fundamental inaccuracies in the original annotations can propagate through the relabeling process. Addressing the overall problem of limited data labels may require integrating additional data sources or utilizing complementary immunohistochemical staining methods.
Although the reported performance metrics are obtained from evaluations on the native test sets of each dataset, it is important to note that the primary application of these classifiers is to perform cross-relabeling, where a model trained on one dataset (e.g., PanNuke) is applied to another (e.g., MoNuSAC) and vice versa. We acknowledge that a more systematic evaluation of cross-dataset generalization is needed and could be performed in future work.

Overall, the refined dataset produced by our approach can enhance the supervised training or fine-tuning of cell segmentation and classification models, especially those that utilize pre-trained foundation models to improve feature extraction robustness. In addition, these models can detect nuanced classes that enable researchers to conduct more detailed analyses of biological processes in computational pathology.

\section{Foundation models for robust cell segmentation and classification}

Accurate cell segmentation and classification in digital pathology are hindered by limited labeled data and the fact that conventional CNNs are unable to capture global contextual information due to their local receptive field constraints \cite{Gheflati_Rivaz_2022,Yang_Marcus_etal.}. Traditional approaches in cell quantification have predominantly relied on CNN encoders, such as ResNet50, given their proven effectiveness in semantic segmentation tasks \cite{Deshmane_2023,Graham_Vu_etal._2019,Mukasheva_Koishiyeva_etal._2024,Stringer_Wang_etal._2021}. However, approaches that include fine-tuning of pretrained CNNs, data augmentation, and stain normalization to partially increase data variability and address staining differences often fail to achieve the necessary generalization and robustness across diverse tissue types and staining conditions \cite{G._Wang_W._Li_etal._2018,Gao_Bagci_etal._2018,Karim_El_Khoury_Martin_Fockedey_etal._2021}.

To overcome these challenges, we leverage an encoder-decoder network that uses a foundation model as the encoder and a CNN upsampling decoder (\hyperref[fig:fig5]{Figure 5}) for simultaneous cell segmentation and classification in 2D patches extracted from WSIs. Foundation models with transformer-based architectures are viable alternatives to CNN-based encoders \cite{Shamshad_Khan_etal._2023,Sourget_2023}. They enable the creation of more advanced architectures that can decode or transform learned features more effectively \cite{Chen_Duan_etal._2023,Cheng_Misra_etal._2022,Xie_Wang_etal._2021}.

\begin{figure}[h!]
    \centering
    \includegraphics[width=\textwidth]{images/Figure_5.pdf}
    \caption{UNETR-like model with foundational model as backbone}
    \label{fig:fig5}
\end{figure}

By utilizing a transformer-based encoder, we incorporate global contextual information into the feature extraction process, which is a key advantage of such architectures \cite{Chen_Lu_etal._2021}. This foundation model integration facilitates accurate pixel-wise segmentation and classification without the need for extensive encoder training, thereby potentially improving generalization across varied cellular structures and tissue types.
In our implementation, we employ a modified UNETR \cite{Hatamizadeh_Tang_etal._2021} architecture that combines a vision transformer (ViT) \cite{Dosovitskiy_Beyer_etal._2021} encoder with a CNN-based decoder. The encoder utilizes the pretrained H-Optimus foundation model, which contains 1.1 billion parameters and is trained on over 500,000 H\&E stained WSIs \cite{Saillard_Jenatton_etal._2024}. We extract outputs from four evenly spaced transformer blocks $Z_i$, where $i \in [1, 14, 26, 38]$, to serve as residual connections for the CNN decoder. We select these blocks based on our observation that features from non-adjacent levels of the encoder lead to better overall performance on the test subset.

The CNN decoder upsamples the feature representations, acquired from the transformer blocks, to generate an intermediate vector that is handled by two task-specific layers that generate cell segmentation and classification masks. The first task-specific layer is the ‘Cellpose head’,  which is used to delineate cell instances. The layer generates horizontal and vertical gradient maps to form vector fields that are refined through gradient tracking in a post-processing step using the Cellpose algorithm \cite{Stringer_Wang_etal._2021}, known for its efficacy in cell segmentation tasks and generalizability across multiple domains \cite{Pachitariu_Stringer_2022,Stringer_Pachitariu_2024}. The second task-specific layer is the "Cell type head", which assigns labels to individual pixels. In the post-processing step, we determine the output classification label of each segmented cell instance by majority voting over the labeled pixels that comprise the cell in the segmentation map.

To evaluate model performance and measure the impact of adding a foundation model as backbone, we compare it to a ResNet50-based model. ResNet50 is a widely used solution for encoders in segmentation architectures in the medical domain \cite{Deshmane_2023,Graham_Vu_etal._2019,Mukasheva_Koishiyeva_etal._2024,Stringer_Wang_etal._2021}. For the H-Optimus-based model, we utilize frozen weights for the encoder and only fine-tune the decoder to take advantage of the extensive pre-training of the foundation model. For the ResNet50-based model we start with ImageNet \cite{Deng_Dong_etal.} weights and train both encoder and decoder parts. Hyperparameters for the training step are set to be identical, where possible, for comparable evaluation. 
For this evaluation, we deliberately use the PanNuke dataset to provide a standardized and controlled comparison between the H‑Optimus and ResNet50-based models (\hyperref[fig:S2]{Appendix Figure S2 (3)}). Specifically, we use two of the default PanNuke dataset splits (66\%) for training and validation, and reserve the third split (33\%) for testing.

To address the challenge of cell class imbalance in the PanNuke dataset, which is a common characteristic in most cell-level H\&E patch datasets, both models’ training processes employ a weighted loss function comprising cross-entropy and focal loss \cite{Lin_Goyal_etal._2018}. The focal loss component is adjusted with coefficients derived from each cell class' instance frequency, emphasizing learning from underrepresented classes and enhancing the model's sensitivity to rare but significant cellular patterns. The cross-entropy loss is augmented with spectral decoupling regularization \cite{Pezeshki_Kaba_etal._2021,Pohjonen_Stürenberg_etal._2022} and spatially varying label smoothing \cite{Islam_Glocker_2021}, which potentially stabilizes training and improves generalization in case of complex tissue morphologies. For optimization, we employ the \textit{AdamW} \cite{Loshchilov_Hutter_2019} to counter unbalanced class scenarios, with cosine annealing learning rate scheduler.

We utilize the scikit-learn library \cite{Van_der_Walt_Schönberger_etal._2014} and HoVer-Net \cite{Graham_Vu_etal._2019} implementations of $R^2$ (the coefficient of determination) and $PQ$ (panoptic quality) to evaluate our experiments. Complete mathematical formulations and detailed explanations of these metrics are provided in \hyperref[chap:S5]{Appendix S5}. To compute confidence intervals, we use nonparametric bootstrapping, where after calculating the metric on the full sample, we generated 1000 bootstrap replicates by resampling with replacement and then determined the 95\% confidence intervals as the 2.5th and 97.5th percentiles of the resulting empirical distribution.

%\hfill

The model comparisons are summarized in \hyperref[tab:2]{Table 2}. The H‑Optimus-based model achieves higher $R^2$ across all cell classes compared to the ResNet50-based model, which means that its predictions are more closely aligned with the PanNuke cell counts, indicating a stronger correlation with the observed data. Notably, the improvement of $R^2_{dead}$ may be an indicator of better global contextual representations provided by the foundation model backbone. In terms of segmentation and classification quality combined, measured by the PQ score, the H‑Optimus-based model demonstrates notable improvements across most cell classes. Overall, the average $R^2$ improved from 0.575 to 0.871, while the average $PQ$ score improved from 0.450 to 0.492, demonstrating better performance of the H-Optimus-based model.

\begin{table}[h!]
\renewcommand{\arraystretch}{1.5}
  \centering
  \caption{Cell quantification metrics for baseline and proposed models (CI 95\%).}
  \label{tab:2}
  \begin{tabular}{|l|c|c|}
    \hline
    %\rowcolor{gray!30}
    Metric             & Resnet50-based            & H-optimus-based              \\
    \hline
    $R^2_{neoplastic}$    & 0.681 (0.576--0.769)       & \textbf{0.941 (0.917--0.960)} \\
    \hline
    $R^2_{inflammatory}$  & 0.863 (0.778--0.903)       & \textbf{0.949 (0.918--0.966)} \\
    \hline
    $R^2_{connective}$    & 0.600 (0.488--0.698)       & 0.609 (0.436--0.772)          \\
    \hline
    $R^2_{dead}$          & 0.097 (-11.389--0.669)     & 0.925 (0.404--0.982)          \\
    \hline
    $R^2_{epithelial}$    & 0.635 (0.490--0.747)       & \textbf{0.930 (0.886--0.964)} \\
    \hline
    $PQ_{neoplastic}$       & 0.517 (0.499--0.535)       & \textbf{0.589 (0.575--0.604)} \\
    \hline
    $PQ_{inflammatory}$     & 0.455 (0.429--0.482)       & \textbf{0.528 (0.507--0.549)} \\
    \hline
    $PQ_{connective}$       & 0.416 (0.400--0.431)       & \textbf{0.451 (0.436--0.465)} \\
    \hline
    $PQ_{dead}$             & 0.374 (0.342--0.408)       & 0.292 (0.209--0.365)          \\
    \hline
    $PQ_{epithelial}$       & 0.488 (0.460--0.519)       & \textbf{0.599 (0.579--0.618)} \\
    \hline
  \end{tabular}
\end{table}

Our results  show that integrating the H‑Optimus foundation model within the UNETR architecture enhances the model's ability to segment and classify cells across diverse tissues from PanNuke data. The pretrained transformer encoder provides robust feature representations, resulting in higher average $R^2$ and $PQ$ scores compared to the CNN-based model. This leads to more reliable cell quantification and more accurate downstream analysis. Additionally, the streamlined fine-tuning process reduces computational overhead and training time, making the model more adaptable for new data.

Despite these advancements, the foundation model-based approach does not fully resolve all challenges related to cell segmentation and classification. We observe lower metric scores for underrepresented classes in the training data. Furthermore, foundation models typically encompass billions of parameters, resulting in substantial computational and memory requirements. It therefore poses challenges for deployment in resource-constrained environments, limiting their practical applicability in certain clinical settings.

\section{Model optimization via Knowledge Distillation}

To address the limitations posed by the extensive size of foundation models, we implement knowledge distillation — a model compression technique that leverages the teacher-student paradigm \cite{Hinton_Vinyals_etal._2015}. By training a smaller, more efficient student model to replicate the output of a larger, pre-trained teacher model, we retain performance while significantly reducing the model's complexity and resource requirements (\hyperref[fig:fig6]{Figure 6}).

\begin{figure}[h!]
    \centering
    \includegraphics[width=\textwidth, height=0.45\textheight, keepaspectratio]{images/Figure_6.pdf}
    \caption{Knowledge distillation framework for training a student model using a pre-trained teacher}
    \label{fig:fig6}
\end{figure}

We employ knowledge distillation to compress the H‑Optimus-based teacher model into a more efficient student model. The teacher model is the modified UNETR architecture with the H‑Optimus foundation model described in the previous chapter. The student model is based on a UNet architecture augmented with residual connections and incorporates a smaller ViT encoder with 9 million parameters \cite{Steiner_Kolesnikov_etal._2022,Wightman_2019}. 

First, we fine-tune the teacher model using the refined dataset from the cross-relabeling procedure (Section 2). Initially we train the decoder of the teacher model while keeping the encoder weights frozen. We split the refined dataset into train (70\%), validation (20\%) and test (10\%) subsets (\hyperref[fig:S2]{Appendix Figure S2 (4)}). During fine-tuning, we use the train and validation subsets, while leaving the test subset for model evaluation. We set the training procedure and model hyperparameters to be identical to those that were used to demonstrate the utility of foundation models for the simultaneous cell segmentation and classification task.

Next, we perform knowledge distillation from teacher to student using the refined dataset used to fine-tune the teacher model. The student model is trained to replicate the teacher model's outputs. We utilize a specialized loss function that aligns the student's predicted probability distribution with the teacher's, incorporating the teacher's class probability distribution derived from the output. Following the methodology of Hinton et al. \cite{Hinton_Vinyals_etal._2015}, we experiment with various hyperparameter settings for the temperature ($T$) and the balancing coefficients ($\alpha$ and $\beta$) in the loss function. We vary $T$ from 1 to 20 and adjust $\alpha$ and $\beta$ to balance the distillation and student losses. Through iterative tuning and evaluation, we identify that setting $T=14$, $\alpha=0.3$, and $\beta=0.7$ yields a configuration that converges and closely approximates the teacher model's performance during training.

Finally, we assess the performance of both models using the $R^2$ and $PQ$ (defined in \hyperref[chap:S5]{Appendix S5}) on the test set of the refined dataset (\hyperref[tab:3]{Table 3}). We observe that the 95\% confidence intervals overlap for most cell types, so we cannot claim statistically significant performance differences between the teacher and student models. One exception appears in the neoplastic class. The teacher model produces an $R^2$ of 0.919, while the student model shows an $R^2$ of 0.852. In addition, the student model achieves higher $PQ$ values for the neoplastic and connective classes, though the confidence intervals show overlap.

\begin{table}[h!]
\renewcommand{\arraystretch}{1.5}
  \centering
  \caption{Cell quantification metrics for teacher and distilled student models (CI 95\%).}
  \label{tab:3}
  \begin{tabular}{|l|c|c|}
    \hline
    %\rowcolor{gray!30}
    Metric & Teacher & Student \\
    \hline
    $R^2_{neoplastic}$    & \textbf{0.919} (0.898--0.939) & 0.852 (0.800--0.891) \\
    \hline
    $R^2_{lymphocyte}$    & 0.969 (0.956--0.977)         & 0.969 (0.956--0.978) \\
    \hline
    $R^2_{connective}$    & 0.694 (0.548--0.809)         & 0.618 (0.469--0.741) \\
    \hline
    $R^2_{dead}$          & 0.755 (0.400--0.908)         & 0.424 (0.100--0.731) \\
    \hline
    $R^2_{epithelial}$    & 0.922 (0.870--0.958)         & 0.843 (0.738--0.917) \\
    \hline
    $R^2_{macrophage}$    & 0.384 (-0.369--0.724)        & 0.704 (0.352--0.859) \\
    \hline
    $R^2_{neutrofil}$     & 0.854 (0.578--0.929)         & 0.833 (0.502--0.925) \\
    \hline
    $PQ_{neoplastic}$       & 0.581 (0.569--0.593)         & 0.601 (0.588--0.613) \\
    \hline
    $PQ_{lymphocyte}$       & 0.536 (0.520--0.553)         & 0.563 (0.544--0.579) \\
    \hline
    $PQ_{connective}$       & 0.436 (0.421--0.451)         & 0.457 (0.441--0.474) \\
    \hline
    $PQ_{dead}$             & 0.272 (0.235--0.315)         & 0.279 (0.201--0.369) \\
    \hline
    $PQ_{epithelial}$       & 0.522 (0.500--0.545)         & 0.530 (0.506--0.555) \\
    \hline
    $PQ_{macrophage}$       & 0.524 (0.459--0.588)         & 0.474 (0.405--0.543) \\
    \hline
    $PQ_{neutrofil}$        & 0.541 (0.490--0.592)         & 0.565 (0.522--0.607) \\
    \hline
  \end{tabular}
\end{table}


We further decompose the $PQ$ metric into its $SQ$ and $DQ$ components (\hyperref[tab:S6]{Appendix Table S6}). Both models produce nearly identical $SQ$ values, which indicates that they predict instance boundaries with similar precision. Although the student model shows some improvement in $DQ$ scores for certain classes, the confidence intervals overlap and do not confirm a statistically significant difference.

We observe that the student and teacher models yield comparable detection performance despite the student model using a much smaller and simpler architecture. A model with fewer parameters reduces the risk of overfitting when training data are scarce relative to the model’s complexity \cite{Farias_Ludermir_etal._2022}. The knowledge distillation process also encourages the student model to focus on the most generalizable detection features learned from the teacher. These factors enable the student model to achieve similar detection performance across different cell types.

Additionally, considering the model sizes reported in \hyperref[tab:4]{Table 4}, the distilled model achieves a significant reduction compared to the teacher model, with a 48-fold decrease in parameter count and a 5.5-fold reduction in on-disk size. In inference mode, the teacher model requires 16 GB of VRAM for a batch size of 32, while the distilled model only needs 3 GB of VRAM for the same batch size. These reductions make the distilled model significantly more practical for fine-tuning and deployment in resource-constrained environments.

\begin{table}[h!]
\renewcommand{\arraystretch}{1.5}
  \centering
  \caption{Parameter counts and size of teacher and distilled model}
  \label{tab:4}
  \adjustbox{max width=\textwidth}{%
  \begin{tabular}{|l|c|c|c|}
    \hline
    %\rowcolor{gray!30}
    Metric & H-optimus-based (Teacher) & mobileViT-based (Student) & Magnitude of difference \\
    \hline
    Parameters count       & 1,158,917,906   & \textbf{24,093,393}   & \textbf{48x}  \\
    \hline
    Estimated Total Size (MB) & 87,912       & \textbf{15,935}    & \textbf{5.5x} \\
    \hline
  \end{tabular}%
}
\end{table}

%\hfill

With recent advancements in complex network architectures and the use of pretrained encoders to achieve state-of-the-art performance \cite{Baumann_Dislich_etal._2024,Hörst_Rempe_etal._2024} in cell segmentation and classification tasks, model size, computational complexity, and processing times have increased. This limits the scalability and accessibility of these models. As we demonstrate, this may be mitigated using knowledge distillation. Studies in the field of natural language processing have demonstrated the efficacy of knowledge distillation in retaining the capabilities of the teacher model while achieving significant reductions in size and complexity \cite{Huangpu_Gao_2024,Sun_Yu_etal.}. 

We demonstrate the feasibility of knowledge distillation in digital pathology, specifically for cell segmentation and classification tasks. Moreover, we achieve this performance while also significantly reducing the parameter count. In addressing the challenge of knowledge transfer, we found that distillation from a transformer-based model to a smaller transformer is more straightforward than attempting to map transformer features to CNN blocks. In our experiments, using a CNN-based network as a student results in worse cell quantification performance due to the structural constraints of CNN feature space dimensions. 

Although our primary approach relies on a transformer-based student model that performs well, it can be further optimized to incorporate advantages from CNN architectures. For example, employing alternative techniques such as using ViT adapters \cite{Chen_Duan_etal._2023} or $1 \times 1$ convolutions to adjust feature map sizes may be beneficial for harnessing CNN advantages like enhanced local feature extraction. Moreover, if additional performance improvements are desired, the process can be further enhanced by applying supplementary knowledge distillation techniques, such as self-distillation \cite{Zhang_Song_etal._2019} or online distillation \cite{Houyon_Cioppa_etal._2023}.

Despite these promising results, further validation on independent datasets is necessary to fully understand the model's limitations. Underrepresented classes may pose challenges when addressing complex cases. Pathologists need to validate these models to adopt them in clinical settings. While the distilled models are smaller and more deployable, a technological gap persists because pathologists traditionally rely on established methods for inspecting WSIs and diagnosing diseases. Addressing the complexities involved in deploying models for inference and supporting pathologists in adopting new tools is essential for integrating these models into clinical workflows.

\section{Model integration with QuPath}
Digital pathology tools with graphical user interfaces are essential for visualizing and analyzing WSIs. To make our student model useful in clinical pathology workflows, it needs to be integrated into a tool that enables inspecting regions, creating annotations, and providing quantitative analyses of biomarkers. Therefore, we integrate the trained student model from the previous chapter into the QuPath open‑source platform \cite{Bankhead_Loughrey_etal._2017}. QuPath provides the required annotation, visualization, and analysis tools to interpret complex histological data, including workflows for cell segmentation, classification, and quantification (\hyperref[fig:fig7]{Figure 7}). 

\begin{figure}[h!]
    \centering
    \includegraphics[width=\textwidth]{images/Figure_7.pdf}
    \caption{Visualization of model-generated cell quantification annotations (left) and the corresponding unannotated slide (right) in QuPath}
    \label{fig:fig7}
\end{figure}

To identify the regions in a WSI critical for prognosticating tumor development, such as specific tumor areas or border regions without overlapping healthy tissue, the pathologist uses QuPath to outline these regions. Then, the pathologist initiates a cell segmentation and classification script through the QuPath interface for the selected regions. The resulting annotations and quantified cell information are then directly overlaid onto the WSI in the QuPath interface. Additional design and implementation details are in \hyperref[chap:S7]{Appendix S7}. 

Two common approaches for integrating deep learning models into QuPath are Java‑based native QuPath extensions \cite{Goldsborough_Philps_etal._2024} and the execution of RESTful API requests to a model server coupled with handling the response via an extension, as demonstrated in the application of cell segmentation models applied to immunofluorescence images \cite{Sugawara_2023}. While the community is actively working on these integration strategies, there is currently no universal solution that fully addresses all integration and performance requirements.

Extensions may offer better integration with QuPath, allowing slightly improved performance and more widespread usage of the built-in QuPath models, but they lack the flexibility to customize models and modify their behavior. For example, the newest version of QuPath includes models such as StarDist \cite{Weigert_Schmidt} and InstanSeg \cite{Goldsborough_Philps_etal._2024} that can perform cell segmentation. Both models pose limitations when applied to simultaneous cell segmentation and classification. StarDist performs well only on convex, round shapes by design, whereas some neoplastic, inflammatory, and connective cells exhibit complex and non-convex shapes. InstanSeg provides only semantic segmentation without assigning classes to the segmented cells.

%\hfill

In contrast, our approach offers an alternative integration strategy. It utilizes the paquo library to directly interact with QuPath’s internal application programming interface from within Python. This enables data exchange and processing without the need for intermediate conversion steps and provides greater control over model customization, retraining, and the incorporation of custom processing steps.

The integration of our custom model with QuPath underscores its potential to significantly enhance the diagnostic process by reducing the time burden on pathologists and enabling them to focus on more complex interpretative tasks using familiar software. Leveraging a tool that is already well-established among pathologists increases the likelihood of its adoption into daily clinical workflows. The quantitative data generated through the automated workflow is critical for both clinical decision-making and research, facilitating more accurate biomarker analysis, enabling robust statistical evaluations, and supporting hypothesis generation and testing. Additionally, by streamlining cell segmentation and classification, the tool enhances the scalability and reproducibility of pathological assessments, ultimately contributing to improved diagnostic accuracy and patient outcomes.

\section{Conclusion and future work}

In this study, we address critical challenges in digital pathology and tackle the usability and deployment issues of the developed models in standard computing environments without the need for high-performance computing systems. Our multi-faceted approach encompasses data refinement through cross-relabeling, leveraging foundation models for robust cell segmentation and classification, optimizing model performance via knowledge distillation, and integrating the optimized model into the QuPath software for practical application. This approach is used to construct a capable, versatile, and adjustable model for cell segmentation and classification, with enhanced performance and usability.

\begin{sloppypar}
While our approach shows potential in the field of computational pathology, certain limitations persist. 
For example, our implementation currently exhibits lower performance in detecting macrophages. 
This serves as an instance of the broader challenge of accurately identifying complex cell types. In order to address this issue, extending our approach to incorporate additional data sources, exploring alternative modeling approaches, and integrating other imaging modalities such as immunohistochemical staining may help improve detection accuracy. Moreover, although the distilled model reduces computational demands, integrating advanced deep learning models into clinical practice requires addressing technological gaps and potential resistance to adopting new tools within established diagnostic processes.
\end{sloppypar}

Future work could focus on several key areas to refine the proposed approach and facilitate its adoption in clinical environments. Enhancing the cell-relabeling process with additional datasets \cite{Graham_Jahanifar_etal._2021} could improve the representation of underrepresented cell types and enhance overall model performance. Also, incorporating additional data sources, such as multi-modal imaging or complementary staining methods, may address limitations related to cell type differentiation and class imbalance. Exploring other foundation models \cite{Vorontsov_Bozkurt_etal._2024,Zimmermann_Vorontsov_etal._2024} or introducing additional modalities \cite{Ding_Wagner_etal._2024,Vaidya_Zhang_etal._2025} may provide alternative architectures better suited to specific tasks or offer improved efficiency. Implementing more complex knowledge distillation techniques \cite{Houyon_Cioppa_etal._2023,Zhang_Song_etal._2019} could further optimize the model's performance and adaptability. Additionally, deeper integration with QuPath or other digital pathology software could provide pathologists more control over cell quantification analysis directly within the QuPath interface, thereby increasing accessibility and usability. Such enhancements would not only refine model performance but also ensure greater adaptability and scalability within various clinical environments. Finally, extensive validation of the model by pathologists and benchmarking against independent datasets are essential steps toward establishing the model's reliability and fostering confidence in its clinical utility.

\section*{Acknowledgments} 
This work was funded in part by the Research Council of Norway grant no. 309439 SFI Visual Intelligence, and the North Norwegian Health Authority grant no. HNF1521-20.

\bibliographystyle{IEEEtran}
\begin{sloppypar}
\begin{thebibliography}{99}

\bibitem{chaplot2020neural} Chaplot, Devendra Singh, et al. "Neural topological slam for visual navigation." Proceedings of the IEEE/CVF conference on computer vision and pattern recognition. 2020.

\bibitem{maksymets2021thda} Maksymets, Oleksandr, et al. "Thda: Treasure hunt data augmentation for semantic navigation." Proceedings of the IEEE/CVF International Conference on Computer Vision. 2021.

\bibitem{mezghan2022memory} Mezghan, Lina, et al. "Memory-augmented reinforcement learning for image-goal navigation." 2022 IEEE/RSJ International Conference on Intelligent Robots and Systems (IROS). IEEE, 2022.

\bibitem{al2022zero} Al-Halah, Ziad, Santhosh Kumar Ramakrishnan, and Kristen Grauman. "Zero experience required: Plug \& play modular transfer learning for semantic visual navigation." Proceedings of the IEEE/CVF Conference on Computer Vision and Pattern Recognition. 2022.

\bibitem{ye2021auxiliary} Ye, Joel, et al. "Auxiliary tasks and exploration enable objectgoal navigation." Proceedings of the IEEE/CVF international conference on computer vision. 2021.

\bibitem{chaplot2020object} Chaplot, Devendra Singh, et al. "Object goal navigation using goal-oriented semantic exploration." Advances in Neural Information Processing Systems 33 (2020)

\bibitem{ramakrishnan2022poni} Ramakrishnan, Santhosh Kumar, et al. "Poni: Potential functions for objectgoal navigation with interaction-free learning." Proceedings of the IEEE/CVF Conference on Computer Vision and Pattern Recognition. 2022.

\bibitem{ramrakhya2022habitat} Ramrakhya, Ram, et al. "Habitat-web: Learning embodied object-search strategies from human demonstrations at scale." Proceedings of the IEEE/CVF Conference on Computer Vision and Pattern Recognition. 2022.

\bibitem{mousavian2019visual} Mousavian, Arsalan, et al. "Visual representations for semantic target driven navigation." 2019 International Conference on Robotics and Automation (ICRA). IEEE, 2019.

\bibitem{dhariwal2021diffusion} Dhariwal, Prafulla, and Alexander Nichol. "Diffusion models beat gans on image synthesis." Advances in neural information processing systems 34 (2021)

\bibitem{ho2022classifier} Ho, Jonathan, and Tim Salimans. "Classifier-free diffusion guidance." arXiv preprint arXiv:2207.12598 (2022).

\bibitem{nichol2021glide} Nichol, Alex, et al. "Glide: Towards photorealistic image generation and editing with text-guided diffusion models." arXiv preprint arXiv:2112.10741 (2021)

\bibitem{brooks2023instructpix2pix} Brooks, Tim, Aleksander Holynski, and Alexei A. Efros. "Instructpix2pix: Learning to follow image editing instructions." Proceedings of the IEEE/CVF Conference on Computer Vision and Pattern Recognition. 2023.

\bibitem{fu2023guiding} Fu, Tsu-Jui, et al. "Guiding instruction-based image editing via multimodal large language models." arXiv preprint arXiv:2309.17102 (2023).

\bibitem{geng2024instructdiffusion} Geng, Zigang, et al. "Instructdiffusion: A generalist modeling interface for vision tasks." Proceedings of the IEEE/CVF Conference on Computer Vision and Pattern Recognition. 2024.

\bibitem{zhou2024minedreamer} Zhou, Enshen, et al. "Minedreamer: Learning to follow instructions via chain-of-imagination for simulated-world control." arXiv preprint arXiv:2403.12037 (2024).

\bibitem{zhou2023esc} Zhou, Kaiwen, et al. "Esc: Exploration with soft commonsense constraints for zero-shot object navigation." International Conference on Machine Learning. PMLR, 2023.

\bibitem{yu2023l3mvn} Yu, Bangguo, Hamidreza Kasaei, and Ming Cao. "L3mvn: Leveraging large language models for visual target navigation." 2023 IEEE/RSJ International Conference on Intelligent Robots and Systems (IROS). IEEE, 2023.

\bibitem{gadre2023cows} Gadre, Samir Yitzhak, et al. "Cows on pasture: Baselines and benchmarks for language-driven zero-shot object navigation." Proceedings of the IEEE/CVF Conference on Computer Vision and Pattern Recognition. 2023.

\bibitem{shah2023navigation} Shah, Dhruv, et al. "Navigation with large language models: Semantic guesswork as a heuristic for planning." Conference on Robot Learning. PMLR, 2023.

\bibitem{cai2024bridging} Cai, Wenzhe, et al. "Bridging zero-shot object navigation and foundation models through pixel-guided navigation skill." 2024 IEEE International Conference on Robotics and Automation (ICRA). IEEE, 2024.

\bibitem{yu2023co} Yu, Bangguo, Hamidreza Kasaei, and Ming Cao. "Co-NavGPT: Multi-robot cooperative visual semantic navigation using large language models." arXiv preprint arXiv:2310.07937 (2023).

\bibitem{wu2024voronav} Wu, Pengying, et al. "Voronav: Voronoi-based zero-shot object navigation with large language model." arXiv preprint arXiv:2401.02695 (2024).

\bibitem{qin2023mp5} Qin, Yiran, et al. "Mp5: A multi-modal open-ended embodied system in minecraft via active perception." arXiv preprint arXiv:2312.07472 (2023).

\bibitem{du2024learning} Du, Yilun, et al. "Learning universal policies via text-guided video generation." Advances in Neural Information Processing Systems 36 (2024).

\bibitem{ajay2024compositional} Ajay, Anurag, et al. "Compositional foundation models for hierarchical planning." Advances in Neural Information Processing Systems 36 (2024).

\bibitem{liang2024skilldiffuser} Liang, Zhixuan, et al. "Skilldiffuser: Interpretable hierarchical planning via skill abstractions in diffusion-based task execution." Proceedings of the IEEE/CVF Conference on Computer Vision and Pattern Recognition. 2024.

\bibitem{heusel2017gans} Heusel, Martin, et al. "Gans trained by a two time-scale update rule converge to a local nash equilibrium." Advances in neural information processing systems 30 (2017).

\bibitem{zhang2018unreasonable} Zhang, Richard, et al. "The unreasonable effectiveness of deep features as a perceptual metric." Proceedings of the IEEE conference on computer vision and pattern recognition. 2018.

\bibitem{brown2020language} Brown, Tom B. "Language models are few-shot learners." arXiv preprint arXiv:2005.14165 (2020).

\bibitem{podell2023sdxl} Podell, Dustin, et al. "Sdxl: Improving latent diffusion models for high-resolution image synthesis." arXiv preprint arXiv:2307.01952 (2023).

\bibitem{brohan2022rt} Brohan, Anthony, et al. "Rt-1: Robotics transformer for real-world control at scale." arXiv preprint arXiv:2212.06817 (2022).

\bibitem{brohan2023rt} Brohan, Anthony, et al. "Rt-2: Vision-language-action models transfer web knowledge to robotic control." arXiv preprint arXiv:2307.15818 (2023).

\bibitem{li2024manipllm} Li, Xiaoqi, et al. "Manipllm: Embodied multimodal large language model for object-centric robotic manipulation." Proceedings of the IEEE/CVF Conference on Computer Vision and Pattern Recognition. 2024.

\bibitem{shah2023vint} Shah, Dhruv, et al. "ViNT: A foundation model for visual navigation." arXiv preprint arXiv:2306.14846 (2023).

\bibitem{liu2024visual} Liu, Haotian, et al. "Visual instruction tuning." Advances in neural information processing systems 36 (2024).

\bibitem{hu2021lora} Hu, Edward J., et al. "Lora: Low-rank adaptation of large language models." arXiv preprint arXiv:2106.09685 (2021).

\bibitem{qin2023supfusion} Qin, Yiran, et al. "SupFusion: Supervised LiDAR-camera fusion for 3D object detection." Proceedings of the IEEE/CVF International Conference on Computer Vision. 2023.

\bibitem{qin2024worldsimbench} Qin, Yiran, et al. "Worldsimbench: Towards video generation models as world simulators." arXiv preprint arXiv:2410.18072 (2024).

\bibitem{yu2025gamefactory} Yu, Jiwen, et al. "GameFactory: Creating New Games with Generative Interactive Videos." arXiv preprint arXiv:2501.08325 (2025).

\bibitem{zhou2024code} Zhou, Enshen, et al. "Code-as-Monitor: Constraint-aware Visual Programming for Reactive and Proactive Robotic Failure Detection." arXiv preprint arXiv:2412.04455 (2024).

\bibitem{zhang2024ad} Zhang, Zaibin, et al. "AD-H: Autonomous Driving with Hierarchical Agents." arXiv preprint arXiv:2406.03474 (2024).

\bibitem{wang2024toward} Wang, Chaoqun, et al. "Toward Accurate Camera-based 3D Object Detection via Cascade Depth Estimation and Calibration." arXiv preprint arXiv:2402.04883 (2024).

\bibitem{huang2024story3d} Huang, Yuzhou, et al. "Story3d-agent: Exploring 3d storytelling visualization with large language models." arXiv preprint arXiv:2408.11801 (2024).

\bibitem{savinov2018semi} Savinov, Nikolay, Alexey Dosovitskiy, and Vladlen Koltun. "Semi-parametric topological memory for navigation." arXiv preprint arXiv:1803.00653 (2018).

\bibitem{majumdar2022zson} Majumdar, Arjun, et al. "Zson: Zero-shot object-goal navigation using multimodal goal embeddings." Advances in Neural Information Processing Systems 35 (2022): 32340-32352.

\bibitem{yadav2023offline} Yadav, Karmesh, et al. "Offline visual representation learning for embodied navigation." Workshop on Reincarnating Reinforcement Learning at ICLR 2023. 2023.

\bibitem{yadav2023ovrl} Yadav, Karmesh, et al. "Ovrl-v2: A simple state-of-art baseline for imagenav and objectnav." arXiv preprint arXiv:2303.07798 (2023).

\bibitem{sun2024fgprompt} Sun, Xinyu, et al. "FGPrompt: fine-grained goal prompting for image-goal navigation." Advances in Neural Information Processing Systems 36 (2024).

\bibitem{zhu2017target} Zhu, Yuke, et al. "Target-driven visual navigation in indoor scenes using deep reinforcement learning." 2017 IEEE international conference on robotics and automation (ICRA). IEEE, 2017.

\bibitem{koh2024generating} Koh, Jing Yu, Daniel Fried, and Russ R. Salakhutdinov. "Generating images with multimodal language models." Advances in Neural Information Processing Systems 36 (2024).

\bibitem{krantz2022instance} Krantz, Jacob, et al. "Instance-specific image goal navigation: Training embodied agents to find object instances." arXiv preprint arXiv:2211.15876 (2022).

\bibitem{schulman2017proximal} Schulman, John, et al. "Proximal policy optimization algorithms." arXiv preprint arXiv:1707.06347 (2017).

\bibitem{anderson2018evaluation} Anderson, Peter, et al. "On evaluation of embodied navigation agents." arXiv preprint arXiv:1807.06757 (2018).

\bibitem{lin2024navcot} Lin, Bingqian, et al. "NavCoT: Boosting LLM-Based Vision-and-Language Navigation via Learning Disentangled Reasoning." arXiv preprint arXiv:2403.07376 (2024).

\bibitem{NavGPT} Zhou, Gengze, Yicong Hong, and Qi Wu. "Navgpt: Explicit reasoning in vision-and-language navigation with large language models." Proceedings of the AAAI Conference on Artificial Intelligence.

\bibitem{hahn2021no} Hahn, Meera, et al. "No rl, no simulation: Learning to navigate without navigating." Advances in Neural Information Processing Systems 34 (2021): 26661-26673.

\bibitem{li2025t2isafety} Li, Lijun, et al. "T2ISafety: Benchmark for Assessing Fairness, Toxicity, and Privacy in Image Generation." arXiv preprint arXiv:2501.12612 (2025).

\bibitem{an2024agfsync} An, Jingkun, et al. "AGFSync: Leveraging AI-Generated Feedback for Preference Optimization in Text-to-Image Generation." arXiv preprint arXiv:2403.13352 (2024).


\end{thebibliography}
\end{sloppypar}

\clearpage
\beginsupplement
\section*{Appendix}
\renewcommand{\thesubsection}{S\arabic{subsection}}

\subsection{\label{chap:S1}PanNuke and MoNuSAC preprocessing}
The PanNuke dataset comprises a set of 7,901 RGB patches, each with dimensions of $256 \times 256$ pixels, which we set as the standard patch size for our analysis. In contrast, the MoNuSAC dataset encompasses 294 images of heterogeneous dimensions. To standardize the MoNuSAC images with our experiments, we implement a standardization protocol. Specifically, for images exceeding the dimensions of $256 \times 256$ pixels, we segment them into equal-sized patches and apply mirror padding to the remaining portions to avoid information loss at the peripherals. Patches with dimensions less than $128 \times 128$ pixels are excluded from the dataset due to the insufficient resolution to capture relevant cellular details. For patches where either dimension falls between 128 and 256 pixels, we employ upsampling to achieve the standard patch size. As a result, we obtain a total of 2,823 RGB patches derived from the MoNuSAC dataset for subsequent analysis. For additional details on the MoNuSAC data preparation process, refer to the source code \cite{Shvetsov_2025a}.
\clearpage

\subsection{\label{chap:S2}Data usage for the methodology}

\counterwithin{figure}{subsection}
\renewcommand{\thefigure}{S\arabic{subsection}}

\begin{figure}[h!]
    \centering
    \includegraphics[width=\textwidth, height=0.85\textheight, keepaspectratio]{images/A2.pdf}
    \caption{Overview of the methodology for cross-labeling, dataset refinement, and model comparison. (1) Cross-relabeling - training and testing cell classification models, (2) Cross-relabeling - using cell classification models to create refined dataset, (3) Fine-tuning and training models for comparison, (4) Student knowledge distillation with refined dataset}
    \label{fig:S2}
\end{figure}
\clearpage

\subsection{\label{chap:S3}Confusion matrices for classification models}
\counterwithin{figure}{subsection}
\renewcommand{\thefigure}{S\arabic{subsection}.\arabic{figure}}

\begin{figure}[h!]
    \centering
    \includegraphics[width=\textwidth, height=0.4\textheight, keepaspectratio]{images/A3_1.pdf}
    \caption{Confusion matrix for PanNuke trained model}
    \label{fig:S3.1}
\end{figure}

\begin{figure}[h!]
    \centering
    \includegraphics[width=\textwidth, height=0.4\textheight, keepaspectratio]{images/A3_2.pdf}
    \caption{Confusion matrix for MoNuSAC trained model}
    \label{fig:S3.2}
\end{figure}

\clearpage

\subsection{\label{chap:S4}Datasets cell counts}

\counterwithin{table}{subsection}
\renewcommand{\thetable}{S\arabic{subsection}}

\begin{table}[h!]
\renewcommand{\arraystretch}{2.0}
\centering
\caption{\label{tab:S4}Cell counts for PanNuke, MoNuSAC and refined datasets. Numbers in parentheses indicate preprocessed cell counts for cell classifier models training and testing.}
%\adjustbox{max width=\textwidth}{%
\begin{tabular}{|l|c|c|c|}
\hline
%\rowcolor{gray!30}
Cell type & PanNuke & MoNuSAC & Refined \\
\hline
Neoplastic & 77,403 (68,031) & - & 105,451 \\
\hline
Epithelial & 26,572 (23,207) & - & 29,926 \\
\hline
Epithelial (benign and malignant) & - & 31,402 & - \\
\hline
Inflammatory & 32,276 & - & - \\
\hline
Lymphocytes & - & 37,045 (33,104) & 65,275 \\
\hline
Neutrophils & - & 1,355 (1,252) & 3,833 \\
\hline
Macrophage & - & 1,842 (1,695) & 3,410 \\
\hline
Dead & 2,908 & - & 2,908 \\
\hline
Connective & 50,585 & - & 50,585 \\
\hline
\end{tabular}
%
%}
\end{table}



\clearpage

\subsection{\label{chap:S5}Definition of validation metrics}
\counterwithin{equation}{subsection}
\renewcommand{\theequation}{\arabic{equation}}

\subsubsection{\label{chap:S5.1}R\textsuperscript{2}}
The coefficient of determination, denoted as $R^2$, is a statistical measure that represents the proportion of variance in the dependent variable that is predictable from the independent variables. In the context of cell quantification in pathology, $R^2$ is used to assess how well the predicted quantities of different cell types in a patch align with the actual quantities observed in the ground truth data, with higher values representing more accurate quantification. $R^2$ is defined as
\begin{equation*}
R^2 = 1 - \frac{\sum_{i=1}^n (y_i - \hat{y}_i)^2}{\sum_{i=1}^n (y_i - \bar{y})^2},
\end{equation*}
where $y_i$ represents the actual number of cells of a specific type in the $i$-th image, $\hat{y}_i$ represents the predicted number of cells of that type in the $i$-th image, $\bar{y}$ is the mean of the actual numbers across all images, and $n$ is the total number of images in the dataset.

The $R^2$ metric has a range of $(-\infty, 1]$. An $R^2$ of 1 indicates perfect prediction, where all predicted values exactly match the actual values. An $R^2$ of 0 suggests that the model explains none of the variability of the response data around its mean. If $R^2$ is negative, it indicates that the model performs worse than a model that simply predicts the mean of the actual values for all observations.

\subsubsection{\label{chap:S5.2}PQ}
Panoptic Quality ($PQ$) is a comprehensive metric used to evaluate the performance of segmentation models in tasks that require both instance segmentation and classification. $PQ$ provides a single score that encapsulates both the detection accuracy (i.e., how many objects were correctly identified) and the segmentation quality (i.e., how accurately the objects' boundaries were delineated). This metric is particularly useful in multiclass scenarios where each pixel is classified into distinct categories, such as different cell types in pathology images.

$PQ$ is calculated as the product of two terms: Detection Quality ($DQ$) and Segmentation Quality ($SQ$). It can be expressed as
\begin{equation*}
PQ = DQ \cdot SQ,
\end{equation*}
where
\begin{equation*}
DQ = \frac{TP}{TP + 0.5\, FP + 0.5\, FN},
\end{equation*}
\begin{equation*}
SQ = \frac{\sum_{(p, g) \in \mathcal{M}} IoU(p, g)}{TP}.
\end{equation*}
In these formulas, $TP$ denotes the number of correctly matched instances between ground truth and prediction, $FP$ denotes the predicted instances that have no corresponding ground truth, $FN$ denotes the ground truth instances that were not detected, $IoU(p, g)$ is the Intersection over Union for a pair of matched instances $p$ (prediction) and $g$ (ground truth), and $\mathcal{M}$ is the set of matched pairs.

The $PQ$ metric is calculated for each class and is averaged across classes to provide a global performance measure.

The $PQ$ score has a range of $[0, 1.0]$, where a higher score indicates better performance in both detecting and segmenting the instances correctly. A $PQ$ of 1 signifies perfect identification and segmentation of all instances, whereas a $PQ$ of 0 indicates that no instances were correctly identified and segmented.

\clearpage

\subsection{\label{chap:S6}Segmentation and Detection quality metrics for teacher and student models}

\begin{table}[h!]
\renewcommand{\arraystretch}{2.0}
\centering
\caption{Segmentation and detection quality for student and teacher models (CI 95\%)}
\label{tab:S6}
%\adjustbox{max width=\textwidth}{%
\begin{tabular}{|l|c|c|}
\hline
%\rowcolor{gray!30}
Metric & Teacher & Student \\
\hline
$SQ_{neoplastic}$ & 0.819 (0.815--0.823) & 0.824 (0.819--0.828) \\
\hline
$SQ_{lymphocyte}$ & 0.795 (0.788--0.802) & 0.790 (0.783--0.796) \\
\hline
$SQ_{connective}$ & 0.770 (0.762--0.776) & 0.780 (0.772--0.786) \\
\hline
$SQ_{dead}$ & 0.659 (0.623--0.688) & 0.657 (0.624--0.695) \\
\hline
$SQ_{epithelial}$ & 0.780 (0.770--0.790) & 0.788 (0.779--0.797) \\
\hline
$SQ_{macrophage}$ & 0.788 (0.760--0.810) & 0.757 (0.730--0.783) \\
\hline
$SQ_{neutrofil}$ & 0.782 (0.761--0.801) & 0.775 (0.759--0.792) \\
\hline
$DQ_{neoplastic}$ & 0.706 (0.692--0.719) & 0.727 (0.712--0.741) \\
\hline
$DQ_{lymphocyte}$ & 0.675 (0.656--0.698) & 0.713 (0.691--0.734) \\
\hline
$DQ_{connective}$ & 0.566 (0.546--0.584) & 0.583 (0.565--0.602) \\
\hline
$DQ_{dead}$ & 0.410 (0.361--0.465) & 0.435 (0.306--0.561) \\
\hline
$DQ_{epithelial}$ & 0.668 (0.639--0.694) & 0.673 (0.644--0.702) \\
\hline
$DQ_{macrophage}$ & 0.657 (0.583--0.727) & 0.615 (0.531--0.703) \\
\hline
$DQ_{neutrofil}$ & 0.691 (0.625--0.753) & 0.729 (0.679--0.778) \\
\hline
\end{tabular}
%
%}
\end{table}

\clearpage

\subsection{\label{chap:S7}QuPath integration method}
We adopt an integration strategy leveraging the paquo \cite{Bayer_AG} library, a Python package that enables direct interaction with QuPath’s internal API, thereby facilitating seamless data exchange without intermediate conversion steps. The data processing pipeline (\hyperref[fig:S7]{Appendix Figure S7}) begins with the acquisition of WSIs and their associated annotations from QuPath, which are represented as Shapely \cite{Gillies_Wel_etal._2024} polygons. Utilizing paquo, we directly read, create, and modify these annotations and detections within a QuPath project in the Python environment. Images are then cropped using these polygons and processed by cell segmentation and classification models employing standard vision processing toolkits such as OpenCV, pyvips, and PyTorch. Additionally, QuPath employs Groovy scripts to initiate a Python process that starts the entire pipeline from QuPath graphical interface: fetching polygons, extracting images from them, and running deep learning model inference on the cropped images. 
The results are returned to QuPath, leveraging paquo's Python bindings to manipulate QuPath data while minimizing the computational overhead typically associated with cross-environment communication.

\counterwithin{figure}{subsection}
\renewcommand{\thefigure}{S\arabic{subsection}}

\begin{figure}[h!]
    \centering
    \includegraphics[width=\textwidth]{images/A7.pdf}
    \caption{QuPath integration workflow using Python environment}
    \label{fig:S7}
\end{figure}

Compared to traditional workflows that involve exporting annotations as GeoJSON, classifying them in Python, and reimporting them into QuPath, our approach offers several advantages. We eliminate the need to switch between programming languages, providing a cohesive and streamlined development process entirely within QuPath software and removing the necessity to use other tools. Meanwhile, we avoid storing annotations as intermediate JSON files unless required for external use or archiving. By conducting the entire inference and post-processing workflow within the Python environment, we leverage the power and flexibility of Python libraries for image processing and machine learning. This approach also enables adjustments to any set of labels and models, thereby improving its applicability.

%\hfill

The distilled model and QuPath integration code are packaged into a Docker container, enabling streamlined execution with the Docker engine. Detailed integration code and deployment instructions can be found in the GitHub repository \cite{Shvetsov_2025b}.

Despite these benefits, we acknowledge that the paquo library is a proof‑of‑concept project in its early development stage and has not been tested across all versions of QuPath.

\clearpage

\subsection{\label{chap:S8}Data and code availability statement}
All datasets, models, and code used in this study are publicly available and can be obtained from the repositories listed below. 
The PanNuke \cite{Gamper_Koohbanani_etal._2019} and MoNuSAC \cite{Verma_Kumar_etal._2021} datasets are publicly accessible, and download information along with detailed descriptions can be found in their respective articles. Preprocessing scripts for PanNuke and MoNuSAC data, as well as individual cell extraction scripts, are available on GitHub \cite{Shvetsov_2025a}. The H-Optimus foundation model used in our experiments can be downloaded from the HuggingFace repository \cite{hoptimus2024}, and model information is available on GitHub \cite{Saillard_Jenatton_etal._2024}. In addition, the integration code for QuPath and the distilled model packaged in a Docker container are provided in the repository \cite{Shvetsov_2025b}, and paquo Python library is available from the authors GitHub repository \cite{Bayer_AG}.
\clearpage

\end{document}




\clearpage

\appendix

\section{Prompts used}
\label{annex:prompts}


The English prompt used for identifying elements in English texts:

\noindent\fbox{%
    \parbox{\textwidth}{%
Your task is to analyze a text from a financial document to identify any linguistic phenomena that would make it difficult to translate the text accurately without surrounding context. Imagine the text is being translated sentence-by-sentence in a random order by a professional translator who cannot see the preceding or following sentences. \\
\\
The key types of context-dependent linguistic phenomena to look for are:\\
\\
- Anaphoras: Pronouns that refer back to an antecedent mentioned earlier \\
- Terminology consistency: Terms that need to be translated consistently even if the target language has synonyms \\
- Ellipsis: Omitted parts of the text that refer to something mentioned previously \\
- Polysemous words: Words with multiple meanings where the correct meaning depends on context \\
- Any other phenomena that would make translation ambiguous without more context \\
\\
Here is the text to analyze:\\
\\
<text>\\
{text}\\
</text>\\
\\
Please read the text carefully and identify any of the linguistic phenomena described above. For each phenomenon you find:\\
\\
1. Describe what type of phenomenon it is. \\ 
2. Explain why this particular instance would be difficult to translate accurately without the surrounding context. \\
3. Explain what information would the translator need to determine the correct translation. \\


After analyzing the full text, please provide an overall score from 1-5 indicating how prevalent these context-dependent translation challenges are in the text: \\
 \\
1 = No challenging phenomena identified  \\
2 = One or two minor instances \\ 
3 = Multiple instances that could lead to some ambiguity \\
4 = Significant amount of ambiguity that would make translation difficult \\
5 = Pervasive issues that would make coherent translation nearly impossible without context \\
\\
Provide your 1-5 score surrounded by tags <score> score here </score> without any further explanation.
    }%
}

\clearpage

The French prompt:


\noindent\fbox{%
    \parbox{\textwidth}{%
Ta tâche consiste à analyser un texte tiré d'un document financier afin d'identifier tout phénomène linguistique qui rendrait difficile la traduction du texte sans le contexte. Imagine que le texte est traduit phrase par phrase dans un ordre aléatoire par un traducteur professionnel qui ne peut pas voir les phrases précédentes ou suivantes. \\

Les principaux types de phénomènes linguistiques dépendant du contexte à rechercher sont les suivants :\\
\\
- Les anaphores : Les pronoms qui renvoient à un antécédent mentionné plus tôt. \\
- Cohérence terminologique : Termes qui doivent être traduits de manière cohérente, même si la langue cible contient des synonymes.\\
- Ellipse : Parties du texte omises qui renvoient à un élément mentionné précédemment.\\
- Mots polysémiques : Mots à sens multiples dont le sens correct dépend du contexte.\\
- Tout autre phénomène qui rendrait la traduction ambiguë en l'absence de contexte.\\
\\
Voici le texte à analyser :\\
\\
<texte>\\
{text}\\
</text>\\
\\
Lis attentivement le texte et identifie les phénomènes linguistiques décrits ci-dessus. Pour chaque phénomène identifié :\\

1. Décris de quel type de phénomène il s'agit.\\
2. Explique pourquoi cet exemple particulier serait difficile à traduire avec précision sans le contexte environnant.\\
3. Explique quelles sont les informations dont le traducteur aurait besoin pour déterminer la traduction correcte.\\

Après avoir analysé le texte complet, attribue une note globale de 1 à 5 indiquant le degré de présence de ces difficultés de traduction liées au contexte :\\
 \\
1 = Aucun phénomène problématique identifié \\
2 = Un ou deux cas mineurs\\
3 = Plusieurs cas susceptibles d'entraîner une certaine ambiguïté\\
4 = Beaucoup d'ambiguïté qui rendrait la traduction difficile\\
5 = Problèmes omniprésents qui rendraient une traduction cohérente presque impossible sans contexte\\
 \\
Indique la note de 1 à 5 entourée des balises <score> score ici </score> sans autre explication.
    }%
}

\clearpage

\section{Sub-domains of finance}
\label{annex:sub-domains}

The types of documents produced in the financial sector refer to different sub-domains of finance. Based on the document type, we grouped them into in 4 larger sub-domains: 1)~Fund Prospectus, 2)~KIID - PRRIPS, 3)~Investment Comments and 4)~Fund Annual Report. They are different in their content, style, degree of formality and even terminology. Therefore, sometimes it is advisable to treat the different sub-domains separately. We provide the sub-domain as meta-data for every segment to allow further analysis. The test set contains 3591, 3667, 1344, 1157 segments for each sub-domain respectively. Table \ref{table:stats_subdomains} states the number of segments  per document type.




\begin{table*}[t]
    \centering
    \begin{tabular}{lcc}
        \hline
        \textbf{Document type} & \textbf{\# Seg.} & \textbf{\# Avg Sent.} \\
        \hline
        Prospectus & 3378 & 10.21 \\
        PRIIP Key Information Document & 1974 & 14.13 \\
        Key Information Document & 1693 & 16.56 \\
        Monthly Report & 729 & 12.99 \\
        Annual Report & 719 & 13.21 \\
        Semi-annual Report & 438 & 16.5 \\
        Additional Information for  Investors & 218 & 8.18 \\
        Terms of Contract & 173 & 9.65 \\
        Monthly Manager Commentary & 169 & 15.04 \\
        Fund Profile & 94 & 11.97 \\
        SFDR Pre-Contractual Disclosure Document & 46 & 12.53 \\
        Legal Message & 40 & 8.64 \\
        Supplement Prospectus & 40 & 9.35 \\
        Quarterly Report & 34 & 14.46 \\
        ESG Factsheet & 14 & 11 \\
        \hline
    \end{tabular}
\caption[Caption for LOF]{Statistics per document types, summed among languages. \textit{\# Seg.} denotes the amount of segments within the subdomain, \textit{\# Avg Sent.} the average number of sentences per segment.}\label{table:stats_subdomains}
    \label{table:stats_subdomains}
\end{table*}




\section{Choice of a PDF extractor}
\label{annex:pdf_extractor}

When choosing the PDF extraction tool, the performance of two tools was compared: PyMuPDF and Apryse. This evaluation was done as part of a separate project, the tools were tested in a study for a different task (identification of relevant texts to a query) that also required PDF to text extraction. The evaluation was done manually on 58 documents and also automatically, by measuring how the quality of the texts obtained from the two tools affected the score of the final task. We discovered that texts obtained by PyMuPDF yielded an accuracy score of 51, while texts from Apryse attained 68. The manual analysis showed that this was mainly due to the fact that Apryse is able to detect which text is a title and therefore enables to construct sections, while PyMuPDF only extracts plain text, no matter the functionality of the text within the document. For building our testset we needed the information about the titles and sections, which is why we chose the tool Apryse.


\end{document}
