\section{Discussion}
This work studies learning in a competitive setting where 
classifiers are trained to increase market share.
From a learning perspective, our main message is that
while maximizing accuracy {\naive}ly is likely not a good strategy,
optimizing a weighted accuracy objective that correctly encodes competition
can be very effective.
Although technically similar, the transition to market-induced objectives
has implications on the market and consequently on user welfare.
In our market model competition promotes welfare,
but this relies on model transparency, efficient information flow,
and calculated user decisions.
Realistic markets are likely to fall short of such ideals:
firms may prefer to keep models private,
informational advantages can be exploited,
and user behavior can be far from rational.
The fact that most service sectors currently include only a few competing platforms
(consider media, social, e-commerce, finance, housing, etc.)
should raise concerns of oligopolistic behavior, notably collusion and lock-in practices. 
This requires deliberation of appropriate regulation
for these emerging accuracy markets.
\squeeze


