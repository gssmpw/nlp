\section{Introduction}

% - competition is prevalent - creates market for accuracy

Machine learning models play an essential role in consumer markets of today.
Across many domains,
firms that offer products and services can significantly gain by attaining better predictions about their users, or providing better predictions for them.
In a sense, this has made accuracy itself a commodity which consumers pursue;
consider how user choices have come to depend on the quality of personalized predictions in
recommendation systems,
media platforms, online marketplaces,
% (as both buyers and sellers),
or health analytics services.
%  depend highly on the quality of predictions that 
% drive her personal experience.
% she is offered.
The increasing demand for accurate predictions incentivizes firms to improve their predictions, which
in turn creates a `supply' of accuracy---%
a process which results in the formation of what we refer to as \emph{accuracy markets}.
% \squeeze

% \tocitec{papers on ML and markets}

% attaining a better understanding of the behaviors and preferences of potential consumers.
% The insights provided by accurate predictions that are conditioned on past consumer data are then placed at the forefront of central company decisions, such as which products to sell, which consumer demographic to target, how to anticipate growth, and more.


In accuracy markets, firms seek to maximize their market share by 
competing with other firms over who provides more users with accurate predictions.
This paper aims to study such markets from three perspectives,
namely of:
(i) the firms (or `\emph{learners}') that compete in the market,
(ii) the population of potential users,
and (iii) the market itself.
We follow the market model proposed by \citet{ben2017best,ben2019regression},
but focus on classification (rather than regression),
which we argue is more natural for this setting.
The main idea is that users choose a firm if it provides them with accurate predictions;
if multiple such firms exist, then ties are broken randomly.
When there is only one firm in the market (a monopoly),
then maximal market share can be attained by maximizing accuracy directly---which is also optimal for users.
However, once there is competition, each firm must take into account
the predictions of all other firms, as it depends on their choice of classifiers.
This dependence forms an oligopoly market in which the interests of the firms
no longer necessarily align with user welfare,
defined as the ratio of users for whom at least one firm is accurate.
% who can obtain accurate predictions.

The main result of \citet{ben2019regression} is that if firms in such markets can successfully compute a best (or better) response % predictor
(i.e., find a model that maximizes market share conditioned on all other models remaining fixed), then dynamics will converge to a pure Nash equilibrium.
%  in a finite number of rounds.
This is insightful, but leaves many questions unanswered:
When can best-response classifiers be found efficiently, and how?%
\footnote{\citet{ben2019regression} give an algorithm that applies to linear regression and relies on integer linear programming.}
What kind of equilibria can be reached, and what will happen on the way there?
How will the market be shared by the different firms?
And will outcomes be beneficial for users?
Our goal is to shed light on these issues and others,
using theoretical analysis and empirical evaluation,
and as they relate to the learning firms, the users, and the market.
\squeeze

From the perspective of the learner, 
our results show that finding the best-response classifier is as hard---%
but also not harder than---%
solving a standard classification problem.
% optimizing a standard supervised learning objective.
In fact, given the predictions (or classifiers) of all other players,
maximizing market share corresponds to maximizing a particular weighted accuracy objective.
This means that although solving this exactly is hard,
standard learning techniques (e.g., using proxy losses) can be highly effective in practice.
It also provides the learner flexibility in choosing the model class to work with,
as it sees fit under standard considerations (e.g., data size, compute capacity).
Interestingly, however, we see that the choice of \emph{when} to respond bears significant implications on outcomes for the learner.
This brings to focus questions regarding the pioneering of a new market,
entering an existing one, and maintaining user loyalty
(or its harmful counterpart of user lock-in practices).
\squeeze

% \todo{mention connection to perf pred?}

From the perspective of the market, 
our analysis suggests that most markets will exhibit strong anti-coordination
(the other alternative being the existence of a single dominant strategy,
which is possible but rare).
In particular, market forces push firms to secure exclusive access to certain user sectors, and firms compete over who secures the larger sectors.
Notably, gaining access \emph{first} does not guarantee exclusivity;
in fact, for simple markets with two firms we show that firms engage in a chicken-like game, where moving first is disadvantageous.
Our analysis reveals
that, despite competition, 
firms are in a sense \emph{cooperating}:
when one firms acts to increase its market share,
this also serves to increase the market shares of the other.
Empirically, we observe that for larger markets with more firms
outcomes are more nuanced,
although the order of play remains highly significant.
% The most intriguing aspect of our analysis is 
% Because responding to the current market requires information on the other firm's classifier, 
% We observe this holds strongly in experiments.
\squeeze


From the perspective of users, a direct result of the market is that competition improves welfare.
% is favorable for users.
What may be surprising is how efficient the market is:
Empirically, we observe that welfare increases quickly and attains the maximum possible value with only a few firms,
and after one round of updates.
For the latter, we give theoretical grounding for why this can happen.
One reason is that our model for the market enables efficient outcomes to materialize---as long as information flows freely.
This has policy implications:
a social planner that seeks to maximize welfare should incentive firms
(or introduce regulation) to make their models public.
% \todo{say something about information sharing -- what is the most we can claim?}
\extended{%
But this may not be necessary---as we show, under certain conditions,
it is in a firm's best interest to fully disclose its classifier to others.
Here the intuition is that revealing the model signals the firm's `marked territory'
to deter others from targeting those users.
}%
Thus, transparency becomes an operational consideration 
which, in a utilitarian sense,
works in favor of both firms \emph{and} users.
% the population of users.
% \squeeze

We end with a series of experiments using synthetic and real data 
that demonstrate the underlying mechanics of accuracy markets and how they operate.
Our results demonstrate that learning in such markets can be feasible,
that competition converges quickly, and that the market is typically highly efficient and favorable to users.
Results also highlight the importance of adjusting the objective to account for competition, and show how lacking to do so (and optimizing accuracy {\naive}ly) 
can be detrimental to both firms and users.
For the market, we present analysis revealing how it
decomposes across firms, and measure concentration and market power.
We also show the importance of timing market entry and model updates,
the relation between performance and model class capacity,
and the constructive role of information sharing.
These results underscore the dynamics of accuracy  markets and showcase the importance of adapting to competition.
\squeeze


