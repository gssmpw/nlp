\section*{Broader impact}

Our work considers the role of machine learning in fostering markets
in which utility to consumers derives from personalized accuracy.
The market model we study is inspired by real markets of this type,
but as a model, makes several simplifying assumptions
that merit consideration before drawing conclusions about actual markets.
One assumption is that there is a single underlying distribution that is fixed and accessible to all providers. 
Although this is a common assumption in standard machine learning,
under competition it has further implications.
When data distributions differ across providers,
or even when the distribution is the same but samples are different (which is plausible),
it is no longer clear if pure equilibrium exists or is reachable through reasonable dynamics.
Another assumption is that users choose a provider who is accurate on their own input.
This applies in some cases, such as personalized recommendations:
users likely know their preferences, but do not know if (and which) content items match them---but can make conclusions once recommended.
More generally however, we consider our model as a simplification of outcomes that materialize and stabilize over time,
such as provider reputation, social learning, or confirmation in hindsight.
Another alternative is that users interact with a platform not once but many times;
if the platform has access to some personalized examples,
then competition can revolve around future expected outcomes.
A final assumption is that users are rational and choose by maximizing utility
independently at each time step.
This can be a reasonable assumption in settings where users have
both incentive and resources to invest effort in bettering their choices,
and when sufficient time passes between rounds to enable switching providers.
More generally, user behavior is likely to play a key role in market outcomes,
and can benefit from more realistic modeling.
It is also likely that competition can drive providers to exploit
users' behavioral weaknesses---an additional reason for establishing appropriate regulations and norms.



% \red{%
% limitations: \\
% - we assume same distribution and sample -- but can be different per provider. in this case we don't know about equilibrium \\
% - need to justify users "knowing" their $y$ and all predictions for it -- what are use cases where this holds? (for example: (1) personalization of content: you know what you want (say comedy), but not exact item (recommended).
% (2) choose in hindsight and/or reputation (real estate pricing))
% what are slightly relaxed realistic settings? (for example, each user $u$ has distribution over content $p_u(x,y)$, training set has some examples from this distribution, goal is to be good in expectation) \\
% -- users choosing UAR at each step, no `sticking' to previous provider. need more realistic user choice models, lock-in practices or cost to leave. our assumption is maybe more appropriate for when there is much time that passes between rounds and user choices settle down.
% }