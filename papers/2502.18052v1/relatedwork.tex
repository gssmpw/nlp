\section{Related work}
% \todo{...}
% COMPETITION: \\
Studying the dynamics of machine learning models competing for market share  has been a budding line of research. \citet{ben2017best,ben2019regression} 
% \nir{tip: use {\textbackslash}citet if the authors' names should be read inline, and {\textbackslash}citep to put parentheses. just {\textbackslash}cite is ambiguous.}
present a regression learning task where providers wish to maximize their market share of users by reducing prediction errors to below a given threshold.
Their focus is on the equilibrium dynamics in the induced game between the providers.
Employing a similar setting, \citet{jagadeesan2024improved} focus on the effects of competition dynamics on social welfare, showing that better data representation does not necessarily translate to better welfare;
some of our results echo theirs.
\citet{Feng_2022} study the bias-variance tradeoff in competitive settings.
\citet{yao2023bad,yao2024rethinking,yao2024user} introduce a competition setting for content creation and study welfare,  equilibrium behaviors,
and best-response dynamics. \citet{ginart2021competing}, \citet{pmlr-v238-dean24a}, and \citet{pmlr-v235-su24a} study how competitors specialize when user choices influence the observable data of each competitor. 
Our work puts emphasis on the learning task itself,
and studies its effects on market dynamics and outcomes.

\squeeze

% SC:\\
More generally, our research relates to the growing literature on strategic learning.
% a growing field centered around analysis of machine learning algorithms taking place in settings with strategic interactions.
The majority of work in this field models users as strategic agents that can manipulate their features \citep[e.g.,][]{hardt2016strategic,levanon2021strategic}.
% This idea is generalized by \cite{chen2024strategic}, who offer a general model of external influence on the model.
In contrast, we model users as choosing among alternatives,
and put emphasis on the strategic role that learning must assume to contend with competition.
The idea that users can choose a provider has been considered in 
\citet{koren2023gatekeeper} and \citet{horowitz2024classification},
but only for binary choice (i.e., join or drop out) and under uncertainty.
% study a setting where the user can choose to join a firm or drop out, whereas we consider a setting of choices between firms.
Our work differs in that it supports choices between multiple firms in a competitive market.
In a recent paper, \citet{chen2025strategic} study strategic learning with externalities; interestingly, their construction also gives rise to a potential game (as ours does), but between users (rather than providers).
Our work also draws connections to the field of performative prediction \citep{perdomo2020performative}, which studies how (re)training models can gradually change the underlying data distribution.
As we show, this perspective applies to our approach when considered
from the viewpoint of a single competing provider.
\squeeze


% - OBP papers \\
% Similarities:
% - Same problem setup
% - same motivation for what consumers want
% Differences:
% - Solution works only for linear classifiers 

% - jagadeesan paper(s?) \\
% - Fan yao - 3 papers on the C3 competition
% - sarah dean papers \\
% - haifeng - has related papers? check
% - pred multiplicity

% strategic: \\
% - strat self-selection - choose firm or not; we extend to multi-choice (but in different setting) \\
% - sc with externalities - potential games!

% - perf pred - ?