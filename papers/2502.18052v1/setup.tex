\section{Setup}
\label{sec:setup}
In our competitive learning setting,
users are described by features $x \in \X$ and labels $y \in \Y$,
over which there is an unknown joint distribution $p(x,y)$.
There are $n$ service providing firms, $s_1,\dots,s_n$,
who provide prediction services to users, and together form a market.
Given a training set $\smplst=\{(x_i,y_i)\}_{i=1}^m$,
each service provider $s_i$ learns a classifier $h_i$ from some model class $H_i$.
Each user $x$ then chooses a provider among those offering an accurate prediction:
\squeeze
% Formally, each user $x$ chooses a provider from the set:
\begin{equation}
\label{eq:user_choice}
% s(x;\h) 
s(x)
% = \argmin_{i \in [n]} \loss(y,h_i(x))
\in \{ s_i \,:\, h_i(x)=y \}
\end{equation}
% \todo{$s(x;\h)$?}
% where
% $\h=(f_1,\dots,f_n)$ and
% $\ell$ is a loss function, such as the 0/1 loss for classification
% or squared loss for regression.
% If multiple providers attain the minimal loss (which is likely in classification), 
% then ties are assumed to be broken \blue{uniformly} at random.
If multiple providers are accurate,
then $s(x)$ is determined by a random tie-breaking rule.
When no providers offer an accurate prediction, we denote the null choice by
$s(x)=\varnothing$.
Welfare is defined as the ratio of users that obtain service:
% \squeeze
\begin{equation}
\label{eq:welfare}
\welf(\h) = \expect{p}{\one{s(x) \ge 1}}
\end{equation}
where $\h = (h_1,\dots,h_n)$ are all classifiers in the market.
\squeeze

% \paragraph{Service providers.}
The goal of service providers is to maximize their market share,
defined as the expected ratio of users that choose them.
For each provider $s_i$,
this depends on its choice of learned model $h_i$,
but also on the set of all other models, $h_{-i}$.
Formally, the market share of service provider $s_i$ is defined as:
\squeeze
\begin{equation}
\label{eq:sp_utility_exp}
\ms_i = \ms(h_i \mid h_{-i})
= \expect{p}{\prob{}{s(x)=s_i}}
% = \expect{p}{\one{s(x)=i}}
\end{equation}
where probability is w.r.t. how $s(x)$ is chosen from the set of accurate providers in Eq.~\eqref{eq:user_choice}.
For simplicity we assume ties are broken uniformly at random,
namely $\prob{}{s(x)=s_i} = 1 / \kappa(x)$ where 
% $\kappa(x)=|\{s_j : h_j(x)=y \}|$.
$\kappa(x)=|\{s_j : s(x) \in s_j \}|$.
This conforms to the setting of \citet{ben2017best,ben2019regression}.
% \squeeze

When the market includes only a single provider,
maximizing market share is equivalent to maximizing accuracy:
\begin{equation}
\label{eq:exp_accuracy}
\argmax\nolimits_{h \in H} \expect{p}{\one{y=h(x)}}
\end{equation}
which is the standard objective of supervised learning,
and in this case also maximizes welfare by definition.
However, once there is competition,
this connection breaks since
% in a competitive market with multiple providers,
each provider's market share becomes dependent on all others.
% providers no longer pursue accuracy,
% and interests diverge from welfare.
% the set of all other classifiers in the market.
Thus, the \naive\ approach of maximizing accuracy
as a proxy
% via Eq.~\eqref{eq:exp_accuracy}
becomes suboptimal,
and the question of how providers maximize their market share
% (Eq.~\eqref{eq:sp_utility_exp}) 
must be considered 
jointly.
% collectively.
\squeeze


% \blue{but our results extend to any stochastic choice rule (Appendix?).}
% \blue{This is like taking expectation over users who randomly choose between equally good providers}


% \nir{\textbf{Qs}:\\%
% - does the distribution have to be the same for all firms? what about the training set(s)? \\
% - need to justify users "knowing" their $y$ and all predictions for it?
% % \\- (related) should we define eq 1 as expectation over $p(y|x)$?
% }


% \todo{add appendix with more general setup that supports:\\
% 1. general tie-breaking rules \\
% 2. regression with (a) threshold choices, (b) random choice by score (eg softmin) \\
% 3. $p(y|x)$ - lets users choose by empirical estimate on train set, but utility is expected on distribution
% }


% \todo{when $m=1$ (and users choose if pred is right, or lowest loss+thresh) then provider objective becomes acc}

% \nir{\\%
% % - notation: utility $U$ or market share $\mu$? \\
% % - define welfare here? \\
% - users as scarce resource
% }

\paragraph{Competitive learning as a game.}
For a given distribution $p$,
if we think of each provider $s_i$ as a player and interpret Eq.~\eqref{eq:sp_utility_exp} as their utility, 
then this defines a game, which we refer to as an \emph{accuracy game}.
%\citep{ben2017best}.
The strategy space for each $s_i$ is the set of all models in its model class $H_i$, and each tuple $\h = (h_1,\dots,h_n)$ defines a game state.
We will assume the game is played on 
the empirical distribution induced by $\smplst$,
but that final payoffs are given by expected market share w.r.t. $p$.
Note the game remains well-defined when players have their own $\smplst_i \sim p_i$, although current equilibrium results do not hold for this setting.
In terms of information, we work in the full information setting
where all players have complete access to the payoff matrix.
However, as we will see, 
optimal strategies for providers require strictly less information.
\extended{it will suffice to assume that players have only local information---and that sharing such information is in their best interest.}
% In terms of information, our analysis will begin with assuming that the distribution $p$ is known,
% but our ultimate goal is to study games in which the $h_i$ are chosen on the basis of a finite sample set
% (though utility will still be defined w.r.t. $p$).


\paragraph{Dynamics.}
To understand how game states progress,
we will explore dynamics in which providers can update their predictive models over time,
and in response to others.
% updates made by other providers at earlier time steps.
Assume w.l.o.g. that providers are ordered,
$s_1 \prec s_2 \prec \dots \prec s_n$.
Then at round $t$,
% Specifically, at discrete time step $t$, 
each provider in turn chooses their $h_i$ by playing \emph{best response}, defined as:
% \squeeze
\begin{equation}
\label{eq:br}
h_i^{t} 
= \br(h^{t}_{-i})
= \argmax_{h \in H_i} \, \ms(h \mid h_{-i}^{t})
\end{equation}
That is, providers respond by choosing the optimal classifier $h_i$ assuming all others classifiers remain fixed,
namely $h_{-i}^t = (h_1^{t}, \dots h_{i-1}^{t}, h_{i+1}^{t-1}, h_n^{t-1})$ and for some choice of initial classifiers $\{h_i^0\}$.
% \ohad{we need to clarify who moves at t, just $h_i$, or everyone?}
We refer to $h_i^t$ as the \emph{best-response classifier} of $s_i$,
but note it need not be unique.
% \ohad{How do we define their starting positions $h^0 = h_{opt}$?}
Since solving Eq.~\eqref{eq:br} can be computationally infeasible,
% and requires access to the distribution $p$.
we will also consider approximate best responses that replace $\ms(h \mid h_{-i}^{t})$ with a tractable surrogate objective (see Sec.~\ref{sec:method}).

% \blue{%
% Best response dynamics will be useful for us for two reasons.
% First, as we will show in Sec.~\toref,
% for \blue{accuracy games/markets} (and under certain conditions),
% such dynamics are guaranteed to converge to a PNE.
% Second, since market share is driven by accuracy,
% Eq.~\eqref{eq:br} can be solved by adapting standard empirical risk minimization methods to our purposes.
% }


\paragraph{Equilibrium.}
We will be interested in studying the game's equilibria,
focusing mostly on pure Nash equilibrium (PNE).
These are defined as states $\h$ in which no provider has incentive to unilaterally deviate from its chosen strategy:
\squeeze
\begin{equation}
\label{eq:pne}
\forall \, i\in [n], \,\, h' \in H_i: \quad\,\,
\ms(h_i \mid h_{-i}) \ge \ms(h' \mid h_{-i}) %\quad\,
\end{equation}
\citet{ben2017best} prove that the game is a type of \emph{potential game} \citep{monderer1996potential}.%
\footnote{For completeness, in Appendix~\ref{appx:congestion}
we give the game's characterization as a \emph{congestion game} by identifying the appropriate congestible resources and constructing an explicit cost function.}
Since the game is played on the empirical distribution,
which implies that the set of all possible predictions is finite (even if $H$ is not),
a direct result is that a PNE exists
and is reachable via a finite sequence of best responses (Eq.~\eqref{eq:br}).
Note that multiple equilibria may exist, and that these may differ significantly
in market shares, market concentration, and induced welfare.
Furthermore, not all equilibria can necessarily be reached via best-response dynamics,
and the equilibrium that is reached can depend on the initial game state (i.e., choice of first classifiers) and the order of play.

% Our goal will be to gain further insight into such equilibria.


% Our goal 
% will be to characterize equilibria,
% understand their properties  as they depend on the game's parameters,
% and explore if and when dynamics can converge to such states.
% What is unique about our game is that the utility of providers, 
% and therefore their behavior,
% relies on attaining high predictive accuracy by employing supervised learning algorithms.
% \squeeze

\extended{%
\nir{\textbf{Qs:}\\
% - what else do we assume players know? \\
- mixed equilibrium? correlated? \\
- fresh sample set(s) at each step? \\
- does learning use all past data, or just current? (is this important?) \\
% - be precise about what each provider knows \\
- better response? for proxy loss and/or empirical objective
% - sync or a-sync updates?
}
}
