\section{Related Work}
\label{sec:relatedwork}

Extensive research has been conducted on the application of AI for Leukemia detection. This section sheds light on the previous work done on Leukemia detection using different Deep Learning architectures, along with their strengths and weaknesses.

Hosseini et al.____ strove to detect B-cell acute lymphoblastic leukemia (B-ALL) cases, including the subtypes, using a deep CNN. After leveraging K-means clustering and segmentation for image preprocessing on a dataset consisting of benign and malignant B-ALL cases, he compared the efficiency of three lightweight CNN models (EfficientNetB0, MobileNetV2, and NASNet Mobile) using the training and testing data. Eventually, segmented and original images were combined and fed as inputs through two channels to extract maximum feature space, which enhanced the models' accuracy. MobileNetV2 was selected for achieving 100\% accuracy and having the smallest size, making it suitable for implementation on mobile devices.

Talaat et al.____ exploited the attention mechanism to detect and classify leukemia cells. The A2M-LEUK algorithm involved image preprocessing, feature extraction using CNN, and an attention mechanism-based machine learning algorithm applied to the extracted features. The C-NMC 2019 ____ dataset was utilized, and classifiers such as SVM or a neural network were used in the proposed algorithm. After evaluating the precision, recall, accuracy, and specificity of the A2M-LEUK algorithm against KNN, SVM, Random Forest, and Naïve Bayes, the proposed model demonstrated superior performance, achieving nearly 100\% in all four metrics. However, the paper did not specify which classification model was used with A2M-LEUK to achieve that accuracy.

Yan____ presented a study on the single-cell dataset C-NMC 2019____ to classify normal and cancerous white blood cells using three different models: YOLOv4, YOLOv8, and a CNN model. Data augmentation was applied to the CNN and YOLOv4 models. The CNN model, consisting of convolutional and max pooling layers, fully connected layers, and ReLU activation functions, achieved 93\% accuracy, while YOLOv4 and YOLOv8 both achieved accuracies above 95\%.

Devi et al.____ utilized a combination of custom-designed and pretrained CNN architectures to detect ALL in the ALL image dataset____ after applying augmentation. The custom-designed CNN was used to extract hierarchical features, while VGG-19 was used to extract high-level features. VGG-19 performed the classification task, and the proposed model achieved 97.85\% accuracy. On the other hand,____ applied image processing and the Fuzzy Rule-Based inference system to tackle the same topic.

Rahmani et al.____ opted for the C-NMC 2019 dataset, where the data was preprocessed using methods such as grayscaling and masking, followed by feature extraction through transfer learning with models like VGG19, ResNet50, ResNet101, ResNet152, EfficientNetB3, DenseNet-121, and DenseNet-201. Feature selection was then applied using Random Forest, Genetic Algorithms, and the Binary Ant Colony Optimization metaheuristic algorithm. The classification was conducted through a multilayer perceptron, achieving an accuracy slightly above 90\%.

Kumar et al.____ contributed to the classification of different types of blood cancer in white blood cells, such as ALL and Multiple Myeloma. He applied preprocessing and augmentation methods to the data, followed by feature selection. The study used the SelectKBest class to select K specific features. The proposed model consisted of two blocks, each containing a convolutional layer and a max pooling layer, followed by fully connected layers and a classification layer. This architecture achieved 97.2\% accuracy.

Saikia et al.____ introduced VCaps-Net, a fine-tuned VGG16 model combined with a capsule network for ALL detection. Two datasets were used: ALL-IDB1 ____ and a private dataset. The proposed model integrates the powerful structure of VGG16 with a capsule network, which represents unit positions in images using vectors to maintain spatial relationships often lost due to max pooling. VCaps-Net achieved an accuracy of 98.64\%.

The ALL-IDB dataset was also used in a study by Alsaykhan et al.____ to detect ALL using a hybrid algorithm. The approach combined support vector machine (SVM) and particle swarm optimization algorithms to optimize the results by selecting the best parameters to minimize errors. As a result, an accuracy of 97\% was achieved.

In____, Abhishek et al. classified different types of leukemia, including CLL, ALL, CML, and AML. He utilized the transfer learning approach, freezing the initial layers of pretrained CNNs as feature extractors (a process known as fine-tuning). The feature extractors used were ResNet152V2, MobileNet, DenseNet121, VGG16, InceptionV3, and Xception, which were trained on ImageNet____. These extractors were then combined with classifiers such as Support Vector Machines, Random Forest, and new fully connected layers to improve classification performance. The accuracies for various combinations of classifiers ranged from 74\% to 84\%.

Vogado et al.____ conducted a study using multiple datasets of different natures, focusing on multi-cell and single-cell images. CNNs were used for feature extraction from the original images, and SVM was applied for classification without prior image segmentation. The pre-trained models included AlexNet____, CaffeNet____, and VGG-f____. The feature vectors were then passed to the selected classifier for final predictions.