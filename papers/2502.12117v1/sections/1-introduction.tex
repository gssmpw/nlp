\section{Introduction}\label{sec:intro}



Putting the right bidders in an auction is crucial for its success \citep{milgrom_2004_putting_auctiontheory_to_work}:
\begin{center}
   \textit{``The success of transactions depends even more on what happens before and after the auction. Understanding the transaction as a whole requires one to ask who participates...''}
\end{center}
We approach this problem from the perspective of how the auctioneer can \textit{prescreen} bidders before the auction based on additional information received. Specifically, the auctioneer employs a noisy predictor to estimate bidders' valuations. This noisy predictor might be based on generative AI that can simulate agents' behaviors \citep{immorlica_2024_generative}, on historical bidding data \citep{munoz_2017_auctions_predictions, lu_2024_competitiveauctions_predictions}, or on players' submitted credentials in all-pay contests (auctions).



We consider a situation in which bidders have independent and identically distributed (i.i.d.) privately informed valuations. For analytical tractability, we assume a Bernoulli predictor: with a given probability, referred to as the \emph{prediction accuracy}, the designer receives a signal that is fully informative (i.e., equal to the bidder's true valuation), while with the remaining probability the noisy signal is completely uninformative.




Based on the posterior expectations of bidders' valuations after receiving the noisy signals, the designer admits only the top players. The designer chooses the admitted number to maximize his utility. This prescreening mechanism not only affects the participated number but also functions as an information disclosure: the admitted players are more likely to have high valuations especially when the prediction accuracy is high.


We study two commonly used auction formats: first-price auctions and all-pay auctions. First-price auctions are widely used in practice; for example, Google Ads currently employs a first-price auction,\footnote{https://www.publift.com/blog/what-will-googles-first-price-auction-mean-for-publishers} and we focus on revenue maximization in this setting. On the other hand, (crowdsourcing) contests are often modeled as all-pay auctions \citep{Moldovanu_2001_AER_WTA_Optimial, moldovanu_2006_contest_architecture, Liu_2023_optimal_reward_negativeprizes}, where bids correspond to levels of effort. In this setting, every participant incurs a cost for exerting effort regardless of whether they win, and common objectives include maximizing the expected total effort (revenue) or the expected highest effort (bid). The latter objective reflects settings in which only the best submission matters.%, as in machine-learning crowdsourcing contests on \href{https://www.kaggle.com/competitions}{Kaggle}.





We find that even in a setting with i.i.d. prior, the participated players' posterior beliefs are \textit{asymmetric} and depend on their own private valuations (in a parametric form). We characterize these posterior beliefs in closed form: the posterior density is given by the product of the prior density and the admission probability, up to a normalizing constant. Moreover, the admission probability is a symmetric function that is piecewise additive and locally supermodular.

We show that the prescreening game is equivalent to a standard auction (without prescreening) but with \emph{correlated} valuations sampled from a symmetric joint distribution. 
In other words, prescreening endogenously induces correlation among bidders' valuations. 
In the equivalent formulation, the joint density is the product of the i.i.d. density and the admission probability, up to a normalizing constant. Notably, when the prediction accuracy equals one, referred to as \emph{perfect prescreening},  the correlated valuations are \emph{affiliated} in the sense of \citet{milgrom_1982_auctiontheory_competitive_bidding}. This basically implies that ``a high value in one bidder’s estimate makes high values in the others’ estimates more likely.''

For all-pay auctions, we find that a symmetric and strictly monotone (SSM) equilibrium strategy exists provided that the \emph{inflated types}---defined as the product of a bidder's true type and the normalizing term in his posterior belief---are increasing in the true types. In the case of perfect prescreening, this condition is also necessary. When an SSM equilibrium exists, it is unique, and we provide an explicit characterization.


We further show that when the prediction accuracy is one, \textit{all} participated players' equilibrium bids decrease with the admitted number.
As a result, it is optimal to admit only two players in terms of the expected highest bid. Perhaps surprisingly, despite the reduction in the potential pool of paid bidders, admitting only two players remains optimal in terms of revenue. On the other hand, when the prediction accuracy is zero, referred to as \emph{blind prescreening}, admitting all players maximizes the expected revenue. However, regarding the expected highest bid, admitting fewer players can strictly benefit the designer, and we explicitly characterize the optimal number for power law prior distributions. For situations between these two extremes, our closed-form characterizations of beliefs and other relevant terms allow us to determine the optimal number of admitted players in a computationally tractable manner. In particular, this involves evaluating a one-dimensional integral $(m-1)$ times when the total number of bidders
is $m$, where the integrands have closed-form expressions.

In first-price auctions, we derive a condition that guarantees the existence of SSM equilibrium strategies. Interestingly, we identify a scenario in which the existence conditions commonly assumed in the literature fail, yet an SSM equilibrium strategy still exists. When such an equilibrium exists, it is unique, and an explicit characterization is provided. Furthermore, we demonstrate that in first-price auctions, admitting all players maximizes the expected revenue under both perfect and blind prescreening. Our numerical experiments indicate that even beyond these two extremes it is also often optimal to admit all players in first-price auctions---a stark contrast to the all-pay auction setting.

Finally, we show that the revenue under all-pay auctions with optimal prescreening can exceed the revenue in first-price auctions (without prescreening) even when the latter has one additional bidder, particularly when the total number of bidders is large. By the seminal Bulow-Klemperer result \citep{bulow_1994_auctions_negotiations}, this finding implies that the revenue generated by all-pay auctions with optimal prescreening can surpass that of the \textit{optimal auction format} (without prescreening), thereby underscoring the advantage of prescreening.




\noindent
\textbf{Releated Works.}
Our work is related to \textit{auction analysis} with interdependent private valuations. Although we start from independent private valuations, prescreening endogenously induces correlation among bidders' valuations. \citet{milgrom_1982_auctiontheory_competitive_bidding} proposes a tractable approach to studying interdependent valuations with positive correlation—namely, \textit{affiliated} valuations (their theory applies beyond private-valuation auctions). They show that with affiliated valuations, there always exists an SSM equilibrium strategy in first-price auctions and establish some revenue dominance (ranking) results among different auction formats. \citet{krishna_1997_all_pay_affiliation} extends this idea to all-pay auctions and finds that the existence of an SSM equilibrium with affiliated valuations may fail. They identify a sufficient condition for the existence of an SSM equilibrium and show that, under this condition, all-pay auctions yield higher revenue than first-price auctions. A more recent examination of positively correlated types beyond affiliation is provided by \citet{castro_2007_affiliation_positive_dependence}.





Our work is also related to \textit{auctions with entry}. A substantial literature has investigated such auctions. For example, the first of those, \citet{samuelson_1985_entrycosts} considers a setting, where potential bidders first learn their private values and then decide whether to participate in the auction with an entry cost. In contrast, \citet{levin_1994_aer_auctions_entry} assume that potential bidders choose, via a mixed strategy, whether to participate in the auction (and incur the entry cost) before observing any information about their valuations. Furthermore, \citet{gentry_2014_affiliated_signal_entry} generalizes these frameworks by allowing the signals observed prior to making entry decisions to be (more generally) affiliated with the bidders' valuations.


\citet{ye_2007_GEB_indicative_twostage_auctions} considers a two-stage auction with costly entry and investigate how the designer should determine the number of admitted bidders. In the first stage, all bidders receive partial information about their (true) valuations and submit bids. The players with top bidding are then promoted to the second stage.
The designer discloses the highest rejected bids from the first stage. Upon entering the second stage, the admitted bidders incur a cost (for example, due to the expense of closely examining the asset being sold) and receive additional ``additive'' information about their valuations. 
The authors find that if the first stage is non-binding (i.e., admitted bidders pay nothing in the first stage), no SSM equilibrium strategy exists. Conversely, when the first stage is binding, an SSM equilibrium strategy does exist under certain conditions. In these cases, they establish revenue equivalence among standard auction formats and demonstrate that admitting fewer bidders can improve the designer's revenue.
\citet{quint_2018_theory_indicativebidding} demonstrates that when the first-stage auction is non-binding and the second-stage auction operates as a second-price auction, a weakly increasing equilibrium strategy exists.



The primary distinction between our work and the aforementioned studies is that we assume the designer, rather than the bidders, observes additional information before the auction begins. In our model, bidders take no pre-auction actions and are fully aware of their true valuations.





Our paper is related to \textit{information disclosure}. There is a fast-growing literature on persuasion under the commitment assumption \citep{kamenica_2011_persuasion,kamenica_2019_bayesian_review,candogan_2020_info_operations,bergemann_2022_info_disclosure_auctions,sun_2024_lead_selling}. 
Our work is related to this literature, as prescreening functions as a form of information disclosure. However, it differs in that the prescreening mechanism simultaneously influences both the number of participants and the players' beliefs, albeit in a limited way.



Our analysis of all-pay auctions is closely related to the literature on \textit{contest design}. Much of the existing work on all-pay contests (auctions) with private valuations focuses on the design of prize structures \citep{Moldovanu_2001_AER_WTA_Optimial, Liu_2023_optimal_reward_negativeprizes, Jason_Optimal_Crowdsourcing_Contest}. In contrast, a few studies have examined elimination mechanisms. For example, \citet{moldovanu_2006_contest_architecture} considers a two-stage model in which players are randomly divided into subgroups in the first stage, and the highest bidder from each subgroup proceeds to the second stage. They show that under certain conditions, it is optimal to implement a winner-take-all prize structure in the second stage and to partition the players into two groups in the first stage. A notable feature of the model in \citet{moldovanu_2006_contest_architecture} is that the admitted players' beliefs remain i.i.d. In contrast, in our model, the prescreening process induces asymmetry and correlations in the admitted players' posterior beliefs.
\citet{sun_2024_contests} considers prescreening contests assuming that the designer knows the ranking of the players' private types. Mathematically, this is equivalent to our perfect prescreening setting. Our results regarding the optimal number of admitted players in all-pay auctions in the case of perfect prescreening, in terms of the expected highest bid, thus follow the analysis in \citet{sun_2024_contests}.



\textit{Mechanism design with predictions} is a recent but active area. The basic idea is that the designer employs a predictor—for example, a machine-learning model trained on bidders' historical behaviors—that may erroneously forecast bidders' valuations. This approach allows for algorithmic mechanism design that leverages prediction errors to bypass worst-case analysis \citep{munoz_2017_auctions_predictions,xu_2022_mechanismdesign_predictions,balkanski_2023_online_mechanism_predictions,lu_2024_competitiveauctions_predictions}.
These previous works mainly focus on prediction-based algorithm design.
Our work contributes to this stream by focusing on the design of the number of admitted bidders.


\noindent
\textbf{Organization.} We introduce the formal setup in \Cref{sec:model}. Posterior beliefs are analyzed in \Cref{sec:beliefs}. We first analyze all-pay auctions in \Cref{sec:all-pay_auctions} and then examine first-price auctions in \Cref{sec:firstprice_auctions}. In \Cref{sec:rev_ranking}, we compare these two auction formats and derive a revenue ranking result in the case of perfect prescreening. \Cref{sec:conclusion} concludes the paper. Auxiliary results are provided in \Cref{app_sec:auxiliary_results}, and all proofs can be found in \Cref{app_sec:proofs}.

