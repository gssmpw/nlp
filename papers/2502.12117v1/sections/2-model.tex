\section{The Model}\label{sec:model}




We consider two auction formats for an indivisible item: first-price auctions and all-pay auctions separately.\footnote{The analysis of prescreening in second-price auctions is trivial because bidders know their valuations, and truth-telling is a dominant strategy. Consequently, prescreening does not affect the equilibrium bids in second-price auctions.
}
We investigate how the designer can “prescreen” players in order to incentivize them to submit higher bids in each setting. In the first-price auction, again, all participating players (bidders) simultaneously submit their bids; the player with the highest bid wins the auction and pays her bid. In the all-pay auction, all participating players simultaneously submit their bids; the player with the highest bid wins the auction, but every participating player pays their bids no matter whether they win the item or not.
The equilibrium notion is pure-strategy Bayesian-Nash. 
The formal definition will be introduced later.
The tie-breaking rule can be arbitrary.
We assume that the players (bidders) and the designer (auctioneer) are all risk-neutral.
We may use the terms ``players'' and ``bidders'' interchangeably. Similarly, the terms ``auctioneer'' and ``designer'' are used exchangeably.

There are a total of $m$ players, where $m$ is exogenously given and common knowledge. Let $\I^0$ be the set of these $m$ players. Each player has a privately known type (valuation) $V_i\in[0,1]$, independently and identically drawn according to an absolutely continuous distribution $F$ with strictly positive density $f$.
Let $v_i$ denote the realization of $V_i$.

The designer receives a noisy signal $\zeta_i(v_i)$ about player $i$'s type $v_i$ in a Bernoulli manner:
\begin{enumerate}[(i)]
    \item With probability $\gamma\in[0,1]$, the signal $\zeta_i(v_i)$ is fully informative; that is, $\zeta_i(v_i)=v_i$.
    \item With probability $1-\gamma$, the signal is completely uninformative in the sense that $\zeta_i(v_i)\sim F(\cdot)$.
\end{enumerate}
The probability $\gamma$ reflects the prediction accuracy. 
For simplicity, we may write $\zeta_i(v_i)$ as $\zeta_i$ when there is no risk of confusion. It can be shown that the designer's posterior expectation about the type $v_i$ after receiving the signal $\zeta_i$ is a convex combination of the signal $\zeta_i$ and the prior expectation $\mathbb{E}[V_i]$:
\begin{align}
\label{eq:posterior_expectation_designer}
    \mathbb{E}[v_i\mid \zeta_i] = \gamma \zeta_i + (1-\gamma)\, \mathbb{E}[V_i].
\end{align}



The designer aims to incentivize players to submit higher bids by deciding the number of admitted players $n \leq m$: the designer selects top $n$ players based on her posterior expectations of players' valuations after receiving the noisy signals $\zeta$. That is, the selection is based on the ranking of $(\mathbb{E}[v_i\mid \zeta_i])_{i\in \I^0}$, which is equivalent to the ranking of $\zeta=(\zeta_i)_{i\in \I^0}$ by \eqref{eq:posterior_expectation_designer}.
To avoid trivial cases, we assume $n\geq 2$.
We will analyze the expected revenue in both first-price and all-pay auctions. Additionally, we consider maximizing the expected highest bid in all-pay auctions (contests), which is a commonly used metric in the contests literature \citep{moldovanu_2006_contest_architecture,Jason_Optimal_Crowdsourcing_Contest}.

When $\gamma = 0$, the signals $\zeta$ are completely uninformative. In this case, the prescreening is conducted uniformly at random, referred to as \emph{blind prescreening}. When $\gamma = 1$, the signals $\zeta$ are fully informative, and the ranking of $\zeta$ is identical to the ranking of the players' true valuations $v^0=(v_i)_{i\in \I^0}$, referred to as \emph{perfect prescreening}.

The prediction accuracy $\gamma$ and the above prescreening mechanism are common knowledge, while the signals $\zeta$ are observed \textit{only} by the designer.
% ; hence, players do not know the ranking of $\zeta$. 
Aside from the common knowledge and their own private valuations, players only know which ones are admitted and which are not. The timeline is as follows:
\begin{enumerate}[(i)]
    \item The designer commits to an admission number $n$, which may depend on the total number $m$ but cannot depend on $v^0$ or $\zeta$.
    \item Each player $i\in \I^0$’s valuation $v_i$ is drawn i.i.d.\ from $F$ and is privately known.
    \item The designer observes a noisy signal $\zeta_i$ about $v_i$ for each player $i\in \I^0$.\footnote{It can be shown that the signal $\zeta_i$ is drawn from a distribution with CDF $ \Pr\{\zeta_i\leq t\}=\gamma\cdot \mathds{1}\{v_i\leq t\}+(1-\gamma)\cdot F(t).$ }
    
    \item The designer selects the top $n$ players among all $m$ players based on the rankings of $(\mathbb{E}[v_i\mid \zeta_i])_{i\in \I^0}$.
    \item Admitted players update their beliefs about their opponents' private valuations via Bayes' rule and simultaneously submit their bids.
\end{enumerate}

The prescreening mechanism serves two roles: (1) it affects the number of players who actually participate in the auction; and (2) it functions as an information disclosure mechanism. In particular, admitted players know that all admitted players are among the top in terms of the ranking of $\zeta$, and this knowledge affects their beliefs about their opponents' private valuations. To illustrate, consider the case of perfect prescreening, i.e., $\gamma=1$. In this case, the admitted players' true valuations are the top $n$ among all $m$ players (although the exact ordering among them remains unknown).
This admission event would alter players' beliefs. Furthermore, since each player knows her own valuation, her posterior beliefs will differ and should depend on her individual private valuation.
In \Cref{sec:beliefs}, we demonstrate that this is indeed the case.

\textit{Notations and Glossary.}
Let $\C^n_k := \binom{n}{k}$ be the binomial coefficient. Given a vector $x\in \mathbb{R}^n$ and a set $\I\subseteq\{1,2,\cdots,n\}$, let $x_{\I}$ denote the subvector of $x$ whose elements are indexed by $\I$.
A function $g:\mathbb{R}^n\to \mathbb{R}$ is \textit{symmetric} if for every permutation (i.e., every bijective mapping) $\pi:\{1,2,\cdots,n\}\to \{1,2,\cdots,n\}$, $g(x_1,\cdots,x_n)=g(x_{\pi(1)},\cdots,x_{\pi(n)}).$
A function $g$ is \textit{supermodular} in the domain $\mathcal{X}\subseteq\mathbb{R}^n$ if for any $x,x^\prime\in \mathcal{X}$, $g(x\wedge x^\prime)+ g(x\vee x^\prime)\geq g(x)+g(x^\prime),$
where $x\wedge x^\prime$ and $x\vee x^\prime$ denote the element-wise minimum and maximum of $x$ and $x^\prime$, respectively.
A function $g$ is \emph{affiliated} in the domain $\mathcal{X}\subseteq\mathbb{R}^n$ if for any $x,x^\prime\in \mathcal{X}$, $g(x\wedge x^\prime)\, g(x\vee x^\prime)\geq g(x)\, g(x^\prime).$
For a strictly positive function $g$, affiliation is equivalent to the property that $\ln g$ is supermodular, which is referred to as \textit{log-supermodular}.
When the affiliated function $g$ is a joint density, we say the random vector $V$ sampled from $g$ is affiliated.