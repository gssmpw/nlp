
%%%%%%%%%%%% All-pay Auctions
\section{All-pay Auctions}\label{sec:all-pay_auctions}





We first characterize the equilibrium strategy in \Cref{subsec:equilibrium_allpay} for a given admitted number $n$. We then discuss the optimal number $n^\ast$ in \Cref{subsec:optimal_number_allpay}.

%%%%%%%%%%%%%%%%%%%%%%%%% equilibrium analysis: allpay
\subsection{Equilibrium Analysis}
\label{subsec:equilibrium_allpay}

\begin{definition}[Equilibrium Definition]
\label{def:equilibrium_allpay}
For all-pay auctions, a strategy $\sigma:[0,1]\to \mathbb{R}_+$ is a symmetric Bayeisan-Nash equilibrium strategy if for all $i\in \I$ and all $v_i\in[0,1 ]$, 
\begin{align}
\label{eq:def_equilibrium_all_pay}
  \sigma(v_i)\in {\normalfont{\argmax}}_{b_i\in \mathbb{R}_+}~  \Pr\{\sigma(V_j)<b_i,\forall j\in \I_{-i}\} \cdot v_i - b_i,  
\end{align}
where $V_{-i}$ draws from player $i$'s belief $\beta(\cdot\mid \I,v_i;n,\gamma)$ given by \eqref{eq:beta_posterior}.
\end{definition}

We assume that the strategy $\sigma(\cdot)$ is differentiable, as is standard in the auction literature \citep{milgrom_1982_auctiontheory_competitive_bidding,milgrom_2004_putting_auctiontheory_to_work}.
When $\sigma(\cdot)$ is strictly increasing, the winning probability $\Pr\{\sigma(V_j)<b_i,\forall j\in \I_{-i}\}$
can be simplified as follows:
\begin{align*}
 \Pr\{\sigma(V_j)<b_i,\forall j\in \I_{-i}\}
  = \Pr\left\{ \max_{j\in \I_{-i}}V_j < \sigma^{-1}(b_i)\right\} 
  = H(\sigma^{-1}(b_i)\mid v_i;n,\gamma),
\end{align*}
where the second equality holds because $H(\cdot \mid v_i;n,\gamma)$ is the CDF of $\max_{j\in \I_{-i}}V_j$ by the definition \eqref{eq:H_CDF}.
By \eqref{eq:beta_posterior}, we can further write the winning probability as follows:
\begin{align*}
  H(\sigma^{-1}(b_i)\mid v_i;n,\gamma)
  =& \kappa(v_i;n,\gamma)\cdot \int_{[0,\sigma^{-1}(b_i)]^{n-1}} \psi(v;n,\gamma) \prod_{j\in \I_{-i}}f(v_j)dv_{-i}\\
  :=& \kappa(v_i;n,\gamma)\cdot \hat{H}(\sigma^{-1}(b_i)\mid v_i;n,\gamma).
\end{align*}
Plugging the above formulation back into \eqref{eq:def_equilibrium_all_pay}, we have player $i$'s utility when she bids $b_i$ while all other players follow the strategy $\sigma(\cdot)$ as follows:
\begin{align}
\label{eq:utility_inflatedtype_allpay}
 \hat{H}(\sigma^{-1}(b_i)\mid v_i;n,\gamma)\cdot \underbrace{\kappa(v_i;n,\gamma)v_i}_{\textrm{``inflated'' type}} - b_i.
\end{align}

From the perspective of \eqref{eq:utility_inflatedtype_allpay}, player $i$ appears to have a new type $\kappa(v_i;n,\gamma)v_i:=\theta(v_i;n,\gamma)$. We call it the \textit{inflated type} since $\kappa(v_i;n,\gamma)\geq \C^{m-1}_{n-1} \geq 1$ established by the following lemma.
\begin{lemma}
\label{lem:kappa_vi}
For any $n\in \{2,3,\cdots,m\}$ and any $\gamma\in[0,1]$, we have 
\begin{enumerate}[(i)]
    \item $\kappa(v_i;n,\gamma)$ is non-increasing in $v_i\in [0,1]$.
    \item $\kappa(v_i;n,\gamma)\geq \frac{\C^m_n}{(1-\gamma) + \gamma \frac{m}{n}} \geq \C^{m-1}_{n-1}$.
\end{enumerate}
\end{lemma}

Our next result provides sufficient conditions to guarantee the existence of a symmetric and strictly monotone\footnote{Obviously, there does not exist a strictly decreasing equilibrium strategy. The monotonicity here refers to increasing.} (SSM) equilibrium, and provides explicitly characterizations when it exists.

\begin{theorem}[Equilibrium Strategy]
\label{thm:strictlymonotone_equilibrium_allpay}
In all-pay auctions, for any $n\in \{2,3,\cdots,m\}$ and any $\gamma\in [0,1]$, 
\begin{enumerate}[(i)]
    \item There exists a symmetric and strictly monotone (SSM) equilibrium if $v_i h(\tilde{v}_i\mid v_i;n,\gamma) - \tilde{v}_i h(\tilde{v}_i\mid \tilde{v}_i;n,\gamma)$ is non-negative for $\tilde{v}_i\in [0,v_i]$ and non-positive for $\tilde{v}_i\in [v_i,1]$ for any given $v_i\in [0,1]$.
    \item If the inflated type $\theta(v_i;n,\gamma)$ is non-decreasing in $v_i$ for {\normalfont{all}} $v_i\in [0,1]$, the condition in (i) holds.
    \item If there exist SSM equilibrium strategies, they are unique and admit the following formula:
    \begin{align}
    \label{eq:SSM_equilibrium_all_pay}
        \sigma^{\mathsf{AP}}(v_i;n,\gamma)
  =\int_0^{v_i} x h(x\mid x;n,\gamma)dx,~ \forall v_i\in [0,1].
    \end{align}
\end{enumerate}
\end{theorem}

The expression $v_i h(\tilde{v}_i\mid v_i;n,\gamma) - \tilde{v}_i h(\tilde{v}_i\mid \tilde{v}_i;n,\gamma)$ in \Cref{thm:strictlymonotone_equilibrium_allpay} (i) is actually the derivative of $U^{\mathsf{AP}}(\tilde{v}_i,v_i;\sigma^{\mathsf{AP}}(\cdot;n,\gamma))$ with respect to $\tilde{v}_i$, where $U^{\mathsf{AP}}(\tilde{v}_i,v_i;\sigma^{\mathsf{AP}}(\cdot;n,\gamma))$ is the expected utility of player $i$ with valuation $v_i$ when she bids $\sigma^{\mathsf{AP}}(\tilde{v}_i;n,\gamma)$ for some $\tilde{v}_i$ while all other players follow the strategy $\sigma^{\mathsf{AP}}(\cdot;n,\gamma)$. The condition (i) guarantees that player $i$'s utility $U^{\mathsf{AP}}(\tilde{v}_i,v_i;\sigma^{\mathsf{AP}}(\cdot;n,\gamma))$ is first increasing in $\tilde{v}_i\in [0,v_i]$ and then decreasing in $\tilde{v}_i\in [v_i,1]$. That is, her utility is maximized when she bids $\sigma^{\mathsf{AP}}(v_i;n,\gamma)$. We want to comment that \Cref{thm:strictlymonotone_equilibrium_allpay} (i) holds beyond the prescreening setting: it applies to any private-valuation all-pay auction with arbitrarily correlated types.

\Cref{thm:strictlymonotone_equilibrium_allpay} (ii) provides an easier-to-verify sufficient condition related to the inflated type $\theta(v_i;n,\gamma)$. Combining \Cref{lem:kappa_vi} (i) that $\kappa(v_i;n,\gamma)$ is (weakly) decreasing, to make the inflated type $\theta(v_i;n,\gamma)=v_i\cdot \kappa(v_i;n,\gamma)$ (weakly) increasing, the term $\kappa(v_i;n,\gamma)$ should decrease sublinearly.

The uniqueness in \Cref{thm:strictlymonotone_equilibrium_allpay} (iii), if there indeed exist SSM equilibrium strategies, comes from the fact that there exists a unique solution of the ordinary differential equation originating from the first-order condition. The sufficient conditions in (i) and (ii) basically guarantee that the solution from the FOC is indeed a maximizer. When $n=m$, i.e., without prescreening, the problem degenerates to a standard i.i.d. all-pay auction with $m$ players. In such a situation, 
\begin{align*}
 h(\tilde{v}_i;v_i;n=m,\gamma)=mF^{m-1}(\tilde{v}_i)f(\tilde{v}_i),
\end{align*}
and condition (i) is always satisfied; thus, an SSM equilibrium strategy always exists when $m=n$. The strategy \eqref{eq:SSM_equilibrium_all_pay} in this situation degenerates to: for any prediction accuracy $\gamma\in [0,1]$,
\begin{align}
\label{eq:equilibrium_standard_allpay}
 \sigma^{\mathsf{AP}}(v_i;n=m,\gamma) = \int_0^{v_i}x\,dF^{m-1}(x),~ \forall v_i\in[0,1].   
\end{align}




We now discuss the special case $\gamma=1$ and show that the increasing property of the inflated type is also \textit{necessary} for the existence of an SSM equilibrium strategy.

\begin{proposition}[Perfect Prescreening]
\label{prop:SSM_gamma_1_allpay}
In all-pay auctions, when $\gamma=1$, for any $n\in \{2,3,\cdots,m\}$:
\begin{enumerate}[(i)]
    \item There exists an SSM equilibrium strategy {\normalfont{if and only if}} the inflated type $\theta(v_i;n,\gamma=1)$ is non-decreasing with $v_i\in [0,1]$.
    \item If there exist SSM equilibrium strategies, they are unique and admit the following formula:
    \begin{align}
    \label{eq:SSM_equilibrium_gamma_1_all_pay}
        \sigma^{\mathsf{AP}}(v_i;n,\gamma=1)
   = \frac{1}{\C^{m-1}_{n-1}}\int_0^{v_i}\theta(v_i;n,\gamma=1)dF^{m-1}(x)
   = \int_0^{v_i} \frac{x}{J(F(x),n,m)} dF^{m-1}(x),
    \end{align}
    where $J(F(x),n,m) = \C^{m-1}_{n-1}/\kappa(x;n,\gamma=1)$ satisfies the following properties:
    \begin{enumerate}
        \item $J(x,m,m)=1$ for any $x\in [0,1]$ and $J(x,n,m)\leq J(1,n,m) = 1$ for any $x\in[0,1]$ and any $n\in \{2,3,\cdots,m\}$.
        \item $J(x,n,m)$ is non-decreasing with $x\in [0,1]$ for any given $n\in \{2,3,\cdots,m\}$.
        \item $J(x,n,m)$ is non-decreasing with $n\in \{2,3,\cdots,m\}$ for any given $x\in[0,1]$.
    \end{enumerate}
\end{enumerate}
\end{proposition}




The necessity in \Cref{prop:SSM_gamma_1_allpay} (i) is based on the following key equation, as shown in the proof:
\begin{align}
\nonumber
 \frac{\partial U^{\mathsf{AP}}(\tilde{v}_i,v_i;\sigma^{\mathsf{AP}}(\cdot;n,\gamma=1))}{\partial \tilde{v}_i}
  \propto \theta(v_i;n,\gamma=1) - \theta(\tilde{v}_i;n,\gamma=1),\quad \forall \tilde{v}_i\in [0,v_i].
\end{align}
This equation shows that the derivative of player $i$'s utility (with respect to $\tilde{v}_i$) is, up to a constant, equal to the difference between the inflated type evaluated at $v_i$ and at $\tilde{v}_i$. Based on this, if there exists an interval on which $\theta(\cdot;n,\gamma=1)$ is strictly decreasing, then for a player with valuation $v_i$ equal to the right boundary of this interval, her utility is decreasing in $\tilde{v}_i$ in this interval.
Thus, this player would have an incentive to deviate from the equilibrium strategy. 
This leads to the necessity in (i).


When $\gamma=1$, the SSM equilibrium strategy (if it exists) given in \eqref{eq:SSM_equilibrium_gamma_1_all_pay} has a clear structure: all effects of prescreening are captured by the denominator $J(F(x),n,m)$ (observe that the only difference between \eqref{eq:SSM_equilibrium_gamma_1_all_pay} and \eqref{eq:equilibrium_standard_allpay} is the denominator $J(F(x),n,m)$), and this denominator is both (weakly) increasing in $x$ and in $n$.

\Cref{prop:SSM_gamma_1_allpay} (ii).(b) mainly comes from \Cref{lem:kappa_vi} (i), which shows that $\kappa(v_i;n,\gamma)$ is (weakly) decreasing in $v_i$. Thus, $J(x,n,m)$, being proportional to $1/\kappa(v_i;n,\gamma=1)$, is (weakly) increasing in $v_i$. This essentially means that the inflation effect in the inflated types of players $\theta(v_i;n,\gamma)=v_i\kappa(v_i;n,\gamma)$ diminishes as $v_i$ increases. Finally, \Cref{prop:SSM_gamma_1_allpay} (ii).(c), combined with \eqref{eq:SSM_equilibrium_gamma_1_all_pay}, implies that admitting fewer players would incentivize a participating player to submit a higher bid. This is one of the crucial properties used to characterize optimal prescreening, as discussed immediately in the following subsection.







%%%%% optimal prescreening mechanism for all-pay auctions
\subsection{Optimal Prescreening}
\label{subsec:optimal_number_allpay}





We focus on two metrics: expected revenue and expected highest bid.  
Given $(\gamma,F,m)$, let 
\begin{align}
\label{eq:rev_n_allpay}
 \R^{\AP}(n;\gamma,F,m) = n\cdot \mathbb{E}_{V_i\sim G^{\mar}(\cdot ; n,\gamma)}[\sigmaAP(V_i;n,\gamma)]
\end{align}
be the expected revenue when the admitted number is $n$.
Define the optimal expected revenue as
\begin{align*}
  \R_\ast^{\AP}(\gamma,F,m) := \max_{n=2,3,\cdots,m} \R^{\AP}(n;\gamma,F,m).
\end{align*}
Similarly, define the expected highest bid and the optimal expected highest bid as
\begin{align*}
\HB^{\AP}(n;\gamma,F,m) = \mathbb{E}_{V_i\sim G^{\lar}(\cdot ;n,\gamma)}[\sigmaAP(V_i;n,\gamma)],
\quad \textrm{and}
\quad 
 \HB_\ast^{\AP}(\gamma,F,m) := \max_{n=2,3,\cdots,m} \HB^{\AP}(n;\gamma,F,m).
\end{align*}
Note that the expectations
% \footnote{Recall that $G^{\mar}(\cdot;n,\gamma)$ is the marginal CDF of the joint distribution $g(\cdot;n,\gamma)$ given by \eqref{eq:joint_dist_g}, and $G^{\lar}(\cdot ;n,\gamma)$ is the CDF of the largest order statistic among variables sampled from $g(\cdot;n,\gamma)$.} 
are taken with respect to distributions related to the joint density $g(\cdot;n,\gamma)$ rather than over the prior distribution $F$. This is because the designer commits to the number $n$ \textit{before} any realizations occur. When making the decision about $n$, the designer has no additional information and must contemplate the entire game process.\footnote{A similar reasoning is also found in the settings of multi-stage all-pay auctions (contests) \citep{moldovanu_2006_contest_architecture} and in the Bayesian persuasion literature \citep{kamenica_2011_persuasion}.}
We may drop $m$ in the above notations when there is no confusion.



\begin{theorem}[Optimal Prescreening]
\label{thm:opt_admittednumber_allpay}
In all-pay auctions, given a total number of players $m$:
\begin{enumerate}[(i)]
\item When $\gamma=1$, suppose there exists an SSM equilibrium strategy for all $n\in \{2,\cdots,m\}$, the unique optimal solution is $n^\ast=2$ in terms of both the expected revenue and the expected highest bid.

\item When $\gamma=0$, 
\begin{enumerate}
    \item in terms of the expected revenue, $n^\ast=m$;
    \item in terms of the expected highest bid,
    for the power law prior distribution $F(x)=x^c,c>0$,
    if $c\in (0,3], n^\ast=m;$ if $c\in \left(3,\frac{5+\sqrt{33}}{2}\right), n^\ast=m \wedge \frac{2(c-1)^2}{c^2-3c}$;\footnote{When $\frac{2(c-1)^2}{c^2-3c}$ is not an integer, $n^\ast$ is either $m \wedge \bigg\lceil{\frac{2(c-1)^2}{c^2-3c}}\bigg\rceil$ or $n^* = m \wedge \bigg\lfloor{\frac{2(c-1)^2}{c^2-3c}}\bigg\rfloor$.} if $c\geq\frac{5+\sqrt{33}}{2}$, $n^\ast=2$.
\end{enumerate}

\item When $m=3$, for a power law prior distribution $F(x)=x^c,c>0$, suppose $c$ guarantees the existence of an SSM equilibrium strategy for all $\gamma\in[0,1]$ when $n=2$. Then, 
if it is optimal to admit only two players under a small $\gamma$, it is also optimal to admit only two players under a larger $\gamma^\prime>\gamma$.
\end{enumerate}
\end{theorem}



It is worth emphasizing that \Cref{thm:opt_admittednumber_allpay} (i) holds for any total number $m$ and any prior distribution. The result regarding the expected highest bid follows mainly from \Cref{prop:SSM_gamma_1_allpay}, from which we know that when $\gamma=1$, fewer admitted players induce higher bids from all participants. A formal proof is provided in the appendix.

Perhaps surprisingly, in terms of expected revenue, the optimal admitted number is also two when $\gamma=1$. Although admitting fewer players leads to higher bids by each individual (due to the inflated type), it also means that fewer players are paying the bid. At first glance, it is unclear which effect would dominate; however, \Cref{thm:opt_admittednumber_allpay} (i) suggests that the effect of the inflated type always dominates the reduction in the number of bidders.

We briefly discuss the proof of \Cref{thm:opt_admittednumber_allpay} (i) regarding the expected revenue. When $\gamma=1$, the expected revenue can be simplified as follows.
\begin{align*}
 \R^{\AP}(n;\gamma=1,F) & = n\cdot \mathbb{E}_{V_i\sim G^{\mar}(\cdot\mid n,\gamma)}[\sigmaAP(V_i;n,\gamma)]
 = n\int_0^1 \int_0^{v_i} \frac{x}{J(F(x),n,m)}\, dF^{m-1}(x)\, dG^{\mar}(v_i;n,\gamma)\\[1mm]
 &= \int_0^1 \frac{n}{J(F(x),n,m)} \cdot \left(1-G^{\mar}(x; n,\gamma)\right) \cdot x\, dF^{m-1}(x).
\end{align*}
The last equality holds by the Fubini theorem. Observe that all effects of prescreening on the expected revenue are captured by the term 
\[
\frac{n}{J(F(x),n,m)}\cdot \left(1-G^{\mar}(x;n,\gamma)\right).
\]
We discover a key monotonicity property of this term: for any given $x$, any $m$, and any prior distribution $F$, the term $\frac{n}{J(F(x),n,m)}\cdot \left(1-G^{\mar}(x;n,\gamma)\right)$ is decreasing in $n\in \{2,3,\cdots,m\}$.
This leads to $n^\ast=2$.

\Cref{thm:opt_admittednumber_allpay} (ii) discusses the situation of blind prescreening, i.e., $\gamma=0$. In this case, the beliefs of participating players do not change and an SSM equilibrium strategy always exists, which is the same as that in a standard auction with i.i.d.\ priors. By revenue equivalence in such a setting \citep{myerson_1981_optimal_auction}, the expected revenue of all-pay auctions equals the expected revenue in a second-price auction, which is the expectation of the second largest order statistic among $n$ i.i.d. random variables. It can be shown that this expectation increases with $n$. 
Thus, in terms of the expected revenue in the case of blind prescreening, $n^\ast=m$.
However, the expected highest bid in all-pay auctions is \textit{not} monotone in the number of participants $n$ and it relies heavily on the prior distribution. \Cref{thm:opt_admittednumber_allpay} (ii).(b) shows that when $c$ is small (i.e., the environment is less competitive), it is optimal to admit all players; when $c$ is large (i.e., the environment is very competitive), it is optimal to admit only two players. This implies that even when the prediction accuracy is zero, it can be strictly beneficial for the designer to conduct prescreening.


\Cref{thm:opt_admittednumber_allpay} (iii) is illustrated in \Cref{fig:m_3_allpay}. It shows that the optimal admitted number $n^\ast$ is weakly decreasing in $\gamma$. In addition, the left panel of \Cref{fig:m_3_allpay} indicates that the optimal expected revenue (represented by the color) is weakly increasing with the prediction accuracy $\gamma$. Specifically, in the region where $n^\ast=3$, the optimal expected revenue does not depend on $\gamma$, whereas when $n^\ast=2$, the optimal expected revenue is strictly increasing with $\gamma$. This observation roughly implies that admitting \textit{truly top} players improves the designer's revenue more. Furthermore, the optimal expected revenue is increasing in $c$, which is intuitive because a larger $c$ makes it more likely that bidders have higher valuations and, hence, are more inclined to bid more. Similar insights hold for the expected highest bid. In fact, by comparing the left and right panels of \Cref{fig:m_3_allpay}, we find that if it is optimal to admit only two players in terms of the expected revenue, then it is also optimal in terms of the expected highest bid.





\begin{figure}[ht]
    \centering
 \includegraphics[width=\textwidth]{fig/allpay/rev_hb_m_3.pdf}
    \caption{Optimal Prescreening for All-pay Auctions with $m=3$. 
    }
    \label{fig:m_3_allpay}
    \vspace{0.5em}
        Note.\textit{ The prior distribution is $F(x)=x^c,c\in(0,1]$, where $c\leq 1$ is to guarantee there exists an SSM equilibrium strategy for all $n\in\{2,3\}$ and for {\normalfont{all}} $\gamma\in[0,1]$.
    The color represents the optimal expected revenue $\R_\ast^{\AP}(\gamma,c)$ in the left panel (the optimal highest bid $\HB_\ast^{\AP}(\gamma,c)$ in the right panel) given each $(\gamma,c)$.}
\end{figure}





Although analytically characterizing the optimal admitted number remains elusive for settings with general $\gamma$ and $m$, it is computationally tractable, as explained by our following remark.


\begin{figure}[ht]
    \centering
    \begin{subfigure}[b]{0.49\textwidth} % Set the width to slightly less than half the text width
    \includegraphics[width=\textwidth]{fig/allpay/rev_hb_m_4.pdf}
    \caption{$m=4$}
    \label{fig:rev_m_4_allpay}
    \end{subfigure}
    \hfill 
    \begin{subfigure}[b]{0.49\textwidth}
    \includegraphics[width=\textwidth]{fig/allpay/rev_hb_m_5.pdf}
    \caption{$m=5$}
    \label{fig:hb_m_5_allpay}
    \end{subfigure}

\begin{subfigure}[b]{0.49\textwidth}
    \includegraphics[width=\textwidth]{fig/allpay/rev_hb_m_6.pdf}
    \caption{$m=6$}
    \label{fig:hb_m_6_allpay}
    \end{subfigure}
    \hfill
    \begin{subfigure}[b]{0.49\textwidth}
    \includegraphics[width=\textwidth]{fig/allpay/rev_hb_m_7.pdf}
    \caption{$m=7$}
    \label{fig:hb_m_7_allpay}
    \end{subfigure}
    
    \caption{Optimal Prescreening for All-pay Auctions. 
    }
    \label{fig:general_m_allpay}
\vspace{0.5em}    
      Note. \textit{The prior distribution is $F(x)=x^c,c\in(0,1/(m-2)]$ when the total number is $m$. The condition $c\leq 1/(m-2)$ is to guarantee there exists an SSM equilibrium strategy for all $n\in\{2,3,\cdots,m\}$ and for {\normalfont{all}} $\gamma\in[0,1]$.
    The color represents the optimal expected revenue $\R_\ast^{\AP}(\gamma,c)$ in the left panel (optimal highest bid $\HB_\ast^{\AP}(\gamma,c)$ in the right panel) given each $(\gamma,c)$.}
\end{figure}


\begin{remark}[Computational Tractability]
\label{remark:generalsetting_allpay}
By exchanging the order of integration, we can simplify the expected revenue in \eqref{eq:rev_n_allpay} as follows:
\begin{align*}
   \R^{\AP}(n;\gamma,F) = \int_0^1 x\, h(x\mid x;n,\gamma)\cdot \left(1-G^{\mar}(x;n,\gamma)\right)dx,
\end{align*}
where we have the closed-form expressions of $h(x\mid x;n,\gamma)$ and $G^{\mar}(x;n,\gamma)$ provided in \Cref{app_sec:auxiliary_results}. Thus, to find the optimal admitted number $n$, we only need to compute the \emph{one-dimensional} integral for each $n\in\{2,\ldots,m\}$, which is computationally tractable. 
\emph{However, without our closed-form expressions of $h(x\mid x;n,\gamma)$ and $G^{\mar}(x;n,\gamma)$, $\R^{\AP}(n;\gamma,F)$ can involve up to $(m-1)^2$-dimensional integrals, 
% (see the proof for detailed arguments), 
which are intractable especially for large $n$.}
Notice that, with our closed forms, no matter what $m$ is, we only need to compute \emph{one-dimensional} integral (for $m-1$ times).
A similar logic applies to the expected highest bid.
\end{remark}




We provide further numerical results in \Cref{fig:general_m_allpay}, where we adopt the power law prior distribution and restrict $c$ to the range that guarantees the existence of an SSM equilibrium strategy for \textit{all} $\gamma\in[0,1]$ (i.e., for small $c$). The prescreening mechanism can improve the revenue substantially. For example, when $m=7$, $c=0.2$, and $\gamma=1$, admitting only two players increases revenue by 31\% compared with admitting all players. 
Another interesting observation is that in terms of both expected revenue and expected highest bid, when $c$ is small, optimal prescreening often involves either admitting all players ($n=m$) or admitting only two players ($n=2$). However, although admitting $n\notin\{2,m\}$ does not simultaneously outperform both extremes, it can yield a higher outcome than one of the extremes even when $c$ is small. Moreover, if we consider the $\gamma$-dependent SSM existence condition, there exist situations in which an SSM equilibrium strategy exists but the optimal admitted number satisfies $n^\ast\notin \{2,m\}$ (see \Cref{thm:opt_admittednumber_allpay} (ii) for an example).





