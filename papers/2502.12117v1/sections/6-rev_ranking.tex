

\section{Revenue Ranking}
\label{sec:rev_ranking}





We now compare the optimal revenue in all-pay auctions and the optimal revenue in first-price auctions.
Given $(\gamma,F,m)$, let $\R^{\FP}(n;\gamma,F,m)$ be the revenue in first-price auctions when admitting $n$ players. 
Let $\R_\ast^{\FP}(\gamma,F,m):=\max_{n\in \{2,\cdots,m\}}\R^{\FP}(n;\gamma,F,m)$ be the revenue under optimal $n$.

\begin{theorem}[Revenue Ranking]
\label{thm:rev_ranking}
When $\gamma=1$, suppose there exists an SSM equilibrium strategy under all-pay auctions.
\begin{enumerate}[(i)]
    \item $\R_\ast^{\AP}(\gamma=1,F,m)>\R_\ast^{\FP}(\gamma=1,F,m)=\R^{\FP}(n=m;\gamma,F,m)$.
    That is, the revenue of all-pay auctions with optimal prescreening is {\normalfont{strictly}} higher than the revenue of first-price auctions with optimal prescreening in the case of $\gamma=1$.
    
    \item When $m$ is large enough,  $\R_\ast^{\AP}(\gamma=1,F,m)> \R^{\FP}(n=m+1;\gamma,F,m+1)$.
    That is, the revenue of all-pay auctions with optimal prescreening is {\normalfont{strictly}} higher than the revenue of first-price auctions without prescreening but with one additional bidder when $m$ is large enough and $\gamma=1$.
\end{enumerate}
\end{theorem}



Regarding (i), notice that when admitting all $m$ players, the joint distribution in the equivalent game (see \Cref{subsec:equivalent_game}) is the same as the prior beliefs. By the revenue equivalence theorem in the independent private-valuation setting \citep{myerson_1981_optimal_auction}, we have 
$\R^{\FP}(n=m;\gamma,F,m)=\R^{AP}(n=m;\gamma,F,m)$. By \Cref{thm:opt_admittednumber_allpay}, we know that 
$\R_\ast^{\AP}(\gamma=1,F,m)=\R^{AP}(n=2;\gamma=1,F,m)>\R^{AP}(n=m;\gamma=1,F,m)$. Also, by \Cref{thm:opt_admittednumber_firstprice}, prescreening does not improve the expected revenue in first-price auctions. Combining the above properties, we have (i).

\Cref{thm:rev_ranking} (ii) shows that in the case of perfect prescreening, when the total number $m$ is large enough, the revenue of all-pay auctions with the optimal admitted number is strictly larger than the revenue in first-price auctions with even one additional bidder. Combining this with the Bulow-Klemperer result \citep{bulow_1994_auctions_negotiations}—which states that (in the absence of prescreening) the revenue of a first-price auction with $m+1$ bidders is greater than the revenue of the optimal auction format with $m$ bidders when the prior $F$ is regular—we have the following remark.
\begin{remark}
 If the prior $F$ is regular, i.e., the virtual value is non-decreasing, then when $\gamma=1$ and the total number $m$ is large enough, the revenue of all-pay auctions when admitting only two players is greater than the revenue of the optimal auction format (i.e., the second-price auction with a reserve price) when admitting all players.
\end{remark}

We highlight that \Cref{thm:rev_ranking} (ii) does \textit{not} require the regularity condition on the prior distribution $F$ as long as $m$ is large enough and an SSM equilibrium strategy exists. This is illustrated in \Cref{fig:rev_comparison}, where the prior distributions are irregular. Besides, as we can see, the total number $m$ does \textit{not} need to be very large to guarantee that \Cref{thm:rev_ranking} (ii) holds.
We provide additional numerical results in \Cref{app_subsec:numerical} beyond the case of perfect prescreening, where we find a similar revenue-ranking result holds.







\begin{figure}[ht]
    \centering
    % Resize the imported TikZ figure to the width of the text
    \resizebox{0.8\textwidth}{!}{%
        \input{fig/revenue_comparison/gamma=1/rev_comparison.tikz}%
    }
    \caption{Revenue Comparison}
    \label{fig:rev_comparison}
\end{figure}
