\section{First-price Auctions}
\label{sec:firstprice_auctions}




We now proceed to first-price auctions. We first characterize the equilibrium and then discuss the optimal prescreening.
The equilibrium definition for first-price auctions is almost the same as \Cref{def:equilibrium_allpay} for all-pay auctions except for replacing \eqref{eq:def_equilibrium_all_pay} with
\begin{align*}
  \Pr\{\sigma(V_j)<b_i,\forall j\in \I_{-i}\} \cdot \left[v_i - b_i  \right].
\end{align*}
That is, only the winner makes the payment.

Define
\begin{align}
\label{eq:L}
    L(x\mid v_i;n,\gamma)
 :=\exp\left(\int_{v_i}^x \frac{h(t\mid t;n,\gamma)}{H(t\mid t;n,\gamma)}dt\right)
 =\exp\left(-\int_{x}^{v_i} \frac{h(t\mid t;n,\gamma)}{H(t\mid t;n,\gamma)}dt\right),~\forall x\in [0,v_i].
\end{align}
Observe that $L(x\mid v_i;n,\gamma)$ is increasing in $x$ and $L(v_i\mid v_i;n,\gamma) =1$.
Besides, define
\begin{align}
\FP(\tilde{v}_i,v_i;n,\gamma) :=
h(\tilde{v}_i\mid v_i;n,\gamma) \left[v_i - \tilde{v}_i\right]
 +H(\tilde{v}_i\mid v_i;n,\gamma) \int_0^{\tilde{v}_i}L(x\mid \tilde{v}_i;n)dx
\left[
 \frac{h(\tilde{v}_i\mid v_i;n,\gamma)}{H(\tilde{v}_i\mid v_i;n,\gamma)}  - \frac{h(\tilde{v}_i\mid \tilde{v}_i;n,\gamma)}{H(\tilde{v}_i\mid \tilde{v}_i;n,\gamma)}
 \right].\nonumber   
\end{align}

\begin{theorem}[Equillibrium Strategy]
\label{thm:SSM_firstprice}
In first-price auctions:
\begin{enumerate}[(i)]
    \item There exists an SSM equilibrium strategy if $\FP(\tilde{v}_i,v_i;n,\gamma)$ is non-negative for $\tilde{v}_i\in [0,v_i]$ and non-positive for $\tilde{v}_i\in [v_i,1]$ for any $v_i\in [0,1]$.
    \item If there exist SSM equilibrium strategies, they are unique and admit the following formula:
    \begin{align}
    \label{eq:SSM_equilibrium_firstprice}
     \sigma^{\FP}(v_i;n,\gamma) = v_i - \int_0^{v_i}L(x\mid v_i;n,\gamma)dx,~ \forall v_i \in [0,1] .  
    \end{align}
    \item When $\gamma=1$, the condition in (i) always holds and thus there exists an SSM equilibrium strategy. Besides, the equilibrium strategy is the same as the standard first-price auctions without prescreening, i.e.,
    \begin{align*}
     \sigma^{\FP}(v_i;n,\gamma=1) = \sigma^{\FP}(v_i;n=m,\gamma) = v_i - \int_0^{v_i} \frac{F^{m-1}(x)}{F^{m-1}(v_i)}  dx,~\forall v_i\in [0,1].
    \end{align*}
    \item For the power law prior distribution $F(x)=x^c$, if $m=3$ and $c\in (0,1]$, the condition in (i) holds for any $n\in \{2,m\}$. 
\end{enumerate}
\end{theorem}

Let $U^{\FP}(\tilde{v}_i,v_i;\sigma^{\FP}(\cdot;n,\gamma))$ be the utility of player $i$ with valuation $v_i$ when she bids $\sigma^{\FP}(\tilde{v}_i;n,\gamma)$ for some $\tilde{v}_i\in [0,1]$ while all other participating players follow the strategy $\sigma^{\FP}(\cdot;n,\gamma)$ given by \eqref{eq:SSM_equilibrium_firstprice}. The term $\FP(\tilde{v}_i,v_i;n,\gamma)$ is actually the derivative of the utility $U^{\FP}(\tilde{v}_i,v_i;\sigma^{\FP}(\cdot;n,\gamma))$ with respect to $\tilde{v}_i$, that is,
\begin{align*}
  \FP(\tilde{v}_i,v_i;n,\gamma) =\frac{\partial U^{\FP}(\tilde{v}_i,v_i;\sigma^{\FP}(\cdot;n,\gamma))}{\partial \tilde{v}_i}.
\end{align*}
Thus, the condition in \Cref{thm:SSM_firstprice} (i) guarantees that player $i$'s utility $U^{\FP}(\tilde{v}_i,v_i;\sigma^{\FP}(\cdot;n,\gamma))$ increases in $\tilde{v}_i\in [0,v_i]$ and decreases in $\tilde{v}_i\in [v_i,1]$, which implies that it is optimal for player $i$ to follow the strategy $\sigma^{\FP}(\cdot;n,\gamma)$. This certifies $\sigma^{\FP}(\cdot;n,\gamma)$ as an equilibrium.

\Cref{thm:SSM_firstprice} (ii) is proven by solving the first-order condition (FOC) and showing the uniqueness of its solution. Notice that a strategy is certified as an equilibrium if, for any player with \textit{any} valuation $v_i$, there is no incentive to deviate when all other players follow this strategy. Based on this, it is easy to show that the FOC condition is necessary for a strategy to be an equilibrium.

The first part in \Cref{thm:SSM_firstprice} (iii) regarding the existence of an SSM equilibrium strategy comes from \Cref{prop:g_property}, which shows that when $\gamma=1$, the joint density $g(\cdot;n,\gamma)$ is affiliated in $v\in [0,1]^n$. Any private-value first-price auction with affiliated types admits an SSM equilibrium as established by \citet{milgrom_1982_auctiontheory_competitive_bidding}.

Perhaps surprisingly, the second part in \Cref{thm:SSM_firstprice} (iii) shows that the equilibrium strategy does \textit{not} depend on the admitted number when $\gamma=1$. This mainly comes from the fact that when $\gamma=1$, 
\begin{align*}
 H(t\mid t;n,\gamma=1) = \frac{F^{m-1}(t)}{J(F(t),n,m)}   \quad
 \textrm{and} \quad
  h(t\mid t;n,\gamma=1) = \frac{(F^{m-1}(t))^\prime}{J(F(t),n,m)},
\end{align*}
where the function $J(F(t),n,m)$ is defined in \Cref{prop:SSM_gamma_1_allpay}. By the definition of \eqref{eq:H_CDF}, $H(t\mid t;n,\gamma=1)$ is the winning probability of a player with valuation $t$ when all players (including him) follow a strictly monotone strategy. Although this winning probability is inflated by the denominator $J(F(t),n,m)\leq 1$ (note that $H(t\mid t;n=m,\gamma)=F^{m-1}(t)$), the term $h(t\mid t;n,\gamma=1)$ is also inflated by the same denominator. Thus, the fraction $\frac{h(t\mid t;n,\gamma=1)}{H(t\mid t;n,\gamma=1)}$ does not depend on $n$, leading to $L(x\mid v_i;n,\gamma)$ in \eqref{eq:L} also being independent of $n$.
As a result, the equilibrium strategy in \eqref{eq:SSM_equilibrium_firstprice} does not depend on $n$.



\Cref{thm:SSM_firstprice} (iv) shows that when the environment is not very competitive (i.e., for small $c$), for \textit{any} $\gamma\in[0,1]$, there exists an SSM equilibrium strategy when $n=2$ and $m=3$.\footnote{For any $\gamma$, $m$, and prior distribution, when $n=m$, there exists an SSM equilibrium strategy.} We emphasize that this result is non-trivial because it violates all the standard conditions used in the literature, yet an SSM equilibrium strategy still exists. Specifically, \citet{milgrom_1982_auctiontheory_competitive_bidding} shows that an SSM equilibrium strategy exists when the joint density $g(\cdot;n,\gamma)$ is affiliated. However, as mentioned in \Cref{subsec:equivalent_game}, when $\gamma\in (0,1)$, the joint density is \textit{not} affiliated. Furthermore, \citet{castro_2007_affiliation_positive_dependence} establishes a weaker condition for the existence of an SSM equilibrium, namely that $\frac{h(x\mid v_i;n,\gamma)}{H(x\mid x;n,\gamma)}$ is weakly increasing in $v_i\in [0,1]$ for any given $x\in [0,1]$ (referred to as the \textit{hazard rate increasing condition}). However, this condition also fails in our setting. 
To prove \Cref{thm:SSM_firstprice} (iv), we use the inequality that $L(x\mid v_i;n,\gamma)\leq 1$.
Since this bound is independent of the prior, we naturally require the restriction of prior distributions; the condition we find that works under this bound is $c\leq 1$. Nevertheless, our numerical results show that an SSM equilibrium strategy always exists.




Based on \Cref{thm:SSM_firstprice}, we are able to characterize the optimal admitted number in first-price auctions.

\begin{theorem}[Optimal Prescreening]
\label{thm:opt_admittednumber_firstprice}
In first-price auctions, in terms of the expected revenue, 
\begin{enumerate}[(i)]
    \item for $\gamma=0$, it is optimal to admit all players, i.e., $n^\ast=m$; 
    \item for $\gamma=1$, admitting any number of players yields the same expected revenue.
\end{enumerate}
\end{theorem}

The case of $\gamma=0$ in first-price auctions is the same as the all-pay auctions by revenue equivalence in the i.i.d. setting \citep{myerson_1981_optimal_auction}. In contrast, when $\gamma=1$, first-price auctions exhibit a sharp difference from all-pay auctions: in first-price auctions, prescreening does not affect the outcome and hence need not be conducted, whereas in all-pay auctions it is optimal to admit only two players by \Cref{thm:opt_admittednumber_allpay}.
In general settings beyond $\gamma\in \{0,1\}$, our numerical results show that it is usually optimal to admit all players, see \Cref{app_subsec:numerical} for details. 



