

%%%%%%%%%%%% Conclusion
\section{Closing Remarks}
\label{sec:conclusion}

We study a prescreening mechanism in which the designer admits only the top players after receiving the noisy signals. Our analysis demonstrates that prescreening can significantly enhance the designer's utility, particularly in competitive environments, e.g., in all-pay auctions (rather than first-price auctions), and in a setting with a larger total number of bidders and with a higher prior distribution in terms of stochastic dominance.





There are several promising avenues for future research. First, although our analysis indicates that all-pay auctions may fail to admit a symmetric and strictly monotone equilibrium strategy in some cases, we conjecture that under certain conditions a symmetric but \emph{weakly} monotone equilibrium strategy may exist. Second, our finding that admitting fewer players can substantially improve revenue in all-pay auctions—but not necessarily in first-price auctions—naturally motivates a joint study of auction design and prescreening. Third, it would be interesting to apply the prescreening to other settings, such as competitive pricing among firms. 
For instance, under the San Francisco government's Powered Scooter Share Permit Program,\footnote{https://www.sfmta.com/projects/powered-scooter-share-permit-program} only two scooter company licenses are sponsored. These companies then compete for customer demand through pricing (and probably other dimensions). Investigating how prescreening can be applied in such settings to maximize social welfare would be a valuable direction.
Lastly, our work offers opportunities for further research in algorithmic mechanism design with predictions. Although generalizing our Bernoulli predictor (noisy signals) to more complex models would be analytically intractable, exploring algorithm design in these settings remains a promising direction.





