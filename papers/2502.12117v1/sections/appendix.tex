
\part*{Appendix} 
\parttoc % Insert the appendix TOC

\setcounter{equation}{0}
\setcounter{proposition}{0}
\setcounter{lemma}{0}
\setcounter{figure}{0}


\section{Auxiliary Results}
\label{app_sec:auxiliary_results}

We provide additional numerical results in \Cref{app_subsec:numerical}. Further discussions about first-price auctions are given in \Cref{app_subsec:condition_firstprice_SSM}. The closed-form expressions for belief-related terms can be found in  \Cref{app_subsec:beliefs}. Finally, we provide additional technical results in \Cref{app_subec:identity}.





\subsection{Additional Numerical Results}
\label{app_subsec:numerical}




\subsubsection{First-price Auctions}

We now present additional numerical results for values of $\gamma$ beyond 1. From these results, we observe that the optimal admitted number, $n^\ast$, is typically equal to $m$ (see \Cref{fig:rev_3d_first-price}).

\begin{figure}[ht]
    \centering
    \includegraphics[width=\textwidth]{fig/firstprice/3D_Revenue_m_3456.pdf}
    \caption{The Revenue of First-price Auctions with Prior Distribution $F(x)=x^c$, where $c>0$.}
    \label{fig:rev_3d_first-price}
\end{figure}


\subsubsection{Revenue Comparison}

We present numerical results for values of $\gamma$ beyond 1 in \Cref{fig:rev_comparison_general_gamma}. As shown, \Cref{thm:rev_ranking} (ii) is robust with respect to the value of $\gamma$.

\begin{figure}[ht]
    \centering
    % First subfigure for gamma=1
    \begin{subfigure}[b]{0.8\textwidth}
        \centering
        \resizebox{\textwidth}{!}{%
            \input{fig/revenue_comparison/general_gamma/rev_comparison_gamma=0.9.tikz}%
        }
        \caption{ \(\gamma=0.9\)}
     
    \end{subfigure}
 
    % Second subfigure for gamma=2
    \begin{subfigure}[b]{0.8\textwidth}
        \centering
        \resizebox{\textwidth}{!}{%
            \input{fig/revenue_comparison/general_gamma/rev_comparison_gamma=0.85.tikz}%
        }
        \caption{\(\gamma=0.85\)}
       
    \end{subfigure}
    \caption{Revenue Comparison}
    \label{fig:rev_comparison_general_gamma}
\end{figure}




%%%%%%%%%%%%%%%%%%% subsec: further discussions  on the existence of SSM in first-price auctions
\subsection{Further Discussions on the Existence of SSM Equilibrium in First-price Auctions}
\label{app_subsec:condition_firstprice_SSM}


We note that our results in \Cref{thm:SSM_firstprice} (i) and (ii) also hold for general first-price auctions with \textit{arbitrarily} correlated distributions (densities) $g$. The definitions of $h$, $H$, and $L$ for a general correlated density $g$ follow in the same way as in our setting.
We drop the parameters $(n,\gamma)$ from these functions in the following.
We now summarize the conditions for the existence of an SSM equilibrium strategy in first-price auctions with arbitrarily correlated types that have been established in the literature, and compare them with our condition:
\begin{align}
  \quad & \textrm{Existence of an SSM equilibrium strategy in first-price auctions with \textit{arbitrarily} correlated types} \nonumber \\
\Longleftarrow \quad &\textrm{The condition in \Cref{thm:SSM_firstprice} (i)} \nonumber\\
\Longleftarrow \quad & \textrm{For any } v_i^\prime \leq \tilde{v}_i \leq v_i,\quad  \frac{h(\tilde{v}_i\mid v_i^\prime)}{H(\tilde{v}_i\mid v_i^\prime)} 
\leq \frac{h(\tilde{v}_i\mid \tilde{v}_i)}{H(\tilde{v}_i\mid \tilde{v}_i)} 
\leq 
 \frac{h(\tilde{v}_i\mid v_i)}{H(\tilde{v}_i\mid v_i)} \label{eq:condition_two_inequalities}\\
\Longleftarrow \quad & \textrm{For any given } \tilde{v}_i \in [0,1],\; \frac{h(\tilde{v}_i\mid v_i)}{H(\tilde{v}_i\mid v_i)} \textrm{ is weakly increasing in } v_i \in [0,1] \tag{\textit{hazard rate increasing}}\\
\Longleftarrow \quad & 
\textrm{$h(\tilde{v}_i\mid v_i)$ is affiliated in $(\tilde{v}_i,v_i) \in [0,1]^2$,} \tag{\textit{affiliation of the largest order statistics sampled from the conditional density $g(\cdot;v_i)$}} \\
\Longleftarrow \quad & 
\textrm{$g(v)$ is affiliated in $v \in [0,1]^n$.} \tag{\textit{joint affiliation}}  
\end{align}

The joint affiliation was first proposed in \citet{milgrom_1982_auctiontheory_competitive_bidding}. The condition in \eqref{eq:condition_two_inequalities} is formalized in \citet{castro_2007_affiliation_positive_dependence}. However, this condition fails in our setting except in special cases (e.g., when $\gamma\in \{0,1\}$ or $n=m$). In fact, when $m=3$ and $n=2$, we find that $\frac{h(\tilde{v}_i \mid v_i)}{H(\tilde{v}_i\mid v_i)}$ is decreasing in $v_i\in [\tilde{v}_i,1]$. In other words, the second inequality in \eqref{eq:condition_two_inequalities} is completely \textit{reversed}. Moreover, there is no monotonicity when $v_i$ lies in the range $[0,\tilde{v}_i]$. Therefore, the \Cref{thm:SSM_firstprice} (iv) is challenging to prove. We briefly outline the proof sketch below.

We first show that, given any $\tilde{v}_i$, the ratio $\frac{h(\tilde{v}_i \mid v_i)}{H(\tilde{v}_i\mid v_i)}$ first decreases and then increases as a function of $v_i$ for $v_i\leq \tilde{v}_i$. In addition, we establish that 
\[
\frac{h(\tilde{v}_i \mid 0)}{H(\tilde{v}_i\mid 0)}\leq \frac{h(\tilde{v}_i \mid \tilde{v}_i)}{H(\tilde{v}_i\mid \tilde{v}_i)}.
\]
These two properties confirm the first inequality in \eqref{eq:condition_two_inequalities}. Since the second inequality in \eqref{eq:condition_two_inequalities} does not hold, we work directly with $\FP(\tilde{v}_i,v_i)$ by using the upper bound $L(x\mid v_i)\leq 1$. The detailed proof can be found in \Cref{app_sec:proofs}.






\subsection{Beliefs Releated}\label{app_subsec:beliefs}

\begin{lemma} [Additive Property]
\label{lem:closedforms_admission_prob}
The admission probability has the additive property, i.e.,  
\begin{align*}
 \psi(v;n,\gamma)  & = \hat{\psi}_0(n,\gamma)+\sum_{k=1}^n \hat{\psi}_k(v_{(k)};n,\gamma),
\end{align*}
where $v_{(k)}$ is the $k^{\mathsf{th}}$ smallest element among all elements in the vector $v$, $\hat{\psi}_0(n,\gamma) =  \frac{(1-\gamma)^n}{\C^{m}_n}$, and
\begin{align*}
 \hat{\psi}_k(v_{(k)};n,\gamma) =& (m-n) F^{m-n} \left(v_{(k)}\right)\mathds{1}\{k\leq n-1\}\sum_{j=1}^{n-k} \C^{n-k}_{j-1}\gamma^j
 (1-\gamma)^{n-j}\sum_{i=0}^{n-j}\frac{(-1)^{i}\C^{n-j}_i}{m-n+i}F^{i}\left(v_{(k)}\right)\\
 &+ (m-n) \gamma^{n-k+1}
 (1-\gamma)^{k-1}F^{m-n}\left(v_{(k)}\right)\sum_{j=0}^{k-1}\frac{(-1)^{j}\C^{k-1}_j}{m-n+j}F^{j}\left(v_{(k)}\right).  
\end{align*}

And the normalizing term $\kappa(v_i;n,\gamma)$ is given by
\begin{align*}
 \kappa(v_i;n,\gamma)  = 1/\left( \frac{1-\gamma}{\C^m_n} + (m-n)\gamma \cdot \sum_{i=0}^{n-1}\frac{(-1)^i\C^{n-1}_i}{m-n+i} F^{m-n+i}(v_i)\right) . 
\end{align*}
\end{lemma}

Given \Cref{lem:closedforms_admission_prob}, we can directly obtain the beliefs of the cases $\gamma = 0$ and $\gamma = 1$.

\begin{corollary}\label{cor:special_gamma}
 When $\gamma=1$, $\psi(v;n,\gamma=1) = F^{m-n}(v_{(1)})$ and 
 \begin{align*}
     \frac{1}{\kappa(x; n, \gamma = 1)} =& (m-n) \sum_{i=0}^{n-1}\frac{(-1)^i\C^{n-1}_i}{m-n+i} F^{m-n+i}(v_i)\\
     =&  (n-1)\int_0^{F(x)}t^{m-n}(1-t)^{n-2}dt + F^{m-n}(x)(1-F(x))^{n-1}.
 \end{align*}
 When $\gamma=0$,
 \begin{align*}
   \psi(v;n,\gamma=0) = \frac{1}{\C^m_n},\quad \textrm{and} \quad
   \kappa(v;n,\gamma=0) = \C^m_n.
 \end{align*}
\end{corollary}

The following lemma provides a key observation that is useful for analyzing the existence of equilibrium for all-pay auctions.

\begin{lemma}[Properties of $h(\cdot\mid v_i;n,\gamma)$]
\label{lem:property_h} $h(\cdot\mid v_i;n,\gamma)$ satisfies that the following properties.

\begin{enumerate}
    \item[(i)] Let $v_{-i,-j}=[v_k]_{k\in \I_{-i}\setminus\{j\}}$ for $j\neq i$. Then $h(y\mid v_i;n, \gamma)$ can yield to
  \begin{align*}
      h(y\mid v_i;n, \gamma) =  \kappa(v_i;n,\gamma)
 \cdot \omega\left(y\mid v_i; n, \gamma\right),
  \end{align*}
where $\omega\left(y \mid v_i; n, \gamma\right) = (n-1)\cdot\int_{[0,y]^{n-2}}
\left(\prod_{k\in \I_{-i}\setminus \{j\}} f(v_k)\right)f(y)
 \cdot 
\psi(v;n,\gamma)\mid _{v_j = y}dv_{-i, -j}$.
\item[(ii)] For any $y \in [0,1]$, $\omega\left(y\mid v_i; n, \gamma\right)$ is increasing with $v_i \in [0,1]$.
\end{enumerate}
\end{lemma}



For numerical efficiency of calculating the expected revenue and expected highest bid, we now give the closed form of $H(\cdot\mid v_i;n,\gamma)$ in \eqref{eq:H_CDF}, its partial derivative $h(\cdot\mid v_i;n,\gamma)$, marginal CDF $G^{\mar}(\cdot; n,\gamma)$ of the joint distribution $g(\cdot;n,\gamma)$ in \eqref{eq:joint_dist_g} and the CDF $G^{\lar}(\cdot;n,\gamma)$ of $\max_{i\in\I}V_i$, where $V\sim g(\cdot;n,\gamma)$.

\begin{lemma}
 \label{lem:H_h}
(i). When $x \in [0,v_i]$,
\begin{align*}
\frac{H(x\mid v_i;n,\gamma)}{\kappa(v_i;n,\gamma)} &=   
 (m-n)F^{m-1}(x)\sum_{k=1}^{n-1}  \sum_{j=1}^{n-k}\C^{n-k}_{j-1}\gamma^j
 (1-\gamma)^{n-j}\sum_{i=0}^{n-j}\frac{(-1)^{i}\C^{n-j}_i\C^{n-1}_{n-k}}{(m-n+i)\C^{m+i-1}_{n-k}}F^i\left(x\right)\\
 & \quad + (m-n)F^{m-1}(x)\sum_{k=1}^{n-1} \gamma^{n-k+1}
 (1-\gamma)^{k-1}\sum_{j=0}^{k-1}\frac{(-1)^{j}\C^{k-1}_j\C^{n-1}_{n-k}}{(m-n+j)\C^{m+j-1}_{n-k}}F^{j}\left(x\right) \\
 &\quad + F^{n-1}(x)\left((m-n)\gamma(1-\gamma)^{n-1}F^{m-n}(v_i)\sum_{j=0}^{n-1}\frac{(-1)^j\C^{n-1}_j}{m-n+j}F^j(v_i) + \frac{(1-\gamma)^n}{\C^m_n}\right).
\end{align*}
When $x \in [v_i, 1]$, 
{\footnotesize{
\begin{align*}
&\frac{H(x\mid v_i;n,\gamma)}{\kappa(v_i;n,\gamma)}\\
=& \frac{(1-\gamma)^n}{\C^m_n}F^{n-1}(x) + (m-n)\sum_{k=1}^{n-1} \C^{n-1}_{k-1}F^{m-n+k-1}(v_i)(F(x) - F(v_i))^{n-k} \sum_{j=1}^{n-k} \C^{n-k}_{j-1}\gamma^j
 (1-\gamma)^{n-j}\sum_{t=0}^{n-j}\frac{(-1)^{t}\C^{n-j}_t}{m-n+t}F^{t}\left(v_{i}\right)\\
 & + (m-n)\sum_{k=1}^{n}\C^{n-1}_{k-1} \gamma^{n-k+1}
 (1-\gamma)^{k-1}F^{m-n+k-1}\left(v_{i}\right)(F(x) - F(v_i))^{n-k}\sum_{j=0}^{k-1}\frac{(-1)^{j}\C^{k-1}_j}{m-n+j}F^{j}\left(v_{i}\right)\\
 & + (m-n) \sum_{k=2}^{n}\C^{n-1}_{k-1}\sum_{s=1}^{k-1}\sum_{j=1}^{n-s} \C^{n-s}_{j-1}\gamma^j
 (1-\gamma)^{n-j}\sum_{t=0}^{n-j}\frac{(-1)^{t}\C^{n-j}_t\C^{k-1}_{k-s}}{(m-n+t)\C^{m-n+t+k-1}_{k-s}}F^{m-n+t+k-1}(v_i)(F(x) - F(v_i))^{n-k}\\
 & + (m-n)\sum_{k=2}^{n} \C^{n-1}_{k-1}\sum_{s=1}^{k-1}\gamma^{n-s+1}
 (1-\gamma)^{s-1}\sum_{j=0}^{s-1}\frac{(-1)^{j}\C^{s-1}_j\C^{k-1}_{k-s}}{(m-n+j)\C^{m-n+j+k-1}_{k-s}}F^{m-n+j+k-1}(v_i)(F(x) - F(v_i))^{n-k}\\
 & + (m-n)\sum_{k=1}^{n-2} \C^{n-1}_{k-1}\sum_{s=k+1}^{n-1}\sum_{j=1}^{n-s} \C^{n-s}_{j-1}\gamma^j
 (1-\gamma)^{n-j}\sum_{t=0}^{n-j}\frac{(-1)^{t}\C^{n-j}_t}{m-n+t}\sum_{r=0}^{m-n+t}\frac{\C^{n-k}_{n-s+1}\C^{m-n+t}_r}{\C^{n+r-k}_{n-s+1}}F^{m-n+t+k-r-1}(v_i)(F(x) - F(v_i))^{n+r-k}\\
 & + (m-n)\sum_{k=1}^{n-1}\C^{n-1}_{k-1} \sum_{s=k+1}^{n}\gamma^{n-s+1}
 (1-\gamma)^{s-1}\sum_{j=0}^{s-1}\frac{(-1)^{j}\C^{s-1}_j}{m-n+j}\sum_{r=0}^{m-n+j}\frac{\C^{n-k}_{n-s+1}\C^{m-n+j}_r}{\C^{n+r-k}_{n-s+1}}F^{m-n+j+k-r-1}(v_i)(F(x) - F(v_i))^{n+r-k}.
\end{align*}}
}
(ii). When $x \in [0,v_i]$,
 \begin{align*}
 \frac{h(x\mid v_i;n,\gamma)}{\kappa(v_i;n,\gamma)} &= (m-n)F^{m-2}(x)f(x)\sum_{k=1}^{n-1}  \sum_{j=1}^{n-k}\C^{n-k}_{j-1}\gamma^j
 (1-\gamma)^{n-j}\sum_{i=0}^{n-j}\frac{(-1)^{i}(m+i-1)\C^{n-j}_i\C^{n-1}_{n-k}}{(m-n+i)\C^{m+i-1}_{n-k}}F^i\left(x\right)\\
 & \quad + (m-n)F^{m-2}(x)f(x)\sum_{k=1}^{n-1} \gamma^{n-k+1}
 (1-\gamma)^{k-1}\sum_{j=0}^{k-1}\frac{(-1)^{j}(m+j-1)\C^{k-1}_j\C^{n-1}_{n-k}}{(m-n+j)\C^{m+j-1}_{n-k}}F^{j}\left(x\right) \\
 &\quad + (n-1)F^{n-2}(x)f(x)\left((m-n)\gamma(1-\gamma)^{n-1}F^{m-n}(v_i)\sum_{j=0}^{n-1}\frac{(-1)^j\C^{n-1}_j}{m-n+j}F^j(v_i) + \frac{(1-\gamma)^n}{\C^m_n}\right). 
 \end{align*}
When $x \in [v_i, 1]$, 
{\tiny {
\begin{align*}
&\frac{h(x\mid v_i;n,\gamma)}{\kappa(v_i;n,\gamma)}\\
=& \frac{(1-\gamma)^n}{\C^m_n}(n-1)F^{n-2}(x)f(x)\\
&+ (m-n)f(x)\sum_{k=1}^{n-1}\C^{n-1}_{k-1} F^{m-n+k-2}(v_i)(F(x) - F(v_i))^{n-k-1}\sum_{j=1}^{n-k} \C^{n-k}_{j-1}\gamma^j
 (1-\gamma)^{n-j}\sum_{t=0}^{n-j}\frac{(-1)^{t}(m-n+k+t-1)\C^{n-j}_t}{m-n+t}F^{t}\left(v_{i}\right)\\
 & + (m-n)f(x)\sum_{k=1}^{n}\C^{n-1}_{k-1} F^{m-n+k-2}(v_i)(F(x) - F(v_i))^{n-k-1}\gamma^{n-k+1}
 (1-\gamma)^{k-1}\sum_{j=0}^{k-1}\frac{(-1)^{j}(m-n+k+j-1)\C^{k-1}_j}{m-n+j}F^{j}\left(v_{i}\right)\\
 & + (m-n) f(x)\sum_{k=2}^{n}(n-k)\C^{n-1}_{k-1}\sum_{s=1}^{k-1}\sum_{j=1}^{n-s} \C^{n-s}_{j-1}\gamma^j
 (1-\gamma)^{n-j}\sum_{t=0}^{n-j}\frac{(-1)^{t}\C^{n-j}_t\C^{k-1}_{k-s}}{(m-n+t)\C^{m-n+t+k-1}_{k-s}}F^{m-n+t+k-1}(v_i)(F(x) - F(v_i))^{n-k-1}\\
 & + (m-n)f(x)\sum_{k=2}^{n} (n-k)\C^{n-1}_{k-1}\sum_{s=1}^{k-1}\gamma^{n-s+1}
 (1-\gamma)^{s-1}\sum_{j=0}^{s-1}\frac{(-1)^{j}\C^{s-1}_j\C^{k-1}_{k-s}}{(m-n+j)\C^{m-n+j+k-1}_{k-s}}F^{m-n+j+k-1}(v_i)(F(x) - F(v_i))^{n-k-1}\\
 & + (m-n)f(x)\sum_{k=1}^{n-2}\C^{n-1}_{k-1} \sum_{s=k+1}^{n-1}\sum_{j=1}^{n-s} \C^{n-s}_{j-1}\gamma^j
 (1-\gamma)^{n-j}\sum_{t=0}^{n-j}\frac{(-1)^{t}\C^{n-j}_t}{m-n+t}\sum_{r=0}^{m-n+t}\frac{(n+r-k)\C^{n-k}_{n-s+1}\C^{m-n+t}_r}{\C^{n+r-k}_{n-s+1}}F^{m-n+t+k-r-1}(v_i)(F(x) - F(v_i))^{n+r-k-1}\\
 & + (m-n)f(x)\sum_{k=1}^{n-1} \C^{n-1}_{k-1}\sum_{s=k+1}^{n}\gamma^{n-s+1}
 (1-\gamma)^{s-1}\sum_{j=0}^{s-1}\frac{(-1)^{j}\C^{s-1}_j}{m-n+j}\sum_{r=0}^{m-n+j}\frac{(n+r-k)\C^{n-k}_{n-s+1}\C^{m-n+j}_r}{\C^{n+r-k}_{n-s+1}}F^{m-n+j+k-r-1}(v_i)(F(x) - F(v_i))^{n+r-k-1}.
\end{align*}}
} 
\end{lemma}

\begin{corollary}[Special Cases]\label{cor:special_case_H_h}
The closed forms and properties of 
$H(\cdot\mid v_i;n,\gamma = 1)$ and $h(\cdot\mid v_i;n,\gamma = 1)$
for some special cases are given as follows.

\medskip


\noindent (i). When $\gamma = 1$, we have 
    \begin{align*}
      &H(x\mid v_i;n,\gamma = 1) \\
      = & \mathds{1}\{x\leq v_i\}\kappa(v_i; n, \gamma = 1)\frac{1}{\C^{m-1}_{n-1}}F^{m-1}(x)\\
      &+ \mathds{1}\{x> v_i\} \kappa(v_i; n, \gamma = 1) \left((n - 1)\int_0^{F(v_i)}t^{n_1-n_2}\left(F(x)-t \right)^{n - 2}dt + F^{m-n}(v_i)(F(x) - F(v_i))^{n - 1} \right),
    \end{align*}
and
{\small {
    \begin{align*}
      &h(x\mid v_i;n,\gamma = 1) \\
      = & \mathds{1}\{x\leq v_i\}\kappa(v_i; n, \gamma = 1)\frac{m-1}{\C^{m-1}_{n-1}}F^{m-2}(x)f(x)\\
      &+ \mathds{1}\{x> v_i\} \kappa(v_i; n, \gamma = 1)(n - 1) \left((n-2)\int_0^{F(v_i)}t^{n_1-n_2}\left(F(x)-t \right)^{n - 3}dt + F^{m-n}(v_i)(F(x) - F(v_i))^{n - 2} \right) f(x).
    \end{align*}}}
\noindent (ii). When $m=3$ and $n=2$, we have
{\scriptsize{
\begin{align*} 
 &H\left(x\mid v_i;n = 2,\gamma\right) \\
 =& \mathds{1} \{x\leq v_i\}\kappa(v_i;n = 2,\gamma)
 \left(\frac{ (1-\gamma)^2}{3} F\left(x\right) + \gamma (1-\gamma)F\left(v_i\right)F\left(x\right)
 - \frac{\gamma(1-\gamma)}{2} F^2\left(v_i\right)F\left(x\right) + \frac{\gamma}{2}  F^2\left(x\right)  - \frac{\gamma(1-\gamma)}{6} F^3\left(x\right)\right)
 \\
 & + \mathds{1}\{x\leq v_i\}\kappa(v_i;n = 2,\gamma)\left(\frac{(1-\gamma)^2}{3}F\left(x\right) + \frac{\gamma(1-\gamma)}{2} F^2\left(x\right) - \frac{\gamma(1-\gamma)}{6} F^3\left(x\right) + \gamma F\left(v_i\right)F\left(x\right) - \frac{\gamma^2}{2} F^2\left(v_i\right) - \frac{\gamma (1-\gamma)}{2} F^2\left(v_i\right) F\left(x\right)\right),
\end{align*}
}
}
and
{\small { 
\begin{align*}
 &h(x,v_i;n=2)\\
 =& \mathds{1}\{x\leq v_i\} \kappa(v_i;n = 2,\gamma) f(x) \left(\frac{ (1-\gamma)^2}{3}  + \gamma (1-\gamma)F\left(v_i\right) - \frac{\gamma(1-\gamma)}{2} F^2\left(v_i\right) + \gamma F\left(x\right) - \frac{\gamma(1-\gamma)}{2} F^2\left(x\right)\right)\\
 & + \mathds{1}\{x > v_i\} \kappa(v_i;n = 2,\gamma) f(x)\left(\frac{(1-\gamma)^2}{3} + \gamma(1-\gamma) F\left(x\right) - \frac{\gamma(1-\gamma)}{2} F^2\left(x\right) + \gamma F\left(v_i\right) - \frac{\gamma (1-\gamma)}{2} F^2\left(v_i\right) \right). 
\end{align*}}
}
Furthermore,
\begin{enumerate}[(a)]
     \item For any $v_i^\prime\leq \tilde{v}_i\leq v_i$, we have
\begin{align*}  \frac{h(\tilde{v}_i\mid v_i^\prime;n=2,\gamma)}{H(\tilde{v}_i\mid v_i^\prime;n=2,\gamma)} 
\leq \frac{h(\tilde{v}_i\mid \tilde{v}_i;n=2,\gamma)}{H(\tilde{v}_i\mid \tilde{v}_i;n=2,\gamma)} 
\geq 
 \frac{h(\tilde{v}_i\mid v_i;n=2,\gamma)}{H(\tilde{v}_i\mid v_i;n=2,\gamma)}.
\end{align*}
\item $\frac{h(x\mid v_i;n=2,\gamma)}{H(x\mid v_i;n=2,\gamma)}v_i$ is non-decreasing with $v_i \in [x,1]$ for $F(x) = x^c$ ($0< c\leq 1$).
\end{enumerate}

\end{corollary}




\begin{lemma}
\label{lem:marginals_joint_g} (i). The marginal density and CDF of the joint distribution $g(\cdot;n,\gamma)$ in \eqref{eq:joint_dist_g} are 
\begin{align*}
    g^{\mar}(x ;n,\gamma)
 = (1-\gamma)f(x) + (m-n)\C^m_n\gamma \cdot \sum_{i=0}^{n-1}\frac{(-1)^i\C^{n-1}_i}{m-n+i} F^{m-n+i}(x)f(x),
\end{align*}
and 
  \begin{align*}
   G^{\mar}(x; n,\gamma)  = (1-\gamma)F(x) + (m-n)\C^m_n\gamma \cdot \sum_{i=0}^{n-1}\frac{(-1)^i\C^{n-1}_i}{(m-n+i)(m-n+i+1)} F^{m-n+i+1}(x).
 \end{align*}
(ii). The density and CDF of $\max_{i\in\I}V_i$
are  
\begin{align*}
    g^{\lar}(x;n,\gamma) =& n(1-\gamma)^n F^{n-1}(x)f(x)\\
    &
   +(m-n)\C^m_nF^m(x)f(x)\sum_{i=0}^n\C^n_i\frac{(-1)^i}{m-n+i}\left[i (1-\gamma)[\gamma + (1-\gamma)F(x)]^{i-1}-i(1-\gamma)^nF^{i-1}(x) \right],
\end{align*}
and 
 \begin{align*}
  & G^{\lar}(x; n,\gamma) =(1-\gamma)^nF^n(x) + (m-n)\C^m_nF^m(x)\sum_{i=0}^n\C^n_i\frac{(-1)^i}{m-n+i}\left[[\gamma + (1-\gamma)F(x)]^i -(1-\gamma)^nF^i(x)\right].
 \end{align*}
\end{lemma}



%%%%%%%%%%%%%%%%%%%% subsec: a useful indentity
\subsection{A Useful Identity}
\label{app_subec:identity}
The following identity plays a key role in the proofs of \Cref{app_subsec:beliefs}.
\begin{lemma}
\label{lem:useful_identity} Let $v_i = v_{(k)}$ be the $k^{\mathsf{th}}$ smallest element among all elements in the vector $v$. For $j\in \I_{-i}$, $x\in [v_i, 1]$, and positive integer $s$, 
\begin{align*}
    &\int_{[0,v_i]^{k-1}\times[v_i,x]^{n-k}}F^s\left(v_{(j)}\right) \prod_{\ell \in \I_{-i}}f(v_{\ell})dv_{-i}\\
    =&\mathds{1}\{j\leq k-1\}\frac{\C^{k-1}_{k-j}}{\C^{s+k-1}_{k-j}} F^{s+k-1}(v_i)\left(F(x)-F(v_i)\right)^{n-k} \\
    & + \mathds{1}\{k+1\leq j \leq n\} \sum_{r=0}^s\frac{\C_{n-j+1}^{n-k}\C_r^s}{\C_{n-j+1}^{n+r-k}}F^{s+k-r-1}(v_i)(F(x)-F(v_i))^{n+r-k}.
\end{align*}
Specifically, when $k=n$, 
\begin{align*}
\int_{ [0,v_i]^{n-1}}F^s\left(v_{(j)}\right) \prod_{\ell \in \I_{-i}}f(v_{\ell})dv_{-i} = \mathds{1}\{j\leq n-1\}\frac{\C^{n-1}_{n-j}}{\C^{s+n-1}_{n-j}} F^{s+n-1}(v_i).
\end{align*}
\end{lemma}



%%%%%%%% proofs
\section{Proofs}
\label{app_sec:proofs}


\subsection{Proofs for \Cref{sec:beliefs}}
For simplicity, we may drop the dependence on $n$ and $\gamma$ from the definition of belief in the following proofs.

\begin{proof}[Proof of \Cref{thm:beta_posterior}] Let
$\tilde{\I}$ be a random variable presenting the set of admitted players. Observe that the event $\left\{n, \tilde{\I}= \I\right\}$ equals to the event $\left\{\tilde{\I}= \I\right\}$, then by the tower property of conditional expectation and the Bayes' theorem, we have
\begin{align*}
 \beta(v_{-i}\mid n,\tilde{\I}=\I,v_i) &=  \beta(v_{-i}\mid\tilde{\I}= \I,v_i)  = \frac{\Pr\left(\tilde{\I}=\I\mid v_{-i}, v_i\right)\beta\left(v_{-i} \mid v_i\right)}{\int_{[0,1]^{n-1}}\Pr\left(\tilde{\I}=\I\mid v_{-i}, v_i\right)\beta\left(v_{-i} \mid v_i\right)dv_{-i}} \\
 &= \frac{\Pr\left\{
\max_{j\in \barI} \zeta_j \leq \min_{i\in \I}\zeta_i
\mid v\right\}\beta\left(v_{-i} \mid v_i\right)}{\int_{[0,1]^{n-1}}\Pr\left\{
\max_{j\in \barI} \zeta_j \leq \min_{i\in \I}\zeta_i
\mid v\right\}\beta\left(v_{-i} \mid v_i\right)dv_{-i}}\\
&=\underbrace{\kappa(v_i;n,\gamma)}_{\textrm{normalizing term}}
\cdot 
\underbrace{\psi(v;n,\gamma)}_{\textrm{admission probability}}
 \cdot 
 \underbrace{\prod_{j\in \I_{-i}} f(v_j)}_{\textrm{prior}},
\end{align*}
where $\beta\left(v_{-i} \mid v_i\right)$ is the belief of the other $n-1$ players' types condition on player $i$' type is $v_i$ (but not condition on the admission set), $\psi(v;n,\gamma)=  \Pr\left\{
\max_{j\in \barI} \zeta_j \leq \min_{i\in \I}\zeta_i
\mid v\right\}$ and $$\kappa(v_i;n,\gamma) = \int_{[0,1]^{n-1}}\Pr\left\{
\max_{j\in \barI} \zeta_j \leq \min_{i\in \I}\zeta_i
\mid v\right\}\beta\left(v_{-i} \mid v_i\right)dv_{-i}$$ is the normalizing term. Notice that the formula $\beta\left(v_{-i} \mid v_i\right)$ neither conditions on the $\tilde{\I}=\I$ nor conditions on the admitted number $n$
Thus, the admitted player $i$ can not obtain any additional information and $\beta\left(v_{-i} \mid v_i\right)$ is simply the prior $\prod_{j\in \I_{-i}} f(v_j)$.
Therefore, the last equality follows.

Now we show that 
\begin{align}\label{eq:psi_formula}
\psi(v;n,\gamma)=
\int_0^1\prod_{i\in \I}[1-F(x;v_i,\gamma)] dF^{m-n}(x),
\end{align}
where $F(x;v_i,\gamma):=\gamma\cdot \mathds{1}\{v_i\leq x\} + (1-\gamma)\cdot F(x) $ is the CDF of $\zeta_i$. To see this, by Fubini theorem, we have 
\begin{align*}
    \psi(v;n,\gamma) =&  \Pr\left\{
\max_{j\in \barI} \zeta_j \leq \min_{i\in \I}\zeta_i
\mid v\right\} 
=\int_0^1 d\Pr\left\{\max_{j\in \barI} \zeta_j\leq x \mid v\right\} \int_x^1 d\Pr\left\{\min_{i\in \I}\zeta_i
\leq y \mid v\right\}\\
=&\int_0^1
\Pr\left\{\min_{i\in \I}\zeta_i
\geq x \mid v\right\}
d\Pr\left\{\max_{j\in \barI} \zeta_j\leq x\right\},
\end{align*}
$\Pr\left\{\left(\min_{i\in \I}\zeta_i
\mid v\right) \geq x\right\} = \prod_{i\in \I}\Pr\left\{\zeta_i
 \geq x \mid v \right\}= \prod_{i\in \I}[1-F(x;v_i,\gamma)]$, and 
 \begin{align*}
\Pr\left\{\max_{j\in \barI} \zeta_j\leq x\right\} = \prod_{j\in \barI} \Pr\left\{\zeta_j\leq x\right\} =  \prod_{j\in \barI} \int_0^1\Pr\left\{\zeta_j\leq x \mid v_j\right\}dF(v_j) = F^{m-n}(x),   
 \end{align*}
then equation \eqref{eq:psi_formula} follows, as desired.
\end{proof}

%%%%%%%%% proof of properties of the admission probability
\begin{proof}[Proof of \Cref{prop:admissionprob_property}]
From \Cref{lem:closedforms_admission_prob} regarding the expression of the admission probability, we know that the admission probability depends only on the order of the elements in $v$ and is \textit{independent} of the specific dimension. Thus, the admission probability is a symmetric function. Moreover, in any domain $\mathcal{V}$ where the order of elements in $v\in \mathcal{V}$ remains fixed, the expression for the admission probability is invariant with respect to each dimension $i$. This immediately implies differentiability almost everywhere and also \Cref{prop:admissionprob_property} (ii) and (iii).
When $\gamma=1$, by \eqref{eq:gamma_1_admissionprob}, we have $\psi(v;n,\gamma)=F\left(\min_{i\in \I}v_i\right)$. Then, supermodularity and log-supermodularity can be directly verified from this expression. 
\end{proof}



%%%%%%%% proof of joint_dis_g 
\begin{proof}[{Proof of \Cref{lem:joint_dis_g}}]
Suppose there exists a symmetric joint density $g(\cdot;n,\gamma)$ such that $g(v_{-i}\mid v_i;n,\gamma) = \beta(v_{-i}\mid \I, v_i)$. Then by symmetry, all marginal densities have the same formula, denoted as $g^{\mathsf{mar}}(\cdot;n,\gamma)$. Hence by definition,
\begin{align*}
 g(v;n,\gamma) =& g^{\mathsf{mar}}(\cdot;n,\gamma)\cdot \beta(v_{-i}\mid \I, v_i) =  g^{\mathsf{mar}}(v_i;n,\gamma)\cdot
\kappa(v_i;n,\gamma)
 \cdot 
\left(\prod_{k\in \I_{-i}} f(v_k)\right)
 \cdot 
 \phi(v;n,\gamma)\\
 =& g^{\mathsf{mar}}(v_j;n,\gamma)\cdot
\kappa(v_j;n,\gamma)
 \cdot 
\left(\prod_{k\in \I_{-j}} f(v_k)\right)
 \cdot 
 \phi(v;n,\gamma),
\end{align*}
which implies that 
\begin{align*}
\frac{g^{\mathsf{mar}}(v_i;n,\gamma)\cdot
\kappa(v_i;n,\gamma)}{f(v_i)} = \frac{g^{\mathsf{mar}}(v_j;n,\gamma)\cdot
\kappa(v_j;n,\gamma)}{f(v_j)}
\end{align*}
for \textit{any} $v_i,v_j\in[0,1]$.
This indicates that $g^{\mathsf{mar}}(v_i;n,\gamma)\cdot
\kappa(v_i;n,\gamma) = Cf(v_i)$ for some constant $C$. Therefore, if a symmetric joint density $g(\cdot;n,\gamma)$ exists, then $g(\cdot;n,\gamma)$ must have the form $g(v;n,\gamma) = C \cdot \left( \prod_{i\in\I} f(v_i)\right)
 \cdot 
 \phi(v;n,\gamma)$. Such $g(\cdot;n,\gamma)$ is unique since $C$ is the normalizing constant, and
 \begin{align*}
    \frac{1}{C} =& {\int_{v\in [0,1]^n}\left( \prod_{i\in\I} f(v_i)\right)
 \cdot 
 \phi(v;n,\gamma)dv} ={\int_0^1dv_i \int_{[0,1]^{n-1}} \prod_{j\in\I_{-i}} f(v_j)
 \cdot 
\phi(v;n,\gamma)dv_{-i}}\\
=& {\int_0^1 \frac{1}{\kappa(v_i;n,\gamma)}dF(v_i)}
=\int_0^1 \left( \frac{1-\gamma}{\C^m_n} + (m-n)\gamma \cdot \sum_{i=0}^{n-1}\frac{(-1)^i\C^{n-1}_i}{m-n+i} F^{m-n+i}(v_i)\right) dF(v_i)\\
=& \frac{1-\gamma}{\C^m_n} + (m-n)\gamma\int_0^1 \int_0^{F(v_i)}t^{m-n-1}(1-t)^{n-1}dt dF(v_i)\\
=& \frac{1-\gamma}{\C^m_n} + (m-n)\gamma \int_0^1 t^{m-n-1}(1-t)^{n}dt \\
=& \frac{1}{\C^m_n},
 \end{align*}
where the fourth equality is by \Cref{cor:special_gamma} and the penultimate is by Fubini theorem. This completes the proof.
\end{proof}


%%%%%%%%%% proof of the properties of joint density
\begin{proof}[Proof of \Cref{prop:g_property}]
By \Cref{prop:admissionprob_property} (iv), we know that $\psi(v;n,\gamma)$ is affiliated when $\gamma=1$. Combining this with the fact that the product of two affiliated functions is also affiliated (see Theorem 1 in \citet{milgrom_1982_auctiontheory_competitive_bidding}), we conclude that the joint density $g(v;n,\gamma)$ in \eqref{eq:joint_dist_g} is affiliated (noting that the function $\prod_{i\in \I} f(v_i)$ is clearly affiliated in $v \in [0,1]^n$). When the prior is uniform, i.e., $f(v_i)=1$, the joint density $g(v;n,\gamma)$ simplifies to $\C^m_n \psi(v;n,\gamma)$. Thus, any properties of $\psi(v;n,\gamma)$ stated in \Cref{prop:admissionprob_property} also hold for the joint density $g(\cdot;n,\gamma)$.
\end{proof}





%%%%%%%% proof of stochastic_domiance_gamma_1 
\begin{proof}[{Proof of \Cref{prop:stochastic_domiance_gamma_1}}] When $\gamma=1$, by \Cref{prop:g_property} (i), we know that $g(v;n,\gamma) = F^{m-n}(v_{(1)})$ is affiliated in $v\in [0,1]^n$.
Since the conditioning and marginalization preserve affiliation \citep{karlin_1980_MTP2}, then the conditional density $g(v_{-i}\mid v_i;n,\gamma) = \beta^{\mar}(\cdot \mid \I,v_i)$ is also affiliated in $v\in [0,1]^n$. Moreover, the marginal density of $\beta^{\mar}(\cdot \mid \I,v_i)$ is affiliated in $(x,v_i)\in [0,1]^2$. Then, the first $\fosd$ follows by Proposition 3.1 in \citet{castro_2007_affiliation_positive_dependence}. 

We now show the second $\fosd$. It's equivalent to show that 
\begin{align*}
\kappa(v_i;n,\gamma=1)  \int_{ [0,1]^{n-2}\times[0,x]} \psi(v;n,\gamma=1) \prod_{j\in \I_{-i}} f(v_j) dv_{-i} \leq F(x).   
\end{align*}
Firstly, by Fubini theorem,
\begin{align*}
 &\int_{ [0,1]^{n-2}\times[0,x]} \psi(v;n,\gamma) \prod_{j\in \I_{-i}} f(v_j) dv_{-i}\\
 =& \int_{ [0,1]^{n-2}\times [0,x]}  \left(\int_0^1\prod_{i\in \I}\left[1-F(t;v_i,\gamma)\right] dF^{m-n}(t)\right)\prod_{j\in \I_{-i}} f(v_j) dv_{-i}\\
 =& \int_0^1 [1-F(t;v_i,\gamma)]\prod_{k\in \I_{-i}\backslash \{j\}}\left(\int_0^1 [1-F(t;v_k,\gamma)]dF(v_k)\right)\int_0^x [1-F(t;v_j,\gamma)]dF(v_j)dF^{m-n}(t).
\end{align*}
Then by the similar calculations as in the proof of \Cref{lem:closedforms_admission_prob} and (ii) of \Cref{lem:marginals_joint_g}, we have $\int_0^1 [1-F(t;v_j,\gamma)]dF(v_j) = 1-F(t)$, and
\begin{align*}
\int_0^x [1-F(t;v_j,\gamma)]dF(v_j) 
=\mathds{1}\{x\leq t\} (1-\gamma)F(x)(1-F(t)) + \mathds{1}\{x> t\} \left[F(x) - F(t)(\gamma + (1-\gamma)F(x))\right].
\end{align*}
Hence when $\gamma = 1$, 
\begin{align*}
\int_{ [0,1]^{n-2}\times[0,x]} \psi(v;n,\gamma = 1) \prod_{j\in \I_{-i}} f(v_j) dv_{-i} =& \int_0^1 \mathds{1}\{v_i>t\} \mathds{1}\{x>t\}(F(x) - F(t)) (1-F(t))^{n-2}dF^{m-n}(t) \\ 
\leq &F(x)\int_0^1 \mathds{1}\{v_i>t\} (1-F(t))^{n-1}dF^{m-n}(t)\\
=& F(x) \cdot (m-n)\sum_{i=0}^{n-1}\frac{(-1)^iC^{n-1}_i}{m-n+i} F^{m-n+i}(v_i)\\
=&\frac{F(x)}{\kappa(v_i;n,\gamma = 1)},
\end{align*}
where in the first inequality we use $\mathds{1}\{x>t\}(F(x)-F(t))\leq F(x)(1-F(t))$ for any $x,t \in [0,1]$, and in the last equality we use \Cref{cor:special_gamma}.
This completes the proof.
\end{proof}


\subsection{Proofs for \Cref{sec:all-pay_auctions}}

%%%%%%%% proof of kappa_vi 
\begin{proof}[{Proof of \Cref{lem:kappa_vi}}]
(i). By the proof of \Cref{lem:closedforms_admission_prob}, we have 
\begin{align}\label{eq:formula_kappa}
\frac{1}{\kappa(v_i;n,\gamma)} 
=& \frac{1-\gamma}{C^m_n} + \gamma \cdot \int_0^{v_i} (1-F(x))^{n-1}dF^{m-n}(x),
\end{align}
hence $\frac{1}{\kappa(v_i;n,\gamma)}$ is non-decreasing in $v_i\in [0,1]$, which implies that $\kappa(v_i;n,\gamma)$ is non-increasing in $v_i\in [0,1]$.

\medskip

\noindent (ii). By \Cref{eq:formula_kappa}, we can conclude that 
\begin{align*}
 \frac{1}{\kappa(v_i;n,\gamma)} 
\leq& \frac{1-\gamma}{C^m_n} + \gamma \cdot \int_0^{1} (1-F(x))^{n-1}dF^{m-n}(x) =  \frac{1-\gamma}{C^m_n} + \frac{\gamma}{C^{m-1}_{n-1}} = \frac{1-\gamma + \frac{m}{n}\gamma}{C^m_n}, 
\end{align*}
hence $\kappa(v_i;n,\gamma)\geq \frac{\C^m_n}{(1-\gamma) + \gamma \frac{m}{n}} \geq \C^{m-1}_{n-1}$, as desired.
\end{proof}

%%%%%%%%%% proof of thm:strictlymonotone_equilibrium_allpay
\begin{proof}[{Proof of \Cref{thm:strictlymonotone_equilibrium_allpay}}]
The proof includes four steps: i) derive the unique SSM equilibrium $\sigma^{\mathsf{AP}}(v_i;n,\gamma)$ if exists; ii) check the obtained function $\sigma^{\mathsf{AP}}(v_i;n,\gamma)$ is indeed strictly increasing; iii) show that $\sigma^{\mathsf{AP}}(v_i;n,\gamma)$ is indeed an equilibrium under the condition that $v_i h(\tilde{v}_i\mid v_i;n,\gamma) - \tilde{v}_i h(\tilde{v}_i\mid \tilde{v}_i;n,\gamma)$ is non-negative for $\tilde{v}_i\in [0,v_i]$ and non-positive for $\tilde{v}_i\in [v_i,1]$ for any given $v_i\in [0,1]$; iv) show that if the inflated type
$\theta\left(v_i; n, \gamma\right)$ is non-decreasing, the condition in iii) holds.

 \medskip

\noindent\textit{\underline{Step 1: SSM equilibrium if exists.}}

 \medskip

\noindent When all $n-1$ other admitted players follow an SSM equilibrium strategy $\sigma^{\mathsf{AP}}(\cdot;n,\gamma)$ and player $i$ bids $ b_i =  \sigma^{\mathsf{AP}}(\tilde{v}_i;n,\gamma)$ for some $\tilde{v}_i\in[0,1]$, the objective of player $i \in \mathcal{I}$ is to
\begin{align}\label{eq:utility function all pay}
    \argmax_{b_i\in \mathbb{R}_+}~  \Pr\left\{\sigma^{\mathsf{AP}}({V}_j;n,\gamma)\leq b_i,\forall j\in \I_{-i}\right\} \cdot v_i - b_i,
\end{align}
where $\Pr\left\{\sigma^{\mathsf{AP}}({V}_j;n,\gamma)\leq b_i,\forall j\in \I_{-i}\right\}$ is the winning probability of player $i$.  Observe that 
\begin{align*}
    \Pr\left\{\sigma^{\mathsf{AP}}({V}_j;n,\gamma)\leq b_i,\forall j\in \I_{-i}\right\} & = \Pr\left(\text{$b_i$ ranks  highest in $\I_{-i}$} \right) \nonumber = H(\tilde{v}_i\mid v_i;n,\gamma),
\end{align*}
where the last equality holds because $H(\cdot \mid v_i;n,\gamma)$ is the CDF of $\max_{j\in \I_{-i}}V_j$ by the definition \eqref{eq:H_CDF} and $\sigma^{\mathsf{AP}}(\cdot;n,\gamma)$ is strictly increasing. Hence Equation (\ref{eq:utility function all pay}) is equivalent to 
\begin{equation}
\label{eq:utility function all pay used}
\max_{\tilde{v}_i}~U^{\mathsf{AP}}(\tilde{v}_i,v_i;\sigma^{\mathsf{AP}}(\cdot;n,\gamma)):= v_i H(\tilde{v}_i\mid v_i; n,\gamma) - \sigma^{\mathsf{AP}}(\tilde{v}_i;n,\gamma).
\end{equation}
If there exists a unique SSM equilibrium strategy, by the first order condition, $\sigma^{\mathsf{AP}}(\cdot;n,\gamma)$ must satisfy that\
\begin{align*}
    \frac{\partial U^{\mathsf{AP}}(\tilde{v}_i,v_i;\sigma^{\mathsf{AP}}(\cdot;n,\gamma))}{\partial \tilde{v}_i}\bigg |_{\tilde{v}_i = v_i} = \left(v_ih(\tilde{v}_i\mid v_i; n,\gamma) - \frac{d \sigma^{\mathsf{AP}}(\tilde{v}_i;n,\gamma)}{d\tilde{v}_i}\right)\bigg |_{\tilde{v}_i = v_i} = 0,
\end{align*}
which implies that $\sigma^{\mathsf{AP}}(v_i;n,\gamma)
  =\int_0^{v_i} x h(x\mid x;n,\gamma)dx$.

 \medskip 

\noindent\textit{\underline{Step 2: $\sigma^{\mathsf{AP}}(\cdot;n, \gamma)$ is indeed strictly increasing.}}

 \medskip
\noindent By part (i) of \Cref{lem:property_h}, we have  
$\frac{d \sigma^{\mathsf{AP}}({v}_i;n,\gamma)}{d{v}_i} = v_i h(v_i\mid v_i; n,\gamma)>0$ for $v_i \in \left(0,1\right]$, hence $\sigma^{\mathsf{AP}}(v_i;n, \gamma)$ is strictly increasing with $v_i$.

 \medskip
\noindent\textit{\underline{Step 3: $\sigma^{\mathsf{AP}}(v_i;n, \gamma)$ is indeed an equilibrium under the condition that $v_i h(\tilde{v}_i\mid v_i;n,\gamma) - \tilde{v}_i h(\tilde{v}_i\mid \tilde{v}_i;n,\gamma)$}}

\noindent\textit{\underline{is non-negative for $\tilde{v}_i\in [0,v_i]$ and non-positive for $\tilde{v}_i\in [v_i,1]$ for any given $v_i\in [0,1]$.}}
\medskip

\noindent To show this, it's equivalent to show that
\begin{align*}
    U^{\mathsf{AP}}(\tilde{v}_i,v_i;\sigma^{\mathsf{AP}}(\cdot;n,\gamma))= v_i H(\tilde{v}_i\mid v_i; n, \gamma) - \sigma^{\mathsf{AP}}(\tilde{v}_i;n, \gamma) = v_i H(\tilde{v}_i\mid v_i; n, \gamma) - \int_0^{\tilde{v}_i} x h(x\mid x;n, \gamma)dx.
\end{align*}
indeed obtains the maximum at $\tilde{v}_i = v_i$. It's sufficient to show that $U^{\mathsf{AP}}(\tilde{v}_i,v_i;\sigma^{\mathsf{AP}}(\cdot;n,\gamma))$ non-decreases with $\tilde{v}_i \in [0, v_i]$, and non-increases with $\tilde{v}_i \in [v_i, 1]$.

Observe that
$$\frac{\partial U^{\mathsf{AP}}(\tilde{v}_i,v_i;\sigma^{\mathsf{AP}}(\cdot;n,\gamma))}{\partial \tilde{v}_i} = v_ih(\tilde{v}_i\mid {v}_i;n, \gamma) -  \tilde{v}_ih(\tilde{v}_i \mid \tilde{v}_i;n, \gamma),$$ 
hence if $v_i h(\tilde{v}_i\mid v_i;n,\gamma) - \tilde{v}_i h(\tilde{v}_i\mid \tilde{v}_i;n,\gamma)$ is non-negative for $\tilde{v}_i\in [0,v_i]$ and non-positive for $\tilde{v}_i\in [v_i,1]$ for any given $v_i\in [0,1]$, we have $\frac{\partial U^{\mathsf{AP}}(\tilde{v}_i,v_i;\sigma^{\mathsf{AP}}(\cdot;n,\gamma))}{\partial \tilde{v}_i} \geq 0$ for $\tilde{v}_i 
\in [0, v_i]$, and $\frac{U^{\mathsf{AP}}(\tilde{v}_i,v_i;\sigma^{\mathsf{AP}}(\cdot;n,\gamma))}{\partial \tilde{v}_i} \leq 0$ for $\tilde{v}_i \in  [v_i,1]$. Therefore, $U^{\mathsf{AP}}(\tilde{v}_i,v_i;\sigma^{\mathsf{AP}}(\cdot;n,\gamma))$ non-decreases with $\tilde{v}_i \in [0, v_i]$, and non-increases with $\tilde{v}_i \in [v_i, 1]$, as desired.

\medskip 

\noindent\textit{\underline{Step 4: The condition in Step 3 holds if the inflated type
$\theta\left(v_i; n, \gamma\right)$ is non-decreasing.}}

 \medskip

\noindent If $\theta\left(v_i; n, \gamma\right)$ is non-decreasing, combining with \Cref{lem:property_h}, we have $$v_ih(\tilde{v}_i\mid {v}_i;n, \gamma) = v_i\kappa(v_i;n,\gamma)
 \cdot \omega\left(\tilde{v}_i\mid v_i; n, \gamma\right)= \theta\left(v_i; n, \gamma\right)\cdot \omega\left(\tilde{v}_i\mid v_i; n, \gamma\right)$$
is non-decreasing with $v_i$ for any $\tilde{v}_i \in [0,1]$, which implies that the condition in Step 3 holds.
\end{proof}


%%%%%%%%% proof of prop:SSM_gamma_1_allpay

\begin{proof}[{Proof of \Cref{prop:SSM_gamma_1_allpay}}] (i). By (ii) in \Cref{thm:strictlymonotone_equilibrium_allpay}, we know that if the inflated type $\theta(v_i;n,\gamma = 1)$ is non-decreasing with $v_i \in [0,1]$, then there exists an SSM equilibrium strategy. Now we show the other direction, that is, if there exists an SSM equilibrium, $\theta(v_i;n,\gamma = 1)$ must be non-decreasing.

According to (iii) in \Cref{thm:strictlymonotone_equilibrium_allpay}, if there exists an SSM equilibrium, the equilibrium must be $\sigma^{\mathsf{AP}}(v_i;n,\gamma = 1)
  =\int_0^{v_i} x h(x\mid x;n,\gamma = 1)dx$. Hence we need to show that $\sigma^{\mathsf{AP}}(v_i;n,\gamma = 1)$ is the equilibrium implying $\theta(v_i;n,\gamma = 1)$ is non-decreasing. We derive this by making a contradiction.

\noindent If $\theta(v_i;n,\gamma = 1)$ is not weakly increasing, by the continuity, there exists some interval $[\underline{v}, \overline{v}]$ on which $\theta(v_i;n,\gamma = 1)$ is strictly decreasing. Pick $v_i \in (\underline{v}, \overline{v})$, it's sufficient to show that 
\begin{align}\label{eq:non_optimity_condition}
    {\argmax}_{\tilde{v}_i}~\frac{\partial U^{\mathsf{AP}}(\tilde{v}_i,v_i;\sigma^{\mathsf{AP}}(\cdot;n,\gamma))}{\partial \tilde{v}_i} \neq v_i.
\end{align}
When $\tilde{v}_i \leq v_i$, by  \Cref{cor:special_case_H_h}, we have $H(\tilde{v}_i \mid v_i;n,\gamma = 1) = \frac{\kappa(v_i; n, \gamma = 1)}{\C^{m-1}_{n-1}}F^{m-1}(\tilde{v}_i)$ and $h(\tilde{v}_i \mid v_i;n,\gamma = 1) = \frac{m-1}{\C^{m-1}_{n-1}}\kappa(v_i; n, \gamma = 1)F^{m-2}(\tilde{v}_i)f(\tilde{v}_i)$. Therefore, by combining the fact $\theta(v_i;n,\gamma = 1)$ is strictly decreasing on $[\underline{v}, \overline{v}]$, and Lagrangian's mean value theorem, there exists $\xi \in (\underline{v}, v_i)$ such that  
\begin{align*}
    &U^{\mathsf{AP}}(\underline{v},v_i;\sigma^{\mathsf{AP}}(\cdot;n,\gamma)) - U^{\mathsf{AP}}({v}_i,v_i;\sigma^{\mathsf{AP}}(\cdot;n,\gamma))\\
    =& \frac{\partial U^{\mathsf{AP}}(\tilde{v}_i,v_i;\sigma^{\mathsf{AP}}(\cdot;n,\gamma))}{\partial \tilde{v}_i}\bigg|_{\tilde{v}_i = \xi} (\underline{v} - v_i)
    =\left(v_ih(\xi\mid {v}_i;n, \gamma) -  \xi h(\xi \mid \xi;n, \gamma)\right)(\underline{v} - v_i)\\
    =& \left(\theta(v_i;n,\gamma = 1) - \theta(\xi;n,\gamma = 1)\right)(\underline{v} - v_i)\frac{m-1}{\C^{m-1}_{n-1}}F^{m-2}(\xi)f(\xi)>0,
\end{align*}
which implies \eqref{eq:non_optimity_condition}, as desired.

(ii). Since $h(x\mid x;n,\gamma = 1) = \frac{m-1}{\C^{m-1}_{n-1}}\kappa(x; n, \gamma = 1)F^{m-2}(x)f(x)$, we have 
\begin{align*}
    \sigma^{\mathsf{AP}}(v_i;n,\gamma = 1)
  =&\int_0^{v_i} x h(x\mid x;n,\gamma = 1)dx 
  = \int_0^{v_i} x \frac{\kappa(x; n, \gamma = 1)}{\C^{m-1}_{n-1}}dF^{m-1}(x) \\=& \int_0^{v_i} x \frac{1}{J(F(x), n, m)}dF^{m-1}(x), 
\end{align*}
where $J(F(x), n, m) = \frac{\C^{m-1}_{n-1}}{\kappa(x; n, \gamma = 1)}$.

(a). By \Cref{cor:special_gamma}, we know that 
\begin{align*}
 \frac{1}{\kappa(x; n, \gamma = 1)} =& (n-1)\int_0^{F(x)}t^{m-n}(1-t)^{n-2}dt + F^{m-n}(x)(1-F(x))^{n-1},  
\end{align*}
hence 
\begin{align*}
    J(x, m, m) = \C^{m-1}_{m-1}\left((m-1)\int_0^{x}t^{m-m}(1-t)^{m-2}dt + x^{m-m}(1-x)^{m-1} \right) = 1,
\end{align*}
and we have
\begin{align*}
 J(x, n, m) = & \mathds{1}\{n\leq m-1\}(m-n)\C^{m-1}_{n-1}\int_0^{F(x)}t^{m-n-1}(1-t)^{n-1}dt + \mathds{1}\{n = m\}.   
\end{align*}
Therefore, when $m=n$, $J(x, m, m) = 1$, and when $n\in \{2,\cdots, m-1\}$, 
\begin{align*}
J(x, n, m) = (m-n)\C^{m-1}_{n-1} \int_0^{x}t^{m-n-1}(1-t)^{n-1}dt \leq (m-n)\C^{m-1}_{n-1} \int_0^{1}t^{m-n-1}(1-t)^{n-1}dt = 1,    
\end{align*}
as desired.

(b). Since $J'(x, n, m) = \mathds{1}\{n\leq m-1\}(m-n)\C^{m-1}_{n-1} x^{m-n-1}(1-x)^{n-1} \geq 0$, we conclude that $J(x, n, m)$ is non-decreasing with $x\in[0,1]$.

(c). The key idea is to evaluate $J(x,n, m)-J(x,n-1, m)$. Observe that for $3\leq n\leq m$, 
\begin{align}
     J(x,n-1, m) &= \C^{m-1}_{n-2} (m-n+1)\int_0^{x}t^{m-n}(1-t)^{n-2}dt =\C^{m-1}_{n-1}(n-1)\int_0^{x}t^{m-n}(1-t)^{n-2}dt. \nonumber 
\end{align}
Hence $\forall x \in [0,1]$,
\begin{align}
     J(x,n-1, m)- J(x,n, m)&=-\C^{m-1}_{n-1}x^{m-n}(1-x)^{n-1}\leq 0\nonumber, 
\end{align}
and the inequality is strict when $x\in (0,1)$. This implies that $J(x,n, m)$ is weakly increasing with $n\in \{2, 3, \cdots, m\}$ for $x=0,1$, and strictly increasing for $x\in (0,1)$.

\end{proof}
%%%%%%%%% proof of thm:opt_admittednumber_allpay
\begin{proof}[{Proof of \Cref{thm:opt_admittednumber_allpay}}]

\underline{(i) \textit{We first analyze the case $\gamma = 1$.}}

\medskip

\noindent\textit{Part 1: Analysis of expected revenue.} By \Cref{lem:marginals_joint_g}, we know that  $$G^{\mar}(\cdot; n,\gamma=1) = \C_n^m \cdot \int_0^{F(x)}\kappa(t;n,\gamma=1)dt.$$ Then the expected revenue is
\begin{align*}
 \R^{\AP}(n;\gamma=1,F) & = n\cdot \mathbb{E}_{V_i\sim G^{\mar}(\cdot; n,\gamma=1)}[\sigmaAP(V_i;n,\gamma=1)]\\
 &= n\int_0^1 \int_0^{v_i} \frac{x}{J(F(x),n,m)}\, dF^{m-1}(x)\, dG^{\mar}(v_i;n,\gamma=1)\\[1mm]
 &= \int_0^1 \frac{n}{J(F(x),n,m)} \cdot \left(1-G^{\mar}(x; n,\gamma=1)\right) \cdot x\, dF^{m-1}(x),
\end{align*}
where the last equality is obtained by the Fubini theorem.
Now we are apt to show $\frac{n}{J(F(x),n,m)}\cdot \left(1-G^{\mar}(x;n,\gamma=1)\right)$ is strictly increasing in $n\in [2, m]$ for any $x\in (0,1)$, hence the unique optimal solution is $n^*=2$. Observe that 
\begin{align*}
\frac{n}{J(F(x),n,m)}\cdot (1-G^{\mar}(x;n,\gamma=1))  
& =  \frac{n}{J(F(x),n,m)}\cdot \left[1- \C_n^m \cdot \int_0^{F(x)}\kappa(t;n,\gamma=1)dt\right]\\
& = \frac{n}{J(F(x),n,m)}\cdot \left[1- \frac{m}{n} \cdot \int_0^{F(x)}J(t,n,m)dt\right]\\
& = \frac{n}{J(F(x),n,m)} - \frac{m}{J(F(x),n,m)}\cdot \int_0^{F(x)} J(t,n,m)dt,
\end{align*}
and 
\begin{align*}
\int_0^x J(t,n,m)dt =& (m-n)\C_{n-1}^{m-1}  \cdot\int_{0}^xdt\int_0^t u^{m-n-1}(1-u)^{n-1}du \\
=& (m-n)\C_{n-1}^{m-1} \cdot\int_{0}^xdu\int_u^x u^{m-n-1}(1-u)^{n-1}dt\\
=& (m-n)\C_{n-1}^{m-1} \cdot \left(x\int_0^xu^{m-n-1}(1-u)^{n-1}du - \int_0^xu^{m-n}(1-u)^{n-1}du\right)\\
=& xJ(x,n,m) - \frac{m-n}{m}\cdot J(x,n,m+1),
\end{align*}
hence 
\begin{align*}
    \frac{n}{J(F(x),n,m)}\cdot (1-G^{\mar}(x;n,\gamma=1))   = 
 \frac{n + (m-n)J(F(x),n,m+1)}{J(F(x),n,m)} -mF(x).
\end{align*}
Now for $n\in [3,m]$, it's sufficient to show 
\begin{align*}
   \frac{n-1 + (m-n +1)J(x,n-1,m+1)}{J(x,n-1,m)} \geq  \frac{n + (m-n)J(x,n,m+1)}{J(x,n,m)} \Longleftrightarrow a(x)\geq0, \quad \forall x\in[0,1],
\end{align*}
where $a(x) = J(x,n,m) \left(n-1 + (m-n +1)J(x,n-1,m+1) \right) - J(x,n-1,m) \left(n + (m-n )J(x,n,m+1) \right)$, and the inequality is strict for $x\in(0,1)$. To see this, we analyze $a'(x)$.
\begin{align*}
 a'(x) = & J'(x,n,m) \left(n-1 + (m-n +1)J(x,n-1,m+1) \right)  + (m-n+1)J(x,n,m)J'(x,n-1,m+1) \\
 & - J'(x,n-1,m) \left(n + (m-n )J(x,n,m+1)\right) - (m-n)J(x,n-1,m)J'(x,n,m+1)\\
 =& (n-1)\C_{n-1}^{m-1} x^{m-n-1}(1-x)^{n-2}a_1(x),
\end{align*}
where 
\begin{align*}
a_1(x) =& m-n + mx^2J(x,n,m) + \frac{(m-n+1)(m-n)}{n-1}(1-x) J(x,n-1,m+1) \\
&-  \frac{m(m-n)}{n-1}x(1-x) J(x,n-1,m) - (m-n)x J(x,n,m+1) -mx.
\end{align*}
By calculation, we have 
\begin{align*}
    a'_1(x) =& 2mxJ(x,n,m) - (m-n) J(x,n,m+1) - \frac{m(m-n)}{n-1}(1-2x)J(x,n-1,m)\\
    &- \frac{(m-n)(m-n+1)}{n-1}J(x,n-1,m+1) - m,
\end{align*}
and $a''_1(x) = 2mJ(x,n,m) + \frac{2m(m-n)}{n-1}J(x,n-1,m)\geq 0.$ Hence $a'_1(x) $ is nondecreasing in $x\in[0,1]$. Since $a'_1(0) = -\frac{m(m-n)}{n-1} - m<0$ and $a'_1(1) = 2n>0$, we conclude that $a_1(x)$ first strictly decreases and then strictly increases. Combining the fact $a_1(0) = m-n + \frac{(m-n+1)(m-n)}{n-1}>0$ and $a_1(1) = 0$, we have $a(x)$ first strictly increases and then strictly decreases, which implies that $a(x)> a(0)=a(1)= 0$ for $x\in (0,1)$, as desired.


\medskip

\noindent\textit{Part 2: Analysis of the expected highest bid.} By \Cref{lem:marginals_joint_g}, we know that  $G^{\lar}(\cdot; n,\gamma=1) =F^m(\cdot).$ Then the expected highest bid is

\begin{align*}
 \HB^{\AP}(n;\gamma=1,F) & = \mathbb{E}_{V_i\sim G^{\lar}(\cdot; n,\gamma=1)}[\sigmaAP(V_i;n,\gamma=1)]\\
 &= \int_0^1 \int_0^{v_i} \frac{x}{J(F(x),n,m)}\, dF^{m-1}(x)\, dG^{\lar}(v_i;n,\gamma=1)\\[1mm]
 &= \int_0^1 \int_0^{v_i} \frac{x}{J(F(x),n,m)}\, dF^{m-1}(x)\, dF^{m}(v_i).
\end{align*}
By (c) of \Cref{prop:SSM_gamma_1_allpay}, we know that $J(F(x),n,m)$ is strictly increasing for $n\in \{2,3,\cdots,m\}$ when $x\in (0,1)$, and weakly increasing for $n\in \{2,3,\cdots,m\}$ when $x=0,1$. Hence we conclude that $\HB^{\AP}(n;\gamma=1,F)$ is strictly increasing with $n$.

\medskip
\noindent 
\underline{(ii) \textit{We now consider the case $\gamma = 0$.}} By \Cref{lem:H_h} and \Cref{lem:marginals_joint_g}, we know that $h(x\mid x;n,\gamma=0) = (n-1)F^{n-2}(x)f(x)$, $G^{\mar}(\cdot; n,\gamma=0) = F(x)$ and $G^{\lar}(\cdot; n,\gamma=0) = F^n(x)$.

\medskip

\noindent (a). The expected revenue is
\begin{align*}
 \R^{\AP}(n;\gamma=0,F) & = n\cdot \mathbb{E}_{V_i\sim G^{\mar}(\cdot; n,\gamma=0)}[\sigmaAP(V_i;n,\gamma=0)]
 = n\int_0^1 \int_0^{v_i} x\, dF^{n-1}(x)\, dG^{\mar}(v_i;n,\gamma = 0)\\[1mm]
 &= \int_0^1 nx \left(1-G^{\mar}(x; n,\gamma)\right)\, dF^{n-1}(x) = \int_0^1 nx \left(1-F(x)\right)\, dF^{n-1}(x)\\
 & = n \int_0^1 xdF^{n-1}(x) - (n-1) \int_0^1 xdF^{n}(x)\\
 & = (n-1)\int_0^1 F^{n}(x)dx - n\int_0^1 F^{n-1}(x)dx + 1,
\end{align*}
where the third equality is obtained by the Fubini theorem.
Hence 
\begin{align*}
 \R^{\AP}(n+1;\gamma=0,F) - \R^{\AP}(n;\gamma=0,F) = n\int_0^1 F^{n-1}(x) (F(x) - 1)^2  dx \geq 0,
\end{align*}
which implies that $\R^{\AP}(n;\gamma=0,F)$ is increasing with $n$. Therefore, $n^* = m$.

\medskip

\noindent (b). The expected highest bid is
\begin{align*}
 \HB^{\AP}(n;\gamma=0,F) & = \mathbb{E}_{V_i\sim G^{\lar}(\cdot; n,\gamma=0)}[\sigmaAP(V_i;n,\gamma=0)]
 = \int_0^1 \int_0^{v_i} x\, dF^{n-1}(x)\, dG^{\lar}(v_i;n,\gamma = 0)\\[1mm]
 &= \int_0^1 x \left(1-G^{\lar}(x; n,\gamma)\right)\, dF^{n-1}(x) = \int_0^1 x \left(1-F^n(x)\right)\, dF^{n-1}(x)\\
 & = \int_0^1 xdF^{n-1}(x) - \frac{(n-1)}{2n-1} \int_0^1 xdF^{2n-1}(x)\\
 & = \frac{(n-1)}{2n-1}\int_0^1 F^{2n-1}(x)dx - \int_0^1 F^{n-1}(x)dx + \frac{n}{2n-1},
\end{align*}
where the third equality is obtained by the Fubini theorem. When $F(x) = x^c$, 
\begin{align*}
    \HB^{\AP}(n;\gamma=0, F=x^c) = &\frac{n-1}{(2n-1)((2n-1)c+1)} - \frac{1}{(n-1)c+1} + \frac{n}{2n-1} \\
    =& \frac{n(n-1)c^2}{((2n-1)c+1)((n-1)c+1)}.
\end{align*}
Hence 
\begin{align*}
    \frac{ \HB^{\AP}(n+1;\gamma=0, F=x^c)}{ \HB^{\AP}(n;\gamma=0, F=x^c)} = \frac{(n+1)((2n-1)c+1)((n-1)c+1)}{(n-1)(nc+1)((2n+1)c+1)},
\end{align*}
and 
\begin{align*}
 \frac{ \HB^{\AP}(n+1;\gamma=0, F=x^c)}{ \HB^{\AP}(n;\gamma=0, F=x^c)} \geq 1 \Longleftrightarrow -(n-1)c^2 + (3n-1)c + 2 \geq 0 \Longleftrightarrow (3c - c^2)n + c^2-c+2\geq 0.
\end{align*}
Now we talk about the range of $c$.
\begin{enumerate}[(a)]
    \item For $0<c\leq 3$, we have $(3c - c^2)n + c^2-c+2\geq 0 $, hence $\HB^{\AP}(n;\gamma=0, F=x^c)$ is non-decreasing with $n$, which implies that $n^* = m$.
    \item For $3 \leq c \leq \frac{5+\sqrt{33}}{2}$, we have $(3c - c^2)n + c^2-c+2\geq 0$ when $2 \leq n \leq \frac{c^2-c+2}{c^2-3c}$, and $(3c - c^2)n + c^2-c+2\leq 0$ when $ n \geq \frac{c^2-c+2}{c^2-3c}$, hence $n^* = m \wedge \bigg\lceil{\frac{2(c-1)^2}{c^2-3c}}\bigg\rceil$ or $n^* = m \wedge \bigg\lfloor{\frac{2(c-1)^2}{c^2-3c}}\bigg\rfloor$.
    \item For $c \geq \frac{5+\sqrt{33}}{2}$, we have $(3c - c^2)n + c^2-c+2\leq 0$ when $n\geq 2$, hence $n^*=2$.
\end{enumerate}



\medskip

\noindent
\underline{(iii) \textit{We now turn to the case of $m=3$.}}
Notice that when $m=3$:
\begin{align*}
   & h\left(x\mid x;n = 2,\gamma\right)  = \frac{\frac{(1-\gamma)^2}{3} + \gamma(2-\gamma) F(x) - \gamma(1-\gamma)F^2(x)}{\frac{1-\gamma}{3} + \gamma F(x) - \frac{\gamma}{2}F^2(x)}f(x)\\
    & G^{\mathsf{mar}}(v_i;n=2,\gamma) = (1-\gamma) F(v_i) + \frac{3}{2} \gamma F^2(v_i)  - \frac{\gamma}{2}F^3(v_i),
\end{align*}
and 
\begin{align*}
 G^{\mathsf{largest}}(v_i;n=2,\gamma)=
 (1-\gamma)^2 F^2(t) + \gamma(3-2\gamma)\cdot F^3(t)
 -\gamma(1-\gamma)\cdot F^4(t).   
\end{align*}
For the prior distribution $F(x)=x^c$, to guarantee the existence of SSM equilibrium strategy for all $\gamma\in [0,1]$, it can be shown that this is equivalent to $c\leq 1$.
Then, 
\Cref{thm:opt_admittednumber_allpay} (iii) can be proven by analyzing the expected revenue and expected highest bid based on the above closed-form expression.
See \Cref{fig:m_3_allpay} for illustrations.
\end{proof}



\begin{proof}[Proof of \Cref{remark:generalsetting_allpay}]
    \begin{align*}
  &\mathbb{E}_{V_i\sim G^{\mar}}[\sigmaAP(v_i;n,\gamma)]
 = \frac{1}{\kappa(v_i;n,\gamma)}\int_0^1 x h(x\mid x;n,\gamma)  (1-G^{\mar}(x))dx\\
  &=
  \int_0^1 x 
\left(\int_{[0,x]^{n-1}}\psi(t;n,\gamma)\prod_{j\in \I_{-1}}f(t_j) dt_{-1}\right)^\prime
  \left (1-
  \int_{[0,1]^{[n-1]}}\C^m_n\cdot \psi(v;n,\gamma)\cdot \prod_{i\in \I}f(v_i)dv_{-1} 
  \right)dx. 
\end{align*}
The highest integral dimensional is $(1+1+n-2)\cdot (1+n-1) = n^2$.
For $n=m-1$, this involves $(m-1)^2$ dimensional integrals. 
\end{proof}

\subsection{Proofs for \Cref{sec:firstprice_auctions}}
\begin{proof}[Proof of \Cref{thm:SSM_firstprice}]
The proof includes five steps: i) derive the unique SSM equilibrium $\sigma^{\mathsf{FP}}(v_i;n,\gamma)$ if exists; ii) check the obtained function $\sigma^{\mathsf{FP}}(v_i;n,\gamma)$ is indeed strictly increasing; iii) show that $\sigma^{\mathsf{FP}}(v_i;n,\gamma)$ is indeed an equilibrium under the condition that $\FP(\tilde{v}_i,v_i;n,\gamma)$ is non-negative for $\tilde{v}_i\in [0,v_i]$ and non-positive for $\tilde{v}_i\in [v_i,1]$ for any $v_i\in [0,1]$; iv) show that when $\gamma = 1$, the condition in iii) holds, and $\sigma^{\FP}(v_i;n,\gamma=1) = v_i - \int_0^{v_i} \frac{F^{m-1}(x)}{F^{m-1}(v_i)}  dx,~\forall v_i\in [0,1]$; v) show that when $m=3$ and $c\in (0,1]$, the condition in iii) holds for the power law prior distribution $F(x)=x^c$ and  $n\in \{2,m\}$.

 \medskip

\noindent\textit{\underline{Step 1: SSM equilibrium if exists.}}

 \medskip

\noindent When all $n-1$ other admitted players follow an SSM equilibrium strategy $\sigma^{\mathsf{FP}}(\cdot;n,\gamma)$ and player $i$ bids $ b_i =  \sigma^{\mathsf{AP}}(\tilde{v}_i;n,\gamma)$ for some $\tilde{v}_i\in[0,1]$, the objective of player $i \in \mathcal{I}$ is to
\begin{align}\label{eq:utility function_first_price}
    \argmax_{b_i\in \mathbb{R}_+}~  \Pr\left\{\sigma^{\mathsf{FP}}({V}_j;n,\gamma)\leq b_i,\forall j\in \I_{-i}\right\} \cdot [v_i - b_i],
\end{align}
where $\Pr\left\{\sigma^{\mathsf{FP}}({V}_j;n,\gamma)\leq b_i,\forall j\in \I_{-i}\right\}$ is the winning probability of player $i$.  Observe that 
\begin{align*}
    \Pr\left\{\sigma^{\mathsf{FP}}({V}_j;n,\gamma)\leq b_i,\forall j\in \I_{-i}\right\} & = \Pr\left(\text{$b_i$ ranks  highest in $\I_{-i}$} \right) \nonumber = H(\tilde{v}_i\mid v_i;n,\gamma),
\end{align*}
where the last equality holds because $H(\cdot \mid v_i;n,\gamma)$ is the CDF of $\max_{j\in \I_{-i}}V_j$ by the definition \eqref{eq:H_CDF} and $\sigma^{\mathsf{FP}}(\cdot;n,\gamma)$ is strictly increasing. Hence Equation (\ref{eq:utility function_first_price}) is equivalent to 
\begin{equation*}
\max_{\tilde{v}_i}~U^{\mathsf{FP}}(\tilde{v}_i,v_i;\sigma^{\mathsf{FP}}(\cdot;n,\gamma)):=  H(\tilde{v}_i\mid v_i; n,\gamma) \cdot \left[ v_i- \sigma^{\mathsf{FP}}(\tilde{v}_i;n,\gamma)\right].
\end{equation*}
If there exists a unique SSM equilibrium strategy, by the first order condition, $\sigma^{\mathsf{FP}}(\cdot;n,\gamma)$ must satisfy that\
\begin{align*}
    &\frac{\partial U^{\mathsf{FP}}(\tilde{v}_i,v_i;\sigma^{\mathsf{AP}}(\cdot;n,\gamma))}{\partial \tilde{v}_i}\bigg |_{\tilde{v}_i = v_i}\\
    =& \left(v_ih(\tilde{v}_i\mid v_i; n,\gamma) - H(\tilde{v}_i\mid v_i; n,\gamma) \frac{d \sigma^{\mathsf{FP}}(\tilde{v}_i;n,\gamma)}{d\tilde{v}_i} - h(\tilde{v}_i\mid v_i; n,\gamma)\sigma^{\mathsf{FP}}(\tilde{v}_i;n,\gamma)\right)\bigg |_{\tilde{v}_i = v_i} = 0\\
    \Longleftrightarrow\quad & \frac{d \sigma^{\mathsf{FP}}(\tilde{v}_i;n,\gamma)}{d\tilde{v}_i} + \frac{h({v}_i\mid v_i; n,\gamma)}{H({v}_i\mid v_i; n,\gamma)}\sigma^{\mathsf{FP}}(\tilde{v}_i;n,\gamma)-v_i\frac{h(t\mid t;n,\gamma)}{H(t\mid t;n,\gamma)} = 0.
\end{align*}
This is a first-order linear ordinary differential equation, the solution is given by
\begin{align*}
\sigma^{\mathsf{FP}}(v_i;n,\gamma) =&  \exp\left(-\int_{0}^{v_i} \frac{h(t\mid t;n,\gamma)}{H(t\mid t;n,\gamma)}dt\right) \int_{0}^{v_i}x \frac{h(x\mid x;n,\gamma)}{H(x\mid x;n,\gamma)}\exp\left(\int_{0}^{x} \frac{h(t\mid t;n,\gamma)}{H(t\mid t;n,\gamma)}dt\right) dx \\
=& \int_{0}^{v_i}x \frac{h(x\mid x;n,\gamma)}{H(x\mid x;n,\gamma)} \exp\left(-\int_{x}^{v_i} \frac{h(t\mid t;n,\gamma)}{H(t\mid t;n,\gamma)}dt\right) dx
= \int_{0}^{v_i}x  dL(x\mid v_i;n,\gamma) \\
=& v_i L(v_i\mid v_i;n,\gamma) - \int_{0}^{v_i}L(x\mid v_i;n,\gamma) dx\\
=& v_i - \int_{0}^{v_i}L(x\mid v_i;n,\gamma) dx.
\end{align*}

 \medskip

\noindent\textit{\underline{Step 2: $\sigma^{\mathsf{FP}}(v_i;n,\gamma)$ is indeed strictly increasing.}}

 \medskip

\noindent This can be directly checked by observing that 
\begin{align*}
    \frac{d \sigma^{\mathsf{FP}}({v}_i;n,\gamma)}{d{v}_i} = v_i \frac{h(v_i\mid v_i;n,\gamma)}{H(v_i\mid v_i;n,\gamma)} >0 
\end{align*}
when $v_i\in (0,1]$.

 \medskip

\noindent\textit{\underline{Step 3: $\sigma^{\mathsf{FP}}(v_i;n,\gamma)$ is indeed an equilibrium under the condition that $\FP(\tilde{v}_i,v_i;n,\gamma)$ is non-negative}}
\\
\noindent\textit{\underline{for $\tilde{v}_i\in [0,v_i]$ and non-positive for $\tilde{v}_i\in [v_i,1]$ for any $v_i\in [0,1]$.}}

\medskip

\noindent To show this, it's equivalent to show that
\begin{align*}
    U^{\mathsf{FP}}(\tilde{v}_i,v_i;\sigma^{\mathsf{FP}}(\cdot;n,\gamma))= & H(\tilde{v}_i\mid v_i; n, \gamma)  \left( v_i - \sigma^{\mathsf{AP}}(\tilde{v}_i;n, \gamma)\right)\\
    = & H(\tilde{v}_i\mid v_i; n, \gamma)\left( v_i - \tilde{v}_i + \int_0^{\tilde{v}_i} x L(x\mid \tilde{v}_i;n, \gamma)dx\right).
\end{align*}
indeed obtains the maximum at $\tilde{v}_i = v_i$. It's sufficient to show that $U^{\mathsf{FP}}(\tilde{v}_i,v_i;\sigma^{\mathsf{AP}}(\cdot;n,\gamma))$ non-decreases with $\tilde{v}_i \in [0, v_i]$, and non-increases with $\tilde{v}_i \in [v_i, 1]$.
Observe that
\begin{align*}
&\frac{\partial U^{\mathsf{FP}}(\tilde{v}_i,v_i;\sigma^{\mathsf{FP}}(\cdot;n,\gamma))}{\partial \tilde{v}_i} \\ = & h(\tilde{v}_i\mid v_i;n,\gamma)  \left(v_i - \tilde{v}_i  +\int_0^{\tilde{v}_i}L(x\mid \tilde{v}_i;n,\gamma)dx\right)\\
 &+ H(\tilde{v}_i\mid v_i;n,\gamma)  \left(-1+L(\tilde{v}_i\mid \tilde{v}_i;n,\gamma)
 + \int_{0}^{\tilde{v}_i}\frac{\partial L(x\mid \tilde{v}_i;n,\gamma)}{\partial \tilde{v}_i}dx
 \right)
 \\
=&  h(\tilde{v}_i\mid v_i;n,\gamma)  \left(v_i - \tilde{v}_i  +\int_0^{\tilde{v}_i}L(x\mid \tilde{v}_i;n,\gamma)dx\right)
 + H(\tilde{v}_i\mid v_i;n,\gamma)   \int_{0}^{\tilde{v}_i}\frac{\partial L(x\mid \tilde{v}_i;n,\gamma)}{\partial \tilde{v}_i}dx\\
 = & h(\tilde{v}_i\mid v_i;n,\gamma)  \left(v_i - \tilde{v}_i\right)
 +
  h(\tilde{v}_i\mid v_i;n,\gamma)  \int_0^{\tilde{v}_i}L(x\mid \tilde{v}_i;n,\gamma)dx\\
  &
  -  H(\tilde{v}_i\mid v_i;n,\gamma)  \frac{h(\tilde{v}_i\mid \tilde{v}_i;n,\gamma)}{H(\tilde{v}_i\mid \tilde{v}_i;n,\gamma)}  \int_0^{\tilde{v}_i}L(x\mid \tilde{v}_i;n,\gamma)dx\\
  = & h(\tilde{v}_i\mid v_i;n,\gamma)  \left(v_i - \tilde{v}_i\right)
 + 
   \left(
 h(\tilde{v}_i\mid v_i;n,\gamma)  - H(\tilde{v}_i\mid v_i;n,\gamma)  \frac{h(\tilde{v}_i\mid \tilde{v}_i;n,\gamma)}{H(\tilde{v}_i\mid \tilde{v}_i;n,\gamma)}
 \right)\int_0^{\tilde{v}_i}L(x\mid \tilde{v}_i;n,\gamma)dx\\
 =&  h(\tilde{v}_i\mid v_i;n,\gamma)  \left(v_i - \tilde{v}_i\right)
 +H(\tilde{v}_i\mid v_i;n,\gamma)  \int_0^{\tilde{v}_i}L(x\mid \tilde{v}_i;n,\gamma)dx
   \left(
 \frac{h(\tilde{v}_i\mid v_i;n,\gamma)}{H(\tilde{v}_i\mid v_i;n,\gamma)}  - \frac{h(\tilde{v}_i\mid \tilde{v}_i;n,\gamma)}{H(\tilde{v}_i\mid \tilde{v}_i;n,\gamma)}
 \right) \\
 =& \FP(\tilde{v}_i,v_i;n,\gamma), 
\end{align*}
hence if $\FP(\tilde{v}_i,v_i;n,\gamma) $ is non-negative for $\tilde{v}_i\in [0,v_i]$ and non-positive for $\tilde{v}_i\in [v_i,1]$ for any given $v_i\in [0,1]$, we have $\frac{\partial U^{\mathsf{FP}}(\tilde{v}_i,v_i;\sigma^{\mathsf{FP}}( ;n,\gamma))}{\partial \tilde{v}_i} \geq 0$ for $\tilde{v}_i 
\in [0, v_i]$, and $\frac{U^{\mathsf{FP}}(\tilde{v}_i,v_i;\sigma^{\mathsf{FP}}(\cdot;n,\gamma))}{\partial \tilde{v}_i} \leq 0$ for $\tilde{v}_i \in  [v_i,1]$. Therefore, $U^{\mathsf{FP}}(\tilde{v}_i,v_i;\sigma^{\mathsf{FP}}(\cdot;n,\gamma))$ non-decreases with $\tilde{v}_i \in [0, v_i]$, and non-increases with $\tilde{v}_i \in [v_i, 1]$, as desired.



 \medskip

\noindent\textit{\underline{Step 4: When $\gamma = 1$, the condition in Step 3 holds, and $\sigma^{\FP}(v_i;n,\gamma=1) = v_i - \int_0^{v_i} \frac{F^{m-1}(x)}{F^{m-1}(v_i)}  dx$.}}

 \medskip
 
\noindent When $\gamma = 1$, by (ii) of \Cref{prop:g_property}, we have $g(\cdot;n,\gamma)$ is affiliated in $v\in [0,1]^n$. Then by the discussion in \Cref{app_subsec:condition_firstprice_SSM}, we conclude that 
\begin{align*}  \frac{h(\tilde{v}_i\mid v_i^\prime;n,\gamma)}{H(\tilde{v}_i\mid v_i^\prime;n,\gamma)} 
\leq \frac{h(\tilde{v}_i\mid \tilde{v}_i;n,\gamma)}{H(\tilde{v}_i\mid \tilde{v}_i;n,\gamma)} 
\leq 
 \frac{h(\tilde{v}_i\mid v_i;n)}{H(\tilde{v}_i\mid v_i;n,\gamma)}
\end{align*}
for any $v_i^\prime\leq \tilde{v}_i\leq v_i$. This implies that $\FP(\tilde{v}_i,v_i;n,\gamma) $ is non-negative for $\tilde{v}_i\in [0,v_i]$ and non-positive for $\tilde{v}_i\in [v_i,1]$ for any given $v_i\in [0,1]$.

\noindent To obtain $\sigma^{\FP}(v_i;n,\gamma=1) $, by  \Cref{cor:special_case_H_h}, we have $H(x\mid x;n,\gamma = 1) = \frac{\kappa(x; n, \gamma = 1)}{\C^{m-1}_{n-1}}F^{m-1}(x)$, $h(x \mid x;n,\gamma = 1) = \frac{m-1}{\C^{m-1}_{n-1}}\kappa(x; n, \gamma = 1)F^{m-2}(x)f(x)$, and $\frac{h(x \mid x;n,\gamma = 1)}{H(x \mid x;n,\gamma = 1)} = \frac{(m-1)f(x)}{F(x)}$. Hence $L(x\mid \tilde{v}_i;n,\gamma) = \frac{F^{m-1}(x)}{F^{m-1}(v_i)}$, and $\sigma^{\FP}(v_i;n,\gamma=1) = v_i - \int_0^{v_i} \frac{F^{m-1}(x)}{F^{m-1}(v_i)}  dx$.   

\medskip

\noindent\textit{\underline{Step 5: The condition in Step 3 holds for the power law prior distribution $F(x)=x^c$ when $m=3$ and $c\in (0,1]$.}}

\medskip

\noindent By (ii). (a) of \Cref{cor:special_case_H_h}, for any $v_i^\prime\leq \tilde{v}_i\leq v_i$, we have
\begin{align*}  \frac{h(\tilde{v}_i\mid v_i^\prime;n=2,\gamma)}{H(\tilde{v}_i\mid v_i^\prime;n=2,\gamma)} 
\leq \frac{h(\tilde{v}_i\mid \tilde{v}_i;n=2,\gamma)}{H(\tilde{v}_i\mid \tilde{v}_i;n=2,\gamma)} 
\geq 
 \frac{h(\tilde{v}_i\mid v_i;n=2,\gamma)}{H(\tilde{v}_i\mid v_i;n=2,\gamma)}.
\end{align*}
Hence when $\tilde{v}_i \geq v_i$, $\FP(\tilde{v}_i,v_i;n,\gamma) $ is non-positive. Now we show that for $F(x) = x^c$ $(c\leq 1)$, $\FP(\tilde{v}_i,v_i;n,\gamma) $ is non-negative when $\tilde{v}_i \leq v_i$. To see this, observe that $\frac{h(\tilde{v}_i\mid v_i;n,\gamma)}{H(\tilde{v}_i\mid v_i;n,\gamma)}  \leq \frac{h(\tilde{v}_i\mid \tilde{v}_i;n,\gamma)}{H(\tilde{v}_i\mid \tilde{v}_i;n,\gamma)}$ and $\int_0^{\tilde{v}_i}L(x\mid \tilde{v}_i;n,\gamma)dx \leq \tilde{v}_i$, hence 
\begin{align*}
  &\FP(\tilde{v}_i,v_i;n,\gamma)\\  =&h(\tilde{v}_i\mid v_i;n,\gamma)  \left(v_i - \tilde{v}_i\right)
 +H(\tilde{v}_i\mid v_i;n,\gamma)  \int_0^{\tilde{v}_i}L(x\mid \tilde{v}_i;n,\gamma)dx
   \left(
 \frac{h(\tilde{v}_i\mid v_i;n,\gamma)}{H(\tilde{v}_i\mid v_i;n,\gamma)}  - \frac{h(\tilde{v}_i\mid \tilde{v}_i;n,\gamma)}{H(\tilde{v}_i\mid \tilde{v}_i;n,\gamma)}
 \right) \\
 \geq & h(\tilde{v}_i\mid v_i;n,\gamma)  \left(v_i - \tilde{v}_i\right)
 +H(\tilde{v}_i\mid v_i;n,\gamma) \tilde{v}_i \left(
 \frac{h(\tilde{v}_i\mid v_i;n,\gamma)}{H(\tilde{v}_i\mid v_i;n,\gamma)}  - \frac{h(\tilde{v}_i\mid \tilde{v}_i;n,\gamma)}{H(\tilde{v}_i\mid \tilde{v}_i;n,\gamma)}
 \right) \\
 \geq & H(\tilde{v}_i\mid v_i;n,\gamma) \left( \frac{h(\tilde{v}_i\mid {v}_i;n,\gamma)}{H(\tilde{v}_i\mid {v}_i;n,\gamma)}{v}_i -  \frac{h(\tilde{v}_i\mid \tilde{v}_i;n,\gamma)}{H(\tilde{v}_i\mid \tilde{v}_i;n,\gamma)}\tilde{v}_i\right)\\
 \geq & 0,
\end{align*}
where the last inequality is by (ii).(b) of \Cref{cor:special_case_H_h}. This completes the proof.
\noindent 
\end{proof}

%%%%%%%%%%%% proof of thm:opt_admittednumber_firstprice
\begin{proof}[Proof of \Cref{thm:opt_admittednumber_firstprice}] We show the cases for $\gamma = 0$ and $\gamma = 1$, respectively.

\medskip

\noindent\textit{\underline{Part 1: Analysis of the case $\gamma = 0$.}} 

\medskip

\noindent When $ \gamma = 0$, since $g(\cdot;n,\gamma)$ is affiliated in $v\in [0,1]^n$, then by the similar proof as in the Step 4 of \Cref{thm:SSM_firstprice}, there exists an SSM equilibrium strategy. Furthermore, by \Cref{lem:H_h} and \Cref{lem:marginals_joint_g}, we know that $h(x\mid x;n,\gamma=0) = (n-1)F^{n-2}(x)f(x)$, $H(x\mid x;n,\gamma=0) = F^{n-1}(x)$ and $G^{\lar}(\cdot; n,\gamma=0) = F^n(x)$. Hence $\sigma^{\FP}(v_i;n,\gamma) = v_i - \int_0^{v_i} \frac{F^{n-1}(x)}{F^{n-1}(v_i)}  dx$, and the expected revenue is
\begin{align*}
 \R^{\FP}(n;\gamma=0,F)=& \mathbb{E}_{V_i\sim G^{\lar}(\cdot; n,\gamma=0)}\left[\sigmaFP(V_i;n,\gamma=0)\right]
 =\int_0^1 \left(v_i - \int_0^{v_i} \frac{F^{n-1}(x)}{F^{n-1}(v_i)}\, dx\right)\, dG^{\lar}(v_i;n,\gamma = 0)\\[1mm]
 =& \int_0^1 dx \int_x^1 \left(1-  \frac{F^{n-1}(x)}{F^{n-1}(v_i)}\right)dG^{\lar}(v_i;n,\gamma = 0)\\
 =&\int_0^1 \left(1-nF^{n-1}(x) + (n-1)F^n(x)\right)dx,
\end{align*}
where the third equality is obtained by the Fubini theorem. Observe that when $n\geq 2$,
\begin{align*}
    \R^{\FP}(n+1;\gamma=0,F) - \R^{\FP}(n;\gamma=0,F) = n\int_0^1 F^{n-1}(x)(F(x) - 1)^2dx > 0, 
\end{align*}
hence $n^* = m$.

\medskip

\noindent\textit{\underline{Part 2: Analysis of the case $\gamma = 1$.}} 

\medskip

\noindent When $ \gamma = 1$, by (iii) of \Cref{thm:SSM_firstprice}, we know that there exists an SSM equilibrium strategy and $\sigma^{\FP}(v_i;n,\gamma) = v_i - \int_0^{v_i} \frac{F^{m-1}(x)}{F^{m-1}(v_i)}  dx$. 
Furthermore, by  \Cref{lem:marginals_joint_g},  $G^{\lar}(\cdot; n,\gamma=0) = F^m(x)$. Hence the expected revenue is
\begin{align*}
 \R^{\FP}(n;\gamma=0,F)=& \mathbb{E}_{V_i\sim G^{\lar}(\cdot; n,\gamma=0)}\left[\sigmaFP(V_i;n,\gamma=0)\right]
 =\int_0^1 \left(v_i - \int_0^{v_i} \frac{F^{m-1}(x)}{F^{m-1}(v_i)}\, dx\right)\, dF^{m}(x),
\end{align*}
which implies that admitting any number of players
yields the same expected revenue.
\end{proof}

%%%%%%%%%%%%%%%%%%%%% proof of results in section Revenue Ranking
\subsection{Proofs for \Cref{sec:rev_ranking}}
\begin{proof}[Proof of \Cref{thm:rev_ranking}]

\Cref{thm:rev_ranking} (i) directly follows from the discussions after \Cref{thm:rev_ranking}, which are mainly based on \Cref{thm:opt_admittednumber_allpay} and \Cref{thm:opt_admittednumber_firstprice}. We now prove the second part.

By \Cref{thm:opt_admittednumber_allpay} (i), we have
\begin{align*}
 \R_\ast^{\AP}(\gamma,F,m)
 &= \R^{\AP}(n=2;\gamma,F,m)\\[1mm]
 &= \int_0^1 \frac{2}{J(F(x),n=2,m)} \cdot \left(1-G^{\mar}(x; n=2,\gamma)\right) \cdot x \, (m-1) F^{m-2}(x) f(x) \, dx.
\end{align*}
Let $A(x)$ denote the integrand in the above equation.
Besides, we have
\begin{align*}
 \R^{\FP}(n=m+1;\gamma,F,m+1)
 &= \R^{\AP}(n=m+1;\gamma,F,m+1)\\[1mm]
 &= \int_0^1 (m+1) \cdot (1-F(x)) \cdot x \, m F^{m-1}(x) f(x) \, dx,
\end{align*}
where the first equality follows from the revenue equivalence between first-price and all-pay auctions under the i.i.d.\ setting. Let $B(x)$ denote the integrand in the above equation.

To complete the proof, it is sufficient to show that when $m$ is large enough, $A(x) > B(x)$ for all $x \in (0,1)$. For any $x \in (0,1)$, we have
\begin{align*}
  \frac{A(x)}{B(x)}
  &= \frac{\frac{2}{J(F(x),n=2,m)} \cdot \left(1-G^{\mar}(x; n=2,\gamma)\right) \cdot (m-1)}{(m+1) \cdot (1-F(x)) \cdot m \cdot F(x)} \\
  &= \frac{\frac{2}{J(F(x),n=2,m)} \cdot \left(1-G^{\mar}(x; n=2,\gamma)\right)}{m \cdot (1-F(x))} \cdot \frac{m-1}{(m+1)F(x)} \\
  &\overset{(a)}{>} \frac{m-1}{(m+1)F(x)} \\
  &\geq 1, \quad \text{for large enough } m.
\end{align*}
The inequality in (a) holds because $\frac{n}{J(F(x),n,m)} \cdot (1-G^{\mar}(x;n,\gamma))$ is strictly decreasing in $n$ (notice that when $n=m$, this term becomes $m(1-F(x))$); see the proof of \Cref{thm:opt_admittednumber_allpay}. This completes the proof.
\end{proof}







%%%%%%%%%%%%%%%%% proofs for auxiliary results in appendix A
\subsection{Proofs for \Cref{app_sec:auxiliary_results}}

%%%%%%%%%%%% proof ox admission probability
\begin{proof}[Proof of \Cref{lem:closedforms_admission_prob}] Define $v_{(0)} = 0$ and $v_{(n+1)} = 1$. Since when $x \in \left[v_{(k)}, v_{(k+1)}\right)$, 
\begin{align*}
 \prod_{i\in \I}\left[1-F(x;v_i,\gamma)\right]=&\prod_{i\in \I}\left[\gamma\cdot \mathds{1}\{v_i> x\} + (1-\gamma)\cdot (1-F(x))\right]\\
 =& \left[(1-\gamma)\cdot (1-F(x))\right]^k
 \cdot 
 \left[\gamma + (1-\gamma)\cdot (1 -F(x))\right]^{n-k},   
\end{align*}
we have 
\begin{align*}
\psi(v;n,\gamma)&=  \Pr\left\{
\max_{j\in \barI} \zeta_j \leq \min_{i\in \I}\zeta_i
\mid v\right\} =
\int_0^1\prod_{i\in \I}[1-F(x;v_i,\gamma)] dF^{m-n}(x)\\
 &= \sum_{k=0}^n \int_{v_{(k)}}^{v_{(k+1)}}
 \left[(1-\gamma)\cdot (1-F(x))\right]^k
 \cdot 
 \left[\gamma + (1-\gamma)\cdot (1 -F(x))\right]^{n-k}dF^{m-n}(x)\\
 & = \sum_{k=0}^n \int_{F\left(v_{(k)}\right)}^{F\left(v_{(k+1)}\right)}
 (1-\gamma)^k (1-x)^k
 \cdot 
 \left[\gamma + (1-\gamma)\cdot (1 -x)\right]^{n-k}dx^{m-n}\\
 & = \sum_{k=0}^n \sum_{j=0}^{n-k}\int_{F\left(v_{(k)}\right)}^{F\left(v_{(k+1)}\right)}\C^{n-k}_j\gamma^j
 (1-\gamma)^{n-j} (1-x)^{n-j}
 dx^{m-n}\\
 & = (m-n)\sum_{j=0}^n \gamma^j
 (1-\gamma)^{n-j} \sum_{k=0}^{n-j}\C^{n-k}_j\int_{F\left(v_{(k)}\right)}^{F\left(v_{(k+1)}\right)} x^{m-n-1}(1-x)^{n-j}
 dx\\
 & = (m-n)\sum_{j=0}^n \gamma^j
 (1-\gamma)^{n-j} \sum_{k=0}^{n-j}\C^{n-k}_j\left(Q_j\left(v_{(k+1)};n,m\right) - Q_j\left(v_{(k)};n,m\right)\right)\\
 & = (m-n)\sum_{j=0}^n \gamma^j
 (1-\gamma)^{n-j}\left( \sum_{k=1}^{n-j+1}\C^{n-k+1}_jQ_j\left(v_{(k)};n,m\right) - \sum_{k=0}^{n-j}\C^{n-k}_jQ_j\left(v_{(k)};n,m\right)\right)\\
 & = (m-n)\sum_{j=1}^n \gamma^j
 (1-\gamma)^{n-j} \sum_{k=1}^{n-j}\C^{n-k}_{j-1}Q_j\left(v_{(k)};n,m\right) + (m-n)\sum_{j=0}^n \gamma^j
 (1-\gamma)^{n-j} Q_j\left(v_{(n-j+1)};n,m\right)\\
 & = (m-n)\sum_{k=1}^{n-1}  \sum_{j=1}^{n-k}\C^{n-k}_{j-1}\gamma^j
 (1-\gamma)^{n-j}Q_j\left(v_{(k)};n,m\right) + (m-n)\sum_{j=0}^n \gamma^j
 (1-\gamma)^{n-j}Q_j\left(v_{(n-j+1)};n,m\right)\\
 & = (m-n)\sum_{k=1}^{n-1}  \sum_{j=1}^{n-k}\C^{n-k}_{j-1}\gamma^j
 (1-\gamma)^{n-j}\sum_{i=0}^{n-j}\frac{(-1)^{i}\C^{n-j}_i}{m-n+i}F^{m-n+i}\left(v_{(k)}\right)\\
 & \quad + (m-n)\sum_{j=0}^n \gamma^j
 (1-\gamma)^{n-j}\sum_{i=0}^{n-j}\frac{(-1)^{i}\C^{n-j}_i}{m-n+i}F^{m-n+i}\left(v_{(n-j+1)}\right)\\
 & = (m-n)\sum_{k=1}^{n-1} F^{m-n}\left(v_{(k)}\right) \sum_{j=1}^{n-k}\C^{n-k}_{j-1}\gamma^j
 (1-\gamma)^{n-j}\sum_{i=0}^{n-j}\frac{(-1)^{i}\C^{n-j}_i}{m-n+i}F^{i}\left(v_{(k)}\right)\\
 & \quad + (m-n)\sum_{j=0}^{n-1} \gamma^{n-j}
 (1-\gamma)^{j}F^{m-n}\left(v_{(j+1)}\right)\sum_{i=0}^{j}\frac{(-1)^{i}\C^{j}_i}{m-n+i}F^{i}\left(v_{(j+1)}\right) + \frac{(1-\gamma)^n}{\C^{m}_n}\\
  & = (m-n)\sum_{k=1}^{n-1} F^{m-n} \left(v_{(k)}\right)\sum_{j=1}^{n-k}\C^{n-k}_{j-1}\gamma^j
 (1-\gamma)^{n-j}\sum_{i=0}^{n-j}\frac{(-1)^{i}\C^{n-j}_i}{m-n+i}F^{i}\left(v_{(k)}\right)\\
 & \quad + (m-n)\sum_{k=1}^{n} \gamma^{n-k+1}
 (1-\gamma)^{k-1}F^{m-n}\left(v_{(k)}\right)\sum_{j=0}^{k-1}\frac{(-1)^{j}\C^{k-1}_j}{m-n+j}F^{j}\left(v_{(k)}\right) + \frac{(1-\gamma)^n}{\C^{m}_n}\\
 &= \hat{\psi}_0(n,\gamma)+\sum_{k=1}^n \hat{\psi}_k(v_{(k)};n,\gamma),
  \end{align*}
where 
\begin{align*}
    Q_j\left(x;n,m\right) 
= \int_{0}^{F\left(x\right)} t^{m-n-1}(1-t)^{n-j}
 dt = \sum_{i=0}^{n-j}(-1)^{i}\C^{n-j}_i\int_{0}^xt^{m-n+i-1}dt = \sum_{i=0}^{n-j}\frac{(-1)^{i}\C^{n-j}_i}{m-n+i}x^{m-n+i}.
\end{align*}
Now we calculate the normalizing term $\kappa(v_i;n,\gamma)$. By definition,
\begin{align*}
 \frac{1}{\kappa(v_i;n,\gamma)} =& \int_{ [0,1]^{n-1}} \psi(v;n,\gamma) \prod_{j\in \I_{-i}} f(v_j) dv_{-i} \\
 =& \int_{ [0,1]^{n-1}}  \left(\int_0^1\prod_{i\in \I}\left[1-F(x;v_i,\gamma)\right] dF^{m-n}(x)\right)\prod_{j\in \I_{-i}} f(v_j) dv_{-i}\\
 =& \int_0^1 [1-F(x;v_i,\gamma)]\prod_{j\in \I_{-i}}\left(\int_0^1 [1-F(x;v_j,\gamma)]dF(v_j)\right)dF^{m-n}(x),
\end{align*}
where in the last equality we interchange the order of integration in iterated integrals by the Fubini theorem. Observe that 
\begin{align*}
\int_0^1 [1-F(x;v_j,\gamma)]dF(v_j) = &\int_0^1 \gamma \cdot \mathds{1}\{v_j>x\} + (1-\gamma)\cdot (1-F(x))dF(v_j) \\
=&(1-\gamma)(1-F(x)) F(x) +\left(\gamma + (1-\gamma)(1-F(x))\right)(1-F(x))\\
=&1-F(x),
\end{align*}
hence 
\begin{align*}
\frac{1}{\kappa(v_i;n,\gamma)} =&   \int_0^1 \left(\gamma \cdot \mathds{1}\{v_j>x\} + (1-\gamma)\cdot (1-F(x))\right)(1-F(x))^{n-1}dF^{m-n}(x)\\
=& (1-\gamma)\cdot \int_0^1(1-F(x))^ndF^{m-n}(x) + \gamma \cdot \int_0^{v_i} (1-F(x))^{n-1}dF^{m-n}(x)\\
=& \frac{1-\gamma}{\C^m_n} + (m-n)\gamma \cdot \sum_{i=0}^{n-1}\frac{(-1)^i\C^{n-1}_i}{m-n+i} F^{m-n+i}(v_i).
\end{align*}
\end{proof}

%%%%%%%% proof of property of h
\begin{proof}[Proof of \Cref{lem:property_h}]
(i). Since $\prod_{j\in \I_{-i}} f(v_j)$ and $\psi(v;n,\gamma) = \Pr\left\{
\max_{j\in \barI} \zeta_j \leq \min_{i\in \I}\zeta_i
\mid v\right\}$ are symmetric with $v_{-i}$, we have $\left(\prod_{j\in \I_{-i}} f(v_j)\right)
 \cdot 
\psi(v;n,\gamma)$ is symmetric with $v_{-i}$. Combining this fact and the derivative rule of integral with parameters, part (i) follows.

\noindent (ii). This can be proved by observing that $\psi(v;n,\gamma)\mid _{v_j = y} = \Pr\left\{
\max_{j\in \barI} \zeta_j \leq \min_{i\in \I}\zeta_i
\mid v_j = y, v_{-j}\right\}$ is increasing with $v_i$.
\end{proof}

%%%%%%%% proof of H and h
\begin{proof}[Proof of \Cref{lem:H_h}] (i). 
When $x \in [0,v_i]$,
\begin{align*}
\frac{H(x\mid v_i;n,\gamma)}{\kappa(v_i;n,\gamma)} &=   
 \int_{[0,x]^{n-1}}
\psi(v;n,\gamma)\prod_{j\in \I_{-i}} f(v_j) 
dv_{-i}\\
& = \int_{[0,x]^{n-1}}
\left(\hat{\psi}_0(n,\gamma)+\sum_{k=1}^n \hat{\psi}_k(v_{(k)};n,\gamma)\right)\prod_{j\in \I_{-i}} f(v_j)dv_{-i}\\
& = \sum_{k=1}^{n-1} \int_{[0,x]^{n-1}} \hat{\psi}_k(v_{(k)};n,\gamma) \prod_{j\in \I_{-i}} f(v_j)
dv_{-i} + F^{n-1}(x)\left(\hat{\psi}_0(n,\gamma)+\hat{\psi}_n(v_{i};n,\gamma)\right)\\
& = (m-n)\sum_{k=1}^{n-1}  \sum_{j=1}^{n-k}\C^{n-k}_{j-1}\gamma^j
 (1-\gamma)^{n-j}\sum_{i=0}^{n-j}\frac{(-1)^{i}\C^{n-j}_i}{m-n+i}\int_{[0,x]^{n-1}}F^{m-n+i}\left(v_{(k)}\right)\left(\prod_{j\in \I_{-i}} f(v_j)\right)dv_{-i}\\
 & \quad + (m-n)\sum_{k=1}^{n-1} \gamma^{n-k+1}
 (1-\gamma)^{k-1}\sum_{j=0}^{k-1}\frac{(-1)^{j}\C^{k-1}_j}{m-n+j}\int_{[0,x]^{n-1}}F^{m-n+j}\left(v_{(k)}\right)\left(\prod_{j\in \I_{-i}} f(v_j)\right) dv_{-i}\\
 & + F^{n-1}(x)\left((m-n)\gamma(1-\gamma)^{n-1}F^{m-n}(v_i)\sum_{j=0}^{n-1}\frac{(-1)^j\C^{n-1}_j}{m-n+j}F^j(v_i) + \frac{(1-\gamma)^n}{\C^m_n}\right)\\
& = (m-n)F^{m-1}(x)\sum_{k=1}^{n-1}  \sum_{j=1}^{n-k}\C^{n-k}_{j-1}\gamma^j
 (1-\gamma)^{n-j}\sum_{i=0}^{n-j}\frac{(-1)^{i}\C^{n-j}_i\C^{n-1}_{n-k}}{(m-n+i)\C^{m+i-1}_{n-k}}F^i\left(x\right)\\
 & \quad + (m-n)F^{m-1}(x)\sum_{k=1}^{n-1} \gamma^{n-k+1}
 (1-\gamma)^{k-1}\sum_{j=0}^{k-1}\frac{(-1)^{j}\C^{k-1}_j\C^{n-1}_{n-k}}{(m-n+j)\C^{m+j-1}_{n-k}}F^{j}\left(x\right) \\
 &\quad + F^{n-1}(x)\left((m-n)\gamma(1-\gamma)^{n-1}F^{m-n}(v_i)\sum_{j=0}^{n-1}\frac{(-1)^j\C^{n-1}_j}{m-n+j}F^j(v_i) + \frac{(1-\gamma)^n}{\C^m_n}\right),
\end{align*}
where we use \Cref{lem:useful_identity} in the in the last equality.

\medskip

\noindent When $x\in [v_i,1]$,
\begin{align*}
    \frac{H(x\mid v_i;n,\gamma)}{\kappa(v_i;n,\gamma)} &=   
 \int_{[0,x]^{n-1}}
\psi(v;n,\gamma)\prod_{j\in \I_{-i}} f(v_j) 
dv_{-i}\\
& = \int_{[0,x]^{n-1}}
\left(\hat{\psi}_0(n,\gamma)+\sum_{k=1}^n \hat{\psi}_k(v_{(k)};n,\gamma)\right)\prod_{j\in \I_{-i}} f(v_j)dv_{-i}\\
& = \hat{\psi}_0(n,\gamma)F^{n-1}(x) + \sum_{k=1}^{n}\C^{n-1}_{k-1}\int_{[0,v_i]^{k-1}\times [v_i,x]^{n-k}}
\sum_{s=1}^n \hat{\psi}_s(v_{(s)};n,\gamma)\prod_{\ell\in \I_{-i}} f(v_{\ell})dv_{-i}\\
& = \hat{\psi}_0(n,\gamma)F^{n-1}(x) + \sum_{k=1}^n \C^{n-1}_{k-1}\hat{\psi}_k(v_{i};n,\gamma)F^{k-1}(v_i)(F(x) - F(v_i))^{n-k}\\
& \quad + \sum_{k=1}^{n}\C^{n-1}_{k-1}\int_{[0,v_i]^{k-1}\times [v_i,x]^{n-k}}
\sum_{s\neq k} \hat{\psi}_s(v_{(s)};n,\gamma)\prod_{\ell\in \I_{-i}} f(v_{\ell})dv_{-i}.
\end{align*}
Observe that by \Cref{lem:useful_identity} and \Cref{lem:closedforms_admission_prob}, we have 
\begin{align*}
&\sum_{s= 1}^{k-1}\int_{[0,v_i]^{k-1}\times [v_i,x]^{n-k}} \hat{\psi}_s(v_{(s)};n,\gamma)\prod_{\ell\in \I_{-i}} f(v_{\ell})dv_{-i}  \\
=&(m-n) \sum_{s=1}^{k-1}\sum_{j=1}^{n-s} \C^{n-s}_{j-1}\gamma^j
 (1-\gamma)^{n-j}\sum_{t=0}^{n-j}\frac{(-1)^{t}\C^{n-j}_t}{m-n+t} \int_{[0,v_i]^{k-1}\times [v_i,x]^{n-k}} F^{m-n+t}\left(v_{(s)}\right)\prod_{\ell\in \I_{-i}} f(v_{\ell})dv_{-i} \\
 &+ (m-n) \sum_{s=1}^{k-1}\gamma^{n-s+1}
 (1-\gamma)^{s-1}\sum_{j=0}^{s-1}\frac{(-1)^{j}\C^{s-1}_j}{m-n+j}\int_{[0,v_i]^{k-1}\times [v_i,x]^{n-k}} F^{m-n+j}\left(v_{(s)}\right)\prod_{\ell\in \I_{-i}} f(v_{\ell})dv_{-i}\\
 =& (m-n) \sum_{s=1}^{k-1}\sum_{j=1}^{n-s} \C^{n-s}_{j-1}\gamma^j
 (1-\gamma)^{n-j}\sum_{t=0}^{n-j}\frac{(-1)^{t}\C^{n-j}_t\C^{k-1}_{k-s}}{(m-n+t)\C^{m-n+t+k-1}_{k-s}}F^{m-n+t+k-1}(v_i)(F(x) - F(v_i))^{n-k}\\
 & + (m-n) \sum_{s=1}^{k-1}\gamma^{n-s+1}
 (1-\gamma)^{s-1}\sum_{j=0}^{s-1}\frac{(-1)^{j}\C^{s-1}_j\C^{k-1}_{k-s}}{(m-n+j)\C^{m-n+j+k-1}_{k-s}}F^{m-n+j+k-1}(v_i)(F(x) - F(v_i))^{n-k},
\end{align*}
and 
{\small
\begin{align*}
&\sum_{s= k+1}^{n}\int_{[0,v_i]^{k-1}\times [v_i,x]^{n-k}} \hat{\psi}_s(v_{(s)};n,\gamma)\prod_{\ell\in \I_{-i}} f(v_{\ell})dv_{-i}  \\
=&(m-n) \sum_{s=k+1}^{n-1}\sum_{j=1}^{n-s} \C^{n-s}_{j-1}\gamma^j
 (1-\gamma)^{n-j}\sum_{t=0}^{n-j}\frac{(-1)^{t}\C^{n-j}_t}{m-n+t} \int_{[0,v_i]^{k-1}\times [v_i,x]^{n-k}} F^{m-n+t}\left(v_{(s)}\right)\prod_{\ell\in \I_{-i}} f(v_{\ell})dv_{-i} \\
 &+ (m-n) \sum_{s=k+1}^{n}\gamma^{n-s+1}
 (1-\gamma)^{s-1}\sum_{j=0}^{s-1}\frac{(-1)^{j}\C^{s-1}_j}{m-n+j}\int_{[0,v_i]^{k-1}\times [v_i,x]^{n-k}} F^{m-n+j}\left(v_{(s)}\right)\prod_{\ell\in \I_{-i}} f(v_{\ell})dv_{-i}\\
 =& (m-n) \sum_{s=k+1}^{n-1}\sum_{j=1}^{n-s} \C^{n-s}_{j-1}\gamma^j
 (1-\gamma)^{n-j}\sum_{t=0}^{n-j}\frac{(-1)^{t}\C^{n-j}_t}{m-n+t}\sum_{r=0}^{m-n+t}\frac{\C^{n-k}_{n-s+1}\C^{m-n+t}_r}{\C^{n+r-k}_{n-s+1}}F^{m-n+t+k-r-1}(v_i)(F(x) - F(v_i))^{n+r-k}\\
 & + (m-n) \sum_{s=k+1}^{n}\gamma^{n-s+1}
 (1-\gamma)^{s-1}\sum_{j=0}^{s-1}\frac{(-1)^{j}\C^{s-1}_j}{m-n+j}\sum_{r=0}^{m-n+j}\frac{\C^{n-k}_{n-s+1}\C^{m-n+j}_r}{\C^{n+r-k}_{n-s+1}}F^{m-n+j+k-r-1}(v_i)(F(x) - F(v_i))^{n+r-k},
\end{align*}}
hence 
{\footnotesize
\begin{align*}
&\frac{H(x\mid v_i;n,\gamma)}{\kappa(v_i;n,\gamma)}\\
=& \frac{(1-\gamma)^n}{\C^m_n}F^{n-1}(x) + (m-n)\sum_{k=1}^{n-1}\C^{n-1}_{k-1} F^{m-n+k-1}(v_i)(F(x) - F(v_i))^{n-k} \sum_{j=1}^{n-k} \C^{n-k}_{j-1}\gamma^j
 (1-\gamma)^{n-j}\sum_{t=0}^{n-j}\frac{(-1)^{t}\C^{n-j}_t}{m-n+t}F^{t}\left(v_{i}\right)\\
 & + (m-n)\sum_{k=1}^{n}\C^{n-1}_{k-1} \gamma^{n-k+1}
 (1-\gamma)^{k-1}F^{m-n+k-1}\left(v_{i}\right)(F(x) - F(v_i))^{n-k}\sum_{j=0}^{k-1}\frac{(-1)^{j}\C^{k-1}_j}{m-n+j}F^{j}\left(v_{i}\right)\\
 & + (m-n) \sum_{k=2}^{n}\C^{n-1}_{k-1}\sum_{s=1}^{k-1}\sum_{j=1}^{n-s} \C^{n-s}_{j-1}\gamma^j
 (1-\gamma)^{n-j}\sum_{t=0}^{n-j}\frac{(-1)^{t}\C^{n-j}_t\C^{k-1}_{k-s}}{(m-n+t)\C^{m-n+t+k-1}_{k-s}}F^{m-n+t+k-1}(v_i)(F(x) - F(v_i))^{n-k}\\
 & + (m-n)\sum_{k=2}^{n} \C^{n-1}_{k-1}\sum_{s=1}^{k-1}\gamma^{n-s+1}
 (1-\gamma)^{s-1}\sum_{j=0}^{s-1}\frac{(-1)^{j}\C^{s-1}_j\C^{k-1}_{k-s}}{(m-n+j)\C^{m-n+j+k-1}_{k-s}}F^{m-n+j+k-1}(v_i)(F(x) - F(v_i))^{n-k}\\
 & + (m-n)\sum_{k=1}^{n-2}\C^{n-1}_{k-1} \sum_{s=k+1}^{n-1}\sum_{j=1}^{n-s} \C^{n-s}_{j-1}\gamma^j
 (1-\gamma)^{n-j}\sum_{t=0}^{n-j}\frac{(-1)^{t}\C^{n-j}_t}{m-n+t}\sum_{r=0}^{m-n+t}\frac{\C^{n-k}_{n-s+1}\C^{m-n+t}_r}{\C^{n+r-k}_{n-s+1}}F^{m-n+t+k-r-1}(v_i)(F(x) - F(v_i))^{n+r-k}\\
 & + (m-n)\sum_{k=1}^{n-1} \C^{n-1}_{k-1}\sum_{s=k+1}^{n}\gamma^{n-s+1}
 (1-\gamma)^{s-1}\sum_{j=0}^{s-1}\frac{(-1)^{j}\C^{s-1}_j}{m-n+j}\sum_{r=0}^{m-n+j}\frac{\C^{n-k}_{n-s+1}\C^{m-n+j}_r}{\C^{n+r-k}_{n-s+1}}F^{m-n+j+k-r-1}(v_i)(F(x) - F(v_i))^{n+r-k}
\end{align*}
}
(ii). $h(x\mid v_i;n,\gamma)$ can be obtained by directly taking the derivative with $x$ from $H(x\mid v_i;n,\gamma)$.
\end{proof}

%%%%%%%%%%%% proof of special_case_H_h
\begin{proof}[{Proof of \Cref{cor:special_case_H_h}}]

We only give the proof for analyzing $\frac{h(x\mid v_i;n=2,\gamma)}{H(x\mid v_i;n=2,\gamma)}$ and $\frac{h(x\mid v_i;n=2,\gamma)}{H(x\mid v_i;n=2,\gamma)}v_i$. The other parts can be directly obtained by \Cref{lem:H_h}.

\medskip

\noindent\textit{\underline{Step 1: Analysis of $\frac{h(x\mid v_i;n=2,\gamma)}{H(x\mid v_i;n=2,\gamma)}$.}}

\medskip

\noindent (i). When $x\leq v_i$,
\begin{align*}
\frac{h(x\mid v_i;n=2,\gamma)}{H(x\mid v_i;n=2,\gamma)} =& \frac{\frac{ (1-\gamma)^2}{3}  + \gamma (1-\gamma)F\left(v_i\right) - \frac{\gamma(1-\gamma)}{2} F^2\left(v_i\right) + \gamma F\left(x\right) - \frac{\gamma(1-\gamma)}{2} F^2\left(x\right)}{\frac{ (1-\gamma)^2}{3}  + \gamma (1-\gamma)F\left(v_i\right) - \frac{\gamma(1-\gamma)}{2} F^2\left(v_i\right) + \frac{\gamma}{2}  F\left(x\right)  - \frac{\gamma(1-\gamma)}{6}F^2\left(x\right)} \frac{f(x)}{F(x)} \\
=&\left(1 + \frac{\frac{\gamma}{2} F\left(x\right) - \frac{\gamma(1-\gamma)}{3} F^2\left(x\right)}{\frac{ (1-\gamma)^2}{3}  + \gamma (1-\gamma)F\left(v_i\right) - \frac{\gamma(1-\gamma)}{2} F^2\left(v_i\right) + \frac{\gamma}{2}  F\left(x\right)  - \frac{\gamma(1-\gamma)}{6}F^2\left(x\right)} \right)\frac{f(x)}{F(x)}.
\end{align*}
Observe that $\gamma (1-\gamma)F\left(v_i\right) - \frac{\gamma(1-\gamma)}{2} F^2\left(v_i\right)$ is non-increasing with $v_i$ and $\frac{\gamma}{2} F\left(x\right) - \frac{\gamma(1-\gamma)}{3} F^2\left(x\right)\geq 0$, hence $\frac{h(x\mid v_i;n=2,\gamma)}{H(x\mid v_i;n=2,\gamma)}$ is non-increasing with $v_i$. This implies that $\frac{h(\tilde{v}_i\mid v_i;n=2,\gamma)}{H(\tilde{v}_i\mid v_i;n=2,\gamma)}  \leq \frac{h(\tilde{v}_i\mid \tilde{v}_i;n=2,\gamma)}{H(\tilde{v}_i\mid \tilde{v}_i;n=2,\gamma)}$ for $\tilde{v}_i \leq v_i$.

\medskip

\noindent (ii). When $x>v_i$, 
\begin{align*}
&\frac{h(x\mid v_i;n=2,\gamma)}{H(x\mid v_i;n=2,\gamma)}\\
=&\frac{\frac{(1-\gamma)^2}{3} + \gamma(1-\gamma) F\left(x\right) - \frac{\gamma(1-\gamma)}{2} F^2\left(x\right) + \gamma F\left(v_i\right) - \frac{\gamma (1-\gamma)}{2} F^2\left(v_i\right)}{\frac{(1-\gamma)^2}{3}F\left(x\right) + \frac{\gamma(1-\gamma)}{2} F^2\left(x\right) - \frac{\gamma(1-\gamma)}{6} F^3\left(x\right) + \gamma F\left(v_i\right)F\left(x\right) - \frac{\gamma^2}{2} F^2\left(v_i\right) - \frac{\gamma (1-\gamma)}{2} F^2\left(v_i\right) F\left(x\right)} f(x)\\
=& \frac{\frac{(1-\gamma)^2}{3} + \gamma(1-\gamma) F\left(x\right) - \frac{\gamma(1-\gamma)}{2} F^2\left(x\right) + \gamma F\left(v_i\right) - \frac{\gamma (1-\gamma)}{2} F^2\left(v_i\right)}{\frac{(1-\gamma)^2}{3} + \frac{\gamma(1-\gamma)}{2} F\left(x\right) - \frac{\gamma(1-\gamma)}{6} F^2\left(x\right) + \gamma F\left(v_i\right) - \frac{\gamma^2}{2} \frac{F^2\left(v_i\right)}{F(x)} - \frac{\gamma (1-\gamma)}{2} F^2\left(v_i\right) } \frac{f(x)}{F(x)} \\
=&\left(1 + d(x\mid v_i;\gamma) \right)\frac{f(x)}{F(x)},
\end{align*}
where
\begin{align*}
    d(x\mid v_i;\gamma) = \frac{\frac{\gamma^2}{2} \frac{F^2(v_i)}{F(x)} + \frac{\gamma(1-\gamma)}{2} F\left(x\right) - \frac{\gamma(1-\gamma)}{3}F^2(x)}{\frac{(1-\gamma)^2}{3} + \frac{\gamma(1-\gamma)}{2} F\left(x\right) - \frac{\gamma(1-\gamma)}{6} F^2\left(x\right) + \gamma F\left(v_i\right) - \frac{\gamma^2}{2} \frac{F^2\left(v_i\right)}{F(x)} - \frac{\gamma (1-\gamma)}{2} F^2\left(v_i\right) }.
\end{align*}
Define 
\begin{align*}
    d_1(x\mid v_i;\gamma) =& \frac{\gamma^3}{2}F^2(v_i) -\frac{\gamma^2(1-\gamma)}{2}F^2(x) - \frac{\gamma(1-\gamma)}{3}F^3(x) \\
    & + \left(-\frac{\gamma^2(1-\gamma)^2}{3}F^3(x) + \frac{\gamma^2(1-\gamma)(1-2\gamma)}{2}F^2(x) + \gamma^3(1-\gamma)F(x) + \frac{\gamma^2(1-\gamma)^2}{3}\right)F(v_i),
\end{align*}
then we have 
\begin{align*}
\frac{\partial d(x\mid v_i;\gamma)}{\partial v_i} = \frac{d_1(x\mid v_i;\gamma)}{\frac{(1-\gamma)^2}{3} + \frac{\gamma(1-\gamma)}{2} F\left(x\right) - \frac{\gamma(1-\gamma)}{6} F^2\left(x\right) + \gamma F\left(v_i\right) - \frac{\gamma^2}{2} \frac{F^2\left(v_i\right)}{F(x)} - \frac{\gamma (1-\gamma)}{2} F^2\left(v_i\right)} \frac{f(v_i)}{F(x)}.   
\end{align*}
Since $\frac{\partial d_1(x\mid v_i;\gamma)}{\partial v_i}\geq 0$ and $d_1(x\mid v_i=0;\gamma) = -\frac{\gamma^2(1-\gamma)}{2}F^2(x) - \frac{\gamma(1-\gamma)}{3}F^3(x) \leq 0$, then we conclude that $d(x\mid v_i;\gamma)$ and $\frac{h(x\mid v_i;n=2,\gamma)}{H(x\mid v_i;n=2,\gamma)}$
either weakly decreasing or first weakly decreasing then weakly increasing with $v_i \in[0,x]$. 

Observe that 
\begin{align*}
&\frac{h(x\mid 0;n=2,\gamma)}{H(x\mid 0;n=2,\gamma)} - \frac{h(x\mid x;n=2,\gamma)}{H(x\mid x;n=2,\gamma)}\\
=& \frac{\frac{(1-\gamma)^2}{3} + \gamma(1-\gamma) F\left(x\right) - \frac{\gamma(1-\gamma)}{2} F^2\left(x\right) }{\frac{(1-\gamma)^2}{3}F\left(x\right) + \frac{\gamma(1-\gamma)}{2} F^2\left(x\right) - \frac{\gamma(1-\gamma)}{6} F^3\left(x\right) } - \frac{\frac{(1-\gamma)^2}{3} + \gamma(2-\gamma) F\left(x\right) - \gamma(1-\gamma)F^2\left(x\right) }{\frac{(1-\gamma)^2}{3}F\left(x\right) + \frac{\gamma(3-2\gamma)}{2} F^2\left(x\right) - \frac{2\gamma(1-\gamma)}{3} F^3\left(x\right) }\\
=&\frac{\gamma^2(1-\gamma)}{12}\frac{\left(-2(1-\gamma)+3(1-\gamma)F(x)-(7-6\gamma)F^2(x) +2(1-\gamma)F^3(x)\right)F^2(x)}{\left(\frac{(1-\gamma)^2}{3}F\left(x\right) + \frac{\gamma(1-\gamma)}{2} F^2\left(x\right) - \frac{\gamma(1-\gamma)}{6} F^3\left(x\right)\right)\left(\frac{(1-\gamma)^2}{3}F\left(x\right) + \frac{\gamma(3-2\gamma)}{2} F^2\left(x\right) - \frac{2\gamma(1-\gamma)}{3} F^3\left(x\right)\right)}\\
\leq &\frac{\gamma^2(1-\gamma)^2}{12}\frac{\left(-2+6F(x)-6F^2(x) +2F^3(x)\right)F^2(x)}{\left(\frac{(1-\gamma)^2}{3}F\left(x\right) + \frac{\gamma(1-\gamma)}{2} F^2\left(x\right) - \frac{\gamma(1-\gamma)}{6} F^3\left(x\right)\right)\left(\frac{(1-\gamma)^2}{3}F\left(x\right) + \frac{\gamma(3-2\gamma)}{2} F^2\left(x\right) - \frac{2\gamma(1-\gamma)}{3} F^3\left(x\right)\right)}\\
\leq & 0,
\end{align*}
hence $\frac{h(x\mid v_i;n=2,\gamma)}{H(x\mid v_i;n=2,\gamma)}$ must be first weakly decreasing then weakly increasing with $v_i \in[0,x]$, furthermore, we conclude that $\frac{h(x\mid v_i;n=2,\gamma)}{H(x\mid v_i;n=2,\gamma)} \leq \frac{h(x\mid x;n=2,\gamma)}{H(x\mid x;n=2,\gamma)}$.

\medskip

\noindent Combining (i) and (ii), we conclude that for any $v_i^\prime\leq \tilde{v}_i\leq v_i$, 
\begin{align*}  \frac{h(\tilde{v}_i\mid v_i^\prime;n=2,\gamma)}{H(\tilde{v}_i\mid v_i^\prime;n=2,\gamma)} 
\leq \frac{h(\tilde{v}_i\mid \tilde{v}_i;n=2,\gamma)}{H(\tilde{v}_i\mid \tilde{v}_i;n=2,\gamma)} 
\geq 
 \frac{h(\tilde{v}_i\mid v_i;n=2,\gamma)}{H(\tilde{v}_i\mid v_i;n=2,\gamma)}.
\end{align*}

\noindent\textit{\underline{Step 2: $\frac{h(x\mid v_i;n=2,\gamma)}{H(x\mid v_i;n=2,\gamma)}v_i$ is non-decreasing with $v_i \in [x,1]$ for $F(x) = x^c$ ($0< c\leq 1$).}}

\medskip

\noindent Similar to (i) of Step 1, we have 
\begin{align*}
\frac{h(x\mid v_i;n=2,\gamma)}{H(x\mid v_i;n=2,\gamma)}v_i 
=&\left(v_i + \frac{\left(\frac{\gamma}{2} F\left(x\right) - \frac{\gamma(1-\gamma)}{3} F^2\left(x\right)\right)v_i}{\frac{ (1-\gamma)^2}{3}  + \gamma (1-\gamma)F\left(v_{i}\right) - \frac{\gamma(1-\gamma)}{2} F^2\left(v_{i}\right) + \frac{\gamma}{2}  F\left(x\right)  - \frac{\gamma(1-\gamma)}{6}F^2\left(x\right)} \right)\frac{f(x)}{F(x)}\\
=&\left(v_i + \frac{\frac{\gamma}{2} F\left(x\right) - \frac{\gamma(1-\gamma)}{3} F^2\left(x\right)}{\frac{ (1-\gamma)^2}{3v_i}  + \gamma (1-\gamma)v_i^{c-1} - \frac{\gamma(1-\gamma)}{2} v_i^{2c-1} + \frac{\gamma}{2v_i}  F\left(x\right)  - \frac{\gamma(1-\gamma)}{6v_i}F^2\left(x\right)} \right)\frac{f(x)}{F(x)}.
\end{align*}
Observe that 
$\frac{ (1-\gamma)^2}{3v_i}  + \gamma (1-\gamma)v_i^{c-1} - \frac{\gamma(1-\gamma)}{2} v_i^{2c-1} + \frac{\gamma}{2v_i}  F\left(x\right)  - \frac{\gamma(1-\gamma)}{6v_i}F^2\left(x\right)$ is non-increasing with $v_i$ when $c\leq 1$. Hence $\frac{h(x\mid v_i;n=2,\gamma)}{H(x\mid v_i;n=2,\gamma)}v_i$ is non-decreasing with $v_i \in [x,1]$.
\end{proof}

%%%%%%%%%%%% proof of marginals_joint_g
\begin{proof}[{Proof of \Cref{lem:marginals_joint_g}}]
(i). Observe that by the proof of \Cref{lem:joint_dis_g}, 
\begin{align*}
    g^{\mar}(x ;n,\gamma)
 =
 \frac{\C^m_n}{\kappa(x;n,\gamma)} \cdot f(x) = (1-\gamma)f(x) + (m-n)C^m_n\gamma \cdot \sum_{i=0}^{n-1}\frac{(-1)^iC^{n-1}_i}{m-n+i} F^{m-n+i}(x)f(x),
\end{align*}
hence 
\begin{align*}
   G^{\mar}(x ; n,\gamma)  = (1-\gamma)F(x) + (m-n)C^m_n\gamma \cdot \sum_{i=0}^{n-1}\frac{(-1)^iC^{n-1}_i}{(m-n+i)(m-n+i+1)} F^{m-n+i+1}(x).
 \end{align*}
(ii). By definition, we have 
\begin{align*}
    G^{\lar}(x ; n,\gamma) = &\int_{[0,x]^n} g(v;n,\gamma) dv =  \int_{[0,x]^n}  \C^m_n \psi(v;n,\gamma)
 \prod_{j\in\I} f(v_j)dv\\
  =& \C^m_n \int_{ [0,x]^{n}}  \left(\int_0^1\prod_{i\in \I}\left[1-F(t;v_i,\gamma)\right] dF^{m-n}(t)\right)\prod_{j\in \I} f(v_j) dv\\
 =& \C^m_n\int_0^1 \prod_{j\in \I}\left(\int_0^x [1-F(t;v_j,\gamma)]dF(v_j)\right)dF^{m-n}(t).
\end{align*}
Observe that 
\begin{align*}
&\int_0^x [1-F(t;v_j,\gamma)]dF(v_j) \\= &\int_0^x \gamma \cdot \mathds{1}\{v_j>t\} + (1-\gamma)\cdot (1-F(t))dF(v_j) \\
=&\mathds{1}\{x\leq t\} (1-\gamma)F(x)(1-F(t)) + \mathds{1}\{x> t\}\left((1-\gamma)F(t)(1-F(t)) + \left(1-(1-\gamma)F(t)\right)(F(x) - F(t))\right)
\\
=&\mathds{1}\{x\leq t\} (1-\gamma)F(x)(1-F(t)) + \mathds{1}\{x> t\} \left(F(x) - F(t)(\gamma + (1-\gamma)F(x))\right),
\end{align*}
hence 
\begin{align*}
&G^{\lar}(x; n,\gamma)\\ = &\C^m_n\int_0^x  \left( F(x) - F(t)(\gamma + (1-\gamma)F(x)) \right)^ndF^{m-n}(t) + \C^m_n\int_0^1 (1-\gamma)^nF^n(x)(1-F(t))^n dF^{m-n}(t) \\
& - \C^m_n\int_0^x (1-\gamma)^nF^n(x)(1-F(t))^n dF^{m-n}(t)\\
=& (m-n)\C^m_n\int_0^{F(x)}  \sum_{i=0}^n \C^n_i (\gamma + (1-\gamma)F(x))^{i}F^{n-i}(x) t^{m-n + i-1}dt + \C^m_n\frac{(1-\gamma)^n}{\C^m_n}F^n(x)\\
& - (m-n)\C^m_n(1-\gamma)^n F^n(x) \int_0^{F(x)}  \sum_{i=0}^n \C^n_i (-1)^i t^{m-n+i-1}dt\\
=& (1-\gamma)^nF^n(x) + (m-n)C_m^nF^m(x)\sum_{i=0}^nC_n^i\frac{(-1)^i}{m-n+i}\left((\gamma + (1-\gamma)F(x))^i -(1-\gamma)^nF^i(x)\right),
\end{align*}
as desired. $g^{\mar}(x ;n,\gamma)$ then can be obtained by directly taking the derivative of $G^{\lar}(x; n,\gamma)$.
\end{proof}

%%%%%%%%%%%% proof of useful_identity
\begin{proof}[Proof of \Cref{lem:useful_identity}]
When $j\leq k-1$,
\begin{align*}
    &\int_{ [0,v_i]^{k-1}\times[v_i,x]^{n-k}}F^s\left(v_{(j)}\right) \prod_{\ell \in \I_{-i}}f(v_{\ell})dv_{-i} \\
    =& \frac{(k-1)!}{(j-1)!(k-1-j)!}\left(F(x)-F(v_i)\right)^{n-k}\int_0^{v_i}F^{s+j-1}\left(v_{(j)}\right)\left(F(v_i) - F\left(v_{(j)}\right)\right)^{k-1-j}dF\left(v_{(j)}\right)\\
    =& j\C_j^{k-1} \left(F(x)-F(v_i)\right)^{n-k} \int_0^{F(v_i)}t^{s+j-1}(F(v_i) - t)^{k-1-j}dt\\
    =& j\C_j^{k-1} F^{s+k-1}(v_i)\left(F(x)-F(v_i)\right)^{n-k} \int_0^1t^{s+j-1}(1-t)^{k-1-j}dt \\
    =& \frac{\C^{k-1}_{k-j}}{\C^{s+k-1}_{k-j}} F^{s+k-1}(v_i)\left(F(x)-F(v_i)\right)^{n-k}.
\end{align*}
When $k+1\leq j \leq n$,
\begin{align*}
    &\int_{[0,v_i]^{k-1}\times[v_i,x]^{n-k}}F^s\left(v_{(j)}\right) \prod_{\ell \in \I_{-i}}f(v_{\ell})dv_{-i} \\
    =& \frac{(n-k)!}{(n-j)!(j-k-1)!}F^{k-1}(v_i)\int_{v_i}^xF^{s}\left(v_{(j)}\right)\left( F\left(v_{(j)}\right) - F(v_i)\right)^{j-k-1}\left( F(x) - F\left(v_{(j)}\right)\right)^{n-j}dF\left(v_{(j)}\right)\\
    =& (j-k)\C_{n-j}^{n-k} F^{k-1}(v_i)\int_{F(v_i)}^{F(x)}t^{s}\left( t - F(v_i)\right)^{j-k-1}\left( F(x) - t\right)^{n-j}dt\\
    =& (j-k)\C_{n-j}^{n-k} F^{k-1}(v_i)\left(F(x)-F(v_i)\right)^{n-k} \int_0^1\left((F(x)-F(v_i))t+F(v_i)\right)^st^{j-k-1}(1-t)^{n-j}dt \\
    =& (j-k)\C_{n-j}^{n-k} F^{k-1}(v_i)\left(F(x)-F(v_i)\right)^{n-k} \sum_{r=0}^s \C_{r}^s \int_0^1(F(x)-F(v_i))^rF(v_i)^{s-r}t^{j+r-k-1}(1-t)^{n-j}dt\\
    =&\sum_{r=0}^s\frac{\C_{n-j+1}^{n-k}\C_r^s}{\C_{n-j+1}^{n+r-k}}F^{s+k-r-1}(v_i)(F(x)-F(v_i))^{n+r-k}.
\end{align*}
\end{proof}
