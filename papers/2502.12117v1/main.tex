\documentclass[11pt]{article}
\usepackage[utf8]{inputenc}
\usepackage[T1]{fontenc}
\usepackage[english]{babel}
\usepackage[]{algorithm2e}
\usepackage{mwe}
\usepackage{amsmath,mathtools}
\usepackage[dvipsnames]{xcolor}



\usepackage{times}
\usepackage{latexsym}

\usepackage[T1]{fontenc}

\usepackage[utf8]{inputenc}

\usepackage{microtype}

\usepackage{inconsolata}

\usepackage{graphicx}


\usepackage{amsmath}
\usepackage{amssymb}
\usepackage{multirow}
\usepackage{booktabs}
\usepackage{catchfile}

\usepackage[boxed]{algorithm}
\usepackage{varwidth}
\usepackage[noEnd=true,indLines=false]{algpseudocodex}
\usepackage{cleveref}
\makeatletter
\@addtoreset{ALG@line}{algorithm}
\renewcommand{\ALG@beginalgorithmic}{\small}
\algrenewcommand\alglinenumber[1]{\small #1:}
\makeatother

\usepackage[normalem]{ulem}
\usepackage{todonotes}

\usepackage{lipsum}    %
\usepackage{comment}   %
\usepackage{graphicx}  %
\usepackage{pifont}    %

\usepackage[font=small,labelfont=bf]{caption}
\usepackage{float}     %
\usepackage{booktabs}  %
\usepackage{subcaption}  %

\usepackage{listings}

\usepackage{amsthm}  %

%\usepackage[inline]{showlabels}

\usepackage{amsthm,bm}
\usepackage{lineno}
%\linenumbers


\usepackage{tikz,pgfplots}
\pgfplotsset{compat=1.5}
% \usepackage{subfigure}
\usetikzlibrary{intersections}
\usetikzlibrary{patterns}
\usepackage{multirow}
\usepackage{graphicx}



\usepackage{subcaption}
\usepackage{pgfpages}
\usepackage{pgfplotstable}

% \usepackage{subfig}



\usepackage[shortlabels]{enumitem}
\usepackage[colorlinks=true, linkcolor=blue, citecolor=blue, hypertexnames=false]{hyperref}  
\usepackage{nameref}
\usepackage{cleveref}
%\crefname{subsection}{subsection}{subsections}

\usepackage{color}              % Need the color package
% \usepackage[suppress]{color-edits}
\usepackage{color-edits}
\addauthor[Chiwei]{cy}{cyan}



\usepackage[right=1.0in, top=1in, bottom=1.0in, left=1.0in]{geometry}

\usepackage{lmodern}

\usepackage{multirow}

\usepackage{lscape}
\usepackage[final]{pdfpages}
\usepackage{amsthm}
\usepackage[authoryear,round]{natbib}
\usepackage{amsfonts}
\usepackage{mathtools}
\usepackage{bbm}
\usepackage{dsfont}

\usepackage{minitoc}
\renewcommand \thepart{}
\renewcommand \partname{}


\newcommand{\revision}[1]{{\color{red}{#1}}}

\usepackage{bbm}
\newtheorem{lemma}{Lemma}
\newtheorem{proposition}{Proposition}
\newtheorem{corollary}{Corollary}
\newtheorem{theorem}{Theorem}
\newtheorem*{theorem*}{Theorem}
\newtheorem{assumption}{Assumption}
\newtheorem{property}{Property}
\newtheorem{definition}{Definition}
\newtheorem{conjecture}{Conjecture}
\newtheorem{remark}{Remark}
\newtheorem{example}{Example}

\usepackage{authblk}

\usepackage{setspace}
\setstretch{1.35}






% \date{February 10, 2025}
\date{}


\title{The Role of Prescreening in Auctions with Predictions\thanks{The first two authors contributed equally.}}

\begin{document}

\author[1]{Yanwei Sun}
\author[2]{Fupeng Sun}
\author[3]{Chiwei Yan}
\author[4]{Jiahua Wu}
\affil[1]{Imperial College Business School, \texttt{yanwei@imperial.ac.uk}}
\affil[2]{Imperial College Business School, \texttt{f.sun23@imperial.ac.uk}}
\affil[3]{University of California, Berkeley, \texttt{chiwei@berkeley.edu}}
\affil[4]{Imperial College Business School,  \texttt{j.wu@imperial.ac.uk}}

\maketitle

\begin{abstract}

We consider an auction environment with i.i.d. privately known valuations. 
Equipped with a noisy predictor,
the auction designer receives a coarse signal about each player's valuation, where the signal is fully informative with a given probability. Based on the posterior expectation of the valuations, the designer selects the top players to admit—a procedure we call \emph{prescreening}. We show that this prescreening game is equivalent to a standard auction without prescreening but with \textit{correlated} types. Besides, when the signals are always fully informative, these correlated types are \textit{affiliated}. We characterize conditions for the existence of a symmetric and strictly monotone equilibrium strategy in both all-pay and first-price auctions. Our results reveal that prescreening can significantly improve the designer's revenue in all-pay auctions; in fact, when the prediction accuracy is one, admitting only two players is optimal. In contrast, prescreening is usually unnecessary in first-price auctions.
\end{abstract}

\doparttoc % Tell to minitoc to generate a toc for the parts
\faketableofcontents % Run a fake tableofcontents command for the partocs



\section{Introduction}

Despite the remarkable capabilities of large language models (LLMs)~\cite{DBLP:conf/emnlp/QinZ0CYY23,DBLP:journals/corr/abs-2307-09288}, they often inevitably exhibit hallucinations due to incorrect or outdated knowledge embedded in their parameters~\cite{DBLP:journals/corr/abs-2309-01219, DBLP:journals/corr/abs-2302-12813, DBLP:journals/csur/JiLFYSXIBMF23}.
Given the significant time and expense required to retrain LLMs, there has been growing interest in \emph{model editing} (a.k.a., \emph{knowledge editing})~\cite{DBLP:conf/iclr/SinitsinPPPB20, DBLP:journals/corr/abs-2012-00363, DBLP:conf/acl/DaiDHSCW22, DBLP:conf/icml/MitchellLBMF22, DBLP:conf/nips/MengBAB22, DBLP:conf/iclr/MengSABB23, DBLP:conf/emnlp/YaoWT0LDC023, DBLP:conf/emnlp/ZhongWMPC23, DBLP:conf/icml/MaL0G24, DBLP:journals/corr/abs-2401-04700}, 
which aims to update the knowledge of LLMs cost-effectively.
Some existing methods of model editing achieve this by modifying model parameters, which can be generally divided into two categories~\cite{DBLP:journals/corr/abs-2308-07269, DBLP:conf/emnlp/YaoWT0LDC023}.
Specifically, one type is based on \emph{Meta-Learning}~\cite{DBLP:conf/emnlp/CaoAT21, DBLP:conf/acl/DaiDHSCW22}, while the other is based on \emph{Locate-then-Edit}~\cite{DBLP:conf/acl/DaiDHSCW22, DBLP:conf/nips/MengBAB22, DBLP:conf/iclr/MengSABB23}. This paper primarily focuses on the latter.

\begin{figure}[t]
  \centering
  \includegraphics[width=0.48\textwidth]{figures/demonstration.pdf}
  \vspace{-4mm}
  \caption{(a) Comparison of regular model editing and EAC. EAC compresses the editing information into the dimensions where the editing anchors are located. Here, we utilize the gradients generated during training and the magnitude of the updated knowledge vector to identify anchors. (b) Comparison of general downstream task performance before editing, after regular editing, and after constrained editing by EAC.}
  \vspace{-3mm}
  \label{demo}
\end{figure}

\emph{Sequential} model editing~\cite{DBLP:conf/emnlp/YaoWT0LDC023} can expedite the continual learning of LLMs where a series of consecutive edits are conducted.
This is very important in real-world scenarios because new knowledge continually appears, requiring the model to retain previous knowledge while conducting new edits. 
Some studies have experimentally revealed that in sequential editing, existing methods lead to a decrease in the general abilities of the model across downstream tasks~\cite{DBLP:journals/corr/abs-2401-04700, DBLP:conf/acl/GuptaRA24, DBLP:conf/acl/Yang0MLYC24, DBLP:conf/acl/HuC00024}. 
Besides, \citet{ma2024perturbation} have performed a theoretical analysis to elucidate the bottleneck of the general abilities during sequential editing.
However, previous work has not introduced an effective method that maintains editing performance while preserving general abilities in sequential editing.
This impacts model scalability and presents major challenges for continuous learning in LLMs.

In this paper, a statistical analysis is first conducted to help understand how the model is affected during sequential editing using two popular editing methods, including ROME~\cite{DBLP:conf/nips/MengBAB22} and MEMIT~\cite{DBLP:conf/iclr/MengSABB23}.
Matrix norms, particularly the L1 norm, have been shown to be effective indicators of matrix properties such as sparsity, stability, and conditioning, as evidenced by several theoretical works~\cite{kahan2013tutorial}. In our analysis of matrix norms, we observe significant deviations in the parameter matrix after sequential editing.
Besides, the semantic differences between the facts before and after editing are also visualized, and we find that the differences become larger as the deviation of the parameter matrix after editing increases.
Therefore, we assume that each edit during sequential editing not only updates the editing fact as expected but also unintentionally introduces non-trivial noise that can cause the edited model to deviate from its original semantics space.
Furthermore, the accumulation of non-trivial noise can amplify the negative impact on the general abilities of LLMs.

Inspired by these findings, a framework termed \textbf{E}diting \textbf{A}nchor \textbf{C}ompression (EAC) is proposed to constrain the deviation of the parameter matrix during sequential editing by reducing the norm of the update matrix at each step. 
As shown in Figure~\ref{demo}, EAC first selects a subset of dimension with a high product of gradient and magnitude values, namely editing anchors, that are considered crucial for encoding the new relation through a weighted gradient saliency map.
Retraining is then performed on the dimensions where these important editing anchors are located, effectively compressing the editing information.
By compressing information only in certain dimensions and leaving other dimensions unmodified, the deviation of the parameter matrix after editing is constrained. 
To further regulate changes in the L1 norm of the edited matrix to constrain the deviation, we incorporate a scored elastic net ~\cite{zou2005regularization} into the retraining process, optimizing the previously selected editing anchors.

To validate the effectiveness of the proposed EAC, experiments of applying EAC to \textbf{two popular editing methods} including ROME and MEMIT are conducted.
In addition, \textbf{three LLMs of varying sizes} including GPT2-XL~\cite{radford2019language}, LLaMA-3 (8B)~\cite{llama3} and LLaMA-2 (13B)~\cite{DBLP:journals/corr/abs-2307-09288} and \textbf{four representative tasks} including 
natural language inference~\cite{DBLP:conf/mlcw/DaganGM05}, 
summarization~\cite{gliwa-etal-2019-samsum},
open-domain question-answering~\cite{DBLP:journals/tacl/KwiatkowskiPRCP19},  
and sentiment analysis~\cite{DBLP:conf/emnlp/SocherPWCMNP13} are selected to extensively demonstrate the impact of model editing on the general abilities of LLMs. 
Experimental results demonstrate that in sequential editing, EAC can effectively preserve over 70\% of the general abilities of the model across downstream tasks and better retain the edited knowledge.

In summary, our contributions to this paper are three-fold:
(1) This paper statistically elucidates how deviations in the parameter matrix after editing are responsible for the decreased general abilities of the model across downstream tasks after sequential editing.
(2) A framework termed EAC is proposed, which ultimately aims to constrain the deviation of the parameter matrix after editing by compressing the editing information into editing anchors. 
(3) It is discovered that on models like GPT2-XL and LLaMA-3 (8B), EAC significantly preserves over 70\% of the general abilities across downstream tasks and retains the edited knowledge better.
\section{Model, Notation, and Preliminaries}
\label{sec:model}
The goal of constructing an incident database is to determine whether some system that individuals interact with---for example, an (algorithmic) loan decision system, or a medical treatment---results in disproportionate harm to some meaningful subgroups. 
For the incident database associated with a particular system, we will use $\Badevent \in \{0,1\}$ as an indicator variable that denotes the undesirable event corresponding to that system. For example, in loan decisions, this could correspond to the event that a highly-qualified individual was denied a loan; in the medical setting, this may be an adverse physical side effect due to the treatment. 

\paragraph{Subgroup definitions.}
Individuals are characterized with feature vectors $X \in \X$, and we index individuals as 
$X_i$ (``features of individual $i$'') or 
$X_t$ (``features of the individual who reports at time $t$'').
Every individual $X_i$ ``belongs to'' at least one group $G$, and we will denote the event that $X_i$ belongs to $G$ as $\{X_i \in G\}$; we will use $\Groups$ to denote the set of all possible groups. 
This set of possible groups $\Groups$ can be defined arbitrarily as long as all groups can be determined as a function of covariates $\X$. We allow for groups to be overlapping---that is, we allow each individual $X_i$ to be in multiple groups so that $|\{G' \in \Groups: X_i \in G'\}| \geq 1$. 
\iftoggle{icml}{}{For example, it is possible to set $\Groups := 2^{\X}$ as in \citet{hebert2018multicalibration}. }

\paragraph{Reference population.}
The system for which the database is constructed naturally has a corresponding reference population of eligible individuals. For example, this could be everyone who has applied for a loan, or everyone who has been prescribed a certain medication. Thus, given a set of groups $\mathcal G$, we assume that it is possible to compute the composition of the reference population. 
\begin{assumption}[Reference population]
\label{assn:ref}
   For every $G \in \Groups$, the quantity $\Basegroup := \Pr[X \in G]$ is known. Throughout this work, we refer to the set $\{\Basegroup\}_{G \in \Groups}$ as \emph{base preponderances}.
\end{assumption}

\paragraph{Probabilistic model of reporting.}
As the database administrator, the high-level goal is to determine whether there exists some subgroup $G \in \Groups$ where $\Pr[\Badevent\mid X \in G]$ is abnormally high. 
Crucially, the database does not have access to information about every individual who interacts with the system; instead, individuals \textit{may} report to the database if they believe that they experienced bad event $\Badevent$. 
We thus let $R_i$ be a random variable representing whether individual $i$ decides to report (with $R_i = 0$ indicating no report). 

Each report $X_t$ is received sequentially, and assumed to be sampled i.i.d. from some underlying reporting distribution.\iftoggle{icml}{}{\footnote{Note that the events $\{X_t \in G\}$ and $\{X_t \in G'\}$ are correlated for any $G, G' \in \Groups$, i.e. the independence does not hold across groups. The key point in our case is independence across time.}}
Given a group $G$, we denote its corresponding mean among reports $\Pr[X_t \in G \mid R_t = 1]$ as  $\mu_G$.
We will sometimes refer to $\{\mu_G\}_{G \in \Groups}$ as (reporting) preponderances, as they represent the proportion of \textit{reports} that each $G$ comprises. A central claim of this paper is that comparing $\mu_G$ to $\Basegroup$---i.e., the extent to which group $G$ is (over)represented within the reporting database---can be a useful signal for $\Pr[Y \mid G]$ in a wide class of applications.\footnote
{Because we allow groups to overlap, we cannot enforce $\sum_G \Basegroup = 1$ or $\sum_G\mu_G = 1$.} \iftoggle{icml}{}{The i.i.d. model of course simplifies the analysis and exposition, but itself is not intrinsic to modeling the incident reporting problem as a sequential hypothesis test. As we discuss in Appendix \ref{subsec:practical}, the explicit i.i.d. assumption can be relaxed; more generally, any probabilistic model for sequential testing can be adapted to incident reporting. 
}
\section{Beliefs}
\label{sec:beliefs}


In this section, we characterize the admitted players' posterior beliefs, supposing that the admitted number $n$ is given. Let $\I\subseteq \I^0$ be a specific realized set of admitted players with $|\I|=n$. Given $\I$, let $\barI:=\I^0\setminus \I$ denote the set of eliminated players.

For an admitted player $i\in \I$, let $\I_{-i}$ denote the set of all other $n-1$ admitted players. Let $v_{-i}=(v_j)_{j\in \I_{-i}}$ and $v=(v_i,v_{-i})$ be the vectors of all and all other \textit{admitted} players' valuations, respectively. Finally, let $\beta(v_{-i}\mid \I,v_i;n,\gamma)$ be the admitted player $i$'s belief (joint density) about all other admitted players' valuations $V_{-i}$, conditional on the admission set $\I$ and player $i$'s own valuation $v_i$, when the prediction accuracy is $\gamma$.\footnote{Notice that, in the posterior beliefs, there is no need to conditional on the admitted number $n$ again since the admission set $\I$ contains the information of the admitted number.}





\begin{theorem}[Posterior Beliefs]
\label{thm:beta_posterior}
 The admitted player $i$'s posterior belief $\beta(v_{-i}\mid \I,v_i;n,\gamma)$ (joint density) about all other admitted players' private valuations $V_{-i}$ is
 \begin{align}\label{eq:beta_posterior}
 \beta(v_{-i} \mid \I,v_i;n,\gamma)
 = \underbrace{\kappa(v_i;n,\gamma)}_{\normalfont{\textrm{normalizing term}}}
\cdot 
\underbrace{\psi(v;n,\gamma)}_{\normalfont{\textrm{admission probability}}}
 \cdot 
 \underbrace{\prod_{j\in \I_{-i}} f(v_j)}_{\normalfont{\textrm{prior}}}
 ,~\forall v_{-i}\in [0,1]^{n-1}.
\end{align}
The normalizing term $\kappa(v_i;n,\gamma)$ satisfies $\int_{[0,1]^{n-1}}\beta(v_{-i}\mid \I,v_i;n,\gamma)dv_{-i}=1$.
The admission probability satisfies: for all $v\in [0,1]^n$,
\begin{align*}
\psi(v;n,\gamma)=  \Pr\left\{
\max_{j\in \barI} \zeta_j \leq \min_{i\in \I}\zeta_i
\mid v\right\} =
\int_0^1\prod_{i\in \I}[1-F(x;v_i,\gamma)] dF^{m-n}(x),
\end{align*}
where $F(x;v_i,\gamma):=\gamma\cdot \mathds{1}\{v_i\leq x\} + (1-\gamma)\cdot F(x) $ is the CDF of $\zeta_i$.
\end{theorem}



The admitted player $i$'s posterior belief has a clear structure: it is proportional to the prior multiplied by the admission probability $\psi(v;n,\gamma)$, which is the probability that all eliminated players' signals $\zeta_{\barI}$ are smaller than all admitted players' signals $\zeta_{\I}$, given the vector of admitted players' valuations $v$. That is $\psi(v;n,\gamma)=\Pr\{
\max_{j\in \barI} \zeta_j \leq \min_{i\in \I}\zeta_i \mid v\}.$



We provide closed-form expressions for the admission probability $\psi(v;n,\gamma)$ (as well as for the normalizing term $\kappa(v_i;n,\gamma)$) in \Cref{lem:closedforms_admission_prob} in \Cref{app_sec:auxiliary_results}. We find that the admission probability has the following structure:
\begin{align}
\label{eq:admissionprob_summation}
\psi(v;n,\gamma)=\hat{\psi}_0(n,\gamma)+\sum_{k=1}^n \hat{\psi}_k(v_{(k)};n,\gamma),
\end{align}
for some functions $(\hat{\psi}_k)_{k=0}^n$. Here, $v_{(k)}$ denotes the $k^{\mathsf{th}}$ smallest element in the vector $v$. Note that, given $n$ and $\gamma$, no crossing terms appear in $\psi(v;n,\gamma)$; it is simply equal to the summation of the individual terms.

In the special case of blind prescreening, i.e., $\gamma=0$, the selection is performed uniformly at random. In this case, we have $\psi(v;n,\gamma)=\frac{1}{\C^m_n},$
which is independent of $v$, and the posterior belief
$\beta(v_{-i}\mid \I,v_i;n,\gamma)=\prod_{j\in \I_{-i}}f(v_j)$
coincides with the prior.\footnote{When $\gamma=0$, we have $\psi(v;n,\gamma)=\Pr\{
\max_{j\in \barI} \zeta_j \leq \min_{i\in \I}\zeta_i \mid v\}
=\Pr\{
\max_{j\in \barI} \zeta_j \leq \min_{i\in \I}\zeta_i\},$
which is the probability that the minimum of $n$ i.i.d. random variables exceeds the maximum of $m-n$ i.i.d. random variables with the same distribution. This probability equals $\frac{1}{\C^m_n}$.}
Besides, for any prediction accuracy, when $n=m$, i.e., in the case of no prescreening, the posterior belief is the same as the prior.

For the case of perfect prescreening, i.e., $\gamma=1$, the admitted players are those with the highest valuations. In this case, we have
\begin{align}
\label{eq:gamma_1_admissionprob}
\psi(v;n,\gamma=1)
=\Pr\left\{
\max_{j\in \barI} \zeta_j \leq \min_{i\in \I}\zeta_i \mid v\right\}
=\Pr\left\{
\max_{j\in \barI} v_j \leq \min_{i\in \I}v_i \mid v\right\}
=F\left(\min_{i\in \I}v_i\right).
\end{align}





\begin{proposition}
\label{prop:admissionprob_property}
 The admission probability $\psi(v;n,\gamma)$ has the following properties:
 \begin{enumerate}[(i)]
    \item $\psi(v;n,\gamma)$ is symmetric in $v\in [0,1]^n$ and is differentiable almost everywhere.
  
 
     \item For any $i\neq j\in \I$, $\frac{\partial^2  \psi}{\partial v_i\partial v_j}$ exists almost everywhere and equals zero whenever it exists.
    
     \item Local Supermodularity: $\psi$ is supermodular in any domain $\mathcal{V}\subset 
     [0,1]^n$ in which for any $v,v^\prime \in \mathcal{V}$, the order of elements in $v$ is the same as the order of elements in $v^\prime$.


     \item When $\gamma=1$, $\psi$ is both supermodular and log-supermodular in $v\in [0,1]^n$.

 
 \end{enumerate}
\end{proposition}



\Cref{prop:admissionprob_property} (i) and (ii) come from the formula \eqref{eq:admissionprob_summation}. The symmetry is straightforward. The function is differentiable unless $v_i=v_j$ for some $i\neq j$, which is a measure-zero event. Thus, the admission probability is differentiable almost everywhere. \Cref{prop:admissionprob_property} (ii) follows from the fact that there is no crossing term in $\psi(v;n,\gamma)$ as shown in \eqref{eq:admissionprob_summation}.

In any domain $\mathcal{V}$ satisfying the condition in \Cref{prop:admissionprob_property} (iii), the ordering of $v\in \mathcal{V}$ does not change; hence, the formula \eqref{eq:admissionprob_summation} remains unchanged with respect to each element $v_i$. Thus, in this domain $\mathcal{V}$, the mixed partial derivative $\frac{\partial^2 \psi}{\partial v_i\partial v_j}=0$ for any $i\neq j$,
this leads to the supermodularity in the domain $\mathcal{V}$, i.e.,  local supermodularity.\footnote{If a function $\psi$ is twice-differentiable everywhere in the domain $\mathcal{V}$, then the supermodularity is equivalent to $\frac{\partial^2 \psi}{\partial v_i\partial v_j}\geq 0$ for any $i\neq j$.}
\Cref{prop:admissionprob_property} (iv) can be verified directly by equation \eqref{eq:gamma_1_admissionprob}.



By \Cref{prop:admissionprob_property} (i), player $i$'s posterior belief $\beta(v_{-i}\mid \I,v_i;n,\gamma)$ is symmetric in $v_{-i}\in [0,1]^{n-1}$. Consequently, its marginal distributions, regardless of the dimension, are identical; we denote this common marginal distribution by $\beta^{\mathsf{mar}}(\cdot \mid \I,v_i;n,\gamma)$. Notice that, unless $n=m$ or $\gamma=0$, in general, we have
\begin{align*}
\beta(v_{-i}\mid \I,v_i;n,\gamma) \neq \prod_{j\in \I_{-i}}\beta^{\mathsf{mar}}(v_j\mid \I,v_i;n,\gamma).
\end{align*}
In other words, player $i$'s posterior beliefs about the valuations of admitted players $j$ and $k$ are correlated.







\subsection{An Equivalent Game}
\label{subsec:equivalent_game}

Observe that by \eqref{eq:beta_posterior}, if two admitted players have the same valuation, then their posterior beliefs are identical. In other words, the posterior belief is independent of the individual identity and depends only on the player's valuation. Thus, one may wonder if there exists a \textit{symmetric} joint density over \textit{all admitted} players' valuations such that the conditional density derived from this joint density equals each admitted player's posterior belief \eqref{eq:beta_posterior}.
The answer is yes and it is unique.


\begin{lemma}
\label{lem:joint_dis_g}
There exists a unique symmetric joint density $g(\cdot;n,\gamma)$ over all admitted players's valuations $v\in[0,1]^n$ such that the conditional density $g(v_{-i}\mid v_i;n,\gamma)=\beta(v_{-i}\mid \I,v_i;n,\gamma)$ for any $i\in \I$ and any $v\in [0,1]^n$.
This symmetric joint density $g(\cdot ;n,\gamma)$ is given by
\begin{align}
\label{eq:joint_dist_g}
 g(v;n,\gamma) = \C^m_n\cdot \psi(v;n,\gamma)
 \cdot \prod_{j\in\I} f(v_j),~ \forall v\in [0,1]^n
 .   
\end{align}
\end{lemma}



Similar to \Cref{thm:beta_posterior}, the joint density $g(\cdot;n,\gamma)$ is proportional to the prior multiplied by the admission probability $\psi(v;n,\gamma)$. An interesting observation in \eqref{eq:joint_dist_g} is that the normalizing constant $\C^m_n$ is independent of both the prediction accuracy $\gamma$ and the prior $F$, whereas the normalizing constant $\kappa(v_i;n,\gamma)$ in \Cref{thm:beta_posterior} depends on both $\gamma$ and $F$ (see \Cref{lem:closedforms_admission_prob} in the appendix for the closed-form expressions of $\kappa(v_i;n,\gamma)$).

Now, consider a standard auction without prescreening but with \textit{correlated} valuations sampled from $g(\cdot;n,\gamma)$ in \eqref{eq:joint_dist_g}. In this setting, the vector of all $n$ players' valuations is jointly sampled from $g(\cdot;n,\gamma)$, and each player is informed only of her own valuation. Then, player $i$'s belief about the private valuations of the other $n-1$ players is given by the conditional density $g(v_{-i}\mid v_i;n,\gamma).$
Since $g(v_{-i}\mid v_i;n,\gamma)=\beta(v_{-i}\mid \I,v_i;n,\gamma),$
all participating players have the same beliefs in both settings. Thus, the equilibrium strategy and the equilibrium outcome are identical in the two settings. This observation leads to the following remark.
\begin{remark}
The prescreening game with i.i.d. prior $f$ is equivalent to a standard auction without prescreening but with the {\normalfont{joint}} distribution $g(\cdot;n,\gamma)$ in \eqref{eq:joint_dist_g}.
From the perspective of this equivalent game, prescreening allows the designer to select a joint distribution $g(\cdot;n,\gamma)$ while simultaneously reducing the number of participants to $n$.
\end{remark}




We have the following properties about the joint density $g(\cdot;n,\gamma)$.

\begin{proposition}
\label{prop:g_property}
The joint density $g(\cdot;n,\gamma)$ has the following properties:
\begin{enumerate}[(i)]
   
    \item When $\gamma\in \{0,1\}$, $g(v;n,\gamma)$ is affiliated in $v\in [0,1]^n$.
    \item When the prior $F$ is the uniform distribution, \Cref{prop:admissionprob_property} regarding the admission probability applies to $g(v;n,\gamma)$.
\end{enumerate}
\end{proposition}

The case of $\gamma=0$ in (i) is trivial since independent valuations are always affiliated.
The case of $\gamma=1$ follows from \Cref{prop:admissionprob_property} (iv), which shows that the admission probability $\psi(v;n,\gamma)$ is affiliated (log-supermodular) when $\gamma=1$, together with the fact that the product of two affiliated functions is also affiliated.\footnote{Note that the function $\prod_{i\in \I}f(v_i)$ is affiliated in $v\in [0,1]^n$.}
If the joint density $g(\cdot;n,\gamma)$ is affiliated, then a player with a high valuation $v_i$ will perceive that the other players are more likely to have high valuations rather than low valuations. Affiliation is a commonly used assumption in the auction literature \citep{krishna_1997_all_pay_affiliation,fang2002_GEB_affilited_secondprice,kotowski_2014_affiliated_allpay_budget}, as pioneered by \citet{milgrom_1982_auctiontheory_competitive_bidding}. 
% \yscomment{add more literatures about affiliated types here.}
However, beyond the cases of perfect and blind prescreening, the joint density is \textit{not} affiliated.
This makes the equilibrium analysis in our setting more challenging, as shown in the following sections.
\begin{remark}[Unaffiliation]
  When $\gamma\in (0,1)$, the joint density $g(\cdot;n,\gamma)$ is in general {\normalfont{not}} affiliated (unless $n=m$).  
\end{remark}





Based on \Cref{prop:g_property} (i), we have the following property about first-order stochastic dominance ($\fosd$).

\begin{proposition}[Stochastic Dominance]
\label{prop:stochastic_domiance_gamma_1}
In the case of perfect prescreening, i.e., $\gamma=1$, for any $v^\prime \geq v_i$, 
\begin{align*}
\beta^{\mar}(\cdot \mid \I,v_i^\prime;n,\gamma) \succeq_{\fosd} \beta^{\mar}(\cdot \mid \I,v_i;n,\gamma) \succeq_{\fosd} f.
\end{align*}
\end{proposition}

\Cref{prop:stochastic_domiance_gamma_1} states that an admitted player with a higher valuation perceives her opponents as being more likely to have higher valuations and that her marginal posterior belief stochastically dominates the prior.
By \Cref{prop:g_property} (i), we know that when $\gamma=1$, $g(v;n,\gamma)$ is affiliated in $v\in [0,1]^n$.
Since the conditioning and marginalization preserve affiliation \citep{karlin_1980_MTP2}, then the conditional density $g(v_{-i}\mid v_i;n,\gamma)$ is also affiliated in $v\in [0,1]^n$. Moreover, the marginal density of $g(v_{-i}\mid v_i;n,\gamma)$, i.e.,  $\beta^{\mar}(\cdot \mid \I,v_i;n,\gamma)$, is affiliated in $(x,v_i)\in [0,1]^2$. Then, the $\fosd$ follows by Proposition 3.1 in \citet{castro_2007_affiliation_positive_dependence}. The second inequality is established by verifying the definition of $\fosd$.

Let $H(\cdot\mid v_i;n,\gamma)$ be the CDF of the largest order statistic sampled from the conditional density $g(v_{-i}\mid v_i;n,\gamma)$, which is the same as $\beta(v_{-i} \mid \I,v_i;n,\gamma)$ by \Cref{lem:joint_dis_g}. Formally,
\begin{align}
\label{eq:H_CDF}
H(x\mid v_i;n,\gamma):=\int_{[0,x]^{n-1}} g(v_{-i}\mid v_i;n,\gamma) \, dv_{-i}.
\end{align}
Let $h(x\mid v_i;n,\gamma)=\frac{\partial H(x\mid v_i;n,\gamma)}{\partial x}$ be the corresponding density.
The closed-form expressions of $H(x\mid v_i;n,\gamma)$ and $h(x\mid v_i;n,\gamma)$ are provided in \Cref{lem:H_h} in \Cref{app_sec:auxiliary_results}.

A similar $\fosd$ result holds for $h(\cdot \mid v_i;n,\gamma)$ when $\gamma=1$, namely, for any $v^\prime \geq v_i$,
\begin{align*}
h(\cdot \mid v_i^\prime;n,\gamma) \succeq_{\fosd} h(\cdot \mid v_i;n,\gamma) \succeq_{\fosd} f.
\end{align*}
The first inequality follows a similar argument of \Cref{prop:stochastic_domiance_gamma_1} based on affiliation.
The second inequality holds since $h(\cdot \mid v_i;n,\gamma)$ first-order stochastically dominates the marginal density of $g^{\mar}(\cdot\mid v_i;n,\gamma)$, which is equal to $\beta^{\mar}(\cdot \mid \I,v_i;n,\gamma)$, and because $\fosd$ is transitive.

Before proceeding to the equilibrium analysis and characterizing the optimal admitted number, we introduce a few notations that will be used in both all-pay and first-price auctions. Since the joint density $g(\cdot;n,\gamma)$ is symmetric, its marginal densities, regardless of the dimension, are identical.
We denote this marginal CDF by $G^{\mathsf{mar}}(\cdot;n,\gamma)$. Let $G^{\lar}(\cdot; n,\gamma)$ be the CDF of the largest order statistic of random vector $V$ sampled from the joint density $g(\cdot;n,\gamma)$, i.e., it is the CDF of $\max_{i\in \I}V_i$ with $V\sim g(\cdot;n,\gamma)$.
The closed forms of $G^{\mathsf{mar}}(\cdot;n,\gamma)$ and $G^{\lar}(\cdot; n,\gamma)$ are provided in \Cref{lem:marginals_joint_g} in \Cref{app_sec:auxiliary_results}. 

%%%%%%%%%%%% All-pay Auctions
\section{All-pay Auctions}\label{sec:all-pay_auctions}





We first characterize the equilibrium strategy in \Cref{subsec:equilibrium_allpay} for a given admitted number $n$. We then discuss the optimal number $n^\ast$ in \Cref{subsec:optimal_number_allpay}.

%%%%%%%%%%%%%%%%%%%%%%%%% equilibrium analysis: allpay
\subsection{Equilibrium Analysis}
\label{subsec:equilibrium_allpay}

\begin{definition}[Equilibrium Definition]
\label{def:equilibrium_allpay}
For all-pay auctions, a strategy $\sigma:[0,1]\to \mathbb{R}_+$ is a symmetric Bayeisan-Nash equilibrium strategy if for all $i\in \I$ and all $v_i\in[0,1 ]$, 
\begin{align}
\label{eq:def_equilibrium_all_pay}
  \sigma(v_i)\in {\normalfont{\argmax}}_{b_i\in \mathbb{R}_+}~  \Pr\{\sigma(V_j)<b_i,\forall j\in \I_{-i}\} \cdot v_i - b_i,  
\end{align}
where $V_{-i}$ draws from player $i$'s belief $\beta(\cdot\mid \I,v_i;n,\gamma)$ given by \eqref{eq:beta_posterior}.
\end{definition}

We assume that the strategy $\sigma(\cdot)$ is differentiable, as is standard in the auction literature \citep{milgrom_1982_auctiontheory_competitive_bidding,milgrom_2004_putting_auctiontheory_to_work}.
When $\sigma(\cdot)$ is strictly increasing, the winning probability $\Pr\{\sigma(V_j)<b_i,\forall j\in \I_{-i}\}$
can be simplified as follows:
\begin{align*}
 \Pr\{\sigma(V_j)<b_i,\forall j\in \I_{-i}\}
  = \Pr\left\{ \max_{j\in \I_{-i}}V_j < \sigma^{-1}(b_i)\right\} 
  = H(\sigma^{-1}(b_i)\mid v_i;n,\gamma),
\end{align*}
where the second equality holds because $H(\cdot \mid v_i;n,\gamma)$ is the CDF of $\max_{j\in \I_{-i}}V_j$ by the definition \eqref{eq:H_CDF}.
By \eqref{eq:beta_posterior}, we can further write the winning probability as follows:
\begin{align*}
  H(\sigma^{-1}(b_i)\mid v_i;n,\gamma)
  =& \kappa(v_i;n,\gamma)\cdot \int_{[0,\sigma^{-1}(b_i)]^{n-1}} \psi(v;n,\gamma) \prod_{j\in \I_{-i}}f(v_j)dv_{-i}\\
  :=& \kappa(v_i;n,\gamma)\cdot \hat{H}(\sigma^{-1}(b_i)\mid v_i;n,\gamma).
\end{align*}
Plugging the above formulation back into \eqref{eq:def_equilibrium_all_pay}, we have player $i$'s utility when she bids $b_i$ while all other players follow the strategy $\sigma(\cdot)$ as follows:
\begin{align}
\label{eq:utility_inflatedtype_allpay}
 \hat{H}(\sigma^{-1}(b_i)\mid v_i;n,\gamma)\cdot \underbrace{\kappa(v_i;n,\gamma)v_i}_{\textrm{``inflated'' type}} - b_i.
\end{align}

From the perspective of \eqref{eq:utility_inflatedtype_allpay}, player $i$ appears to have a new type $\kappa(v_i;n,\gamma)v_i:=\theta(v_i;n,\gamma)$. We call it the \textit{inflated type} since $\kappa(v_i;n,\gamma)\geq \C^{m-1}_{n-1} \geq 1$ established by the following lemma.
\begin{lemma}
\label{lem:kappa_vi}
For any $n\in \{2,3,\cdots,m\}$ and any $\gamma\in[0,1]$, we have 
\begin{enumerate}[(i)]
    \item $\kappa(v_i;n,\gamma)$ is non-increasing in $v_i\in [0,1]$.
    \item $\kappa(v_i;n,\gamma)\geq \frac{\C^m_n}{(1-\gamma) + \gamma \frac{m}{n}} \geq \C^{m-1}_{n-1}$.
\end{enumerate}
\end{lemma}

Our next result provides sufficient conditions to guarantee the existence of a symmetric and strictly monotone\footnote{Obviously, there does not exist a strictly decreasing equilibrium strategy. The monotonicity here refers to increasing.} (SSM) equilibrium, and provides explicitly characterizations when it exists.

\begin{theorem}[Equilibrium Strategy]
\label{thm:strictlymonotone_equilibrium_allpay}
In all-pay auctions, for any $n\in \{2,3,\cdots,m\}$ and any $\gamma\in [0,1]$, 
\begin{enumerate}[(i)]
    \item There exists a symmetric and strictly monotone (SSM) equilibrium if $v_i h(\tilde{v}_i\mid v_i;n,\gamma) - \tilde{v}_i h(\tilde{v}_i\mid \tilde{v}_i;n,\gamma)$ is non-negative for $\tilde{v}_i\in [0,v_i]$ and non-positive for $\tilde{v}_i\in [v_i,1]$ for any given $v_i\in [0,1]$.
    \item If the inflated type $\theta(v_i;n,\gamma)$ is non-decreasing in $v_i$ for {\normalfont{all}} $v_i\in [0,1]$, the condition in (i) holds.
    \item If there exist SSM equilibrium strategies, they are unique and admit the following formula:
    \begin{align}
    \label{eq:SSM_equilibrium_all_pay}
        \sigma^{\mathsf{AP}}(v_i;n,\gamma)
  =\int_0^{v_i} x h(x\mid x;n,\gamma)dx,~ \forall v_i\in [0,1].
    \end{align}
\end{enumerate}
\end{theorem}

The expression $v_i h(\tilde{v}_i\mid v_i;n,\gamma) - \tilde{v}_i h(\tilde{v}_i\mid \tilde{v}_i;n,\gamma)$ in \Cref{thm:strictlymonotone_equilibrium_allpay} (i) is actually the derivative of $U^{\mathsf{AP}}(\tilde{v}_i,v_i;\sigma^{\mathsf{AP}}(\cdot;n,\gamma))$ with respect to $\tilde{v}_i$, where $U^{\mathsf{AP}}(\tilde{v}_i,v_i;\sigma^{\mathsf{AP}}(\cdot;n,\gamma))$ is the expected utility of player $i$ with valuation $v_i$ when she bids $\sigma^{\mathsf{AP}}(\tilde{v}_i;n,\gamma)$ for some $\tilde{v}_i$ while all other players follow the strategy $\sigma^{\mathsf{AP}}(\cdot;n,\gamma)$. The condition (i) guarantees that player $i$'s utility $U^{\mathsf{AP}}(\tilde{v}_i,v_i;\sigma^{\mathsf{AP}}(\cdot;n,\gamma))$ is first increasing in $\tilde{v}_i\in [0,v_i]$ and then decreasing in $\tilde{v}_i\in [v_i,1]$. That is, her utility is maximized when she bids $\sigma^{\mathsf{AP}}(v_i;n,\gamma)$. We want to comment that \Cref{thm:strictlymonotone_equilibrium_allpay} (i) holds beyond the prescreening setting: it applies to any private-valuation all-pay auction with arbitrarily correlated types.

\Cref{thm:strictlymonotone_equilibrium_allpay} (ii) provides an easier-to-verify sufficient condition related to the inflated type $\theta(v_i;n,\gamma)$. Combining \Cref{lem:kappa_vi} (i) that $\kappa(v_i;n,\gamma)$ is (weakly) decreasing, to make the inflated type $\theta(v_i;n,\gamma)=v_i\cdot \kappa(v_i;n,\gamma)$ (weakly) increasing, the term $\kappa(v_i;n,\gamma)$ should decrease sublinearly.

The uniqueness in \Cref{thm:strictlymonotone_equilibrium_allpay} (iii), if there indeed exist SSM equilibrium strategies, comes from the fact that there exists a unique solution of the ordinary differential equation originating from the first-order condition. The sufficient conditions in (i) and (ii) basically guarantee that the solution from the FOC is indeed a maximizer. When $n=m$, i.e., without prescreening, the problem degenerates to a standard i.i.d. all-pay auction with $m$ players. In such a situation, 
\begin{align*}
 h(\tilde{v}_i;v_i;n=m,\gamma)=mF^{m-1}(\tilde{v}_i)f(\tilde{v}_i),
\end{align*}
and condition (i) is always satisfied; thus, an SSM equilibrium strategy always exists when $m=n$. The strategy \eqref{eq:SSM_equilibrium_all_pay} in this situation degenerates to: for any prediction accuracy $\gamma\in [0,1]$,
\begin{align}
\label{eq:equilibrium_standard_allpay}
 \sigma^{\mathsf{AP}}(v_i;n=m,\gamma) = \int_0^{v_i}x\,dF^{m-1}(x),~ \forall v_i\in[0,1].   
\end{align}




We now discuss the special case $\gamma=1$ and show that the increasing property of the inflated type is also \textit{necessary} for the existence of an SSM equilibrium strategy.

\begin{proposition}[Perfect Prescreening]
\label{prop:SSM_gamma_1_allpay}
In all-pay auctions, when $\gamma=1$, for any $n\in \{2,3,\cdots,m\}$:
\begin{enumerate}[(i)]
    \item There exists an SSM equilibrium strategy {\normalfont{if and only if}} the inflated type $\theta(v_i;n,\gamma=1)$ is non-decreasing with $v_i\in [0,1]$.
    \item If there exist SSM equilibrium strategies, they are unique and admit the following formula:
    \begin{align}
    \label{eq:SSM_equilibrium_gamma_1_all_pay}
        \sigma^{\mathsf{AP}}(v_i;n,\gamma=1)
   = \frac{1}{\C^{m-1}_{n-1}}\int_0^{v_i}\theta(v_i;n,\gamma=1)dF^{m-1}(x)
   = \int_0^{v_i} \frac{x}{J(F(x),n,m)} dF^{m-1}(x),
    \end{align}
    where $J(F(x),n,m) = \C^{m-1}_{n-1}/\kappa(x;n,\gamma=1)$ satisfies the following properties:
    \begin{enumerate}
        \item $J(x,m,m)=1$ for any $x\in [0,1]$ and $J(x,n,m)\leq J(1,n,m) = 1$ for any $x\in[0,1]$ and any $n\in \{2,3,\cdots,m\}$.
        \item $J(x,n,m)$ is non-decreasing with $x\in [0,1]$ for any given $n\in \{2,3,\cdots,m\}$.
        \item $J(x,n,m)$ is non-decreasing with $n\in \{2,3,\cdots,m\}$ for any given $x\in[0,1]$.
    \end{enumerate}
\end{enumerate}
\end{proposition}




The necessity in \Cref{prop:SSM_gamma_1_allpay} (i) is based on the following key equation, as shown in the proof:
\begin{align}
\nonumber
 \frac{\partial U^{\mathsf{AP}}(\tilde{v}_i,v_i;\sigma^{\mathsf{AP}}(\cdot;n,\gamma=1))}{\partial \tilde{v}_i}
  \propto \theta(v_i;n,\gamma=1) - \theta(\tilde{v}_i;n,\gamma=1),\quad \forall \tilde{v}_i\in [0,v_i].
\end{align}
This equation shows that the derivative of player $i$'s utility (with respect to $\tilde{v}_i$) is, up to a constant, equal to the difference between the inflated type evaluated at $v_i$ and at $\tilde{v}_i$. Based on this, if there exists an interval on which $\theta(\cdot;n,\gamma=1)$ is strictly decreasing, then for a player with valuation $v_i$ equal to the right boundary of this interval, her utility is decreasing in $\tilde{v}_i$ in this interval.
Thus, this player would have an incentive to deviate from the equilibrium strategy. 
This leads to the necessity in (i).


When $\gamma=1$, the SSM equilibrium strategy (if it exists) given in \eqref{eq:SSM_equilibrium_gamma_1_all_pay} has a clear structure: all effects of prescreening are captured by the denominator $J(F(x),n,m)$ (observe that the only difference between \eqref{eq:SSM_equilibrium_gamma_1_all_pay} and \eqref{eq:equilibrium_standard_allpay} is the denominator $J(F(x),n,m)$), and this denominator is both (weakly) increasing in $x$ and in $n$.

\Cref{prop:SSM_gamma_1_allpay} (ii).(b) mainly comes from \Cref{lem:kappa_vi} (i), which shows that $\kappa(v_i;n,\gamma)$ is (weakly) decreasing in $v_i$. Thus, $J(x,n,m)$, being proportional to $1/\kappa(v_i;n,\gamma=1)$, is (weakly) increasing in $v_i$. This essentially means that the inflation effect in the inflated types of players $\theta(v_i;n,\gamma)=v_i\kappa(v_i;n,\gamma)$ diminishes as $v_i$ increases. Finally, \Cref{prop:SSM_gamma_1_allpay} (ii).(c), combined with \eqref{eq:SSM_equilibrium_gamma_1_all_pay}, implies that admitting fewer players would incentivize a participating player to submit a higher bid. This is one of the crucial properties used to characterize optimal prescreening, as discussed immediately in the following subsection.







%%%%% optimal prescreening mechanism for all-pay auctions
\subsection{Optimal Prescreening}
\label{subsec:optimal_number_allpay}





We focus on two metrics: expected revenue and expected highest bid.  
Given $(\gamma,F,m)$, let 
\begin{align}
\label{eq:rev_n_allpay}
 \R^{\AP}(n;\gamma,F,m) = n\cdot \mathbb{E}_{V_i\sim G^{\mar}(\cdot ; n,\gamma)}[\sigmaAP(V_i;n,\gamma)]
\end{align}
be the expected revenue when the admitted number is $n$.
Define the optimal expected revenue as
\begin{align*}
  \R_\ast^{\AP}(\gamma,F,m) := \max_{n=2,3,\cdots,m} \R^{\AP}(n;\gamma,F,m).
\end{align*}
Similarly, define the expected highest bid and the optimal expected highest bid as
\begin{align*}
\HB^{\AP}(n;\gamma,F,m) = \mathbb{E}_{V_i\sim G^{\lar}(\cdot ;n,\gamma)}[\sigmaAP(V_i;n,\gamma)],
\quad \textrm{and}
\quad 
 \HB_\ast^{\AP}(\gamma,F,m) := \max_{n=2,3,\cdots,m} \HB^{\AP}(n;\gamma,F,m).
\end{align*}
Note that the expectations
% \footnote{Recall that $G^{\mar}(\cdot;n,\gamma)$ is the marginal CDF of the joint distribution $g(\cdot;n,\gamma)$ given by \eqref{eq:joint_dist_g}, and $G^{\lar}(\cdot ;n,\gamma)$ is the CDF of the largest order statistic among variables sampled from $g(\cdot;n,\gamma)$.} 
are taken with respect to distributions related to the joint density $g(\cdot;n,\gamma)$ rather than over the prior distribution $F$. This is because the designer commits to the number $n$ \textit{before} any realizations occur. When making the decision about $n$, the designer has no additional information and must contemplate the entire game process.\footnote{A similar reasoning is also found in the settings of multi-stage all-pay auctions (contests) \citep{moldovanu_2006_contest_architecture} and in the Bayesian persuasion literature \citep{kamenica_2011_persuasion}.}
We may drop $m$ in the above notations when there is no confusion.



\begin{theorem}[Optimal Prescreening]
\label{thm:opt_admittednumber_allpay}
In all-pay auctions, given a total number of players $m$:
\begin{enumerate}[(i)]
\item When $\gamma=1$, suppose there exists an SSM equilibrium strategy for all $n\in \{2,\cdots,m\}$, the unique optimal solution is $n^\ast=2$ in terms of both the expected revenue and the expected highest bid.

\item When $\gamma=0$, 
\begin{enumerate}
    \item in terms of the expected revenue, $n^\ast=m$;
    \item in terms of the expected highest bid,
    for the power law prior distribution $F(x)=x^c,c>0$,
    if $c\in (0,3], n^\ast=m;$ if $c\in \left(3,\frac{5+\sqrt{33}}{2}\right), n^\ast=m \wedge \frac{2(c-1)^2}{c^2-3c}$;\footnote{When $\frac{2(c-1)^2}{c^2-3c}$ is not an integer, $n^\ast$ is either $m \wedge \bigg\lceil{\frac{2(c-1)^2}{c^2-3c}}\bigg\rceil$ or $n^* = m \wedge \bigg\lfloor{\frac{2(c-1)^2}{c^2-3c}}\bigg\rfloor$.} if $c\geq\frac{5+\sqrt{33}}{2}$, $n^\ast=2$.
\end{enumerate}

\item When $m=3$, for a power law prior distribution $F(x)=x^c,c>0$, suppose $c$ guarantees the existence of an SSM equilibrium strategy for all $\gamma\in[0,1]$ when $n=2$. Then, 
if it is optimal to admit only two players under a small $\gamma$, it is also optimal to admit only two players under a larger $\gamma^\prime>\gamma$.
\end{enumerate}
\end{theorem}



It is worth emphasizing that \Cref{thm:opt_admittednumber_allpay} (i) holds for any total number $m$ and any prior distribution. The result regarding the expected highest bid follows mainly from \Cref{prop:SSM_gamma_1_allpay}, from which we know that when $\gamma=1$, fewer admitted players induce higher bids from all participants. A formal proof is provided in the appendix.

Perhaps surprisingly, in terms of expected revenue, the optimal admitted number is also two when $\gamma=1$. Although admitting fewer players leads to higher bids by each individual (due to the inflated type), it also means that fewer players are paying the bid. At first glance, it is unclear which effect would dominate; however, \Cref{thm:opt_admittednumber_allpay} (i) suggests that the effect of the inflated type always dominates the reduction in the number of bidders.

We briefly discuss the proof of \Cref{thm:opt_admittednumber_allpay} (i) regarding the expected revenue. When $\gamma=1$, the expected revenue can be simplified as follows.
\begin{align*}
 \R^{\AP}(n;\gamma=1,F) & = n\cdot \mathbb{E}_{V_i\sim G^{\mar}(\cdot\mid n,\gamma)}[\sigmaAP(V_i;n,\gamma)]
 = n\int_0^1 \int_0^{v_i} \frac{x}{J(F(x),n,m)}\, dF^{m-1}(x)\, dG^{\mar}(v_i;n,\gamma)\\[1mm]
 &= \int_0^1 \frac{n}{J(F(x),n,m)} \cdot \left(1-G^{\mar}(x; n,\gamma)\right) \cdot x\, dF^{m-1}(x).
\end{align*}
The last equality holds by the Fubini theorem. Observe that all effects of prescreening on the expected revenue are captured by the term 
\[
\frac{n}{J(F(x),n,m)}\cdot \left(1-G^{\mar}(x;n,\gamma)\right).
\]
We discover a key monotonicity property of this term: for any given $x$, any $m$, and any prior distribution $F$, the term $\frac{n}{J(F(x),n,m)}\cdot \left(1-G^{\mar}(x;n,\gamma)\right)$ is decreasing in $n\in \{2,3,\cdots,m\}$.
This leads to $n^\ast=2$.

\Cref{thm:opt_admittednumber_allpay} (ii) discusses the situation of blind prescreening, i.e., $\gamma=0$. In this case, the beliefs of participating players do not change and an SSM equilibrium strategy always exists, which is the same as that in a standard auction with i.i.d.\ priors. By revenue equivalence in such a setting \citep{myerson_1981_optimal_auction}, the expected revenue of all-pay auctions equals the expected revenue in a second-price auction, which is the expectation of the second largest order statistic among $n$ i.i.d. random variables. It can be shown that this expectation increases with $n$. 
Thus, in terms of the expected revenue in the case of blind prescreening, $n^\ast=m$.
However, the expected highest bid in all-pay auctions is \textit{not} monotone in the number of participants $n$ and it relies heavily on the prior distribution. \Cref{thm:opt_admittednumber_allpay} (ii).(b) shows that when $c$ is small (i.e., the environment is less competitive), it is optimal to admit all players; when $c$ is large (i.e., the environment is very competitive), it is optimal to admit only two players. This implies that even when the prediction accuracy is zero, it can be strictly beneficial for the designer to conduct prescreening.


\Cref{thm:opt_admittednumber_allpay} (iii) is illustrated in \Cref{fig:m_3_allpay}. It shows that the optimal admitted number $n^\ast$ is weakly decreasing in $\gamma$. In addition, the left panel of \Cref{fig:m_3_allpay} indicates that the optimal expected revenue (represented by the color) is weakly increasing with the prediction accuracy $\gamma$. Specifically, in the region where $n^\ast=3$, the optimal expected revenue does not depend on $\gamma$, whereas when $n^\ast=2$, the optimal expected revenue is strictly increasing with $\gamma$. This observation roughly implies that admitting \textit{truly top} players improves the designer's revenue more. Furthermore, the optimal expected revenue is increasing in $c$, which is intuitive because a larger $c$ makes it more likely that bidders have higher valuations and, hence, are more inclined to bid more. Similar insights hold for the expected highest bid. In fact, by comparing the left and right panels of \Cref{fig:m_3_allpay}, we find that if it is optimal to admit only two players in terms of the expected revenue, then it is also optimal in terms of the expected highest bid.





\begin{figure}[ht]
    \centering
 \includegraphics[width=\textwidth]{fig/allpay/rev_hb_m_3.pdf}
    \caption{Optimal Prescreening for All-pay Auctions with $m=3$. 
    }
    \label{fig:m_3_allpay}
    \vspace{0.5em}
        Note.\textit{ The prior distribution is $F(x)=x^c,c\in(0,1]$, where $c\leq 1$ is to guarantee there exists an SSM equilibrium strategy for all $n\in\{2,3\}$ and for {\normalfont{all}} $\gamma\in[0,1]$.
    The color represents the optimal expected revenue $\R_\ast^{\AP}(\gamma,c)$ in the left panel (the optimal highest bid $\HB_\ast^{\AP}(\gamma,c)$ in the right panel) given each $(\gamma,c)$.}
\end{figure}





Although analytically characterizing the optimal admitted number remains elusive for settings with general $\gamma$ and $m$, it is computationally tractable, as explained by our following remark.


\begin{figure}[ht]
    \centering
    \begin{subfigure}[b]{0.49\textwidth} % Set the width to slightly less than half the text width
    \includegraphics[width=\textwidth]{fig/allpay/rev_hb_m_4.pdf}
    \caption{$m=4$}
    \label{fig:rev_m_4_allpay}
    \end{subfigure}
    \hfill 
    \begin{subfigure}[b]{0.49\textwidth}
    \includegraphics[width=\textwidth]{fig/allpay/rev_hb_m_5.pdf}
    \caption{$m=5$}
    \label{fig:hb_m_5_allpay}
    \end{subfigure}

\begin{subfigure}[b]{0.49\textwidth}
    \includegraphics[width=\textwidth]{fig/allpay/rev_hb_m_6.pdf}
    \caption{$m=6$}
    \label{fig:hb_m_6_allpay}
    \end{subfigure}
    \hfill
    \begin{subfigure}[b]{0.49\textwidth}
    \includegraphics[width=\textwidth]{fig/allpay/rev_hb_m_7.pdf}
    \caption{$m=7$}
    \label{fig:hb_m_7_allpay}
    \end{subfigure}
    
    \caption{Optimal Prescreening for All-pay Auctions. 
    }
    \label{fig:general_m_allpay}
\vspace{0.5em}    
      Note. \textit{The prior distribution is $F(x)=x^c,c\in(0,1/(m-2)]$ when the total number is $m$. The condition $c\leq 1/(m-2)$ is to guarantee there exists an SSM equilibrium strategy for all $n\in\{2,3,\cdots,m\}$ and for {\normalfont{all}} $\gamma\in[0,1]$.
    The color represents the optimal expected revenue $\R_\ast^{\AP}(\gamma,c)$ in the left panel (optimal highest bid $\HB_\ast^{\AP}(\gamma,c)$ in the right panel) given each $(\gamma,c)$.}
\end{figure}


\begin{remark}[Computational Tractability]
\label{remark:generalsetting_allpay}
By exchanging the order of integration, we can simplify the expected revenue in \eqref{eq:rev_n_allpay} as follows:
\begin{align*}
   \R^{\AP}(n;\gamma,F) = \int_0^1 x\, h(x\mid x;n,\gamma)\cdot \left(1-G^{\mar}(x;n,\gamma)\right)dx,
\end{align*}
where we have the closed-form expressions of $h(x\mid x;n,\gamma)$ and $G^{\mar}(x;n,\gamma)$ provided in \Cref{app_sec:auxiliary_results}. Thus, to find the optimal admitted number $n$, we only need to compute the \emph{one-dimensional} integral for each $n\in\{2,\ldots,m\}$, which is computationally tractable. 
\emph{However, without our closed-form expressions of $h(x\mid x;n,\gamma)$ and $G^{\mar}(x;n,\gamma)$, $\R^{\AP}(n;\gamma,F)$ can involve up to $(m-1)^2$-dimensional integrals, 
% (see the proof for detailed arguments), 
which are intractable especially for large $n$.}
Notice that, with our closed forms, no matter what $m$ is, we only need to compute \emph{one-dimensional} integral (for $m-1$ times).
A similar logic applies to the expected highest bid.
\end{remark}




We provide further numerical results in \Cref{fig:general_m_allpay}, where we adopt the power law prior distribution and restrict $c$ to the range that guarantees the existence of an SSM equilibrium strategy for \textit{all} $\gamma\in[0,1]$ (i.e., for small $c$). The prescreening mechanism can improve the revenue substantially. For example, when $m=7$, $c=0.2$, and $\gamma=1$, admitting only two players increases revenue by 31\% compared with admitting all players. 
Another interesting observation is that in terms of both expected revenue and expected highest bid, when $c$ is small, optimal prescreening often involves either admitting all players ($n=m$) or admitting only two players ($n=2$). However, although admitting $n\notin\{2,m\}$ does not simultaneously outperform both extremes, it can yield a higher outcome than one of the extremes even when $c$ is small. Moreover, if we consider the $\gamma$-dependent SSM existence condition, there exist situations in which an SSM equilibrium strategy exists but the optimal admitted number satisfies $n^\ast\notin \{2,m\}$ (see \Cref{thm:opt_admittednumber_allpay} (ii) for an example).






\section{First-price Auctions}
\label{sec:firstprice_auctions}




We now proceed to first-price auctions. We first characterize the equilibrium and then discuss the optimal prescreening.
The equilibrium definition for first-price auctions is almost the same as \Cref{def:equilibrium_allpay} for all-pay auctions except for replacing \eqref{eq:def_equilibrium_all_pay} with
\begin{align*}
  \Pr\{\sigma(V_j)<b_i,\forall j\in \I_{-i}\} \cdot \left[v_i - b_i  \right].
\end{align*}
That is, only the winner makes the payment.

Define
\begin{align}
\label{eq:L}
    L(x\mid v_i;n,\gamma)
 :=\exp\left(\int_{v_i}^x \frac{h(t\mid t;n,\gamma)}{H(t\mid t;n,\gamma)}dt\right)
 =\exp\left(-\int_{x}^{v_i} \frac{h(t\mid t;n,\gamma)}{H(t\mid t;n,\gamma)}dt\right),~\forall x\in [0,v_i].
\end{align}
Observe that $L(x\mid v_i;n,\gamma)$ is increasing in $x$ and $L(v_i\mid v_i;n,\gamma) =1$.
Besides, define
\begin{align}
\FP(\tilde{v}_i,v_i;n,\gamma) :=
h(\tilde{v}_i\mid v_i;n,\gamma) \left[v_i - \tilde{v}_i\right]
 +H(\tilde{v}_i\mid v_i;n,\gamma) \int_0^{\tilde{v}_i}L(x\mid \tilde{v}_i;n)dx
\left[
 \frac{h(\tilde{v}_i\mid v_i;n,\gamma)}{H(\tilde{v}_i\mid v_i;n,\gamma)}  - \frac{h(\tilde{v}_i\mid \tilde{v}_i;n,\gamma)}{H(\tilde{v}_i\mid \tilde{v}_i;n,\gamma)}
 \right].\nonumber   
\end{align}

\begin{theorem}[Equillibrium Strategy]
\label{thm:SSM_firstprice}
In first-price auctions:
\begin{enumerate}[(i)]
    \item There exists an SSM equilibrium strategy if $\FP(\tilde{v}_i,v_i;n,\gamma)$ is non-negative for $\tilde{v}_i\in [0,v_i]$ and non-positive for $\tilde{v}_i\in [v_i,1]$ for any $v_i\in [0,1]$.
    \item If there exist SSM equilibrium strategies, they are unique and admit the following formula:
    \begin{align}
    \label{eq:SSM_equilibrium_firstprice}
     \sigma^{\FP}(v_i;n,\gamma) = v_i - \int_0^{v_i}L(x\mid v_i;n,\gamma)dx,~ \forall v_i \in [0,1] .  
    \end{align}
    \item When $\gamma=1$, the condition in (i) always holds and thus there exists an SSM equilibrium strategy. Besides, the equilibrium strategy is the same as the standard first-price auctions without prescreening, i.e.,
    \begin{align*}
     \sigma^{\FP}(v_i;n,\gamma=1) = \sigma^{\FP}(v_i;n=m,\gamma) = v_i - \int_0^{v_i} \frac{F^{m-1}(x)}{F^{m-1}(v_i)}  dx,~\forall v_i\in [0,1].
    \end{align*}
    \item For the power law prior distribution $F(x)=x^c$, if $m=3$ and $c\in (0,1]$, the condition in (i) holds for any $n\in \{2,m\}$. 
\end{enumerate}
\end{theorem}

Let $U^{\FP}(\tilde{v}_i,v_i;\sigma^{\FP}(\cdot;n,\gamma))$ be the utility of player $i$ with valuation $v_i$ when she bids $\sigma^{\FP}(\tilde{v}_i;n,\gamma)$ for some $\tilde{v}_i\in [0,1]$ while all other participating players follow the strategy $\sigma^{\FP}(\cdot;n,\gamma)$ given by \eqref{eq:SSM_equilibrium_firstprice}. The term $\FP(\tilde{v}_i,v_i;n,\gamma)$ is actually the derivative of the utility $U^{\FP}(\tilde{v}_i,v_i;\sigma^{\FP}(\cdot;n,\gamma))$ with respect to $\tilde{v}_i$, that is,
\begin{align*}
  \FP(\tilde{v}_i,v_i;n,\gamma) =\frac{\partial U^{\FP}(\tilde{v}_i,v_i;\sigma^{\FP}(\cdot;n,\gamma))}{\partial \tilde{v}_i}.
\end{align*}
Thus, the condition in \Cref{thm:SSM_firstprice} (i) guarantees that player $i$'s utility $U^{\FP}(\tilde{v}_i,v_i;\sigma^{\FP}(\cdot;n,\gamma))$ increases in $\tilde{v}_i\in [0,v_i]$ and decreases in $\tilde{v}_i\in [v_i,1]$, which implies that it is optimal for player $i$ to follow the strategy $\sigma^{\FP}(\cdot;n,\gamma)$. This certifies $\sigma^{\FP}(\cdot;n,\gamma)$ as an equilibrium.

\Cref{thm:SSM_firstprice} (ii) is proven by solving the first-order condition (FOC) and showing the uniqueness of its solution. Notice that a strategy is certified as an equilibrium if, for any player with \textit{any} valuation $v_i$, there is no incentive to deviate when all other players follow this strategy. Based on this, it is easy to show that the FOC condition is necessary for a strategy to be an equilibrium.

The first part in \Cref{thm:SSM_firstprice} (iii) regarding the existence of an SSM equilibrium strategy comes from \Cref{prop:g_property}, which shows that when $\gamma=1$, the joint density $g(\cdot;n,\gamma)$ is affiliated in $v\in [0,1]^n$. Any private-value first-price auction with affiliated types admits an SSM equilibrium as established by \citet{milgrom_1982_auctiontheory_competitive_bidding}.

Perhaps surprisingly, the second part in \Cref{thm:SSM_firstprice} (iii) shows that the equilibrium strategy does \textit{not} depend on the admitted number when $\gamma=1$. This mainly comes from the fact that when $\gamma=1$, 
\begin{align*}
 H(t\mid t;n,\gamma=1) = \frac{F^{m-1}(t)}{J(F(t),n,m)}   \quad
 \textrm{and} \quad
  h(t\mid t;n,\gamma=1) = \frac{(F^{m-1}(t))^\prime}{J(F(t),n,m)},
\end{align*}
where the function $J(F(t),n,m)$ is defined in \Cref{prop:SSM_gamma_1_allpay}. By the definition of \eqref{eq:H_CDF}, $H(t\mid t;n,\gamma=1)$ is the winning probability of a player with valuation $t$ when all players (including him) follow a strictly monotone strategy. Although this winning probability is inflated by the denominator $J(F(t),n,m)\leq 1$ (note that $H(t\mid t;n=m,\gamma)=F^{m-1}(t)$), the term $h(t\mid t;n,\gamma=1)$ is also inflated by the same denominator. Thus, the fraction $\frac{h(t\mid t;n,\gamma=1)}{H(t\mid t;n,\gamma=1)}$ does not depend on $n$, leading to $L(x\mid v_i;n,\gamma)$ in \eqref{eq:L} also being independent of $n$.
As a result, the equilibrium strategy in \eqref{eq:SSM_equilibrium_firstprice} does not depend on $n$.



\Cref{thm:SSM_firstprice} (iv) shows that when the environment is not very competitive (i.e., for small $c$), for \textit{any} $\gamma\in[0,1]$, there exists an SSM equilibrium strategy when $n=2$ and $m=3$.\footnote{For any $\gamma$, $m$, and prior distribution, when $n=m$, there exists an SSM equilibrium strategy.} We emphasize that this result is non-trivial because it violates all the standard conditions used in the literature, yet an SSM equilibrium strategy still exists. Specifically, \citet{milgrom_1982_auctiontheory_competitive_bidding} shows that an SSM equilibrium strategy exists when the joint density $g(\cdot;n,\gamma)$ is affiliated. However, as mentioned in \Cref{subsec:equivalent_game}, when $\gamma\in (0,1)$, the joint density is \textit{not} affiliated. Furthermore, \citet{castro_2007_affiliation_positive_dependence} establishes a weaker condition for the existence of an SSM equilibrium, namely that $\frac{h(x\mid v_i;n,\gamma)}{H(x\mid x;n,\gamma)}$ is weakly increasing in $v_i\in [0,1]$ for any given $x\in [0,1]$ (referred to as the \textit{hazard rate increasing condition}). However, this condition also fails in our setting. 
To prove \Cref{thm:SSM_firstprice} (iv), we use the inequality that $L(x\mid v_i;n,\gamma)\leq 1$.
Since this bound is independent of the prior, we naturally require the restriction of prior distributions; the condition we find that works under this bound is $c\leq 1$. Nevertheless, our numerical results show that an SSM equilibrium strategy always exists.




Based on \Cref{thm:SSM_firstprice}, we are able to characterize the optimal admitted number in first-price auctions.

\begin{theorem}[Optimal Prescreening]
\label{thm:opt_admittednumber_firstprice}
In first-price auctions, in terms of the expected revenue, 
\begin{enumerate}[(i)]
    \item for $\gamma=0$, it is optimal to admit all players, i.e., $n^\ast=m$; 
    \item for $\gamma=1$, admitting any number of players yields the same expected revenue.
\end{enumerate}
\end{theorem}

The case of $\gamma=0$ in first-price auctions is the same as the all-pay auctions by revenue equivalence in the i.i.d. setting \citep{myerson_1981_optimal_auction}. In contrast, when $\gamma=1$, first-price auctions exhibit a sharp difference from all-pay auctions: in first-price auctions, prescreening does not affect the outcome and hence need not be conducted, whereas in all-pay auctions it is optimal to admit only two players by \Cref{thm:opt_admittednumber_allpay}.
In general settings beyond $\gamma\in \{0,1\}$, our numerical results show that it is usually optimal to admit all players, see \Cref{app_subsec:numerical} for details. 






\section{Revenue Ranking}
\label{sec:rev_ranking}





We now compare the optimal revenue in all-pay auctions and the optimal revenue in first-price auctions.
Given $(\gamma,F,m)$, let $\R^{\FP}(n;\gamma,F,m)$ be the revenue in first-price auctions when admitting $n$ players. 
Let $\R_\ast^{\FP}(\gamma,F,m):=\max_{n\in \{2,\cdots,m\}}\R^{\FP}(n;\gamma,F,m)$ be the revenue under optimal $n$.

\begin{theorem}[Revenue Ranking]
\label{thm:rev_ranking}
When $\gamma=1$, suppose there exists an SSM equilibrium strategy under all-pay auctions.
\begin{enumerate}[(i)]
    \item $\R_\ast^{\AP}(\gamma=1,F,m)>\R_\ast^{\FP}(\gamma=1,F,m)=\R^{\FP}(n=m;\gamma,F,m)$.
    That is, the revenue of all-pay auctions with optimal prescreening is {\normalfont{strictly}} higher than the revenue of first-price auctions with optimal prescreening in the case of $\gamma=1$.
    
    \item When $m$ is large enough,  $\R_\ast^{\AP}(\gamma=1,F,m)> \R^{\FP}(n=m+1;\gamma,F,m+1)$.
    That is, the revenue of all-pay auctions with optimal prescreening is {\normalfont{strictly}} higher than the revenue of first-price auctions without prescreening but with one additional bidder when $m$ is large enough and $\gamma=1$.
\end{enumerate}
\end{theorem}



Regarding (i), notice that when admitting all $m$ players, the joint distribution in the equivalent game (see \Cref{subsec:equivalent_game}) is the same as the prior beliefs. By the revenue equivalence theorem in the independent private-valuation setting \citep{myerson_1981_optimal_auction}, we have 
$\R^{\FP}(n=m;\gamma,F,m)=\R^{AP}(n=m;\gamma,F,m)$. By \Cref{thm:opt_admittednumber_allpay}, we know that 
$\R_\ast^{\AP}(\gamma=1,F,m)=\R^{AP}(n=2;\gamma=1,F,m)>\R^{AP}(n=m;\gamma=1,F,m)$. Also, by \Cref{thm:opt_admittednumber_firstprice}, prescreening does not improve the expected revenue in first-price auctions. Combining the above properties, we have (i).

\Cref{thm:rev_ranking} (ii) shows that in the case of perfect prescreening, when the total number $m$ is large enough, the revenue of all-pay auctions with the optimal admitted number is strictly larger than the revenue in first-price auctions with even one additional bidder. Combining this with the Bulow-Klemperer result \citep{bulow_1994_auctions_negotiations}—which states that (in the absence of prescreening) the revenue of a first-price auction with $m+1$ bidders is greater than the revenue of the optimal auction format with $m$ bidders when the prior $F$ is regular—we have the following remark.
\begin{remark}
 If the prior $F$ is regular, i.e., the virtual value is non-decreasing, then when $\gamma=1$ and the total number $m$ is large enough, the revenue of all-pay auctions when admitting only two players is greater than the revenue of the optimal auction format (i.e., the second-price auction with a reserve price) when admitting all players.
\end{remark}

We highlight that \Cref{thm:rev_ranking} (ii) does \textit{not} require the regularity condition on the prior distribution $F$ as long as $m$ is large enough and an SSM equilibrium strategy exists. This is illustrated in \Cref{fig:rev_comparison}, where the prior distributions are irregular. Besides, as we can see, the total number $m$ does \textit{not} need to be very large to guarantee that \Cref{thm:rev_ranking} (ii) holds.
We provide additional numerical results in \Cref{app_subsec:numerical} beyond the case of perfect prescreening, where we find a similar revenue-ranking result holds.







\begin{figure}[ht]
    \centering
    % Resize the imported TikZ figure to the width of the text
    \resizebox{0.8\textwidth}{!}{%
        \input{fig/revenue_comparison/gamma=1/rev_comparison.tikz}%
    }
    \caption{Revenue Comparison}
    \label{fig:rev_comparison}
\end{figure}

\section{Conclusion}

In this paper, we introduce STeCa, a novel agent learning framework designed to enhance the performance of LLM agents in long-horizon tasks. 
STeCa identifies deviated actions through step-level reward comparisons and constructs calibration trajectories via reflection. 
These trajectories serve as critical data for reinforced training. Extensive experiments demonstrate that STeCa significantly outperforms baseline methods, with additional analyses underscoring its robust calibration capabilities.

%\newpage

%\bibliographystyle{agsm}
\bibliographystyle{plainnat}
\bibliography{ref}

\newpage

\appendix

\renewcommand{\theequation}{\thesection-\arabic{equation}}
\renewcommand{\theproposition}{\thesection-\arabic{proposition}}
\renewcommand{\thelemma}{\thesection-\arabic{lemma}}
\renewcommand{\thetheorem}{\thesection-\arabic{theorem}}
\renewcommand{\thedefinition}{\thesection-\arabic{definition}}
\pagenumbering{arabic}
\renewcommand{\thepage}{App-\arabic{page}}

\setcounter{equation}{0}
\setcounter{proposition}{0}
\setcounter{definition}{0}
\setcounter{lemma}{0}
\setcounter{theorem}{0}

\subsection{Lloyd-Max Algorithm}
\label{subsec:Lloyd-Max}
For a given quantization bitwidth $B$ and an operand $\bm{X}$, the Lloyd-Max algorithm finds $2^B$ quantization levels $\{\hat{x}_i\}_{i=1}^{2^B}$ such that quantizing $\bm{X}$ by rounding each scalar in $\bm{X}$ to the nearest quantization level minimizes the quantization MSE. 

The algorithm starts with an initial guess of quantization levels and then iteratively computes quantization thresholds $\{\tau_i\}_{i=1}^{2^B-1}$ and updates quantization levels $\{\hat{x}_i\}_{i=1}^{2^B}$. Specifically, at iteration $n$, thresholds are set to the midpoints of the previous iteration's levels:
\begin{align*}
    \tau_i^{(n)}=\frac{\hat{x}_i^{(n-1)}+\hat{x}_{i+1}^{(n-1)}}2 \text{ for } i=1\ldots 2^B-1
\end{align*}
Subsequently, the quantization levels are re-computed as conditional means of the data regions defined by the new thresholds:
\begin{align*}
    \hat{x}_i^{(n)}=\mathbb{E}\left[ \bm{X} \big| \bm{X}\in [\tau_{i-1}^{(n)},\tau_i^{(n)}] \right] \text{ for } i=1\ldots 2^B
\end{align*}
where to satisfy boundary conditions we have $\tau_0=-\infty$ and $\tau_{2^B}=\infty$. The algorithm iterates the above steps until convergence.

Figure \ref{fig:lm_quant} compares the quantization levels of a $7$-bit floating point (E3M3) quantizer (left) to a $7$-bit Lloyd-Max quantizer (right) when quantizing a layer of weights from the GPT3-126M model at a per-tensor granularity. As shown, the Lloyd-Max quantizer achieves substantially lower quantization MSE. Further, Table \ref{tab:FP7_vs_LM7} shows the superior perplexity achieved by Lloyd-Max quantizers for bitwidths of $7$, $6$ and $5$. The difference between the quantizers is clear at 5 bits, where per-tensor FP quantization incurs a drastic and unacceptable increase in perplexity, while Lloyd-Max quantization incurs a much smaller increase. Nevertheless, we note that even the optimal Lloyd-Max quantizer incurs a notable ($\sim 1.5$) increase in perplexity due to the coarse granularity of quantization. 

\begin{figure}[h]
  \centering
  \includegraphics[width=0.7\linewidth]{sections/figures/LM7_FP7.pdf}
  \caption{\small Quantization levels and the corresponding quantization MSE of Floating Point (left) vs Lloyd-Max (right) Quantizers for a layer of weights in the GPT3-126M model.}
  \label{fig:lm_quant}
\end{figure}

\begin{table}[h]\scriptsize
\begin{center}
\caption{\label{tab:FP7_vs_LM7} \small Comparing perplexity (lower is better) achieved by floating point quantizers and Lloyd-Max quantizers on a GPT3-126M model for the Wikitext-103 dataset.}
\begin{tabular}{c|cc|c}
\hline
 \multirow{2}{*}{\textbf{Bitwidth}} & \multicolumn{2}{|c|}{\textbf{Floating-Point Quantizer}} & \textbf{Lloyd-Max Quantizer} \\
 & Best Format & Wikitext-103 Perplexity & Wikitext-103 Perplexity \\
\hline
7 & E3M3 & 18.32 & 18.27 \\
6 & E3M2 & 19.07 & 18.51 \\
5 & E4M0 & 43.89 & 19.71 \\
\hline
\end{tabular}
\end{center}
\end{table}

\subsection{Proof of Local Optimality of LO-BCQ}
\label{subsec:lobcq_opt_proof}
For a given block $\bm{b}_j$, the quantization MSE during LO-BCQ can be empirically evaluated as $\frac{1}{L_b}\lVert \bm{b}_j- \bm{\hat{b}}_j\rVert^2_2$ where $\bm{\hat{b}}_j$ is computed from equation (\ref{eq:clustered_quantization_definition}) as $C_{f(\bm{b}_j)}(\bm{b}_j)$. Further, for a given block cluster $\mathcal{B}_i$, we compute the quantization MSE as $\frac{1}{|\mathcal{B}_{i}|}\sum_{\bm{b} \in \mathcal{B}_{i}} \frac{1}{L_b}\lVert \bm{b}- C_i^{(n)}(\bm{b})\rVert^2_2$. Therefore, at the end of iteration $n$, we evaluate the overall quantization MSE $J^{(n)}$ for a given operand $\bm{X}$ composed of $N_c$ block clusters as:
\begin{align*}
    \label{eq:mse_iter_n}
    J^{(n)} = \frac{1}{N_c} \sum_{i=1}^{N_c} \frac{1}{|\mathcal{B}_{i}^{(n)}|}\sum_{\bm{v} \in \mathcal{B}_{i}^{(n)}} \frac{1}{L_b}\lVert \bm{b}- B_i^{(n)}(\bm{b})\rVert^2_2
\end{align*}

At the end of iteration $n$, the codebooks are updated from $\mathcal{C}^{(n-1)}$ to $\mathcal{C}^{(n)}$. However, the mapping of a given vector $\bm{b}_j$ to quantizers $\mathcal{C}^{(n)}$ remains as  $f^{(n)}(\bm{b}_j)$. At the next iteration, during the vector clustering step, $f^{(n+1)}(\bm{b}_j)$ finds new mapping of $\bm{b}_j$ to updated codebooks $\mathcal{C}^{(n)}$ such that the quantization MSE over the candidate codebooks is minimized. Therefore, we obtain the following result for $\bm{b}_j$:
\begin{align*}
\frac{1}{L_b}\lVert \bm{b}_j - C_{f^{(n+1)}(\bm{b}_j)}^{(n)}(\bm{b}_j)\rVert^2_2 \le \frac{1}{L_b}\lVert \bm{b}_j - C_{f^{(n)}(\bm{b}_j)}^{(n)}(\bm{b}_j)\rVert^2_2
\end{align*}

That is, quantizing $\bm{b}_j$ at the end of the block clustering step of iteration $n+1$ results in lower quantization MSE compared to quantizing at the end of iteration $n$. Since this is true for all $\bm{b} \in \bm{X}$, we assert the following:
\begin{equation}
\begin{split}
\label{eq:mse_ineq_1}
    \tilde{J}^{(n+1)} &= \frac{1}{N_c} \sum_{i=1}^{N_c} \frac{1}{|\mathcal{B}_{i}^{(n+1)}|}\sum_{\bm{b} \in \mathcal{B}_{i}^{(n+1)}} \frac{1}{L_b}\lVert \bm{b} - C_i^{(n)}(b)\rVert^2_2 \le J^{(n)}
\end{split}
\end{equation}
where $\tilde{J}^{(n+1)}$ is the the quantization MSE after the vector clustering step at iteration $n+1$.

Next, during the codebook update step (\ref{eq:quantizers_update}) at iteration $n+1$, the per-cluster codebooks $\mathcal{C}^{(n)}$ are updated to $\mathcal{C}^{(n+1)}$ by invoking the Lloyd-Max algorithm \citep{Lloyd}. We know that for any given value distribution, the Lloyd-Max algorithm minimizes the quantization MSE. Therefore, for a given vector cluster $\mathcal{B}_i$ we obtain the following result:

\begin{equation}
    \frac{1}{|\mathcal{B}_{i}^{(n+1)}|}\sum_{\bm{b} \in \mathcal{B}_{i}^{(n+1)}} \frac{1}{L_b}\lVert \bm{b}- C_i^{(n+1)}(\bm{b})\rVert^2_2 \le \frac{1}{|\mathcal{B}_{i}^{(n+1)}|}\sum_{\bm{b} \in \mathcal{B}_{i}^{(n+1)}} \frac{1}{L_b}\lVert \bm{b}- C_i^{(n)}(\bm{b})\rVert^2_2
\end{equation}

The above equation states that quantizing the given block cluster $\mathcal{B}_i$ after updating the associated codebook from $C_i^{(n)}$ to $C_i^{(n+1)}$ results in lower quantization MSE. Since this is true for all the block clusters, we derive the following result: 
\begin{equation}
\begin{split}
\label{eq:mse_ineq_2}
     J^{(n+1)} &= \frac{1}{N_c} \sum_{i=1}^{N_c} \frac{1}{|\mathcal{B}_{i}^{(n+1)}|}\sum_{\bm{b} \in \mathcal{B}_{i}^{(n+1)}} \frac{1}{L_b}\lVert \bm{b}- C_i^{(n+1)}(\bm{b})\rVert^2_2  \le \tilde{J}^{(n+1)}   
\end{split}
\end{equation}

Following (\ref{eq:mse_ineq_1}) and (\ref{eq:mse_ineq_2}), we find that the quantization MSE is non-increasing for each iteration, that is, $J^{(1)} \ge J^{(2)} \ge J^{(3)} \ge \ldots \ge J^{(M)}$ where $M$ is the maximum number of iterations. 
%Therefore, we can say that if the algorithm converges, then it must be that it has converged to a local minimum. 
\hfill $\blacksquare$


\begin{figure}
    \begin{center}
    \includegraphics[width=0.5\textwidth]{sections//figures/mse_vs_iter.pdf}
    \end{center}
    \caption{\small NMSE vs iterations during LO-BCQ compared to other block quantization proposals}
    \label{fig:nmse_vs_iter}
\end{figure}

Figure \ref{fig:nmse_vs_iter} shows the empirical convergence of LO-BCQ across several block lengths and number of codebooks. Also, the MSE achieved by LO-BCQ is compared to baselines such as MXFP and VSQ. As shown, LO-BCQ converges to a lower MSE than the baselines. Further, we achieve better convergence for larger number of codebooks ($N_c$) and for a smaller block length ($L_b$), both of which increase the bitwidth of BCQ (see Eq \ref{eq:bitwidth_bcq}).


\subsection{Additional Accuracy Results}
%Table \ref{tab:lobcq_config} lists the various LOBCQ configurations and their corresponding bitwidths.
\begin{table}
\setlength{\tabcolsep}{4.75pt}
\begin{center}
\caption{\label{tab:lobcq_config} Various LO-BCQ configurations and their bitwidths.}
\begin{tabular}{|c||c|c|c|c||c|c||c|} 
\hline
 & \multicolumn{4}{|c||}{$L_b=8$} & \multicolumn{2}{|c||}{$L_b=4$} & $L_b=2$ \\
 \hline
 \backslashbox{$L_A$\kern-1em}{\kern-1em$N_c$} & 2 & 4 & 8 & 16 & 2 & 4 & 2 \\
 \hline
 64 & 4.25 & 4.375 & 4.5 & 4.625 & 4.375 & 4.625 & 4.625\\
 \hline
 32 & 4.375 & 4.5 & 4.625& 4.75 & 4.5 & 4.75 & 4.75 \\
 \hline
 16 & 4.625 & 4.75& 4.875 & 5 & 4.75 & 5 & 5 \\
 \hline
\end{tabular}
\end{center}
\end{table}

%\subsection{Perplexity achieved by various LO-BCQ configurations on Wikitext-103 dataset}

\begin{table} \centering
\begin{tabular}{|c||c|c|c|c||c|c||c|} 
\hline
 $L_b \rightarrow$& \multicolumn{4}{c||}{8} & \multicolumn{2}{c||}{4} & 2\\
 \hline
 \backslashbox{$L_A$\kern-1em}{\kern-1em$N_c$} & 2 & 4 & 8 & 16 & 2 & 4 & 2  \\
 %$N_c \rightarrow$ & 2 & 4 & 8 & 16 & 2 & 4 & 2 \\
 \hline
 \hline
 \multicolumn{8}{c}{GPT3-1.3B (FP32 PPL = 9.98)} \\ 
 \hline
 \hline
 64 & 10.40 & 10.23 & 10.17 & 10.15 &  10.28 & 10.18 & 10.19 \\
 \hline
 32 & 10.25 & 10.20 & 10.15 & 10.12 &  10.23 & 10.17 & 10.17 \\
 \hline
 16 & 10.22 & 10.16 & 10.10 & 10.09 &  10.21 & 10.14 & 10.16 \\
 \hline
  \hline
 \multicolumn{8}{c}{GPT3-8B (FP32 PPL = 7.38)} \\ 
 \hline
 \hline
 64 & 7.61 & 7.52 & 7.48 &  7.47 &  7.55 &  7.49 & 7.50 \\
 \hline
 32 & 7.52 & 7.50 & 7.46 &  7.45 &  7.52 &  7.48 & 7.48  \\
 \hline
 16 & 7.51 & 7.48 & 7.44 &  7.44 &  7.51 &  7.49 & 7.47  \\
 \hline
\end{tabular}
\caption{\label{tab:ppl_gpt3_abalation} Wikitext-103 perplexity across GPT3-1.3B and 8B models.}
\end{table}

\begin{table} \centering
\begin{tabular}{|c||c|c|c|c||} 
\hline
 $L_b \rightarrow$& \multicolumn{4}{c||}{8}\\
 \hline
 \backslashbox{$L_A$\kern-1em}{\kern-1em$N_c$} & 2 & 4 & 8 & 16 \\
 %$N_c \rightarrow$ & 2 & 4 & 8 & 16 & 2 & 4 & 2 \\
 \hline
 \hline
 \multicolumn{5}{|c|}{Llama2-7B (FP32 PPL = 5.06)} \\ 
 \hline
 \hline
 64 & 5.31 & 5.26 & 5.19 & 5.18  \\
 \hline
 32 & 5.23 & 5.25 & 5.18 & 5.15  \\
 \hline
 16 & 5.23 & 5.19 & 5.16 & 5.14  \\
 \hline
 \multicolumn{5}{|c|}{Nemotron4-15B (FP32 PPL = 5.87)} \\ 
 \hline
 \hline
 64  & 6.3 & 6.20 & 6.13 & 6.08  \\
 \hline
 32  & 6.24 & 6.12 & 6.07 & 6.03  \\
 \hline
 16  & 6.12 & 6.14 & 6.04 & 6.02  \\
 \hline
 \multicolumn{5}{|c|}{Nemotron4-340B (FP32 PPL = 3.48)} \\ 
 \hline
 \hline
 64 & 3.67 & 3.62 & 3.60 & 3.59 \\
 \hline
 32 & 3.63 & 3.61 & 3.59 & 3.56 \\
 \hline
 16 & 3.61 & 3.58 & 3.57 & 3.55 \\
 \hline
\end{tabular}
\caption{\label{tab:ppl_llama7B_nemo15B} Wikitext-103 perplexity compared to FP32 baseline in Llama2-7B and Nemotron4-15B, 340B models}
\end{table}

%\subsection{Perplexity achieved by various LO-BCQ configurations on MMLU dataset}


\begin{table} \centering
\begin{tabular}{|c||c|c|c|c||c|c|c|c|} 
\hline
 $L_b \rightarrow$& \multicolumn{4}{c||}{8} & \multicolumn{4}{c||}{8}\\
 \hline
 \backslashbox{$L_A$\kern-1em}{\kern-1em$N_c$} & 2 & 4 & 8 & 16 & 2 & 4 & 8 & 16  \\
 %$N_c \rightarrow$ & 2 & 4 & 8 & 16 & 2 & 4 & 2 \\
 \hline
 \hline
 \multicolumn{5}{|c|}{Llama2-7B (FP32 Accuracy = 45.8\%)} & \multicolumn{4}{|c|}{Llama2-70B (FP32 Accuracy = 69.12\%)} \\ 
 \hline
 \hline
 64 & 43.9 & 43.4 & 43.9 & 44.9 & 68.07 & 68.27 & 68.17 & 68.75 \\
 \hline
 32 & 44.5 & 43.8 & 44.9 & 44.5 & 68.37 & 68.51 & 68.35 & 68.27  \\
 \hline
 16 & 43.9 & 42.7 & 44.9 & 45 & 68.12 & 68.77 & 68.31 & 68.59  \\
 \hline
 \hline
 \multicolumn{5}{|c|}{GPT3-22B (FP32 Accuracy = 38.75\%)} & \multicolumn{4}{|c|}{Nemotron4-15B (FP32 Accuracy = 64.3\%)} \\ 
 \hline
 \hline
 64 & 36.71 & 38.85 & 38.13 & 38.92 & 63.17 & 62.36 & 63.72 & 64.09 \\
 \hline
 32 & 37.95 & 38.69 & 39.45 & 38.34 & 64.05 & 62.30 & 63.8 & 64.33  \\
 \hline
 16 & 38.88 & 38.80 & 38.31 & 38.92 & 63.22 & 63.51 & 63.93 & 64.43  \\
 \hline
\end{tabular}
\caption{\label{tab:mmlu_abalation} Accuracy on MMLU dataset across GPT3-22B, Llama2-7B, 70B and Nemotron4-15B models.}
\end{table}


%\subsection{Perplexity achieved by various LO-BCQ configurations on LM evaluation harness}

\begin{table} \centering
\begin{tabular}{|c||c|c|c|c||c|c|c|c|} 
\hline
 $L_b \rightarrow$& \multicolumn{4}{c||}{8} & \multicolumn{4}{c||}{8}\\
 \hline
 \backslashbox{$L_A$\kern-1em}{\kern-1em$N_c$} & 2 & 4 & 8 & 16 & 2 & 4 & 8 & 16  \\
 %$N_c \rightarrow$ & 2 & 4 & 8 & 16 & 2 & 4 & 2 \\
 \hline
 \hline
 \multicolumn{5}{|c|}{Race (FP32 Accuracy = 37.51\%)} & \multicolumn{4}{|c|}{Boolq (FP32 Accuracy = 64.62\%)} \\ 
 \hline
 \hline
 64 & 36.94 & 37.13 & 36.27 & 37.13 & 63.73 & 62.26 & 63.49 & 63.36 \\
 \hline
 32 & 37.03 & 36.36 & 36.08 & 37.03 & 62.54 & 63.51 & 63.49 & 63.55  \\
 \hline
 16 & 37.03 & 37.03 & 36.46 & 37.03 & 61.1 & 63.79 & 63.58 & 63.33  \\
 \hline
 \hline
 \multicolumn{5}{|c|}{Winogrande (FP32 Accuracy = 58.01\%)} & \multicolumn{4}{|c|}{Piqa (FP32 Accuracy = 74.21\%)} \\ 
 \hline
 \hline
 64 & 58.17 & 57.22 & 57.85 & 58.33 & 73.01 & 73.07 & 73.07 & 72.80 \\
 \hline
 32 & 59.12 & 58.09 & 57.85 & 58.41 & 73.01 & 73.94 & 72.74 & 73.18  \\
 \hline
 16 & 57.93 & 58.88 & 57.93 & 58.56 & 73.94 & 72.80 & 73.01 & 73.94  \\
 \hline
\end{tabular}
\caption{\label{tab:mmlu_abalation} Accuracy on LM evaluation harness tasks on GPT3-1.3B model.}
\end{table}

\begin{table} \centering
\begin{tabular}{|c||c|c|c|c||c|c|c|c|} 
\hline
 $L_b \rightarrow$& \multicolumn{4}{c||}{8} & \multicolumn{4}{c||}{8}\\
 \hline
 \backslashbox{$L_A$\kern-1em}{\kern-1em$N_c$} & 2 & 4 & 8 & 16 & 2 & 4 & 8 & 16  \\
 %$N_c \rightarrow$ & 2 & 4 & 8 & 16 & 2 & 4 & 2 \\
 \hline
 \hline
 \multicolumn{5}{|c|}{Race (FP32 Accuracy = 41.34\%)} & \multicolumn{4}{|c|}{Boolq (FP32 Accuracy = 68.32\%)} \\ 
 \hline
 \hline
 64 & 40.48 & 40.10 & 39.43 & 39.90 & 69.20 & 68.41 & 69.45 & 68.56 \\
 \hline
 32 & 39.52 & 39.52 & 40.77 & 39.62 & 68.32 & 67.43 & 68.17 & 69.30  \\
 \hline
 16 & 39.81 & 39.71 & 39.90 & 40.38 & 68.10 & 66.33 & 69.51 & 69.42  \\
 \hline
 \hline
 \multicolumn{5}{|c|}{Winogrande (FP32 Accuracy = 67.88\%)} & \multicolumn{4}{|c|}{Piqa (FP32 Accuracy = 78.78\%)} \\ 
 \hline
 \hline
 64 & 66.85 & 66.61 & 67.72 & 67.88 & 77.31 & 77.42 & 77.75 & 77.64 \\
 \hline
 32 & 67.25 & 67.72 & 67.72 & 67.00 & 77.31 & 77.04 & 77.80 & 77.37  \\
 \hline
 16 & 68.11 & 68.90 & 67.88 & 67.48 & 77.37 & 78.13 & 78.13 & 77.69  \\
 \hline
\end{tabular}
\caption{\label{tab:mmlu_abalation} Accuracy on LM evaluation harness tasks on GPT3-8B model.}
\end{table}

\begin{table} \centering
\begin{tabular}{|c||c|c|c|c||c|c|c|c|} 
\hline
 $L_b \rightarrow$& \multicolumn{4}{c||}{8} & \multicolumn{4}{c||}{8}\\
 \hline
 \backslashbox{$L_A$\kern-1em}{\kern-1em$N_c$} & 2 & 4 & 8 & 16 & 2 & 4 & 8 & 16  \\
 %$N_c \rightarrow$ & 2 & 4 & 8 & 16 & 2 & 4 & 2 \\
 \hline
 \hline
 \multicolumn{5}{|c|}{Race (FP32 Accuracy = 40.67\%)} & \multicolumn{4}{|c|}{Boolq (FP32 Accuracy = 76.54\%)} \\ 
 \hline
 \hline
 64 & 40.48 & 40.10 & 39.43 & 39.90 & 75.41 & 75.11 & 77.09 & 75.66 \\
 \hline
 32 & 39.52 & 39.52 & 40.77 & 39.62 & 76.02 & 76.02 & 75.96 & 75.35  \\
 \hline
 16 & 39.81 & 39.71 & 39.90 & 40.38 & 75.05 & 73.82 & 75.72 & 76.09  \\
 \hline
 \hline
 \multicolumn{5}{|c|}{Winogrande (FP32 Accuracy = 70.64\%)} & \multicolumn{4}{|c|}{Piqa (FP32 Accuracy = 79.16\%)} \\ 
 \hline
 \hline
 64 & 69.14 & 70.17 & 70.17 & 70.56 & 78.24 & 79.00 & 78.62 & 78.73 \\
 \hline
 32 & 70.96 & 69.69 & 71.27 & 69.30 & 78.56 & 79.49 & 79.16 & 78.89  \\
 \hline
 16 & 71.03 & 69.53 & 69.69 & 70.40 & 78.13 & 79.16 & 79.00 & 79.00  \\
 \hline
\end{tabular}
\caption{\label{tab:mmlu_abalation} Accuracy on LM evaluation harness tasks on GPT3-22B model.}
\end{table}

\begin{table} \centering
\begin{tabular}{|c||c|c|c|c||c|c|c|c|} 
\hline
 $L_b \rightarrow$& \multicolumn{4}{c||}{8} & \multicolumn{4}{c||}{8}\\
 \hline
 \backslashbox{$L_A$\kern-1em}{\kern-1em$N_c$} & 2 & 4 & 8 & 16 & 2 & 4 & 8 & 16  \\
 %$N_c \rightarrow$ & 2 & 4 & 8 & 16 & 2 & 4 & 2 \\
 \hline
 \hline
 \multicolumn{5}{|c|}{Race (FP32 Accuracy = 44.4\%)} & \multicolumn{4}{|c|}{Boolq (FP32 Accuracy = 79.29\%)} \\ 
 \hline
 \hline
 64 & 42.49 & 42.51 & 42.58 & 43.45 & 77.58 & 77.37 & 77.43 & 78.1 \\
 \hline
 32 & 43.35 & 42.49 & 43.64 & 43.73 & 77.86 & 75.32 & 77.28 & 77.86  \\
 \hline
 16 & 44.21 & 44.21 & 43.64 & 42.97 & 78.65 & 77 & 76.94 & 77.98  \\
 \hline
 \hline
 \multicolumn{5}{|c|}{Winogrande (FP32 Accuracy = 69.38\%)} & \multicolumn{4}{|c|}{Piqa (FP32 Accuracy = 78.07\%)} \\ 
 \hline
 \hline
 64 & 68.9 & 68.43 & 69.77 & 68.19 & 77.09 & 76.82 & 77.09 & 77.86 \\
 \hline
 32 & 69.38 & 68.51 & 68.82 & 68.90 & 78.07 & 76.71 & 78.07 & 77.86  \\
 \hline
 16 & 69.53 & 67.09 & 69.38 & 68.90 & 77.37 & 77.8 & 77.91 & 77.69  \\
 \hline
\end{tabular}
\caption{\label{tab:mmlu_abalation} Accuracy on LM evaluation harness tasks on Llama2-7B model.}
\end{table}

\begin{table} \centering
\begin{tabular}{|c||c|c|c|c||c|c|c|c|} 
\hline
 $L_b \rightarrow$& \multicolumn{4}{c||}{8} & \multicolumn{4}{c||}{8}\\
 \hline
 \backslashbox{$L_A$\kern-1em}{\kern-1em$N_c$} & 2 & 4 & 8 & 16 & 2 & 4 & 8 & 16  \\
 %$N_c \rightarrow$ & 2 & 4 & 8 & 16 & 2 & 4 & 2 \\
 \hline
 \hline
 \multicolumn{5}{|c|}{Race (FP32 Accuracy = 48.8\%)} & \multicolumn{4}{|c|}{Boolq (FP32 Accuracy = 85.23\%)} \\ 
 \hline
 \hline
 64 & 49.00 & 49.00 & 49.28 & 48.71 & 82.82 & 84.28 & 84.03 & 84.25 \\
 \hline
 32 & 49.57 & 48.52 & 48.33 & 49.28 & 83.85 & 84.46 & 84.31 & 84.93  \\
 \hline
 16 & 49.85 & 49.09 & 49.28 & 48.99 & 85.11 & 84.46 & 84.61 & 83.94  \\
 \hline
 \hline
 \multicolumn{5}{|c|}{Winogrande (FP32 Accuracy = 79.95\%)} & \multicolumn{4}{|c|}{Piqa (FP32 Accuracy = 81.56\%)} \\ 
 \hline
 \hline
 64 & 78.77 & 78.45 & 78.37 & 79.16 & 81.45 & 80.69 & 81.45 & 81.5 \\
 \hline
 32 & 78.45 & 79.01 & 78.69 & 80.66 & 81.56 & 80.58 & 81.18 & 81.34  \\
 \hline
 16 & 79.95 & 79.56 & 79.79 & 79.72 & 81.28 & 81.66 & 81.28 & 80.96  \\
 \hline
\end{tabular}
\caption{\label{tab:mmlu_abalation} Accuracy on LM evaluation harness tasks on Llama2-70B model.}
\end{table}

%\section{MSE Studies}
%\textcolor{red}{TODO}


\subsection{Number Formats and Quantization Method}
\label{subsec:numFormats_quantMethod}
\subsubsection{Integer Format}
An $n$-bit signed integer (INT) is typically represented with a 2s-complement format \citep{yao2022zeroquant,xiao2023smoothquant,dai2021vsq}, where the most significant bit denotes the sign.

\subsubsection{Floating Point Format}
An $n$-bit signed floating point (FP) number $x$ comprises of a 1-bit sign ($x_{\mathrm{sign}}$), $B_m$-bit mantissa ($x_{\mathrm{mant}}$) and $B_e$-bit exponent ($x_{\mathrm{exp}}$) such that $B_m+B_e=n-1$. The associated constant exponent bias ($E_{\mathrm{bias}}$) is computed as $(2^{{B_e}-1}-1)$. We denote this format as $E_{B_e}M_{B_m}$.  

\subsubsection{Quantization Scheme}
\label{subsec:quant_method}
A quantization scheme dictates how a given unquantized tensor is converted to its quantized representation. We consider FP formats for the purpose of illustration. Given an unquantized tensor $\bm{X}$ and an FP format $E_{B_e}M_{B_m}$, we first, we compute the quantization scale factor $s_X$ that maps the maximum absolute value of $\bm{X}$ to the maximum quantization level of the $E_{B_e}M_{B_m}$ format as follows:
\begin{align}
\label{eq:sf}
    s_X = \frac{\mathrm{max}(|\bm{X}|)}{\mathrm{max}(E_{B_e}M_{B_m})}
\end{align}
In the above equation, $|\cdot|$ denotes the absolute value function.

Next, we scale $\bm{X}$ by $s_X$ and quantize it to $\hat{\bm{X}}$ by rounding it to the nearest quantization level of $E_{B_e}M_{B_m}$ as:

\begin{align}
\label{eq:tensor_quant}
    \hat{\bm{X}} = \text{round-to-nearest}\left(\frac{\bm{X}}{s_X}, E_{B_e}M_{B_m}\right)
\end{align}

We perform dynamic max-scaled quantization \citep{wu2020integer}, where the scale factor $s$ for activations is dynamically computed during runtime.

\subsection{Vector Scaled Quantization}
\begin{wrapfigure}{r}{0.35\linewidth}
  \centering
  \includegraphics[width=\linewidth]{sections/figures/vsquant.jpg}
  \caption{\small Vectorwise decomposition for per-vector scaled quantization (VSQ \citep{dai2021vsq}).}
  \label{fig:vsquant}
\end{wrapfigure}
During VSQ \citep{dai2021vsq}, the operand tensors are decomposed into 1D vectors in a hardware friendly manner as shown in Figure \ref{fig:vsquant}. Since the decomposed tensors are used as operands in matrix multiplications during inference, it is beneficial to perform this decomposition along the reduction dimension of the multiplication. The vectorwise quantization is performed similar to tensorwise quantization described in Equations \ref{eq:sf} and \ref{eq:tensor_quant}, where a scale factor $s_v$ is required for each vector $\bm{v}$ that maps the maximum absolute value of that vector to the maximum quantization level. While smaller vector lengths can lead to larger accuracy gains, the associated memory and computational overheads due to the per-vector scale factors increases. To alleviate these overheads, VSQ \citep{dai2021vsq} proposed a second level quantization of the per-vector scale factors to unsigned integers, while MX \citep{rouhani2023shared} quantizes them to integer powers of 2 (denoted as $2^{INT}$).

\subsubsection{MX Format}
The MX format proposed in \citep{rouhani2023microscaling} introduces the concept of sub-block shifting. For every two scalar elements of $b$-bits each, there is a shared exponent bit. The value of this exponent bit is determined through an empirical analysis that targets minimizing quantization MSE. We note that the FP format $E_{1}M_{b}$ is strictly better than MX from an accuracy perspective since it allocates a dedicated exponent bit to each scalar as opposed to sharing it across two scalars. Therefore, we conservatively bound the accuracy of a $b+2$-bit signed MX format with that of a $E_{1}M_{b}$ format in our comparisons. For instance, we use E1M2 format as a proxy for MX4.

\begin{figure}
    \centering
    \includegraphics[width=1\linewidth]{sections//figures/BlockFormats.pdf}
    \caption{\small Comparing LO-BCQ to MX format.}
    \label{fig:block_formats}
\end{figure}

Figure \ref{fig:block_formats} compares our $4$-bit LO-BCQ block format to MX \citep{rouhani2023microscaling}. As shown, both LO-BCQ and MX decompose a given operand tensor into block arrays and each block array into blocks. Similar to MX, we find that per-block quantization ($L_b < L_A$) leads to better accuracy due to increased flexibility. While MX achieves this through per-block $1$-bit micro-scales, we associate a dedicated codebook to each block through a per-block codebook selector. Further, MX quantizes the per-block array scale-factor to E8M0 format without per-tensor scaling. In contrast during LO-BCQ, we find that per-tensor scaling combined with quantization of per-block array scale-factor to E4M3 format results in superior inference accuracy across models. 




\end{document}