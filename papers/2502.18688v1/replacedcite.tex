\section{Related Work}
\subsection{Human-Robot Collaboration} 

The field of human-robot interaction (HRI) has long studied effective ways for robots to engage in collaborations with humans. 
Prior work human-robot collaboration in terms of conversational dynamics and physical human-robot collaboration.%, joint action, and adaptation. 
For example, prior work has explored how robots can shape conversation dynamics in group collaboration ____ and how robots can improve group learning processes ____. 
Furthermore, physical human-teaming involves physical activities such as moving around, lifting items, moving items from one place to another. 
The literature often frames this as a joint action, adaptation, and entrainment problem which models psychological, neurological, and physical mechanisms by which humans collaborate with robots ____. 
%One example of joint action are measures of synchrony in groups with one and multiple robots to enable robots to adapt to physical motion of human teams ____. 
%Others have explored human membership preferences of working alongside competitive versus relationship-driven robots in teams ____. 
%Although some prior research has focused on modeling human group dynamics and the design of robots with high-level behaviors that support teamwork, there is a lack of work focused on the design of robots for acute care settings. 
Furthermore, prior work often involves evaluating robots in well-controlled environments with minimal consequences for their actions, whereas in acute care settings which are high-risk environments, human actions could result in patient safety risks. 
HRI in \textit{action teams} is the most relevant work as it highlights the importance of well-designed proactive robot behaviors to address operational failures in time-critical contexts (i.e., healthcare and firefighting) ____. 
\textcolor{blue}{Prior research has explored modeling techniques for human intent, human collaborations with robots, and methods that enable robots to anticipate human actions} ____.
As a result, further research is required to understand how robots can assist in team collaborations in ER environments.


\subsection{Collaboration in Medical Teams}

There are many robots designed to support people in terms of health and wellbeing ____.
For example, assistive robots are used as companions to support older adults ____, robotic wheelchairs are used to support patient mobility ____. 
Robots are used to support people with rehabilitative training, patients with psychiatric disabilities by engaging in rehabilitative training ____, older adults to support recreation  ____, and improve patients' motor skills ____.
Robots also perform non-patient-facing tasks, such as fetching and delivering supplies ____ to free up time for HCWs to focus on patient care. 
They are also used to support nurses with triage ____ and lifting patients ____ and telemedicine ____.%jang2021assessment,






\subsection{Embodiment of Robots}

Much prior research in HRI demonstrates that robot embodiment sets the expectations of how robots can interact with people based on robot affordances ____, shapes how people perceive robots ____, and to what extent people adopt robots ____.
For example, robots come in many shapes including humanoids ____, zoomorphic (animal-like) ____, ottomans ____, and even adjustable wall robots ____, and adjustable furniture ____.
Robots can also vary in terms of how human-like or machine-like they appear where those that appear too human-like often appear uncanny to humans ____ and those that appear more machine-like are often viewed as companions or pets ____. 
Most similar to our work is the work done by Ju et al. ____ who designed an automatic drawer open mechanism for robot carts in office spaces.
A key difference between this work and ours is that we focus on building robots for safety-critical healthcare settings and we focus on building robots for team-based interactions, opposed to dyadic interactions.
\textcolor{blue}{The work by Taylor et. al., ____ motivates the use of crash cart form factors as a way to integrate robots into clinical team collaborations. A recent study revealed robotic use-cases in clinical team settings which we build on in our work to conduct iterative rapid prototyping of new MCCRs.}

\subsection{Design Approaches for Field Studies}

Participatory design (PD) has gained increasing popularity in HRI as a way to invite stakeholders to act as co-designers. %, enabling them to generate design ideas collaboratively with researchers. 
%This method has enhanced the HRI community’s appreciation for stakeholders’ contextual knowledge, particularly for underserved populations whose insights have been undervalued, such as older adults.
 Despite this growing interest, PD has been used limitedly just to generate initial design ideas. 
However, PD researchers have emphasized the importance of long-term ongoing efforts, advocating for the PD process as a form of `infrastructuring' ____. 
This concept highlights that prototypes (or existing technologies) are merely entry points situated within the complex networks of the communities ____. %where they are investigated over the long term. 
%Thus, ongoing alignment of technology ____ through long-term collaboration with stakeholders is essential.
Inspired by the concept of infrastructuring, we have focused on benefiting healthcare teams through a long-term collaboration (since 2021) with a healthcare professionals’ organization that conducts an annual bootcamp for interprofessional emergency medicine team training. 
Our goal has involved a entangled process of designing, developing, deploying, and redesigning our crash cart robots to ensure they align well with the needs of our collaborators. This study will demonstrate understudied aspects of PD that view the design process as a continuous dynamic endeavor. 
This new approach is particularly beneficial to the HRI community, considering the importance of prototyping in real-world contexts to incorporate unique factors that differ depending on the setting ____ %environmental, social, and cultural .%, as discussed in previous work 



%%%%%
% Section 3: Methodology / Algorithmic Approach / etc
%
%   GOAL: If a studies or design paper, this explains your methodology. If a technical paper, this section explains your algorithmic approach.
%%%%%