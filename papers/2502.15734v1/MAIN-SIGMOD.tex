%%
%% This is file `sample-sigconf.tex',
%% generated with the docstrip utility.
%%
%% The original source files were:
%%
%% samples.dtx  (with options: `all,proceedings,bibtex,sigconf')
%% 
%% IMPORTANT NOTICE:
%% 
%% For the copyright see the source file.
%% 
%% Any modified versions of this file must be renamed
%% with new filenames distinct from sample-sigconf.tex.
%% 
%% For distribution of the original source see the terms
%% for copying and modification in the file samples.dtx.
%% 
%% This generated file may be distributed as long as the
%% original source files, as listed above, are part of the
%% same distribution. (The sources need not necessarily be
%% in the same archive or directory.)
%%
%%
%% Commands for TeXCount
%TC:macro \cite [option:text,text]
%TC:macro \citep [option:text,text]
%TC:macro \citet [option:text,text]
%TC:envir table 0 1
%TC:envir table* 0 1
%TC:envir tabular [ignore] word
%TC:envir displaymath 0 word
%TC:envir math 0 word
%TC:envir comment 0 0
%%
%%
%% The first command in your LaTeX source must be the \documentclass
%% command.
%%
%% For submission and review of your manuscript please change the
%% command to \documentclass[manuscript, screen, review]{acmart}.
%%
%% When submitting camera ready or to TAPS, please change the command
%% to \documentclass[sigconf]{acmart} or whichever template is required
%% for your publication.
%%
%%
\documentclass[sigconf]{acmart} %authordraft


\usepackage{times}
\usepackage{latexsym}

\usepackage[T1]{fontenc}

\usepackage[utf8]{inputenc}

\usepackage{microtype}

\usepackage{inconsolata}

\usepackage{graphicx}


\usepackage{amsmath}
\usepackage{amssymb}
\usepackage{multirow}
\usepackage{booktabs}
\usepackage{catchfile}

\usepackage[boxed]{algorithm}
\usepackage{varwidth}
\usepackage[noEnd=true,indLines=false]{algpseudocodex}
\usepackage{cleveref}
\makeatletter
\@addtoreset{ALG@line}{algorithm}
\renewcommand{\ALG@beginalgorithmic}{\small}
\algrenewcommand\alglinenumber[1]{\small #1:}
\makeatother

\usepackage[normalem]{ulem}
\usepackage{todonotes}

\usepackage{lipsum}    %
\usepackage{comment}   %
\usepackage{graphicx}  %
\usepackage{pifont}    %

\usepackage[font=small,labelfont=bf]{caption}
\usepackage{float}     %
\usepackage{booktabs}  %
\usepackage{subcaption}  %

\usepackage{listings}

\usepackage{amsthm}  %

% This is for defining terms that we should use consistently through the paper. Also pay attention to the casing. 

%\newcommand{\sys}[0]{\textsc{CacheCraft}\xspace}
\newcommand{\sys}[0]{\textsc{Cache-Craft}\xspace}


\newcommand{\sysomp}[0]{\textsc{SYS}-w/o\textsc{MP}\xspace}
\newcommand{\policy}[0]{\texttt{LCBFU}\xspace}
%Least Computationally Beneficial and Frequently Used(LCBFU) item

\newcommand{\gpt}[0]{\textsc{GPT-Cache}\xspace}

\newcommand{\llama}[0]{\texttt{LLaMA}\xspace}
\newcommand{\vllm}[0]{\texttt{vLLM}\xspace}

\newcommand{\RPE}[0]{\texttt{RPE}\xspace}

\newcommand{\fullcache}[0]{\textsc{Full-Cache}\xspace}

\newcommand{\fullrecomp}[0]{\textsc{Full-Recomp}\xspace}

\newcommand{\ho}[0]{\textsc{Prefill-H2O}\xspace}

\newcommand{\randrecomp}[0]{\textsc{Random-Recomp}\xspace}

\newcommand{\prefixcache}[0]{\textsc{Prefix-Cache}\xspace}

\newcommand{\setcache}[0]{\textsc{Set-Cache}\xspace}

\newcommand{\LLMLingua}[0]{\textsc{Lingua2}\xspace}

\newcommand{\MapReduce}[0]{\textsc{MapReduce}\xspace}


%AEP
\newcommand{\X}{{\sc Sys-X}\xspace}
%Acrobat AI
\newcommand{\Y}{{\sc Sys-Y}\xspace}

\newcommand{\dms}[0]{\text{diffusion-models}\xspace}

\newcommand{\embeddinggenerator}[0]{\texttt{embedding-generator}\xspace}

\newcommand{\matchpredictor}[0]{\texttt{match-predictor}\xspace}

\newcommand{\VDB}[0]{\texttt{VDB}\xspace}
\newcommand{\vdb}[0]{\texttt{VDB}\xspace}

\newcommand{\cacheselector}[0]{\texttt{cache-selector}\xspace}

\newcommand{\cachemaintainer}[0]{\texttt{cache-maintainer}\xspace}

\newcommand{\efs}[0]{\texttt{EFS}\xspace}



% Replace below with iffalse for final version
\iftrue
%\iffalse
\newcommand{\sm}[1]{\todo[inline,color=brown!40]{Subrata: #1}}
\newcommand{\sa}[1]{\todo[inline,color=green!40]{Shubham: #1}}
\newcommand{\re}[1]{\todo[inline,color=red!40]{Remark: #1}}
\newcommand{\rs}[1]{\todo[inline,color=blue!40]{Rounak: #1}}
\newcommand{\ag}[1]{\todo[inline,color=yellow!40]{Archit: #1}}
% \newcommand{\db}[1]{\todo[inline,color=pink!40]{Deb: #1}}
\newcommand{\db}[1]{\textcolor{orange}{\emph{Deb: #1}}}
\newcommand{\sks}[1]{\todo[inline,color=purple!40]{Shiv: #1}}
%CFC - candidate for cut
\definecolor{darkgreen}{rgb}{0.0, 0.6, 0.0}
\newcommand{\CFC}[1]{{{\textcolor{darkgreen}{\{\{CFC:\} #1\}}}\xspace}}
\newcommand{\tong}[1]{{\color{blue}[Tong: #1]}}



\else
\newcommand{\sm}[1]{}
\newcommand{\sa}[1]{}
\newcommand{\re}[1]{}
\newcommand{\rs}[1]{}
\newcommand{\ag}[1]{}

%{{\color{black}{#1}}}
\fi

%%
%% \BibTeX command to typeset BibTeX logo in the docs
\AtBeginDocument{%
  \providecommand\BibTeX{{%
    Bib\TeX}}}

%% Rights management information.  This information is sent to you
%% when you complete the rights form.  These commands have SAMPLE
%% values in them; it is your responsibility as an author to replace
%% the commands and values with those provided to you when you
%% complete the rights form.
% \setcopyright{acmlicensed}
% \copyrightyear{2018}
% \acmYear{2018}
% \acmDOI{XXXXXXX.XXXXXXX}

%% These commands are for a PROCEEDINGS abstract or paper.
% \acmConference[Conference acronym 'XX]{Make sure to enter the correct
%   conference title from your rights confirmation emai}{June 03--05,
%   2018}{Woodstock, NY}
%%
%%  Uncomment \acmBooktitle if the title of the proceedings is different
%%  from ``Proceedings of ...''!
%%
%%\acmBooktitle{Woodstock '18: ACM Symposium on Neural Gaze Detection,
%%  June 03--05, 2018, Woodstock, NY}
% \acmISBN{978-1-4503-XXXX-X/18/06}


%%
%% Submission ID.
%% Use this when submitting an article to a sponsored event. You'll
%% receive a unique submission ID from the organizers
%% of the event, and this ID should be used as the parameter to this command.
%%\acmSubmissionID{123-A56-BU3}

%%
%% For managing citations, it is recommended to use bibliography
%% files in BibTeX format.
%%
%% You can then either use BibTeX with the ACM-Reference-Format style,
%% or BibLaTeX with the acmnumeric or acmauthoryear sytles, that include
%% support for advanced citation of software artefact from the
%% biblatex-software package, also separately available on CTAN.
%%
%% Look at the sample-*-biblatex.tex files for templates showcasing
%% the biblatex styles.
%%

%%
%% The majority of ACM publications use numbered citations and
%% references.  The command \citestyle{authoryear} switches to the
%% "author year" style.
%%
%% If you are preparing content for an event
%% sponsored by ACM SIGGRAPH, you must use the "author year" style of
%% citations and references.
%% Uncommenting
%% the next command will enable that style.
%%\citestyle{acmauthoryear}


%%
%% end of the preamble, start of the body of the document source.


\settopmatter{printfolios=true}
\settopmatter{printacmref=false} % Removes citation info below abstract
\renewcommand\footnotetextcopyrightpermission[1]{} % Removes copyright footnote
\setcopyright{none} % Removes ACM copyright statement

% \renewcommand{\acmConference}[3]{}
% Ensure all metadata fields are completely empty
\acmConference[Preprint]{~}{ArXiv}{2025}
% Clears conference name, location, and year
\acmYear{}            % Clears year
\acmMonth{}           % Clears month
\acmISBN{}            % Clears ISBN
\acmDOI{}             % Clears DOI
\acmPrice{}           % Clears pricing info
\acmSubmissionID{}    % Clears submission ID
\acmBooktitle{}       % Clears book title


\usepackage{soul}
\usepackage{booktabs}
% \usepackage[table,xcdraw,svgnames]{xcolor}

% \usepackage[table,xcdraw]{xcolor}

% \usepackage[svgnames]{xcolor} % For colors
\usepackage{graphicx}         % For including images
\usepackage{tcolorbox}        % For creating colored boxes

\definecolor{darkorange}{RGB}{204, 102, 0}
\renewcommand{\hl}[1]{{\textcolor{black}{#1}}}

%\usepackage[font={bf,it,footnotesize},labelsep=colon, belowskip=-1.0pt, aboveskip=0.3pt]{caption}
\usepackage[shortlabels]{enumitem}
\setlist[enumerate]{nosep}
\setlist{nolistsep,leftmargin=4.0mm}
\setlist[itemize]{left=0pt, topsep=0.1pt, partopsep=0.1pt, parsep=0pt, itemsep=0pt}

%\usepackage[demo]{graphicx}
%\usepackage{floatrow}
%% Using belowskip=-3pt will cause caption to overlap tables
%\captionsetup{font=footnotesize, belowskip=-1.0pt, aboveskip=0.3pt}
%\setlength{\skip\footins}{5pt}
\usepackage{mathtools}
\usepackage[hang,flushmargin]{footmisc} 

\widowpenalty=10000
\clubpenalty=10000


\pagestyle{plain}
\pagenumbering{arabic}
\usepackage{indentfirst}
\begin{document}

% \pagestyle{plain} % Removes all fancy headers and footers

%%
%% The "title" command has an optional parameter,
%% allowing the author to define a "short title" to be used in page headers.



\title{\vspace{-1em}\sys: Managing Chunk-Caches for Efficient Retrieval-Augmented Generation
%\vspace{-1.5em}
}



% The "author" command and its associated commands are used to define
% the authors and their affiliations.
% Of note is the shared affiliation of the first two authors, and the
% "authornote" and "authornotemark" commands
% used to denote shared contribution to the research.

\author{\Large Shubham Agarwal$^1$\footnotemark[1], Sai Sundaresan$^1$\footnotemark[1], Subrata Mitra$^1$\footnotemark[2], Debabrata Mahapatra$^1$}
\author{\Large Archit Gupta$^2$\footnotemark[3], Rounak Sharma$^3$\footnotemark[3], Nirmal Joshua Kapu$^3$\footnotemark[3], Tong Yu$^1$, Shiv Saini$^1$
}


\affiliation{%
    % \vspace{-1em}
  \Large {$^1$Adobe Research} \hspace{20pt} 
  {$^2$IIT Bombay} \hspace{20pt} 
  {$^3$IIT Kanpur}
  \country{}
}







%%
%% The "author" command and its associated commands are used to define
%% the authors and their affiliations.
%% Of note is the shared affiliation of the first two authors, and the
%% "authornote" and "authornotemark" commands
%% used to denote shared contribution to the research.
% \author{Ben Trovato}
% \authornote{Both authors contributed equally to this research.}
% \email{trovato@corporation.com}
% \orcid{1234-5678-9012}
% \author{G.K.M. Tobin}
% \authornotemark[1]
% \email{webmaster@marysville-ohio.com}
% \affiliation{%
%   \institution{Institute for Clarity in Documentation}
%   \city{Dublin}
%   \state{Ohio}
%   \country{USA}
% }

% \author{Lars Th{\o}rv{\"a}ld}
% \affiliation{%
%   \institution{The Th{\o}rv{\"a}ld Group}
%   \city{Hekla}
%   \country{Iceland}}
% \email{larst@affiliation.org}

% \author{Valerie B\'eranger}
% \affiliation{%
%   \institution{Inria Paris-Rocquencourt}
%   \city{Rocquencourt}
%   \country{France}
% }

% \author{Aparna Patel}
% \affiliation{%
%  \institution{Rajiv Gandhi University}
%  \city{Doimukh}
%  \state{Arunachal Pradesh}
%  \country{India}}

% \author{Huifen Chan}
% \affiliation{%
%   \institution{Tsinghua University}
%   \city{Haidian Qu}
%   \state{Beijing Shi}
%   \country{China}}

% \author{Charles Palmer}
% \affiliation{%
%   \institution{Palmer Research Laboratories}
%   \city{San Antonio}
%   \state{Texas}
%   \country{USA}}
% \email{cpalmer@prl.com}

% \author{John Smith}
% \affiliation{%
%   \institution{The Th{\o}rv{\"a}ld Group}
%   \city{Hekla}
%   \country{Iceland}}
% \email{jsmith@affiliation.org}

% \author{Julius P. Kumquat}
% \affiliation{%
%   \institution{The Kumquat Consortium}
%   \city{New York}
%   \country{USA}}
% \email{jpkumquat@consortium.net}

%%
%% By default, the full list of authors will be used in the page
%% headers. Often, this list is too long, and will overlap
%% other information printed in the page headers. This command allows
%% the author to define a more concise list
%% of authors' names for this purpose.

\renewcommand{\shortauthors}{Agarwal et al.}

%%
%% The abstract is a short summary of the work to be presented in the
%% article.



% \begin{abstract}
%   A clear and well-documented \LaTeX\ document is presented as an
%   article formatted for publication by ACM in a conference proceedings
%   or journal publication. Based on the ``acmart'' document class, this
%   article presents and explains many of the common variations, as well
%   as many of the formatting elements an author may use in the
%   preparation of the documentation of their work.
% \end{abstract}

\sethlcolor{yellow!40}

\begin{abstract}  
Test time scaling is currently one of the most active research areas that shows promise after training time scaling has reached its limits.
Deep-thinking (DT) models are a class of recurrent models that can perform easy-to-hard generalization by assigning more compute to harder test samples.
However, due to their inability to determine the complexity of a test sample, DT models have to use a large amount of computation for both easy and hard test samples.
Excessive test time computation is wasteful and can cause the ``overthinking'' problem where more test time computation leads to worse results.
In this paper, we introduce a test time training method for determining the optimal amount of computation needed for each sample during test time.
We also propose Conv-LiGRU, a novel recurrent architecture for efficient and robust visual reasoning. 
Extensive experiments demonstrate that Conv-LiGRU is more stable than DT, effectively mitigates the ``overthinking'' phenomenon, and achieves superior accuracy.
\end{abstract}  


%%
%% The code below is generated by the tool at http://dl.acm.org/ccs.cfm.
%% Please copy and paste the code instead of the example below.
%%
% \begin{CCSXML}
% <ccs2012>
%  <concept>
%   <concept_id>00000000.0000000.0000000</concept_id>
%   <concept_desc>Do Not Use This Code, Generate the Correct Terms for Your Paper</concept_desc>
%   <concept_significance>500</concept_significance>
%  </concept>
%  <concept>
%   <concept_id>00000000.00000000.00000000</concept_id>
%   <concept_desc>Do Not Use This Code, Generate the Correct Terms for Your Paper</concept_desc>
%   <concept_significance>300</concept_significance>
%  </concept>
%  <concept>
%   <concept_id>00000000.00000000.00000000</concept_id>
%   <concept_desc>Do Not Use This Code, Generate the Correct Terms for Your Paper</concept_desc>
%   <concept_significance>100</concept_significance>
%  </concept>
%  <concept>
%   <concept_id>00000000.00000000.00000000</concept_id>
%   <concept_desc>Do Not Use This Code, Generate the Correct Terms for Your Paper</concept_desc>
%   <concept_significance>100</concept_significance>
%  </concept>
% </ccs2012>
% \end{CCSXML}

% \ccsdesc[500]{Do Not Use This Code~Generate the Correct Terms for Your Paper}
% \ccsdesc[300]{Do Not Use This Code~Generate the Correct Terms for Your Paper}
% \ccsdesc{Do Not Use This Code~Generate the Correct Terms for Your Paper}
% \ccsdesc[100]{Do Not Use This Code~Generate the Correct Terms for Your Paper}

% %%
% %% Keywords. The author(s) should pick words that accurately describe
% %% the work being presented. Separate the keywords with commas.
% \keywords{Do, Not, Us, This, Code, Put, the, Correct, Terms, for}
%   Your, Paper}
%% A "teaser" image appears between the author and affiliation
%% information and the body of the document, and typically spans the
%% page.
% \begin{teaserfigure}
%   \includegraphics[width=\textwidth]{Images/sampleteaser.pdf}
%   \caption{Seattle Mariners at Spring Training, 2010.}
%   \Description{Enjoying the baseball game from the third-base
%   seats. Ichiro Suzuki preparing to bat.}
%   \label{fig:teaser}
% \end{teaserfigure}

% \received{20 February 2007}
% \received[revised]{12 March 2009}
% \received[accepted]{5 June 2009}

%%
%% This command processes the author and affiliation and title
%% information and builds the first part of the formatted document.

\maketitle


% \renewcommand{\thefootnote}{\fnsymbol{footnote}}
% \footnotetext[1]{Equal contributions.}
% \footnotetext[3]{Work done at Adobe Research.}

% \renewcommand{\footnoterule}{%
%     \hrule width 0.35\linewidth 
% }

\renewcommand{\thefootnote}{\fnsymbol{footnote}}
\footnotetext[1]{Equal contributions. \hspace{0.2em} \footnotemark[2] Corresponding Author (subrata.mitra@adobe.com).}
\footnotetext[3]{Work done at Adobe Research.}

% \setcounter{page}{1}



\section{Introduction}
\label{sec:introduction}
The business processes of organizations are experiencing ever-increasing complexity due to the large amount of data, high number of users, and high-tech devices involved \cite{martin2021pmopportunitieschallenges, beerepoot2023biggestbpmproblems}. This complexity may cause business processes to deviate from normal control flow due to unforeseen and disruptive anomalies \cite{adams2023proceddsriftdetection}. These control-flow anomalies manifest as unknown, skipped, and wrongly-ordered activities in the traces of event logs monitored from the execution of business processes \cite{ko2023adsystematicreview}. For the sake of clarity, let us consider an illustrative example of such anomalies. Figure \ref{FP_ANOMALIES} shows a so-called event log footprint, which captures the control flow relations of four activities of a hypothetical event log. In particular, this footprint captures the control-flow relations between activities \texttt{a}, \texttt{b}, \texttt{c} and \texttt{d}. These are the causal ($\rightarrow$) relation, concurrent ($\parallel$) relation, and other ($\#$) relations such as exclusivity or non-local dependency \cite{aalst2022pmhandbook}. In addition, on the right are six traces, of which five exhibit skipped, wrongly-ordered and unknown control-flow anomalies. For example, $\langle$\texttt{a b d}$\rangle$ has a skipped activity, which is \texttt{c}. Because of this skipped activity, the control-flow relation \texttt{b}$\,\#\,$\texttt{d} is violated, since \texttt{d} directly follows \texttt{b} in the anomalous trace.
\begin{figure}[!t]
\centering
\includegraphics[width=0.9\columnwidth]{images/FP_ANOMALIES.png}
\caption{An example event log footprint with six traces, of which five exhibit control-flow anomalies.}
\label{FP_ANOMALIES}
\end{figure}

\subsection{Control-flow anomaly detection}
Control-flow anomaly detection techniques aim to characterize the normal control flow from event logs and verify whether these deviations occur in new event logs \cite{ko2023adsystematicreview}. To develop control-flow anomaly detection techniques, \revision{process mining} has seen widespread adoption owing to process discovery and \revision{conformance checking}. On the one hand, process discovery is a set of algorithms that encode control-flow relations as a set of model elements and constraints according to a given modeling formalism \cite{aalst2022pmhandbook}; hereafter, we refer to the Petri net, a widespread modeling formalism. On the other hand, \revision{conformance checking} is an explainable set of algorithms that allows linking any deviations with the reference Petri net and providing the fitness measure, namely a measure of how much the Petri net fits the new event log \cite{aalst2022pmhandbook}. Many control-flow anomaly detection techniques based on \revision{conformance checking} (hereafter, \revision{conformance checking}-based techniques) use the fitness measure to determine whether an event log is anomalous \cite{bezerra2009pmad, bezerra2013adlogspais, myers2018icsadpm, pecchia2020applicationfailuresanalysispm}. 

The scientific literature also includes many \revision{conformance checking}-independent techniques for control-flow anomaly detection that combine specific types of trace encodings with machine/deep learning \cite{ko2023adsystematicreview, tavares2023pmtraceencoding}. Whereas these techniques are very effective, their explainability is challenging due to both the type of trace encoding employed and the machine/deep learning model used \cite{rawal2022trustworthyaiadvances,li2023explainablead}. Hence, in the following, we focus on the shortcomings of \revision{conformance checking}-based techniques to investigate whether it is possible to support the development of competitive control-flow anomaly detection techniques while maintaining the explainable nature of \revision{conformance checking}.
\begin{figure}[!t]
\centering
\includegraphics[width=\columnwidth]{images/HIGH_LEVEL_VIEW.png}
\caption{A high-level view of the proposed framework for combining \revision{process mining}-based feature extraction with dimensionality reduction for control-flow anomaly detection.}
\label{HIGH_LEVEL_VIEW}
\end{figure}

\subsection{Shortcomings of \revision{conformance checking}-based techniques}
Unfortunately, the detection effectiveness of \revision{conformance checking}-based techniques is affected by noisy data and low-quality Petri nets, which may be due to human errors in the modeling process or representational bias of process discovery algorithms \cite{bezerra2013adlogspais, pecchia2020applicationfailuresanalysispm, aalst2016pm}. Specifically, on the one hand, noisy data may introduce infrequent and deceptive control-flow relations that may result in inconsistent fitness measures, whereas, on the other hand, checking event logs against a low-quality Petri net could lead to an unreliable distribution of fitness measures. Nonetheless, such Petri nets can still be used as references to obtain insightful information for \revision{process mining}-based feature extraction, supporting the development of competitive and explainable \revision{conformance checking}-based techniques for control-flow anomaly detection despite the problems above. For example, a few works outline that token-based \revision{conformance checking} can be used for \revision{process mining}-based feature extraction to build tabular data and develop effective \revision{conformance checking}-based techniques for control-flow anomaly detection \cite{singh2022lapmsh, debenedictis2023dtadiiot}. However, to the best of our knowledge, the scientific literature lacks a structured proposal for \revision{process mining}-based feature extraction using the state-of-the-art \revision{conformance checking} variant, namely alignment-based \revision{conformance checking}.

\subsection{Contributions}
We propose a novel \revision{process mining}-based feature extraction approach with alignment-based \revision{conformance checking}. This variant aligns the deviating control flow with a reference Petri net; the resulting alignment can be inspected to extract additional statistics such as the number of times a given activity caused mismatches \cite{aalst2022pmhandbook}. We integrate this approach into a flexible and explainable framework for developing techniques for control-flow anomaly detection. The framework combines \revision{process mining}-based feature extraction and dimensionality reduction to handle high-dimensional feature sets, achieve detection effectiveness, and support explainability. Notably, in addition to our proposed \revision{process mining}-based feature extraction approach, the framework allows employing other approaches, enabling a fair comparison of multiple \revision{conformance checking}-based and \revision{conformance checking}-independent techniques for control-flow anomaly detection. Figure \ref{HIGH_LEVEL_VIEW} shows a high-level view of the framework. Business processes are monitored, and event logs obtained from the database of information systems. Subsequently, \revision{process mining}-based feature extraction is applied to these event logs and tabular data input to dimensionality reduction to identify control-flow anomalies. We apply several \revision{conformance checking}-based and \revision{conformance checking}-independent framework techniques to publicly available datasets, simulated data of a case study from railways, and real-world data of a case study from healthcare. We show that the framework techniques implementing our approach outperform the baseline \revision{conformance checking}-based techniques while maintaining the explainable nature of \revision{conformance checking}.

In summary, the contributions of this paper are as follows.
\begin{itemize}
    \item{
        A novel \revision{process mining}-based feature extraction approach to support the development of competitive and explainable \revision{conformance checking}-based techniques for control-flow anomaly detection.
    }
    \item{
        A flexible and explainable framework for developing techniques for control-flow anomaly detection using \revision{process mining}-based feature extraction and dimensionality reduction.
    }
    \item{
        Application to synthetic and real-world datasets of several \revision{conformance checking}-based and \revision{conformance checking}-independent framework techniques, evaluating their detection effectiveness and explainability.
    }
\end{itemize}

The rest of the paper is organized as follows.
\begin{itemize}
    \item Section \ref{sec:related_work} reviews the existing techniques for control-flow anomaly detection, categorizing them into \revision{conformance checking}-based and \revision{conformance checking}-independent techniques.
    \item Section \ref{sec:abccfe} provides the preliminaries of \revision{process mining} to establish the notation used throughout the paper, and delves into the details of the proposed \revision{process mining}-based feature extraction approach with alignment-based \revision{conformance checking}.
    \item Section \ref{sec:framework} describes the framework for developing \revision{conformance checking}-based and \revision{conformance checking}-independent techniques for control-flow anomaly detection that combine \revision{process mining}-based feature extraction and dimensionality reduction.
    \item Section \ref{sec:evaluation} presents the experiments conducted with multiple framework and baseline techniques using data from publicly available datasets and case studies.
    \item Section \ref{sec:conclusions} draws the conclusions and presents future work.
\end{itemize}
\section{Background}\label{sec:backgrnd}

\subsection{Cold Start Latency and Mitigation Techniques}

Traditional FaaS platforms mitigate cold starts through snapshotting, lightweight virtualization, and warm-state management. Snapshot-based methods like \textbf{REAP} and \textbf{Catalyzer} reduce initialization time by preloading or restoring container states but require significant memory and I/O resources, limiting scalability~\cite{dong_catalyzer_2020, ustiugov_benchmarking_2021}. Lightweight virtualization solutions, such as \textbf{Firecracker} microVMs, achieve fast startup times with strong isolation but depend on robust infrastructure, making them less adaptable to fluctuating workloads~\cite{agache_firecracker_2020}. Warm-state management techniques like \textbf{Faa\$T}~\cite{romero_faa_2021} and \textbf{Kraken}~\cite{vivek_kraken_2021} keep frequently invoked containers ready, balancing readiness and cost efficiency under predictable workloads but incurring overhead when demand is erratic~\cite{romero_faa_2021, vivek_kraken_2021}. While these methods perform well in resource-rich cloud environments, their resource intensity challenges applicability in edge settings.

\subsubsection{Edge FaaS Perspective}

In edge environments, cold start mitigation emphasizes lightweight designs, resource sharing, and hybrid task distribution. Lightweight execution environments like unikernels~\cite{edward_sock_2018} and \textbf{Firecracker}~\cite{agache_firecracker_2020}, as used by \textbf{TinyFaaS}~\cite{pfandzelter_tinyfaas_2020}, minimize resource usage and initialization delays but require careful orchestration to avoid resource contention. Function co-location, demonstrated by \textbf{Photons}~\cite{v_dukic_photons_2020}, reduces redundant initializations by sharing runtime resources among related functions, though this complicates isolation in multi-tenant setups~\cite{v_dukic_photons_2020}. Hybrid offloading frameworks like \textbf{GeoFaaS}~\cite{malekabbasi_geofaas_2024} balance edge-cloud workloads by offloading latency-tolerant tasks to the cloud and reserving edge resources for real-time operations, requiring reliable connectivity and efficient task management. These edge-specific strategies address cold starts effectively but introduce challenges in scalability and orchestration.

\subsection{Predictive Scaling and Caching Techniques}

Efficient resource allocation is vital for maintaining low latency and high availability in serverless platforms. Predictive scaling and caching techniques dynamically provision resources and reduce cold start latency by leveraging workload prediction and state retention.
Traditional FaaS platforms use predictive scaling and caching to optimize resources, employing techniques (OFC, FaasCache) to reduce cold starts. However, these methods rely on centralized orchestration and workload predictability, limiting their effectiveness in dynamic, resource-constrained edge environments.



\subsubsection{Edge FaaS Perspective}

Edge FaaS platforms adapt predictive scaling and caching techniques to constrain resources and heterogeneous environments. \textbf{EDGE-Cache}~\cite{kim_delay-aware_2022} uses traffic profiling to selectively retain high-priority functions, reducing memory overhead while maintaining readiness for frequent requests. Hybrid frameworks like \textbf{GeoFaaS}~\cite{malekabbasi_geofaas_2024} implement distributed caching to balance resources between edge and cloud nodes, enabling low-latency processing for critical tasks while offloading less critical workloads. Machine learning methods, such as clustering-based workload predictors~\cite{gao_machine_2020} and GRU-based models~\cite{guo_applying_2018}, enhance resource provisioning in edge systems by efficiently forecasting workload spikes. These innovations effectively address cold start challenges in edge environments, though their dependency on accurate predictions and robust orchestration poses scalability challenges.

\subsection{Decentralized Orchestration, Function Placement, and Scheduling}

Efficient orchestration in serverless platforms involves workload distribution, resource optimization, and performance assurance. While traditional FaaS platforms rely on centralized control, edge environments require decentralized and adaptive strategies to address unique challenges such as resource constraints and heterogeneous hardware.



\subsubsection{Edge FaaS Perspective}

Edge FaaS platforms adopt decentralized and adaptive orchestration frameworks to meet the demands of resource-constrained environments. Systems like \textbf{Wukong} distribute scheduling across edge nodes, enhancing data locality and scalability while reducing network latency. Lightweight frameworks such as \textbf{OpenWhisk Lite}~\cite{kravchenko_kpavelopenwhisk-light_2024} optimize resource allocation by decentralizing scheduling policies, minimizing cold starts and latency in edge setups~\cite{benjamin_wukong_2020}. Hybrid solutions like \textbf{OpenFaaS}~\cite{noauthor_openfaasfaas_2024} and \textbf{EdgeMatrix}~\cite{shen_edgematrix_2023} combine edge-cloud orchestration to balance resource utilization, retaining latency-sensitive functions at the edge while offloading non-critical workloads to the cloud. While these approaches improve flexibility, they face challenges in maintaining coordination and ensuring consistent performance across distributed nodes.


\section{Bellman Error Centering}

Centering operator $\mathcal{C}$ for a variable $x(s)$ is defined as follows:
\begin{equation}
\mathcal{C}x(s)\dot{=} x(s)-\mathbb{E}[x(s)]=x(s)-\sum_s{d_{s}x(s)},
\end{equation} 
where $d_s$ is the probability of $s$.
In vector form,
\begin{equation}
\begin{split}
\mathcal{C}\bm{x} &= \bm{x}-\mathbb{E}[x]\bm{1}\\
&=\bm{x}-\bm{x}^{\top}\bm{d}\bm{1},
\end{split}
\end{equation} 
where $\bm{1}$ is an all-ones vector.
For any vector $\bm{x}$ and $\bm{y}$ with a same distribution $\bm{d}$,
we have
\begin{equation}
\begin{split}
\mathcal{C}(\bm{x}+\bm{y})&=(\bm{x}+\bm{y})-(\bm{x}+\bm{y})^{\top}\bm{d}\bm{1}\\
&=\bm{x}-\bm{x}^{\top}\bm{d}\bm{1}+\bm{y}-\bm{y}^{\top}\bm{d}\bm{1}\\
&=\mathcal{C}\bm{x}+\mathcal{C}\bm{y}.
\end{split}
\end{equation}
\subsection{Revisit Reward Centering}


The update (\ref{src3}) is an unbiased estimate of the average reward
with  appropriate learning rate $\beta_t$ conditions.
\begin{equation}
\bar{r}_{t}\approx \lim_{n\rightarrow\infty}\frac{1}{n}\sum_{t=1}^n\mathbb{E}_{\pi}[r_t].
\end{equation}
That is 
\begin{equation}
r_t-\bar{r}_{t}\approx r_t-\lim_{n\rightarrow\infty}\frac{1}{n}\sum_{t=1}^n\mathbb{E}_{\pi}[r_t]= \mathcal{C}r_t.
\end{equation}
Then, the simple reward centering can be rewrited as:
\begin{equation}
V_{t+1}(s_t)=V_{t}(s_t)+\alpha_t [\mathcal{C}r_{t+1}+\gamma V_{t}(s_{t+1})-V_t(s_t)].
\end{equation}
Therefore, the simple reward centering is, in a strict sense, reward centering.

By definition of $\bar{\delta}_t=\delta_t-\bar{r}_{t}$,
let rewrite the update rule of the value-based reward centering as follows:
\begin{equation}
V_{t+1}(s_t)=V_{t}(s_t)+\alpha_t \rho_t (\delta_t-\bar{r}_{t}),
\end{equation}
where $\bar{r}_{t}$ is updated as:
\begin{equation}
\bar{r}_{t+1}=\bar{r}_{t}+\beta_t \rho_t(\delta_t-\bar{r}_{t}).
\label{vrc3}
\end{equation}
The update (\ref{vrc3}) is an unbiased estimate of the TD error
with  appropriate learning rate $\beta_t$ conditions.
\begin{equation}
\bar{r}_{t}\approx \mathbb{E}_{\pi}[\delta_t].
\end{equation}
That is 
\begin{equation}
\delta_t-\bar{r}_{t}\approx \mathcal{C}\delta_t.
\end{equation}
Then, the value-based reward centering can be rewrited as:
\begin{equation}
V_{t+1}(s_t)=V_{t}(s_t)+\alpha_t \rho_t \mathcal{C}\delta_t.
\label{tdcentering}
\end{equation}
Therefore, the value-based reward centering is no more,
 in a strict sense, reward centering.
It is, in a strict sense, \textbf{Bellman error centering}.

It is worth noting that this understanding is crucial, 
as designing new algorithms requires leveraging this concept.


\subsection{On the Fixpoint Solution}

The update rule (\ref{tdcentering}) is a stochastic approximation
of the following update:
\begin{equation}
\begin{split}
V_{t+1}&=V_{t}+\alpha_t [\bm{\mathcal{T}}^{\pi}\bm{V}-\bm{V}-\mathbb{E}[\delta]\bm{1}]\\
&=V_{t}+\alpha_t [\bm{\mathcal{T}}^{\pi}\bm{V}-\bm{V}-(\bm{\mathcal{T}}^{\pi}\bm{V}-\bm{V})^{\top}\bm{d}_{\pi}\bm{1}]\\
&=V_{t}+\alpha_t [\mathcal{C}(\bm{\mathcal{T}}^{\pi}\bm{V}-\bm{V})].
\end{split}
\label{tdcenteringVector}
\end{equation}
If update rule (\ref{tdcenteringVector}) converges, it is expected that
$\mathcal{C}(\mathcal{T}^{\pi}V-V)=\bm{0}$.
That is 
\begin{equation}
    \begin{split}
    \mathcal{C}\bm{V} &= \mathcal{C}\bm{\mathcal{T}}^{\pi}\bm{V} \\
    &= \mathcal{C}(\bm{R}^{\pi} + \gamma \mathbb{P}^{\pi} \bm{V}) \\
    &= \mathcal{C}\bm{R}^{\pi} + \gamma \mathcal{C}\mathbb{P}^{\pi} \bm{V} \\
    &= \mathcal{C}\bm{R}^{\pi} + \gamma (\mathbb{P}^{\pi} \bm{V} - (\mathbb{P}^{\pi} \bm{V})^{\top} \bm{d_{\pi}} \bm{1}) \\
    &= \mathcal{C}\bm{R}^{\pi} + \gamma (\mathbb{P}^{\pi} \bm{V} - \bm{V}^{\top} (\mathbb{P}^{\pi})^{\top} \bm{d_{\pi}} \bm{1}) \\  % 修正双重上标
    &= \mathcal{C}\bm{R}^{\pi} + \gamma (\mathbb{P}^{\pi} \bm{V} - \bm{V}^{\top} \bm{d_{\pi}} \bm{1}) \\
    &= \mathcal{C}\bm{R}^{\pi} + \gamma (\mathbb{P}^{\pi} \bm{V} - \bm{V}^{\top} \bm{d_{\pi}} \mathbb{P}^{\pi} \bm{1}) \\
    &= \mathcal{C}\bm{R}^{\pi} + \gamma (\mathbb{P}^{\pi} \bm{V} - \mathbb{P}^{\pi} \bm{V}^{\top} \bm{d_{\pi}} \bm{1}) \\
    &= \mathcal{C}\bm{R}^{\pi} + \gamma \mathbb{P}^{\pi} (\bm{V} - \bm{V}^{\top} \bm{d_{\pi}} \bm{1}) \\
    &= \mathcal{C}\bm{R}^{\pi} + \gamma \mathbb{P}^{\pi} \mathcal{C}\bm{V} \\
    &\dot{=} \bm{\mathcal{T}}_c^{\pi} \mathcal{C}\bm{V},
    \end{split}
    \label{centeredfixpoint}
    \end{equation}
where we defined $\bm{\mathcal{T}}_c^{\pi}$ as a centered Bellman operator.
We call equation (\ref{centeredfixpoint}) as centered Bellman equation.
And it is \textbf{centered fixpoint}.

For linear value function approximation, let define
\begin{equation}
\mathcal{C}\bm{V}_{\bm{\theta}}=\bm{\Pi}\bm{\mathcal{T}}_c^{\pi}\mathcal{C}\bm{V}_{\bm{\theta}}.
\label{centeredTDfixpoint}
\end{equation}
We call equation (\ref{centeredTDfixpoint}) as \textbf{centered TD fixpoint}.

\subsection{On-policy and Off-policy Centered TD Algorithms
with Linear Value Function Approximation}
Given the above centered TD fixpoint,
 mean squared centered Bellman error (MSCBE), is proposed as follows:
\begin{align*}
    \label{argminMSBEC}
 &\arg \min_{{\bm{\theta}}}\text{MSCBE}({\bm{\theta}}) \\
 &= \arg \min_{{\bm{\theta}}} \|\bm{\mathcal{T}}_c^{\pi}\mathcal{C}\bm{V}_{\bm{{\bm{\theta}}}}-\mathcal{C}\bm{V}_{\bm{{\bm{\theta}}}}\|_{\bm{D}}^2\notag\\
 &=\arg \min_{{\bm{\theta}}} \|\bm{\mathcal{T}}^{\pi}\bm{V}_{\bm{{\bm{\theta}}}} - \bm{V}_{\bm{{\bm{\theta}}}}-(\bm{\mathcal{T}}^{\pi}\bm{V}_{\bm{{\bm{\theta}}}} - \bm{V}_{\bm{{\bm{\theta}}}})^{\top}\bm{d}\bm{1}\|_{\bm{D}}^2\notag\\
 &=\arg \min_{{\bm{\theta}},\omega} \| \bm{\mathcal{T}}^{\pi}\bm{V}_{\bm{{\bm{\theta}}}} - \bm{V}_{\bm{{\bm{\theta}}}}-\omega\bm{1} \|_{\bm{D}}^2\notag,
\end{align*}
where $\omega$ is is used to estimate the expected value of the Bellman error.
% where $\omega$ is used to estimate $\mathbb{E}[\delta]$, $\omega \doteq \mathbb{E}[\mathbb{E}[\delta_t|S_t]]=\mathbb{E}[\delta]$ and $\delta_t$ is the TD error as follows:
% \begin{equation}
% \delta_t = r_{t+1}+\gamma
% {\bm{\theta}}_t^{\top}\bm{{\bm{\phi}}}_{t+1}-{\bm{\theta}}_t^{\top}\bm{{\bm{\phi}}}_t.
% \label{delta}
% \end{equation}
% $\mathbb{E}[\delta_t|S_t]$ is the Bellman error, and $\mathbb{E}[\mathbb{E}[\delta_t|S_t]]$ represents the expected value of the Bellman error.
% If $X$ is a random variable and $\mathbb{E}[X]$ is its expected value, then $X-\mathbb{E}[X]$ represents the centered form of $X$. 
% Therefore, we refer to $\mathbb{E}[\delta_t|S_t]-\mathbb{E}[\mathbb{E}[\delta_t|S_t]]$ as Bellman error centering and 
% $\mathbb{E}[(\mathbb{E}[\delta_t|S_t]-\mathbb{E}[\mathbb{E}[\delta_t|S_t]])^2]$ represents the the mean squared centered Bellman error, namely MSCBE.
% The meaning of (\ref{argminMSBEC}) is to minimize the mean squared centered Bellman error.
%The derivation of CTD is as follows.

First, the parameter  $\omega$ is derived directly based on
stochastic gradient descent:
\begin{equation}
\omega_{t+1}= \omega_{t}+\beta_t(\delta_t-\omega_t).
\label{omega}
\end{equation}

Then, based on stochastic semi-gradient descent, the update of 
the parameter ${\bm{\theta}}$ is as follows:
\begin{equation}
{\bm{\theta}}_{t+1}=
{\bm{\theta}}_{t}+\alpha_t(\delta_t-\omega_t)\bm{{\bm{\phi}}}_t.
\label{theta}
\end{equation}

We call (\ref{omega}) and (\ref{theta}) the on-policy centered
TD (CTD) algorithm. The convergence analysis with be given in
the following section.

In off-policy learning, we can simply multiply by the importance sampling
 $\rho$.
\begin{equation}
    \omega_{t+1}=\omega_{t}+\beta_t\rho_t(\delta_t-\omega_t),
    \label{omegawithrho}
\end{equation}
\begin{equation}
    {\bm{\theta}}_{t+1}=
    {\bm{\theta}}_{t}+\alpha_t\rho_t(\delta_t-\omega_t)\bm{{\bm{\phi}}}_t.
    \label{thetawithrho}
\end{equation}

We call (\ref{omegawithrho}) and (\ref{thetawithrho}) the off-policy centered
TD (CTD) algorithm.

% By substituting $\delta_t$ into Equations (\ref{omegawithrho}) and (\ref{thetawithrho}), 
% we can see that Equations (\ref{thetawithrho}) and (\ref{omegawithrho}) are formally identical 
% to the linear expressions of Equations (\ref{rewardcentering1}) and (\ref{rewardcentering2}), respectively. However, the meanings 
% of the corresponding parameters are entirely different.
% ${\bm{\theta}}_t$ is for approximating the discounted value function.
% $\bar{r_t}$ is an estimate of the average reward, while $\omega_t$ 
% is an estimate of the expected value of the Bellman error.
% $\bar{\delta_t}$ is the TD error for value-based reward centering, 
% whereas $\delta_t$ is the traditional TD error.

% This study posits that the CTD is equivalent to value-based reward 
% centering. However, CTD can be unified under a single framework 
% through an objective function, MSCBE, which also lays the 
% foundation for proving the algorithm's convergence. 
% Section 4 demonstrates that the CTD algorithm guarantees 
% convergence in the on-policy setting.

\subsection{Off-policy Centered TDC Algorithm with Linear Value Function Approximation}
The convergence of the  off-policy centered TD algorithm
may not be guaranteed.

To deal with this problem, we propose another new objective function, 
called mean squared projected centered Bellman error (MSPCBE), 
and derive Centered TDC algorithm (CTDC).

% We first establish some relationships between
%  the vector-matrix quantities and the relevant statistical expectation terms:
% \begin{align*}
%     &\mathbb{E}[(\delta({\bm{\theta}})-\mathbb{E}[\delta({\bm{\theta}})]){\bm{\phi}}] \\
%     &= \sum_s \mu(s) {\bm{\phi}}(s) \big( R(s) + \gamma \sum_{s'} P_{ss'} V_{\bm{\theta}}(s') - V_{\bm{\theta}}(s)  \\
%     &\quad \quad-\sum_s \mu(s)(R(s) + \gamma \sum_{s'} P_{ss'} V_{\bm{\theta}}(s') - V_{\bm{\theta}}(s))\big)\\
%     &= \bm{\Phi}^\top \mathbf{D} (\bm{TV}_{\bm{{\bm{\theta}}}} - \bm{V}_{\bm{{\bm{\theta}}}}-\omega\bm{1}),
% \end{align*}
% where $\omega$ is the expected value of the Bellman error and $\bm{1}$ is all-ones vector.

The specific expression of the objective function 
MSPCBE is as follows:
\begin{align}
    \label{MSPBECwithomega}
    &\arg \min_{{\bm{\theta}}}\text{MSPCBE}({\bm{\theta}})\notag\\ 
    % &= \arg \min_{{\bm{\theta}}}\big(\mathbb{E}[(\delta({\bm{\theta}}) - \mathbb{E}[\delta({\bm{\theta}})]) \bm{{\bm{\phi}}}]^\top \notag\\
    % &\quad \quad \quad\mathbb{E}[\bm{{\bm{\phi}}} \bm{{\bm{\phi}}}^\top]^{-1} \mathbb{E}[(\delta({\bm{\theta}}) - \mathbb{E}[\delta({\bm{\theta}})]) \bm{{\bm{\phi}}}]\big) \notag\\
    % &=\arg \min_{{\bm{\theta}},\omega}\mathbb{E}[(\delta({\bm{\theta}})-\omega) \bm{\bm{{\bm{\phi}}}}]^{\top} \mathbb{E}[\bm{\bm{{\bm{\phi}}}} \bm{\bm{{\bm{\phi}}}}^{\top}]^{-1}\mathbb{E}[(\delta({\bm{\theta}}) -\omega)\bm{\bm{{\bm{\phi}}}}]\\
    % &= \big(\bm{\Phi}^\top \mathbf{D} (\bm{TV}_{\bm{{\bm{\theta}}}} - \bm{V}_{\bm{{\bm{\theta}}}}-\omega\bm{1})\big)^\top (\bm{\Phi}^\top \mathbf{D} \bm{\Phi})^{-1} \notag\\
    % & \quad \quad \quad \bm{\Phi}^\top \mathbf{D} (\bm{TV}_{\bm{{\bm{\theta}}}} - \bm{V}_{\bm{{\bm{\theta}}}}-\omega\bm{1}) \notag\\
    % &= (\bm{TV}_{\bm{{\bm{\theta}}}} - \bm{V}_{\bm{{\bm{\theta}}}}-\omega\bm{1})^\top \mathbf{D} \bm{\Phi} (\bm{\Phi}^\top \mathbf{D} \bm{\Phi})^{-1} \notag\\
    % &\quad \quad \quad \bm{\Phi}^\top \mathbf{D} (\bm{TV}_{\bm{{\bm{\theta}}}} - \bm{V}_{\bm{{\bm{\theta}}}}-\omega\bm{1})\notag\\
    % &= (\bm{TV}_{\bm{{\bm{\theta}}}} - \bm{V}_{\bm{{\bm{\theta}}}}-\omega\bm{1})^\top {\bm{\Pi}}^\top \mathbf{D} {\bm{\Pi}} (\bm{TV}_{\bm{{\bm{\theta}}}} - \bm{V}_{\bm{{\bm{\theta}}}}-\omega\bm{1}) \notag\\
    &= \arg \min_{{\bm{\theta}}} \|\bm{\Pi}\bm{\mathcal{T}}_c^{\pi}\mathcal{C}\bm{V}_{\bm{{\bm{\theta}}}}-\mathcal{C}\bm{V}_{\bm{{\bm{\theta}}}}\|_{\bm{D}}^2\notag\\
    &= \arg \min_{{\bm{\theta}}} \|\bm{\Pi}(\bm{\mathcal{T}}_c^{\pi}\mathcal{C}\bm{V}_{\bm{{\bm{\theta}}}}-\mathcal{C}\bm{V}_{\bm{{\bm{\theta}}}})\|_{\bm{D}}^2\notag\\
    &= \arg \min_{{\bm{\theta}},\omega}\| {\bm{\Pi}} (\bm{\mathcal{T}}^{\pi}\bm{V}_{\bm{{\bm{\theta}}}} - \bm{V}_{\bm{{\bm{\theta}}}}-\omega\bm{1}) \|_{\bm{D}}^2\notag.
\end{align}
In the process of computing the gradient of the MSPCBE with respect to ${\bm{\theta}}$, 
$\omega$ is treated as a constant.
So, the derivation process of CTDC is the same 
as for the TDC algorithm \cite{sutton2009fast}, the only difference is that the original $\delta$ is replaced by $\delta-\omega$.
Therefore, the updated formulas of the centered TDC  algorithm are as follows:
\begin{equation}
 \bm{{\bm{\theta}}}_{k+1}=\bm{{\bm{\theta}}}_{k}+\alpha_{k}[(\delta_{k}- \omega_k) \bm{\bm{{\bm{\phi}}}}_k\\
 - \gamma\bm{\bm{{\bm{\phi}}}}_{k+1}(\bm{\bm{{\bm{\phi}}}}^{\top}_k \bm{u}_{k})],
\label{thetavmtdc}
\end{equation}
\begin{equation}
 \bm{u}_{k+1}= \bm{u}_{k}+\zeta_{k}[\delta_{k}-\omega_k - \bm{\bm{{\bm{\phi}}}}^{\top}_k \bm{u}_{k}]\bm{\bm{{\bm{\phi}}}}_k,
\label{uvmtdc}
\end{equation}
and
\begin{equation}
 \omega_{k+1}= \omega_{k}+\beta_k (\delta_k- \omega_k).
 \label{omegavmtdc}
\end{equation}
This algorithm is derived to work 
with a given set of sub-samples—in the form of 
triples $(S_k, R_k, S'_k)$ that match transitions 
from both the behavior and target policies. 

% \subsection{Variance Minimization ETD Learning: VMETD}
% Based on the off-policy TD algorithm, a scalar, $F$,  
% is introduced to obtain the ETD algorithm, 
% which ensures convergence under off-policy 
% conditions. This paper further introduces a scalar, 
% $\omega$, based on the ETD algorithm to obtain VMETD.
% VMETD by the following update:
% \begin{equation}
% \label{fvmetd}
%  F_t \leftarrow \gamma \rho_{t-1}F_{t-1}+1,
% \end{equation}
% \begin{equation}
%  \label{thetavmetd}
%  {{\bm{\theta}}}_{t+1}\leftarrow {{\bm{\theta}}}_t+\alpha_t (F_t \rho_t\delta_t - \omega_{t}){\bm{{\bm{\phi}}}}_t,
% \end{equation}
% \begin{equation}
%  \label{omegavmetd}
%  \omega_{t+1} \leftarrow \omega_t+\beta_t(F_t  \rho_t \delta_t - \omega_t),
% \end{equation}
% where $\rho_t =\frac{\pi(A_t | S_t)}{\mu(A_t | S_t)}$ and $\omega$ is used to estimate $\mathbb{E}[F \rho\delta]$, i.e., $\omega \doteq \mathbb{E}[F \rho\delta]$.

% (\ref{thetavmetd}) can be rewritten as
% \begin{equation*}
%  \begin{array}{ccl}
%  {{\bm{\theta}}}_{t+1}&\leftarrow& {{\bm{\theta}}}_t+\alpha_t (F_t \rho_t\delta_t - \omega_t){\bm{{\bm{\phi}}}}_t -\alpha_t \omega_{t+1}{\bm{{\bm{\phi}}}}_t\\
%   &=&{{\bm{\theta}}}_{t}+\alpha_t(F_t\rho_t\delta_t-\mathbb{E}_{\mu}[F_t\rho_t\delta_t|{{\bm{\theta}}}_t]){\bm{{\bm{\phi}}}}_t\\
%  &=&{{\bm{\theta}}}_t+\alpha_t F_t \rho_t (r_{t+1}+\gamma {{\bm{\theta}}}_t^{\top}{\bm{{\bm{\phi}}}}_{t+1}-{{\bm{\theta}}}_t^{\top}{\bm{{\bm{\phi}}}}_t){\bm{{\bm{\phi}}}}_t\\
%  & & \hspace{2em} -\alpha_t \mathbb{E}_{\mu}[F_t \rho_t \delta_t]{\bm{{\bm{\phi}}}}_t\\
%  &=& {{\bm{\theta}}}_t+\alpha_t \{\underbrace{(F_t\rho_tr_{t+1}-\mathbb{E}_{\mu}[F_t\rho_t r_{t+1}]){\bm{{\bm{\phi}}}}_t}_{{b}_{\text{VMETD},t}}\\
%  &&\hspace{-7em}- \underbrace{(F_t\rho_t{\bm{{\bm{\phi}}}}_t({\bm{{\bm{\phi}}}}_t-\gamma{\bm{{\bm{\phi}}}}_{t+1})^{\top}-{\bm{{\bm{\phi}}}}_t\mathbb{E}_{\mu}[F_t\rho_t ({\bm{{\bm{\phi}}}}_t-\gamma{\bm{{\bm{\phi}}}}_{t+1})]^{\top})}_{\textbf{A}_{\text{VMETD},t}}{{\bm{\theta}}}_t\}.
%  \end{array}
% \end{equation*}
% Therefore, 
% \begin{equation*}
%  \begin{array}{ccl}
%   &&\textbf{A}_{\text{VMETD}}\\
%   &=&\lim_{t \rightarrow \infty} \mathbb{E}[\textbf{A}_{\text{VMETD},t}]\\
%   &=& \lim_{t \rightarrow \infty} \mathbb{E}_{\mu}[F_t \rho_t {\bm{{\bm{\phi}}}}_t ({\bm{{\bm{\phi}}}}_t - \gamma {\bm{{\bm{\phi}}}}_{t+1})^{\top}]\\  
%   &&\hspace{1em}- \lim_{t\rightarrow \infty} \mathbb{E}_{\mu}[  {\bm{{\bm{\phi}}}}_t]\mathbb{E}_{\mu}[F_t \rho_t ({\bm{{\bm{\phi}}}}_t - \gamma {\bm{{\bm{\phi}}}}_{t+1})]^{\top}\\
%   &=& \lim_{t \rightarrow \infty} \mathbb{E}_{\mu}[{\bm{{\bm{\phi}}}}_tF_t \rho_t ({\bm{{\bm{\phi}}}}_t - \gamma {\bm{{\bm{\phi}}}}_{t+1})^{\top}]\\   
%   &&\hspace{1em}-\lim_{t \rightarrow \infty} \mathbb{E}_{\mu}[ {\bm{{\bm{\phi}}}}_t]\lim_{t \rightarrow \infty}\mathbb{E}_{\mu}[F_t \rho_t ({\bm{{\bm{\phi}}}}_t - \gamma {\bm{{\bm{\phi}}}}_{t+1})]^{\top}\\
%   && \hspace{-2em}=\sum_{s} d_{\mu}(s)\lim_{t \rightarrow \infty}\mathbb{E}_{\mu}[F_t|S_t = s]\mathbb{E}_{\mu}[\rho_t\bm{{\bm{\phi}}}_t(\bm{{\bm{\phi}}}_t - \gamma \bm{{\bm{\phi}}}_{t+1})^{\top}|S_t= s]\\   
%   &&\hspace{1em}-\sum_{s} d_{\mu}(s)\bm{{\bm{\phi}}}(s)\sum_{s} d_{\mu}(s)\lim_{t \rightarrow \infty}\mathbb{E}_{\mu}[F_t|S_t = s]\\
%   &&\hspace{7em}\mathbb{E}_{\mu}[\rho_t(\bm{{\bm{\phi}}}_t - \gamma \bm{{\bm{\phi}}}_{t+1})^{\top}|S_t = s]\\
%   &=& \sum_{s} f(s)\mathbb{E}_{\pi}[\bm{{\bm{\phi}}}_t(\bm{{\bm{\phi}}}_t- \gamma \bm{{\bm{\phi}}}_{t+1})^{\top}|S_t = s]\\   
%   &&\hspace{1em}-\sum_{s} d_{\mu}(s)\bm{{\bm{\phi}}}(s)\sum_{s} f(s)\mathbb{E}_{\pi}[(\bm{{\bm{\phi}}}_t- \gamma \bm{{\bm{\phi}}}_{t+1})^{\top}|S_t = s]\\
%   &=&\sum_{s} f(s) \bm{\bm{{\bm{\phi}}}}(s)(\bm{\bm{{\bm{\phi}}}}(s) - \gamma \sum_{s'}[\textbf{P}_{\pi}]_{ss'}\bm{\bm{{\bm{\phi}}}}(s'))^{\top}  \\
%   &&-\sum_{s} d_{\mu}(s) {\bm{{\bm{\phi}}}}(s) * \sum_{s} f(s)({\bm{{\bm{\phi}}}}(s) - \gamma \sum_{s'}[\textbf{P}_{\pi}]_{ss'}{\bm{{\bm{\phi}}}}(s'))^{\top}\\
%   &=&{\bm{\bm{\Phi}}}^{\top} \textbf{F} (\textbf{I} - \gamma \textbf{P}_{\pi}) \bm{\bm{\Phi}} - {\bm{\bm{\Phi}}}^{\top} {d}_{\mu} {f}^{\top} (\textbf{I} - \gamma \textbf{P}_{\pi}) \bm{\bm{\Phi}}  \\
%   &=&{\bm{\bm{\Phi}}}^{\top} (\textbf{F} - {d}_{\mu} {f}^{\top}) (\textbf{I} - \gamma \textbf{P}_{\pi}){\bm{\bm{\Phi}}} \\
%   &=&{\bm{\bm{\Phi}}}^{\top} (\textbf{F} (\textbf{I} - \gamma \textbf{P}_{\pi})-{d}_{\mu} {f}^{\top} (\textbf{I} - \gamma \textbf{P}_{\pi})){\bm{\bm{\Phi}}} \\
%   &=&{\bm{\bm{\Phi}}}^{\top} (\textbf{F} (\textbf{I} - \gamma \textbf{P}_{\pi})-{d}_{\mu} {d}_{\mu}^{\top} ){\bm{\bm{\Phi}}},
%  \end{array}
% \end{equation*}
% \begin{equation*}
%  \begin{array}{ccl}
%   &&{b}_{\text{VMETD}}\\
%   &=&\lim_{t \rightarrow \infty} \mathbb{E}[{b}_{\text{VMETD},t}]\\
%   &=& \lim_{t \rightarrow \infty} \mathbb{E}_{\mu}[F_t\rho_tR_{t+1}{\bm{{\bm{\phi}}}}_t]\\
%   &&\hspace{2em} - \lim_{t\rightarrow \infty} \mathbb{E}_{\mu}[{\bm{{\bm{\phi}}}}_t]\mathbb{E}_{\mu}[F_t\rho_kR_{k+1}]\\  
%   &=& \lim_{t \rightarrow \infty} \mathbb{E}_{\mu}[{\bm{{\bm{\phi}}}}_tF_t\rho_tr_{t+1}]\\
%   &&\hspace{2em} - \lim_{t\rightarrow \infty} \mathbb{E}_{\mu}[  {\bm{{\bm{\phi}}}}_t]\mathbb{E}_{\mu}[{\bm{{\bm{\phi}}}}_t]\mathbb{E}_{\mu}[F_t\rho_tr_{t+1}]\\ 
%   &=& \lim_{t \rightarrow \infty} \mathbb{E}_{\mu}[{\bm{{\bm{\phi}}}}_tF_t\rho_tr_{t+1}]\\
%   &&\hspace{2em} - \lim_{t \rightarrow \infty} \mathbb{E}_{\mu}[ {\bm{{\bm{\phi}}}}_t]\lim_{t \rightarrow \infty}\mathbb{E}_{\mu}[F_t\rho_tr_{t+1}]\\  
%   &=&\sum_{s} f(s) {\bm{{\bm{\phi}}}}(s)r_{\pi} - \sum_{s} d_{\mu}(s) {\bm{{\bm{\phi}}}}(s) * \sum_{s} f(s)r_{\pi}  \\
%   &=&\bm{\bm{\bm{\Phi}}}^{\top}(\textbf{F}-{d}_{\mu} {f}^{\top}){r}_{\pi}.
%  \end{array}
% \end{equation*}


\section{Overview}

\revision{In this section, we first explain the foundational concept of Hausdorff distance-based penetration depth algorithms, which are essential for understanding our method (Sec.~\ref{sec:preliminary}).
We then provide a brief overview of our proposed RT-based penetration depth algorithm (Sec.~\ref{subsec:algo_overview}).}



\section{Preliminaries }
\label{sec:Preliminaries}

% Before we introduce our method, we first overview the important basics of 3D dynamic human modeling with Gaussian splatting. Then, we discuss the diffusion-based 3d generation techniques, and how they can be applied to human modeling.
% \ZY{I stopp here. TBC.}
% \subsection{Dynamic human modeling with Gaussian splatting}
\subsection{3D Gaussian Splatting}
3D Gaussian splatting~\cite{kerbl3Dgaussians} is an explicit scene representation that allows high-quality real-time rendering. The given scene is represented by a set of static 3D Gaussians, which are parameterized as follows: Gaussian center $x\in {\mathbb{R}^3}$, color $c\in {\mathbb{R}^3}$, opacity $\alpha\in {\mathbb{R}}$, spatial rotation in the form of quaternion $q\in {\mathbb{R}^4}$, and scaling factor $s\in {\mathbb{R}^3}$. Given these properties, the rendering process is represented as:
\begin{equation}
  I = Splatting(x, c, s, \alpha, q, r),
  \label{eq:splattingGA}
\end{equation}
where $I$ is the rendered image, $r$ is a set of query rays crossing the scene, and $Splatting(\cdot)$ is a differentiable rendering process. We refer readers to Kerbl et al.'s paper~\cite{kerbl3Dgaussians} for the details of Gaussian splatting. 



% \ZY{I would suggest move this part to the method part.}
% GaissianAvatar is a dynamic human generation model based on Gaussian splitting. Given a sequence of RGB images, this method utilizes fitted SMPLs and sampled points on its surface to obtain a pose-dependent feature map by a pose encoder. The pose-dependent features and a geometry feature are fed in a Gaussian decoder, which is employed to establish a functional mapping from the underlying geometry of the human form to diverse attributes of 3D Gaussians on the canonical surfaces. The parameter prediction process is articulated as follows:
% \begin{equation}
%   (\Delta x,c,s)=G_{\theta}(S+P),
%   \label{eq:gaussiandecoder}
% \end{equation}
%  where $G_{\theta}$ represents the Gaussian decoder, and $(S+P)$ is the multiplication of geometry feature S and pose feature P. Instead of optimizing all attributes of Gaussian, this decoder predicts 3D positional offset $\Delta{x} \in {\mathbb{R}^3}$, color $c\in\mathbb{R}^3$, and 3D scaling factor $ s\in\mathbb{R}^3$. To enhance geometry reconstruction accuracy, the opacity $\alpha$ and 3D rotation $q$ are set to fixed values of $1$ and $(1,0,0,0)$ respectively.
 
%  To render the canonical avatar in observation space, we seamlessly combine the Linear Blend Skinning function with the Gaussian Splatting~\cite{kerbl3Dgaussians} rendering process: 
% \begin{equation}
%   I_{\theta}=Splatting(x_o,Q,d),
%   \label{eq:splatting}
% \end{equation}
% \begin{equation}
%   x_o = T_{lbs}(x_c,p,w),
%   \label{eq:LBS}
% \end{equation}
% where $I_{\theta}$ represents the final rendered image, and the canonical Gaussian position $x_c$ is the sum of the initial position $x$ and the predicted offset $\Delta x$. The LBS function $T_{lbs}$ applies the SMPL skeleton pose $p$ and blending weights $w$ to deform $x_c$ into observation space as $x_o$. $Q$ denotes the remaining attributes of the Gaussians. With the rendering process, they can now reposition these canonical 3D Gaussians into the observation space.



\subsection{Score Distillation Sampling}
Score Distillation Sampling (SDS)~\cite{poole2022dreamfusion} builds a bridge between diffusion models and 3D representations. In SDS, the noised input is denoised in one time-step, and the difference between added noise and predicted noise is considered SDS loss, expressed as:

% \begin{equation}
%   \mathcal{L}_{SDS}(I_{\Phi}) \triangleq E_{t,\epsilon}[w(t)(\epsilon_{\phi}(z_t,y,t)-\epsilon)\frac{\partial I_{\Phi}}{\partial\Phi}],
%   \label{eq:SDSObserv}
% \end{equation}
\begin{equation}
    \mathcal{L}_{\text{SDS}}(I_{\Phi}) \triangleq \mathbb{E}_{t,\epsilon} \left[ w(t) \left( \epsilon_{\phi}(z_t, y, t) - \epsilon \right) \frac{\partial I_{\Phi}}{\partial \Phi} \right],
  \label{eq:SDSObservGA}
\end{equation}
where the input $I_{\Phi}$ represents a rendered image from a 3D representation, such as 3D Gaussians, with optimizable parameters $\Phi$. $\epsilon_{\phi}$ corresponds to the predicted noise of diffusion networks, which is produced by incorporating the noise image $z_t$ as input and conditioning it with a text or image $y$ at timestep $t$. The noise image $z_t$ is derived by introducing noise $\epsilon$ into $I_{\Phi}$ at timestep $t$. The loss is weighted by the diffusion scheduler $w(t)$. 
% \vspace{-3mm}

\subsection{Overview of the RTPD Algorithm}\label{subsec:algo_overview}
Fig.~\ref{fig:Overview} presents an overview of our RTPD algorithm.
It is grounded in the Hausdorff distance-based penetration depth calculation method (Sec.~\ref{sec:preliminary}).
%, similar to that of Tang et al.~\shortcite{SIG09HIST}.
The process consists of two primary phases: penetration surface extraction and Hausdorff distance calculation.
We leverage the RTX platform's capabilities to accelerate both of these steps.

\begin{figure*}[t]
    \centering
    \includegraphics[width=0.8\textwidth]{Image/overview.pdf}
    \caption{The overview of RT-based penetration depth calculation algorithm overview}
    \label{fig:Overview}
\end{figure*}

The penetration surface extraction phase focuses on identifying the overlapped region between two objects.
\revision{The penetration surface is defined as a set of polygons from one object, where at least one of its vertices lies within the other object. 
Note that in our work, we focus on triangles rather than general polygons, as they are processed most efficiently on the RTX platform.}
To facilitate this extraction, we introduce a ray-tracing-based \revision{Point-in-Polyhedron} test (RT-PIP), significantly accelerated through the use of RT cores (Sec.~\ref{sec:RT-PIP}).
This test capitalizes on the ray-surface intersection capabilities of the RTX platform.
%
Initially, a Geometry Acceleration Structure (GAS) is generated for each object, as required by the RTX platform.
The RT-PIP module takes the GAS of one object (e.g., $GAS_{A}$) and the point set of the other object (e.g., $P_{B}$).
It outputs a set of points (e.g., $P_{\partial B}$) representing the penetration region, indicating their location inside the opposing object.
Subsequently, a penetration surface (e.g., $\partial B$) is constructed using this point set (e.g., $P_{\partial B}$) (Sec.~\ref{subsec:surfaceGen}).
%
The generated penetration surfaces (e.g., $\partial A$ and $\partial B$) are then forwarded to the next step. 

The Hausdorff distance calculation phase utilizes the ray-surface intersection test of the RTX platform (Sec.~\ref{sec:RT-Hausdorff}) to compute the Hausdorff distance between two objects.
We introduce a novel Ray-Tracing-based Hausdorff DISTance algorithm, RT-HDIST.
It begins by generating GAS for the two penetration surfaces, $P_{\partial A}$ and $P_{\partial B}$, derived from the preceding step.
RT-HDIST processes the GAS of a penetration surface (e.g., $GAS_{\partial A}$) alongside the point set of the other penetration surface (e.g., $P_{\partial B}$) to compute the penetration depth between them.
The algorithm operates bidirectionally, considering both directions ($\partial A \to \partial B$ and $\partial B \to \partial A$).
The final penetration depth between the two objects, A and B, is determined by selecting the larger value from these two directional computations.

%In the Hausdorff distance calculation step, we compute the Hausdorff distance between given two objects using a ray-surface-intersection test. (Sec.~\ref{sec:RT-Hausdorff}) Initially, we construct the GAS for both $\partial A$ and $\partial B$ to utilize the RT-core effectively. The RT-based Hausdorff distance algorithms then determine the Hausdorff distance by processing the GAS of one object (e.g. $GAS_{\partial A}$) and set of the vertices of the other (e.g. $P_{\partial B}$). Following the Hausdorff distance definition (Eq.~\ref{equation:hausdorff_definition}), we compute the Hausdorff distance to both directions ($\partial A \to \partial B$) and ($\partial B \to \partial A$). As a result, the bigger one is the final Hausdorff distance, and also it is the penetration depth between input object $A$ and $B$.


%the proposed RT-based penetration depth calculation pipeline.
%Our proposed methods adopt Tang's Hausdorff-based penetration depth methods~\cite{SIG09HIST}. The pipeline is divided into the penetration surface extraction step and the Hausdorff distance calculation between the penetration surface steps. However, since Tang's approach is not suitable for the RT platform in detail, we modified and applied it with appropriate methods.

%The penetration surface extraction step is extracting overlapped surfaces on other objects. To utilize the RT core, we use the ray-intersection-based PIP(Point-In-Polygon) algorithms instead of collision detection between two objects which Tang et al.~\cite{SIG09HIST} used. (Sec.~\ref{sec:RT-PIP})
%RT core-based PIP test uses a ray-surface intersection test. For purpose this, we generate the GAS(Geometry Acceleration Structure) for each object. RT core-based PIP test takes the GAS of one object (e.g. $GAS_{A}$) and a set of vertex of another one (e.g. $P_{B}$). Then this computes the penetrated vertex set of another one (e.g. $P_{\partial B}$). To calculate the Hausdorff distance, these vertex sets change to objects constructed by penetrated surface (e.g. $\partial B$). Finally, the two generated overlapped surface objects $\partial A$ and $\partial B$ are used in the Hausdorff distance calculation step.
\chapter{Implementation}{\label{ch:implementation}}
In this chapter, we present the implementation of the final product. We start by discussing how the four steps introduced in \hyperref[ch:high_level_approach]{chapter \ref*{ch:high_level_approach}} are integrated. We then outline the main system components of our score follower, presenting each as an independent, self-contained module. We then combine this into an overall system architecture and finally introduce the open-source score renderer used to display the score and evaluate the score follower.       

% \section{Aims and Requirements}
% The overall aim of the score follower was to 


\section{Score Follower Framework Details}
Our score follower conforms to the high-level framework presented in \hyperref[section:score_follower_framework]{section \ref*{section:score_follower_framework}}. In step 1, two score features are extracted from a MIDI file (see \hyperref[subsection:midi]{subsection \ref*{subsection:midi}}), namely MIDI note numbers\footnote{\href{https://inspiredacoustics.com/en/MIDI_note_numbers_and_center_frequencies}{https://inspiredacoustics.com/en/MIDI\_note\_numbers\_and\_center\_frequencies}} (corresponding to pitch) and note onsets (corresponding to duration). In step 2, the audio is streamed (whether from a file or into a microphone) and audioframes that exceed some predefined energy threshold are extracted. Here, audioframes are groups of contiguous audio samples, whose length can be specified by the argument \verb|frame_length|, usually between 800 and 2000 samples. The period between consecutive audioframes can also be defined by the argument \verb|hop_length|, typically between 2000 and 5000 audio samples. In step 3, score following is performed via a `Windowed' Viterbi algorithm (see  \hyperref[subsection:adjusting_viterbi]{subsection \ref*{subsection:adjusting_viterbi}}) which uses the Gaussian Process (GP) log marginal likelihoods (LMLs) for emission probabilities (see \hyperref[section:state_duration_model]{section \ref*{section:state_duration_model}}) and a state duration model for transition probabilities (see \hyperref[section:state_duration_model]{section \ref*{section:state_duration_model}}). Finally, in step 4 we render our results using an adapted version of the open source user interface, \textit{Flippy Qualitative Testbench}.

\section{Following Modes}
Two modes are available to the user: Pre-recorded Mode and Live Mode. The former requires a pre-recorded $\verb|.wav|$ file, whereas the latter takes an input stream of audio via the device's microphone. Note that both modes are still forms of score following, as opposed to score alignment, since in each mode we receive audioframes at the sampling rate, not all at once.\\

Live Mode offers a practical example of a score follower, displaying a score and position marker which a musician can read off while playing. However, this mode is not suitable for evaluation because the input and results cannot be easily replicated. Even ignoring repeatability, Live Mode is not suitable for one-off testing since a musician using this application may be influenced by the movement of the marker. For instance, the performer may speed up if the score follower `gets ahead' or slow down if the position marker lags or `gets lost'. To avoid this, we use Pre-recorded Mode when evaluating the performance of our score follower. Furthermore, Pre-recorded Mode offers the advantage of testing away from the music room, providing the opportunity to evaluate a variety of recordings available online. 

\section{System Architecture}
Our guiding principle for development was to build modular code in order to create a streamlined system where each component performs a specific task independently. This structure facilitates easy testing and debugging. \hyperref[fig:black_box]{Figure \ref*{fig:black_box}} presents a high-level architecture diagram, where each black box abstracts a key component of the score follower. When operating in Pre-recorded Mode, there is the option to stream the recording during run-time, which outputs to the device's speakers (as indicated by the dashed lines).

\begin{figure}[H]
    \centering
    \includegraphics[width=1\textwidth]{figs/Part_4_Implementation_And_Results/black_box.png}
    \caption{Abstracted system architecture diagram displaying inputs in grey, the 4 main components of the score follower in black and the outputs in green.}
    \label{fig:black_box}
\end{figure}

\subsection{Score Preprocessor}
The architecture for the Score Preprocesor is given in \hyperref[fig:score_preprocessor]{Figure \ref*{fig:score_preprocessor}}. First, MIDI note number and note onset times are extracted from each MIDI event. Simultaneous notes can be gathered into states and returned as a time-sorted list of lists called \verb|score|, where each element of the outer list is a list of simultaneous note onsets. Similarly, a list of note durations calculated as the time difference between consecutive states is returned as \verb|times_to_next|. Finally, all covariance matrices are precalculated and stored in a dictionary, where the key of the dictionary is determined by the notes present. This is because the distribution of notes and chords in a score is not random: notes tend to belong to a home \gls{key} and melodies tend to be repeated or related (similar to subject fields in speech processing). Therefore, states tend to be reused often, allowing us to achieve amortised time and space savings (by avoiding repeated calculation of the same covariance matrices). 

\begin{figure}[H]
    \centering
    \includegraphics[width=1\textwidth]{figs/Part_3_Implementation/Stage_2_Alignment/score_preprocessor.png}
    \caption{System architecture diagram representing the Score Preprocessor with inputs in grey, processes in blue and objects in yellow.}
    \label{fig:score_preprocessor}
\end{figure}


\subsection{Audio Preprocessor}
The architecture for the Audio Preprocessor is illustrated in \hyperref[fig:audio_preprocessor]{Figure \ref*{fig:audio_preprocessor}}. In Pre-recorded Mode, the Slicer receives a $\verb|.wav|$ file and returns audioframes separated by the \verb|hop_length|. These audioframes are periodically added to a multiprocessing queue, \verb|AudioFramesQueue|, to simulate real-time score following. In Live Mode, we use the python module \verb|sounddevice| to receive a stream of audio, using a periodic callback function to place audioframes on \verb|AudioFramesQueue|. 

\begin{figure}[H]
    \centering
    \includegraphics[width=1\textwidth]{figs/Part_4_Implementation_And_Results/audio_preprocessor.png}
    \caption{System architecture diagram representing the Audio Preprocessor with inputs in grey, processes in blue and objects in yellow.}
    \label{fig:audio_preprocessor}
\end{figure}

\subsection{Follower and Backend}
The joint Follower and Backend architecture diagram is shown in \hyperref[fig:follwer_and_backend]{Figure \ref*{fig:follwer_and_backend}}. The Viterbi Follower (detailed in \hyperref[subsection:adjusting_viterbi]{section \ref*{subsection:adjusting_viterbi}}) calculates the most probable state in the score, given audioframes continually taken from \verb|AudioFramesQueue|. These states are placed on another multiprocessing queue, the \verb|FollowerOutputQueue|, for the Backend to process and send. This prevents any bottle-necking occurring at the Follower stage. The Backend first sets up a UDP connection and then reads off values from \verb|FollowerOutputQueue|, sending them via UDP packets to the score renderer.

\begin{figure}[H]
    \centering
    \includegraphics[width=1\textwidth]{figs/Part_4_Implementation_And_Results/follower_and_backend.png}
    \caption{System architecture diagram representing the Follower and Backend processes with processes in blue, objects in yellow and outputs in green.}
    \label{fig:follwer_and_backend}
\end{figure}

\subsection{Player}
In Pre-recorded Mode, the Player sets up a new process and begins streaming the recording once the Follower process begins. This provides a baseline for testing purposes, as a trained musician can observe the score position marker and judge how well it matches the music. 

\subsection{Overall System Architecture}
The overall system architecture is presented in \hyperref[fig:overall_system_architecture]{Figure \ref*{fig:overall_system_architecture}}. Since the Follower runs a real-time, time sensitive process, parallelism is employed to reduce the total system latency. We use two \verb|multiprocessing| queues\footnote{\href{https://docs.python.org/3/library/multiprocessing.html}{https://docs.python.org/3/library/multiprocessing.html}} to avoid bottle-necking, which allows us to run 4 concurrent processes (Audio Preprocessor, Follower, Backend, and Audio Player). Hence, this architecture allows the components to run independently of one another to avoid blocking. Furthermore, this allows the system to take advantage of the multiple cores and high computational power offered by most modern machines.  

\begin{figure}[H]
    \centering
    \includegraphics[width=1\textwidth]{figs/Part_4_Implementation_And_Results/overall_score_follower_2.png}
    \caption{System architecture diagram representing the overall score follower running in Pre-recorded mode, with inputs in grey, processes in blue, objects in yellow and outputs in green.}
    \label{fig:overall_system_architecture}
\end{figure}


\section{Rendering Results}{\label{section:renderer}}
To visualise the results of our score follower, we adapted an open source tool for testing different score followers.\footnote{\href{https://github.com/flippy-fyp/flippy-qualitative-testbench/blob/main/README.md}{https://github.com/flippy-fyp/flippy-qualitative-testbench/blob/main/README.md}} \hyperref[fig:flippy_example]{Figure \ref*{fig:flippy_example}} shows the user interface of the score position renderer, where the green bar indicates score position. 

\begin{figure}[H]
    \centering
    \includegraphics{figs/Part_4_Implementation_And_Results/example_renderer.png}
    \caption{Screenshot of the score renderer user interface which displays a score (here we show a keyboard arrangement of \textit{O Haupt voll Blut und Wunden} by Bach). The green marker represents the score follower position.}
    \label{fig:flippy_example}
\end{figure}



\definecolor{darkgreen}{rgb}{0.0, 0.5, 0.0}
\definecolor{violet}{rgb}{0.56, 0.0, 1.0}
\section{Evaluation}
We apply our methodology to derive counterfactual policies for various MDPs, addressing three main research questions: (1) how does our policy's performance compare to the Gumbel-max SCM approach; (2) how do the counterfactual stability and monotonicity assumptions impact the probability bounds; and (3) how fast is our approach compared with the Gumbel-max SCM method?

\begin{figure*}
    \centering
    %
    \resizebox{0.6\textwidth}{!}{
        \begin{tikzpicture}[scale=1.0, every node/.style={scale=1.0}]
            \draw[thick, black] (-3, -0.25) rectangle (10, 0.25);
            %
            \draw[black, line width=1pt] (-2.5, 0.0) -- (-2,0.0);
            \fill[black] (-2.25,0.0) circle (2pt); %
            \node[right] at (-2,0.0) {\small Observed Path};
            
            %
            \draw[blue, line width=1pt] (1.0,0.0) -- (1.5,0.0);
            \node[draw=blue, circle, minimum size=4pt, inner sep=0pt] at (1.25,0.0) {}; %
            \node[right] at (1.5,0.0) {\small Interval CFMDP Policy};
            
            %
            \draw[red, line width=1pt] (5.5,0) -- (6,0);
            \node[red] at (5.75,0) {$\boldsymbol{\times}$}; %
            \node[right] at (6,0) {\small Gumbel-max SCM Policy};
        \end{tikzpicture}
    }\\
    %
    \subfigure[\footnotesize Lowest cumulative reward: Interval CFMDP ($312$), Gumbel-max SCM ($312$)]{%
        \resizebox{0.76\columnwidth}{!}{
             \begin{tikzpicture}
                \begin{axis}[
                    xlabel={$t$},
                    ylabel={Mean reward at time step $t$},
                    title={Optimal Path},
                    grid=both,
                    width=20cm, height=8.5cm,
                    every axis/.style={font=\Huge},
                    %
                ]
                \addplot[
                    color=black, %
                    mark=*, %
                    line width=2pt,
                    mark size=3pt,
                    error bars/.cd,
                    y dir=both, %
                    y explicit, %
                    error bar style={line width=1pt,solid},
                    error mark options={line width=1pt,mark size=4pt,rotate=90}
                ]
                coordinates {
                    (0, 0.0)  +- (0, 0.0)
                    (1, 0.0)  +- (0, 0.0) 
                    (2, 1.0)  +- (0, 0.0) 
                    (3, 1.0)  +- (0, 0.0)
                    (4, 2.0)  +- (0, 0.0)
                    (5, 3.0) +- (0, 0.0)
                    (6, 5.0) +- (0, 0.0)
                    (7, 100.0) +- (0, 0.0)
                    (8, 100.0) +- (0, 0.0)
                    (9, 100.0) +- (0, 0.0)
                };
                %
                \addplot[
                    color=blue, %
                    mark=o, %
                    line width=2pt,
                    mark size=3pt,
                    error bars/.cd,
                    y dir=both, %
                    y explicit, %
                    error bar style={line width=1pt,solid},
                    error mark options={line width=1pt,mark size=4pt,rotate=90}
                ]
                 coordinates {
                    (0, 0.0)  +- (0, 0.0)
                    (1, 0.0)  +- (0, 0.0) 
                    (2, 1.0)  +- (0, 0.0) 
                    (3, 1.0)  +- (0, 0.0)
                    (4, 2.0)  +- (0, 0.0)
                    (5, 3.0) +- (0, 0.0)
                    (6, 5.0) +- (0, 0.0)
                    (7, 100.0) +- (0, 0.0)
                    (8, 100.0) +- (0, 0.0)
                    (9, 100.0) +- (0, 0.0)
                };
                %
                \addplot[
                    color=red, %
                    mark=x, %
                    line width=2pt,
                    mark size=6pt,
                    error bars/.cd,
                    y dir=both, %
                    y explicit, %
                    error bar style={line width=1pt,solid},
                    error mark options={line width=1pt,mark size=4pt,rotate=90}
                ]
                coordinates {
                    (0, 0.0)  +- (0, 0.0)
                    (1, 0.0)  +- (0, 0.0) 
                    (2, 1.0)  +- (0, 0.0) 
                    (3, 1.0)  +- (0, 0.0)
                    (4, 2.0)  +- (0, 0.0)
                    (5, 3.0) +- (0, 0.0)
                    (6, 5.0) +- (0, 0.0)
                    (7, 100.0) +- (0, 0.0)
                    (8, 100.0) +- (0, 0.0)
                    (9, 100.0) +- (0, 0.0)
                };
                \end{axis}
            \end{tikzpicture}
         }
    }
    \hspace{1cm}
    \subfigure[\footnotesize Lowest cumulative reward: Interval CFMDP ($19$), Gumbel-max SCM ($-88$)]{%
         \resizebox{0.76\columnwidth}{!}{
            \begin{tikzpicture}
                \begin{axis}[
                    xlabel={$t$},
                    ylabel={Mean reward at time step $t$},
                    title={Slightly Suboptimal Path},
                    grid=both,
                    width=20cm, height=8.5cm,
                    every axis/.style={font=\Huge},
                    %
                ]
                \addplot[
                    color=black, %
                    mark=*, %
                    line width=2pt,
                    mark size=3pt,
                    error bars/.cd,
                    y dir=both, %
                    y explicit, %
                    error bar style={line width=1pt,solid},
                    error mark options={line width=1pt,mark size=4pt,rotate=90}
                ]
              coordinates {
                    (0, 0.0)  +- (0, 0.0)
                    (1, 1.0)  +- (0, 0.0) 
                    (2, 1.0)  +- (0, 0.0) 
                    (3, 1.0)  +- (0, 0.0)
                    (4, 2.0)  +- (0, 0.0)
                    (5, 3.0) +- (0, 0.0)
                    (6, 3.0) +- (0, 0.0)
                    (7, 2.0) +- (0, 0.0)
                    (8, 2.0) +- (0, 0.0)
                    (9, 4.0) +- (0, 0.0)
                };
                %
                \addplot[
                    color=blue, %
                    mark=o, %
                    line width=2pt,
                    mark size=3pt,
                    error bars/.cd,
                    y dir=both, %
                    y explicit, %
                    error bar style={line width=1pt,solid},
                    error mark options={line width=1pt,mark size=4pt,rotate=90}
                ]
              coordinates {
                    (0, 0.0)  +- (0, 0.0)
                    (1, 1.0)  +- (0, 0.0) 
                    (2, 1.0)  +- (0, 0.0) 
                    (3, 1.0)  +- (0, 0.0)
                    (4, 2.0)  +- (0, 0.0)
                    (5, 3.0) +- (0, 0.0)
                    (6, 3.0) +- (0, 0.0)
                    (7, 2.0) +- (0, 0.0)
                    (8, 2.0) +- (0, 0.0)
                    (9, 4.0) +- (0, 0.0)
                };
                %
                \addplot[
                    color=red, %
                    mark=x, %
                    line width=2pt,
                    mark size=6pt,
                    error bars/.cd,
                    y dir=both, %
                    y explicit, %
                    error bar style={line width=1pt,solid},
                    error mark options={line width=1pt,mark size=4pt,rotate=90}
                ]
                coordinates {
                    (0, 0.0)  +- (0, 0.0)
                    (1, 1.0)  +- (0, 0.0) 
                    (2, 1.0)  +- (0, 0.0) 
                    (3, 1.0)  +- (0, 0.0)
                    (4, 2.0)  += (0, 0.0)
                    (5, 3.0)  += (0, 0.0)
                    (6, 3.17847) += (0, 0.62606746) -= (0, 0.62606746)
                    (7, 2.5832885) += (0, 1.04598233) -= (0, 1.04598233)
                    (8, 5.978909) += (0, 17.60137623) -= (0, 17.60137623)
                    (9, 5.297059) += (0, 27.09227512) -= (0, 27.09227512)
                };
                \end{axis}
            \end{tikzpicture}
         }
    }\\[-1.5pt]
    \subfigure[\footnotesize Lowest cumulative reward: Interval CFMDP ($14$), Gumbel-max SCM ($-598$)]{%
         \resizebox{0.76\columnwidth}{!}{
             \begin{tikzpicture}
                \begin{axis}[
                    xlabel={$t$},
                    ylabel={Mean reward at time step $t$},
                    title={Almost Catastrophic Path},
                    grid=both,
                    width=20cm, height=8.5cm,
                    every axis/.style={font=\Huge},
                    %
                ]
                \addplot[
                    color=black, %
                    mark=*, %
                    line width=2pt,
                    mark size=3pt,
                    error bars/.cd,
                    y dir=both, %
                    y explicit, %
                    error bar style={line width=1pt,solid},
                    error mark options={line width=1pt,mark size=4pt,rotate=90}
                ]
                coordinates {
                    (0, 0.0)  +- (0, 0.0)
                    (1, 1.0)  +- (0, 0.0) 
                    (2, 2.0)  +- (0, 0.0) 
                    (3, 1.0)  +- (0, 0.0)
                    (4, 0.0)  +- (0, 0.0)
                    (5, 1.0) +- (0, 0.0)
                    (6, 2.0) +- (0, 0.0)
                    (7, 2.0) +- (0, 0.0)
                    (8, 3.0) +- (0, 0.0)
                    (9, 2.0) +- (0, 0.0)
                };
                %
                \addplot[
                    color=blue, %
                    mark=o, %
                    line width=2pt,
                    mark size=3pt,
                    error bars/.cd,
                    y dir=both, %
                    y explicit, %
                    error bar style={line width=1pt,solid},
                    error mark options={line width=1pt,mark size=4pt,rotate=90}
                ]
                coordinates {
                    (0, 0.0)  +- (0, 0.0)
                    (1, 1.0)  +- (0, 0.0) 
                    (2, 2.0)  +- (0, 0.0) 
                    (3, 1.0)  +- (0, 0.0)
                    (4, 0.0)  +- (0, 0.0)
                    (5, 1.0) +- (0, 0.0)
                    (6, 2.0) +- (0, 0.0)
                    (7, 2.0) +- (0, 0.0)
                    (8, 3.0) +- (0, 0.0)
                    (9, 2.0) +- (0, 0.0)
                };
                %
                \addplot[
                    color=red, %
                    mark=x, %
                    line width=2pt,
                    mark size=6pt,
                    error bars/.cd,
                    y dir=both, %
                    y explicit, %
                    error bar style={line width=1pt,solid},
                    error mark options={line width=1pt,mark size=4pt,rotate=90}
                ]
                coordinates {
                    (0, 0.0)  +- (0, 0.0)
                    (1, 0.7065655)  +- (0, 0.4553358) 
                    (2, 1.341673)  +- (0, 0.67091621) 
                    (3, 1.122926)  +- (0, 0.61281824)
                    (4, -1.1821935)  +- (0, 13.82444042)
                    (5, -0.952399)  +- (0, 15.35195457)
                    (6, -0.72672) +- (0, 20.33508414)
                    (7, -0.268983) +- (0, 22.77861454)
                    (8, -0.1310835) +- (0, 26.31013314)
                    (9, 0.65806) +- (0, 28.50670214)
                };
                %
            %
            %
            %
            %
            %
            %
            %
            %
            %
            %
            %
            %
            %
            %
            %
            %
            %
            %
                \end{axis}
            \end{tikzpicture}
         }
    }
    \hspace{1cm}
    \subfigure[\footnotesize Lowest cumulative reward: Interval CFMDP ($-698$), Gumbel-max SCM ($-698$)]{%
         \resizebox{0.76\columnwidth}{!}{
            \begin{tikzpicture}
                \begin{axis}[
                    xlabel={$t$},
                    ylabel={Mean reward at time step $t$},
                    title={Catastrophic Path},
                    grid=both,
                    width=20cm, height=8.5cm,
                    every axis/.style={font=\Huge},
                    %
                ]
                \addplot[
                    color=black, %
                    mark=*, %
                    line width=2pt,
                    mark size=3pt,
                    error bars/.cd,
                    y dir=both, %
                    y explicit, %
                    error bar style={line width=1pt,solid},
                    error mark options={line width=1pt,mark size=4pt,rotate=90}
                ]
                coordinates {
                    (0, 1.0)  +- (0, 0.0)
                    (1, 2.0)  +- (0, 0.0) 
                    (2, -100.0)  +- (0, 0.0) 
                    (3, -100.0)  +- (0, 0.0)
                    (4, -100.0)  +- (0, 0.0)
                    (5, -100.0) +- (0, 0.0)
                    (6, -100.0) +- (0, 0.0)
                    (7, -100.0) +- (0, 0.0)
                    (8, -100.0) +- (0, 0.0)
                    (9, -100.0) +- (0, 0.0)
                };
                %
                \addplot[
                    color=blue, %
                    mark=o, %
                    line width=2pt,
                    mark size=3pt,
                    error bars/.cd,
                    y dir=both, %
                    y explicit, %
                    error bar style={line width=1pt,solid},
                    error mark options={line width=1pt,mark size=4pt,rotate=90}
                ]
                coordinates {
                    (0, 0.0)  +- (0, 0.0)
                    (1, 0.504814)  +- (0, 0.49997682) 
                    (2, 0.8439835)  +- (0, 0.76831917) 
                    (3, -8.2709165)  +- (0, 28.93656754)
                    (4, -9.981082)  +- (0, 31.66825363)
                    (5, -12.1776325) +- (0, 34.53463233)
                    (6, -13.556076) +- (0, 38.62845372)
                    (7, -14.574418) +- (0, 42.49603359)
                    (8, -15.1757075) +- (0, 46.41913968)
                    (9, -15.3900395) +- (0, 50.33563368)
                };
                %
                \addplot[
                    color=red, %
                    mark=x, %
                    line width=2pt,
                    mark size=6pt,
                    error bars/.cd,
                    y dir=both, %
                    y explicit, %
                    error bar style={line width=1pt,solid},
                    error mark options={line width=1pt,mark size=4pt,rotate=90}
                ]
                coordinates {
                    (0, 0.0)  +- (0, 0.0)
                    (1, 0.701873)  +- (0, 0.45743556) 
                    (2, 1.1227805)  +- (0, 0.73433129) 
                    (3, -8.7503255)  +- (0, 30.30257976)
                    (4, -10.722092)  +- (0, 33.17618589)
                    (5, -13.10721)  +- (0, 36.0648089)
                    (6, -13.7631645) +- (0, 40.56553451)
                    (7, -13.909043) +- (0, 45.23829402)
                    (8, -13.472517) +- (0, 49.96270296)
                    (9, -12.8278835) +- (0, 54.38618735)
                };
                %
            %
            %
            %
            %
            %
            %
            %
            %
            %
            %
            %
            %
            %
            %
            %
            %
            %
            %
                \end{axis}
            \end{tikzpicture}
         }
    }
    \caption{Average instant reward of CF paths induced by policies on GridWorld $p=0.4$.}
    \label{fig: reward p=0.4}
\end{figure*}

\subsection{Experimental Setup}
To compare policy performance, we measure the average rewards of counterfactual paths induced by our policy and the Gumbel-max policy by uniformly sampling $200$ counterfactual MDPs from the ICFMDP and generating $10,000$ counterfactual paths over each sampled CFMDP. \jl{Since the interval CFMDP depends on the observed path, we select $4$  paths of varying optimality to evaluate how the observed path impacts the performance of both policies: an optimal path, a slightly suboptimal path that could reach the optimal reward with a few changes, a catastrophic path that enters a catastrophic, terminal state with low reward, and an almost catastrophic path that was close to entering a catastrophic state.} When measuring the average probability bound widths and execution time needed to generate the ICFMDPs, we averaged over $20$ randomly generated observed paths
\footnote{Further training details are provided in Appendix \ref{app: training details}, and the code is provided at \href{https://github.com/ddv-lab/robust-cf-inference-in-MDPs}{https://github.com/ddv-lab/robust-cf-inference-in-MDPs}
%
%
.}.

\subsection{GridWorld}
\jl{The GridWorld MDP is a $4 \times 4$ grid where an agent must navigate from the top-left corner to the goal state in the bottom-right corner, avoiding a dangerous terminal state in the centre. At each time step, the agent can move up, down, left, or right, but there is a small probability (controlled by hyper-parameter $p$) of moving in an unintended direction. As the agent nears the goal, the reward for each state increases, culminating in a reward of $+100$ for reaching the goal. Entering the dangerous state results in a penalty of $-100$. We use two versions of GridWorld: a less stochastic version with $p=0.9$ (i.e., $90$\% chance of moving in the chosen direction) and a more stochastic version with $p=0.4$.}

\paragraph{GridWorld ($p=0.9$)}
When $p=0.9$, the counterfactual probability bounds are typically narrow (see Table \ref{tab:nonzero_probs} for average measurements). Consequently, as shown in Figure \ref{fig: reward p=0.9}, both policies are nearly identical and perform similarly well across the optimal, slightly suboptimal, and catastrophic paths.
%
However, for the almost catastrophic path, the interval CFMDP path is more conservative and follows the observed path more closely (as this is where the probability bounds are narrowest), which typically requires one additional step to reach the goal state than the Gumbel-max SCM policy.
%

\paragraph{GridWorld ($p=0.4$)}
\jl{When $p=0.4$, the GridWorld environment becomes more uncertain, increasing the risk of entering the dangerous state even if correct actions are chosen. Thus, as shown in Figure \ref{fig: reward p=0.4}, the interval CFMDP policy adopts a more conservative approach, avoiding deviation from the observed policy if it cannot guarantee higher counterfactual rewards (see the slightly suboptimal and almost catastrophic paths), whereas the Gumbel-max SCM is inconsistent: it can yield higher rewards, but also much lower rewards, reflected in the wide error bars.} For the catastrophic path, both policies must deviate from the observed path to achieve a higher reward and, in this case, perform similarly.
%
%
%
%
\subsection{Sepsis}
The Sepsis MDP \citep{oberst2019counterfactual} simulates trajectories of Sepsis patients. Each state consists of four vital signs (heart rate, blood pressure, oxygen concentration, and glucose levels), categorised as low, normal, or high.
and three treatments that can be toggled on/off at each time step (8 actions in total). Unlike \citet{oberst2019counterfactual}, we scale rewards based on the number of out-of-range vital signs, between $-1000$ (patient dies) and $1000$ (patient discharged). \jl{Like the GridWorld $p=0.4$ experiment, the Sepsis MDP is highly uncertain, as many states are equally likely to lead to optimal and poor outcomes. Thus, as shown in Figure \ref{fig: reward sepsis}, both policies follow the observed optimal and almost catastrophic paths to guarantee rewards are no worse than the observation.} However, improving the catastrophic path requires deviating from the observation. Here, the Gumbel-max SCM policy, on average, performs better than the interval CFMDP policy. But, since both policies have lower bounds clipped at $-1000$, neither policy reliably improves over the observation. In contrast, for the slightly suboptimal path, the interval CFMDP policy performs significantly better, shown by its higher lower bounds. 
Moreover, in these two cases, the worst-case counterfactual path generated by the interval CFMDP policy is better than that of the Gumbel-max SCM policy,
indicating its greater robustness.
%
\begin{figure*}
    \centering
     \resizebox{0.6\textwidth}{!}{
        \begin{tikzpicture}[scale=1.0, every node/.style={scale=1.0}]
            \draw[thick, black] (-3, -0.25) rectangle (10, 0.25);
            %
            \draw[black, line width=1pt] (-2.5, 0.0) -- (-2,0.0);
            \fill[black] (-2.25,0.0) circle (2pt); %
            \node[right] at (-2,0.0) {\small Observed Path};
            
            %
            \draw[blue, line width=1pt] (1.0,0.0) -- (1.5,0.0);
            \node[draw=blue, circle, minimum size=4pt, inner sep=0pt] at (1.25,0.0) {}; %
            \node[right] at (1.5,0.0) {\small Interval CFMDP Policy};
            
            %
            \draw[red, line width=1pt] (5.5,0) -- (6,0);
            \node[red] at (5.75,0) {$\boldsymbol{\times}$}; %
            \node[right] at (6,0) {\small Gumbel-max SCM Policy};
        \end{tikzpicture}
    }\\
    \subfigure[\footnotesize Lowest cumulative reward: Interval CFMDP ($8000$), Gumbel-max SCM ($8000$)]{%
         \resizebox{0.76\columnwidth}{!}{
             \begin{tikzpicture}
                \begin{axis}[
                    xlabel={$t$},
                    ylabel={Mean reward at time step $t$},
                    title={Optimal Path},
                    grid=both,
                    width=20cm, height=8.5cm,
                    every axis/.style={font=\Huge},
                    %
                ]
                \addplot[
                    color=black, %
                    mark=*, %
                    line width=2pt,
                    mark size=3pt,
                ]
                coordinates {
                    (0, -50.0)
                    (1, 50.0)
                    (2, 1000.0)
                    (3, 1000.0)
                    (4, 1000.0)
                    (5, 1000.0)
                    (6, 1000.0)
                    (7, 1000.0)
                    (8, 1000.0)
                    (9, 1000.0)
                };
                %
                \addplot[
                    color=blue, %
                    mark=o, %
                    line width=2pt,
                    mark size=3pt,
                    error bars/.cd,
                    y dir=both, %
                    y explicit, %
                    error bar style={line width=1pt,solid},
                    error mark options={line width=1pt,mark size=4pt,rotate=90}
                ]
                coordinates {
                    (0, -50.0)  +- (0, 0.0)
                    (1, 50.0)  +- (0, 0.0) 
                    (2, 1000.0)  +- (0, 0.0) 
                    (3, 1000.0)  +- (0, 0.0)
                    (4, 1000.0)  +- (0, 0.0)
                    (5, 1000.0) +- (0, 0.0)
                    (6, 1000.0) +- (0, 0.0)
                    (7, 1000.0) +- (0, 0.0)
                    (8, 1000.0) +- (0, 0.0)
                    (9, 1000.0) +- (0, 0.0)
                };
                %
                \addplot[
                    color=red, %
                    mark=x, %
                    line width=2pt,
                    mark size=6pt,
                    error bars/.cd,
                    y dir=both, %
                    y explicit, %
                    error bar style={line width=1pt,solid},
                    error mark options={line width=1pt,mark size=4pt,rotate=90}
                ]
                coordinates {
                    (0, -50.0)  +- (0, 0.0)
                    (1, 50.0)  +- (0, 0.0) 
                    (2, 1000.0)  +- (0, 0.0) 
                    (3, 1000.0)  +- (0, 0.0)
                    (4, 1000.0)  +- (0, 0.0)
                    (5, 1000.0) +- (0, 0.0)
                    (6, 1000.0) +- (0, 0.0)
                    (7, 1000.0) +- (0, 0.0)
                    (8, 1000.0) +- (0, 0.0)
                    (9, 1000.0) +- (0, 0.0)
                };
                %
                \end{axis}
            \end{tikzpicture}
         }
    }
    \hspace{1cm}
    \subfigure[\footnotesize Lowest cumulative reward: Interval CFMDP ($-5980$), Gumbel-max SCM ($-8000$)]{%
         \resizebox{0.76\columnwidth}{!}{
            \begin{tikzpicture}
                \begin{axis}[
                    xlabel={$t$},
                    ylabel={Mean reward at time step $t$},
                    title={Slightly Suboptimal Path},
                    grid=both,
                    width=20cm, height=8.5cm,
                    every axis/.style={font=\Huge},
                    %
                ]
               \addplot[
                    color=black, %
                    mark=*, %
                    line width=2pt,
                    mark size=3pt,
                ]
                coordinates {
                    (0, -50.0)
                    (1, 50.0)
                    (2, -50.0)
                    (3, -50.0)
                    (4, -1000.0)
                    (5, -1000.0)
                    (6, -1000.0)
                    (7, -1000.0)
                    (8, -1000.0)
                    (9, -1000.0)
                };
                %
                \addplot[
                    color=blue, %
                    mark=o, %
                    line width=2pt,
                    mark size=3pt,
                    error bars/.cd,
                    y dir=both, %
                    y explicit, %
                    error bar style={line width=1pt,solid},
                    error mark options={line width=1pt,mark size=4pt,rotate=90}
                ]
                coordinates {
                    (0, -50.0)  +- (0, 0.0)
                    (1, 50.0)  +- (0, 0.0) 
                    (2, -50.0)  +- (0, 0.0) 
                    (3, 20.0631)  +- (0, 49.97539413)
                    (4, 71.206585)  +- (0, 226.02033693)
                    (5, 151.60797) +- (0, 359.23292559)
                    (6, 200.40593) +- (0, 408.86185176)
                    (7, 257.77948) +- (0, 466.10372804)
                    (8, 299.237465) +- (0, 501.82579506)
                    (9, 338.9129) +- (0, 532.06124996)
                };
                %
                \addplot[
                    color=red, %
                    mark=x, %
                    line width=2pt,
                    mark size=6pt,
                    error bars/.cd,
                    y dir=both, %
                    y explicit, %
                    error bar style={line width=1pt,solid},
                    error mark options={line width=1pt,mark size=4pt,rotate=90}
                ]
                coordinates {
                    (0, -50.0)  +- (0, 0.0)
                    (1, 20.00736)  +- (0, 49.99786741) 
                    (2, -12.282865)  +- (0, 267.598755) 
                    (3, -47.125995)  +- (0, 378.41755832)
                    (4, -15.381965)  +- (0, 461.77616558)
                    (5, 41.15459) +- (0, 521.53189262)
                    (6, 87.01595) +- (0, 564.22243126 )
                    (7, 132.62376) +- (0, 607.31338037)
                    (8, 170.168145) +- (0, 641.48013693)
                    (9, 201.813135) +- (0, 667.29441777)
                };
                %
                %
                %
                %
                %
                %
                %
                %
                %
                %
                %
                %
                %
                %
                %
                %
                %
                %
                %
                \end{axis}
            \end{tikzpicture}
         }
    }\\[-1.5pt]
    \subfigure[\footnotesize Lowest cumulative reward: Interval CFMDP ($100$), Gumbel-max SCM ($100$)]{%
         \resizebox{0.76\columnwidth}{!}{
             \begin{tikzpicture}
                \begin{axis}[
                    xlabel={$t$},
                    ylabel={Mean reward at time step $t$},
                    title={Almost Catastrophic Path},
                    grid=both,
                    every axis/.style={font=\Huge},
                    width=20cm, height=8.5cm,
                    %
                ]
               \addplot[
                    color=black, %
                    mark=*, %
                    line width=2pt,
                    mark size=3pt,
                ]
                coordinates {
                    (0, -50.0)
                    (1, 50.0)
                    (2, 50.0)
                    (3, 50.0)
                    (4, -50.0)
                    (5, 50.0)
                    (6, -50.0)
                    (7, 50.0)
                    (8, -50.0)
                    (9, 50.0)
                };
                %
                %
                \addplot[
                    color=blue, %
                    mark=o, %
                    line width=2pt,
                    mark size=3pt,
                    error bars/.cd,
                    y dir=both, %
                    y explicit, %
                    error bar style={line width=1pt,solid},
                    error mark options={line width=1pt,mark size=4pt,rotate=90}
                ]
                coordinates {
                    (0, -50.0)  +- (0, 0.0)
                    (1, 50.0)  +- (0, 0.0) 
                    (2, 50.0)  +- (0, 0.0) 
                    (3, 50.0)  +- (0, 0.0)
                    (4, -50.0)  +- (0, 0.0)
                    (5, 50.0) +- (0, 0.0)
                    (6, -50.0) +- (0, 0.0)
                    (7, 50.0) +- (0, 0.0)
                    (8, -50.0) +- (0, 0.0)
                    (9, 50.0) +- (0, 0.0)
                };
                %
                \addplot[
                    color=red, %
                    mark=x, %
                    line width=2pt,
                    mark size=6pt,
                    error bars/.cd,
                    y dir=both, %
                    y explicit, %
                    error bar style={line width=1pt,solid},
                    error mark options={line width=1pt,mark size=4pt,rotate=90}
                ]
                coordinates {
                    (0, -50.0)  +- (0, 0.0)
                    (1, 50.0)  +- (0, 0.0) 
                    (2, 50.0)  +- (0, 0.0) 
                    (3, 50.0)  +- (0, 0.0)
                    (4, -50.0)  +- (0, 0.0)
                    (5, 50.0) +- (0, 0.0)
                    (6, -50.0) +- (0, 0.0)
                    (7, 50.0) +- (0, 0.0)
                    (8, -50.0) +- (0, 0.0)
                    (9, 50.0) +- (0, 0.0)
                };
                %
                %
                %
                %
                %
                %
                %
                %
                %
                %
                %
                %
                %
                %
                %
                %
                %
                %
                %
                \end{axis}
            \end{tikzpicture}
         }
    }
    \hspace{1cm}
    \subfigure[\footnotesize Lowest cumulative reward: Interval CFMDP ($-7150$), Gumbel-max SCM ($-9050$)]{%
         \resizebox{0.76\columnwidth}{!}{
            \begin{tikzpicture}
                \begin{axis}[
                    xlabel={$t$},
                    ylabel={Mean reward at time step $t$},
                    title={Catastrophic Path},
                    grid=both,
                    width=20cm, height=8.5cm,
                    every axis/.style={font=\Huge},
                    %
                ]
               \addplot[
                    color=black, %
                    mark=*, %
                    line width=2pt,
                    mark size=3pt,
                ]
                coordinates {
                    (0, -50.0)
                    (1, -50.0)
                    (2, -1000.0)
                    (3, -1000.0)
                    (4, -1000.0)
                    (5, -1000.0)
                    (6, -1000.0)
                    (7, -1000.0)
                    (8, -1000.0)
                    (9, -1000.0)
                };
                %
                %
                \addplot[
                    color=blue, %
                    mark=o, %
                    line width=2pt,
                    mark size=3pt,
                    error bars/.cd,
                    y dir=both, %
                    y explicit, %
                    error bar style={line width=1pt,solid},
                    error mark options={line width=1pt,mark size=4pt,rotate=90}
                ]
                coordinates {
                    (0, -50.0)  +- (0, 0.0)
                    (1, -50.0)  +- (0, 0.0) 
                    (2, -50.0)  +- (0, 0.0) 
                    (3, -841.440725)  += (0, 354.24605512) -= (0, 158.559275)
                    (4, -884.98225)  += (0, 315.37519669) -= (0, 115.01775)
                    (5, -894.330425) += (0, 304.88572805) -= (0, 105.669575)
                    (6, -896.696175) += (0, 301.19954514) -= (0, 103.303825)
                    (7, -897.4635) += (0, 299.61791279) -= (0, 102.5365)
                    (8, -897.77595) += (0, 298.80392585) -= (0, 102.22405)
                    (9, -897.942975) += (0, 298.32920557) -= (0, 102.057025)
                };
                %
                \addplot[
                    color=red, %
                    mark=x, %
                    line width=2pt,
                    mark size=6pt,
                    error bars/.cd,
                    y dir=both, %
                    y explicit, %
                    error bar style={line width=1pt,solid},
                    error mark options={line width=1pt,mark size=4pt,rotate=90}
                ]
            coordinates {
                    (0, -50.0)  +- (0, 0.0)
                    (1, -360.675265)  +- (0, 479.39812699) 
                    (2, -432.27629)  +- (0, 510.38620897) 
                    (3, -467.029545)  += (0, 526.36009628) -= (0, 526.36009628)
                    (4, -439.17429)  += (0, 583.96638919) -= (0, 560.82571)
                    (5, -418.82704) += (0, 618.43027478) -= (0, 581.17296)
                    (6, -397.464895) += (0, 652.67322574) -= (0, 602.535105)
                    (7, -378.49052) += (0, 682.85407033) -= (0, 621.50948)
                    (8, -362.654195) += (0, 707.01412023) -= (0, 637.345805)
                    (9, -347.737935) += (0, 729.29076479) -= (0, 652.262065)
                };
                %
                %
                %
                %
                %
                %
                %
                %
                %
                %
                %
                %
                %
                %
                %
                %
                %
                %
                %
                \end{axis}
            \end{tikzpicture}
         }
    }
    \caption{Average instant reward of CF paths induced by policies on Sepsis.}
    \label{fig: reward sepsis}
\end{figure*}

%
%
%
\subsection{Interval CFMDP Bounds}
%
%
Table \ref{tab:nonzero_probs} presents the mean counterfactual probability bound widths (excluding transitions where the upper bound is $0$) for each MDP, averaged over 20 observed paths. We compare the bounds under counterfactual stability (CS) and monotonicity (M) assumptions, CS alone, and no assumptions. This shows that the assumptions marginally reduce the bound widths, indicating the assumptions tighten the bounds without excluding too many causal models, as intended.
\renewcommand{\arraystretch}{1}

\begin{table}
\centering
\caption{Mean width of counterfactual probability bounds}
\resizebox{0.8\columnwidth}{!}{%
\begin{tabular}{|c|c|c|c|}
\hline
\multirow{2}{*}{\textbf{Environment}} & \multicolumn{3}{c|}{\textbf{Assumptions}} \\ \cline{2-4}
 & \textbf{CS + M} & \textbf{CS} & \textbf{None\tablefootnote{\jl{Equivalent to \citet{li2024probabilities}'s bounds (see Section \ref{sec: equivalence with Li}).}}} \\ \hline
\textbf{GridWorld} ($p=0.9$) & 0.0817 & 0.0977 & 0.100 \\ \hline
\textbf{GridWorld} ($p=0.4$) & 0.552  & 0.638  & 0.646 \\ \hline
\textbf{Sepsis} & 0.138 & 0.140 & 0.140 \\ \hline
\end{tabular}
}
\label{tab:nonzero_probs}
\end{table}


\subsection{Execution Times}
Table \ref{tab: times} compares the average time needed to generate the interval CFMDP vs.\ the Gumbel-max SCM CFMDP for 20 observations.
The GridWorld algorithms were run single-threaded, while the Sepsis experiments were run in parallel.
Generating the interval CFMDP is significantly faster as it uses exact analytical bounds, whereas the Gumbel-max CFMDP requires sampling from the Gumbel distribution to estimate counterfactual transition probabilities. \jl{Since constructing the counterfactual MDP models is the main bottleneck in both approaches, ours is more efficient overall and suitable for larger MDPs.}
\begin{table}
\centering
\caption{Mean execution time to generate CFMDPs}
\resizebox{0.99\columnwidth}{!}{%
\begin{tabular}{|c|c|c|}
\hline
\multirow{2}{*}{\textbf{Environment}} & \multicolumn{2}{c|}{\textbf{Mean Execution Time (s)}} \\ \cline{2-3} 
                                      & \textbf{Interval CFMDP} & \textbf{Gumbel-max CFMDP} \\ \hline
\textbf{GridWorld ($p=0.9$) }                  & 0.261                   & 56.1                      \\ \hline
\textbf{GridWorld ($p=0.4$)  }                 & 0.336                   & 54.5                      \\ \hline
\textbf{Sepsis}                                 & 688                     & 2940                      \\ \hline
\end{tabular}%
}
\label{tab: times}
\end{table}

\section{Discussion of Assumptions}\label{sec:discussion}
In this paper, we have made several assumptions for the sake of clarity and simplicity. In this section, we discuss the rationale behind these assumptions, the extent to which these assumptions hold in practice, and the consequences for our protocol when these assumptions hold.

\subsection{Assumptions on the Demand}

There are two simplifying assumptions we make about the demand. First, we assume the demand at any time is relatively small compared to the channel capacities. Second, we take the demand to be constant over time. We elaborate upon both these points below.

\paragraph{Small demands} The assumption that demands are small relative to channel capacities is made precise in \eqref{eq:large_capacity_assumption}. This assumption simplifies two major aspects of our protocol. First, it largely removes congestion from consideration. In \eqref{eq:primal_problem}, there is no constraint ensuring that total flow in both directions stays below capacity--this is always met. Consequently, there is no Lagrange multiplier for congestion and no congestion pricing; only imbalance penalties apply. In contrast, protocols in \cite{sivaraman2020high, varma2021throughput, wang2024fence} include congestion fees due to explicit congestion constraints. Second, the bound \eqref{eq:large_capacity_assumption} ensures that as long as channels remain balanced, the network can always meet demand, no matter how the demand is routed. Since channels can rebalance when necessary, they never drop transactions. This allows prices and flows to adjust as per the equations in \eqref{eq:algorithm}, which makes it easier to prove the protocol's convergence guarantees. This also preserves the key property that a channel's price remains proportional to net money flow through it.

In practice, payment channel networks are used most often for micro-payments, for which on-chain transactions are prohibitively expensive; large transactions typically take place directly on the blockchain. For example, according to \cite{river2023lightning}, the average channel capacity is roughly $0.1$ BTC ($5,000$ BTC distributed over $50,000$ channels), while the average transaction amount is less than $0.0004$ BTC ($44.7k$ satoshis). Thus, the small demand assumption is not too unrealistic. Additionally, the occasional large transaction can be treated as a sequence of smaller transactions by breaking it into packets and executing each packet serially (as done by \cite{sivaraman2020high}).
Lastly, a good path discovery process that favors large capacity channels over small capacity ones can help ensure that the bound in \eqref{eq:large_capacity_assumption} holds.

\paragraph{Constant demands} 
In this work, we assume that any transacting pair of nodes have a steady transaction demand between them (see Section \ref{sec:transaction_requests}). Making this assumption is necessary to obtain the kind of guarantees that we have presented in this paper. Unless the demand is steady, it is unreasonable to expect that the flows converge to a steady value. Weaker assumptions on the demand lead to weaker guarantees. For example, with the more general setting of stochastic, but i.i.d. demand between any two nodes, \cite{varma2021throughput} shows that the channel queue lengths are bounded in expectation. If the demand can be arbitrary, then it is very hard to get any meaningful performance guarantees; \cite{wang2024fence} shows that even for a single bidirectional channel, the competitive ratio is infinite. Indeed, because a PCN is a decentralized system and decisions must be made based on local information alone, it is difficult for the network to find the optimal detailed balance flow at every time step with a time-varying demand.  With a steady demand, the network can discover the optimal flows in a reasonably short time, as our work shows.

We view the constant demand assumption as an approximation for a more general demand process that could be piece-wise constant, stochastic, or both (see simulations in Figure \ref{fig:five_nodes_variable_demand}).
We believe it should be possible to merge ideas from our work and \cite{varma2021throughput} to provide guarantees in a setting with random demands with arbitrary means. We leave this for future work. In addition, our work suggests that a reasonable method of handling stochastic demands is to queue the transaction requests \textit{at the source node} itself. This queuing action should be viewed in conjunction with flow-control. Indeed, a temporarily high unidirectional demand would raise prices for the sender, incentivizing the sender to stop sending the transactions. If the sender queues the transactions, they can send them later when prices drop. This form of queuing does not require any overhaul of the basic PCN infrastructure and is therefore simpler to implement than per-channel queues as suggested by \cite{sivaraman2020high} and \cite{varma2021throughput}.

\subsection{The Incentive of Channels}
The actions of the channels as prescribed by the DEBT control protocol can be summarized as follows. Channels adjust their prices in proportion to the net flow through them. They rebalance themselves whenever necessary and execute any transaction request that has been made of them. We discuss both these aspects below.

\paragraph{On Prices}
In this work, the exclusive role of channel prices is to ensure that the flows through each channel remains balanced. In practice, it would be important to include other components in a channel's price/fee as well: a congestion price  and an incentive price. The congestion price, as suggested by \cite{varma2021throughput}, would depend on the total flow of transactions through the channel, and would incentivize nodes to balance the load over different paths. The incentive price, which is commonly used in practice \cite{river2023lightning}, is necessary to provide channels with an incentive to serve as an intermediary for different channels. In practice, we expect both these components to be smaller than the imbalance price. Consequently, we expect the behavior of our protocol to be similar to our theoretical results even with these additional prices.

A key aspect of our protocol is that channel fees are allowed to be negative. Although the original Lightning network whitepaper \cite{poon2016bitcoin} suggests that negative channel prices may be a good solution to promote rebalancing, the idea of negative prices in not very popular in the literature. To our knowledge, the only prior work with this feature is \cite{varma2021throughput}. Indeed, in papers such as \cite{van2021merchant} and \cite{wang2024fence}, the price function is explicitly modified such that the channel price is never negative. The results of our paper show the benefits of negative prices. For one, in steady state, equal flows in both directions ensure that a channel doesn't loose any money (the other price components mentioned above ensure that the channel will only gain money). More importantly, negative prices are important to ensure that the protocol selectively stifles acyclic flows while allowing circulations to flow. Indeed, in the example of Section \ref{sec:flow_control_example}, the flows between nodes $A$ and $C$ are left on only because the large positive price over one channel is canceled by the corresponding negative price over the other channel, leading to a net zero price.

Lastly, observe that in the DEBT control protocol, the price charged by a channel does not depend on its capacity. This is a natural consequence of the price being the Lagrange multiplier for the net-zero flow constraint, which also does not depend on the channel capacity. In contrast, in many other works, the imbalance price is normalized by the channel capacity \cite{ren2018optimal, lin2020funds, wang2024fence}; this is shown to work well in practice. The rationale for such a price structure is explained well in \cite{wang2024fence}, where this fee is derived with the aim of always maintaining some balance (liquidity) at each end of every channel. This is a reasonable aim if a channel is to never rebalance itself; the experiments of the aforementioned papers are conducted in such a regime. In this work, however, we allow the channels to rebalance themselves a few times in order to settle on a detailed balance flow. This is because our focus is on the long-term steady state performance of the protocol. This difference in perspective also shows up in how the price depends on the channel imbalance. \cite{lin2020funds} and \cite{wang2024fence} advocate for strictly convex prices whereas this work and \cite{varma2021throughput} propose linear prices.

\paragraph{On Rebalancing} 
Recall that the DEBT control protocol ensures that the flows in the network converge to a detailed balance flow, which can be sustained perpetually without any rebalancing. However, during the transient phase (before convergence), channels may have to perform on-chain rebalancing a few times. Since rebalancing is an expensive operation, it is worthwhile discussing methods by which channels can reduce the extent of rebalancing. One option for the channels to reduce the extent of rebalancing is to increase their capacity; however, this comes at the cost of locking in more capital. Each channel can decide for itself the optimum amount of capital to lock in. Another option, which we discuss in Section \ref{sec:five_node}, is for channels to increase the rate $\gamma$ at which they adjust prices. 

Ultimately, whether or not it is beneficial for a channel to rebalance depends on the time-horizon under consideration. Our protocol is based on the assumption that the demand remains steady for a long period of time. If this is indeed the case, it would be worthwhile for a channel to rebalance itself as it can make up this cost through the incentive fees gained from the flow of transactions through it in steady state. If a channel chooses not to rebalance itself, however, there is a risk of being trapped in a deadlock, which is suboptimal for not only the nodes but also the channel.

\section{Conclusion}
This work presents DEBT control: a protocol for payment channel networks that uses source routing and flow control based on channel prices. The protocol is derived by posing a network utility maximization problem and analyzing its dual minimization. It is shown that under steady demands, the protocol guides the network to an optimal, sustainable point. Simulations show its robustness to demand variations. The work demonstrates that simple protocols with strong theoretical guarantees are possible for PCNs and we hope it inspires further theoretical research in this direction.
\section{RELATED WORK}
\label{sec:relatedwork}
In this section, we describe the previous works related to our proposal, which are divided into two parts. In Section~\ref{sec:relatedwork_exoplanet}, we present a review of approaches based on machine learning techniques for the detection of planetary transit signals. Section~\ref{sec:relatedwork_attention} provides an account of the approaches based on attention mechanisms applied in Astronomy.\par

\subsection{Exoplanet detection}
\label{sec:relatedwork_exoplanet}
Machine learning methods have achieved great performance for the automatic selection of exoplanet transit signals. One of the earliest applications of machine learning is a model named Autovetter \citep{MCcauliff}, which is a random forest (RF) model based on characteristics derived from Kepler pipeline statistics to classify exoplanet and false positive signals. Then, other studies emerged that also used supervised learning. \cite{mislis2016sidra} also used a RF, but unlike the work by \citet{MCcauliff}, they used simulated light curves and a box least square \citep[BLS;][]{kovacs2002box}-based periodogram to search for transiting exoplanets. \citet{thompson2015machine} proposed a k-nearest neighbors model for Kepler data to determine if a given signal has similarity to known transits. Unsupervised learning techniques were also applied, such as self-organizing maps (SOM), proposed \citet{armstrong2016transit}; which implements an architecture to segment similar light curves. In the same way, \citet{armstrong2018automatic} developed a combination of supervised and unsupervised learning, including RF and SOM models. In general, these approaches require a previous phase of feature engineering for each light curve. \par

%DL is a modern data-driven technology that automatically extracts characteristics, and that has been successful in classification problems from a variety of application domains. The architecture relies on several layers of NNs of simple interconnected units and uses layers to build increasingly complex and useful features by means of linear and non-linear transformation. This family of models is capable of generating increasingly high-level representations \citep{lecun2015deep}.

The application of DL for exoplanetary signal detection has evolved rapidly in recent years and has become very popular in planetary science.  \citet{pearson2018} and \citet{zucker2018shallow} developed CNN-based algorithms that learn from synthetic data to search for exoplanets. Perhaps one of the most successful applications of the DL models in transit detection was that of \citet{Shallue_2018}; who, in collaboration with Google, proposed a CNN named AstroNet that recognizes exoplanet signals in real data from Kepler. AstroNet uses the training set of labelled TCEs from the Autovetter planet candidate catalog of Q1–Q17 data release 24 (DR24) of the Kepler mission \citep{catanzarite2015autovetter}. AstroNet analyses the data in two views: a ``global view'', and ``local view'' \citep{Shallue_2018}. \par


% The global view shows the characteristics of the light curve over an orbital period, and a local view shows the moment at occurring the transit in detail

%different = space-based

Based on AstroNet, researchers have modified the original AstroNet model to rank candidates from different surveys, specifically for Kepler and TESS missions. \citet{ansdell2018scientific} developed a CNN trained on Kepler data, and included for the first time the information on the centroids, showing that the model improves performance considerably. Then, \citet{osborn2020rapid} and \citet{yu2019identifying} also included the centroids information, but in addition, \citet{osborn2020rapid} included information of the stellar and transit parameters. Finally, \citet{rao2021nigraha} proposed a pipeline that includes a new ``half-phase'' view of the transit signal. This half-phase view represents a transit view with a different time and phase. The purpose of this view is to recover any possible secondary eclipse (the object hiding behind the disk of the primary star).


%last pipeline applies a procedure after the prediction of the model to obtain new candidates, this process is carried out through a series of steps that include the evaluation with Discovery and Validation of Exoplanets (DAVE) \citet{kostov2019discovery} that was adapted for the TESS telescope.\par
%



\subsection{Attention mechanisms in astronomy}
\label{sec:relatedwork_attention}
Despite the remarkable success of attention mechanisms in sequential data, few papers have exploited their advantages in astronomy. In particular, there are no models based on attention mechanisms for detecting planets. Below we present a summary of the main applications of this modeling approach to astronomy, based on two points of view; performance and interpretability of the model.\par
%Attention mechanisms have not yet been explored in all sub-areas of astronomy. However, recent works show a successful application of the mechanism.
%performance

The application of attention mechanisms has shown improvements in the performance of some regression and classification tasks compared to previous approaches. One of the first implementations of the attention mechanism was to find gravitational lenses proposed by \citet{thuruthipilly2021finding}. They designed 21 self-attention-based encoder models, where each model was trained separately with 18,000 simulated images, demonstrating that the model based on the Transformer has a better performance and uses fewer trainable parameters compared to CNN. A novel application was proposed by \citet{lin2021galaxy} for the morphological classification of galaxies, who used an architecture derived from the Transformer, named Vision Transformer (VIT) \citep{dosovitskiy2020image}. \citet{lin2021galaxy} demonstrated competitive results compared to CNNs. Another application with successful results was proposed by \citet{zerveas2021transformer}; which first proposed a transformer-based framework for learning unsupervised representations of multivariate time series. Their methodology takes advantage of unlabeled data to train an encoder and extract dense vector representations of time series. Subsequently, they evaluate the model for regression and classification tasks, demonstrating better performance than other state-of-the-art supervised methods, even with data sets with limited samples.

%interpretation
Regarding the interpretability of the model, a recent contribution that analyses the attention maps was presented by \citet{bowles20212}, which explored the use of group-equivariant self-attention for radio astronomy classification. Compared to other approaches, this model analysed the attention maps of the predictions and showed that the mechanism extracts the brightest spots and jets of the radio source more clearly. This indicates that attention maps for prediction interpretation could help experts see patterns that the human eye often misses. \par

In the field of variable stars, \citet{allam2021paying} employed the mechanism for classifying multivariate time series in variable stars. And additionally, \citet{allam2021paying} showed that the activation weights are accommodated according to the variation in brightness of the star, achieving a more interpretable model. And finally, related to the TESS telescope, \citet{morvan2022don} proposed a model that removes the noise from the light curves through the distribution of attention weights. \citet{morvan2022don} showed that the use of the attention mechanism is excellent for removing noise and outliers in time series datasets compared with other approaches. In addition, the use of attention maps allowed them to show the representations learned from the model. \par

Recent attention mechanism approaches in astronomy demonstrate comparable results with earlier approaches, such as CNNs. At the same time, they offer interpretability of their results, which allows a post-prediction analysis. \par


\section{Conclusion}
In this work, we propose a simple yet effective approach, called SMILE, for graph few-shot learning with fewer tasks. Specifically, we introduce a novel dual-level mixup strategy, including within-task and across-task mixup, for enriching the diversity of nodes within each task and the diversity of tasks. Also, we incorporate the degree-based prior information to learn expressive node embeddings. Theoretically, we prove that SMILE effectively enhances the model's generalization performance. Empirically, we conduct extensive experiments on multiple benchmarks and the results suggest that SMILE significantly outperforms other baselines, including both in-domain and cross-domain few-shot settings.

\newpage


% %%
% %% The next two lines define the bibliography style to be used, and
%% the bibliography file.
\bibliographystyle{ACM-Reference-Format}
\bibliography{main}
%\newpage

\end{document}
