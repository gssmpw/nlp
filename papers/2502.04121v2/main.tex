\documentclass[a4paper, amsfonts, amssymb, amsmath, onecolumn, showkeys, nofootinbib, twoside, superscriptaddress]{revtex4-2}
\usepackage[english]{babel}
\usepackage[utf8]{inputenc}
\usepackage[colorinlistoftodos, color=green!40, prependcaption]{todonotes}
\usepackage{cancel,amsmath}
\usepackage{amsthm}
\usepackage{mathtools}
\usepackage{physics}
\usepackage{xcolor}
\usepackage{graphicx}
\usepackage{float}
\usepackage{soul}

\usepackage{gensymb}
\usepackage{multirow}
\usepackage[left=23mm,right=13mm,top=35mm,columnsep=15pt]{geometry} 
\usepackage{adjustbox}
\usepackage{placeins}
\usepackage[T1]{fontenc}

\usepackage{csquotes}

\usepackage[pdftex, pdftitle={Article}, pdfauthor={Author}]{hyperref} % For hyperlinks in the PDF

\renewcommand\floatpagefraction{0.8} %% default value: 0.5
\renewcommand\topfraction{0.8}       %% default value: 0.7
\bibliographystyle{apsrev4-1}

% \usepackage{changes}
% \usepackage[final]{changes}

% cleveref! \cref{x} gives the result of "ref type"~\ref{x}
\usepackage[capitalize]{cleveref}
\DeclareMathOperator*{\argmax}{arg\,max}


\begin{document}
\title{First-Passage Approach to Optimizing Perturbations for Improved Training of Machine Learning Models}


\author{Sagi Meir}
    \affiliation{School of Chemistry, Tel Aviv University, Tel Aviv 6997801, Israel.}
    \affiliation{The Center for Physics and Chemistry of Living Systems, Tel Aviv University, Tel Aviv 6997801, Israel.}
\author{Tommer D. Keidar}
    \affiliation{School of Chemistry, Tel Aviv University, Tel Aviv 6997801, Israel.}
    \affiliation{The Center for Physics and Chemistry of Living Systems, Tel Aviv University, Tel Aviv 6997801, Israel.}
\author{Shlomi Reuveni}
    \affiliation{School of Chemistry, Tel Aviv University, Tel Aviv 6997801, Israel.}
    \affiliation{The Center for Physics and Chemistry of Living Systems, Tel Aviv University, Tel Aviv 6997801, Israel.}
    \affiliation{The Center for Computational Molecular and Materials Science, Tel Aviv University, Tel Aviv 6997801, Israel.}
\author{Barak Hirshberg}
    \email{hirshb@tauex.tau.ac.il}
    \affiliation{School of Chemistry, Tel Aviv University, Tel Aviv 6997801, Israel.}
    \affiliation{The Center for Physics and Chemistry of Living Systems, Tel Aviv University, Tel Aviv 6997801, Israel.}
    \affiliation{The Center for Computational Molecular and Materials Science, Tel Aviv University, Tel Aviv 6997801, Israel.}


\begin{abstract}
    Machine learning models have become indispensable tools in applications across the physical sciences. Their training is often time-consuming, vastly exceeding the inference timescales. Several protocols have been developed to perturb the learning process and improve the training, such as shrink and perturb, warm restarts, and stochastic resetting. For classifiers, these perturbations have been shown to result in enhanced speedups or improved generalization. However, the design of such perturbations is usually done \textit{ad hoc} by intuition and trial and error. To rationally optimize training protocols, we frame them as first-passage processes and consider their response to perturbations. 
    We show that if the unperturbed learning process reaches a quasi-steady state, the response at a single perturbation frequency can predict the behavior at a wide range of frequencies. 
    We employ this approach to a CIFAR-10 classifier using the ResNet-18 model and  identify a useful perturbation and frequency among several possibilities. 
    Our work allows optimization of perturbations for improving the training of machine learning models using a first-passage approach.
\end{abstract}

\keywords{machine learning, neural networks, perturbations, stochastic resetting, first-passage, response theory}

\maketitle


\section{Introduction}
Machine learning algorithms and deep neural network (NN) models have become powerful tools across chemistry and physics to tackle challenges that were once computationally or experimentally prohibitive. Examples include protein structure prediction~\cite{jumper2021highly}, entropy calculation for physical systems~\cite{nir2020machine}, machine learning potentials~\cite{behler2007generalized}, and many others~\cite{carleo2017solving, noe2019boltzmann, mate2024neural, anelli2024robust, lagemann2021deep, ravuri2021skilful, wang2020machine, rupp2012fast, tsai2020learning, geiger2013neural}. Unfortunately, these capabilities come with a time-consuming price spent on model training. Second-order optimization algorithms are unfeasible for deep NN~\cite{fukumizu2000local,loshchilov2016sgdr}, and currently, stochastic gradient descent (SGD) and its variants, remain a cornerstone of NN training~\cite{peng2019accelerating, loshchilov2016sgdr, nguyen2019machine, rumelhart1986learning, duchi2011adaptive, sutskever2013importance, zeiler2012adadelta, kingma2014adam}. 

SGD by itself can be slow and prone to getting stuck in local minima. Several common practices are used to address these limitations. First is the incorporation of momentum and adaptive learning rates as done by modern optimizers, usually leading to faster convergence and improved performance~\cite{sutskever2013importance, zeiler2012adadelta, kingma2014adam, leimkuhler2019partitioned}. The second, which is the focus of this work, is to ``perturb'' the training process, which may lead to speedup or improved generalization. Examples include shrink \& perturb (S\&P)~\cite{ash2020warm, zaidi2023does}, warm restarts~\cite{loshchilov2016sgdr}, stochastic resetting (SR)~\cite{bae2024stochastic} and continual backpropagation~\cite{dohare2024loss}. 

Ash et al.~\cite{ash2020warm} first suggested the S\&P protocol to improve generalization during online learning. More recently, Zaidi et al.~\cite{zaidi2023does} investigated when such re-initialization might help the training, and concluded that while it is clear S\&P is helpful in some cases, a general theory of why it works is missing. Bae et al.~\cite{bae2024stochastic} applied SR  to a dynamically-updated checkpoint during the training of an NN with a resetting time sampled from an exponential distribution. They concluded that their strategy decreases overfitting on noisy labels and leads to better generalization. Yet, predicting the optimal resetting rate a priori was not possible. Loshchilov et al.~\cite{loshchilov2016sgdr} showed that a protocol with warm restarts converges up to four times faster to the same test accuracy compared to standard SGD training. 
Overall, despite their usefulness, a theory capable of predicting the effect of such perturbation protocols is missing, and their design is often done by intuition and empirical trial and error. 

The resemblance of SGD to Langevin dynamics~\cite{feng2021inverse,cheng2020stochastic} has been utilized to describe and analyze the learning process of NNs with concepts and methods from thermodynamics and statistical physics~\cite{seung1992statistical, choromanska2015loss, alemi2018therml, stephan2017stochastic, feng2021inverse, jules2023charting, zdeborova2020understanding, karniadakis2021physics, carleo2019machine}.
For example, Feng et al. found an inverse fluctuation-dissipation relation between weights' fluctuations and the flatness of the loss landscape. Based on their finding, they developed an algorithm that delays catastrophic forgetting in sequential learning tasks~\cite{feng2021inverse}.

Another example is the work of Stephan et al. where they showed that SGD with a constant learning rate can be used as an approximate Bayesian posterior inference algorithm~\cite{stephan2017stochastic}. Their result was obtained by viewing SGD as a Markov chain with a stationary distribution.
However, despite the advancements in applying statistical mechanical tools in machine learning, they have not been previously used to design perturbations to improve the performance of SGD optimizers.

Recently, Keidar et al. developed a response theory to predict how rare perturbations affect the completion of an arbitrary stochastic process~\cite{keidar2024universal}. Their theory focused on first-passage processes~\cite{Redner_2001,metzler2014first,bray2013persistence,reuveni2016optimal, pal2017first, evans2020stochastic, blumer2022stochastic, blumer2024combining}, in which a stochastic process has a predefined distribution of initial conditions and a well-defined target. The stochastic nature of the process gives rise to a distribution of first-passage times, i.e., the first time the process reaches its target. In this work, we treat the SGD-based training of NNs up to a predefined target accuracy as a first-passage process. We treat protocols such as S\&P, warm restarts, and stochastic resetting as perturbations to the first-passage process that occur every $P$ epochs during the training. We employ the theory of Keidar et al. for the first time to analyze and improve the training of NN models using such perturbations. We show that, given a set of protocols, we can determine which would lead to the highest acceleration and identify the optimal perturbation time interval $P$. We focus on three types of protocols: S\&P, partial re-initialization of small weights (partial SR), and full SR. Finally, we suggest a methodology for testing and analyzing new perturbation protocols.


\section{Theory}

In this work, we treat the NN training in the absence of a perturbation as a stochastic process that is characterized by a propagator $G(\boldsymbol{\theta},t)$, representing the probability of being in state $\boldsymbol{\theta}$ at time $t$. We emphasize that $\boldsymbol{\theta}$ characterizes the overall state of the system, i.e., it may represent the weights, biases, hyperparameters, etc. We focus on first-passage processes (see \cref{fig:LR_Theory}A), and define the first-passage time (FPT)  as the first instance in which the test accuracy reaches a certain threshold, which is treated as an absorbing boundary condition. Namely, once the test accuracy of a model reaches the target, training is stopped. Since it is a stochastic process, there will be an FPT distribution. We denote with $T$ the random variable representing the FPT of the unperturbed training process. We define the survival probability $\Psi_T(t)$ as the fraction of models which did not reach the threshold at time $t$,
\begin{equation}
    \Psi_T(t) = \int_\Theta G(\boldsymbol{\theta},t)\mathrm{d}\boldsymbol{\theta},
\label{eq:Psi-definition}
\end{equation}
where $\Theta$ symbolizes that the integration domain is over all possible states. 


\begin{figure}[h]
\begin{center}
\centerline{\includegraphics[width=0.7\linewidth]{LR_theory_pic.pdf}}
\caption{Training NNs as a first-passage process. Panel A presents the test accuracy as a function of the number of epochs for training without perturbations (orange line). Here, $T$ is the FPT to reach the target accuracy (dashed blue line). Panel B presents the test accuracy as a function of the number of epochs for training with a perturbation every $P$ epochs (orange line). $T_P$ is the perturbed FPT to the target accuracy (dashed blue line) and $\tau_P(\boldsymbol{\theta})$ is the residual time to reach the target accuracy after the first perturbation.}
\label{fig:LR_Theory}
\end{center}
\end{figure}

\cref{fig:LR_Theory}B shows the training process with a perturbation that is applied every $P$ epochs. We denote by $T_P$ the random variable representing the FPT of the perturbed process to the target accuracy. The perturbation can be of any form, e.g., it can affect the network weights, hyperparameters, activation functions, or any other component of the training. Keidar et al. showed that the perturbed FPT is connected to the unperturbed FPT through~\cite{keidar2024universal},
\begin{equation}
T_P= \begin{cases}T & \text { if } T \leq P, \\ P+\tau_P(\boldsymbol{\theta}) & \text { if } T>P,\end{cases}
\label{eq:T_P_cases_description}
\end{equation}
where $\tau_P(\boldsymbol{\theta})$ is the random variable representing the residual number of epochs it takes to reach the target after the perturbation has been first applied (see \cref{fig:LR_Theory}B). It depends on $\boldsymbol{\theta}$, the state of the network at time $P$, which might be different for every realization of the process. 

Applying the law of total expectation to \cref{eq:T_P_cases_description} (see the derivation in the appendix), we obtain the mean FPT under perturbations, $\mathbb{E}[T_P]$,
\begin{equation}
    \mathbb{E}[T_P]=\sum_{t=0}^{P-1}\Psi_T(t)  + \Psi_T(P) \Bar{\tau}_{P},
\label{eq:LR-prediction}
\end{equation}
where $\Bar{\tau}_{P}$ is given by
\begin{equation}
    \Bar{\tau}_{P} = \int_{\Theta} \mathbb{E}[\tau_P(\boldsymbol{\theta})] \frac{G(\boldsymbol{\theta},P)}{\Psi_T(P)} \mathrm{d}\boldsymbol{\theta}.
\label{eq:tau_P-definition}
\end{equation}
In \cref{eq:tau_P-definition}, $\mathbb{E}[\tau_P(\boldsymbol{\theta})]$ is the average residual time after the first perturbation over all possible noise realizations of the stochastic training process after the perturbation was applied to a given state $\boldsymbol{\theta}$ at time $P$.  Hence, $\Bar{\tau}_{P}$ can be understood as the average of $\mathbb{E}[\tau_P(\boldsymbol{\theta})]$ over all possible $\boldsymbol{\theta}$ generated by the unperturbed process at time $P$ right before the perturbation has been applied.

\cref{eq:LR-prediction} formally decomposes the mean FPT with the perturbation to two contributions. The first term, $\sum_{t=0}^{P-1}\Psi_T(t)$, sets a lower bound on $\mathbb{E}[T_P]$ that only depends on the unperturbed process. The second term encodes all the effects of the perturbation on the mean FPT. These equations allow, in principle to predict the effect of a perturbation on the training for all $P$. 

Consider, for example, the specific case of SR as the perturbation. Then, every $P$ epochs, the state of the system is restarted by resampling the initial conditions for training. As a result, for any training process, $\tau_P(\boldsymbol{\theta})$ does not depend on the state at time $P$, and is simply an independent and identically distributed copy of $T_P$. In that case, $\Bar{\tau}_P=\mathbb{E}[T_P]$, and substituting it into \cref{eq:LR-prediction} gives~\cite{eliazar2020mean},
\begin{equation}
    \mathbb{E}[T_P]_{SR}=\frac{1}{1-\Psi_T(P)}\sum_{t=0}^{P-1}\Psi_T(t) .
\label{eq:SR-prediction}
\end{equation}
\cref{eq:SR-prediction} shows that, for SR, the mean FPT with the perturbation can be predicted entirely from the survival probability of the unperturbed learning process. 
To derive \cref{eq:SR-prediction}, we used the specific properties of SR. For other perturbations, one must assume something about the dynamics of the underlying training process as we do in the next section. 


\section{Results and Discussion}

\subsection{The quasi-steady state hypothesis}
\label{sec:qss-theory}
To proceed beyond \cref{eq:LR-prediction}, we consider a stochastic process in which the propagator reaches a quasi-steady state (QSS) after a typical relaxation time $t_r$,
\begin{equation}
    G(\boldsymbol{\theta},t)\simeq \phi(\boldsymbol{\theta})\Psi_T(t) \quad \text { for } t>t_r.
\label{eq:quasi-steady state}
\end{equation}
In \cref{eq:quasi-steady state}, $\phi(\boldsymbol{\theta})$ is a time-independent probability density function of the network state $\boldsymbol{\theta}$. In other words, a QSS is defined by constant relative populations of all states $\boldsymbol{\theta}$, although the overall number of models that survived declines over time through $\Psi_T(t)$~\cite{nitzan2024chemical}, as illustrated in \cref{fig:qss_drawing}. The QSS approximation is commonly applied to analyze chemical kinetics~\cite{nitzan2024chemical, ji2021stiff, kramers1940brownian, hanggi1986escape, hanggi1990reaction}, but has not been used to explain NN training before. Next, we use it to show that for rare perturbations, the mean residual time $\Bar{\tau}_P$ is independent of $P$. Therefore, by using samples of the training at a single value of $P$, i.e., a single perturbation frequency, we can predict the mean FPT at a wide range of perturbation frequencies.

\begin{figure}[h]
\begin{center}
\centerline{\includegraphics[width=0.69\linewidth]{QSS_drawing_2ndVER.pdf}}
\caption{An illustration of a QSS of a propagator $G(\boldsymbol{\theta},t)$ at times $t_2>t_1>t_r$. At QSS, the probability density function $\phi(\boldsymbol{\theta})$ (purple lines) maintains the same shape, while the fraction of models that survived (blue areas) declines over time through panels A-B, i.e., $\Psi_T(t_1)>\Psi_T(t_2)$.}
\label{fig:qss_drawing}
\end{center}
\end{figure}


To show this, we consider three timescales in the system: the relaxation time $t_r$, the perturbation time interval $P$ and the mean residual time $\Bar{\tau}_P$. Considering a perturbation that is done at a time $P > t_r$, and utilizing QSS, we observe that the perturbation acts on a state sampled from $\phi(\boldsymbol{\theta})$ irrespective of $P$. We can then plug \cref{eq:quasi-steady state} into \cref{eq:tau_P-definition} and get
\begin{equation}
    \Bar{\tau}_{P} = \int_{\Theta} \mathbb{E}[\tau_P(\boldsymbol{\theta})] \phi(\boldsymbol{\theta}) \mathrm{d}\boldsymbol{\theta}.
\label{eq:tau-indep-P_step1}
\end{equation}
Note that the mean residual time for completion after the perturbation is applied still depends on $P$ through $\mathbb{E}[\tau_P(\boldsymbol{\theta})]$. Next, if the perturbation is rare enough, i.e., $P > \Bar{\tau}_P$, we effectively perturb the system only once on average before the training is complete. As a result, $\tau_P(\boldsymbol{\theta})=\tau(\boldsymbol{\theta})$, i.e., does not depend on $P$, leading to
\begin{equation}
    \Bar{\tau}_{P} \approx \Bar{\tau} = \int_{\Theta} \mathbb{E}[\tau(\boldsymbol{\theta})] \phi(\boldsymbol{\theta}) \mathrm{d}\boldsymbol{\theta} \quad \text { for } P>\max\left(t_r, \Bar{\tau}_P\right).
\label{eq:tau-indep-P}
\end{equation}
%\textcolor{red}{Note that we obtained \cref{eq:tau-indep-P} by assuming a QSS, but hypothetically there could be non-QSS scenarios for which \cref{eq:tau-indep-P} would hold. Fortunately, our method applies also to such cases.} 
Plugging \cref{eq:tau-indep-P} into \cref{eq:LR-prediction} leads to
\begin{equation}
    \mathbb{E}[T_P]=\sum_{t=0}^{P-1}\Psi_T(t)  + \Psi_T(P) \Bar{\tau}.
\label{eq:LR-prediction-rarepert}
\end{equation}
\cref{eq:LR-prediction-rarepert} sets the recipe for designing useful perturbations and the frequencies at which to apply them. It tells us to perform simulations applying the perturbation only once at some $P^*>\max\left(t_r, \Bar{\tau}_P\right)$. These simulations provide the unbiased survival function until time $P^*$ and $\Bar{\tau}$. Combining the two, we can predict the mean FPT for all values $\max\left(t_r, \Bar{\tau}_P\right) \leq P \leq P^*$. 

Below, we first test the QSS hypothesis when training a CIFAR-10~\cite{krizhevsky2009learning} classifier using the ResNet-18 model~\cite{he2016deep}. Then, for two perturbations, S\&P and partial SR, we show that indeed $\Bar{\tau}_{P} \approx \Bar{\tau}$ for a wide range of $P$. Finally, we use \cref{eq:LR-prediction-rarepert} to predict the mean FPT under these perturbations at a wide range of $P$, and benchmark them against brute-force training. We show that this procedure can be used to select a perturbation and time interval $P$ that lead to faster training.


\subsection{Experimental test of the quasi-steady-state}
A naive approach to demonstrate a QSS when training an NN model would be to sample several learning trajectories while keeping track of the parameters to obtain $G(\boldsymbol{\theta},t)$. However, it is impractical to keep track of millions of parameters, and the finite sample size introduces noise to the estimation of the distribution of $G(\boldsymbol{\theta},t)$. This noise increases with the dimensionality of $\boldsymbol{\theta}$, making it infeasible to demonstrate a QSS convergence from the trajectory data of $\boldsymbol{\theta}$. 
Instead, we draw inspiration from enhanced sampling of free energy surfaces~\cite{barducci2011metadynamics, valsson2014variational, bussi2020using} and define a collective variable,  $A(\boldsymbol{\theta})$, representing the state of the system and look at its marginal distribution, 
\begin{equation}
    G(A,t) = \int_{\Theta} G(\boldsymbol{\theta},t)\delta(A(\boldsymbol{\theta})-A)\mathrm{d}\boldsymbol{\theta}.
\label{eq:acc_propagator-def}
\end{equation}
In practice, we will use the test set accuracy as a collective variable. 
If $G(\boldsymbol{\theta},t)$ reaches a QSS, and we plug \cref{eq:quasi-steady state} into \cref{eq:acc_propagator-def}, we get a QSS also in $G(A,t)$,
\begin{equation}
    G(A,t) \simeq \Psi_T(t)\int_{\Theta} \phi(\boldsymbol{\theta})\delta(A(\boldsymbol{\theta})-A)\mathrm{d}\boldsymbol{\theta} = \Psi_T(t)\phi(A), \quad \text { for } t>t_r.
\label{eq:acc_quasi-steay}
\end{equation}
In \cref{eq:acc_quasi-steay}, $\phi(A)=\int_{\Theta} \phi(\boldsymbol{\theta})\delta(A(\boldsymbol{\theta})-A)\mathrm{d}\boldsymbol{\theta}$ is the time-independent probability density function of $A$. Therefore, we will use the dynamics of $G(A,t)$ as a proxy to justify the QSS hypothesis.


We used the ResNet-18 architecture as our NN model for the classification task of the CIFAR-10 dataset. We trained 1000 models using random initialization and plotted the test accuracy as a function of the number of epochs, which we will refer to as trajectories. We used standard SGD training, without perturbations. See the Computational Details section for the full training setup. To obtain the propagator $G(A,t)$ from the trajectory data for a specific target test accuracy, we treat it as an absorbing boundary. In other words, trajectories that reached the target accuracy are stopped and excluded from $G(A,t)$ at later times. The survival $\Psi_T(t)$ is estimated by the fraction of models that did not reach the target at epoch $t$ of the training (see \cref{eq:Psi-definition}). 
%For example, if the target test accuracy is set to an unreachable value then $\Psi_T(t)=1,\, \forall t$. Particularly in this case, any QSS is a steady state.

In \cref{fig:acc_densities_over_time}A, we start by analyzing the specific case of a target test accuracy of 75\% (dashed blue line). For this accuracy, no model reaches the target up to 100 epochs, i.e., $\Psi_T(t)=1 \, \forall t$ (blue line \cref{fig:acc_densities_over_time}C). The violins of \cref{fig:acc_densities_over_time}A represent the density distribution of $A$ as a function of time. It narrows down with time, until epoch $\sim20$, and then remains with a fixed shape and height, which suggests that $G(A,t)=\phi(A)$ from that epoch onwards. 
%We define the relaxation time according to the convergence of the mean, i.e., $t_r\sim20$ epochs. 
Besides this qualitative shape comparison, we also quantitatively compare the cumulative distribution functions (CDFs) of $A$ to the average CDF of $A$ over epochs 20 to 100 with the Kolmogorov–Smirnov (KS) test~\cite{massey1951kolmogorov, tiwary2015role, blumer2024short}, shown in \cref{fig:acc_densities_over_time}D (blue line). We define the relaxation time as having a p-value larger than 0.05 in the KS test, which gives a relaxation time of $t_r=17$ epochs. Alternative statistical tests result in similar relaxation times (see the appendix).

\begin{figure}[h]
\centering
    \begin{minipage}[t]{0.78\linewidth}
    \centering
    \includegraphics[width=\linewidth]{acc_densities_over_time.pdf}
    \end{minipage}
    \hspace{-0.0262\linewidth}
    \begin{minipage}[t]{0.2195\linewidth}
    \centering
    \includegraphics[width=\linewidth]{qss_distance.pdf}
    \end{minipage}
\caption{Experimental evidence for a quasi-steady-state. Panels A and B are violin plots of the density distributions of the test accuracy as a function of the number of epochs, for models that did not reach the target. Blue dots represent the mean accuracy and the blue areas are the distributions rotated by $90^{\circ}$. The dashed blue lines are the target accuracies, set to 75\% and 72\% in panels A and B, respectively. In both cases, after $\sim$20-30 epochs, the system reaches a QSS, i.e., $G(A,t)\simeq \phi(A)\Psi_T(t)$. 
Panel A is a special case in which $\Psi_T(t)=1$, while in panel B, $\Psi_T(t)$ decays slowly. 
Panel C shows the survivals $\Psi_T(t)$ for the 75\% (blue), 72\% (orange), and 71.5\% (green) target test accuracies. 
Panel D is the Kolmogorov–Smirnov (KS) test between the CDFs of the accuracy $A$ at different epochs to the average CDF of $A$ over epochs 20 to 100. 
}
\label{fig:acc_densities_over_time}
\end{figure}


In \cref{fig:acc_densities_over_time}B we set the target test accuracy to 72\% which leads to a decay of $\Psi_T(t)$ with the number of epochs, i.e., some trajectories reach the target during the training (orange line in \cref{fig:acc_densities_over_time}C). Although there is a decline in the number of models that did not reach the target, after $\sim30$ epochs the shape of the density distribution again remains approximately fixed while the height decreases, as expected from a QSS (\cref{eq:acc_quasi-steay}). 
%Additional, indication to a QSS is the orange line in \cref{fig:acc_densities_over_time}C that changes its decay behavior at $t_r$ to a slow decay. 
We compare the CDFs of $A$ to the average CDF of $A$ over epochs 20 to 100 with the KS test~\cite{massey1951kolmogorov, tiwary2015role, blumer2024short}, conditioned on model survival.
Similarly to the previous case, we plot the p-value of the KS-test in \cref{fig:acc_densities_over_time}D (orange line) and obtain $t_r = 28$ epochs.
We also checked the QSS hypothesis for a lower target accuracy of $71.5\%$ and plot the survival function and KS-test p-value in \cref{fig:acc_densities_over_time}C,D (green lines), respectively. Again, we obtain a similar relaxation time of $t_r = 26$ epochs.
All these examples are consistent with our hypothesis that the training of a CIFAR-10 classifier using a ResNet-18 model reaches a QSS for various target accuracies.


\subsection{Experimental test that the residual time is $P$-independent}
We now show that if $P$ is large enough, introducing a perturbation would result in $P$-independent residual times, according to \cref{eq:tau-indep-P}. Here, we show it experimentally by focusing on the 72\% target test accuracy  case (\cref{fig:acc_densities_over_time}B) where the system reaches a QSS after $t_r=28$ epochs. We analyze the dependence of the residual time to reach the target on the perturbation time $P$. To do so, we take only the models that did not reach the target after $P=100,\,50,\,20,$ and 10 epochs, introduce the perturbation, and continue the training until reaching the target. We consider two perturbations, S\&P and partial SR. 

The S\&P perturbation takes the weights and biases of the NN at the moment it is applied, shrinks them by a factor $\lambda$ and adds an independent and identically distributed copy sampled from the initial conditions distribution, multiplied by a factor $\gamma$. We used the same values as in ref. \cite{zaidi2023does}, $\lambda=0.4$ and $\gamma=0.1$. The partial SR perturbation takes only a fraction of the parameters of the NN and re-initializes them to an independent and identically distributed copy sampled from the initial conditions distribution. We chose to initialize 30\% of the weights, those with the smallest absolute value at the moment the perturbation is applied.

In \cref{fig:s_and_p_constant_tau}, we first plot the mean test accuracy of the models as a function of the epoch. The top and bottom rows correspond to the S\&P and the partial resetting perturbations, respectively. In all cases, we find that the residual time of the mean trajectory to reach the target is smaller than $P$ and that the perturbation is only applied once before reaching the target. 
As a result, when $P>t_r$, we expect that the mean residual time will be $P$-independent, as explained in \cref{sec:qss-theory}. 
 In panels E and J we center all the mean trajectories with respect to their perturbation time $P$. We find that all curves approximately collapse on top of each other regardless of $P$, reaching the target accuracy after the same number of epochs since the perturbation has been applied (gray area). We emphasize that the residual time is different for the two perturbations, as shown in panels E and J, but is $P$-independent in both cases.

\begin{figure}[h]
\begin{center}
\centerline{\includegraphics[width=0.99\linewidth]{constant_tau.pdf}}
\caption{The mean trajectory of the test accuracy and its top and bottom deciles (shaded areas), for models that did not reach the target after $P$ epochs. From left to right $P=100$, $50$, $20$, $10$ (see legend). The dashed black lines represent 72\% test accuracy (the target). At epoch number $P$ we apply the S\&P protocol (upper row, panels A-E) or the partial SR protocol (bottom row, panels F-J). Panels E and J center the mean trajectories with respect to their $P$. The gray areas are from the moment of perturbation until a the first time that the average trajectory reaches the target accuracy.}
\label{fig:s_and_p_constant_tau}
\end{center}
\end{figure}

Next, instead of plotting the average training curve, we plot the value of the residual time $\Bar{\tau}_P$ as a function of $P$ in \cref{fig:tau_P_vs_P} for a larger range of $P=1\text{--}100$. 
We observe that $\Bar{\tau}_P$ is roughly constant over nearly two orders of magnitude in $P$ for both perturbations. Surprisingly, $\Bar{\tau}_P \approx \Bar{\tau}$ even for values of $P$ that are smaller than the relaxation time, beyond the expected range according to \cref{eq:tau-indep-P}. However, when $P$ becomes smaller than the residual time ($\sim3$ and $\sim5$ for S\&P and partial SR, respectively), denoted by the shaded area in \cref{fig:tau_P_vs_P}, the process is perturbed more than once on average before reaching the target and the residual time is no longer $P$-independent. 


\begin{figure}[h]
\begin{center}
\centerline{\includegraphics[width=0.4\linewidth]{tau_P_vs_P.pdf}}
\caption{The mean residual time $\Bar{\tau}_P$ versus the perturbation time. The values of $\Bar{\tau}_P$ for S\&P (orange circles) and partial SR (green triangles) are approximately constant when $P>\Bar{\tau}_P$ (outside the yellow area). Dashed dotted lines are averages of all $\Bar{\tau}_P<P$.}
\label{fig:tau_P_vs_P}
\end{center}
\end{figure}



\subsection{Prediction of the mean FPT}
In this section, we show how to predict the mean FPT with a perturbation for a wide range of $P$ from as few experiments as possible. 
We begin by assuming that several training trajectories are already available, and we wish to assess whether applying a perturbation is worthwhile, and select at what time interval to apply it. The simplest perturbation to analyze is SR. It comes at no additional cost, through \cref{eq:SR-prediction}, since we already sampled the survival function. To quantify the acceleration in reaching a target accuracy by applying a perturbation, we define the speedup as the ratio between the mean FPT without and with the perturbation, respectively. We plot the predicted speedup for SR in \cref{fig:Speedup} (blue line) and compare it with numerical experiments (blue squares). We find that SR leads to a maximal speedup of $\sim3$ using a perturbation time interval of $\sim20$. 

To go beyond SR to other perturbations that could potentially lead to higher speedups, we must measure their residual time.
A naive approach of simply evaluating $\Bar{\tau}_P$ for every $P$ and using \cref{eq:LR-prediction} would mean running the training for all perturbations at all $P$ values, which is costly. Instead, we would like to use the fact that the residual time is $P$-independent. To that end, we sample the training with the perturbation applied once at some $P^*$ that is large enough. We then use \cref{eq:LR-prediction-rarepert} to predict the mean FPT for a wide range of $P < P^*$. 
We plot the predicted speedup in \cref{fig:Speedup} for the S\&P (orange line) and partial SR (green line) perturbations, using their value of $\Bar{\tau}_{P^*}$ at $P^*=100$. Going from right to left on the $P$ axis, the predictions fit the experimental values of the S\&P (orange circles) and partial SR (green triangles) protocols for $P\geq\Bar{\tau}_{P^*}$, as expected.

Using these predictions, we identify that S\&P leads to a speedup of $\sim16$ at a perturbation time interval of 3, while partial SR leads to a speedup of $\sim8$ at $P=7$. These predictions correctly identify S\&P as the preferred strategy that leads to the highest speedup. Although the experimental maximal speedup is slightly larger ($\sim21$), our predictions are a computationally efficient way of determining useful perturbations and time intervals that lead to high speedups.


\begin{figure}[h]
\begin{center}
\centerline{\includegraphics[width=0.4\linewidth]{Speedup_noMomentum.pdf}}
\caption{The speedup gained by using different perturbation protocols versus the perturbation time. Orange and green solid lines are the theoretical predictions of \cref{eq:LR-prediction-rarepert} for S\&P and partial SR, respectively. For the predictions, we used the value of $\Bar{\tau}_{P^*}$ for $P^*=100$. We plot the predictions only down to $P=\Bar{\tau}_{P^*}$.
The blue solid line is the theoretical prediction for the SR case (\cref{eq:SR-prediction}). Symbols represent brute force training at every value of $P$.}
\label{fig:Speedup}
\end{center}
\end{figure}

We suggest a general methodology that can examine different kinds of perturbations and pick the optimal one and an efficient $P$. Consider for example a classification task for which the architecture of the NN and the hyperparameters of the SGD-based optimization are given. Then, it is common practice to train in parallel an ensemble of models and terminate any training process that did not reach a minimal desired test accuracy after a tolerance time of $P^*$ epochs. 
After this process is completed, we propose to estimate the survival function, $\Psi_T(t)$ for $t\leq P^*$. This is already enough information to evaluate whether SR leads to any acceleration. 
For other perturbations, evaluate first if the training accuracy of the unperturbed process  reaches a QSS and determine $t_r < P^*$ using the KS test. If that is the case, introduce several candidate perturbations to the models that did not reach the target accuracy and measure $\Bar{\tau}_{P^*}$ for each one. 
This way, $\Psi_T(t\leq P^*)$ and $\Bar{\tau}_{P^*}$ are obtained simultaneously for several perturbation protocols. According to \cref{eq:LR-prediction-rarepert}, the perturbation with the minimal $\Bar{\tau}_{P^*}$ is the preferred one. Finally, to obtain an efficient perturbation time interval, plug $\Psi_T(t\leq P^*)$ and $\Bar{\tau}_{P^*}$ into \cref{eq:LR-prediction-rarepert}, and evaluate the speedup for $\max(t_r,\Bar{\tau}_{P^*}) \leq P \leq P^*$. Choose the time interval that leads to the highest speedup. 

\section{Summary and Conclusion}
We developed an approach based on a recent response theory by Keidar et al.~\cite{keidar2024universal} to design useful perturbations for accelerating the training of NNs.  
To that end, we treated the training as a first-passage process to a target test accuracy.
For the case of a CIFAR-10 classifier using the ResNet-18 architecture, we demonstrated that the unperturbed training test accuracy reached a QSS after a relaxation time of 20-30 epochs. We showed that, as a result, we can predict the mean FPT for a wide range of perturbation times from measurements at a single perturbation time. We focused on three perturbations: S\&P, SR and partial SR, but our method is general and can be used for other perturbations.
Lastly, we proposed a strategy to examine different kinds of perturbations and pick the optimal one and an efficient $P$. We showed that this strategy correctly selects S\&P as a better perturbation than partial SR or SR for a CIFAR-10 classifier and identified a useful perturbation time. 
Our work provides a first-passage framework capable of identifying perturbations that result in a speedup when training an NN model up to a target accuracy. It allows for a more rational design of perturbation protocols, based on physical insights. 

\section{Computational details}
The CIFAR-10 dataset contains 50,000 training and 10,000 test RGB images each belonging to one of 10 different classes. The number of images per class across the train and test sets is equal, and each image is 32$\times$32 pixels in size. For classification, we used a common modified version of the ResNet-18 architecture, because the original architecture was designed for much larger images. The modification includes reducing the kernel size of the first convolution layer from 7 to 3, the stride from 2 to 1, and the padding from 3 to 1. We also removed the subsequent max pooling layer to maintain everything else similar to the original architecture (as was done in the work of Zaidi et al.~\cite{zaidi2023does}). 

We did not use any data augmentation techniques except for standard data normalization (by the channel-wise mean and variance of the train images) for both train and test datasets. We initialized the weights and biases with the default initialization of Pytorch, i.e., uniform distribution, and used the FFCV library for faster training~\cite{leclerc2023ffcv}. We trained each model with an SGD optimizer with a learning rate of 0.02. Implementation of the perturbations was straightforward according to their definitions above. The test accuracy curves for each perturbative protocol with $P=20$ are presented in supporting information.


\section*{Acknowledgements}
B.H. acknowledges support from the Israel Science Foundation (grants No. 1037/22 and 1312/22), the Pazy Foundation of the IAEC-UPBC (grant No. 415-2023), and Tel Aviv University Center for Artificial Intelligence and Data Science (TAD).This project has received funding from the European Research Council (ERC) under the European Union’s Horizon 2020 research and innovation program (grant agreement No. 947731 to S.R.). S.M. acknowledges support from the Quantum Science and Technology Center of Tel Aviv University. B.H. and S.M. thank Sheheryar Zaidi for sharing his code. The authors thank Yohai Bar Sinai, Tomer Koren, and Rotem Widman for fruitful discussions.

\bibliography{refs}

\section{Supplementary figures}
\begin{figure}[!htb]
    \centering
    \includegraphics[width=0.5\linewidth]{figures/correct.pdf} 
    \caption{Correct ratio of property-to-molecule generation. We treat the generated molecule as a correct one if $\lvert v' - v \lvert \leq \delta$, where $v'$ is its property value and $v$ is the input value. $\delta$ is set to 0 for HBA, HBD, RotBonds, 0.05 for QED and FSP3, and 5 for TPSA. }
    \label{fig:prop2mol_correct}
\end{figure}
\clearpage
\begin{figure}[!htb]
\centering
\subfigure[QED]{
\includegraphics[width=0.5\linewidth]{figures/QED.pdf} 
}%
\subfigure[HBA]{
\includegraphics[width=0.5\linewidth]{figures/HBA.pdf} 
}
\subfigure[HBD]{
\includegraphics[width=0.5\linewidth]{figures/HBD.pdf} 
}%
\subfigure[Rotatable bonds]{
\includegraphics[width=0.5\linewidth]{figures/RotBonds.pdf} 
}
\caption{Violin plot of basic molecular properties for molecule generation, including QED, the number of hydrogen bond acceptors (HBA), the number of hydrogen bond donors (HBD) and the number of rotatable bonds.}
\label{fig:basic_to_cmpd_violinplot}
\end{figure}

\clearpage
\begin{figure}[!htb]
\centering
\subfigure[QED=0.8, FSP3=0.4]{
\includegraphics[width=0.5\linewidth]{figures/QED=0.8.FSP3=0.4.pdf}
}%
\subfigure[QED=0.8, FSP3=0.6]{
\includegraphics[width=0.5\linewidth]{figures/QED=0.8.FSP3=0.6.pdf}
}
\caption{Heatmap of molecule generation based on QED and fraction of sp³ (FSP3) properties. Each generated compound's QED and FSP3 values are calculated using RDKit and visualized in the heatmap.}
\label{fig:qed_fsp3_joint_optim}
\end{figure}
\begin{figure}[!htb]
    \centering
    \includegraphics[width=0.7\linewidth]{figures/8b_targets.pdf} 
    \caption{Bar plot of the proportion of correct, equal and wrong generated molecules. Molecules evaluated by retrieval and molecules evaluated by docking are distinguished using different colors.}
    \label{fig:binding_docking}
\end{figure}
\begin{figure}[!htb]
    \centering
    \includegraphics[width=0.7\linewidth]{figures/compare.pdf} 
    \caption{Bar plot of the correct ratio of \ourM{} (1B), \ourM{} (8B) and \ourM{} (8x7B) on each target.}
    \label{fig:binding_correct}
\end{figure}

\begin{figure}
    \centering
    \subfigure[\ourM{} (1B)]{
    \includegraphics[width=0.45\linewidth]{figures/mat_uncon_ehull_1b.pdf}
    }%
    \subfigure[\ourM{} (8B)]{
    \includegraphics[width=0.45\linewidth]{figures/mat_uncon_ehull_8b.pdf}
    }%
    \vskip\baselineskip
    \subfigure[\ourM{} (8x7B)]{
    \includegraphics[width=0.45\linewidth]{figures/mat_uncon_ehull_8x7b.pdf}
    }
    \subfigure[Accumulated distribution]{
    \includegraphics[width=0.45\linewidth]{figures/mat_uncon_ehull_accumulate_ehull.pdf}
    }
    \caption{Energy above hull (ehull) distribution for unconditional material generation.}
    \label{fig:mat_uncon}
\end{figure}

\begin{figure}
    \centering
    \subfigure[\ourM{} (1B)]{
    \includegraphics[width=0.45\linewidth]{figures/mat_bulk_to_mat_ehull_1b.pdf}
    }%
    \subfigure[\ourM{} (8B)]{
    \includegraphics[width=0.45\linewidth]{figures/mat_bulk_to_mat_ehull_8b.pdf}
    }%
    \vskip\baselineskip
    \subfigure[\ourM{} (8x7B)]{
    \includegraphics[width=0.45\linewidth]{figures/mat_bulk_to_mat_ehull_8x7b.pdf}
    }
    \subfigure[Accumulated distribution]{
    \includegraphics[width=0.45\linewidth]{figures/mat_bulk_to_mat_accumulate_ehull.pdf}
    }
    \caption{Energy above hull (ehull) distribution for bulk modulus to material generation.}
    \label{fig:mat_bulk_to_mat_ehull}
\end{figure}

\begin{figure}[!htbp]
    \centering
    \includegraphics[width=0.8\linewidth]{figures/mat_novelty.pdf}
    \caption{Novel materials w.r.t generated materials.}
    \label{fig:mat_novelty}
\end{figure}

\clearpage
\begin{figure}[!htbp]
\centering
\includegraphics[width=\linewidth]{figures/caseStudy_smi2iupac_online.pdf}
\caption{We selected SMILES strings from PubChem with IDs 172655007 and 172655008, which were available as of February 24, 2025, and were excluded from our training set. The performance of \ourM{}, DeepSeek-R1 \cite{deepseekai2025r1}, GPT-4o, GPT-4.5-preview, and o3-mini was evaluated for SMILES-to-IUPAC translation. The generated IUPAC names are presented in the accompanying figure. These IUPAC names were subsequently converted back to SMILES for validation. The IUPAC name produced by o3-mini could not be processed due to the high structural complexity of the corresponding molecule. \ourM{} successfully generated the correct result. It is important to emphasize that our objective is not to criticize the limitations of general language models but to better understand their current capabilities and explore how they can be complemented by \ourM{} for enhanced performance. The molecular structures were visualized using the ChemDB Chemoinformatics Portal \cite{Chen2007-il} \url{https://cdb.ics.uci.edu/cgibin/Smi2DepictWeb.py}.}
%Case study on SMILES-to-IUPAC translation. We selected SMILES strings from PubChem with IDs 172655007 and 172655008, which were available as of February 24, 2025, and were not included in our training set. We evaluated the performance of \ourM{}, DeepSeek-R1 \cite{deepseekai2025r1}, GPT-4o, GPT-4.5-preview and  o3-mini for SMILES-to-IUPAC translation, with the generated IUPAC names listed in the figure. These IUPAC names were then converted back to SMILES for validation. We were not unable to process the IUPAC name produced by o3-mini due to the high structural complexity. \ourM{} successfully generated the correct result. Our goal is NOT to highlight the limitations of general language models but to better understand their current capabilities and explore how they can be complemented by \ourM{} for improved performance. The molecules are visualzzed by ChemDB Chemoinformatics Portal \cite{Chen2007-il} \url{https://cdb.ics.uci.edu/cgibin/Smi2DepictWeb.py}  }
\label{fig:case_study_iupac_to_smiles}
\end{figure}



\clearpage

\begin{figure}[!htbp]
\centering
\includegraphics[width=1.0\linewidth]
{figures/NatureLM_retro_example2_rdkit.pdf}
% {figures/NatureLM_retro_example2.pdf}


\caption{Additional examples on retrosynthesis prediction. 
We evaluated the performance of \ourM{}, DeepSeek-R1, and o3-mini-high using a reaction from U.S. Patent ID US11999726B2, granted to Eli Lilly on June 04, 2024. 
The product features two ring systems with a protecting functional group, suggesting that the previous synthesis step likely involved a reaction to connect these rings. Notably, an ether bond links the two rings, with a pyrazine ring on one side and a piperidine ring on the other. Substitution on the pyrazine ring is a common strategy due to its electrophilicity, which often leads to substitution reactions. In this case, NatureLM accurately predicted the cleavage site of the molecule, incorporated a common chlorine atom on the pyrazine ring, and preserved the molecule's stereochemistry, providing a reasonable synthetic strategy. In contrast, both DeepSeek-R1 and o3-mini-high models correctly identified the reactive sites but failed to predict the correct reactants due to poor handling of SMILES representations. For instance, DeepSeek-R1 predicted the pyrazine as pyrimidine, altering the nitrogen atom's position, while o3-mini-high converted the six-membered pyrazine directly into a five-membered imidazole. These errors indicate that these general-purpose language models do not fully understand the relationship between chemical structures and their SMILES representations, hindering their ability to perform accurate reaction predictions.
% \ourM{} successfully proposed the ground-truth reactants from the patent.
}
\label{fig:case_study_reaction2}
\end{figure}

\clearpage 


\begin{figure}[!htpb]
\centering
    \includegraphics[trim=1cm 0 3cm 0, clip, width=\linewidth]{figures/heme_showCase_more.pdf}
    \caption{Additional examples of designing heme-binding proteins based on text or SMILES instructions are shown. The first two rows display results from the text-based design, while the second row corresponds to the SMILES-based design. The yellow models represent structures generated by \ourM{}, whereas the blue models are the reference structures retrieved using the built-in Chimera function. The structures of the generated proteins were predicted using Protenix \cite{Protenix2025}. }
    \label{fig:SI:moreHemeCases}
\end{figure}

\clearpage 




\begin{figure}[!htpb]
\centering
\includegraphics[width=0.6\linewidth]{figures/compare_heme_hemec.png}
\caption{Comparison of the complex structure of the generated protein with heme (yellow model) and heme C (pink model). The protein was obtained using the SMILES-to-protein approach described in Section \ref{sec:heme_case_study}. We observe that they share common structural features. The structures of the generated proteins were predicted using Protenix \cite{Protenix2025}. \\
For the retrieved PDB structure 3MK7 in Fig. \ref{fig:heme_bind_prot}, our generated protein aligns to the pocket region that binds to heme C. To further validate this, we used Protenix to predict the binding of our generated protein to both heme C (PubChem CID: 11987638) and heme. The results demonstrate that heme C fits properly into the designed pocket, supporting the structural compatibility of the generated protein with heme C.\\
This discrepancy arises from the high structural similarity between heme and heme C, as their SMILES representations are nearly identical. Despite this slight misalignment, the output remains biologically relevant because heme-binding proteins often interact with multiple heme derivatives. Furthermore, generating a protein that binds to heme C from the SMILES of heme highlights the algorithm's ability to capture the inherent structural flexibility and functional overlap within the heme family. We will continue improving the algorithm to enhance ligand specificity in future iterations.}
\label{fig:SI:prot_hem_hec}
\end{figure}


\clearpage
\begin{figure}[!htpb]
    \centering
    \includegraphics[trim=5cm 0.5cm 8cm 2cm, clip, width=\linewidth]{figures/comparison_apo_holo.pdf}
    \caption{Comparison of the apo structure of the generated protein, the holo structure in complex with heme, and their aligned structures. Key residues, such as histidine and methionine, occupy similar positions in the pocket region in both the apo and holo structures. This observation suggests that the generated proteins are not only capable of binding heme but also exhibit a structurally pre-formed or conserved binding pocket even in the absence of the ligand. These findings validate the structural plausibility of the designed proteins and their suitability for heme binding.}
    \label{fig:compare_apo_holo}
\end{figure}

\clearpage


\begin{figure}[!h]
\centering
\includegraphics[width=0.75\linewidth]{figures/protein_unconditioned_length_distribution.png}
\caption{Sequence length distribution of generated proteins. The \ourM{} models demonstrate a more natural distribution that closely resembles the reference UR50 sequences, while Mixtral 8x7B and GPT-4 tend to generate shorter sequences.}
\label{fig:protein:unconditioned_generation_sequence_length}
\end{figure}


\begin{figure}
\centering
\includegraphics[width=\linewidth]{figures/prot2rna.boxplot.pdf}
\caption{The distribution of the predicted scores for the RNA sequences in the test set and the generated RNA sequences shows a clear trend. In terms of median values, larger models consistently achieve better predicted scores, indicating stronger binding affinity.}
\label{fig:enter-label}
\end{figure}


\clearpage

\begin{figure}[h]  
    \centering  
    \begin{mdframed}[backgroundcolor=white, linecolor=black, linewidth=1pt]  
    % \textbf{\textcolor{white}{\rule{0pt}{1em}\textcolor{black}{Soluble}}} \\[5pt]
    \textsc{Stable}
    \textit{
    \begin{itemize} % [topsep=0pt]
        \item[-] Please produce a protein sequence that exhibits stability.
        \item[-] I require a stable protein sequence, kindly generate one.
        \item[-] Generate a protein sequence ensuring its stability.
        \item[-] I need a protein sequence that's stable. Please generate it.
        \item[-] Create a stable protein sequence.
        \item[-] Produce a stable protein sequence.
        \item[-] Kindly generate a protein sequence with stability.
        \item[-] I would like you to generate a stable protein sequence.
        \item[-] Please create a protein sequence that ensures stability.
        \item[-] Make a protein sequence that is stable.
    \end{itemize}
    }
    \textsc{Soluble}
    \textit{
    \begin{itemize} % [topsep=0pt]
        \item[-] Generate a soluble protein sequence.
        \item[-] Produce a protein sequence that is soluble.
        \item[-] Create a soluble protein sequence, please.
        \item[-] I require a soluble protein sequence, kindly generate one.
        \item[-] Please produce a protein sequence that exhibits solubility.
        \item[-] Make a protein sequence that is soluble.
        \item[-] Kindly generate a protein sequence with solubility.
        \item[-] I need a protein sequence that's soluble. Please generate it.
        \item[-] I would like you to generate a soluble protein sequence.
        \item[-] Please create a protein sequence that ensures solubility.
    \end{itemize}
    }
    % \textbf{\textcolor{white}{\rule{0pt}{1em}\textcolor{black}{Insoluble}}} \\[5pt]
    \textsc{Insoluble}
    \textit{
    \begin{itemize}
        \item[-] Generate an insoluble protein sequence.
        \item[-] Produce a protein sequence that is not soluble.
        \item[-] Create an insoluble protein sequence, please.
        \item[-] I require an insoluble protein sequence, kindly generate one.
        \item[-] Please produce a protein sequence that lacks solubility.
        \item[-] Make a protein sequence that is not soluble.
        \item[-] Kindly generate a protein sequence without solubility.
        \item[-] I need a protein sequence that's insoluble. Please generate it.
        \item[-] I would like you to generate an insoluble protein sequence.
        \item[-] Please create a protein sequence that ensures insolubility.
    \end{itemize}
    }
    \end{mdframed}  
    \caption{Prompts for conditioned generation} \label{fig:protein:conditioned_prompts_full}
\end{figure} 

\clearpage

\begin{mdframed}[backgroundcolor=white, linecolor=black, linewidth=1pt]
    \textbf{Instruction:} \textit{Tell me about Cas9 regarding its cutting mechanism?} \\
    \textbf{Response:} \\
    Cas9 is a CRISPR-associated protein that can be programmed to cut DNA at a specific location.  It recognizes a short DNA sequence, called a protospacer, and cuts both strands of the DNA at that site. This DNA cutting is done using two nuclease domains, HNH and RuvC, which are repsonsible for cutting the top and bottom strands of the DNA, respectively. The HNH nuclease domain cleaves the top strand, while the RuvC nuclease domain cleaves the bottom strand. The cut made by the nuclease domains leaves the DNA with a sticky end that can ligate to another DNA molecule, such as the chromosome of the cell.  \\

    \textbf{Instruction:} \textit{Could you provide me the advantages and disadvantages of Cas12a, another widely used CRISPR protein?} \\
    \textbf{Response:} \\
    Advantages: \\
    1. High specificty: Cas12a has higher specificity compared to Cas 9, which reduces off-target effects. \\
    2. Simpler PAM requirements: Cas 12a requires a simpler PAM sequence than Cas9, making it easier to target specific genomic loci. \\
    3. Multiplexing: Cas12a can be used for multiplexing genome editing, meaning multiple genomic loci can be edited in a single reaction. \\
    Disadvantages: \\
    4. Lower efficiency: Cas12a is less efficient in terms of indel formation than Cas9. \\
    5. Larger size: Cas12a is a larger protein than Cas9, which can make it harder to deliver the Cas12a system into cells. \\
    6. Lack of donor specificity: Cas12a has a lower efficiency of HDR than Cas9, which makes it difficult to target specific nucleotides for insertion or deletion.
    7. Higher toxicity: Cas12a may casue higher levels of cell death than Cas9, which could impact the overall efficiency of genome editing. \\
    8. Differing DNA repair mechanism: Cas12a and Cas9 utilize DNA different repair mechanisms, which can lead to different outcomes when targeting specific genomic loci. \\
    9. Less versatile: Cas12a is not as versatile as Cas9, as it can only target DNA sequences that contain a TTN PAM sequence. \\
    10. Unclear off-target effects: Cas12a has not been fully characterized in terms of off-target effects, which could lead to unwanted changes in the genome. \\
    11. Difficulty in designing gRNAs: Cas12 requires a longer gRNA sequence compared to Cas9, making it more difficult to design effective gRNAs. \\
\end{mdframed}  
\captionof{figure}{Response of CRISPR gene editing.}\label{SI_fig:case_study:CRISPR}

\begin{mdframed}[backgroundcolor=white, linecolor=black, linewidth=1pt]  
    \noindent
    \textbf{Instruction:} \textit{PD-L1, Programmed Death Ligand-1, is a protein that expressed on cancer cells.  Design a peptide with 10 amino acids that binds to PD-L1.} \\ 
    \textbf{Response:} 
        \text{$\langle$protein$\rangle$}\text{TDEDVLQKIVKRATEDSSYTVKLSNGQTLSVTE} \\
        \text{DSEIKKIQKFRS}\text{$\langle$/protein$\rangle$} \\
\end{mdframed}  
\captionof{figure}{Response of cancer immunotherapy.} \label{SI_fig:case_study:cancer_immunotherapy}


\begin{mdframed}[backgroundcolor=white, linecolor=black, linewidth=1pt]
    \textbf{Instruction:} \textit{Could you explain Lipinski's Rule of Five and its role in evaluating the drug-likeness of chemical compounds?} \\
    \textbf{Response:} Lipinski's Rule of Five, also known as the Rule of Five, is a set ofules developed by Dr. Lipinski that outlines the physical properties of successful orally active drugs. It is a set of simple guidelines that can be used to evaluate the drug-likeness of chemical compounds. It indicates whether a chemical compound with a certain set of properties will have desirable drug-like properties and is likely to be successful as an oral drug. The fiveules that comprise Lipinski's Rule of Five are:
1. The number of hydrogen bond donors should be limited to no more than 
2. The number ofotatable bonds should be limited to no more than 
3. The molecular weight should be less than 500 daltons.
4. The number of hydrogen bond acceptors should be limited to no more than 10.
5. The logD value (a logarithmic measure of a compound's hydrophobicity) should be less than 3.
The application of Lipinski's Rule of Five has been widely adopted in the pharmaceutical industry to prioritize lead compounds for drug development. It is a tool that is used to identify potentially promising compounds and can help in the elimination of compounds with poor pharmaceutical properties.\\
\end{mdframed}  
\captionof{figure}{Response of drug discovery.} \label{SI_fig:case_study:drug_discovery}

\clearpage

\begin{mdframed}[backgroundcolor=white, linecolor=black, linewidth=1pt]  
    \textbf{Instruction:} \textit{How do I wrap a present neatly?} \\
    \textbf{\ourM{}:} To wrap a present neatly, start by deciding on the type of wrapping paper you want to use. Next, tear the edges of the paper to make it easier to fold. Wrap the paper around the gift and use tape to hold it in place. Fold the ends of the paper over to form a neat tuck. Finally, tie a bow around the gift to finish it off.\\
    \textbf{Mixtral:} To wrap a present neatly, follow these steps: 1. Gather your supplies: wrapping paper, scissors, tape, and a ruler or measuring tape. 2. Measure the length and width of the present, and cut a piece of wrapping paper that is at least 2 inches longer and wider than the present. 3. Place the present in the center of the wrapping paper, with the bottom of the present facing up. 4. Fold one side of the wrapping paper over the present, and tape it in place. 5. Fold the other side of the wrapping paper over the present, and tape it in place. 6. Fold the top and bottom edges of the wrapping paper over the present, and tape them in place.7. Flip the present over, and repeat steps 4-6 on the other side. 8. Add a ribbon or bow to the top of the present for a finishing touch.

    \textbf{Instruction:} \textit{Write a captivating movie review for a recently released science fiction film, discussing its plot, characters, and special effects.} \\
    \textbf{\ourM{}:} It's still quite worth seeing.\\
    \textbf{Mixtral:} Captivating Movie Review: "Interstellar" "Interstellar" is a science fiction film directed by Christopher Nolan, starring Matthew McConaughey, Anne Hathaway, and Jessica Chastain. The film follows a group of astronauts as they travel through a wormhole in search of a new home for humanity. The plot of "Interstellar" is complex and thought-provoking, exploring themes of time, space, and the human condition. The characters are well-developed and relatable, with McConaughey delivering a powerful performance as the lead astronaut, Cooper. The special effects in "Interstellar" are breathtaking, with stunning visuals of space and otherworldly landscapes. The film also features a haunting score by Hans Zimmer, which adds to the emotional impact of the story. Overall, "Interstellar" is a captivating and thought-provoking film that will leave audiences pondering the mysteries of the universe long after the credits roll.
    
\end{mdframed}  
\captionof{figure}{Response of AlpacaEval.} \label{fig:case_study:alpacaeval}


\clearpage
\section{Supplementary tables}


\begin{table}[!htbp]
\centering
    % \footnotesize
    % \setlength{\tabcolsep}{0.5mm}
\begin{tabular}{lcccc}
\toprule
& Samples   & Tokens  & Samples & Tokens \\
Data source & (by million) &  (by billion) &  (\%)& (\%)\\
\midrule
Text           & 46.3  & 14.4 & 11.6 & 10.0\\
Small molecule & 68.0  &  4.2 & 17.0 & 2.9 \\
Protein        & 192.0 & 65.2 & 47.9 & 45.3 \\
DNA            & 13.4  & 19.8 & 3.3  & 13.8\\
RNA            & 37.8  & 27.5 & 9.4  & 19.1\\
Material       & 1.1   & 0.02 & 0.3  & 0.014\\
Cross-domain & 41.9  & 12.7 & 10.5 & 8.8\\
\midrule
Total & 400.5& 143.8 & 100 & 100 \\
\bottomrule
\end{tabular}
\caption{Tokens numbers and their distribution of each domain. }
\label{tab:statistics_pretrain_data}
\end{table}

\begin{table}[!htbp]
\centering
\begin{tabular}{cccccccc}
\toprule
Model Parameters & 1B & 8B & 8x7B \\
\midrule
Learning Rate & 1e-4 & 1e-4 & 2e-4 \\
Batch Size (Sentences) & 4096 & 2048 & 1536 \\
Context Length (Tokens) & 8192 & 8192 & 8192 \\
GPU number (H100) & 64 & 256 & 256 \\
\bottomrule
\end{tabular}
\caption{Training recipe of different models.}
\label{tab:training_recipe}
\end{table}

\begin{table}[!htbp]
\centering
\begin{tabular}{lcc}
\toprule
Porperty & Value \\
\midrule
QED & 0.5, 0.6, 0.7, 0.8, 0.9, 1.0\\
HBA & 0, 1, 2, 3, 4, 5, 6, 7, 8, 9, 10\\
HBD & 0, 1, 2, 3, 4, 5\\
FSP3 & 0.0, 0.1, 0.2, 0.3, 0.4, 0.5, 0.6, 0.7, 0.8, 0.9, 1.0\\
RotBonds & 0, 1, 2, 3, 4, 5, 6, 7, 8, 9, 10\\
TPSA & 20, 40, 60, 80, 100, 120\\
\bottomrule
\end{tabular}
\caption{Input property values for property-to-molecule generation}
\label{tab:property_values}
\end{table}

\begin{table}[!htbp]
\centering
\begin{tabular}{lc}
\toprule
Target & Spearman correlation\\
\midrule
Pancreatic alpha-amylase &0.569\\
Large T antigen &0.572\\
DNA (cytosine-5)-methyltransferase 1 &0.517\\
Chaperone protein PapD &0.739\\
Catechol O-methyltransferase &0.638\\
Glyceraldehyde-3-phosphate dehydrogenase, glycosomal &0.503\\
Phosphoenolpyruvate carboxykinase cytosolic &0.501\\
FK506-binding protein 1A &0.606\\
Beta-lactamase class C &0.560\\
OXA-48 &0.680\\
Ubiquitin carboxyl-terminal hydrolase 7 &0.764\\
MAP/microtubule affinity-regulating kinase 4 &0.782\\
\bottomrule
\end{tabular}
\caption{Spearman correlation between docking scores and binding affinity on the selected targets for evaluation.}
\label{tab:targets}
\end{table}

\begin{table}[]
\centering
\begin{tabular}{lcccccccccc}
\toprule
Basic property & QED & QED & donor & donor & LogP & LogP \\
Enzyme & CYP2C9 & CYP3A4  & CYP2C9 & CYP3A4 & CYP2C9 & CYP3A4 & Average \\
\midrule
1B & 0.352 & 0.357 & 0.501 & 0.497 & 0.276 & 0.280 & 0.377 \\
8B & 0.404 & 0.428 & 0.548 & 0.522 & 0.332 & 0.340 & 0.429 \\
8x7B & 0.429 & 0.427 & 0.515 & 0.501 & 0.355 & 0.347 & 0.429 \\
\bottomrule
\end{tabular}
\caption{Joint optimization of metabolism and a basic property.}
\label{tab:joint_basic_cyp}
\end{table}

\clearpage
\begin{table}[b]
\centering
\begin{tabular}{ ccc }
\toprule
Property Name & Training samples & Testing samples \\ 
\midrule
BBBP          & 1272             & 199             \\  
BACE          & 90677            & 152             \\  
LogP          & 8491             & 473             \\  
Donor         & 8526             & 478             \\  
QED           & 8466             & 476             \\  
CYP1A2        & 8076             & 103             \\  
CYP2C9        & 21589            & 199             \\  
CYP2D6        & 8067             & 165             \\  
CYP3A4        & 24376            & 171             \\ 
\midrule
Total         & 179540           & 2416            \\ 
\bottomrule
\end{tabular}
\caption{Statistics of preference data used in RLHF}
\label{tab:data-rlhf}
\end{table}




\clearpage
\section{Supplementary notes}


\subsection{Text-guided basic property optimization of small molecule compounds}
We focus on optimizing the basic molecular properties in this section. The input of \ourM{} includes a text command and a SMILES sequence to be optimized.  We evaluate the optimization results of Quantitative Estimation of Drug-likeness (QED), LogP, and the number of hydrogen bond donors. Following DrugAssist \cite{ye2023drugassist}, we curated a fine-grained procedure. An illustrative example is provided below and the example is from DrugAssist \cite{ye2023drugassist}:

\begin{example}
\noindent\texttt{Instruction: With a molecule represented by the SMILES string }
\newline
\mol{}CC(N)=[NH+]CC(=O)N1CCC(O)(Cn2cnc3c(cnn3-c3ccc(N4CCC5(CCOCC5)CC4)cc3)c2=O)CC1\emol{}, \texttt{propose adjustments that can increase its QED value by at least 0.1 compared to the pre-optimized value to make it more drug-like. }   
\newline
\texttt{Response:}\mol{}CC(C)(C)OC(=O)N1CCC(c2ncc(-c3ccc(CC[B-](F)(F)F)cc3)cn2)CC1\emol{}.
\end{example}

For QED and hydrogen bond donor property optimization, our instructions cover the following scenarios: (i) increase or decrease the property by $\delta$, where both $\delta=0$ and $\delta>0$ are considered, aiming to verify the ability of the model; (2) maintain the properties. For LogP, the instruction is to adjust the LogP value from one specified region to another. 

\begin{table}[!htbp]
\centering
\begin{tabular}{lccc}
\toprule
Model            & QED     & \#Donor & LogP \\
\midrule 
%GPT4  \\ 
LLAMA 3 8B$^*$   & 0.62 / 0.43    &  0.75 / 0.43   & 0.84 / 0.45   \\ 
\ourM{} (1B)    & 0.58 / 0.57    &  0.74 / 0.58   & 0.63 / 0.60 \\ 
\ourM{} (8B)         & 0.65 / 0.45    &  0.81 / 0.44   & 0.80 / 0.42 \\ 
\ourM{} (8x7B) & 0.66 / 0.48 & 0.80 / 0.47  & 0.80 / 0.47  \\ 
\bottomrule
\end{tabular}
\caption{Comparison between the basic property optimization. In each cell, the success rate and uniqueness ratio are reported.}
\label{tab:basic_property_optimization}
\end{table}



The results are in Table \ref{tab:basic_property_optimization}. Notably, as the model size of \ourM{} increases, there is a marked improvement in performance metrics across all properties. For instance, \ourM{} (8B) surpasses \ourM{} (1B) in all categories, indicating enhanced comprehension and manipulation of molecular structures and properties as model complexity grows. Despite DrugAssist$^*$ achieving the highest scores overall, our results demonstrate that by further increasing the model size and fine-tuning the training process, there is significant potential to outperform this baseline. The trend observed with the \ourM{} models underscores the importance of model scale and suggests that with continued advancements in model architecture and training methodologies, even better optimization outcomes can be achieved. This validates the proficiency of \ourM{} in understanding and applying the given instructions to revise molecular properties accordingly. 


\subsection{Supplementary information of RNA generation}\label{app:rna_generation}

Minimum free energy (MFE) calculation: 

\texttt{./ViennaRNA-2.7.0/src/bin/RNAfold -p --MEA \$\{input\_file\}}

Usage of cmscan: 

\texttt{cmscan --rfam --cut\_ga --nohmmonly --tblout results\_tblout --fmt 2 --clanin Rfam/Rfam.clanin Rfam/Rfam.cm \$\{input\_file\}}


\subsection{POSCAR files of crystal structures in Fig. \ref{fig:bulk_caseStudy}}
{{
\footnotesize
\begin{example}
\begin{verbatim}
Generated by VASPKIT code
 1.000000
    7.1831247561033589    0.0000000000000000    0.0000000000000000
    0.0000000000000000    1.4245311887791383    2.4673932588460490
    0.0000000000000000   -1.4245311887791383    2.4673932588460490
   Re   C 
     3     1
Direct
    0.5000000000000000    0.6666666666666643    0.6666666666666643     Re1
    0.8049243600558619    0.3333333333333357    0.3333333333333357     Re2
    0.1950756399441381    0.3333333333333357    0.3333333333333357     Re3
    0.0000000000000000    0.6666666666666643    0.6666666666666643      C1
\end{verbatim}
\end{example}
\begin{example}
\begin{verbatim}
Generated by VASPKIT code
 1.000000
    8.7432980292995008    0.0000000000000000    0.0000000000000000
    0.0000000000000000    1.3846334542329621    2.3982883660601972
    0.0000000000000000   -1.3846334542329621    2.3982883660601972
   Re   Os
     1     3
Direct
    0.0000000000000000    0.3333333333333355    0.3333333333333355     Re1
    0.7492665073023750    0.6666666666666643    0.6666666666666643     Os1
    0.2507334926976250    0.6666666666666643    0.6666666666666643     Os2
    0.5000000000000000    0.3333333333333355    0.3333333333333355     Os3
\end{verbatim}
\end{example}
}}


\subsection{Supplementary information for evaluation metrics}\label{app:more_eval_method}
\subsubsection*{Success Rate for BBBP and CYP Optimization}

For BBBP optimization, our goal is to enhance the BBBP ability of the given compounds. These compounds are selected from the test set of the BBBP dataset in MoleculeNet, and initially, none can cross the BBB. For compounds generated by our AI method, we use BioT5 to predict their ability to cross the BBB. If a compound is predicted to cross, the optimization is considered successful.

For CYP optimization, the objective is to decrease the inhibition ability. Our prediction model uses a sigmoid function in the final layer, where $0$ indicates inhibition and $1$ indicates no inhibition. For an input molecule A and output molecule B, with predicted values $p_a$ and $p_B$, if $p_a>p_b$, the optimization is deemed successful.



\subsection{Shift the focus from general text to scientific sequences}\label{app:compare_with_galactica}
Although there are certain sequence-based foundation models for scientific tasks, their main focus is on text-based tasks and scientific understanding, instead of scientific discovery, i.e., discovering new molecules, proteins, and material. In Table~\ref{tab:comparison_galactica_ourM}, we compare \ourM{} with several sequence models.    %BioGPT~\cite{biogpt2022}, MolXPT~\cite{liu2023molxpt}, and Galactica~\cite{galactica2022} and the details are summarized 

\begin{table}[!htbp]  
    \centering
    \begin{tabular}{@{}p{1.9cm}p{5cm}p{5cm}@{}} % Adjust widths as necessary  
        \toprule  
        \textbf{Model} & \textbf{BioGPT} & \textbf{MolXPT} \\   
        \midrule  
        \textbf{Scope} & Biomedical literature & Text and SMILES \\   
        \midrule  
        \textbf{Core Capabilities} &   
        \begin{tabular}[t]{@{}p{5cm}@{}}  
            Biomedical natural language processing
        \end{tabular} &   
        \begin{tabular}[t]{@{}p{5cm}@{}}
        SMILES understanding and generation
        \end{tabular} \\   
        \midrule
        \textbf{Representative Tasks} &   
        \begin{tabular}[t]{@{}p{5cm}@{}}  
            \textbullet\ Biomedical relation extraction \\  
            \textbullet\ Biomedical question answering \\
            \textbullet\ Biomedical document classification \\
        \end{tabular} &   
        \begin{tabular}[t]{@{}p{5cm}@{}} 
            \textbullet\ Molecule property prediction \\
            \textbullet\ Text-molecule translation \\  
        \end{tabular} \\   
        \midrule
        \textbf{Training Data} &   
        \begin{tabular}[t]{@{}p{5cm}@{}}  
            \textbullet\ Text only\\ 
            \textbullet\ PubMed items before 2021\\ 
            \textbullet\ 15M paper titles and abstracts \\
        \end{tabular} &   
        \begin{tabular}[t]{@{}p{5cm}@{}} 
        \textbullet\ 67\% pure text tokens \\
            \textbullet\ 30M paper titles and abstracts from PubMed \\  
            \textbullet\ 30M SMILES from PubChem \\
            \textbullet\ 8M interleaved sequences between SMILES and text
        \end{tabular} \\   
        \midrule  
        \textbf{Training Strategy} &   
        Trained from scratch &  Trained from scratch \\
        \bottomrule
    \end{tabular}  
    \begin{tabular}{@{}p{1.9cm}p{5cm}p{5cm}@{}} % Adjust widths as necessary  
        \toprule  
        \textbf{Model} & \textbf{Galactica} & \textbf{\ourM{}} \\   
        \midrule  
        \textbf{Scope} & Academic literature & Broader ``language of nature'' \\   
        \midrule  
        \textbf{Core Capabilities} &   
        \begin{tabular}[t]{@{}p{5cm}@{}}  
            \textbullet\ Scientific knowledge and reasoning \\  
            \textbullet\ Scientific writing assistance \\  
        \end{tabular} &   
        \begin{tabular}[t]{@{}p{5cm}@{}}  
            \textbullet\ Scientific entity generation \\  
            \textbullet\ Scientific entity optimization \\ 
        \end{tabular} \\   
        \midrule
        \textbf{Representative Tasks} &   
        \begin{tabular}[t]{@{}p{5cm}@{}}  
            \textbullet\ Scientific Q\&A \\  
            \textbullet\ Citation prediction \\
            \textbullet\ Equation recall \\
        \end{tabular} &   
        \begin{tabular}[t]{@{}p{5cm}@{}} 
            \textbullet\ Molecule optimization \\
            \textbullet\ Protein-to-molecule design \\  
            \textbullet\ Guide RNA engineering \\ 
        \end{tabular} \\   
        \midrule
        \textbf{Training Data} &   
        \begin{tabular}[t]{@{}p{5cm}@{}}  
            %\textbullet\ 106 billion tokens \\  
            \textbullet\ More than 90\% pure text tokens \\  
            \textbullet\ Academic text (e.g., papers, knowledge bases)  
        \end{tabular} &   
        \begin{tabular}[t]{@{}p{5cm}@{}}  
            %\textbullet\ 143 billion tokens \\  
            \textbullet\ 10\% pure text tokens \\  
            \textbullet\ Diverse scientific sequences (e.g., SMILES, FASTA, DNA, RNA, material, text.)  
        \end{tabular} \\   
        \midrule  
        \textbf{Training Strategy} &   
        Trained from scratch &  Continual pre-training on existing LLMs. Incorporates domain-specific instructions \\
        \bottomrule
    \end{tabular}  
    \caption{Comparison between existing sequence models and \ourM{}.}  
    \label{tab:comparison_galactica_ourM}  
\end{table}  

%\paragraph{Overview}
%BioGPT is a domain-specific large language model developed for biological tasks. The model concentrates on biomedical natural language processing tasks, such as biomedical question answering. MolXPT is another domain-specific language model that focuses on text and SMILES. The model has demonstrated significant improvements in molecular understanding and generation tasks.
%Galactica is a large language model designed specifically for science, with a focus on absorbing, reasoning, and generating knowledge from scientific literature and data. It excels in text-based tasks such as equation recall, scientific Q\&A, and citation prediction, leveraging a highly curated academic corpus. 

%\paragraph{Scope and training data}
BioGPT~\cite{biogpt2022} and MolXPT~\cite{liu2023molxpt} are designed for the biomedical and (small) molecular domains. BioGPT is trained with titles and abstracts from PubMed items. MolXPT is trained with PubMed items as well as SMILES from PubChem. Their core capabilities are natural language tasks.
Galactica~\cite{galactica2022} is primarily designed for understanding and reasoning about academic literature. Its core capabilities include recalling equations, answering scientific questions, and performing domain-specific reasoning such as predicting chemical reactions and deriving mathematical proofs. 
% It is trained on a highly curated corpus of approximately 106 billion tokens, predominantly composed of academic text, including papers (e.g., arXiv, PMC), reference materials, knowledge bases, LaTeX equations, and structured factual data. 
It is trained on a highly curated corpus, primarily consisting of academic texts such as research papers (e.g., arXiv, PMC), reference materials, knowledge bases, LaTeX equations, and structured factual datasets. 
Notably, \textbf{over 90\%} of Galactica's training data consists of pure text, reflecting its emphasis on ``academic text'' and its key application in scientific writing.

In contrast, \ourM{} envisions a broader ``language of nature'' that unifies multiple scientific domains and modalities. It is explicitly designed to process diverse sequence-based data, including small molecules (SMILES), proteins (FASTA), materials (composition, space group, and atomic coordinates), as well as DNA and RNA sequences. 	Unlike Galactica, which focuses on understanding and reasoning within scientific text, \ourM{} focuses on generative tasks for scientific discovery, especially cross-domain generation and optimization tasks, such as protein-to-molecule design or guide RNA engineering. 

% \ourM{} is trained on 143 billion tokens, of which \textbf{only 10\%} is pure text. 
Only \textbf{10\%} training data of \ourM{} is pure text. The remaining \textbf{90\%} consists of scientific entities and cross-domain sequences. Furthermore, \ourM{} incorporates cross-domain data where text is interlinked with SMILES, FASTA, and material representations, enabling it to span multiple scientific disciplines through sequence-based formats. This emphasis on structured scientific data allows \ourM{} to bridge multiple domains and facilitates discovery-oriented tasks beyond text-based scientific reasoning.

%BioGPT, MolXPT, and Galactica are trained from scratch on its curated corpus. In contrast, \ourM{} leverages continual pretraining on top of existing large language models, inheriting general language capabilities while specializing in scientific domains. Additionally, \ourM{} incorporates extensive domain-specific instructions—such as “optimize a molecule’s LogP”—to enhance its performance in specialized scientific tasks, a strategy not emphasized in Galactica’s training paradigm.

% \subsubsection{Scope and domain coverage}
% Galactica is designed as a language model for science that focuses on absorbing, combining, and reasoning about scientific literature and data. Its primary emphasis is on tasks such as recalling equations, answering scientific questions, and performing domain‐specific reasoning (e.g. chemical reactions and mathematical equations) from a vast curated scientific corpus.

% \ourM{} envisions a ``language of nature'' that unifies multiple scientific domains.
% It is explicitly built to work across diverse sequence‑based modalities, including small molecules (SMILES), proteins (FASTA), materials (composition plus space group), DNA, and RNA.
% \ourM{} is set up not only to understand science but to enable cross-domain generation and optimization (for example, protein-to-molecule or guide RNA design).




% \subsubsection{Training data}

% Galactica is trained on a highly curated scientific corpus mostly composed of papers (e.g., arXiv, PMC), reference material, knowledge bases, etc., yielding about 106 billion tokens. Galactica emphasizes high-quality ``academic text'', including LaTeX equations, factual data, and reference citations. \textbf{Over 89\%} of the tokens in its training corpus are pure text.

% In contrast, \ourM{} is trained on 143 billion tokens of both scientific text plus ``sequence data'' (e.g., small molecules with SMILES, protein FASTA sequences, material crystals, DNA/RNA). 
% Only \textbf{10\%} of \ourM{}’s training data is pure text, with the remaining data being sequences of scientific entities and cross-domain sequences. Additionally, \ourM{} incorporates cross-domain data where text is interlinked with, or integrated into, SMILES/FASTA/material representations, enabling it to span multiple scientific disciplines through sequence-based formats.












% \begin{myexample}{SFM-based fragment generation}{TamGen_fragment}  
%     \textbf{Instruction: }\\Design a compound with reference to the target \\  
%     $\langle$\texttt{protein}$\rangle$DTKEQRILR$\cdots$EKAIYQGP$\langle$\texttt{/protein}$\rangle$ and the fragment  $\langle$\texttt{fragA}$\rangle$O=c1[nH]cnc2c(O)cc([*:1])c([*:2])c12$\langle$\texttt{/fragA}$\rangle$\\  
%     \textbf{Response: }\\$\langle$\texttt{fragB}$\rangle$Fc1ccc([*:1])cc1.Fc1ccc([*:2])cc1$\langle$\texttt{/fragB}$\rangle$  
% \end{myexample}  

  

 

\end{document}
