% \& Motivation
% https://ieeexplore.ieee.org/stamp/stamp.jsp?tp=&arnumber=6696456
\vspace{-0.3cm}
\section{Background}

\subsection{Delivery Drone Precise Landing Primer}
% The process of drone-based instant delivery operates as follows. 
% First, the package is manually placed on the drone. The drone then takes off from the departure platform located at the self-collection airport on top of the mall. It ascends to cruising altitude and navigates towards the designated destination, usually a self-collection station near the customer. As the drone nears the destination, it starts a precise landing sequence and begins its descent.
% Upon safely landing on the platform of the self-collection station, the drone releases the package into the station for customer retrieval. 
% The drone then departs and returns to the airport, landing on the landing platform of self-collection airport.
The process of drone-based instant delivery operates as follows:
$(i)$ The package is manually loaded onto the drone;
$(ii)$ The drone then takes off from the departure platform located at the self-collection airport usually on top of the mall;
$(iii)$ It ascends to cruising altitude and navigates toward the designated destination, typically a self-collection station near the customer;
$(iv)$ As the drone approaches the destination, it initiates a precise landing sequence and begins its descent;
$(v)$ Upon safely landing at the self-collection station, the drone releases the package for customer retrieval; 
$(vi)$ The drone then departs, returning to the self-collection airport and executing a precise landing on the landing platform.

% The consequences of not landing precisely cannot be overstated.
% Missing the platform can result in the drone losing control and crashing, risking damage or injury. 
% 控制Localization latency and localization error are vital to drone landing to avoid similar mishaps, as increased errors and delays in location updates can destabilize the system. 
% Specifically, significant localization latency provide the drone with minimal reaction time, and large localization errors can cause the flight controller to make incorrect maneuvers.
The importance of precise landing cannot be emphasized enough. 
Failing to land accurately can lead to the drone losing control and crashing, potentially causing harm or damage \cite{Russiandrone}. 
Minimizing localization error and latency is crucial for safe drone landing, as any increase in errors or delays in location updates can destabilize the system, leading to the failure happens. 
Specifically, reducing localization latency provides the drone with more reaction time, while minimizing localization errors prevents the flight controller from executing incorrect maneuvers.

% Instant delivery with a drone unfolds as follows. After the package通过人工的方式 is placed onboard , the drone takes off from departing platform of self-collection airport on the top of the mall, climbs to cruising altitude, and heads towards the destination. 
% The latter is typically a self-collection station near the customer. When the drone is close to the destination, a precise landing procedure is initiated, in which the drone starts the descent.
% Once the drone safely docks onto the landing platform of the self-collection station, the package is dropped into the self-collection station, where the customer fetches it.
% 随后,drone从self-collection station起飞,heading towards the airport and docks on the landing platform of self-collection airport.

% The consequences of not landing precisely can only be underestimated. 
% Loss of the transported good is the most obvious. Should the drone miss the landing platform even partly, it would quickly lose control and crash, possibly damaging objects or hurting people.
% Both the localization latency and localization error are crucial to drone landing, as evidence shows that increasing error of location estimation and latency of location updates represent sources of system instability.. 
% A long localization delay leaves the drone very short time to react, and a large localization error misleads the flight controller to execute a wrong maneuver.

%要不要加呢??
% \subsection{Limitations of the Current Practice}
% \noindent \textbf{Examining the current practice.}
% During aerial navigation, the flight control system primarily utilizes GPS or RTK for location information. 
% As the drone descends, precise control according to location information is crucial to remain centered over the landing platform. 
% However, in environments like urban canyons, GPS or RTK accuracy may degrade, making them unreliable at lower altitudes where drone docking.
% To address this, auxiliary anchor-based systems like visual markers assist during descent.
% These markers are detected by downward-facing cameras on the drone. 
% However, as these methods rely on RGB cameras, their effectiveness is limited by camera inherent frame exposure time. 
% Additionally, factors like camera motion blurring, field of view and drone alignment may lead to incomplete marker capture, causing localization errors.
% Overall, existing methods fail to meet the requirements for localization accuracy and latency during the drone docking phase.

% \noindent \textbf{What is a feasible solution? } 
% As the drone descends to dock, achieving precise and low latency adjust based on localization result is crucial for accurate landing. 
% Thus, there are the key requirements in updating localization result for drones: 
% $(i)$ \textit{cm-level} localization accuracy and $(ii)$ \textit{ms-level} localization latency.


% \subsection{Fusing Event and mmWave: Opportunities and Challenges}
% \noindent $\bullet$ \textbf{C1: Noisy sensing results impairs drone detection and tracking.}
% Environmental changes often introduce irrelevant information in the sensing results from the event camera and mmWave radar, hindering the system's ability to identify signals changes caused by the drones to be tracked. 
% Additionally, modern flight control loops operate at frequencies exceeding 400 Hz, requiring the system to detect and track drones rapidly. 
% However, traditional noise filtering, object detection and tracking algorithms have high time complexities, resulting in precision and tracking frequency bottlenecks. 
% \todo{Explanation in figures or tables.}

% \noindent $\bullet$ \textbf{C2: Heterogeneous data fusion delays drone localization.}
% Once the drone is detected, accurate 3D spatial localization of it is essential, which is more time-consuming than detection and tracking due to additional processing. 
% Moreover, event cameras provide asynchronous event streams, while mmWave radar generates sparse point clouds with relatively low spatial resolution. 
% Previous fusion methods (\eg extended Kalman filters and particle filters) often suffer from significant cumulative drift error and lengthy processing times, rendering them inadequate for high-frequency and high-precision tracking.
% \todo{Explanation in figures or tables.}

\vspace{-0.3cm}
\subsection{HFPL: System goals}
HFPL takes a solid step forward in ensuring safely takeoff and landing of drone. 
We list the system goals below.

\noindent $\bullet$ \textbf{Goal 1: Low latency \& High precision localization.}
HFPL should provide low-latency and accurate localization results for the drone. This allows the drone's flight control loops to take advantage of HFPL and operate the drone to land at the designated location.\\
\noindent $\bullet$ \textbf{Goal 2: Plug and play.}
HFPL should be implemented as a plug-in module, seamlessly integrating with existing GPS/RTK and visual marker localization systems. This prevents deeply embedded code changes that could potentially compromise the system’s stability. \\
\noindent $\bullet$ \textbf{Goal 3: Resource overhead reduction.}
HFPL should effectively reduce the resource overhead spanning filtering and fusion for both sensing modalities, while maintaining the high-precision and low-latency performance.