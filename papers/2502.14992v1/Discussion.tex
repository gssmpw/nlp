\vspace{-0.5cm}
\section{Discussion}\label{8}
\revise{
% We provide a detailed explanation of rationale behind key design decisions in design and implementation of mmE-Loc.

\noindent \textbf{How does mmE-Loc relate to visual markers?}
Currently, delivery drones utilize visual markers and onboard cameras for self-localization. 
In contrast, mmE-Loc focuses on ground-based drone localization, enabling the landing pad to determine the spatial relationship between the drone and itself for precise adjustments. 
In practice, mmE-Loc operates alongside visual markers to enhance reliability and accuracy of localization service.
% Currently, delivery drones incorporate with visual markers and drone-equipped camera for drone onboard localization.
% mmE-Loc focuses on drone ground localization, which aims at enabling landing pad to obtain spatial relationship between drone and pad for further adjustment. mmE-Loc works in conjunction with visual markers to provide a more reliable and accurate localization service.

\noindent \textbf{Is it feasible to design an onboard sensor system for drone landing localization?}
It's theoretically feasible. 
However, designing an onboard system for drone localization using an event camera and mmWave radar presents several challenges. 
Given the continuous motion of the drone, the system must address:  
$(i)$ motion compensation for event data,  
$(ii)$ reliable feature extraction and matching within the event stream despite its lack of semantic information, and  
$(iii)$ the sparsity of radar measurements and noise induced by specular reflections, diffraction, and multi-path effects.
% There are several challenges need to be solved to design an onboard system for drone localization based on event camera and mmWave radar. 
% Since drone is continuous moving, the system needs to deal with $(i)$ motion compensation of event data; $(ii)$ reliable feature extraction and matching with event stream while event data lack semantic information; $(iii)$ sparse measurements of radar and radar noise caused by specular reflections, diffraction and multi-path.

% Additionally, we will discuss strategies for handling simultaneous drone landings, and investigate the reasons for the continuous increase in memory usage. 

\noindent \textbf{How to manage simultaneous drone landings?}
% Currently, each delivery drone airport has a single landing pad, and drones land sequentially while those waiting hover at a designated altitude. When mmE-Loc is deployed on the landing pad, it continuously tracks the landing drone with the largest bounding box.  
As multiple drones land simultaneously, mmE-Loc will initialize multiple trackers within the event tracking model to track each drone, associate the results with radar tracking model, and subsequently perform localization and optimization for each drone individually.
% In the future, as multiple drones land simultaneously, mmE-Loc will initialize multiple trackers within the event tracking model to track different drones and associate these results with tracking results from the radar tracking model, and then launch localization and optimization process for each drone.
% Currently, each delivery drone airport has one landing pad and drones land sequently.
% Meanwhile, drones waiting for landing will hover at a designated height. 
% Therefore, when mmE-Loc is deployed on the landing pad, it will continuously track the landing drone which has the biggest bounding box.
% In the future, when multi drones landing simultaneously, mmE-Loc will initiates multi trackers in event tracking model to track different drones and associate tracking results with mmWave tracking results from radar tracking model.
% As the drone landing, it will become bigger in the FoV and generates more events.
% Therefore, we keep tracking the drone with the biggest bounding box in the FoV when multi drones show up.


\noindent \textbf{How does network latency impact the system, and how can potential delay-related issues be addressed?}
mmE-Loc is designed to assist ground platforms in locating drones and guiding them to land accurately at designated spots.
Network latency may impact the location update rate for the drone. 
To mitigate this issue, airports should deploy access points near the landing pad and utilize multiple wireless links, including Wi-Fi and cellular networks, to ensure reliable and fast communication.
Additionally, an optics-based communication system could be integrated into the system to ensure a high communication frequency \cite{wang2024towards}.
% The network latency may impact the location update rate for the drone.
% In order to relieve this problem, airports are supposed to deploy access point near the landing pad.
% Meanwhile, the optics-based communication system may be considered to integrate into the system to ensure the communication frequency \cite{wang2024towards}.
% However, if frequent location adjustments are needed, an optics-based communication system could be integrated to enhance communication frequency \cite{wang2024towards}.
% rather than enabling the drones to localize themselves. As a result, only when the drone exhibits a location bias does the ground platform send commands for adjustment.  
% While network latency may affect command transmission, high-frequency sending are not required, making the latency tolerable. 
% mmE-Loc is designed to assist ground platforms to locate drones and guide them to accurately land at designated spots instead of assisting the drone to locating itself.
% Therefore, only the drone experience location bias then the ground platform send commands to drone for location adjustment.
% The network latency may impact the commands sending, but the commands have no need to be sent with high frequency, so the network latency is tolerable.
% If one want to adjust drone location with high frequency, optics-based communication system may be considered to integrate into the system to ensure the communication frequency.

\noindent \textbf{How does strong light affect system performance?}
Lighting conditions primarily affect event cameras.
Strong light has minimal impact on event cameras, as they detect only changes in light intensity. 
In contrast, weak illumination affects performance more significantly since intensity changes are less pronounced.
Our experimental results on illumination effects (Fig. 12) confirm that weak illumination introduces greater errors than strong light.
% The strong light have little effect on event camera since it only detect intensity change of light. 
% The weak illumination have more effect on event camera because light intensity change isn't obvious.
% According to our experiment results on impact Env. \& Illu, weak illumination leads to more error than light illumination.
% Thanks to high dynamic range of event camera, it can handle different light condition.

% \noindent \textbf{What is the system's computational cost?}
% mmE-Loc runs on a PC with Ubuntu 20.04, equipped with an Intel i7-12900K CPU, 32GB of RAM, and an NVIDIA GeForce GTX 1070 GPU. 
% As shown in Fig. 19 and Fig. 20, CPU usage remains below 20\%, while memory consumption stays under 120MB.
}
