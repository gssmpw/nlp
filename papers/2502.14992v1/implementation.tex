\section{Implementation} \label{I}
\subsection{Push the Limit of Accuracy and Latency} \label{I-1}
\textbf{Factor graph solving.}
% Now we express the estimation problem \eqn \ref{inter_frame} and \eqn \ref{local_location} using a graphical model. 
% When solving both nonlinear least-squares problems, we linearize the observation model and solve the least squares formulation as follows:
% \begin{equation}
% \hat{\boldsymbol{\mathcal{X}}}=\arg \min _{\boldsymbol{\mathcal{X}}}\|A \boldsymbol{\mathcal{X}}-\mathbf{b}\|^2,
% \end{equation}
% where the matrix $A \in \mathbb{R}^{m \times n}$ is a measurement Jacobian and $\mathbf{b} \in \mathbb{R}^m$ is the right-hand side vector \tocite.
% The QR matrix factorization $A = Q[R, 0]^T$ is then utilized, and the least squares problem $R \hat{\boldsymbol{\mathcal{X}}}=\mathbf{d}$ is solved through backsubstitution to get optimized locations $\hat{\boldsymbol{\mathcal{X}}}$, where $R \in \mathbb{R}^{n \times n}$ is the upper triangular square root information matrix, $Q \in \mathbb{R}^{m \times m}$ is an orthogonal matrix and $\textbf{d} \in \mathbb{R}^n$. 
% More detail can refer to \tocite.
% Although re-linearization and re-generate $R$ as new measurements comes can reduce system error, for the problem \eqn \ref{local_location}, which requires joint optimization of multiple locations, this process can be computationally expensive \tocite.
We represent the estimation problems \eqn \eqref{inter_frame} and \eqn \eqref{local_location} using a factor graph model. To solve the nonlinear least-squares problems, we linearize the observation model and solve the least-squares formulation:
\begin{equation}
\vspace{-0.1cm}
\hat{\boldsymbol{\mathcal{X}}}=\arg \min _{\boldsymbol{\mathcal{X}}}\|A \boldsymbol{\mathcal{X}}-\mathbf{b}\|^2,
\end{equation}
where $A \in \mathbb{R}^{m \times n}$ is the measurement Jacobian and $\mathbf{b} \in \mathbb{R}^m$ is the right-hand side vector. We then utilize QR matrix factorization \cite{bischof1998computing} as $A = Q[R, 0]^T$ and solve the least squares problem $R \hat{\boldsymbol{\mathcal{X}}}=\mathbf{d}$ through backsubstitution to obtain optimized locations $\hat{\boldsymbol{\mathcal{X}}}$, where $R \in \mathbb{R}^{n \times n}$ is the upper triangular square root information matrix, $Q \in \mathbb{R}^{m \times m}$ is an orthogonal matrix, and $\textbf{d} \in \mathbb{R}^n$ \cite{kaess2008isam}. 
% While re-linearization and regeneration of $R$ with new measurements can reduce error, for problem \eqn \eqref{local_location}, this can be computationally expensive.
Although re-linearizing and regenerating \( R \) with new measurements can help mitigate errors, applying this approach to problem \eqref{local_location} can be computationally expensive, as it requires frequent updates and increased processing overhead, limiting real-time performance.


% A batch solution solves the complete problem at every step, including all previous measurements, which performs unnecessary calculations.
% A  exploits incrementally updating the square root information matrix R with new measurements.

% \vspace{-0.3cm}
\begin{algorithm}

\caption{Adaptively Optimization method}
\setlength{\abovedisplayskip}{3pt}
\setlength{\belowdisplayskip}{-1cm}
\label{algorithm}
\KwData{Original factor graph $G$; New measurements $D^i, \vec{v}^i, U^i_E$; square root information matrix $R$}
\KwResult{Updated locations $\hat{\mathcal{X}}$}
$G \leftarrow \mathtt{AddFactorToGraph}(G^i, D^i, \vec{v}, U^i_E)$\;
$R \leftarrow \mathtt{\textbf{IncrementalUpdate}}(G)$\;
$\hat{\mathcal{X}} \leftarrow \mathtt{Backsubstitution}(R)$\;
$L \leftarrow \emptyset$;  $\quad\quad \triangleright \textit{Set of nodes need to be linearized}$\;
\For{all $t^i_{ED} \in \mathcal{X}$ and all $\hat{t}^i_{ED} \in \hat{\mathcal{X}}$}
{
\If{$\hat{t}^i_{ED} - t^i_{ED} \geq \delta$}
{$L \leftarrow L \cup t^i_{ED}$\;}
}
\If{$|L| \geq L_T$ or $||\hat{\mathcal{X}} - \mathcal{X} || \geq \Delta$}
{
\For{all $t^i_{ED} \in \mathcal{X}$}
{
$\mathtt{UpdateLinearizationPoint}(t^i_{ED})$\;
}
$R \leftarrow \mathtt{\textbf{FullUpdate}}(G)$\;
$\hat{\mathcal{X}} \leftarrow \mathtt{Backsubstitution}(R)$\;
}
\end{algorithm}


% \noindent 
\vspace{-0.2cm}
\textbf{Adaptive optimization method.}
% To address this problem, we propose the Adaptively Optimization method, based on the observation that \textit{new measurements often have a localized impact, leaving remote parts of the graph unaffected}, which enable us to incrementally update $R$ \tocite. 
% When solving local location tracking at each step, this method adaptively combines incrementally updated $R$ and re-generated $R$, reducing latency and improving accuracy.
To tackle this, we propose an adaptive optimization method, leveraging the insight that \textit{new measurements mainly impact localized areas, leaving distant parts unchanged}. 
This allows us to incrementally update $R$ during local optimization \cite{kaess2012isam2}. 
By adaptively combining updated and regenerated $R$, this method reduces latency and enhances accuracy.
% of \textit{GAJO}.

% \alg \ref{algorithm} shows how Adaptively Optimization method solves local location tracking problem.
% Line 1-3 represents local location tracking with incrementally updated $R$ \tocite.
% Line 4-16 show local location tracking with re-generated $R$.
% Specifically, when receive new measurements, function $\mathtt{AddFactorToGraph}$ updates factor graph, and the function $\mathtt{IncrementalUpdate}$ incrementally update $R$ with new measurements \tocite. 
% We then solve local location tracking with this incrementally updated $R$.
% When one of two conditions is met, we solve local location tracking with re-generated $R$: 
% $(i)$ we tracks locations that have changed significantly in a set $L = \{t_{ED}^i: \hat{t^i_{ED}} - t^i_{ED} \geq \delta\}$. If enough locations have undergone significant changes (\ie $|L| \geq L_T$), we solve local location tracking with re-generated $R$ output by function $\mathtt{FullUpdate}$ \tocite;
% $(ii)$ if the norm of the total locations changes becomes too large (\ie $||\hat{\mathcal{X}} - \mathcal{X} || \geq \Delta$), we solve local location tracking with re-generated $R$; 
% Since the local location tracking involves repeatedly solving linear equations, this condition keeps the current solution from diverging too far from the optimal solution.
% We monitor locations with notable changes in $L = \{t_{ED}^i: \hat{t^i_{ED}} - t^i_{ED} \geq \delta\}$. If $|L| \geq L_T$, indicating sufficient changes, we employ $\mathtt{FullUpdate}$ to regenerate $R$ for local location tracking.

\alg \ref{algorithm} shows how adaptively optimization method solves local location optimization problem.
Lines 1-3 depict tracking with incrementally updated $R$, while lines 4-16 show tracking with re-generated $R$. 
% When new measurements arrive, $\mathtt{AddFactorToGraph}$ updates the factor graph, and $\mathtt{IncrementalUpdate}$ adjusts $R$ incrementally. 
Specifically, when receiving new measurements, function $\mathtt{AddFactorToGraph}$ updates the factor graph, and the function $\mathtt{IncrementalUpdate}$ incrementally updates $R$ with new measurements \cite{kaess2012isam2}. 
% Tracking proceeds with this updated $R$. 
% If conditions are met, tracking switches to re-generated $R$ via $\mathtt{FullUpdate}$:
We then solve local location tracking with this incrementally updated $R$.
When one of two conditions is met, we solve local location tracking with re-generated $R$: 
% $(i)$ Significant changes in a set $L$ of tracked locations (\ie $|L| \geq L_T$);
$(i)$ we track locations that have changed significantly in a set $L = \{t_{ED}^i: \hat{t^i_{ED}} - t^i_{ED} \geq \delta\}$. If enough locations have undergone significant changes (\ie, $|L| \geq L_T$), we solve location tracking with re-generated $R$ output by function $\mathtt{FullUpdate}$.
$(ii)$ The norm of total location changes exceeds a threshold $\Delta$ (\ie, $||\hat{\mathcal{X}} - \mathcal{X} || \geq \Delta$).
% This ensures the solution doesn't stray too far from optimality during repeated linear equation solving.
Since the local location tracking involves repeatedly solving linear equations, this condition keeps current solution from diverging too far from optimal solution.

\vspace{-0.2cm}
\subsection{Platform Implementation}
\revise{
In this section, we will detail the platform implementation.

\textbf{Sensor platform configuration.}
As illustrated in \fig \ref{setup}, we implement our sensing platform with multiple sensors including 
$(i)$ A Prophesee EVK4 HD evaluation kit, featuring the IMX636ES event-based vision sensor for HD event data (1280 $\times$ 720 pixels) with 47.0\degree  FoV.
$(ii)$ A Texas Instruments (TI) IWR1843 board for transmitting and receiving mmWave signals within the frequency range of 76 $GHz$ to 81 $GHz$ with three transmitting antennas and four receiving antennas.
These antennas are arranged in two linear configurations on the horizontal plane.
$(iii)$ An Intel D435i Depth camera for RGB image capture used in the baseline method.

\textbf{Deployment detail.}
During the experiments, the sensor platform is deployed at the center of the landing pad. 
In the case study at a real-world airport, the platform is positioned at the edge of the landing pad for safety reasons, with the distance between the sensor platform and the center of the pad being less than 1$m$. 
After takeoff, the drone appears within FoV of the event camera. 
All sensors are synchronized through the Robot Operating System (ROS). 
mmE-Loc is deployed on a PC with Ubuntu 20.04, featuring an Intel i7-12900K CPU, 32GB of RAM, and an NVIDIA GeForce GTX 1070 GPU. 
For practical deployments, the sensor platform is expected to be positioned at the center of the landing pad, with its height aligned to that of that of the pad, ensuring the drone remains within the event camera’s FoV at all times for continuous localization.
% During the experiments, the sensor platform is deployed at the center of the landing pad.
% In the case study at the real-work airport, the sensor platform is deployed at the edge of the landing pad due to the safety consideration and the distance between sensor platform and center of the landing pad is less than 1$m$.
% The drone shows in the FoV of event camera after the takeoff operation.
% All the sensors are synchronized through Robot Operating System (ROS).
% The mmE-Loc operates on a PC equipped with Ubuntu 20.04, an Intel i7-12900K CPU, 32GB RAM, and NVIDIA GeForce GTX 1070 GPU.
% In the practical deployment, the sensor platform is expected to have same height with landing pad and be deployed in the center of the landing pad  so that the drone will show in FoV of event camera all the time.

}