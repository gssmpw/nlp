% 如果您使用 overleaf,您只需单击“提交”图标并选择“Arxiv”选项,它将为您提供所需的所有文件。


% edit link: https://www.overleaf.com/7243917635qvjxbtpbhxbg#f5184e

%%
%% This is file `sample-sigconf.tex',
%% generated with the docstrip utility.
%%
%% The original source files were:
%%
%% samples.dtx  (with options: `sigconf')
%% 
%% IMPORTANT NOTICE:
%% 
%% For the copyright see the source file.
%% 
%% Any modified versions of this file must be renamed
%% with new filenames distinct from sample-sigconf.tex.
%% 
%% For distribution of the original source see the terms
%% for copying and modification in the file samples.dtx.
%% 
%% This generated file may be distributed as long as the
%% original source files, as listed above, are part of the
%% same distribution. (The sources need not necessarily be
%% in the same archive or directory.)
%%
%% The first command in your LaTeX source must be the \documentclass command.
\documentclass[sigconf,9pt]{acmart}

\makeatletter
\def\@ACM@checkaffil{% Only warnings
    \if@ACM@instpresent\else
    \ClassWarningNoLine{\@classname}{No institution present for an affiliation}%
    \fi
    \if@ACM@citypresent\else
    \ClassWarningNoLine{\@classname}{No city present for an affiliation}%
    \fi
    \if@ACM@countrypresent\else
        \ClassWarningNoLine{\@classname}{No country present for an affiliation}%
    \fi
}
\makeatother

% \documentclass[sigconf,9pt]{manuscript}
% \documentclass[manuscript, anonymous, sigconf]{acmart}
%% NOTE that a single column version may be required for 
%% submission and peer review. This can be done by changing
%% the \doucmentclass[...]{acmart} in this template to 
%% \documentclass[manuscript,screen]{acmart}
%% 
%% To ensure 100% compatibility, please check the white list of
%% approved LaTeX packages to be used with the Master Article Template at
%% https://www.acm.org/publications/taps/whitelist-of-latex-packages 
%% before creating your document. The white list page provides 
%% information on how to submit additional LaTeX packages for 
%% review and adoption.
%% Fonts used in the template cannot be substituted; margin 
%% adjustments are not allowed.
%%
%%
%% \BibTeX command to typeset BibTeX logo in the docs
\AtBeginDocument{%
  \providecommand\BibTeX{{%
    \normalfont B\kern-0.5em{\scshape i\kern-0.25em b}\kern-0.8em\TeX}}}


% \renewcommand\footnotetextcopyrightpermission[1]{}
\setcopyright{acmcopyright}
\settopmatter{printacmref=true, printccs=true, printfolios=false}
% Rights management information.  This information is sent to you
% when you complete the rights form.  These commands have SAMPLE
% values in them; it is your responsibility as an author to replace
% the commands and values with those provided to you when you
% complete the rights form.
% \setcopyright{acmcopyright}
% \copyrightyear{2018}
% \acmYear{2018}
% \acmDOI{XXXXXXX.XXXXXXX}
% \renewcommand{\baselinestretch}{0.95}


% These commands are for a PROCEEDINGS abstract or paper.
% \acmConference[Conference acronym 'XX]{Make sure to enter the correct
%   conference title from your rights confirmation emai}{June 03--05,
%   2018}{Woodstock, NY}

 % Uncomment \acmBooktitle if th title of the proceedings is different
 % from ``Proceedings of ...''!

% \acmBooktitle{Woodstock '18: ACM Symposium on Neural Gaze Detection,
%  June 03--05, 2018, Woodstock, NY} 
% \acmPrice{15.00}
% \acmISBN{978-1-4503-XXXX-X/18/06}
% \renewcommand{\baselinestretch}{.95} % line space
% Systems (SenSys '22), November 6--9, 2022, Boston, MA, USA}

\copyrightyear{2025}
\acmYear{2025}
\setcopyright{rightsretained}
\acmConference[ACM SenSys '25]{The 23th ACM Conference on Embedded Networked Sensor Systems}{May 6-9, 2025}{Irvine, USA}
\acmBooktitle{The 23th ACM Conference on Embedded Networked Sensor Systems (SenSys '25), May 6-9, 2025, Irvine, USA}
\acmDOI{xx.xxxx/xxxxxxx.xxxxxxx}
\acmISBN{xxx-x-xxxx-xxxx-x/xx/xx}
%%
%% Submission ID.
%% Use this when submitting an article to a sponsored event. You'll
%% receive a unique submission ID from the organizers
%% of the event, and this ID should be used as the parameter to this command. 
%%\acmSubmissionID{123-A56-BU3}

%%
%% The majority of ACM publications use numbered citations and
%% references.  The command \citestyle{authoryear} switches to the
%% "author year" style.
%%
%% If you are preparing content for an event
%% sponsored by ACM SIGGRAPH, you must use the "author year" style of
%% citations and references.
%% Uncommenting
%% the next command will enable that style.
%%\citestyle{acmauthoryear}

%%
%% end of the preamble, start of the body of the document source.
\usepackage[ruled,linesnumbered]{algorithm2e}
\usepackage{subfigure}
\usepackage{multicol}
% \usepackage{subcaption}
\setlength\columnsep{0.33in}
% \renewcommand{\baselinestretch}{0.97}
% \usepackage{algorithm}
\usepackage{algorithmic}
\usepackage{caption}
\captionsetup{labelfont=bf}
\usepackage{bbding}
\usepackage{pifont}
\usepackage{graphicx} % 引入 graphicx 包
\usepackage{url}
\usepackage{gensymb}
% \usepackage[colorlinks,linkcolor=blue]{hyperref}


%=== Editing tools ============
\ifodd 1
\newcommand{\rev}[1]{{\color{blue}#1}} %revise of the text
\newcommand{\com}[1]{\textbf{\color{red}(COMMENT: #1)}} %comment of the 
\else
\newcommand{\rev}[1]{#1}
\newcommand{\com}[1]{}
\fi

\newcommand{\term}[1]{{\it #1}}
\newcommand{\empha}[1]{{\bf #1}}
\newcommand{\head}[1]{{\bf #1}}
\newcommand{\scare}[1]{`#1'}
\newcommand{\code}[1]{{\tt #1}}
\newcommand{\lnote}[1]{Lognfei: {\color{red}{#1}}}
\newcommand{\notsure}[1]{{\color[rgb]{0.7,0.7,0.7}{#1}}}
\newcommand{\todo}[1]{{\color{red}{\textbf{TODO:} #1}}}
\newcommand{\notice}[1]{{\color{red}{#1}}}
\newcommand{\bsymb}[1]{{\boldsymbol #1}}
\newcommand{\tocite}{\rev{[?]}}
\newcommand{\toref}{\rev{?}}
% \newcommand{\revise}[1]{{\color{blue}{#1}}}
\newcommand{\revise}[1]{{\color{black}{#1}}}

\def\fig{Fig.}
\def\sec{Section}
\def\tab{Table}
\def\eg{e.g.}
\def\ie{i.e.}
\def\aka{a.k.a.}
\def\eqn{Eqn.}
\def\eq{Eq.}
\def\alg{Algorithm}
\def\etc{etc.}

% \copyrightyear{2022}
% \acmYear{2022}
% \setcopyright{rightsretained}
% \acmConference[SenSys '22]{The 20th ACM Conference on Embedded
% Networked Sensor Systems}{November 6--9, 2022}{Boston, MA, USA}
% \acmBooktitle{The 20th ACM Conference on Embedded Networked Sensor
% Systems (SenSys '22), November 6--9, 2022, Boston, MA, USA}
% \acmDOI{10.1145/3560905.3568432}
% \acmISBN{978-1-4503-9886-2/22/11}

\begin{document}
% \begin{figure*}[t]
%     \setlength{\abovecaptionskip}{-0cm} % height above Figure X caption
%     \setlength{\belowcaptionskip}{-0.24cm}
%     \setlength{\subfigcapskip}{-0.25cm}
%     \centering
%         \includegraphics[width=1.\columnwidth]{Figs/introall.png}
%         % \vspace{-0.2cm}
%     \caption{Illustration of the airport and landing of the delivery drone. \textnormal{(a) The self-collection airport is equipped with multiple delivery drones for package distribution. (b) A delivery drone descends onto the landing platform.}}
%     \label{intro}
%     % \vspace{-0.2cm}
% \end{figure*} 

% \let\oldtwocolumn\twocolumn
% \renewcommand\twocolumn[1][]{%
%     \oldtwocolumn[{#1}{
%     \begin{center}
%     \vspace{-0.5cm}
%     \setlength{\belowcaptionskip}{0.24cm}
%     \setlength{\abovecaptionskip}{-0.cm}
%            \includegraphics[width=2.1\columnwidth]{Figs/introall.png}
%            \captionof{figure}{Illustration of airport, landing of the delivery drone, sensors and algorithms performance. \textnormal{(a) The self-collection airport is equipped with multiple delivery drones for package distribution. (b) A delivery drone lands onto the platform. (c) Enhancing mmWave radar with event camera could have high spatial-temporal resolution and depth sensing capability. (d) Field study system performance comparison: green circles represent frame camera-based solutions, purple for mmWave radar-based, blue for radar-camera fusion, and red for event camera-based solutions.}}
%            \label{intro}
%         \end{center}
%     }]
% }


% \thispagestyle{empty} % no page number for the first page
% \pagestyle{empty}  % no page number for the second and the later pages

\newcommand{\mycustomsize}{\fontsize{21}{\baselineskip}\selectfont}
% \newcommand{\mycustomsize}{\fontsize{20}{\baselineskip}\selectfont}

% \title[EventLoc]{\mycustomsize{EventLoc: Unleashing the Potential of Event Camera in Low Latency 3D Object Localization with mmWave Radar}}
% \title[mmE-Loc]{\mycustomsize{High-Resolution Temporal Consistency Matters: Enhancing mmWave Radar with Event Camera for Accurate Drone Landing }}
% \title[mmE-Loc]{\mycustomsize{Ultra-High-Frequency Harmony: mmWave Radar and \\ Event Camera Orchestrate Accurate Drone Landing}}
\title[Ultra-High-Frequency Harmony: \\mmWave Radar and Event Camera Orchestrate Accurate Drone Landing]
{Ultra-High-Frequency Harmony: mmWave Radar and \\ Event Camera Orchestrate Accurate Drone Landing }
% 1. EventLoc: Enhancing mmWave Radars with Event Cameras for Low Latency Landing Drone Ground Localization 
% 2. mmEvent: Enhancing mmWave Radars with Event Cameras for Low Latency Landing Drone Ground Localization 
% 3. mmE-Loc: Enhancing mmWave Radars with Event Cameras for Low Latency Landing Drone Ground Localization 
% 4. FLASH:Enhancing mmWave Radars with Event Cameras for Low Latency Landing Drone Ground Localization 
% 5. mmE-$L^4$oc:Enhancing mmWave Radars with Event Cameras for Low Latency Landing Drone Ground Localization 
% 6. FLARE:Enhancing mmWave Radars with Event Cameras for Fast Landing Drone Ground Localization 
% 7. EventNest:Enhancing mmWave Radars with Event Cameras for Fast Landing Drone Ground Localization  

\author{Haoyang Wang$^{1}$, Jingao Xu$^{2}$, Xinyu Luo$^{1}$, Xuecheng Chen$^{1}$, Ting Zhang$^{1}$, \\ Ruiyang Duan$^3$, Yunhao Liu$^{4}$, Xinlei Chen$^{1, 5, 6}$\textsuperscript{\Envelope}}

\renewcommand{\authors}{Haoyang Wang, Jingao Xu, Xinyu Luo, Xuecheng Chen, Ting Zhang, Ruiyang Duan, Yunhao Liu, Xinlei Chen}

\affiliation{%
    \institution{$^1$ Shenzhen International Graduate School, Tsinghua University, China; $^2$ Carnegie Mellon University; \\$^3$ Meituan Academy of Robotics Shenzhen, China; $^4$ School of Software, Tsinghua University, China; \\$^5$ Pengcheng Laboratory, Shenzhen, China; $^6$ RISC-V International Open Source Laboratory, Shenzhen, China}
    \country{}
    \city{}
}

% \affiliation{%
%  \institution{$^1$ Shenzhen International Graduate School, Tsinghua University, China; $^2$ Carnegie Mellon University; $^3$ Meituan Technology, Shenzhen, China; $^4$ School of Software, Tsinghua University, China; $^5$ Pengcheng Laboratory, Shenzhen, China; $^6$ RISC-V International Open Source Laboratory, Shenzhen, China}
% }

\affiliation{%
  \institution{Email: \{haoyang-22, luo-xy23, chenxc21, yunhao\}@mails.tsinghua.edu.cn, \{xujingao13, zhangt2112\}@gmail.com, \\ duanruiyang@meituan.com, chen.xinlei@sz.tsinghua.edu.cn}
}

\renewcommand{\shortauthors}{Haoyang Wang, et al.}

% mmE$L^4$oc

% mmE-Loc: Enhancing mmWave Radars with Event Cameras for Low Latency Landing Drone Ground Localization 

% Low Latency 
% 3D Localization
% Landing drone ground Localization

% Temporal Consistency Matters: Enhancing mmWave Radars with Event Cameras for Low Latency Landing Drone Ground Localization 

% Temporal Consistency Matters: Enhancing mmWave Radar with Event Camera for Accurate Drone Landing 

% High Temporal Resolution Consistency Matters: Enhancing mmWave Radar with Event Camera for Accurate Drone Landing 

% Baseline: 
% High-Resolution Temporal Consistency Matters: Enhancing mmWave Radar with Event Camera for Accurate Drone Landing 
% 缩写:mmE-Loc

% \title[EventLoc]{EventLoc: Enhancing Event Cameras for Low Latency 3D Object Localization with mmWave Radar Integration}
% EventLoc: Leveraging mmWave Radar for Enhanced Low Latency 3D Object Localization with Event Cameras
% EventLoc: Enhancing Low Latency 3D Object Localization in Event Cameras with mmWave Radar Integration
% Unleashing the Potential of Event Camera in 3D localization with mmWave Radar: A Case for Landing Drone Tracking

% \author{Anonymous}

% \authornote{Both authors contributed equally to this research.}
% \affiliation{%
%   \institution{Shenzhen International Graduate School, Tsinghua University}
%     \city{Shenzhen}
%   \country{China}
% }
% \email{haoyang-22@mails.tsinghua.edu.cn}

% \author{Xuecheng Chen}
% \authornotemark[1]
% \affiliation{%
%   \institution{Tsinghua-Berkeley Shenzhen Institute, Tsinghua University}
%   \city{Shenzhen}
%   \country{China}
% }
% \email{chenxc21@mails.tsinghua.edu.cn}

% \author{Yuhan Cheng}
% \affiliation{%
%   \institution{Shenzhen International Graduate School, Tsinghua University}
%   \city{Shenzhen}
%   \country{China}
% }
% \email{cyh22@mails.tsinghua.edu.cn}

% \author{Chenye Wu}
% \affiliation{%
%   \institution{The Chinese University of Hong Kong, Shenzhen}
%   \city{Shenzhen}
%   \country{China}
% }
% \email{chenyewu@yeah.net}

% \author{Fan Dang}
% \affiliation{%
%   \institution{Global Innovation Exchange, Tsinghua University}
%     \city{Beijing}
%   \country{China}
% }
% \email{dangfan@tsinghua.edu.cn}

% \author{Xinlei Chen}
% \authornote{Xinlei Chen is the corresponding author.}
% \affiliation{%
%   \institution{Shenzhen International Graduate School, Tsinghua University}
%    \institution{Peng Cheng Laboratory}
%     \city{Shenzhen}
%   \country{China}
% }
% \email{chen.xinlei@sz.tsinghua.edu.cn}


%%
%% By default, the full list of authors will be used in the page
%% headers. Often, this list is too long, and will overlap
%% other information printed in the page headers. This command allows
%% the author to define a more concise list
%% of authors' names for this purpose.

% \renewcommand{\shortauthors}{Wang, et al.}



\begin{abstract}

For precise, efficient, and safe drone landings, ground platforms should real-time, accurately locate descending drones and guide them to designated spots.
% While mmWave sensing combined with cameras improves localization accuracy, 
% % frame cameras introduce latency and throughput bottlenecks that limit location update rates.
% the lower sampling frequencies of traditional frame cameras compared to mmWave radar introduce localization latency and create bottlenecks in system throughput.
While mmWave sensing combined with cameras improves localization accuracy, the lower sampling frequency of traditional frame cameras compared to mmWave radar creates bottlenecks in system throughput. 
In this work, we replace the traditional frame camera with event camera, a novel sensor that harmonizes in sampling frequency with mmWave radar within the ground platform setup, and introduce mmE-Loc, a high-precision, low-latency ground localization system designed for drone landings.
To fully leverage the \textit{temporal consistency} and \textit{spatial complementarity} between these modalities, we propose two innovative modules, \textit{consistency-instructed collaborative tracking} and \textit{graph-informed adaptive joint optimization}, for accurate drone measurement extraction and efficient sensor fusion.
% We propose two novel algorithms: Consistency-Instructed Collaborative Tracking and Graph-Informed Adaptive Joint Optimization, which leverage \textit{temporal consistency} and \textit{spatial complementarity} between modalities for accurate drone measurements extraction and efficient sensor fusion.
% Extensive real-world experiments in landing scenarios from, conducted with a leading drone delivery company demonstrate that mmE-Loc outperforms state-of-the-art methods in both localization accuracy and latency.
Extensive real-world experiments in landing scenarios from a leading drone delivery company demonstrate that mmE-Loc outperforms state-of-the-art methods in both localization accuracy and latency. 
\end{abstract}

\begin{CCSXML}
<ccs2012>
   <concept>
       <concept_id>10010520.10010553.10003238</concept_id>
       <concept_desc>Computer systems organization~Sensor networks</concept_desc>
       <concept_significance>500</concept_significance>
       </concept>
   <concept>
       <concept_id>10010147.10010178.10010199.10010201</concept_id>
       <concept_desc>Computing methodologies~Planning under uncertainty</concept_desc>
       <concept_significance>500</concept_significance>
       </concept>
 </ccs2012>
\end{CCSXML}

\ccsdesc[500]{Computer systems organization~Embedded systems}
\ccsdesc[500]{Information systems~Location based services}
% \ccsdesc[500]{Computing methodologies~Planning under uncertainty}

%
% Keywords. The author(s) should pick words that accurately describe
% the work being presented. Separate the keywords with commas.
\keywords{Drone Ground Localization; Event Camera; mmWave Radar}

\maketitle 

\renewcommand{\thefootnote}{}
\footnotetext{\Envelope\ Corresponding author.}
\footnotetext{Project homepage: \href{https://mmE-Loc.github.io/}{\color{blue}{https://mmE-Loc.github.io/}}}

\vspace{-0.8cm}
\section{Introduction}

Projected to soar to a \$1 trillion market by 2040 \cite{low_altitude_eco}, the drone-driven low-altitude economy is transforming sectors with revolutionary applications such as on-demand delivery \cite{wang2022micnest, chen2024ddl, chen2022deliversense}, meticulous industrial inspections \cite{xu2022swarmmap, li2024quest, xu2019vehicle, liu2024mobiair}, and rapid relief-and-rescue \cite{zhang2023rf, chi2022wi, chen2024soscheduler}. 
Of paramount importance within this burgeoning sector is the \textit{landing phase}, where ground platforms locate drones descending from below 10 meters and guide them to accurately land at designated spots (\fig\ref{intro}a) \cite{he2023acoustic, sun2022aim}.
Situated near populated and commercial zones, these operations emphasize safety and reliability: our research with a leading drone delivery company reveals that a landing bias of just 10$cm$ will result in drones damaging delivery targets or missing their charging ports \cite{gonzalez2021visual}. 
Such inaccuracies disrupt the operational efficiency of this swiftly growing economic sector, with potentially severe consequences.

\begin{figure}[t]
    \setlength{\abovecaptionskip}{0.25cm} % height above Figure X caption
    \setlength{\belowcaptionskip}{-0.3cm}
    \setlength{\subfigcapskip}{-0.5cm}
    \centering
        \includegraphics[width=0.92\columnwidth]{Figs/intro_v.png}
        \vspace{-0.2cm}
    \caption{Snapshot of drone landing phase, airport, and sensors performance. \textnormal{(a) A delivery drone lands on the platform. (b) The real-world drone airport is equipped with multiple drones for package delivery. 
    (c) Integrating mmWave radar with event camera combines reliable depth sensing and 2D imaging at ultra-high sampling frequencies, enabling high spatial-temporal resolution and depth sensing, while ensuring compatibility with flight controllers.}}
    \label{intro}
\end{figure} 

Widely adopted and straightforward approaches involve installing cameras at the center or edges of drone landing pads and employing computer vision algorithms for drone localization \cite{li2018real,xia2023anemoi,zhang2019eye}.
However, traditional frame cameras' Achilles heel is capturing only 2D images without depth information, leading to scale uncertainty that limits the 3D localization accuracy \cite{zhang2022mobidepth,xie2023mozart,10.1145/3517260}.
To address this shortcoming, current practices have incorporated mmWave sensing to provide the lacked depth information for better localization accuracy and reliability in various conditions \cite{deng2022geryon, lu2020smokerobustindoormapping,zhang2023mmhawkeye, sie2023batmobility, iizuka2023millisign, lu2020see, lu2020milliego}.

Albeit inspiring, our benchmark study with a world-leading drone delivery company in landing scenarios (\fig \ref{intro}b) reveals another critical drawback (\fig \ref{intro}c): the exposure times of frame cameras (>20$ms$) prevent their sampling rates from matching the high frequency of mmWave radars (\eg, 200Hz). 
This limitation creates system efficiency and throughput bottlenecks, restricting drone location updates to below 50Hz.
In contrast, drone flight controllers typically require location input rates over 150Hz to precisely adjust the drone's flight attitude for safe landing \cite{8412592, 10.1145/3570361.3592532}.
The inefficiency originates from the inherent physical limitations of conventional frame cameras and cannot be easily solved by software solutions.


\noindent \textbf{Upgrade frame camera to event camera.}
Event cameras are bio-inspired sensors that report pixel-wise intensity changes with $ms$-level resolution \cite{gallego2020event, ruan2024distill}, capturing high-speed motions without blurring \cite{he2024microsaccade}, ideal for fast-tracking tasks \cite{xu2023taming, luo2024eventtracker}.
Event cameras offer $ms$-level sampling latency, which harmonizes exceptionally with the high sampling frequencies of mmWave radar.
Their 2D imaging capability also complements radars' limited spatial resolution, similar to how traditional frame cameras operate.
Such \textit{temporal-consistency} and \textit{spatial-complementarity} across both modalities inspire us to upgrade frame cameras with event cameras to pair with radar for accurate and fast drone localization.


\noindent \textbf{Our work.}
Following the above insight, we present \textbf{mmE-Loc}, the first active, high-precision, and low-latency landing drone ground localization system that enhances mmWave radar functionality with event cameras. 
mmE-Loc works in scenarios where urban canyon environments degrade the accuracy of GPS or RTK systems as altitude decreases, rendering them nearly ineffective for the landing phase. 
With mmE-Loc, drones achieve reliable localization even under challenging conditions (\eg, weak illumination), ensuring stable and efficient landing.

However, our benchmark study at a real-world drone delivery airport (\fig \ref{relatedwork}a) highlights several challenges that have been solved in making \textbf{mmE-Loc} a viable system outdoors:
$(i)$ \textit{How to accurately extract drone-related measurements} given the immense noisy output of event cameras and mmWave radars, which also lack inherent drone semantic information and differ greatly in dimension and pattern?
Both modalities are sensitive to environmental variations (\eg, changes in lighting conditions), as shown in \fig \ref{relatedwork}b.
Existing algorithms \cite{cao2024virteach, liu2024pmtrack, wang2021asynchronous, alzugaray2018asynchronous} are typically designed for single-modality, resulting in low noise filtering rates (recall and precision < 65\% in \fig \ref{relatedwork}c).
$(ii)$ \textit{How to efficiently fuse event camera and mmWave readings} that are heterogeneous in measurement precision, scale, and density? 
Existing EKF (extended Kalman filter) or PF (particle filter) based approaches \cite{falanga2020dynamic,zhao20213d, mitrokhin2018event}, suffer from cumulative drift errors, making them insufficient for precise localization (\fig \ref{relatedwork}d).
$(iii)$ \textit{How to optimize the efficiency of the fusion algorithm} to achieve high-frequency drone ground localization, given the limited computational resources on landing platforms?
Existing methods experience significant processing delays, rendering them unsuitable for low-latency localization tasks (\fig \ref{relatedwork}d) \cite{zhao20213d, falanga2020dynamic, mitrokhin2018event}.

To solve the above challenges, the design and implementation of mmE-Loc excel in the three aspects of drone ground localization:

\noindent $\bullet$ \textit{On system architecture front.}
Upgrading frame camera to event camera with $ms$-level latency to pair the mmWave radar, mmE-Loc improves drone ground localization at data source.
The system architecture tightly integrates both modalities, from early-stage noise filtering and drone detection to later-stage fusion and optimization, fully leveraging the unique advantages of both sensors (§\ref{3.2}).\\
\noindent $\bullet$ \textit{On system algorithm front.}
We introduce a Consistency-Instructed Collaborative Tracking (\textit{CCT}) algorithm, which leverages the drone's periodic micro-motion and cross-modal \textit{temporal-consistency} to filter environment-triggered noise, achieving accurate drone detection (§\ref{4.1}). 
We then present a Graph-Informed Adaptive Joint Optimization (\textit{GAJO}) algorithm, which fuses \textit{spatial-complementarity} with a novel factor graph to boost drone ground localization, resulting in a trajectory with minimal bias and low cumulative drift(§\ref{4.2}). \\
\noindent $\bullet$ \textit{On system implementation front.}
We further analyze the sources of latency and propose an Adaptive Optimization method to improve the efficiency of the \textit{GAJO} algorithm.
This approach allows \textit{GAJO} to dynamically optimize a set of locations, maintaining accuracy while reducing latency (§\ref{I}).

\begin{figure}[t]
    \setlength{\belowcaptionskip}{-0.3cm}
    \setlength{\subfigcapskip}{-0.25cm}
    \centering
        \includegraphics[width=0.9\columnwidth]{Figs/relatedall_4.png}
        \vspace{-0.4cm}
    \caption{
    Benchmark study on drone localization.
    \textnormal{(a) Benchmark study at a real-world drone delivery airport; (b) Both sensors are sensitive to environmental variations; (c) Existing algorithms suffer from low noise filtering rates; (d) Existing algorithms experience cumulative drift errors and delays. }}
    \label{relatedwork}
    \vspace{-0.5cm}
\end{figure} 

\begin{figure*}[t]
    \setlength{\abovecaptionskip}{0.2cm} % height above Figure X caption
    \setlength{\belowcaptionskip}{-0.5cm}
    \setlength{\subfigcapskip}{-0.25cm}
    \centering
        \includegraphics[width=1.9\columnwidth]{Figs/overview.png}
        \vspace{-0.1cm}
    \caption{System architecture of mmE-Loc.}
    \label{overview}
    % \vspace{-0.2cm}
\end{figure*} 


We fully implement mmE-Loc using a COTS event camera and mmWave radar. 
Over 30+ hours of indoor and outdoor experiments under various drone flight conditions assess its localization accuracy and end-to-end latency performance against four SOTA methods.
mmE-Loc achieves an average localization accuracy of 0.083$m$ and latency of 5.15$ms$, surpassing baselines by >48\% and >62\%, respectively, and showing minimal sensitivity to drone type and environment.
We also deploy mmE-Loc at a real-world drone delivery airport (\fig \ref{relatedwork}a) for 10 hours, demonstrating its practicality for commercial-level drone landing requirements. 

In summary, this paper makes the following contributions.\\
\noindent $(1)$ We explore a fresh sensor configuration, event camera plus mmWave radar, that embraces and harmonizes ultra-high sampling frequencies and propose mmE-Loc, a ground localization system for drone landings that delivers $cm$ accuracy and $ms$ latency.\\
\noindent $(2)$ We present $CCT$, which leverages \textit{temporal consistency} and the drone's periodic micro-motions for precise drone detection; and $GAJO$, which employs \textit{spatial complementarity} with a novel factor graph to enhance drone localization.\\
\noindent $(3)$ We implement and extensively evaluate mmE-Loc by comparing it with four SOTA methods, showing its effectiveness. We also deploy mmE-Loc in a real-world drone delivery airport, demonstrating feasibility of mmE-Loc.

% \vspace{-0.2cm}
\section{System Overview}
The mmE-Loc enhances the mmWave radar with an event camera to achieve accurate and low-latency drone ground localization, allowing the drone to rapidly adjust its location state and perform a precise landing.
% Given the paramount importance of safety in commercial drone operations, mmE-Loc can collaborate with RTK or visual markers to ensure precise landings.
Given the critical importance of safety in commercial drone operations, mmE-Loc can work in conjunction with RTK or visual markers to ensure precise landing performance.
In this section, we mathematically introduce the problem that mmE-Loc tries to address and provide an overview of the system design.

% mmE-Loc leverages the integration of mmWave radar and event camera data to achieve accurate, low-latency drone ground localization, allowing the drone to quickly adjust its position and perform precise landings. Given the critical importance of safety in commercial drone operations, mmE-Loc can also work in conjunction with RTK or visual markers to ensure precise landing performance.

% enabling a reliable and accurate localization service for drones (\fig \ref{overview}).

\vspace{-0.3cm}
\subsection{Problem Formulation}
In this section, we illustrate key variables in mmE-Loc and introduce the system's inputs and outputs.

% \begin{figure}[t]
%     \setlength{\abovecaptionskip}{-0.1cm} % height above Figure X caption
%     \setlength{\belowcaptionskip}{-0.3cm}
%     \setlength{\subfigcapskip}{-0.25cm}
%     \centering
%         \includegraphics[width=1\columnwidth]{Figs/reference.png}
%         \vspace{-0.2cm}
%     \caption{Illustration of reference systems and essential variables in mmE-Loc.}
%     \label{reference}
%     % \vspace{-0.6cm}
% \end{figure} 

\textbf{Reference systems.} \label{3.2}
% As shown in \fig \ref{reference}, t
There are four reference (\aka, coordinate) systems in mmE-Loc: 
$(i)$ the Event camera reference system $\mathtt{E}$; 
$(ii)$ the Radar reference system $\mathtt{R}$; 
$(iii)$ the Object reference system $\mathtt{O}$;
$(iv)$ the Drone reference system $\mathtt{D}$.
Note that a drone can be considered as an object.
For clarity, before an object is identified as a drone, we utilize $\mathtt{O}$. 
Once confirmed as a drone, we use $\mathtt{D}$ for the drone and continue using $\mathtt{O}$ for other objects.
Throughout the operation of system, $\mathtt{E}$ and $\mathtt{R}$ remain stationary and are rigidly attached together, while $\mathtt{O}$ and $\mathtt{D}$ undergo changes in accordance with movement of the object and the drone, respectively. 
The transformation from $\mathtt{R}$ to $\mathtt{E}$ can be readily obtained from calibration \cite{wang2023vital}. 

\textbf{Goal of mmE-Loc.}
The goal of mmE-Loc is to determine 3D location of the drone, defined as $t_{\mathtt{ED}}$, the translation from coordinate system $\mathtt{D}$ to $\mathtt{E}$.
Specifically, mmE-Loc optimizes and reports 3D location of drone $(l_x, l_y, l_z)$ at each timestamp $i$ with input from event stream and radar sample.
% where $t^i_{\mathtt{ED}}$ represents the translation vector from $\mathtt{D}$ to $\mathtt{E}^3$. 
$t_{\mathtt{ED}}$ and ($l_x$, $l_y$, $l_z$) are equivalent representations of the drone’s location and can be inter-converted with Rodrigues’ formula \cite{min2021joint}. 
The former representation is adopted in the paper, as it is commonly used in drone flight control systems.

\begin{figure*}[t]
    \setlength{\abovecaptionskip}{0.3cm} % height above Figure X caption
    \setlength{\belowcaptionskip}{-0.2cm}
    \setlength{\subfigcapskip}{-0.25cm}
    \centering
        \includegraphics[width=1.85\columnwidth]{Figs/trackingmodel.png}
        \vspace{-0.3cm}
    \caption{Illustration of tracking models in Consistency-instructed Collaborative Tracking algorithm.}
    \label{CCT}
    \vspace{-0.2cm}
\end{figure*} 

% We first briefly describe and illustrate some essential variables in mmE-Loc and the problem of localizing the landing drone,which is the 3D location ($l_x$, $l_y$, $l_z$).
% \todo{As shown in \fig )}, there are three reference (\aka, coordinate) systems in mmE-Loc: $(i)$ the Event camera reference system \mathtt{E}; $(ii)$ the Radar reference system \mathtt{R}; $(iii)$ the Drone reference system \mathtt{D}.


% \subsection{mmE-Loc: System goals}
% \noindent $\bullet$ \textbf{How to detect and track the drones from noisy sensing results at a high frequency?}
% Environmental changes often introduce irrelevant information in the sensing results from the event camera and mmWave radar, hindering the system's ability to identify signals changes caused by the drones to be tracked. 
% Additionally, modern flight control loops operate at frequencies exceeding 400 Hz, requiring the system to detect and track drones rapidly. 
% However, traditional noise filtering, object detection and tracking algorithms have high time complexities, resulting in precision and tracking frequency bottlenecks. 
% \todo{Explanation in figures or tables.}

% \noindent $\bullet$ \textbf{How to fuse two types of heterogeneous data to precisely localize drones at a high frequency?} 
% Once the drone is detected, accurate 3D spatial localization of it is essential, which is more time-consuming than detection and tracking due to additional processing. 
% Moreover, event cameras provide asynchronous event streams, while mmWave radar generates sparse point clouds with relatively low spatial resolution. 
% Previous fusion methods (\eg extended Kalman filters and particle filters) often suffer from significant cumulative drift error and lengthy processing times, rendering them inadequate for high-frequency and high-precision tracking.
% \todo{Explanation in figures or tables.}

\vspace{-0.3cm}
\subsection{Overview}
% mmE-Loc is a localization system designed for high-frequency and precise localization of drones, enabling them to rapidly adjust their location state and achieve accurate landing. 
As illustrated in \fig \ref{overview}, mmE-Loc comprises two key modules: 
% $(i)$ \textit{CCT} (\textbf{C}onsistency-instructed \textbf{C}ollaborative \textbf{T}racking) for noise filtering and drone detection and 
% $(ii)$ \textit{GAO} (\textbf{G}raph-informed \textbf{A}daptive \textbf{O}ptimization) for localization of drones from the integrated event stream and mmWave 3D point cloud. \todo{Fix the figure.}

\noindent $\bullet$ 
The \textit{CCT} (\textbf{C}onsistency-instructed \textbf{C}ollaborative \textbf{T}racking) for noise filtering, drone detection, and preliminary localization of the drone.
This module utilizes time-synchronized event streams and mmWave radar measurements as inputs. 
Subsequently, the \textit{Radar Tracking Model} processes radar measurements to generate a sparse 3D point cloud. 
Meanwhile, the \textit{Event Tracking Model} takes into the stream of asynchronous events for event filtering, drone detection, and tracking. 
% Finally, \textit{Consistency-instructed Measurements Filter} aligns the results of the both tracking models leverage the consistency between both modalities, and then utilizes periodic micro motion of drone to extract drone-related measurements,  eliminate noise and error detections, and roughly localize the drone.
Finally, \textit{Consistency-instructed Measurements Filter} aligns the outputs of both tracking models by leveraging \textit{temporal-consistency} between the two modalities. 
It then utilizes the drone's periodic micro-motion to extract drone-specific measurements and achieve drone preliminary localization.

\noindent $\bullet$
The \textit{GAJO} (\textbf{G}raph-informed \textbf{A}daptive \textbf{J}oint \textbf{O}ptimization) for fine localization and trajectory optimization of the drone.
% GAJO initially derives preliminary estimates of drone motion and location, using radar measurements and event camera tracking results, respectively. 
% GAJO includes a carefully designed \textit{factor graph-based optimization} method, which 通过深入挖掘事件相机与radar的潜力, jointly fuses and refines the outputs from both the \textit{Event Tracking Model} and \textit{Radar Tracking Model} to accurately determine the location of drone. 
Based on the operational principles of two sensors and their respective noise distributions, \textit{GAJO} incorporates a meticulously designed \textit{factor graph-based optimization} method. 
This module employs the \textit{spatial-complementarity} from both modalities to unleash the potential of event camera and mmWave radar in drone ground localization.
Specifically, \textit{GAJO} jointly fuses the preliminary location estimation from the \textit{Event Tracking Model} and the \textit{Radar Tracking Model} and adaptively refines them, determining the fine location of drone with $ms$-level processing time.
% Finally, to expedite the process of the \textit{factor graph-based location optimization} method, we transform factor graph into a Bayes tree through elimination and optimize the drone's location adaptively via a \textit{Incremental Optimization method}. 
\section{Design of \chatiot}\label{sec:design}
In this section, we first describe the construction of \iotrgen\ in \S~\ref{sec:retrieval}, followed by an overview of the data processing toolkit in \S~\ref{sec:toolkit}.
Finally, we present the use case specifications and explain how \chatiot\ is designed to be user-friendly in \S~\ref{sec:usecase}.


\subsection{Construction of \iotrgen}\label{sec:retrieval}

%\begin{algorithm}[t]
\caption{\iotrgen}\label{alg:iotrgen}
\begin{algorithmic}[1]
\REQUIRE
User \textsf{role}, query $Q$, large language model $\mathcal{M}$, and retrievers $\{\mathcal{R}_i\}_{i=1}^n$.

\ENSURE
Generated answer $A$.

\STATE
\textcolor{gray}{$\triangleright$ \textbf{Procedure of Selector:}}

\STATE
Set prompt $\mathsf{SPT} = (\mathsf{Task}, \mathsf{role}, Q, \{\mathcal{R}_i\}_{i=1}^n)$, where $\mathsf{role}$ includes user's background implicitly and $\mathcal{R}_i$ indicates its description here.
\STATE
Inputting $\mathsf{SPT}$ to $\mathcal{M}$ and get $\{S_i\}_{i=1}^n = \mathcal{M}(\mathsf{SPT})$, where $S_i\in \{\mathsf{True}, \mathsf{False}\}$.

\STATE 
\textcolor{gray}{$\triangleright$ \textbf{Procedure of Self-Querying Retrieval:}}
\FORALL{$i=1, \dots, n$}
\IF{$\mathcal{S}_i = \mathsf{True}$}
\STATE
Execute self-querying retriever $\mathcal{R}_i$ and get documents $D_i=\mathcal{R}_i(Q)$
\ELSE
\STATE
Set $D_i = \mathsf{NULL}$.
\ENDIF
\ENDFOR

\STATE
\textcolor{gray}{$\triangleright$ \textbf{Procedure of Guided Generation:}}

\STATE
Set prompt $\mathsf{UPT} = (\mathsf{Task}, \mathsf{role}, Q, \{\mathcal{D}_i\}_{i=1}^n)$

\RETURN
$A = \mathcal{M}(\mathsf{UPT})$.


\end{algorithmic}
\end{algorithm}

As shown in Figure.~\ref{fig:system}, \iotrgen\ consists of \textit{Adaptive Retrieval} and \textit{Guided Generation}.
We present their detailed constructions as follows.
%The third component utilizes LLM $\mathcal{M}$ to generate the response based on retrieved documents, role, and query, following a straightforward process. 
%This subsection will provide the detailed constructions of the Retriever Selector and Retrievers.

\subsubsection{Adaptive Retrieval}\label{sec:rca}
Recall that we have multiple retrievers, each dedicated to retrieving documents generated from a specific dataset. We achieve adaptive retrieval mechanism works in two aspects: 
\romannumeral1) selecting which retrievers should be activated and \romannumeral2) trying to retrieve only relevant documents while discarding irrelevant ones, even for the activated retrievers.

\smallskip
\noindent \textbf{Design of Selector.}
When user \textsf{role} submits a query $Q$, a straightforward and \textit{static} approach is to $Q$ to all retrievers and gather retrieved documents from them. 
However, this method has the following drawbacks:
\romannumeral1) \textit{Irrelevant Retrievers.} Documents of some retrievers may not be relevant to the query. Retrieving them not only fails to improve the quality of the generated answer but may even negatively affect it.
\romannumeral2) \textit{Resource Costs.} Retrieving unnecessary retrievers leads to increased resource costs, such as computational overhead, during both the retrieval and generation processes.

To address these issues, we introduce an LLM-based adaptive \textit{Selector}. As shown in Figure~\ref{fig:system}, we put the user \textsf{role} with the background, the query $Q$, and the descriptions of all retrievers as \textit{Selector Prompt (SPT)} and input SPT to LLM $\mathcal{M}$, which generates $\{\mathcal{S}_i\}_{i=1}^n$,
where $\mathcal{S}_i = \mathsf{True}$ implies that the retriever $\mathcal{R}_i$ should be activated, and $\mathsf{False}$ indicates not. 
In Table~\ref{tab:selectorexample}, we present the configurations generated by LLaMA3:8B for some example queries submitted by different users.

\iffalse
\iffalse
\begin{table}[]
    \centering
    \caption{The configurations $\{\mathcal{S}_i\}_{i=1}^n$ generated by LLaMA3:8B for different user role and example queries. \ding{52} is for \texttt{True} and \ding{56} is for \texttt{False}.}
    \label{tab:selectorexample}
    \resizebox{0.4\textwidth}{!}{
    \begin{tabular}{@{}c@{\hskip 0.1cm}|
    @{\hskip 0.1cm}>{\centering\arraybackslash}p{4.5cm}@{\hskip 0.1cm}|
    @{\hskip 0.1cm}c
    @{\hskip 0.1cm}c
    @{\hskip 0.1cm}c
    @{\hskip 0.1cm}c
    @{\hskip 0.1cm}c
    @{}}
    \toprule
    \toprule
         Role & Example Query & $\mathcal{S}_1$ & $\mathcal{S}_2$ & $\mathcal{S}_3$ & $\mathcal{S}_4$ & $\mathcal{S}_5$  \\ \midrule
         { \centered{Consumer}} & \centered{ Is it secure to use Signify Smart \\  Lighting in home?} & \ding{52} & \ding{56} & \ding{52} & \ding{56} & \ding{52} \\ \midrule
         {\centered{Security\\ Analyst}} & \centered{ Conduct a security assessment, \\  including vulnerability, exploits, \\  threats, and others, for\\  TP-Link AX6000 Wi-Fi Router.} & \ding{52} & \ding{52} & \ding{52} & \ding{56} & \ding{56} \\ \midrule
         { \centered{Technical\\ Officer}} & \centered{ Check TTPs and security labeling \\  of the company's WiFi Routers, \\  including TP-Link, D-Link, \\  and ASUS in Singapore.} & \ding{52} & \ding{52} & \ding{56} & \ding{52} & \ding{52} \\ \midrule
         { Developer} & \centered{ Develop a security enhancement \\  roadmap for the next generation\\  of TP-Link Wi-Fi routers.} & \ding{52} & \ding{56} & \ding{52} & \ding{56} & \ding{52} \\ \midrule
         { Trainer} & \centered{ Prepare a guide on the importance \\   of cybersecurity labeling for smart \\  locks like the August Smart Lock.} & \ding{52} & \ding{56} & \ding{56} & \ding{56} & \ding{52}  \\
         \bottomrule
         \bottomrule
    \end{tabular}}
\end{table}
\fi 
\fi 

\begin{table}[ht]
    \centering
    \caption{Selector configurations $\{\mathcal{S}_i\}_{i=1}^n$ generated by LLaMA3:8B for different user roles and example queries. \ding{52} indicates \texttt{True} and \ding{56} is for \texttt{False}.}
    \label{tab:selectorexample}
    \resizebox{0.48\textwidth}{!}{
    \begin{tabular}{@{}p{1.5cm}|p{6cm}|c c c c c}
    \toprule
    \toprule
    \multicolumn{1}{c|}{\textbf{Role}} & \multicolumn{1}{c|}{\textbf{Example Query}} & $\mathcal{S}_1$ & $\mathcal{S}_2$ & $\mathcal{S}_3$ & $\mathcal{S}_4$ & $\mathcal{S}_5$  \\ \midrule
    Consumer & Is it secure to use Signify Smart Lighting in home? & \ding{52} & \ding{56} & \ding{52} & \ding{56} & \ding{52} \\ \midrule
    Security Analyst & Conduct a security assessment, including vulnerability, exploits, and threats, for TP-Link AX6000 Wi-Fi Router. & \ding{52} & \ding{52} & \ding{52} & \ding{56} & \ding{56} \\ \midrule
    Technical Officer & Check TTPs and security labeling of the company's WiFi Routers, including TP-Link, D-Link, and ASUS in Singapore. & \ding{52} & \ding{52} & \ding{56} & \ding{52} & \ding{52} \\ \midrule
    Developer & Develop a security enhancement roadmap for the next generation of TP-Link Wi-Fi routers. & \ding{52} & \ding{56} & \ding{52} & \ding{56} & \ding{52} \\ \midrule
    Trainer & Prepare a guide on the importance of cybersecurity labeling for smart locks like the August Smart Lock. & \ding{52} & \ding{56} & \ding{56} & \ding{56} & \ding{52}  \\
    \bottomrule
    \bottomrule
    \end{tabular}
    }
\end{table}



\begin{figure}[h]
    \centering
    \includegraphics[width=0.48\textwidth]{Figures/retriever.pdf}
    \caption{The construction of self-querying retriever based on LangChain and Elastic. When $\mathcal{S}_i=\mathsf{True}$, retrieve documents that are semantically similar to query $Q$ and filtered by metadata.}
    \label{fig:retri}
\end{figure}


\smallskip
\noindent \textbf{Self-Querying Retrievers.}
However, even the activated retrievers may return information that does not meet requirements, \eg, mismatch \textit{id} and \textit{products}. To address this problem, we make use of a self-querying technique to filter documents by \textsf{metadata}. 



\begin{figure*}[h]
    \centering 
\includegraphics[width=\linewidth]{Figures/code3.jpg}
\caption{Metadata fields information and examples for the self-querying retriever corresponding to the VARIoT vulnerabilities dataset.}
\label{fig:variotvulcode}
\end{figure*}


\begin{figure*}
    \centering
    \includegraphics[width=\linewidth]{Figures/code4.jpg}
    \caption{The generated queries and filters for VARIoT vulnerabilities and CLS list for the submitted query "What are the security issues with DLink DCS-942 camera?" by a Security Analyst.}
    \label{fig:querytrans}
\end{figure*}

As illustrated in Figure~\ref{fig:retri}, the self-querying retriever is composed of a Query Constructor, Query Translator, Search Params, and VectorStore.
When the user's query $Q$ is passed to Query Constructor, LLM will generate \textit{internal query language elements} based on pre-defined \textit{metadata\_field information} and \textit{metadata\_examples}.
Query Translator converts these elements into a \textit{structured query} with appropriate filters. 
Finally, the structured query and search parameters are applied to the Vector Store to retrieve documents. 
To specialize the self-querying retriever for IoT security, we take the following steps:
\begin{itemize}
    \item[\textcircled{1}] \textbf{Metadata \& Examples.} We provide the metadata field information and examples from relevant datasets. For instance, in the VARIoT vulnerabilities dataset, fields \textit{id} and \textit{products} are utilized as metadata. The corresponding \textit{metadata\_field\_info} and \textit{examples} are illustrated in Figure~\ref{fig:variotvulcode}, And details for other datasets are available in Appendix~\ref{appendix:metadataother}. This ensures the LLM can gain the necessary IoT security-specific knowledge to generate effective internal query language elements from the user's query.
    
    \item[\textcircled{2}] \textbf{Create Structured Queries.} Based on the above customized internal query language elements, Query Translator can create specific structured queries for each retriever. 
    Figure~\ref{fig:querytrans} shows how to enable the retrieval of VARIoT vulnerabilities and CLS lists that are both semantically similar to the query and appropriately filtered by their respective metadata.
\end{itemize}


%Specifically, given any query $Q$ in natural language, the retriever uses LLM-based to extract filters $F$ from $Q$, generate the structured query $\mathcal{Q}$ from $Q$ and $F$, and finally applies $\mathcal{Q}$ to its underlying VectorStore. 

\begin{figure}[h]
    \centering    
{\footnotesize
\begin{tikzpicture}
% Draw rounded rectangle with shadow
\node[rectangle, rounded corners, draw=black, fill=black!3!white, text width=0.43\textwidth, inner sep=12pt, align=left] (box) {
\textbf{\texttt{"Knowledge"}}: "The consumer may not have formal technical training but are familiar with using IoT devices for daily convenience such as smart home systems. The consumer has a basic understanding of device operation but may not be aware of the intricate security risks that exist."
\vspace{5pt}\\
\textbf{\texttt{"Goals"}}: "The primary aim is to understand whether a device is secure and how to maintain or improve its security, ensure safety, security, and reliability
of IoT devices within their homes or personal environments."
\vspace{5pt}\\
\textbf{\texttt{"Requirements"}}: "The answers should be practical, easy to follow, and focused on actionable steps the general user can take."
};
\end{tikzpicture}
}
\caption{The background for consumer utilized to guide the \chatiot\ to generate consumer-friendly outputs.}
\label{fig:usecasedef-consumer}
\end{figure}

\subsubsection{Guided Generation}\label{sec:resgen}

After retrieval, one direct step is feeding the retrieved documents and query to LLM to generate the answer. 
However, this simple approach is likely to result in user-unfriendly outputs. 
For example, consumers often lack the expertise needed to fully comprehend highly technical content, making such answers unhelpful and not actionable for them.

To address this issue, we incorporate user-specific backgrounds, including knowledge, goals, and requirements, into each user type's user-friendly prompt template \textit{UPT}. This adjustment guides LLM in generating answers tailored to the user's needs.
The specific background for the general consumer is shown in Figure~\ref{fig:usecasedef-consumer}, demonstrating how we simplify content for easy understanding. The background specifications of other user types can be referred to Appendix~\ref{appendix:bk}.


\begin{algorithm}[t]
\caption{\iotrgen}\label{alg:iotrgen}
\begin{algorithmic}[1]
\REQUIRE
User \textsf{role}, query $Q$, large language model $\mathcal{M}$, and retrievers $\{\mathcal{R}_i\}_{i=1}^n$.

\ENSURE
Generated answer $A$.

\STATE
\textcolor{gray}{$\triangleright$ \textbf{Procedure of Selector:}}

\STATE
Set prompt $\mathsf{SPT} = (\mathsf{Task}, \mathsf{role}, Q, \{\mathcal{R}_i\}_{i=1}^n)$, where $\mathsf{role}$ includes user's background implicitly and $\mathcal{R}_i$ indicates its description here.
\STATE
Inputting $\mathsf{SPT}$ to $\mathcal{M}$ and get $\{S_i\}_{i=1}^n = \mathcal{M}(\mathsf{SPT})$, where $S_i\in \{\mathsf{True}, \mathsf{False}\}$.

\STATE 
\textcolor{gray}{$\triangleright$ \textbf{Procedure of Self-Querying Retrieval:}}
\FORALL{$i=1, \dots, n$}
\IF{$\mathcal{S}_i = \mathsf{True}$}
\STATE
Execute self-querying retriever $\mathcal{R}_i$ and get documents $D_i=\mathcal{R}_i(Q)$
\ELSE
\STATE
Set $D_i = \mathsf{NULL}$.
\ENDIF
\ENDFOR

\STATE
\textcolor{gray}{$\triangleright$ \textbf{Procedure of Guided Generation:}}

\STATE
Set prompt $\mathsf{UPT} = (\mathsf{Task}, \mathsf{role}, Q, \{\mathcal{D}_i\}_{i=1}^n)$

\RETURN
$A = \mathcal{M}(\mathsf{UPT})$.


\end{algorithmic}
\end{algorithm}

Formally, we summarize and show the workflow of \iotrgen\ in algorithm~\ref{alg:iotrgen}.


\begin{figure*}
    \centering
    \includegraphics[width=\linewidth]{Figures/dataprocess.pdf}
    \caption{The construction of data processing toolkit \datakit. After collecting various IoT security datasets, \datakit\ first parses the raw data to get elements of multi-modal (step~\textcircled{1}), and then converts the multi-modal elements into text by utilizing LLM (step~\textcircled{2}). Finally, \datakit\ uses LLM to select fields for the page\_content and metadata of documents (step~\textcircled{3}), and optimizes the chunking strategy (step~\textcircled{4}).}
    \label{fig:dataprocess}
\end{figure*}

\subsection{Data Processing Toolkit}\label{sec:toolkit}

The raw data of the collected datasets are of different formats, \eg, PDF and JSON. These formats are not suitable for RAG processing directly, and thus we develop our end-to-end data processing \datakit. As shown in Figure~\ref{fig:dataprocess}, \datakit\ works in four steps:
\begin{itemize}
    \item[\ding{172}] \textbf{Parse Raw Data.} The initial step involves parsing raw data into distinct content elements such as text, tables, figures, and code. Leveraging existing tools like the unstructured library\footnote{\scriptsize \url{https://pypi.org/project/unstructured/}} helps in extracting textual content from threat reports, while the JSON library\footnote{\scriptsize \url{https://python.readthedocs.io/en/v2.7.2/library/json.html}} is useful for handling structured data from sources like VARIoT and MITRE ATT\&CK.
    
    \item[\ding{173}] \textbf{Convert Multi-Modal Elements to Text.} Once parsed, any multi-modal elements (\eg, tables, figures, code) must be converted into text descriptions for further processing. LLMs like LLaVA (for images)~\cite{liu2023llava,liu2023improvedllava,liu2024llavanext}, LLaMA3:8B (for tables), and CodeLlama (for code)~\cite{roziere2023code} are employed to generate these descriptions. This modular approach allows for easy integration of other LLMs to handle different types of multi-modal content.

    \item[\ding{174}] \textbf{Field Selection for Page\_Content \& Metadata.} In structured formats like JSON, content is often stored across multiple fields.
    Instead of using all fields, it is crucial to identify and utilize the most relevant ones for retrievers. This is done by sampling example items from each field and using an LLM (\eg, LLaMA3:8B) to intelligently select fields that best represent the document's page\_content and metadata, and the prompt is shown in Figure~\ref{fig:dataprocess}.
    Concretely, the selected fields for page\_content and metadata of each dataset are shown in \S~\ref{sec:exp-field}.

    \item[\ding{175}] \textbf{Optimize Chunking Strategy.} The final step involves selecting an appropriate chunking strategy, including the chunking size, overlap, and splitting method. The Ragas library~\cite{es2023ragas} is used to optimize this process, and the details are shown in algorithm~\ref{alg:chunkopt}.
\end{itemize}

\begin{algorithm}[t]
\caption{Optimize Chunking Strategy}\label{alg:chunkopt}
\begin{algorithmic}[1]
\REQUIRE
Documents $D$, chunking $sizes$, $overlaps$, and $splitters$ functions, $metrics=[precision, recall]$, and LLM $\mathcal{M}$.

\ENSURE
Optimized $(size^*, overlap^*, splitter^*)$.

\FORALL{$splitter \in splitters$}
\FORALL{$size \in chunk\_sizes$}
\FORALL{$overlap\in overlaps$}
\IF{$overlap < size$}
\STATE
\textcolor{gray}{$\triangleright$ \textbf{Split $D$ as small chunks:}}
\STATE
$\{d_i\}_i = splitter(D, size, overlap)$.
\STATE
$(p, r) = \text{Ragas.evaluate}(\{d_i\}_i, \mathcal{M}, metrics)$.
\ELSE
\STATE
\texttt{break.} \textcolor{gray}{$\triangleright$ \textbf{Go to next chunk size.}}
\ENDIF
\ENDFOR
\ENDFOR
\ENDFOR
\RETURN
chunking $(size^*, overlap^*, splitter^*)$ with a optimized trade-off between $(p, r)$.
\end{algorithmic}
\end{algorithm}

After obtaining the optimized chunking strategy, we split the documents into small chunks and use the all-MiniLM model for chunked text embedding. 
The documents are composed of chunked text, embedding, and metadata.
This approach ensures that the IoT security and threat datasets are processed efficiently and ready for further analysis or use in LLM. 
Note while many of these technologies are adapted from existing works, we foucs on putting them all together to develop an end-to-end data processing toolkit, which might be of independent interest and useful in practical applications.

\begin{table}[!hbpt]
    \centering
    \footnotesize
    \begin{tabular}{llp{0.4\linewidth}}
    \hline
         Use Case & Factor & Description \\
         \hline
         \multicolumn{3}{l}{\textbf{Professional Use Cases}}\\
         \quad Lawyer & Doctoral/Professional Degree & Advises clients on digital legal proceedings/transactions. \\
         \quad Elementary School Teacher & Bachelor’s degree & Teaches academic skills at the elementary school level. \\
         \quad IT Support Specialist & Some college, no degree & Maintains computer networks and provides technical help. \\
         \quad Government Eligibility Interviewer & High school diploma & Determine eligibility for government programs/resources. \\
         \quad Telemarketer & No formal education & Solicits donations or orders over the telephone. \\
         \hline
         \multicolumn{3}{l}{\textbf{Personal Use Cases}}\\
         \quad Digital Medical Advice & High Risk & Provide medical assessments prior to medical consultations. \\
         \quad Customized Lifestyle Coach & High / Limited Risk & Personalized advice for healthy living and wellness. \\
         \quad Personal Health Research & Limited Risk & Summarizes research related to personal health issues. \\
         \quad Nutrition Optimizer & Limited / Low Risk & Personalize meals and optimize nutritional intake. \\
         \quad Flavorful Swaps & Low Risk & Suggest delicious and healthy alternatives food options. \\
    \hline
    \end{tabular}
    \caption{Use cases selected for our study by categories. Use case descriptions were shortened for brevity.}
    \label{tab:use-cases}
\end{table}


\chapter{Implementation}{\label{ch:implementation}}
In this chapter, we present the implementation of the final product. We start by discussing how the four steps introduced in \hyperref[ch:high_level_approach]{chapter \ref*{ch:high_level_approach}} are integrated. We then outline the main system components of our score follower, presenting each as an independent, self-contained module. We then combine this into an overall system architecture and finally introduce the open-source score renderer used to display the score and evaluate the score follower.       

% \section{Aims and Requirements}
% The overall aim of the score follower was to 


\section{Score Follower Framework Details}
Our score follower conforms to the high-level framework presented in \hyperref[section:score_follower_framework]{section \ref*{section:score_follower_framework}}. In step 1, two score features are extracted from a MIDI file (see \hyperref[subsection:midi]{subsection \ref*{subsection:midi}}), namely MIDI note numbers\footnote{\href{https://inspiredacoustics.com/en/MIDI_note_numbers_and_center_frequencies}{https://inspiredacoustics.com/en/MIDI\_note\_numbers\_and\_center\_frequencies}} (corresponding to pitch) and note onsets (corresponding to duration). In step 2, the audio is streamed (whether from a file or into a microphone) and audioframes that exceed some predefined energy threshold are extracted. Here, audioframes are groups of contiguous audio samples, whose length can be specified by the argument \verb|frame_length|, usually between 800 and 2000 samples. The period between consecutive audioframes can also be defined by the argument \verb|hop_length|, typically between 2000 and 5000 audio samples. In step 3, score following is performed via a `Windowed' Viterbi algorithm (see  \hyperref[subsection:adjusting_viterbi]{subsection \ref*{subsection:adjusting_viterbi}}) which uses the Gaussian Process (GP) log marginal likelihoods (LMLs) for emission probabilities (see \hyperref[section:state_duration_model]{section \ref*{section:state_duration_model}}) and a state duration model for transition probabilities (see \hyperref[section:state_duration_model]{section \ref*{section:state_duration_model}}). Finally, in step 4 we render our results using an adapted version of the open source user interface, \textit{Flippy Qualitative Testbench}.

\section{Following Modes}
Two modes are available to the user: Pre-recorded Mode and Live Mode. The former requires a pre-recorded $\verb|.wav|$ file, whereas the latter takes an input stream of audio via the device's microphone. Note that both modes are still forms of score following, as opposed to score alignment, since in each mode we receive audioframes at the sampling rate, not all at once.\\

Live Mode offers a practical example of a score follower, displaying a score and position marker which a musician can read off while playing. However, this mode is not suitable for evaluation because the input and results cannot be easily replicated. Even ignoring repeatability, Live Mode is not suitable for one-off testing since a musician using this application may be influenced by the movement of the marker. For instance, the performer may speed up if the score follower `gets ahead' or slow down if the position marker lags or `gets lost'. To avoid this, we use Pre-recorded Mode when evaluating the performance of our score follower. Furthermore, Pre-recorded Mode offers the advantage of testing away from the music room, providing the opportunity to evaluate a variety of recordings available online. 

\section{System Architecture}
Our guiding principle for development was to build modular code in order to create a streamlined system where each component performs a specific task independently. This structure facilitates easy testing and debugging. \hyperref[fig:black_box]{Figure \ref*{fig:black_box}} presents a high-level architecture diagram, where each black box abstracts a key component of the score follower. When operating in Pre-recorded Mode, there is the option to stream the recording during run-time, which outputs to the device's speakers (as indicated by the dashed lines).

\begin{figure}[H]
    \centering
    \includegraphics[width=1\textwidth]{figs/Part_4_Implementation_And_Results/black_box.png}
    \caption{Abstracted system architecture diagram displaying inputs in grey, the 4 main components of the score follower in black and the outputs in green.}
    \label{fig:black_box}
\end{figure}

\subsection{Score Preprocessor}
The architecture for the Score Preprocesor is given in \hyperref[fig:score_preprocessor]{Figure \ref*{fig:score_preprocessor}}. First, MIDI note number and note onset times are extracted from each MIDI event. Simultaneous notes can be gathered into states and returned as a time-sorted list of lists called \verb|score|, where each element of the outer list is a list of simultaneous note onsets. Similarly, a list of note durations calculated as the time difference between consecutive states is returned as \verb|times_to_next|. Finally, all covariance matrices are precalculated and stored in a dictionary, where the key of the dictionary is determined by the notes present. This is because the distribution of notes and chords in a score is not random: notes tend to belong to a home \gls{key} and melodies tend to be repeated or related (similar to subject fields in speech processing). Therefore, states tend to be reused often, allowing us to achieve amortised time and space savings (by avoiding repeated calculation of the same covariance matrices). 

\begin{figure}[H]
    \centering
    \includegraphics[width=1\textwidth]{figs/Part_3_Implementation/Stage_2_Alignment/score_preprocessor.png}
    \caption{System architecture diagram representing the Score Preprocessor with inputs in grey, processes in blue and objects in yellow.}
    \label{fig:score_preprocessor}
\end{figure}


\subsection{Audio Preprocessor}
The architecture for the Audio Preprocessor is illustrated in \hyperref[fig:audio_preprocessor]{Figure \ref*{fig:audio_preprocessor}}. In Pre-recorded Mode, the Slicer receives a $\verb|.wav|$ file and returns audioframes separated by the \verb|hop_length|. These audioframes are periodically added to a multiprocessing queue, \verb|AudioFramesQueue|, to simulate real-time score following. In Live Mode, we use the python module \verb|sounddevice| to receive a stream of audio, using a periodic callback function to place audioframes on \verb|AudioFramesQueue|. 

\begin{figure}[H]
    \centering
    \includegraphics[width=1\textwidth]{figs/Part_4_Implementation_And_Results/audio_preprocessor.png}
    \caption{System architecture diagram representing the Audio Preprocessor with inputs in grey, processes in blue and objects in yellow.}
    \label{fig:audio_preprocessor}
\end{figure}

\subsection{Follower and Backend}
The joint Follower and Backend architecture diagram is shown in \hyperref[fig:follwer_and_backend]{Figure \ref*{fig:follwer_and_backend}}. The Viterbi Follower (detailed in \hyperref[subsection:adjusting_viterbi]{section \ref*{subsection:adjusting_viterbi}}) calculates the most probable state in the score, given audioframes continually taken from \verb|AudioFramesQueue|. These states are placed on another multiprocessing queue, the \verb|FollowerOutputQueue|, for the Backend to process and send. This prevents any bottle-necking occurring at the Follower stage. The Backend first sets up a UDP connection and then reads off values from \verb|FollowerOutputQueue|, sending them via UDP packets to the score renderer.

\begin{figure}[H]
    \centering
    \includegraphics[width=1\textwidth]{figs/Part_4_Implementation_And_Results/follower_and_backend.png}
    \caption{System architecture diagram representing the Follower and Backend processes with processes in blue, objects in yellow and outputs in green.}
    \label{fig:follwer_and_backend}
\end{figure}

\subsection{Player}
In Pre-recorded Mode, the Player sets up a new process and begins streaming the recording once the Follower process begins. This provides a baseline for testing purposes, as a trained musician can observe the score position marker and judge how well it matches the music. 

\subsection{Overall System Architecture}
The overall system architecture is presented in \hyperref[fig:overall_system_architecture]{Figure \ref*{fig:overall_system_architecture}}. Since the Follower runs a real-time, time sensitive process, parallelism is employed to reduce the total system latency. We use two \verb|multiprocessing| queues\footnote{\href{https://docs.python.org/3/library/multiprocessing.html}{https://docs.python.org/3/library/multiprocessing.html}} to avoid bottle-necking, which allows us to run 4 concurrent processes (Audio Preprocessor, Follower, Backend, and Audio Player). Hence, this architecture allows the components to run independently of one another to avoid blocking. Furthermore, this allows the system to take advantage of the multiple cores and high computational power offered by most modern machines.  

\begin{figure}[H]
    \centering
    \includegraphics[width=1\textwidth]{figs/Part_4_Implementation_And_Results/overall_score_follower_2.png}
    \caption{System architecture diagram representing the overall score follower running in Pre-recorded mode, with inputs in grey, processes in blue, objects in yellow and outputs in green.}
    \label{fig:overall_system_architecture}
\end{figure}


\section{Rendering Results}{\label{section:renderer}}
To visualise the results of our score follower, we adapted an open source tool for testing different score followers.\footnote{\href{https://github.com/flippy-fyp/flippy-qualitative-testbench/blob/main/README.md}{https://github.com/flippy-fyp/flippy-qualitative-testbench/blob/main/README.md}} \hyperref[fig:flippy_example]{Figure \ref*{fig:flippy_example}} shows the user interface of the score position renderer, where the green bar indicates score position. 

\begin{figure}[H]
    \centering
    \includegraphics{figs/Part_4_Implementation_And_Results/example_renderer.png}
    \caption{Screenshot of the score renderer user interface which displays a score (here we show a keyboard arrangement of \textit{O Haupt voll Blut und Wunden} by Bach). The green marker represents the score follower position.}
    \label{fig:flippy_example}
\end{figure}



\section{Evaluation}

% Our proposed framework was compared with Apollo \cite{b7Apollo1, b7Apollo2}, which demonstrates that it can model analytic operators using data content. Two loss functions were utilized, the root-mean-square deviation error (RMSE), and the mean absolute error (MAE). The selection of these two loss functions is because they fulfil the disadvantages of each other, while RMSE is sensitive to outlier MAE is not and the MAE cannot take into account the direction of the error while the RMSE can achieve it. Speedup was computed to determine how quickly our framework can model the operator $\Phi$. We utilised the \textit{Speedup} and \textit{Amortized Speedup}, which assesses the require time to approximate each operator in comparison to exhaustively executing them on all datasets (more is better). Particularly, the speedup is equalled $\frac{T{^{(i)}_{op}}}{T{^{(i)}_{SimOp} + T_{vec} + T_{sim} + T_{pred}}}$, where $T{^{(i)}_{op}}$ is the execution time for operator $i$, across all the datasets, $T{^{(i)}_{SimOp}}$ is the time needed to model the operator with the datasets selected from the similarity search, $T_{vec}$  is the time needed to compute the vector embedding for each dataset, $T_{sim}$, is the time needed to perform similarity search, and $T_{pred}$ is the time needed to predict on the dataset $D_o$. In addition to the dataset vectorisation, which is done once for each data lake, we calculate amortised speedup. Furthermore, an experimental evaluation of our proposed model for dataset vectorization NumTabData2Vec has been performed to show that our approach can transform a dataset to a vector embedding representation space $z$. For the evaluation experiments, three different NumTabData2Vec were built to project the dataset representation with vector sizes of $100$, $200$, and $300$. Each model has eight transformer layers and is trained parallel using four NVIDIA A10s GPUs, and trained for fifty epochs.
We compared our framework with Apollo \cite{b7Apollo1, b7Apollo2}, which models analytic operators using data content. Two loss functions to measure prediction accuracy are employed: root-mean-square error (RMSE) and mean absolute error (MAE). RMSE is sensitive to outliers, while MAE is not; conversely, RMSE accounts for error direction, which MAE cannot. Speedup metrics are also used to evaluate how efficiently our framework models operator $\Phi$. Specifically, \textit{Speedup} and \textit{Amortized Speedup} measure the time required to approximate each operator versus exhaustively executing them on all datasets. Speedup is defined as $\frac{T{^{(i)}_{op}}}{T{^{(i)}_{SimOp} + T_{vec} + T_{sim} + T_{pred}}}$, where $T{^{(i)}_{op}}$ is the time to execute operator $i$ on all datasets, $T{^{(i)}_{SimOp}}$ is the time to model the operator with datasets from similarity search, $T_{vec}$ is the vector embedding computation time, $T_{sim}$ is the similarity search time, and $T_{pred}$ is the prediction time for $D_o$. Amortized speedup includes dataset vectorization, performed once per data lake for multiple operators (in our case two operators).
We also evaluate our dataset vectorization model, NumTabData2Vec, which projects datasets into vector embedding space $z$. Three versions were built with vector sizes of $100$, $200$, and $300$, each featuring eight transformer layers. The models were trained for 50 epochs on four NVIDIA A10 GPUs in parallel.

\subsection{Evaluation Setup}
Our framework is deployed over an AWS EC2 virtual machine server running with 48 VCPUs of AMD EPYC 7R32 processors at 2.40GHz, and four A10s GPUs with 24GB of memory each, $192GB$ of RAM memory, and $2TB$ of storage, running over Ubuntu 24.4 LTS. Our code is written in Python (v.3.9.1) and PyTorch modules (v.2.4.0). Apollo was deployed in a virtual machine with 8 VCPUs Intel Xeon E5-2630 @ 2.30GHz, $64GB$ of RAM memory, and $250GB$ of storage, running Ubuntu 24.4 LTS like in their experimental evaluation. 

\subsection{Datasets}
\begin{table}[!ht]
    \centering
    \setlength\doublerulesep{0.5pt}
    \caption{Dataset properties for experimental evaluation}
    \label{tab:table-evaluation-datasets}
    \begin{tabular}{||c|c|c|c||}
        \hline
         \makecell{Dataset Name}& \makecell{\# Files} & \makecell{\# Tuples} & \makecell{\# Columns}\\ \hline\hline
         Household Power & & & \\
         Consumption \cite{b21HPCdataset} & $401$ & $2051$ & 7\\
         \hline
         Adult \cite{b22AdultDataset} & $100$ & $228$ & 14\\
         \hline
         Stocks \cite{b23StockMarketDataset} & $508$ & $1959 - 13$ & 7 \\
         \hline
         Weather \cite{b23WeatherDataset} & $49$ & $516$ & 7 \\ \hline
    \end{tabular}

\end{table}

We evaluated our framework using four diverse datasets to represent real-world scenarios. Table \ref{tab:table-evaluation-datasets} summarizes these datasets and their attributes. The vectorization module, NumTabData2Vec, was trained on data separate from the experimental evaluation data, split $60\%$ for training and $40\%$ for testing.
The Household Power Consumption (HPC) dataset \cite{b21HPCdataset} contains electric power usage measurements from a household in Sceaux, France. It includes $401$ datasets, each with $2051$ tuples and seven features recorded at one-minute intervals. The Adult dataset \cite{b22AdultDataset}, commonly used for binary classification, predicts whether an individual earns more or less than $50K$ annually. It comprises $100$ datasets, each with $228$ individuals and various socio-economic features.
The Stock Market dataset \cite{b23StockMarketDataset} includes daily NASDAQ stock prices obtained from Yahoo Finance, with $508$ datasets. Each dataset contains $13$ to $1959$ tuples, each describing seven feature attributes. The Weather dataset \cite{b23WeatherDataset} provides hourly weather measurements from $36$ U.S. cities between $2012$ and $2017$, split into $49$ datasets, each with $516$ tuples and seven features.


\begin{figure}[!t]
     \centering
     \begin{subfigure}[b]{0.24\textwidth}
         \centering
         \includegraphics[width=\textwidth]{Figures/Results/Sim_Search/HPC/HPC_LR_RMSE_Loss_fig.pdf}
         \caption{Linear Regression RMSE error loss}
         \label{fig:HPC-LR-RMSE}
     \end{subfigure}
     \hfill 
     \begin{subfigure}[b]{0.24\textwidth}
         \centering
         \includegraphics[width=\textwidth]{Figures/Results/Sim_Search/HPC/HPC_LR_MAE_Loss_fig.pdf}
         \caption{Linear Regression MAE error loss}
         \label{fig:HPC-LR-MAE}
     \end{subfigure}
        
     \begin{subfigure}[b]{0.24\textwidth}
         \centering
         \includegraphics[width=\textwidth]{Figures/Results/Sim_Search/HPC/HPC_MLP_RMSE_Loss_fig.pdf}
         \caption{MLP for Regression RMSE error loss}
         \label{fig:HPC-MLP-RMSE}
     \end{subfigure}
     \hfill 
     \begin{subfigure}[b]{0.24\textwidth}
         \centering
         \includegraphics[width=\textwidth]{Figures/Results/Sim_Search/HPC/HPC_MLP_MAE_Loss_fig.pdf}
         \caption{MLP for Regression MAE error loss}
         \label{fig:HPC-MLP-MAE}
     \end{subfigure}
        \caption{Household power consumption dataset prediction error loss}
        \label{fig:HPC-EVAL-RES}
\end{figure}

Our framework was evaluated by registering the accuracy of predicting the output of various ML operators over multiple datasets in $D$ without actually executing the operator on them. To evaluate our scheme and its parameters, we use all four datasets, ranging the size of the produced vectors as well as the similarity functions used.
We project all datasets into $k$-dimensional spaces with varying vector dimensions ($100$, $200$, and $300$). For each dataset in Table \ref{tab:table-evaluation-datasets}, we model different operators: For the regression datasets (Household Power Consumption and Stock Market), we model Linear Regression (LR) and Multi-Layer Perceptron (MLP) operators; for the classification datasets (Weather and Adult), we model the Support Vector Machine (SVM) and MLP classifier operators. Each experiment has been executed $10$ times and we report the average of the error loss, as well as the speedup. 

\begin{figure}[!t]
     \centering
     \begin{subfigure}[b]{0.24\textwidth}
         \centering
         \includegraphics[width=\textwidth]{Figures/Results/Sim_Search/Stocks/Stocks_LR_RMSE_Loss_fig.pdf}
         \caption{Linear Regression RMSE error loss}
         \label{fig:Stock-LR-RMSE}
     \end{subfigure}
     \hfill 
     \begin{subfigure}[b]{0.24\textwidth}
         \centering
         \includegraphics[width=\textwidth]{Figures/Results/Sim_Search/Stocks/Stocks_LR_MAE_Loss_fig.pdf}
         \caption{Linear Regression MAE error loss}
         \label{fig:Stock-LR-MAE}
     \end{subfigure}
        
     \begin{subfigure}[b]{0.24\textwidth}
         \centering
         \includegraphics[width=\textwidth]{Figures/Results/Sim_Search/Stocks/Stocks_MLP_RMSE_Loss_fig.pdf}
         \caption{MLP for Regression RMSE error loss}
         \label{fig:Stock-MLP-RMSE}
     \end{subfigure}
     \hfill 
     \begin{subfigure}[b]{0.24\textwidth}
         \centering
         \includegraphics[width=\textwidth]{Figures/Results/Sim_Search/Stocks/Stocks_MLP_MAE_Loss_fig.pdf}
         \caption{MLP for Regression MAE error loss}
         \label{fig:Stock-MLP-MAE}
     \end{subfigure}
        \caption{Stock market dataset prediction error loss}
        \label{fig:Stock-EVAL-RES}
\end{figure}
\subsection{Evaluation Results}



Figures \ref{fig:HPC-EVAL-RES}, \ref{fig:Stock-EVAL-RES}, \ref{fig:Weather-EVAL-RES}, and \ref{fig:Adult-EVAL-RES} present the evaluation results for each method, comparing the performance of different similarity search techniques across various vector embedding representation spaces. The red (with hatches), brown, and blue bars correspond to vector embeddings of size 100, 200, and 300 respectively. In each sub-figure, the y-axis represents the error loss value, while the x-axis displays the similarity search method applied over the vector embeddings. Figures \ref{fig:HPC-EVAL-RES} and \ref{fig:Stock-EVAL-RES} show the results for the Stock market and Household power consumption datasets, where the bottom sub-figure demonstrates the MLP regression model, and the top sub-figure presents the LR model. Figures \ref{fig:Weather-EVAL-RES} and \ref{fig:Adult-EVAL-RES} depict the evaluation results for the Weather and Adult datasets. In these Figures, the top sub-figure shows the SVM with SGD results, while the bottom sub-figure shows the MLP classifier. The left sub-figures in all Figures use the RMSE loss function, whereas the right sub-figures use the MAE loss function. 



Figure \ref{fig:HPC-EVAL-RES}, we show, for the HPC dataset, shows as increase the vector dimension size there is slightly lower prediction error for all the operator modelling. While for different similarity methods did not result in any significant differences in the prediction error loss for all the operator modelling. This suggests that, regardless the similarity selection method, our framework effectively selects the most optimal subset of data to improve model predictions on the unseen input dataset $D_o$. Additionally, we observe higher error loss with a vector size of 100, which can be attributed to the reduced representation capacity of lower-dimensional vectors. This limitation results in fewer ``right" datasets being selected.

For the stock market dataset, Figure \ref{fig:Stock-EVAL-RES} depicts that a vector embedding representation of size $300$ models more accurate operators, with cosine similarity performing best in the similarity search and modelling the most optimal operator. However, due to the inherent volatility in Stock market data from different days, all models in the stock market dataset experiments exhibit high loss values. 

In the weather dataset, the SVM operator results from sub-figures \ref{fig:Weather-SVM-RMSE} and \ref{fig:Weather-SVM-MAE} show that using $300$ vectors in the representation space consistently led to more accurate operator models across all similarity methods. Specifically, cosine similarity in combination with the $300$-dimensional vector embedding reduced the error rate in operator predictions, demonstrating that projecting datasets into this representation space and applying cosine similarity improves the prediction accuracy on the modelled operator. For the MLP classifier from sub-figures \ref{fig:Weather-MLP-RMSE} and \ref{fig:Weather-MLP-MAE}, the results illustrate that using vector embeddings of size $200$ and K-Means clustering produced the most accurate MLP classifier operators.

% Overall, we observe that the error loss was minimized 
% (** what do you mean, minimized? In general, here you should comment on the effect of similarity function, the effect of vector size and the effect of different operators to the accuracy of prediction. E.g., in Household dataset shows little effect in all bars, but in Stock, the cosine seems better and larger size of vectors leads to better performance etc. **)
% in most cases, indicating that our framework effectively selects the most relevant datasets from the data lake $D$, thereby improving data quality and reducing $\Phi$ prediction errors on the target dataset $D_o$. This demonstrates that the datasets are accurately transformed into the vector embedding representation space, allowing for the selection of datasets most similar to $D_o$. 

%Adult


%Weather
\begin{figure}[t!]
     \centering
     \begin{subfigure}[b]{0.24\textwidth}
         \centering
         \includegraphics[width=\textwidth]{Figures/Results/Sim_Search/Weather/Weather_SVM_RMSE_Loss_fig.pdf}
         \caption{SVM with SGD RMSE error loss}
         \label{fig:Weather-SVM-RMSE}
     \end{subfigure}
     \hfill 
     \begin{subfigure}[b]{0.24\textwidth}
         \centering
         \includegraphics[width=\textwidth]{Figures/Results/Sim_Search/Weather/Weather_SVM_MAE_Loss_fig.pdf}
         \caption{SVM with SGD MAE error loss}
         \label{fig:Weather-SVM-MAE}
     \end{subfigure}
        
     \begin{subfigure}[b]{0.24\textwidth}
         \centering
         \includegraphics[width=\textwidth]{Figures/Results/Sim_Search/Weather/Weather_MLP_RMSE_Loss_fig.pdf}
         \caption{MLP RMSE error loss}
         \label{fig:Weather-MLP-RMSE}
     \end{subfigure}
     \hfill 
     \begin{subfigure}[b]{0.24\textwidth}
         \centering
         \includegraphics[width=\textwidth]{Figures/Results/Sim_Search/Weather/Weather_MLP_MAE_Loss_fig.pdf}
         \caption{MLP MAE error loss}
         \label{fig:Weather-MLP-MAE}
     \end{subfigure}
        \caption{Weather dataset prediction error loss}
        \label{fig:Weather-EVAL-RES}
\end{figure}

On the other hand, the Adult dataset shows the lowest error rates, with error loss values consistently below $0.5$ across all vector embedding dimensions and similarity search methods (see Figure \ref{fig:Adult-EVAL-RES}). The Adult dataset, besides exhibiting a high number of rows, also has a higher number of columns, which demonstrates that our framework performs consistently well even with larger datasets.
Additionally, we observe that the lowest prediction error across all datasets occurs when using higher-dimensional vector embeddings. With a trade-off between accuracy and execution time as the difference to generate all data lake available datasets vector embedding representation between $100$ and $300$ size dimension in the vector representation space to be less than $60$ seconds. This confirms that a higher number of vector dimensions leads to more accurate predictions, consistent with findings in previous research \cite{b8Word2Vec}.


\begin{figure}[!t]
     \centering
     \begin{subfigure}[b]{0.24\textwidth}
         \centering
         \includegraphics[width=\textwidth]{Figures/Results/Sim_Search/Adult/Adult_MLP_RMSE_Loss_fig.pdf}
         \caption{SVM with SGD RMSE error loss}
         \label{fig:Adult-LR-RMSE}
     \end{subfigure}
     \hfill 
     \begin{subfigure}[b]{0.24\textwidth}
         \centering
         \includegraphics[width=\textwidth]{Figures/Results/Sim_Search/Adult/Adult_MLP_RMSE_Loss_fig.pdf}
         \caption{SVM with SGD MAE error loss}
         \label{fig:Adult-LR-MAE}
     \end{subfigure}
     
     \begin{subfigure}[b]{0.24\textwidth}
         \centering
         \includegraphics[width=\textwidth]{Figures/Results/Sim_Search/Adult/Adult_MLP_RMSE_Loss_fig.pdf}
         \caption{MLP RMSE error loss}
         \label{fig:Adult-MLP-RMSE}
     \end{subfigure}
     \hfill 
     \begin{subfigure}[b]{0.24\textwidth}
         \centering
         \includegraphics[width=\textwidth]{Figures/Results/Sim_Search/Adult/Adult_MLP_MAE_Loss_fig.pdf}
         \caption{MLP MAE error loss}
         \label{fig:Adult-MLP-MAE}
     \end{subfigure}
        \caption{Adult dataset prediction error loss}
        \label{fig:Adult-EVAL-RES}
\end{figure}





We conducted an experimental evaluation using the Sampling Ratio (SR) approach, similar to Apollo \cite{b7Apollo1}, but employed neural networks built from the vector embeddings of each dataset. The SR approach involves a unified random selection of $l\%$ datasets from the vector representation space, using this subset to construct a neural network for predicting operator outputs. We tested SR values of $0.1$, $0.2$, and $0.4$, as well as vector embedding dimensions of $100$, $200$, and $300$, across all datasets. 
Figure \ref{fig:SR-EVAL-RES} presents the sampling ratio results for the Adult dataset using MLP (sub-figure \ref{fig:Adult-SR-RMSE}) and for the Weather dataset using LR (sub-figure \ref{fig:Weather-SR-SVM-MAE}). In each sub-figure the y-axis represents the RMSE prediction error loss while the x-axis denotes the vector dimension



\begin{figure}[htpb!]
     \centering
     \begin{subfigure}[b]{0.24\textwidth}
         \centering
         \includegraphics[width=\textwidth]{Figures/Results/SR/Adult/Adult_MLP_SR_RMSE_Loss_fig.pdf}
         \caption{Adult Dataset MLP Operator RMSE error loss}
         \label{fig:Adult-SR-RMSE}
     \end{subfigure}
     \hfill 
     \begin{subfigure}[b]{0.24\textwidth}
         \centering
         \includegraphics[width=\textwidth]{Figures/Results/SR/HPC/HPC_LR_SR_RMSE_Loss_fig.pdf}
         \caption{HPC dataset LR Operator RMSE error loss}
         \label{fig:Weather-SR-SVM-MAE}
     \end{subfigure}
     \caption{Sampling Ratio prediction results}
        \label{fig:SR-EVAL-RES}
\end{figure}

Both experiments demonstrate that as the vector embedding dimension increases, coupled with a larger sampling ratio (SR) value, there is a slight decrease in the prediction error loss. This improvement occurs because higher-dimensional vector embeddings provide a more accurate representation of the datasets in k-dimensions, with better dataset selection leading to enhanced prediction accuracy. Comparing the SR approach to our similarity search method for the HPC dataset, the SR approach was approximately $15\%$ less accurate in operator prediction across all vector embedding dimensions. A similar trend was observed in the Weather dataset. However, the Stock dataset exhibited a much larger discrepancy, with the SR approach performing about $70\%$ worse in prediction accuracy across all vector embedding dimensions. Likewise, in the Adult dataset, the SR approach delivered the poorest performance, with nearly $90\%$ lower prediction accuracy compared to the similarity search methods.

\begin{table*}[htbp]
    \centering
        \caption{Evaluation results of our framework exported analytic operator with lowest prediction error in comparison with Apollo}
    \label{tab:table-eval-res}
    % \scalebox{0.8}{
    \setlength\doublerulesep{0.5pt}
    % \begin{adjustbox}{width=\linewidth,center}
    \begin{tabular}{|c|c|c|c|c|c|c|}
    \hline
         \makecell{Dataset\\Name} & Method & Operator & RMSE &  MAE & Speedup  & Amortized Speedup \\
         \hline\hline
         \multirow{7}{*}{\makecell{Household\\Power\\Consumption}}& \makecell{$300$V Cosine} & LR & $\mathbf{6.61}$ & $\mathbf{5.42}$ & $0.0017$ & $\mathbf{1.99}$ \\ \cline{2-7}
                  & \makecell{$300$V SR-$0.2$} & LR & $7.77$ & $6.66$ &  $0.0018$  & $1.42$\\ \cline{2-7} 
        & \makecell{Apollo-SR $0.1$} & LR & $2968.01$ &  $2352.55$ & $\mathbf{0.015}$ & $0.024$ \\ \cline{2-7}
         & \makecell{Apollo-SR $0.2$} & LR & $2811.49$ &  $2229.50$ & $0.015$ & $0.024$ \\ \cline{2-7}\cline{2-7}
         & \makecell{$300$V K-Means} & MLP Regr. & $\mathbf{6.70}$ & $\mathbf{3.38}$ &  $0.9249$  & $\mathbf{1.99}$\\ \cline{2-7}
         & \makecell{Apollo-SR $0.1$} & MLP Regr. & $3322.05$ &  $2606.99$ & $2.38$ & $1.74$ \\ \cline{2-7}
         & \makecell{Apollo-SR $0.2$} & MLP Regr. & $3850.01$ &  $2609.36$ & $\mathbf{2.38}$ & $1.74$\\ \cline{1-7} \cline{1-7} 
         % Stock
         % \multirow{5}{*}{\makecell{Stock}}& \multirow{1}{*}{ \makecell{$100$V Euclidean}} & LR & $229388.93$ & $193066.03$ \\ \cline{2-5}
        \multirow{7}{*}{\makecell{Stock}} &  \makecell{$300$V Cosine} & LR & $306382.28$ & $125335.65$ & $0.00085$ & $\mathbf{1.91}$\\ \cline{2-7}
        & \makecell{$300$V SR-$0.4$} & LR & $21861625.91$ & $5674215.265$ &  $0.00087$  & $0.33$\\ \cline{2-7}
        & \makecell{Apollo-SR $0.1$} & LR & $\mathbf{153665.92}$ &  $\mathbf{118236.48}$ & $\mathbf{0.00093}$ & $0.00096$\\ \cline{2-7}
         & \makecell{Apollo-SR $0.2$} & LR & $166844.95$ &  $133306.68$ & $0.00093$ & $0.00096$\\ \cline{2-7}\cline{2-7}
         &  \makecell{$300$V Cosine} & MLP Regr. & $\mathbf{140236.47}$ & $\mathbf{123571.12}$ & $0.63$ & $\mathbf{1.91}$\\ \cline{2-7}
         & \makecell{Apollo-SR $0.1$} & MLP Regr. &  $175150.82$ &  $145123.09$ & $\mathbf{0.93}$ & $0.96$\\ \cline{2-7}
         & \makecell{Apollo-SR $0.2$} & MLP Regr. & $174390.81$ &  $146338.73$ & $0.93$ & $0.96$\\ \cline{1-7} \cline{1-7}
         % Weather
         \multirow{7}{*}{\makecell{Weather}}& \multirow{1}{*}{ \makecell{$300$V Cosine}} & \makecell{SVM SGD}& $\mathbf{14.13}$ & $\mathbf{7.63}$ & $1.06$ & $\mathbf{22.8}$ \\ \cline{2-7}
               & \makecell{Apollo-SR $0.1$} & SVM & $69.51$ &  $25.52$ & $\mathbf{2.10}$ &  $1.16$\\ \cline{2-7}
                        & \makecell{Apollo-SR $0.2$} & SVM & $68.70$ &  $22.81$ & $2.10$ & $1.16$\\ \cline{2-7} \cline{2-7}
       &  \multirow{1}{*}{ \makecell{$200$V Cosine}}& MLP & $\mathbf{14.29}$ & $\mathbf{4.03}$ & $1.03$  & $\mathbf{22.8}$\\ \cline{2-7}
        &  \multirow{1}{*}{ \makecell{$200$V SR-$0.4$}}& MLP & $15.95$ & $13.31$ & $1.02$  & $1.77$\\ \cline{2-7}
         & \makecell{Apollo-SR $0.1$} & MLP & $69.62$ &  $23.10$ & $\mathbf{1.34}$ & $1.14$ \\ \cline{2-7}
         & \makecell{Apollo-SR $0.2$} & MLP & $673.56$ &  $\mathbf{84.70}$ & $1.32$ & $1.14$\\ \cline{1-7} \cline{1-7}
         
         % Adult
         \multirow{7}{*}{\makecell{Adult}}& \multirow{1}{*}{ \makecell{$300$V Cosine}} & \makecell{SVM SGD}& $\mathbf{0.36}$ & $\mathbf{0.2}$ & $0.37$   & $\mathbf{2.78}$\\ \cline{2-7}
                  & \makecell{Apollo-SR $0.1$} & SVM & $68.32$ &  $22.95$ & $\mathbf{0.75}$ & $0.85$ \\ \cline{2-7}
                 & \makecell{Apollo-SR $0.2$} & SVM & $68.88$ &  $22.88$ & $0.74$ & $0.85$\\ \cline{2-7} \cline{2-7}

         &  \multirow{1}{*}{ \makecell{$300$V K-Means}}& MLP & $\mathbf{0.36}$ & $\mathbf{0.19}$ & $0.30$ & $2.78$ \\ \cline{2-7}
        & \makecell{$300$V SR-$0.2$} & MLP & $6.01$ & $6.00$ &  $0.54$  & $\mathbf{3.54}$\\ \cline{2-7}
         & \makecell{Apollo-SR $0.1$} & MLP & $71.11$ &  $26.51$ & $\mathbf{1.07}$ & $1.31$\\ \cline{2-7}
         & \makecell{Apollo-SR $0.2$} & MLP & $70.16$ &  $25.74$ & $1.05$ & $1.31$\\ \cline{1-7}
         
    \end{tabular}
    % }
\end{table*}

% Table \ref{tab:table-eval-res} illustrates the model operators for each dataset and each loss function, amortized speedup and speedup from our framework in comparison with the same model operators from the Apollo \cite{b7Apollo1, b7Apollo2} framework with SR of $0.1$ and $0.2$. The values $100$V, $200$V, and $300$V in the method column correspond to the dimensions of the vector embedding used for each dataset. The lowest prediction error for each modelled operator in each dataset is highlighted in the method that is used in the similarity search step from our pipeline. Apollo outperforms our framework only on the stock dataset for SR equal with $0.1$ in the LR analytic operator for both RMSE and MAE loss function which performs $50\%$ and $6\%$ better on each loss function equivalent. While our framework for the MLP for Regression outperforms the Apollo modelled operator for $20\%$ and $84\%$ for RMSE and MAE loss functions. However, this difference in the Stock dataset for LR operator modelling is not significant. In the remaining datasets, our framework illustrates that it can outperform Apollo for different values of SR. This makes us confirm that our similarity search using similarity functions selects the most similar datasets $D_r$ from data lake directory $D$, increasing data quality and minimising $\Phi$ prediction errors on the dataset $D_o$. For the Adult dataset, our model operators also perform better, which indicates our method's advantage with an increased number of dataset features (columns). In term of speedup we can see that Apollo outperformed our framework of all modelled operators. In terms of speedup we can see that Apollo outperformed our framework of all modelled operators. This is due to the vectorisation method of our framework which consists of big complexity time. Furthermore, in amortized speedup in most of the amortized speedup in which the vectorization is not counted because it is executed only one time and can be reused our framework surpasses Apollo framework in most of the operators with a big difference with our framework to be between $10\%$ and $60\%$ faster than Apollo. Additionally, most datasets demonstrate better amortized speedup when using the SR approach within our framework. This is because the prediction process relies solely on the vector representation, rather than leveraging all dataset tuples as done in the similarity search method for operator modelling. However, in terms of prediction accuracy, the SR approach does not perform as well as the similarity search method, which achieves superior results.

Table \ref{tab:table-eval-res} compares model operators, loss functions, and speedup metrics for our framework and Apollo at SR values of $0.1$ and $0.2$. Methods $100$V, $200$V, and $300$V denote vector embedding dimensions. The lowest prediction errors align with our pipeline's similarity search method.
Apollo outperforms our framework on the Stock dataset for the LR analytic operator at SR equals with $0.1$ (with $50\%$ and $6\%$ improvements for RMSE and MAE, respectively). However, our framework excels with the MLP regression operator, improving RMSE and MAE by $20\%$ and $17\%$, respectively. The LR operator's performance gap on the Stock dataset is minor.
For other datasets, our framework consistently surpasses Apollo across different SR values. This demonstrates the effectiveness of our similarity search approach, which enhances data quality and reduces $\Phi$ prediction errors by identifying relevant datasets $D_r$ from the data lake directory $D$. The Adult dataset also highlights our framework's advantage with increasing feature dimensions.
Although Apollo achieves better raw speedup due to the higher complexity of our framework's vectorization step, our framework outperforms it in amortized speedup. By excluding the reusable vectorization process, it achieves speed gains of $10\%$ to $60\%$ for most operators.
The SR approach, leveraging vector embedding representations, enhances operator prediction compared to Apollo and achieves greater amortized speedup. However, the similarity search method outperforms both Apollo and the SR approach in prediction accuracy and amortized speedup, establishing its clear superiority across most datasets and operator scenarios.

\subsection{NumTabData2Vec Evaluation Results}

\begin{figure}[!ht]
    \centering
    \includegraphics[width=0.4\textwidth]{Figures/Results/Representation/V200_representation.pdf}
    \caption{Vector representation for each dataset from NumTabData2Vec}
    \label{fig:eval-data-repr}
\end{figure}


\begin{table}[!htp]
    \centering
    \caption{Similarity between vectors of different datasets scenarios}
    \label{tab:vec-rep-sim}
    \setlength\doublerulesep{0.5pt}
    \begin{tabular}{||c|c||}
    \hline
    Model Name & Similarity \\
    \hline\hline
     \makecell{NumTabData2Vec\\$100$ Vector size} & $0.54$\\
     \hline
      \makecell{NumTabData2Vec\\$200$ Vector size}   & $0.18$\\
      \hline
       \makecell{NumTabData2Vec\\$300$ Vector size}  & $0.16$\\ \hline
    \end{tabular}
\end{table}

% Our proposed model, \textit{NumTabData2Vec}, for dataset vectorization is compared between all the available dataset scenarios to determine whether it can effectively distinguish between them based on qualitative differences. The comparison involves selecting $n$ random datasets for each detaset scenario and projecting them into their respective vector embedding representations. Then for each dataset scenario, it gains the average vector embedding representation by the average vector embedding representation of the $n$ random datasets. The vector embedding representation for each dataset scenario depicted in Figure \ref{fig:eval-data-repr} in from the $k$-dimensional space (size of $200$) transformed to the 3d space using the PCA. Figure \ref{fig:eval-data-repr} demonstrates that each dataset occupies a distinct dimension, with non-overlapping or clustering closely together. This indicates that \textit{NumTabData2Vec} can identify the datasets from various situations and does not have a close representation like previous methods achieved it with the same accuracy but on different data types (such as word, and graphs) \cite{b8Word2Vec, b9Graph2Vec} and not in an entire dataset. Table \ref{tab:vec-rep-sim}, further illustrates the average cosine similarity between the vector embeddings of all datasets, demonstrating how dissimilar are the datasets in their vector representation. As the size dimension of the vector embedding representation increases, the model's ability to distinguish across datasets improves as their average similarity decreases. Furthermore, this indicates that larger vector dimension sizes are unneeded since between $100$ and $300$ is sufficient.

Our proposed model, \textit{NumTabData2Vec}, was evaluated to determine its ability to distinguish dataset scenarios based on qualitative differences. For each scenario, $n$ random datasets were selected, and their vector embeddings averaged to represent the scenario. These embeddings, initially in a 200-dimensional space, were projected into 3D using PCA and are shown in Figure \ref{fig:eval-data-repr}. The figure illustrates that each dataset scenario occupies a distinct space, with minimal overlap or clustering. This demonstrates that \textit{NumTabData2Vec} effectively distinguishes datasets, outperforming prior methods like Word2Vec and Graph2Vec \cite{b8Word2Vec, b9Graph2Vec}, which achieved similar accuracy but on different data types (e.g., words, graphs) rather than entire datasets. Table \ref{tab:vec-rep-sim} further highlights the average cosine similarity between dataset embeddings, showing greater dissimilarity as vector dimensions increase. However, results suggest that dimensions between $100$ and $300$ are sufficient for accurate distinction, avoiding the need for larger vector sizes.

\begin{figure}[!ht]
    \centering
    \includegraphics[width=0.4\textwidth]{Figures/Results/Representation/plot_representation_200Vectors.pdf}
    \caption{Synthetic data vector embedding representation}
    \label{fig:eval-sd-data-repr}
\end{figure}

To evaluate \textit{NumTabData2Vec}'s ability to distinguish datasets with varying row and column counts, we generated synthetic numerical tabular datasets of different dimensions and vectorized them. Figure \ref{fig:eval-sd-data-repr} shows datasets with columns ranging from three to thirty and rows from ten to one thousand, projected from a $200$-dimensional space to 2D using PCA. Each bullet caption c and r corresponds to the columns and rows of the dataset, respectively. Datasets with the same number of columns cluster closely in the representation space, and a similar pattern is observed for datasets with the same number of rows. These results indicate that our method effectively distinguishes datasets based on size during vectorization.

\begin{table}[!htp]
    \centering
    \caption{NumTabData2Vec execution time for different dataset dimensions and different vector sizes }

    \begin{adjustbox}{width=\columnwidth,center}
    \label{tab:vec-exec-time}
    \setlength\doublerulesep{0.5pt}
    \begin{tabular}{||c|c|c|c|c||}
    \hline
     \makecell{\# of columns} & \makecell{\# of rows} & \makecell{$50$ Vectors\\Execution time} & \makecell{$100$ Vectors\\Execution time} & \makecell{$200$ Vectors\\Execution time} \\
    \hline\hline
     $3$ & $100$ & $0.0004$ sec & $0.00042$ sec & $0.00051$ sec\\ \hline
     $3$ & $500$ & $0.0004$ sec & $0.00041$ sec & $0.00049$ sec\\ \hline
     $3$ & $1000$ & $0.0004$ sec & $0.00041$ sec & $0.00049$ sec\\ \hline
     $3$ & $1500$ & $0.0004$ sec & $0.00041$ sec & $0.00055$ sec\\ \hline
     $3$ & $1800$ & $0.0004$ sec & $0.00041$ sec & $0.00055$ sec\\ \hline
     \hline
     $10$ & $100$ & $0.0004$ sec & $0.0004$ sec & $0.00057$ sec\\ \hline
     $10$ & $500$ & $0.00039$ sec & $0.0004$ sec & $0.00051$ sec\\ \hline
     $10$ & $1000$ & $0.00041$ sec & $0.00042$ sec & $0.00052$ sec\\ \hline
     $10$ & $1500$ & $0.00041$ sec & $0.00042$ sec & $0.00055$ sec\\ \hline
     $10$ & $1800$ & $0.00041$ sec & $0.00042$ sec & $0.00052$ sec\\ \hline
     \hline
     $20$ & $100$ & $0.0004$ sec & $0.00042$ sec & $0.0005$ sec\\ \hline
     $20$ & $500$ & $0.0004$ sec & $0.00042$ sec & $0.0005$ sec\\ \hline
     $20$ & $1000$ & $0.00042$ sec & $0.00043$ sec & $0.00052$ sec\\ \hline
     $20$ & $1500$ & $0.00043$ sec & $0.00044$ sec & $0.00054$ sec\\ \hline
     $20$ & $1800$ & $0.00044$ sec & $0.00044$ sec & $0.00054$ sec\\ \hline    
     \hline\hline
    \end{tabular}
    \end{adjustbox}
\end{table}

To evaluate how dataset dimensions affect the execution time of \textit{NumTabData2Vec}, we created synthetic datasets with varying numbers of rows ($100$, $500$, $1000$, $1500$, and $1800$) and columns ($3$, $10$, and $20$). These datasets were vectorized into different dimensions, and the execution times were recorded. Table \ref{tab:vec-exec-time} shows that increasing the k-dimension requires approximately $20\%$ more time to generate the vector embeddings. This is expected, as a higher k-dimension involves more hyperparameters, which naturally increases computation time.

Interestingly, varying the number of columns did not significantly impact execution time. However, increasing the number of rows resulted in approximately $5\%$ additional execution time. This is because larger datasets require the extraction of more features, which has a modest impact on the model's execution time.

\begin{figure}[!ht]
    \centering
    \includegraphics[width=0.4\textwidth]{Figures/Results/Representation/plot_representation_noise_data_200Vectors.pdf}
    \caption{HPC Dataset vector embedding representation with addition of Noise}
    \label{fig:eval-nd-data-repr}
\end{figure}

To evaluate \textit{NumTabData2Vec}'s ability to distinguish datasets based on different properties like distribution and order, we introduced Gaussian noise to random $l\%$ of data tuples in an HPC dataset. Figure \ref{fig:eval-nd-data-repr} visualises the original and noise-modified datasets, projected from a 200-dimensional space to 2D using PCA. Each bullet caption g denotes the percentage of Gaussian noise added in the dataset. As noise increases, the representation space shifts further from the original dataset, indicating that \textit{NumTabData2Vec} effectively captures distribution differences. Additionally, since the HPC dataset has an inherent order, the model's sensitivity to noise demonstrates its ability to distinguish datasets based on ordering as well.

\begin{figure}[!ht]
    \centering
    \includegraphics[width=0.4\textwidth]{Figures/Results/Representation/plotrepresentationnoisedata1col200Vectors.pdf}
    \caption{HPC Dataset vector embedding representation with addition of Noise in the first column}
    \label{fig:eval-nd-data-repr-1col}
\end{figure}

To evaluate how fine-grained as distinction can be, we introduced noise into a single column and repeated the previous experiment, with the difference being that noise was added exclusively to the first column. Figure \ref{fig:eval-nd-data-repr-1col} visualizes the dataset's 2D vector space. The amount of Gaussian noise added to the dataset's first column is indicated by g in the bullet caption. The results show that as more noise is introduced to the column, the vector representation moves further away from the original dataset. In contrast to the previous experiment shown in Figure \ref{fig:eval-nd-data-repr}, the noisy dataset's representation stays closest to the original when only a single column is modified. Also in this experiment the dataset points in the 2-dimension are more grouped between them instead the previous experiment. 

%closely grouped compared to the previous experiment.
\section{Threats To Validity}\label{sec:ttv}
Our SLR aims to be comprehensive, but some limitations should be acknowledged. While we searched popular repositories (IEEE Xplore, ACM Digital Library, Springer, and ScienceDirect) and employed both backward and forward snowballing techniques on recent publications, the possibility of unintentional inclusion or exclusion of relevant studies remains. The authors carefully evaluated publications that fell on the borderline of inclusion/exclusion criteria to mitigate this risk.

Furthermore, our SLR focused exclusively on peer-reviewed journal articles and conference publications published in English. This decision was made to streamline the review process and ensure a high standard of research quality. However, it is important to acknowledge that relevant information may exist in other sources, such as books, theses, and non-English publications, which were not included in this review.

In addition, we deliberately excluded publications that primarily addressed network security, architecture, or systems that utilized SDN as a component. We aimed to maintain a focused review on the software security aspects of SDN itself. This means that studies analyzing SDN's role in broader contexts, such as cloud security or Internet of Things (IoT) networks, were not included. While this approach ensured a clear research focus, it may have overlooked valuable insights on the broader implications of SDN software security.

Overall, while we believe our SLR provides a comprehensive overview of the current state of research on SDN software security, readers should be aware of these limitations when interpreting our findings.
\section{ Task Generalization Beyond i.i.d. Sampling and Parity Functions
}\label{sec:Discussion}
% Discussion: From Theory to Beyond
% \misha{what is beyond?}
% \amir{we mean two things: in the first subsection beyond i.i.d subsampling of parity tasks and in the second subsection beyond parity task}
% \misha{it has to be beyond something, otherwise it is not clear what it is about} \hz{this is suggested by GPT..., maybe can be interpreted as from theory to beyond theory. We can do explicit like Discussion: Beyond i.i.d. task sampling and the Parity Task}
% \misha{ why is "discussion" in the title?}\amir{Because it is a discussion, it is not like separate concrete explnation about why these thing happens or when they happen, they just discuss some interesting scenraios how it relates to our theory.   } \misha{it is not really a discussion -- there is a bunch of experiments}

In this section, we extend our experiments beyond i.i.d. task sampling and parity functions. We show an adversarial example where biased task selection substantially hinders task generalization for sparse parity problem. In addition, we demonstrate that exponential task scaling extends to a non-parity tasks including arithmetic and multi-step language translation.

% In this section, we extend our experiments beyond i.i.d. task sampling and parity functions. On the one hand, we find that biased task selection can significantly degrade task generalization; on the other hand, we show that exponential task scaling generalizes to broader scenarios.
% \misha{we should add a sentence or two giving more detail}


% 1. beyond i.i.d tasks sampling
% 2. beyond parity -> language, arithmetic -> task dependency + implicit bias of transformer (cannot implement this algorithm for arithmatic)



% In this section, we emphasize the challenge of quantifying the level of out-of-distribution (OOD) differences between training tasks and testing tasks, even for a simple parity task. To illustrate this, we present two scenarios where tasks differ between training and testing. For each scenario, we invite the reader to assess, before examining the experimental results, which cases might appear “more” OOD. All scenarios consider \( d = 10 \). \kaiyue{this sentence should be put into 5.1}






% for parity problem




% \begin{table*}[th!]
%     \centering
%     \caption{Generalization Results for Scenarios 1 and 2 for $d=10$.}
%     \begin{tabular}{|c|c|c|c|}
%         \hline
%         \textbf{Scenario} & \textbf{Type/Variation} & \textbf{Coordinates} & \textbf{Generalization accuracy} \\
%         \hline
%         \multirow{3}{*}{Generalization with Missing Pair} & Type 1 & \( c_1 = 4, c_2 = 6 \) & 47.8\%\\ 
%         & Type 2 & \( c_1 = 4, c_2 = 6 \) & 96.1\%\\ 
%         & Type 3 & \( c_1 = 4, c_2 = 6 \) & 99.5\%\\ 
%         \hline
%         \multirow{3}{*}{Generalization with Missing Pair} & Type 1 &  \( c_1 = 8, c_2 = 9 \) & 40.4\%\\ 
%         & Type 2 & \( c_1 = 8, c_2 = 9 \) & 84.6\% \\ 
%         & Type 3 & \( c_1 = 8, c_2 = 9 \) & 99.1\%\\ 
%         \hline
%         \multirow{1}{*}{Generalization with Missing Coordinate} & --- & \( c_1 = 5 \) & 45.6\% \\ 
%         \hline
%     \end{tabular}
%     \label{tab:generalization_results}
% \end{table*}

\subsection{Task Generalization Beyond i.i.d. Task Sampling }\label{sec: Experiment beyond iid sampling}

% \begin{table*}[ht!]
%     \centering
%     \caption{Generalization Results for Scenarios 1 and 2 for $d=10, k=3$.}
%     \begin{tabular}{|c|c|c|}
%         \hline
%         \textbf{Scenario}  & \textbf{Tasks excluded from training} & \textbf{Generalization accuracy} \\
%         \hline
%         \multirow{1}{*}{Generalization with Missing Pair}
%         & $\{4,6\} \subseteq \{s_1, s_2, s_3\}$ & 96.2\%\\ 
%         \hline
%         \multirow{1}{*}{Generalization with Missing Coordinate}
%         & \( s_2 = 5 \) & 45.6\% \\ 
%         \hline
%     \end{tabular}
%     \label{tab:generalization_results}
% \end{table*}




In previous sections, we focused on \textit{i.i.d. settings}, where the set of training tasks $\mathcal{F}_{train}$ were sampled uniformly at random from the entire class $\mathcal{F}$. Here, we explore scenarios that deliberately break this uniformity to examine the effect of task selection on out-of-distribution (OOD) generalization.\\

\textit{How does the selection of training tasks influence a model’s ability to generalize to unseen tasks? Can we predict which setups are more prone to failure?}\\

\noindent To investigate this, we consider two cases parity problems with \( d = 10 \) and \( k = 3 \), where each task is represented by its tuple of secret indices \( (s_1, s_2, s_3) \):

\begin{enumerate}[leftmargin=0.4 cm]
    \item \textbf{Generalization with a Missing Coordinate.} In this setup, we exclude all training tasks where the second coordinate takes the value \( s_2 = 5 \), such as \( (1,5,7) \). At test time, we evaluate whether the model can generalize to unseen tasks where \( s_2 = 5 \) appears.
    \item \textbf{Generalization with Missing Pair.} Here, we remove all training tasks that contain both \( 4 \) \textit{and} \( 6 \) in the tuple \( (s_1, s_2, s_3) \), such as \( (2,4,6) \) and \( (4,5,6) \). At test time, we assess whether the model can generalize to tasks where both \( 4 \) and \( 6 \) appear together.
\end{enumerate}

% \textbf{Before proceeding, consider the following question:} 
\noindent \textbf{If you had to guess.} Which scenario is more challenging for generalization to unseen tasks? We provide the experimental result in Table~\ref{tab:generalization_results}.

 % while the model struggles for one of them while as it generalizes almost perfectly in the other one. 

% in the first scenario, it generalizes almost perfectly in the second. This highlights how exposure to partial task structures can enhance generalization, even when certain combinations are entirely absent from the training set. 

In the first scenario, despite being trained on all tasks except those where \( s_2 = 5 \), which is of size $O(\d^T)$, the model struggles to generalize to these excluded cases, with prediction at chance level. This is intriguing as one may expect model to generalize across position. The failure  suggests that positional diversity plays a crucial role in the task generalization of Transformers. 

In contrast, in the second scenario, though the model has never seen tasks with both \( 4 \) \textit{and} \( 6 \) together, it has encountered individual instances where \( 4 \) appears in the second position (e.g., \( (1,4,5) \)) or where \( 6 \) appears in the third position (e.g., \( (2,3,6) \)). This exposure appears to facilitate generalization to test cases where both \( 4 \) \textit{and} \( 6 \) are present. 



\begin{table*}[t!]
    \centering
    \caption{Generalization Results for Scenarios 1 and 2 for $d=10, k=3$.}
    \resizebox{\textwidth}{!}{  % Scale to full width
        \begin{tabular}{|c|c|c|}
            \hline
            \textbf{Scenario}  & \textbf{Tasks excluded from training} & \textbf{Generalization accuracy} \\
            \hline
            Generalization with Missing Pair & $\{4,6\} \subseteq \{s_1, s_2, s_3\}$ & 96.2\%\\ 
            \hline
            Generalization with Missing Coordinate & \( s_2 = 5 \) & 45.6\% \\ 
            \hline
        \end{tabular}
    }
    \label{tab:generalization_results}
\end{table*}

As a result, when the training tasks are not i.i.d, an adversarial selection such as exclusion of specific positional configurations may lead to failure to unseen task generalization even though the size of $\mathcal{F}_{train}$ is exponentially large. 


% \paragraph{\textbf{Key Takeaways}}
% \begin{itemize}
%     \item Out-of-distribution generalization in the parity problem is highly sensitive to the diversity and positional coverage of training tasks.
%     \item Adversarial exclusion of specific pairs or positional configurations can lead to systematic failures, even when most tasks are observed during training.
% \end{itemize}




%################ previous veriosn down
% \textit{How does the choice of training tasks affect the ability of a model to generalize to unseen tasks? Can we predict which setups are likely to lead to failure?}

% To explore these questions, we crafted specific training and test task splits to investigate what makes one setup appear “more” OOD than another.

% \paragraph{Generalization with Missing Pair.}

% Imagine we have tasks constructed from subsets of \(k=3\) elements out of a larger set of \(d\) coordinates. What happens if certain pairs of coordinates are adversarially excluded during training? For example, suppose \(d=5\) and two specific coordinates, \(c_1 = 1\) and \(c_2 = 2\), are excluded. The remaining tasks are formed from subsets of the other coordinates. How would a model perform when tested on tasks involving the excluded pair \( (c_1, c_2) \)? 

% To probe this, we devised three variations in how training tasks are constructed:
%     \begin{enumerate}
%         \item \textbf{Type 1:} The training set includes all tasks except those containing both \( c_1 = 1 \) and \( c_2 = 2 \). 
%         For this example, the training set includes only $\{(3,4,5)\}$. The test set consists of all tasks containing the rest of tuples.

%         \item \textbf{Type 2:} Similar to Type 1, but the training set additionally includes half of the tasks containing either \( c_1 = 1 \) \textit{or} \( c_2 = 2 \) (but not both). 
%         For the example, the training set includes all tasks from Type 1 and adds tasks like \(\{(1, 3, 4), (2, 3, 5)\}\) (half of those containing \( c_1 = 1 \) or \( c_2 = 2 \)).

%         \item \textbf{Type 3:} Similar to Type 2, but the training set also includes half of the tasks containing both \( c_1 = 1 \) \textit{and} \( c_2 = 2 \). 
%         For the example, the training set includes all tasks from Type 2 and adds, for instance, \(\{(1, 2, 5)\}\) (half of the tasks containing both \( c_1 \) and \( c_2 \)).
%     \end{enumerate}

% By systematically increasing the diversity of training tasks in a controlled way, while ensuring no overlap between training and test configurations, we observe an improvement in OOD generalization. 

% % \textit{However, the question is this improvement similar across all coordinate pairs, or does it depend on the specific choices of \(c_1\) and \(c_2\) in the tasks?} 

% \textbf{Before proceeding, consider the following question:} Is the observed improvement consistent across all coordinate pairs, or does it depend on the specific choices of \(c_1\) and \(c_2\) in the tasks? 

% For instance, consider two cases for \(d = 10, k = 3\): (i) \(c_1 = 4, c_2 = 6\) and (ii) \(c_1 = 8, c_2 = 9\). Would you expect similar OOD generalization behavior for these two cases across the three training setups we discussed?



% \paragraph{Answer to the Question.} for both cases of \( c_1, c_2 \), we observe that generalization fails in Type 1, suggesting that the position of the tasks the model has been trained on significantly impacts its generalization capability. For Type 2, we find that \( c_1 = 4, c_2 = 6 \) performs significantly better than \( c_1 = 8, c_2 = 9 \). 

% Upon examining the tasks where the transformer fails for \( c_1 = 8, c_2 = 9 \), we see that the model only fails at tasks of the form \((*, 8, 9)\) while perfectly generalizing to the rest. This indicates that the model has never encountered the value \( 8 \) in the second position during training, which likely explains its failure to generalize. In contrast, for \( c_1 = 4, c_2 = 6 \), while the model has not seen tasks of the form \((*, 4, 6)\), it has encountered tasks where \( 4 \) appears in the second position, such as \((1, 4, 5)\), and tasks where \( 6 \) appears in the third position, such as \((2, 3, 6)\). This difference may explain why the model generalizes almost perfectly in Type 2 for \( c_1 = 4, c_2 = 6 \), but not for \( c_1 = 8, c_2 = 9 \).



% \paragraph{Generalization with Missing Coordinates.}
% Next, we investigate whether a model can generalize to tasks where a specific coordinate appears in an unseen position during training. For instance, consider \( c_1 = 5 \), and exclude all tasks where \( c_1 \) appears in the second position. Despite being trained on all other tasks, the model fails to generalize to these excluded cases, highlighting the importance of positional diversity in training tasks.



% \paragraph{Key Takeaways.}
% \begin{itemize}
%     \item OOD generalization depends heavily on the diversity and positional coverage of training tasks for the parity problem.
%     \item adversarial exclusion of specific pairs or positional configurations in the parity problem can lead to failure, even when the majority of tasks are observed during training.
% \end{itemize}


%################ previous veriosn up

% \paragraph{Key Takeaways} These findings highlight the complexity of OOD generalization, even in seemingly simple tasks like parity. They also underscore the importance of task design: the diversity of training tasks can significantly influence a model’s ability to generalize to unseen tasks. By better understanding these dynamics, we can design more robust training regimes that foster generalization across a wider range of scenarios.


% #############


% Upon examining the tasks where the transformer fails for \( c_1 = 8, c_2 = 9 \), we see that the model only fails at tasks of the form \((*, 8, 9)\) while perfectly generalizing to the rest. This indicates that the model has never encountered the value \( 8 \) in the second position during training, which likely explains its failure to generalize. In contrast, for \( c_1 = 4, c_2 = 6 \), while the model has not seen tasks of the form \((*, 4, 6)\), it has encountered tasks where \( 4 \) appears in the second position, such as \((1, 4, 5)\), and tasks where \( 6 \) appears in the third position, such as \((2, 3, 6)\). This difference may explain why the model generalizes almost perfectly in Type 2 for \( c_1 = 4, c_2 = 6 \), but not for \( c_1 = 8, c_2 = 9 \).

% we observe a striking pattern: generalization fails entirely in Type 1, regardless of the coordinate pair (\(c_1, c_2\)). However, in Type 2, generalization varies: \(c_1 = 4, c_2 = 6\) achieves 96\% accuracy, while \(c_1 = 8, c_2 = 9\) lags behind at 70\%. Why? Upon closer inspection, the model struggles specifically with tasks like \((*, 8, 9)\), where the combination \(c_1 = 8\) and \(c_2 = 9\) is entirely novel. In contrast, for \(c_1 = 4, c_2 = 6\), the model benefits from having seen tasks where \(4\) appears in the second position or \(6\) in the third. This suggests that positional exposure during training plays a key role in generalization.

% To test whether task structure influences generalization, we consider two variations:
% \begin{enumerate}
%     \item \textbf{Sorted Tuples:} Tasks are always sorted in ascending order.
%     \item \textbf{Unsorted Tuples:} Tasks can appear in any order.
% \end{enumerate}

% If the model struggles with generalizing to the excluded position, does introducing variability through unsorted tuples help mitigate this limitation?

% \paragraph{Discussion of Results}

% In \textbf{Generalization with Missing Pairs}, we observe a striking pattern: generalization fails entirely in Type 1, regardless of the coordinate pair (\(c_1, c_2\)). However, in Type 2, generalization varies: \(c_1 = 4, c_2 = 6\) achieves 96\% accuracy, while \(c_1 = 8, c_2 = 9\) lags behind at 70\%. Why? Upon closer inspection, the model struggles specifically with tasks like \((*, 8, 9)\), where the combination \(c_1 = 8\) and \(c_2 = 9\) is entirely novel. In contrast, for \(c_1 = 4, c_2 = 6\), the model benefits from having seen tasks where \(4\) appears in the second position or \(6\) in the third. This suggests that positional exposure during training plays a key role in generalization.

% In \textbf{Generalization with Missing Coordinates}, the results confirm this hypothesis. When \(c_1 = 5\) is excluded from the second position during training, the model fails to generalize to such tasks in the sorted case. However, allowing unsorted tuples introduces positional diversity, leading to near-perfect generalization. This raises an intriguing question: does the model inherently overfit to positional patterns, and can task variability help break this tendency?




% In this subsection, we show that the selection of training tasks can affect the quality of the unseen task generalization significantly in practice. To illustrate this, we present two scenarios where tasks differ between training and testing. For each scenario, we invite the reader to assess, before examining the experimental results, which cases might appear “more” OOD. 

% % \amir{add examples, }

% \kaiyue{I think the name of scenarios here are not very clear}
% \begin{itemize}
%     \item \textbf{Scenario 1:  Generalization Across Excluded Coordinate Pairs (\( k = 3 \))} \\
%     In this scenario, we select two coordinates \( c_1 \) and \( c_2 \) out of \( d \) and construct three types of training sets. 

%     Suppose \( d = 5 \), \( c_1 = 1 \), and \( c_2 = 2 \). The tuples are all possible subsets of \( \{1, 2, 3, 4, 5\} \) with \( k = 3 \):
%     \[
%     \begin{aligned}
%     \big\{ & (1, 2, 3), (1, 2, 4), (1, 2, 5), (1, 3, 4), (1, 3, 5), \\
%            & (1, 4, 5), (2, 3, 4), (2, 3, 5), (2, 4, 5), (3, 4, 5) \big\}.
%     \end{aligned}
%     \]

%     \begin{enumerate}
%         \item \textbf{Type 1:} The training set includes all tuples except those containing both \( c_1 = 1 \) and \( c_2 = 2 \). 
%         For this example, the training set includes only $\{(3,4,5)\}$ tuple. The test set consists of tuples containing the rest of tuples.

%         \item \textbf{Type 2:} Similar to Type 1, but the training set additionally includes half of the tuples containing either \( c_1 = 1 \) \textit{or} \( c_2 = 2 \) (but not both). 
%         For the example, the training set includes all tuples from Type 1 and adds tuples like \(\{(1, 3, 4), (2, 3, 5)\}\) (half of those containing \( c_1 = 1 \) or \( c_2 = 2 \)).

%         \item \textbf{Type 3:} Similar to Type 2, but the training set also includes half of the tuples containing both \( c_1 = 1 \) \textit{and} \( c_2 = 2 \). 
%         For the example, the training set includes all tuples from Type 2 and adds, for instance, \(\{(1, 2, 5)\}\) (half of the tuples containing both \( c_1 \) and \( c_2 \)).
%     \end{enumerate}

% % \begin{itemize}
% %     \item \textbf{Type 1:} The training set includes tuples \(\{1, 3, 4\}, \{2, 3, 4\}\) (excluding tuples with both \( c_1 \) and \( c_2 \): \(\{1, 2, 3\}, \{1, 2, 4\}\)). The test set contains the excluded tuples.
% %     \item \textbf{Type 2:} The training set includes all tuples in Type 1 plus half of the tuples containing either \( c_1 = 1 \) or \( c_2 = 2 \) (e.g., \(\{1, 2, 3\}\)).
% %     \item \textbf{Type 3:} The training set includes all tuples in Type 2 plus half of the tuples containing both \( c_1 = 1 \) and \( c_2 = 2 \) (e.g., \(\{1, 2, 4\}\)).
% % \end{itemize}
    
%     \item \textbf{Scenario 2: Scenario 2: Generalization Across a Fixed Coordinate (\( k = 3 \))} \\
%     In this scenario, we select one coordinate \( c_1 \) out of \( d \) (\( c_1 = 5 \)). The training set includes all task tuples except those where \( c_1 \) is the second coordinate of the tuple. For this scenario, we examine two variations:
%     \begin{enumerate}
%         \item \textbf{Sorted Tuples:} Task tuples are always sorted (e.g., \( (x_1, x_2, x_3) \) with \( x_1 \leq x_2 \leq x_3 \)).
%         \item \textbf{Unsorted Tuples:} Task tuples can appear in any order.
%     \end{enumerate}
% \end{itemize}




% \paragraph{Discussion of Results.} In the first scenario, for both cases of \( c_1, c_2 \), we observe that generalization fails in Type 1, suggesting that the position of the tasks the model has been trained on significantly impacts its generalization capability. For Type 2, we find that \( c_1 = 4, c_2 = 6 \) performs significantly better than \( c_1 = 8, c_2 = 9 \). 

% Upon examining the tasks where the transformer fails for \( c_1 = 8, c_2 = 9 \), we see that the model only fails at tasks of the form \((*, 8, 9)\) while perfectly generalizing to the rest. This indicates that the model has never encountered the value \( 8 \) in the second position during training, which likely explains its failure to generalize. In contrast, for \( c_1 = 4, c_2 = 6 \), while the model has not seen tasks of the form \((*, 4, 6)\), it has encountered tasks where \( 4 \) appears in the second position, such as \((1, 4, 5)\), and tasks where \( 6 \) appears in the third position, such as \((2, 3, 6)\). This difference may explain why the model generalizes almost perfectly in Type 2 for \( c_1 = 4, c_2 = 6 \), but not for \( c_1 = 8, c_2 = 9 \).

% This position-based explanation appears compelling, so in the second scenario, we focus on a single position to investigate further. Here, we find that the transformer fails to generalize to tasks where \( 5 \) appears in the second position, provided it has never seen any such tasks during training. However, when we allow for more task diversity in the unsorted case, the model achieves near-perfect generalization. 

% This raises an important question: does the transformer have a tendency to overfit to positional patterns, and does introducing more task variability, as in the unsorted case, prevent this overfitting and enable generalization to unseen positional configurations?

% These findings highlight that even in a simple task like parity, it is remarkably challenging to understand and quantify the sources and levels of OOD behavior. This motivates further investigation into the nuances of task design and its impact on model generalization.


\subsection{Task Generalization Beyond Parity Problems}

% \begin{figure}[t!]
%     \centering
%     \includegraphics[width=0.45\textwidth]{Figures/arithmetic_v1.png}
%     \vspace{-0.3cm}
%     \caption{Task generalization for arithmetic task with CoT, it has $\d =2$ and $T = d-1$ as the ambient dimension, hence $D\ln(DT) = 2\ln(2T)$. We show that the empirical scaling closely follows the theoretical scaling.}
%     \label{fig:arithmetic}
% \end{figure}



% \begin{wrapfigure}{r}{0.4\textwidth}  % 'r' for right, 'l' for left
%     \centering
%     \includegraphics[width=0.4\textwidth]{Figures/arithmetic_v1.png}
%     \vspace{-0.3cm}
%     \caption{Task generalization for the arithmetic task with CoT. It has $d =2$ and $T = d-1$ as the ambient dimension, hence $D\ln(DT) = 2\ln(2T)$. We show that the empirical scaling closely follows the theoretical scaling.}
%     \label{fig:arithmetic}
% \end{wrapfigure}

\subsubsection{Arithmetic Task}\label{subsec:arithmetic}











We introduce the family of \textit{Arithmetic} task that, like the sparse parity problem, operates on 
\( d \) binary inputs \( b_1, b_2, \dots, b_d \). The task involves computing a structured arithmetic expression over these inputs using a sequence of addition and multiplication operations.
\newcommand{\op}{\textrm{op}}

Formally, we define the function:
\[
\text{Arithmetic}_{S} \colon \{0,1\}^d \to \{0,1,\dots,d\},
\]
where \( S = (\op_1, \op_2, \dots, \op_{d-1}) \) is a sequence of \( d-1 \) operations, each \( \op_k \) chosen from \( \{+, \times\} \). The function evaluates the expression by applying the operations sequentially from left-to-right order: for example, if \( S = (+, \times, +) \), then the arithmetic function would compute:
\[
\text{Arithmetic}_{S}(b_1, b_2, b_3, b_4) = ((b_1 + b_2) \times b_3) + b_4.
\]
% Thus, the sequence of operations \( S \) defines how the binary inputs are combined to produce an integer output between \( 0 \) and \( d \).
% \[
% \text{Arithmetic}_{S} 
% (b_1,\,b_2,\,\dots,b_d)
% =
% \Bigl(\dots\bigl(\,(b_1 \;\op_1\; b_2)\;\op_2\; b_3\bigr)\,\dots\Bigr) 
% \;\op_{d-1}\; b_d.
% \]
% We now introduce an \emph{Arithmetic} task that, like the sparse parity problem, operates on $d$ binary inputs $b_1, b_2, \dots, b_d$. Specifically, we define an arithmetic function
% \[
% \text{Arithmetic}_{S}\colon \{0,1\}^d \;\to\; \{0,1,\dots,d\},
% \]
% where $S = (i_1, i_2, \dots, i_{d-1})$ is a sequence of $d-1$ operations, each $i_k \in \{+,\,\times\}$. The value of $\text{Arithmetic}_{S}$ is obtained by applying the prescribed addition and multiplication operations in order, namely:
% \[
% \text{Arithmetic}_{S}(b_1,\,b_2,\,\dots,b_d)
% \;=\;
% \Bigl(\dots\bigl(\,(b_1 \;i_1\; b_2)\;i_2\; b_3\bigr)\,\dots\Bigr) 
% \;i_{d-1}\; b_d.
% \]

% This is an example of our framework where $T = d-1$ and $|\Theta_t| = 2$ with total $2^d$ possible tasks. 




By introducing a step-by-step CoT, arithmetic class belongs to $ARC(2, d-1)$: this is because at every step, there is only $\d = |\Theta_t| = 2$ choices (either $+$ or $\times$) while the length is  $T = d-1$, resulting a total number of $2^{d-1}$ tasks. 


\begin{minipage}{0.5\textwidth}  % Left: Text
    Task generalization for the arithmetic task with CoT. It has $d =2$ and $T = d-1$ as the ambient dimension, hence $D\ln(DT) = 2\ln(2T)$. We show that the empirical scaling closely follows the theoretical scaling.
\end{minipage}
\hfill
\begin{minipage}{0.4\textwidth}  % Right: Image
    \centering
    \includegraphics[width=\textwidth]{Figures/arithmetic_v1.png}
    \refstepcounter{figure}  % Manually advances the figure counter
    \label{fig:arithmetic}  % Now this label correctly refers to the figure
\end{minipage}

Notably, when scaling with \( T \), we observe in the figure above that the task scaling closely follow the theoretical $O(D\log(DT))$ dependency. Given that the function class grows exponentially as \( 2^T \), it is truly remarkable that training on only a few hundred tasks enables generalization to an exponentially larger space—on the order of \( 2^{25} > 33 \) Million tasks. This exponential scaling highlights the efficiency of structured learning, where a modest number of training examples can yield vast generalization capability.





% Our theory suggests that only $\Tilde{O}(\ln(T))$ i.i.d training tasks is enough to generalize to the rest of unseen tasks. However, we show in Figure \ref{fig:arithmetic} that transformer is not able to match  that. The transformer out-of distribution generalization behavior is not consistent across different dimensions when we scale the number of training tasks with $\ln(T)$. \hongzhou{implicit bias, optimization, etc}
 






% \subsection{Task generalization Beyond parity problem}

% \subsection{Arithmetic} In this setting, we still use the set-up we introduced in \ref{subsec:parity_exmaple}, the input is still a set of $d$ binary variable, $b_1, b_2,\dots,b_d$ and ${Arithmatic_{S}}:\{0,1\}\rightarrow \{0, 1, \dots, d\}$, where $S = (i_1,i_2,\dots,i_{d-1})$ is a tuple of size $d-1$ where each coordinate is either add($+
% $) or multiplication ($\times$). The function is as following,

% \begin{align*}
%     Arithmatic_{S}(b_1, b_2,\dots,b_d) = (\dots(b1(i1)b2)(i3)b3\dots)(i{d-1})
% \end{align*}
    


\subsubsection{Multi-Step Language Translation Task}

 \begin{figure*}[h!]
    \centering
    \includegraphics[width=0.9\textwidth]{Figures/combined_plot_horiz.png}
    \vspace{-0.2cm}
    \caption{Task generalization for language translation task: $\d$ is the number of languages and $T$ is the length of steps.}
    \vspace{-2mm}
    \label{fig:language}
\end{figure*}
% \vspace{-2mm}

In this task, we study a sequential translation process across multiple languages~\cite{garg2022can}. Given a set of \( D \) languages, we construct a translation chain by randomly sampling a sequence of \( T \) languages \textbf{with replacement}:  \(L_1, L_2, \dots, L_T,\)
where each \( L_t \) is a sampled language. Starting with a word, we iteratively translate it through the sequence:
\vspace{-2mm}
\[
L_1 \to L_2 \to L_3 \to \dots \to L_T.
\]
For example, if the sampled sequence is EN → FR → DE → FR, translating the word "butterfly" follows:
\vspace{-1mm}
\[
\text{butterfly} \to \text{papillon} \to \text{schmetterling} \to \text{papillon}.
\]
This task follows an \textit{AutoRegressive Compositional} structure by itself, specifically \( ARC(D, T-1) \), where at each step, the conditional generation only depends on the target language, making \( D \) as the number of languages and the total number of possible tasks is \( D^{T-1} \). This example illustrates that autoregressive compositional structures naturally arise in real-world languages, even without explicit CoT. 

We examine task scaling along \( D \) (number of languages) and \( T \) (sequence length). As shown in Figure~\ref{fig:language}, empirical  \( D \)-scaling closely follows the theoretical \( O(D \ln D T) \). However, in the \( T \)-scaling case, we observe a linear dependency on \( T \) rather than the logarithmic dependency \(O(\ln T) \). A possible explanation is error accumulation across sequential steps—longer sequences require higher precision in intermediate steps to maintain accuracy. This contrasts with our theoretical analysis, which focuses on asymptotic scaling and does not explicitly account for compounding errors in finite-sample settings.

% We examine task scaling along \( D \) (number of languages) and \( T \) (sequence length). As shown in Figure~\ref{fig:language}, empirical scaling closely follows the theoretical \( O(D \ln D T) \) trend, with slight exceptions at $ T=10 \text{ and } 3$ in Panel B. One possible explanation for this deviation could be error accumulation across sequential steps—longer sequences require each intermediate translation to be approximated with higher precision to maintain test accuracy. This contrasts with our theoretical analysis, which primarily focuses on asymptotic scaling and does not explicitly account for compounding errors in finite-sample settings.

Despite this, the task scaling is still remarkable — training on a few hundred tasks enables generalization to   $4^{10} \approx 10^6$ tasks!






% , this case, we are in a regime where \( D \ll T \), we observe  that the task complexity empirically scales as \( T \log T \) rather than \( D \log T \). 


% the model generalizes to an exponentially larger space of \( 2^T \) unseen tasks. In case $T=25$, this is $2^{25} > 33$ Million tasks. This remarkable exponential generalization demonstrates the power of structured task composition in enabling efficient generalization.


% In the case of parity tasks, introducing CoT effectively decomposes the problem from \( ARC(D^T, 1) \) to \( ARC(D, T) \), significantly improving task generalization.

% Again, in the regime scaling $T$, we again observe a $T\log T$ dependency. Knowing that the function class is scaling as $D^T$, it is remarkable that training on a few hundreds tasks can generalize to $4^{10} \approx 1M$ tasks. 





% We further performed a preliminary investigation on a semi-synthetic word-level translation task to show that (1) task generalization via composition structure is feasible beyond parity and (2) understanding the fine-grained mechanism leading to this generalization is still challenging. 
% \noindent
% \noindent
% \begin{minipage}[t]{\columnwidth}
%     \centering
%     \textbf{\scriptsize In-context examples:}
%     \[
%     \begin{array}{rl}
%         \textbf{Input} & \hspace{1.5em} \textbf{Output} \\
%         \hline
%         \textcolor{blue}{car}   & \hspace{1.5em} \textcolor{red}{voiture \;,\; coche} \\
%         \textcolor{blue}{house} & \hspace{1.5em} \textcolor{red}{maison \;,\; casa} \\
%         \textcolor{blue}{dog}   & \hspace{1.5em} \textcolor{red}{chien \;,\; perro} 
%     \end{array}
%     \]
%     \textbf{\scriptsize Query:}
%     \[
%     \begin{array}{rl}
%         \textbf{Input} & \textbf{Output} \\
%         \hline
%         \textcolor{blue}{cat} & \hspace{1.5em} \textcolor{red}{?} \\
%     \end{array}
%     \]
% \end{minipage}



% \begin{figure}[h!]
%     \centering
%     \includegraphics[width=0.45\textwidth]{Figures/translation_scale_d.png}
%     \vspace{-0.2cm}
%     \caption{Task generalization behavior for word translation task.}
%     \label{fig:arithmetic}
% \end{figure}


\vspace{-1mm}
\section{Conclusions}
% \misha{is it conclusion of the section or of the whole paper?}    
% \amir{The whole paper. It is very short, do we need a separate section?}
% \misha{it should not be a subsection if it is the conclusion the whole thing. We can just remove it , it does not look informative} \hz{let's do it in a whole section, just to conclude and end the paper, even though it is not informative}
%     \kaiyue{Proposal: Talk about the implication of this result on theory development. For example, it calls for more fine-grained theoretical study in this space.  }

% \huaqing{Please feel free to edit it if you have better wording or suggestions.}

% In this work, we propose a theoretical framework to quantitatively investigate task generalization with compositional autoregressive tasks. We show that task to $D^T$ task is theoretically achievable by training on only $O (D\log DT)$ tasks, and empirically observe that transformers trained on parity problem indeed achieves such task generalization. However, for other tasks beyond parity, transformers seem to fail to achieve this bond. This calls for more fine-grained theoretical study the phenomenon of task generalization specific to transformer model. It may also be interesting to study task generalization beyond the setting of in-context learning. 
% \misha{what does this add?} \amir{It does not, i dont have any particular opinion to keep it. @Hongzhou if you want to add here?}\hz{While it may not introduce anything new, we are following a good practice to have a short conclusion. It provides a clear closing statement, reinforces key takeaways, and helps the reader leave with a well-framed understanding of our contributions. }
% In this work, we quantitatively investigate task generalization under autoregressive compositional structure. We demonstrate that task generalization to $D^T$ tasks is theoretically achievable by training on only $\tilde O(D)$ tasks. Empirically, we observe that transformers trained indeed achieve such exponential task generalization on problems such as parity, arithmetic and multi-step language translation. We believe our analysis opens up a new angle to understand the remarkable generalization ability of Transformer in practice. 

% However, for tasks beyond the parity problem, transformers appear to fail to reach this bound. This highlights the need for a more fine-grained theoretical exploration of task generalization, especially for transformer models. Additionally, it may be valuable to investigate task generalization beyond the scope of in-context learning.



In this work, we quantitatively investigated task generalization under the autoregressive compositional structure, demonstrating both theoretically and empirically that exponential task generalization to $D^T$ tasks can be achieved with training on only $\tilde{O}(D)$ tasks. %Our theoretical results establish a fundamental scaling law for task generalization, while our experiments validate these insights across problems such as parity, arithmetic, and multi-step language translation. The remarkable ability of transformers to generalize exponentially highlights the power of structured learning and provides a new perspective on how large language models extend their capabilities beyond seen tasks. 
We recap our key contributions  as follows:
\begin{itemize}
    \item \textbf{Theoretical Framework for Task Generalization.} We introduced the \emph{AutoRegressive Compositional} (ARC) framework to model structured task learning, demonstrating that a model trained on only $\tilde{O}(D)$ tasks can generalize to an exponentially large space of $D^T$ tasks.
    
    \item \textbf{Formal Sample Complexity Bound.} We established a fundamental scaling law that quantifies the number of tasks required for generalization, proving that exponential generalization is theoretically achievable with only a logarithmic increase in training samples.
    
    \item \textbf{Empirical Validation on Parity Functions.} We showed that Transformers struggle with standard in-context learning (ICL) on parity tasks but achieve exponential generalization when Chain-of-Thought (CoT) reasoning is introduced. Our results provide the first empirical demonstration of structured learning enabling efficient generalization in this setting.
    
    \item \textbf{Scaling Laws in Arithmetic and Language Translation.} Extending beyond parity functions, we demonstrated that the same compositional principles hold for arithmetic operations and multi-step language translation, confirming that structured learning significantly reduces the task complexity required for generalization.
    
    \item \textbf{Impact of Training Task Selection.} We analyzed how different task selection strategies affect generalization, showing that adversarially chosen training tasks can hinder generalization, while diverse training distributions promote robust learning across unseen tasks.
\end{itemize}



We introduce a framework for studying the role of compositionality in learning tasks and how this structure can significantly enhance generalization to unseen tasks. Additionally, we provide empirical evidence on learning tasks, such as the parity problem, demonstrating that transformers follow the scaling behavior predicted by our compositionality-based theory. Future research will  explore how these principles extend to real-world applications such as program synthesis, mathematical reasoning, and decision-making tasks. 


By establishing a principled framework for task generalization, our work advances the understanding of how models can learn efficiently beyond supervised training and adapt to new task distributions. We hope these insights will inspire further research into the mechanisms underlying task generalization and compositional generalization.

\section*{Acknowledgements}
We acknowledge support from the National Science Foundation (NSF) and the Simons Foundation for the Collaboration on the Theoretical Foundations of Deep Learning through awards DMS-2031883 and \#814639 as well as the  TILOS institute (NSF CCF-2112665) and the Office of Naval Research (ONR N000142412631). 
This work used the programs (1) XSEDE (Extreme science and engineering discovery environment)  which is supported by NSF grant numbers ACI-1548562, and (2) ACCESS (Advanced cyberinfrastructure coordination ecosystem: services \& support) which is supported by NSF grants numbers \#2138259, \#2138286, \#2138307, \#2137603, and \#2138296. Specifically, we used the resources from SDSC Expanse GPU compute nodes, and NCSA Delta system, via allocations TG-CIS220009. 

We present RiskHarvester, a risk-based tool to compute a security risk score based on the value of the asset and ease of attack on a database. We calculated the value of asset by identifying the sensitive data categories present in a database from the database keywords. We utilized data flow analysis, SQL, and Object Relational Mapper (ORM) parsing to identify the database keywords. To calculate the ease of attack, we utilized passive network analysis to retrieve the database host information. To evaluate RiskHarvester, we curated RiskBench, a benchmark of 1,791 database secret-asset pairs with sensitive data categories and host information manually retrieved from 188 GitHub repositories. RiskHarvester demonstrates precision of (95\%) and recall (90\%) in detecting database keywords for the value of asset and precision of (96\%) and recall (94\%) in detecting valid hosts for ease of attack. Finally, we conducted an online survey to understand whether developers prioritize secret removal based on security risk score. We found that 86\% of the developers prioritized the secrets for removal with descending security risk scores.
\section{Acknowledgement}
We sincerely thank the anonymous shepherd for constructive comments and feedback in improving this work. 
This paper was supported by Yunnan Forestry and Grassland Science and Technology Innovation Joint Special Project (grant NO. 202404CB090017), Natural Science Foundation of China under Grant 62371269, Guangdong Innovative and Entrepreneurial Research Team Program (2021ZT09 L197), Meituan Academy of Robotics Shenzhen.

\newpage
\balance



% \section{Related Work}
\label{sec:RelatedWork}

Within the realm of geophysical sciences, super-resolution/downscaling is a challenge that scientists continue to tackle. There have been several works involved in downscaling applications such as river mapping \cite{Yin2022}, coastal risk assessment \cite{Rucker2021}, estimating soil moisture from remotely sensed images \cite{Peng2017SoilMoisture} and downscaling of satellite based precipitation estimates \cite{Medrano2023PrecipitationDownscaling} to name a few. We direct the reader to \cite{Karwowska2022SuperResolutionSurvey} for a comprehensive review of satellite based downscaling applications and methods. Pertaining to our objective of downscaling \acp{WFM}, we can draw comparisons with several existing works. 
In what follows, we provide a brief review of functionally adjacent works to contrast the novelty of our proposed model and its role in addressing gaps in literature. 

When it comes to downscaling \ac{WFM}, several works use statistical downscaling techniques. These works downscale images by using statistical techniques that utilize relationships between neighboring water fraction pixels. For instance, \cite{Li2015SRFIM} treat the super-resolution task as a sub-pixel mapping problem, wherein the input fraction of inundated pixels must be exactly mapped to the output patch of inundated pixels. 
% In doing so, they are able to apply a discrete particle swarm optimization method to maximize the Flood Inundation Spatial Dependence Index (FISDI). 
\cite{Wang2019} improved upon these approaches by including a spectral term to fully utilize spectral information from multi spectral remote sensing image band. \cite{Wang2021} on the other hand also include a spectral correlation term to reduce the influence of linear and non-linear imaging conditions. All of these approaches are applied to water fraction obtained via spectral unmixing \cite{wang2013SpectralUnmixing} and are designed to work with multi spectral information from MODIS. However, we develop our model with the intention to be used with water fractions directly derived from the output of satellites. One such example is NOAA/VIIRS whose water fraction extraction method is described in \cite{Li2013VIIRSWFM}. \cite{Li2022VIIRSDownscaling} presented a work wherein \ac{WFM} at 375-m flood products from VIIRS were downscaled 30-m flood event and depth products by expressing the inundation mechanism as a function of the \ac{DEM}-based water area and the VIIRS water area.

On the other hand, the non-linear nature of the mapping task lends itself to the use of neural networks. Several models have been adapted from traditional single image digital super-resolution in computer vision literature \cite{sdraka2022DL4downscalingRemoteSensing}. Existing deep learning models in single image super-resolution are primarily dominated by \ac{CNN} based models. Specifically, there has been an upward trend in residual learning models. \acp{RDN} \cite{Zhang2018ResidualDenseSuperResolution} introduced residual dense blocks that employed a contiguous memory mechanism that aimed to overcome the inability of very deep \acp{CNN} to make full use of hierarchical features. 
\acp{RCAN} \cite{Zhang2018RCANSuperResolution} introduced an attention mechanism to exploit the inter-channel dependencies in the intermediate feature transformations. There have also been some works that aim to produce more lightweight \ac{CNN}-based architectures \cite{Zheng2019IMDN,Xiaotong2020LatticeNET}. Since the introduction of the vision transformer \cite{Vaswani2017Attention} that utilized the self-attention mechanism -- originally used for modeling text sequences -- by feeding a sequence 2D sub-image extracted from the original image. Using this approach \cite{LuESRT2022} developed a light-weight and efficient transformer based approach for single image super-resolution. 


For the task of super-resolution of \acp{WFM}, we discuss some works whose methodology is similar to ours even though they differ in their problem setting. \cite{Yin2022} presented a cascaded spectral spatial model for super-resolution of MODIS imagery with a scaling factor 10. Their architecture consists of two stages; first multi-spectral MODIS imagery is converted into a low-resolution \ac{WFM} via spectral unmixing by passing it through a deep stacked residual \ac{CNN}. The second stage involved the super-resolution mapping of these \acp{WFM} using a nested multi-level \ac{CNN} model. Similar to our work, the input fraction images are obtained with zero errors which may not be reflective of reality since there tends to be sensor noise, the spatial distribution of whom cannot be easily estimated. We also note that none of these works directly tackle flood inundation since they've been trained with river map data during non-flood circumstance and \textit{ergo} do not face a data scarcity problem as we do. 
% In this work, apart from the final product of \acp{WFM}, we are not presented with any additional spectral information about the low resolution image. This was intended to work directly with products that can generate \ac{WFM} either directly (VIIRS) or indirectly (Landsat).
\cite{Jia2019} used a deep \ac{CNN} for land mapping that consists of several classes such as building, low vegetation, background and trees. 
\cite{Kumar2021} similarly employ a \ac{CNN} based model for downscaling of summer monsoon rainfall data over the Indian subcontinent. Their proposed Super-Resolution Convolutional Neural Network (SRCNN) has a downscaling factor of 4. 
\cite{Shang2022} on the other hand, proposed a super-resolution mapping technique using Generative Adversarial Networks (GANs). They first generate high resolution fractional images, somewhat analogous to our \ac{WFM}, and are then mapped to categorical land cover maps involving forest, urban, agriculture and water classes. 
\cite{Qin2020} interestingly approach lake area super-resolution for Landsat and MODIS data as an unsupervised problem using a \ac{CNN} and are able to extend to other scaling factors. \cite{AristizabalInundationMapping2020} performed flood inundation mapping using \ac{SAR} data obtained from Sentinel-1. They showed that \ac{DEM}-based features helped to improve \ac{SAR}-based predictions for quadratic discriminant analysis, support vector machines and k-nearest neighbor classifiers. While almost all of the aforementioned works can be adapted to our task. We stand out in the following ways (i) We focus on downscaling of \acp{WFM} directly, \textit{i.e.,} we do not focus on the algorithm to compute the \ac{WFM} from multi-channel satellite data and (ii) We focus on producing high resolution maps only for instances of flood inundation. The latter point produces a data scarcity issue which we seek to remedy with synthetic data. 


%%%%%%%%%%%%%%%%% Additional unused information %%%%%%%%%%%%%%%%


%     \item[\cite{Wang2021}] Super-Resolution Mapping Based on Spatial–Spectral Correlation for Spectral Imagery
%     \begin{itemize}
%         \item Not a deep neural network approach. SRM based on spatial–spectral correlation (SSC) is proposed in order to overcome the influence of linear and nonlinear imaging conditions and utilize more accurate spectral properties.
%         \item (fig 1) there are two main SRM types: (1) the initialization-then-optimization SRM, where the class labels are allocated randomly to subpixels, and the location of each subpixel is optimized to obtain the final SRM result. and (2)soft-then-hard SRM, which involves two steps: the subpixel sharpening and the class allocation.  
%         \item SSC procedures: (1) spatial correlation is performed by the MSAM to reduce the influences of linear imaging conditions on image quality. (2) A spectral correlation that utilizes spectral properties based on the nonlinear KLD is proposed to reduce the influences of nonlinear imaging conditions. (3) spatial and spectral correlations are then combined to obtain an optimization function with improved linear and nonlinear performances. And finally (4) by maximizing the optimization function, a class allocation method based on the SA is used to assign LC labels to each subpixel, obtaining the final SRM result.
%         \item (Comparable) 
%     \end{itemize}
%     %--------------------------------------------------------------------
% \cite{Wang2021} account for the influence of linear and non-linear imaging conditions by involving more accurate spectral properties. 
%     %--------------------------------------------------------------------
%     \item[\cite{Yin2022}] A Cascaded Spectral–Spatial CNN Model for Super-Resolution River Mapping With MODIS Imagery
%     \begin{itemize}
%         \item produce  Landsat-like  fine-resolution (scale of 10)  river  maps  from  MODIS images. Notice the original coarse-resolution remotely sensed images, not the river fraction images.
%         \item combined  CNN  model that  contains  a spectral  unmixing  module  and  an  SRM  module, and the SRM module is made up of an encoder and a decoder that are connected through a series of convolutional blocks. 
%         \item With an adaptive cross-entropy loss function to address class imbalance.	
%         \item The overall accuracy, the omission error, the  commission  error,  and  the  mean  intersection  over  union (MIOU)  calculated  to  assess  the results.
%         \item partially comparable with ours, only the SRM module part
%     %--------------------------------------------------------------------

% To decouple the description of the objective and the \ac{ML} model architecture, the motivation for the model architecture is described in \secref{sec:Methodology}. 


%     \item[\cite{Wang2019}] Improving Super-Resolution Flood Inundation Mapping for Multi spectral Remote Sensing Image by Supplying More Spectral 
%     \begin{itemize}
%         \item proposed the SRFIM-MSI,where a new spectral term is added to the traditional SRFIM to fully utilize the spectral information from multi spectral remote sensing image band. 
%         \item The original SRFIM \cite{Huang2014, Li2015} obtains the sub pixel spatial distribution of flood inundation within mixed pixels by maximizing their spatial correlation while maintaining the original proportions of flood inundation within the mixed pixels. The SRFIM is formulated as a maximum combined optimization issue according to the principle of spatial correlation.
%         \item follow the terminology in \cite{Wang2021}, this is an initialization-then-optimization SRM. 
%         \item (Comparable) 
%     \end{itemize}
%     %--------------------------------------------------------------------


%--------------------------------------------------------------------
%     \item[\cite{Jia2019}] Super-Resolution Land Cover Mapping Based on the Convolutional Neural Network
%     \begin{itemize}
%         \item SRMCNN (Super-resolution mapping CNN) is proposed to obtain fine-scale land cover maps from coarse remote sensing images. Specifically, an encoder-decoder CNN is used to determined the labels (i.e., land cover classes) of the subpixels within mixed pixels.
%         \item There were three main parts in SRMCNN. The first part was a three-sequential convolutional layer with ReLU and pooling. The second part is up-sampling, for which a multi transposed-convolutional layer was adopted. To keep the feature learned in the previous layer, a skip connection was used to concatenate the output of the corresponding convolution layer. The last part was the softmax classifier, in which the feature in the antepenultimate layer was classified and class probabilities are obtained.
%         \item The loss: the optimal allocation of classes to the subpixels of mixed pixel is achieved by maximizing the spatial dependence between neighbor pixels under constraint that the class proportions within the mixed pixels are preserved.
%         \item (Preferred), this paper is designed to classify background, Building, Low Vegetation, or Tree in the land. But we can easily adapt to our problem and should compare with this paper.
%     \end{itemize}
%     %--------------------------------------------------------------------

%     \item[\cite{Kumar2021}] Deep learning–based downscaling of summer monsoon rainfall data over Indian region
%     \begin{itemize}
%         \item down-scaling (scale of 4) rainfall data. The output image is not binary image.
%         \item three algorithms: SRCNN, stacked SRCNN, and DeepSD are employed, based on \cite{Vandal2019}
%         \item mean square error and pattern correlation coefficient are used as evaluation metrics.
%         \item SRCNN: super-resolution-based convolutional neural networks (SRCNN) first upgrades the low-resolution image to the higher resolution size by using bicubic interpolation. Suppose the interpolated image is referred to as Y; SRCNNs’task is to retrieve from Y an image F(Y) which is close to the high-resolution ground truth image X.
%         \item stacked SRCNN: stack 2 or more SRCNN blocks to increasing the scaling factor.
%         \item DeepSD: uses topographies as an additional input to stacked SRCNN.
%         \item These algorithms are not designed for binary output images, but if prefer, the ``modified'' stacked SRCNN or DeepSD can be used as baseline algorithms.
%     
%     \item[\cite{Shang2022}] Super resolution Land Cover Mapping Using a Generative Adversarial Network
%     \begin{itemize}
%         \item propose an end-to-end SRM model based on a generative adversarial network (GAN), that is, GAN-SRM, to improve the two-step learning-based SRM methods. 
%         \item Two-step SRM method: The first step is fraction-image super-resolution (SR), which reconstructs a high-spatial-resolution fraction image from the low input, methods like SVR, or CNN has been widely adopted. The second step is converting the high-resolution fraction images to a categorical land cover map, such as with a soft-max function to assign each high-resolution pixel to a unique category value.
%         \item The proposed GAN-SRM model includes a generative network and a discriminative network, so that both the fraction-image SR and the conversion of the fraction images to categorical map steps are fully integrated to reduce the resultant uncertainty. 
%         \item applied to the National Land Cover Database (NLCD), which categorized land into four typical classes:forest, urban, agriculture,and water. scale factor of 8. 
%         \item (Preferred), we should compare with this work.
%     \end{itemize}
%     %--------------------------------------------------------------------

%   \item[\cite{Qin2020}] Achieving Higher Resolution Lake Area from Remote Sensing Images Through an Unsupervised Deep Learning Super-Resolution Method
%   \begin{itemize}
%       \item propose an unsupervised deep gradient network (UDGN) to generate a higher resolution lake area from remote sensing images.
%       \item UDGN models the internal recurrence of information inside the single image and its corresponding gradient map to generate images with higher spatial resolution. 
%       \item A single image super-resolution approach, not comparable
%   \end{itemize}
%     %--------------------------------------------------------------------




%     \item[\cite{Demiray2021}] D-SRGAN: DEM Super-Resolution with Generative Adversarial Networks
%     \begin{itemize}
%         \item A GAN based model is proposed to increase the spatial resolution of a given DEM dataset up to 4 times without additional information related to data.
%         \item Rather than processing each image in a sequence independently, our generator architecture uses a recurrent layer to update the state of the high-resolution reconstruction in a manner that is consistent with both the previous state and the newly received data. The recurrent layer can thus be understood as performing a Bayesian update on the ensemble member, resembling an ensemble Kalman filter. 
%         \item A single image super-resolution approach, not comparable
%     \end{itemize}
%     %--------------------------------------------------------------------
%     \item[\cite{Leinonen2021}] Stochastic Super-Resolution for Downscaling Time-Evolving Atmospheric Fields With a Generative Adversarial Network
%     \begin{itemize}
%         \item propose a super-resolution GAN that operates on sequences of two-dimensional images and creates an ensemble of predictions for each input. The spread between the ensemble members represents the uncertainty of the super-resolution reconstruction.
%         \item for sequence of input images, not comparable with ours.
%     \end{itemize} 
%     %--------------------------------------------------------------------

% \end{itemize}





% \section{Acknowledgments}

% Identification of funding sources and other support, and thanks to
% individuals and groups that assisted in the research and the
% preparation of the work should be included in an acknowledgment
% section, which is placed just before the reference section in your
% document.

% This section has a special environment:
% \begin{verbatim}
%   \begin{acks}
%   ...
%   \end{acks}
% \end{verbatim}
% so that the information contained therein can be more easily collected
% during the article metadata extraction phase, and to ensure
% consistency in the spelling of the section heading.

% Authors should not prepare this section as a numbered or unnumbered {\verb|\section|}; please use the ``{\verb|acks|}'' environment.

% \section{Appendices}

% If your work needs an appendix, add it before the
% ``\verb|\end{document}|'' command at the conclusion of your source
% document.

% Start the appendix with the ``\verb|appendix|'' command:
% \begin{verbatim}
%   \appendix
% \end{verbatim}
% and note that in the appendix, sections are lettered, not
% numbered. This document has two appendices, demonstrating the section
% and subsection identification method.

% \section{SIGCHI Extended Abstracts}

% The ``\verb|sigchi-a|'' template style (available only in \LaTeX\ and
% not in Word) produces a landscape-orientation formatted article, with
% a wide left margin. Three environments are available for use with the
% ``\verb|sigchi-a|'' template style, and produce formatted output in
% the margin:
% \begin{itemize}
% \item {\verb|sidebar|}:  Place formatted text in the margin.
% \item {\verb|marginfigure|}: Place a figure in the margin.
% \item {\verb|margintable|}: Place a table in the margin.
% \end{itemize}

% %%
% %% The acknowledgments section is defined using the "acks" environment
% %% (and NOT an unnumbered section). This ensures the proper
% %% identification of the section in the article metadata, and the
% %% consistent spelling of the heading.
% \begin{acks}
% To Robert, for the bagels and explaining CMYK and color spaces.
% \end{acks}

% %%
% %% The next two lines define the bibliography style to be used, and
% %% the bibliography file.
\bibliographystyle{unsrt}
% \bibliography{sample-base}
\bibliography{reference}

% %%
% %% If your work has an appendix, this is the place to put it.
% \appendix

% \section{Research Methods}

% \subsection{Part One}

% Lorem ipsum dolor sit amet, consectetur adipiscing elit. Morbi
% malesuada, quam in pulvinar varius, metus nunc fermentum urna, id
% sollicitudin purus odio sit amet enim. Aliquam ullamcorper eu ipsum
% vel mollis. Curabitur quis dictum nisl. Phasellus vel semper risus, et
% lacinia dolor. Integer ultricies commodo sem nec semper.

% \subsection{Part Two}

% Etiam commodo feugiat nisl pulvinar pellentesque. Etiam auctor sodales
% ligula, non varius nibh pulvinar semper. Suspendisse nec lectus non
% ipsum convallis congue hendrerit vitae sapien. Donec at laoreet
% eros. Vivamus non purus placerat, scelerisque diam eu, cursus
% ante. Etiam aliquam tortor auctor efficitur mattis.

% \section{Online Resources}

% Nam id fermentum dui. Suspendisse sagittis tortor a nulla mollis, in
% pulvinar ex pretium. Sed interdum orci quis metus euismod, et sagittis
% enim maximus. Vestibulum gravida massa ut felis suscipit
% congue. Quisque mattis elit a risus ultrices commodo venenatis eget
% dui. Etiam sagittis eleifend elementum.

% Nam interdum magna at lectus dignissim, ac dignissim lorem
% rhoncus. Maecenas eu arcu ac neque placerat aliquam. Nunc pulvinar
% massa et mattis lacinia.

\end{document}
\endinput
%%
%% End of file `sample-sigconf.tex'.
