% 如果您使用 overleaf,您只需单击“提交”图标并选择“Arxiv”选项,它将为您提供所需的所有文件。


% edit link: https://www.overleaf.com/7243917635qvjxbtpbhxbg#f5184e

%%
%% This is file `sample-sigconf.tex',
%% generated with the docstrip utility.
%%
%% The original source files were:
%%
%% samples.dtx  (with options: `sigconf')
%% 
%% IMPORTANT NOTICE:
%% 
%% For the copyright see the source file.
%% 
%% Any modified versions of this file must be renamed
%% with new filenames distinct from sample-sigconf.tex.
%% 
%% For distribution of the original source see the terms
%% for copying and modification in the file samples.dtx.
%% 
%% This generated file may be distributed as long as the
%% original source files, as listed above, are part of the
%% same distribution. (The sources need not necessarily be
%% in the same archive or directory.)
%%
%% The first command in your LaTeX source must be the \documentclass command.
\documentclass[sigconf,9pt]{acmart}

\makeatletter
\def\@ACM@checkaffil{% Only warnings
    \if@ACM@instpresent\else
    \ClassWarningNoLine{\@classname}{No institution present for an affiliation}%
    \fi
    \if@ACM@citypresent\else
    \ClassWarningNoLine{\@classname}{No city present for an affiliation}%
    \fi
    \if@ACM@countrypresent\else
        \ClassWarningNoLine{\@classname}{No country present for an affiliation}%
    \fi
}
\makeatother

% \documentclass[sigconf,9pt]{manuscript}
% \documentclass[manuscript, anonymous, sigconf]{acmart}
%% NOTE that a single column version may be required for 
%% submission and peer review. This can be done by changing
%% the \doucmentclass[...]{acmart} in this template to 
%% \documentclass[manuscript,screen]{acmart}
%% 
%% To ensure 100% compatibility, please check the white list of
%% approved LaTeX packages to be used with the Master Article Template at
%% https://www.acm.org/publications/taps/whitelist-of-latex-packages 
%% before creating your document. The white list page provides 
%% information on how to submit additional LaTeX packages for 
%% review and adoption.
%% Fonts used in the template cannot be substituted; margin 
%% adjustments are not allowed.
%%
%%
%% \BibTeX command to typeset BibTeX logo in the docs
\AtBeginDocument{%
  \providecommand\BibTeX{{%
    \normalfont B\kern-0.5em{\scshape i\kern-0.25em b}\kern-0.8em\TeX}}}


% \renewcommand\footnotetextcopyrightpermission[1]{}
\setcopyright{acmcopyright}
\settopmatter{printacmref=true, printccs=true, printfolios=false}
% Rights management information.  This information is sent to you
% when you complete the rights form.  These commands have SAMPLE
% values in them; it is your responsibility as an author to replace
% the commands and values with those provided to you when you
% complete the rights form.
% \setcopyright{acmcopyright}
% \copyrightyear{2018}
% \acmYear{2018}
% \acmDOI{XXXXXXX.XXXXXXX}
% \renewcommand{\baselinestretch}{0.95}


% These commands are for a PROCEEDINGS abstract or paper.
% \acmConference[Conference acronym 'XX]{Make sure to enter the correct
%   conference title from your rights confirmation emai}{June 03--05,
%   2018}{Woodstock, NY}

 % Uncomment \acmBooktitle if th title of the proceedings is different
 % from ``Proceedings of ...''!

% \acmBooktitle{Woodstock '18: ACM Symposium on Neural Gaze Detection,
%  June 03--05, 2018, Woodstock, NY} 
% \acmPrice{15.00}
% \acmISBN{978-1-4503-XXXX-X/18/06}
% \renewcommand{\baselinestretch}{.95} % line space
% Systems (SenSys '22), November 6--9, 2022, Boston, MA, USA}

\copyrightyear{2025}
\acmYear{2025}
\setcopyright{rightsretained}
\acmConference[ACM SenSys '25]{The 23th ACM Conference on Embedded Networked Sensor Systems}{May 6-9, 2025}{Irvine, USA}
\acmBooktitle{The 23th ACM Conference on Embedded Networked Sensor Systems (SenSys '25), May 6-9, 2025, Irvine, USA}
\acmDOI{xx.xxxx/xxxxxxx.xxxxxxx}
\acmISBN{xxx-x-xxxx-xxxx-x/xx/xx}
%%
%% Submission ID.
%% Use this when submitting an article to a sponsored event. You'll
%% receive a unique submission ID from the organizers
%% of the event, and this ID should be used as the parameter to this command. 
%%\acmSubmissionID{123-A56-BU3}

%%
%% The majority of ACM publications use numbered citations and
%% references.  The command \citestyle{authoryear} switches to the
%% "author year" style.
%%
%% If you are preparing content for an event
%% sponsored by ACM SIGGRAPH, you must use the "author year" style of
%% citations and references.
%% Uncommenting
%% the next command will enable that style.
%%\citestyle{acmauthoryear}

%%
%% end of the preamble, start of the body of the document source.
\usepackage[ruled,linesnumbered]{algorithm2e}
\usepackage{subfigure}
\usepackage{multicol}
% \usepackage{subcaption}
\setlength\columnsep{0.33in}
% \renewcommand{\baselinestretch}{0.97}
% \usepackage{algorithm}
\usepackage{algorithmic}
\usepackage{caption}
\captionsetup{labelfont=bf}
\usepackage{bbding}
\usepackage{pifont}
\usepackage{graphicx} % 引入 graphicx 包
\usepackage{url}
\usepackage{gensymb}
% \usepackage[colorlinks,linkcolor=blue]{hyperref}


%=== Editing tools ============
\ifodd 1
\newcommand{\rev}[1]{{\color{blue}#1}} %revise of the text
\newcommand{\com}[1]{\textbf{\color{red}(COMMENT: #1)}} %comment of the 
\else
\newcommand{\rev}[1]{#1}
\newcommand{\com}[1]{}
\fi

\newcommand{\term}[1]{{\it #1}}
\newcommand{\empha}[1]{{\bf #1}}
\newcommand{\head}[1]{{\bf #1}}
\newcommand{\scare}[1]{`#1'}
\newcommand{\code}[1]{{\tt #1}}
\newcommand{\lnote}[1]{Lognfei: {\color{red}{#1}}}
\newcommand{\notsure}[1]{{\color[rgb]{0.7,0.7,0.7}{#1}}}
\newcommand{\todo}[1]{{\color{red}{\textbf{TODO:} #1}}}
\newcommand{\notice}[1]{{\color{red}{#1}}}
\newcommand{\bsymb}[1]{{\boldsymbol #1}}
\newcommand{\tocite}{\rev{[?]}}
\newcommand{\toref}{\rev{?}}
% \newcommand{\revise}[1]{{\color{blue}{#1}}}
\newcommand{\revise}[1]{{\color{black}{#1}}}

\def\fig{Fig.}
\def\sec{Section}
\def\tab{Table}
\def\eg{e.g.}
\def\ie{i.e.}
\def\aka{a.k.a.}
\def\eqn{Eqn.}
\def\eq{Eq.}
\def\alg{Algorithm}
\def\etc{etc.}

% \copyrightyear{2022}
% \acmYear{2022}
% \setcopyright{rightsretained}
% \acmConference[SenSys '22]{The 20th ACM Conference on Embedded
% Networked Sensor Systems}{November 6--9, 2022}{Boston, MA, USA}
% \acmBooktitle{The 20th ACM Conference on Embedded Networked Sensor
% Systems (SenSys '22), November 6--9, 2022, Boston, MA, USA}
% \acmDOI{10.1145/3560905.3568432}
% \acmISBN{978-1-4503-9886-2/22/11}

\begin{document}
% \begin{figure*}[t]
%     \setlength{\abovecaptionskip}{-0cm} % height above Figure X caption
%     \setlength{\belowcaptionskip}{-0.24cm}
%     \setlength{\subfigcapskip}{-0.25cm}
%     \centering
%         \includegraphics[width=1.\columnwidth]{Figs/introall.png}
%         % \vspace{-0.2cm}
%     \caption{Illustration of the airport and landing of the delivery drone. \textnormal{(a) The self-collection airport is equipped with multiple delivery drones for package distribution. (b) A delivery drone descends onto the landing platform.}}
%     \label{intro}
%     % \vspace{-0.2cm}
% \end{figure*} 

% \let\oldtwocolumn\twocolumn
% \renewcommand\twocolumn[1][]{%
%     \oldtwocolumn[{#1}{
%     \begin{center}
%     \vspace{-0.5cm}
%     \setlength{\belowcaptionskip}{0.24cm}
%     \setlength{\abovecaptionskip}{-0.cm}
%            \includegraphics[width=2.1\columnwidth]{Figs/introall.png}
%            \captionof{figure}{Illustration of airport, landing of the delivery drone, sensors and algorithms performance. \textnormal{(a) The self-collection airport is equipped with multiple delivery drones for package distribution. (b) A delivery drone lands onto the platform. (c) Enhancing mmWave radar with event camera could have high spatial-temporal resolution and depth sensing capability. (d) Field study system performance comparison: green circles represent frame camera-based solutions, purple for mmWave radar-based, blue for radar-camera fusion, and red for event camera-based solutions.}}
%            \label{intro}
%         \end{center}
%     }]
% }


% \thispagestyle{empty} % no page number for the first page
% \pagestyle{empty}  % no page number for the second and the later pages

\newcommand{\mycustomsize}{\fontsize{21}{\baselineskip}\selectfont}
% \newcommand{\mycustomsize}{\fontsize{20}{\baselineskip}\selectfont}

% \title[EventLoc]{\mycustomsize{EventLoc: Unleashing the Potential of Event Camera in Low Latency 3D Object Localization with mmWave Radar}}
% \title[mmE-Loc]{\mycustomsize{High-Resolution Temporal Consistency Matters: Enhancing mmWave Radar with Event Camera for Accurate Drone Landing }}
% \title[mmE-Loc]{\mycustomsize{Ultra-High-Frequency Harmony: mmWave Radar and \\ Event Camera Orchestrate Accurate Drone Landing}}
\title[Ultra-High-Frequency Harmony: \\mmWave Radar and Event Camera Orchestrate Accurate Drone Landing]
{Ultra-High-Frequency Harmony: mmWave Radar and \\ Event Camera Orchestrate Accurate Drone Landing }
% 1. EventLoc: Enhancing mmWave Radars with Event Cameras for Low Latency Landing Drone Ground Localization 
% 2. mmEvent: Enhancing mmWave Radars with Event Cameras for Low Latency Landing Drone Ground Localization 
% 3. mmE-Loc: Enhancing mmWave Radars with Event Cameras for Low Latency Landing Drone Ground Localization 
% 4. FLASH:Enhancing mmWave Radars with Event Cameras for Low Latency Landing Drone Ground Localization 
% 5. mmE-$L^4$oc:Enhancing mmWave Radars with Event Cameras for Low Latency Landing Drone Ground Localization 
% 6. FLARE:Enhancing mmWave Radars with Event Cameras for Fast Landing Drone Ground Localization 
% 7. EventNest:Enhancing mmWave Radars with Event Cameras for Fast Landing Drone Ground Localization  

\author{Haoyang Wang$^{1}$, Jingao Xu$^{2}$, Xinyu Luo$^{1}$, Xuecheng Chen$^{1}$, Ting Zhang$^{1}$, \\ Ruiyang Duan$^3$, Yunhao Liu$^{4}$, Xinlei Chen$^{1, 5, 6}$\textsuperscript{\Envelope}}

\renewcommand{\authors}{Haoyang Wang, Jingao Xu, Xinyu Luo, Xuecheng Chen, Ting Zhang, Ruiyang Duan, Yunhao Liu, Xinlei Chen}

\affiliation{%
    \institution{$^1$ Shenzhen International Graduate School, Tsinghua University, China; $^2$ Carnegie Mellon University; \\$^3$ Meituan Academy of Robotics Shenzhen, China; $^4$ School of Software, Tsinghua University, China; \\$^5$ Pengcheng Laboratory, Shenzhen, China; $^6$ RISC-V International Open Source Laboratory, Shenzhen, China}
    \country{}
    \city{}
}

% \affiliation{%
%  \institution{$^1$ Shenzhen International Graduate School, Tsinghua University, China; $^2$ Carnegie Mellon University; $^3$ Meituan Technology, Shenzhen, China; $^4$ School of Software, Tsinghua University, China; $^5$ Pengcheng Laboratory, Shenzhen, China; $^6$ RISC-V International Open Source Laboratory, Shenzhen, China}
% }

\affiliation{%
  \institution{Email: \{haoyang-22, luo-xy23, chenxc21, yunhao\}@mails.tsinghua.edu.cn, \{xujingao13, zhangt2112\}@gmail.com, \\ duanruiyang@meituan.com, chen.xinlei@sz.tsinghua.edu.cn}
}

\renewcommand{\shortauthors}{Haoyang Wang, et al.}

% mmE$L^4$oc

% mmE-Loc: Enhancing mmWave Radars with Event Cameras for Low Latency Landing Drone Ground Localization 

% Low Latency 
% 3D Localization
% Landing drone ground Localization

% Temporal Consistency Matters: Enhancing mmWave Radars with Event Cameras for Low Latency Landing Drone Ground Localization 

% Temporal Consistency Matters: Enhancing mmWave Radar with Event Camera for Accurate Drone Landing 

% High Temporal Resolution Consistency Matters: Enhancing mmWave Radar with Event Camera for Accurate Drone Landing 

% Baseline: 
% High-Resolution Temporal Consistency Matters: Enhancing mmWave Radar with Event Camera for Accurate Drone Landing 
% 缩写:mmE-Loc

% \title[EventLoc]{EventLoc: Enhancing Event Cameras for Low Latency 3D Object Localization with mmWave Radar Integration}
% EventLoc: Leveraging mmWave Radar for Enhanced Low Latency 3D Object Localization with Event Cameras
% EventLoc: Enhancing Low Latency 3D Object Localization in Event Cameras with mmWave Radar Integration
% Unleashing the Potential of Event Camera in 3D localization with mmWave Radar: A Case for Landing Drone Tracking

% \author{Anonymous}

% \authornote{Both authors contributed equally to this research.}
% \affiliation{%
%   \institution{Shenzhen International Graduate School, Tsinghua University}
%     \city{Shenzhen}
%   \country{China}
% }
% \email{haoyang-22@mails.tsinghua.edu.cn}

% \author{Xuecheng Chen}
% \authornotemark[1]
% \affiliation{%
%   \institution{Tsinghua-Berkeley Shenzhen Institute, Tsinghua University}
%   \city{Shenzhen}
%   \country{China}
% }
% \email{chenxc21@mails.tsinghua.edu.cn}

% \author{Yuhan Cheng}
% \affiliation{%
%   \institution{Shenzhen International Graduate School, Tsinghua University}
%   \city{Shenzhen}
%   \country{China}
% }
% \email{cyh22@mails.tsinghua.edu.cn}

% \author{Chenye Wu}
% \affiliation{%
%   \institution{The Chinese University of Hong Kong, Shenzhen}
%   \city{Shenzhen}
%   \country{China}
% }
% \email{chenyewu@yeah.net}

% \author{Fan Dang}
% \affiliation{%
%   \institution{Global Innovation Exchange, Tsinghua University}
%     \city{Beijing}
%   \country{China}
% }
% \email{dangfan@tsinghua.edu.cn}

% \author{Xinlei Chen}
% \authornote{Xinlei Chen is the corresponding author.}
% \affiliation{%
%   \institution{Shenzhen International Graduate School, Tsinghua University}
%    \institution{Peng Cheng Laboratory}
%     \city{Shenzhen}
%   \country{China}
% }
% \email{chen.xinlei@sz.tsinghua.edu.cn}


%%
%% By default, the full list of authors will be used in the page
%% headers. Often, this list is too long, and will overlap
%% other information printed in the page headers. This command allows
%% the author to define a more concise list
%% of authors' names for this purpose.

% \renewcommand{\shortauthors}{Wang, et al.}



\begin{abstract}

For precise, efficient, and safe drone landings, ground platforms should real-time, accurately locate descending drones and guide them to designated spots.
% While mmWave sensing combined with cameras improves localization accuracy, 
% % frame cameras introduce latency and throughput bottlenecks that limit location update rates.
% the lower sampling frequencies of traditional frame cameras compared to mmWave radar introduce localization latency and create bottlenecks in system throughput.
While mmWave sensing combined with cameras improves localization accuracy, the lower sampling frequency of traditional frame cameras compared to mmWave radar creates bottlenecks in system throughput. 
In this work, we replace the traditional frame camera with event camera, a novel sensor that harmonizes in sampling frequency with mmWave radar within the ground platform setup, and introduce mmE-Loc, a high-precision, low-latency ground localization system designed for drone landings.
To fully leverage the \textit{temporal consistency} and \textit{spatial complementarity} between these modalities, we propose two innovative modules, \textit{consistency-instructed collaborative tracking} and \textit{graph-informed adaptive joint optimization}, for accurate drone measurement extraction and efficient sensor fusion.
% We propose two novel algorithms: Consistency-Instructed Collaborative Tracking and Graph-Informed Adaptive Joint Optimization, which leverage \textit{temporal consistency} and \textit{spatial complementarity} between modalities for accurate drone measurements extraction and efficient sensor fusion.
% Extensive real-world experiments in landing scenarios from, conducted with a leading drone delivery company demonstrate that mmE-Loc outperforms state-of-the-art methods in both localization accuracy and latency.
Extensive real-world experiments in landing scenarios from a leading drone delivery company demonstrate that mmE-Loc outperforms state-of-the-art methods in both localization accuracy and latency. 
\end{abstract}

\begin{CCSXML}
<ccs2012>
   <concept>
       <concept_id>10010520.10010553.10003238</concept_id>
       <concept_desc>Computer systems organization~Sensor networks</concept_desc>
       <concept_significance>500</concept_significance>
       </concept>
   <concept>
       <concept_id>10010147.10010178.10010199.10010201</concept_id>
       <concept_desc>Computing methodologies~Planning under uncertainty</concept_desc>
       <concept_significance>500</concept_significance>
       </concept>
 </ccs2012>
\end{CCSXML}

\ccsdesc[500]{Computer systems organization~Embedded systems}
\ccsdesc[500]{Information systems~Location based services}
% \ccsdesc[500]{Computing methodologies~Planning under uncertainty}

%
% Keywords. The author(s) should pick words that accurately describe
% the work being presented. Separate the keywords with commas.
\keywords{Drone Ground Localization; Event Camera; mmWave Radar}

\maketitle 

\renewcommand{\thefootnote}{}
\footnotetext{\Envelope\ Corresponding author.}
\footnotetext{Project homepage: \href{https://mmE-Loc.github.io/}{\color{blue}{https://mmE-Loc.github.io/}}}

\vspace{-0.8cm}
\section{Introduction}

Projected to soar to a \$1 trillion market by 2040 \cite{low_altitude_eco}, the drone-driven low-altitude economy is transforming sectors with revolutionary applications such as on-demand delivery \cite{wang2022micnest, chen2024ddl, chen2022deliversense}, meticulous industrial inspections \cite{xu2022swarmmap, li2024quest, xu2019vehicle, liu2024mobiair}, and rapid relief-and-rescue \cite{zhang2023rf, chi2022wi, chen2024soscheduler}. 
Of paramount importance within this burgeoning sector is the \textit{landing phase}, where ground platforms locate drones descending from below 10 meters and guide them to accurately land at designated spots (\fig\ref{intro}a) \cite{he2023acoustic, sun2022aim}.
Situated near populated and commercial zones, these operations emphasize safety and reliability: our research with a leading drone delivery company reveals that a landing bias of just 10$cm$ will result in drones damaging delivery targets or missing their charging ports \cite{gonzalez2021visual}. 
Such inaccuracies disrupt the operational efficiency of this swiftly growing economic sector, with potentially severe consequences.

\begin{figure}[t]
    \setlength{\abovecaptionskip}{0.25cm} % height above Figure X caption
    \setlength{\belowcaptionskip}{-0.3cm}
    \setlength{\subfigcapskip}{-0.5cm}
    \centering
        \includegraphics[width=0.92\columnwidth]{Figs/intro_v.png}
        \vspace{-0.2cm}
    \caption{Snapshot of drone landing phase, airport, and sensors performance. \textnormal{(a) A delivery drone lands on the platform. (b) The real-world drone airport is equipped with multiple drones for package delivery. 
    (c) Integrating mmWave radar with event camera combines reliable depth sensing and 2D imaging at ultra-high sampling frequencies, enabling high spatial-temporal resolution and depth sensing, while ensuring compatibility with flight controllers.}}
    \label{intro}
\end{figure} 

Widely adopted and straightforward approaches involve installing cameras at the center or edges of drone landing pads and employing computer vision algorithms for drone localization \cite{li2018real,xia2023anemoi,zhang2019eye}.
However, traditional frame cameras' Achilles heel is capturing only 2D images without depth information, leading to scale uncertainty that limits the 3D localization accuracy \cite{zhang2022mobidepth,xie2023mozart,10.1145/3517260}.
To address this shortcoming, current practices have incorporated mmWave sensing to provide the lacked depth information for better localization accuracy and reliability in various conditions \cite{deng2022geryon, lu2020smokerobustindoormapping,zhang2023mmhawkeye, sie2023batmobility, iizuka2023millisign, lu2020see, lu2020milliego}.

Albeit inspiring, our benchmark study with a world-leading drone delivery company in landing scenarios (\fig \ref{intro}b) reveals another critical drawback (\fig \ref{intro}c): the exposure times of frame cameras (>20$ms$) prevent their sampling rates from matching the high frequency of mmWave radars (\eg, 200Hz). 
This limitation creates system efficiency and throughput bottlenecks, restricting drone location updates to below 50Hz.
In contrast, drone flight controllers typically require location input rates over 150Hz to precisely adjust the drone's flight attitude for safe landing \cite{8412592, 10.1145/3570361.3592532}.
The inefficiency originates from the inherent physical limitations of conventional frame cameras and cannot be easily solved by software solutions.


\noindent \textbf{Upgrade frame camera to event camera.}
Event cameras are bio-inspired sensors that report pixel-wise intensity changes with $ms$-level resolution \cite{gallego2020event, ruan2024distill}, capturing high-speed motions without blurring \cite{he2024microsaccade}, ideal for fast-tracking tasks \cite{xu2023taming, luo2024eventtracker}.
Event cameras offer $ms$-level sampling latency, which harmonizes exceptionally with the high sampling frequencies of mmWave radar.
Their 2D imaging capability also complements radars' limited spatial resolution, similar to how traditional frame cameras operate.
Such \textit{temporal-consistency} and \textit{spatial-complementarity} across both modalities inspire us to upgrade frame cameras with event cameras to pair with radar for accurate and fast drone localization.


\noindent \textbf{Our work.}
Following the above insight, we present \textbf{mmE-Loc}, the first active, high-precision, and low-latency landing drone ground localization system that enhances mmWave radar functionality with event cameras. 
mmE-Loc works in scenarios where urban canyon environments degrade the accuracy of GPS or RTK systems as altitude decreases, rendering them nearly ineffective for the landing phase. 
With mmE-Loc, drones achieve reliable localization even under challenging conditions (\eg, weak illumination), ensuring stable and efficient landing.

However, our benchmark study at a real-world drone delivery airport (\fig \ref{relatedwork}a) highlights several challenges that have been solved in making \textbf{mmE-Loc} a viable system outdoors:
$(i)$ \textit{How to accurately extract drone-related measurements} given the immense noisy output of event cameras and mmWave radars, which also lack inherent drone semantic information and differ greatly in dimension and pattern?
Both modalities are sensitive to environmental variations (\eg, changes in lighting conditions), as shown in \fig \ref{relatedwork}b.
Existing algorithms \cite{cao2024virteach, liu2024pmtrack, wang2021asynchronous, alzugaray2018asynchronous} are typically designed for single-modality, resulting in low noise filtering rates (recall and precision < 65\% in \fig \ref{relatedwork}c).
$(ii)$ \textit{How to efficiently fuse event camera and mmWave readings} that are heterogeneous in measurement precision, scale, and density? 
Existing EKF (extended Kalman filter) or PF (particle filter) based approaches \cite{falanga2020dynamic,zhao20213d, mitrokhin2018event}, suffer from cumulative drift errors, making them insufficient for precise localization (\fig \ref{relatedwork}d).
$(iii)$ \textit{How to optimize the efficiency of the fusion algorithm} to achieve high-frequency drone ground localization, given the limited computational resources on landing platforms?
Existing methods experience significant processing delays, rendering them unsuitable for low-latency localization tasks (\fig \ref{relatedwork}d) \cite{zhao20213d, falanga2020dynamic, mitrokhin2018event}.

To solve the above challenges, the design and implementation of mmE-Loc excel in the three aspects of drone ground localization:

\noindent $\bullet$ \textit{On system architecture front.}
Upgrading frame camera to event camera with $ms$-level latency to pair the mmWave radar, mmE-Loc improves drone ground localization at data source.
The system architecture tightly integrates both modalities, from early-stage noise filtering and drone detection to later-stage fusion and optimization, fully leveraging the unique advantages of both sensors (§\ref{3.2}).\\
\noindent $\bullet$ \textit{On system algorithm front.}
We introduce a Consistency-Instructed Collaborative Tracking (\textit{CCT}) algorithm, which leverages the drone's periodic micro-motion and cross-modal \textit{temporal-consistency} to filter environment-triggered noise, achieving accurate drone detection (§\ref{4.1}). 
We then present a Graph-Informed Adaptive Joint Optimization (\textit{GAJO}) algorithm, which fuses \textit{spatial-complementarity} with a novel factor graph to boost drone ground localization, resulting in a trajectory with minimal bias and low cumulative drift(§\ref{4.2}). \\
\noindent $\bullet$ \textit{On system implementation front.}
We further analyze the sources of latency and propose an Adaptive Optimization method to improve the efficiency of the \textit{GAJO} algorithm.
This approach allows \textit{GAJO} to dynamically optimize a set of locations, maintaining accuracy while reducing latency (§\ref{I}).

\begin{figure}[t]
    \setlength{\belowcaptionskip}{-0.3cm}
    \setlength{\subfigcapskip}{-0.25cm}
    \centering
        \includegraphics[width=0.9\columnwidth]{Figs/relatedall_4.png}
        \vspace{-0.4cm}
    \caption{
    Benchmark study on drone localization.
    \textnormal{(a) Benchmark study at a real-world drone delivery airport; (b) Both sensors are sensitive to environmental variations; (c) Existing algorithms suffer from low noise filtering rates; (d) Existing algorithms experience cumulative drift errors and delays. }}
    \label{relatedwork}
    \vspace{-0.5cm}
\end{figure} 

\begin{figure*}[t]
    \setlength{\abovecaptionskip}{0.2cm} % height above Figure X caption
    \setlength{\belowcaptionskip}{-0.5cm}
    \setlength{\subfigcapskip}{-0.25cm}
    \centering
        \includegraphics[width=1.9\columnwidth]{Figs/overview.png}
        \vspace{-0.1cm}
    \caption{System architecture of mmE-Loc.}
    \label{overview}
    % \vspace{-0.2cm}
\end{figure*} 


We fully implement mmE-Loc using a COTS event camera and mmWave radar. 
Over 30+ hours of indoor and outdoor experiments under various drone flight conditions assess its localization accuracy and end-to-end latency performance against four SOTA methods.
mmE-Loc achieves an average localization accuracy of 0.083$m$ and latency of 5.15$ms$, surpassing baselines by >48\% and >62\%, respectively, and showing minimal sensitivity to drone type and environment.
We also deploy mmE-Loc at a real-world drone delivery airport (\fig \ref{relatedwork}a) for 10 hours, demonstrating its practicality for commercial-level drone landing requirements. 

In summary, this paper makes the following contributions.\\
\noindent $(1)$ We explore a fresh sensor configuration, event camera plus mmWave radar, that embraces and harmonizes ultra-high sampling frequencies and propose mmE-Loc, a ground localization system for drone landings that delivers $cm$ accuracy and $ms$ latency.\\
\noindent $(2)$ We present $CCT$, which leverages \textit{temporal consistency} and the drone's periodic micro-motions for precise drone detection; and $GAJO$, which employs \textit{spatial complementarity} with a novel factor graph to enhance drone localization.\\
\noindent $(3)$ We implement and extensively evaluate mmE-Loc by comparing it with four SOTA methods, showing its effectiveness. We also deploy mmE-Loc in a real-world drone delivery airport, demonstrating feasibility of mmE-Loc.

\section{System Overview}
\label{sec:overview}

In this section, we present the control system architecture of the proposed framework, shown in Fig. \ref{fig:controlArchi}. 
Empirically, humanoid kino-dynamics MPC explicitly optimizes the joint states through kinematics constraints \cite{gu2025humanoid}, while traditional centroidal-dynamics MPC often requires subsequent inverse kinematics solver or whole-body control for motion execution. Both approaches employ nonlinear approaches to solve the optimization problem. In our framework, we proposed a Gait-Net-augmented sequential CMPC algorithm that translates the original nonlinear problem into convex sequential subproblems. With the additional assistance of Gait-Net, we reduce the optimization variable and mimic a natural step duration decision in each iteration. 

The control framework converts user commands and contact sequence into joint space references $\{\mathbf q_k^\text{ref} \in \mathbb R^{6+n_j},\: \dot{\mathbf q}_k^\text{ref} \in \mathbb R^{6+n_j}\}^h_{k = 0}$ and foot location reference trajectory $\{\bm p_f^\text{ref}\in \mathbb R^{3n_i}\}^h_{k = 0}$, where $n_j$ is the number of joints, $n_i$ is the number of contact/foot, and $h$ is a finite number of horizon. These joint-space trajectories, along with joint-space feedback states, are then translated into spatial momenta $\bm h\in \mathbb R^6$ and their primitive, the centroidal pose $\bm H\in \mathbb R^6$, which are the state variables used in the Gait-Net-augmented kino-dynamic MPC. Within the MPC, we break down the nonlinear dynamics constraints into sequential CMPC subproblems that can be solved through QP solvers. In each sequential iteration $j$, the Gait-Net predicts and updates the MPC sampling time $dt$ towards convergence and enables variable-frequency walking.
The spatial momentum and pose trajectories are updated at each iteration to reflect the kinematic configuration based on the iterative solution of $dt$, CoM location $\bm p_c \in \mathbb R^3$, and foot locations $\bm p_f\in \mathbb R^{3n_i}$,
providing a kinematically feasible reference. Once the terminal condition is met in the custom sequential solver, the control inputs are then mapped to motor commands in low-level control, which incorporates standard techniques such as inverse kinematics, contact Jacobian mapping, and joint-PD swing leg control \cite{di2018dynamic}. Notably, the full Gait-Net-augmented Kino-dynamic MPC is run at the beginning of each footstep to determine the step duration, the rest of the duration will incorporate the kino-dynamic MPC with the same MPC $dt$ throughout this very footstep. 


 

\vspace{-0.3cm}
\section{System Design} \label{4}
In this section, we introduce \textit{CCT}: Consistency-instructed Collaborative Tracking for noise filtering, detection, and preliminary localization of drone (§ \ref{4.1}). 
Subsequently, we delve into \textit{GAJO}: Graph-informed Adaptive Joint Optimization for fine localization and trajectory optimization of drone (§ \ref{4.2}).

\vspace{-0.2cm}
\subsection{\textit{CCT}: Consistency-instructed \\ Collaborative Tracking} \label{4.1}

% \noindent \textbf{Challenge.} 
% Events and mmWave samples contain noise. 
% Event camera asynchronously measures per-pixel brightness changes, often triggered by non-drone factors like shadows. 
% The mmWave radar suffers from signal multipath effects, causing erroneous point clouds. 
% The event camera captures per-pixel brightness changes asynchronously, which are frequently influenced by non-drone factors such as shadows. 
The mmWave radar is prone to signal multipath effects, leading to inaccurate point cloud data.
Meanwhile, the event camera captures per-pixel brightness changes asynchronously, which are frequently influenced by non-drone factors such as shadows.  
% However, the lack of inherent drone semantic information and great difference in dimension and pattern of these two modalities pose challenges to noise filtering, leading to detection bottlenecks and further diminishing the efficiency and precision of localization.
However, the absence of intrinsic drone semantic information, combined with significant differences in dimension and patterns between these two modalities, presents challenges for noise filtering. 
This results in drone detection bottlenecks, which further reduce the efficiency and accuracy of localization.
Therefore, in this part, we focus on enhancing noise filtering and drone detection, while providing preliminary localization of the drone.
% Millisecond latency amplify this phenomenon.
% Millisecond latency adds complexity.
% Previous methods filter noise separately but fall short in achieving low-latency, accurate noise reduction for both sensors. 
% These elements pose challenges to noise filtering, leading to detection bottlenecks and diminishing the efficiency and precision of localization.


% \textit{How to accurately extract drone-related measurements} given the immense noisy output of event cameras and mmWave radars, which also lack inherent drone semantic information and differ greatly in dimension and pattern?
% Both modalities are sensitive to environmental variations (\eg, changes in lighting conditions).

To address this challenge, we explore the operational principles of both sensors. 
Our design is based on observations: \textit{(i) Event camera and mmWave radar demonstrate temporal consistency and distinct response mechanisms.}
% Event camera and mmWave radar are consistent in $ms-level$ latency.
% Meanwhile, event camera are not affected by multipath effects, while mmWave radars are immune to changes in brightness. 
Event camera and mmWave radar maintain $ms$-level latency.
Additionally, event cameras are unaffected by multipath effects, whereas mmWave radar remains impervious to changes in brightness.
\textit{(ii) Drone exhibits periodic micro motion features (\eg, propeller rotation),} which can serve as stable and distinctive features of drone.
% These enable efficient cross-modal noise filtering and drone detection by aligning event camera and mmWave radar measurements to extract drone-specific data.
These facilitate efficient cross-modal noise filtering by aligning measurements from the event camera and mmWave radar and enable drone detection by extracting drone measurements through periodic micro-motions.
% These enable efficient cross-modal noise filtering and detection by aligning event camera and mmWave radar measurements and extracting drone-specific data with periodic micro motion.

% These enable efficient cross-modal noise filtering by aligning event camera and mmWave radar measurements, and detection by extracting drone-specific data with periodic micro motion.
% These enable a light-weight cross-modal noise filtering and drone detection by aligning event camera and mmWave radar measurements, and extract drone-related measurements.

% This enables a light-weight map synchronization by avoiding transferring massive map-point data and the bulky geographical descriptors such as their spatial locations, features, observation relationships with keyframes

% To address this challenge, we explore the operational principles of both sensors. 
% Our design is based on the observation: \textit{both the event camera and mmWave radar respond to dynamic objects, albeit through different mechanisms.}
% The events are triggered by brightness changes, while radar generates point clouds from frequency differences. 
% Event cameras are not affected by multipath effects, while mmWave radars are immune to changes in brightness. 
% Moreover, moving objects cause both brightness variations and frequency differences, eliciting responses from both modalities.
% This allows cross-modal noise filtering by employing \textit{consistent information} of both modalities.

% Compared to current methods, our design leverages the advantages of both sensors, which achieves efficient noise filtering, enabling detection and rough localization of the drone.
\begin{figure}[t]
    \setlength{\abovecaptionskip}{0.2cm} % height above Figure X caption
    \setlength{\belowcaptionskip}{-0.3cm}
    \setlength{\subfigcapskip}{-0.4cm}
    \centering
        \includegraphics[width=0.85\columnwidth]{Figs/event.png}
        % \vspace{-0.18cm}
    \caption{Illustration of synchronous frames and asynchronous events. \textnormal{Frame cameras use a global shutter to capture images at fixed intervals, while each pixel in an event camera responds independently, generating events asynchronously when intensity changes exceed a threshold.}}
    \label{event}
    \vspace{-0.4cm}
\end{figure} 

\begin{figure*}[t]
    \setlength{\abovecaptionskip}{0.4cm} % height above Figure X caption
    \setlength{\belowcaptionskip}{-0.34cm}
    \setlength{\subfigcapskip}{-0.25cm}
    \centering
        \includegraphics[width=2\columnwidth]{Figs/performance.png}
        \vspace{-0.28cm}
    \caption{Step-by-step filtering performance. \textnormal{The \textit{CCT} module in mmE-Loc eliminate noise events, mmWave point cloud and erroneous detection by employing \textit{temporal-consistency} of both modalities.}}
    \label{performance}
    \vspace{-0.2cm}
\end{figure*}

% \vspace{-0.38cm}
To realize this idea, we design \textit{CCT}, a lightweight cross-modal drone detector and tracker.
% optimized for efficient noise filtering, drone detection, and preliminary ground localization of drones. 
\textit{CCT} includes several components:
$(i)$ a Radar Tracking Model (§\ref{4.1.1}) providing sparse point cloud indicating distance and direction information of objects;
$(ii)$ an Event Tracking Model (§\ref{4.1.2}) for event filtering, detection, and tracking of objects;
% $(iii)$ a Consistency-Instructed Measurements Filter (§\ref{4.1.3}) leverages periodic micro motion feature of the drone and consistency of both modalities to efficiently eliminate erroneous detections and point cloud, enabling rough localization of the drone.
$(iii)$ a Consistency-instructed Measurements Filter (§\ref{4.1.3}) utilizes temporal consistency between both modalities and drone's periodic micro-motion features to extract detection and point cloud of drone, facilitating the preliminary localization.

\subsubsection{\textbf{Radar Tracking Model}} \label{4.1.1}
In this part, we calculate the distance $D$ and direction vector $\vec{v}$ between the radar and objects, along with a preliminary estimation of the object's location, as depicted in \fig \ref{CCT}a and \fig \ref{CCT}b.

% \noindent 
\textbf{Distance calculation.} 
% The difference in frequency between the transmitted signal (TX signal) and received signal (RX signal) reflects the signal propagation time, providing insight into the distance between object and radar.
As shown in \fig \ref{CCT}a, the frequency difference between the transmitted (TX) and received (RX) signals indicates the signal propagation time, revealing the distance between the object and the radar.
Denoting $D^i$ as the distance at time $i$, TX and RX signals as:
\begin{equation}
\begin{aligned}
S_{TX}^i\!=\!\exp \left[j\left(2 \pi f_c i+\pi K i^2\right)\right], 
S_{RX}^i\!=\!\alpha S_{TX}\left[i-2D^i/{c}\right],
\end{aligned}
\end{equation}
where $\alpha$ denotes the attenuation rate, $f_c$ is the initial frequency, $K$ represents the chirp slope of the FMCW signal, and $c$ stands for speed of light.
The TX and RX signals undergo mixing and low-pass filter (LPF) to extract intermediate frequency signal (IF signal) $s(t)$: 
% given as:
\begin{equation}
S_{IF}^i=LPF(S_{TX}^{i*} S_{RX}^{i}) \approx \alpha \exp \left[j 2 \pi\left(2KD^i/c\right)i\right].
\end{equation}
The frequency value $f_{IF}$ within $S_{IF}^i$ encapsulates distance information. 
After the Range-FFT operation $S_{IF}^i$, $f_{IF}$ is extracted, facilitating distance calculation $D^i=c f_{IF} / 2K$.


% \noindent
\textbf{Direction calculation.}
% With a fixed antenna array, mmWave radar determines the object's direction using two orthogonal linear antenna arrays. 
% As shown in \fig \ref{CCT}c, each linear array captures an Angle of Arrival (AoA), calculated from the phase difference between adjacent antennas spaced apart by $d$ as $cos \theta = \Delta \phi \lambda/2 \pi d$, where $\theta$ is AoA, $\lambda$ is the wavelength, and $\Delta \phi$ is the phase difference. 
% Having two orthogonal arrays allows the radar to obtain two AoAs, $\theta_x$ and $\theta_y$. The unit vector indicating the object’s direction at time $i$ is then given by:
Using a fixed antenna array, the mmWave radar determines the object's direction by employing two orthogonal linear arrays. 
As depicted in \fig \ref{CCT}b, each linear array captures an Angle of Arrival (AoA), calculated from the phase difference between adjacent antennas spaced apart by $d$ as $cos \theta = \Delta \phi \lambda/2 \pi d$, where $\theta$ represents AoA, $\lambda$ denotes the wavelength and $\Delta \phi$ indicates the phase difference. 
With two orthogonal arrays, the radar obtains two AoAs, $\theta_x$ and $\theta_y$. The unit vector indicating the object's direction at time $i$ is given by
$\vec{v}^i=[\cos \theta_x \cos \theta_y \sqrt{1-\cos ^2 \theta_x-\cos ^2 \theta_y}]^{\mathrm{T}}$.

% \noindent \textbf{Preliminary estimation.}
Using the distance and angle information obtained above, along with the spatial relationship between radar and event camera, we can determine the preliminary 3D location estimation of the object in $\mathtt{E}$ as $P_E = D\vec{v}+t_{ER}$.
% , as depicted in \fig \ref{performance}d.
% We t the 3D location of the object at each timestamp for object tracking and compute the translation $t_{\mathtt{EO}}$ of the object from $\mathtt{O}$ to $\mathtt{E}$ at each timestamp as:
We then leverage the mmWave radar for object 3D location tracking, estimating the translation $t_{\mathtt{EO}}$ of the object from $\mathtt{O}$ to $\mathtt{E}$ at time $i$:
\begin{equation}
\begin{aligned}
\vspace{-0.2cm}
t_{\mathtt{EO}}^i & =t_{\mathtt{EO}}^{i-1}+U_{\mathtt{E}}^{i}+w^i + w^{i-1} \\
& =t_{\mathtt{EO}}^{i-1}+\left(P_{\mathtt{E}}^i-P_{\mathtt{E}}^{i-1}\right)+w^i + w^{i-1}.
\vspace{-0.2cm}
\end{aligned}
\end{equation}
$U_{\mathtt{E}}^{i}$ is discrepancy between two radar calculation results at times $i$ and ${i-1}$ in $\mathtt{E}$. $w_i$ and $w_{i-1}$ signify the measurement noise.

% Althourgh the mmWave radars have the  capability to excel in accurately estimating the depth of objects along the radial direction, they struggle to precisely capture motion in the tangential direction, which encompasses horizontal and vertical movements. 
% 为了解决该问题,我们引入event cameras,with similar latency to the mmWave radars,but uses a completely different sensing principle,以高空间分辨率 detect 无人机,弥补 mmWave radar 在tangential direction方向上的不足。
\revise{
While mmWave radars excel at estimating object depth along the radial direction, they struggle to accurately capture horizontal and vertical (tangential) motion \cite{qian20203d, zhang2023push}.
To address this issue, we introduce the event camera, which has similar latency but a different sensing principle. 
With high spatial resolution, the event camera detects objects and compensates for mmWave radars' limitations in the tangential direction.
}

% \vspace{-0.4cm}
\subsubsection{\textbf{Event Tracking Model}} \label{4.1.2}
% Compare with frame cameras which use a global shutter to capture images at fixed intervals, event cameras report pixel-wise intensity changes with $ms$-level resolution and $ms$-level sampling latency, capturing high-speed motions without blurring (\fig \ref{event}).
In this part, we demonstrate the process of noise filtering from a stream of asynchronous events, and how to detect and track objects with the filtered events, as depicted in \fig \ref{CCT}c.
Compared to frame cameras that use a global shutter to capture images at fixed intervals, event cameras record pixel-wise intensity changes with $ms$-level resolution and sampling latency, enabling high-speed motion capture without blurring but adding complexity to noise filtering and object detection (Fig. \ref{event}).

% enabling the capture of high-speed motion without blurring, but add complexity to noise filtering, detect and track objects (\fig \ref{event}).
% estimate objects' states based on the detection and tracking results, 

% \noindent 
\textbf{Similarity-informed event filtering.}
Event cameras are prone to noise from transistor circuits and other non-idealities, requiring pre-processing filtering. 
For the $i^{th}$ event $e^i_{(x, y)}$ with the timestamp $t^i_{(x, y)}$, we assess the timestamp ($t^i_{n(x, y)}$) of the most recent neighboring event in all directions. 
Events with a time difference less than the threshold $T_n$ are retained, indicating object activity, while those exceeding it are discarded as noise (\fig \ref{performance}b, \fig \ref{performance}c).
\revise{
We utilize the Surface of Active Events (SAE) \cite{lin2020efficient} to manage events, mapping coordinates $(x, y)$ to timestamps $(t_l, t_r)$.
Upon a new event's arrival, $t_l$ updates accordingly, and $t_r$ updates only if the previous event at the same location occurred outside the time window $T_k$ or had a different polarity. 
Events that update value of $t_r$ are retained.
The event stream, segregated by polarity, is processed with distinct SAEs. 
This method ensures precise spatial-temporal representation, reducing events and conserving computational resources.
}

% Event cameras are prone to noise due to transistor circuit noise, and non-idealities, \etc. Implementing a pre-processing filtering block is essential to mitigate these effects.  
% Initially, for the $i^{th}$ event $e^i_{(x, y)}$ at coordinates $(x, y)$ with timestamp $t^i_{(x, y)}$, we assess the timestamp ($t^i_{n(x, y)}$) of the most recent neighboring event $e^i_{n(x, y)}$ in all directions (horizontal, vertical, and diagonal). 
% Events with a time difference less than the threshold $T_n$ are retained, indicating a connection to neighborhood activity and likely to the object. 
% Events exceeding this threshold are likely noise and discarded.
% To manage events efficiently, we employ a Surface of Active Events (SAE), summarizing the event stream. 
% The SAE $\mathcal{S}$ maps coordinates $(x, y)$ to timestamps $(t_r, t_l)$. 
% The event stream is segregated by polarity, processed separately with distinct SAEs. 
% Upon a new event arrival at time $t$, $t_l$ updates accordingly, while $t_r$ updates only if the previous event at the same location occurred outside the time window $\mathtt{k}$ or had a different polarity. 
% This method ensures precise spatial and temporal representation while drastically reducing the event stream by eliminating redundancies, thus conserving computational resources.

% Event cameras, like other vision sensors, are prone to noise due to shot noise in photons, transistor circuit noise, and non-idealities. Implementing a pre-processing filtering block is essential to mitigate these effects. 
% Initially, for the $i^{th}$ event $e^i_{(x, y)}$ at coordinates $(x, y)$ with timestamp $t^i_{(x, y)}$, we assess the timestamp ($t^i_{n(x, y)}$) of the most recent neighboring event $e^i_{n(x, y)}$ in all directions (horizontal, vertical, and diagonal). 
% Events with a time difference less than the threshold $T_n$ are retained, indicating a connection to neighborhood activity and likely to the drone. 
% Events exceeding this threshold are likely noise and discarded.

% We further analyze previous events to determine if $e^i_{(x, y)}$ is linked to the drone.
% To efficiently manage past events, we utilize a Surface of Active Events (SAE) \tocite, summarizing the event stream. 
% The SAE $\mathcal{S}$ is defined as $\mathcal{S}: (x, y) \in \mathbb{R}^2 \mapsto (t_r, t_l) \in \mathbb{R}^2$, where $t_l$ is the timestamp of the latest event, and $t_r$ is the reference time.
% We segregate the event stream by polarity, processing each set separately with distinct SAEs. 
% Upon arrival of a new event at time $t$, $t_l$ updates to $t$, while $t_r$ updates only if the previous event at the same location occurred outside the time window $\mathtt{k}$ or had a different polarity. 
% Events that update the value of $t_r$ are considered in the subsequent algorithm.
% This method ensures precise spatial and temporal representation while drastically reducing the event stream by eliminating redundancies, thus conserving computational resources.

% Event cameras, akin to other vision sensors, are susceptible to noise stemming from inherent shot noise in photons, transistor circuit noise, and non-idealities. Employing a pre-processing filtering block is crucial to mitigate these effects.
% Initially, for the $i^{th}$ event $e^i_{(x, y)}$ occurring at coordinates $(x, y)$ with timestamp $t^i_{(x, y)}$, we examine the timestamp ($t^i_{n(x, y)}$) of the most recent neighboring event $e^i_{n(x, y)}$ in all directions (horizontal, vertical, and diagonal). If the time difference between $t^i_{(x, y)}$ and $t^i_{n(x, y)}$ is less than the threshold $T_n$, the event is retained.
% Filtered events signify their association with neighborhood activity, implying a probable connection to the drone. 
% Events surpassing the threshold are likely noise and thus discarded.

% We further analyze previously triggered events in the stream to determine if $e^i_{(x, y)}$ is connected to the drone. To handle the large volume of past events efficiently, we utilize a Surface of Active Events (SAE) \tocite to summarize the event stream at any given moment. The SAE $\mathcal{S}$ is defined as $\mathcal{S}: (x, y) \in \mathbb{R}^2 \mapsto (t_r, t_l) \in \mathbb{R}^2$, where $t_l$ represents the timestamp of the latest event triggered at pixel location $(x, y)$, and $t_r$ is the reference time.
% We partition the event stream based on polarity and process each set independently using separate SAEs. 
% When a new event arrives at time $t$, the value of $t_l$ at that location is always updated ($t_l \leftarrow t$) in $\mathcal{S}$. However, the reference time $t_r$ is only updated if the previous event at the same location occurred outside the time window $\mathtt{k}$, indicated by $t_r \leftarrow t$ if $t > t_l + \mathtt{k}$, or if the polarity of the latest event differs from that of the incoming one. Events that update the value of $t_r$ in $\mathcal{S}$ are considered in the subsequent algorithm.
% This approach ensures more accurate spatial and temporal representation of high contrast regions while significantly reducing the event stream by eliminating redundant events, thus saving computational resources.


% \noindent

\textbf{Filter-based detection and tracking.} 
We employ a grid-based method to cluster events to facilitate object detection. 
The camera's field of view is partitioned into elementary cells sized $c_w \times c_h$. 
For each cell, we compare the event count within a specified time interval ($c_{\Delta t}$) to an activation threshold $c_{thres}$. 
Cells surpassing $c_{thres}$ are marked as active and connected to form clusters, serving as object detection results, including those generated by the drone.
\revise{
For tracking, we deploy Kalman filter-based trackers with a constant velocity motion model, as the Kalman filter provides low-latency estimates with minimal computational cost.
% For tracking, we deploy the Kalman filter-based trackers built on a constant velocity motion model due to their simplicity and efficiency, as Kalman filter providing low-latency estimates with minimal computational cost.
A tracker predicts the state of the current object and associates it with the input cluster that has the largest Intersection Over the Union area. 
The input cluster corrects tracker state, generating bounding boxes, and effectively tracking moving objects, including the drone.
}

% We cluster events with a grid-based method to facilitate object detection. 
% Initially, we partition the camera's field of view (FOV) into elementary cells with a regular grid, each cell sized $c_w \times c_h$. 
% For each cell, we compare the event count within it over a specified time interval ($c_{\Delta t}$) to an activation threshold $c_{thres}$. Cells surpassing $c_{thres}$ are marked as active. 
% Active cells are then connected to form clusters, serving as detection results, including those generated by the drone.
% For object tracking, we deploy trackers based on a constant velocity motion model and Kalman filter. These trackers generate bounding box proposals from input clusters. 
% Initially, a tracker predicts the current object state using the constant velocity motion model. Upon receiving a cluster input, we associate it with the tracker exhibiting the largest Intersection Over Union (IOU) area compared to the cluster. 
% Finally, we use the input cluster as an observation state to update the current tracker state, and the trackers then generate bounding box proposals. 
% This method facilitates tracking of moving objects, including drones.
% In cases where no IOU association is feasible but the distance between the tracker and the cluster is less than $d_{iou}$, the cluster is linked to the nearest tracker. 

% \noindent \textbf{Preliminary estimation.}
\revise{
Using bounding box proposals and the pinhole camera model with projection function $\pi$, we estimate the preliminary 3D locations of objects.
Specifically, the projection function $\pi$ transforms a 3D point $\textbf{X}_\mathtt{E}$ in $\mathtt{E}$ into a 2D pixel $x$ in the image plane as: 
% $x\!=\!\pi\left(\textbf{X}_\mathtt{E}\right)\!=\![f_x X_\mathtt{E} / Z_\mathtt{E}+c_x,
% f_y Y_\mathtt{E} / Z_\mathtt{E}+c_y]^T, 
% \textbf{X}_\mathtt{E}\!=\![X_\mathtt{E}, Y_\mathtt{E}, Z_\mathtt{E} ]^T,$
\begin{equation}
x\!=\!\pi\left(\textbf{X}_\mathtt{E}\right)\!=\![f_x X_\mathtt{E} / Z_\mathtt{E}+c_x,
f_y Y_\mathtt{E} / Z_\mathtt{E}+c_y]^T, 
\textbf{X}_\mathtt{E}\!=\![X_\mathtt{E}, Y_\mathtt{E}, Z_\mathtt{E} ]^T,
\end{equation}
where $[f_x, f_y]^T$ is the focal length of the event camera, and $[c_x, c_y]^T$ denotes the principal point, both being intrinsic camera parameters. 
% The $Z_\mathtt{E}$ is measured by mmWave radar ($t_{EO}^i)$.
Then, the object's preliminary location at time $i$ is estimated using the center point of bounding box proposal $x^i$ as: 
% $x^i =\pi(\textbf{X}_\mathtt{E}^i)+v^i =\pi(\textbf{X}_\mathtt{O}^i+t_{\mathtt{EO}}^i)+v^i, $
\begin{equation}
x^i =\pi(\textbf{X}_\mathtt{E}^i)+v^i =\pi(\textbf{X}_\mathtt{O}^i+t_{\mathtt{EO}}^i)+v^i,
\end{equation}
% \begin{equation}
% \begin{aligned}
% x^i & =\pi\left(\textbf{X}_\mathtt{E}^i\right)+v^i =\pi\left(\textbf{X}_\mathtt{O}
% ^i+t_{\mathtt{EO}}^i\right)+v^i,
% \end{aligned}
% \end{equation}
% $t_{\mathtt{EO}}^i$ is mmWave radar measurement,
where $\textbf{X}_\mathtt{O}^i$ represents the corresponding 3D point of center point $x^i$ in the object reference $\mathtt{O}$, $v^i$ denotes the random noise.
% Eq. (3) and Eq. (5)
% of center point.
When extracting center points from bounding box proposals, we first undistort their coordinates. 
}

% However, the event camera encounter a scale uncertainty issue as it cannot determine the object's depth $Z_\mathtt{E}$ from a single central point of detection result. 
% The mmWave radar, which operates with similar latency to the event camera but uses a completely different sensing principle, generating sparse point clouds with depth information.

% 同时,mmWave radar 没有语义信息,无法直接从 objects measurements中分辨出 drone 相关的 measurements。


% Utilizing bounding box proposals and a conventional pinhole camera model with a projection function $\pi$, we can determine the preliminary 3D location estimation of the object. 
% $\pi$ transforms a 3D point $X_\mathtt{E}$ in $\mathtt{E}$, into a 2D pixel $x$ in image plane as:
% \begin{equation}
% \pi\left(X_\mathtt{E}\right)=\left[\begin{array}{l}
% f_x \frac{X_\mathtt{E}}{Z_\mathtt{E}}+c_x \\
% f_y \frac{Y_\mathtt{E}}{Z_\mathtt{E}}+c_y
% \end{array}\right], \quad X_\mathtt{E}=\left[X_\mathtt{E} \space Y_\mathtt{E} \space Z_\mathtt{E} \right]^T,
% \end{equation}
% where $\left[f_x, f_y\right]^T$ is the focal length of the event camera and $\left[c_x, c_y\right]^T$ is the principle point. They are the camera intrinsic parameters.
% When extracting the center points from bounding box proposals, we first undistort their coordinates. 
% Subsequently, we can obtain the object's location estimation under $\mathtt{E}$ at time $i$ with the center point of bounding box proposal $x^i$ as:
% \begin{equation}
% \begin{aligned}
% x^i & =\pi\left(X_\mathtt{E}^i\right)+v^i =\pi\left(X_\mathtt{O}
% ^i+t_{\mathtt{EO}}^i\right)+v^i,
% \end{aligned}
% \end{equation}
% where $X_\mathtt{O}^i$ is the corresponding 3D point of $x^i$ in object reference, and $v^i$ is the random noise of the feature point.
% It's worth noting that we encounter a scale issue because we only have one central point.

% We perform event clustering with a grid-based method to aid in objects detection.
% Firstly, we divide the camera's field of view (FOV) into elementary cells using a regular grid, with each cell sized according to $c_w \times c_h$.
% Then, for each cell, we compare the number of events within it over a specified time interval ($c_{\Delta t}$) against an activation threshold $c_{thres}$.
% If this count exceeds $c_{thres}$, the cell is marked as active.
% Finally, we connect active cells to form clusters as detection results, containing clusters generated by drones.
% We employ trackers built upon a constant velocity motion model and Kalman filter to track objects. 
% These trackers generate bounding box proposals based on input clusters. 
% Specifically, a tracker initially employs the constant velocity motion model to predict the current state of the object based on its previous state.
% Subsequently, upon receiving a cluster input, we associate it with the tracker exhibiting the largest Intersection Over Union (IOU) area compared to the cluster.
% In cases where no IOU association is feasible but the distance between the tracker and the cluster is less than $d_{iou}$, the cluster is linked to the nearest tracker. 
% Finally, we utilize the input cluster as an observation state to correct the current state of the tracker. 
% This approach enables the tracking of moving objects, including drones.




\vspace{-0.8cm}
\subsubsection{\textbf{Consistency-instructed Measurements Filter}} \label{4.1.3}
% The output of the \textit{Event Tracking Model} includes detection results for drones as well as other objects in the environment that cause changes in light intensity, such as indicator lights near drone landing sites or shadows cast by drones and objects passing near the event camera. 
% The system needs to separate the detection results for the landing drone from the detections of various objects.
% Similarly, the 3D point cloud output by the \textit{Radar Tracking Model} contains points for the landing drone as well as noise points generated by multipath effects. 
% It is necessary to extract the points relevant to the landing drone from this noisy point cloud.

% The \textit{Event Tracking Model} detects drones and other objects causing light intensity changes, such as indicator lights or shadows near the event camera. The system must distinguish the detection results for the landing drone from other objects. 
% Similarly, the \textit{Radar Tracking Model} outputs a 3D point cloud with points for the drone and noise from multipath effects. It is necessary to extract the relevant points for the landing drone from this noisy cloud.

The \textit{Event Tracking Model} detects drones and other objects causing light changes, such as indicator lights or shadows. The system must distinguish the landing drone from these objects. Similarly, the \textit{Radar Tracking Model} outputs a 3D point cloud containing both the drone and noise from multipath effects, requiring extraction of the drone’s relevant points.

% \noindent 
\textbf{Consistency-instructed alignment.} 
Utilizing the \textit{temporal-consistency} from the event camera and radar, and their distinct mechanisms respond to dynamic objects, we filter event camera results affected by lighting variations on stationary objects and vice versa for radar points influenced by multipath effects.
Specifically, we align synchronized radar points (\textit{Radar Tracking Model}) to each event bounding box (\textit{Event Tracking Model}) (\fig \ref{performance}d). 
Using event camera's projection, we determine that object's location lies along the ray from the camera's optical center through bounding box center. 
The system then identifies the nearest radar points along this ray to isolate the object-associated points.
If no radar point is detected, the bounding box is treated as noise and disregarded.

% Leveraging the \textit{consistent information} of event camera and mmWave radar respond to dynamic objects with different mechanisms, we filter out event camera tracking results caused by light variations of stationary objects using radar tracking results and, conversely, filter out radar points generated by multipath effects using event camera tracking results. 
% Specifically, for synchronized bounding boxes from the \textit{Event Tracking Model} and point clouds from the \textit{Radar Tracking Model}, we match radar points to each event camera tracking result, as shown in \fig \ref{performance}e. 
% Referring to projection function of event camera, we establish that the object's location lies along the ray originating from the event camera optical center and passing through the center point of the bounding box on the image plane. 
% Our system then identifies the radar points closest to this ray, enabling us to isolate the radar point associated with the object while disregarding other measurements.
% If no radar point is detected, we consider the bounding box to be the result of unexpected noise and disregard this.\\
% \noindent 

\textbf{Periodic micro motion-aid measurements extraction.} 
% Each platform supports one drone landing at a time, we need to find a feature of drone, and utilize it to extract landing drone-specific measurements from the aligned tracking results. 
% This feature must effectively distinguish drones from noise.
Since each platform supports one drone landing at a time, we need to identify a distinguishing feature of the landing drone, which effectively differentiates the drone from noise, and use it to extract landing drone-specific measurements from the aligned tracking results. 
% This feature must effectively differentiate drones from noise.
Our finding is that drones exhibit periodic micro-motions (\eg, propeller rotation), which can serve as stable and distinctive features of the drone. 
We transform the spatio-temporal distribution of events into a heatmap and apply statistical metrics to isolate drone measurements leveraging this feature. 
% Specifically, events within a time window $[i, i + \delta i]$ are binned into a 2D histogram, with each bin representing a spatial region (\eg, $5\times 5$ pixels). 
% Bins with periodic micro-motions contain more events due to rapid light changes. 
% Meanwhile, periodic micro-motions produces bipolar events in a bin, whereas background motion and noise tends to be unipolar (\eg, flying birds)
% We 根据event counts and the proportion of positive events in each bin select  bins with periodic micro-motions which exhibit higher event counts and a more balanced proportion of event polarities.
% 随后,我们选择包含 bins with periodic micro-motions最多的event tracking results and corresponding point clouds ($t_{EO}$) as the drone tracking result ($t_{ED}$) to roughly localize the drone.
Specifically, within a time window $[i, i + \delta i]$, events are binned into a 2D histogram where each bin corresponds to a spatial region (\eg., $5\times 5$ pixels). 
Bins containing propeller rotation tend to accumulate more events due to rapid light intensity changes. 
Meanwhile, these propeller rotations generate bipolar events within a bin, while background motion and noise typically result in unipolar events (\eg, from flying birds).
Therefore, we select bins with propeller rotation based on event counts and the proportion of positive events, favoring those with higher counts and a more balanced ratio. 
Finally, we identify event tracking results with the most bins indicative of propeller rotation and corresponding point clouds ($t_{EO}$), designating them as drone tracking results ($t_{ED}$) for preliminary localization from two models as shown in \fig \ref{performance}e.
% The drone near to the platform
% This approach reduces tracking latency by leveraging both sensors, enhancing accuracy and reliability.
When multiple drones are scheduled to land, they descend and land sequentially. 
This method accurately identifies the landing drone and extracts relevant measurements.
% This method accurately determines the drone nearest to the platform, identifying the landing drone.
% This method accurately determines the landing drone, extract measurements related to it.
% Assuming the number of landing drones is provided and no other moving objects are present, we select the tracking results with the largest bounding box and associated point clouds ($t_{EO}$) as the drones tracking result ($t_{ED}$), which are used to roughly localize the drone. 
% \notice{select the tracking results with the rotation part. Only one drone landing at one time}
% % This method effectively reduces tracking latency by combining the strengths of both sensors.
% This innovative approach significantly minimizes tracking latency by leveraging the combined strengths of both sensors, resulting in enhanced accuracy and reliability.

\vspace{-0.2cm}
\subsection{\textit{GAJO}: Graph-informed Adaptive \\ Joint Optimization} \label{4.2}

% \noindent \textbf{Challenge.} 
% So far, we attain a preliminary estimation results of drone location. 
% However, results of both event camera tracking model and radar tracking model suffer from severe location tracking bias. 
% The event camera estimations are hampered by scale uncertainty and restricted resolutions, while radar estimations grapple with the challenges of low spatial resolution and accumulating drift. 
% Additionally, data from both sensors is heterogeneous in nature.
% In addition to this, accurate 3D localization proves to be more time-consuming than detection and tracking due to additional processing.
% Thus, in mmE-Loc, we focus on the accurate localization of drone and latency minimization.
The preliminary drone location estimations from the event and radar tracking models suffer from biases. 
Specifically, event camera estimations face scale uncertainty, while radar estimations struggle with limited spatial resolution, scatter center drift, and accumulating drift. 
% Meanwhile, measurements from different tracking models are heterogeneous in precision, scale, and density, which adds complexity to fusion and optimization. 
Additionally, estimations from different models are heterogeneous in precision, scale, and density, complicating the fusion and optimization.
Therefore, in this part, we prioritize accurate drone ground localization and trajectory tracking.


% \noindent \textbf{Observation.} 
% Our design is based on an observation that \textit{the event camera tracking and radar tracking models leverage two distinct modalities and features, respectively.} 
% Consequently, the two models benefit from their individual yet complementary advantages, and thus a joint optimization would enhance the overall performance, yielding a trajectory that exhibits both low bias and low cumulative drift.

% (1) Supplementing the design section with more design details, providing insight into how the spatial complementarity features of the radar and event camera are fused;
\revise{

Our design is founded on the insight that \textit{the Event Tracking Model and Radar Tracking Model provide distinct features that are spatial-complementarity to each other.} 
As a result, the 2D imaging capability of event cameras and the depth sensing capability of mmWave radar mutually enhance each other when combined, as demonstrated in \fig \ref{relationship}. 
Since both the event stream and mmWave samples are drone-related, fully leveraging the \textit{spatial- complementarity} of these two modalities through joint optimization offers the potential to significantly improve performance. This leads to a trajectory with reduced bias and minimized cumulative drift.
% Our design is founded on the insight that \textit{the Event Tracking Model and Radar Tracking Model provide distinct features that are spatial-complementarity to each other.}
% As a result, the 2D imaging capability of event cameras and depth sensing capability of mmWave radar are mutually beneficial when combined, as shown in \fig \ref{relationship}.
% Since the event stream and mmWave sample are all drone-related, fully harnessing the \textit{spatial-complementarity} of both modalities through a joint optimization holds promise for comprehensively enhancing performance, resulting in a trajectory characterized by reduced bias and decreased cumulative drift.
}
% 充分挖掘两个model的潜力,同时Integrating these models through a joint optimization process is promising to enhance performance comprehensively, resulting in a trajectory characterized by reduced bias and a decrease in cumulative drift. 

To realize this idea and push the limit of localization accuracy while minimizing latency, we introduce a \textit{GAJO}, a factor graph-based location optimization framework designed for low-latency and accurate drone 3D localization (§\ref{4.2.1}).
\textit{GAJO} includes two parallel tightly coupled modules: $(i)$ short-term (inter-SAE tracking) and $(ii)$ long-term (local location optimization) optimizations, collectively enhancing location tracking precision (§\ref{4.2.3}).
Beyond the capabilities of \textit{Event Tracking model} and \textit{Radar Tracking model}, \textit{GAJO} assimilates prior knowledge of drone's flight dynamics to refine the trajectory for enhanced smoothness and accuracy (§\ref{4.2.2}).


\subsubsection{\textbf{Factor graph-based optimization}}\label{4.2.1}

% ChatGPT
% A factor graph consists of two types of nodes: $(i)$ the variable nodes which indicate the states to be optimized (\eg, $t_{ED}^i$); $(ii)$ the factor nodes which represent the probability of certain states given a measurement result.
% In mmE-Loc, these measurements come from the ET (abbr. for Event Tracking) model ($x^i$) and RT (abbr. for Radar Tracking) model ($D^i$, $\vec{v}^i$ and $U_E^{i}$).
% In order to estimate the values of a certain set of variable nodes $\boldsymbol{\mathcal{X}} = \{t_{ED}^i | i \in \mathcal{T}\}$ given measurements $\boldsymbol{\mathcal{Z}} = \{x^i, D^i, \vec{v}^i, U_E^{i} | i \in \mathcal{T}\}$, \textit{GAJO} optimizes all the factor nodes connected with them based on maximum a posteriori estimation:
A factor graph comprises variable nodes, indicating the states to be optimized (\eg, $t_{ED}^i$), and factor nodes, representing the probability of certain states given a measurement result. 
In mmE-Loc, measurements are derived from the Event Tracking (ET) model ($x^i$) and Radar Tracking (RT) model ($D^i$, $\vec{v}^i$, and $U_E^{i}$).
To estimate the values of a set of variable nodes $\boldsymbol{\mathcal{X}} = \{t_{ED}^i | i \in \mathcal{T}\}$ given measurements $\boldsymbol{\mathcal{Z}} = \{x^i, D^i, \vec{v}^i, U_E^{i} | i \in \mathcal{T}\}$, \textit{GAJO} optimizes all connected factor nodes based on maximum a posteriori estimation:
\begin{align}
\begin{split}
\hat{\boldsymbol{\mathcal{X}}} & =\underset{\boldsymbol{\mathcal{X}}}{\arg \max } \ p(\boldsymbol{\mathcal{X}} \mid \boldsymbol{\mathcal{Z}}) = \underset{\boldsymbol{\mathcal{X}}} {\arg \max } \  p(\boldsymbol{\mathcal{X}}) \ p(\boldsymbol{\mathcal{Z}} \mid \boldsymbol{\mathcal{X}}) \\
& =\underset{\boldsymbol{\mathcal{X}}}{\arg \max } \ 
p(\boldsymbol{\mathcal{X}}) \prod_{i \in \mathcal{T}} \ p\left(x^i \mid t_{ED}^i\right) p\left(D^i, \vec{v}^i, U_E^{i} \mid t_{ED}^i\right),
\end{split}
\label{factor_graph}
\end{align}
which follows the Bayes theorem.
% , and all measurements are independent.
$p(\boldsymbol{\mathcal{X}})$ is the prior information over $\boldsymbol{\mathcal{X}} = \{t_{ED}^i | i \in \mathcal{T}\}$, which is inferred from drone flight characteristics.
The $p\left(x^i \mid t_{ED}^i\right)$ is the likelihood of the ET model measurements. The $p\left(D^i \mid t_{ED}^i\right)$, $p\left(\vec{v}^i \mid t_{ED}^i\right)$ and $p\left(U_E^{i} \mid t_{ED}^i\right)$ are likelihood of the RT model measurements.

\begin{figure}[t]
    % \setlength{\abovecaptionskip}{-0.1cm} % height above Figure X caption
    \setlength{\belowcaptionskip}{-0.2cm}
    \setlength{\subfigcapskip}{-0.6cm}
    \centering
        \includegraphics[width=0.95\columnwidth]{Figs/relationship.png}
        \vspace{-0.4cm}
    \caption{Illustration of relationship between \textit{GAJO} and \textit{CCT}. \textnormal{The \textit{GAJO} module harness the \textit{spatial-complementarity} of both modalities through a join optimization.}}
    \label{relationship}
    \vspace{-0.5cm}
\end{figure} 


\begin{figure*}[t]
    \setlength{\abovecaptionskip}{0.05cm} % height above Figure X caption
    \setlength{\belowcaptionskip}{-0.3cm}
    \setlength{\subfigcapskip}{-0.25cm}
    \centering
        \includegraphics[width=1.92\columnwidth]{Figs/factorgraph.png}
        % \vspace{-0.1cm}
    \caption{Long-short term optimization based on the factor graph.}
    \label{factorgraph}
    \vspace{-0.25cm}
\end{figure*} 


\subsubsection{\textbf{Probabilistic Representation}} \label{4.2.2}
Inferring the drone's location requires prior term and likelihood term in \eqn \eqref{factor_graph}.

% \noindent 
\textbf{Prior term.} 
% The prior term $p(t_{ED}^i)$ indicates the probability distribution of the drone’s location time $i$ without knowing any measurement result. 
% Based on the kinetic characteristics of the drone, the constant velocity model, which has been widely used in both flight control and SLAM system, is adopted to derive the prior term. Specifically, the drone is assumed to move at an approximately constant speed during a short period of time. On this basis, the prior location can be inferred from 
% The prior term regarding the drone's position probability distribution  at time $i$, not influenced by current measurements, is expressed with $p(t_{ED}^i)$. 
% This prior is derived from the constant velocity model used in modern flight control system, revealing that the drone likely maintains steady speed over short intervals. 
% This assumption allows us to predict the prior location through the relation
% The prior term, $p(t_{ED}^i)$, represents the probability distribution of the drone's position at time $i$ unaffected by current measurements. It is derived from a constant velocity model commonly used in modern flight control systems, indicating that the drone likely maintains a steady speed over short intervals. This assumption enables us to predict the prior location:
The prior term, $p(t_{ED}^i)$, represents the drone's location probability distribution at time $i$ unaffected by current measurements. Derived from a constant velocity model, it suggests the drone maintains steady speed over short intervals, allowing us to predict the prior location using:
\begin{equation}
\vspace{-0.1cm}
\bar{t}_{\mathrm{ED}}^i-t_{\mathrm{ED}}^{i-1}=t_{\mathrm{ED}}^{i-1}-t_{\mathrm{ED}}^{i-2}.
% \vspace{-0.1cm}
\end{equation}

% \noindent 
\textbf{ET model likelihood.} 
% The likelihood of the ET model $p(x^i|t_{ED}^i)$ indicates the distribution of the center point at a given drone location.
% Most existing vision-based systems treat random noise of center point $v^i$ as Gaussian distribution (\ie normal distribution). The assumption has been proved to be effective in many tracking systems.
% Therefore, the ET model measurement likelihood can be presented as follows:
The likelihood $p(x^i|t_{ED}^i)$ from ET model represents the center point distribution at a given drone location. 
In many tracking systems \cite{campos2021orb}, center point noise $v^i$ is assumed Gaussian, proving effective. Thus, likelihood of ET model is:
% $p(x^i|t_{ED}^i) \sim \mathcal{N}(\pi(\textbf{X}_E^i), \sigma_{ET})$,
% \vspace{-0.4cm}
\begin{equation}
p(x^i|t_{ED}^i) \sim \mathcal{N}(\pi(\textbf{X}_E^i), \sigma_{ET}),
% \vspace{-0.4cm}
\end{equation}
where $\sigma_{ET}$ is the center point standard deviation.
% where the $\sigma_{ET}$ is the standard deviation of center point measurement.

% \noindent 
\textbf{RT model likelihood.}
The likelihood of the RT model $p(D^i \mid t_{ED}^i)$, $p(\vec{v}^i \mid t_{ED}^i)$ and $p(U_E^{i} \mid t_{ED}^i)$ indicates the distribution of the measured distance, angle, and motion at a given drone location.
The distance, angle, and motion from RT model likelihood are:
% \vspace{-0.2cm}
\begin{equation}
\begin{aligned}
p(D^i \mid t_{ED}^i&) \sim  \mathcal{N}(||t_{ED}^i||, \sigma_{D}), \quad p(\vec{v}^i \mid t_{ED}^i) \sim \mathcal{N}(\vec{v}_{t_{ED}^i}, \sigma_{\vec{v}}), \\
& p(U_E^{i} \mid t_{ED}^i) \sim \mathcal{N}(t_{ED}^i - t_{ED}^{i - 1}, \sigma_{U_E}),
\vspace{-0.4cm}
\end{aligned}
\end{equation}
where $\sigma_{D}$, $\sigma_{\vec{v}}$ and $\sigma_{U_E^{i}}$ are the standard deviation of distance, angle, and motion measurements respectively.

\subsubsection{\textbf{Fusion-based Tracking}} \label{4.2.3}
% As illustrated in \fig \todo{Figure}, two types of fusion schemes are adopted in mmE-Loc.
% Specifically, the inter-frame tracking infers the drone’s location in real-time. In contrast, the local pose tracking focuses on the overall accuracy of the flight trajectory over a period of time.
% In mmE-Loc, two fusion schemes are utilized, as shown in \fig \ref{factorgraph}. 
% The first scheme, inter-frame tracking, aims to estimate the real-time location of the drone. 
% On the other hand, the second scheme, local pose tracking, focuses on ensuring the overall accuracy of the flight trajectory over a certain time period.
\revise{
In mmE-Loc, two fusion schemes are employed for sensor fusion and optimization, as depicted in \fig \ref{factorgraph}. 
The first, inter-SAE tracking, aims for instant drone location estimation by minimizing errors across different tracking models simultaneously.  
The second, local location optimization, enhances overall trajectory accuracy through the joint optimization of a selected set of locations.
}
% The first, inter-SAE tracking, aims for instant drone location estimation by minimizing the error of different tracking model at the same time.
% The second, local location optimization, ensures overall trajectory accuracy through joint optimization of selected location set.

% \noindent 
\textbf{Inter-SAE tracking.}
% (\aka projection error term)
Once the measurements of ET model and RT model $(x^i, D^i, \vec{v}^i, U_E^i)$ received, the prior factor, ET factor and the RT factor are formulated as follows:
% \vspace{-0.5cm}
\begin{equation}
\begin{aligned}
E^i_{\text {Prior }} & =-\log p\left(t_{ED}^i\right) \propto \left\|t_{ED}^i-\bar{t}_{ED}^i\right\|_{\sigma_{t_{ED}}}^2, \\
E^i_{\mathrm{ET}} & =-\log p\left(x^i \mid t_{ED}^i\right) \propto \rho(\left\| x^i - \pi(\textbf{X}_E^i) \right\|^2_{\Sigma_E}), \\
E^i_{\mathrm{RT}} & =-\log p\left(D^i, \vec{v}^i, U_E^i \mid t_{ED}^i\right) \\
\propto & \left\| ||t_{ED}^i|| \!-\! D^i \right\|^2_{\sigma_D} \!+\!  \left\| \vec{v}_{t_{ED}^i} \!-\! \vec{v}^i \right\|^2_{\sigma_{\vec{v}}} \!+\! \left\| (t_{ED}^i - t_{ED}^{i-1}) \!-\! U_E^i \right\|^2_{\sigma_{U_E}},
\end{aligned}
\end{equation}
% where $\left\|e \right\|^2_\Omega=e^T\Omega e$. The symbol $\Omega_E$ represents the information matrix, which is the inverse of the covariance matrix associated with the event camera measurements.
where $\left\|e \right\|^2_{\Sigma_E}=e^T\Sigma^{-1} e$.
% is defined as the squared Mahalanobis distance with covariance matrix $\Sigma_E$.  
The symbol $\Sigma_E$ represents the covariance matrix associated with the event camera measurements.

On this basis, the inter-SAE tracking in \fig \ref{factorgraph}a is performed to give an instant location result based on \eqn \eqref{factor_graph} as follows:
% \vspace{-0.5cm}
\begin{equation}
\begin{aligned}
& \hat{t}_{ED}^i \!=\! \underset{\boldsymbol{t_{ED}^i}}{\arg \max } \ p ( t_{ED}^i \!\mid\! t_{ED}^{i-1}, t_{ED}^{i-2} ) p(x^i \!\mid\! t_{ED}^i) \ p(D^i, \vec{v}^i, U_E^{i} \!\mid\! t_{ED}^i) \\
\vspace{1ex}
& = \underset{\boldsymbol{t_{ED}^i}}{\arg \min } \!-\!\log \!\left(p ( t_{ED}^i \!\mid\! t_{ED}^{i-1}, t_{ED}^{i-2} ) p(x^i \!\mid\! t_{ED}^i) p(D^i\!,\! \vec{v}^i\!,\! U_E^{i} \!\mid\! t_{ED}^i)\right) \\
\vspace{1ex}
& = \underset{\boldsymbol{t_{ED}^i}}{\arg \min } \left( E^i_{\text {prior }} + E^i_{\mathrm{ET}} + E^i_{\mathrm{RT}}\right).
\end{aligned}
\label{inter_frame}
\end{equation}
% Inter-SAE tracking

% \noindent 
\textbf{Local location optimization.}
% For every few frames, the local location tracking is triggered to correct the cumulative drift. 
% Local location tracking takes several frames and jointly optimizes their locations:
% Denote the set of frames as $\mathcal{T}$, the optimization problem can be formulated as follows:
% To address cumulative drift, periodic local location tracking is performed, which corrects the estimated locations based on several consecutive frames. 
% This optimization process involves jointly optimizing the locations of a set of frames denoted as $\mathcal{X}=\underset{i \in \mathcal{T}}{\bigcup}\left\{t_{ED}^i\right\}$. 
% The formulation of the optimization problem is as follows:
To mitigate cumulative drift, periodic local location optimization is conducted, correcting estimated locations based on multiple consecutive SAEs. 
This optimization entails jointly optimizing the locations of a SAE set denoted as $\mathcal{X}=\underset{i \in \mathcal{T}}{\bigcup}\left\{t_{ED}^i\right\}$, as shown in \fig \ref{factorgraph}b, where $W=|\mathcal{T}|$.
The optimization problem formulation is as follows:
\begin{equation}
\begin{aligned}
\hat{\boldsymbol{\mathcal{X}}} & =\underset{\boldsymbol{\mathcal{X}}}{\arg \max } \ p(\boldsymbol{\mathcal{X}}) \prod_{i \in \mathcal{T}} \ p\left(x^i \mid t_{ED}^i\right) p\left(D^i, \vec{v}^i, U_E^{i} \mid t_{ED}^i\right), \\
& = \underset{\boldsymbol{\mathcal{X}}}{\arg \min } \sum_{i \in \mathcal{T}}\left(E_i^{\mathrm{prior}}+E_i^{\mathrm{ET}}+E_i^{\mathrm{RT}}\right) .
\end{aligned}
\label{local_location}
\end{equation}
It is worth noting that $(i)$ when the local location optimization is triggered, $(ii)$ what is the size of $\mathcal{T}$ ($W$ = $|\mathcal{T}|$), and $(iii)$ how to solve the inter-SAE tracking and local location optimization problems affect the latency and accuracy of localization.
Hence, we enhance the efficiency of \textit{GAJO} through an adaptive optimization method.

% \subsubsection{\textbf{Adaptively Optimization method}}
% Now we express the estimation problem \eqn \ref{inter_frame} and \eqn \ref{local_location} using a graphical model. 
% When solving both nonlinear least-squares problems, we linearize the observation model and solve the least squares formulation as follows:
% \begin{equation}
% \hat{\boldsymbol{\mathcal{X}}}=\arg \min _{\boldsymbol{\mathcal{X}}}\|A \boldsymbol{\mathcal{X}}-\mathbf{b}\|^2,
% \end{equation}
% where the matrix $A \in \mathbb{R}^{m \times n}$ is a measurement Jacobian and $\mathbf{b} \in \mathbb{R}^m$ is the right-hand side vector \tocite.
% The QR matrix factorization $A = Q[R, 0]^T$ is then utilized, and the least squares problem $R \hat{\boldsymbol{\mathcal{X}}}=\mathbf{d}$ is solved through backsubstitution to get optimized locations $\hat{\boldsymbol{\mathcal{X}}}$, where $R \in \mathbb{R}^{n \times n}$ is the upper triangular square root information matrix, $Q \in \mathbb{R}^{m \times m}$ is an orthogonal matrix and $\textbf{d} \in \mathbb{R}^n$. 
% More detail can refer to \tocite.
% % A batch solution solves the complete problem at every step, including all previous measurements, which performs unnecessary calculations.
% % A  exploits incrementally updating the square root information matrix R with new measurements.
% Although re-linearization and re-generate $R$ as new measurements comes can reduce system error, for the problem \eqn \ref{local_location}, which requires joint optimization of multiple locations, this process can be computationally expensive \tocite.

% To address this problem, we propose the Adaptively Optimization method, based on the observation that \textit{new measurements often have a localized impact, leaving remote parts of the graph unaffected}, which enable us to incrementally update $R$ \tocite. 
% When solving local location tracking at each step, this method adaptively combines incrementally updated $R$ and re-generated $R$, reducing latency and improving accuracy.
% \alg \ref{algorithm} shows how Adaptively Optimization method solves local location tracking problem.
% Line 1-3 represents local location tracking with incrementally updated $R$ \tocite.
% Line 4-16 show local location tracking with re-generated $R$.
% Specifically, when receive new measurements, function $\mathtt{AddFactorToGraph}$ updates factor graph, and function $\mathtt{IncrementalUpdate}$ incrementally update $R$ with new measurements \tocite. 
% We then solve local location tracking with this incrementally updated $R$.
% When one of two conditions is met, we solve local location tracking with re-generated $R$: 
% $(i)$ we tracks locations that have changed significantly in a set $L = \{t_{ED}^i: \hat{t^i_{ED}} - t^i_{ED} \geq \delta\}$. If enough locations have undergone significant changes (\ie $|L| \geq L_T$), we solve local location tracking with re-generated $R$ output by function $\mathtt{FullUpdate}$ \tocite;
% $(ii)$ if the norm of the total locations changes becomes too large (\ie $||\hat{\mathcal{X}} - \mathcal{X} || \geq \Delta$), we solve local location tracking with re-generated $R$; 
% Since the local location tracking involves repeatedly solving linear equations, this condition keeps the current solution from diverging too far from the optimal solution.


% \begin{algorithm}[t]
% \caption{Adaptively Optimization method}
% \label{algorithm}
% \KwData{Original factor graph $G$; New measurements $D, \vec{v}, U^i_E$; square root information matrix $R$}
% \KwResult{Updated locations $\hat{\mathcal{X}}$}
% $G \leftarrow \mathtt{AddFactorToGraph}(G, D, \vec{v}, U^i_E)$\;
% $R \leftarrow \mathtt{\textbf{IncrementalUpdate}}(G)$\;
% $\hat{\mathcal{X}} \leftarrow \mathtt{Backsubstitution}(R)$\;
% $L \leftarrow \emptyset$;  $\quad\quad \triangleright \textit{Set of nodes need to be linearized}$\;
% \For{all $t^i_{ED} \in \mathcal{X}$ and all $\hat{t^i_{ED}} \in \hat{\mathcal{X}}$}
% {
% \If{$\hat{t^i_{ED}} - t^i_{ED} \geq \delta$}
% {$L \leftarrow L \cup t^i_{ED}$\;}
% }
% \If{$|L| \geq L_T$ or $||\hat{\mathcal{X}} - \mathcal{X} || \geq \Delta$}
% {
% \For{all $t^i_{ED} \in \mathcal{X}$}
% {
% $\mathtt{UpdateLinearizationPoint}(t^i_{ED})$\;
% }
% $R \leftarrow \mathtt{\textbf{FullUpdate}}(G)$\;
% $\hat{\mathcal{X}} \leftarrow \mathtt{Backsubstitution}(R)$\;
% }
% \end{algorithm}
\section{Implementation Environment}
\label{sec:implementation_environment}

Here we introduce the detailed implementation details and environment for reproducibility purpose. For our model, we choose hyperparameters based on the performance on validation set (Document classification task in the main paper explains how we split validation set). The results in the main paper are obtain by 5 independent runs. The standard deviations reported in the main paper are 1-sigma error bars and are obtained by calling its corresponding function in Excel library. All the experiments were done on Linux server with an NVIDIA A40 GPU with 46,068 MiB. Its operating system is CentOS Linux 7 (Core). We implemented our proposed model GTFormer using Python 3.10 as programming language and PyTorch 2.0.0 as deep learning library. Other frameworks include NumPy 1.23.1, sklearn 0.23.2, and scipy 1.5.2. We emphasize that the main focus of our model is effectiveness, instead of running efficiency. But for completeness, we still make a short comment on execution time. Our model is efficient, on the largest dataset Web, the training takes less than 40 hours to converge. We will release code and datasets upon publication.
\section{Evaluation}
% In light of experiments of CacheBlend (\S\ref{eval:1}) and EPIC (\S\ref{eval:2}), we design our experiments (\S\ref{eval:3}).

% \noindent\textbf{LLM Dataset.} 2WikiMQA, MuSiQue, HotpotQA, SAMSum, MultiNews.

% \noindent\textbf{LLM Baselines.} Full KV recompute, Prefix caching, Full KV reuse, CacheBlend, EPIC.
% \subsection{CacheBlend evaluation}\label{eval:1}
% \begin{itemize}
%     \item TTFT-Score Comparison.
%     \item RPS-TTFT Comparison.
%     \item Sensitivity Analysis. (1) chunk number; (2) chunk length; (3) batch size; (4) recompute ratio; (5) storage device (CPU RAM / slower Disk).
% \end{itemize}
% \subsection{EPIC evaluation}\label{eval:2}
% \begin{itemize}
%     \item TTFT-Score Comparison.
%     \item (CCR+RPS)-TTFT/Throughput Comparison.
%     \item Context length-TTFT Comparison.
%     \item Semantic-based / fixed-token-based splitting.
% \end{itemize}
% \subsection{\sys}\label{eval:3}\
% \noindent\textbf{VLM Model.} InternVL 2.5-8B \cite{chen2024internvl}, Qwen2-VL-7B \cite{wang2024qwen2vl}, LLaVA-1.6-vicuna-7B, LLaVA-1.6-Mistral-7B \cite{liu2024llavanext}.

% \noindent\textbf{VLM Dataset.} SparklesDialogueCC, SparklesDialogueVG \cite{huang2024sparkles}, MMDU \cite{liu2024mmdu}.

% \noindent\textbf{VLM Baselines.} CacheBlend, Prefix caching, Full KV reuse, \sys.

% \begin{itemize}
%     \item TTFT-Score Comparison.
%     \item RPS-TTFT/Throughput Comparison.
%     \item Sensitivity Analysis: Image number.
%     \item Why does CacheBlend fail to work when serving MLLM?
% \end{itemize}

In this section, we evaluate \sys~in terms of response time and generation quality. We also investigate whether \sys~is applicable when the number of images is large.
\subsection{Experimental settings}
We select two prevalent MLLMs in the experiments: LLaVA-1.6-vicuna-7B and LLaVA-1.6-mistral-7B \cite{liu2024llavanext}. All experiments are run on a server with 1 NVIDIA H800-80 GB GPU, 20-core Intel(R) Xeon(R) Platinum CPUs, and 100GB DRAM.

Two datasets are used in our evaluation. (1) \textbf{MMDU} \cite{liu2024mmdu}: This dataset aims to evaluate MLLMs' abilities in multi-turn and multi-image conversations. Each conversation stitches together multiple images and sentence-level text (e.g., ``IMAGE\#1, IMAGE\#2. Can you describe these images as detailed as possible?"). (2) \textbf{SparklesEval} \cite{huang2024sparkles}: This is also a dataset for assessing MLLMs' conversational competence across multiple images and conversation turns. Unlike MMDU, SparklesEval integrates multiple images at word level (e.g., ``Can you link the celebration occurring in IMAGE\#1 and the dirt bike race in IMAGE\#2 ?"). As shown in the examples, the prompts of two datasets are open questions. Previous works adopt GPT score to evaluate the quality of MLLMs' responses to the open questions \cite{liu2024mmdu, huang2024sparkles}. GPT score is a GPT-assisted evaluation that uses a powerful judge model (e.g., GPT-4o, Qwen, etc.) to assess the answers. We also employ this metric and their evaluation prompt, as listed in Appendix~\ref{prompt}.
% (3) \textbf{V*Bench} \cite{wu2024v}:  A dataset specifically designed to evaluate
% MLLMs in their ability to process high-resolution images and focus on visual details. Each sample contains a high-resolution image, a question, and four options.
% We select 100 samples from each of the above datasets for testing, each including 1 to 5 images.

% We use the following metrics to measure the performance of algorithms. (1) Time-To-First-Token (TTFT) refers to the time it takes for LLMs, to generate and return the first token after receiving an request. This metric is designed to measure the time spent in the prefill stage, which can be optimized by addressing the PIC problem. (2) GPT score \cite{liu2024mmdu, huang2024sparkles} is a GPT-assited evaluation  that uses a judge model (e.g., GPT-4o, Qwen, etc.) to assess the quality of model-generated responses. We employ this metric to assess the quality of MLLMs' responses to the open questions in MMDU and SparklesEval. We apply the evaluation prompts in MMDU \cite{liu2024mmdu} to guide the judge model for scoring in the range of 10. 

% (3) F1 score is a metric used to evaluate the similarity between MLLMs’ output and the groundtruth answer. We employ this metric to assess the accuracy of the MLLMs' answers to the multiple-choice questions in V*Bench.

We compare \sys-$k$ with three existing CC algorithms: prefix caching, full reuse, and CacheBlend \cite{yao2024cacheblend}. CacheBlend is also a position-independent algorithm designed for RAG system. It recomputes $r$\% of total tokens with largest KV deviation, so we denote it as CacheBlend-$r$. The primary focus of CacheBlend is the KV deviation, while the \sys's selection process involves the identification of tokens that exhibit both high attention scores and significant KV deviation. We implement the four CC algorithms based on vLLM 0.6.4 \cite{kwon2023efficient}.

% (1) Prefix Caching: This algorithm merely stores and reuses the KV chche of the prefix. And the KV cache of non-prefix tokens needs to be computed during prefill. (2) Full Reuse: This algorithm reduces TTFT by fully reusing the entire KV cache regardless of the position of multimodal data. (3) CacheBlend \cite{yao2024cacheblend}: This is a state-of-the-art partial reuse algorithm that achieves a trade-off between TTFT and generation quality by dynamically selecting partial tokens to recompute.
% Additionally, we evaluate various variants of CacheBlend, denoted as CacheBlend-r, where $r$ represents the ratio of tokens recomputed. Similarly, we test different variants of InfoBlend, denoted as InfoBlend-k, where $k$ indicates the number of tokens recomputed at each chunk boundary.

\subsection{Effectiveness of \sys}
Based on vLLM offline inference, we compare the performance of all algorithms. Specifically, we process all requests sequentially and evaluate their generation quality and processing time for prefill. The workflow initiates with the precomputation of the relevant KV cache for images. Subsequently, we send the user's query along with the cache\_ids of the images to the serving system. Prefix caching will process the query with the KV cache of system prompt only. \sys~concatenates the dummy cache and stored cache, and computes the first output token using selective attention mechanism in single step. Full reuse and CacheBlend first compute the KV cache of text, and then produce the first output token with the concatenated KV cache. We record the processing time of the algorithms and finally score for each response.
\begin{figure}[t]
    \centering
    \includegraphics[width=\columnwidth]{figs/legend_result.pdf}
    % \vskip -0.2in
    \includegraphics[width=\columnwidth]{figs/results.pdf}
    \caption{Comparison of TTFT ($\downarrow$ Better) and Score ($\uparrow$ Better) using different models on different datasets. }
    \label{fig:ttft-score}
    % \vskip -0.2in
\end{figure}

\figurename~\ref{fig:ttft-score} presents the experimental results of all algorithms across different models and datasets. The results indicate that \sys~consistently outperforms CacheBlend in terms of both TTFT and score across various configurations. \sys-32 reduces TTFT by up to 54.1\% while maintaining a loss of score within 13.6\% compared to prefix caching. Additionally, it is clear that \sys~exhibits a slight decrease in TTFT compared to full reuse, since \sys~is a single-step process. Overall, compared to other algorithms, \sys~achieves the best trade-off between TTFT and score.

\subsection{Sensitivity analysis}
In order to achieve a more profound comprehension of \sys, a subsequent analysis is necessary to ascertain how the number of images impacts overall performance. We divide the dataset of MMDU into 10 groups in terms of the number of images. We evaluate the TTFT and score of \sys~and baselines on each group. The average value of results are shown in \figurename~\ref{fig:10}. The TTFT of \sys~is consistently shorter than that of prefix caching. When the number of images is 10, \sys~achieves 54.7\% reduction in TTFT. Furthermore, the performance of \sys~remains unaffected by the number of images, exhibiting negligible or no accuracy degradation.
\begin{figure}[t]
    \centering
    \includegraphics[width=0.9\columnwidth]{figs/legend_image_num.pdf}
    \vskip -0.2in
    \subfloat[]{
        \includegraphics[width=0.42\columnwidth]{figs/TTFT_all.pdf}
        \label{fig:10a}
    }
    \subfloat[]{
        \includegraphics[width=0.4\columnwidth]{figs/Score_all.pdf}
        \label{fig:10b}
    }
    \caption{The performance of \sys~as the number of images increases. For clarity, we only present the results of \sys-32. Other variants of \sys~show similar patterns.}
    \label{fig:10}
    % \vskip -0.2in
\end{figure}

% \subsection{Latency and throughput performance of InfoBlend}
% To assess Infoblend's latency and throughput performance, we leverage VLLM's OpenAI-compatible API server to simulate real-world user request patterns. We first select $n$ samples from MMDU and pre-generate KV caches for their contexts. Subsequently, we simulate user request behavior by repeatedly sending the user queries along with the cache\_ids of these 
% $n$ samples at a specified request rate over a period of time. by varying the request rate, We measure the latency and throughput across different experimental conditions.

% In Figure, we present a comparison of latency and throughput between InfoBlend and CacheBlend at varying request rates.  Compared to CacheBlend, InfoBlend achieves up to 80\% reduction in TTFT and 2-3 $\times$ improvement in throughput. This gap increases as the request rate rises.
\section{Related work}

The literature related to our work can be classified into two
categories: general purpose DR techniques
(\autoref{sec:relatedWorkGeneralPurpose}) and topology-aware techniques
(\autoref{sec:relatedWorkTopology}).

\subsection{General purpose dimensionality reduction}
\label{sec:relatedWorkGeneralPurpose}

Numerous DR techniques have been proposed and the related literature has been
summarized in several books~\cite{borg97, dimensionReductionBook} and surveys
\cite{surveyDimensionReduction2, surveyDimensionReduction1, NonatoA19}.
Principal Component Analysis (PCA)~\cite{pearson1901liii} is by far the most
popular linear DR technique.
Although it is an indispensable tool for data analysis,
its linear nature does not always allow it to apprehend complex non-linear
phenomena. One of the first non linear DR methods is the multidimensional
scaling (MDS)~\cite{torgerson1952multidimensional}. It aims at preserving as far
as possible the pairwise distances in the high- and low-dimensional point
clouds.
Another approach, particularly related to our work,
consists in optimizing an autoencoder neural network~\cite{hinton_reducing_2006}.
The \textit{encoder} is used to represent the explicit projection map from the
high-dimensional input space to the low-dimensional representation
space, while the \textit{decoder} tries to reconstruct the input data
from its encoded representation.
We will refer to these methods as \emph{global} methods.

Global methods have been used successfully in many applications, but
they do not take into
account the possible distribution of the input points over some implicit,
unknown manifold. This may lead to the unwanted preservation of distances
between points that are close in the ambient space but far apart on this
manifold. \emph{Locally topology-aware} methods have therefore been
introduced to address this issue. For instance,
Isomap~\cite{tenenbaum_global_2000}
preserves geodesic distances on a captured manifold structure of the
input data.
%\remove{Because it suffers from computational
%inefficiencies, Isomap was sped up with the use of landmark points (L-Isomap
%\cite{silva2003global}).}
Other methods also feature neighborhood preservation objectives.
For example, Local Linear Embedding (LLE)~\cite{roweis2000nonlinear} relies
on linear reconstructions of local neighborhoods.
Other methods leverage additional landmarks~\cite{silva2003global} or user-provided
control points~\cite{joia:tvcg:2011}.
%Some local methods additionally support user
%constraints expressed as control points~\cite{joia:tvcg:2011}.

All these methods try to preserve local
Euclidean distances when projecting to a lower dimension.
However, this can sometimes lack relevance in the applications,
especially with high-dimensional datasets for which
the distribution of pairwise Euclidean distances tend to be uniform.
For such cases, local distance preservation fails at characterizing
well relevant local relations.
To alleviate this issue, SNE~\cite{hinton2002stochastic} and later
t\nobreakdash-SNE~\cite{van2008visualizing} use a conditional probability
formulation to represent similarities between points and try to
have similar distributions both in high- and low-dimension thanks to a
Kullback--Leibler divergence minimization.
More recently UMAP has been introduced~\cite{mcinnes2018umap} along a
theoretical foundation on category theory.
It provides results that are similar visually to t-SNE, but in a more
scalable way.
Variants were later introduced to better preserve the global structure in the embedding, such as TriMAP~\cite{amid2022trimap} that constrains the proximity order within triplets of points, or PaCMAP~\cite{wang_understanding_2021} that adds constraints on more distant point pairs.
Although these methods succeed in preserving the local topology, they are not
explicitly aware of the global structure
of the input, which can lead to the loss of noteworthy global,
topological features.

\subsection{Globally topology-aware dimensionality reduction}
\label{sec:relatedWorkTopology}

Topology-based methods have become popular over the last
two decades in data analysis and
visualization~\cite{heine16} and have been applied to various areas:
astrophysics~\cite{sousbie11, shivashankar2016felix},
biological imaging~\cite{beiBrain18, carr04, topoAngler},
quantum chemistry~\cite{chemistry_vis14,harshChemistry, D2CP05893F},
fluid dynamics~\cite{kasten_tvcg11, NauleauVBBT22},
material sciences~\cite{gyulassy_vis07, gyulassy_vis15, SolerPDPCT19},
turbulent combustion~\cite{gyulassy_ev14, laney_vis06}. They leverage tools that
define concise signatures of the data based on its topological properties and
that have been summarized in topological data analysis reference
books ~\cite{edelsbrunner_computational_2010, zomorodian_computational_2010}
and surveys~\cite{chazal_introduction_2021}.

Several DR methods have been proposed
by the visualization community to preserve specific topological signatures
of the input data. For instance, terrain metaphors have been
investigated for the visualization of an input high-dimensional scalar
field, in the form of a three-dimensional terrain, whose elevation yields an
identical contour tree~\cite{Weber:2007} or an identical set of separatrices
\cite{gerber2010, gerber2013}.
This framework has been extended to density
estimators~\cite{OesterlingHJS10,
OesterlingSTHKEW10, OHJSH11, Oesterling0WS13} for the support of
high-dimensional point clouds. However, such metaphors completely discard
the metric information of the input space~\cite{OesterlingHJS10}, possibly
placing next to each other points which are arbitrarily far
in the input space (and reciprocally). Yan et al.~\cite{abs-1806-08460}
introduced a DR approach driven by the Mapper structure~\cite{SinghMC07}, an
approximation of the Reeb graph~\cite{reeb46}, which can capture in practice
large handles in the data, however without guarantees, since the number of handles in the considered manifold is only an upper bound on the number of loops in the Reeb graph~\cite{edelsbrunner_computational_2010}.

To incorporate the metric information from the input data while
preserving at the same time some of their topological characteristics, several
approaches have focused on driving the projection by
the \emph{persistence diagram}
of the Rips filtration of the point cloud (see \autoref{sec:persistentHomology}
for  a technical description).
Carriere et al.~\cite{carriere2021optimizing} presented a generic persistence
optimization framework with an application to dimensionality reduction.
Their approach explicitly minimizes the Wasserstein distance
(\autoref{sec:persistentHomology}) between the $1$-dimensional persistence
diagrams in high and low dimensions. However, this approach solely focuses on
this penalization term. As a result,
although the number and persistence of cycles  may be well-preserved,
the solver tends to produce cycles in low dimensions which involve arbitrary
points (e.g., which were not necessarily located along the cycles in high
dimensions), which challenges visual interpretation, as later
detailed in \autoref{sec:results:analysis}.

To enforce a correspondence between the topological
features at the data point level, additional structures need to be preserved.
For the specific case of $0$-dimensional persistent homology (\PH{0}),
Doraiswamy et al. introduced \emph{TopoMap}~\cite{doraiswamy2020topomap}, an
algorithm which constructively preserves the \emph{persistence pairs}
(\autoref{sec:persistentHomology}) through the preservation of the minimum
spanning tree of the data. An accelerated version, with improved layouts, has
recently been proposed~\cite{guardieiro2024topomap++}.
Alternative approaches have considered the usage of an optimization framework
(typically based on an autoencoder neural network
\cite{hinton_reducing_2006}), with the integration of specific topology-aware
losses~\cite{moor2020topological,barannikov2021representation,
nelson2022topology,trofimov2023learning,schonenberger2020witness}. Among them,
a prominent approach is the \emph{Topological Autoencoders}
(TopoAE)~\cite{moor2020topological}. Its loss aims at preserving
the diameter of the simplices involved in
persistence pairs, when going from high to low dimensions and reciprocally.
However, the above techniques focused in practice on the
preservation of \PH{0} and did not, to our knowledge, report experiments
regarding the preservation of higher dimensional PH.
Specifically, we show in \autoref{sec:analysis} that, while a zero
TopoAE loss indeed implies a preservation of the persistence pairs for \PH{0},
it is not the case for higher dimensional PH. We provide a counter example for
\PH{1}, which is addressed by our novel, generalized loss.

\section{Discussion and Future Work}\label{sec:discussion}
This paper pioneers the novel approach of selective response, showing that withholding responses can be a powerful tool for GenAI systems. By opting not to answer every query as accurately as it can---particularly when new or complex topics emerge---GenAI can encourage user participation on community-driven platforms and thereby generate more high-quality data for future training. This mechanism ultimately enhances GenAI's long-term performance and revenue. From a welfare perspective, our results indicate that such selective engagement can also benefit users, leading to better solutions and increased overall satisfaction. Since this work is the first to address selective response strategies for GenAI, numerous promising directions remain for future research; we highlight some of them below. 

First, from a technical standpoint, all of the results in this paper rely on Assumption~\ref{assumption: data lip}, involving the lipshitz condition of the accuracy function and the sensitivity parameter $\beta$. Future work could seek to relax this assumption. Furthermore, our constrained optimization approach in Subsection~\ref{sec: welfare constrained revenue maximization} could be extended to approximate the optimal (continuous) strategy instead of the optimal discrete strategy.

Second, our stylized model adopts the simplifying---though unrealistic---assumption that only a single GenAI platform exists. Admittedly, this makes it easier to focus on the idea of selective responses, and indeed, this assumption is pivotal in keeping our analysis tractable. Future research could explore scenarios with multiple GenAI platforms and human-centered forums. In such settings, one platform's selective response might redirect users not only to forums but also to competing GenAI platforms, leading to the tragedy of the commons \cite{hardin1968tragedy}: Although all GenAI platforms benefit from fresh data generation, none may choose to respond selectively if it means losing users to competitors. 

Third, we assumed Forum behaves non-strategically. In reality, human-centered platforms often monetize their data by selling it to GenAI platforms, adding a further layer of strategic interaction for GenAI. Moreover, data transfer between the platforms can form the basis for collaboration: GenAI could employ selective response to bolster Forum content creation, and Forum could, in turn, attribute that content to GenAI for subsequent use in retraining.


%Third, we make the (again) simplifying assumption that Forum is non-strategic. However, in practice, human-centered platforms can sell their data to GenAI platforms. This adds additional considerations for GenAI. Furthermore, data transmission between the platforms can also become the basis for collaboration: GenAI can use selective response to ensure enough content is generated in Forum, and Forum could provide the data attributed to this mechanism back to GenAI. 


%Second, this paper makes the simplifying yet unrealistic assumption of the existence of one GenAI platform. Indeed, this simplifies many aspects and allows us to analyze selective responses. Future work could address the data generation process with more than one GenAI platform and possibly several human-centered forums. In such a case, selective response of one GenAI platform can either drive users to forums or to other GenAI platforms; thus, we might face a tragedy of the commons situation~\ref{hardin1968tragedy}, where all GenAI platforms are interested in fresh data generation but none volunteer to selectively respond and lose users. 

%This paper examines the competition between a generative AI platform and human-based platforms, challenging the assumption that always providing answers is optimal. We analyzed the impact of withholding answers on GenAI's revenue and developed an efficient approximately optimal algorithm for this purpose. We further explored how withholding affects users, showing that it can lead to better outcomes compared to always answering. Specifically, we demonstrated that withholding can Pareto-dominate this strategy and derived the necessary and sufficient conditions for that. Finally, we proposed a second approximately optimal algorithm that maximizes GenAI's revenue while ensuring users are better off than when GenAI answers all queries.

%On a more conceptual level, our model assumes that GenAI’s data comes solely from the competing platform (Forum). Future research could explore a scenario where GenAI can purchase additional data from a third party. This extension could provide valuable insights into the interplay between withholding answers and data purchasing, and whether these two strategies can complement each other or must be traded off.
Software development is increasingly conceived as a collaboration activity between developers and AIs. Indeed, IDEs already implement features to enable interactive development, with AI suggesting implementations that are reused by developers.

Although multiple studies show this interaction can be successful, there is still limited understanding of how the models must be configured and used in the context of code generation tasks. This study addresses this gap, systematically investigating the impact of several key parameters, including the repeated submission of a prompt to accommodate for the non-deterministic nature of the models.

Our study reveals several key findings about the usage of ChatGPT. In particular, we discovered how creativity, although up to a limited extent, is useful to increase the range of methods whose code can be generated correctly. A major role is played by parameter top-p, which is commonly underrated, and instead has a major impact on the correctness of the results, with lower values producing better results. Finally, prompts should be submitted multiple times, with $5$ repetitions combined with a temperature of $1.2$ resulting in an effective configuration in our experiments.  

Future work concerns two main research directions. One is about replicating this experiment with other AI assistants, to validate our findings in multiple contexts. The second research direction concerns finding strategies to deal with the need to submit the same prompt multiple times to obtain a useful result, and thus developing approaches able to select or merge multiple responses automatically. 
\section{Acknowledgement}
We sincerely thank the anonymous shepherd for constructive comments and feedback in improving this work. 
This paper was supported by Yunnan Forestry and Grassland Science and Technology Innovation Joint Special Project (grant NO. 202404CB090017), Natural Science Foundation of China under Grant 62371269, Guangdong Innovative and Entrepreneurial Research Team Program (2021ZT09 L197), Meituan Academy of Robotics Shenzhen.

\newpage
\balance



% \section{Related Work}

\subsection{View-Dependent Control}
View-dependent representations have a long history in computer graphics.
In his pioneering work, Rademacher proposed interpolating between \textit{key viewpoints} and associated \textit{key deformations} to manipulate 3D models~\cite{rademacher1999view}.
Other researchers have extended the idea to create 3D animation systems~\cite{10.1111:j.1467-8659.2004.00772.x}, streamline the modeling process~\cite{DBLP:journals/corr/abs-2103-15472}, and integrate physical simulation\cite{koyama2013view}.
Of particular note, Rivers et al.~\cite{rivers25Dcartoonmodels} introduced \textit{2.5D Cartoon Models}, a combination of planar meshes transformed, based upon view angle, so as to appears three dimensional.
Our work draws upon these works but is, to our knowledge, the first work to attempt to use view-dependent techniques to retarget 3D motion onto 2D characters.   

\subsection{Animation from 2D Images}

% output is still 2D
Many researchers have proposed different methods for creating animations from 2D images. Hornung et al.~\cite{Hornung2007anim2Dpicmotion} presented a method to deform a character from a photograph given user-provided joint annotations.
\textit{Toonsynth}~\cite{Dvoroznak18-SIG} and \textit{Neural Puppet}~\cite{poursaeed2020neural} both present methods to create new images of hand-drawn characters from examples.
% output is 3D model
Other researchers have proposed methods of obtaining 3D geometry from 2D sketches~\cite{igarashi2006teddy, Dvoroznak20-SA} and images~\cite{ArtiSketch,weng2019photo}.
% focus on sketches specifically
A number of works have specifically focused on childlike drawings.
Lingens et al.~\cite{lingens2020towards} proposed an evolutionary algorithm for animating children's drawings. 
\textit{MagicToon}~\cite{feng2017magictoon} creates a 3D model from childlike drawings for AR applications.
Similar to our work, Smith et al.~\cite{SmithHodgins} proposed a method for animating childlike drawings using 3D skeletal motion. 
However, the resulting animations are only suitable for use in 2D applications and the type of motions it supports are limited.

Unlike these previous works, our solution can be used in 3D contexts but does not create a 3D model. We instead relying upon a view-dependent formulation of the animated character.

% \section{Acknowledgments}

% Identification of funding sources and other support, and thanks to
% individuals and groups that assisted in the research and the
% preparation of the work should be included in an acknowledgment
% section, which is placed just before the reference section in your
% document.

% This section has a special environment:
% \begin{verbatim}
%   \begin{acks}
%   ...
%   \end{acks}
% \end{verbatim}
% so that the information contained therein can be more easily collected
% during the article metadata extraction phase, and to ensure
% consistency in the spelling of the section heading.

% Authors should not prepare this section as a numbered or unnumbered {\verb|\section|}; please use the ``{\verb|acks|}'' environment.

% \section{Appendices}

% If your work needs an appendix, add it before the
% ``\verb|\end{document}|'' command at the conclusion of your source
% document.

% Start the appendix with the ``\verb|appendix|'' command:
% \begin{verbatim}
%   \appendix
% \end{verbatim}
% and note that in the appendix, sections are lettered, not
% numbered. This document has two appendices, demonstrating the section
% and subsection identification method.

% \section{SIGCHI Extended Abstracts}

% The ``\verb|sigchi-a|'' template style (available only in \LaTeX\ and
% not in Word) produces a landscape-orientation formatted article, with
% a wide left margin. Three environments are available for use with the
% ``\verb|sigchi-a|'' template style, and produce formatted output in
% the margin:
% \begin{itemize}
% \item {\verb|sidebar|}:  Place formatted text in the margin.
% \item {\verb|marginfigure|}: Place a figure in the margin.
% \item {\verb|margintable|}: Place a table in the margin.
% \end{itemize}

% %%
% %% The acknowledgments section is defined using the "acks" environment
% %% (and NOT an unnumbered section). This ensures the proper
% %% identification of the section in the article metadata, and the
% %% consistent spelling of the heading.
% \begin{acks}
% To Robert, for the bagels and explaining CMYK and color spaces.
% \end{acks}

% %%
% %% The next two lines define the bibliography style to be used, and
% %% the bibliography file.
\bibliographystyle{unsrt}
% \bibliography{sample-base}
\bibliography{reference}

% %%
% %% If your work has an appendix, this is the place to put it.
% \appendix

% \section{Research Methods}

% \subsection{Part One}

% Lorem ipsum dolor sit amet, consectetur adipiscing elit. Morbi
% malesuada, quam in pulvinar varius, metus nunc fermentum urna, id
% sollicitudin purus odio sit amet enim. Aliquam ullamcorper eu ipsum
% vel mollis. Curabitur quis dictum nisl. Phasellus vel semper risus, et
% lacinia dolor. Integer ultricies commodo sem nec semper.

% \subsection{Part Two}

% Etiam commodo feugiat nisl pulvinar pellentesque. Etiam auctor sodales
% ligula, non varius nibh pulvinar semper. Suspendisse nec lectus non
% ipsum convallis congue hendrerit vitae sapien. Donec at laoreet
% eros. Vivamus non purus placerat, scelerisque diam eu, cursus
% ante. Etiam aliquam tortor auctor efficitur mattis.

% \section{Online Resources}

% Nam id fermentum dui. Suspendisse sagittis tortor a nulla mollis, in
% pulvinar ex pretium. Sed interdum orci quis metus euismod, et sagittis
% enim maximus. Vestibulum gravida massa ut felis suscipit
% congue. Quisque mattis elit a risus ultrices commodo venenatis eget
% dui. Etiam sagittis eleifend elementum.

% Nam interdum magna at lectus dignissim, ac dignissim lorem
% rhoncus. Maecenas eu arcu ac neque placerat aliquam. Nunc pulvinar
% massa et mattis lacinia.

\end{document}
\endinput
%%
%% End of file `sample-sigconf.tex'.
