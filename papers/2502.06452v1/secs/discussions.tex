\section{Discussions}

\subsection{The Asymmetry of the Point Spread Function (PSF) in Microscopy}

In the ideal imaging model, the Point Spread Function (PSF) of a microscope is symmetric with respect to the focal plane. 
This symmetry allows algorithms to estimate the absolute defocus distance but prevents them from determining whether the defocus is above or below the focal plane, thereby rendering one-shot autofocusing seemingly impractical.
However, in real optical microscopy systems, the PSF often exhibits significant asymmetry due to refractive index mismatches among the different media in the imaging path, such as the slide, sample, cover slip, and the surrounding medium like air or immersion oil. 
These mismatches introduce aberrations, including spherical aberration, coma, astigmatism, field curvature, and distortion, which disrupt the ideal symmetric distribution of the PSF. 
Figures~\ref{fig:psf}(a) and \ref{fig:psf}(b) present the 3D PSF and 2D PSF of our developed Whole Slide Imaging (WSI) device.
These visualizations are generated using the Gibson \& Lanni PSF model~\cite{Gibson:89} within the open-source software Fiji~\cite{Schindelin2012-jh} and the PSF Generator plugin\footnote{http://bigwww.epfl.ch/publications/kirshner1103.html}.
Figure~\ref{fig:psf}(c) shows images at symmetric defocus distances on both sides of the focal plane.
Figure~\ref{fig:psf}(d) illustrates the differences in pixel grayscale values at the same position for images at 5\si{\micro\meter} and -5\si{\micro\meter}.
These visualizations also demonstrate that, in real optical microscopy systems, the PSF is asymmetric.
Additional theoretical analysis on the PSF is provided in the supplementary materials.

\begin{figure}[H]
	\centering
	\includegraphics[width=\linewidth]{figs/PSF_fire_sq.pdf}
	\caption{\textbf{The asymmetry of PSF.} (a) and (b) illustrate the PSF of a microscopy imaging system, highlighting its asymmetry with respect to the focal plane. (c) demonstrates the defocused imaging relative to the focal plane, and (d) presents the comparison of the gray values of pixels at the same positions corresponding to 5\si{\micro\meter} and -5\si{\micro\meter}. Both of them provide corroborative evidence for the disparities in images at the corresponding locations.}
	\label{fig:psf}
\end{figure}

The asymmetry of the PSF, though potentially detrimental to image quality, presents a unique opportunity for one-shot autofocusing. This phenomenon results in images with positive or negative defocus on either side of the focal plane exhibiting distinct characteristics. Although these differences are subtle, the sophisticated feature extraction capabilities of deep learning can effectively discern them. By capitalizing on this physical principle, we propose a one-shot learning-based network designed to estimate both the defocus distance and direction from a single image.

\subsection{Autofocus for Thick Specimens}

Autofocus is generally designed for a specific focal plane, assuming that most samples exhibit little variation in elevation over a field of view.
However, for very thick samples, such as those resulting from the slicing of pathological sections, different regions within the same field of view may lie on different focal planes (see supplementary material). This can lead to a scenario where focusing on one region causes others to appear blurry, complicating autofocus efforts.
To address such challenges, strategies may contain: 1) Designate a specific region of interest for the autofocus algorithm to target exclusively; 2) Employ z-stack image fusion strategy, capturing and fusing images at various z-axis positions to achieve a uniformly sharp image across the entire field of view.


