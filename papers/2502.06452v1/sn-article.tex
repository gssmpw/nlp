%Version 3 October 2023
% See section 11 of the User Manual for version history
%
%%%%%%%%%%%%%%%%%%%%%%%%%%%%%%%%%%%%%%%%%%%%%%%%%%%%%%%%%%%%%%%%%%%%%%
%%                                                                 %%
%% Please do not use \input{...} to include other tex files.       %%
%% Submit your LaTeX manuscript as one .tex document.              %%
%%                                                                 %%
%% All additional figures and files should be attached             %%
%% separately and not embedded in the \TeX\ document itself.       %%
%%                                                                 %%
%%%%%%%%%%%%%%%%%%%%%%%%%%%%%%%%%%%%%%%%%%%%%%%%%%%%%%%%%%%%%%%%%%%%%

%%\documentclass[referee,sn-basic]{sn-jnl}% referee option is meant for double line spacing

%%=======================================================%%
%% to print line numbers in the margin use lineno option %%
%%=======================================================%%

%%\documentclass[lineno,sn-basic]{sn-jnl}% Basic Springer Nature Reference Style/Chemistry Reference Style

%%======================================================%%
%% to compile with pdflatex/xelatex use pdflatex option %%
%%======================================================%%

%%\documentclass[pdflatex,sn-basic]{sn-jnl}% Basic Springer Nature Reference Style/Chemistry Reference Style


%%Note: the following reference styles support Namedate and Numbered referencing. By default the style follows the most common style. To switch between the options you can add or remove “Numbered” in the optional parenthesis. 
%%The option is available for: sn-basic.bst, sn-vancouver.bst, sn-chicago.bst%  
 
% \documentclass[sn-nature]{sn-jnl}% Style for submissions to Nature Portfolio journals
%%\documentclass[sn-basic]{sn-jnl}% Basic Springer Nature Reference Style/Chemistry Reference Style
\documentclass[sn-mathphys-num]{sn-jnl}% Math and Physical Sciences Numbered Reference Style 
%%\documentclass[sn-mathphys-ay]{sn-jnl}% Math and Physical Sciences Author Year Reference Style
%%\documentclass[sn-aps]{sn-jnl}% American Physical Society (APS) Reference Style
%%\documentclass[sn-vancouver,Numbered]{sn-jnl}% Vancouver Reference Style
%%\documentclass[sn-apa]{sn-jnl}% APA Reference Style 
%%\documentclass[sn-chicago]{sn-jnl}% Chicago-based Humanities Reference Style

%%%% Standard Packages
%%<additional latex packages if required can be included here>

\usepackage{graphicx}%
\usepackage{multirow}%
\usepackage{amsmath,amssymb,amsfonts}%
\usepackage{amsthm}%
\usepackage{mathrsfs}%
\usepackage[title]{appendix}%
\usepackage{xcolor}%
\usepackage{textcomp}%
\usepackage{manyfoot}%
\usepackage{booktabs}%
\usepackage{algorithm}%
\usepackage{algorithmicx}%
\usepackage{algpseudocode}%
\usepackage{listings}%
%\usepackage{lineno} % Include the lineno package

\usepackage{siunitx} % for SI
%%%%

%%%%%=============================================================================%%%%
%%%%  Remarks: This template is provided to aid authors with the preparation
%%%%  of original research articles intended for submission to journals published 
%%%%  by Springer Nature. The guidance has been prepared in partnership with 
%%%%  production teams to conform to Springer Nature technical requirements. 
%%%%  Editorial and presentation requirements differ among journal portfolios and 
%%%%  research disciplines. You may find sections in this template are irrelevant 
%%%%  to your work and are empowered to omit any such section if allowed by the 
%%%%  journal you intend to submit to. The submission guidelines and policies 
%%%%  of the journal take precedence. A detailed User Manual is available in the 
%%%%  template package for technical guidance.
%%%%%=============================================================================%%%%

%% as per the requirement new theorem styles can be included as shown below
\theoremstyle{thmstyleone}%
\newtheorem{theorem}{Theorem}%  meant for continuous numbers
%%\newtheorem{theorem}{Theorem}[section]% meant for sectionwise numbers
%% optional argument [theorem] produces theorem numbering sequence instead of independent numbers for Proposition
\newtheorem{proposition}[theorem]{Proposition}% 
%%\newtheorem{proposition}{Proposition}% to get separate numbers for theorem and proposition etc.

\theoremstyle{thmstyletwo}%
\newtheorem{example}{Example}%
\newtheorem{remark}{Remark}%

\theoremstyle{thmstylethree}%
\newtheorem{definition}{Definition}%

\raggedbottom
%%\unnumbered% uncomment this for unnumbered level heads

\begin{document}

\title[Article Title]{SparseFocus: Learning-based One-shot Autofocus for Microscopy with Sparse Content}

%%=============================================================%%
%% GivenName	-> \fnm{Joergen W.}
%% Particle	-> \spfx{van der} -> surname prefix
%% FamilyName	-> \sur{Ploeg}
%% Suffix	-> \sfx{IV}
%% \author*[1,2]{\fnm{Joergen W.} \spfx{van der} \sur{Ploeg} 
%%  \sfx{IV}}\email{iauthor@gmail.com}
%%=============================================================%%


% 匿名投稿
% ------------------------------
\author*[1]{\fnm{Yongping} \sur{Zhai}}\email{zhaiyongping08@nudt.edu.cn}
\equalcont{These authors contributed equally to this work.}

\author[1]{\fnm{Xiaoxi} \sur{Fu}}
\equalcont{These authors contributed equally to this work.}

\author[1]{\fnm{Qiang} \sur{Su}}
\equalcont{These authors contributed equally to this work.}

\author[1]{\fnm{Jia} \sur{Hu}}

\author[1]{\fnm{Yake} \sur{Zhang}}

\author[1]{\fnm{Yunfeng} \sur{Zhou}}

\author[1]{\fnm{Chaofan} \sur{Zhang}}

\author[1]{\fnm{Xiao} \sur{Li}}

\author[1]{\fnm{Wenxin} \sur{Wang}}

% Department of Information, Daping Hospital, Army Medical University, Chongqing 400042, China, wudd01@tmmu.edu.cn
\author*[2]{\fnm{Dongdong} \sur{Wu}}\email{wudd01@tmmu.edu.cn}

\author*[3]{\fnm{Shen} \sur{Yan}}\email{yanshen12@nudt.edu.cn}

\affil[1]{\orgdiv{College of Advanced Interdisciplinary Studies}, \orgname{National University of Defense Technology}, \orgaddress{\city{Changsha}, \postcode{410073}, \state{Hunan}, \country{China}}}

\affil[2]{\orgdiv{Department of Information}, \orgname{Army Medical University}, \orgaddress{\city{Chongqing}, \postcode{400042}, \state{Chongqing}, \country{China}}}

\affil[3]{\orgdiv{College of System Engineering}, \orgname{National University of Defense Technology}, \orgaddress{\city{Changsha}, \postcode{410073}, \state{Hunan}, \country{China}}}
% ----------------------------
% \author{Anonymous Author(s)}

\abstract{Autofocus is necessary for high-throughput and real-time scanning in microscopic imaging. Traditional methods rely on complex hardware or iterative hill-climbing algorithms. Recent learning-based approaches have demonstrated remarkable efficacy in a one-shot setting, avoiding hardware modifications or iterative mechanical lens adjustments. 
However, in this paper, we highlight a significant challenge that the richness of image content can significantly affect autofocus performance. When the image content is sparse, previous autofocus methods, whether traditional climbing-hill or learning-based, tend to fail. 
To tackle this, we propose a content-importance-based solution, named SparseFocus, featuring a novel two-stage pipeline. The first stage measures the importance of regions within the image, while the second stage calculates the defocus distance from selected important regions. To validate our approach and benefit the research community, we collect a large-scale dataset comprising millions of labelled defocused images, encompassing both dense, sparse and extremely sparse scenarios. Experimental results show that SparseFocus surpasses existing methods, effectively handling all levels of content sparsity. Moreover, we integrate SparseFocus into our Whole Slide Imaging (WSI) system that performs well in real-world applications. The code and dataset will be made available upon the publication of this paper.}


\keywords{Autofocus, Whole Slide Imaging, One-shot Autofocus, Learning-based Autofocus, Autofocus with sparse content, Autofocus for Microscopy}

\maketitle

\section{Introduction}
\label{sec:introduction}
The business processes of organizations are experiencing ever-increasing complexity due to the large amount of data, high number of users, and high-tech devices involved \cite{martin2021pmopportunitieschallenges, beerepoot2023biggestbpmproblems}. This complexity may cause business processes to deviate from normal control flow due to unforeseen and disruptive anomalies \cite{adams2023proceddsriftdetection}. These control-flow anomalies manifest as unknown, skipped, and wrongly-ordered activities in the traces of event logs monitored from the execution of business processes \cite{ko2023adsystematicreview}. For the sake of clarity, let us consider an illustrative example of such anomalies. Figure \ref{FP_ANOMALIES} shows a so-called event log footprint, which captures the control flow relations of four activities of a hypothetical event log. In particular, this footprint captures the control-flow relations between activities \texttt{a}, \texttt{b}, \texttt{c} and \texttt{d}. These are the causal ($\rightarrow$) relation, concurrent ($\parallel$) relation, and other ($\#$) relations such as exclusivity or non-local dependency \cite{aalst2022pmhandbook}. In addition, on the right are six traces, of which five exhibit skipped, wrongly-ordered and unknown control-flow anomalies. For example, $\langle$\texttt{a b d}$\rangle$ has a skipped activity, which is \texttt{c}. Because of this skipped activity, the control-flow relation \texttt{b}$\,\#\,$\texttt{d} is violated, since \texttt{d} directly follows \texttt{b} in the anomalous trace.
\begin{figure}[!t]
\centering
\includegraphics[width=0.9\columnwidth]{images/FP_ANOMALIES.png}
\caption{An example event log footprint with six traces, of which five exhibit control-flow anomalies.}
\label{FP_ANOMALIES}
\end{figure}

\subsection{Control-flow anomaly detection}
Control-flow anomaly detection techniques aim to characterize the normal control flow from event logs and verify whether these deviations occur in new event logs \cite{ko2023adsystematicreview}. To develop control-flow anomaly detection techniques, \revision{process mining} has seen widespread adoption owing to process discovery and \revision{conformance checking}. On the one hand, process discovery is a set of algorithms that encode control-flow relations as a set of model elements and constraints according to a given modeling formalism \cite{aalst2022pmhandbook}; hereafter, we refer to the Petri net, a widespread modeling formalism. On the other hand, \revision{conformance checking} is an explainable set of algorithms that allows linking any deviations with the reference Petri net and providing the fitness measure, namely a measure of how much the Petri net fits the new event log \cite{aalst2022pmhandbook}. Many control-flow anomaly detection techniques based on \revision{conformance checking} (hereafter, \revision{conformance checking}-based techniques) use the fitness measure to determine whether an event log is anomalous \cite{bezerra2009pmad, bezerra2013adlogspais, myers2018icsadpm, pecchia2020applicationfailuresanalysispm}. 

The scientific literature also includes many \revision{conformance checking}-independent techniques for control-flow anomaly detection that combine specific types of trace encodings with machine/deep learning \cite{ko2023adsystematicreview, tavares2023pmtraceencoding}. Whereas these techniques are very effective, their explainability is challenging due to both the type of trace encoding employed and the machine/deep learning model used \cite{rawal2022trustworthyaiadvances,li2023explainablead}. Hence, in the following, we focus on the shortcomings of \revision{conformance checking}-based techniques to investigate whether it is possible to support the development of competitive control-flow anomaly detection techniques while maintaining the explainable nature of \revision{conformance checking}.
\begin{figure}[!t]
\centering
\includegraphics[width=\columnwidth]{images/HIGH_LEVEL_VIEW.png}
\caption{A high-level view of the proposed framework for combining \revision{process mining}-based feature extraction with dimensionality reduction for control-flow anomaly detection.}
\label{HIGH_LEVEL_VIEW}
\end{figure}

\subsection{Shortcomings of \revision{conformance checking}-based techniques}
Unfortunately, the detection effectiveness of \revision{conformance checking}-based techniques is affected by noisy data and low-quality Petri nets, which may be due to human errors in the modeling process or representational bias of process discovery algorithms \cite{bezerra2013adlogspais, pecchia2020applicationfailuresanalysispm, aalst2016pm}. Specifically, on the one hand, noisy data may introduce infrequent and deceptive control-flow relations that may result in inconsistent fitness measures, whereas, on the other hand, checking event logs against a low-quality Petri net could lead to an unreliable distribution of fitness measures. Nonetheless, such Petri nets can still be used as references to obtain insightful information for \revision{process mining}-based feature extraction, supporting the development of competitive and explainable \revision{conformance checking}-based techniques for control-flow anomaly detection despite the problems above. For example, a few works outline that token-based \revision{conformance checking} can be used for \revision{process mining}-based feature extraction to build tabular data and develop effective \revision{conformance checking}-based techniques for control-flow anomaly detection \cite{singh2022lapmsh, debenedictis2023dtadiiot}. However, to the best of our knowledge, the scientific literature lacks a structured proposal for \revision{process mining}-based feature extraction using the state-of-the-art \revision{conformance checking} variant, namely alignment-based \revision{conformance checking}.

\subsection{Contributions}
We propose a novel \revision{process mining}-based feature extraction approach with alignment-based \revision{conformance checking}. This variant aligns the deviating control flow with a reference Petri net; the resulting alignment can be inspected to extract additional statistics such as the number of times a given activity caused mismatches \cite{aalst2022pmhandbook}. We integrate this approach into a flexible and explainable framework for developing techniques for control-flow anomaly detection. The framework combines \revision{process mining}-based feature extraction and dimensionality reduction to handle high-dimensional feature sets, achieve detection effectiveness, and support explainability. Notably, in addition to our proposed \revision{process mining}-based feature extraction approach, the framework allows employing other approaches, enabling a fair comparison of multiple \revision{conformance checking}-based and \revision{conformance checking}-independent techniques for control-flow anomaly detection. Figure \ref{HIGH_LEVEL_VIEW} shows a high-level view of the framework. Business processes are monitored, and event logs obtained from the database of information systems. Subsequently, \revision{process mining}-based feature extraction is applied to these event logs and tabular data input to dimensionality reduction to identify control-flow anomalies. We apply several \revision{conformance checking}-based and \revision{conformance checking}-independent framework techniques to publicly available datasets, simulated data of a case study from railways, and real-world data of a case study from healthcare. We show that the framework techniques implementing our approach outperform the baseline \revision{conformance checking}-based techniques while maintaining the explainable nature of \revision{conformance checking}.

In summary, the contributions of this paper are as follows.
\begin{itemize}
    \item{
        A novel \revision{process mining}-based feature extraction approach to support the development of competitive and explainable \revision{conformance checking}-based techniques for control-flow anomaly detection.
    }
    \item{
        A flexible and explainable framework for developing techniques for control-flow anomaly detection using \revision{process mining}-based feature extraction and dimensionality reduction.
    }
    \item{
        Application to synthetic and real-world datasets of several \revision{conformance checking}-based and \revision{conformance checking}-independent framework techniques, evaluating their detection effectiveness and explainability.
    }
\end{itemize}

The rest of the paper is organized as follows.
\begin{itemize}
    \item Section \ref{sec:related_work} reviews the existing techniques for control-flow anomaly detection, categorizing them into \revision{conformance checking}-based and \revision{conformance checking}-independent techniques.
    \item Section \ref{sec:abccfe} provides the preliminaries of \revision{process mining} to establish the notation used throughout the paper, and delves into the details of the proposed \revision{process mining}-based feature extraction approach with alignment-based \revision{conformance checking}.
    \item Section \ref{sec:framework} describes the framework for developing \revision{conformance checking}-based and \revision{conformance checking}-independent techniques for control-flow anomaly detection that combine \revision{process mining}-based feature extraction and dimensionality reduction.
    \item Section \ref{sec:evaluation} presents the experiments conducted with multiple framework and baseline techniques using data from publicly available datasets and case studies.
    \item Section \ref{sec:conclusions} draws the conclusions and presents future work.
\end{itemize}

\begin{table*}[t]
\centering
\fontsize{11pt}{11pt}\selectfont
\begin{tabular}{lllllllllllll}
\toprule
\multicolumn{1}{c}{\textbf{task}} & \multicolumn{2}{c}{\textbf{Mir}} & \multicolumn{2}{c}{\textbf{Lai}} & \multicolumn{2}{c}{\textbf{Ziegen.}} & \multicolumn{2}{c}{\textbf{Cao}} & \multicolumn{2}{c}{\textbf{Alva-Man.}} & \multicolumn{1}{c}{\textbf{avg.}} & \textbf{\begin{tabular}[c]{@{}l@{}}avg.\\ rank\end{tabular}} \\
\multicolumn{1}{c}{\textbf{metrics}} & \multicolumn{1}{c}{\textbf{cor.}} & \multicolumn{1}{c}{\textbf{p-v.}} & \multicolumn{1}{c}{\textbf{cor.}} & \multicolumn{1}{c}{\textbf{p-v.}} & \multicolumn{1}{c}{\textbf{cor.}} & \multicolumn{1}{c}{\textbf{p-v.}} & \multicolumn{1}{c}{\textbf{cor.}} & \multicolumn{1}{c}{\textbf{p-v.}} & \multicolumn{1}{c}{\textbf{cor.}} & \multicolumn{1}{c}{\textbf{p-v.}} &  &  \\ \midrule
\textbf{S-Bleu} & 0.50 & 0.0 & 0.47 & 0.0 & 0.59 & 0.0 & 0.58 & 0.0 & 0.68 & 0.0 & 0.57 & 5.8 \\
\textbf{R-Bleu} & -- & -- & 0.27 & 0.0 & 0.30 & 0.0 & -- & -- & -- & -- & - &  \\
\textbf{S-Meteor} & 0.49 & 0.0 & 0.48 & 0.0 & 0.61 & 0.0 & 0.57 & 0.0 & 0.64 & 0.0 & 0.56 & 6.1 \\
\textbf{R-Meteor} & -- & -- & 0.34 & 0.0 & 0.26 & 0.0 & -- & -- & -- & -- & - &  \\
\textbf{S-Bertscore} & \textbf{0.53} & 0.0 & {\ul 0.80} & 0.0 & \textbf{0.70} & 0.0 & {\ul 0.66} & 0.0 & {\ul0.78} & 0.0 & \textbf{0.69} & \textbf{1.7} \\
\textbf{R-Bertscore} & -- & -- & 0.51 & 0.0 & 0.38 & 0.0 & -- & -- & -- & -- & - &  \\
\textbf{S-Bleurt} & {\ul 0.52} & 0.0 & {\ul 0.80} & 0.0 & 0.60 & 0.0 & \textbf{0.70} & 0.0 & \textbf{0.80} & 0.0 & {\ul 0.68} & {\ul 2.3} \\
\textbf{R-Bleurt} & -- & -- & 0.59 & 0.0 & -0.05 & 0.13 & -- & -- & -- & -- & - &  \\
\textbf{S-Cosine} & 0.51 & 0.0 & 0.69 & 0.0 & {\ul 0.62} & 0.0 & 0.61 & 0.0 & 0.65 & 0.0 & 0.62 & 4.4 \\
\textbf{R-Cosine} & -- & -- & 0.40 & 0.0 & 0.29 & 0.0 & -- & -- & -- & -- & - & \\ \midrule
\textbf{QuestEval} & 0.23 & 0.0 & 0.25 & 0.0 & 0.49 & 0.0 & 0.47 & 0.0 & 0.62 & 0.0 & 0.41 & 9.0 \\
\textbf{LLaMa3} & 0.36 & 0.0 & \textbf{0.84} & 0.0 & {\ul{0.62}} & 0.0 & 0.61 & 0.0 &  0.76 & 0.0 & 0.64 & 3.6 \\
\textbf{our (3b)} & 0.49 & 0.0 & 0.73 & 0.0 & 0.54 & 0.0 & 0.53 & 0.0 & 0.7 & 0.0 & 0.60 & 5.8 \\
\textbf{our (8b)} & 0.48 & 0.0 & 0.73 & 0.0 & 0.52 & 0.0 & 0.53 & 0.0 & 0.7 & 0.0 & 0.59 & 6.3 \\  \bottomrule
\end{tabular}
\caption{Pearson correlation on human evaluation on system output. `R-': reference-based. `S-': source-based.}
\label{tab:sys}
\end{table*}



\begin{table}%[]
\centering
\fontsize{11pt}{11pt}\selectfont
\begin{tabular}{llllll}
\toprule
\multicolumn{1}{c}{\textbf{task}} & \multicolumn{1}{c}{\textbf{Lai}} & \multicolumn{1}{c}{\textbf{Zei.}} & \multicolumn{1}{c}{\textbf{Scia.}} & \textbf{} & \textbf{} \\ 
\multicolumn{1}{c}{\textbf{metrics}} & \multicolumn{1}{c}{\textbf{cor.}} & \multicolumn{1}{c}{\textbf{cor.}} & \multicolumn{1}{c}{\textbf{cor.}} & \textbf{avg.} & \textbf{\begin{tabular}[c]{@{}l@{}}avg.\\ rank\end{tabular}} \\ \midrule
\textbf{S-Bleu} & 0.40 & 0.40 & 0.19* & 0.33 & 7.67 \\
\textbf{S-Meteor} & 0.41 & 0.42 & 0.16* & 0.33 & 7.33 \\
\textbf{S-BertS.} & {\ul0.58} & 0.47 & 0.31 & 0.45 & 3.67 \\
\textbf{S-Bleurt} & 0.45 & {\ul 0.54} & {\ul 0.37} & 0.45 & {\ul 3.33} \\
\textbf{S-Cosine} & 0.56 & 0.52 & 0.3 & {\ul 0.46} & {\ul 3.33} \\ \midrule
\textbf{QuestE.} & 0.27 & 0.35 & 0.06* & 0.23 & 9.00 \\
\textbf{LlaMA3} & \textbf{0.6} & \textbf{0.67} & \textbf{0.51} & \textbf{0.59} & \textbf{1.0} \\
\textbf{Our (3b)} & 0.51 & 0.49 & 0.23* & 0.39 & 4.83 \\
\textbf{Our (8b)} & 0.52 & 0.49 & 0.22* & 0.43 & 4.83 \\ \bottomrule
\end{tabular}
\caption{Pearson correlation on human ratings on reference output. *not significant; we cannot reject the null hypothesis of zero correlation}
\label{tab:ref}
\end{table}


\begin{table*}%[]
\centering
\fontsize{11pt}{11pt}\selectfont
\begin{tabular}{lllllllll}
\toprule
\textbf{task} & \multicolumn{1}{c}{\textbf{ALL}} & \multicolumn{1}{c}{\textbf{sentiment}} & \multicolumn{1}{c}{\textbf{detoxify}} & \multicolumn{1}{c}{\textbf{catchy}} & \multicolumn{1}{c}{\textbf{polite}} & \multicolumn{1}{c}{\textbf{persuasive}} & \multicolumn{1}{c}{\textbf{formal}} & \textbf{\begin{tabular}[c]{@{}l@{}}avg. \\ rank\end{tabular}} \\
\textbf{metrics} & \multicolumn{1}{c}{\textbf{cor.}} & \multicolumn{1}{c}{\textbf{cor.}} & \multicolumn{1}{c}{\textbf{cor.}} & \multicolumn{1}{c}{\textbf{cor.}} & \multicolumn{1}{c}{\textbf{cor.}} & \multicolumn{1}{c}{\textbf{cor.}} & \multicolumn{1}{c}{\textbf{cor.}} &  \\ \midrule
\textbf{S-Bleu} & -0.17 & -0.82 & -0.45 & -0.12* & -0.1* & -0.05 & -0.21 & 8.42 \\
\textbf{R-Bleu} & - & -0.5 & -0.45 &  &  &  &  &  \\
\textbf{S-Meteor} & -0.07* & -0.55 & -0.4 & -0.01* & 0.1* & -0.16 & -0.04* & 7.67 \\
\textbf{R-Meteor} & - & -0.17* & -0.39 & - & - & - & - & - \\
\textbf{S-BertScore} & 0.11 & -0.38 & -0.07* & -0.17* & 0.28 & 0.12 & 0.25 & 6.0 \\
\textbf{R-BertScore} & - & -0.02* & -0.21* & - & - & - & - & - \\
\textbf{S-Bleurt} & 0.29 & 0.05* & 0.45 & 0.06* & 0.29 & 0.23 & 0.46 & 4.2 \\
\textbf{R-Bleurt} & - &  0.21 & 0.38 & - & - & - & - & - \\
\textbf{S-Cosine} & 0.01* & -0.5 & -0.13* & -0.19* & 0.05* & -0.05* & 0.15* & 7.42 \\
\textbf{R-Cosine} & - & -0.11* & -0.16* & - & - & - & - & - \\ \midrule
\textbf{QuestEval} & 0.21 & {\ul{0.29}} & 0.23 & 0.37 & 0.19* & 0.35 & 0.14* & 4.67 \\
\textbf{LlaMA3} & \textbf{0.82} & \textbf{0.80} & \textbf{0.72} & \textbf{0.84} & \textbf{0.84} & \textbf{0.90} & \textbf{0.88} & \textbf{1.00} \\
\textbf{Our (3b)} & 0.47 & -0.11* & 0.37 & 0.61 & 0.53 & 0.54 & 0.66 & 3.5 \\
\textbf{Our (8b)} & {\ul{0.57}} & 0.09* & {\ul 0.49} & {\ul 0.72} & {\ul 0.64} & {\ul 0.62} & {\ul 0.67} & {\ul 2.17} \\ \bottomrule
\end{tabular}
\caption{Pearson correlation on human ratings on our constructed test set. 'R-': reference-based. 'S-': source-based. *not significant; we cannot reject the null hypothesis of zero correlation}
\label{tab:con}
\end{table*}

\section{Results}
We benchmark the different metrics on the different datasets using correlation to human judgement. For content preservation, we show results split on data with system output, reference output and our constructed test set: we show that the data source for evaluation leads to different conclusions on the metrics. In addition, we examine whether the metrics can rank style transfer systems similar to humans. On style strength, we likewise show correlations between human judgment and zero-shot evaluation approaches. When applicable, we summarize results by reporting the average correlation. And the average ranking of the metric per dataset (by ranking which metric obtains the highest correlation to human judgement per dataset). 

\subsection{Content preservation}
\paragraph{How do data sources affect the conclusion on best metric?}
The conclusions about the metrics' performance change radically depending on whether we use system output data, reference output, or our constructed test set. Ideally, a good metric correlates highly with humans on any data source. Ideally, for meta-evaluation, a metric should correlate consistently across all data sources, but the following shows that the correlations indicate different things, and the conclusion on the best metric should be drawn carefully.

Looking at the metrics correlations with humans on the data source with system output (Table~\ref{tab:sys}), we see a relatively high correlation for many of the metrics on many tasks. The overall best metrics are S-BertScore and S-BLEURT (avg+avg rank). We see no notable difference in our method of using the 3B or 8B model as the backbone.

Examining the average correlations based on data with reference output (Table~\ref{tab:ref}), now the zero-shoot prompting with LlaMA3 70B is the best-performing approach ($0.59$ avg). Tied for second place are source-based cosine embedding ($0.46$ avg), BLEURT ($0.45$ avg) and BertScore ($0.45$ avg). Our method follows on a 5. place: here, the 8b version (($0.43$ avg)) shows a bit stronger results than 3b ($0.39$ avg). The fact that the conclusions change, whether looking at reference or system output, confirms the observations made by \citet{scialom-etal-2021-questeval} on simplicity transfer.   

Now consider the results on our test set (Table~\ref{tab:con}): Several metrics show low or no correlation; we even see a significantly negative correlation for some metrics on ALL (BLEU) and for specific subparts of our test set for BLEU, Meteor, BertScore, Cosine. On the other end, LlaMA3 70B is again performing best, showing strong results ($0.82$ in ALL). The runner-up is now our 8B method, with a gap to the 3B version ($0.57$ vs $0.47$ in ALL). Note our method still shows zero correlation for the sentiment task. After, ranks BLEURT ($0.29$), QuestEval ($0.21$), BertScore ($0.11$), Cosine ($0.01$).  

On our test set, we find that some metrics that correlate relatively well on the other datasets, now exhibit low correlation. Hence, with our test set, we can now support the logical reasoning with data evidence: Evaluation of content preservation for style transfer needs to take the style shift into account. This conclusion could not be drawn using the existing data sources: We hypothesise that for the data with system-based output, successful output happens to be very similar to the source sentence and vice versa, and reference-based output might not contain server mistakes as they are gold references. Thus, none of the existing data sources tests the limits of the metrics.  


\paragraph{How do reference-based metrics compare to source-based ones?} Reference-based metrics show a lower correlation than the source-based counterpart for all metrics on both datasets with ratings on references (Table~\ref{tab:sys}). As discussed previously, reference-based metrics for style transfer have the drawback that many different good solutions on a rewrite might exist and not only one similar to a reference.


\paragraph{How well can the metrics rank the performance of style transfer methods?}
We compare the metrics' ability to judge the best style transfer methods w.r.t. the human annotations: Several of the data sources contain samples from different style transfer systems. In order to use metrics to assess the quality of the style transfer system, metrics should correctly find the best-performing system. Hence, we evaluate whether the metrics for content preservation provide the same system ranking as human evaluators. We take the mean of the score for every output on each system and the mean of the human annotations; we compare the systems using the Kendall's Tau correlation. 

We find only the evaluation using the dataset Mir, Lai, and Ziegen to result in significant correlations, probably because of sparsity in a number of system tests (App.~\ref{app:dataset}). Our method (8b) is the only metric providing a perfect ranking of the style transfer system on the Lai data, and Llama3 70B the only one on the Ziegen data. Results in App.~\ref{app:results}. 


\subsection{Style strength results}
%Evaluating style strengths is a challenging task. 
Llama3 70B shows better overall results than our method. However, our method scores higher than Llama3 70B on 2 out of 6 datasets, but it also exhibits zero correlation on one task (Table~\ref{tab:styleresults}).%More work i s needed on evaluating style strengths. 
 
\begin{table}%[]
\fontsize{11pt}{11pt}\selectfont
\begin{tabular}{lccc}
\toprule
\multicolumn{1}{c}{\textbf{}} & \textbf{LlaMA3} & \textbf{Our (3b)} & \textbf{Our (8b)} \\ \midrule
\textbf{Mir} & 0.46 & 0.54 & \textbf{0.57} \\
\textbf{Lai} & \textbf{0.57} & 0.18 & 0.19 \\
\textbf{Ziegen.} & 0.25 & 0.27 & \textbf{0.32} \\
\textbf{Alva-M.} & \textbf{0.59} & 0.03* & 0.02* \\
\textbf{Scialom} & \textbf{0.62} & 0.45 & 0.44 \\
\textbf{\begin{tabular}[c]{@{}l@{}}Our Test\end{tabular}} & \textbf{0.63} & 0.46 & 0.48 \\ \bottomrule
\end{tabular}
\caption{Style strength: Pearson correlation to human ratings. *not significant; we cannot reject the null hypothesis of zero corelation}
\label{tab:styleresults}
\end{table}

\subsection{Ablation}
We conduct several runs of the methods using LLMs with variations in instructions/prompts (App.~\ref{app:method}). We observe that the lower the correlation on a task, the higher the variation between the different runs. For our method, we only observe low variance between the runs.
None of the variations leads to different conclusions of the meta-evaluation. Results in App.~\ref{app:results}.
Our findings challenge the conjecture that code-comment coherence, as measured by SIDE \cite{mastropaolo2024evaluating}, is a critical quality attribute for filtering instances of code summarization datasets. By selecting $\langle code, summary \rangle$ pairs with high-coherence for training allow to achieve the same results that would be achieved by randomly selecting such a number of instances. At the same time, we observed that reducing the datasets size up to 50\% of the training instances does not significantly affect the effectiveness of the models, even when the instances are randomly selected. These results have several implications.

First, code-comment consistency might not be a problem in state-of-the-art datasets in the first place, as also suggested in the results of RQ$_0$. Also, the DL models we adopted (and, probably, bigger models as well) are not affected by inconsistent code-comment pairs, even when these inconsistencies are present in the training set.
Despite the theoretical benefits of filtering by SIDE \cite{mastropaolo2024evaluating}, that is the state-of-the-art metric for measuring code-comment alignment, our results indicate its limitations in improving the \textit{overall} quality of the training sets for code summarization task.
Nevertheless, other quality aspects of code and comments that have not been explored yet (such as readability) may be important for smartly selecting the training instances.
Future work should explore such quality aspects further.

Our results clearly indicate that state-of-the-art datasets contain instances that do not contribute to improving the models' effectiveness. This finding is related to a general phenomenon observed in Machine Learning and Deep Learning. Models reach convergence when they are trained for a certain amount of time (epochs). Additional training provides smaller improvements and increases the risk of overfitting. We show that the same is true for data. In terms of effectiveness, model convergence is achieved with fewer training instances than previously assumed. Limiting the number of epochs may make it possible to reach model convergence with a subset of training data, maintaining model effectiveness, reducing resource demands and minimizing the risk of overfitting.
Future work could explore different criteria for data selection that identify the most informative subsets for training.
Conversely, this insight suggests that currently available datasets suffer from poor diversity (thus causing the previously discussed phenomenon).
This latter insight constitutes a clear warning for researchers interested in building code summarization datasets, which should include instances that add relevant information instead of adding more data, which might turn out to be useless.

Finally, it is worth pointing out that another benefit of the reduction we performed is the environmental impact. Reducing the number of training instances implies a reduced training time, which, in turn, lowers the resources necessary to perform training and, thus, energy consumption and CO$_2$ emissions.
We performed a rough estimation of the training time across different selections of \textit{TL-CodeSum} and \textit{Funcom} datasets and estimated a proxy of the CO$_2$ emissions for each model training phase by relying on the ML CO$_2$ impact calculator\footnote{\url{https://mlco2.github.io/impact/\#compute}} \cite{lacoste2019quantifying}. Such a calculator considers factors such as the total training time, the infrastructure used, the carbon efficiency, and the amount of carbon offset purchased. The estimation of CO$_{2}$ emissions needed to train the model with the \textit{Full} selection of \textit{Funcom} ($\sim$ 200 hours) is equal to 26.05 Kg, while with the optimized training set, \ie $SIDE_{0.9}$ ($\sim$ 90 hours), the estimation is 11.69 Kg of $CO_2$ (-55\% emissions).
While we recognize that this method provides an estimation rather than a precise measurement, it offers a glimpse into the environmental impact of applying data reduction.

\subsection{Greedies}
We have two greedy methods that we're using and testing, but they both have one thing in common: They try every node and possible resistances, and choose the one that results in the largest change in the objective function.

The differences between the two methods, are the function. The first one uses the median (since we want the median to be >0.5, we just set this as our objective function.)

Second one uses a function to try to capture more nuances about the fact that we want the median to be over 0.5. The function is:

\[
\text{score}(\text{opinion}) =
\begin{cases} 
\text{maxScore}, & \text{if } \text{opinion} \geq 0.5 \\
\min\left(\frac{50}{0.5 - \text{opinion}}, \frac{\text{maxScore}}{2}\right), & \text{if } \text{opinion} < 0.5 
\end{cases}
\] 

Where we set maxScore to be $10000$.

\subsection{find-c}
Then for the projected methods where we use Huber-Loss, we also have a $find-c$ version (temporary name). This method initially finds the c for the rest of the run. 

The way it does it it randomly perturbs the resistances and opinions of every node, then finds the c value that most closely approximates the median for all of the perturbed scenarios (after finding the stable opinions). 

\section{Conclusion}
In this work, we propose a simple yet effective approach, called SMILE, for graph few-shot learning with fewer tasks. Specifically, we introduce a novel dual-level mixup strategy, including within-task and across-task mixup, for enriching the diversity of nodes within each task and the diversity of tasks. Also, we incorporate the degree-based prior information to learn expressive node embeddings. Theoretically, we prove that SMILE effectively enhances the model's generalization performance. Empirically, we conduct extensive experiments on multiple benchmarks and the results suggest that SMILE significantly outperforms other baselines, including both in-domain and cross-domain few-shot settings.

\bmhead{Data availability}
The datasets captured and analyzed during the current study are available from the corresponding author on request.

\bmhead{Code availability}
The code supporting the findings of this study is available from the corresponding author upon request.

% \bibliography{sn-bibliography}% common bib file
%% if required, the content of .bbl file can be included here once bbl is generated
%%%%%%%%%%%%%%%%%%%%%%%%%%%%%%%%%%%%%%%%%%%%%%%%%%%%%%%%%%%%%%%%%%%%%
%%                                                                 %%
%% Please do not use \input{...} to include other tex files.       %%
%% Submit your LaTeX manuscript as one .tex document.   ,          %%
%%                                                                 %%
%% All additional figures and files should be attached             %%
%% separately and not embedded in the \TeX\ document itself.       %%
%%                                                                 %%
%%%%%%%%%%%%%%%%%%%%%%%%%%%%%%%%%%%%%%%%%%%%%%%%%%%%%%%%%%%%%%%%%%%%%

%%\documentclass[referee,sn-basic]{sn-jnl}% referee option is meant for double line spacing

%%=======================================================%%
%% to print line numbers in the margin use lineno option %%
%%=======================================================%%

%%\documentclass[lineno,sn-basic]{sn-jnl}% Basic Springer Nature Reference Style/Chemistry Reference Style

%%======================================================%%
%% to compile with pdflatex/xelatex use pdflatex option %%
%%======================================================%%

% \documentclass[pdflatex,sn-vancouver,iicol]{sn-jnl}
\documentclass[iicol,sn-basic]{sn-jnl}
% \renewcommand{\bibfont}{\small} % Adjust to match the journal style

%%\documentclass[sn-basic]{sn-jnl}% Basic Springer Nature Reference Style/Chemistry Reference Style
%\documentclass[sn-standardnature]{sn-jnl}% Math and Physical Sciences Reference Style
%%\documentclass[sn-aps]{sn-jnl}% American Physical Society (APS) Reference Style
%%\documentclass[sn-vancouver]{sn-jnl}% Vancouver Reference Style
%%\documentclass[sn-apa]{sn-jnl}% APA Reference Style
%%\documentclass[sn-chicago]{sn-jnl}% Chicago-based Humanities Reference Style
%%\documentclass[sn-standardnature]{sn-jnl}% Standard Nature Portfolio Reference Style
%%\documentclass[default]{sn-jnl}% Default
%%\documentclass[default,iicol]{sn-jnl}% Default with double column layout

%%%% Standard Packages
%%<additional latex packages if required can be included here>
%%%%

%%%%%=============================================================================%%%%
%%%%  Remarks: This template is provided to aid authors with the preparation
%%%%  of original research articles intended for submission to journals published
%%%%  by Springer Nature. The guidance has been prepared in partnership with
%%%%  production teams to conform to Springer Nature technical requirements.
%%%%  Editorial and presentation requirements differ among journal portfolios and
%%%%  research disciplines. You may find sections in this template are irrelevant
%%%%  to your work and are empowered to omit any such section if allowed by the
%%%%  journal you intend to submit to. The submission guidelines and policies
%%%%  of the journal take precedence. A detailed User Manual is available in the
%%%%  template package for technical guidance.
%%%%%=============================================================================%%%%
 % \usepackage{tikz}
% \usetikzlibrary{shapes, positioning}
 % \usepackage{neuralnetwork}
%\usepackage{natbib}
\usepackage{graphicx}
%\usepackage{algorithm}
%\usepackage{algorithmic}
%\usepackage[noend]{algpseudocode}
%\usepackage[boxruled]{algorithm2e}
\graphicspath{ {./images/} }
\usepackage{esvect}
\usepackage{amsmath}
\usepackage[utf8]{inputenc}
\usepackage[tight,footnotesize]{subfigure}
%\usepackage{algpseudocode}
\usepackage{appendix}
\usepackage{lscape}
\usepackage{textcomp}
\usepackage{amssymb}
\usepackage{lscape}
\usepackage{bbm}
\usepackage{hhline}
\usepackage{multirow}
\usepackage{adjustbox}
\usepackage{makecell}
\usepackage{subcaption} 

\jyear{2024}%

%% as per the requirement new theorem styles can be included as shown below
\theoremstyle{thmstyleone}%
\newtheorem{theorem}{Theorem}%  meant for continuous numbers
%%\newtheorem{theorem}{Theorem}[section]% meant for sectionwise numbers
%% optional argument [theorem] produces theorem numbering sequence instead of independent numbers for Proposition
\newtheorem{proposition}[theorem]{Proposition}%
%%\newtheorem{proposition}{Proposition}% to get separate numbers for theorem and proposition etc.

\theoremstyle{thmstyletwo}%
\newtheorem{example}{Example}%
\newtheorem{remark}{Remark}%

\theoremstyle{thmstylethree}%
\newtheorem{definition}{Definition}%

\raggedbottom
%%\unnumbered% uncomment this for unnumbered level heads

\begin{document}

\title{SincPD: An Explainable Method based on Sinc Filters to Diagnose Parkinson's Disease Severity by Gait Cycle Analysis}

%%=============================================================%%
%% Prefix	-> \pfx{Dr}
%% GivenName	-> \fnm{Joergen W.}
%% Particle	-> \spfx{van der} -> surname prefix
%% FamilyName	-> \sur{Ploeg}
%% Suffix	-> \sfx{IV}
%% NatureName	-> \tanm{Poet Laureate} -> Title after name
%% Degrees	-> \dgr{MSc, PhD}
%% \author*[1,2]{\pfx{Dr} \fnm{Joergen W.} \spfx{van der} \sur{Ploeg} \sfx{IV} \tanm{Poet Laureate}
%%                 \dgr{MSc, PhD}}\email{iauthor@gmail.com}
%%=============================================================%%

\author*[1]{\fnm{Armin} \sur{Salimi-Badr}}\email{a\_salimibadr$@$sbu.ac.ir}

\author[1]{\fnm{Mahan} \sur{Veisi}}%\email{mohamm.hashemi@mail.sbu.ac.ir}


\author[1]{\fnm{Sadra} \sur{Berangi}}%\email{h.saffari@mail.sbu.ac.ir}

%\author[2]{\fnm{Athena} \sur{Abdi}}

\affil*[1]{\orgdiv{Faculty of Computer Science and Engineering}, \orgname{Shahid Beheshti University},  \city{Tehran}, \country{Iran}}

%\affil[2]{\orgdiv{Faculty of Computer Engineering}, \orgname{K. N. Toosi University of Technology},  \city{Tehran}, \country{Iran}}


%%==================================%%
%% sample for unstructured abstract %%
%%==================================%%

\abstract{In this paper, an explainable deep learning-based classifier based on adaptive sinc filters for Parkinson's Disease diagnosis (PD) along with determining its severity, based on analyzing the gait cycle (SincPD) is presented. Considering the effects of PD on the gait cycle of patients, the proposed method utilizes raw data in the form of vertical Ground Reaction Force (vGRF) measured by wearable sensors placed in soles of subjects' shoes. The proposed method consists of Sinc layers that model adaptive bandpass filters to extract important frequency-bands in gait cycle of patients along with healthy subjects. Therefore, by considering these frequencies, the reasons behind the classification a person as a patient or healthy can be explained. In this method, after applying some preprocessing processes, a large model equipped with many filters is first trained. Next, to prune the extra units and reach a more explainable and parsimonious structure, the extracted filters are clusters based on their cut-off frequencies using a centroid-based clustering approach. Afterward, the medoids of the extracted clusters are considered as the final filters. Therefore, only 15 bandpass filters for each sensor are derived to classify patients and healthy subjects. Finally, the most effective filters along with the sensors are determined by comparing the energy of each filter encountering patients and healthy subjects. %The final Accuracy of the proposed method is \textbf{}\% for PD diagnosis and \textbf{}\% for severity detection.
}

%%================================%%
%% Sample for structured abstract %%
%%================================%%

% \abstract{\textbf{Purpose:} The abstract serves both as a general introduction to the topic and as a brief, non-technical summary of the main results and their implications. The abstract must not include subheadings (unless expressly permitted in the journal's Instructions to Authors), equations or citations. As a guide the abstract should not exceed 200 words. Most journals do not set a hard limit however authors are advised to check the author instructions for the journal they are submitting to.
%
% \textbf{Methods:} The abstract serves both as a general introduction to the topic and as a brief, non-technical summary of the main results and their implications. The abstract must not include subheadings (unless expressly permitted in the journal's Instructions to Authors), equations or citations. As a guide the abstract should not exceed 200 words. Most journals do not set a hard limit however authors are advised to check the author instructions for the journal they are submitting to.
%
% \textbf{Results:} The abstract serves both as a general introduction to the topic and as a brief, non-technical summary of the main results and their implications. The abstract must not include subheadings (unless expressly permitted in the journal's Instructions to Authors), equations or citations. As a guide the abstract should not exceed 200 words. Most journals do not set a hard limit however authors are advised to check the author instructions for the journal they are submitting to.
%
% \textbf{Conclusion:} The abstract serves both as a general introduction to the topic and as a brief, non-technical summary of the main results and their implications. The abstract must not include subheadings (unless expressly permitted in the journal's Instructions to Authors), equations or citations. As a guide the abstract should not exceed 200 words. Most journals do not set a hard limit however authors are advised to check the author instructions for the journal they are submitting to.}

\keywords{Parkinson's Disease, SincNet, Explainability, Wearable Sensors, Gait Cycle}

%%\pacs[JEL Classification]{D8, H51}

%%\pacs[MSC Classification]{35A01, 65L10, 65L12, 65L20, 65L70}

\maketitle

%\end{titlepage}
\section{Introduction}

Parkinson's Disease (PD) is the second most common neurodegenerative disease affecting many elderly individuals worldwide \citep{neuro-fuzzy,Liu2021CNN_and_LSTM,khoury2019data,pan2012parkinson,GOYAL2020103955}. It primarily originates from the loss of dopaminergic neurons in the Substantia Nigra pars Compacta within the Basal Ganglia \citep{Hall2020,salimi2017possible,salimi2018system}.

PD treatment typically begins after symptoms manifest years post-infection, necessitating more complex treatments like Deep Brain Stimulus (DBP) instead of simpler lifestyle changes. Early diagnosis can enable more effective and economical treatments. However, diagnosis is a challenging task that prompts the use of machine learning to enhance healthcare diagnostics.

PD patients often exhibit movement issues such as Bradykinesia, Akinesia, stiffness, and resting tremor \citep{Hall2020}. Consequently, machine learning methods focus on movement analysis for diagnosis, including \textit{speech disorders} \citep{Kuresan2021,Pramanik2022,Yousif2023-gy,Liu2022}, \textit{g`ait changes} \citep{khoury2018cdtw,salimipd,deep1d,LSTM,SalimiBadr-MultiLSTM,Liu2021CNN_and_LSTM}, \textit{handwritten records} \citep{Yousif2023-gy}, \textit{typing speed} \citep{prashanth2016high}, \textit{eye movement changes} \citep{Farashi2021}, multi-modal biomedical time-series analysis \citep{JUNAID2023107495}, and postural stability assessment using RGB-Depth cameras \citep{Ferraris2024-up}.

Machine Learning, particularly Deep Learning, has been effectively applied in medical diagnosis \citep{deep1d,cnnlstm,Liu2021CNN_and_LSTM}. Explainable methods like neuro-fuzzy systems \citep{neuro-fuzzy} are limited by their reliance on high-level clinical features. Deep models extract complex features but lack interpretability \citep{xai1,sensors1}. However, explainability is crucial in medical applications to support expert diagnosis \citep{JUNAID2023107495,Saadatinia2024}.

Convolutional Neural Networks (CNNs) extract abstract features through learned filters. The first convolutional layers are critical as they process raw signals and form higher-level features. \textit{SincNet} \citep{SincNet} enhances interpretability by using sinc-shaped bandpass filters with only two parameters: center and width. This allows identification of important frequency bands influencing network decisions \citep{hung2022using,ravanelli2018interpretable}.

This paper presents an explainable AI approach to detect PD and assess its severity using sinc-layers in CNNs to analyze vertical-Ground Reaction Force (vGRF) signals from wearable sensors (SincPD). The Sinc layers model adaptive bandpass filters to extract key frequency bands in gait cycles, enabling interpretation of the network's decisions.

Our SincPD involves preprocessing, training a large filter-rich model, pruning redundant filters via centroid-based clustering, and selecting cluster medoids as final filters. We analyze important frequencies by calculating filter energy for patients and healthy subjects.

The paper is organized as follows: Section \ref{sec3} reviews related machine learning studies for PD detection. Section \ref{sec2} presents preliminaries. Section \ref{sec4} details the proposed methodology, including preprocessing, SincNet architecture, learning, and pruning. Section \ref{sec5} discusses experimental results and filter analysis. Conclusions are in Section \ref{sec6}.

% \section{Introduction}



%  Parkinson's Disease (PD) is the the second most common neurodegenerative diseases which threats the lives of a many elderly persons in the world \citep{neuro-fuzzy,cnnlstm2,khoury2019data,pan2012parkinson,GOYAL2020103955}. the destruction of the dopaminergic neurons of the Substantia Nigra pars Compacta in the Basal Ganglia is the main origin of this chronic disease \citep{guyton2010,salimi2017possible,salimi2018system,khoury2019data}. 
 
%  Treatment of PD is usually started after detecting its main symptoms which appear many years after infection. Therefore, this diagnosis delay causes necessity for more complicated and aggressive treatments, including surgical therapy like the Deep Brain Stimulus (DBP), instead of the existing simpler methods, like applying appropriate lifestyle changes \citep{khoury2019data}. Therefore, early diagnosis can help patients to have more effective, simpler, and more economic treatments. On the other hand, the diagnosis task is challenging with reported misdiagnosis rate of $25\%$ of cases \citep{das2010comparison}. These challenges have motivated researches to apply machine learning methods to increase the quality of healthcare services and illness diagnoses.

% The majority of patients suffering from PD have movement problems such as Bradykinesia (slowness), Akinesia (impaired power of voluntary movements), stiffness, and resting tremor \citep{guyton2010}. These movement problems have lead most of machine learning methods to study the movements of subjects for diagnosis including  \textit{speech disorders (vocal impairment)} \citep{das2010comparison,aastrom2011parallel,hariharan2014new}, \textit{gait cycle changing} \citep{khoury2019data,khoury2018cdtw,salimipd,neuro-fuzzy,deep1d,LSTM,SalimiBadr-MultiLSTM,Liu2021CNN_and_LSTM}, speed of typing \citep{prashanth2016high}, \textit{eye movements changing} \citep{Farashi2021}, and multi-modal biomedical time-series analysis \citep{JUNAID2023107495}.

% Recently, Machine Learning approaches including Deep Learning methods are successfully applied for medical diagnosis \citep{deep1d,cnnlstm,Liu2021CNN_and_LSTM}. In this context the explainable methods like utilizing neuro-fuzzy systems \citep{neuro-fuzzy} have a limited performance due to use high-level expert understandable clinical features extracted from the recorded raw data \citep{salimipd,36lee2012parkinson,qbp}. However, the deep structures with a higher performance can extract complex and more abstract features through representation learning from the raw data, but they lack interpretability and transparency due to working as a black-box \citep{xai1,neuro-fuzzy,sensors1}. On the other hand, explainabilty is necessary in medical applications to provide experts with more information from the model than a simple binary prediction for supporting their diagnosis \citep{xai3,xai2,neuro-fuzzy,JUNAID2023107495,samek2019explainable}. 

% Convolutional Neural Networks (CNNs) are popular deep neural networks that try to extract more abstract and proper features by learning suitable filters in their different layers. The extraction ability of the first convolutional layers is very important in most applications because they directly process the input raw waveform signal. These layers represent the effectiveness of the initial low-dimensional variables to form higher-level features extracted in next layers. To make it more interpretable while reducing the number of parameters in the deep layers, in \citep{SincNet} \textit{SincNet} has been proposed that consider these mentioned first layers of a CNN as ideal bandpass filters (that have sinc-shape impulse responses along with corresponding square-shape frequency responses). These filters have only two adaptive variables to determine their cut-off frequencies: a center and a width. By analysing their effects on the input signal, it would be possible to detect important frequency bands that lead to different decisions during the inference process. Consequently, using sinc-layers can increase the interpretability of the network by encouraging the first layers to learn more meaningful filters \citep{hung2022using,ravanelli2018interpretable}.





% In this paper, we propose an explainable artificial intelligence (XAI) approach to detect patients suffering from PD and determine its severity, based on utilizing the sinc-layers (bandpass filters) in the first layers of a convolutional neural network to analyze the waveform of vertical-Ground Reaction Force (vGRF) signals measured by wearable sensors placed in soles of subjects' shoes. The Sinc layers used as the first layers of the proposed architecture, model adaptive bandpass filters to extract important frequency-bands in gait cycle of patients along with healthy subjects. Therefore, the reason behind the network's inference can be explained by studying these frequencies. 

% In this method, after applying some preprocessing processes, first a large model equipped with lots of filters is trained. Next, to prune the extra units and reach a more explainable and compact structure, the extracted filters are clusters based on their cut-off frequencies using a centroid-based clustering approach. Afterward, the medoids of the extracted clusters are considered as the final filters. To analyze the important frequencies in the network's inference process, the energy of each filter encountering patients and healthy subjects is calculated and studies. 

% The rest of this paper is organized as follows: First, in section \ref{sec3} the related studies in detecting PD based on applying machine learning techniques, are reviewed. Next, preliminaries are presented in section \ref{sec2}. Afterward, in section \ref{sec4} the proposed paradigm, including the preprocessing, the utilized SincNet architecture, the learning method, and pruning processes are explained. The experimental results including comparing the performance of the proposed method with some other approaches in both diagnosis and severity measurement are presented in section \ref{sec5}. Moreover, the main filters of the main sensors are extracted in section \ref{sec5}. Finally, conclusions are presented in section \ref{sec6}.

\section{Related Work}
\label{sec3}
% Attributable to factors that are not yet thoroughly elucidated, the rate and extensive occurrence of this Parkinson's disease have surged substantially in the last twenty years\citep{collaborators2019n,dorsey2018emerging}.
% Mahan: PD and gait

% Parkinson's Disease (PD) is a progressive neurodegenerative condition that primarily affects motor functions but also involves several non-motor symptoms. This disease arises from the loss of dopamine-producing brain cells, placing it among a group of motor system disorders. PD is notably the second most common neurodegenerative disease\citep{LEE2016}, with its prevalence estimated at approximately 0.3\% in the general population. This prevalence increases to around 1\% in individuals over 60 years old and approximately 3\% in those aged 80 and above\citep{LEE2016}. Early detection of PD is crucial for effective management and slowing disease progression.


% PD is characterized by a range of early indicators that precede the more obvious motor signs like tremors and rigidity. Vocal impairments, for instance, are among the earliest signs, with approximately 90\% of patients experiencing changes in voice and speech patterns in the initial stages\citep{early_vocal_90_percent}. This has led to the development of diagnostic tools focusing on speech analysis, which provide non-invasive and cost-effective early detection.

% Gait analysis has emerged as a particularly informative diagnostic tool for Parkinson's Disease (PD). The disease affects gait mechanics, leading to distinctive alterations in walking patterns, such as reduced stride length, decreased speed, and increased gait variability. Advanced technologies involving sensor-based systems and motion capture are now extensively used to quantify these deviations, enabling not only early detection but also monitoring of disease progression and response to treatment.

% In clinical settings, gait analysis is employed to assess and treat individuals with various conditions that impair their walking ability, including poor health, advanced age, body size, weight, and speed of movement. To effectively evaluate gait characteristics and other related phenomena in PD patients, a standardized assessment is necessary. Such assessments typically involve measurements of gait count, walking speed, and step length.

% Pistacchi et al. conducted a comprehensive study on early PD patients using 3D gait analysis, examining several temporal parameters. The study found that the cadence in PD patients was 102.46 ± 13.17 steps/min, compared to 113.84 ± 4.30 steps/min in healthy individuals. Stride duration for PD patients was measured at 1.19 ± 0.18 seconds for the right limb and 1.19 ± 0.19 seconds for the left limb, while for healthy subjects, it was significantly shorter, at 0.426 ± 0.16 seconds for the right limb and 0.429 ± 0.23 seconds for the left limb. The stance duration in PD patients was 0.74 ± 0.14 seconds for the right limb and 0.74 ± 0.16 seconds for the left limb, in contrast to 1.34 ± 1.1 seconds and 0.83 ± 0.6 seconds in healthy subjects, respectively. Additionally, the velocity of PD patients was markedly lower at 0.082 ± 0.29 m/s compared to 1.33 ± 0.06 m/s in healthy individuals \citep{pistacchi2017gait}.

% Further research by Sofuwa et al. indicated that PD significantly reduces step length and walking speed when compared to non-PD controls\citep{SOFUWA20051007}. Lescano et al. focused on analyzing various gait parameters, including stance and swing phase duration, and the vertical component of ground reaction force, to determine statistically significant differences between PD patients at stages 2 and 2.5 on the modified Hoehn and Yahr scale\citep{lescano2016possible}.


% % Sadra: ML and ai in PD
% The use of ground reaction force sensors embedded in shoes to distinguish between healthy individuals and patients with Parkinson's disease has been extensively studied\citep{deep1d,LSTM,SalimiBadr-MultiLSTM,Applied_Soft_SVM,Applied_Soft_severity,cnnlstm,cnnlstm2}. Vidya et al.\citep{Applied_Soft_SVM} developed a decision support system for gait classification employing a multi-class support vector machine (MCSVM) and utilized spatiotemporal features for classification. In a different approach, Daliri\citep{32daliri2012automatic} extracted statistical features, including the minimum, maximum, mean, and standard deviation for each time series, which were then fed into binary machine learning classifiers like Support Vector Machines (SVM). Furthermore, an interpretable method for diagnosing Parkinson’s disease using clinical features from vertical Ground Reaction Force (vGRF) data is presented in \citep{neuro-fuzzy}. The approach employs type-2 fuzzy logic and a novel learning method, achieving high classification performance. El Maachi et al.\citep{deep1d} proposed a deep learning-based approach for Parkinson's disease detection and severity prediction using a 1D convolutional neural network (1D-Convnet). The proposed model processes gait data and demonstrates high accuracy and efficiency in classification tasks. Zhao et al.\citep{ZHAO201891} used a dual-channel LSTM to diagnose some diseases, including PD, based on gait dynamics, achieving a classification accuracy of 97.43\%. However, their approach was limited by the partial acquisition of gait dynamics due to the use of only one sensor per foot. Balaji et al.\citep{LSTM} proposed an LSTM network for diagnosing and rating the severity of Parkinson's disease using gait patterns. The method avoids hand-crafted features and effectively classifies PD with high accuracy, demonstrating a significant improvement over related techniques. Vidya and Sasikumar\citep{BVIDYA2022105099} developed a CNN-LSTM model for Parkinson's disease diagnosis using gait analysis. By leveraging VGRF signals and employing empirical mode decomposition. This hybrid approach effectively combines CNN's spatial feature extraction with LSTM's temporal sequence learning, outperforming previous methods in PD severity classification.
% While numerous machine learning approaches have been developed for Parkinson's disease (PD) diagnosis and severity assessment, most rely heavily on hand-crafted features or complex black-box models that lack interpretability. Our proposed method addresses these limitations by introducing an explainable deep learning approach utilizing Sinc filters in a convolutional neural network (CNN). This approach extracts significant frequency bands from raw vertical Ground Reaction Force (vGRF) signals, offering both high accuracy in classification and transparency in the decision-making process. This work aims to enhance the interpretability and reliability of PD diagnosis and severity measurement tools, ensuring that the insights provided by the model are understandable and actionable for medical professionals.
%new edit

Parkinson's Disease (PD) is a progressive neurodegenerative disorder marked by motor symptoms (e.g., tremors, rigidity) and non-motor manifestations. Due to the movement disorders, gait analysis have been one of the most popular approaches to study the Parkinson's disease in the literature. 

Gait analysis has proven valuable for PD diagnosis due to its ability to identify changes such as reduced stride length, slower speed, and increased variability. \cite{pistacchi2017gait} observed significant differences in cadence, stride and stance duration, and gait velocity in early-stage PD patients compared to controls. \cite{SOFUWA20051007} reported shorter step length and slower walking speed, while \cite{lescano2016possible} found deviations in stance and swing phases and ground reaction force at Hoehn and Yahr stages 2--2.5.


Numerous sensor-based systems have been developed for automated PD detection via gait analysis, extensively utilizing vGRF captured by in-shoe sensors~\citep{deep1d,LSTM,SalimiBadr-MultiLSTM,Applied_Soft_SVM,Applied_Soft_severity,cnnlstm,Liu2021CNN_and_LSTM}.\cite{Applied_Soft_SVM} employed spatiotemporal features with a multi-class support vector machine. \cite{neuro-fuzzy} presented a type-2 fuzzy logic approach using vGRF data. and \cite{deep1d} proposed a 1D convolutional neural network for PD detection and severity estimation. \cite{ZHAO201891} utilized a dual-channel LSTM for gait classification, though limited by partial gait acquisition. \cite{LSTM} applied LSTM networks for PD diagnosis and severity rating without hand-crafted features, and \cite{BVIDYA2022105099} introduced a CNN-LSTM model based on empirical mode decomposition of vGRF signals.

Although these machine learning methods have advanced PD diagnosis, many depend on hand-crafted features or lack interpretability.
 To overcome these limitations, we propose an explainable deep learning model using Sinc filters in a convolutional neural network (CNN) to extract salient frequency bands from raw vGRF data. This approach combines high classification accuracy with transparency, improving the clinical relevance of PD diagnosis and severity assessment.




% The deterioration of executive functions and movement disorders in patients with PD have been shown extensively \citep{litvan2012diagnostic,marras2013measuring,baiano2020prevalence}. Yogev et al. \citep{yogev2005dual} studied the impacts of different types of dual-tasking and cognitive function on the gait of patients with PD and control subjects. They also showed contrasting measures of gait rhythmicity for patients with PD in comparison to other features. Additionally, in \citep{yogev2007gait} it is investigated that Parkinson's disease has a great impact on the left-right symmetry of gait. Yogev et al. \citep{yogev2007gait} conducted a similar walking condition for both patients with PD and healthy control subjects and they demonstrated that asymmetry of gait increased mainly during the dual-task condition in patients with PD but not in the healthy control subjects. Considering the Hausdorff et al. \citep{hausdorff1998gait} studies on gait variability and Basal Ganglia disorders, it can be concluded that the ability to maintain a steady gait with low stride-to-stride variability of gait cycle timing, would be decreased in patients with PD. Parkinson's disease symptoms also include speech disorders as well as cognitive impairments \citep{pahwa2013handbook,harel2004variability}. In addition, $90\%$ of the patients with PD themselves report speech impairments as one of the most significant symptoms \citep{hartelius1994speech}.

% To classify the patients with PD and healthy control subjects based on their gait cycles, both wearable \citep{10jeon2008classification,11ashhar2017wearable,12nieuwboer2004electromyographic,14hong2009kinematic,15saito2004lifecorder,17salarian2004gait,18mariani2012shoe} and non-wearable \citep{19latash1995anticipatory,20cho2009vision,21pachoulakis2014building,22galna2014accuracy,23dror2014automatic,24dyshel2015quantifying,25antonio2015abnormal,26song2012altered,27foreman2012improved,28muniz2010comparison,29vaugoyeau2003coordination} sensors have been used in various experiments \citep {GOYAL2020103955}. For instance, Jean et al. \citep{10jeon2008classification}, conducted the classification using Spatial-Temporal Image of Plantar pressure (STIP) among normal step and patient steps with PD. In \citep{18mariani2012shoe} the wearable sensors on-shoe along with some algorithms are presented to characterize the Parkinson's disease motor symptoms.

% Accordingly, the classification of healthy control subjects and patients with PD using ground reaction force sensors placed in shoes has been extensively studied \citep{30su2015characterizing,31zeng2016parkinson,32daliri2012automatic,36lee2012parkinson,salimipd,khoury2018cdtw,khoury2019data}. In \citep{32daliri2012automatic}, the vGRFs measurements of both left and right foot were used to extract statistical features including minimum, maximum, average, and the standard deviation of each time series. Their extracted features then were fed to machine learning binary classifier including Support Vector Machine (SVM). In \citep{30su2015characterizing} the gait asymmetry (GA) was calculated based on the difference of the ground reaction force (GRF) features of the left and right feet. This was done by decomposition of the GRF into components of different frequency sub-bands via the wavelet transform and Multi-Layer Perceptron (MLP) models.  In \citep{36lee2012parkinson}, Lee et al. utilized the gait characteristics of idiopathic PD patients who shuffle their feet while they are walking to classify patients with PD and healthy control subjects. They trained a neural network with weighted fuzzy membership functions (NEWFM) using extracted 40 statistical and wavelet-based features.

% Different supervised and unsupervised methods such as Decision Tree (DT), Support Vector Machine (SVM), K-Nearest Neighbors (KNN), Gaussian-Mixture Model (GMM), and K-Means are extensively utilized to classify patients with PD and healthy control subjects. In \citep{37joshi2017automatic}, automatic noninvasive identification of PD is used with the combination of wavelet analysis and SVM which led to an accuracy of $90.32\%$. Although most prior research focused on time-domain and frequency-domain features, only the clinical features extracted from vertical Ground Reaction Forces (vGRFs) were considered in \citep{khoury2019data}. Accordingly, nineteen statistical features are extracted and used as the input of machine learning-based classifiers.

% A  time-delay neural network classifier learned by a Q-back propagation learning approach was proposed in \citep{qbp} using temporal information of vGRF time-series to classify the PD patients and healthy subjects. Data from three Parkinson's disease research projects are used for evaluation of this approach \citep{yogev2005dual,frenkel2005treadmill,hausdorff2007rhythmic}. The accuracy on the three sub-datasets reached $91.49\%$, $92.19\%$, and $90.91\%$, respectively.

% Although most of the previous works tried to extract feature vectors based on some human knowledge, recently published method tried to extract high-level features based temporal and spatial analysis of the gait-cycle using deep neural networks \citep{salimipd,deep1d,cnnlstm,cnnlstm2}. In \citep{salimipd}, we have demonstrated that the spatial correlation among different sensors data during time placed in each left and right foot are useful for diagnosing the patients. To consider the temporal dependencies, a structure using Long-Short Term Memory (LSTM) cell layers has been proposed to build a Recurrent Neural Network (RNN). In \citep{deep1d}, a deep neural network consists of one dimensional convolutional layers is used to classify patients. The final accuracy of the method reaches $98.7\%$. In \citep{cnnlstm} and \citep{cnnlstm2}, a combination of Long-short Term Memory (LSTM) and Convolutional Neural Network (CNN) are used to detect patients suffering from PD with accuracy equal to $98.61\%$ and $99.22\%$, respectively. Although these deep methods perform efficiently, they have many parameters that must be determined based on a supervised learning method. Therefore, to learn these deep structures, considering the small size of available datasets, authors of these methods have segmented the sequence related each subject into very small pieces (10 to 20 steps) to increase the number of training samples. Moreover, these deep structures lack the interpretability and their reasoning for classification is not clear and verifiable for an expert.

% Most previous studies have not used clinical features or have not applied an interpretable method for classifying PD patients. Using an interpretable machine learning approach that the base of its decision making is transparent and expert understandable is more agreeable in a clinical application \citep{xai3}. In \citep{neuro-fuzzy} a self-organizing interval type-2 fuzzy neural network is presented based on analysing the clinical features extracted from the recorded vGRF signal. This method also reports its extracted fuzzy rules that can be verified and modified based on the knowledge of experts. On the other hand, experts can utilize the extracted rules to detect patients suffering from PD.

% Although the method presented in \citep{neuro-fuzzy} is a an interpretable machine learning method, it is limited to utilize the clinical features extracted from the recorded signals. In this paper, an interpretable deep learning method is proposed based on using adaptive bandpass filters in its first layers to detect patients suffering from PD based on studying their vGRF signals directly. Furthermore, the proposed method is utilized to determine the severity level of PD. 


% -
% -
% -
% all together

\section{Preliminaries}
\label{sec2}

In this section, we review the preliminaries of the proposed method, including the concept of bandpass filters along with the neural architecture based on them (SincNet).

% \textbf{Signal's Energy.} Energy in signal processing quantifies the total power contained within a signal over a specific time interval or frequency range. Mathematically, the energy \( E \) of a signal \( x(t) \) over a time interval \([t_1, t_2]\) is calculated as \citep{oppenheim2017signals}:

% \[
% E = \int_{t_1}^{t_2} x(t) \cdot x^*(t) \, dt
% \]

% where \( x^*(t) \) is the conjugate value of \( x(t) \). To compute the energy, we square the signal amplitudes and integrate over time, yielding a measure of the signal's power.

\noindent\textbf{SincNet Model Architecture.} \label{subsec:sincnet}SincNet introduces a novel approach by incorporating parameterized sinc functions into a convolutional neural network (CNN) architecture to efficiently create bandpass filters. This method allows SincNet to learn from only two parameters per filter—lower and upper cutoff frequencies \( f_1 \) and \( f_2 \)—significantly reducing model complexity.

\begin{figure}[ht]
    \centering
    \begin{minipage}{0.493\columnwidth}
        \centering
        \includegraphics[width=\textwidth]{ sinc_time.eps}
    \end{minipage}
    \hfill
    \begin{minipage}{0.493\columnwidth}
        \centering
        \includegraphics[width=\textwidth]{ sinc_freq.eps}
    \end{minipage}
    \caption{The impulse-response along with the frequency-response of a bandpass filter that allows selective frequency passage from \( f_1 \) to \( f_2 \).}
    \label{fig:sinc}
\end{figure}

\noindent\textbf{Bandpass Filter Configuration.} In SincNet, each filter’s impulse response is modeled by the difference between two scaled sinc functions, representing a bandpass filter \citep{SincNet}:

\begin{equation}
h[n] = 2f_2 \cdot \text{sinc}(2\pi f_2 n) - 2f_1 \cdot \text{sinc}(2\pi f_1 n),
\end{equation}

where the sinc function is defined as follows:

\begin{equation}
\text{sinc}(\omega) = 
\begin{cases} 
\frac{\sin(\omega)}{\omega} & \text{if } \omega \neq 0, \\
1 & \text{if } \omega = 0.
\end{cases}
\end{equation}

This impulse response \( h[n] \) allows selective frequency passage from \( f_1 \) to \( f_2 \), effectively attenuating frequencies outside this range. The form of a sample of this impulse response along with its frequency response is presented in Fig.~\ref{fig:sinc}. The design is inspired by the properties of sinc, which acts as an ideal mathematical model for bandpass filters due to its sharp frequency domain characteristics.

\noindent\textbf{Frequency-Domain Characterization.} The frequency response of the bandpass filter designed by SincNet is fundamentally a rectangular function (see Fig.~\ref{fig:sinc}), characterized as:
\begin{equation}
G(f) = \text{rect}\left(\frac{f}{2f_2}\right) - \text{rect}\left(\frac{f}{2f_1}\right),
\end{equation}
where \( \text{rect}(\cdot) \) is the rectangular function. The equivalent time-domain representation is shown in Equation~(4), illustrating how the sinc functions are used to implement bandpass behavior. The filters are initialized to cover a range of frequencies between 0 and half the sampling rate (\( f_s/2 \)), guided by psychoacoustic principles such as the equivalent rectangular bandwidth (ERB), which aids in setting the filters' bandwidth more precisely.
 
% \section{Preliminaries}
% \label{sec2}
% In this section we review the preliminaries of the proposed method including the the concept of a signal's energy, and the concept of bandpass filters along with the neural architecture based on this type of filters (SincNet).

% %\subsection{Normalization Technique: Z-score\label{subsec:zscore}}
% %Z-score normalization, also known as standardization, is a preprocessing technique applied to ensure that features are on a comparable scale. This normalization is crucial in many machine learning algorithms to aid the convergence and stability of optimization algorithms, as it balances the feature scales, preventing any single feature from dominating the gradient updates. It involves transforming each feature to have a mean of zero and a standard deviation of one, according to the formula:

% %\begin{equation}
% %x^i_s(t) = \frac{x^i(t) - \bar{x}^i}{\sigma(x^i)}
% %\end{equation}

% %Here, $x^i(t)$ represents the difference in signal between the recorded Vertical Ground Reaction Forces (VGRF) of the $i^{th}$ sensor on the left and right feet. The terms $\bar{x}^i$ and $\sigma(x^i)$ denote the mean and standard deviation of this signal over a specified interval, respectively.


% \subsection{Signal's Energy\label{subsubsec:energy}}
% Energy in signal processing quantifies the total power contained within a signal over a specific time interval or frequency range. Mathematically, the energy $E$ of a signal $x(t)$ over a time interval $[t_1, t_2]$ is calculated as \citep{oppenheim2017signals}:

% \[E = \int_{t_1}^{t_2} x(t).x^*(t) dt\]
% where $x^*(t)$ is the conjugate value of $x(t)$.
% To compute the energy, we square the signal amplitudes and integrate over time, yielding a measure of the signal's power.

% \subsection{SincNet Model Architecture\label{subsec:sincnet}}
% SincNet introduces a novel approach by incorporating parameterized sinc functions into a convolutional neural network (CNN) architecture to efficiently create bandpass filters. This method allows SincNet to learn from only two parameters per filter—lower and upper cutoff frequencies $f_1$ and $f_2$), significantly reducing model complexity.

% \begin{figure}[ht]
%     \centering
%     \begin{minipage}{0.493\columnwidth}
%         \centering
%         \includegraphics[width=\textwidth]{ sinc_time.eps}
%     \end{minipage}
%     \hfill
%     \begin{minipage}{0.493\columnwidth}
%         \centering
%         \includegraphics[width=\textwidth]{ sinc_freq.eps}
%     \end{minipage}
%     \caption{The impulse-response along with the frequency-response of a bandpass filter that allows selective frequency passage from $f_1$ to $f_2$.}
%     \label{fig:sinc}
% \end{figure}


% \subsubsection{bandpass Filter Configuration}

% In SincNet, each filter’s impulse response is modeled by the difference between two scaled sinc functions, representing a bandpass filter \citep{SincNet}:
% \begin{equation}
% h[n] = 2f_2 \cdot \text{sinc}(2\pi f_2 n) - 2f_1 \cdot \text{sinc}(2\pi f_1 n),
% \end{equation}
% where sinc function is defined as follows:
% \begin{equation}
% \text{sinc}(\omega) = 
% \begin{cases} 
% \frac{\sin(\omega)}{\omega} & \text{if } x \neq 0, \\
% 1 & \text{if } x = 0.
% \end{cases}
% \end{equation}
% This impulse response $h[n]$ allows selective frequency passage from $f_1$ to $f_2$, effectively attenuating frequencies outside this range. The form of a sample of this impulse response along with its frequency response is presented in Fig. \ref{fig:sinc}. The design is inspired by the properties of sinc, which acts as an ideal mathematical model for bandpass filters due to its sharp frequency domain characteristics.

% \subsubsection{Frequency-Domain Characterization}
% The frequency response of the bandpass filter designed by SincNet is fundamentally a rectangular function (see Fig. \ref{fig:sinc}), characterized as:
% \begin{equation}
% G(f) = \text{rect}\left(\frac{f}{2f_2}\right) - \text{rect}\left(\frac{f}{2f_1}\right),
% \end{equation}
% where $\text{rect}(\cdot)$ is the rectangular function. The equivalent time-domain representation is shown in equation (4), illustrating how the sinc functions are used to implement bandpass behavior. The filters are initialized to cover a range of frequencies between 0 and half the sampling rate ($f_s/2$), guided by psychoacoustic principles such as the equivalent rectangular bandwidth (ERB), which aids in setting the filters' bandwidth more precisely.

% \subsubsection{Feature Extraction by Applying Convolution}
% SincNet applies the following convolution operation to extract features:
% \begin{equation}
% y[n] = (x[n] * h[n]) = \sum_{m=-\infty}^{\infty} x[m] \cdot h[n - m],
% \end{equation}
% where $*$ indicates the convolution operation, $x[n]$ is the input signal, $h[n]$ is the impulse response of the filter, and $y[n]$ is the output signal. This operation transforms the input into feature maps that encode specific frequency characteristics, making the network adept at handling tasks that require detailed frequency analysis, such as image, audio, and biomedical signal processing.

% \subsubsection{Integration into CNN Architecture}
% Post-convolution, the feature maps ($y[n]$) are input to subsequent CNN layers, which further refine these features for complex tasks like classification or regression. The architecture benefits from reduced parameter count and faster convergence due to the efficient initial feature extraction provided by the Sinc layers.

% \begin{algorithm}
% \caption{Processing Pipeline of SincNet}
% \begin{algorithmic}[1]
%     \State Initialize filter parameters $f_1$ and $f_2$ within the range [0, $f_s/2$].
%     \State Compute bandpass filters using sinc functions.
%     \State Apply windowing to filters to minimize spectral leakage.
%     \State Execute convolution to derive feature maps from input signals.
%     \State Propagate feature maps through additional CNN layers for advanced processing.
%     \State Utilize fully connected layers to generate the final output based on refined features.
% \end{algorithmic}
% \end{algorithm}
% \newpage
% \begin{landscape}
% \begin{figure}
%     \centering
%     \includegraphics[width=.9\linewidth]{ Main_model_plot.pdf}
%     \caption{Architecture of the inital model. The model architecture begins with an input layer followed by SincConv1D layers, which are specialized convolutional layers for handling 1D signals. The SincConv1D layers are normalized and activated using Leaky ReLU activation function. Following the SincConv1D layers, traditional convolutional layers with batch normalization, Leaky ReLU activation, and dropout are employed to extract higher-level features from the learned representations. Finally, the model concludes with dense layers for classification, including regularization to prevent overfitting.}
%     \label{fig:model_architecture}
% \end{figure}
% \end{landscape}


\section{Materials and Methods}
\label{sec4}

In this section, we detail the methodology employed in our study, encompassing the preprocessing steps, the architecture of the proposed deep learning model, and the pruning process.


\subsection{Preprocessing}
\label{sec42}

The preprocessing stage ensures input data quality and suitability for the neural network. We prepared the dataset by partitioning it into ten-second segments, filtering out incomplete or inconsistent data, and standardizing the signals. This process ensured clean and normalized inputs for further analysis.

% Cumulative force signals for the left and right feet of healthy and PD subjects, shown in Fig.~\ref{fig:combined_plots}, highlight differences in gait variability and force magnitude, supporting the need for detailed signal analysis.

To reduce dimensionality without data loss, we calculated the difference between left and right sensor signals. This approach reduced 16 arrays per subject to 8, effectively preserving critical gait patterns while capturing meaningful inter-sensor variations. 

\iffalse
The standardization process is defined as:
\begin{equation}
x^i_s(t) = \frac{x^i(t) - \bar{x}^i}{\sigma(x^i)}
\end{equation}
where $x^i(t)$ represents the difference in signals from the $i^{th}$ sensor, and $\bar{x}^i$ and $\sigma(x^i)$ are the mean and standard deviation over a specified interval, respectively.
% This reduction is illustrated in Fig.~\ref{fig:combined_plots}.
\fi
Moreover, stratified splitting was employed to maintain class balance in the training and test datasets, ensuring robustness across patient and healthy subject categories. In summary, Fig.~\ref{fig:pre} demonstrates an overview of the preprocessing pipeline, including chunking, cleaning, and standardization steps.

% \begin{figure*}
%     \centering
%     % First two plots side by side
%     \begin{minipage}{0.25\textwidth}
%         \centering
%         \includegraphics[width=\textwidth]{ Left Gait Cycle.pdf}
%         % \caption*{Left Gait Cycle}
%     \end{minipage}
%     \hfill
%     \begin{minipage}{0.25\textwidth}
%         \centering
%         \includegraphics[width=\textwidth]{ Right Gait Cycle.pdf}
%         % \caption*{Right Gait Cycle}
%     \end{minipage}
%     \hfill
%     % Second two plots side by side
%     \begin{minipage}{0.24\textwidth}
%         \centering
%         \includegraphics[width=\textwidth]{ Left_VGRF1_plot.pdf}
%         % \caption*{a)}
%     \end{minipage}
%     \hfill
%     \begin{minipage}{0.24\textwidth}
%         \centering
%         \includegraphics[width=\textwidth]{ Left_VGRF1-Right_VGRF1.pdf}
%         % \caption*{Left-Right VGRF Difference}
%     \end{minipage}
%     \caption{Cumulative force signals for the left and right feet of healthy and PD subjects, and their differences for a sampled data chunk, illustrating reduced dimensionality while preserving gait patterns. ((((((((((TBD))))))))))))) }
%     \label{fig:combined_plots}
% \end{figure*}


\begin{figure}
    \centering
    \includegraphics[width=3.in]{ preprocessing2.pdf}
    \caption{Overview of the preprocessing pipeline.}
    \label{fig:pre}
\end{figure}

\subsection{Proposed Architecture}
\label{sec43}

\begin{figure*}[t]
    \centering
    \includegraphics[width=0.9\textwidth]{ Main_model_plot.pdf}
    \caption{Architecture of the SincPD. The model begins with an input layer followed by SincConv1D layers, specialized for handling 1D signals. These layers are normalized and activated using the Leaky ReLU function. Following the SincConv1D layers, traditional convolutional layers with batch normalization extract higher-level features. Finally, dense layers are utilized for classification. }
    \label{fig:model_architecture}
\end{figure*}

The initial model architecture, depicted in Fig.~\ref{fig:model_architecture}, is designed to efficiently extract and refine meaningful features from vGRF data using a sequential framework. Traditional convolutional layers, while powerful for feature extraction, often require a high number of learnable parameters and lack mathematical interpretability, making them less suitable for tasks prioritizing explainability and efficiency.

To address these challenges, the first layers of our model utilize SincConv1D filters. These filters are particularly effective in capturing critical low-level features essential for subsequent processing while maintaining interpretability due to their frequency domain meaning (adaptive bandpass filters). Each filter is parameterized by only two learnable parameters (cutoff frequencies), reducing model complexity and allowing direct analysis of the learned frequency bands. The subsequent layers and their configurations, described in this section, refine these features into robust predictive insights.


\subsubsection{SincConv1D Layers}
\label{subsubsec:sincconv1d_layers}

We employ eight SincConv1D layers (one per preprocessed signal) to extract frequency-specific features from vGRF data. In the initial version of the model, each layer is configured with 100 filters of length 101. The extracted features are normalized, activated with Leaky ReLU, and pooled to reduce computational load. Finally, all eight outputs are concatenated, forming a unified representation for subsequent network layers.


% \subsubsection{Convolutional, Dense Layers, and Model Compilation}
\subsubsection{Model Structure }
\label{subsubsec:conv_dense_combined}

Following the frequency-focused SincConv1D feature extraction, two standard Conv1D layers (128 and 256 filters) further refine the learned representations. These layers are followed by Batch Normalization, Leaky ReLU, Dropout, and Max Pooling to improve model generalization and computational efficiency.

The outputs are then flattened and passed through three Dense layers, with 128, 64, and 1 neuron(s), respectively. The first two layers use ReLU activation, while the final layer employs a sigmoid activation function to output a probability score between 0 and 1. Batch Normalization and L2 regularization are applied to enhance stability and reduce overfitting. Finally, the Adam optimizer~\citep{kingma2014adam} is applied to learn model's parameters by minimizing the cross-entropy.

% \subsection{Pruning Method}
    
%     In this section, to address the high number of filters in the trained SincConv1D layers, which can be challenging to interpret, we propose a pruning method that reduces the filters to the minimum necessary while maintaining model performance. This is achieved by clustering the filters and constructing a new architecture based on the centroids of these clusters.

%     As described previously~\ref{subsec:sincnet}, the SincConv1D layer learns two parameters: the center frequency ($f_c$) and the bandwidth ($b$) of the sinc filters, which control their frequency response. These parameters form the basis for clustering and identifying redundant filters.
    
%     To group similar pairs of ($f_c$, $b$) parameters across filters, We employ K-means clustering \citep{kmeans} and  retain only the most significant clusters that impact performance. To approximate the range of important clusters, we first optimize the filter count for Sensor~1 using the elbow method and silhouette scores \citep{geron2022hands}. Subsequently, the optimal $k$ value is determined for each sensor by examining silhouette diagrams for layers \texttt{sinc\_conv1d\_0} through \texttt{sinc\_conv1d\_7}. Finally, we cluster each sensor’s filter data and visualize the results (filters and centroids) to confirm that the retained filters effectively capture critical patterns.
    
%     With the clusters identified, we construct a new model architecture using the centroids of these clusters. The new architecture retains the essential filter patterns while significantly reducing the number of parameters, thereby improving computational efficiency without sacrificing performance. After identifying the clusters, we use the centroids of these clusters as the weights of our SincConv1D layers, replacing the rest of the sinc filters in the trained model. The remaining architecture is adapted without changes, connecting to the new SincConv1D layers, and the model is retrained for a few epochs to fine-tune performance.
    \subsection{Pruning Method}
    
        High number of filters decreases the interpretability. Therefore, we propose a pruning method to prune extra filters. To realize this, we cluster filters based on their cut-off frequencies and reconstruct the architecture based on the clusters' centroids.
        
        As previously described~\ref{subsec:sincnet}, SincConv1D layers learn the center frequency ($f_c$) and bandwidth ($b$) of sinc filters, which define their frequency response. These parameters are used for clustering and identifying redundancy. Using K-means clustering~\citep{kmeans}, we group similar ($f_c$, $b$) parameters and retain only the most significant clusters.
        
        To determine the optimal number of clusters, we apply the elbow method and silhouette scores~\citep{geron2022hands}, initially optimizing for Sensor~1. For each sensor, the optimal $k$ value is identified by analyzing silhouette diagrams for Sinc layers. The clusters and centroids are then visualized to ensure the retained filters capture critical patterns.
        
        The new architecture is constructed using the cluster centroids as filter weights in the SincConv1D layers, reducing parameters while preserving essential patterns. The model is retrained for a few epochs to fine-tune performance, with the rest of the architecture unchanged.
        
    

    
    \section{Experimental Results}
    \label{sec5}
    In this section, the performance of the proposed method is evaluated and compared to some previous methods in both diagnosis and severity determination. All experiments have been conducted in CoLab runtime environment based on Python, using the TensorFlow platform along with the SKLearn package.
    
    \subsection{Data and Evaluation Metrics}
    \label{subsec:Data_Evaluation}
    
    This study employs gait cycle data from PhysioNet\footnote{https://physionet.org/content/gaitpdb/1.0.0/}, comprising vGRF signals recorded via 16 sensors under participants’ feet. The dataset includes individuals with PD and healthy controls, capturing approximately two minutes of walking at a self-selected pace. Gait cycles (Fig.~\ref{gate_phases_fig}), divided into stance and swing  phases, highlight PD-related alterations in stride length and variability.
    
    A total of 166 subjects (93 PD, 73 controls) participated, with data aggregated from three separate studies (Table~\ref{tab:dataset}). PD severity was assessed based on the modified Hoehn and Yahr scale (Table~\ref{tab:PDseverity}), ranging from Stage~1 (unilateral involvement) to Stage~5 (complete disability).
    
    To evaluate the proposed approach, standard classification metrics including Accuracy, Precision, Recall, and F1 Score were employed.

    
    \begin{figure}
        \centering
        \includegraphics[height= 1.5in,width=3in]{ Gate_phase.pdf}
        \caption{Illustration of the gait cycle phases. The gait cycle consists of the stance phase (heel strike, loading response, mid-stance, terminal stance, pre-swing includes $60\%$ of the cycle) and the swing phase (toe-off, mid-swing, terminal swing include $40\%$ of the cycle). }
        \label{gate_phases_fig}
    \end{figure}
    
    
    \begin{table}
        \centering
        \caption{Number of participants in datasets categorized by severity level.}
        \captionsetup{font={footnotesize,rm}}

        \label{tab:dataset}
        \footnotesize
        \begin{tabular}{|c|c|c|c|c|}
        \hline
        \textbf{Dataset} & \textbf{Healthy} & \textbf{Stage 2} & \textbf{Stage 2.5} & \textbf{Stage 3} \\
        \hline
        Ga & 18 & 15 & 8 & 6 \\
        Ju & 26 & 12 & 13 & 4 \\
        Si & 29 & 29 & 6 & 0 \\
        \hline
        \end{tabular}
    \end{table}
    
    \begin{table}[tb]
    \renewcommand{\arraystretch}{1.2} % Adjust row height
    \centering
    \captionsetup{font={scriptsize,rm}}

    \caption{PD severity classification based on the modified H\&Y scale.}
    \label{tab:PDseverity}
    \resizebox{\columnwidth}{!}{%
        \begin{tabular}{|c|l|l|}
        \hline
        \textbf{Scale} & \textbf{Description} & \textbf{Stage} \\
        \hline
        1 & One side only & No functional disability \\
        1.5 & One side + axial symptoms & Early stage \\
        2 & Bilateral involvement & No balance impairment \\
        2.5 & Mild bilateral, recovery on pull test & Balance impairment \\
        3 & Mild/moderate bilateral & Impaired postural reflexes \\
        4 & Severe disability & Still capable of walking \\
        5 & Bedridden or wheelchair-bound & Completely disabled \\
        \hline
        \end{tabular}
    }
    \end{table}
    

    
    \subsection{Training the Initial Model}
    
        The explained model in section \ref{sec4} is built and trained using the preprocessed data for 1000 epochs. Training and validation accuracy progression (Fig.~\ref{fig:training_plots}) shows steady improvement, demonstrating effective learning and reliable generalization.
    
        \begin{figure}
            \centering
            \includegraphics[width=3in]{ train_acc.pdf}
            \caption {Training and validation accuracy progression over 1000 epochs}
            \label{fig:training_plots}
        \end{figure}
        
        \begin{figure}
            \centering
            \begin{minipage}{0.493\columnwidth}
                \centering
                \includegraphics [width=\textwidth]{ Confusion_Matrix_for_model_befor_pruning20.pdf}
            \end{minipage}
            \hfill
            \begin{minipage}{0.493\columnwidth}
                \centering
                \includegraphics[width=\textwidth]{ Confusion_Matrix_for_model_after_pruning20.pdf}
            \end{minipage}
            \caption{Confusion matrices: Left, classification performance before pruning; Right, classification performance after pruning.}
            \label{fig:combined_confusion}
        \end{figure}
        
    
        The performance of the trained SincNet model for binary classification is illustrated through a confusion matrix (Fig.~\ref{fig:combined_confusion}). The model achieves a high classification accuracy, correctly predicting 97\% of samples belong to the negative class and 99.55\% of the positive ones, with minimal misclassifications.
    
    \subsection{Filter Optimization}
    
        To optimize the number of filters in the SincConv1D layers, we utilize the elbow method and silhouette scores, as described in Section~\ref{sec42}, to estimate an initial range for the optimal number of clusters.
    
        \begin{figure}
            \centering
            \begin{minipage}{0.493\columnwidth}
                \centering
                \includegraphics[width=\textwidth]{ elbowS1.png}
            \end{minipage}
            \hfill
            \begin{minipage}{0.493\columnwidth}
                \centering
                \includegraphics[width=\textwidth]{ avgsilhouetteS1.png}
            \end{minipage}
            \caption{Left: Inertia plot for Sensor 1, using the elbow method to suggest an initial cluster range. Right: Silhouette scores for Sensor 1, peaking around \( k=4 \), confirming well-separated clusters.}
            \label{fig:elbow_silhouette_sensor1}
        \end{figure}
    
        From the analysis of the plots sampled for Sensor 1 (Fig.~\ref{fig:elbow_silhouette_sensor1}), the elbow method indicates a potential range of 3 to 7 clusters, while the silhouette scores peak around $k=4$. These observations suggest that the optimal number of filters lies within this range, which should be refined and evaluated for each sensor individually. Thus, we analyze the optimal $k$ using the silhouette diagram for each sensor. As shown in Fig.~\ref{fig:silhouette_diagram_sensor1}, the silhouette plots for Sensor 1 suggest $k=3$ as the best choice, providing optimal cluster cohesion and separation. Repeating this procedure for each sensor yields an efficient clustering configuration, improving the overall filter performance.
        
        \begin{figure}
            \centering
            \includegraphics[width=3in]{ silhouetteS1.png}
            \caption{Silhouette Diagram for Sensor 1, visually representing cluster cohesion and separation at the optimal $k$.}
            \label{fig:silhouette_diagram_sensor1}
        \end{figure}
    
        For two sample sensors (Sensor 1 and Sensor 5), scatter plots illustrate the distribution of filter weights and their centroids after clustering. For Sensor 1, the Silhouette Diagram analysis determined $k=3$, and the corresponding centroids were obtained using the k-means algorithm. Similarly, for Sensor 5, $k=4$ was identified as optimal, resulting in four centroid pairs of $(f_c, b)$. Fig.~\ref{fig:scatter_plots_all_sensors} displays these clustered filters and centroids.
        \begin{figure}
            \centering
            \includegraphics[width=1.\linewidth]{ scatter_plots_all_sensors_cut.pdf}
            \caption{Scatter plots for two sample sensors (Sensor 1 and Sensor 5), showing clustered filters and their centroids post-clustering. The x-axis represents the bandwidth parameter ($b$) and the y-axis represents the center frequency parameter ($f_c$).}
            \label{fig:scatter_plots_all_sensors}
        \end{figure}

        As described in Section~\ref{subsec:sincnet}, after retraining the model with the pruned filters, the updated architecture maintains nearly the same classification performance as the initial model, despite using significantly fewer SincNet filters. The confusion matrix (Fig.~\ref{fig:combined_confusion}, right) shows that the model retains 97\% accuracy for the negative class while experiencing only a minor decrease in accuracy for the positive class, from 99.55\% to 98.66\%.
    
        This minimal reduction in performance is achieved while reducing the total number of filters across the SincNet layers from 800 to approximately 30, significantly decreasing the parameters in the feature extraction layers.

        After retraining the pruned model, the preserved filters reveal key insights into the feature extraction process. Fig.~\ref{fig:sinc_filters} illustrates the trained Sinc filters for sensors numbered 5 to 8. These filters reflect the optimized cluster sizes determined during pruning, resulting in a varying number of filters per layer.
    
        \begin{figure}
            \centering
            \includegraphics[width=3in]{ plot_Sinc_filters_times_cut2.PDF}
            \caption{Visualization of Sinc Filters in Different Layers In Time (First Row) and Frequency (Second Row) Domain}
            \label{fig:sinc_filters}
        \end{figure}
        
        
        The top row of Fig.~\ref{fig:sinc_filters} displays the time-domain representations of the learned Sinc filters, while the bottom row shows their corresponding frequency responses. These frequency bands reveal how the pruned filters capture distinct features relevant to distinguishing healthy and patient signals.
    
        % To provide a comprehensive overview of all the remaining trained Sinc filters in our model, we present a 3D visualization in Figure \ref{fig:sinc_filters_3d}.
        
        % \begin{figure}
        %     \centering
        %     \includegraphics[width=2.5in]{ plot_Sinc_filters_3d_times.PDF}
        %     \caption{3D Visualization of All Sinc Filters}
        %     \label{fig:sinc_filters_3d}
        % \end{figure}
    

    \subsection{Effect of filters and sensors}
    
    After training the pruned model, we focus on interpreting the feature extraction process within the SincNet layers. By studying the energy distributions of the signals passing through these layers, we aim to uncover how the model distinguishes between healthy and patient signals. This analysis emphasizes the role of the pruned Sinc filters as high-level feature extractors, effectively capturing critical frequency-based patterns for classification.
    
    We aim to calculate the signal energy from the outputs of the pruned SincNet layers, treating these as abstract processed representations of the input signals. This enables us to identify key distinctions in energy patterns between the two classes, offering insights into the discriminatory power of the filters and sensors.
    
    Initially, we seek to identify representative signals for patient and healthy classes. One intuitive approach is to compute the mean signal energy for each class. However, using the mean of all signals may incorporate noise, resulting in similar energy distributions for both classes. This similarity can obscure the distinctions necessary to identify significant sensors and critical signals that enhance classification accuracy. To address this issue, we employ clustering methods such as DBSCAN~\citep{dbscan} to identify core signals that effectively represent each class.

    By employing DBSCAN and focusing on clustroids, we identify the most representative real samples of healthy and patient signals. These clustroids serve as class representatives, enabling accurate evaluation of energy distributions and highlighting key differences between PD and healthy signals. Fig.~\ref{fig:dbscan} depicts the clustroids obtained using DBSCAN clustering.
    
    \begin{figure}
        \centering
        \includegraphics[width=2.75in]{healthy_and_patient_centroids_DBSCAN.PDF}
        \caption{DBSCAN clustroids representing healthy and patient signals.}
        \label{fig:dbscan}
    \end{figure}
        
   Once the representative signals for patient and healthy classes are identified, they are passed through the pruned SincNet layers to analyze the output energy distributions. This analysis helps determine which specific filters within each sensor are most significant for distinguishing between the two classes. For example, in Sensor 1 (top-left corner of Fig.~\ref{fig:energy_filters_centroids}), the output energies of the first, second, and fourth filters exhibit noticeable differences between healthy and patient signals. These differences suggest that the corresponding Sinc filters are targeting specific frequency ranges crucial for classification. To systematically identify such impactful filters, we calculate the difference in output energy between healthy and patient signals across all filters of the eight sensors.
    
    \begin{figure}
        \centering
        \includegraphics[width=2.5in]{ main_diff_energy.pdf}
        \caption{Energy distribution for patient and healthy signals using centroids for two sample sensors (Sensor 1 and Sensor 4, first row) and the corresponding differences in energy values across filters (second row).}
        \label{fig:energy_filters_centroids}
    \end{figure}
    

    As illustrated in Fig.~\ref{fig:energy_filters_centroids} (second row), the corresponding differences in output energy values for two sample sensors (Sensor 1 and Sensor 4) are shown, highlighting some filters with notable variations between the two classes.

    We then analyze the distribution of these energy differences across all sensors and their pruned filters to identify the discriminatory power of individual filters and sensors. Fig.~\ref{fig:energy_difference_distribution} illustrates the distribution of energy differences for all sensors and filters. From this analysis, we observe that certain sensors and filters exhibit larger energy differences, making them critical for further investigation. To better understand the characteristics of these impactful filters, we focus on analyzing the top 20\% based on their energy difference metric.
    These filters could provide valuable information for understanding the mechanisms of the model's decision-making process. likely capturing crucial features for distinguishing between healthy and patient signals. 
    
    \begin{figure}
        \centering
        \includegraphics[width=2.25in]{ energy_difference_between_not_PD_and_PD_combined_one_figure.PDF}
        \caption{Distribution of Energy Differences Across Sensors and Filters}
        \label{fig:energy_difference_distribution}
    \end{figure}    
    
    \begin{figure}
        \centering
        \includegraphics[width=3in]{ top__Sinc_filters_in_times.PDF}
        \caption{Top Filters Based on Energy Differences (Time Domain)}
        \label{fig:top_filters_times}
    \end{figure}
    
    \begin{figure}
        \centering
        \includegraphics[width=3in]{ top_filters_in_frequency.PDF}
        \caption{Top Filters Based on Energy Differences (Frequency Domain)}
        \label{fig:top_filters_freq}
    \end{figure}
    
     Fig.~\ref{fig:top_filters_times} and Fig. \ref{fig:top_filters_freq} display the top filters identified in the time and frequency domains, respectively. From Fig.~\ref{fig:top_filters_freq}, the first two top filters, primarily associated with Sensor 7 (ball of the foot) and Sensor 2 (heel), as indicated in Fig.\ref{fig:pre} for sensor locations, focus on bandpassing frequencies in the range of approximately 0.4 to 0.6 Hz while attenuating other frequencies. Similarly, the fourth filter, derived from Sensor 7, targets a narrower band of 0.2 to 0.5 Hz. These frequency ranges highlight filter selectivity in targeting class-specific features, underlining their role in distinguishing between healthy and patient gait signals.

    Furthermore, based on Fig.~\ref{fig:energy_difference_distribution}, sensors positioned at the front (e.g., Sensors 6 and 7) and back (e.g., Sensors 1 to 3) of the foot demonstrate higher energy differences. This suggests that these regions contribute more prominently to the model's feature extraction process, potentially due to their biomechanical significance in gait patterns.
    
    % Analyzing the characteristics of these top filters, such as their frequency response and spatial distribution, can aid in understanding the discriminative features learned by the model and facilitate the development of interpretable diagnostic tools.
    
    
\subsection{Severity Model Integration}

We integrate the severity model for Parkinson’s disease into our framework using transfer learning. A pretrained cluster model, obtained via the pruning method described in Section~3.5, provides the initial weights for the severity model. To preserve learned representations, we freeze the clustered model’s layers and align the severity model’s layers to these pre-trained weights. This setup leverages the feature extraction capabilities already established by the clustered model.

After initialization, the severity model undergoes further training or evaluation to classify PD severity levels. It achieves a 97.22\% accuracy in multi-class classification. The confusion matrix for these severity predictions is shown in Fig.~\ref{fig:confusion_matrix_Severity}.


    \begin{figure}
        \centering
        \includegraphics[width=3in]{ Confusion_Matrix_for_model_sevirity_row.pdf}
        \caption{ Confusion matrix for severity model}
        \label{fig:confusion_matrix_Severity}
\end{figure}

\subsection{Performance Comparison}

Our proposed model demonstrates superior performance in both binary and multi-class classification tasks compared to state-of-the-art methods, as summarized in Tables~\ref{tab:performance_comparison_binary} and~\ref{tab:performance_comparison_severity}.


\begin{table}[ht]
\scriptsize
\centering
\captionsetup{font={scriptsize,rm}}
\caption{Model Performance Comparison Before and After Pruning State of Art Methods for Binary Classification}
\resizebox{\columnwidth}{!}{
  \begin{tabular}{|c|c|c|c|c|c|}
  \hline
  \textbf{Method} & \textbf{Acc (\%)} & \textbf{Prec (\%)} & \textbf{Recall (\%)} & \textbf{F1 (\%)} & \textbf{Params} \\
  \hline
  LSTM\textsuperscript{1} & \underline{98.60} & 98.23 & 96.6 & 98.95 & \underline{1.12M} \\ \hline
  Multi LSTM\textsuperscript{2} & 91.95 & 88.75 & 96.66 & 92.53 & 33.86K \\ \hline
  CNN+LSTM\textsuperscript{3} & 98.09 & \textbf{99.22} & \textbf{100} & 98.04 & 89.2M \\ \hline
  ResNet-101\textsuperscript{4} & 97.56 & - & 97.73 & - & 44.5M \\ \hline
  \textbf{SincPD} & \textbf{98.77} & \underline{98.67} & \underline{99.55} & \textbf{99.11} & 1.17M \\
  \textit{Before Pruning} & & & & & \\ \hline
  \textbf{SincPD} & 98.15 & 98.66 & 98.66 & 98.66 & \textbf{872K} \\
  \textit{After Pruning} & & & & & \\ 
  \hline
  \end{tabular}
}
\label{tab:performance_comparison_binary}
\end{table}





% \begin{table}[ht]
% \scriptsize
% \centering
% \captionsetup{font={scriptsize,rm}}
% \caption{Comparison of Model Performance with State of Art Deep Learning Methods for Multi-Classification}
% \resizebox{\columnwidth}{!}{
%   \begin{tabular}{|c|c|c|c|c|}
%   \hline
%   \textbf{Method} & \textbf{Acc (\%)} & \textbf{Prec (\%)} & \textbf{Recall (\%)} & \textbf{F1 (\%)} \\
%   \hline
%   1D CNN\textsuperscript{5} & 85.23 & 87.30 & 85.3 & 85.3  \\ \hline
%   LSTM\textsuperscript{2} & 96.60 & 98.70 & 96.20 & 97.43  \\ \hline
%   LSTM and CNN\textsuperscript{6} & 98.32 & 98.10 & 98.29 & 98.19 \\ \hline
%   ANN-FFT\textsuperscript{7} & 97.00 & 98.00 & 94.00 & 96.00 \\ \hline
%   \textbf{Proposed Model} & 97.22 & 97.30 & 97.22 & 97.24  \\ \hline
%   \end{tabular}
% }
% \label{tab:performance_comparison_severity}
% \end{table}
% \footnotetext{* \textsuperscript{1} \citep{LSTM}, \textsuperscript{2} \citep{SalimiBadr-MultiLSTM}, \textsuperscript{3} \citep{Liu2021CNN_and_LSTM}, \textsuperscript{4} \citep{resnet-101}, 
% \textsuperscript{5} \citep{deep1d}, 
% \textsuperscript{6} \citep{BVIDYA2022105099}, \textsuperscript{7} \citep{ANN-FFT}.}

\begin{table}[ht]
\scriptsize
\centering
\captionsetup{font={scriptsize,rm}}
\caption{Comparison of Model Performance with State of Art Deep Learning Methods for Multi-Classification}
\resizebox{\columnwidth}{!}{
  \begin{tabular}{|c|c|c|c|c|}
  \hline
  \textbf{Method} & \textbf{Acc (\%)} & \textbf{Prec (\%)} & \textbf{Recall (\%)} & \textbf{F1 (\%)} \\
  \hline
  1D CNN\textsuperscript{5} & 85.23 & 87.30 & 85.3 & 85.3  \\ \hline
  LSTM\textsuperscript{2} & 96.60 & \textbf{98.70} & \underline{96.20} & \textbf{97.43}  \\ \hline

  ANN-FFT\textsuperscript{6} & \underline{97.00} & \underline{98.00} & 94.00 & 96.00 \\ \hline
  \textbf{SincPD} & \textbf{97.22} & 97.30 & \textbf{97.22} & \underline{97.24}  \\ \hline
  \end{tabular}
}
\label{tab:performance_comparison_severity}
\end{table}
\footnotetext{* \textsuperscript{1} \citep{LSTM}, \textsuperscript{2} \citep{SalimiBadr-MultiLSTM}, \textsuperscript{3} \citep{Liu2021CNN_and_LSTM}, \textsuperscript{4} \citep{resnet-101}, 
\textsuperscript{5} \citep{deep1d}, 
\textsuperscript{6} \citep{ANN-FFT}.}


\noindent Tables~\ref{tab:performance_comparison_binary} and~\ref{tab:performance_comparison_severity} illustrate that the proposed model outperforms existing state-of-the-art methods in both binary and multi-class classification tasks. For binary classification, it achieves an accuracy of 98.77\% before pruning and maintains 98.15\% post-pruning with only 872K parameters, surpassing model like LSTM  in both performance and efficiency. In multi-class classification, the model attains a 97.22\% accuracy, which is competitive with top methods such as LSTM and LSTM-CNN combinations. Additionally, the proposed model offers enhanced explainability compared to existing approaches. These results demonstrate the proposed model’s superior accuracy, precision, recall, and F1 scores while significantly reducing model complexity, underscoring its effectiveness, scalability, and interpretability in diverse classification scenarios.



\section{Conclusions}
\label{sec6}
In this paper an interpretable deep structure is proposed to classify patients with Parkinson's disease and healthy subjects according to their gait cycle pattern. The proposed method can also determine the severity of the disease. 

The proposed method is a deep structure that takes the low-level raw vertical Ground Reaction Force (vGRF) signal recorded by 16 sensors put in the subjects' shoes as the input and extract higher level features based on applying various filters. Our proposed method applies bandpass filters with sinc-shape impulse responses in its first layers. Consequently, model is encouraged to learn the main frequencies of each sensor which is various between patients and healthy movements. This extraction leads to extract more meaningful features. These filters have lower number of parameters that make the method a light-weight deep model. Moreover, based on analyzing the active filters during the inference process, the important filters and sensors are determined. This information can be utilized to explain the network's output.

To train the proposed method, first a large model with a lot of filters is trained. Next, the extracted bandpass filters are studied and clustered based on their cut-off frequencies. Based on this clustering, the extra filters are pruned by replacing them with the medoids of the extracted clusters. 

The model achieved an accuracy of 98.77\% and 98.15\%  before and after applying the pruning process in PD diagnosis, and an accuracy of 97.22\% in severity detection problem. The proposed method outperforms previous models with a more parsimonious structure while providing more explanations on the reasons behind its decisions.
\newpage
\section*
\noindent\textbf{Authors' Contributions}
\textbf{Armin Salimi-Badr}: Conceptualization, Methodology, Formal analysis, Theoretical analysis, Validation, Investigation, Supervision, Writing - original draft, Writing – review \& editing, Project administration.
\textbf{Mahan Veisi} and \textbf{Sadra Berangi}: Methodology, Software, Validation, Data curation, Visualization, Writing - original draft. 

\noindent\textbf{Funding}
The is not any funding .

\noindent\textbf{Data Availability}  The dataset used during the current study is an open access dataset from free public database, available in PhysioNet\footnote{https:$\slash$$\slash$physionet.org$\slash$content$\slash$gaitpdb$\slash$1.0.0$\slash$}. 

\section*{Declarations} 
\textbf{Conflict of Interest}  None.

\noindent \textbf{Ethical Approval} Not applicable.

\noindent\textbf{Consent to Participate} Not applicable.

\noindent\textbf{Consent for Publication} Not applicable.

\noindent\textbf{Clinical Trial Number} Not applicable.

%\bibliographystyle{elsarticle-num}
%\bibliographystyle{elsarticle-num-names}
%\bibliographystyle{model5-names}
%\bibliographystyle{spmpsci}
% \bibliographystyle{IEEEtran}
% \bibliographystyle{sn-basic}
\setlength{\bibsep}{0pt} % No space between entries
\bibliography{myref.bib}


\end{document} 

% 匿名投稿
% -------------------------------------
\bmhead{Acknowledgements}
This study is supported by the Independent Innovation Science Fund of National University of Defense Technology (NO.22-ZZCX-064) and the National Natural Science Foundation of China (No. 62406331).

\bmhead{Author contributions}
Y.-P.Z. conceived the idea, contributed to writing the manuscript, conducted experiments, and developed the WSI system. Q.S. performed experiments, wrote code, analyzed data, and prepared figures. X.-X.F. contributed to code writing, data analysis, and figure drawing. J.H., Y.-K.Z., and Y.-F.Z. were involved in data analysis. C.-F.Z., X.L., and W.-X.W. participated in discussions. D.-D.W. prepared and provided the microscope slide samples. S.Y. contributed to writing, discussions, and data analysis.

\bmhead{Materials \& Correspondence}
Correspondence and requests for materials should be addressed to Yongping Zhai.
% ---------------------------


\bmhead{Competing interests}
The authors declare no competing interests.

\end{document}
