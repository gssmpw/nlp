\pdfoutput=1

\documentclass[11pt]{article}

\usepackage[]{acl}

\usepackage{times}
\usepackage{latexsym}

\usepackage[T1]{fontenc}

\usepackage[utf8]{inputenc}

\usepackage{microtype}

\usepackage{inconsolata}

\usepackage{graphicx}
\usepackage{subcaption}
\usepackage{multirow}
\usepackage{tabularx}
\usepackage{todonotes}
\usepackage{amsmath}
\usepackage{booktabs}
\usepackage{tcolorbox}
\usepackage{mathtools}
\usepackage{amsfonts}
\usepackage{bbm}

 \newcommand*\iftodonotes{\if@todonotes@disabled\expandafter\@secondoftwo\else\expandafter\@firstoftwo\fi}  %
\makeatother
\newcommand{\noindentaftertodo}{\iftodonotes{\noindent}{}}
\newcommand{\fixme}[2][]{\todo[color=yellow,size=\scriptsize,fancyline,caption={},#1]{#2}} %
\newcommand{\note}[4][]{\todo[author=#2,color=#3,size=\scriptsize,fancyline,caption={},#1]{#4}} %
\newcommand{\mathtutorbench}{\textsc{MathTutorBench}}
\newcommand{\Nico}[2][]{\note[#1]{nico}{blue!40}{#2}}
\newcommand\nico[1]{\textcolor{violet}{[Nico\@: #1]}}
\newcommand{\Jakub}[2][]{\note[#1]{jakub}{orange}{#2}}
\newcommand\jakub[1]{\textcolor{orange}{[Jakub\@: #1]}}
\newcommand{\Ido}[2][]{\note[#1]{ido}{red}{#2}}
\newcommand\ido[1]{\textcolor{blue!70,}{[Ido\@: #1]}}
\newcommand\revision[1]{\textcolor{blue}{#1}}


%%
%% Common definitions
%% Only contains the definitions, not formatting information
%%

\usepackage{xspace}
\usepackage{bbm}
%\usepackage{mathrsfs}
%% \usepackage{semantic}
%% \usepackage{bm}
\input{mathlig}
\usepackage{mathtools}
\usepackage{relsize}
\usepackage{mathrsfs}
\usepackage{dsfont}
%% Inner product
\DeclarePairedDelimiterX{\inp}[2]{\langle}{\rangle}{#1, #2}
\makeatletter
\newcommand*\bigcdot{\mathpalette\bigcdot@{.5}}
\newcommand*\bigcdot@[2]{\mathbin{\vcenter{\hbox{\scalebox{#2}{$\m@th#1\bullet$}}}}}
\makeatother

%% Small spacing
\newcommand{\muspace}{\mspace{1mu}}

%% |
\DeclareRobustCommand{\scond}{\mathchoice{\muspace\vert\muspace}{\vert}{\vert}{\vert}}
\mathlig{|}{\scond}
\newcommand{\cond}{\mathchoice{\,\vert\,}{\mspace{2mu}\vert\mspace{2mu}}{\vert}{\vert}}
\newcommand{\bigcond}{\,\big\vert\,}

%% :
\DeclareRobustCommand{\discint}{\mathchoice{\mspace{-1.5mu}:\mspace{-1.5mu}}{\mspace{-1.5mu}:\mspace{-1.5mu}}{:}{:}}
\mathlig{::}{\discint}
\newcommand{\suchthat}{\mathchoice{\colon}{\colon}{:\mspace{1mu}}{:}}
\newcommand{\mappingfrom}{\colon}

%% A few operator definitions
\def\tr{\mathop{\rm tr}\nolimits}%
\def\var{\mathop{\rm Var}\nolimits}%
\def\Var{\mathop{\rm Var}\nolimits}%
\def\cov{\mathop{\rm Cov}\nolimits}%
\def\Cov{\mathop{\rm Cov}\nolimits}%
\def\diag{\mathop{\rm diag}\nolimits}%
\def\Re{\mathop{\rm Re}\nolimits}%
\def\Im{\mathop{\rm Im}\nolimits}%
\def\rank{\mathop{\rm rank}\nolimits}%
\def\essinf{\mathop{\rm ess\,inf}}%
\def\sgn{\mathop{\rm sgn}\nolimits}%
\def\erfc{\mathop{\rm erfc}\nolimits}%
\def\co{\mathop{\rm co}\nolimits}%

%% Mathcal (discrete sets)
\newcommand{\Ac}{\mathcal{A}}
\newcommand{\Bc}{\mathcal{B}}
\newcommand{\Cc}{\mathcal{C}}
\newcommand{\cc}{\text{\footnotesize $\mathcal{C}$} }
\newcommand{\Dc}{\mathcal{D}}
\newcommand{\Ec}{\mathcal{E}}
\newcommand{\Fc}{\mathcal{F}}
\newcommand{\Gc}{\mathcal{G}}
\newcommand{\Hc}{\mathcal{H}}
\newcommand{\Ic}{\mathcal{I}}
\newcommand{\Jc}{\mathcal{J}}
\newcommand{\Kc}{\mathcal{K}}
\newcommand{\Lc}{\mathcal{L}}
\newcommand{\Mc}{\mathcal{M}}
\newcommand{\Nc}{\mathcal{N}}
\newcommand{\Pc}{\mathcal{P}}
\newcommand{\Qc}{\mathcal{Q}}
\newcommand{\Rc}{\mathcal{R}}
\newcommand{\Sc}{\mathcal{S}}
\newcommand{\Tc}{\mathcal{T}}
\newcommand{\Uc}{\mspace{1.5mu}\mathcal{U}}
\newcommand{\Vc}{\mathcal{V}}
\newcommand{\Wc}{\mathcal{W}}
\newcommand{\Xc}{\mathcal{X}}
\newcommand{\Yc}{\mathcal{Y}}
\newcommand{\Zc}{\mathcal{Z}}

%% Mathcal bold (random discrete sets)
\newcommand{\Acb}{\boldsymbol{\mathcal{A}}}
\newcommand{\Bcb}{\boldsymbol{\mathcal{B}}}
\newcommand{\Ccb}{\boldsymbol{\mathcal{C}}}
\newcommand{\Dcb}{\boldsymbol{\mathcal{D}}}
\newcommand{\Ecb}{\boldsymbol{\mathcal{E}}}
\newcommand{\Fcb}{\boldsymbol{\mathcal{F}}}
\newcommand{\Gcb}{\boldsymbol{\mathcal{G}}}
\newcommand{\Hcb}{\boldsymbol{\mathcal{H}}}
\newcommand{\Jcb}{\boldsymbol{\mathcal{J}}}
\newcommand{\Kcb}{\boldsymbol{\mathcal{K}}}
\newcommand{\Ncb}{\boldsymbol{\mathcal{N}}}
\newcommand{\Pcb}{\boldsymbol{\mathcal{P}}}
\newcommand{\Qcb}{\boldsymbol{\mathcal{Q}}}
\newcommand{\Rcb}{\boldsymbol{\mathcal{R}}}
\newcommand{\Scb}{\boldsymbol{\mathcal{S}}}
\newcommand{\Tcb}{\boldsymbol{\mathcal{T}}}
\newcommand{\Ucb}{\boldsymbol{\mathcal{U}}}
\newcommand{\Vcb}{\boldsymbol{\mathcal{V}}}
\newcommand{\Wcb}{\boldsymbol{\mathcal{W}}}
\newcommand{\Xcb}{\boldsymbol{\mathcal{X}}}
\newcommand{\Ycb}{\boldsymbol{\mathcal{Y}}}
\newcommand{\Zcb}{\boldsymbol{\mathcal{Z}}}


%% Script sets: Capacity region, set of probs, achievable rate region, etc.
\newcommand{\Ar}{\mathscr{A}}
\newcommand{\Cr}{\mathscr{C}}
\newcommand{\Dr}{\mathscr{D}}
\renewcommand{\Pr}{\mathscr{P}}
\newcommand{\Rr}{\mathscr{R}}
\newcommand{\Ur}{\mathscr{U}}
\newcommand{\Kr}{\mathscr{K}}
\newcommand{\Gr}{\mathscr{G}}
\newcommand{\Sr}{\mathscr{S}}

%% Boldface vectors
%%
\newcommand{\Lv}{{\bf L}}
\newcommand{\Xv}{{\bf X}}
\newcommand{\Yv}{{\bf Y}}
\newcommand{\Zv}{{\bf Z}}
\newcommand{\Uv}{{\bf U}}
\newcommand{\Wv}{{\bf W}}
\newcommand{\Vv}{{\bf V}}
\newcommand{\Sv}{{\bf S}}
\newcommand{\av}{{\bf a}}
\newcommand{\bv}{{\bf b}}
\newcommand{\cv}{{\bf c}}
\newcommand{\fv}{{\bf f}}
\newcommand{\gv}{{\bf g}}
\newcommand{\lv}{{\boldsymbol l}}
\newcommand{\mv}{{\bf m}}
\newcommand{\xv}{{\bf x}}
\newcommand{\yv}{{\bf y}}
\newcommand{\zv}{{\bf z}}
\newcommand{\uv}{{\bf u}}
\newcommand{\vv}{{\bf v}}
\newcommand{\sv}{{\bf s}}
\newcommand{\alphav}{{\boldsymbol \alpha}}
\newcommand{\betav}{{\boldsymbol \beta}}
\newcommand{\gammav}{{\boldsymbol \gamma}}
\newcommand{\lambdav}{{\boldsymbol \lambda}}
\newcommand{\muv}{{\boldsymbol \mu}}
\newcommand{\nuv}{{\boldsymbol \nu}}
\newcommand{\onev}{{\boldsymbol 1}}

\newcommand{\ph}{{\hat{p}}}
\newcommand{\qh}{{\hat{q}}}
\newcommand{\pemp}{\ph_{\mathrm{emp}}}
\newcommand{\qemp}{q_{\mathrm{emp}}}
\newcommand{\memp}{\hat{m}_{\mathrm{emp}}}

\newcommand{\xbayes}{\hat{x}^*_{\Lambdab}}
\newcommand{\sbayes}{\hat{s}^*_{\Lb}}

%% Boldface vectors and more (added by Jongha)
%%

\newcommand{\cb}{{\mathbf c}}
\newcommand{\Cb}{{\mathbf C}}
\newcommand{\eb}{{\mathbf e}}
\newcommand{\hb}{{\mathbf h}}
\newcommand{\Lb}{{\mathbf L}}
\newcommand{\pb}{{\mathbf p}}
\newcommand{\Pb}{{\mathbf P}}
\newcommand{\qb}{{\mathbf q}}
\newcommand{\Qb}{{\mathbf Q}}
\newcommand{\rb}{{\mathbf r}}
\newcommand{\Sb}{{\mathbf S}}
\newcommand{\tb}{\mathbf{t}}
\newcommand{\ub}{{\mathbf u}}
\newcommand{\Ub}{{\mathbf U}}
\newcommand{\vb}{{\mathbf v}}
\newcommand{\Vb}{{\mathbf V}}
\newcommand{\wb}{{\mathbf w}}
\newcommand{\Wb}{{\mathbf W}}
\newcommand{\wbh}{\hat{\wb}}
\newcommand{\xb}{{\mathbf x}}
\newcommand{\Xb}{{\mathbf X}}
\newcommand{\yb}{{\mathbf y}}
\newcommand{\Yb}{\mathbf{Y}}
\newcommand{\zb}{{\mathbf z}}
\newcommand{\Zb}{{\mathbf Z}}

\newcommand{\chib}{\boldsymbol{\chi}}
\newcommand{\chibh}{\hat{\chib}}

% For DUDE
\newcommand{\rhob}{\boldsymbol{\rho}}
\newcommand{\Lbar}{\bar{\Lb}}
\newcommand{\Lambdab}{\boldsymbol{\Lambda}}
\newcommand{\Lambdabar}{\bar{\Lambdab}}

% For online learning?
\newcommand{\regret}{\mathsf{Reg}}
\newcommand{\Rbar}{\overline{\regret}}
\newcommand{\redundancy}{\mathsf{Red}}

%% Acronyms
%%
\newcommand{\whp}{\textnormal{w.h.p. }}
\newcommand{\wrt}{\textnormal{w.r.t. }}
\newcommand{\nn}{\textnormal{-NN}}
\newcommand{\knn}{k\nn}



%% Hats
\newcommand{\Kh}{{\hat{K}}}
\newcommand{\Lh}{{\hat{L}}}
\newcommand{\Mh}{{\hat{M}}}
\newcommand{\Rh}{{\hat{R}}}
\newcommand{\Sh}{{\hat{S}}}
\newcommand{\Uh}{{\hat{U}}}
\newcommand{\Vh}{{\hat{V}}}
\newcommand{\Wh}{{\hat{W}}}
\newcommand{\Xh}{{\hat{X}}}
\newcommand{\Yh}{{\hat{Y}}}
\newcommand{\Zh}{{\hat{Z}}}
\renewcommand{\dh}{{\hat{d}}}
\newcommand{\kh}{{\hat{k}}}
\newcommand{\lh}{{\hat{l}}}
\newcommand{\mh}{{\hat{m}}}
\newcommand{\sh}{{\hat{s}}}
\newcommand{\uh}{{\hat{u}}}
\newcommand{\vh}{{\hat{v}}}
\newcommand{\wh}{{\hat{w}}}
\newcommand{\xh}{{\hat{x}}}
\newcommand{\yh}{{\hat{y}}}
\newcommand{\zh}{{\hat{z}}}
\newcommand{\gh}{{\hat{g}}}

%% Tildes
\newcommand{\At}{{\tilde{A}}}
\newcommand{\Ct}{{\tilde{C}}}
\newcommand{\Kt}{{\tilde{K}}}
\newcommand{\Lt}{{\tilde{L}}}
\newcommand{\Mt}{{\tilde{M}}}
\newcommand{\Nt}{{\tilde{N}}}
\newcommand{\Qt}{{\tilde{Q}}}
\newcommand{\Rt}{{\tilde{R}}}
\newcommand{\St}{{\tilde{S}}}
\newcommand{\Ut}{{\tilde{U}}}
\newcommand{\Vt}{{\tilde{V}}}
\newcommand{\Wt}{{\tilde{W}}}
\newcommand{\Xt}{{\tilde{X}}}
\newcommand{\Yt}{{\tilde{Y}}}
\newcommand{\Zt}{{\tilde{Z}}}
\newcommand{\bt}{{\tilde{b}}}
\newcommand{\dt}{{\tilde{d}}}
\newcommand{\kt}{{\tilde{k}}}
\newcommand{\lt}{{\tilde{l}}}
\newcommand{\mt}{{\tilde{m}}}
\newcommand{\pt}{{\tilde{p}}}
\newcommand{\rt}{{\tilde{r}}}
\newcommand{\std}{{\tilde{s}}}  % to avoid clash with soul.sty
\newcommand{\ut}{{\tilde{u}}}
\newcommand{\vt}{{\tilde{v}}}
\newcommand{\wt}{{\tilde{w}}}
\newcommand{\xt}{{\tilde{x}}}
\newcommand{\yt}{{\tilde{y}}}
\newcommand{\zt}{{\tilde{z}}}
\newcommand{\gt}{{\tilde{g}}}

%% Greek
\def\a{\alpha}
\def\b{\beta}
\def\g{\gamma}
\def\d{\delta}
\def\e{\epsilon}
\def\eps{\epsilon}
% \def\l{\lambda}
\def\th{\theta}
\def\Th{\Theta}

%% Probability and expectation
\DeclareMathOperator\E{\mathsf{E}}
\let\P\relax
\DeclareMathOperator\P{\mathsf{P}}
\DeclareMathOperator\Q{\mathsf{Q}}
\DeclareMathOperator\Prob{\mathrm{Pr}}

%% Gaussian capacity, rate-dist, binary entropy
%\DeclareMathOperator\C{\textsf{C}}
\DeclareMathOperator\R{\mathsf{R}}
%\DeclareMathOperator\C{C}
%\DeclareMathOperator\R{R}

%% Error symbol
\def\error{\mathrm{e}}

%% Probability distributions
%%\newcommand{\Br}{\mathrm{Bern}}
\newcommand{\Bern}{\mathrm{Bern}}
\newcommand{\Geom}{\mathrm{Geom}}
\newcommand{\Poisson}{\mathrm{Poisson}}
\newcommand{\Exp}{\mathrm{Exp}}
%%\newcommand{\Norm}{\mathrm{N}}
\newcommand{\N}{\mathrm{N}}
\newcommand{\Unif}{\mathrm{Unif}}
\newcommand{\Binom}{\mathrm{Binom}}
%% Misc
\newcommand\eg{e.g.,\xspace}
\newcommand\ie{i.e.,\xspace}
\def\textiid{i.i.d.\@\xspace}
\newcommand\iid{\ifmmode\text{ i.i.d. } \else \textiid \fi}
\let\textand\and
%\renewcommand\and{\ifmmode{\text{ and }}{\textand}}
\let\textor\or
%\renewcommand\or{\ifmmode{\text{ or }}{\textor}}
\newcommand{\geql}{\succeq}
\newcommand{\leql}{\preceq}
\newcommand{\Complex}{\mathbb{C}}
\newcommand{\Ff}{\mathbb{F}}
\newcommand{\Zz}{\mathbb{Z}}
\newcommand{\Real}{\mathbb{R}}
\newcommand{\Natural}{\mathbb{N}}
\newcommand{\Integer}{\mathbb{Z}}
\newcommand{\Rational}{\mathbb{Q}}
\newcommand{\ones}{\mathds{1}}
%\newcommand{\defeq}{=}
\newcommand{\carddot}{\mathchoice{\!\cdot\!}{\!\cdot\!}{\cdot}{\cdot}}

%%-------------------------------------------------------
%% Fractions
%%
% \newcommand{\sfrac}[2]{\mbox{\small$\displaystyle\frac{#1}{#2}$}}
\newcommand{\half}{\frac{1}{2}}%\sfrac{1}{2}

%%-------------------------------------------------------
%% Cups and caps
%%
\def\medcupmath{\mbox{\small$\displaystyle\bigcup$}}
\def\medcuptext{\raisebox{1pt}{\small$\textstyle\bigcup$}}
\def\medcup{\mathop{\mathchoice{\medcupmath}{\medcuptext}{\bigcup}{\bigcup}}}

\def\medcapmath{\raisebox{0.5pt}{\small$\displaystyle\bigcap$}}
\def\medcaptext{\raisebox{1pt}{\small$\textstyle\bigcap$}}
\def\medcap{\mathop{\mathchoice{\medcapmath}{\medcaptext}{\bigcap}{\bigcap}}}


%%-------------------------------------------------------
%% Products
%%
\def\medtimesmath{\mbox{\small$\displaystyle\bigtimes$}}
\def\medtimestext{\raisebox{1pt}{\small$\textstyle\bigtimes$}}
\def\medtimes{\mathop{\mathchoice{\medtimesmath}{\medtimestext}{\bigtimes}{\bigtimes}}}


%%--------------------------------------------------------
%% Spacing
%%
\def\mathllap{\mathpalette\mathllapinternal}
\def\mathllapinternal#1#2{%
  \llap{$\mathsurround=0pt#1{#2}$}}

\def\mathrlap{\mathpalette\mathrlapinternal}
\def\mathrlapinternal#1#2{%
  \rlap{$\mathsurround=0pt#1{#2}$}}

\def\clap#1{\hbox to 0pt{\hss#1\hss}}
\def\mathclap{\mathpalette\mathclapinternal}
\def\mathclapinternal#1#2{%
  \clap{$\mathsurround=0pt#1{#2}$}}

%% For lecture notes
\def\labelitemi{$\bullet$}
\def\labelitemii{$\bullet$}
%\def\labelitemii{$\circ$}

%%-------------------------------------------------------
%% Stackrel
%%

\let\oldstackrel\stackrel
\renewcommand{\stackrel}[2]{\oldstackrel{\mathclap{#1}}{#2}}

%-------------------------------------------------------------------------
% Interchapter macros

% Relative entropy and f-divergence
\DeclarePairedDelimiterX{\infdivx}[2]{(}{)}{%
  #1\;\delimsize\|\;#2%
}
\newcommand{\D}{D\infdivx*}
\newcommand{\Df}{D_f\infdivx*}
% Total variation distance
\def\dtv{\delta}

%-------------------------------------------------------------------------
% Overline
\newcommand{\Hbar}{H\mathllap{\overline{\vphantom{H}\hphantom{\rule{6.9pt}{0pt}}}\mspace{0.7mu}}}
\newcommand{\Dbar}{D\mathllap{\overline{\vphantom{D}\hphantom{\rule{0.7em}{0pt}}}\mspace{0.7mu}}\divx}
\newcommand{\Gbar}{G\mathllap{\overline{\vphantom{G}\hphantom{\rule{6.25pt}{0pt}}}\mspace{1.1mu}}}
\renewcommand{\hbar}{h\mathllap{\overline{\vphantom{h}\hphantom{\rule{4.6pt}{0pt}}}\mspace{0.77mu}}}

%------------------------------------
% URL tilde
\catcode`~=11 % make LaTeX treat tilde (~) like a normal character
\newcommand{\urltilde}{\kern -.06em\lower -.06em\hbox{~}\kern .02em}
\catcode`~=13 % revert back to treating tilde (~) as an active character

%------------------------------------
% Hyphenation
\hyphenation{Gauss-ian}
\hyphenation{qua-dra-tic}
\hyphenation{Vis-wa-nath}
\hyphenation{non-trivial}
\hyphenation{multi-letter}
\hyphenation{Gauss-ian}
\hyphenation{super-posi-tion}
\hyphenation{de-cod-er}
\hyphenation{Nara-yan}
\hyphenation{multi-message}
\hyphenation{Dimi-tris}
\hyphenation{Pol-ty-rev}
\hyphenation{multi-cast}
\hyphenation{multi-user}
\hyphenation{multi-plex-ing}
\hyphenation{bi-directional}
\hyphenation{comput}

%------------------------------------
% Added by J. Ryu
\DeclareMathOperator*{\argmin}{arg\,min}
\DeclareMathOperator*{\argmax}{arg\,max}
\newcommand{\ind}[1]{\ones_{#1}}
\newcommand{\indi}[1]{\ones_{\{#1\}}}

% \DeclarePairedDelimiter{\norm}{\lVert}{\rVert}
% \DeclarePairedDelimiter{\abs}{\lvert}{\rvert}
\DeclarePairedDelimiterX{\norm}[1]{\lVert}{\rVert}{#1}
\DeclarePairedDelimiterX{\abs}[1]{\lvert}{\rvert}{#1}

\usepackage{xparse}

\newcommand*\diff{\mathop{}\!\mathrm{d}}
\newcommand*\Diff[1]{\mathop{}\!\mathrm{d^#1}}
\let\oldpartial\partial
\renewcommand*{\partial}{\mathop{}\!\oldpartial}
%
% \DeclarePairedDelimiterX{\set}[1]{\{}{\}}{\setargs{#1}}
% \NewDocumentCommand{\setargs}{>{\SplitArgument{1}{;}}m}
% {\setargsaux#1}
% \NewDocumentCommand{\setargsaux}{mm}
% {\IfNoValueTF{#2}{#1} {#1\,\delimsize|\,\mathopen{}#2}}%{#1\:;\:#2}

% Swap the definition of \abs* and \norm*, so that \abs
% and \norm resizes the size of the brackets, and the
% starred version does not.
% \makeatletter
% \let\oldabs\abs
% \def\abs{\@ifstar{\oldabs}{\oldabs*}}
% %
% \let\oldnorm\norm
% \def\norm{\@ifstar{\oldnorm}{\oldnorm*}}
% %
% \let\oldset\set
% \def\set{\@ifstar{\oldset}{\oldset*}}
% \makeatother

\newcommand{\vle}{\rotatebox[origin=c]{90}{$\ge$}}

\newcommand{\indep}{\rotatebox[origin=c]{90}{$\models$}}

% \newcommand{\defeq}{\vcentcolon{=}}
% \newcommand{\eqdef}{=\vcentcolon{}}
\newcommand{\defeq}{\mathrel{\mathop{:}}=}
\newcommand{\eqdef}{\mathrel{\mathop{=}}:}
\newcommand{\defeqdef}{\vcentcolon{=}\vcentcolon}
\newcommand{\parallelsum}{\mathbin{\!/\mkern-5mu/\!}}
\newcommand{\supp}{\textnormal{supp}}

%% Emojis
%%

%% Footnote w/o tag
%%
\newcommand\blfootnote[1]{%
  \begingroup
  \renewcommand\thefootnote{}\footnote{#1}%
  \addtocounter{footnote}{-1}%
  \endgroup
}

%% Excerpted from braket.sty
%%
\def\bra#1{\mathinner{\langle{#1}|}}
\def\ket#1{\mathinner{|{#1}\rangle}}
\def\braket#1{\mathinner{\langle{#1}\rangle}}
\def\Bra#1{\left\langle#1\right|}
\def\Ket#1{\left|#1\right\rangle}

%% For consistent typography
% \newcommand\keyword{\textsf}

\newcommand{\revision}[1]{\textcolor{blue}{#1}}

% Command \coloredmidrule consists of 3 rules (top colour white, middle colour black, bottom colour tablerowcolor)
\colorlet{tablerowcolor}{lightblue}
\newcommand{\coloredmidrule}{\arrayrulecolor{white}\specialrule{\aboverulesep}{0pt}{0pt}%
            \arrayrulecolor{black}\specialrule{\lightrulewidth}{0pt}{0pt}%
            \arrayrulecolor{tablerowcolor}\specialrule{\belowrulesep}{0pt}{0pt}%
            \arrayrulecolor{black}}
\newcommand{\coloredmidrulecw}{\arrayrulecolor{tablerowcolor}\specialrule{\aboverulesep}{0pt}{0pt}%
            \arrayrulecolor{black}\specialrule{\heavyrulewidth}{0pt}{0pt}%
            \arrayrulecolor{white}\specialrule{\belowrulesep}{0pt}{0pt}%
            \arrayrulecolor{black}}
% Command \coloredbottomline consists of 2 rules (top colour
\newcommand{\coloredbottomrule}{\arrayrulecolor{tablerowcolor}\specialrule{\aboverulesep}{0pt}{0pt}%
            \arrayrulecolor{black}\specialrule{\heavyrulewidth}{0pt}{\belowbottomsep}}%
\title{MathTutorBench: A Benchmark for Measuring Open-ended\\ Pedagogical Capabilities of LLM Tutors}




\author{
    Jakub Macina$^{1, 2}$ \quad
    Nico Daheim$^{1, 3}$ \quad
    Ido Hakimi$^{1, 2}$ \quad \\
    \textbf{
    Manu Kapur$^4$ \quad
    Iryna Gurevych$^3$ \quad
    Mrinmaya Sachan$^{1}$
    } \\ \text{} \\
  $^{1}$Department of Computer Science, ETH Zurich \quad
  $^2$ETH AI Center \\
  $^{3}$Ubiquitous Knowledge Processing Lab (UKP Lab), Department of Computer Science\\ and Hessian Center for AI (hessian.AI), TU Darmstadt \\
  $^4$Professorship for Learning Sciences and Higher Education, ETH Zurich \\
  \texttt{jakub.macina@ai.ethz.ch}
}

\begin{document}
\maketitle
\begin{abstract}
Evaluating the pedagogical capabilities of AI-based tutoring models is critical for making guided progress in the field. Yet, we lack a reliable, easy-to-use, and simple-to-run evaluation that reflects the pedagogical abilities of models.
To fill this gap, we present \mathtutorbench, an open-source benchmark for holistic tutoring model evaluation. \mathtutorbench\ contains a collection of datasets and metrics that broadly cover tutor abilities as defined by learning sciences research in dialog-based teaching. To score the pedagogical quality of open-ended teacher responses, we train a reward model and show it can discriminate expert from novice teacher responses with high accuracy.
We evaluate a wide set of closed- and open-weight models on \mathtutorbench\ and find that subject expertise, indicated by solving ability, does not immediately translate to good teaching.
Rather, pedagogy and subject expertise appear to form a trade-off that is navigated by the degree of tutoring specialization of the model.
Furthermore, tutoring appears to become more challenging in longer dialogs, where simpler questioning strategies begin to fail.
We release the benchmark, code, and leaderboard openly
to enable rapid benchmarking of future models. 




\end{abstract}

\hspace{0em}\includegraphics[width=0.9em,height=0.9em]{figures/github.png}\hspace{.25em}\parbox{\dimexpr\linewidth-2\fboxsep-2\fboxrule}{\href{https://github.com/eth-lre/mathtutorbench}{github.com/eth-lre/mathtutorbench}}


\section{Introduction}%

Decision-making is at the heart of artificial intelligence systems, enabling agents to navigate complex environments, achieve goals, and adapt to changing conditions. Traditional decision-making frameworks often rely on associations or statistical correlations between variables, which can lead to suboptimal outcomes when the underlying causal relationships are ignored \citep{pearl2009causal}. 
The rise of causal inference as a field has provided powerful frameworks and tools to address these challenges, such as structural causal models and potential outcomes frameworks \citep{rubin1978bayesian,pearl2000causality}. 
Unlike traditional methods, \textit{causal decision-making} focuses on identifying and leveraging cause-effect relationships, allowing agents to reason about the consequences of their actions, predict counterfactual scenarios, and optimize decisions in a principled way \citep{spirtes2000causation}. In recent years, numerous decision-making methods based on causal reasoning have been developed, finding applications in diverse fields such as recommender systems \citep{zhou2017large}, clinical trials \citep{durand2018contextual}, finance \citep{bai2024review}, and ride-sharing platforms \citep{wan2021pattern}. Despite these advancements, a fundamental question persists: 

\begin{center}
    \textit{When and why do we need causal modeling in decision-making?}
\end{center} 

% Numerous decision-making methods based on causal reasoning have been developed recently with wide applications 
% %Decision makings based on causal reasoning have been widely applied 
% in a variety of fields, including 
% recommender systems \citep{zhou2017large}, clinical trials \citep{durand2018contextual}, 
% finance \citep{bai2024review}, 
% ride-sharing platforms \citep{wan2021pattern}, and so on. 


 

% At the intersection of these fields, causal decision-making seeks to answer critical questions: How can agents make decisions when causal knowledge is incomplete? How do we integrate learning and reasoning about causality into real-world decision-making systems? What role do interventions, counterfactuals, and observational data play in guiding decisions? 

% Our review is structured as follows: 
 

This question is closely tied to the concept of counterfactual thinking—reasoning about what might have happened under alternative decisions or actions. Counterfactual analysis is crucial in domains where the outcomes of unchosen decisions are challenging, if not impossible, to observe. For instance, a business leader selecting one marketing strategy over another may never fully know the outcome of the unselected option \citep{rubin1974estimating, pearl2009causal}. Similarly, in econometrics, epidemiology, psychology, and social sciences, \textit{the inability to observe counterfactuals directly often necessitates causal approaches} \citep{morgan2015counterfactuals, imbens2015causal}. 
Conversely, non-causal analysis may suffice in scenarios where alternative outcomes are readily determinable. For example, a personal investor's actions may have negligible impact on stock market dynamics, enabling potential outcomes of alternate investment decisions to be inferred from existing stock price time series \citep{angrist2008mostly}. However, even in cases where counterfactual outcomes are theoretically calculable—such as in environments with known models like AlphaGo—exhaustively computing all possible outcomes is computationally infeasible \citep{silver2017mastering, silver2018general}. 
In such scenarios, causal modeling remains advantageous by offering \textit{structured ways to infer outcomes efficiently and make robust decisions}. 


%This perspective not only enhances the interpretability of decisions but also provides a principled framework for addressing uncertainty, guiding actions, and improving performance across a broad range of applications.

% Data-driven decision-making exists before the causal revolution. \textit{So when and why do we need causal modelling in decision-making?} 
% This is closely related to the presence of counterfactuals in many applications. 
% The counterfactual thinking involves considering what would have happened in an alternate scenario where a different decision or action was taken. 
% In many fields, including econometrics, epidemiology, psychology, and social sciences, accessing outcomes from unchosen decisions is often challenging if not impossible. 
% For example, a business leader who selects one marketing strategy over another may never know the outcome of the unselected option. 
% Conversely, non-causal analysis may be adequate in situations where potential outcomes of alternate actions are more readily determinable: for example, the investment of a personal investor may have minimal impact on the market, therefore her counterfactual investment decision's outcomes can still be calculated with the data of stock price time series. 
% However, it is important to note that even when counterfactuals are theoretically calculable, as in environments with known models like AlphaGo, computing all possible outcomes may not be feasible. 
% In such scenarios, a causal perspective  remains beneficial. 


 

% 1. significance of decision making
% 2. role of causal in decision making
% 3. refer to the https://jair.org/index.php/jair/article/view/13428/26917

% Decision makings based on causal reasoning have been widely applied in a variety of fields, including recommender systems \citep{zhou2017large}, clinical trials \citep{durand2018contextual}, 
% business economics scenarios \citep{shen2015portfolio}, 
% ride-sharing platforms \citep{wan2021pattern}, and so on. 
% However, most existing works primarily assume either sophisticated prior knowledge or strong causal models to conduct follow-up decision-making. To make effective and trustworthy decisions, it is critical to have a thorough understanding of the causal connections between actions, environments, and outcomes.

\begin{figure}[!t]
    \centering
    \includegraphics[width = .75\linewidth]{Figure/3Steps_V2.png}
    \caption{Workflow of the \acrlong{CDM}. $f_1$, $f_2$, and $f_3$ represent the impact sizes of the directed edges. Variables enclosed in solid circles are observed, while those in dashed circles are actionable.}\label{fig:cdm}
\end{figure}


Most existing works primarily assume either sophisticated prior knowledge or strong causal models to conduct follow-up decision-making. To make effective and trustworthy decisions, it is critical to have a thorough understanding of the causal relationships among actions, environments, and outcomes. This review synthesizes the current state of research in \acrfull{CDM}, providing an overview of foundational concepts, recent advancements, and practical applications. Specifically, this work discusses the connections of \textbf{three primary components of decision-making} through a causal lens: 1) discovering causal relationships through \textit{\acrfull{CSL}}, 2) understanding the impacts of these relationships through \textit{\acrfull{CEL}}, and 3) applying the knowledge gained from the first two aspects to decision making via \textit{\acrfull{CPL}}. 

Let $\boldsymbol{S}$ denote the state of the environment, which includes all relevant feature information about the environment the decision-makers interact with, $A$ the action taken, $\pi$ the action policy that determines which action to take, and $R$ the reward observed after taking action $A$. As illustrated in Figure \ref{fig:cdm}, \acrshort{CDM} typically begins with \acrshort{CSL}, which aims to uncover the unknown causal relationships among various variables of interest. Once the causal structure is established, \acrshort{CEL} is used to assess the impact of a specific action on the outcome rewards. To further explore more complex action policies and refine decision-making strategies, \acrshort{CPL} is employed to evaluate a given policy or identify an optimal policy. In practice, it is also common to move directly from \acrshort{CSL} to \acrshort{CPL} without conducting \acrshort{CEL}. Furthermore, \acrshort{CPL} has the potential to improve both \acrshort{CEL} and \acrshort{CSL} by facilitating the development of more effective experimental designs \citep{zhu2019causal,simchi2023multi} or adaptively refining causal structures \citep{sauter2024core}. %However, these are beyond the scope of this paper.

\begin{figure}[!t]
    \centering
    \includegraphics[width = .9\linewidth]{Figure/Table_of_Six_Scenarios_S.png}
    \caption{Common data dependence structures (paradigms) in \acrshort{CDM}. Detailed notations and explanations can be found in Section \ref{sec:paradigms}.}
    \label{Fig:paradigms}
\end{figure}
Building on this framework, decision-making problems discussed in the literature can be further categorized into \textbf{six paradigms}, as summarized in Figure \ref{Fig:paradigms}. These paradigms summarize the common assumptions about data dependencies frequently employed in practice. Paradigms 1-3 describe the data structures in offline learning settings, where data is collected according to an unknown and fixed behavior policy. In contrast, paradigms 4-6 capture the online learning settings, where policies dynamically adapt to newly collected data, enabling continuous policy improvement. These paradigms also reflect different assumptions about state dependencies. The simplest cases, paradigms 1 and 4, assume that all observations are independent, implying no long-term effects of actions on future observations. To account for sequental dependencies, the \acrfull{MDP} framework, summarized in paradigms 2 and 5, assumes Markovian state transition. Specifically, it assumes that given the current state-action pair $(S_t, A_t)$, the next state $S_{t+1}$ and reward $R_t$ are independent of all prior states $\{S_j\}_{j < t}$ and actions $\{A_j\}_{j < t}$. When such independence assumptions do not hold, paradigms 3 and 6 account for scenarios where all historical observations may impact state transitions and rewards. This includes but not limited to researches on \acrfull{POMDP} \citep{hausknecht2015deep, littman2009tutorial}, panel data analysis \citep{hsiao2007panel,hsiao2022analysis}, \acrfull{DTR} with finite stages \citep{chakraborty2014dynamic, chakraborty2013statistical}. 

Each \acrshort{CDM} task has been studied under different paradigms, with \acrshort{CSL} extensively explored within paradigm 1. \acrshort{CEL} and offline \acrshort{CPL} encompass paradigms 1-3, while online \acrshort{CPL} spans paradigms 4-6. By organizing the discussion around these three tasks and six paradigms, this review aims to provide a cohesive framework for understanding the field of \acrlong{CDM} across diverse tasks and data structures.

%Recognizing the importance of long-term effects in decision-making

%Further discussions on these paradigms and their connections to various causal decision-making problems are provided in Section \ref{sec:paradigms}.


\textbf{Contribution.} In this paper, we conduct a comprehensive survey of \acrshort{CDM}. 
Our contributions are as follows. 
\begin{itemize}
    \item We for the first time organize the related causal decision-making areas into three tasks and six paradigms, connecting previously disconnected areas (including economics, statistics, machine learning, and reinforcement learning) using a consistent language. For each paradigm and task, we provide a few taxonomies to establish a unified view of the recent literature.
    \item We provide a comprehensive overview of \acrshort{CDM}, covering all three major tasks and six classic problem structures, addressing gaps in existing reviews that either focus narrowly on specific tasks or paradigms or overlook the connection between decision-making and causality (detailed in Section \ref{sec::related_work}).
    %\item We outline three key challenges that emerge when utilizing CDM in practice. Moreover, we delve into a comprehensive discussion on the recent advancements and progress made in addressing these challenges. We also suggest six future directions for these problems.
    \item We provide real-world examples to illustrate the critical role of causality in decision-making and to reveal how \acrshort{CSL}, \acrshort{CEL} and \acrshort{CPL} are inherently interconnected in daily applications, often without explicit recognition.
    \item We are actively maintaining and expanding a GitHub repository and online book, providing detailed explanations of key methods reviewed in this paper, along with a code package and demos to support their implementation, with URL: \url{https://causaldm.github.io/Causal-Decision-Making}.
\end{itemize}
% Our review is structured as follows: 


%%%%%%%%%%%%%%%%%%%%%%%%%%%%%%%%%%
%  causal helps over "Correlational analysis"
%Correlational analysis, though widely used in various fields, has inherent limitations, particularly when it comes to decision-making. While it identifies relationships between variables, it fails to establish causality, often leading to misinterpretations and misguided decisions. For example, the positive correlation between ice cream sales and drowning incidents is a classic example of how correlational data can be misleading, as both are influenced by a third factor, temperature, rather than causing each other. Such spurious correlations, due to oversight of confounding variables, underscore the necessity of causal modeling in decision making. Causal models excel where correlational analysis falls short, offering predictive power and a deeper understanding of underlying mechanisms. They enable us to predict the outcomes of interventions, even under untested conditions, and provide insights into the processes leading to these outcomes, thereby informing more effective strategies. Moreover, causal models are good at generalizing findings across different contexts, a capability often limited in purely correlational studies. 

%  causal helps in causal RL 
%From another complementary angle, although causal concepts have traditionally not been explicitly incorporated in fields like online bandits \citep{lattimore2020bandit} and \acrfull{RL} \citep{sutton2018reinforcement}, much of the literature in these areas implicitly relies on basic assumptions outlined in Section \ref{sec:prelim_assump} to utilize observed data in place of potential outcomes in their analyses, and there is also a growing recognition of the significance of the causal perspective \citep{lattimore2016causal, zeng2023survey} in these areas. 
% \textbf{Read causal RL survey and summarize. } However, by integrating causal concepts and leverging existing methodologies, we open up possibilities for developing more robust models to remove spurious correlation and selection bias \citep{xu2023instrumental, forney2017counterfactual}, designing more sample-efficient \citep{sontakke2021causal, seitzer2021causal} and robust \citep{dimakopoulou2019balanced, ye2023doubly} algorithms, and improving the generalizability \citep{zhang2017transfer, eghbal2021learning}, explanability \citep{foerster2018counterfactual, herlau2022reinforcement}, and fairness \citep{zhang2018fairness,huang2022achieving,balakrishnan2022scales} of these methods. %, and safety \cite{hart2020counterfactual}

%


%\subsection{Paper Structure}
The remainder of this paper is organized as follows: Section \ref{sec::related_work} provides an overview of related survey papers. Section \ref{sec:preliminary} introduces the foundational concepts, assumptions, and notations that form the foundation for the subsequent discussions. In Section \ref{sec:3task6paradigm}, we offer a detailed introduction to the three key tasks and six learning paradigms in \acrshort{CDM}. Sections \ref{Sec:CSL} through \ref{sec:Online CPL} form the core of the paper, with each section dedicated to a specific topic within \acrshort{CDM}: \acrshort{CSL}, \acrshort{CEL}, Offline \acrshort{CPL}, and Online \acrshort{CPL}, respectively. Section \ref{sec:assump_violated} then explores extensions needed when standard causal assumptions are violated. To illustrate the practical application of the \acrshort{CDM} framework, Section \ref{sec:real_data} presents two real-world case studies. Finally, Section \ref{sec:conclusion} concludes the paper with a summary of our contributions and a discussion of additional research directions that are actively being explored.


\begin{figure*}[t!]%
	\centering
	\resizebox{.99\textwidth}{!}{\includegraphics{figures/figure1_v4.pdf}}
	\caption{Overview of the \textit{MathTutorBench} benchmark. Each benchmark task defines a dataset, system prompts with problem and dialog, metric, and ground-truth teacher responses. A reward model is used to score the pedagogical quality over teacher responses (win rate). The right part of the figure shows the outcome as a performance comparison of selected LLMs. While they all perform well in a simple problem-solving setting, most of them lack in correct detection of mistakes and generating pedagogical responses. 
    }\label{fig:overview}
\end{figure*}

\section{Related Work}

\subsection{LLM-Based Dialog Tutoring}
A good tutor should scaffold student learning in a structured way rather than just provide correct answers. 
Current approaches to dialog tutoring using LLMs try to achieve this by differnent means: prompt-based elicitation of pedagogical behavior, finetuning models on pedagogical conversations, and alignment with pedagogical preferences. 

First, most existing works use LLMs with a carefully chosen prompt which enumerates desired pedagogical behavior. Bridge~\cite{bridge24} analyzes the teacher behavior and proposes to structure the prompt into a sequence of decisions, similar to real teachers, first to infer the error type and then to determine the pedagogical intent. Other works mostly directly write extensive prompts~\cite{class2023,kargupta-etal-2024-instruct}. However, defining such prompts is tedious, sensitive to small changes and difficult to test~\cite{jurenka2024towards}.

Second, several approaches finetune models on real or mostly synthetically generated data.
SocraticLM~\cite{socraticlm2024} uses a GPT-4 judge to evaluate the quality of teacher guidance using correctness and Socratic principles. A similar approach is to role-model teacher and student conversations based on textbook data~\cite{tutorchat24,book2dial2024}. MathDial~\cite{mathdial2023} is one of the few works that use teachers' utterances when interacting with students to finetune models. However, it is expensive to collect such data on a larger scale. Therefore, LearnLM~\cite{team2024learnlm}
uses an empirically validated mixture of synthetic and teacher-created datasets. However, for capturing high-quality tutoring, teacher-created data is essential and therefore upweighted in their final data mix.
Finally, LLMs can be aligned for pedagogical preferences during post-training~\cite{team2024learnlm}, because these are usually tacit. However, no datasets are openly available or they rely on larger models such as GPT-4 as a judge which limits its generalizability.

Our benchmark contributes an important missing ingredient in the development of LLM-based tutors -- the ability to quickly evaluate and compare models on key pedagogical aspects. 



\subsection{Automatic \& Human Evaluation}
Several works rely on automatic NLG metrics %
such as BLEU~\cite{papineni2002bleu} or BERTScore~\cite{Zhang2020BERTScore} for evaluation which require human-annotated ground truths. However, since tutoring has the goal of helping students learn %
~\cite{opportunities2023}, it is %
very open-ended and there is no single best pedagogical approach at each turn~\cite{jurenka2024towards}. This results in noisy and unreliable scores from automatic metrics~\cite{beasharedtask2023}. There exists an educational-specific classifier of active teacher listening~\cite{uptake2021}, however, it is limited to only this one dimension of teaching and does not account for the entire dialog history.
Therefore, recent finetuned tutoring LLMs~\cite{tutorchat24,socraticlm2024} rely on GPT-4-as-a-judge on dimensions like helpfulness and presentation. Some works on reasoning~\cite{mathchat2024} also focus on multi-turn model abilities judged by GPT-4 but they lack an educational focus. %


Pedagogical quality annotation requires hiring teachers, but it is time-consuming and hard to compare across trials. Two papers recently addressed the issue~\cite{tack2022ai,mrbench2024} by providing evaluation taxonomies but only present one-off static snapshots of current models' performance without the possibility to automatically evaluate new models. 
Finally, measuring student learning gains directly focuses on the end goal. However, learner studies are costly, time-consuming~\cite{schmucker2024ruffle}, and with strict ethics and privacy requirements. There is a growing interest in designing proper evaluation guidelines~\cite{beasharedtask2023,jurenka2024towards}, however, there is still a need for a unified automatic evaluation for scalable model development. 

\mathtutorbench\ addresses the limitations of existing automatic metrics by focusing specifically on tutoring, is simple to run, replicable, and could serve as a proxy for deciding which models to focus on in human studies. Moreover, our benchmark only uses data collected from real teachers.




\begin{table*}[h]
    \centering
    \resizebox{1.\linewidth}{!}{
        \begin{tabular}{l|cc|ccc|cc}
            & \multicolumn{2}{c}{\textbf{Math Expertise}} & \multicolumn{3}{c}{\textbf{Student Understanding}} & \multicolumn{2}{c}{\textbf{Teacher Response Generation}} \\
            \hline
             \textbf{Task(s)} & Problem  & Socratic  & Solution  & Mistake  & Mistake  & \multicolumn{1}{c}{Scaffolding Gen.,} & \multicolumn{1}{c}{Scaffolding Gen.,} \\
             & solving & questioning &  correctness &  location &  correction & \multicolumn{1}{c}{Ped. instr. following} & \multicolumn{1}{c}{Ped. instr. following}  \\
            \hline
            \textbf{Dataset} & GSM8k & GSM8k & StepVerify &  StepVerify & StepVerify & MathDialBridge & MathDialBridge[hard]  \\
             \textbf{Input} & $\vp$ & $\vp$ & $\vp$, $\widehat{\vs}$ & $\vp$, $\widehat{\vs}$ & $\vp$, $\mathcal{H}$ & $\vp$, $\mathcal{H}$ & $\vp$, $\mathcal{H}$  \\
            \textbf{Type} & generation & generation & bin. clas. (bal.) &  multi-cl. & generation & generation & generation  \\
         \textbf{Ground Truth} & $\va$ & $\vq_1,\dots,\vq_N$ & $\mathbbm{1}(e\neq 0)$ & $e$ & $\va$ & $\vu_{teacher}$ & $\vu_{teacher}$   \\
        \textbf{Instances} & 1319 & 1319 & 2004 &  2004 & 1002 & 1150 & 327   \\
        \textbf{Avg. turns} & - & - & - &  - & 3.04 & 3.08 & 5.78   \\
            \hline
        \end{tabular}
    }
    \caption{Datasets used in the benchmark and their statistics. Notation defined in Section~\ref{sec:background}. }\label{tab:datasets_in_benchmark}
\end{table*}


\section{Background}\label{sec:background}
\subsection{Next Teacher Utterance Generation}

We focus on educational dialogues between a student and a teacher, where a student is trying to solve a multi-step problem $\smash{\vp\in \mathcal{V}^\ast}$. %
The problem has a single numerical solution $\va$ and a sequence of solution steps $\smash{\vs=(\vs_1, \dots, \vs_N)}$, where each $\smash{\vs_n \in\mathcal{V}^\ast}$ and $\smash{\vs_N}$ contains the final answer
$\va$. 
A student solution consists of steps $\smash{\widehat{\vs}=(\widehat{\vs}_1, \dots, \widehat{\vs}_M)}$ and the first step with a mistake is $e \in \{0, 1,\dots, M\}$, where $e=0$ means no mistake. 

The goal of dialog tutoring is to continue an existing teacher-student dialog $\mathcal{H} \coloneqq (\vu_1,\dots,\vu_{T-1})$ consisting of $T-1$ turns $\vu_{t} \in \mathcal{V}^\ast$ with a new turn $\vu_{T} \in \mathcal{V}^\ast$ that simulates the teacher and guides the student towards solving a problem.
This is usually done by learning a model
$p_{\text{\vparam}}(\vu_T\mid \mathcal{H}, \mathcal{K}, \vi_t)$ with parameters $\vparam \in \real^d$, which is optionally conditioned on background knowledge $\mathcal{K}$ (in our case only the problem $\vp$) and a teacher intent $\vi_T$, and using a decoding strategy, such as greedy decoding or sampling according to $p_{\text{\vparam}}$ to generate an output.
The turn $\vu_T$ should then fulfill the desiderate laid out in Section~\ref{sec:ls_principles}.
The goal of this work is to present a benchmark to understand the quality of various $\vu_T$ generated by different models $\vparam$.

\subsection{Learning Sciences Principles}\label{sec:ls_principles}
We focus on 1:1 multi-turn teacher-student interactions where teachers promote active learning~\cite{freeman2014active} by engaging students through scaffolding nudges, hints, and Socratic questioning. Based on effective teaching research~\cite{LEPPER2002135,chi2014icap,nye2014autotutor,jurenka2024towards}, we define the following pedagogical principles: \textit{(a) correctness:} the teacher should guide the student towards the correct answer and not state incorrect facts; \textit{(b) scaffolding instead of giving away the answer:} the teacher should help the student to cognitively engage with the problem and discover the answer on their own; \textit{(c) encourage self-correction:} by correctly identifying the student mistake and first giving the student the opportunity to self-correct and learn from a mistake; \textit{(d) not overload student:} manage cognitive load by not giving too much information at once.




\section{MathTutorBench} %
We introduce \mathtutorbench, a benchmark that evaluates the tutoring capabilities of tutoring models. \mathtutorbench\ consists of three high-level skills that a good human teacher needs to have~\cite{bommasani2021opportunities}: \textit{Expertise}, \textit{Student Understanding} and \textit{Pedagogical Abilities}. These skills are tested by seven different tasks, each consisting of a dataset, prompt, and metric. All tasks in \mathtutorbench\ are related to math tutoring. The problems are mostly sourced from GSM8k~\cite{cobbe2021gsm8k}. 
Table~\ref{tab:datasets_in_benchmark} summarizes the datasets and tasks. The prompts used for the tasks in the benchmark are shown in Appendix~\ref{sec:appendix-task-prompts}.



\begin{table*}[t!]
    \centering
    \resizebox{1.\linewidth}{!}{
        \begin{tabular}{l|cc||ccc||cc|cc}
            & \multicolumn{2}{c}{\textbf{Math Expertise}} & \multicolumn{3}{c}{\textbf{Student Understanding}} & \multicolumn{4}{c}{\textbf{Pedagogy}} \\
            \hline
            Model & Problem  & Socratic  & Solution  & Mistake  & Mistake  & \multicolumn{4}{c}{Teacher response generation } \\
             & solving & questioning &  correctness &  location &  correction & scaff. & ped.IF & scaff. [hard] & ped.IF [hard] \\
            \hline
             Metric & accuracy & bleu & F1 &  micro F1 & accuracy & win rate & win rate  & win rate & win rate   \\
            \hline
            \hline
            LLaMA3.2-3B-Instruct  & 0.60  & 0.29 & 0.67 & 0.41  & 0.13  &  \textbf{0.64} & 0.63 & 0.45 & 0.40   \\
            LLaMA3.1-8B-Instruct  & 0.70 & 0.29 & 0.63 & 0.29  & 0.09  & 0.61 & 0.67 & 0.46 & 0.49  \\
            LLaMA3.1-70B-Instruct  & 0.91 & 0.29 & 0.71 & 0.56 & 0.19 & 0.63 & 0.70 & 0.49 & 0.49  \\
            GPT-4o  & 0.90 & \textbf{0.48} & 0.67 & 0.37 & \textbf{0.84} & 0.50 & \textbf{0.82} & 0.46 & \textbf{0.70} \\
            \hline
            LearnLM-1.5-Pro  & \textbf{0.94}  & 0.32 & \textbf{0.75} & \textbf{0.57} & 0.74 & \textbf{0.64} & 0.68 & \textbf{0.66} & 0.67  \\
            Llemma-7B-ScienceTutor  & 0.62  & 0.29 & 0.66  & 0.29  & 0.16 & 0.37 & 0.48 & 0.38 & 0.42  \\
            Qwen2.5-7B-SocraticLM  & 0.73  & 0.32 & 0.05  & 0.39  & 0.23  & 0.39 &  0.39 & 0.28 & 0.28  \\
            \hline
            Qwen2.5-Math-7B-Instruct  & 0.88  & 0.35 & 0.43  & 0.47  & 0.49 & 0.06 & 0.07 & 0.05 & 0.05 \\
            \hline
        \end{tabular}
    }
    \caption{We find that expertise and student understanding form a trade-off with pedagogy in tutor response generation. Models are grouped into general, specialized tutoring, and math reasoning models. The win rate is computed as the rate of the reward model preferring model responses over teacher responses. IF = Instruction Following.}
    \label{tab:benchmark_overall_results}
\end{table*}


\subsection{Tasks}
This section explains each task and complements Table~\ref{tab:datasets_in_benchmark} with the rationale for including it.

\noindent\textbf{1. Problem Solving.} We %
include a math word problem solving task that measures the accuracy of the final numeric answer generated with chain-of-thought~\cite{wei2022chain} compared to the answer $\va$. Even though this type of evaluation is popular, saturated, and contaminated, in \mathtutorbench\ it serves as an indicator of a balance between expertise and pedagogical abilities.

\noindent\textbf{2. Socratic Questioning.} Socratic questioning is related to the problem decomposition to smaller and more manageable parts.  
This task is to evaluate whether a model generates for each correct step $\vs_n$ at least one corresponding guidance question $\vq_n$ towards the correct answer, which could be posed to the student instead of simply providing the answer~\cite{shridhar2022automatic,socraticlm2024}. 

\noindent\textbf{3. Student Solution Correctness.}
This task evaluates a teacher's ability to verify the correctness of a student's answer. %
Framed as a balanced binary classification 
task based on student solution chain $\widehat{s}$, this dimension ensures that the model can objectively discern whether a student's reply is correct or incorrect, a crucial prerequisite for providing accurate feedback and identification of misconceptions~\cite{bridge24}.

\noindent\textbf{4. Student Mistake Location.}
Mistake location is a critical component of effective tutoring, focusing on a teacher's ability to accurately identify the exact location of the first mistake in a student's response $\widehat{\vs}$~\cite{verifiers2024}. This task assesses whether a tutoring model can pinpoint where a student's reasoning has gone wrong, enabling timely and precise feedback. By detecting steps with mistakes, the model can help students understand their misconceptions and steer the conversation to mitigate them, thus fostering a more productive learning experience~\cite{kapur2016examining,bridge24}.

\noindent\textbf{5. Student Mistake Correction.}
This task measures the performance of a model to generate a reasoning chain with a correct final numeric answer $\va$ even though the student proposes an incorrect answer in the dialog history $\mathcal{H}$. The conditioning on dialog history is the difference to Problem Solving. We test the models' ability to handle incorrect solutions. Models should not get derailed by students' incorrect steps. From a broader perspective, even if there is an incorrect step in a dialog history $\mathcal{H}$, this tests the recovery of a model from mistakes.    

\noindent\textbf{6. Scaffolding Generation (scaff.).}
The task is to generate the next teacher utterance $\vu_T$ as a continuation of the dialog. As it is an open-ended task, we use a reward model to score generations over teacher responses (explained in Section~\ref{sec:reward_model}) to estimate its' pedagogical quality. 
The tasks consist of two variations. \textit{Scaffolding generation} focuses on generating an immediate response to a student's incorrect solution.
We use a simple prompt for this version asking models to respond to a student as ``an experienced math teacher in a useful and caring way``~\cite{bridge24}. The second version is \textit{scaffolding generation [hard]}, a variant with a longer conversation history (avg. 5.78 turns).

\noindent\textbf{7. Pedagogical Instruction Following (IF) for Scaffolding Generation.} The task refers to the ability of the model to follow pedagogical instructions in prompts and steer the model generations to be more desired~\cite{team2024learnlm}. In this task, we use the LearnLM `extended` prompt~\cite{jurenka2024towards} which specifically enumerates desired behaviors 
such as ``nudging students'', ``asking guiding questions'', and ``not overwhelm student''. 
Therefore, in contrast to a simple prompt from \textit{scaffolding generation}, we hypothesize that models should improve their generations to be more aligned with our set of guiding principles from Section~\ref{sec:background}. The same is applied to the hard portion of the dataset.

\subsection{Datasets}
The requirements for the dataset included in the benchmark are to focus on middle school math content and contain 1:1 tutoring conversations written by human teachers. We found two datasets that fit the criteria, Bridge~\cite{bridge24} and MathDial~\cite{mathdial2023}. We excluded NCTE~\cite{demszky-hill-2023-ncte} dataset because it is multi-persona.
Bridge~\cite{bridge24} contains 700 snippets of real online tutoring conversations by novice teachers, where each response is revised by an expert teacher. MathDial~\cite{mathdial2023} consists of 2.9k tutoring conversations collected by human teachers who interacted with simulated students. Both datasets focus on math, Bridge uses various problem sources and MathDial sources problems from GSM8k~\cite{cobbe2021gsm8k}; a dataset of math word problems that we used in the expertise task. We combine Bridge and MathDial datasets into a combined dataset called \textit{MathDialBridge} which we further split into one with a maximum of 4 utterances and the rest we put into \textit{MathDialBridge[hard]}.
Finally, we use the StepVerify~\cite{verifiers2024} dataset which builds on top of the MathDial student incorrect solutions and introduces annotation of the first erroneous step in a student solution. Table~\ref{tab:datasets_in_benchmark} describes all the datasets and their statistics.










\subsection{Scaffolding Score}\label{sec:reward_model}
Evaluating pedagogical abilities in tutoring is inherently challenging due to the open-ended nature of the involved tasks. Unlike more structured domains like factual question answering, pedagogy requires assessing the quality of responses such as questioning guidance to the root cause of a mistake, and actionability of productive scaffolding. In other words, we need an efficient and lightweight mechanism, a critic model, that can assign a meaningful score to a generative model's output based on its pedagogical effectiveness.

\subsubsection{Criteria-based Scoring}
\begin{figure}[t]
    \centering
    \includegraphics[width=\columnwidth]{figures/model_reward_scores_comparison.pdf}
    \caption{Models performance on pairwise judgment of teacher responses. We compute accuracy on an independent test set 
    based on Bridge dataset~\cite{bridge24} 
    as a proportion of Expert teacher responses preferred over Novice teacher responses. 
    Extended prompt enumerates our pedagogical criteria (Figure ~\ref{fig:prompt-reward-model}).}
    \label{fig:model_reward_scores_comparison}
\end{figure}
The most straightforward approach is to train individual critic models for each pedagogical task using labeled data. 
For an evaluation taxonomy with $n$ total evaluation criteria,
for each criterion $i$ we train a binary classifier $C_i(\vy)$ that outputs a binary prediction of whether the criteria is present or not in response $\vy$. To combine these into a final score for a response, we aggregate them as $\sum_{i=1}^{n} C_i(\vy)$, which represents a discrete score of the total number of predicted desired criteria for the response. 
For example, MRBench~\cite{mrbench2024} is a small dataset annotated with 8 criteria such as the presence of guidance, actionability, and telling the answer. 
However, the scale of the required data and sparse features pose significant challenges.

\subsubsection{Pairwise Ranking of Teacher Responses}
Since labeled data for each criterion can be scarce, we here explore a more unified strategy. Instead of training a separate model for each criterion, where each annotation criterion has inherent subjectivity, we relax the objective and train a single critic model that aggregates multiple criteria into a pairwise comparison. 
We train a reward model using binary ranking loss by following~\citet{ouyang2022training}:
\begin{equation}
\mathcal{L}_{\text{rank}} = -\log \sigma\left( r_{\theta}(\vx, \vy_{c}) - r_{\theta}(\vx, \vy_{r}) - m \right)
\end{equation}
where $r_{\theta}(\vx, \vy)$ is the scalar score for prompt $\vx$ and generation $\vy$, $\vy_{c}$ and $\vy_{r}$ are preferred (chosen) and rejected generations respectively. The margin $m(\vy_c,\vy_r)$ represents the numerical quality difference between the chosen and rejected response but may also be set to $0$.


\subsubsection{Pairwise Preference Data Pipeline}

To create pairwise preference data, we follow our pedagogical criteria from Section~\ref{sec:ls_principles}. For example, a response is preferred if it is a Socratic question $\vq_t$ or it has dialog intent $\vi_t$ which probes student understanding.
Contrary, a response is chosen as dispreferred if it contains part(s) of the reference solution $\vs$ or has a lower number of desired criteria. 

To formalize this, for a given dialog history $\mathcal{H}$ and a taxonomy with $n$ criteria, we define a score for each response $\vy$:
\begin{equation}
f(\vy) = \sum_{i=1}^{n} \mathbbm{1}(\vy \text{ has desired criterion } i)
\end{equation}
where \( \mathbbm{1}(\cdot) \) is the indicator function that equals 1 if $\cdot$ holds and 0 otherwise. 
The condition within the indicator function is determined by: a) human criteria annotations (for MRBench~\cite{mrbench2024}), b) dialog intent annotations $\vi_t$ of the used pedagogical strategy (for MathDial~\cite{mathdial2023}), and c) subquestion annotation $\vq_t$ (for GSM8k~\cite{cobbe2021gsm8k}). For each pair of responses $(\vy_i, \vy_j)$, we construct a dataset of preference-label pairs $\mathcal{D} =\{(\vy_i, \vy_j) \mid f(\vy_i) > f(\vy_j)\}$, where the margin is defined as $m(\vy_i,\vy_j) = f(\vy_i) - f(\vy_j)$. The dataset captures the relative preference between responses based on the number of desired criteria they exhibit. The description of the datasets used for training and testing is found in Table~\ref{tab:rm_dataset_overview}.











\begin{table}[t]
    \centering
    \resizebox{\linewidth}{!}{\begin{tabular}{lcc}
         \textbf{Data Mix \& Setting} & \textbf{Accuracy} & \textbf{Avg. margin} \\
        \hline
        GSM8k inpaint (22k) & 0.60  & 3.26  \\
        \hline
        MathDial (3.6k) & 0.77  & 1.57  \\
        MRBench (4.5k) & 0.80 & 2.60 \\
        \;\; + margin in loss (4.5k) & 0.79  & \textbf{7.68} \\
        \;\; + pretrain (16.7k) & 0.80  & 3.09  \\
        \;\; + MathDial (8.1k) & \textbf{0.84}  & 5.75  \\
        \hline
    \end{tabular}}
    \caption{Ablation of \texttt{Qwen2.5-1.5B-Instruct} reward model. Total number of training instances in brackets. + indicates an addition to the model. Pretraining uses 20\% of Ultrafeedback~\cite{cui2024ultrafeedback}. We select the most accurate model to calculate the Scaffolding score. }
    \label{tab:rewards-abliation}
\end{table}

\begin{figure*}[t]
    \centering
    \begin{subfigure}{0.32\textwidth}
        \centering
        \includegraphics[width=\linewidth]{figures/Qwen2.5-1.5B_vanilla.pdf}
        \caption{Prompted}
        \label{fig:reward_vanilla}
    \end{subfigure}
    \hfill
    \begin{subfigure}{0.32\textwidth}
        \centering
        \includegraphics[width=\linewidth]{figures/Qwen2.5-1.5B_prompted.pdf}
        \caption{With Extended Prompt}
        \label{fig:reward_prompted}
    \end{subfigure}
    \hfill
    \begin{subfigure}{0.32\textwidth}
        \centering
        \includegraphics[width=\linewidth]{figures/Qwen2.5-1.5B_finetuned.pdf}
        \caption{Finetuned}
        \label{fig:reward_finetuned}
    \end{subfigure}
    \caption{Reward model distribution scores for expert and novice teachers across prompted (prompt in Figure~\ref{fig:simple-prompt-rm}), with extended prompt (prompt in Figure~\ref{fig:prompt-reward-model}), and finetuned Qwen2.5-1.5B-Instruct models.}
    \label{fig:reward_model_distributions}
\end{figure*}

\section{Experiments}


\subsection{Models}
\textit{MathTutorBench} includes an evaluation of three groups of models: general LLMs, LLM tutors, and math reasoners.
General LLMs such as open-weight \texttt{Llama3.1} 70B and 8B, newer \texttt{Llama3.2} 3B model, and closed source \texttt{gpt-4o-mini}. We use specialized tutoring models, namely closed-sourced \texttt{LearnLM-1.5-Pro} and recent open-source tutoring models \texttt{Qwen2.5-7B-SocraticLM}~\cite{socraticlm2024} and \texttt{Llemma-7B-32K-MathMix (ScienceTutor)}~\cite{tutorchat24}. 
To measure the importance of specially finetuned tutoring models, we evaluate the \texttt{Qwen2.5-Math-7B-Instruct}, which is optimized for math reasoning and was used for finetuning the specialized tutor model SocraticLM.



\subsection{Scaffolding Score - Test Set and Metrics}
The goal of the Scaffolding score is to estimate the pedagogical quality of the teacher response generation.
To validate it, we build a test set containing 482 examples based on Bridge~\cite{bridge24} which contains student dialogs with novice teachers. The test set has no instance or problem overlap with our training data. In Bridge, novice teacher responses are improved by expert teachers following an expert-defined decision-making process. The process first identifies the type of the error and then determines the pedagogical strategy and intent. For example, while novice teachers tend to explicitly correct student mistakes by giving away correct answers to students, expert teachers use various scaffolding nudges such as the Socratic method, use hints, or ask for further elaboration of the problematic part. 
We use the following formula to compute the accuracy of pairwise ranking between expert teacher and novice teacher:
\begin{equation}
    \frac{1}{N} \sum_{i=1}^{N} \mathbbm{1}(\vy_{\text{expert},i} > \vy_{\text{novice}, i}).
    \label{eq:reward_evaluation_score}
\end{equation}


\subsection{Scaffolding Score - Models and Baselines}
We use LLM-as-a-judge prompting as a baseline, similar to~\citet{jurenka2024towards}. For this, we use \texttt{Llama-3.1-70B-Instruct}, \texttt{GPT-4o-mini}, and the specialized judge model \texttt{Prompetheus-7b-v2.0}~\cite{kim2024prometheus}. Moreover, we pick well-performing existing preference-tuned reward models with high scores from the RewardBench~\cite{lambert2024rewardbench} on a variety of chat comparisons, namely, \texttt{Internlm2-7b-reward} and \texttt{Skywork-Reward-Llama-3.1-8B-v0.2}. 
To finetune single criteria-based binary classifiers we use \texttt{ModernBERT$_\text{base}$}~\citep{warner2024smarterbetterfasterlonger} with a classification head. Finally, we use \texttt{Qwen2.5-0.5B-Instruct} and \texttt{Qwen2.5-1.5B-Instruct} for finetuning on preference data, which are small enough to run fast as a part of the benchmark. 






\section{Results}
In this section, we showcase our core findings on \mathtutorbench\ and demonstrate the robustness and quality of the scaffolding reward model.

\subsection{Comparing SotA LLMs (Table~\ref{tab:benchmark_overall_results})}

\paragraph{Math expertise does not translate directly to student understanding and pedagogy.}
Our evaluations reveal a striking imbalance in current language models. While these models exhibit impressive domain knowledge and excel at Problem solving, as evidenced by their performance on datasets like GSM8K, they consistently fall short in Scaffolding generation task. This is particularly clear for Qwen2.5-Math and GPT4o. 


\paragraph{Specialized tutoring models improve in pedagogy but do not retain the full solving abilities.}
The specialized tutoring model SocraticLM achieves good Scaffolding scores for its size and big improvements over the base model (Qwen2.5-Math). However, it degrades in all Student 
Understanding tasks. Compared to SocraticLM, the ScienceTutor degrades in math expertise but has significantly better Student correctness solution and pedagogical instruction following. Closed-sourced LearnLM achieves a more reasonable balance across all skills and tasks.








\paragraph{Tutoring is more challenging on longer dialogs.} 
As indicated by the drop in performance in the win rate of tasks, indicated with `hard`, the longer the context it is more difficult for more to adapt. For example, it might be important to guide students differently than with a simple Socratic questioning. Only LearnLM can keep consistent performance.

\paragraph{Majority of models suffer by limited pedagogical instruction following.}
When we compare scaffolding generation with instruction following win rate (in base and hard splits), we notice that GPT4o follows the pedagogical instructions and gains a significant improvement (similarly, there is a smaller improvement for ScienceTutor). However, other models such as the SocraticLM, LearnLM, or Llama models show decreased or similar performance
suggesting a limited ability to follow pedagogical instructions defined in prompt.   







\subsection{Scaffolding Score - Results}


Figure~\ref{fig:model_reward_scores_comparison} shows a comparison between various models evaluated on the task of scoring expert teacher responses higher than novice teacher responses, see Equation~\ref{eq:reward_evaluation_score}. 
LLM-as-a-judge models are sensitive to prompts and positional bias, so we randomize the order. We report simple and extended prompts with detailing pedagogical guidelines (Figure~\ref{fig:simple-prompt-rm} and~\ref{fig:prompt-reward-model}) but their accuracy is lower than 0.7. 
Performance of reward models from RewardBench~\cite{lambert2024rewardbench} on the pedagogical preferences is only slightly higher than random.
We also train a combination of criteria-based ModernBERT binary classifiers aggregated into a summed final score, however, it lags behind extended-prompted LLM-as-a-judge models (for individual criterion performance see Table~\ref{tab:single-feature-results}). We hypothesize the single criterion data are highly sparse, noisy and imbalanced, and do not have sufficient data size to work. 




To summarize, Figure~\ref{fig:model_reward_scores_comparison} shows that finetuning reward models on pedagogical preference data is essential, as these finetuned reward models outperform both LLMs-as-a-judge models and SoTA reward models from RewardBench, consistent with~\cite{xu-etal-2024-promises}. We hypothesize that this is because of the lack of pedagogical datasets and a fundamental shift between a better chat response and a better pedagogical response.










\paragraph{Ablation of finetuning data.} Table~\ref{tab:rewards-abliation} shows the results for various data mixtures of pedagogical preference data. We see that synthetic inpainted data~\cite{inpainting2022} using stepwise questions and answers from GSM8k do not lead to a significant improvement over the base model. However, using pedagogical preference pairs based on human annotators scores~\cite{mrbench2024} improves the score to $0.8$, more than any other baseline in Table~\ref{fig:model_reward_scores_comparison}. However, as this dataset contains mostly model generations, only one of the responses is from a human teacher, and they are highly underrepresented. Therefore, we also include conversations from the  MathDial training set~\cite{mathdial2023}, which is filtered by desired dialog acts (more details in Table~\ref{tab:rm_dataset_overview}). The resulting finetuned model achieves the best accuracy of 0.84. As the test set is completely separate and no problems are shared between the train and test set, we pick this reward model as our final model for computing the Scaffolding score for model generation win rates over teacher responses (proportion of model generations preferred over teacher responses). 

\paragraph{Scores distribution.} Additionaly, we plot in Figure~\ref{fig:reward_model_distributions} the model distribution over scores on the test set. The prompted model with extended prompt and the vanilla model cannot separate the teacher and novice responses as well as the finetuned model. This supports the idea that pedagogical criteria are unique compared to general preference data and we need high-quality pedagogical preference data.

















\section{Conclusion}
In this work we propose \mathtutorbench, a holistic benchmark for quick and cost-effective assessment of the educational capabilities of LLM tutoring models.
It fills a crucial gap in the literature, as it allows fast prototyping of models
by using only lightweight automatic and learned metrics to evaluate pedagogy.
The goal is to not replace human studies measuring learning outcomes, but rather to serve as a measure of which models to use and compare. Finally, we benchmark various
models and report a trade-off between expertise, understanding, and pedagogy, as well as diminishing results on longer tutoring conversations.



\section*{Limitations}
Our work focuses on high school math tutoring and limits the insights of the benchmark to multi-step math problems. Despite a limited number of available conversation dataset in other domains, we plan to extend the benchmark to further STEM domains to generalize its applicability and reach. 

The conversational data in the benchmark does not contain conversations longer than 10 turns and thus can miss to evaluate very long educational conversations with long-term dependencies which might be present in online tutoring classes.

We study 1:1 conversational tutoring between teacher and student in this work. Specifically, we focus on a teacher using hints and nudges to aid student learning and provide engaging learning opportunities for students. However, there are additional functions of a teacher that we decided not to model, for example building rapport or trust with less engaged students. 

The benchmark does not contain all possible dimensions for educational evaluation. For example, it is missing a safety evaluation of potentially harmful tutor responses. It is an extensive research area and not the goal of this work. However, as the benchmark is open-source we plan to extend it to include more safety evaluations. 



\section*{Ethics Statement}
\paragraph{Intended usage} The goal of the benchmark is to evaluate new and existing dialog tutoring models on the skills related to math expertise, student understanding, and pedagogical capabilities. We will release the code and the dataset under CC-BY-4.0 license. This follows the licences of all the datasets which we are using in the benchmark.

\paragraph{Accessibility and Potential Misuse} The main goal of our work is to encourage the community to use the benchmark to improve existing tutoring models by balancing expertise, student understanding, and proper pedagogical guidance. However, there are potential risks related to the data and the scoring reward model. Models could optimize for reward hacking which could lead to suboptimal tutoring behaviour. Moreover, if the data contains some unknown pattern, the risk is that this could be exploited by new models to achieve higher scores. 
However, we tried to mitigate this by including several various data sources in the benchmark and in the training data, mostly human-annotated. We encourage the deployment of tutoring models in any case with appropriate safeguards.

\section*{Acknowledgements}
Jakub Macina acknowledges funding from the ETH AI Center Doctoral Fellowship, Asuera Stiftung, and the ETH Zurich Foundation. 
We thank Shehzaad Dhuliawala for valuable feedback and discussions.

\bibliography{custom}

\newpage
\appendix



\begin{table*}[h]
\centering
\resizebox{\textwidth}{!}{
\begin{tabular}{|l|l|l|l|>{\raggedright\arraybackslash}p{3cm}|>{\raggedright\arraybackslash}p{3cm}|>{\raggedright\arraybackslash}p{4cm}|}
\hline
\textbf{Dataset} & \textbf{Split} & \textbf{Pref. pairs} & \textbf{Avg. turns} & \textbf{Preferred resp.} & \textbf{Rejected resp.} & \textbf{Settings} \\ \hline
GSM8k-inpainted & all & 22,753 & 4.38 & Subquestion $q_t$  & Solution steps $s_{t:}$ & Math Word Problems with matching solutions steps $s_t$ to subquestions $q_t$ \\
\hline
\multicolumn{4}{l}{\textbf{Training datasets}} \\ 
\hline
MathDial   & train          & 3,615                & 2.93               & Teacher utterances with $\vi_t$ annotated as probing and focus in the first 3 teacher turns. & Reference sol. $s$ & Tutoring conversations created by human teachers interacting with LLM students \\ \hline
MRBench           & N/A            & 4,521                & 3.74               & Response with a higher number of desired criteria & Response with fewer desired criteria & Human annotation across 8 desired tutoring criteria - \textit{guidance, actionability, answer reveal, mistake identification, mistake location, coherence, tone, humanness} \\ \hline
\multicolumn{4}{l}{\textbf{Testing dataset}} \\
\hline
Bridge            & all            & 482                  & 2.79               & Expert teacher response            & Novice teacher response           & Original novice teacher responses and revisions by expert teachers  \\ \hline
\end{tabular}
}
\caption{Datasets used to create pedagogical pairwise preference data.  }
\label{tab:rm_dataset_overview}
\end{table*}


\section{Scaffolding scores qualitative examples}
Table~\ref{tab:rm-example-scores} shows assigned scores for various model and teacher responses given the problem and previous dialog. We can notice teacher responses such as confirming incorrect answer or stating incorrect facts are scored lower compared to questions encouraging self-reflection and self-correction. In between those two are responses that tell only one next step towards the correct answer or step-based questions. 
Similarly, Table~\ref{tab:quartile-scores-test-set} has examples of novice teacher responses from test set categorized into score quartiles. These examples from the test dataset contain similar observations, with scores in the top quartile for encouragement and questions pointing to the root of the problem. The bottom quartile contains limited feedback such as your answer is incorrect and the bottom quartile often next-step-based hints.

\begin{figure*}[h] 
    \centering
    \includegraphics[width=0.99\textwidth]{figures/prompts2_3.pdf}
    \caption{Prompts for each task. 
    }
    \label{fig:prompts}
\end{figure*}

\begin{table*}[h]
    \centering
    \small
    \renewcommand{\arraystretch}{1.2}
    \begin{tabular}{p{3cm} p{10cm} c}
        \toprule
        \multicolumn{3}{l}{\textbf{Problem}} \\
        \midrule
        \multicolumn{3}{p{14cm}}{Margaret wants to serve chicken salad sandwiches using mini croissants. Each committee member (24 people) will have 2 sandwiches. The bakery sells 12 mini croissants for \$8.00. How much will she spend on croissants?} \\
        \midrule
        \multicolumn{3}{l}{\textbf{Reference Solution}} \\
        \midrule
        \multicolumn{3}{p{14cm}}{
            \begin{itemize}
                \item Total sandwiches: \( 24 \times 2 = 48 \)
                \item Mini croissants needed: \( 48 \div 12 = 4 \) dozen
                \item Cost: \( 4 \times 8 = 32 \). \textbf{Final Answer:}  \$32.00 
            \end{itemize}
        } \\
        \midrule
        \multicolumn{3}{l}{\textbf{Dialogue}} \\
        \midrule
        \multicolumn{3}{l}{\textbf{Teacher:} \textit{"Walk me through your solution."}} \\
        \multicolumn{3}{p{15cm}}{\textbf{Student:} \textit{"I started by calculating how many sandwiches Margaret needs to make. Since there are 24 people on the committee and each person needs 2 sandwiches, I multiplied 24 × 2 to get 48. Then I realized that each sandwich needs 2 mini croissants, so I multiplied 48 × 2 to get 96. Since the bakery offers 12 mini croissants for \$8.00, I divided 96 by 12 to get 8 sets of 12 mini croissants. Finally, I multiplied 8 × \$8.00 to get my final answer of \$64.00."}} \\
        \midrule
        \textbf{Model} & \textbf{Generated Response} & \textbf{Score} \\
        \midrule
        LearnLM-simple & \textit{"That's a fantastic approach to solving the problem! Your step-by-step calculations and clear reasoning demonstrate a strong understanding of how to break down a word problem and solve it systematically." \textcolor{red}{confirms incorrect answer}} & -5.09 \\
        GPT4o-simple & \textit{"Great job on breaking down the problem step by step! Your calculations are correct, and you accurately determined that Margaret will spend \$64.00 on croissants."\textcolor{red}{confirms incorrect answer}} & -4.25 \\
        Qwen-Math-7b-extended & \textit{"It looks like you have a solid understanding of the problem. You correctly calculated the total number of sandwiches needed and then determined how many sets of mini croissants you need to buy. Your final answer is \$64.00." \textcolor{red}{incorrect}} & -3.68 \\
        Human Teacher & \textit{"A dozen croissants cost \$8.00 and she needs 4 dozens so that will cost 8*4 =" \textcolor{blue}{telling part of the answer}} & -1.91 \\
        Qwen-Math-7b-simple & \textit{"Calculate the total number of sandwiches needed: There are 24 people on the committee. Each person needs 2 sandwiches. \( 24 \times 2 = 48 \) sandwiches." \textcolor{blue}{telling part of the correct answer}} & -0.36 \\
        SocraticLM-simple & \textit{"How many sandwiches does Margaret need to make?" \textcolor{blue}{reiterate first step}} & 4.59 \\
        GPT4o-extended & \textit{"Great job breaking down the problem step by step! Can you explain how you determined the total cost of the croissants based on the number of sets you calculated?" \textcolor{blue}{ask to re-check}} & 4.62 \\
        LearnLM-extended & \textit{"That's a great start! Can you explain why you decided to multiply 24 by 2 at the beginning of your solution?" \textcolor{green}{encourages to reflect and self-correct on the exact error step}} & 7.47 \\
        SocraticLM-extended & \textit{"You correctly calculated the total number of sandwiches needed. Now, how many mini croissants does each sandwich require?" \textcolor{green}{encourages to reflect and self-correct on the exact error step}} & 7.66 \\
        \bottomrule
    \end{tabular}
    \caption{Example scaffolding reward model scores. Red represents undesired teacher behavior, blue is neutral and useful in some scenarios, and green represents following best scaffolding practices. Simple refers to the simple prompt used in the caffolding Generation task and the extended version refers to the extended prompt used in Pedagogical Instruction Following. }\label{tab:rm-example-scores}
\end{table*}


\section{Task Prompts}\label{sec:appendix-task-prompts}
The exact prompts used in the benchmark are shown in Figure~\ref{fig:prompts}. Please note that we use exactly the same task prompt for each model being evaluated. Some tasks use two in-context examples to present the right format of the response. The cost to run the full benchmark with GPT4o-mini is less than 3\$. To run the open-weight models we use the vllm library~\cite{kwon2023efficient}. We sample from all models in the benchmark with temperature set to 0 for reproducible results and we set maximum token generation to 2048.

\subsection{Details of Benchmarked Models}
Specific versions of closed models we use are \texttt{gpt-4o-mini-2024-07-18} version and \texttt{learnlm-1.5-pro-experimental}. We use these exact versions of open-weight models loaded from Huggingface model hub~\cite{wolf-etal-2020-transformers}: \texttt{LLaMA3.2-3B-Instruct}, \texttt{LLaMA3.1-8B-Instruct}, \texttt{Llama-3.1-70B-Instruct}, \texttt{CogBase-USTC/SocraticLM}, \texttt{princeton-nlp/Llemma-7B-32K-MathMix}, and \texttt{Qwen2.5-Math-7B-Instruct}.



\section{Reward Model Details}
\subsection{Training data}
Training data used for training the Scaffolding reward model and its ablation are in Table~\ref{tab:rm_dataset_overview}.




\subsection{Implementation details}
We finetune all models using Huggingface transformers library~\cite{wolf-etal-2020-transformers} and
using the checkpoints from the Huggingface Model Hub respecting corresponding license agreements.

We finetune all models with a learning rate of $1\cdot10^{-5}$ for 1 training epoch and with a batch size of 16. We use the AdamW optimizer~\citep{loshchilov2018decoupled}.

We used an NVIDIA A100 80GB GPU and finetuning takes around 1 hour for each model.

\begin{figure*}[h]
    \centering
    \small
    \begin{tcolorbox}
        Judge the pedagogical quality of the responses provided by two teachers. Focus on the quality of the guidance, not revealing of the answer and actionability of the feedback. Be as objective as possible. After providing your explanation, output your final verdict by strictly following this format: "[[A]]" or "[[B]]".\\
        Problem: \{problem\}\\
        Reference Solution: \{solution\} \\
        \{conversation\}\\
        \\
        \lbrack The Start of Response A\rbrack\\
        \{responseA\} \\
        \lbrack The End of Response A\rbrack\\
        \\
        \lbrack The Start of Response B\rbrack \\
        \{responseB\} \\
        \lbrack The End of Response B \rbrack\\
    \end{tcolorbox}
    \caption{
    A simple baseline prompt is used in LLM-as-a-judge and preference reward models. 
    \label{fig:simple-prompt-rm}}
\end{figure*}

\begin{figure*}[h]
    \centering
    \small
    \begin{tcolorbox}
        Judge the pedagogical quality of the responses provided by two teachers. Focus on the quality of the scaffolding guidance, correctness, and actionability of the feedback through nudges, questions, and hints. Do not give high scores for revealing the full answer.\\
        Problem: \{problem\}\\
        Reference Solution: \{solution\} \\
        \{conversation\}\\
        Teacher: \{utterance\_to\_score\}
    \end{tcolorbox}
    \caption{
    Extended prompt used by the reward models, LLM-as-a-judge, and preference-tuned reward models. \{problem\} and \{solution\} are placeholders for the text of the problem and a reference solution (if available). \{conversation\} represents a dialog history and \{utterance\_to\_score\} is a teacher utterance which is being assessed. For LLM-as-a-judge, two utterances are listed the same way as in Figure~\ref{fig:simple-prompt-rm}.
    \label{fig:prompt-reward-model}}
\end{figure*}




\section{Details on Single-Criteria Classifiers}
The results of individual criteria classifiers on the separate test set are shown in Figure~\ref{tab:single-feature-results}. For training of the single-criteria classifiers we binarize the data from MRBench~\cite{mrbench2024}. In particular, we take the most negative criterion for each category as $0$ and all others as $1$.
We train \texttt{ModernBERT$_\text{base}$} with 149M parameters on NVIDIA V100 GPUs.
Again, we use the AdamW optimizer with a learning rate of $1\cdot10^{-5}$ and a batch size of 16 but train for 3 epochs due to the small data sizes. Training takes only ca. 15 minutes.


\begin{table}[h]
    \centering
    \begin{tabular}{lc}
         \textbf{Model} & \textbf{Accuracy} \\ %
        \hline
        \hline
        Actionability & 0.78 \\
        Guidance & 0.44 \\
        Tone & 0.46 \\
        Mistake Identification & 0.61 \\
        Mistake Location & 0.63 \\
        Revealing & 0.39 \\
        Aggregated ens. & 0.66 \\
        Aggregated ens. (best 3) & 0.68 \\
        \hline
    \end{tabular}
    \caption{Results of the criteria-based binary classifiers on the test set. All models are finetuned ModernBERT$_\text{base}$ models, the last two rows represent ensembles (ens.) with aggregated discrete binary predictions. The criteria are a subset of criteria from  MRBench~\cite{mrbench2024}.}
    \label{tab:single-feature-results}
\end{table}

\begin{table*}[h!]
\centering
\renewcommand{\arraystretch}{1.5}
\begin{tabular}{|c|p{10cm}|}
\hline
\textbf{Quartile} & \textbf{Example} \\
\hline
\textbf{Top (75th)} & You made a good try. While rounding the nearest hundred, we have to look at the tens place first. Is the value in the tens place below 5? \\
\cline{2-2}
 & Your answer is a little bit off. There are 4 points in this graph. The x-axis moves on the graph horizontally or right to left. What direction does the y-axis move on the graph? \\
\cline{2-2}
 & That is great! +1 point for your effort. The division is the part of the question. What is the dividend? \\
\hline
\textbf{Mid (25-75th)} & Very good try! 1 day = \_\_\_ hours. \\
\cline{2-2}
 & That was a good try. Plus 1 point. Let me explain it to you. Here, we have to find the value of 10 divided by 5. \\
\cline{2-2}
 & You got an incorrect answer. Let me show you. The area of the top rectangle is 10. Add the areas of the two sections together. The final answer is 45 square feet. Did you understand? \\
\cline{2-2}
 & That's a good try. Multiplication is also called repeated addition. \\
\hline
\textbf{Bottom (25th)} & Your answer is incorrect. The volume is 70 cubic units. Does the step make sense? \\
\cline{2-2}
 & Incorrect answer [STUDENT], but good try. \\
\cline{2-2}
 & That was a good try. \\
\hline
\end{tabular}
\caption{Examples of reward model scores for novice teacher responses from the test set, categorized into quartiles.}
\label{tab:quartile-scores-test-set}
\end{table*}

\end{document}
