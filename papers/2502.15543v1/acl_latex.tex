% This must be in the first 5 lines to tell arXiv to use pdfLaTeX, which is strongly recommended.
\pdfoutput=1
% In particular, the hyperref package requires pdfLaTeX in order to break URLs across lines.

\documentclass[11pt]{article}

% Change "review" to "final" to generate the final (sometimes called camera-ready) version.
% Change to "preprint" to generate a non-anonymous version with page numbers.
% \usepackage[review]{acl}
\usepackage[table]{xcolor} 

\usepackage[final]{acl}

% Standard package includes
\usepackage{times}
\usepackage{latexsym}

% For proper rendering and hyphenation of words containing Latin characters (including in bib files)
\usepackage[T1]{fontenc}
% For Vietnamese characters
% \usepackage[T5]{fontenc}
% See https://www.latex-project.org/help/documentation/encguide.pdf for other character sets

% This assumes your files are encoded as UTF8
\usepackage[utf8]{inputenc}

% This is not strictly necessary, and may be commented out,
% but it will improve the layout of the manuscript,
% and will typically save some space.
\usepackage{microtype}

% This is also not strictly necessary, and may be commented out.
% However, it will improve the aesthetics of text in
% the typewriter font.
\usepackage{inconsolata}

%Including images in your LaTeX document requires adding
%additional package(s)
\usepackage{graphicx}


% Ours
\usepackage{amssymb}
\usepackage{amsfonts}
\usepackage{bm}       % 提供加粗数学符号的命令
\usepackage{amsmath}  % 引入amsmath包
\usepackage{enumitem}  % 列表管理
\usepackage{pifont}
\usepackage{adjustbox} % 对其bar

\usepackage{enumitem}
\usepackage{booktabs} % For professional looking tables
% \usepackage{float}

% \usepackage{ulem}
\usepackage{CJKutf8}
\usepackage{booktabs} 
\usepackage{multirow}
% \usepackage{dashrule}
\usepackage{tabularx}
\usepackage{makecell}
\usepackage{siunitx}

% framework
% \def\mylogo{\scalerel*{\includegraphics{figures/fire.png}}{X}}
\usepackage{tcolorbox}

% \usepackage{subcaption}
\usepackage{subfigure}

% \def\mylogo{\resizebox{0.45cm}{!}{\includegraphics{figures/fire.png}}} % 宽度为2cm,高度按比例缩放
% \def\mylogosnow{\resizebox{0.45cm}{!}{\includegraphics{figures/snow.png}}} % 宽度为2cm,高度按比例缩放
% \def\mylogonarrow{\resizebox{0.5cm}{!}{\includegraphics{figures/narrow.jpg}}} % 宽度为2cm,高度按比例缩放
\newlength{\hseg}
\setlength{\hseg}{0.6cm}

\usepackage{scalerel} % 用于 \scalerel 命令
% \usepackage{hyperref}
\definecolor{reasoning-side}{RGB}{167,210,165}
\definecolor{reasoning-inside}{RGB}{206,222,186}
\definecolor{lang-side}{RGB}{159,164,248}
\definecolor{lang-inside}{RGB}{217,217,252}
\usepackage{tikz}
\newcommand{\inlinecolorboxone}{
    \tikz[baseline=-0.1cm]{
        \node[draw=lang-side, fill=lang-inside, minimum width=0.4cm, minimum height=0.1cm, line width=0.5pt, rounded corners=1pt] (char1) {};
    }
}
\newcommand{\inlinecolorboxtwo}{
    \tikz[baseline=-0.1cm]{
        \node[draw=reasoning-side, fill=reasoning-inside, minimum width=0.4cm, minimum height=0.1cm, line width=0.5pt, rounded corners=1pt] (char2) {};
    }
}

%figures
\usepackage{pgfplots}
\definecolor{tiffanyblue}{RGB}{129,216,208}
\definecolor{bangdiblue}{RGB}{0,149,182}
\definecolor{kleinblue}{RGB}{0,47,167}
\definecolor{purple}{RGB}{138,43,226}
\usetikzlibrary{shapes}
\usepackage{circledtext}

\usetikzlibrary{arrows,decorations.pathmorphing,backgrounds,positioning,fit,petri}
\usetikzlibrary{arrows.meta,fit,shapes.arrows}
% \usepackage{caption} 
\usepackage{pgfplots}
\usepackage{tikz}
\usetikzlibrary{positioning,shapes,arrows,shadows,patterns}
\usetikzlibrary{calc}
% \usepackage{tkz-kiviat,pgfplots}
\pgfplotsset{compat=newest}




\usepackage{xspace}


\newcommand{\eg}{\emph{e.g.}}
\newcommand{\hpc}[1]{\textcolor{purple}{[hpc: #1]}}
\newcommand{\ysj}[1]{\textcolor{red}{[ysj: #1]}}

\def\method{PIP-KAG}
\def\dataset{CoConflictQA} % Knowledge conFliCt qa
\def\attrprompt{$\text{Attr}_{\text{prompt}}$}
\def\oiprompt{$\text{O\&I}_{\text{prompt}}$}
% ====================================
\usepackage[normalem]{ulem}


\title{\includegraphics[width=1.25em]{figs/pip.pdf}-KAG: Mitigating Knowledge Conflicts in Knowledge-Augmented Generation via Parametric Pruning}




\author{Pengcheng Huang$^{1}$, Zhenghao Liu$^{1}$\thanks{ \ \ indicates corresponding author.}, Yukun Yan$^{2}$, Xiaoyuan Yi$^{3}$, \\ 
\textbf{Hao Chen$^{2}$, Zhiyuan Liu$^{2}$, Maosong Sun$^{2}$, Tong Xiao$^{1}$, Ge Yu$^{1}$, Chenyan Xiong$^{4}$} \\
$^1$Department of Computer Science and Technology, Northeastern University, China \\
$^2$Department of Computer Science and Technology, Institute for AI, Tsinghua University, China \\
$^3$Microsoft Research Asia, Beijing, China \\
$^4$Language Technologies Institute, Carnegie Mellon University, United States
}


\begin{document}
\maketitle

\begin{abstract}

% Recent works to jointly reconstruct 3D human and object from a single RGB image, are mostly model-based, that fail to capture the fine details of the clothed human body and object surface. In this paper, we introduce ReCHOR, a novel, model-free, first-method to produce realistic clothed human-object reconstructions from a monocular view. This is extremely challenging due to human-object occlusions, diverse interactions and depth ambiguity, as it needs to infer both 3D spatial awareness and high resolution details. Our core idea is based on estimating neural implicit representations for human and object respectively by an attention-based neural implicit model that attends to pixel-aligned features from both the global human-object image for spatial awareness and  the local separate view of human and object images for high quality details. Additionally, the network is conditioned on semantic features from an initial estimated human-object pose prior and a generative diffusion model that inpaints occluded regions, thus enabling the retrieval of details from them.
% We also propose a synthetic dataset with rendered scenes of diverse, inter-occluded 3D human and object scans, to train our network. We evaluate our method on the synthetic and real world BEHAVE dataset. Our experiments show that our method outperforms the SOTA in achieving realistic clothed human-object reconstructions.
Recent approaches to jointly reconstruct 3D humans and objects from a single RGB image represent 3D shapes with template-based or coarse models, which fail to capture details of loose clothing on human bodies. In this paper, we introduce a novel implicit approach for jointly reconstructing realistic 3D clothed humans and objects from a monocular view. For the first time, we model both the human and the object with an implicit representation, allowing to capture more realistic details such as clothing. This task is extremely challenging due to human-object occlusions and the lack of 3D information in 2D images, often leading to poor detail reconstruction and depth ambiguity. To address these problems, we propose a novel attention-based neural implicit model that leverages image pixel alignment from both the input human-object image for a global understanding of the human-object scene and from local separate views of the human and object images to improve realism with, for example, clothing details. Additionally, the network is conditioned on semantic features derived from an estimated human-object pose prior, which provides 3D spatial information about the shared space of humans and objects. To handle human occlusion caused by objects, we use a generative diffusion model that inpaints the occluded regions, recovering otherwise lost details. For training and evaluation, we introduce a synthetic dataset featuring rendered scenes of inter-occluded 3D human scans and diverse objects. Extensive evaluation on both synthetic and real-world datasets demonstrates the superior quality of the proposed human-object reconstructions over competitive methods.
\end{abstract}
\section{Introduction}\label{sec:intro}

In computational finance, Monte Carlo simulations are used extensively to estimate the expected value of financial payoffs based on the solution of stochastic differential equations (SDEs) which model the evolution of stock prices, interest rates, exchange rates and other quantities \cite{glasserman04}.  Monte Carlo methods are very general and flexible, but for high accuracy it requires generating a large number of costly SDE path approximations, which has motivated research into a number of variance reduction or, equivalently, cost reduction techniques. One such method is
Multilevel Monte Carlo (MLMC), which was proposed in \cite{GILES2008} and was adapted for various applications that are summarised in \cite{Giles_overview17} and successfully combined with other methods such as quasi-Monte Carlo methods. The main idea of MLMC is to approximate the payoff using different time stepping resolutions when numerically solving the underlying SDE and to generate an optimal number of samples on each level, such that the overall computational cost is minimised subject to the desired bound on the variance. %, such that the total computational cost is minimised. 
The computational savings come from the fact that most samples are computed on the coarser levels and hence are less expensive while only a few samples from the finest levels are required \cite{GILES2008}.


Among the directions in which the computational cost 
of MLMC methods could further be reduced, an important avenue is the use of lower precision calculations, especially for the first Monte Carlo levels where the targeted accuracy is relatively low. 
 An overview of the research on mixed precision for the standard Monte Carlo (MC) framework is provided in \cite{ChowMixedPrecisionStandardMC} but only a few references study the potential of low precision computation in the MLMC framework \cite{Rounding_error_oliver}. To the best of our knowledge, the only MLMC framework with customised precision in the literature is \cite{brugger2014mixed}, but they use a uniform precision for all operations on each Monte Carlo level instead of optimising 
 the precision of each intermediary variable to reduce as much as possible the cost of path generation.
 
An important motivation for an MLMC framework with variable precision would be performing the low precision computations on reconfigurable hardware devices such as Field Programmable Gate Arrays (FPGAs). FPGAs contain customizable logic blocks and connectors that make it easy to adapt the digital circuit architecture for a specific application, leading to a highly parallel and optimised implementation. Therefore they are successfully exploited in applications that require high speed and have high computational workload, such as signal processing \cite{woods2008fpga}, and real time applications like high frequency trading \cite{HFT1,HFT2}. That is why a number of previous works in hardware architecture design implemented the MLMC algorithm to price financial options using FPGAs as accelerators, which resulted in improved speed and power efficiency compared to full CPU architectures \cite{Schryver2013AMM}. The paper \cite{lindsey2016domain} also proposed 
a Domain Specific Language to automate the configuration of FPGAs for this specific application. However, only \cite{brugger2014mixed} proposed a heuristic to reduce the precision in calculations.

In addition, all aforementioned works considered that the random number generation (RNG) is performed in single or double precision. Yet in most cases an important portion of the workload in the overall MLMC simulation comes from the RNG and in \cite{brugger2014mixed} this limited the total computational savings.
To reduce the cost of MLMC simulations in particular those based on the Geometric Brownian Motion (GBM), \cite{approximateICDF_Oliver, NestedOliver} have proposed to use approximate random numbers that are generated by applying an approximation of the inverse CDF to uniform random numbers. In \cite{NestedOliver}, the authors proposed a way to integrate these lower precision random variables into a \textit{nested} MLMC framework and completed a numerical analysis to bound the resulting error at each MC level by a product of the time step and the error in the random number approximation. The same authors show in \cite{approximateICDF_Oliver} that using approximate random variables reduces the cost of path generation by a factor 7.


In this paper we propose a nested MLMC framework that combines the use of approximate random normal variables and lower precision calculations to reduce the computational cost of MLMC even further than \cite{brugger2014mixed,NestedOliver}. We illustrate the efficiency of our framework in Matlab, after making several assumptions on the cost of operations and size of the errors that we carefully justify. We focus on the case of GBM and use the approximate RNG methods presented in \cite{approximateICDF_Oliver} as well as a new slightly modified method that combines CDF inversion and the central limit theorem. To choose the precision of the variables in the low precision path generation, we introduce a novel method to optimise the bit-widths. This optimisation is performed before the main path generation loop is executed and is based on a linear model of the payoff error  
due to rounding when computing in low precision. The error model relies on algorithmic differentiation in a similar manner to \cite{unifying-bwoptim,bitwidth-AD,ADAPT}. The bit-width optimisation procedure can be performed off-line, so this stage can be excluded from the on-line time complexity of our framework. The user specified desired accuracy is then enforced by calculating on-line the number of samples that need to be generated.

In terms of hardware design, we suggest implementing the low precision path generation on FPGAs and the full-precision ones on a CPU or GPU. 
The FPGA offers enough flexibility to define a separate bit-width for every variable in the low precision path generation, and can be reconfigured periodically to update the bit-widths when the market parameters have changed considerably. 


The paper is organized as follows : \Cref{sec:MLMC} introduces MLMC and nested MLMC to make clear the estimator that is implemented in our framework. Then in \Cref{sec:RNG} we detail the methods that could be used to obtain approximate random normally distributed numbers very cheaply for the low precision path generation. In \Cref{sec:error_model} and \Cref{sec:costModel} we propose an error model and a cost model (resp.) that we then use to formulate the optimisation problem that is solved to obtain the optimal bit-widths of fixed point variables in \Cref{sec:optimisation}. Finally we summarise our results and future directions in \Cref{sec:conclusion}.



% \section{VLM$^2$-Bench}
VLM$^2$-Bench is a benchmark designed to assess models' ability to visually link matching cues when processing multiple images or videos. This section introduces the three main categories of VLM$^2$-Bench—\textit{general cue} (\S\ref{gc}), \textit{object-centric cue} (\S\ref{oc}), and \textit{person-centric cue} (\S\ref{pc})—detailing their associated subtasks, data collection process, and QA pair construction.

\begin{figure*}[t]
  \centering
  \includegraphics[width=0.99\textwidth]{img/general-pipe.png} 
  % \vspace{-0.8cm}
  \caption{Construction of \textbf{GC}: (i) We start by manually verifying the edited image data based on three key criteria. (ii) A VLM is then prompted to generate captions for each image, followed by salient score-based filtering to retain the challenging cases. (iii) Finally, visual cues are extracted from two sources and incorporated into a QA prompt, guiding an LLM to generate both positive and negative answer pairs. 
  % \yi{"a event hall" is hard to read in the top right -- rule of thumb is need to make paper read-friendly for color blind ppl and when ppl print in black and white (otherwise reviewer might crib); turn that shade into darker shade of color please}
  }
  \label{fig:gc construction}
\end{figure*}

\subsection{General Cue (GC)}
\label{gc}
GC is designed to assess a model's ability to link matching cues across diverse contexts, encompassing a broad range of \textit{general cues}. Given two images containing both matched and mismatched cues, an ideal model should accurately identify mismatched ones and associate matched ones.

\paragraph{Subtasks.}
Here we introduce two subtasks: (i) \textbf{\textit{Matching (Mat)}} evaluates a model’s ability to link corresponding visual cues across two images to determine whether they match. Instead of merely identifying differences, the model must associate identical visual elements in both images to recognize what has remained the same and what has changed. 
(ii) \textbf{\textit{Tracking (Trk)}} focuses on a model’s ability to track a specific visual cue that appears in only one of the two images and determine how it has changed. Rather than simply detecting a difference, the model must link the cue across contexts to understand the transformation process. 



\paragraph{Data Collection.} 
We repurpose data from two image editing datasets~\citep{wei2024omniedit,ku2023imagenhub}, where each data sample includes an original image $I_{ori}$, an edited image with subtle modifications $I_{edit}$, and a corresponding edit instruction $\mathcal{P}$ describing the changes. Our data collection is carried out across two dimensions. First, to ensure diversity in the mismatched cues, GC encompasses various types of changes, such as instance-level modifications (e.g., add/remove, swap, attribute change), which focus on specific items, as well as environment-level changes.

\paragraph{QA Construction.}
We predefine a T/F question template for \textit{Mat} and \textit{Trk} with a placeholder for the candidate answer (refer to Appendix~\ref{appendix: more details on benchmark construction}). Figure~\ref{fig:gc construction} illustrates the construction process, which follows a three-stage approach. 

\textit{Manual Screening \& Refinement:} We ensure that $\mathcal{P}$ accurately reflects the changes (correctness), corresponds uniquely to the modified cues (uniqueness), and is unambiguous (clarity).


\textit{Salient Sampling:} Here, we automate the removal of overly simple cases (e.g., mismatched cues are too salient). To achieve this, a VLM first generates separate descriptions for \( I_{ori} \) and \( I_{edit} \), denoted as \( Cap_{ori} \) and \( Cap_{edit} \). These descriptions are then combined with \( \mathcal{P} \) into a single passage using a predefined template \(\mathcal{T}\) (see Table \ref{template salient score} for details). The probability assigned by a language model (e.g., Llama3-8B~\citep{dubey2024llama}) to \( \mathcal{P} \) given this text-based information is used to compute the salient score, formulated as:

\vspace{-13pt}
\begin{equation}
S_{\text{salient}} = \frac{1}{|\mathcal{P}|} \sum_{i=1}^{|\mathcal{P}|} \log P_{\theta}(p_i \mid C \cup p_{<i}),
\end{equation}

\noindent where \( \mathcal{P} = \{p_1, p_2, ..., p_{|\mathcal{P}|}\} \) represents the tokenized \(\mathcal{P}\), and \( C = \mathcal{T} (Cap_{ori}, Cap_{edit}) \) denotes the context filled with template \(\mathcal{T}\). Samples with scores below \( \theta \) (-2.0 here) are retained, ensuring that the benchmark includes more challenging examples requiring nuanced visual cue association. 
% \yi{what's your actual threshold used? which LLM exactly did you use to compute salient score and construct your dataset here? I know it's tempting to leave many details into appendix but if you don't provide sufficient information for readers to follow and reproduce your methods at bare minimum from the 8 main pgs then you may put the paper at greater risk for reviewer attack (things to consider) -- good to cross reference other benchmark construction papers and see how they write these details}


\textit{Pair-wise Answer Generation:} Finally, we extract visual cues using a dual-level approach. First, cues parsed from VLM-generated descriptions compensate for the limitations of open-set detectors when handling out-of-distribution scenes. Meanwhile, the open-set detector~\citep{wu2022grit} extracts fine-grained cues that VLMs might overlook. With these extracted cues, we prompt an LLM to generate a pair of answers for \textit{Mat} and \textit{Trk}, each consisting of one positive and one negative answer.




\subsection{Object-centric Cue (OC)}
\label{oc}
OC aims to assess a model's ability to link matching cues associated with everyday objects using \textit{object-centric cues}. Even when encountering an object for the first time, a well-aligned model should be able to leverage its unique visual cues to establish associations, enabling it to recognize and track the object across different scenes. This capability is essential for coherent perception and interaction in real-world deployments.

\paragraph{Subtasks.} 
\label{subtasks}
Based on the complexity of linking cues to solve the problem, we define three subtasks in OC. (i) \textbf{\textit{Comparison (Cpr)}} requires the model to determine whether the objects appearing in different images are the same. This task primarily assesses the model’s ability to perceive visual consistency or change. Notably, we observe that models exhibit significant model-specific bias when making a binary decision~\citep{pair, ye2024justice, song2024large,li2024naturalbench}, leading to discrepancies between results and their actual capabilities. To mitigate this, we introduce consistency-pair validation, where for each statement (e.g., ``X is Y”, with the answer being T), we generate a corresponding negation (e.g., ``X is not Y”, with the answer being F). The model is only considered correct if it correctly answers both statements, ensuring consistency in its decision-making.
(ii) \textbf{\textit{Counting (Cnt)}} involves identifying the number of unique objects, requiring the model not only to recognize variations or consistencies but also to track distinct cues to avoid double-counting the same object. (iii) \textbf{\textit{Grouping (Grp)}}, the most challenging one, requires the model to identify all instances of the same object, building on precise cue matching across multiple images.

\paragraph{Data Collection.}  
We manually collect various categories of everyday objects (e.g., pets, cups). For each category, we define multiple subcategories and collect a set of images \( \mathcal{I}_{O_i}\)—four images that depict the same object in different scenarios. Additionally, we also collect a set \( \mathcal{I}_{\neg O_i} \), consisting of four images of different objects, each containing some matching visual cues with \( \mathcal{I}_{O_i} \), which are used as distractors.
% We manually collect 8 major categories of everyday objects (e.g., pets, cups) from various online sources. Within each category, we defined multiple subcategories \( O_i \) and collected \( \mathcal{I}_{O_i} = \{ I_1, I_2, I_3, I_4 \} \)—4 images that depict the same object in different scenarios. Additionally, we also collect \( \mathcal{I}_{\neg O_i} \), a set of 4 images featuring different objects, each containing some matching visual cues with \( \mathcal{I}_{O_i} \). \( \mathcal{I}_{\neg O_i} \) are used as distractors to test the model's ability to differentiate between objects that share many cues but are distinct.


\paragraph{QA Construction.}
For each subtask, we define a question template that includes a placeholder for \( \mathcal{I}_{O_i} \), which allows us to tailor the question based on different objects (see Appendix~\ref{appendix: more details on benchmark construction}). For answer generation, we first curate the multi-image sequences according to predefined rules. For each specific sequence, we generate the ground truth answers for the questions related to \textit{Cpr}, \textit{Cnt}, and \textit{Grp}.




\subsection{Person-centric Cue (PC)}
\label{pc}
PC aims to evaluate a model's ability to link \textit{person-centric cues}. While a model cannot memorize every individual, it should possess the capability to associate the same person across different images or frames by leveraging distinctive visual cues such as facial features, clothing, or body posture. This ability is essential for ensuring coherent perception of human actions and is a fundamental requirement for real-world VLM applications.

\paragraph{Subtasks.} 
Similar to OC's subtasks (refer to \S\ref{subtasks}), PC includes (i) \textbf{\textit{Comparison (Cpr)}}, (ii) \textbf{\textit{Counting (Cnt)}}, and (iii) \textbf{\textit{Grouping (Grp)}}. However, unlike objects, individuals can be observed through their actions in videos. Therefore, we introduce (iv) \textbf{\textit{Video Identity Describing (VID)}}. This subtask assesses whether a model can correctly link the same person by analyzing its description of a video containing that person.


\paragraph{Data Collection.}  
We manually select several individuals, each denoted as \( \mathcal{P}_i \). For each individual, we collect \( \mathcal{I}_{\mathcal{P}_i}\)—4 images depicting the same individual. For each image \( I_i  \in  \mathcal{I}_{\mathcal{P}_i} \), we select the distractor images \( I_{\neg i} \notin \mathcal{I}_{\mathcal{P}_i} \) that has the highest CLIP similarity~\citep{hessel2021clipscore}. This allows us to obtain images of different individuals where most cues are matched.
% We manually select 30 different individuals, each denoted as \( \mathcal{P}_i \). For each individual, we collect \( \mathcal{I}_{\mathcal{P}_i} = \{ I_1, I_2, I_3, I_4 \} \)—4 images depicting the same individual in different scenarios. For each image \( I_i  \in  \mathcal{I}_{\mathcal{P}_i} \), we select the distractor images \( I_{\neg i} \notin \mathcal{I}_{\mathcal{P}_i} \) that has the highest CLIP similarity. This allows us to obtain images of different individuals where most cues are matched.
For the subtask of \textit{VID}, we collect videos of different individuals, denoted as \( V_{\mathcal{P}_i} \), and pair each with another video \( V_{\neg \mathcal{P}_i} \) featuring a different individual with highly similar cues (e.g., actions, scene, clothing). We then construct two video sequences: 
(i) \( \mathcal{P}_i \xrightarrow{} \neg \mathcal{P}_i \), assessing the model's ability to distinguish individuals. 
(ii) \( \mathcal{P}_i \xrightarrow{} \neg \mathcal{P}_i \xrightarrow{} \mathcal{P}_i \), evaluating whether the model detects changes and links the final occurrence of \( \mathcal{P}_i \) to its first appearance.

% Additionally, for the subtask of \textit{VID}, we collect videos of different individuals, denoted as \( V_{\mathcal{P}_i} \). For each \( V_{\mathcal{P}_i} \), we find another video of a different individual, denoted as \( V_{\neg \mathcal{P}_i} \), with highly similar cues (such as actions, scene, clothing, etc.). We then construct two types of videos based on different person sequences. (i) \( \mathcal{P}_i \xrightarrow{} \neg \mathcal{P}_i \), which tests the model's ability to distinguish between the two individuals. (ii) \( \mathcal{P}_i \xrightarrow{} \neg \mathcal{P}_i \xrightarrow{} \mathcal{P}_i \), which examines whether the model can detect changes between \( \mathcal{P}_i \) and \( \neg \mathcal{P}_i \), and link the final occurrence of \( \mathcal{P}_i \) to its first appearance.


\paragraph{QA Construction.} The construction for the overall QA in PC’s \textit{Cpr}, \textit{Cnt}, and \textit{Grp} subtasks follows a similar approach to OC. For the \textit{VID} task, we emphasize the model's ability to describe individuals when designing open-ended questions, aiming to better test the model's capacity to link individuals appearing in different scenes.

\begin{figure}[hb]
  \centering
  \includegraphics[width=0.49\textwidth]{img/bench-stastics.png} 
  % \vspace{-0.8cm}
  \caption{Statistical overview of \textbf{VLM$^2$-Bench}. The pie chart shows the distribution of 9 subtasks across the 3 main categories of visual cues. The bar plot illustrates the percentage breakdown by question format. 
  }
  \label{fig:bench statistics main}
\end{figure}

\begin{figure*}[t]
    \captionsetup{type=table}
    % \vspace{-0.4cm}
    \centering
    \begin{minipage}{0.99\textwidth}
        \centering
        \resizebox{1\textwidth}{!}{%
        \begin{tabular}{l||cc|ccc|cccc|cc}
            \toprule
            \textbf{Baselines or Models} & \multicolumn{2}{c|}{\textbf{GC}} & \multicolumn{3}{c|}{\textbf{OC}} & \multicolumn{4}{c|}{\textbf{PC}} & \multicolumn{2}{c}{\textbf{Overall*}}\\
            % \midrule
            & \textit{Mat} & \textit{Trk} & \textit{Cpr} & \textit{Cnt} & \textit{Grp} & \textit{Cpr} & \textit{Cnt} & \textit{Grp} & \textit{VID} & Avg & \textbf{$\Delta_{human}$} \\
            \midrule
            \textcolor{black!50}{Chance-Level} & 
            \textcolor{black!50}{25.00} & 
            \textcolor{black!50}{25.00} & 
            \textcolor{black!50}{50.00} & 
            \textcolor{black!50}{34.88} & 
            \textcolor{black!50}{25.00} & 
            \textcolor{black!50}{50.00} & 
            \textcolor{black!50}{34.87} & 
            \textcolor{black!50}{25.00} & 
            \textcolor{black!50}{-} & 
            \textcolor{black!50}{33.72} & 
            \textcolor{black!50}{-61.44} \\
            Human-Level  & 95.06 & 98.11 & 96.02 & 94.23 & 91.92 & 97.08 & 92.87 & 91.17 & 100.00 & 95.16 & 0.00 \\
            \midrule
            LLaVA-OneVision-7B    & 16.60 & 13.70 & 47.22 & 56.17 & 27.50 & \cellcolor{yellow!45}62.00 & 46.67 & 37.00 & \cellcolor{yellow!15}47.25 & 39.35 & -55.81 \\
            LLaVA-Video-7B        & 18.53 & 12.79 & 54.72 & \cellcolor{yellow!15}62.47 & 28.50 & \cellcolor{yellow!45}62.00 & \cellcolor{yellow!45}66.91 & 25.00 & \cellcolor{yellow!45}59.00 & 43.32 &  -51.84 \\
            LongVA-7B             & 14.29 & 19.18 & 26.67 & 42.53 & 18.50 & 21.50 & 38.90 & 18.00 & 3.75  & 22.59 & -72.57 \\
            mPLUG-Owl3-7B         & 17.37 & 18.26 & 49.17 & \cellcolor{yellow!45}62.97 & 31.00 & \cellcolor{yellow!15}63.50 & 58.86 & 26.00 & 13.50 & 37.85 & -57.31 \\
            Qwen2-VL-7B  & 27.80 & 19.18 & \cellcolor{yellow!15}68.06 & 45.99 & 35.00 & 61.50 & 58.59 & 49.00 & 16.25 & 42.37 & -52.79 \\
            Qwen2.5-VL-7B & \cellcolor{yellow!45}35.91 & \cellcolor{yellow!75}43.38 & \cellcolor{yellow!45}71.39 & 41.72 & \cellcolor{yellow!15}47.50 & \cellcolor{yellow!75}80.00 & 57.98 & \cellcolor{yellow!75}69.00 & 46.50 & \cellcolor{yellow!45}54.82 & -40.34 \\
            InternVL2.5-8B        & 21.24 & 26.03 & 53.33 & 55.23 & 46.50 & 51.50 & \cellcolor{yellow!15}60.00 & \cellcolor{yellow!15}52.00 & 5.25  & 41.23 & -53.93 \\
            InternVL2.5-26B        & \cellcolor{yellow!15}30.50 & \cellcolor{yellow!15}30.59 & 43.33 & 51.48 & \cellcolor{yellow!45}52.50 & 59.50 & 59.70 & \cellcolor{yellow!45}61.00 & 21.75 & \cellcolor{yellow!15}45.59 & -49.57 \\
            \midrule
            GPT-4o                & \cellcolor{yellow!75}37.45 & \cellcolor{yellow!45}39.27 & \cellcolor{yellow!75}74.17 & \cellcolor{yellow!75}80.62 & \cellcolor{yellow!75}57.50 & 50.00 & \cellcolor{yellow!75}90.50 & 47.00 & \cellcolor{yellow!75}66.75 & \cellcolor{yellow!75}60.36 & -34.80 \\
            \bottomrule
        \end{tabular}
        }
    \end{minipage}
    % \hfill
    % \begin{minipage}[c]{0.22\textwidth}
    %     \centering
    %     \vspace{-0.5em}
    %     \includegraphics[width=\textwidth]{img/radar_chart.png}
    % \end{minipage}
    % \vspace{-0.2cm}
    \caption{Evaluation results on \textbf{VLM$^2$-Bench}, covering \textit{Mat} (Matching), \textit{Trk} (Tracking), \textit{Cpr} (Comparison), \textit{Cnt} (Counting), \textit{Grp} (Grouping), and \textit{VID} (Video Identity Describing). The \colorbox{yellow!75}{highest}, \colorbox{yellow!45}{second}, and \colorbox{yellow!15}{third} highest scores are highlighted. *: Overall excludes the \textit{VID} due to the lack of a chance-level baseline for open-ended tasks.}
    \label{exp:main_exp}
    % \vspace{-0.4cm}
\end{figure*}

\subsection{Benchmark Statistics}
\label{bench_statistics_main}
Our benchmark is organized into three main categories, comprising a total of 9 subtasks. After careful verification, it contains 3,060 question-answer pairs, with varying formats including T/F, multi-choice (MC), numerical (Nu), and open-ended (Oe). To ensure the quality of the annotations, we perform an inter-annotator agreement (IAA) evaluation~\citep{thorne2018fever} involving three annotators, resulting in a high Fleiss' Kappa score~\citep{fleiss1971measuring} of 0.983. Figure~\ref{fig:bench statistics main} presents the distribution of these subtasks across the three categories, along with the breakdown of different question formats. For additional details, refer to Appendix~\ref{appendix: statistics}.

% \yi{TODO: add overall descriptive stats for dataset, and interannotator agreement}


\section{Related Work}
\label{sec:related_work}

The original investigation \cite{gibson1979ecological} on the relationship between visual perception and human action defines \emph{affordance} as the opportunities for interaction with the surrounding environment. Behavioral studies on regular and cognitively impaired persons have shown evidence that perception results in both visual and motor signals in the human brain. An extended study \cite{anderson2002attentional} shows that visual attention to the spatial characteristics of the perceived objects initiates automatic motor signals for different actions. In computer vision, human affordance learning involves novel pose prediction such that the estimated pose represents a valid human action within the scene context. The task is fundamental to many problems requiring robust semantic reasoning about the environment, such as human motion synthesis \cite{wang2021scene} and scene-aware human pose generation \cite{wang2017binge, roy2016multi, zhang2022inpaint, yao2023scene}.

Earlier methods of affordance learning have explored knowledge mining \cite{zhu2014reasoning} and multimodal feature cues \cite{roy2016multi} to address the problem. In \cite{zhu2014reasoning}, the authors use a Markov Logic Network for constructing a knowledge base by extracting several object attributes from different image and metadata sources, which can perform various downstream visual inference tasks without any additional classifier, including zero-shot affordance prediction. In \cite{roy2016multi}, the authors use depth map, surface normals, and segmentation map as multimodal cues to train a multi-scale convolutional neural network (CNN) for scene-level semantic label assignment associated with specific human actions. In \cite{do2018affordancenet}, the authors design a multi-branch end-to-end CNN with two separate pathways for object detection and affordance label assignment to achieve high real-time inference throughput. Researchers \cite{chuang2018learning} have also explored socially imposed constraints for affordance learning. In \cite{chuang2018learning}, the authors propose a graph neural network (GNN) to propagate contextual scene information from egocentric views for action-object affordance reasoning.

Probabilistic modeling of scene-aware human motion generation also involves semantic reasoning of human interaction with the environment. Initial works on human motion synthesis have taken different architectural approaches, such as sequence-to-sequence models \cite{barsoum2018hp}, generative adversarial networks (GAN) \cite{barsoum2018hp, cai2018deep, yang2018pose}, graph convolutional networks (GCN) \cite{yan2019convolutional}, and variational autoencoders (VAE) \cite{guo2020action2motion}. However, these methods have mostly ignored the role of environmental semantics. Due to potential uncertainty in human motion, in a recent approach \cite{wang2021scene}, the authors address such motion synthesis with a GAN conditioned on scene attributes and motion trajectory to predict probable body pose dynamics.

One key challenge of human affordance generation in 2D scenes is the lack of large-scale datasets with rich pose annotations. In \cite{wang2017binge}, the authors compile the only public dataset of annotated human body poses in complex 2D indoor scenes by extracting frames from sitcom videos. Aiming to generate a contextually valid human affordance at a user-defined location, the authors propose sampling the scale and deformation parameters for an existing human pose template using a VAE conditioned on the localized image patches as scene context. In \cite{zhang2022inpaint}, the authors introduce a two-stage GAN architecture for achieving a similar goal by estimating the affine bounding box parameters to localize a probable human in the scene and then generating a potential body pose at that location. The method uses the input scene, corresponding depth, and segmentation maps as semantic guidance. In \cite{yao2023scene}, the authors propose a transformer-based approach with knowledge distillation for generating human affordances in 2D indoor scenes.




\section{Methodology}
\paragraph{Preliminaries.}
We primarily focus on the homologous model merging, in which $\boldsymbol{\theta}_i$ all come from the same base model $\boldsymbol{\theta}_{\rm{base}}$. Given $K$ tasks $\{T_1,T_2,\cdots,T_K\}$ and $K$ corresponding fine-tuned models with parameters $\{\boldsymbol{\theta}_1,\boldsymbol{\theta}_2,\cdots,\boldsymbol{\theta}_K\}$, model merging aims to combine $K$ fine-tuned models into one single model simultaneously performing on $\{T_1,T_2,\cdots,T_K\}$ without post-training~\cite{method_p1_1,method_p1_2}.
Task vector~\cite{ilharco2023editing,yang2024adamerging} is a key element in merging method which could enhances the base model‘s ability or enable the model to handle other tasks. Specifically, for task $T_i$, the task vector $\boldsymbol\tau_i\in \mathbb{R}^D$ is defined as the vector obtained by subtracting the SFT weights $\boldsymbol{\theta}_i$ from the base model weight
$\boldsymbol{\theta}_{\rm{base}}$, \emph{i.e.}, $\boldsymbol\tau_i=\boldsymbol{\theta}_i-\boldsymbol{\theta}_{\rm{base}}$. The merged model could be denoted as $\boldsymbol{\theta}_m=\boldsymbol{\theta}_{\rm{base}}+\sum_i \lambda_i\boldsymbol{\tau}_i$, which $\lambda_i$ is the scaling factor measuring the importance of task vector. For clarification, we also denote the neuron set in $\boldsymbol{\theta}_i$ as $\mathcal{N}_i$, the neuron set in $\boldsymbol{\tau}_i$ as $\mathcal{T}_i$.



\begin{algorithm}[!ht]
    \caption{LED-Merging}
    \label{alg1}
    \begin{algorithmic}[1]
        \REQUIRE  base model $\boldsymbol{\theta}_{\rm{base}}$, SFT models $\{\boldsymbol{\theta}_{i}\mid i\in [K]\}$, mask ratios \{$r_{i} \mid i\in [K]\}$, scaling factors $\{\lambda_i\mid i\in[K]\}$, location datasets $\{\mathcal{X}_{i}\mid i\in[K]\}$
        \ENSURE merged parameter $\boldsymbol{\theta}_{m}$
        \STATE $\mathcal{M}\leftarrow\phi$
        \STATE $\boldsymbol{\theta}_{m}\leftarrow \boldsymbol{\theta}_{\rm{base}}$
        \FOR{$i\in [K]$}
        \STATE $I(\boldsymbol{\theta}_i)=\mathbb{E}_{x\sim \mathcal{X}_i}|\boldsymbol{\theta}_{i}\odot \nabla_{\boldsymbol{\theta}_i}\mathcal{L}(x)|$
        \STATE $I(\boldsymbol{\theta}_{\rm{base}})=\mathbb{E}_{x\sim \mathcal{X}_i}|\boldsymbol{\theta}_{\rm{base}}\odot \nabla_{\boldsymbol{\theta}_{\rm{base}}}\mathcal{L}(x)|$
        
        \STATE calculate $\mathcal{T}^{r_i}_{i}$ following Equation \ref{vote}
        \STATE  $\mathcal{M}\leftarrow \mathcal{M}\cup\{\mathcal{T}^{r_i}_i\}$
       
        
   
        
        
        \ENDFOR  
        \FOR{$i\in [K]$}
        
        \STATE calculate $\text{Disjoint}(\mathcal{T}_i^{r_i})$ use Equation~\ref{disjoint_safety}
        \STATE $\boldsymbol{m}_i \leftarrow \boldsymbol{0}$
        \FOR{$d\in \mathcal{T}_i^{r_i}$}
        \STATE $\boldsymbol{m}_{i,d}=1$
        \ENDFOR
        \STATE $\boldsymbol{\theta}_{m}\leftarrow \boldsymbol{\theta}_{m}+\lambda_i \boldsymbol{\tau}_i\odot \boldsymbol{m}_{i}$
        \ENDFOR
    \end{algorithmic}
\end{algorithm}
    %\vspace{-5pt}
\begin{figure*}[h!]
    \centering
    \includegraphics[width=\linewidth]{figs/pipeline_v2.pdf}
    \vspace{-40mm}
    \caption{Overview of our two-stage training pipeline {\ours}.}
    \label{fig:pipeline}
\end{figure*}


\paragraph{LED-Merging: Location, Election, and Disjoint Merging}
To address the neuron misidentification and interference issues in existing model merging methods, we propose LED-Merging (Location, Election, and Disjoint Merging). Specifically, previous studies \cite{modelstock, ilharco2023editing, tiesmerging} fail to accurately identify safety-related neurons in task vectors with a single magnitude score, namely \textit{neuron misidentification}. Meanwhile, there exists an interference between safety-related and utility-related task vector neurons during the merging process, namely \textit{neuron interference}. To address neuron misidentification, we first locate important neurons both in the base and fine-tuned models and then elect neurons from the task vector considering these two scores together. Subsequently, to mitigate the interference, we introduce a disjoint step, isolating these important neurons so that they influence different base neurons. The whole process is illustrated in Figure~\ref{fig:method}. 




In the location and election step, we consider the importance score from base and fine-tuned models simultaneously to locate task-specific neurons. In this way, it is more accurate than relying on the magnitude score alone because task-specific neurons with high importance score in the fine-tuned model may not necessarily score high in the base model, and vice versa.

{\textbf{Location}}.  We first calculate importance scores for each neuron in a base/fine-tuned model. Given a location dataset $\mathcal{X}_i=\{(x,y)_k\}$, where $x$ is the question and $y$ is the answer, we calculate the importance scores for the weight $\boldsymbol{\theta}_i\in\mathbb{R}^D$ in any  layer as follows~\cite{snip,spareseGPT,sun2024a}:
\begin{equation}
    I(\boldsymbol{\theta}_i)=\mathbb{E}_{x\sim \mathcal{X}_i}[\boldsymbol{\theta}_i\odot \nabla _{\boldsymbol{\theta}_i}\mathcal{L}(x)],
    \label{location}
\end{equation}
which $\mathcal{L}(x)=-\log p(y\mid x)$ is the conditional negative log-likelihood loss. We choose the SNIP score~\cite{snip} because it balances computational efficiency and performance~\cite{cq}. Please refer to Sec.~\ref{sec:ablation} for the comparison between different location methods. After computing importance scores, we choose top-$r_i$ neurons as the important neuron subset $\mathcal{N}_{i}^{r_i}$ from $I(\boldsymbol{\theta}_i)$.
 
 % After computing locating scores, we select the neurons scoring both high in base and fine-tuned models as important neurons in task vectors. Then in the disjoint step,  with preventing  polysemantic neurons  from receiving gradient updates towards different directions,
 % we use set difference to isolate the safety   and utility-related neurons  and construct corresponding masks for merging process,

{\textbf{Election}}. A natural question is how to select important neurons in the task vector $\boldsymbol{\tau}_i$ based on $I(\boldsymbol{\theta}_{\rm{base}})$ and $I(\boldsymbol{\theta}_{i})$. The important neurons in the base model may be different from neurons in the fine-tuned model. Therefore, we introduce the following election strategy to select neurons with high scores in both base and fine-tuned models:
\begin{equation}
    \mathcal{T}_i^{r_i}=\mathcal{N}_i^{r_i}\cap \mathcal{N}_{\rm{base}}^{r_i}.
    \label{vote}
\end{equation}
\emph{Remark}. We compare different choosing methods, including scoring low or high in base or fine-tuned model in Section~\ref{sec:ablation} and find that Equation \ref{vote} achieves the best performance.





{\textbf{Disjoint}}. As important neurons from different task vectors may conflict with each other at the same position, we use the set difference to disjoint the neurons from others to prevent interference:
\begin{equation}
    \text{Disjoint}(\mathcal{T}^{r_i}_{i})=\mathcal{T}^{r_i}_{i}-\mathop{\cup}\limits_{{J}\subsetneqq [K],|J|\geq 2}\mathop{\cap}\limits_{j\in {J}}\mathcal{T}^{r_j}_{j}.
    \label{disjoint_safety}
\end{equation}

Next, we construct a mask $\boldsymbol{m}_i\in\mathbb{R}^D$ to implement disjoint in the merging process. Specifically, this mask $\boldsymbol{m}_i$ is used to select neurons from $\mathcal{T}_i$. The mask ratio is $r_i$, where $r\in(0,1]$. The mask $\boldsymbol{m}_i$ can be derived from:
\begin{equation}
    \boldsymbol{m}_{i,d}=\begin{aligned} &\left\{ \begin{array}{ll} 1, & \text{if } d\in \text{Disjoint}(\mathcal{T}_{i}^{r_i}), \\ 0, & \text{otherwise}. \end{array} \right. \end{aligned}
    \label{mask_safety}
\end{equation}


% \subsection{Merging Models with Masks}
{\textbf{Merging}}. The final
merged task vector $\boldsymbol{\tau}_m$ is as follows:
\begin{equation}
    \boldsymbol{\tau}_m= \sum_i \lambda_i\boldsymbol{\tau}_{i}\odot\boldsymbol{m}_i.
    \label{merged_task_vector}
\end{equation}
We summarize the workflow in Algorithm \ref{alg1}.





\section{\dataset{}: A Consistency-Filtered Conflict Knowledge QA Dataset} \label{sec:benchmark}



In this section, we introduce the \textbf{Co}nsistency-filtered \textbf{Conflict} knowledge \textbf{QA} (\dataset{}) dataset, specifically designed to analyze LLMs' behavior in real-world knowledge conflict scenarios. We begin by highlighting the key differences between \dataset{} and existing works, followed by an overview of our knowledge conflict construction method, and conclude with a detailed description of the dataset construction process.


\textbf{Motivations of \dataset{}.}
Previous studies typically simulate knowledge conflicts by synthesizing counterfactual contexts that contradict accurate parametric knowledge~\cite{longpre2021entity,si2022prompting,xie2023adaptive}, thereby forcing LLMs to accept these pseudo facts provided in the input contexts. In contrast, our approach collects knowledge conflicts from the hallucinations and inaccurate memories of LLMs, rather than relying on these synthesized counterfactuals. Additionally, we ensure data quality by applying a self-consistency based data filtering method and utilizing a GPT-4o-mini-based conflict verification mechanism, distinguishing our method from previous works~\cite{yuan2024discerning,kortukovstudying}.



\textbf{Knowledge Conflict Construction.}
We propose a consistency-based knowledge conflict filtering method to identify and collect queries that induce conflicts between the parametric memory of LLMs and external information, thereby constructing a real-world knowledge conflict benchmark. Our framework consists of the following two steps.

\textit{Parametric Knowledge Elicitation.} 
We adopt a closed-book QA setup and implement a self-consistency mechanism to capture the parametric knowledge of LLMs. Specifically, we first prompt the LLMs $n$ times with the same query and identify the response generated most frequently (i.e., the majority answer) as the model's dominant parametric knowledge for that query. Queries where the majority answer appears fewer than $\frac{n}{2}$ times are filtered out to ensure the quality of the elicited parametric knowledge. In Appendix~\ref{append:data_freq}, we show that retaining queries with higher answering consistency can benefit knowledge conflict construction.



\textit{Conflict Data Selection.} Next, we select instances where the parametric knowledge of LLMs conflicts with the contextual answer. Specifically, we exclude cases where the answer is either fully correct or only partially correct. This ensures that \dataset{} contains queries that can lead to clear and substantial knowledge conflicts. Specifically, we use GPT-4o-mini to identify these conflicts, with the prompt used for this process detailed in Appendix~\ref{append:prompts}. To ensure high precision, we manually spot-check a subset of the detected conflicts and compare them against human annotations (Appendix~\ref{append:human_eval}).


\textbf{\dataset{} Dataset Construction.}
We construct the \dataset{} benchmark using the above conflict-filtering framework. This benchmark is designed to assess the context-faithfulness of LLMs in KAG scenarios.

\begin{table}[t!]
    \centering
    


% \renewcommand{\arraystretch}{1.2} % 调整行高
% \setlength{\tabcolsep}{10pt}  % 调整列间距
\small
\begin{tabular}{lrr}
        \toprule
        \textbf{Dataset} & \textbf{Full Size*} & \textbf{\dataset{}} \\
        \midrule
        HotpotQA  & 5,901  & 1,476 {\footnotesize \textcolor{gray}{(25\%)}}  \\
        NewsQA    & 4,212  & 934  {\footnotesize \textcolor{gray}{(22\%)}}  \\
        NQ        & 7,314  & 1,479 {\footnotesize \textcolor{gray}{(20\%)}}  \\
        SearchQA  & 16,980  & 1,497 {\footnotesize \textcolor{gray}{(9\%)}}  \\
        SQuAD     & 10,490  & 2,351 {\footnotesize \textcolor{gray}{(22\%)}}  \\
        TriviaQA  & 7,785  & 792  {\footnotesize \textcolor{gray}{(10\%)}}  \\
        \bottomrule
    \end{tabular}




  \caption{Statistics of \dataset{} dataset. \textbf{Full Size*} represents the deduplicated dataset size.}
  \label{tab:our_bench_stats}
\end{table}

Specifically, the \dataset{} dataset is built upon the 2019 MrQA Shared Task~\cite{fisch2019mrqa}. The shared task covers a variety of QA tasks: Natural Questions (NQ)~\cite{kwiatkowski2019natural}, SQuAD~\cite{rajpurkar2016squad}, NewsQA~\cite{trischler2016newsqa}, TriviaQA~\cite{joshi2017triviaqa}, SearchQA~\cite{dunn2017searchqa}, and HotpotQA~\cite{yang2018hotpotqa}. The dataset statistics are shown in Table~\ref{tab:our_bench_stats}.





\section{Experimental Methodology}
In this section, we describe the datasets, evaluation metrics, baselines and implementation details used in our experiments.


\textbf{Datasets.}
For our experiments, we use the \dataset{} dataset for both training and evaluation. In addition, we utilize the ConFiQA~\cite{bi2024context} dataset during evaluation, which serves as an out-of-domain test scenario to assess the generalization ability of different models. ConFiQA is designed to evaluate the context faithfulness of LLMs in adhering to counterfactual contexts. It consists of three subsets: Question-Answering, Multi-hop Reasoning, and Multi-Conflicts, each containing 6,000 instances.


\textbf{Evaluation.}
Following previous work~\cite{longpre2021entity, zhou2023context}, we employ multiple evaluation metrics to assess the generated responses from two aspects: correctness and context faithfulness. To ensure more accurate evaluations, we normalize both responses and answers according to the method described by \citet{li2024rag}.

For accuracy assessment, we use $\text{EM}(\uparrow)$, which measures whether the generated responses exactly match the ground truth answers.
To evaluate context faithfulness, we adopt two metrics: the recall of context (\text{ConR}$\uparrow$) and the recall of memory (\text{MemR}$\downarrow$). Specifically, \text{ConR} assesses whether the generated responses align with the provided context, while \text{MemR} assesses the alignment with parametric memories. Additionally, we adopt the memorization ratio $\text{MR}(\downarrow) = \frac{\text{MemR}}{\text{MemR} + \text{ConR}}$, which captures the tendency to rely on internal memory.

\begin{table*}[t!]
    
\centering
\resizebox{\textwidth}{!}{%
% \rowcolors{3}{gray!10}{white} % 交替背景色
\begin{tabular}{lc|cccc|cccc|cccc}
\toprule
% \rowcolor{gray!20}
\multirow{2}{*}{\textbf{Models}} & \multirow{2}{*}{\textbf{Param.}} & \multicolumn{4}{c|}{\textbf{HotPotQA}} & \multicolumn{4}{c|}{\textbf{NQ}} & \multicolumn{4}{c}{\textbf{NewsQA}} \\
\cmidrule(lr){3-6} \cmidrule(lr){7-10} \cmidrule(lr){11-14}

& &  ConR $\uparrow$ &  MemR $\downarrow$ &  MR $\downarrow$ &  EM $\uparrow$ &  ConR $\uparrow$ &  MemR $\downarrow$ &  MR $\downarrow$ & EM $\uparrow$ &  ConR $\uparrow$ &  MemR $\downarrow$ &  MR $\downarrow$ & EM $\uparrow$ \\
\midrule
\rowcolor{gray!10}
Vanilla-KAG~\shortcite{ram2023context} & 8.03B & 65.72 & 15.04 & 18.62 & 16.40 & 64.33 & 13.71 & 17.57 & 4.75 & 62.42 & 8.89 & 12.46 & 6.42 \\
\attrprompt{}~\shortcite{zhou2023context} & 8.03B & 65.31 & 15.58 & 19.26 & 6.23 & 66.24 & 11.55 & 14.85 & 0.65 & 62.74 & 8.35 & 11.75 & 3.85 \\
\rowcolor{gray!10}
\oiprompt{}~\shortcite{zhou2023context} & 8.03B & 60.98 & 15.58 & 20.35 & 2.44 & 64.41 & 9.76 & 13.15 & 1.27 & 62.10 & 8.46 & 11.99 & 2.46 \\
SFT~\shortcite{wei2021finetuned} & 8.03B & \uline{70.26} & \uline{7.05} & \uline{9.11} & \uline{64.84} & 68.45 & 8.57 & 11.13 & 67.31 & \uline{63.28} & 5.78 & 8.37 & 53.10 \\
\rowcolor{gray!10}
KAFT~\shortcite{li2022large} & 8.03B & 68.29 & 7.18 & 9.52 & 64.16 & \uline{68.86} & 8.37 & 10.84 & \uline{67.39} & 63.60 & \uline{5.35} & \uline{7.76} & \uline{53.75} \\
DPO~\shortcite{bi2024context}  & 8.03B & 67.21 & 8.74 & 11.51 & 50.20 & 67.55 & \textbf{7.76} & \uline{10.30} & 49.80 & 58.35 & \textbf{5.03} & 7.94 & 40.15 \\

\midrule
% our_{param.} & \textbf{6.97B} & \textbf{71.95} & \uline{6.71} & \uline{8.53} & \uline{65.79} & 61.05 & 9.87 & 13.92 & 59.03 & \textbf{65.95} & \uline{5.14} & \textbf{7.23} & 56.21 \\
% \rowcolor{gray!20}
\rowcolor{gray!10}
\method{} & \textbf{6.97B} & \textbf{71.95} & \textbf{6.44} & \textbf{8.21} & \textbf{67.41} & \textbf{69.88} & \uline{8.00} & \textbf{10.27} & \textbf{69.10} & \textbf{65.74} & 5.46 & \textbf{7.67} & \textbf{55.89} \\
\midrule
% \rowcolor{gray!20}

\multirow{2}{*}{\textbf{Models}} & \multirow{2}{*}{\textbf{Param.}} & \multicolumn{4}{c|}{\textbf{SearchQA}} & \multicolumn{4}{c|}{\textbf{SQuAD}} & \multicolumn{4}{c}{\textbf{TriviaQA}} \\
\cmidrule(lr){3-6} \cmidrule(lr){7-10} \cmidrule(lr){11-14}
% \rowcolor{gray!10}
& &  ConR $\uparrow$ &  MemR $\downarrow$ &  MR $\downarrow$ & EM $\uparrow$ &  ConR $\uparrow$ &  MemR $\downarrow$ &  MR $\downarrow$ & EM $\uparrow$ &  ConR $\uparrow$ &  MemR $\downarrow$ &  MR $\downarrow$ & EM $\uparrow$ \\
\midrule
\rowcolor{gray!10}
Vanilla-KAG~\shortcite{ram2023context} & 8.03B & 67.80 & 12.56 & 15.63 & 24.92 & 79.18 & 7.39 & 8.54 & 14.23 & \textbf{62.50} & 12.88 & 17.09 & 16.41 \\
\attrprompt{}~\shortcite{zhou2023context} & 8.03B & 65.87 & 12.22 & 15.65 & 8.68 & 78.93 & 7.60 & 8.79 & 4.38 & \uline{61.62} & 11.49 & 15.72 & 5.43 \\
\rowcolor{gray!10}
\oiprompt{}~\shortcite{zhou2023context} & 8.03B & 59.25 & 12.16 & 17.03 & 4.28 & \textbf{82.63} & 7.09 & 7.91 & 2.72 & 58.33 & 11.87 & 16.91 & 1.77 \\
SFT~\shortcite{wei2021finetuned} & 8.03B & \uline{75.28} & \uline{7.68} & \uline{9.26} & \uline{71.28} & 79.61 & 4.50 & 5.35 & 69.88 & 59.47 & 10.73 & 15.29 & 54.42 \\
\rowcolor{gray!10}
KAFT~\shortcite{li2022large} & 8.03B & 73.28 & 8.22 & 10.08 & 69.54 & 80.54 & \uline{4.21} & \uline{4.96} & \uline{71.03} & 60.10 & 9.72 & 13.92 & \uline{54.55} \\
DPO~\shortcite{bi2024context} & 8.03B & 66.67 & 8.02 & 10.73 & 51.24 & 78.93 & 5.01 & 5.97 & 52.63 & 58.84 & \uline{8.59} & \uline{12.73} & 44.95 \\
\midrule
% \rowcolor{gray!20}
% our_{param.} & 6.97B & 77.56 & 7.55 & 8.87 & 73.55 & 81.92 & 3.96 & 4.61 & 71.71 & 59.85 & 10.86 & 15.36 & 54.04 \\
\rowcolor{gray!10}
\method{} & \textbf{6.97B} & \textbf{78.22} & \textbf{7.01} & \textbf{8.23} & \textbf{75.68} & \uline{81.56} & \textbf{3.99} & \textbf{4.67} & \textbf{71.71} & \uline{61.62} & \textbf{7.95} & \textbf{11.43} & \textbf{57.32} \\
\bottomrule
\end{tabular}%
}

    \caption{Performance on the \dataset{} dataset. The highest scores are highlighted in \textbf{bold}, while the second-highest scores are \uline{underlined}. ``Param.'' refers to the total number of model parameters.}
     \label{tab:main_res_id}
\end{table*}

\begin{table*}[t!]
    
\centering
% \caption{Model performance on ConfiQA datasets.}
\resizebox{\textwidth}{!}{%
\begin{tabular}{lc|cccc|cccc|cccc}
\toprule

% \multirow{2}{*}{\textbf{Models}} & \multirow{2}{*}{\textbf{Param}} & \multicolumn{4}{c|}{\textbf{HotPotQA}} & \multicolumn{4}{c|}{\textbf{NQ}} & \multicolumn{4}{c}{\textbf{NewsQA}} \\
% \cmidrule(lr){3-6} \cmidrule(lr){7-10} \cmidrule(lr){11-14}


\multirow{2}{*}{\textbf{Models}} & \multirow{2}{*}{\textbf{Param.}} & \multicolumn{4}{c|}{\textbf{Question Answering}} & \multicolumn{4}{c|}{\textbf{Multi-hop Reasoning}} & \multicolumn{4}{c}{\textbf{Multi-Conflicts}}  \\
\cmidrule(lr){3-6} \cmidrule(lr){7-10} \cmidrule(lr){11-14}

& &  ConR $\uparrow$ &  MemR $\downarrow$ &  MR $\downarrow$ & EM $\uparrow$ &  ConR $\uparrow$ &  MemR $\downarrow$ &  MR $\downarrow$ & EM $\uparrow$ &  ConR $\uparrow$ &  MemR $\downarrow$ &  MR $\downarrow$ & EM $\uparrow$  \\
\midrule
\rowcolor{gray!10}
Vanilla-KAG~\shortcite{ram2023context} & 8.03B & 31.29 & 40.71 & 56.54 & 6.49 & 26.69 & 32.49 & 54.90 & 0.73 & 9.58 & 20.24 & 67.88 & 0.18  \\
\attrprompt{}~\shortcite{zhou2023context} & 8.03B & 47.71 & 28.40 & 37.31 & 0.84 & 27.56 & 30.02 & 52.14 & 0.02 & 9.33 & 19.16 & 67.24 & 0.07  \\
\rowcolor{gray!10}
\oiprompt{}~\shortcite{zhou2023context} & 8.03B & 62.64 & 15.96 & 20.30 & 10.04 & 22.11 & 23.31 & 51.32 & 0.20 & 12.89 & 16.11 & 55.56 & 0.25 \\
SFT~\shortcite{wei2021finetuned} & 8.03B & 79.36 & \uline{5.09} & 6.03 & \textbf{74.98} & 61.31 & \uline{13.84} & 18.42 & 63.44 & 62.89 & \uline{9.20} & 12.76 & 61.55   \\
\rowcolor{gray!10}
KAFT~\shortcite{li2022large} & 8.03B & \textbf{85.04} & 5.31 & \uline{5.88} & \uline{62.76} & \textbf{64.96} & 13.91 & \uline{17.64} & \textbf{67.91} & \textbf{66.27} & 9.42 & \uline{12.45} & \textbf{66.60}   \\
DPO~\shortcite{bi2024context} & 8.03B & \uline{80.71} & 5.69 & 6.58 & 61.00 & 63.38 & 14.00 & 18.09 & 47.96 & 64.87 & 9.84 & 13.18 & 46.90  \\

\midrule
\rowcolor{gray!10}
\method{} & \textbf{6.97B} & 80.56 & \textbf{4.51} & \textbf{5.30} & \textbf{74.98} & \uline{64.73} & \textbf{13.07} & \textbf{16.80} & \uline{65.07} & \uline{66.00} & \textbf{8.89} & \textbf{11.87} & \uline{61.84}   \\
\bottomrule
\end{tabular}%
}





% \centering
% % \caption{Model performance on ConfiQA datasets.}
% \resizebox{\textwidth}{!}{%
% \begin{tabular}{ll|cccc|cccc|cccc}
% \toprule
% \rowcolor{gray!20}
% \textbf{Model} & \textbf{Param} & \multicolumn{4}{c|}{\textbf{Confiqa\_QA}} & \multicolumn{4}{c|}{\textbf{Confiqa\_MR}} & \multicolumn{4}{c}{\textbf{Confiqa\_MC}} \\
% \cmidrule(lr){3-6} \cmidrule(lr){7-10} \cmidrule(lr){11-14}
% \rowcolor{gray!10}
% & & Pc & Po & MR & EM & Pc & Po & MR & EM & Pc & Po & MR & EM \\
% \midrule
% Base & 8.03B & 31.29 & 40.71 & 56.54 & 6.49 & 26.69 & 32.49 & 54.90 & 0.73 & 9.58 & 20.24 & 67.88 & 0.18 \\
% Attr. & 8.03B & 47.71 & 28.40 & 37.31 & 0.84 & 27.56 & 30.02 & 52.14 & 0.02 & 9.33 & 19.16 & 67.24 & 0.07 \\
% O\&I & 8.03B & 62.64 & 15.96 & 20.30 & 10.04 & 22.11 & 23.31 & 51.32 & 0.20 & 12.89 & 16.11 & 55.56 & 0.25 \\
% sft & 8.03B & 79.36 & 5.09 & 6.03 & \textbf{74.98} & 61.31 & 13.84 & 18.42 & 63.44 & 62.89 & 9.20 & 12.76 & 61.55 \\
% dpo & 8.03B & 80.71 & 5.69 & 6.58 & 61.00 & 63.38 & 14.00 & 18.09 & 47.96 & 64.87 & 9.84 & 13.18 & 46.90 \\
% kaft & 8.03B & \textbf{85.04} & 5.31 & 5.88 & 62.76 & \textbf{64.96} & 13.91 & 17.64 & \textbf{67.91} & \textbf{66.27} & 9.42 & 12.45 & \textbf{66.60} \\
% \midrule
% our_{param.} & 6.97B & 80.71 & 4.73 & 5.54 & 74.67 & 62.6 & 13.98 & 18.25 & 64.40 & 64.87 & 9.11 & 12.32 & 61.96 \\
% our_{ffn} & \textbf{6.97B} & 80.56 & \textbf{4.51} & \textbf{5.30} & \textbf{74.98} & 64.73 & \textbf{13.07} & \textbf{16.80} & 65.07 & 66.00 & \textbf{8.89} & \textbf{11.87} & 61.84 \\
% \bottomrule
% \end{tabular}%
% }
% \label{tab:main_res_ood}


    \caption{Performance on the testing sets of ConFiQA.}
     \label{tab:main_res_ood}
\end{table*}

\textbf{Baselines.}
We evaluate \method{} against five baselines, which are categorized into three groups: (1) Prompt-based approaches, including the attributed prompt (\attrprompt) and the combined opinion-based and instruction-based prompt (\oiprompt) from \citet{zhou2023context}; (2)  Fine-tuning methods, consisting of standard Supervised Fine-Tuning (SFT) and Knowledge Aware Fine-Tuning (KAFT)~\cite{li2022large}. KAFT enhances context faithfulness through counterfactual data augmentation; and (3) the Context-DPO~\cite{bi2024context} utilizes DPO method~\cite{rafailov2023direct} to strengthen context-grounded responses while penalizing those relying on parametric memory.

\textbf{Implementation Details.}
In our experiments, we use LLaMA3-8B-Instruct as the backbone for all methods. To train \method{}, we utilize \dataset{} to identify inaccurate parametric knowledge. The pruning threshold $\alpha$ is set to 0.05. The hyperparameters, $\lambda_1$ and $\lambda_2$, are set to 0.5, which are used to balance $\mathcal{L}_{\text{KAT}}$ and $\mathcal{L}_{\text{KPO}}$ in Eq.~\ref{eq:final_loss}. Additionally, we exclude the last three layers when selecting layers for pruning, as previous studies have shown that removing these layers significantly impacts model performance~\cite{lad2024remarkable,chen2024compressing,zhang2024finercut}. More implementation details are provided in Appendix~\ref{append:implementation}.




\section{Results}

\subsection{Homogeneous Model Merging}

\subsubsection{Averaging vs. Dare / Ties}
\label{sec:rq1}
% We merge 4 models (Base, Math \llama, Code \llama, Knowledge \llama) with the same architecture into a unified MoE with different merging methods and fine-tune it as described in Section \ref{sec:model_pretraining}. 

\textbf{Replacing simple averaging with Dare or Ties merging obtains better performance.} \quad
In this section, we demonstrate the superiority of our proposed Ties and Dare merging MoE over the BTX merging method.
We present the performance of MoE models with \textbf{Dare merging} or \textbf{Ties merging} on non-FFN layers and other baselines in Table \ref{table:homo_results}. The details of training cost for each method are presented in Table \ref{table:1_training_cost} in Appendix.

\begin{table}[!htb]
\centering
\resizebox{\columnwidth}{!}{%
\begin{tabular}{lccccccc}
\hline
Method             & MBPP           & HumanEval      & MATH          & GSM8K         & NQ            & TriviaQA       & Avg.           \\ \hline
\multicolumn{8}{c}{\textbf{Dense Model}}                                                                                               \\ \hline
\llamab & 4.60            & 3.04           & 2.42          & 1.44          & \textbf{6.61} & 26.72          & 7.47           \\ \hline
Code \llama        & \textbf{10.2}  & \textbf{8.53}  & 2.42          & 2.57          & 3.11          & 16.70           & 7.26           \\ \hline
Math \llama        & 9.80            & 6.71           & \textbf{7.81} & \textbf{6.36} & 5.48          & 19.86          & \textbf{9.34}  \\ \hline
Knowledge \llama   & 3.60            & 4.26           & 2.62          & 2.04          & 5.65          & \textbf{28.71} & 7.81           \\ \hline
\multicolumn{8}{c}{\textbf{MoE Merging}}                                                                                                 \\ \hline
Random Routing     & 4.00           & 6.10           & 2.78          & 2.05          & 4.86          & 21.75          & 6.92           \\ \hline
Router Fine-tuning & 3.60           & 6.71           & 2.42          & 2.96          & 5.82          & 25.98          & 7.92           \\ \hline
BTX merging        & 12.40          & 11.58          & 6.74          & 7.73          & \textbf{6.78} & 25.10           & 11.72          \\ \hline
Ties merging       & 14.20          & \textbf{11.98} & 6.74          & 7.81          & 6.72          & 27.66          & 12.52          \\ \hline
Dare merging       & \textbf{14.20} & 10.98          & \textbf{6.82} & \textbf{7.96} & 6.50           & \textbf{30.68} & \textbf{12.86} \\ \hline
\multicolumn{8}{c}{\textbf{MoE from Scratch}}                                                                                                \\ \hline
MoE Upcycling & 18.40  & 12.20 & 7.80 & 12.21 & 8.37  &  37.33 & 16.05 \\
\hline
\end{tabular}
}
\caption{\label{table:homo_results} \textbf{Performance of proposed Dare and Ties merged MoE and other baselines across six datasets.} The best performance of Dense and MoE model is marked in bold. Results of Dare and Ties merged MoE outperform the BTX MoE and other baseline methods.}
\end{table}

% Regarding dense model performance, from Table \ref{table:homo_results}, we find that individual expert LLMs achieve the best performance in their respective domain in most cases, as expected. However, CPTed \llama models also suffer from catastrophic forgetting. For example, both Code and Math \llama behaved worse than the \llamab model on the TriviaQA and NQ datasets.

From Table \ref{table:homo_results}, we see that individual experts generally achieve the best performance in their respective domains, as expected. However, CPTed \llama models experience catastrophic forgetting. For instance, both Code and Math \llama perform worse than \llamab on the TriviaQA and NQ datasets.


% The results of the MoEs in Table \ref{table:homo_results} suggest that by using the Ties or Dare merging to replace the average merging method, the proposed MoE performance exceeds the MoE from the BTX pipeline in almost all datasets and obtains a relative improvement of 6.94\% and 9.72\% compared to the BTX method in average performance. This observation implies that advanced merging methods mitigate the interference of model weights during merging and increase performance. 

The results in Table \ref{table:homo_results} show that using Ties or Dare merging significantly improves MoE performance over the BTX pipeline across almost all datasets, with a relative improvement of 6.94\% and 9.72\% in average performance. This suggests that advanced merging methods reduce weight interference and enhance performance.

% In addition to these baseline results, we include the results of MoE sparse upcycling \cite{komatsuzaki2022sparse} in the last row as a reference. In this approach, the MoE model is initialized from the base model by creating four identical copies of the FFN layers. We then CPT the initialized MoE model on the same training data (340B tokens) as used in the proposed pipeline. Note that we do not compare our results with the upcycling method, since that method pretrains the entire MoE on all data with significantly higher additional cost.

As a reference, we include the results of MoE sparse upcycling \cite{komatsuzaki2022sparse} in the last row of Table \ref{table:homo_results}. This approach initializes the MoE model by creating four identical copies of the FFN layers from the base model and then CPT on the same 340B tokens used in our pipeline. However, we do not directly compare our results with the upcycling method, as it involves pretraining the entire MoE on all data, incurring significantly higher costs.
We also visualize the average performance for each merging method with different fine-tuning token numbers in Figure \ref{fig:result_token} in Appendix \ref{sec:supp_routing}.
In Figure \ref{fig:result_token}, we observe that the Dare and Ties merging MoE models consistently outperform the BTX merging MoE throughout fine-tuning, especially in the earlier stages of fine-tuning. 
% There is more performance gain when fine-tuning with a smaller number of tokens, which follows our expectation since we regard Dare or Ties-merging methods as a better initialization method and have a larger effect during the early stage of training. This observation implies that more advanced merging techniques should be preferred over unweighted average especially when the fine-tuning budget is limited.  

\begin{figure}[!t]
    \centering
    \begin{subfigure}[b]{0.48\columnwidth}
        \centering
        \includegraphics[width=\textwidth]{figure/routing_probabilities_GSM8K.pdf}
        \caption{GSM8K}
        \label{fig:gsm8k_route}
    \end{subfigure}
    \hfill
    \begin{subfigure}[b]{0.48\columnwidth}
        \centering
        \includegraphics[width=\textwidth]{figure/routing_probabilities_MATH.pdf}
        \caption{MATH}
        \label{fig:math_route}
    \end{subfigure}
    \vspace{-0.5em}
    \caption{Routing probability of experts on GSM8K and MATH for different merging methods.}
    \label{fig:routing_prob}
\end{figure}


\noindent \textbf{MoE with Dare or Ties merging routes more tokens to domain experts.}\quad
To further explore the effectiveness of Dare and Ties merging MoE, we evaluate MoEs on multiple benchmarks and calculate the routing probability averaged from each layer and token. We visualize the routing probability of each method of two math datasets (MATH and GSM8K) in Figure \ref{fig:routing_prob} and for other datasets, we put the results in Figure \ref{fig:supp_routing_prob} in Appendix \ref{sec:supp_routing}.


Compared to MoEs with BTX merging, where the base model accepts the most routing decisions, the Dare and Ties merging method routes tokens to domain experts more frequently, as suggested in Figure \ref{fig:routing_prob}. For example, for the GSM8K dataset, the routing probability for math expert increases from 0.28 to 0.35 or 0.46 when replacing simple averaging with the Ties or Dare merging. This finding suggests that the more effective MoE with the more advanced merging method should be attributed to more optimized routing decisions.


\subsubsection{Merging without Fine-tuning}

In this part, we will evaluate our proposed routing heuristics in Section \ref{sec:merging_wo_ft} for MoE without fine-tuning.
Before we evaluate the overall performance of each benchmark, we will first examine the routing decision with our proposed heuristics. We present the routing probability for PPL routing heuristics for each dataset in Table \ref{table:routing_heuristic}.

\begin{table}[!htb]
\centering
\resizebox{0.8\columnwidth}{!}{%
\begin{tabular}{lcccc}
\hline
% \multicolumn{5}{c}{Heuristic 1: Task Vector Routing}     \\ \hline
Benchmark      & Base       & Code          & Math          & Knowledge     \\ \hline
GSM8K     & 23\%       & 2\%           & \textbf{43\%} & {\ul 32\%}    \\ \hline
MATH      & 22\%       & 2\%           & \textbf{49\%} & {\ul 27\%}    \\ \hline
MBPP      & 19\%       & {\ul 22\%}    & \textbf{44\%} & 15\%          \\ \hline
HumanEval & 5\%        & {\ul 43\%}    & \textbf{45\%} & 7\%           \\ \hline
NQ        & {\ul 43\%} & 4\%           & 10\%          & \textbf{43\%} \\ \hline
TriviaQA  & {\ul 50\%} & 0\%           & 0\%           & \textbf{50\%} \\ \hline
\end{tabular}
}
\caption{\label{table:routing_heuristic} \textbf{Routing probability of PPL routing for each dataset.} The largest probability are in bold, and the second-largest are underlined.}
\end{table}

\noindent \textbf{Routing heuristic effectively assigns tokens to the corresponding experts.} \quad 
Table \ref{table:routing_heuristic} demonstrates that PPL routing generally achieves the desired routing patterns, effectively directing inputs from a specific domain to the specialized expert models, except in the case of the MBPP dataset. Since our heuristics rely solely on inference inputs without fine-tuning, they can be considered reliable strategies. We also visualize the routing probability for both PPL and task vector routing heuristics for each dataset in Figure \ref{fig:routing_heuristic} in Appendix \ref{sec:supp_routing}. We find that PPL routing consistently produces better results than the task vector routing.

Next, we evaluate the performance on each dataset with different combinations of merging methods and routing heuristics, compared to the baseline methods. We prepare three dense fine-tuning baselines: \textbf{Dare Dense}, \textbf{Ties Dense} and \textbf{Random Routing} (details in Section \ref{sec:baseline}). We also evaluate the ablation methods: merging attention layers without separation and task vector routing. We present the results of each method across datasets in Table \ref{table:moe_wo_finetune}. The details of training cost for each method are presented in Table \ref{table:3_training_cost} in Appendix.

\begin{table}[!htb]
\centering
\resizebox{\columnwidth}{!}{%
\begin{tabular}{llccccccc}
\hline
Merging      & Routing     & MBPP         & HumanEval     & MATH          & GSM8K         & NQ            & TriviaQA       & Avg.          \\ \hline
\multicolumn{9}{c}{\textbf{Dense Merging}}                                                                                                          \\ \hline
Dare       & N/A         & 6.20         & 6.70          & 2.22          & 2.27          & 4.80          & 20.45          & 7.11          \\ \hline
Ties        & N/A         & 6.00         & 6.70          & 2.48          & 2.19          & 3.62          & 20.86          & 6.98          \\ \hline
\multicolumn{9}{c}{\textbf{MoE Merging}}                                                                                                            \\ \hline
Merge attention     & random      & 4.00         & 6.10          & 2.78          & 2.05          & 4.86          & 21.75          & 6.92          \\ \hline
Merge attention     & task vector & 6.60          & 4.87          & 3.06          & 1.44          & \textbf{6.05} & 21.39          & 7.24          \\ \hline
Merge attention     & PPL         & 6.40          & 4.87          & 2.86          & 1.13          & 5.93          & 22.71          & 7.32          \\ \hline
Separate attention & task vector & 4.00            & 7.32          & \textbf{2.98} & 2.5           & 5.37          & 20.11          & 7.05          \\ \hline
Separate attention & PPL         & \textbf{6.80} & \textbf{7.92} & 2.88          & \textbf{2.95} & 4.74          & \textbf{23.21} & \textbf{8.08} \\ \hline
\end{tabular}
}
\caption{\label{table:moe_wo_finetune} \textbf{Performance of proposed merging and routing methods for MoE without substantial fine-tuning and other baselines across six datasets.} Separating attention layers and perplexity routing heuristics get the best average performance.}
\end{table}

% \noindent \textbf{Proposed MoE method without fine-tuning perform better then dense merging baseline.} \quad From the results in Table \ref{table:moe_wo_finetune}, we find that with the perplexity routing heuristic and the option of separating the attention layers, we get the best average result among all baseline methods. Compared to Random Routing and SoTA dense merging method: Dare, our best methods: PPL routing + separating the attention layers bring a relative improvement of 16.8\% and 13.6\%, respectively, on the average performance. Better results with PPL routing than with task vector routing align with the observation in Figure \ref{fig:routing_heuristic}, where PPL routing routes input more precisely to the desired experts. 
% Moreover, better result for merging without attention layers also follows our expectation that this merging option eliminates the inconsistency of task vector effect as suggested in Section \ref{sec:merging_wo_ft}.

\noindent \textbf{Proposed MoE method without fine-tuning outperforms the dense merging baseline.} \quad From Table \ref{table:moe_wo_finetune}, we observe that using the PPL routing heuristic and separating attention layers achieves the best average results among all baseline methods. Compared to Random Routing and the SoTA dense merging method (Dare), our best method - PPL routing + separating attention layers - yields relative improvements of 16.8\% and 13.6\%, respectively. The superior performance of PPL routing aligns with Figure \ref{fig:routing_heuristic} in Appendix \ref{sec:supp_routing}, where PPL routing more accurately directs input to the appropriate experts. 
Moreover, the better results of separating attention layers support our expectation that this approach resolves the inconsistency of task vector counts, as discussed in Section \ref{sec:merging_wo_ft}.


\subsection{Heterogeneous Model Merging}

% We prepare two MoE models with heterogeneous model merging for this part. We use 3 models (\llamab, Code \llama, and Knowledge \llama) with the same architecture and one Math Olmo model for the first MoE. For the second MoE, we merge 3 models and one Math TinyLlama model. 
% For the tokenizer, we choose the \llamab tokenizer due to the majority of experts. 
% We also prepare two 3-expert MoE models (\llamab, Code \llama, and Knowledge \llama) as the baseline method to show the functionality of Olmo or TinyLlama experts. One 3-expert MoE is fine-tuned on the same data source, and the other one is fine-tuned only on the data without math data. Both baselines are fine-tuned with 100B tokens. 

\textbf{MoE merged with heterogeneous models outperforms the corresponding experts.} \quad
After showing the superiority of our homogeneous model merging method, our next question is whether the proposed heterogeneous expert merging is also effective.
We present the performance of the dense, MoE and baseline methods in Table \ref{table:hetero_results}. The details of training cost for each method are presented in Table \ref{table:4_training_cost} in Appendix.

\begin{table}[!htb]
\centering
\resizebox{\columnwidth}{!}{%
\begin{tabular}{lccccccc}
\toprule
Method                                                           & MBPP           & HumanEval      & MATH          & GSM8K         & NQ            & TriviaQA      & Avg.           \\ \hline
\multicolumn{8}{c}{\textbf{Dense Model}}                                                                                                                                            \\ \hline
\llamab                                                      & 4.60           & 3.04           & 2.42          & 1.44          & 6.61          & 26.72         & 7.47           \\ \hline
Base TinyLlama                                                   & 5.40           & 5.27           & 2.26          & 2.2           & 8.53          & 34.27         & 9.66           \\ \hline
Base Olmo                                                        & 2.80           & 2.64           & 2.46          & 2.42          & 6.16          & 29.21         & 7.62           \\ \hline
Code \llama                                                       & 10.20          & 8.53           & 2.42          & 2.57          & 3.11          & 16.7          & 7.26           \\ \hline
Math TinyLlama                                                   & 15.60          & 9.76           & 4.18          & 5.91          & 6.05          & 21.12         & 10.44          \\ \hline
Math Olmo                                                        & 0.00           & 0.00           & 4.82          & 5.08          & 3.61          & 11.25         & 4.13           \\ \hline
Knowledge \llama                                                  & 3.60            & 4.26           & 2.62          & 2.04          & 5.65          & 28.71         & 7.81           \\ \hline

\multicolumn{8}{c}{\textbf{Homogeneous Expert Merging}}                                                                                                                                              \\ \hline
% BTX merging        & 12.40          & 11.58          & 6.74          & 7.73          & 6.78 & 25.1           & 11.72          \\ \hline
\begin{tabular}[c]{@{}l@{}}3-expert MoE \\ (same data)\end{tabular}  & 9.14           & 10.8           & 4.42          & 5.16          & 6.95          & 26.78         & 10.54          \\ \hline
\begin{tabular}[c]{@{}l@{}}3-expert MoE \\ (w/o math)\end{tabular}   & 12.00             & 9.76           & 2.38          & 1.74          & 6.22          & \textbf{33.20} & 10.88          \\ \hline


\multicolumn{8}{c}{\textbf{Heterogeneous Expert merging}}                                                                                                                                              \\ \hline
\begin{tabular}[c]{@{}l@{}}(Ours) MoE w/ \\ Math Olmo\end{tabular}      & 13.60           & 10.98          & 4.86          & 6.14          & 5.43 & 26.01         & 11.17          \\ \hline

\begin{tabular}[c]{@{}l@{}} (Ours) MoE w/ \\ Math TinyLlama\end{tabular} & \textbf{15.80} & \textbf{11.59} & \textbf{5.42} & \textbf{6.29} & \textbf{8.25} & 32.71         & \textbf{13.34} \\ \bottomrule
\end{tabular}
}
\caption{\label{table:hetero_results} \textbf{Performance of proposed heterogeneous merged MoE and other baselines.} The merged MoE is comparable or outperform the dense or 3-expert baselines on the benchmark from the corresponding domain.}
\end{table}

% From Table \ref{table:hetero_results}, compared to the domain expert models, our merged heterogeneous MoE models are comparable to or exceed the expert results in their corresponding domains. For example, MoE with Math Olmo and Math TinyLlama achieves the performance of 6.14\% and 6.29\% on GSM8K datasets and our CPTed dense model: Math Olmo and Math TinyLlama obtain 5.91\% and 5.08\% accuracy. For average performance, our merging framework also brings a relative improvement of 43.02\% and 27.78\% between the best-performed experts and the merged MoE models for MoE with Olmo and TinyLlama, respectively.
% As for the MoE baseline: 3-expert MoE, both heterogeneous merged MoEs outperform their performance,  especially in the math domains, demonstrating the functionality of including the math expert, which suggests the effectiveness of our merged pipeline. 

Table \ref{table:hetero_results} shows that our merged MoE models are comparable to or outperform the domain expert models in their respective domains. For instance, the MoE merged with Math Olmo and Math TinyLlama achieves 6.14\% and 6.29\% accuracy on GSM8K, compared to 5.91\% and 5.08\% for their dense counterparts. On average, our MoEs with Olmo and TinyLlama improves performance by 43.02\% and 27.78\% relative to the best dense experts, respectively. Both MoEs with heterogeneous experts also outperform the 3-expert MoE baseline, particularly in math, highlighting the effectiveness of including math experts in the pipeline.

\noindent \textbf{MoE merged with heterogeneous experts show the desired routing patterns in most cases.} \quad We also perform a similar routing analysis as described in Section \ref{sec:rq1}. We visualize the routing probability of two MoEs when evaluating on GSM8K and MATH datasets in Figure \ref{fig:hetero_routing_prob} and for other datasets, we visualize the results in Figure \ref{fig:supp_hetero_routing_prob} in Appendix \ref{sec:supp_routing}.

\begin{figure}[!t]
    \centering
    \begin{subfigure}[b]{0.48\columnwidth}
        \centering
        \includegraphics[width=\textwidth]{figure/hetero_routing_probabilities_GSM8K.pdf}
        \caption{GSM8K}
        \label{fig:hetero_gsm8k_route}
    \end{subfigure}
    \hfill
    \begin{subfigure}[b]{0.48\columnwidth}
        \centering
        \includegraphics[width=\textwidth]{figure/hetero_routing_probabilities_MATH.pdf}
        \caption{MATH}
        \label{fig:hetero_math_route}
    \end{subfigure}
    \caption{Routing probability of experts on GSM8K and MATH for the MoE w/ Olmo and MoE w/ TinyLlama.}
    \label{fig:hetero_routing_prob}
\end{figure}

% From the routing analysis in Figures \ref{fig:hetero_routing_prob} and \ref{fig:supp_hetero_routing_prob}, for the coding and knowledge datasets, most of the tokens will be routed to the corresponding experts.
% However, unlike the homogeneous model merging where the math expert shares the highest routing probability in math datasets, the math expert (Olmo or TinyLlama) obtains the second highest routing probability. This discrepancy should be attributed to the difference between the embedding output from MoE models and expert models. Since the embedding layer of MoE is merged from 3 \llama models and 1 other model, the embedding output of MoE is expected to be closer to the embedding output of \llama models. Therefore, the router network is more likely to route the inputs to the \llama model. Adding the load balancing loss may be the possible solution to this caveat \cite{sukhbaatar2024branchtrainmixmixingexpertllms, fedus2022switch}, which ensures a more uniform distribution of the routing load, and we leave that exploration to future work.

As shown in Figures \ref{fig:hetero_routing_prob} and \ref{fig:supp_hetero_routing_prob}, most tokens in the coding and knowledge datasets are routed to the corresponding experts. However, unlike homogeneous model merging where the math expert has the highest routing probability for math datasets, Math Olmo or Math TinyLlama ranks second. This discrepancy is likely due to the difference in embedding outputs between the MoE and expert models. Since the MoE's embedding layer is merged from 3 \llama models and 1 other model, its output is closer to that of the \llama models, making the router more likely to select them. Adding a load balancing loss is a possible solution to address this issue \cite{sukhbaatar2024branchtrainmixmixingexpertllms, fedus2022switch}, ensuring a more uniform routing distribution. We leave this for future exploration

\subsection{Effectiveness of \method{} in Leveraging Contextual Knowledge} \label{sec:analysis}

In Figure~\ref{fig:analysis}, we further evaluate the ability of different LLMs to utilize contextual knowledge. We compare the performance of three models: the vanilla LLM, the Uninstall model (\method{} w/o Adaptation), and our \method{}.

\begin{figure}[!t]
    \centering
    % \section{Analysis}
\label{sec:analysis}
\subsection{Quantifying the Influence of Adversarial Suffixes}
In our earlier experiments, we established that features extracted from benign datasets can be harnessed to manipulate large language models (LLMs) into producing harmful outputs, effectively executing successful jailbreak attacks. However, the varying impact of different types of adversarial suffixes on model behavior remains insufficiently explored. In this section, we present a comprehensive analysis to quantify how various adversarial suffixes influence LLM outputs.

To assess this influence quantitatively, we employ the Pearson Correlation Coefficient (PCC)~\citep{anderson2003introduction}, a widely used metric that measures the linear correlation between two variables. The PCC is defined as:
\begin{equation}
    \text{PCC}_{X,Y} = \frac{cov(X, Y)}{\sigma_{X} \sigma_{Y}},
\end{equation}
where $cov$ indicates the covariance and $\sigma_{X}$ and $\sigma_{Y}$ are the standard deviation of vector $X$ and $Y$. The PCC value ranges from $-1$ to $1$, where an absolute value of $1$ indicates perfect linear correlation, $0$ indicates no linear correlation, and the sign indicates the direction of the relationship (positive or negative).
\begin{figure}[!t]
\centering
    % First row
    \begin{minipage}[b]{0.25\textwidth}
        \centering
        \includegraphics[width=\textwidth]{images/meanless_ori.pdf}\\
        \includegraphics[width=\textwidth]{images/meanless_suffix.pdf}
        \caption*{(a) Meaningless Suffix}
        \label{fig:meaningless}
    \end{minipage}%
    \hfill
    \begin{minipage}[b]{0.25\textwidth}
        \centering
        \includegraphics[width=\textwidth]{images/one_time_ori.pdf}\\
        \includegraphics[width=\textwidth]{images/one_time_suffix.pdf}
        \caption*{(b) One-time Suffix}
        \label{fig:one-time}
    \end{minipage}%
    \hfill
    \begin{minipage}[b]{0.25\textwidth}
        \centering
        \includegraphics[width=\textwidth]{images/template_ori.pdf}\\
        \includegraphics[width=\textwidth]{images/template_suffix.pdf}
        \caption*{(c) Template Suffix}
        \label{fig:template}
    \end{minipage}

    \vspace{1em} % Add some vertical space between rows

    % Second row
    \begin{minipage}[b]{0.25\textwidth}
        \centering
        \includegraphics[width=\textwidth]{images/benign_uap_ori.pdf}\\
        \includegraphics[width=\textwidth]{images/benign_uap_suffix.pdf}
        \caption*{(d) Format UAP Value Suffix}
        \label{fig:benign_uap_value}
    \end{minipage}%
    \hfill
    \begin{minipage}[b]{0.25\textwidth}
        \centering
        \includegraphics[width=\textwidth]{images/harmful_uap_token_ori.pdf}\\
        \includegraphics[width=\textwidth]{images/harmful_uap_token_suffix.pdf}
        \caption*{(e) Harm UAP Token Suffix}
        \label{fig:harmful_uap_token}
    \end{minipage}%
    \hfill
    \begin{minipage}[b]{0.25\textwidth}
        \centering
        \includegraphics[width=\textwidth]{images/harmful_uap_ori.pdf}\\
        \includegraphics[width=\textwidth]{images/harmful_uap_suffix.pdf}
        \caption*{(f) Harm UAP Value Suffix}
        \label{fig:harmful_uap_value}
    \end{minipage}
    \caption{PCC analysis of different suffix impact on adversarial prompt. Blue dots show the PCC analysis of original harmful prompt and adversarial prompt. Red dots show PCC analysis of suffix and adversarial prompt.}
    \label{fig:pcc_analysis}
\end{figure}

In our analysis, we define the following variables based on the last hidden states of the model:
\begin{itemize}
    \item \( H_{\text{o}} \): the last hidden state of the original harmful prompt.
    \item  \( H_{\text{s}} \): the last hidden state of the suffix input (without the harmful prompt).
    \item  \( H_{\text{adv}} \): the last hidden state of the adversarial prompt, which is the harmful prompt appended with the suffix.
\end{itemize}

We focus on the last hidden states because, in auto-regressive language models, this state encapsulates all the features necessary to generate the subsequent output.

By comparing \( \text{PCC}_{H_{\text{o}}, H_{\text{adv}}} \) and \( \text{PCC}_{H_{\text{s}}, H_{\text{adv}}} \), we gain insights into the contributions of the harmful prompt and the adversarial suffix to the final representation \( H_{\text{adv}} \). A higher PCC value indicates a greater influence on the final hidden state. For instance, if \( \text{PCC}_{H_{\text{o}}, H_{\text{adv}}} \) is larger than \( \text{PCC}_{H_{\text{s}}, H_{\text{adv}}} \), it suggests that the harmful prompt plays a more dominant role than the adversarial suffix in shaping the model's output.

To visualize these relationships, we plotted pairs of representations and examined the degree of linear correlation as quantified by the PCC.

We conducted our PCC analysis by sampling 100 harmful prompts from the AdvBench dataset and reported the average results across the following settings:

\begin{itemize}
    \item \textbf{Prompt + Meaningless Suffix}:

    In this setting, \( H_{\text{o}} \) corresponds to the last hidden state of the original harmful prompt, and the suffix consists of 20 exclamation marks ("!"). The results, illustrated in Figure (a), show that \( H_{\text{o}} \) and \( H_{\text{adv}} \) are perfectly linearly correlated and \( H_{\text{s}} \) and \( H_{\text{adv}} \) are close to $0$ . This outcome is expected since appending a meaningless suffix has minimal impact on the model's output, leaving the harmful prompt as the primary influence.

    \item \textbf{Prompt + One-Time Suffix}:

    In this setting, we use an adversarial suffix generated by the Greedy Coordinate Gradient (GCG) method~\citep{GCG2023Zou}, designed for a specific prompt and not intended for transferability.  Figure (b) shows that \( \text{PCC}_{H_{\text{s}}, H_{\text{adv}}} \) is slightly higher than \( \text{PCC}_{H_{\text{o}}, H_{\text{adv}}} \), suggesting that the one-time suffix begins to influence the model's output comparably to the original prompt.

    \item \textbf{Prompt + Template Suffix}:

    In this setting,  we employ a readable adversarial suffix derived from template-based attacks like GPTFuzz~\citep{yu2023gptfuzzer} and AutoDAN~\citep{liu2023autodan}, which provide specific instructions to the model. Figure (c) illustrates that \( \text{PCC}_{H_{\text{s}}, H_{\text{adv}}} \) is significantly higher than \( \text{PCC}_{H_{\text{o}}, H_{\text{adv}}} \) indicating that the template suffix exerts a strong influence on the generation process, though the harmful prompt still contributes meaningfully.

    \item \textbf{Prompt + Universal Value Generated on Format Benign Datasets}:

    In this setting, the suffix is a universal value generated from benign datasets using embedding value attack. Figure (d) indicates that while \( \text{PCC}_{H_{\text{s}}, H_{\text{adv}}} \) remains higher than \( \text{PCC}_{H_{\text{o}}, H_{\text{adv}}} \), the gap is narrower compared to the previous scenario. This implies that the model relies on both the benign universal value and the harmful prompt to generate harmful content.
    
    \item \textbf{Prompt + Universal Token Generated on Harmful Datasets}:

    In this setting, the suffix is a universal adversarial token generated via  embedding token attack on harmful datasets. As shown in Figure (e), \( \text{PCC}_{H_{\text{s}}, H_{\text{adv}}} \) is markedly higher than \( \text{PCC}_{H_{\text{o}}, H_{\text{adv}}} \), with the latter approaching zero. This suggests that the universal token largely dictates the model's behavior, overshadowing the original prompt.

    \item \textbf{Prompt + Universal Value Generated on Harmful Datasets}:

    Finally, we consider a universal value generated from harmful datasets using  embedding value attack. Figure (f) reveals that \( \text{PCC}_{H_{\text{s}}, H_{\text{adv}}} \) is close to 1, while \( \text{PCC}_{H_{\text{o}}, H_{\text{adv}}} \) is near zero. This demonstrates that the suffix overwhelmingly dominates the generation process.
\end{itemize}

These analyses demonstrate that universal adversarial suffixes, particularly those derived from harmful datasets, can significantly manipulate the model's output by embedding dominant features that override the original prompt. Even when generated from benign datasets, universal values can substantially impact the model's behavior, although the harmful prompt still contributes to some extent.




% \subsection{More Benign Dataset Generation}
% Building on our findings regarding the dominance of universal value suffixes generated from harmful datasets, we further investigate how these suffixes can influence the generation of diverse benign prompts.

% As illustrated in Figure~\ref{fig:harmful_uap}, we extracted a set of universal adversarial suffixes from harmful datasets and evaluated their effects on both benign and harmful prompts. Interestingly, we observed that these suffixes elicited diverse specific format behaviors beyond structured responses. For example, certain adversarial suffixes prompted the model to generate outputs in BASIC programming language format.

% Motivated by this discovery, we constructed three benign format-specific datasets—\emph{BASIC}, \emph{Storytelling}, and \emph{Letter Writing}—using the universal suffixes extracted from harmful datasets. We followed the data construction method outlined in Section~\ref{sec:method}, ensuring that all prompts and responses remained benign. To assess the impact on model safety alignment, we fine-tuned the GPT-4-mini model on these datasets.

% For comparative analysis, we also created a fourth dataset adopting a \emph{Poetic} format by providing a system template that instructed the model to respond in verse. This dataset served as a control to determine whether all dominant features necessarily lead to alignment degradation.
% \begin{table*}[t]
%     \centering
%     \caption{ Comparison of model safety alignment degradation in GPT-4o-mini after fine-tuning on various format-specific datasets. }
%     \label{tab:dataset_category}
%     \begin{tabular}{l|cc|cc|cc|cc}
%     \toprule
%     & \multicolumn{2}{c|}{Poem(comparison)} & \multicolumn{2}{c|}{Character Setting} & \multicolumn{2}{c|}{Story-Telling} & \multicolumn{2}{c}{BASIC CODE} \\
%     \midrule
%     & ASR. & Harm. & ASR. & Harm. & ASR. & Harm. & ASR. & Harm. \\
%     \midrule
%     GPT-4o-mini & 6.3\% & 1.09 &   70.2\% & 3.44   & 96.3\% & 4.75 & 91.9\% & 4.44 \\
%     \bottomrule
%     \end{tabular}
% \end{table*}

% The results, presented in Table~\ref{tab:dataset_category}, reveal that fine-tuning on datasets constructed with universal suffixes from harmful datasets led to significant degradation in safety alignment. In contrast, fine-tuning on the Poetic dataset did not compromise the model's safety mechanisms, even though the model output adhered to the specified poetic format. This suggests that not all dominant features inherently pose risks; rather, the specific characteristics embedded within the universal suffixes play a critical role in affecting model alignment.


% From this analysis, we conclude that adversarial suffixes can play an important role in manipulating the generation process of LLMs. Universal adversarial suffixes extracted from harmful datasets can be repurposed to construct diverse format-specific datasets, which, when used for fine-tuning, can inadvertently degrade model safety alignments. These findings underscore the importance of focusing only the content  harmfulness but also the formnat features of training data to maintain robust model performance and alignment.



    
\centering
    \subfigure[Response Similarity with Parametric Knowledge.]
    {
     \label{fig:sim_2_pm_ans} \includegraphics[width=0.46\linewidth]{figs/sim_to_pm_ans.pdf} }  
    \hspace{0.005\linewidth} 
    \subfigure[Response Similarity with Contextual Answer.] 
    { 
    \label{fig:sim_2_context_ans} 
    \includegraphics[width=0.46\linewidth]{figs/sim_to_ans.pdf}
    }  

    \subfigure[Perplexity w/o Context.] 
    { 
    \label{fig:ppl_wo_context} 
    \includegraphics[width=0.46\linewidth]{figs/ppl_wo_context.pdf}
    }  
    \hspace{0.005\linewidth} 
    \subfigure[Perplexity w/ Context.] 
    { 
    \label{fig:ppl_w_context} 
    \includegraphics[width=0.46\linewidth]{figs/ppl_w_context.pdf}
    }  






    % \includegraphics[width=0.48\textwidth]{figs/diff_model_double.pdf}
  \caption{Evaluation of knowledge utilization of different models. We assess the response similarity with parametric knowledge and contextual knowledge (Figure~\ref{fig:sim_2_pm_ans} and Figure~\ref{fig:sim_2_context_ans}), and compute the Perplexity (PPL) score when reproducing the ground truth answer (Figure~\ref{fig:ppl_wo_context} and Figure~\ref{fig:ppl_w_context}). The ``Uninstall'' model refers to \method{} w/o Adaption, which only incorporates the knowledge uninstallation.
  }
  \label{fig:analysis}
\end{figure}

First, we compute the semantic similarity between the outputs of different models and two knowledge sources--parametric knowledge from the model (Figure~\ref{fig:sim_2_pm_ans}) and contextual answers (Figure~\ref{fig:sim_2_context_ans})--to analyze their knowledge preference. Additionally, the performance of vanilla KAG model is provided as a reference. As shown in Figure~\ref{fig:sim_2_pm_ans}, compared to vanilla LLM, the Uninstall model exhibits the lowest similarity with parametric knowledge among all models, indicating that the knowledge uninstallation process effectively reduces the LLM's reliance on internal memory. Figure~\ref{fig:sim_2_context_ans} further compares the similarity between LLM responses and the contextual answers. \method{} achieves a significantly higher similarity score with contextual answers than other models, demonstrating its ability to effectively guide LLMs in leveraging external knowledge through knowledge-augmented adaptation.

To further investigate the knowledge utilization of different models, we ask each model to reproduce the ground truth answers and calculate the Perplexity (PPL) score both without (Figure~\ref{fig:ppl_wo_context}) and with (Figure~\ref{fig:ppl_w_context}) contextual knowledge. A lower PPL score indicates that the model is more confident to produce contextual answers. When external knowledge is not provided, the Uninstall model shows a significant increase in PPL, while the vanilla LLM maintains a relatively low PPL in the absence of context, showcasing the effectiveness of knowledge uninstallation in freeing the LLM's internal knowledge storage. In contrast, \method{} reaches an ``Inf'' PPL score without contextual knowledge provided but demonstrates a significant reduction in PPL score when external knowledge is provided. This highlights the effectiveness of our knowledge-augmented adaptation module in optimizing the LLM's reliance on external context for knowledge utilization.


\begin{figure}[t!]
    \centering
    \includegraphics[width=0.93\linewidth]{figs/act_llama_step2.pdf}
    \caption{Neuron activation across different layers. We present the absolute inhibition ratio $|\Delta R|$ under two conditions: when the input incorporates context knowledge (w/ context) and when it does not (w/o context).}
    \label{fig:llama3-8b-neuron_activation}
\end{figure}


\subsection{Neuron Activation in LLMs} 
As shown in Figure~\ref{fig:llama3-8b-neuron_activation}, we visualize the ratio of activated neurons and the absolute inhibition ratio $
|\Delta R|$ (Eq.~\ref{eq:difference}) in LLaMA3-8B-Instruct. The neuron activation ratios of different LLMs are provided in Appendix~\ref{app:activation}.

The results reveal a significant reduction in overall neuron activation levels when external context is provided. This reduction likely suggests that certain neurons associated with parametric knowledge become inhibited in the presence of external knowledge. We further observe that these inhibited neurons are predominantly concentrated in the upper layers of the model, which aligns with prior findings that factual knowledge is predominantly stored in the upper layers of transformer-based models~\cite{geva2020transformer,wang2024knowledge}. 
While parametric knowledge plays a crucial role in generating responses, it may introduce risks when it is outdated or conflicts with external information provided by KAG, potentially degrading the KAG performance. This work explores a pruning-based approach to mitigate the impact of parametric memory by removing these neurons that are inactive after feeding contextual knowledge, offering a new perspective on mitigating knowledge conflicts.

\section{Summary and Conclusion}
\label{sec:conclusion}


In this paper, we introduced \ToolName{}, a method for discovering fine-grained \emph{sub-activities} from unlabeled smart home sensor data without relying on pre-segmentation. Our pipeline is organized into two core steps: Clustering and Labeling. 
The \textbf{Clustering step} consists of:

\begin{itemize}
    \item \textbf{Encoder Pre-Training:} We leverage a pre-trained BERT model adapted with sensor-specific tokens and train it using a masked language modeling (MLM) objective to generate context-rich embeddings for raw sensor sequences.
    
    \item \textbf{Clustering Model Fine-Tuning:} Using the SCAN loss function, we fine-tune these embeddings to form more homogeneous and distinct clusters of sensor sequences.
\end{itemize}

The \textbf{Labeling step} comprises:

\begin{itemize}
    \item \textbf{Cluster Centroid Annotation:} Representative sequences from each cluster are visualized with a custom tool, enabling expert annotators to assign meaningful sub-activity labels to the centroids.
    
    \item \textbf{Label Propagation:} The centroid labels are propagated to all sequences within their respective clusters, resulting in a fully labeled dataset with minimal manual effort.
    
    \item \textbf{Re-annotation of Original Time-Series Data:} 
    Finally, these propagated labels are mapped back onto the original time-series data, preserving temporal continuity and facilitating the analysis of longitudinal activity patterns.
\end{itemize}


Our approach addresses important challenges in HAR, including the high cost and effort of manual data annotation, the limitations of coarse activity labels, and the need for scalable and generalizable models. \ToolName{} offers an open source tool that facilitates the HAR annotation and re-annotation process and enables the dynamic discovery and validation of sub-activities, thus capturing a broader spectrum of behaviors observed in real homes.
\section*{Limitation}
This paper uninstalls knowledge by removing entire FFN layers after identifying those with greater neuron inhibition. While \method{} demonstrates promising results, the pruning process may possibly affect some non-knowledge-related neurons. Regarding evaluation, we have employed ConR and MemR metrics, consistent with established practices in the field. Although these rule-based metrics provide valuable insights, we acknowledge that more sophisticated evaluation methodologies could offer more comprehensive evaluations. 

\section*{Ethics Statement}
Our data construction process involves prompting LLMs to elicit their parametric knowledge for studying knowledge conflicts. This process may result in some hallucinated content. We commit to the careful distribution of the data generated through our research, ensuring it is strictly used for research purposes. Our goal is to encourage responsible use of LLMs while advancing understanding of knowledge conflicts. Additionally, no personally identifiable information or offensive content is included in our dataset. We adhere to ethical guidelines for responsible AI research and data sharing.

We also employed human evaluation to assess the reliability of GPT-4o-mini in identifying knowledge conflicts. Evaluation data was carefully distributed to human evaluators solely for research purposes, ensuring it adheres to ethical standards and contains no content that violates these standards.

\bibliography{acl_latex}

\clearpage
\appendix
\appendix
\begin{table}[t!]
  \centering
  


% \renewcommand{\arraystretch}{1.2} % 调整行高
% \setlength{\tabcolsep}{10pt}  % 调整列间距
\resizebox{0.48\textwidth}{!}{%
\begin{tabular}{lcrr}
        \toprule
        \textbf{Dataset} & \textbf{Full Size*} & \textbf{Consistency}  & \textbf{\dataset{}} \\
        \midrule
        HotpotQA  & 5,901 & 2,973 {\footnotesize \textcolor{gray}{(50\%)}}  & 1,476 {\footnotesize \textcolor{gray}{(25\%)}}  \\
        NewsQA    & 4,212 & 1,260 {\footnotesize \textcolor{gray}{(30\%)}} & 934  {\footnotesize \textcolor{gray}{(22\%)}}  \\
        NQ        & 7,314 & 4,419 {\footnotesize \textcolor{gray}{(60\%)}}  & 1,479 {\footnotesize \textcolor{gray}{(20\%)}}  \\
        SearchQA  & 16,980 & 12,133 {\footnotesize \textcolor{gray}{(71\%)}} & 1,497 {\footnotesize \textcolor{gray}{(9\%)}}  \\
        SQuAD     & 10,490 & 5,024 {\footnotesize \textcolor{gray}{(48\%)}}  & 2,351 {\footnotesize \textcolor{gray}{(22\%)}}  \\
        TriviaQA  & 7,785 & 6654 {\footnotesize \textcolor{gray}{(85\%)}}  & 792  {\footnotesize \textcolor{gray}{(10\%)}}  \\
        \bottomrule
    \end{tabular}
}




 \caption{Number of instances at each stage in the \dataset{} construction pipeline.}
 \label{tab:our_bench_stats_each_step}
\end{table}
\section{Appendix}
\subsection{License}
We present the licenses of the datasets used in this study: Natural Questions (CC BY-SA 3.0 license), NewsQA (MIT License), SearchQA and TriviaQA (Apache License 2.0), HotpotQA and SQuAD (CC BY-SA 4.0 license).

All these licenses and agreements permit the use of their data for academic purposes.

\subsection{Details of Data Constructing}
\label{append:prompts}
In this section, we detail the two main steps in constructing \dataset{}. The dataset sizes at each stage of the pipeline are shown in Table~\ref{tab:our_bench_stats_each_step}.


\textbf{Parametric Knowledge Elicitation.} First, we elicit the LLM's parametric knowledge by prompting it in a closed-book setting (i.e., without any context). To ensure the reliability of the elicited knowledge, we apply a consistency-based filtering method. Specifically, for each query, the LLM is prompted five times, and the frequency of each response is recorded. The response with the highest frequency is identified as the majority answer. Queries where the majority answer appears fewer than three times are discarded, in order to filter out inconsistent responses and enhance data quality. The following prompt is used to instruct the LLM:
\begin{tcolorbox}
[title=Prompt for eliciting parametric knowledge,colback=blue!10,colframe=blue!50!black,arc=1mm,boxrule=1pt,left=1mm,right=1mm,top=1mm,bottom=1mm]
Answer the question \textcolor{blue}{\{\textit{brevity\_instruction}\}} and provide supporting evidence.

Question: \textcolor{blue}{\{\textit{question}\}}
\end{tcolorbox}
\noindent The ``\textit{brevity\_instruction}'' is used to guide the LLM to generate responses in a more concise form.

\textbf{Conflict Data Selection.} Next, we filter the data to retain only instances where the LLM's parametric knowledge directly conflicts with the contextual answer. Specifically, we categorize the data obtained from the previous step into two groups, conflicting and non-conflicting instances, based on the detailed results of conflict detection. All non-conflicting instances are discarded. GPT-4o-mini is then used to detect the presence of a conflict, using the following prompt:

\begin{tcolorbox}
[title=Prompt for identifying conflict knowledge,colback=blue!10,colframe=blue!50!black,arc=1mm,boxrule=1pt,left=1mm,right=1mm,top=1mm,bottom=1mm]
\small
You are tasked with evaluating the correctness of a model-generated answer based on the given information. 

\small
Context: \textcolor{blue}{\{\textit{context}\}}

Question: \textcolor{blue}{\{\textit{question}\}}

Contextual Answer: \textcolor{blue}{\{\textit{contextual\_answer}\}}

Model-Generated Answer: \textcolor{blue}{\{\textit{Model-Generated\_answer}\}}

\textcolor{blue}{[\textit{Detailed task description...}]}

Output Format:

Evaluate result: (Correct / Partially Correct / Incorrect) 
\end{tcolorbox}




\subsection{Assessing the Reliability of GPT-4o-mini in Knowledge Conflict Identification}
\label{append:human_eval}
In this subsection, we conduct the human evaluation to assess the reliability of GPT-4o-mini in identifying knowledge conflicts, which is a critical task in our data construction process to guarantee the data quality.

We randomly sampled 100 examples from each of the six subsets of \dataset{}, yielding a total of 600 samples. Six senior computational linguistics researchers were then asked to evaluate whether a knowledge conflict was present in each example. For each instance, the evaluators were provided with the question, the contextual answer, the model-generated response, and the corresponding supporting evidence. The results were classified into three categories: No Conflict, Somewhat Conflict, and High Conflict. The detailed annotation instructions are as follows:

\begin{tcolorbox}
[title=Annotation Instruction,colback=blue!10,colframe=blue!50!black,arc=1mm,boxrule=1pt,left=1mm,right=1mm,top=1mm,bottom=1mm]
\small
You are tasked with determining whether the parametric knowledge of LLMs conflicts with the given context to facilitate the study of knowledge conflicts in large language models.

Each data instance contains the following fields: 

Question: \textcolor{blue}{\{\textit{question}\}}


Answers: \textcolor{blue}{\{\textit{answers}\}}


Context: \textcolor{blue}{\{\textit{context}\}}

Parametric\_knowledge: \textcolor{blue}{\{\textit{LLMs' parametric\_knowledge }\}} 

The annotation process consists of two steps. 

\textbf{Step 1}: Compare the model-generated answer with the ground truth answers, based on the given question and context, to determine whether the model’s parametric knowledge conflicts with the context.

\textbf{Step 2}: Classify the results into one of three categories: 

\textcolor{blue}{\{\textit{No Conflict}\}} if the model-generated answer is consistent with the ground truth answers and context, 

\textcolor{blue}{\{\textit{Somewhat Conflict}\}}  if it is partially inconsistent

\textcolor{blue}{\{\textit{High Conflict}\}} if it significantly contradicts the ground truth answers or context.
\end{tcolorbox}


The evaluation results, shown in Table~\ref{tab:append_human_eval}, reveal a high level of agreement between the human annotators and GPT-4o-mini. Over 85\% of the examples reach consensus among the annotators, with an average agreement rate of 85.6\% across all subsets. These findings underscore the reliability of GPT-4o-mini as an effective tool for identifying knowledge conflicts.




\begin{table}[t]
  \centering
  
\centering
\begin{tabular}{l c}
\toprule
\textbf{Subset} & \textbf{Agreement (\%)} \\ \midrule
HotpotQA        & 81.4                        \\
NewsQA          & 72.7                        \\
NQ              & 88.7                        \\
SearchQA        & 95.3                        \\
SQuAD           & 86.1                        \\
TriviaQA        & 90.7                        \\ \midrule
\textbf{Average} & \textbf{85.6}            \\ \bottomrule
\end{tabular}

 \caption{Agreement between human annotators and GPT-4o-mini across different subsets of our \dataset{} benchmark.}
 \label{tab:append_human_eval}
\end{table}



\subsection{Evaluating the Effectiveness of Our Consistency-Based Filtering Method}
\label{append:data_freq}

In this subsection, we evaluate the effectiveness of our consistency-based knowledge conflict filtering method. As described in Appendix~\ref{append:prompts}, for each query, we prompt the model five times and record the most frequently generated answer along with its occurrence frequency. Based on this frequency, we divide the data into sub-datasets, where all queries within each sub-dataset share the same answer frequency. We then apply ``Conflict Data Selection'' to each sub-dataset, retaining only instances where knowledge conflicts occur. Finally, we evaluate ConR and MemR on these sub-datasets.

As shown in Figure~\ref{fig:diff_freq}, a clear trend emerges: as answer frequency increases, ConR consistently decreases, while MemR increases. This pattern indicates that as answer frequency rises, the model becomes increasingly reliant on its internal knowledge. Notably, for data with an answer frequency of 1, MemR is only 3\%, indicating minimal dependence on internal knowledge. Retaining only high-answer-frequency data improves the quality of \dataset{}. This data construction approach distinguishes our methodology from previous studies~\cite{longpre2021entity,xie2023adaptive}.

\begin{figure}[t!]
  \centering
  \includegraphics[width=0.4\textwidth]{figs/diff_freq.pdf}
  \caption{Performance comparison of ConR and MemR across sub-datasets grouped by the answer frequency of LLMs.}
  \label{fig:diff_freq}
\end{figure}





\subsection{Additional Implementation Details of Our Experiments}
\label{append:implementation}
This subsection outlines the training prompt, describes more details of the training data, and provides details of the experimental setup used in our experiments.

\textbf{Training Prompts.}
We adopt a simple QA-format training prompt following~\citet{zhou2023context} for all methods except \attrprompt{} and \oiprompt{}.
\begin{tcolorbox}
[title=Base Prompt ,colback=blue!10,colframe=blue!50!black,arc=1mm,boxrule=1pt,left=1mm,right=1mm,top=1mm,bottom=1mm]
% \small
\textcolor{blue}{\{\textit{context}\}} 
Q: \textcolor{blue}{\{\textit{question}\}} ? 
A: \textcolor{blue}{\{\textit{answer}\}}.
\end{tcolorbox}


\textbf{Training Datasets.} During \method{}, we randomly sample 32,580 instances from the training set of the MRQA 2019 benchmark~\cite{fisch2019mrqa} to construct our training data.



\textbf{Experimental Setup.} In this work, all models are trained for 2,100 steps with a total batch size of 32 and a learning rate of 1e-4. To enhance training efficiency, we implemented \method{} with LoRA~\cite{hu2021lora}, setting both the rank $\text{r}$ and scaling factor $\text{alpha}$ to 64. For \method{}, we set $\alpha$ to 0.1 (Eq.~\ref{eq:selct_layers}), which determines the minimum activation ratio difference required for a layer to be pruned. Additionally, we adopt a dynamic $\gamma$ in $\mathcal{L}_{\text{KC}}$ (Eq.~\ref{eq:kc_loss}), which linearly transitions from an initial margin ($\gamma_{0}=1$) to a final margin ($\gamma^*=5$) as training progresses. This adaptive strategy gradually reduces the model's reliance on internal parametric knowledge, encouraging it to rely more on external knowledge provided by the KAG system.


\subsection{Implementation Details of Baselines}
\label{append:baseline}
This subsection describes the implementation details of all baseline methods.

We adopt two prompt-based baselines: the attributed prompt ($\text{Attr}_{\text{prompt}}$) and a combination of opinion-based and instruction-based prompts ($\text{O\&I}_{\text{prompt}}$). The corresponding prompt templates are as follows:

\begin{tcolorbox}
[title=Attr based prompt ,colback=blue!10,colframe=blue!50!black,arc=1mm,boxrule=1pt,left=1mm,right=1mm,top=1mm,bottom=1mm]
% \small
\textcolor{blue}{\{\textit{context}\}} Q: \textcolor{blue}{\{\textit{question}\}} based on the given text? A: \textcolor{blue}{\{\textit{answer}\}}.
\end{tcolorbox}

\begin{tcolorbox}
[title=O\&I based prompt ,colback=blue!10,colframe=blue!50!black,arc=1mm,boxrule=1pt,left=1mm,right=1mm,top=1mm,bottom=1mm]

Bob said ``\textcolor{blue}{\{\textit{context}\}}'' Q: \textcolor{blue}{\{\textit{question}\}} in Bob's opinion? A: \textcolor{blue}{\{\textit{answer}\}}.
\end{tcolorbox}
For the SFT baseline, we incorporate context during training, similar to \method{}, while keeping the remaining experimental settings identical. To construct preference pairs for DPO training, we use contextually aligned answers from the dataset as ``preferred responses'' to ensure the consistency with the provided context. The ``rejected responses'' are generated by identifying parametric knowledge conflicts through our data construction methodology (Sec.~\ref{sec:benchmark}).

For KAFT, we employ a hybrid dataset containing both counterfactual and factual data. Specifically, we integrate the counterfactual data developed by \citet{xie2023adaptive}, leveraging their advanced data construction framework.

By maintaining equivalent dataset sizes and ensuring comparable data quality across all baselines, we provide a rigorous and fair comparison with our proposed \method{}.




\subsection{Extending \method{} to More LLMs}
\label{append:diff_model_performance}


\begin{figure}[t!]
  \centering
  
\subfigure[ConR Results]{
        \label{fig:diff_model:llama_conr}
        \includegraphics[width=0.462\linewidth]{append_fig/llama_conr.pdf}
    }
    \hspace{0.0005\linewidth} 
    \subfigure[MemR Results]{
        \label{fig:diff_model:llama_memr}
        \includegraphics[width=0.462\linewidth]{append_fig/llama_memr.pdf}
    }


  % \includegraphics[width=0.48\textwidth]{figs/diff_model_double.pdf}
 \caption{Average ConR and MemR across different models implemented by LLMs of LLaMA series, before and after applying \method{}.
 }
 \label{fig:diff_model_double_llama}
\end{figure}

\begin{figure}[t]
  \centering
  \subfigure[ConR Results]{
        \label{fig:diff_model:qwen_conr}
        \includegraphics[width=0.462\linewidth]{append_fig/qwen_conr.pdf}
    }
    \hspace{0.0005\linewidth} 
    \subfigure[MemR Results]{
        \label{fig:diff_model:qwen_memr}
        \includegraphics[width=0.462\linewidth]{append_fig/qwen_memr.pdf}
    }
  % \includegraphics[width=0.48\textwidth]{figs/diff_model_double.pdf}
 \caption{Average ConR and MemR across different models implemented by LLMs of Qwen series, before and after applying \method{}.
 }
 \label{fig:diff_model_double_qwen}
\end{figure}






We extend \method{} to a diverse range of LLMs, encompassing multiple model families and sizes. 

Specifically, our evaluation includes LLaMA3-8B-Instruct, LLaMA3.2-1B-Instruct, LLaMA3.2-3B-Instruct, Qwen2.5-0.5B-Instruct, Qwen2.5-1.5B-Instruct, Qwen2.5-3B-Instruct, Qwen2.5-7B-Instruct, and Qwen2.5-14B-Instruct. The results on ConR and MemR are summarized in Figures~\ref{fig:diff_model_double_llama} and \ref{fig:diff_model_double_qwen}, while Table~\ref{tab:append:all_model_res} presents the average performance of all models on \dataset{} and ConFiQA. Additionally, Table~\ref{tab:diff_model_param} provides detailed parameter information and specifies the layers selected for pruning for each model. This comprehensive evaluation demonstrates the versatility and scalability of \method{} across a wide spectrum of model architectures and sizes.

\begin{table}[!t]
  
    \resizebox{0.48\textwidth}{!}{%
\begin{tabular}{l|c|c|c}
\toprule
\textbf{Models}     & \textbf{Param.} & \textbf{\method{} Param.} & \textbf{Selected Layers} \\
\midrule
\rowcolor{gray!10}
LLaMA3.2-1B        & 1.24B  & 1.08B \small\textcolor{gray}{(87\%)}   & [12, 14]                 \\
LLaMA3.2-3B        & 3.21B  & 2.60B \small\textcolor{gray}{(81\%)}   &  [18, 25]   \\
\rowcolor{gray!10}
LLaMA3-8B          & 8.03B  & 6.97B \small\textcolor{gray}{(87\%)}   & [24, 29]      \\
LLaMA3.1-8B          & 8.03B  & 6.27B \small\textcolor{gray}{(78\%)}   & [20, 29]      \\
\rowcolor{gray!10}
Qwen2.5-0.5B         & 0.49B  & 0.44B \small\textcolor{gray}{(90\%)}   &  [19, 22]       \\
Qwen2.5-1.5B         & 1.54B  & 1.34B \small\textcolor{gray}{(87\%)}   & [21, 25]        \\
\rowcolor{gray!10}
Qwen2.5-3B         & 3.09B  & 2.68B \small\textcolor{gray}{(87\%)}   & [29, 34]        \\
Qwen2.5-7B         & 7.61B  & 7.21B \small\textcolor{gray}{(95\%)}   &   [25, 26 ]     \\
\rowcolor{gray!10}
Qwen2.5-14B        & 14.70B & 12.43B \small\textcolor{gray}{(85\%)}  &  [35, 45]   \\
\bottomrule
\end{tabular}
}

% \end{sidewaystable}

% \end{document}

  \caption{The total number of parameters for various models before and after applying \method{}. \textcolor{gray}{\small$(\cdot)\%$} represents the proportion relative to the original model, and the last column lists the layers selected for pruning.}
   \label{tab:diff_model_param}
\end{table}

These experimental results illustrate several key insights: 1) Larger models tend to rely more on parametric memory. As model size increases in both the LLaMA and Qwen families, MemR also grows, indicating a tendency to overlook external knowledge in favor of internal parameters. \method{} counteracts this behavior, decreasing larger models' MemR score to even below that of smaller models. 2) \method{} consistently benefits all evaluated models. Across both LLaMA and Qwen model families, \method{} outperforms Vanilla-KAG by boosting accuracy and context faithfulness, underscoring its broad applicability and effectiveness. 3) Not all parameters in KAG models are essential. Pruning parametric knowledge not only reduces computation costs but also fosters better generalization without sacrificing accuracy, highlighting the potential of building a parameter-efficient LLM within the KAG framework.




\begin{table*}[!t]
  
\centering
\resizebox{0.96\textwidth}{!}{%
\begin{tabular}{l|c|cccc|cccc}
\toprule
\multirow{2}{*}{\textbf{Models}} & \multirow{2}{*}{\textbf{Param.}} & \multicolumn{4}{c|}{\textbf{\dataset{}}} & \multicolumn{4}{c}{\textbf{ConFiQA}} \\ 
\cmidrule(lr){3-6}  \cmidrule(lr){7-10}
 &  & ConR $\uparrow$ & MemR $\downarrow$ & MR $\downarrow$ & EM $\uparrow$ & ConR $\uparrow$ & MemR $\downarrow$ & MR $\downarrow$ & EM $\uparrow$ \\ 
\midrule
LLaMA3-8B   & 8.03B  & 66.99  & 11.75  & 14.99  & 13.83  & 22.52  & 31.15  & 59.77  & 2.47 \\
\rowcolor{gray!10}
+\method{}    & 6.97B  & 71.50  & 6.48   & 8.41   & 66.19  & 70.43  & 8.82   & 11.32  & 67.29 \\
LLaMA3.1-8B & 8.03B  & 63.15  & 11.69  & 15.93  & 21.85  & 15.38  & 29.97  & 68.98  & 6.69 \\
\rowcolor{gray!10}
+\method{}   & 6.27B  & 70.41  & 6.95   & 9.17   & 63.58  & 71.12  & 9.01   & 11.44  & 66.61 \\
LLaMA3.2-1B & 1.24B  & 39.06  & 10.49  & 21.83  & 5.13   & 32.09  & 18.32  & 36.28  & 7.15 \\
\rowcolor{gray!10}
+\method{}   & 1.08B  & 51.75  & 6.51   & 11.34  & 47.60  & 62.70  & 7.63   & 11.38  & 61.85 \\
LLaMA3.2-3B & 3.21B  & 56.75  & 11.53  & 17.11  & 12.69  & 26.16  & 23.47  & 49.05  & 9.84 \\
\rowcolor{gray!10}
+\method{}   & 2.60B  & 67.00  & 6.80   & 9.35   & 61.59  & 69.61  & 8.39   & 11.09  & 66.53 \\
Qwen2.5-0.5B & 0.49B  & 47.17  & 11.36  & 19.48  & 2.06   & 50.72  & 17.15  & 26.20  & 3.78 \\
\rowcolor{gray!10}
+\method{}   & 0.44B  & 58.13  & 6.63   & 10.41  & 52.56  & 67.54  & 8.04   & 11.03  & 66.33 \\
Qwen2.5-1.5B & 1.54B  & 58.08  & 11.28  & 16.48  & 10.30  & 51.69  & 19.87  & 28.23  & 10.78 \\
\rowcolor{gray!10}
+\method{}   & 1.34B  & 63.78  & 6.74   & 9.76   & 57.67  & 69.61   & 8.35   & 11.05   & 66.04 \\
Qwen2.5-3B   & 3.09B  & 62.22  & 14.45  & 18.88  & 0.10   & 25.47  & 29.34  & 55.70  & 0.01 \\
\rowcolor{gray!10}
+\method{}     & 2.68B  & 66.31  & 6.75   & 9.38   & 59.42  & 66.30   & 8.62  & 11.94   & 63.03 \\
Qwen2.5-7B    & 7.61B  & 65.46  & 14.93  & 18.57  & 0.80   & 24.75  & 33.09  & 59.04  & 0.10 \\
\rowcolor{gray!10}
+\method{}      & 6.60B  & 67.75  & 6.60   & 9.01   & 61.77  & 69.54  & 8.85   & 11.58  & 66.68 \\
Qwen2.5-14B   & 14.70B & 65.75  & 16.13  & 19.75  & 0.00   & 7.86   & 32.88  & 83.71  & 0.01 \\
\rowcolor{gray!10}
+\method{}     & 12.43B & 70.01  & 6.43   & 8.55   & 64.43  & 71.70  & 8.90   & 11.29  & 68.40 \\
\bottomrule
\end{tabular}%
}


  \caption{Average performance of LLMs on \dataset{} and ConFiQA before and after applying \method{}.}
   \label{tab:append:all_model_res}
\end{table*}

\subsection{Neuron Activations in Different LLMs}\label{app:activation}
We present the neuron activations for the LLaMA family models, including LLaMA-3.2-1B-Instruct, LLaMA-3.2-3B-Instruct, LLaMA-3-8B-Instruct, and LLaMA-3.1-8B-Instruct, as well as the Qwen family models, including Qwen-2.5-0.5B-Instruct, Qwen-2.5-1.5B-Instruct, Qwen-2.5-3B-Instruct, Qwen-2.5-7B-Instruct, and Qwen-2.5-14B-Instruct, in Figures~\ref{fig:act_llama} and \ref{fig:act_qwen}, respectively. 
% 我们发现qwen系列模型


\begin{figure*}[t]
  \centering
  \subfigure[Neuron activations of LLaMA-3.2-1B-Instruct]{
        \label{fig:act_llama:3.2-1b}
        \includegraphics[width=0.9\linewidth]{append_fig/act_llama32_1b_all.pdf}
    }
\subfigure[Neuron activations of LLaMA-3.2-3B-Instruct]{
        \label{fig:act_llama:3.2-3b}
        \includegraphics[width=0.9\linewidth]{append_fig/act_llama32_3b_all.pdf}
    }
 \subfigure[Neuron activations of LLaMA-3-8B-Instruct]{
        \label{fig:act_llama:3-8b}
        \includegraphics[width=0.9\linewidth]{append_fig/act_llama_3_8b.pdf}
    }
 \subfigure[Neuron activations of LLaMA-3.1-8B-Instruct]{
        \label{fig:act_llama:3.1-8b}
        \includegraphics[width=0.9\linewidth]{append_fig/act_llama_31_8b.pdf}
    }
 

 \caption{Neuron activations across different layers of the LLaMA series models. We present the inhibition ratio $\Delta R$ under two conditions: with contextual knowledge input (w/ context) and without it (w/o context).}
 \label{fig:act_llama}
\end{figure*}

\begin{figure*}[t]
  \centering
  \subfigure[Neuron activations of Qwen-2.5-0.5B-Instruct]{
        \label{fig:act_qwen:2.5-0.5b}
        \includegraphics[width=0.75\linewidth]{append_fig/act_qwen25_0_5b_all.pdf}
    }
\subfigure[Neuron activations of Qwen-2.5-1.5B-Instruct]{
        \label{fig:act_qwen:2.5-1.5b}
        \includegraphics[width=0.75\linewidth]{append_fig/act_qwen25_1_5b_all.pdf}
    }
\subfigure[Neuron activations of Qwen-2.5-3B-Instruct]{
        \label{fig:act_qwen:2.5-3b}
        \includegraphics[width=0.75\linewidth]{append_fig/act_qwen25_3b_all.pdf}
    }
\subfigure[Neuron activations of Qwen-2.5-7B-Instruct]{
        \label{fig:act_qwen:2.5-7b}
        \includegraphics[width=0.75\linewidth]{append_fig/act_qwen25_7b_all.pdf}
    }
\subfigure[Neuron activations of Qwen-2.5-14B-Instruct]{
        \label{fig:act_qwen:2.5-14b}
        \includegraphics[width=0.75\linewidth]{append_fig/act_qwen25_14b_all.pdf}
    }


 \caption{Neuron activations across different layers of the Qwen series models. We present the inhibition ratio $\Delta R$ under two conditions: with contextual knowledge input (w/ context) and without it (w/o context). }
 \label{fig:act_qwen}
\end{figure*}


\end{document}
