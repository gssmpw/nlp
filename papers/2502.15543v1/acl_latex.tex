% This must be in the first 5 lines to tell arXiv to use pdfLaTeX, which is strongly recommended.
\pdfoutput=1
% In particular, the hyperref package requires pdfLaTeX in order to break URLs across lines.

\documentclass[11pt]{article}

% Change "review" to "final" to generate the final (sometimes called camera-ready) version.
% Change to "preprint" to generate a non-anonymous version with page numbers.
% \usepackage[review]{acl}
\usepackage[table]{xcolor} 

\usepackage[final]{acl}

% Standard package includes
\usepackage{times}
\usepackage{latexsym}

% For proper rendering and hyphenation of words containing Latin characters (including in bib files)
\usepackage[T1]{fontenc}
% For Vietnamese characters
% \usepackage[T5]{fontenc}
% See https://www.latex-project.org/help/documentation/encguide.pdf for other character sets

% This assumes your files are encoded as UTF8
\usepackage[utf8]{inputenc}

% This is not strictly necessary, and may be commented out,
% but it will improve the layout of the manuscript,
% and will typically save some space.
\usepackage{microtype}

% This is also not strictly necessary, and may be commented out.
% However, it will improve the aesthetics of text in
% the typewriter font.
\usepackage{inconsolata}

%Including images in your LaTeX document requires adding
%additional package(s)
\usepackage{graphicx}


% Ours
\usepackage{amssymb}
\usepackage{amsfonts}
\usepackage{bm}       % 提供加粗数学符号的命令
\usepackage{amsmath}  % 引入amsmath包
\usepackage{enumitem}  % 列表管理
\usepackage{pifont}
\usepackage{adjustbox} % 对其bar

\usepackage{enumitem}
\usepackage{booktabs} % For professional looking tables
% \usepackage{float}

% \usepackage{ulem}
\usepackage{CJKutf8}
\usepackage{booktabs} 
\usepackage{multirow}
% \usepackage{dashrule}
\usepackage{tabularx}
\usepackage{makecell}
\usepackage{siunitx}

% framework
% \def\mylogo{\scalerel*{\includegraphics{figures/fire.png}}{X}}
\usepackage{tcolorbox}

% \usepackage{subcaption}
\usepackage{subfigure}

% \def\mylogo{\resizebox{0.45cm}{!}{\includegraphics{figures/fire.png}}} % 宽度为2cm,高度按比例缩放
% \def\mylogosnow{\resizebox{0.45cm}{!}{\includegraphics{figures/snow.png}}} % 宽度为2cm,高度按比例缩放
% \def\mylogonarrow{\resizebox{0.5cm}{!}{\includegraphics{figures/narrow.jpg}}} % 宽度为2cm,高度按比例缩放
\newlength{\hseg}
\setlength{\hseg}{0.6cm}

\usepackage{scalerel} % 用于 \scalerel 命令
% \usepackage{hyperref}
\definecolor{reasoning-side}{RGB}{167,210,165}
\definecolor{reasoning-inside}{RGB}{206,222,186}
\definecolor{lang-side}{RGB}{159,164,248}
\definecolor{lang-inside}{RGB}{217,217,252}
\usepackage{tikz}
\newcommand{\inlinecolorboxone}{
    \tikz[baseline=-0.1cm]{
        \node[draw=lang-side, fill=lang-inside, minimum width=0.4cm, minimum height=0.1cm, line width=0.5pt, rounded corners=1pt] (char1) {};
    }
}
\newcommand{\inlinecolorboxtwo}{
    \tikz[baseline=-0.1cm]{
        \node[draw=reasoning-side, fill=reasoning-inside, minimum width=0.4cm, minimum height=0.1cm, line width=0.5pt, rounded corners=1pt] (char2) {};
    }
}

%figures
\usepackage{pgfplots}
\definecolor{tiffanyblue}{RGB}{129,216,208}
\definecolor{bangdiblue}{RGB}{0,149,182}
\definecolor{kleinblue}{RGB}{0,47,167}
\definecolor{purple}{RGB}{138,43,226}
\usetikzlibrary{shapes}
\usepackage{circledtext}

\usetikzlibrary{arrows,decorations.pathmorphing,backgrounds,positioning,fit,petri}
\usetikzlibrary{arrows.meta,fit,shapes.arrows}
% \usepackage{caption} 
\usepackage{pgfplots}
\usepackage{tikz}
\usetikzlibrary{positioning,shapes,arrows,shadows,patterns}
\usetikzlibrary{calc}
% \usepackage{tkz-kiviat,pgfplots}
\pgfplotsset{compat=newest}




\usepackage{xspace}


\newcommand{\eg}{\emph{e.g.}}
\newcommand{\hpc}[1]{\textcolor{purple}{[hpc: #1]}}
\newcommand{\ysj}[1]{\textcolor{red}{[ysj: #1]}}

\def\method{PIP-KAG}
\def\dataset{CoConflictQA} % Knowledge conFliCt qa
\def\attrprompt{$\text{Attr}_{\text{prompt}}$}
\def\oiprompt{$\text{O\&I}_{\text{prompt}}$}
% ====================================
\usepackage[normalem]{ulem}


\title{\includegraphics[width=1.25em]{figs/pip.pdf}-KAG: Mitigating Knowledge Conflicts in Knowledge-Augmented Generation via Parametric Pruning}




\author{Pengcheng Huang$^{1}$, Zhenghao Liu$^{1}$\thanks{ \ \ indicates corresponding author.}, Yukun Yan$^{2}$, Xiaoyuan Yi$^{3}$, \\ 
\textbf{Hao Chen$^{2}$, Zhiyuan Liu$^{2}$, Maosong Sun$^{2}$, Tong Xiao$^{1}$, Ge Yu$^{1}$, Chenyan Xiong$^{4}$} \\
$^1$Department of Computer Science and Technology, Northeastern University, China \\
$^2$Department of Computer Science and Technology, Institute for AI, Tsinghua University, China \\
$^3$Microsoft Research Asia, Beijing, China \\
$^4$Language Technologies Institute, Carnegie Mellon University, United States
}


\begin{document}
\maketitle

\begin{abstract}


The choice of representation for geographic location significantly impacts the accuracy of models for a broad range of geospatial tasks, including fine-grained species classification, population density estimation, and biome classification. Recent works like SatCLIP and GeoCLIP learn such representations by contrastively aligning geolocation with co-located images. While these methods work exceptionally well, in this paper, we posit that the current training strategies fail to fully capture the important visual features. We provide an information theoretic perspective on why the resulting embeddings from these methods discard crucial visual information that is important for many downstream tasks. To solve this problem, we propose a novel retrieval-augmented strategy called RANGE. We build our method on the intuition that the visual features of a location can be estimated by combining the visual features from multiple similar-looking locations. We evaluate our method across a wide variety of tasks. Our results show that RANGE outperforms the existing state-of-the-art models with significant margins in most tasks. We show gains of up to 13.1\% on classification tasks and 0.145 $R^2$ on regression tasks. All our code and models will be made available at: \href{https://github.com/mvrl/RANGE}{https://github.com/mvrl/RANGE}.

\end{abstract}


\section{Introduction}

Video generation has garnered significant attention owing to its transformative potential across a wide range of applications, such media content creation~\citep{polyak2024movie}, advertising~\citep{zhang2024virbo,bacher2021advert}, video games~\citep{yang2024playable,valevski2024diffusion, oasis2024}, and world model simulators~\citep{ha2018world, videoworldsimulators2024, agarwal2025cosmos}. Benefiting from advanced generative algorithms~\citep{goodfellow2014generative, ho2020denoising, liu2023flow, lipman2023flow}, scalable model architectures~\citep{vaswani2017attention, peebles2023scalable}, vast amounts of internet-sourced data~\citep{chen2024panda, nan2024openvid, ju2024miradata}, and ongoing expansion of computing capabilities~\citep{nvidia2022h100, nvidia2023dgxgh200, nvidia2024h200nvl}, remarkable advancements have been achieved in the field of video generation~\citep{ho2022video, ho2022imagen, singer2023makeavideo, blattmann2023align, videoworldsimulators2024, kuaishou2024klingai, yang2024cogvideox, jin2024pyramidal, polyak2024movie, kong2024hunyuanvideo, ji2024prompt}.


In this work, we present \textbf{\ours}, a family of rectified flow~\citep{lipman2023flow, liu2023flow} transformer models designed for joint image and video generation, establishing a pathway toward industry-grade performance. This report centers on four key components: data curation, model architecture design, flow formulation, and training infrastructure optimization—each rigorously refined to meet the demands of high-quality, large-scale video generation.


\begin{figure}[ht]
    \centering
    \begin{subfigure}[b]{0.82\linewidth}
        \centering
        \includegraphics[width=\linewidth]{figures/t2i_1024.pdf}
        \caption{Text-to-Image Samples}\label{fig:main-demo-t2i}
    \end{subfigure}
    \vfill
    \begin{subfigure}[b]{0.82\linewidth}
        \centering
        \includegraphics[width=\linewidth]{figures/t2v_samples.pdf}
        \caption{Text-to-Video Samples}\label{fig:main-demo-t2v}
    \end{subfigure}
\caption{\textbf{Generated samples from \ours.} Key components are highlighted in \textcolor{red}{\textbf{RED}}.}\label{fig:main-demo}
\end{figure}


First, we present a comprehensive data processing pipeline designed to construct large-scale, high-quality image and video-text datasets. The pipeline integrates multiple advanced techniques, including video and image filtering based on aesthetic scores, OCR-driven content analysis, and subjective evaluations, to ensure exceptional visual and contextual quality. Furthermore, we employ multimodal large language models~(MLLMs)~\citep{yuan2025tarsier2} to generate dense and contextually aligned captions, which are subsequently refined using an additional large language model~(LLM)~\citep{yang2024qwen2} to enhance their accuracy, fluency, and descriptive richness. As a result, we have curated a robust training dataset comprising approximately 36M video-text pairs and 160M image-text pairs, which are proven sufficient for training industry-level generative models.

Secondly, we take a pioneering step by applying rectified flow formulation~\citep{lipman2023flow} for joint image and video generation, implemented through the \ours model family, which comprises Transformer architectures with 2B and 8B parameters. At its core, the \ours framework employs a 3D joint image-video variational autoencoder (VAE) to compress image and video inputs into a shared latent space, facilitating unified representation. This shared latent space is coupled with a full-attention~\citep{vaswani2017attention} mechanism, enabling seamless joint training of image and video. This architecture delivers high-quality, coherent outputs across both images and videos, establishing a unified framework for visual generation tasks.


Furthermore, to support the training of \ours at scale, we have developed a robust infrastructure tailored for large-scale model training. Our approach incorporates advanced parallelism strategies~\citep{jacobs2023deepspeed, pytorch_fsdp} to manage memory efficiently during long-context training. Additionally, we employ ByteCheckpoint~\citep{wan2024bytecheckpoint} for high-performance checkpointing and integrate fault-tolerant mechanisms from MegaScale~\citep{jiang2024megascale} to ensure stability and scalability across large GPU clusters. These optimizations enable \ours to handle the computational and data challenges of generative modeling with exceptional efficiency and reliability.


We evaluate \ours on both text-to-image and text-to-video benchmarks to highlight its competitive advantages. For text-to-image generation, \ours-T2I demonstrates strong performance across multiple benchmarks, including T2I-CompBench~\citep{huang2023t2i-compbench}, GenEval~\citep{ghosh2024geneval}, and DPG-Bench~\citep{hu2024ella_dbgbench}, excelling in both visual quality and text-image alignment. In text-to-video benchmarks, \ours-T2V achieves state-of-the-art performance on the UCF-101~\citep{ucf101} zero-shot generation task. Additionally, \ours-T2V attains an impressive score of \textbf{84.85} on VBench~\citep{huang2024vbench}, securing the top position on the leaderboard (as of 2025-01-25) and surpassing several leading commercial text-to-video models. Qualitative results, illustrated in \Cref{fig:main-demo}, further demonstrate the superior quality of the generated media samples. These findings underscore \ours's effectiveness in multi-modal generation and its potential as a high-performing solution for both research and commercial applications.
% 

\section{M3DA Benchmark}
\label{sec:bench}


We consider a semantic segmentation problem of 3D medical images, which we call a downstream task. Any downstream model works with input samples $x \in X$ and the corresponding segmentation masks $y \in Y$, where $X$ and $Y$ are some input image and label spaces. If $x \in \R^{H \times W \times D}$, segmentation mask is of the same spatial size $y \in \R^{H \times W \times D}$, where every element belongs to a predefined set of labels $y^{(h,w,d)} \in \{ 0, 1, \dots, C \}$, $0$ is background and $C$ is the number of foreground classes.




\begin{table*}[ht]
    \centering
    \caption{Comparison to the existing benchmark (CrossMoDA) and datasets that are commonly used for Domain Adaptation in 3D medical image segmentation. Our proposed benchmark (M3DA) covers all primary domain shifts with the largest publicly available datasets.}

    \resizebox{\textwidth}{!}{%
    \begin{tabular}{lccccc}
        \toprule
        \textbf{Paper} & \textbf{Domain shifts} & \textbf{Datasets} & \textbf{Modalities} & \textbf{Region of interest} \\
        \midrule
        Jiang at al., 2020 \cite{jiang2020unified} & inter-modality & BTCV, CHAOS & MRI, CT &  4 thoracic organs  \\
        Al et al., 2021 \cite{al2021olva} & inter-modality & MM-WHS & MRI, CT & heart \\
        Weihsbach at al., 2024 \cite{dg_tta} & inter-modality & AMOS, MM-WHS,  spine & MRI, CT & 15 thoracic organs, brain, spine \\
        Liu et al., 2020 \cite{liu2020shape}             & MRI settings, scanners & 6 public datasets      & MRI              &  prostate         \\
        Gu et al., 2021 \cite{gu2021domain}             & MRI settings, scanners & SAML                   & MRI              &  prostate         \\
        Chen et al., 2022 \cite{chen2022maxstyle}         & MRI settings, scanners & ACDC, M\&Ms            & MRI              &  heart            \\
        Lennartz et al., 2023 \cite{lennartz2023segmentation} & MRI settings, scanners & CC359, ACDC, M\&Ms     & MRI              &  brain, heart     \\
        Wong et al., 2023 \cite{wong2023hartleymha}       & MRI settings, scanners & BraTS                  & MRI              &  brain            \\
        Liu et al., 2022 \cite{liu2022act}               & MRI inter-sequence & BraTS                  & MRI              &  brain            \\
        CrossMoDA, 2023 \cite{crossmoda}                & MRI inter-sequence & Vestibular Schwannoma  & MRI              &  brain            \\
        
        \midrule
        
        \multirow{2}{*}{M3DA (ours)} & inter-modality, inter-sequence & \multirow{2}{*}{AMOS, CC359, BraTS, LIDC} & \multirow{2}{*}{MRI, CT} & brain, tumors, \\
        & CT and MRI settings, scanners, contrast & & & 15 thoracic organs \\
        
        \bottomrule
    \end{tabular}}
    
    \label{tab:benchmarks}
\end{table*}




We follow the standard unsupervised domain adaptation (UDA) problem setting, as in \cite{dann}. We assume that two distributions $\mathcal{S}(x, y)$ and $\mathcal{T}(x, y)$ exist on $X \otimes Y$, called \textit{source} and \textit{target} distributions. At the training time, we have a set of source training samples $X^s = \{ x_i^s \}_{i=1}^n$ with the corresponding masks $Y^s = \{ y_i^s \}_{i=1}^n$ and a set of target training samples $X^t_{tr}$ without annotations; source images and masks are considered to be sampled from $\mathcal{S}$, $(x_i^s, y_i^s) \sim \mathcal{S}(x, y)$. Our goal is to predict segmentations $y$ given the input from the marginal distribution of target images, $x \sim \mathcal{T}(x)$. To evaluate algorithms, we have target testing samples $X^t_{ts}$ with masks $Y^t_{ts}$ \textit{available only for evaluation purposes}.


Thus, given domains A and B, one trains a supervised model on domain A while having access to unannotated samples from domain B for adaptation. The goal of UDA is to develop a model that makes accurate predictions on domain B. Importantly, this setup prohibits incorporating annotations from the target domain into the training routine.


\subsection{UDA problems motivation}
\label{ssec:constructing}




\begin{table}[ht]
    \centering
    \caption{Summary of the datasets commonly used in DA studies. ROI stands for region of interest, and \#cls denotes the number of foreground segmentation classes. Several datasets do not contain inner domain shifts, i.e., \textit{single source}; they are used in multi-dataset setups.}

    \resizebox{\linewidth}{!}{%
    \begin{tabular}{@{}lcccc@{}}
    \toprule
    \textbf{Dataset} & \textbf{Modality} & \textbf{ROI} & \textbf{\#cls} & \textbf{\#cases} \\
    \midrule
    WMH \cite{wmh} & MRI & Brain & 2 & 60 \\
    BraTS \cite{brats} & MRI & Brain & 3 & 1251 \\
    CC359 \cite{cc359} & MRI & Brain & 5 & 359 \\
    ACDC \cite{bernard2018deep} & MRI & Heart & 2 & 150 \\
    M\&Ms \cite{campello2021multi} & MRI & Heart & 6 & 375 \\
    SCGM \cite{prados2017spinal} & MRI & Spinal Cord & 2 & 80 \\
    IVDM3Seg \cite{IVDM3Seg2018} & MRI & Spinal Cord & 2 & 16 \\
    BTCV \cite{btcv} & CT & Abdomen & 13 & 30 \\
    AMOS \cite{amos} & CT, MRI & Abdomen & 15 & 500, 100 \\
    CHAOS \cite{kavur2021chaos} & CT, MRI & Thorax & 4 & 50 \\
    MM-WHS \cite{zhuang2016multi} & CT, MRI & Heart & 8 & 20, 20 \\
    \bottomrule
    \end{tabular}}
    \label{tab:datasets}
\end{table}


\paragraph{CT \textbf{$\leftrightarrow$} MRI}
First, we include domain shift from MRI to CT and vice versa. Although the use of CT scans is often clinically justified, it is associated with additional risks, such as potentially increasing the risk of cancer \cite{cao2022ct,brenner2007computed}. In contrast, MRI is a safer imaging modality that does not involve radiation exposure \cite{nie2016estimating}. While CT is critical for various clinical applications like radiotherapy treatment planning, there is a recent transition to MRI for these applications \cite{paczona2023magnetic}. Thus, developing algorithms that use decades of collected CT data and adapt them for newly acquired MRI scans is an important avenue of research.


The inverse problem of estimating MRI from CT is also an important application. CT is a much faster imaging modality compared to MRI, making it a better solution in emergency scenarios such as stroke. However, MRI provides more sensitive brain visualization \cite{vymazal2012comparison}. Therefore, having universal algorithms that can adapt to the needed modality at hand is highly beneficial.


While these examples highlight the clinical relevance of domain adaptation between CT and MRI, for the purpose of this benchmark, we utilize a different dataset focusing on thoracic organ segmentation. This choice is motivated by the availability of a dataset that provides both MR and CT images with corresponding segmentation maps for thoracic organs, which is essential for evaluating the performance of UDA algorithms. Despite the difference in the target application, the underlying principles of domain adaptation remain the same, and the insights gained from this benchmark can be applied to various clinical scenarios.


\paragraph{CT $\rightarrow$ low-dose CT}Second, we include a CT to low-dose CT (LDCT) shift, motivated by the increasing popularity of LDCT. LDCT produces images with a lower signal-to-noise ratio but are still diagnostically effective, resulting in several-fold lower radiation dosage exposure compared to regular CT (allowing for screening purposes \cite{lidc,kubo2016standard}), faster scanning time, mobility to scan underserved populations \cite{raghavan2020initial}, and cost-effectiveness \cite{mohammadshahi2019cost}. Similar to the CT $\leftrightarrow$ MRI domain shift, utilizing publicly available annotated regular CT scans can accelerate the development of automated segmentation models for LDCT. As demonstrated in Table \ref{tab:metrics_pure}, methods trained on regular CT perform poorly on LDCT. This shift is the only one obtained via simulation, where we algorithmically simulate low-dose CT from regular ones.

 % \reconsider{no contrast enhancement}} Why?
\paragraph{Contrast enhancement $\rightarrow$ no contrast enhancement} Third, we include two tasks, MRI and CT, involving domain transfer from a contrast-enhanced (CE) image to an image without contrast enhancement (native). CE injection is a labor-intensive step, requiring additional training for personnel and carrying a small but additional risk for patients \cite{andreucci2014side,costelloe2020risks}. Again, we suggest benchmarking DA methods against the scenario where models that utilized richer imaging modalities (CE) during supervised training are adapted for safer modalities (non-contrast-enhanced).


\paragraph{MRI settings} Finally, we include three setups that address the domain shifts caused by variable MRI scanner settings, which are among the most common sources of domain shift encountered in practice \cite{yan2020mri,medim_da_survey_2023}. These setups cover different field strengths (1.5T vs. 3T), different scanner manufacturers, and a combination of both. Domain shifts arising from variations in scanner settings are ubiquitous in multi-source MRI datasets, as differences in field strength and manufacturer-specific acquisition parameters can significantly impact the appearance and quality of the resulting images. Addressing these shifts is crucial for developing robust and generalizable segmentation models that can handle the heterogeneity of MRI data encountered in real-world clinical scenarios.


\subsection{Datasets selection}
\label{ssec:data}
% \reconsider{Intro sentence, may be add link to some medim datasets summary paper/github, comment on what is necessary to use dataset for DA setup (annotation, metadata)}
We base the inclusion of datasets into the benchmark on two criteria. \textbf{Relevance}, we aim to cover as many relevant domain shifts as possible; see Section~\ref{ssec:limitations} for a list of domain shifts not included in our benchmark. \textbf{Scale}, we prefer a dataset with a larger number of samples, when deciding between two datasets that are both relevant and include similar domain shifts. 
% \textbf{Economy}, we aim to cover selected shifts with a preferably smaller number of datasets.
All reviewed and selected datasets are summarized in Table~\ref{tab:datasets}, and all technical details (e.g., links, download instructions, and licenses) are provided in Supplementary materials.


\begin{table}[h]
    \centering
    \caption{Details of eight tasks in M3DA benchmark. The last three columns correspond to the numbers of cases in \textit{source}, \textit{target train}, and \textit{target test} folds, respectively.}

    \resizebox{\linewidth}{!}{%
    \begin{tabular}{@{}lccccc@{}}
        \toprule
        \textbf{Task name} & \textbf{Domain shift} & \textbf{Dataset} & $|X^s|$ & $|X^t_{tr}|$ & $|X^t_{ts}|$ \\
        \midrule
        MR$\ra$CT & inter-modality & AMOS & 60 & 200 & 150 \\
        CT$\ra$MR & inter-modality & AMOS & 150 & 40 & 60 \\
        CT$\ra$LDCT & CT settings & AMOS & 150 & 200 & 150 \\
        CE CT $\ra$ CT & settings (contrast) & LIDC & 289 & 297 & 297 \\
        % T1$\ra$T2 & intra-modality & BraTS & 625 & 313 & 313 \\
        % T2$\ra$T1 & intra-modality & BraTS & 625 & 313 & 313 \\
        % T1$\ra$T1c & settings (contrast) & BraTS & 625 & 313 & 313 \\
        T1 CE$\ra$T1 & settings (contrast) & BraTS & 625 & 313 & 313 \\
        T1 F & MRI settings & CC359 & 30 & 30 & 30 \\
        % T1 Sc & MRI scanners & CC359 & 30 & 20 & 21 \\  % add unlbl
        T1 Sc & MRI scanners & CC359 & 30 & 30 & 30 \\
        T1 Mix & settings, scanners & CC359 & 29 & 30 & 30 \\
        \bottomrule
    \end{tabular}}
    
    \label{tab:setup}
\end{table}


We start by selecting a dataset for MRI to CT conversion. This allows for several alternatives. Many authors use datasets such as BTCV and CHAOS for these tasks, both of which include images of the thoracic region. BTCV consists of 30 CT scans with 13 organ annotations, and CHAOS of 40 MRI and 40 CT scans with 4 organ annotations. Another option is MM-WHS, which consists of 20 MRI and 20 CT scans of the heart with 8 annotated classes. Finally, there is the newer AMOS dataset, which consists of 500 CT and 100 MRI scans with 15 annotated thoracic organs. Following our criteria, we include AMOS as it is the largest option. We also use AMOS to simulate LDCT data.

To cover CE CT to native CT task, we add Lung Image Database Consortium image collection (LIDC) \cite{lidc}. LIDC contains chest CT images with and without contrast enhancement with segmentation annotation of lung nodules. LIDC dataset covers lung nodules - an oncology pathology, one of the most common reasons for using contrast enhancement \cite{purysko2016does}. Then, we cover the similar CE-based data shift in multi-sequence MRI data, from T1 CE to T1 modality. BraTS 2021, being one of the largest and most widely used datasets in the medical imaging community, emerges as the natural choice, satisfying our criteria of relevance, and scale.

Finally, we cover variability in the single-sequence MRI acquisition. As evident from Tables \ref{tab:benchmarks} and \ref{tab:datasets}, common choices are the ACDC and M\&Ms datasets. Both include segmentation classes of the heart and are relatively large, consisting of 150 and 375 annotated samples, respectively. Both have several concretely defined MRI domains (e.g., different scanners, parameters, field strengths). Another option is CC359, which has the same rich variability in MRI parameters and is similarly sized, including 359 annotated samples. Both ACDC and M\&Ms images have $1\times 1\times 9~ \text{mm}^3$ spacing, for which a 2D algorithm would often be a more viable choice. In contrast, a significant advantage of CC359 is the fine-grade and consistent voxel spacing, approximately $1\times 1\times 1~ \text{mm}^3$, which concludes our selection. UDA setups from the selected datasets are summarized in Table~\ref{tab:setup}.

\section{Related Work}

\subsection{Large 3D Reconstruction Models}
Recently, generalized feed-forward models for 3D reconstruction from sparse input views have garnered considerable attention due to their applicability in heavily under-constrained scenarios. The Large Reconstruction Model (LRM)~\cite{hong2023lrm} uses a transformer-based encoder-decoder pipeline to infer a NeRF reconstruction from just a single image. Newer iterations have shifted the focus towards generating 3D Gaussian representations from four input images~\cite{tang2025lgm, xu2024grm, zhang2025gslrm, charatan2024pixelsplat, chen2025mvsplat, liu2025mvsgaussian}, showing remarkable novel view synthesis results. The paradigm of transformer-based sparse 3D reconstruction has also successfully been applied to lifting monocular videos to 4D~\cite{ren2024l4gm}. \\
Yet, none of the existing works in the domain have studied the use-case of inferring \textit{animatable} 3D representations from sparse input images, which is the focus of our work. To this end, we build on top of the Large Gaussian Reconstruction Model (GRM)~\cite{xu2024grm}.

\subsection{3D-aware Portrait Animation}
A different line of work focuses on animating portraits in a 3D-aware manner.
MegaPortraits~\cite{drobyshev2022megaportraits} builds a 3D Volume given a source and driving image, and renders the animated source actor via orthographic projection with subsequent 2D neural rendering.
3D morphable models (3DMMs)~\cite{blanz19993dmm} are extensively used to obtain more interpretable control over the portrait animation. For example, StyleRig~\cite{tewari2020stylerig} demonstrates how a 3DMM can be used to control the data generated from a pre-trained StyleGAN~\cite{karras2019stylegan} network. ROME~\cite{khakhulin2022rome} predicts vertex offsets and texture of a FLAME~\cite{li2017flame} mesh from the input image.
A TriPlane representation is inferred and animated via FLAME~\cite{li2017flame} in multiple methods like Portrait4D~\cite{deng2024portrait4d}, Portrait4D-v2~\cite{deng2024portrait4dv2}, and GPAvatar~\cite{chu2024gpavatar}.
Others, such as VOODOO 3D~\cite{tran2024voodoo3d} and VOODOO XP~\cite{tran2024voodooxp}, learn their own expression encoder to drive the source person in a more detailed manner. \\
All of the aforementioned methods require nothing more than a single image of a person to animate it. This allows them to train on large monocular video datasets to infer a very generic motion prior that even translates to paintings or cartoon characters. However, due to their task formulation, these methods mostly focus on image synthesis from a frontal camera, often trading 3D consistency for better image quality by using 2D screen-space neural renderers. In contrast, our work aims to produce a truthful and complete 3D avatar representation from the input images that can be viewed from any angle.  

\subsection{Photo-realistic 3D Face Models}
The increasing availability of large-scale multi-view face datasets~\cite{kirschstein2023nersemble, ava256, pan2024renderme360, yang2020facescape} has enabled building photo-realistic 3D face models that learn a detailed prior over both geometry and appearance of human faces. HeadNeRF~\cite{hong2022headnerf} conditions a Neural Radiance Field (NeRF)~\cite{mildenhall2021nerf} on identity, expression, albedo, and illumination codes. VRMM~\cite{yang2024vrmm} builds a high-quality and relightable 3D face model using volumetric primitives~\cite{lombardi2021mvp}. One2Avatar~\cite{yu2024one2avatar} extends a 3DMM by anchoring a radiance field to its surface. More recently, GPHM~\cite{xu2025gphm} and HeadGAP~\cite{zheng2024headgap} have adopted 3D Gaussians to build a photo-realistic 3D face model. \\
Photo-realistic 3D face models learn a powerful prior over human facial appearance and geometry, which can be fitted to a single or multiple images of a person, effectively inferring a 3D head avatar. However, the fitting procedure itself is non-trivial and often requires expensive test-time optimization, impeding casual use-cases on consumer-grade devices. While this limitation may be circumvented by learning a generalized encoder that maps images into the 3D face model's latent space, another fundamental limitation remains. Even with more multi-view face datasets being published, the number of available training subjects rarely exceeds the thousands, making it hard to truly learn the full distibution of human facial appearance. Instead, our approach avoids generalizing over the identity axis by conditioning on some images of a person, and only generalizes over the expression axis for which plenty of data is available. 

A similar motivation has inspired recent work on codec avatars where a generalized network infers an animatable 3D representation given a registered mesh of a person~\cite{cao2022authentic, li2024uravatar}.
The resulting avatars exhibit excellent quality at the cost of several minutes of video capture per subject and expensive test-time optimization.
For example, URAvatar~\cite{li2024uravatar} finetunes their network on the given video recording for 3 hours on 8 A100 GPUs, making inference on consumer-grade devices impossible. In contrast, our approach directly regresses the final 3D head avatar from just four input images without the need for expensive test-time fine-tuning.


\section{Study Design}
% robot: aliengo 
% We used the Unitree AlienGo quadruped robot. 
% See Appendix 1 in AlienGo Software Guide PDF
% Weight = 25kg, size (L,W,H) = (0.55, 0.35, 06) m when standing, (0.55, 0.35, 0.31) m when walking
% Handle is 0.4 m or 0.5 m. I'll need to check it to see which type it is.
We gathered input from primary stakeholders of the robot dog guide, divided into three subgroups: BVI individuals who have owned a dog guide, BVI individuals who were not dog guide owners, and sighted individuals with generally low degrees of familiarity with dog guides. While the main focus of this study was on the BVI participants, we elected to include survey responses from sighted participants given the importance of social acceptance of the robot by the general public, which could reflect upon the BVI users themselves and affect their interactions with the general population \cite{kayukawa2022perceive}. 

The need-finding processes consisted of two stages. During Stage 1, we conducted in-depth interviews with BVI participants, querying their experiences in using conventional assistive technologies and dog guides. During Stage 2, a large-scale survey was distributed to both BVI and sighted participants. 

This study was approved by the University’s Institutional Review Board (IRB), and all processes were conducted after obtaining the participants' consent.

\subsection{Stage 1: Interviews}
We recruited nine BVI participants (\textbf{Table}~\ref{tab:bvi-info}) for in-depth interviews, which lasted 45-90 minutes for current or former dog guide owners (DO) and 30-60 minutes for participants without dog guides (NDO). Group DO consisted of five participants, while Group NDO consisted of four participants.
% The interview participants were divided into two groups. Group DO (Dog guide Owner) consisted of five participants who were current or former dog guide owners and Group NDO (Non Dog guide Owner) consisted of three participants who were not dog guide owners. 
All participants were familiar with using white canes as a mobility aid. 

We recruited participants in both groups, DO and NDO, to gather data from those with substantial experience with dog guides, offering potentially more practical insights, and from those without prior experience, providing a perspective that may be less constrained and more open to novel approaches. 

We asked about the participants' overall impressions of a robot dog guide, expectations regarding its potential benefits and challenges compared to a conventional dog guide, their desired methods of giving commands and communicating with the robot dog guide, essential functionalities that the robot dog guide should offer, and their preferences for various aspects of the robot dog guide's form factors. 
For Group DO, we also included questions that asked about the participants' experiences with conventional dog guides. 

% We obtained permission to record the conversations for our records while simultaneously taking notes during the interviews. The interviews lasted 30-60 minutes for NDO participants and 45-90 minutes for DO participants. 

\subsection{Stage 2: Large-Scale Surveys} 
After gathering sufficient initial results from the interviews, we created an online survey for distributing to a larger pool of participants. The survey platform used was Qualtrics. 

\subsubsection{Survey Participants}
The survey had 100 participants divided into two primary groups. Group BVI consisted of 42 blind or visually impaired participants, and Group ST consisted of 58 sighted participants. \textbf{Table}~\ref{tab:survey-demographics} shows the demographic information of the survey participants. 

\subsubsection{Question Differentiation} 
Based on their responses to initial qualifying questions, survey participants were sorted into three subgroups: DO, NDO, and ST. Each participant was assigned one of three different versions of the survey. The surveys for BVI participants mirrored the interview categories (overall impressions, communication methods, functionalities, and form factors), but with a more quantitative approach rather than the open-ended questions used in interviews. The DO version included additional questions pertaining to their prior experience with dog guides. The ST version revolved around the participants' prior interactions with and feelings toward dog guides and dogs in general, their thoughts on a robot dog guide, and broad opinions on the aesthetic component of the robot's design. 

\section{Benchmark}

Based on our scanners, scanning-pattern, and pose optimization procedures discussed in Sec.~\ref{sec:dataset}, we have prepared grouped sequences for benchmarking NVS methods on multi-lane scenarios. First of all, we discuss on implementation details of our selected NVS methods in Sec.~\ref{sec:bench:methods}, and then share our protocols and metrics chosen for cross-lane NVS application in Sec.~\ref{sec:bench:design}.

\subsection{Benchmarking Methods}
\label{sec:bench:methods}

% \noindent \textbf{Unified Settings:} 
We select a range of methods, specifically those designed for autonomous driving datasets and based on either NeRF or 3DGS, as our benchmarking methods.
For all methods discussed below, we use the combination of LiDAR $\mathcal{L}^\mathbb{G}$ and SfM $\mathcal{P}^\mathbb{G}$ points to initialize the Gaussians. 
In order to eliminate the influence of dynamic objects, we filter out all the cars and pedestrians in point clouds and images based on semantic labels.
Since the original 3DGS and most of its extensions only support focal points at the absolute center, we rectify the focal point to the center of corresponding images during the post-processing of our dataset.

\textbf{3DGS~\cite{Kerbl20233dgs}:} 
We utilize the implementation from the official release to evaluate our dataset and employ the AdamW optimizer with a learning rate of \(10^{-3}\). The model is trained for 30,000 steps.

\textbf{GaussianPro~\cite{cheng2024gaussianpro}:} 
We train the models for 30,000 iterations across all scenes, adhering to the original training schedule and hyperparameters. The interval step for the progressive propagation strategy is set to 20, with propagation performed three times. 

\textbf{Scaffold-GS~\cite{Lu2024scaffoldgs}:} 
Both the appearance and feature dimensions in the MLP are set to 32, and voxel size is set to 0.005. We adjust the initial and hierarchy factors for anchor growing to 16 and 4, respectively. The model is trained for 30,000 steps.

\textbf{2DGS~\cite{Huang20242DGS}:} 
We keep most parameters consistent with the original implementation. For densification, we adjust the gradient threshold to \(3 \times 10^{-4}\) and set the final densification iterations to 13,000. The model is trained for 30,000 steps.

\textbf{Street Gaussians~\cite{yan2024street}:} To ensure a fair comparison with other methods, we omit the handling of dynamic objects from the original scene graph method. Additionally, we do not utilize the sky mask for an equitable comparison. The parameters remain consistent with those in the official implementation.

\textbf{PVG~\cite{chen2023pvg}:} 
We utilize the Adam optimizer and keep a comparable learning rate for most parameters, consistent with the original implementation. We adjust the gradient threshold to \(3 \times 10^{-4}\) and set the final densification iterations to 13,000. The maximum number of Gaussian spheres is configured to \(10^{6}\). {We omit multi-resolution downsampling, using images at their original resolution for training.}

\textbf{EmerNeRF~\cite{yang2023emernerf}:} 
We train EmerNeRF for 30,000 iterations using its original parameters. The flow branch and temporal interpolation are activated, with both the feature levels of the hash encoder for the static and dynamic branches set to 4.

\subsection{Experimental Protocols and Metrics}
\label{sec:bench:design}


To perform a comprehensive benchmark of all the aforementioned methods on our proposed dataset, we meticulously group all sequences and organize the benchmarks across five different tracks for each method. Specifically, the tracks are categorized as follows: (1) Single lane regression, (2) Adjacent lane prediction, (3) Second-adjacent lane prediction, (4) Adjacent lane prediction (trained from two lanes), and (5) Sandwich lane prediction (trained from two side lanes). A figure illustrating the experimental protocols is provided in Fig.~\ref{fig:expset}. For each track, we uniformly sample 200 frames from training sequences for model learning and 25 frames from test sequences as the ground truth.
% As we have grouped these sequences recording parallel lanes in the same direction, and most road groups have at least 3 parallel lanes, we perform five NVS ability tests listed in Fig.~\ref{fig:expset} for each method: (1) Single lane regression. (2) Adjacent lane prediction. (3) Secondly-adjacent lane prediction. (4) Adjacent lane prediction (train from two lanes). (5) Sandwich lane prediction (train from two side lanes). 

We adhere to the widely used metrics for evaluating the performance of NVS as outlined in \cite{li2024xld}, which includes Peak Signal-to-Noise Ratio (PSNR), Structural Similarity Index (SSIM), and Learned Perceptual Image Patch Similarity (LPIPS).

\subsection{Experimental Results and Notes}
\label{sec:bench:main}

\begin{figure}[t]
\includegraphics[width=\linewidth]{sec/images/expset.pdf}
\caption{{Five evaluation tracks using different combinations of lanes for training (colored in blue) and testing (colored in red).}}
\label{fig:expset}
\end{figure}

\begin{table*}[t]
\caption{Quantitative results of different Gaussian reconstruction methods on our proposed Para-Lane dataset.}
\centering
{
\fontsize{8pt}{9.6pt}\selectfont
\setlength{\tabcolsep}{2.2pt}
\begin{tabular}{l|ccc|ccc|ccc|ccc|ccc}
\toprule
Method                  &   \hmerge{Single}     &  \hmerge{Adjacent}    &  \hmerge{Sec-Adj.}    &  \hmerge{Two-for-One} & \hmergl{Sandwich}       \\
Metrics                 &       \metrics        &      \metrics         &      \metrics         &      \metrics         &      \metrics           \\ 
\midrule
% EmerNeRF                &       &       &       &       &       &       &       &       &       &       &       &       &       &       &         \\
% \midrule
3DGS                    & \cccf{22.99} & \cccf{0.689} & \cccs{0.344} & \ccct{17.05} & \cccs{0.524} & \cccf{0.446} & \ccct{16.26} & \ccct{0.505} & \cccs{0.472} & \ccct{17.85} & \ccct{0.551} & \ccct{0.440} & 18.74 & \ccct{0.563} & \cccs{0.424}    \\
% PVG                     & 22.46 & 0.669 & 0.376 & 15.64 & 0.507 & 0.493 & 14.80 & 0.498 & 0.515 & 15.00 & 0.498 & 0.510 & 15.80 & 0.511 & 0.489    \\
GaussianPro             & \ccct{22.93} & \cccs{0.687} & \cccf{0.343} & 17.01 & 0.521 & \cccf{0.446} & \cccs{16.29} & \ccct{0.505} & \cccs{0.472} & 17.83 & \ccct{0.551} & \cccs{0.439} & 18.66 & 0.562 & \cccs{0.424}   \\
Scaffold-GS             & \cccs{22.96} & \ccct{0.675} & 0.364 & \cccf{17.59} & \cccf{0.538} & \ccct{0.450} & \cccf{17.09} & \cccf{0.525} & \cccf{0.470} & \cccf{18.62} & \cccf{0.565} & \cccf{0.437} & \cccf{19.20} & \cccf{0.574} & \cccf{0.423}   \\
2DGS                    & 22.29 & 0.651 & 0.395 & 16.79 & \ccct{0.523} & 0.469 & 16.01 & \cccs{0.510} & 0.494 & 17.46 & 0.548 & 0.466 & \cccs{19.04} & \cccs{0.572} & 0.451   \\
Street Gaussians        & 22.56 & 0.643 & \ccct{0.353} & \cccs{17.50} & 0.510 & 0.456 & 16.16 & 0.496 & 0.480 & \cccs{17.87} & \cccs{0.555} & 0.453 & \ccct{18.91} & 0.561 & 0.443   \\
\bottomrule
\end{tabular}
}

\label{tab:eval}\vspace{-10pt}
\end{table*}

\begin{figure*}[t]
\includegraphics[width=\linewidth]{sec/images/Exp-results.pdf}
\caption{Comparisons for NVS quality between different methods and designs, see our supplementary video for results on more sequences.}
\label{fig:eval}
\end{figure*}

We benchmark the methods described in Sec.~\ref{sec:bench:methods} according to the framework outlined in Sec.~\ref{sec:bench:design} to evaluate performance for cross-lane NVS. A series of experiments were conducted using an NVIDIA GeForce RTX 3090 24GB GPU. For quantitative metrics across all tested methods and different designs, refer to Tab.~\ref{tab:eval}. Throughout the experiments, we discovered several interesting insights:
\vspace{4pt}

\noindent \textbf{The quality of NVS is significantly affected by the view distribution of the training set.} 
From Tab.~\ref{tab:eval}, we found {\emph{exactly}} the same conclusion for all methods: the performance gradually decreases in the following sequence: Single $>$ Sandwich $>$ Two-for-One $>$ Adjacent $>$ Second-Adjacent. When the training and testing views are on the same trajectory, all methods achieve the best NVS results. However, when the testing viewpoint undergoes lateral shifts, the results are compromised to varying degrees. 
{Besides, although the number of images used for training is the same, our Sandwich {track}, which evenly distributes training views on both sides of the test views, consistently achieves superior rendering quality compared to the Two-for-One {track}, where training views are located only on one side of the testing views.}
This can be attributed to the fact that when training data is more evenly distributed and closer to the target render pose, the learned radiance functions are less likely to overfit to a specific viewpoint. 
Therefore, from an application perspective, to generate images of a target scene from arbitrary viewpoints, it is advisable to utilize multiple passes of data to reconstruct the target scene, thereby minimizing potential artifacts during novel view synthesis.
{Fig.~\ref{fig:eval} presents a visual comparison of novel view synthesis results under different designs, and our supplementary video provides additional examples. }

% \noindent \textbf{The quality of NVS is significantly affected by the view distribution of the training set.} 
% In our experiments, 
% This underscores the importance of gathering training images with a sufficiently even distribution of perspectives and training a scalable Gaussian model that accommodates such diversity when synthesizing novel views in real-world scenarios.

\noindent \textbf{Domain gap between synthetic~\cite{li2024xld} and real datasets.} Most of the methods tested here were also evaluated on the XLD dataset~\cite{li2024xld}, a synthetic cross-lane dataset; however, they generally did not perform as well on our dataset as they did on XLD. We attribute this difference to the domain gaps between synthetic and real-world data for several reasons. Firstly, synthetic data perfectly ensures the accuracy of all parameters, including extrinsic, intrinsic, and timestamps. However, in the real world, despite the optimizations discussed in Sec.~\ref{sec:dataset:slam}, there inevitably remains a gap between our final estimations and the ground truth values. Secondly, the sequences in a group were collected over different time intervals, leading to minor variations in brightness, water stain shapes, and other trivial factors among the cross-lane sequences. This also brings errors that require NVS approaches to handle. Finally, in real-world data, special attention must be given to dynamic objects. Unlike synthetic data, we cannot capture observations of the same dynamic object simultaneously across different locations. Although we used SAM to mask dynamic objects, the model's output is not always precise, and some noise remains.

\noindent \textbf{Scaffold-GS achieves the best NVS performance on our benchmark.} 
Unlike other methods, Scaffold-GS~\cite{Lu2024scaffoldgs} utilizes anchor points to distribute 3D Gaussians. Each anchor point is associated with multiple neural Gaussians. The attributes of these neural Gaussians—such as position, opacity, quaternion, scaling, and color—are determined by a multi-layer perceptron (MLP). The input features for the MLP include the relative poses from the anchor points to the camera views. Our cross-lane experimental results demonstrate that this method, which establishes correlations between view poses and rendering outcomes, is effective.

\subsection{Handling Dynamic Objects}

Besides the main experiment in Sec.~\ref{sec:bench:main} that reflects performance of methods in masked static scenes, we perform another experiment on those experiments capable of handling dynamic objects.

EmerNeRF~\cite{yang2023emernerf} and PVG~\cite{chen2023pvg} are two representative methods that contain procedures on handling dynamic objects in a self-supervised manner, we compare the two methods using {the Single track. We perform reconstruction through both methods with and without automatically labeled mask~\cite{Kirillov2023sam}}
The dataset used in this part includes 6 groups, which are part of the entire 25 groups dataset.

\textbf{Results and analysis.} From Tab.~\ref{tab:eval_dynamic}, we can find that EmerNeRF achieves better PSNR and LPIPS scores. This is probably because of its novel and effective approach of static-dynamic decomposition. On the other hand, PVG excels in SSIM for both tests. This exceptional performance is likely to due to the adaptable design for Gaussian points. 
Qualitative results are shown in Fig.~\ref{fig:pvg}. We can find that the two methods are comparable for nearby scenes. However, PVG results are inferior for distant locations, such as buildings and trees. This is likely due to that PVG advocates to utilize the larger points for faraway scenes as described in its draft~\cite{chen2023pvg}, which results in inadequate expression of details and cause blur.

\begin{table}[htbp]
\centering
{
\fontsize{8pt}{9.6pt}\selectfont
\setlength{\tabcolsep}{5pt}
% \fontsize{9pt}{9pt}\selectfont
% \setlength{\tabcolsep}{0.9pt}
\begin{tabular}{l|ccc}
\toprule
Method                       &   \hmergl{Single}     \\
Metrics                      &       \metrics        \\ 
\midrule
EmerNeRF                     & \cccf{23.67} & 0.668 & \cccf{0.350} \\
PVG                          & 22.08 & \cccf{0.672} & 0.408 \\
\midrule
EmerNeRF (static only)       & \cccf{23.76} & 0.678 & \cccf{0.346} \\
PVG (static only)            & 22.77 & \cccf{0.684} & 0.368 \\
\bottomrule
\end{tabular}
}
\caption{Quantitative results on our proposed Para-Lane dataset with dynamic objects. We perform reconstruction with and without mask for ablation study.}
\label{tab:eval_dynamic}
\end{table}

\begin{figure}[htbp]
\includegraphics[width=\linewidth]{sec/images/pvg-cross.pdf}
\caption{Comparisons for NVS quality in Single lane test between EmerNeRF and PVG.}
\label{fig:pvg}
\end{figure}


\subsection{Limitations}

This section describes the limitations of our current dataset and benchmark. The dynamic object masks provided in the dataset are not manually labeled, so there are a small number of omissions and mislabeling. We also do not yet have 3D bounding box and tracking labels for all dynamic objects. The diversity of the dataset can also be enhanced by collecting more data in the future.
Given the fact that current works rarely support dynamic/static decomposition, we only tested EmerNeRF~\cite{yang2023emernerf} and PVG~\cite{chen2023pvg} in handling dynamic objects. A more comprehensive benchmark for dynamic scenes would be beneficial as more dynamic methods are developed in the future.
% We have not yet conducted more studies about dynamic/static decomposition capabilities, which may involve methods such as PVG, EmerNerf, 4D gaussian~\cite{wu20244dgaussiansplattingrealtime}, etc.
\section{Experimental Methodology}
In this section, we describe the datasets, evaluation metrics, baselines and implementation details used in our experiments.


\textbf{Datasets.}
For our experiments, we use the \dataset{} dataset for both training and evaluation. In addition, we utilize the ConFiQA~\cite{bi2024context} dataset during evaluation, which serves as an out-of-domain test scenario to assess the generalization ability of different models. ConFiQA is designed to evaluate the context faithfulness of LLMs in adhering to counterfactual contexts. It consists of three subsets: Question-Answering, Multi-hop Reasoning, and Multi-Conflicts, each containing 6,000 instances.


\textbf{Evaluation.}
Following previous work~\cite{longpre2021entity, zhou2023context}, we employ multiple evaluation metrics to assess the generated responses from two aspects: correctness and context faithfulness. To ensure more accurate evaluations, we normalize both responses and answers according to the method described by \citet{li2024rag}.

For accuracy assessment, we use $\text{EM}(\uparrow)$, which measures whether the generated responses exactly match the ground truth answers.
To evaluate context faithfulness, we adopt two metrics: the recall of context (\text{ConR}$\uparrow$) and the recall of memory (\text{MemR}$\downarrow$). Specifically, \text{ConR} assesses whether the generated responses align with the provided context, while \text{MemR} assesses the alignment with parametric memories. Additionally, we adopt the memorization ratio $\text{MR}(\downarrow) = \frac{\text{MemR}}{\text{MemR} + \text{ConR}}$, which captures the tendency to rely on internal memory.

\begin{table*}[t!]
    
\centering
\resizebox{\textwidth}{!}{%
% \rowcolors{3}{gray!10}{white} % 交替背景色
\begin{tabular}{lc|cccc|cccc|cccc}
\toprule
% \rowcolor{gray!20}
\multirow{2}{*}{\textbf{Models}} & \multirow{2}{*}{\textbf{Param.}} & \multicolumn{4}{c|}{\textbf{HotPotQA}} & \multicolumn{4}{c|}{\textbf{NQ}} & \multicolumn{4}{c}{\textbf{NewsQA}} \\
\cmidrule(lr){3-6} \cmidrule(lr){7-10} \cmidrule(lr){11-14}

& &  ConR $\uparrow$ &  MemR $\downarrow$ &  MR $\downarrow$ &  EM $\uparrow$ &  ConR $\uparrow$ &  MemR $\downarrow$ &  MR $\downarrow$ & EM $\uparrow$ &  ConR $\uparrow$ &  MemR $\downarrow$ &  MR $\downarrow$ & EM $\uparrow$ \\
\midrule
\rowcolor{gray!10}
Vanilla-KAG~\shortcite{ram2023context} & 8.03B & 65.72 & 15.04 & 18.62 & 16.40 & 64.33 & 13.71 & 17.57 & 4.75 & 62.42 & 8.89 & 12.46 & 6.42 \\
\attrprompt{}~\shortcite{zhou2023context} & 8.03B & 65.31 & 15.58 & 19.26 & 6.23 & 66.24 & 11.55 & 14.85 & 0.65 & 62.74 & 8.35 & 11.75 & 3.85 \\
\rowcolor{gray!10}
\oiprompt{}~\shortcite{zhou2023context} & 8.03B & 60.98 & 15.58 & 20.35 & 2.44 & 64.41 & 9.76 & 13.15 & 1.27 & 62.10 & 8.46 & 11.99 & 2.46 \\
SFT~\shortcite{wei2021finetuned} & 8.03B & \uline{70.26} & \uline{7.05} & \uline{9.11} & \uline{64.84} & 68.45 & 8.57 & 11.13 & 67.31 & \uline{63.28} & 5.78 & 8.37 & 53.10 \\
\rowcolor{gray!10}
KAFT~\shortcite{li2022large} & 8.03B & 68.29 & 7.18 & 9.52 & 64.16 & \uline{68.86} & 8.37 & 10.84 & \uline{67.39} & 63.60 & \uline{5.35} & \uline{7.76} & \uline{53.75} \\
DPO~\shortcite{bi2024context}  & 8.03B & 67.21 & 8.74 & 11.51 & 50.20 & 67.55 & \textbf{7.76} & \uline{10.30} & 49.80 & 58.35 & \textbf{5.03} & 7.94 & 40.15 \\

\midrule
% our_{param.} & \textbf{6.97B} & \textbf{71.95} & \uline{6.71} & \uline{8.53} & \uline{65.79} & 61.05 & 9.87 & 13.92 & 59.03 & \textbf{65.95} & \uline{5.14} & \textbf{7.23} & 56.21 \\
% \rowcolor{gray!20}
\rowcolor{gray!10}
\method{} & \textbf{6.97B} & \textbf{71.95} & \textbf{6.44} & \textbf{8.21} & \textbf{67.41} & \textbf{69.88} & \uline{8.00} & \textbf{10.27} & \textbf{69.10} & \textbf{65.74} & 5.46 & \textbf{7.67} & \textbf{55.89} \\
\midrule
% \rowcolor{gray!20}

\multirow{2}{*}{\textbf{Models}} & \multirow{2}{*}{\textbf{Param.}} & \multicolumn{4}{c|}{\textbf{SearchQA}} & \multicolumn{4}{c|}{\textbf{SQuAD}} & \multicolumn{4}{c}{\textbf{TriviaQA}} \\
\cmidrule(lr){3-6} \cmidrule(lr){7-10} \cmidrule(lr){11-14}
% \rowcolor{gray!10}
& &  ConR $\uparrow$ &  MemR $\downarrow$ &  MR $\downarrow$ & EM $\uparrow$ &  ConR $\uparrow$ &  MemR $\downarrow$ &  MR $\downarrow$ & EM $\uparrow$ &  ConR $\uparrow$ &  MemR $\downarrow$ &  MR $\downarrow$ & EM $\uparrow$ \\
\midrule
\rowcolor{gray!10}
Vanilla-KAG~\shortcite{ram2023context} & 8.03B & 67.80 & 12.56 & 15.63 & 24.92 & 79.18 & 7.39 & 8.54 & 14.23 & \textbf{62.50} & 12.88 & 17.09 & 16.41 \\
\attrprompt{}~\shortcite{zhou2023context} & 8.03B & 65.87 & 12.22 & 15.65 & 8.68 & 78.93 & 7.60 & 8.79 & 4.38 & \uline{61.62} & 11.49 & 15.72 & 5.43 \\
\rowcolor{gray!10}
\oiprompt{}~\shortcite{zhou2023context} & 8.03B & 59.25 & 12.16 & 17.03 & 4.28 & \textbf{82.63} & 7.09 & 7.91 & 2.72 & 58.33 & 11.87 & 16.91 & 1.77 \\
SFT~\shortcite{wei2021finetuned} & 8.03B & \uline{75.28} & \uline{7.68} & \uline{9.26} & \uline{71.28} & 79.61 & 4.50 & 5.35 & 69.88 & 59.47 & 10.73 & 15.29 & 54.42 \\
\rowcolor{gray!10}
KAFT~\shortcite{li2022large} & 8.03B & 73.28 & 8.22 & 10.08 & 69.54 & 80.54 & \uline{4.21} & \uline{4.96} & \uline{71.03} & 60.10 & 9.72 & 13.92 & \uline{54.55} \\
DPO~\shortcite{bi2024context} & 8.03B & 66.67 & 8.02 & 10.73 & 51.24 & 78.93 & 5.01 & 5.97 & 52.63 & 58.84 & \uline{8.59} & \uline{12.73} & 44.95 \\
\midrule
% \rowcolor{gray!20}
% our_{param.} & 6.97B & 77.56 & 7.55 & 8.87 & 73.55 & 81.92 & 3.96 & 4.61 & 71.71 & 59.85 & 10.86 & 15.36 & 54.04 \\
\rowcolor{gray!10}
\method{} & \textbf{6.97B} & \textbf{78.22} & \textbf{7.01} & \textbf{8.23} & \textbf{75.68} & \uline{81.56} & \textbf{3.99} & \textbf{4.67} & \textbf{71.71} & \uline{61.62} & \textbf{7.95} & \textbf{11.43} & \textbf{57.32} \\
\bottomrule
\end{tabular}%
}

    \caption{Performance on the \dataset{} dataset. The highest scores are highlighted in \textbf{bold}, while the second-highest scores are \uline{underlined}. ``Param.'' refers to the total number of model parameters.}
     \label{tab:main_res_id}
\end{table*}

\begin{table*}[t!]
    
\centering
% \caption{Model performance on ConfiQA datasets.}
\resizebox{\textwidth}{!}{%
\begin{tabular}{lc|cccc|cccc|cccc}
\toprule

% \multirow{2}{*}{\textbf{Models}} & \multirow{2}{*}{\textbf{Param}} & \multicolumn{4}{c|}{\textbf{HotPotQA}} & \multicolumn{4}{c|}{\textbf{NQ}} & \multicolumn{4}{c}{\textbf{NewsQA}} \\
% \cmidrule(lr){3-6} \cmidrule(lr){7-10} \cmidrule(lr){11-14}


\multirow{2}{*}{\textbf{Models}} & \multirow{2}{*}{\textbf{Param.}} & \multicolumn{4}{c|}{\textbf{Question Answering}} & \multicolumn{4}{c|}{\textbf{Multi-hop Reasoning}} & \multicolumn{4}{c}{\textbf{Multi-Conflicts}}  \\
\cmidrule(lr){3-6} \cmidrule(lr){7-10} \cmidrule(lr){11-14}

& &  ConR $\uparrow$ &  MemR $\downarrow$ &  MR $\downarrow$ & EM $\uparrow$ &  ConR $\uparrow$ &  MemR $\downarrow$ &  MR $\downarrow$ & EM $\uparrow$ &  ConR $\uparrow$ &  MemR $\downarrow$ &  MR $\downarrow$ & EM $\uparrow$  \\
\midrule
\rowcolor{gray!10}
Vanilla-KAG~\shortcite{ram2023context} & 8.03B & 31.29 & 40.71 & 56.54 & 6.49 & 26.69 & 32.49 & 54.90 & 0.73 & 9.58 & 20.24 & 67.88 & 0.18  \\
\attrprompt{}~\shortcite{zhou2023context} & 8.03B & 47.71 & 28.40 & 37.31 & 0.84 & 27.56 & 30.02 & 52.14 & 0.02 & 9.33 & 19.16 & 67.24 & 0.07  \\
\rowcolor{gray!10}
\oiprompt{}~\shortcite{zhou2023context} & 8.03B & 62.64 & 15.96 & 20.30 & 10.04 & 22.11 & 23.31 & 51.32 & 0.20 & 12.89 & 16.11 & 55.56 & 0.25 \\
SFT~\shortcite{wei2021finetuned} & 8.03B & 79.36 & \uline{5.09} & 6.03 & \textbf{74.98} & 61.31 & \uline{13.84} & 18.42 & 63.44 & 62.89 & \uline{9.20} & 12.76 & 61.55   \\
\rowcolor{gray!10}
KAFT~\shortcite{li2022large} & 8.03B & \textbf{85.04} & 5.31 & \uline{5.88} & \uline{62.76} & \textbf{64.96} & 13.91 & \uline{17.64} & \textbf{67.91} & \textbf{66.27} & 9.42 & \uline{12.45} & \textbf{66.60}   \\
DPO~\shortcite{bi2024context} & 8.03B & \uline{80.71} & 5.69 & 6.58 & 61.00 & 63.38 & 14.00 & 18.09 & 47.96 & 64.87 & 9.84 & 13.18 & 46.90  \\

\midrule
\rowcolor{gray!10}
\method{} & \textbf{6.97B} & 80.56 & \textbf{4.51} & \textbf{5.30} & \textbf{74.98} & \uline{64.73} & \textbf{13.07} & \textbf{16.80} & \uline{65.07} & \uline{66.00} & \textbf{8.89} & \textbf{11.87} & \uline{61.84}   \\
\bottomrule
\end{tabular}%
}





% \centering
% % \caption{Model performance on ConfiQA datasets.}
% \resizebox{\textwidth}{!}{%
% \begin{tabular}{ll|cccc|cccc|cccc}
% \toprule
% \rowcolor{gray!20}
% \textbf{Model} & \textbf{Param} & \multicolumn{4}{c|}{\textbf{Confiqa\_QA}} & \multicolumn{4}{c|}{\textbf{Confiqa\_MR}} & \multicolumn{4}{c}{\textbf{Confiqa\_MC}} \\
% \cmidrule(lr){3-6} \cmidrule(lr){7-10} \cmidrule(lr){11-14}
% \rowcolor{gray!10}
% & & Pc & Po & MR & EM & Pc & Po & MR & EM & Pc & Po & MR & EM \\
% \midrule
% Base & 8.03B & 31.29 & 40.71 & 56.54 & 6.49 & 26.69 & 32.49 & 54.90 & 0.73 & 9.58 & 20.24 & 67.88 & 0.18 \\
% Attr. & 8.03B & 47.71 & 28.40 & 37.31 & 0.84 & 27.56 & 30.02 & 52.14 & 0.02 & 9.33 & 19.16 & 67.24 & 0.07 \\
% O\&I & 8.03B & 62.64 & 15.96 & 20.30 & 10.04 & 22.11 & 23.31 & 51.32 & 0.20 & 12.89 & 16.11 & 55.56 & 0.25 \\
% sft & 8.03B & 79.36 & 5.09 & 6.03 & \textbf{74.98} & 61.31 & 13.84 & 18.42 & 63.44 & 62.89 & 9.20 & 12.76 & 61.55 \\
% dpo & 8.03B & 80.71 & 5.69 & 6.58 & 61.00 & 63.38 & 14.00 & 18.09 & 47.96 & 64.87 & 9.84 & 13.18 & 46.90 \\
% kaft & 8.03B & \textbf{85.04} & 5.31 & 5.88 & 62.76 & \textbf{64.96} & 13.91 & 17.64 & \textbf{67.91} & \textbf{66.27} & 9.42 & 12.45 & \textbf{66.60} \\
% \midrule
% our_{param.} & 6.97B & 80.71 & 4.73 & 5.54 & 74.67 & 62.6 & 13.98 & 18.25 & 64.40 & 64.87 & 9.11 & 12.32 & 61.96 \\
% our_{ffn} & \textbf{6.97B} & 80.56 & \textbf{4.51} & \textbf{5.30} & \textbf{74.98} & 64.73 & \textbf{13.07} & \textbf{16.80} & 65.07 & 66.00 & \textbf{8.89} & \textbf{11.87} & 61.84 \\
% \bottomrule
% \end{tabular}%
% }
% \label{tab:main_res_ood}


    \caption{Performance on the testing sets of ConFiQA.}
     \label{tab:main_res_ood}
\end{table*}

\textbf{Baselines.}
We evaluate \method{} against five baselines, which are categorized into three groups: (1) Prompt-based approaches, including the attributed prompt (\attrprompt) and the combined opinion-based and instruction-based prompt (\oiprompt) from \citet{zhou2023context}; (2)  Fine-tuning methods, consisting of standard Supervised Fine-Tuning (SFT) and Knowledge Aware Fine-Tuning (KAFT)~\cite{li2022large}. KAFT enhances context faithfulness through counterfactual data augmentation; and (3) the Context-DPO~\cite{bi2024context} utilizes DPO method~\cite{rafailov2023direct} to strengthen context-grounded responses while penalizing those relying on parametric memory.

\textbf{Implementation Details.}
In our experiments, we use LLaMA3-8B-Instruct as the backbone for all methods. To train \method{}, we utilize \dataset{} to identify inaccurate parametric knowledge. The pruning threshold $\alpha$ is set to 0.05. The hyperparameters, $\lambda_1$ and $\lambda_2$, are set to 0.5, which are used to balance $\mathcal{L}_{\text{KAT}}$ and $\mathcal{L}_{\text{KPO}}$ in Eq.~\ref{eq:final_loss}. Additionally, we exclude the last three layers when selecting layers for pruning, as previous studies have shown that removing these layers significantly impacts model performance~\cite{lad2024remarkable,chen2024compressing,zhang2024finercut}. More implementation details are provided in Appendix~\ref{append:implementation}.




\section{Results}

\subsection{Homogeneous Model Merging}

\subsubsection{Averaging vs. Dare / Ties}
\label{sec:rq1}
% We merge 4 models (Base, Math \llama, Code \llama, Knowledge \llama) with the same architecture into a unified MoE with different merging methods and fine-tune it as described in Section \ref{sec:model_pretraining}. 

\textbf{Replacing simple averaging with Dare or Ties merging obtains better performance.} \quad
In this section, we demonstrate the superiority of our proposed Ties and Dare merging MoE over the BTX merging method.
We present the performance of MoE models with \textbf{Dare merging} or \textbf{Ties merging} on non-FFN layers and other baselines in Table \ref{table:homo_results}. The details of training cost for each method are presented in Table \ref{table:1_training_cost} in Appendix.

\begin{table}[!htb]
\centering
\resizebox{\columnwidth}{!}{%
\begin{tabular}{lccccccc}
\hline
Method             & MBPP           & HumanEval      & MATH          & GSM8K         & NQ            & TriviaQA       & Avg.           \\ \hline
\multicolumn{8}{c}{\textbf{Dense Model}}                                                                                               \\ \hline
\llamab & 4.60            & 3.04           & 2.42          & 1.44          & \textbf{6.61} & 26.72          & 7.47           \\ \hline
Code \llama        & \textbf{10.2}  & \textbf{8.53}  & 2.42          & 2.57          & 3.11          & 16.70           & 7.26           \\ \hline
Math \llama        & 9.80            & 6.71           & \textbf{7.81} & \textbf{6.36} & 5.48          & 19.86          & \textbf{9.34}  \\ \hline
Knowledge \llama   & 3.60            & 4.26           & 2.62          & 2.04          & 5.65          & \textbf{28.71} & 7.81           \\ \hline
\multicolumn{8}{c}{\textbf{MoE Merging}}                                                                                                 \\ \hline
Random Routing     & 4.00           & 6.10           & 2.78          & 2.05          & 4.86          & 21.75          & 6.92           \\ \hline
Router Fine-tuning & 3.60           & 6.71           & 2.42          & 2.96          & 5.82          & 25.98          & 7.92           \\ \hline
BTX merging        & 12.40          & 11.58          & 6.74          & 7.73          & \textbf{6.78} & 25.10           & 11.72          \\ \hline
Ties merging       & 14.20          & \textbf{11.98} & 6.74          & 7.81          & 6.72          & 27.66          & 12.52          \\ \hline
Dare merging       & \textbf{14.20} & 10.98          & \textbf{6.82} & \textbf{7.96} & 6.50           & \textbf{30.68} & \textbf{12.86} \\ \hline
\multicolumn{8}{c}{\textbf{MoE from Scratch}}                                                                                                \\ \hline
MoE Upcycling & 18.40  & 12.20 & 7.80 & 12.21 & 8.37  &  37.33 & 16.05 \\
\hline
\end{tabular}
}
\caption{\label{table:homo_results} \textbf{Performance of proposed Dare and Ties merged MoE and other baselines across six datasets.} The best performance of Dense and MoE model is marked in bold. Results of Dare and Ties merged MoE outperform the BTX MoE and other baseline methods.}
\end{table}

% Regarding dense model performance, from Table \ref{table:homo_results}, we find that individual expert LLMs achieve the best performance in their respective domain in most cases, as expected. However, CPTed \llama models also suffer from catastrophic forgetting. For example, both Code and Math \llama behaved worse than the \llamab model on the TriviaQA and NQ datasets.

From Table \ref{table:homo_results}, we see that individual experts generally achieve the best performance in their respective domains, as expected. However, CPTed \llama models experience catastrophic forgetting. For instance, both Code and Math \llama perform worse than \llamab on the TriviaQA and NQ datasets.


% The results of the MoEs in Table \ref{table:homo_results} suggest that by using the Ties or Dare merging to replace the average merging method, the proposed MoE performance exceeds the MoE from the BTX pipeline in almost all datasets and obtains a relative improvement of 6.94\% and 9.72\% compared to the BTX method in average performance. This observation implies that advanced merging methods mitigate the interference of model weights during merging and increase performance. 

The results in Table \ref{table:homo_results} show that using Ties or Dare merging significantly improves MoE performance over the BTX pipeline across almost all datasets, with a relative improvement of 6.94\% and 9.72\% in average performance. This suggests that advanced merging methods reduce weight interference and enhance performance.

% In addition to these baseline results, we include the results of MoE sparse upcycling \cite{komatsuzaki2022sparse} in the last row as a reference. In this approach, the MoE model is initialized from the base model by creating four identical copies of the FFN layers. We then CPT the initialized MoE model on the same training data (340B tokens) as used in the proposed pipeline. Note that we do not compare our results with the upcycling method, since that method pretrains the entire MoE on all data with significantly higher additional cost.

As a reference, we include the results of MoE sparse upcycling \cite{komatsuzaki2022sparse} in the last row of Table \ref{table:homo_results}. This approach initializes the MoE model by creating four identical copies of the FFN layers from the base model and then CPT on the same 340B tokens used in our pipeline. However, we do not directly compare our results with the upcycling method, as it involves pretraining the entire MoE on all data, incurring significantly higher costs.
We also visualize the average performance for each merging method with different fine-tuning token numbers in Figure \ref{fig:result_token} in Appendix \ref{sec:supp_routing}.
In Figure \ref{fig:result_token}, we observe that the Dare and Ties merging MoE models consistently outperform the BTX merging MoE throughout fine-tuning, especially in the earlier stages of fine-tuning. 
% There is more performance gain when fine-tuning with a smaller number of tokens, which follows our expectation since we regard Dare or Ties-merging methods as a better initialization method and have a larger effect during the early stage of training. This observation implies that more advanced merging techniques should be preferred over unweighted average especially when the fine-tuning budget is limited.  

\begin{figure}[!t]
    \centering
    \begin{subfigure}[b]{0.48\columnwidth}
        \centering
        \includegraphics[width=\textwidth]{figure/routing_probabilities_GSM8K.pdf}
        \caption{GSM8K}
        \label{fig:gsm8k_route}
    \end{subfigure}
    \hfill
    \begin{subfigure}[b]{0.48\columnwidth}
        \centering
        \includegraphics[width=\textwidth]{figure/routing_probabilities_MATH.pdf}
        \caption{MATH}
        \label{fig:math_route}
    \end{subfigure}
    \vspace{-0.5em}
    \caption{Routing probability of experts on GSM8K and MATH for different merging methods.}
    \label{fig:routing_prob}
\end{figure}


\noindent \textbf{MoE with Dare or Ties merging routes more tokens to domain experts.}\quad
To further explore the effectiveness of Dare and Ties merging MoE, we evaluate MoEs on multiple benchmarks and calculate the routing probability averaged from each layer and token. We visualize the routing probability of each method of two math datasets (MATH and GSM8K) in Figure \ref{fig:routing_prob} and for other datasets, we put the results in Figure \ref{fig:supp_routing_prob} in Appendix \ref{sec:supp_routing}.


Compared to MoEs with BTX merging, where the base model accepts the most routing decisions, the Dare and Ties merging method routes tokens to domain experts more frequently, as suggested in Figure \ref{fig:routing_prob}. For example, for the GSM8K dataset, the routing probability for math expert increases from 0.28 to 0.35 or 0.46 when replacing simple averaging with the Ties or Dare merging. This finding suggests that the more effective MoE with the more advanced merging method should be attributed to more optimized routing decisions.


\subsubsection{Merging without Fine-tuning}

In this part, we will evaluate our proposed routing heuristics in Section \ref{sec:merging_wo_ft} for MoE without fine-tuning.
Before we evaluate the overall performance of each benchmark, we will first examine the routing decision with our proposed heuristics. We present the routing probability for PPL routing heuristics for each dataset in Table \ref{table:routing_heuristic}.

\begin{table}[!htb]
\centering
\resizebox{0.8\columnwidth}{!}{%
\begin{tabular}{lcccc}
\hline
% \multicolumn{5}{c}{Heuristic 1: Task Vector Routing}     \\ \hline
Benchmark      & Base       & Code          & Math          & Knowledge     \\ \hline
GSM8K     & 23\%       & 2\%           & \textbf{43\%} & {\ul 32\%}    \\ \hline
MATH      & 22\%       & 2\%           & \textbf{49\%} & {\ul 27\%}    \\ \hline
MBPP      & 19\%       & {\ul 22\%}    & \textbf{44\%} & 15\%          \\ \hline
HumanEval & 5\%        & {\ul 43\%}    & \textbf{45\%} & 7\%           \\ \hline
NQ        & {\ul 43\%} & 4\%           & 10\%          & \textbf{43\%} \\ \hline
TriviaQA  & {\ul 50\%} & 0\%           & 0\%           & \textbf{50\%} \\ \hline
\end{tabular}
}
\caption{\label{table:routing_heuristic} \textbf{Routing probability of PPL routing for each dataset.} The largest probability are in bold, and the second-largest are underlined.}
\end{table}

\noindent \textbf{Routing heuristic effectively assigns tokens to the corresponding experts.} \quad 
Table \ref{table:routing_heuristic} demonstrates that PPL routing generally achieves the desired routing patterns, effectively directing inputs from a specific domain to the specialized expert models, except in the case of the MBPP dataset. Since our heuristics rely solely on inference inputs without fine-tuning, they can be considered reliable strategies. We also visualize the routing probability for both PPL and task vector routing heuristics for each dataset in Figure \ref{fig:routing_heuristic} in Appendix \ref{sec:supp_routing}. We find that PPL routing consistently produces better results than the task vector routing.

Next, we evaluate the performance on each dataset with different combinations of merging methods and routing heuristics, compared to the baseline methods. We prepare three dense fine-tuning baselines: \textbf{Dare Dense}, \textbf{Ties Dense} and \textbf{Random Routing} (details in Section \ref{sec:baseline}). We also evaluate the ablation methods: merging attention layers without separation and task vector routing. We present the results of each method across datasets in Table \ref{table:moe_wo_finetune}. The details of training cost for each method are presented in Table \ref{table:3_training_cost} in Appendix.

\begin{table}[!htb]
\centering
\resizebox{\columnwidth}{!}{%
\begin{tabular}{llccccccc}
\hline
Merging      & Routing     & MBPP         & HumanEval     & MATH          & GSM8K         & NQ            & TriviaQA       & Avg.          \\ \hline
\multicolumn{9}{c}{\textbf{Dense Merging}}                                                                                                          \\ \hline
Dare       & N/A         & 6.20         & 6.70          & 2.22          & 2.27          & 4.80          & 20.45          & 7.11          \\ \hline
Ties        & N/A         & 6.00         & 6.70          & 2.48          & 2.19          & 3.62          & 20.86          & 6.98          \\ \hline
\multicolumn{9}{c}{\textbf{MoE Merging}}                                                                                                            \\ \hline
Merge attention     & random      & 4.00         & 6.10          & 2.78          & 2.05          & 4.86          & 21.75          & 6.92          \\ \hline
Merge attention     & task vector & 6.60          & 4.87          & 3.06          & 1.44          & \textbf{6.05} & 21.39          & 7.24          \\ \hline
Merge attention     & PPL         & 6.40          & 4.87          & 2.86          & 1.13          & 5.93          & 22.71          & 7.32          \\ \hline
Separate attention & task vector & 4.00            & 7.32          & \textbf{2.98} & 2.5           & 5.37          & 20.11          & 7.05          \\ \hline
Separate attention & PPL         & \textbf{6.80} & \textbf{7.92} & 2.88          & \textbf{2.95} & 4.74          & \textbf{23.21} & \textbf{8.08} \\ \hline
\end{tabular}
}
\caption{\label{table:moe_wo_finetune} \textbf{Performance of proposed merging and routing methods for MoE without substantial fine-tuning and other baselines across six datasets.} Separating attention layers and perplexity routing heuristics get the best average performance.}
\end{table}

% \noindent \textbf{Proposed MoE method without fine-tuning perform better then dense merging baseline.} \quad From the results in Table \ref{table:moe_wo_finetune}, we find that with the perplexity routing heuristic and the option of separating the attention layers, we get the best average result among all baseline methods. Compared to Random Routing and SoTA dense merging method: Dare, our best methods: PPL routing + separating the attention layers bring a relative improvement of 16.8\% and 13.6\%, respectively, on the average performance. Better results with PPL routing than with task vector routing align with the observation in Figure \ref{fig:routing_heuristic}, where PPL routing routes input more precisely to the desired experts. 
% Moreover, better result for merging without attention layers also follows our expectation that this merging option eliminates the inconsistency of task vector effect as suggested in Section \ref{sec:merging_wo_ft}.

\noindent \textbf{Proposed MoE method without fine-tuning outperforms the dense merging baseline.} \quad From Table \ref{table:moe_wo_finetune}, we observe that using the PPL routing heuristic and separating attention layers achieves the best average results among all baseline methods. Compared to Random Routing and the SoTA dense merging method (Dare), our best method - PPL routing + separating attention layers - yields relative improvements of 16.8\% and 13.6\%, respectively. The superior performance of PPL routing aligns with Figure \ref{fig:routing_heuristic} in Appendix \ref{sec:supp_routing}, where PPL routing more accurately directs input to the appropriate experts. 
Moreover, the better results of separating attention layers support our expectation that this approach resolves the inconsistency of task vector counts, as discussed in Section \ref{sec:merging_wo_ft}.


\subsection{Heterogeneous Model Merging}

% We prepare two MoE models with heterogeneous model merging for this part. We use 3 models (\llamab, Code \llama, and Knowledge \llama) with the same architecture and one Math Olmo model for the first MoE. For the second MoE, we merge 3 models and one Math TinyLlama model. 
% For the tokenizer, we choose the \llamab tokenizer due to the majority of experts. 
% We also prepare two 3-expert MoE models (\llamab, Code \llama, and Knowledge \llama) as the baseline method to show the functionality of Olmo or TinyLlama experts. One 3-expert MoE is fine-tuned on the same data source, and the other one is fine-tuned only on the data without math data. Both baselines are fine-tuned with 100B tokens. 

\textbf{MoE merged with heterogeneous models outperforms the corresponding experts.} \quad
After showing the superiority of our homogeneous model merging method, our next question is whether the proposed heterogeneous expert merging is also effective.
We present the performance of the dense, MoE and baseline methods in Table \ref{table:hetero_results}. The details of training cost for each method are presented in Table \ref{table:4_training_cost} in Appendix.

\begin{table}[!htb]
\centering
\resizebox{\columnwidth}{!}{%
\begin{tabular}{lccccccc}
\toprule
Method                                                           & MBPP           & HumanEval      & MATH          & GSM8K         & NQ            & TriviaQA      & Avg.           \\ \hline
\multicolumn{8}{c}{\textbf{Dense Model}}                                                                                                                                            \\ \hline
\llamab                                                      & 4.60           & 3.04           & 2.42          & 1.44          & 6.61          & 26.72         & 7.47           \\ \hline
Base TinyLlama                                                   & 5.40           & 5.27           & 2.26          & 2.2           & 8.53          & 34.27         & 9.66           \\ \hline
Base Olmo                                                        & 2.80           & 2.64           & 2.46          & 2.42          & 6.16          & 29.21         & 7.62           \\ \hline
Code \llama                                                       & 10.20          & 8.53           & 2.42          & 2.57          & 3.11          & 16.7          & 7.26           \\ \hline
Math TinyLlama                                                   & 15.60          & 9.76           & 4.18          & 5.91          & 6.05          & 21.12         & 10.44          \\ \hline
Math Olmo                                                        & 0.00           & 0.00           & 4.82          & 5.08          & 3.61          & 11.25         & 4.13           \\ \hline
Knowledge \llama                                                  & 3.60            & 4.26           & 2.62          & 2.04          & 5.65          & 28.71         & 7.81           \\ \hline

\multicolumn{8}{c}{\textbf{Homogeneous Expert Merging}}                                                                                                                                              \\ \hline
% BTX merging        & 12.40          & 11.58          & 6.74          & 7.73          & 6.78 & 25.1           & 11.72          \\ \hline
\begin{tabular}[c]{@{}l@{}}3-expert MoE \\ (same data)\end{tabular}  & 9.14           & 10.8           & 4.42          & 5.16          & 6.95          & 26.78         & 10.54          \\ \hline
\begin{tabular}[c]{@{}l@{}}3-expert MoE \\ (w/o math)\end{tabular}   & 12.00             & 9.76           & 2.38          & 1.74          & 6.22          & \textbf{33.20} & 10.88          \\ \hline


\multicolumn{8}{c}{\textbf{Heterogeneous Expert merging}}                                                                                                                                              \\ \hline
\begin{tabular}[c]{@{}l@{}}(Ours) MoE w/ \\ Math Olmo\end{tabular}      & 13.60           & 10.98          & 4.86          & 6.14          & 5.43 & 26.01         & 11.17          \\ \hline

\begin{tabular}[c]{@{}l@{}} (Ours) MoE w/ \\ Math TinyLlama\end{tabular} & \textbf{15.80} & \textbf{11.59} & \textbf{5.42} & \textbf{6.29} & \textbf{8.25} & 32.71         & \textbf{13.34} \\ \bottomrule
\end{tabular}
}
\caption{\label{table:hetero_results} \textbf{Performance of proposed heterogeneous merged MoE and other baselines.} The merged MoE is comparable or outperform the dense or 3-expert baselines on the benchmark from the corresponding domain.}
\end{table}

% From Table \ref{table:hetero_results}, compared to the domain expert models, our merged heterogeneous MoE models are comparable to or exceed the expert results in their corresponding domains. For example, MoE with Math Olmo and Math TinyLlama achieves the performance of 6.14\% and 6.29\% on GSM8K datasets and our CPTed dense model: Math Olmo and Math TinyLlama obtain 5.91\% and 5.08\% accuracy. For average performance, our merging framework also brings a relative improvement of 43.02\% and 27.78\% between the best-performed experts and the merged MoE models for MoE with Olmo and TinyLlama, respectively.
% As for the MoE baseline: 3-expert MoE, both heterogeneous merged MoEs outperform their performance,  especially in the math domains, demonstrating the functionality of including the math expert, which suggests the effectiveness of our merged pipeline. 

Table \ref{table:hetero_results} shows that our merged MoE models are comparable to or outperform the domain expert models in their respective domains. For instance, the MoE merged with Math Olmo and Math TinyLlama achieves 6.14\% and 6.29\% accuracy on GSM8K, compared to 5.91\% and 5.08\% for their dense counterparts. On average, our MoEs with Olmo and TinyLlama improves performance by 43.02\% and 27.78\% relative to the best dense experts, respectively. Both MoEs with heterogeneous experts also outperform the 3-expert MoE baseline, particularly in math, highlighting the effectiveness of including math experts in the pipeline.

\noindent \textbf{MoE merged with heterogeneous experts show the desired routing patterns in most cases.} \quad We also perform a similar routing analysis as described in Section \ref{sec:rq1}. We visualize the routing probability of two MoEs when evaluating on GSM8K and MATH datasets in Figure \ref{fig:hetero_routing_prob} and for other datasets, we visualize the results in Figure \ref{fig:supp_hetero_routing_prob} in Appendix \ref{sec:supp_routing}.

\begin{figure}[!t]
    \centering
    \begin{subfigure}[b]{0.48\columnwidth}
        \centering
        \includegraphics[width=\textwidth]{figure/hetero_routing_probabilities_GSM8K.pdf}
        \caption{GSM8K}
        \label{fig:hetero_gsm8k_route}
    \end{subfigure}
    \hfill
    \begin{subfigure}[b]{0.48\columnwidth}
        \centering
        \includegraphics[width=\textwidth]{figure/hetero_routing_probabilities_MATH.pdf}
        \caption{MATH}
        \label{fig:hetero_math_route}
    \end{subfigure}
    \caption{Routing probability of experts on GSM8K and MATH for the MoE w/ Olmo and MoE w/ TinyLlama.}
    \label{fig:hetero_routing_prob}
\end{figure}

% From the routing analysis in Figures \ref{fig:hetero_routing_prob} and \ref{fig:supp_hetero_routing_prob}, for the coding and knowledge datasets, most of the tokens will be routed to the corresponding experts.
% However, unlike the homogeneous model merging where the math expert shares the highest routing probability in math datasets, the math expert (Olmo or TinyLlama) obtains the second highest routing probability. This discrepancy should be attributed to the difference between the embedding output from MoE models and expert models. Since the embedding layer of MoE is merged from 3 \llama models and 1 other model, the embedding output of MoE is expected to be closer to the embedding output of \llama models. Therefore, the router network is more likely to route the inputs to the \llama model. Adding the load balancing loss may be the possible solution to this caveat \cite{sukhbaatar2024branchtrainmixmixingexpertllms, fedus2022switch}, which ensures a more uniform distribution of the routing load, and we leave that exploration to future work.

As shown in Figures \ref{fig:hetero_routing_prob} and \ref{fig:supp_hetero_routing_prob}, most tokens in the coding and knowledge datasets are routed to the corresponding experts. However, unlike homogeneous model merging where the math expert has the highest routing probability for math datasets, Math Olmo or Math TinyLlama ranks second. This discrepancy is likely due to the difference in embedding outputs between the MoE and expert models. Since the MoE's embedding layer is merged from 3 \llama models and 1 other model, its output is closer to that of the \llama models, making the router more likely to select them. Adding a load balancing loss is a possible solution to address this issue \cite{sukhbaatar2024branchtrainmixmixingexpertllms, fedus2022switch}, ensuring a more uniform routing distribution. We leave this for future exploration
% 在Hybrid Cross-Chain Smart Contract Deployment Protocol 协议中,relayer对需要被跨链调用intra-chain DApp的逻辑合约进行了迁移。relayer在bridging合约中通过已经部署的合约地址获得其字节码,合约的 API 文件用来定义与合约交互的接口,通常在智能合约部署时就会公开,因此可以直接获取。通过 API 文件以及链上合约的字节码,relayer可以重新在其他区块链上部署该合约。在IntegratedX系统中,Hybrid Cross-Chain Smart Contract Deployment Protocol 协议内的任意relayer都可以进行迁移操作,然而这些relayer并不一定是可信任的,这就导致了可能有恶意的relayer在迁移过程中故意更改字节码造成安全性漏洞,因此我们通过on-chain verification进行链上合约验证来保证字节码没有被更改。
% 我们设计使用bridging合约比较迁移后合约的字节码哈希值和原合约的字节码哈希值。合约字节码哈希值可以通过智能合约的链内调用获得,并且将会由relayer传输到intra-chain DApp所在的链上,由于存在至少一个可以运行的relayer,该哈希值最终会被传输到目标链上并由目标链进行验证并将验证结果回传。由于区块链数据的不可篡改性,使用链上智能合约验证可以保证验证结果无法被更改。只有通过验证的合约才会完成完整的登记流程并可以被跨链应用链内调用。由于区块链数据具有可追溯性,因此对于验证通过的合约,进行该次迁移的relayer将会获得奖励,如果验证失败,将由新的relayer进行迁移,并对迁移失败的relayre进行惩罚。惩罚措施包含建立黑名单机制,一定次数的验证失败将无法进行迁移且此次迁移的费用将会由relayer承担。通过这样的on-chain verification,可以保证逻辑合约迁移后的正确性。由于至少存在一个relayer可以正常工作,因此该逻辑合约最后一定会被成功迁移。
\section{Security Analysis}\label{security_analysis}
% In this section, we analyzed how \texttt{IntegrateX} can safely and efficiently perform complex cross-chain calls.

\subsection{Security in Hybrid Cross-Chain Smart Contract Deployment Protocol}

\begin{theorem}
The Hybrid Cross-Chain Smart Contract Deployment Protocol ensures reliability, verifiability, and consistency, if the proportion of Byzantine nodes in each blockchain is less than its fault tolerance threshold, and at least one functional relayer is present.
% As long as the number of Byzantine nodes in each blockchain remains within the Byzantine fault tolerance limits, and at least one functional relayer is present, the reliability, verifiability, and consistency of the Hybrid Cross-Chain Smart Contract Deployment Protocol can thus be assured.
\end{theorem}

\begin{proof}
\noindent
\textbf{Reliability. }
The Hybrid Cross-Chain Smart Contract Deployment Protocol ensures reliability, meaning that when an execution chain issues a request for cross-chain deployment of a contract, the requested contract will eventually be deployed to the execution chain. 
Within the Hybrid Cross-Chain Smart Contract Deployment Protocol, multiple relayers listen for such requests. Even if malicious relayers intentionally ignore the requests, assuming that at least one functional relayer exists, this relayer will ultimately handle the logic clone and deployment of the requested contract. Thus, even in the worst-case scenario, the protocol ensures that the contract will be successfully deployed to the execution chain.

\vspace{3pt}
\noindent
\textbf{Verifiability. }
The Hybrid Cross-Chain Smart Contract Deployment Protocol ensures verifiability, meaning that both the execution chain and the invoked chain are able to verify the cross-chain transactions transmitted by the relayers. 
Additionally, both chains can verify the hash values of the contract bytecode before and after the cross-chain clone and deployment to ensure that the contract has been deployed correctly. 

In the Hybrid Cross-Chain Smart Contract Deployment Protocol, multiple relayers listen to the request and relay the messages. 
% However, relayers are not responsible for verifying the cross-chain transactions. 
% Instead, 
The invoked chain (bridging contract) validates the authenticity of the transactions using the Merkle proof attached with the cross-chain transactions, thereby preventing the relayers from altering the cross-chain data. 
By parsing the transactions, the invoked chain can obtain the bytecode hash of the cloned contract and compare it with the bytecode hash of the original contract on the chain to ensure the correctness of the cross-chain deployment. 
Additionally, the nonce value associated with each transaction prevents malicious replay attacks.

\vspace{3pt}
\noindent
\textbf{Consistency. }
The Hybrid Cross-Chain Smart Contract Deployment Protocol ensures consistency, meaning that during the off-chain clone and deployment as well as the on-chain verification process, both the execution chain and the invoked chain reach agreement on the outcome of the cross-chain requests.
% meaning that the agreement between the execution chain and the invoked chain on the outcome of the cross-chain request during off-chain clone and deployment and on-chain verification. 
In the Hybrid Cross-Chain Smart Contract Deployment Protocol, the proportion of malicious nodes on both the execution chain and the invoked chain remains within the fault tolerance threshold.
Moreover, the protocol requires to wait until consensus on one chain is finalized (or highly likely to be finalized) before committing the cross-chain transaction to another chain. 
% both the execution chain and the invoked chain maintain within the Byzantine fault tolerance limits. 
As a result, even in the presence of Byzantine nodes, each chain can still achieve consensus on cross-chain transactions and finalize them, ensuring that consistency is not undermined by malicious nodes. 
This guarantees that the outcomes of cross-chain requests remain consistent.
\end{proof}

\subsection{Security in Cross-Chain Smart Contract Integrated Execution Protocol}
%Cross-Chain Smart Contract Integrated Execution Protocol实现了跨链调用的原子执行,IntegrateX系统通过类似于 2PC 的状态同步机制对invoked的合约进行状态锁定和更新,在cross-chain应用中当用户在execution chain上的调用cross-chain DApp合约后,bridging合约将向所有invoked chain上的invoked合约发送状态请求,如果任何一个invoked合约未能锁定状态,将会返回失败信息,bridging合约将会取消本次调用并且发送跨链请求解锁所有其他合约,被锁定的合约将被解锁,而当尚未完成锁定的合约先接收到了解锁请求,将会在忽略后续接收到的该次锁定请求,以此保证合约状态锁定的一致性。
% 当所有被请求的合约都成功锁定状态后,cross-chain DApp合约将在execution chain上进行集成执行并将结果返回给所有的invoked合约,invoked 合约的新的状态将会首先返回给invoked chain上的bridging合约,并由bridging合约返回状态更新成功信息,如果任意一条invoked chain上的状态更新失败,所有链上的bridging合约都会丢弃更新的状态,此次跨链调用将失败,只有所有invoked chain上都完成了状态更新,cross-chain DApp合约才会输出最终结果,invoked合约解锁状态并从本链的bridging合约进行状态更新,标志着此次调用成功,由此保证完整的跨链调用原子性。
\begin{theorem}
The Hybrid Cross-Chain Smart Contract Integrated Execution Protocol ensures overall atomicity, reliability, verifiability, and consistency, if the proportion of Byzantine nodes in each blockchain is less than its fault tolerance threshold, and at least one functional relayer is present.
% As long as the number of Byzantine nodes in each blockchain remains within the Byzantine fault tolerance limits, and at least one functional relayer is present, the atomicity, reliability, verifiability, and consistency of the Cross-Chain Smart Contract Integrated Execution Protocol can thus be assured.
\end{theorem}

\begin{proof}
\noindent
\textbf{Overall Atomicity. }
The Cross-Chain Smart Contract Integrated Execution Protocol guarantees overall atomicity, which means that in the selected CCSCI process, state changes on both the execution chain and the invoked chain either all succeed or all fail, preventing any situation where one chain's state changes while the other does not. 
We employ the atomic integrated execution mechanism, similar to the 2PC scheme. 
For one cross-chain dApp, during this process, all the states required by the invocation on the invoked chains will be locked. 
If any contract has already been locked by another invocation, this invocation will fail, and all other invoked contracts will be unlocked. 
If the execution chain obtains all the necessary states but the execution fails due to insufficient gas or other reasons, the execution will be aborted, and all related locked contracts will be unlocked without any state changes. 
Furthermore, the protocol incorporates a timeout scheme: 
If any of the invoked chains fails to return the required state within the specified time frame by dApp, or if the execution transaction on the execution chain fails to complete within the specified time limit, the execution chain will abort the invocation and unlock all related contract states, ignoring any subsequent state responses from the invoked chains. 
% In cases where an invoked chain receives a lock request after receiving an unlock request, it will also ignore the lock request, ensuring that all invoked contract states are either locked or unlocked simultaneously. 
As a result, only when the execution chain has successfully acquired all required states and completed execution will it issue state updates to all related invoked chains, thereby ensuring the atomicity of the entire CCSCI process.

\vspace{3pt}
\noindent
\textbf{Reliability. }
The Cross-Chain Smart Contract Integrated Execution Protocol ensures reliability, which means that when an execution chain initiates a CCSCI request, the invoked chain will eventually receive the cross-chain transaction, and the state returned by the invoked chain will likewise be received by the execution chain. 
In the Cross-Chain Smart Contract Integrated Execution Protocol, multiple relayers monitor CCSCI requests. 
Even in the presence of malicious relayers who deliberately fail to respond to the request, the assumption of at least one functional relayer ensures that, in the worst-case scenario, this relayer will relay the cross-chain transaction to the invoked chain, ensuring that the transaction is eventually received. 
Similarly, even in the worst-case scenario, at least one relayer will transmit the state returned by the invoked chain back to the execution chain, thereby guaranteeing the reliability of the CCSCI process.

\vspace{3pt}
\noindent
\textbf{Verifiability. }
The Cross-Chain Smart Contract Integrated Execution Protocol ensures verifiability, which means the ability of the invoked chain to verify the authenticity of cross-chain transactions transmitted by relayers from the execution chain, while the execution chain can also verify the transactions returned by the invoked chain. 
In this protocol, multiple relayers listen for the request and relay messages.
% , but relayers themselves are not responsible for verifying the cross-chain transactions. 
Both the execution chain and the invoked chain can independently validate the authenticity of the cross-chain transactions using the Merkle proof attached with the transactions. 
Additionally, the use of transaction nonce values prevents malicious replay attacks, ensuring the integrity of the cross-chain interaction.

\vspace{3pt}
\noindent
\textbf{Consistency. }
The Cross-Chain Smart Contract Integrated Execution Protocol guarantees consistency, ensuring that both the execution chain and the invoked chain agree on the result of the requested operation during a cross-chain smart contract invocation. 
In this protocol, the proportion of malicious nodes on both chains remains within the fault tolerance threshold, allowing consensus to be achieved even if there exist Byzantine nodes. 
Additionally, the protocol requires waiting until consensus on one chain has been finalized before committing the cross-chain transaction to the other chain.
Therefore, cross-chain transactions on the blockchain cannot be maliciously altered, ensuring that both the execution chain and the invoked chain reach a unified agreement on the outcome of cross-chain operations.
\end{proof}


% \subsection{Efficiency Between \texttt{IntegrateX} and GPACT}
% %在\texttt{IntegrateX}中通过迁移合约逻辑将跨链调用转换为链内调用可以提高调用效率,因为通过集成执行的方式,合约的状态将会被同步锁定,提高了不同invoked chain上对合约状态并行处理的效率,从而减少了链式调用的等待时间,从而达到减少整体跨链调用latency的目的。考虑在GPACT协议agency应用的调用过程,如图所示。
% %在GPACT中进行跨链调用的过程中,整个跨链调用过程是一个串联调用过程,由于一次跨链调用需要发送receipt来确定信息是否已经送达,因此一次跨链通信时间至少需要2个区块时间,分别为接收方获得跨链信息所需的1个区块时间和发送方确认receipt所需的1个区块时间。在GPACT中跨链调用必须从调用树的根节点合约进行状态锁定,并在到达叶子节点合约时遵从叶子节点到根节点的执行顺序进行执行,假设调用树的深度为d,则此过程需要2d次信息传递,在执行结束后的更新合约状态则需要进行1次跨链信息传递,因此总共需要2d+1次信息传递,即4d+2个区块时间,考虑在调用根节点合约时还需要一个区块时间处理该交易,因此总共需要4d+3个区块时间。在agency应用中,调用树的深度为2,因此需要进行5次跨链信息传递 ,总共需要11个区块时间。
% In \texttt{IntegrateX}, by migrating contract logic, cross-chain calls can be converted into intra-chain calls, which improves call efficiency. This is because, through integrated execution, contract states are synchronously locked, enhancing the efficiency of parallel processing of contract states on different invoked chains. This reduces the waiting time associated with chained calls, ultimately lowering the overall latency of cross-chain calls. Consider the call process in the agency application under the GPACT protocol, as illustrated in Fig. \ref{gpact}.
% \begin{figure}[htbp]
%     \centering
%     \includegraphics[width=0.35\textwidth]{Figures/GPACT_process.png}
%     \caption{The process of the agency application in the GPACT protocol.}
%     \label{gpact}
% \end{figure} 

% In GPACT, the cross-chain call process is sequential. Since each cross-chain call requires sending a receipt to confirm whether the information has been delivered, a single cross-chain communication requires at least 2 block times: 1 block time for the recipient to receive the cross-chain information and 1 block time for the sender to confirm the receipt. In GPACT, cross-chain calls must begin by locking the state at the root contract of the call tree, and the execution follows the sequence from the leaf node contract back to the root node contract. Assuming the depth of the call tree is $d$, this process requires $2d$ transmissions of information. After the execution, updating the contract state requires an additional cross-chain transmission, making a total of $2d+1$ transmissions, which corresponds to $4d+2$ block times. Considering that processing the transaction at the root contract also requires 1 block time, the total time required is $4d+3$ block times. In the agency application, where the depth of the call tree is 2, 5 cross-chain transmissions are needed, resulting in a total of 11 block times.

% %在\texttt{IntegrateX}中进行跨链调用的过程中,整个跨链调用过程是一个并行调用过程,如图所示。在\texttt{IntegrateX}中,跨链调用的调用树的根节点合约将同时对所有调用树中的其他节点合约发起状态锁定请求,并在获得所有其他节点合约状态后在本链进行integrated execution,并在执行结束后同时更新其他节点合约的状态。假设调用树的深度为d,则锁定和获取其他合约状态需要2次信息传递,在执行结束后的更新合约状态则需要进行1次跨链信息传递,因此总共需要3次信息传递,即6个区块时间,考虑在调用根节点合约时还需要一个区块时间处理该交易,因此总共需要7个区块时间,是一个常数,与调用树深度无关。
% In \texttt{IntegrateX}, the cross-chain call process is parallel, as illustrated in Fig. \ref{IntegrateX}. In \texttt{IntegrateX}, the root contract of the call tree simultaneously sends state lock requests to all other node contracts in the call tree. After obtaining the states of all other node contracts, integrated execution is performed on the same chain, and after execution, the states of the other node contracts are updated simultaneously. Assuming the depth of the call tree is $d$, locking and obtaining the states of other contracts require 2 transmissions of information. After execution, updating the contract states requires an additional cross-chain transmission, making a total of 3 transmissions, corresponding to 6 block times. Considering that processing the transaction at the root contract requires 1 block time, the total time needed is 7 block times, which is a constant and independent of the depth of the call tree.
% \begin{figure}[htbp]
%     \centering
%     \includegraphics[width=0.35\textwidth]{Figures/IntegrateX_process.png}
%     \caption{The process of the agency application in the \texttt{IntegrateX} protocol.}
%     \label{IntegrateX}
% \end{figure}

% %在agency应用中,\texttt{IntegrateX}协议完整跨链调用只需要7个区块时间,而GPACT需要11个区块时间,由于处理根节点合约的调用并不需要一个完整的区块时间,因此\texttt{IntegrateX}的最短时间将会介于6-7个区块时间,而GPACT则是介于4d+2-4d+3个区块时间。考虑到这只是一个调用树深度为2跨链调用,在 INtegratedX 中,跨链等待时间并不会随着调用树深度的增加而增大而在GPACT中跨链调用的latency会随着调用树的深度增加而增大。因此随着调用树的深度增加,\texttt{IntegrateX}将会减少更多的时间。然而在实际场景中,由于存在网络波动,跨链信息传递的发送和接收并不能保证一定会在一个区块时间内完成,因此\texttt{IntegrateX}和GPACT所需要的区块时间将会大于最短区块时间。
% In the agency application, the \texttt{IntegrateX} protocol requires only 7 block times for a complete cross-chain call, whereas GPACT requires 11 block times. Since processing the root contract's call does not necessarily require a full block time, the minimum time for \texttt{IntegrateX} will range between 6 to 7 block times, while for GPACT, it will range between $4d+2$ to $4d+3$ block times. Considering that this example is a cross-chain call with a call tree depth of 2, in \texttt{IntegrateX}, the cross-chain waiting time does not increase with the depth of the call tree, whereas in GPACT, the latency of cross-chain calls increases with the depth of the call tree. Therefore, as the depth of the call tree increases, \texttt{IntegrateX} will save more time. However, in real-world scenarios, due to network fluctuations, the transmission and reception of cross-chain information cannot always be guaranteed to complete within a single block time. Consequently, the actual block time required for \texttt{IntegrateX} and GPACT will be greater than the minimum block time.

\section{Conclusion \& Future Work}\label{conclusion}
This work presents XAMBA, the first framework optimizing SSMs on COTS NPUs, removing the need for specialized accelerators. XAMBA mitigates key bottlenecks in SSMs like CumSum, ReduceSum, and activations using ActiBA, CumBA, and ReduBA, transforming sequential operations into parallel computations. These optimizations improve latency, throughput (Tokens/s), and memory efficiency. Future work will extend XAMBA to other models, explore compression, and develop dynamic optimizations for broader hardware platforms.



% This work introduces XAMBA, the first framework to optimize SSMs on COTS NPUs, eliminating the need for specialized hardware accelerators. XAMBA addresses key bottlenecks in SSM execution, including CumSum, ReduceSum, and activation functions, through techniques like ActiBA, CumBA, and ReduBA, which restructure sequential operations into parallel matrix computations. These optimizations reduce latency, enhance throughput, and improve memory efficiency. 
% Experimental results show up to 2.6$\times$ performance improvement on Intel\textregistered\ Core\texttrademark\ Ultra Series 2 AI PC. 
% Future work will extend XAMBA to other models, incorporate compression techniques, and explore dynamic optimization strategies for broader hardware platforms.


% This work presents XAMBA, an optimization framework that enhances the performance of SSMs on NPUs. Unlike transformers, SSMs rely on structured state transitions and implicit recurrence, which introduce sequential dependencies that challenge efficient hardware execution. XAMBA addresses these inefficiencies by introducing CumBA, ReduBA, and ActiBA, which optimize cumulative summation, ReduceSum, and activation functions, respectively, significantly reducing latency and improving throughput. By restructuring sequential computations into parallelizable matrix operations and leveraging specialized hardware acceleration, XAMBA enables efficient execution of SSMs on NPUs. Future work will extend XAMBA to other state-space models, integrate advanced compression techniques like pruning and quantization, and explore dynamic optimization strategies to further enhance performance across various hardware platforms and frameworks.
% This work presents XAMBA, an optimization framework that enhances the performance of SSMs on NPUs. Key techniques, including CumBA, ReduBA, and ActiBA, achieve significant latency reductions by optimizing operations like cumulative summation, ReduceSum, and activation functions. Future work will focus on extending XAMBA to other state-space models, integrating advanced compression techniques, and exploring dynamic optimization strategies to further improve performance across various hardware platforms and frameworks.

% This work introduces XAMBA, an optimization framework for improving the performance of Mamba-2 and Mamba models on NPUs. XAMBA includes three key techniques: CumBA, ReduBA, and ActiBA. CumBA reduces latency by transforming cumulative summation operations into matrix multiplication using precomputed masks. ReduBA optimizes the ReduceSum operation through matrix-vector multiplication, reducing execution time. ActiBA accelerates activation functions like Swish and Softplus by mapping them to specialized hardware during the DPU’s drain phase, avoiding sequential execution bottlenecks. Additionally, XAMBA enhances memory efficiency by reducing SRAM access, increasing data reuse, and utilizing Zero Value Compression (ZVC) for masks. The framework provides significant latency reductions, with CumBA, ReduBA, and ActiBA achieving up to 1.8X, 1.1X, and 2.6X reductions, respectively, compared to the baseline.
% Future work includes extending XAMBA to other state-space models (SSMs) and exploring further hardware optimizations for emerging NPUs. Additionally, integrating advanced compression techniques like pruning and quantization, and developing adaptive strategies for dynamic optimization, could enhance performance. Expanding XAMBA's compatibility with other frameworks and deployment environments will ensure broader adoption across various hardware platforms.
\section*{Limitation}
This paper uninstalls knowledge by removing entire FFN layers after identifying those with greater neuron inhibition. While \method{} demonstrates promising results, the pruning process may possibly affect some non-knowledge-related neurons. Regarding evaluation, we have employed ConR and MemR metrics, consistent with established practices in the field. Although these rule-based metrics provide valuable insights, we acknowledge that more sophisticated evaluation methodologies could offer more comprehensive evaluations. 

\section*{Ethics Statement}
Our data construction process involves prompting LLMs to elicit their parametric knowledge for studying knowledge conflicts. This process may result in some hallucinated content. We commit to the careful distribution of the data generated through our research, ensuring it is strictly used for research purposes. Our goal is to encourage responsible use of LLMs while advancing understanding of knowledge conflicts. Additionally, no personally identifiable information or offensive content is included in our dataset. We adhere to ethical guidelines for responsible AI research and data sharing.

We also employed human evaluation to assess the reliability of GPT-4o-mini in identifying knowledge conflicts. Evaluation data was carefully distributed to human evaluators solely for research purposes, ensuring it adheres to ethical standards and contains no content that violates these standards.

\bibliography{acl_latex}

\clearpage
\appendix


\newpage
\appendix
\section{Applicability of SparseTransX for dense graphs} 
\label{A:density}
Even for fully dense graphs, our KGE computations remain highly sparse. This is because our SpMM leverages the incidence matrix for triplets, rather than the graph's adjacency matrix. In the paper, the sparse matrix $A \in \{-1,0,1\}^{M \times (N+R)}$ represents the triplets, where $N$ is the number of entities, $R$ is the number of relations, and $M$ is the number of triplets. This representation remains extremely sparse, as each row contains exactly three non-zero values (or two in the case of the "ht" representation). Hence, the sparsity of this formulation is independent of the graph's structure, ensuring computational efficiency even for dense graphs.

\section{Computational Complexity}
\label{A:complexity}
 For a sparse matrix $A$ with $m \times k$ having $nnz(A)=$ number of non zeros and dense matrix $X$ with $k \times n$ dimension, the computational complexity of the SpMM is $O(nnz(A) \cdot n)$ since there are a total of $nnz(A)$ number of dot products each involving $n$ components. Since our sparse matrix contains exactly three non-zeros in each row, $nnz(A) = 3m$. Therefore, the complexity of SpMM is $O(3m \cdot n)$ or $O(m \cdot n)$, meaning the complexity increases when triplet counts or embedding dimension is increased. Memory access pattern will change when the number of entities is increased and it will affect the runtime, but the algorithmic complexity will not be affected by the number of entities/relations.

\section{Applicability to Non-translational Models}
\label{A:non_trans}
Our paper focused on translational models using sparse operations, but the concept extends broadly to various other knowledge graph embedding (KGE) methods. Neural network-based models, which are inherently matrix-multiplication-based, can be seamlessly integrated into this framework. Additionally, models such as DistMult, ComplEx, and RotatE can be implemented with simple modifications to the SpMM operations. Implementing these KGE models requires modifying the addition and multiplication operators in SpMM, effectively changing the semiring that governs the multiplication.   

In the paper, the sparse matrix $A \in \{-1,0,1\}^{M \times (N+R)}$ represents the triplets, and the dense matrix $E \in \mathbb{R}^{(N+R) \times d}$ represents the embedding matrix, where $N$ is the number of entities, $R$ is the number of relations, and $M$ is the number of triplets. TransE’s score function, defined as $h + r - t$, is computed by multiplying $A$ and $E$ using an SpMM followed by the L2 norm. This operation can be generalized using a semiring-based SpMM model: $Z_{ij} = \bigoplus_{k=1}^{n} (A_{ik} \otimes E_{kj})$

Here, $\oplus$ represents the semiring addition operator, and $\otimes$ represents the semiring multiplication operator. For TransE, these operators correspond to standard arithmetic addition and multiplication, respectively.

\subsection*{DistMult} 
DistMult’s score function has the expression $h \odot r \odot t$. To adapt SpMM for this model, two key adjustments are required: The sparse matrix $A$ stores $+1$ at the positions corresponding to $h_{\text{idx}}$, $t_{\text{idx}}$, and $r_{\text{idx}}$. Both the semiring addition and multiplication operators are set to arithmetic multiplication. These changes enable the use of SpMM for the DistMult score function.

\subsection*{ComplEx} 
ComplEx’s score function has $h \odot r \odot \bar{t}$, where embeddings are stored as complex numbers (e.g., using PyTorch). In this case, the semiring operations are similar to DistMult, but with complex number multiplication replacing real number multiplication.

\subsection*{RotatE} 
RotatE’s score function has $h \odot r - t$. For this model, the semiring requires both arithmetic multiplication and subtraction for $\oplus$. With minor modifications to our SpMM implementation, the semiring addition operator can be adapted to compute $h \odot r - t$.

\subsection*{Support from other libraries}
Many existing libraries, such as GraphBLAS (Kimmerer, Raye, et al., 2024), Ginkgo (Anzt, Hartwig, et al., 2022), and Gunrock (Wang, Yangzihao, et al., 2017), already support custom semirings in SpMM. We can leverage C++ templates to extend support for KGE models with minimal effort.


\begin{figure*}[t]
\centering     %%% not \center
\includegraphics[width=\textwidth]{figures/all-eval.pdf}
\caption{Loss curve for sparse and non-sparse approach. Sparse approach eventually reaches the same loss value with similar Hits@10 test accuracy.}
\label{fig:loss_curve}
\end{figure*}

\section{Model Performance Evaluation and Convergence}
\label{A:eval}
SpTransX follows a slightly different loss curve (see Figure \ref{fig:loss_curve}) and eventually converges with the same loss as other non-sparse implementations such as TorchKGE. We test SpTransX with the WN18 dataset having embedding size 512 (128 for TransR and TransH due to memory limitation) and run 200-1000 epochs. We compute average Hits@10 of 9 runs with different initial seeds and a learning rate scheduler. The results are shown below. We find that Hits@10 is generally comparable to or better than the Hits@10 achieved by TorchKGE.

\begin{table}[h]
\centering
\caption{Average of 9 Hits@10 Accuracy for WN18 dataset}
\begin{tabular}{|c|c|c|}
\hline
\textbf{Model} & \textbf{TorchKGE} & \textbf{SpTransX} \\ \hline
TransE         & 0.79 ± 0.001700   & 0.79 ± 0.002667   \\ \hline
TransR         & 0.29 ± 0.005735   & 0.33 ± 0.006154   \\ \hline
TransH         & 0.76 ± 0.012285   & 0.79 ± 0.001832   \\ \hline
TorusE         & 0.73 ± 0.003258   & 0.73 ± 0.002780   \\ \hline
\end{tabular}
\label{table:perf_eval}
\end{table}

% We also plot the loss curve for different models in Figure \ref{fig:loss_curve}. We observe that the sparse approach follows a similar loss curve and eventually converges to the same final loss.

\section{Distributed SpTransX and Its Applicability to Large KGs}
\label{A:dist}
SpTransX framework includes several features to support distributed KGE training across multi-CPU, multi-GPU, and multi-node setups. Additionally, it incorporates modules for model and dataset streaming to handle massive datasets efficiently. 

Distributed SpTransX relies on PyTorch Distributed Data Parallel (DDP) and Fully Sharded Data Parallel (FSDP) support to distribute sparse computations across multiple GPUs. 

\begin{table}[h]
\centering
\caption{Average Time of 15 Epochs (seconds). Training time of TransE model with Freebase dataset (250M triplets, 77M entities. 74K relations, batch size 393K)  on 32 NVIDIA A100 GPUs. FSDP enables model training with larger embedding when DDP fails.}
\begin{tabular}{|p{2cm}|p{2.5cm}|p{2.5cm}|}
\hline
\textbf{Embedding Size} & \textbf{DDP (Distributed Data Parallel)} & \textbf{FSDP (Fully Sharded Data Parallel)} \\ \hline
16                      & 65.07 ± 1.641                            & 63.35 ± 1.258                               \\ \hline
20                      & Out of Memory                            & 96.44 ± 1.490                               \\ \hline
\end{tabular}
\end{table}

We run an experiment with a large-scale KG to showcase the performance of distributed SpTransX. Freebase (250M triplets, 77M entities. 74K relations, batch size 393K) dataset is trained using the TransE model on 32 NVIDIA A100 GPUs of NERSC using various distributed settings. SpTransX’s Streaming dataset module allows fetching only the necessary batch from the dataset and enables memory-efficient training. FSDP enables model training with larger embedding when DDP fails.

\section{Scaling and Communication Bottlenecks for Large KG Training}
\label{A:scaling}
Communication can be a significant bottleneck in distributed KGE training when using SpMM. However, by leveraging Distributed Data-Parallel (DDP) in PyTorch, we successfully scale distributed SpTransX to 64 NVIDIA A100 GPUs with reasonable efficiency. The training time for the COVID-19 dataset with 60,820 entities, 62 relations, and 1,032,939 triplets is in Table \ref{table:scaling}. 
% \vspace{-.3cm}
\begin{table}[h]
\centering
\caption{Scaling TransE model on COVID-19 dataset}
\begin{tabular}{|c|c|}
\hline
\textbf{Number of GPUs} & \textbf{500 epoch time (seconds)} \\ \hline
4                       & 706.38                            \\ \hline
8                       & 586.03                            \\ \hline
16                      & 340.00                               \\ \hline
32                      & 246.02                            \\ \hline
64                      & 179.95                            \\ \hline
\end{tabular}
\label{table:scaling}
\end{table}
% \vspace{-.2cm}
It indicates that communication is not a bottleneck up to 64 GPUs. If communication becomes a performance bottleneck at larger scales, we plan to explore alternative communication-reducing algorithms, including 2D and 3D matrix distribution techniques, which are known to minimize communication overhead at extreme scales. Additionally, we will incorporate model parallelism alongside data parallelism for large-scale knowledge graphs.

\section{Backpropagation of SpMM}
\label{A:backprop}
 Our main computational kernel is the sparse-dense matrix multiplication (SpMM). The computation of backpropagation of an SpMM w.r.t. the dense matrix is also another SpMM. To see how, let's consider the sparse-dense matrix multiplication $AX = C$ which is part of the training process. As long as the computational graph reduces to a single scaler loss $\mathfrak{L}$, it can be shown that $\frac{\partial C}{\partial X} = A^T$. Here, $X$ is the learnable parameter (embeddings), and $A$ is the sparse matrix. Since $A^T$ is also a sparse matrix and $\frac{\partial \mathfrak{L}}{\partial C}$ is a dense matrix, the computation $\frac{\partial \mathfrak{L}}{\partial X} = \frac{\partial C}{\partial X} \times \frac{\partial \mathfrak{L}}{\partial C} = A^T \times \frac{\partial \mathfrak{L}}{\partial C} $ is an SpMM. This means that both forward and backward propagation of our approach benefit from the efficiency of a high-performance SpMM.

\subsection*{Proof that $\frac{\partial C}{\partial X} = A^T$}
 To see why $\frac{\partial C}{\partial X} = A^T$ is used in the gradient calculation, we can consider the following small matrix multiplication without loss of generality.
\begin{align*}
A &= \begin{bmatrix}
a_1 & a_2 \\
a_3 & a_4
\end{bmatrix} \\ 
 X &= \begin{bmatrix}
x_1 & x_2 \\
x_3 & x_4
\end{bmatrix} \\
 C &=  \begin{bmatrix}
c_1 & c_2 \\
c_3 & c_4
\end{bmatrix}
\end{align*}
Where $C=AX$, thus-
\begin{align*}
c_1&=f(x_1, x_3) \\
c_2&=f(x_2, x_4) \\
c_3&=f(x_1, x_3) \\
c_4&=f(x_2, x_4) \\
\end{align*}
Therefore-
\begin{align*}
\frac{\partial \mathfrak{L}}{\partial x_1} &= \frac{\partial \mathfrak{L}}{\partial c_1} \times \frac{\partial c_1}{\partial x_1} + \frac{\partial \mathfrak{L}}{\partial c_2} \times \frac{\partial c_2}{\partial x_1} + \frac{\partial \mathfrak{L}}{\partial c_3} \times \frac{\partial c_3}{\partial x_1} + \frac{\partial \mathfrak{L}}{\partial c_4} \times \frac{\partial c_4}{\partial x_1}\\
&= \frac{\partial \mathfrak{L}}{\partial c_1} \times \frac{\partial \mathfrak{c_1}}{\partial x_1} + 0 + \frac{\partial \mathfrak{L}}{\partial c_3} \times \frac{\partial \mathfrak{c_3}}{\partial x_1} + 0\\
&= a_1 \times \frac{\partial \mathfrak{L}}{\partial c_1} + a_3 \times \frac{\partial \mathfrak{L}}{\partial c_3}\\
\end{align*}

Similarly-
\begin{align*}
\frac{\partial \mathfrak{L}}{\partial x_2}
&= a_1 \times \frac{\partial \mathfrak{L}}{\partial c_2} + a_3 \times \frac{\partial \mathfrak{L}}{\partial c_4}\\
\frac{\partial \mathfrak{L}}{\partial x_3}
&= a_2 \times \frac{\partial \mathfrak{L}}{\partial c_1} + a_4 \times \frac{\partial \mathfrak{L}}{\partial c_3}\\
\frac{\partial \mathfrak{L}}{\partial x_4}
&= a_2 \times \frac{\partial \mathfrak{L}}{\partial c_2} + a_4 \times \frac{\partial \mathfrak{L}}{\partial c_4}\\
\end{align*}
This can be expressed as a matrix equation in the following manner-
\begin{align*}
\frac{\partial \mathfrak{L}}{\partial X} &= \frac{\partial C}{\partial X} \times \frac{\partial \mathfrak{L}}{\partial C}\\
\implies \begin{bmatrix}
\frac{\partial \mathfrak{L}}{\partial x_1} & \frac{\partial \mathfrak{L}}{\partial x_2} \\
\frac{\partial \mathfrak{L}}{\partial x_3} & \frac{\partial \mathfrak{L}}{\partial x_4}
\end{bmatrix} &= \frac{\partial C}{\partial X} \times \begin{bmatrix}
\frac{\partial \mathfrak{L}}{\partial c_1} & \frac{\partial \mathfrak{L}}{\partial c_2} \\
\frac{\partial \mathfrak{L}}{\partial c_3} & \frac{\partial \mathfrak{L}}{\partial c_4}
\end{bmatrix}
\end{align*}
By comparing the individual partial derivatives computed earlier, we can say-

\begin{align*}
\begin{bmatrix}
\frac{\partial \mathfrak{L}}{\partial x_1} & \frac{\partial \mathfrak{L}}{\partial x_2} \\
\frac{\partial \mathfrak{L}}{\partial x_3} & \frac{\partial \mathfrak{L}}{\partial x_4}
\end{bmatrix} &= \begin{bmatrix}
a_1 & a_3 \\
a_2 & a_4
\end{bmatrix} \times \begin{bmatrix}
\frac{\partial \mathfrak{L}}{\partial c_1} & \frac{\partial \mathfrak{L}}{\partial c_2} \\
\frac{\partial \mathfrak{L}}{\partial c_3} & \frac{\partial \mathfrak{L}}{\partial c_4}
\end{bmatrix}\\
\implies \begin{bmatrix}
\frac{\partial \mathfrak{L}}{\partial x_1} & \frac{\partial \mathfrak{L}}{\partial x_2} \\
\frac{\partial \mathfrak{L}}{\partial x_3} & \frac{\partial \mathfrak{L}}{\partial x_4}
\end{bmatrix} &= A^T \times \begin{bmatrix}
\frac{\partial \mathfrak{L}}{\partial c_1} & \frac{\partial \mathfrak{L}}{\partial c_2} \\
\frac{\partial \mathfrak{L}}{\partial c_3} & \frac{\partial \mathfrak{L}}{\partial c_4}
\end{bmatrix}\\
\implies \frac{\partial \mathfrak{L}}{\partial X} &= A^T \times \frac{\partial \mathfrak{L}}{\partial C}\\
\therefore \frac{\partial C}{\partial X} &= A^T \qed
\end{align*}


\end{document}
