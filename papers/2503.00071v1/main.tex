\pdfoutput=1

\documentclass[11pt]{article}

\usepackage{acl}

\usepackage{amsmath}
\usepackage{amssymb}
\usepackage{times}
\usepackage{latexsym}

\usepackage{enumitem} %
\usepackage{booktabs} %

\usepackage[T1]{fontenc}

\usepackage[utf8]{inputenc}

\usepackage{microtype}

\usepackage{inconsolata}

\usepackage{graphicx}
\usepackage{natbib}
\usepackage{subcaption}
\usepackage{authblk}




\newcommand{\cmark}{\ding{51}}%
\newcommand{\xmark}{\ding{55}}%
\newcommand{\R}{\mathbb{R}}
\newcommand{\D}{\mathbb{D}}
\newcommand{\etal}{\textit{et al.}}
\newcommand{\ie}{\textit{i.e.}}
\newcommand{\eg}{\textit{e.g.}}

\newcommand{\raq}[1]{\textcolor{blue}{\small [RF: #1]}}
\newcommand{\esam}[1]{\textcolor{red}{\small [EG: #1]}}
\newcommand{\bulat}[1]{\textcolor{cyan}{\small [BK: #1]}}
\renewcommand\Authands{, } 

\title{\textit{I see what you mean} \\
Co-Speech Gestures for Reference Resolution in Multimodal Dialogue}

\author[1,2]{\textbf{Esam Ghaleb}}
\author[3]{\textbf{Bulat Khaertdinov}}
\author[1,2]{\textbf{Asl\i~Özy\"{u}rek}}
\author[4]{\textbf{Raquel Fern\'andez}}

\affil[1]{Multimodal Language Department, Max Planck Institute for Psycholinguistics}
\affil[2]{Donders Institute for Brain, Cognition and Behaviour, Radboud University}
\affil[3]{Department of Advanced Computing Sciences, Maastricht University}
\affil[4]{Institute for Logic, Language and Computation, University of Amsterdam}
\affil{\small \textbf{Correspondence:} \href{mailto:esam.ghaleb@mpi.nl}{esam.ghaleb@mpi.nl}}

% \author{
%  \textbf{Esam Ghaleb\textsuperscript{1, 2}},
%  \textbf{Bulat Khaertdinov\textsuperscript{3}},
%  \textbf{Asl\i~Özy\"{u}rek\textsuperscript{1,2}},
%  \textbf{Raquel Fern\'andez\textsuperscript{4}}
% \\
%  \textsuperscript{1}Multimodal Language Department, Max Planck Institute for Psycholinguistics \\
%  \textsuperscript{2}Donders Institute for Brain, Cognition and Behaviour, Radboud University\\
%   \textsuperscript{3}Department of Advanced Computing Sciences, Maastricht University\\
%   \textsuperscript{4}Institute for Logic, Language and Computation, University of Amsterdam\\
%  \small{
%    \textbf{Correspondence:} \href{mailto:esam.ghaleb@mpi.nl}{esam.ghaleb@mpi.nl}
%  }
% }

\begin{document}
\maketitle
\begin{abstract}
In face-to-face interaction, we use multiple modalities, including speech and gestures, to communicate information and resolve references to objects. However, how representational co-speech gestures refer to objects remains understudied from a computational perspective. In this work, we address this gap by introducing a multimodal reference resolution task centred on representational gestures, while simultaneously tackling the challenge of learning robust gesture embeddings. We propose a self-supervised pre-training approach to gesture representation learning that grounds body movements in spoken language. Our experiments show that the learned embeddings align with expert annotations and have significant predictive power. Moreover, reference resolution accuracy further improves when (1) using multimodal gesture representations, even when speech is unavailable at inference time, and (2) leveraging dialogue history. Overall, our findings highlight the complementary roles of gesture and speech in reference resolution, offering a step towards more naturalistic models of human-machine interaction.
\end{abstract}

\begin{figure}
    \centering
    \includegraphics[width=1\linewidth]{figures/ACL_CABB_Setup.pdf}
    \caption{Example from the CABB dataset \cite{rasenberg2022primacy}, illustrating how participants resolve references through speech and gestures in face-to-face dialogue. The speaker on the right says ``there is a circle on the front'' while performing a representational 
    gesture. 
    The object is shown for illustration but not visible to the listener; the orange highlight 
    marks the referent as annotated by experts. 
    Our work draws on these interactions to model \emph{multimodal reference resolution}.}
    \label{fig:example_interaction}
\end{figure}

\section{Introduction}

Referring to objects is common in everyday communication. 
In face-to-face interaction, when we need to collaborate on new tasks or refer to new objects, we rely on verbal (\ie, speech) and non-verbal (\eg, gestures and gaze) cues to describe salient object features and direct the listener's attention.
Among the non-verbal cues are \textit{representational co-speech gestures}, \ie, iconic hand movements semantically and pragmatically related to co-occurring speech \cite{kendon2004gesture}. 
Studies have shown that representational gestures facilitate language comprehension \cite{drijvers2017visual, arbona2023semantically} and help listeners identify referents more quickly than speech alone \cite{campana2005real}.
Along with speech, gestures are used to refer to novel objects and build shared understanding \cite{rasenberg2022primacy,Akamine2024sp}, as shown in Figure~\ref{fig:example_interaction}. 
These multimodal abilities are inherent ingredients of our communicative interactions \cite{ozyurek2014hearing}. Hence, developing computational approaches that can interpret such cues is important for naturalistic human-machine collaboration: in situated dialogue, an artificial agent must recognize multimodal inputs and speaker references to meet human needs \cite{moon2020situated, kontogiorgos2018multimodal}. 

However, how to computationally represent and interpret co-speech gestures remains an understudied problem, particularly within Natural Language Processing. Other gestural forms, such as deictic gestures, \ie, pointing \cite{gatt-paggio-2013-empirical, KenningtonSchlangen2017, chen2021yourefit, kontogiorgos2018multimodal} or beat gestures, \ie, rhythmic movements without semantic content \cite[e.g.,][]{sinclair2021linguistic, abzaliev2022towards}, have received some attention. 
However, the challenges posed by iconic, representational gestures and their contribution to reference resolution have hardly been tackled from a data-driven perspective. 

In this paper, we propose an approach to learning embeddings for representational gestures that exploits not only the body movements that make up a gesture, but also the semantically related speech that is typically produced simultaneously with it. 
We combine contemporary self-supervised learning techniques to train a Transformer-based gesture encoder and ground it in information from features extracted by text or speech large language models. Then, we test the effectiveness of the pre-trained gesture embeddings in the downstream task of reference resolution in face-to-face dialogue, showing that gestures---as learned by our proposed multimodal approach---have significant predictive power that complements the verbal modality. More concretely, we make the following contributions:

\begin{itemize}[leftmargin=10pt, itemsep=-1pt, topsep=0pt] %

\item We propose three model architectures for gesture representation learning that exploit a version of the motion encoder DSTFormer \cite{zhu2023motionbert}, which we adapt to allow for the integration of speech through cross-modal attention.
\item We show that the resulting pre-trained gesture embeddings are aligned with expert knowledge present in manual annotations, clearly surpassing earlier approaches to gesture representation learning \cite{ghaleb2024learning}.
\item We introduce a novel multimodal reference resolution task and demonstrate that learning gesture representations by jointly exploiting body movements and the semantics of the concurrent speech results in more accurate models, even when speech is not available at prediction time. 
\item Our reference resolution experiments also show that leveraging dialogue history improves model prediction and that, when speech is present at test time, gestures provide complementary information that enhances reference resolution accuracy. 
\item Our experiments make use of the CABB dataset \cite{rasenberg2022primacy,eijk2022cabb}, collected by cognitive scientists. We make available the pre-processed data and the code to reproduce all our results via a public GitHub repository, providing valuable resources to  community.\footnote{\href{https://github.com/EsamGhaleb/MultimodalReferenceResolution}{https://github.com/EsamGhaleb/ReferenceResolution}}
\end{itemize}








\section{Related Work}
\label{sec:related}

\paragraph{Learning multimodal representations.}
Despite the importance of gestures in multimodal communication, learning gesture representations remains challenging and understudied in both computer vision and NLP.  Some existing work has used formal approaches to integrate gestures into discourse semantics \cite{lascarides2009formal,lai2024encoding}, while a few other works have employed data-driven methods. For example, \citet{abzaliev2022towards} jointly learned gesture and word embeddings from TED talks using contrastive learning, and showed that function words, discourse markers, and the language of the speaker can be predicted from non-representational gestures. 
Self-supervised contrastive learning techniques \cite{chen2020simple,radford2021learning} have been widely adopted in the field of multimedia to learn representations of human movements from skeletal joint coordinates unimodally \cite{thoker2021skeleton,zhu2023motionbert} and in combination with other data modalities \cite{brinzea2022contrastive, liu2024multi}, while \cite{lee2021crossmodal} used self-supervised learning to learn gesture embeddings as a pre-training stage for gesture generation.

Our approach to learning gesture representations is most closely related to the preliminary work of \citet{ghaleb2024learning}, who proposed to learn embeddings for representational gestures by grounding them in co-occurring speech. We substantially extend this work by replacing their skeleton encoder with a Transformer-based encoder, allowing us to integrate not only speech but also text-based semantic embeddings with higher temporal granularity and using a much larger amount of data samples. Furthermore, unlike this work, we exploit the learned gesture embeddings for the downstream task of reference resolution, here formulated as the problem of identifying the object referred to by a gesture in face-to-face dialogue.


\paragraph{Reference resolution in dialogue.}

Reference resolution in dialogue has mostly been modelled as the task of identifying the referent of text-based linguistic expressions, ignoring non-verbal cues. %
For example, \citet{skantze2022collie} proposed COLLIE, a continual learning method that adjusts language embeddings to accommodate new language use for new referents; in an earlier study  \cite{shore2018using}, the authors found that leveraging dialogue history in the form of previous referring expressions improves model prediction, similarly to \citet{takmaz2020refer}. 
Resolving referring linguistic expressions in the visual
modality has also been studied in computer vision thanks to datasets such as ReferIt~\cite{kazemzadeh2014referitgame}, Flicker30k Entities~\cite{plummer2015flickr30k},
and Visual Genome~\cite{krishna2017visual}, which map referring expressions to regions in an image.

In this work, we focus on reference resolution in face-to-face communication, where linguistic expressions interact with non-verbal signals like gestures. The large majority of work in this domain has been concerned with deictic pointing gestures. For instance, \citet{kennington2017simple} combined linguistic information with gaze and deictic gestures by treating them as separate resolution models and then fusing their predictions via interpolation. Similarly, \citet{kontogiorgos2018multimodal} used multisensory input in a collaborative assembly task to assess the contribution of various cues--such as eye gaze, head direction, and pointing gestures--to reference resolution. They found that deictic gestures, when combined with speech, reliably located objects, while gaze and head direction were only useful for approximating the general location of the intended object when paired with speech. More recently, within the computer vision community, \citet{chen2021yourefit} found that referential expressions were more discriminative when both visual context and pointing gestures were considered, compared to using visual context alone.

In this paper, we tackle reference resolution by means of iconic representational gestures rather than pointing, calling attention to the importance of modelling such gestures to identify objects in multimodal interaction.

\section{Data}
\label{sec:data}
For our study, we use the CABB dataset \cite{eijk2022cabb,rasenberg2022primacy}, which consists of face-to-face conversations in Dutch between two dialogue participants who play a reference game. The setup is shown in Figure \ref{fig:example_interaction}. The participants' task is to identify 16 objects without conventional names that are made up of different geometrical parts (see Appendix~\ref{app:objects}). Each dyad plays the game for six rounds, exchanging the roles of `director' (who describes one of the target objects) and `matcher' (who attempts to identify the director's intended referent among the 16 candidate objects displayed on a screen). 
The participants are free to communicate as they like, which elicits spontaneous speech and gestures. Speakers were video recorded from different angles and we make use of the semi-frontal views shown in Figure \ref{fig:example_interaction}, as well as the audio recordings.

We use two different subsets of this data, which we refer to as CABB-S and CABB-L, plus an extension of the latter which we call CABB-XL: 

\paragraph{CABB-S} \cite{rasenberg2022primacy}
consists of 19 dialogues by 38 individuals, corresponding to over 8 hours of recordings. The dataset includes manual speech transcriptions and manual segmentation of gesture strokes, 
with 4949 gesture segments in total. 
Approximately 97\% of these segments are accompanied by concurrent speech.
CABB-S also includes manual annotations of gesture strokes with 
two types of information:\footnote{For further details, see \citet{rasenberg2022primacy}.} 

\begin{itemize}[itemsep=0pt,topsep=0pt,left=0pt] %
    \item \textbf{Referent:} the object subpart referred to by the gesture. The candidate objects and their sub-parts are shown in  Appendix~\ref{app:objects}, Figure \ref{fig:all_object_and_subparts}.
\item \textbf{Form similarity:} 419 pairs of gestures with the same referent are annotated with five low-level binary features indicating whether two semantically related gestures are similar regarding shape, movement, rotation, position, and use of hands.
\end{itemize}

\noindent
We use CABB-S for evaluating our gesture representation pre-training approach (Section~\ref{sec:eval_pretraining}) and for the experiments on reference resolution (Section~\ref{sec:ref_resolution}). 

\paragraph{CABB-L} \cite{eijk2022cabb}
contains an additional 49 dialogues by 98 different subjects, with about 42,5 hours of recordings. It is therefore much larger than CABB-S. Only 42 dialogues are manually transcribed and no manual annotations regarding gestures are present. To identify gestures, we use the segmentation model by \citet{ghaleb2024le}, resulting in 30k automatically segmented gestures. To increase the amount of data available for pre-training, we oversample 
by selecting 1-second time windows overlapping more than 50\% with the automatically segmented gestures. This results in approximately 400k data samples, which we refer to as \textbf{CABB-XL}. 
We use Whisper-X \cite{bain2022whisperx} to automatically generate speech transcriptions when manual transcriptions are unavailable. 83\% of the gestures are accompanied by speech. We use CABB-L and CABB-XL for pre-training the models introduced in Section~\ref{sec:models}. 




\paragraph{Pre-processing}

To process body movements, we apply the procedure used by \citet{ghaleb2024learning} to CABB-S, CABB-L, and CABB-XL. Concretely, we sample 1-second time windows centered around each segmented gesture and use MMPose \cite{sengupta2020mm} to extract skeletal information, \ie, 2D keypoint coordinates 
for 27 upper body and hand joints. 
To process the verbal modality, we extract 2-second windows centered around the sampled gestures to account for the temporal asynchrony between speech and gestures \cite{holler2019multimodal}, and use both the raw speech and the transcriptions as described in the next section. 
 










\section{Gesture Representation Learning}
\label{sec:models}


\begin{figure*}
    \centering
    \includegraphics[width=0.99\linewidth]{figures/cross-modal_approach.pdf}
    \caption{Our multimodal-X architecture. The left branch encodes semantic information (text or speech) and fuses it with skeleton embeddings via the proposed cross-attention blocks in our adapted skeleton encoder. The architecture is trained by minimizing contrastive losses.}
    \label{fig:multimodal-x_approach}
\end{figure*}

In this section, we present our approach to learning robust gesture representations in a self-supervised fashion. To do so in the context of \textit{multimodal} communication, we experiment with three types of input: gestures themselves (\ie, skeletal information corresponding to body movement), raw speech, and text-based semantics (Section~\ref{subsec:encoders}). We propose three pre-training model architectures that use these input types to different degrees and with different multimodal integration strategies (Section~\ref{sec:architectures}). We train these models on CABB-L/XL and evaluate them against expert annotations using the unseen gestures in CABB-S (Section~\ref{sec:eval_pretraining}). In Section~\ref{sec:ref_resolution}, we then test the effectiveness of our pre-training approach for the task of reference resolution.  

\subsection{Modality Encoders}
\label{subsec:encoders}

We use three encoders
to extract representations of speech, text, and body movements, respectively.

\paragraph{Speech.} 
As speech encoder, we use multilingual wav2vec-2 (version  \texttt{wav2vec2-xlsr-300}), a masked-language model pre-trained on a large number of speech datasets in multiple languages \cite{baevski2020wav2vec, conneau2020unsupervised}. Similarly to  \citet{pepino2021emotion}, we aggregate the embeddings across all Transformer layers using a learnable weighted average and pass the output through two point-wise CNN layers to fuse signals along the temporal dimension. 

\paragraph{Semantics.} 
Although wav2vec-2 representations may capture diverse linguistic properties including prosody, phonetics, and to some extent semantics \cite{tsai-etal-2022-superb, zaiem2025speech}, they are less semantically rich than word embeddings learned from text. Therefore, we also experiment with the word embeddings 
from a pre-trained Dutch BERT-based model \cite{devries2019bertje}.

\paragraph{Skeleton.}
We adapt DSTFormer \cite{zhu2023motionbert} to encode sequences of body movements. 
The original model has two parallel branches: one applies temporal self-attention followed by spatial self-attention, and the other one spatial followed by temporal. 
To reduce overhead, in each encoder, we keep only one temporal layer and one spatial layer in each branch and replace the second layer with an optional cross-attention module. This optional cross-attention takes semantic or speech embeddings as keys and values, as schematically illustrated in Figure~\ref{fig:multimodal-x_approach}. 



\subsection{Model Architectures}
\label{sec:architectures}

We propose three pre-training strategies to learn gesture representations. The first one is unimodal, in the sense that it learns representations only considering the body movements that make up a gesture. The other two (multimodal and multimodal-x) are motivated by a more holistic conception of co-speech gestures as multimodal acts \cite{HollerLevinson2019, ozyurek2014hearing}, and therefore exploit both skeletal and concurrent verbal input. We describe the gist of each architecture here and provide further technical details in Appendix~\ref{app:models}.

\paragraph{Unimodal architecture.}
This model jointly optimizes a masked reconstruction loss and a unimodal contrastive loss.  For the former, we follow the original procedure for pre-training DSTFormer \cite{zhu2023motionbert} by randomly masking portions of the 2D keypoint skeletal input and learning to reconstruct them. The unimodal contrastive loss pulls representations of two views of augmented skeletal data closer while pushing them away from other negative samples in a batch.
A detailed diagram of this architecture is shown in Figure \ref{fig:unimodal_approach}, Appendix~\ref{app:models}.

\paragraph{Multimodal architecture.}
This model combines the two losses from the unimodal architecture with a multimodal contrastive loss 
that integrates information from skeletons and either speech or semantics. For the latter, we use a CLIP-like contrastive objective \cite{radford2021learning} mapping global representations of gestures and co-occurring utterances (as either raw speech or semantics) in a joint feature space.

\paragraph{Multimodal-X architecture.}
This approach is optimized by a combination of two contrastive losses as shown in Figure \ref{fig:multimodal-x_approach}. The first (blue line in the figure) aligns skeletal embeddings with global semantic representations.
The second loss (green line), in turn, aligns unimodal gesture representations with fused gesture--crossmodal embeddings. These embeddings are obtained by ingesting semantic information (e.g. text tokens) into our adapted DSTFormer via cross-attention blocks, previously introduced in Section \ref{subsec:encoders}.


\paragraph{Training and implementation details.}
We train the three architectures above using CABB-L and CABB-XL, which allows us to test the impact of increasing the size of the training data. In Appendix \ref{app:implementation_details}, we provide further details about the backbone models, projection heads, and the parameters used in the learning objectives, along with the implementation details.

\begin{figure}
    \centering
    \includegraphics[width=1\linewidth]{figures/correlation_impact_of_dataset_size.pdf}
    \caption{Spearman correlation between the number of form features shared by a pair of gestures and their cosine similarity using embeddings from skeleton-speech, skeleton-semantics, and unimodal models. Pre-training was conducted on CABB-L and CABB-XL, while the correlation scores were computed on CABB-S. All coefficients are statistically significant ($p\ll0.05$).}
    \label{fig:cabb_correlation}
\end{figure}


\subsection{Evaluation}
\label{sec:eval_pretraining}
We evaluate the gesture representations learned with our pre-training architectures using the CABB-S dataset, which contains manually annotated information on gestures unseen during model pre-training.
To monitor pre-training progress and save the best model variants across epochs, we conducted evaluations using form similarity as correlation.
Following \citet{ghaleb2024learning}, we compute Spearman's correlation between the number of form features a pair of gestures share according to experts' annotations and cosine similarity between the gestures' learned representations. 



Figure~\ref{fig:cabb_correlation} shows the correlation results, yielded by the best models obtained during pre-training. The figure shows that the variants with the highest performance are the multimodal-x and multimodal architectures where gesture representations are jointly learned with text-based semantics from concurrent speech, using the large CABB-XL as training data. The other model variants---unimodal, multimodal (with raw speech), and multimodal-x with raw speech---do not benefit as much from an increase in the amount of training data. In fact, when exploiting raw speech, the best correlation coefficient is obtained with the multimodal architecture and CABB-L.



\paragraph{Comparison with related work.} 
We compare our models against the framework by \citet{ghaleb2024learning}. This work proposed a pre-training approach to gesture representation learning based on Spatio-Temporal Graph Convolutional Networks (ST-GCN) using unimodal and multimodal contrastive learning with co-occurring raw speech. For comparability with our approach, when reproducing this framework, in addition to raw speech we extend it to also use text-based semantics and pre-train it on CABB-XL.\footnote{Due to the architecture of ST-GCN, it is not possible to combine it with the multimodal-x architecture introduced in our work (Figure \ref{fig:multimodal-x_approach}).} 
The comparison of correlation coefficients for form similarity is shown in Figure~\ref{fig:cabb-xl_baseline_corr}. As can be observed, the gesture representations learned by our transformer-based encoder are more aligned with form-based expert annotations as evidenced by higher correlation values across the board. 


\begin{figure}[t]
    \centering
    \includegraphics[width=1\linewidth]{figures/CABB_XL_correlation_against_baseline.pdf}
    \caption{Pre-training on CABB-XL: comparing best models against \cite{ghaleb2024learning}. All Spearman correlation coefficients $\rho$ are statistically significant ($p\ll0.05$).}
    \label{fig:cabb-xl_baseline_corr}
\end{figure}


Overall, the correlation analysis indicates that the best pre-training strategies combine skeletal data with semantic information—using either multimodal or multimodal-x approaches—specifically when trained on a large dataset like CABB-XL. In the next section, we focus on model variants trained on CABB-XL that use semantic embeddings, with unimodal models serving as a baseline.

\section{Reference Resolution}
\label{sec:ref_resolution}
An important question in situated interactions is to what extent representational gestures complement or supplement speech in reference resolution. Here we shed light on this question by leveraging our pre-trained gesture models for the downstream task of reference resolution. We investigate whether models that have learned gestures by exploiting multimodal information (from body movements and concurrent speech) have more predictive power than models that represent gestures exclusively from body movements. Moreover, we test whether gestures contribute complementary information to the verbal modality when identifying referents. 

More concretely, given a gesture unseen during pre-training, we investigate two scenarios: 
(1) We measure how accurate a reference resolution system that only has access to the gesture (\ie, to skeletal input) is at predicting the gesture's referent. Here the gesture embedding is extracted zero-shot with our models, some of which exploited raw speech or text semantics during pre-training---but importantly verbal input is not provided at inference time in this scenario. 
(2) In the second scenario, at inference time the reference resolution system has access to both the unseen gesture to be resolved and any concurrent speech. We operationalise this by concatenating the gesture embedding extracted with our models and a semantic embedding of the concurrent speech, and then measure whether this leads to higher reference resolution accuracy than only exploiting the semantics of concurrent speech.

\paragraph{Resolution model and evaluation setup}
The resolution model leverages our model architectures pre-trained on CABB-XL without any fine-tuning. The model is implemented as a multi‐class MLP classifier with two hidden layers of size 300 and 150, respectively, and it is trained on CABB-S. Given a gesture encoded with our pre-trained models, we train the MLP to predict one referent among 70 possible object sub‐parts (see Appendix~\ref{app:objects} for details) using a batch size of 32 and a learning rate of $10^{-4}$ with the Adam optimizer for 200 epochs. 
Recall from Section~\ref{sec:data} that each dialogue consists of six rounds. We use leave-one-round-out cross-validation, holding out the gestures in one round as a test set and training on the gestures in the remaining rounds across all dialogues in CABB-S. 
We use accuracy as an evaluation metric. Chance accuracy is approximately 1.4\%, and a model using random gesture embeddings (without access to our pre-trained models) achieves around 3\% accuracy.



\subsection{Gesture-Only Reference Resolution}
\label{subsect:unimodal_evals}

\begin{figure}
    \centering
    \includegraphics[width=1\linewidth]{figures/reference_resolution_with_gestures.pdf}
    \caption{Average reference resolution accuracies for gesture-only embeddings are shown for unimodal, multimodal, and multimodal‑x representations. The multimodal and multimodal‑x models are pre‑trained with text‑based semantic input, and the embeddings are derived only from skeletal data.}
\label{fig:reference_resolution_from_gestures}
\end{figure}

We first evaluate the resolution model when it only has access to gestural information (\ie, skeletal data) as input.
As shown in Figure~\ref{fig:reference_resolution_from_gestures}, 
when a gesture is encoded with our unimodal model, the average resolution accuracy is 16\%, significantly above the random baselines. 
Using embeddings from models 
that were pre-trained jointly with text-based semantics
significantly increases resolution accuracy to around 19\%, with no statistically significant difference between the multimodal and multimodal-x approaches.
These results show that our pre-trained gesture representations capture information that is useful to identify referents. Moreover, they indicate that learning gesture representations by jointly exploiting body movements and the semantics of co-occurring speech enhances their reference resolution potential, even when information about concurrent speech is not provided at prediction time.

\begin{figure}
    \centering
    \includegraphics[width=1\linewidth]{figures/reference_resolution_with_semantic.pdf}
     \caption{Average accuracies for reference resolution using gestures and co-occurring verbal information are reported for unimodal gesture embeddings, multimodal‑x gesture embeddings, semantic embeddings, and their concatenated representations. The multimodal‑x gesture embeddings are learned through pre‑training with semantic embeddings.}
    \label{fig:reference_resolution_with_semantic}
\end{figure}

\subsection{Reference Resolution with Gestural and Co-occurring Verbal Information}
\label{subsec:resolution2}
Next, we assess the resolution model when it has access to gestures and the speech co-occurring with them. speech %
We operationalise this scenario by concatenating gesture embedding (extracted with either unimodal or multimodal-x pre-trained models)\footnote{Given the lack of statistically significant difference between multimodal and multimodal-x in Section~\ref{subsect:unimodal_evals} Fig.~\ref{fig:reference_resolution_from_gestures}, we focus on the multimodal-x model for this experiment. } with a semantic embedding derived from the transcribed co-occurring speech using BERTje \cite{devries2019bertje}. 
The results are shown in Figure~\ref{fig:reference_resolution_with_semantic}, where we also include a condition where the reference resolution model exclusively uses the concurrent speech in the form of a semantic embedding. In that condition, resolution accuracy is 24\%. That is, the concurrent information present in the verbal modality has stronger predictive power to identify referents than body movements alone, which is to be expected in spoken conversations. 
Importantly, when both the verbal and gestural modalities are combined, we observe a significant increase in reference resolution accuracy, reaching 31\% when gestures are encoded with our multimodal-x model, pre-trained with text-based semantics. These findings confirm the complementary roles of gesture and speech in reference resolution and highlight the benefits of exploiting such complementarity for gesture representation learning.


 
\subsection{Impact of Dialogue History}
\label{subsec:history}

It is well known that, in referential communication tasks, participants tend to reuse the same referential expressions over the course of the dialogue, creating dialogue-specific conventions \cite{clark1986referring, brennan1996conceptual}. Such `alignment' has been observed for both speech and gestures \cite{Akamine2024sp}. Hence, a system tasked with identifying the referent of a gesture is expected to achieve higher accuracy if it has access to other gestures previously used within the same dialogue than if such dialogue history is not available. 
To test whether our approach to gesture representation learning gives rise to this pattern, we train two versions of our reference resolution model: a \textit{baseline} model and a \textit{dialogue-specific} model.  The baseline model is trained on all dialogues in CABB-S, except the target dialogue---thus, referent prediction for the gestures in the target dialogue is carried out without dialogue history. In contrast, the dialogue-specific model is progressively adapted over the dialogue rounds: \ie, in round 1 it is identical to the baseline model, but 
by round $n$, it has additionally seen data from all previous dialogue rounds up to $n-1$.
To keep the amount of training data comparable between the baseline and dialogue-specific models, when new round data is added, we proportionally reduce the amount of data drawn from other dialogues during the re-training of the dialogue-specific model. As a result, both models are trained on an identical number of samples in every round.


\begin{figure}
    \centering    \includegraphics[width=1\linewidth]{figures/classification_with_dialogue_history.pdf}
    \caption{Average reference resolution accuracy over dialogue rounds using gesture embeddings from the unimodal (green) and multimodal-x (blue) pre-trained models. Dotted lines are used for the corresponding baseline models.}
    \label{fig:classification_with_dialogue_history}
\end{figure}

To isolate the impact of dialogue history on gesture-driven reference resolution, in this experiment we focus on identifying referents with only gestural information as input (as in Section~\ref{subsect:unimodal_evals}),\footnote{The impact of dialogue history on text-based reference resolution has already been extensively studied, \eg, \citet{shore2018using,haber2019photobook,takmaz2020refer, hawkins-etal-2020-continual}.} 
comparing our unimodal and multimodal-x pre-trained models. 
Figure~\ref{fig:classification_with_dialogue_history} shows that as the conversation unfolds over the rounds, the dialogue-specific reference resolution models clearly outperform the baselines (dotted lines). Our statistical analysis shows that there is a significant difference in accuracy between the two (independent t-test yielding $t=2.9$, $p\ll 0.05$ for both the unimodal and multimodal-x models). It also indicates that the pattern of increased accuracy, as more dialogue history becomes available, is more pronounced when the gestures are encoded with the multimodal-x pre-trained model (Spearman correlation between accuracy values and dialogue round numbers: $\rho=0.32$ for the unimodal model and $\rho=0.35$ for the multimodal-x model, with $p\ll 0.05$ in both instances).\footnote{Note that there is no statistically significant difference in accuracy between rounds 5 and 6, despite the apparent drop.}


Overall, the results indicate that our gesture representations, particularly when learned via multimodal-x pre-training, encode features that capture the subtle increase in similarity between gestures referring to the same object within a given dialogue. In other words, to some extent the models capture gesture entrainment, which results in an advantage for the task of reference resolution.  From a practical point of view, this suggests that access to dialogue history can be an asset to agents deployed with a gesture resolution model.  

















\section{Conclusion}

In this work, we have studied representational co-speech gestures in collaborative dialogue, using an existing dataset of face-to-face interactions collected by cognitive scientists. We introduced a novel reference resolution task formulated as the problem of identifying the intended referent of a co-speech gesture, while addressing key challenges in gesture representation learning. We proposed a self-supervised Transformer-based approach to learning pre-trained gesture embeddings by jointly exploiting skeletal information and concurrent language encoded with text or speech large language models. Our experiments showed that the resulting gesture embeddings effectively contribute to reference resolution. Representing gestures by exclusively exploiting skeletal information has significant predictive power, and grounding body movements in concurrent speech during pre-training further improves resolution accuracy, even when speech is not provided at test time. Moreover, we showed that reference resolution from representational gestures can benefit from having access to gestures previously used within a dialogue, thus providing empirical support for the presence of gestural entrainment in face-to-face interaction. 

Taken together, these findings emphasize the multimodal character of conversation \cite{HollerLevinson2019, ozyurek2014hearing} and the importance of capturing the complementarity between gestures and speech in naturalistic human-machine interaction. 
Further work is needed to test the extent to which the proposed pre-training approach would transfer to other referential domains and other tasks---a step we leave to future research. 




\section*{Limitations}
The current work focuses on Dutch-speaking task-oriented dialogues, thus contributing to linguistic diversity in the current English-centric NLP landscape. We nevertheless acknowledge that it is an open question how well the proposed models may generalise to other languages, cultural contexts, tasks, as well as open-domain dialogues. On the methodological front, while we employ and adapt a state-of-the-art motion encoder and show that our pre-training objectives and architecture choices are effective, further optimisation and integration with more advanced speech and semantic encoders may give additional improvements.
Finally, the current approach does not ground the reference resolution models in the visual properties of the referents, which might further improve resolution performance. We leave this aspect to future work.

\section*{Acknowledgements}
We thank the members of the Dialogue Modelling Group at the ILLC, University of Amsterdam, and the Multimodal Language Department at the Max Planck Institute for Psycholinguistics for their feedback.
Raquel Fern\'{a}ndez acknowledges support from the European Research Council, ERC Consolidator Grant No.~819455.

\bibliography{custom}

\appendix
\subsection{Lloyd-Max Algorithm}
\label{subsec:Lloyd-Max}
For a given quantization bitwidth $B$ and an operand $\bm{X}$, the Lloyd-Max algorithm finds $2^B$ quantization levels $\{\hat{x}_i\}_{i=1}^{2^B}$ such that quantizing $\bm{X}$ by rounding each scalar in $\bm{X}$ to the nearest quantization level minimizes the quantization MSE. 

The algorithm starts with an initial guess of quantization levels and then iteratively computes quantization thresholds $\{\tau_i\}_{i=1}^{2^B-1}$ and updates quantization levels $\{\hat{x}_i\}_{i=1}^{2^B}$. Specifically, at iteration $n$, thresholds are set to the midpoints of the previous iteration's levels:
\begin{align*}
    \tau_i^{(n)}=\frac{\hat{x}_i^{(n-1)}+\hat{x}_{i+1}^{(n-1)}}2 \text{ for } i=1\ldots 2^B-1
\end{align*}
Subsequently, the quantization levels are re-computed as conditional means of the data regions defined by the new thresholds:
\begin{align*}
    \hat{x}_i^{(n)}=\mathbb{E}\left[ \bm{X} \big| \bm{X}\in [\tau_{i-1}^{(n)},\tau_i^{(n)}] \right] \text{ for } i=1\ldots 2^B
\end{align*}
where to satisfy boundary conditions we have $\tau_0=-\infty$ and $\tau_{2^B}=\infty$. The algorithm iterates the above steps until convergence.

Figure \ref{fig:lm_quant} compares the quantization levels of a $7$-bit floating point (E3M3) quantizer (left) to a $7$-bit Lloyd-Max quantizer (right) when quantizing a layer of weights from the GPT3-126M model at a per-tensor granularity. As shown, the Lloyd-Max quantizer achieves substantially lower quantization MSE. Further, Table \ref{tab:FP7_vs_LM7} shows the superior perplexity achieved by Lloyd-Max quantizers for bitwidths of $7$, $6$ and $5$. The difference between the quantizers is clear at 5 bits, where per-tensor FP quantization incurs a drastic and unacceptable increase in perplexity, while Lloyd-Max quantization incurs a much smaller increase. Nevertheless, we note that even the optimal Lloyd-Max quantizer incurs a notable ($\sim 1.5$) increase in perplexity due to the coarse granularity of quantization. 

\begin{figure}[h]
  \centering
  \includegraphics[width=0.7\linewidth]{sections/figures/LM7_FP7.pdf}
  \caption{\small Quantization levels and the corresponding quantization MSE of Floating Point (left) vs Lloyd-Max (right) Quantizers for a layer of weights in the GPT3-126M model.}
  \label{fig:lm_quant}
\end{figure}

\begin{table}[h]\scriptsize
\begin{center}
\caption{\label{tab:FP7_vs_LM7} \small Comparing perplexity (lower is better) achieved by floating point quantizers and Lloyd-Max quantizers on a GPT3-126M model for the Wikitext-103 dataset.}
\begin{tabular}{c|cc|c}
\hline
 \multirow{2}{*}{\textbf{Bitwidth}} & \multicolumn{2}{|c|}{\textbf{Floating-Point Quantizer}} & \textbf{Lloyd-Max Quantizer} \\
 & Best Format & Wikitext-103 Perplexity & Wikitext-103 Perplexity \\
\hline
7 & E3M3 & 18.32 & 18.27 \\
6 & E3M2 & 19.07 & 18.51 \\
5 & E4M0 & 43.89 & 19.71 \\
\hline
\end{tabular}
\end{center}
\end{table}

\subsection{Proof of Local Optimality of LO-BCQ}
\label{subsec:lobcq_opt_proof}
For a given block $\bm{b}_j$, the quantization MSE during LO-BCQ can be empirically evaluated as $\frac{1}{L_b}\lVert \bm{b}_j- \bm{\hat{b}}_j\rVert^2_2$ where $\bm{\hat{b}}_j$ is computed from equation (\ref{eq:clustered_quantization_definition}) as $C_{f(\bm{b}_j)}(\bm{b}_j)$. Further, for a given block cluster $\mathcal{B}_i$, we compute the quantization MSE as $\frac{1}{|\mathcal{B}_{i}|}\sum_{\bm{b} \in \mathcal{B}_{i}} \frac{1}{L_b}\lVert \bm{b}- C_i^{(n)}(\bm{b})\rVert^2_2$. Therefore, at the end of iteration $n$, we evaluate the overall quantization MSE $J^{(n)}$ for a given operand $\bm{X}$ composed of $N_c$ block clusters as:
\begin{align*}
    \label{eq:mse_iter_n}
    J^{(n)} = \frac{1}{N_c} \sum_{i=1}^{N_c} \frac{1}{|\mathcal{B}_{i}^{(n)}|}\sum_{\bm{v} \in \mathcal{B}_{i}^{(n)}} \frac{1}{L_b}\lVert \bm{b}- B_i^{(n)}(\bm{b})\rVert^2_2
\end{align*}

At the end of iteration $n$, the codebooks are updated from $\mathcal{C}^{(n-1)}$ to $\mathcal{C}^{(n)}$. However, the mapping of a given vector $\bm{b}_j$ to quantizers $\mathcal{C}^{(n)}$ remains as  $f^{(n)}(\bm{b}_j)$. At the next iteration, during the vector clustering step, $f^{(n+1)}(\bm{b}_j)$ finds new mapping of $\bm{b}_j$ to updated codebooks $\mathcal{C}^{(n)}$ such that the quantization MSE over the candidate codebooks is minimized. Therefore, we obtain the following result for $\bm{b}_j$:
\begin{align*}
\frac{1}{L_b}\lVert \bm{b}_j - C_{f^{(n+1)}(\bm{b}_j)}^{(n)}(\bm{b}_j)\rVert^2_2 \le \frac{1}{L_b}\lVert \bm{b}_j - C_{f^{(n)}(\bm{b}_j)}^{(n)}(\bm{b}_j)\rVert^2_2
\end{align*}

That is, quantizing $\bm{b}_j$ at the end of the block clustering step of iteration $n+1$ results in lower quantization MSE compared to quantizing at the end of iteration $n$. Since this is true for all $\bm{b} \in \bm{X}$, we assert the following:
\begin{equation}
\begin{split}
\label{eq:mse_ineq_1}
    \tilde{J}^{(n+1)} &= \frac{1}{N_c} \sum_{i=1}^{N_c} \frac{1}{|\mathcal{B}_{i}^{(n+1)}|}\sum_{\bm{b} \in \mathcal{B}_{i}^{(n+1)}} \frac{1}{L_b}\lVert \bm{b} - C_i^{(n)}(b)\rVert^2_2 \le J^{(n)}
\end{split}
\end{equation}
where $\tilde{J}^{(n+1)}$ is the the quantization MSE after the vector clustering step at iteration $n+1$.

Next, during the codebook update step (\ref{eq:quantizers_update}) at iteration $n+1$, the per-cluster codebooks $\mathcal{C}^{(n)}$ are updated to $\mathcal{C}^{(n+1)}$ by invoking the Lloyd-Max algorithm \citep{Lloyd}. We know that for any given value distribution, the Lloyd-Max algorithm minimizes the quantization MSE. Therefore, for a given vector cluster $\mathcal{B}_i$ we obtain the following result:

\begin{equation}
    \frac{1}{|\mathcal{B}_{i}^{(n+1)}|}\sum_{\bm{b} \in \mathcal{B}_{i}^{(n+1)}} \frac{1}{L_b}\lVert \bm{b}- C_i^{(n+1)}(\bm{b})\rVert^2_2 \le \frac{1}{|\mathcal{B}_{i}^{(n+1)}|}\sum_{\bm{b} \in \mathcal{B}_{i}^{(n+1)}} \frac{1}{L_b}\lVert \bm{b}- C_i^{(n)}(\bm{b})\rVert^2_2
\end{equation}

The above equation states that quantizing the given block cluster $\mathcal{B}_i$ after updating the associated codebook from $C_i^{(n)}$ to $C_i^{(n+1)}$ results in lower quantization MSE. Since this is true for all the block clusters, we derive the following result: 
\begin{equation}
\begin{split}
\label{eq:mse_ineq_2}
     J^{(n+1)} &= \frac{1}{N_c} \sum_{i=1}^{N_c} \frac{1}{|\mathcal{B}_{i}^{(n+1)}|}\sum_{\bm{b} \in \mathcal{B}_{i}^{(n+1)}} \frac{1}{L_b}\lVert \bm{b}- C_i^{(n+1)}(\bm{b})\rVert^2_2  \le \tilde{J}^{(n+1)}   
\end{split}
\end{equation}

Following (\ref{eq:mse_ineq_1}) and (\ref{eq:mse_ineq_2}), we find that the quantization MSE is non-increasing for each iteration, that is, $J^{(1)} \ge J^{(2)} \ge J^{(3)} \ge \ldots \ge J^{(M)}$ where $M$ is the maximum number of iterations. 
%Therefore, we can say that if the algorithm converges, then it must be that it has converged to a local minimum. 
\hfill $\blacksquare$


\begin{figure}
    \begin{center}
    \includegraphics[width=0.5\textwidth]{sections//figures/mse_vs_iter.pdf}
    \end{center}
    \caption{\small NMSE vs iterations during LO-BCQ compared to other block quantization proposals}
    \label{fig:nmse_vs_iter}
\end{figure}

Figure \ref{fig:nmse_vs_iter} shows the empirical convergence of LO-BCQ across several block lengths and number of codebooks. Also, the MSE achieved by LO-BCQ is compared to baselines such as MXFP and VSQ. As shown, LO-BCQ converges to a lower MSE than the baselines. Further, we achieve better convergence for larger number of codebooks ($N_c$) and for a smaller block length ($L_b$), both of which increase the bitwidth of BCQ (see Eq \ref{eq:bitwidth_bcq}).


\subsection{Additional Accuracy Results}
%Table \ref{tab:lobcq_config} lists the various LOBCQ configurations and their corresponding bitwidths.
\begin{table}
\setlength{\tabcolsep}{4.75pt}
\begin{center}
\caption{\label{tab:lobcq_config} Various LO-BCQ configurations and their bitwidths.}
\begin{tabular}{|c||c|c|c|c||c|c||c|} 
\hline
 & \multicolumn{4}{|c||}{$L_b=8$} & \multicolumn{2}{|c||}{$L_b=4$} & $L_b=2$ \\
 \hline
 \backslashbox{$L_A$\kern-1em}{\kern-1em$N_c$} & 2 & 4 & 8 & 16 & 2 & 4 & 2 \\
 \hline
 64 & 4.25 & 4.375 & 4.5 & 4.625 & 4.375 & 4.625 & 4.625\\
 \hline
 32 & 4.375 & 4.5 & 4.625& 4.75 & 4.5 & 4.75 & 4.75 \\
 \hline
 16 & 4.625 & 4.75& 4.875 & 5 & 4.75 & 5 & 5 \\
 \hline
\end{tabular}
\end{center}
\end{table}

%\subsection{Perplexity achieved by various LO-BCQ configurations on Wikitext-103 dataset}

\begin{table} \centering
\begin{tabular}{|c||c|c|c|c||c|c||c|} 
\hline
 $L_b \rightarrow$& \multicolumn{4}{c||}{8} & \multicolumn{2}{c||}{4} & 2\\
 \hline
 \backslashbox{$L_A$\kern-1em}{\kern-1em$N_c$} & 2 & 4 & 8 & 16 & 2 & 4 & 2  \\
 %$N_c \rightarrow$ & 2 & 4 & 8 & 16 & 2 & 4 & 2 \\
 \hline
 \hline
 \multicolumn{8}{c}{GPT3-1.3B (FP32 PPL = 9.98)} \\ 
 \hline
 \hline
 64 & 10.40 & 10.23 & 10.17 & 10.15 &  10.28 & 10.18 & 10.19 \\
 \hline
 32 & 10.25 & 10.20 & 10.15 & 10.12 &  10.23 & 10.17 & 10.17 \\
 \hline
 16 & 10.22 & 10.16 & 10.10 & 10.09 &  10.21 & 10.14 & 10.16 \\
 \hline
  \hline
 \multicolumn{8}{c}{GPT3-8B (FP32 PPL = 7.38)} \\ 
 \hline
 \hline
 64 & 7.61 & 7.52 & 7.48 &  7.47 &  7.55 &  7.49 & 7.50 \\
 \hline
 32 & 7.52 & 7.50 & 7.46 &  7.45 &  7.52 &  7.48 & 7.48  \\
 \hline
 16 & 7.51 & 7.48 & 7.44 &  7.44 &  7.51 &  7.49 & 7.47  \\
 \hline
\end{tabular}
\caption{\label{tab:ppl_gpt3_abalation} Wikitext-103 perplexity across GPT3-1.3B and 8B models.}
\end{table}

\begin{table} \centering
\begin{tabular}{|c||c|c|c|c||} 
\hline
 $L_b \rightarrow$& \multicolumn{4}{c||}{8}\\
 \hline
 \backslashbox{$L_A$\kern-1em}{\kern-1em$N_c$} & 2 & 4 & 8 & 16 \\
 %$N_c \rightarrow$ & 2 & 4 & 8 & 16 & 2 & 4 & 2 \\
 \hline
 \hline
 \multicolumn{5}{|c|}{Llama2-7B (FP32 PPL = 5.06)} \\ 
 \hline
 \hline
 64 & 5.31 & 5.26 & 5.19 & 5.18  \\
 \hline
 32 & 5.23 & 5.25 & 5.18 & 5.15  \\
 \hline
 16 & 5.23 & 5.19 & 5.16 & 5.14  \\
 \hline
 \multicolumn{5}{|c|}{Nemotron4-15B (FP32 PPL = 5.87)} \\ 
 \hline
 \hline
 64  & 6.3 & 6.20 & 6.13 & 6.08  \\
 \hline
 32  & 6.24 & 6.12 & 6.07 & 6.03  \\
 \hline
 16  & 6.12 & 6.14 & 6.04 & 6.02  \\
 \hline
 \multicolumn{5}{|c|}{Nemotron4-340B (FP32 PPL = 3.48)} \\ 
 \hline
 \hline
 64 & 3.67 & 3.62 & 3.60 & 3.59 \\
 \hline
 32 & 3.63 & 3.61 & 3.59 & 3.56 \\
 \hline
 16 & 3.61 & 3.58 & 3.57 & 3.55 \\
 \hline
\end{tabular}
\caption{\label{tab:ppl_llama7B_nemo15B} Wikitext-103 perplexity compared to FP32 baseline in Llama2-7B and Nemotron4-15B, 340B models}
\end{table}

%\subsection{Perplexity achieved by various LO-BCQ configurations on MMLU dataset}


\begin{table} \centering
\begin{tabular}{|c||c|c|c|c||c|c|c|c|} 
\hline
 $L_b \rightarrow$& \multicolumn{4}{c||}{8} & \multicolumn{4}{c||}{8}\\
 \hline
 \backslashbox{$L_A$\kern-1em}{\kern-1em$N_c$} & 2 & 4 & 8 & 16 & 2 & 4 & 8 & 16  \\
 %$N_c \rightarrow$ & 2 & 4 & 8 & 16 & 2 & 4 & 2 \\
 \hline
 \hline
 \multicolumn{5}{|c|}{Llama2-7B (FP32 Accuracy = 45.8\%)} & \multicolumn{4}{|c|}{Llama2-70B (FP32 Accuracy = 69.12\%)} \\ 
 \hline
 \hline
 64 & 43.9 & 43.4 & 43.9 & 44.9 & 68.07 & 68.27 & 68.17 & 68.75 \\
 \hline
 32 & 44.5 & 43.8 & 44.9 & 44.5 & 68.37 & 68.51 & 68.35 & 68.27  \\
 \hline
 16 & 43.9 & 42.7 & 44.9 & 45 & 68.12 & 68.77 & 68.31 & 68.59  \\
 \hline
 \hline
 \multicolumn{5}{|c|}{GPT3-22B (FP32 Accuracy = 38.75\%)} & \multicolumn{4}{|c|}{Nemotron4-15B (FP32 Accuracy = 64.3\%)} \\ 
 \hline
 \hline
 64 & 36.71 & 38.85 & 38.13 & 38.92 & 63.17 & 62.36 & 63.72 & 64.09 \\
 \hline
 32 & 37.95 & 38.69 & 39.45 & 38.34 & 64.05 & 62.30 & 63.8 & 64.33  \\
 \hline
 16 & 38.88 & 38.80 & 38.31 & 38.92 & 63.22 & 63.51 & 63.93 & 64.43  \\
 \hline
\end{tabular}
\caption{\label{tab:mmlu_abalation} Accuracy on MMLU dataset across GPT3-22B, Llama2-7B, 70B and Nemotron4-15B models.}
\end{table}


%\subsection{Perplexity achieved by various LO-BCQ configurations on LM evaluation harness}

\begin{table} \centering
\begin{tabular}{|c||c|c|c|c||c|c|c|c|} 
\hline
 $L_b \rightarrow$& \multicolumn{4}{c||}{8} & \multicolumn{4}{c||}{8}\\
 \hline
 \backslashbox{$L_A$\kern-1em}{\kern-1em$N_c$} & 2 & 4 & 8 & 16 & 2 & 4 & 8 & 16  \\
 %$N_c \rightarrow$ & 2 & 4 & 8 & 16 & 2 & 4 & 2 \\
 \hline
 \hline
 \multicolumn{5}{|c|}{Race (FP32 Accuracy = 37.51\%)} & \multicolumn{4}{|c|}{Boolq (FP32 Accuracy = 64.62\%)} \\ 
 \hline
 \hline
 64 & 36.94 & 37.13 & 36.27 & 37.13 & 63.73 & 62.26 & 63.49 & 63.36 \\
 \hline
 32 & 37.03 & 36.36 & 36.08 & 37.03 & 62.54 & 63.51 & 63.49 & 63.55  \\
 \hline
 16 & 37.03 & 37.03 & 36.46 & 37.03 & 61.1 & 63.79 & 63.58 & 63.33  \\
 \hline
 \hline
 \multicolumn{5}{|c|}{Winogrande (FP32 Accuracy = 58.01\%)} & \multicolumn{4}{|c|}{Piqa (FP32 Accuracy = 74.21\%)} \\ 
 \hline
 \hline
 64 & 58.17 & 57.22 & 57.85 & 58.33 & 73.01 & 73.07 & 73.07 & 72.80 \\
 \hline
 32 & 59.12 & 58.09 & 57.85 & 58.41 & 73.01 & 73.94 & 72.74 & 73.18  \\
 \hline
 16 & 57.93 & 58.88 & 57.93 & 58.56 & 73.94 & 72.80 & 73.01 & 73.94  \\
 \hline
\end{tabular}
\caption{\label{tab:mmlu_abalation} Accuracy on LM evaluation harness tasks on GPT3-1.3B model.}
\end{table}

\begin{table} \centering
\begin{tabular}{|c||c|c|c|c||c|c|c|c|} 
\hline
 $L_b \rightarrow$& \multicolumn{4}{c||}{8} & \multicolumn{4}{c||}{8}\\
 \hline
 \backslashbox{$L_A$\kern-1em}{\kern-1em$N_c$} & 2 & 4 & 8 & 16 & 2 & 4 & 8 & 16  \\
 %$N_c \rightarrow$ & 2 & 4 & 8 & 16 & 2 & 4 & 2 \\
 \hline
 \hline
 \multicolumn{5}{|c|}{Race (FP32 Accuracy = 41.34\%)} & \multicolumn{4}{|c|}{Boolq (FP32 Accuracy = 68.32\%)} \\ 
 \hline
 \hline
 64 & 40.48 & 40.10 & 39.43 & 39.90 & 69.20 & 68.41 & 69.45 & 68.56 \\
 \hline
 32 & 39.52 & 39.52 & 40.77 & 39.62 & 68.32 & 67.43 & 68.17 & 69.30  \\
 \hline
 16 & 39.81 & 39.71 & 39.90 & 40.38 & 68.10 & 66.33 & 69.51 & 69.42  \\
 \hline
 \hline
 \multicolumn{5}{|c|}{Winogrande (FP32 Accuracy = 67.88\%)} & \multicolumn{4}{|c|}{Piqa (FP32 Accuracy = 78.78\%)} \\ 
 \hline
 \hline
 64 & 66.85 & 66.61 & 67.72 & 67.88 & 77.31 & 77.42 & 77.75 & 77.64 \\
 \hline
 32 & 67.25 & 67.72 & 67.72 & 67.00 & 77.31 & 77.04 & 77.80 & 77.37  \\
 \hline
 16 & 68.11 & 68.90 & 67.88 & 67.48 & 77.37 & 78.13 & 78.13 & 77.69  \\
 \hline
\end{tabular}
\caption{\label{tab:mmlu_abalation} Accuracy on LM evaluation harness tasks on GPT3-8B model.}
\end{table}

\begin{table} \centering
\begin{tabular}{|c||c|c|c|c||c|c|c|c|} 
\hline
 $L_b \rightarrow$& \multicolumn{4}{c||}{8} & \multicolumn{4}{c||}{8}\\
 \hline
 \backslashbox{$L_A$\kern-1em}{\kern-1em$N_c$} & 2 & 4 & 8 & 16 & 2 & 4 & 8 & 16  \\
 %$N_c \rightarrow$ & 2 & 4 & 8 & 16 & 2 & 4 & 2 \\
 \hline
 \hline
 \multicolumn{5}{|c|}{Race (FP32 Accuracy = 40.67\%)} & \multicolumn{4}{|c|}{Boolq (FP32 Accuracy = 76.54\%)} \\ 
 \hline
 \hline
 64 & 40.48 & 40.10 & 39.43 & 39.90 & 75.41 & 75.11 & 77.09 & 75.66 \\
 \hline
 32 & 39.52 & 39.52 & 40.77 & 39.62 & 76.02 & 76.02 & 75.96 & 75.35  \\
 \hline
 16 & 39.81 & 39.71 & 39.90 & 40.38 & 75.05 & 73.82 & 75.72 & 76.09  \\
 \hline
 \hline
 \multicolumn{5}{|c|}{Winogrande (FP32 Accuracy = 70.64\%)} & \multicolumn{4}{|c|}{Piqa (FP32 Accuracy = 79.16\%)} \\ 
 \hline
 \hline
 64 & 69.14 & 70.17 & 70.17 & 70.56 & 78.24 & 79.00 & 78.62 & 78.73 \\
 \hline
 32 & 70.96 & 69.69 & 71.27 & 69.30 & 78.56 & 79.49 & 79.16 & 78.89  \\
 \hline
 16 & 71.03 & 69.53 & 69.69 & 70.40 & 78.13 & 79.16 & 79.00 & 79.00  \\
 \hline
\end{tabular}
\caption{\label{tab:mmlu_abalation} Accuracy on LM evaluation harness tasks on GPT3-22B model.}
\end{table}

\begin{table} \centering
\begin{tabular}{|c||c|c|c|c||c|c|c|c|} 
\hline
 $L_b \rightarrow$& \multicolumn{4}{c||}{8} & \multicolumn{4}{c||}{8}\\
 \hline
 \backslashbox{$L_A$\kern-1em}{\kern-1em$N_c$} & 2 & 4 & 8 & 16 & 2 & 4 & 8 & 16  \\
 %$N_c \rightarrow$ & 2 & 4 & 8 & 16 & 2 & 4 & 2 \\
 \hline
 \hline
 \multicolumn{5}{|c|}{Race (FP32 Accuracy = 44.4\%)} & \multicolumn{4}{|c|}{Boolq (FP32 Accuracy = 79.29\%)} \\ 
 \hline
 \hline
 64 & 42.49 & 42.51 & 42.58 & 43.45 & 77.58 & 77.37 & 77.43 & 78.1 \\
 \hline
 32 & 43.35 & 42.49 & 43.64 & 43.73 & 77.86 & 75.32 & 77.28 & 77.86  \\
 \hline
 16 & 44.21 & 44.21 & 43.64 & 42.97 & 78.65 & 77 & 76.94 & 77.98  \\
 \hline
 \hline
 \multicolumn{5}{|c|}{Winogrande (FP32 Accuracy = 69.38\%)} & \multicolumn{4}{|c|}{Piqa (FP32 Accuracy = 78.07\%)} \\ 
 \hline
 \hline
 64 & 68.9 & 68.43 & 69.77 & 68.19 & 77.09 & 76.82 & 77.09 & 77.86 \\
 \hline
 32 & 69.38 & 68.51 & 68.82 & 68.90 & 78.07 & 76.71 & 78.07 & 77.86  \\
 \hline
 16 & 69.53 & 67.09 & 69.38 & 68.90 & 77.37 & 77.8 & 77.91 & 77.69  \\
 \hline
\end{tabular}
\caption{\label{tab:mmlu_abalation} Accuracy on LM evaluation harness tasks on Llama2-7B model.}
\end{table}

\begin{table} \centering
\begin{tabular}{|c||c|c|c|c||c|c|c|c|} 
\hline
 $L_b \rightarrow$& \multicolumn{4}{c||}{8} & \multicolumn{4}{c||}{8}\\
 \hline
 \backslashbox{$L_A$\kern-1em}{\kern-1em$N_c$} & 2 & 4 & 8 & 16 & 2 & 4 & 8 & 16  \\
 %$N_c \rightarrow$ & 2 & 4 & 8 & 16 & 2 & 4 & 2 \\
 \hline
 \hline
 \multicolumn{5}{|c|}{Race (FP32 Accuracy = 48.8\%)} & \multicolumn{4}{|c|}{Boolq (FP32 Accuracy = 85.23\%)} \\ 
 \hline
 \hline
 64 & 49.00 & 49.00 & 49.28 & 48.71 & 82.82 & 84.28 & 84.03 & 84.25 \\
 \hline
 32 & 49.57 & 48.52 & 48.33 & 49.28 & 83.85 & 84.46 & 84.31 & 84.93  \\
 \hline
 16 & 49.85 & 49.09 & 49.28 & 48.99 & 85.11 & 84.46 & 84.61 & 83.94  \\
 \hline
 \hline
 \multicolumn{5}{|c|}{Winogrande (FP32 Accuracy = 79.95\%)} & \multicolumn{4}{|c|}{Piqa (FP32 Accuracy = 81.56\%)} \\ 
 \hline
 \hline
 64 & 78.77 & 78.45 & 78.37 & 79.16 & 81.45 & 80.69 & 81.45 & 81.5 \\
 \hline
 32 & 78.45 & 79.01 & 78.69 & 80.66 & 81.56 & 80.58 & 81.18 & 81.34  \\
 \hline
 16 & 79.95 & 79.56 & 79.79 & 79.72 & 81.28 & 81.66 & 81.28 & 80.96  \\
 \hline
\end{tabular}
\caption{\label{tab:mmlu_abalation} Accuracy on LM evaluation harness tasks on Llama2-70B model.}
\end{table}

%\section{MSE Studies}
%\textcolor{red}{TODO}


\subsection{Number Formats and Quantization Method}
\label{subsec:numFormats_quantMethod}
\subsubsection{Integer Format}
An $n$-bit signed integer (INT) is typically represented with a 2s-complement format \citep{yao2022zeroquant,xiao2023smoothquant,dai2021vsq}, where the most significant bit denotes the sign.

\subsubsection{Floating Point Format}
An $n$-bit signed floating point (FP) number $x$ comprises of a 1-bit sign ($x_{\mathrm{sign}}$), $B_m$-bit mantissa ($x_{\mathrm{mant}}$) and $B_e$-bit exponent ($x_{\mathrm{exp}}$) such that $B_m+B_e=n-1$. The associated constant exponent bias ($E_{\mathrm{bias}}$) is computed as $(2^{{B_e}-1}-1)$. We denote this format as $E_{B_e}M_{B_m}$.  

\subsubsection{Quantization Scheme}
\label{subsec:quant_method}
A quantization scheme dictates how a given unquantized tensor is converted to its quantized representation. We consider FP formats for the purpose of illustration. Given an unquantized tensor $\bm{X}$ and an FP format $E_{B_e}M_{B_m}$, we first, we compute the quantization scale factor $s_X$ that maps the maximum absolute value of $\bm{X}$ to the maximum quantization level of the $E_{B_e}M_{B_m}$ format as follows:
\begin{align}
\label{eq:sf}
    s_X = \frac{\mathrm{max}(|\bm{X}|)}{\mathrm{max}(E_{B_e}M_{B_m})}
\end{align}
In the above equation, $|\cdot|$ denotes the absolute value function.

Next, we scale $\bm{X}$ by $s_X$ and quantize it to $\hat{\bm{X}}$ by rounding it to the nearest quantization level of $E_{B_e}M_{B_m}$ as:

\begin{align}
\label{eq:tensor_quant}
    \hat{\bm{X}} = \text{round-to-nearest}\left(\frac{\bm{X}}{s_X}, E_{B_e}M_{B_m}\right)
\end{align}

We perform dynamic max-scaled quantization \citep{wu2020integer}, where the scale factor $s$ for activations is dynamically computed during runtime.

\subsection{Vector Scaled Quantization}
\begin{wrapfigure}{r}{0.35\linewidth}
  \centering
  \includegraphics[width=\linewidth]{sections/figures/vsquant.jpg}
  \caption{\small Vectorwise decomposition for per-vector scaled quantization (VSQ \citep{dai2021vsq}).}
  \label{fig:vsquant}
\end{wrapfigure}
During VSQ \citep{dai2021vsq}, the operand tensors are decomposed into 1D vectors in a hardware friendly manner as shown in Figure \ref{fig:vsquant}. Since the decomposed tensors are used as operands in matrix multiplications during inference, it is beneficial to perform this decomposition along the reduction dimension of the multiplication. The vectorwise quantization is performed similar to tensorwise quantization described in Equations \ref{eq:sf} and \ref{eq:tensor_quant}, where a scale factor $s_v$ is required for each vector $\bm{v}$ that maps the maximum absolute value of that vector to the maximum quantization level. While smaller vector lengths can lead to larger accuracy gains, the associated memory and computational overheads due to the per-vector scale factors increases. To alleviate these overheads, VSQ \citep{dai2021vsq} proposed a second level quantization of the per-vector scale factors to unsigned integers, while MX \citep{rouhani2023shared} quantizes them to integer powers of 2 (denoted as $2^{INT}$).

\subsubsection{MX Format}
The MX format proposed in \citep{rouhani2023microscaling} introduces the concept of sub-block shifting. For every two scalar elements of $b$-bits each, there is a shared exponent bit. The value of this exponent bit is determined through an empirical analysis that targets minimizing quantization MSE. We note that the FP format $E_{1}M_{b}$ is strictly better than MX from an accuracy perspective since it allocates a dedicated exponent bit to each scalar as opposed to sharing it across two scalars. Therefore, we conservatively bound the accuracy of a $b+2$-bit signed MX format with that of a $E_{1}M_{b}$ format in our comparisons. For instance, we use E1M2 format as a proxy for MX4.

\begin{figure}
    \centering
    \includegraphics[width=1\linewidth]{sections//figures/BlockFormats.pdf}
    \caption{\small Comparing LO-BCQ to MX format.}
    \label{fig:block_formats}
\end{figure}

Figure \ref{fig:block_formats} compares our $4$-bit LO-BCQ block format to MX \citep{rouhani2023microscaling}. As shown, both LO-BCQ and MX decompose a given operand tensor into block arrays and each block array into blocks. Similar to MX, we find that per-block quantization ($L_b < L_A$) leads to better accuracy due to increased flexibility. While MX achieves this through per-block $1$-bit micro-scales, we associate a dedicated codebook to each block through a per-block codebook selector. Further, MX quantizes the per-block array scale-factor to E8M0 format without per-tensor scaling. In contrast during LO-BCQ, we find that per-tensor scaling combined with quantization of per-block array scale-factor to E4M3 format results in superior inference accuracy across models. 


\end{document}
