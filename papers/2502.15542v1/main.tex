    %%
%% This is file `sample-sigconf-authordraft.tex',
%% generated with the docstrip utility.
%%
%% The original source files were:
%%
%% samples.dtx  (with options: `all,proceedings,bibtex,authordraft')
%% 
%% IMPORTANT NOTICE:
%% 
%% For the copyright see the source file.
%% 
%% Any modified versions of this file must be renamed
%% with new filenames distinct from sample-sigconf-authordraft.tex.
%% 
%% For distribution of the original source see the terms
%% for copying and modification in the file samples.dtx.
%% 
%% This generated file may be distributed as long as the
%% original source files, as listed above, are part of the
%% same distribution. (The sources need not necessarily be
%% in the same archive or directory.)
%%
%%
%% Commands for TeXCount
%TC:macro \cite [option:text,text]
%TC:macro \citep [option:text,text]
%TC:macro \citet [option:text,text]
%TC:envir table 0 1
%TC:envir table* 0 1
%TC:envir tabular [ignore] word
%TC:envir displaymath 0 word
%TC:envir math 0 word
%TC:envir comment 0 0
%%
%%
%% The first command in your LaTeX source must be the \documentclass
%% command.
%%
%% For submission and review of your manuscript please change the
%% command to \documentclass[manuscript, screen, review]{acmart}.
%%
%% When submitting camera ready or to TAPS, please change the command
%% to \documentclass[sigconf]{acmart} or whichever template is required
%% for your publication.
%%
%%
% \documentclass[sigconf,authordraft,article]{acmart}
\documentclass[sigconf]{acmart}

\usepackage[utf8]{inputenc}
\usepackage{tabularray}
\usepackage{booktabs}
\usepackage{multirow}
\usepackage{makecell}
\usepackage{float}
\usepackage{color}
\settopmatter{printacmref=false}
\setcopyright{none}
\renewcommand\footnotetextcopyrightpermission[1]{}
\pagestyle{plain}

%%
%% \BibTeX command to typeset BibTeX logo in the docs
\AtBeginDocument{%
  \providecommand\BibTeX{{%
    Bib\TeX}}}

%% Rights management information.  This information is sent to you
%% when you complete the rights form.  These commands have SAMPLE
%% values in them; it is your responsibility as an author to replace
%% the commands and values with those provided to you when you
%% complete the rights form.
\setcopyright{acmlicensed}
\copyrightyear{2018}
\acmYear{2018}
\acmDOI{XXXXXXX.XXXXXXX}

%% These commands are for a PROCEEDINGS abstract or paper.
\acmConference[Conference acronym 'XX]{Make sure to enter the correct
  conference title from your rights confirmation emai}{June 03--05,
  2018}{Woodstock, NY}
%%
%%  Uncomment \acmBooktitle if the title of the proceedings is different
%%  from ``Proceedings of ...''!
%%
%%\acmBooktitle{Woodstock '18: ACM Symposium on Neural Gaze Detection,
%%  June 03--05, 2018, Woodstock, NY}
\acmISBN{978-1-4503-XXXX-X/18/06}


%%
%% Submission ID.
%% Use this when submitting an article to a sponsored event. You'll
%% receive a unique submission ID from the organizers
%% of the event, and this ID should be used as the parameter to this command.
%%\acmSubmissionID{123-A56-BU3}

%%
%% For managing citations, it is recommended to use bibliography
%% files in BibTeX format.
%%
%% You can then either use BibTeX with the ACM-Reference-Format style,
%% or BibLaTeX with the acmnumeric or acmauthoryear sytles, that include
%% support for advanced citation of software artefact from the
%% biblatex-software package, also separately available on CTAN.
%%
%% Look at the sample-*-biblatex.tex files for templates showcasing
%% the biblatex styles.
%%

%%
%% The majority of ACM publications use numbered citations and
%% references.  The command \citestyle{authoryear} switches to the
%% "author year" style.
%%
%% If you are preparing content for an event
%% sponsored by ACM SIGGRAPH, you must use the "author year" style of
%% citations and references.
%% Uncommenting
%% the next command will enable that style.
%%\citestyle{acmauthoryear}


%%
%% end of the preamble, start of the body of the document source.
\begin{document}

%%
%% The "title" command has an optional parameter,
%% allowing the author to define a "short title" to be used in page headers.
% \title{PTMRec: A Prompt Tuning-Based Multimodal Recommendation Framework with Training Decoupling and Knowledge Transfer}
% \title{PTMRec: Parameter-efficient Tuning for Multimodal Recommendation}
\title{Bridging Domain Gaps between Pretrained Multimodal Models and Recommendations}
%%
%% The "author" command and its associated commands are used to define
%% the authors and their affiliations.
%% Of note is the shared affiliation of the first two authors, and the
%% "authornote" and "authornotemark" commands
%% used to denote shared contribution to the research.
% \author{Ben Trovato}
% \authornote{Both authors contributed equally to this research.}
% \email{trovato@corporation.com}
% \orcid{1234-5678-9012}
% \author{G.K.M. Tobin}
% \authornotemark[1]
% \email{webmaster@marysville-ohio.com}
% \affiliation{%
%   \institution{Institute for Clarity in Documentation}
%   \city{Dublin}
%   \state{Ohio}
%   \country{USA}
% }


\author{Wenyu Zhang}
\authornote{Co-first authors with equal contribution to refining the theory and experimental design.}
\orcid{0009-0001-1457-1707}
% \authornotemark[1]
\affiliation{%
  \institution{University of Science and Technology of China}
  \city{Hefei}
  \country{China}}
\email{wenyuz@mail.ustc.edu.cn}


\author{Jie Luo}
\authornotemark[1]

\affiliation{%
  \institution{University of Science and Technology of China}
  \city{Hefei}
  \country{China}}
\email{luojie2000@mail.ustc.edu.cn}

\author{Xinming Zhang}
\authornote{Corresponding authors.}

% \authornotemark[2]
% \authornote{Corresponding authors.}
\orcid{0000-0002-8136-6834}
% \authornotemark[1]
\affiliation{%
  \institution{University of Science and Technology of China}
  \city{Hefei}
  \country{China}}
\email{xinming@ustc.edu.cn}

\author{Yuan Fang}
\authornotemark[2]
% \authornote{Corresponding authors.}
% \authornotemark[1]
\affiliation{%
  \institution{Singapore Management University}
  \country{Singapore}}
\email{yfang@smu.edu.sg}


% \author{Lars Th{\o}rv{\"a}ld}
% \affiliation{%
%   \institution{The Th{\o}rv{\"a}ld Group}
%   \city{Hekla}
%   \country{Iceland}}
% \email{larst@affiliation.org}

% \author{Valerie B\'eranger}
% \affiliation{%
%   \institution{Inria Paris-Rocquencourt}
%   \city{Rocquencourt}
%   \country{France}
% }

% \author{Aparna Patel}
% \affiliation{%
%  \institution{Rajiv Gandhi University}
%  \city{Doimukh}
%  \state{Arunachal Pradesh}
%  \country{India}}

% \author{Huifen Chan}
% \affiliation{%
%   \institution{Tsinghua University}
%   \city{Haidian Qu}
%   \state{Beijing Shi}
%   \country{China}}

% \author{Charles Palmer}
% \affiliation{%
%   \institution{Palmer Research Laboratories}
%   \city{San Antonio}
%   \state{Texas}
%   \country{USA}}
% \email{cpalmer@prl.com}

% \author{John Smith}
% \affiliation{%
%   \institution{The Th{\o}rv{\"a}ld Group}
%   \city{Hekla}
%   \country{Iceland}}
% \email{jsmith@affiliation.org}

% \author{Julius P. Kumquat}
% \affiliation{%
%   \institution{The Kumquat Consortium}
%   \city{New York}
%   \country{USA}}
% \email{jpkumquat@consortium.net}

%%
%% By default, the full list of authors will be used in the page
%% headers. Often, this list is too long, and will overlap
%% other information printed in the page headers. This command allows
%% the author to define a more concise list
%% of authors' names for this purpose.
\renewcommand{\shortauthors}{Zhang et al.}

%%
%% The abstract is a short summary of the work to be presented in the
%% article.
% \begin{abstract}


% With the proliferation of multimodal content online, pre-trained visual-language models show great promise for multimodal recommendation. However, direct joint training underperforms baseline models due to significant domain gaps between pre-training and personalized recommendation. we propose \textbf{P}arameter-efficient \textbf{T}uning for \textbf{M}ultimodal \textbf{Rec}ommendation (\textbf{PTMRec}), a framework that bridges this domain gap through a knowledge-guided dual-stage parameter-efficient training strategy. The first stage leverages pre-trained vision-language features to train the recommendation model, while the second stage utilizes this knowledge to guide parameter-efficient fine-tuning of the pre-trained model.Experiments show that PTMRec significantly improves Recall@10 across multiple models—averaging +9.6\% on Baby, +4.0\% on Sports, and +15.0\% on Clothing—while tuning only 20 prompt tokens, offering a practical solution for domain adaptation in multimodal recommendation. The code is available at \href{https://anonymous.4open.science/r/PTMRec-F861}{here}.
% \end{abstract}
\begin{abstract}
With the explosive growth of multimodal content online, pre-trained visual-language models have shown great potential for multimodal recommendation. However, while these models achieve decent performance when applied in a frozen manner, surprisingly, due to significant domain gaps (e.g., feature distribution discrepancy and task objective misalignment) between pre-training and personalized recommendation, adopting a joint training approach instead leads to performance worse than baseline. Existing approaches either rely on simple feature extraction or require computationally expensive full model fine-tuning, struggling to balance effectiveness and efficiency. To tackle these challenges, we propose \textbf{P}arameter-efficient \textbf{T}uning for \textbf{M}ultimodal \textbf{Rec}ommendation (\textbf{PTMRec}), a novel framework that bridges the domain gap between pre-trained models and recommendation systems through a knowledge-guided dual-stage parameter-efficient training strategy. In the first stage, we leverage features extracted from pre-trained vision-language models to facilitate multimodal recommendation training, which enables the recommendation model to adapt to the representation space of pre-trained features while allowing ID embeddings to capture personalized user-item matching patterns. This knowledge is then leveraged in the second stage to guide parameter-efficient fine-tuning of the pre-trained model, thereby achieving effective domain adaptation while maintaining computational efficiency. This framework not only eliminates the need for costly additional pre-training but also flexibly accommodates various parameter-efficient tuning methods. Experiments on three public datasets demonstrate that PTMRec significantly improves recommendation performance (average 10.6\% gain in Recall@10) by training only x\% of parameters compared to the original CLIP model. Our work provides a practical and general framework for addressing domain adaptation challenges in multimodal recommendation.
\end{abstract}
%%
%% The code below is generated by the tool at http://dl.acm.org/ccs.cfm.
%% Please copy and paste the code instead of the example below.
%%
\begin{CCSXML}
<ccs2012>
   <concept>
       <concept_id>10002951.10003227.10003351.10003269</concept_id>
       <concept_desc>Information systems~Collaborative filtering</concept_desc>
       <concept_significance>500</concept_significance>
       </concept>
   <concept>
       <concept_id>10010147.10010178</concept_id>
       <concept_desc>Computing methodologies~Artificial intelligence</concept_desc>
       <concept_significance>500</concept_significance>
       </concept>
 </ccs2012>
\end{CCSXML}

\ccsdesc[500]{Information systems~Collaborative filtering}
\ccsdesc[500]{Computing methodologies~Artificial intelligence}
%%
%% Keywords. The author(s) should pick words that accurately describe
%% the work being presented. Separate the keywords with commas.
\keywords{MultiModal Recommendation, Parameter-efficient Tuning, Knowledge Transfer}
%% A "teaser" image appears between the author and affiliation
%% information and the body of the document, and typically spans the
%% page.
% \begin{teaserfigure}
%   \includegraphics[width=\textwidth]{sampleteaser}
%   \caption{Seattle Mariners at Spring Training, 2010.}
%   \Description{Enjoying the baseball game from the third-base
%   seats. Ichiro Suzuki preparing to bat.}
%   \label{fig:teaser}
% \end{teaserfigure}

% \received{20 February 2007}
% \received[revised]{12 March 2009}
% \received[accepted]{5 June 2009}

%%
%% This command processes the author and affiliation and title
%% information and builds the first part of the formatted document.
\maketitle

\section{Introduction}
Recommender systems play a crucial role in personalized services \cite{50DBLP:journals/csur/ZhangYST19}, with traditional ID-based methods\cite{1_koren2009matrix,2_BPR,3_lightgcn} achieving significant success. However, as online platforms evolve \cite{51DBLP:conf/kdd/YingHCEHL18}, the prevalence of multimodal data challenges conventional approaches \cite{10_DBLP:journals/corr/abs-2302-04473}, driving the development of multimodal recommendation systems \cite{5_DBLP:conf/icmlc/ChenWHT17,6_DBLP:conf/aaai/FuPWXL19,52DBLP:conf/ijcai/DuLLZ22,56DBLP:journals/corr/abs-2411-10332,57DBLP:journals/corr/abs-2405-15304}. Recent advances in multimodal recommendation systems have spawned various approaches utilizing pre-trained models. From the perspective of leveraging pre-trained models, we categorize existing work into three paradigms: (1) Pre-extracted, (2) Pre-train/Fine-tuning, and (3) Pre-extracted with Parameter-Efficient Fine-tuning (PEFT). 


\begin{figure}[t]
% \vspace{5mm}
  \centering
  % \fbox{\rule{0pt}{2in} \rule{0.9\linewidth}{0pt}}
   %\includegraphics[width=0.8\linewidth]{egfigure.eps}
   \includegraphics[width=\linewidth]{figure/fig1.pdf}
   \caption{Modality Feature Extraction Paradigms in Multimodal Recommendation.}
   \label{fig:1}
\end{figure}

First, the pre-extracted paradigm uses pre-trained encoders to extract features offline, which, as shown in Fig.~\ref{fig:1}a, are used to initialize the feature embedding layer of recommendation models. VBPR \cite{7_DBLP:conf/aaai/HeM16} pioneered the introduction of pre-trained CNN features, MMGCN \cite{wei2019mmgcn} enhanced feature interaction through modality-specific graph structures, and Freedom \cite{15_DBLP:conf/mm/ZhouS23} improved performance via contrastive learning and cross-modal attention. However this paradigm mainly relies on early lightweight models (such as VGG-16 \cite{58DBLP:journals/corr/SimonyanZ14a} and Sentence-BERT \cite{18_DBLP:conf/emnlp/ReimersG19}), and the static nature of features limits the potential of recommendation models.

Second, as shown in Fig.~\ref{fig:1}b, the pre-training/fine-tuning paradigm employs pre-trained models such as BERT\cite{19_DBLP:conf/naacl/DevlinCLT19}, ViT~\cite{20_DBLP:conf/iclr/DosovitskiyB0WZ21}, and CLIP~\cite{21_radford2021learning} as modal encoders and fine-tunes them during the recommendation training process \cite{62yang2024exploring}. One line of work directly fine-tunes pre-trained models as modal encoders (e.g., MoRec~\cite{4_DBLP:conf/sigir/YuanYSLFYPN23}). Another line of work addresses domain transfer through source domain pre-training and target domain fine-tuning (e.g., TransRec~\cite{54DBLP:conf/apweb/WangYCJYKWHL24} and PMMRec~\cite{55DBLP:conf/icde/LiDNZGY024}). Notably, AdapterRec~\cite{25_DBLP:conf/wsdm/FuY0YCCZWP24} incorporates parameter-efficient adapters~\cite{27_DBLP:conf/icml/HoulsbyGJMLGAG19} during target domain fine-tuning to improve efficiency. Direct fine-tuning methods inevitably require substantial computational resources \cite{63DBLP:journals/pami/YangZC22}, while domain transfer approaches demand significant resources particularly in the source domain pre-training phase.

Third, the pre-extracted with PEFT paradigm adapts to recommendation scenarios through parameter-efficient methods such as prompts \cite{26_DBLP:conf/acl/LiL20}, adapters \cite{27_DBLP:conf/icml/HoulsbyGJMLGAG19}, and LoRA \cite{28_DBLP:conf/iclr/HuSWALWWC22}. As shown in Fig~\ref{fig:1}c, this paradigm primarily enhances pre-extracted features by introducing lightweight trainable modules. For instance, PromptMM \cite{22_DBLP:conf/www/WeiTXJH24} employs trainable prompt tokens and knowledge distillation to bridge modality content and collaborative signals, while MISSRec \cite{24_DBLP:conf/mm/WangZWWLLYZZX23} adopts multimodal adapters to enhance sequence representation. Furthermore, if PEFT methods can be effectively utilized to transfer powerful pre-trained models to the recommendation domain, it would significantly enhance the performance of recommendation models. However, our experiments show that directly applying PEFT methods for joint training of pre-trained encoders and recommender systems leads to performance degradation (detailed results in Section 3.3 Ablation study), indicating that the substantial domain gap between pre-trained models and recommender systems cannot be effectively bridged through traditional PEFT methods.


In summary, directly applying pre-trained models in existing recommender systems still faces key challenges: while \textbf{pre-extraction} and \textbf{PEFT-enhanced pre-extraction} methods are computationally efficient, they struggle to fully unleash the representational potential of pre-trained models; the \textbf{pre-train/fine-tuning} paradigm shows superior performance but comes with enormous computational costs. The core dilemma lies in: \textbf{how to construct a parameter-efficient paradigm that enables joint training of pre-trained models and recommender systems while maintaining efficiency?}


To address these limitations, we propose a Parameter-efficient Tuning for Multimodal Recommendation framework. As illustrated in Fig. \ref{fig:1}d, PTMRec introduces an innovative knowledge-guided dual-stage parameter-efficient training strategy that effectively bridges the domain gap between pre-trained models and recommendation systems while maintaining computational efficiency. In the first stage (Personalized Preference Learning), we train a lightweight recommendation model with learnable parameters using pre-trained features to capture task-specific knowledge about user preferences and item characteristics. In the second stage (Knowledge-guided Prompt Learning), we guide the tuning of pre-trained models through knowledge transfer optimization using personalized preference knowledge. This two-stage design not only eliminates the need for expensive additional pre-training but also provides a flexible framework that can accommodate various parameter-efficient tuning methods while maintaining the coupling between feature extraction and recommendation objectives.


\section{Method}

\subsection{Preliminaries}
Let $\mathcal{U} = {u_1, u_2, \ldots }$ denote the user set and $\mathcal{I} = {i_1, i_2, \ldots }$ denote the item set. For each user u, their historical interactions form an item set $\mathcal{I}^u$ representing positive feedback. Each item i contains visual and textual modalities ($\mathbf{X}_i^{\text{v}}$ and $\mathbf{X}_i^{\text{t}}$) besides ID information. For representations, we use embeddings: $\mathbf{e}_u^{\text{id}} \in \mathbb{R}^{d}$ and $\mathbf{e}_i^{\text{id}} \in \mathbb{R}^{d}$ for user and item IDs respectively, where d is the embedding dimension. Item multimodal information is represented by $\mathbf{e}_i^{\text{v}}$ and $\mathbf{e}_i^{\text{t}}$ from visual and textual features.



For optimization, recommendation systems typically employ BPR loss:

\begin{equation}
\mathcal{L}_{BPR} = -\sum_{(u,i,j)\in\mathcal{D}} \ln \sigma(f_u(i) - f_u(j))
\end{equation}

where $\sigma(\cdot)$ is the sigmoid function, $(u,i,j)$ represents a training triplet with user u, interacted item i, and non-interacted item j.

Vision-language pre-training typically adopts InfoNCE loss for image-text alignment:
\begin{equation}
\mathcal{L}_{NCE} = -\frac{1}{N}\sum_{i=1}^{N} \log\frac{\exp(sim(\mathbf{v}_i, \mathbf{t}_i)/\tau)}{\sum_{j=1}^{N}\exp(sim(\mathbf{v}_i, \mathbf{t}_j)/\tau)}
\end{equation}

where $\mathbf{v}_i$ and $\mathbf{t}_i$ are visual and textual features, $sim(\cdot,\cdot)$ is cosine similarity, and $\tau$ is temperature. These objectives reflect different goals: BPR focuses on learning personalized preferences through user-item interactions, while InfoNCE aims for general vision-language alignment. 



\begin{figure*}[h!]
  \centering
  % 使用scale选项调整图像大小
  \includegraphics[scale=0.5]{figure/Framework.pdf} % 0.8是新计算的缩放因子

  \caption{The architectures of PTMRec.}
  \label{fig:3}
\end{figure*}


\subsection{Personalized Preference Learning}

As shown in Figure \ref{fig:3}, in the first stage of our framework, we aim to learn basic recommendation patterns while preserving the general semantic knowledge from pre-trained models. Specifically, we utilize a frozen CLIP model for multimodal feature extraction to initialize and train the recommendation model.

For each item i, we first extract its visual and textual features using the pre-trained CLIP model, where $\mathbf{X}_i^{\text{v}}$ and $\mathbf{X}_i^{\text{t}}$ denote the raw image and text description of item i respectively. The image encoding divides the image $\mathbf{X}_i^{\text{v}}$ into $M$ fixed-size blocks, which are projected as image block embeddings 
$\mathbf{E}^{\text{v}} =[\text{e}_0^{\text{v}},\text{e}_1^{\text{v}}, \cdots,\text{e}_N^{\text{v}}]$. A learnable class (CLS) token $\mathbf{c}$ is initialized and input into the image encoder together with the image block embeddings. The image representation $\mathbf{e}_i^{\text{v}}$ is obtained by mapping the class token $\mathbf{c}'$ from the last Transformer layer through $\texttt{VisualProj}$:
\begin{equation} [{\mathbf{c'}},\mathbf{E'}^{\text{v}}]=\text{VisualEnc}([\mathbf{c},\text{E}^{\text{v}}]). \end{equation}
\begin{equation} \mathbf{e}_i^{\text{v}}=\text{VisualProj}({\mathbf{c'}}). \end{equation}

The CLIP text encoder uses a tokenizer to project the text into token $\text{E}^{\text{t}} =[\text{e}_0^{\text{t}},\text{e}_1^{\text{t}}, \cdots,\text{e}_N^{\text{t}}]$. This is input into the text encoder \text{TextEnc}. The final text representation $\mathbf{e}_i^{\text{t}}$ projects the last token output from the last Transformer layer through $\texttt{TextProj}$ into the general multimodal embedding space.
% \begin{equation}
% \mathbf{e}_i^{\text{t}} = \text{TextEncoder}(\mathbf{X}_i^{\text{t}})
% \end{equation}
\begin{equation} \mathbf{E'}^{\text{t}}=\text{TextEnc}(\mathbf{E}^{\text{t}}). \end{equation}
\begin{equation} \mathbf{e}_i^{\text{t}}=\text{TextProj}({\mathbf{e'}_{N}^{\text{t}}}). 
\end{equation}

These encoders are kept frozen during training. Different recommendation models can be flexibly integrated into our framework and trained with their own specific objective functions.




\subsection{Knowledge-guided Prompt Learning}

Building upon the basic recommendation patterns learned in the first stage, we design a knowledge-guided parameter-efficient tuning strategy to enhance model performance. Our framework is flexible and can incorporate various parameter-efficient methods (e.g., LoRA, Adapter, or other methods). Through extensive experimental (Table \ref{tab:PEFT}) analysis balancing computational efficiency and model performance, we find that prompt learning is the most suitable choice for recommendation scenarios. Therefore, we focus on achieving efficient domain adaptation by introducing learnable prompts into CLIP encoders.

As shown in Fig.~\ref{fig:3}, we follow the design of MaPLe \cite{29_DBLP:conf/cvpr/KhattakR0KK23} and insert visual and textual prompts, denoted as $p^m=\langle{p^m_{0},p^m_{1},\cdots,p^m_n}\rangle $, into the first i layers of the modality encoder, where $ m \in (t,v)$. Different from MaPLe, we remove the prompt coupling function for modality alignment since recommendation systems focus more on personalized perception.
\begin{equation}
[{\mathbf{c'}}, \mathbf{E}_l^{\prime \text{v}}, \_] = \text{VisualEnc}_l([\mathbf{c_l},\text{E}_l^{\text{v}}, p^v_l]).   
 ~  l=1, \cdots, i \\
\end{equation}
\begin{equation} 
[{\mathbf{c}^{\prime}},\mathbf{E}_{l+1}^{\prime \text{v}}, p^v_{l+1}] = \text{VisualEnc}_l
([\mathbf{c_{l+1}},\text{E}_{l}^{\text{v}}, p^v_l]).   ~  l=i+1, \cdots, L 
\end{equation}

For the text encoder, it select the last token as the text representation. We insert $p^t$ before the text tokens.
\begin{equation}
[\_, \mathbf{E'}^{\text{t}}] = \text{TextEnc}_l([p^t_l,\text{E}_l^{\text{t}}]).  ~  l=1, \cdots, i 
\end{equation}

\begin{equation}
[p^\top_{l+1},\mathbf{E'}_{l}^{\text{t}}] = \text{TextEnc}_l([p^t_l, \text{E}_l^{\text{t}}]).  ~  l=i+1, \cdots, L 
\end{equation}

In the first i layers of the modality encoder, the modality prompt is only referenced within the current layer, and a new prompt is introduced for each layer. All prompts are initialized using the standard random initialization method. After the i-th transformer layer, the subsequent layers process the output of the previous layer. 

\subsection{Knowledge Transfer Optimization}

To bridge the domain gap between pre-trained models and recommendation models, we propose an optimization strategy based on in-batch knowledge transfer. Specifically, in each training batch, we construct positive user-item pairs from interaction records while treating other items within the batch as negative samples. The user-item ID interaction distributions learned in the first stage contain core recommendation patterns. Through in-batch knowledge transfer, we effectively transfer these patterns to modal feature learning without additional computational overhead. The target distribution is defined as:
\begin{equation}
\text{probs}^{id} = \text{stopgrad}(\text{softmax}(\mathbf{e^u}^\top \cdot \mathbf{e}_{id})),
\end{equation}
where $\mathbf{e}^\top_u \cdot \mathbf{e}_{id}$ computes similarity scores between user and item features, which are normalized into probability distributions through softmax. The stopgrad operation prevents gradients from flowing through ID features, thus maintaining a stable target distribution and avoiding representation collapse \cite{60DBLP:conf/cvpr/ChenH21}.

For modal features, we compute their interaction probability distributions as follows:
\begin{equation}
\text{prob}^{m} = \text{log\_softmax}(\mathbf{e^u}^\top \cdot \text{Linear}(\mathbf{e}^m)), \quad \text{where} \ m \in {(t, v)},
\end{equation}
where we obtain dimension-aligned feature representations by applying linear transformation to modal features $\mathbf{e}^m$. Then, we minimize KL divergence to guide modal features in learning user-item interaction patterns. The specific computation is as follows:
\begin{equation}
\mathcal{L}_{KT} = \text{KL}(\text{prob}^t, \text{prob}^{id}) + \text{KL}(\text{prob}^v, \text{prob}^{id}).
\end{equation}

\begin{table*}[t]
	%\fontsize{9}{10}\selectfont
	\caption{Overall performance comparison by different recommendation methods in terms of Recall and NDCG. Results are averaged over 5 runs with different random seeds. $^{*}$  indicates statistical significance (p < 0.05).}
	\footnotesize
	\label{Table:perform}
	% \renewcommand\arraystretch{1.0}
	\begin{center}
{
	\begin{tabular}{ccccccccccccc}
		\toprule
		\multirow{2}{*}{\textbf{Dataset}} & \multirow{2}{*}{\textbf{Metric}} & \multicolumn{3}{c}{\textbf{General models}} & \multicolumn{8}{c}{\textbf{Multi-modal models}} \cr
			\cmidrule(lr){3-5} \cmidrule(lr){6-7} \cmidrule(lr){8-9} \cmidrule(lr){10-11} \cmidrule(lr){12-13}  & & \textbf{BPR} & \textbf{LightGCN} & \textbf{BUIR} & \textbf{VBPR} & \textbf{$\text{VBPR}_{\text{PTM}}$}   & \textbf{Freedom} & \textbf{$\text{Freedom}_{\text{PTM}}$}  & \textbf{MGCN} & \textbf{$\text{MGCN}_{\text{PTM}}$} & \textbf{$\text{SMORE}$}  & \textbf{$\text{SMORE}_{\text{PTM}}$} \\
            \hline
			\multirow{4}{*}{Baby} 
                 & Recall@10 & 0.0357         & 0.0479   & 0.0506 & 0.0412             & \textbf{0.0532$^{*}$}    & 0.0646  & \textbf{0.0686$^{*}$}   & 0.0638 & \textbf{0.0649} &0.0678   &\textbf{0.0688$^{*}$}  \\
                 & Recall@20 & 0.0575         & 0.0754   & 0.0788 & 0.0672             & \textbf{0.0817$^{*}$ }  & 0.0982  & \textbf{0.1042$^{*}$}     & 0.0982 &\textbf{ 0.1014$^{*}$}  &0.1030  &\textbf{0.1060$^{*}$} \\
                 & NDCG@10   & 0.0192         & 0.0257   & 0.0269 & 0.0226             & \textbf{0.0283$^{*}$ }  & 0.0349  & \textbf{0.0369$^{*}$}   & 0.0344 & \textbf{0.0352$^{*}$}  &0.0371    &\textbf{0.0375$^{*}$}  \\
                 & NDCG@20   & 0.0249         & 0.0328   & 0.0342 & 0.0293             &\textbf{ 0.0356$^{*}$}    & 0.0436  & \textbf{0.0461$^{*}$}    & 0.0433 & \textbf{0.0446$^{*}$}  &0.0461  &\textbf{0.0470$^{*}$} \\
			\hline
			\multirow{4}{*}{Sports} 
                 & Recall@10 & 0.0432         & 0.0569   & 0.0467 & 0.0534             & \textbf{0.0582$^{*}$}    & 0.0715  & \textbf{0.0737$^{*}$}    & 0.0745 & \textbf{0.0762$^{*}$}  &0.0751 &\textbf{0.0764$^{*}$}  \\
                 & Recall@20 & 0.0653         & 0.0864   & 0.0733 & 0.0813             & \textbf{0.0896$^{*}$ }   & 0.1081  & \textbf{0.1107$^{*}$}     & 0.1108 & \textbf{0.1125$^{*}$}  &0.1122 &\textbf{0.1142$^{*}$} \\
                 & NDCG@10   & 0.0241         & 0.0311   & 0.0260  & 0.0286             & \textbf{0.0308$^{*}$}   & 0.0388  & \textbf{0.0396$^{*}$}     & 0.0409 & \textbf{0.0419$^{*}$}  &0.0412 &\textbf{0.0419$^{*}$} \\
                 & NDCG@20   & 0.0298         & 0.0387   & 0.0329 & 0.0358             & \textbf{0.0389$^{*}$}   & 0.0482  & \textbf{0.0491$^{*}$}   & 0.0502 & \textbf{0.0513$^{*}$}  &0.0508 &\textbf{0.0516$^{*}$} \\
			\hline
			\multirow{4}{*}{Clothing} 
                 & Recall@10 & 0.0235         & 0.0363   & 0.0332 & 0.0302             &\textbf{ 0.0457$^{*}$}   & 0.0629  & \textbf{0.0655$^{*}$ }   & 0.0665 & \textbf{0.0677$^{*}$}  &0.0656 &\textbf{0.0674$^{*}$} \\
                 & Recall@20 & 0.0367         & 0.0540    & 0.0514 & 0.0444             & \textbf{0.0696$^{*}$ }  & 0.0929  & \textbf{0.0969$^{*}$}    & 0.0965 & \textbf{0.1000$^{*}$}  &0.0971 &\textbf{0.1003$^{*}$} \\
                 & NDCG@10   & 0.0127         & 0.0204   & 0.0185 & 0.0169            & \textbf{0.0242$^{*}$}   & 0.0342  & \textbf{0.0355$^{*}$}     & 0.0364 & \textbf{0.0367$^{*}$}  &0.0358 &\textbf{0.0367$^{*}$} \\
                 & NDCG@20   & 0.0161         & 0.0250    & 0.0232 & 0.0205             & \textbf{0.0302$^{*}$}    & 0.0418  & \textbf{0.0435$^{*}$ }   & 0.0440 & \textbf{0.0449$^{*}$}  &0.0438  &\textbf{0.0451$^{*}$} \\
		\bottomrule			
		
		\end{tabular}}
	\end{center}
\end{table*}




\section{Experiment}

\subsection{Experiment setting}

\subsubsection{Baseline}
To validate our framework's effectiveness, we conduct comparative experiments with two categories of representative models: traditional recommendation models including BPR (UAI'09) \cite{2_BPR}, LightGCN (SIGIR'20) \cite{3_lightgcn}, BUIR (SIGIR'21) \cite{DBLP:conf/sigir/LeeKJPY21}, and multimodal recommendation models including VBPR (AAAI'16) \cite{7_DBLP:conf/aaai/HeM16}, Freedom (MM'23) \cite{15_DBLP:conf/mm/ZhouS23}, MGCN (MM'23) \cite{16_DBLP:conf/mm/Yu0LB23}, SMORE (WSDM'25) \cite{61ong2024spectrum}, comparing their performance both in original form and after integration with our framework.

\subsubsection{Implementation details.}
We selected three categories from the Amazon review dataset: Baby, Sports, and Clothing. All data is preprocessed through the MMRec framework \cite{10_DBLP:journals/corr/abs-2302-04473} and image is download from link. Consistent with existing models, We sets the embedding dimensions for users and items to 64 and employs the Adam optimizer with a learning rate of 0.001. To ensure model convergence during training, the number of epochs is set to 1000, and an early stopping strategy. The batch size for the first stage is set to 2048 (128 for the second stage with gradient accumulation steps of 12). The model evaluation was conducted on an NVIDIA Tesla V100 32 GB GPU.



\subsection{Performance Comparison}
% 如表3 所示,我们
As shown in Tab. \ref{Table:perform}, we compared conventional recommendation methods with popular multimodal recommendation methods across three datasets. Consistent with existing multimodal recommendation studies, we adopted Recall@K and NDCG@K (K=10,20) as evaluation metrics. The \textbf{PTM} indicates that our framework was utilized in the model. The results reveal that significant improvements were achieved when the multimodal model was incorporated into our framework. The simpler the model, the greater the performance enhancement. Furthermore, the enhanced performance indicates that our framework not only improves the overall performance of the models but also narrows the performance gap between them. The experimental results validate our claim that our method can effectively facilitate the migration of basic models to recommendation domains with substantial domain gaps, solely with a limited number of prompts.

\subsection{ Model Analysis}



\begin{table}[]
\caption{Ablation study of PEFT Methods\textsuperscript{*}.}
\label{tab:PEFT}
\footnotesize
\begin{tabular}{llccccccc}
\toprule
Datasets & Method     & Recall@20  & Time/E\textsuperscript{1}     & Param & MU\textsuperscript{2} \\
\midrule
\multirow{3}{*}{Baby} 
         & adapter    & 0.1046       & 20.5m    & 169M  & 23g   \\
         & lora & 0.103     & 23.5m    & 160M  & 28g \\
         & prompt   &0.1042     & 8m    & 9M  & 12g \\


\midrule
\multirow{3}{*}{Sports}
         & adapter    & 0.1107       &38m    & 182M  & 23g   \\
         & lora & 0.1095     & 44m    & 173M  & 28g \\
         & prompt   &0.1107      & 12.5m    & 22M  & 12g \\

\midrule
\multirow{3}{*}{Clothing}
         & adapter    & 0.0953       & 34m    & 188M  & 23g   \\
         & lora & 0.0955   &39m    & 178M  & 28g \\
         & prompt   &0.0969     &11m    & 28M  & 12g \\

\bottomrule
\end{tabular}
\\
\textsuperscript{*}Time/E and MU denote training time per epoch  and memory usage, respectively.
\end{table}



\subsubsection{Analysis of PEFT Methods }
To select an appropriate Parameter-Efficient Fine-Tuning (PEFT) method, we conducted comparative experiments on model performance, computational efficiency, and memory consumption across three datasets. As shown in Table \ref{tab:PEFT}, Prompt Tuning demonstrates significant advantages in memory usage and computational efficiency while maintaining outstanding performance, making it our chosen implementation approach. The experimental results also verify that our framework can be flexibly integrated with different PEFT methods. (Note: Sports and Clothing datasets have more parameters than the Baby dataset due to their larger item counts.)



\begin{figure}[h]
% \vspace{5mm}
  \vspace{-2mm}

  \centering
  % \fbox{\rule{0pt}{2in} \rule{0.9\linewidth}{0pt}}
   %\includegraphics[width=0.8\linewidth]{egfigure.eps}
   \includegraphics[width=\linewidth]{figure/ablation.pdf}
   \caption{ Ablation study of PTMRec.}
   \label{fig:ablation}
\end{figure}

\subsubsection{Ablation Study }

To validate the contribution of each component, we conducted ablation studies based on the Freedom model across three datasets. Specifically, we tested variants including CLIP feature initialization (\textbf{CLIP-frozen}), joint training with CLIP pre-prompt (\textbf{CLIP-prompt}), two-stage training without knowledge transfer (\textbf{PTMRec w/o loss}), and the complete framework (\textbf{PTMRec}). As shown in Fig. \ref{fig:ablation}: (1) while CLIP-extracted modality features can enhance model performance, joint training with pre-prompt leads to performance degradation, indicating that simple prompts struggle to bridge the domain gap between recommendation and image-text matching; (2) although the two-stage training strategy partially alleviates the domain gap, optimizing prompts to obtain recommendation-suitable multimodal features remains challenging without domain knowledge guidance; (3) the complete framework, combining training decoupling and knowledge transfer loss, effectively mitigates domain differences, enabling the pre-trained model to transfer to domains with significant gaps.


\subsection{Visualization }
\begin{figure}[h]
% \vspace{5mm}
  \vspace{-2mm}

  \centering
  % \fbox{\rule{0pt}{2in} \rule{0.9\linewidth}{0pt}}
   %\includegraphics[width=0.8\linewidth]{egfigure.eps}
   \includegraphics[width=\linewidth]{figure/features-visualization.pdf}
   \caption{ Distribution of modal feature representations of Freedom in the Sports dataset under different settings. 
}
   \label{fig:visualize}
\end{figure}
% 


We visualized feature distributions on the Amazon Sports dataset using t-SNE and Gaussian kernel density estimation (see Fig.~\ref{fig:visualize}). The results show that: the original Freedom model's features exhibit high clustering but significant dispersion, with dramatic fluctuations in density curves indicating poor feature alignment; after introducing the CLIP encoder, the feature distribution becomes smoother with enhanced local cohesion; furthermore, the complete PTMRec framework strengthens both local cohesion and multimodal alignment of features, leading to tighter clustering of similar items' features and thus improved recommendation performance.

% The introduction of multimodal prompts and the knowledge transfer loss not only facilitates better alignment of multimodal features but also brings similar item features closer together, significantly contributing to the improvement of the recommendation system’s performance.

\section{Conclusion}
This paper proposes PTMRec, a framework addressing the high training costs and domain gaps when integrating pretrained multimodal models into recommendation systems. Based on the CLIP modality encoder, we employ parameter-efficient tuning to reduce training expenses, and enhance user preference perception through a two-stage training strategy with knowledge transfer regularization.

% \section{Ethical Considerations}
% During the conduct of this research, we have thoroughly considered potential ethical issues and societal impacts. We acknowledge that issues of privacy could arise if data is mishandled or misused. To mitigate these risks, we have strictly adhered to all relevant data protection laws and guidelines during data processing. 

%%
%% The next two lines define the bibliography style to be used, and
%% the bibliography file.
\bibliographystyle{ACM-Reference-Format}
\bibliography{main}



\end{document}
\endinput
%%
%% End of file `sample-sigconf-authordraft.tex'.
