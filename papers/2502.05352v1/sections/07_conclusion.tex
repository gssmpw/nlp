\section{Discussion and Conclusion}
\label{sec:discuss}

We presented \bench, the first framework  and experimental platform to benchmark AI Agents for IT automation tasks. \bench strives to capture the complexity of modern IT systems and the diversity of IT tasks. The reproducibility of \bench ensures the community-driven effort despite inherent nondeterminism of large-scale IT systems. 

One of the key design principles of \bench is ensuring its flexibility to support diverse areas of different IT systems
and its extensibility to new scenarios. While current scope of \bench is comprehensive and representative, we plan to further enrich the benchmark suites by adding other important processes essential to modern IT automation. Furthermore, we plan to expand our benchmarking beyond event-triggered scenarios. 
We are actively working to expand scenario coverage for the supported processes and promote growth through open-community contributions.
 We invite the community to reproduce their real-world-inspired incidents in a synthetic sandboxed environment leveraging the \bench. We expect that everyone contributing can bring their expertise to the table.

We expect \bench to drive the innovations of AI agent-based techniques with a direct impact on the safety, efficiency, and intelligence of today’s IT infrastructures. 
With \bench, we are starting to explore many deep, exciting open problems: How to develop domain-specific AI agents that specialize in certain types of IT tasks? How to orchestrate multiple agents with various expertise to collaborate on bigger projects? How can we ensure safety of agent-driven solutions? How can we effectively use human-in-the-loop while developing diverse adaptive agents? We invite everyone to participate in answering these questions and realizing the vision of using AI agents to automate critical IT tasks.

