%%
%% This is file `sample-sigconf.tex',
%% generated with the docstrip utility.
%%
%% The original source files were:
%%
%% samples.dtx  (with options: `all,proceedings,bibtex,sigconf')
%% 
%% IMPORTANT NOTICE:
%% 
%% For the copyright see the source file.
%% 
%% Any modified versions of this file must be renamed
%% with new filenames distinct from sample-sigconf.tex.
%% 
%% For distribution of the original source see the terms
%% for copying and modification in the file samples.dtx.
%% 
%% This generated file may be distributed as long as the
%% original source files, as listed above, are part of the
%% same distribution. (The sources need not necessarily be
%% in the same archive or directory.)
%%
%%
%% Commands for TeXCount
%TC:macro \cite [option:text,text]
%TC:macro \citep [option:text,text]
%TC:macro \citet [option:text,text]
%TC:envir table 0 1
%TC:envir table* 0 1
%TC:envir tabular [ignore] word
%TC:envir displaymath 0 word
%TC:envir math 0 word
%TC:envir comment 0 0
%%
%%
%% The first command in your LaTeX source must be the \documentclass
%% command.
%%
%% For submission and review of your manuscript please change the
%% command to \documentclass[manuscript, screen, review]{acmart}.
%%
%% When submitting camera ready or to TAPS, please change the command
%% to \documentclass[sigconf]{acmart} or whichever template is required
%% for your publication.
%%
%%
\documentclass[sigconf]{acmart}

%% JPN tables
%\usepackage[table,xcdraw]{xcolor}
%\usepackage[table,xcdraw]{xcolor}
%\usepackage{xcolor}
\usepackage{xcolor,colortbl}
%\usepackage[ampersand]{easylist} % JPN added
%\usepackage[ampersand]{easylist}


%%
%% \BibTeX command to typeset BibTeX logo in the docs
\AtBeginDocument{%
  \providecommand\BibTeX{{%
    Bib\TeX}}}

%% Rights management information.  This information is sent to you
%% when you complete the rights form.  These commands have SAMPLE
%% values in them; it is your responsibility as an author to replace
%% the commands and values with those provided to you when you
%% complete the rights form.
%\setcopyright{acmlicensed}
%\copyrightyear{2018}
%\acmYear{2018}
%\acmDOI{XXXXXXX.XXXXXXX}

%% These commands are for a PROCEEDINGS abstract or paper.
% \acmConference[WebSci '25]{Make sure to enter the correct
%   conference title from your rights confirmation email}{June 03--05,
%   2018}{Woodstock, NY}
%%
%%  Uncomment \acmBooktitle if the title of the proceedings is different
%%  from ``Proceedings of ...''!
%%
%%\acmBooktitle{Woodstock '18: ACM Symposium on Neural Gaze Detection,
%%  June 03--05, 2018, Woodstock, NY}
%\acmISBN{978-1-4503-XXXX-X/18/06}
\copyrightyear{2025}
\acmYear{2025}
\setcopyright{cc}
\setcctype{by}
\acmConference[Websci '25]{17th ACM Web Science Conference}{May 20--24, 2025}{New Brunswick, NJ, USA}
\acmBooktitle{17th ACM Web Science Conference (Websci '25), May 20--24, 2025, New Brunswick, NJ, USA}
\acmDOI{10.1145/3717867.3717908}
\acmISBN{979-8-4007-1483-2/2025/05}

%%
%% Submission ID.
%% receive a unique submission ID from the organizers
%% of the event, and this ID should be used as the parameter to this command.
%%\acmSubmissionID{123-A56-BU3}

%%
%% For managing citations, it is recommended to use bibliography
%% files in BibTeX format.
%%
%% You can then either use BibTeX with the ACM-Reference-Format style,
%% or BibLaTeX with the acmnumeric or acmauthoryear sytles, that include
%% support for advanced citation of software artefact from the
%% biblatex-software package, also separately available on CTAN.
%%
%% Look at the sample-*-biblatex.tex files for templates showcasing
%% the biblatex styles.
%%

%%
%% The majority of ACM publications use numbered citations and
%% references.  The command \citestyle{authoryear} switches to the
%% "author year" style.
%%
%% If you are preparing content for an event
%% sponsored by ACM SIGGRAPH, you must use the "author year" style of
%% citations and references.
%% Uncommenting
%% the next command will enable that style.
%%\citestyle{acmauthoryear}

% Astro references
\newcommand{\bu}{$\bullet$ }
\newcommand{\apj}{ApJ}
\newcommand{\apjl}{ApJL}
\newcommand{\mnras}{MNRAS}
\newcommand{\aj}{Astron. J.}
\newcommand{\aap}{A \& A}
\newcommand{\apjs}{ApJSS}
\newcommand{\araa}{ARAA}
\newcommand{\baas}{Bulletin of the American Astronomical Society}
\newcommand{\nar}{NaR}
\newcommand{\nat}{Nature}
\newcommand{\pasp}{PASP}
\newcommand{\na}{NA}

\usepackage{enumitem}

% JPN's stuffs
%\usepackage[ampersand]{easylist}
\newcommand{\rt}[1]{{\textcolor{black}{{#1}}}}
\newcommand{\rtc}[1]{{\textcolor{red}{({#1})}}}
%\usepackage{xcolor}

\usepackage{multirow}
% \usepackage{amsmath} % for \boxed and \smash[b] macros
% \usepackage{booktabs}% for \midrule and \cmidrule macros
% \newcommand\headercell[1]{%
%    \smash[b]{\begin{tabular}[t]{@{}c@{}} #1 \end{tabular}}}

%% important numbers and whatnot
\newcommand{\nsentences}{1296} % number of sentences in the dataset (including duplicates)
\newcommand{\nreviews}{1043} % number of total reviews in dataset
\newcommand{\nmovies}{6} % number of movies, NOT including Hubble
\newcommand{\nmoviesword}{six} % number of movies, NOT including Hubble
\newcommand{\nsentencesNoTBDs}{1286} % no emoji's/pure punctuation AND TBDs removed
\newcommand{\noNoDups}{1190} % number of sentences which are not duplicated (i.e. there is only one of these in the dataset) -- this number is probably not used
\newcommand{\nsentencesNoDups}{1213} % all unique sentences in our dataset
\newcommand{\movielist}{SuperTornado: Anatomy of a MegaDisaster\footnote{\url{https://www.imdb.com/title/tt5297036/}}, Birth of Planet Earth\footnote{\url{https://www.imdb.com/title/tt12036106/}}, Solar Superstorms\footnote{\url{https://www.imdb.com/title/tt5201632/}, \url{https://www.spitzcreativemedia.com/shows/solar-superstorms/}}, Seeing the Beginning of Time\footnote{\url{https://www.imdb.com/title/tt10553756/}}, Space Junk\footnote{\url{https://www.imdb.com/title/tt2180529/}}, and The Jupiter Enigma\footnote{\url{https://www.imdb.com/title/tt12787166/}}} 

\newcommand{\impact}{\textit{Impact}}
\newcommand{\sentiment}{\textit{Sentiment}}
\newcommand{\avlref}{Advanced Visualization Lab (NCSA). }
%\newcommand{\avlref}{[Anonymized]}

% % for citing movies
% \DeclareLabelname[movie]{
%   \field{director}
%   \field{producer}
% }

\def\sectionautorefname{Section}
\def\subsectionautorefname{Section}
\def\subsubsectionautorefname{Section}


%
% These are are recommended to typeset listings but not required. See the subsubsection on listing. Remove this block if you don't have listings in your paper.
\usepackage{newfloat}
\usepackage{listings}
\lstset{%
	basicstyle={\footnotesize\ttfamily},%footnotesize acceptable for monospace
	numbers=left,numberstyle=\footnotesize,xleftmargin=2em,% show line numbers, remove this entire line if you don't want the numbers.
	aboveskip=0pt,belowskip=0pt,%
	showstringspaces=false,tabsize=2,breaklines=true}
\floatstyle{ruled}
\newfloat{listing}{tb}{lst}{}
\floatname{listing}{Listing}

%%%% JPN tables
%\usepackage[table,tabular,xcdraw]{xcolor}



%%
%% end of the preamble, start of the body of the document source.
\begin{document}

%%
%% The "title" command has an optional parameter,
%% allowing the author to define a "short title" to be used in page headers.
\title[Beyond the Lens]{Beyond the Lens: Quantifying the Impact of Scientific Documentaries through Amazon Reviews}

%%
%% The "author" command and its associated commands are used to define
%% the authors and their affiliations.
%% Of note is the shared affiliation of the first two authors, and the
%% "authornote" and "authornotemark" commands
%% used to denote shared contribution to the research.
%\author{Anonymized}
% \author{Ben Trovato}
% \authornote{Both authors contributed equally to this research.}
% \email{trovato@corporation.com}
% \orcid{1234-5678-9012}
% \author{G.K.M. Tobin}
% \authornotemark[1]
% \email{webmaster@marysville-ohio.com}
% \affiliation{%
%   \institution{Institute for Clarity in Documentation}
%   \city{Dublin}
%   \state{Ohio}
%   \country{USA}
% }

% AUTHORS: Jill Naiman, Aria Pessianzadeh, Hanyu Zhao, Aj Christensen, Alistair Nunn, Shriya Srikanth, Anushka Gami, Emma Maxwell, Louisa Zhang, Sri Nithya Yeragorla and Shadi Rezapour

\author{Jill P. Naiman}
\affiliation{%
  \institution{University of Illinois}
  \city{Champaign}
  \country{USA}}
\email{jnaiman@illinois.edu}

\author{Aria Pessianzadeh}
\affiliation{%
  \institution{Drexel University}
  \city{Philadelphia}
  \country{USA}}
%\email{ap3943@drexel.edu}


\author{Hanyu Zhao}
\affiliation{%
  \institution{Duke University}
  \city{Durham}
  \country{USA}}
%\email{hz291@duke.edu}

\author{{AJ} Christensen}
\affiliation{%
  \institution{NASA Scientific Visualization Studio}
  \city{Baltimore}
  \country{USA}}
%\email{andrew.j.christensen@nasa.gov}

\author{Kalina Borkiewicz}
\affiliation{%
  \institution{The University of Utah}
  \city{Salt Lake City}
  \country{USA}}
%\email{kalina@cs.utah.edu}

\author{Shriya Srikanth}
\affiliation{%
  \institution{Harvard}
  \city{Cambridge}
  \country{USA}}
%\email{ssrikanth@jd27.law.harvard.edu}

\author{Anushka Gami}
\affiliation{%
  \institution{University of Illinois}
  \city{Champaign}
  \country{USA}}
%\email{anushkagami@gmail.com}

\author{Emma Maxwell}
\affiliation{%
  \institution{University of Illinois}
  \city{Champaign}
  \country{USA}}
%\email{elm6@illinois.edu}

\author{Louisa Zhang}
\affiliation{%
  \institution{University of Illinois}
  \city{Champaign}
  \country{USA}}
%\email{louisaz2@illinois.edu}

\author{Sri Nithya Yeragorla}
\affiliation{%
  \institution{University of Illinois}
  \city{Champaign}
  \country{USA}}
%\email{srinithya.yeragorla@gmail.com}

\author{Rezvaneh Rezapour}
\affiliation{%
  \institution{Drexel University}
  \city{Philadelphia}
  \country{USA}}
\email{sr3563@drexel.edu}

% \author{Valerie B\'eranger}
% \affiliation{%
%   \institution{Inria Paris-Rocquencourt}
%   \city{Rocquencourt}
%   \country{France}
% }

% \author{Aparna Patel}
% \affiliation{%
%  \institution{Rajiv Gandhi University}
%  \city{Doimukh}
%  \state{Arunachal Pradesh}
%  \country{India}}

% \author{Huifen Chan}
% \affiliation{%
%   \institution{Tsinghua University}
%   \city{Haidian Qu}
%   \state{Beijing Shi}
%   \country{China}}

% \author{Charles Palmer}
% \affiliation{%
%   \institution{Palmer Research Laboratories}
%   \city{San Antonio}
%   \state{Texas}
%   \country{USA}}
% \email{cpalmer@prl.com}

% \author{John Smith}
% \affiliation{%
%   \institution{The Th{\o}rv{\"a}ld Group}
%   \city{Hekla}
%   \country{Iceland}}
% \email{jsmith@affiliation.org}

% \author{Julius P. Kumquat}
% \affiliation{%
%   \institution{The Kumquat Consortium}
%   \city{New York}
%   \country{USA}}
% \email{jpkumquat@consortium.net}

%%
%% By default, the full list of authors will be used in the page
%% headers. Often, this list is too long, and will overlap
%% other information printed in the page headers. This command allows
%% the author to define a more concise list
%% of authors' names for this purpose.
\renewcommand{\shortauthors}{Naiman et al.}

%%
%% The abstract is a short summary of the work to be presented in the
%% article.
% \begin{abstract}
\begin{abstract}
Out-of-distribution (OOD) detection and OOD generalization are widely studied in Deep Neural Networks (DNNs), yet their relationship remains poorly understood. We empirically show that the degree of Neural Collapse (NC) in a network layer is inversely related with these objectives: stronger NC improves OOD detection but degrades generalization, while weaker NC enhances generalization at the cost of detection. This trade-off suggests that a single feature space cannot simultaneously achieve both tasks. To address this, we develop a theoretical framework linking NC to OOD detection and generalization. We show that entropy regularization mitigates NC to improve generalization, while a fixed Simplex Equiangular Tight Frame (ETF) projector enforces NC for better detection. Based on these insights, we propose a method to control NC at different DNN layers. In experiments, our method excels at both tasks across OOD datasets and DNN architectures. 

\begin{comment}   

Out-of-distribution (OOD) detection and OOD generalization are critical for deploying machine learning models in real-world scenarios. While substantial progress has been made in addressing these problems independently, few works have attempted to tackle them jointly. However, existing methods often rely on auxiliary OOD training data and primarily focus on covariate-shifted OOD data that share labels with in-distribution (ID) data. In contrast, we tackle the more realistic and challenging task of jointly detecting and generalizing to semantic OOD data with disjoint labels from the ID data, without auxiliary OOD training data.
Achieving both objectives simultaneously is inherently difficult due to a fundamental conflict — OOD generalization requires enhanced transferability, while OOD detection necessitates the inhibition of transfer.
To address this, we leverage insights from neural collapse (NC) — a phenomenon in deep networks where top-layer representations suppress feature variability and adopt a Simplex Equiangular Tight Frame (ETF) structure, impairing transferability. By controlling NC, we unify OOD detection and generalization: preventing NC enhances OOD transfer while inducing NC improves OOD detection.
Our proposed method excels at both tasks across various OOD datasets and architectures. 

\end{comment}


\end{abstract}

% \end{abstract}

%%
%% The code below is generated by the tool at http://dl.acm.org/ccs.cfm.
%% Please copy and paste the code instead of the example below.
%%

\begin{CCSXML}
<ccs2012>
   <concept>
       <concept_id>10010147.10010178.10010179</concept_id>
       <concept_desc>Computing methodologies~Natural language processing</concept_desc>
       <concept_significance>500</concept_significance>
       </concept>
   <concept>
       <concept_id>10010405.10010469.10010474</concept_id>
       <concept_desc>Applied computing~Media arts</concept_desc>
       <concept_significance>500</concept_significance>
       </concept>
   <concept>
       <concept_id>10003120</concept_id>
       <concept_desc>Human-centered computing</concept_desc>
       <concept_significance>500</concept_significance>
       </concept>
 </ccs2012>
\end{CCSXML}

\ccsdesc[500]{Computing methodologies~Natural language processing}
\ccsdesc[500]{Applied computing~Media arts}
\ccsdesc[500]{Human-centered computing}

%%
%% Keywords. The author(s) should pick words that accurately describe
%% the work being presented. Separate the keywords with commas.
\keywords{Impact analysis, scientific films, natural language processing, review analysis, large language models}
%% A "teaser" image appears between the author and affiliation
%% information and the body of the document, and typically spans the
%% page.
% \begin{teaserfigure}
%   \includegraphics[width=\textwidth]{sampleteaser}
%   \caption{Seattle Mariners at Spring Training, 2010.}
%   \Description{Enjoying the baseball game from the third-base
%   seats. Ichiro Suzuki preparing to bat.}
%   \label{fig:teaser}
% \end{teaserfigure}

% \received{20 February 2007}
% \received[revised]{12 March 2009}
% \received[accepted]{5 June 2009}

%%
%% This command processes the author and affiliation and title
%% information and builds the first part of the formatted document.
\maketitle

% \begin{abstract}
In recent years, a variety of methods based on Transformer and state space model (SSM) architectures have been proposed, advancing foundational DNA language models. 
% However, there is a lack of comparison between these recent approaches and the widely used convolutional networks (CNNs) on foundation model benchmarks.
However, there is a lack of comparison between these recent approaches and the classical architecture---convolutional networks (CNNs)---on foundation model benchmarks.
This raises the question: \textit{ Are CNNs truly being surpassed by these recent approaches based on transformer and SSM architectures?} In this paper, we develop a simple but well-designed CNN-based method, termed \textbf{ConvNova}. ConvNova identifies and proposes three effective designs: 1) dilated convolutions, 2) gated convolutions, and 3) a dual-branch framework for gating mechanisms. 
Through extensive empirical experiments, we demonstrate that ConvNova significantly outperforms recent methods on more than half of the tasks across several foundation model benchmarks. For example, in histone-related tasks, ConvNova exceeds the second-best method by an average of 5.8\%, while generally utilizing fewer parameters and allowing faster computation.  
% Additionally, our exploration of the receptive field in histone-related tasks yields further insights. 
% Additionally, the experiments observed that the appropriate receptive field size for CNNs may be related to the biological characteristics of the task.
In addition, the experiments observed findings that may be related to biological characteristics.
This indicates that CNNs are still a strong competitor compared to Transformers and SSMs. We anticipate that this work will spark renewed interest in CNN-based methods for DNA foundation models. Code is available at: \url{https://github.com/aim-uofa/ConvNova}
\end{abstract}
\section{Introduction}
Communicating science is an important goal of many societies and institutions \citep[e.g.,][]{national2017communicating}. 
Scientific documentaries have long served as an important communication medium for disseminating scientific knowledge to the general public. They bridge the gap between complex scientific concepts and public understanding, often influencing public opinion and policy on critical issues such as climate change, public health, and technology~\citep{barnes2008,punzo2015,vogt2016,borkiewicz2019cinematic,ytini}. 
Cinematic treatments of data visualization in the field of ``Cinematic Scientific Visualization'' (CSV) \citep{borkiewicz2022introducing} such as documentaries and IMAX films can be especially effective in engaging the general public \citep{dubeck2004fantastic,arroio2010,franconeri2021science,lee2022affective}.
From a media studies perspective, the narrative techniques and cinematographic strategies employed in documentaries can significantly enhance viewer engagement and retention of information~\cite{barrett2008assessing,yeo2018inconvenient,saputra2022impactful}. 
Research shows that the storytelling aspect of documentaries can lead to greater empathetic understanding and cognitive retention of scientific facts~\cite{ginting2024effects,gaunkar2022exploring}. The emotional responses elicited by documentaries are seen as key drivers for behavioral change and advocacy, influencing how individuals and communities respond to scientific challenges like environmental conservation or disease prevention~\cite{bieniek2019communicating}.

The impact of media was studied from the perspective of sociology, examining how documentaries reinforce or challenge cultural norms and societal structures~\cite{rezapour2017classification,diesner2015social}. The results show that documentaries can act as catalysts for social change by highlighting underrepresented issues and providing a platform for marginalized voices. This fosters a more inclusive public dialogue around scientific and technological advancements~\cite{bouzoubaa2024euphoria,conrad2022breaking,diesner2016assessing}.
These impacts extend beyond individual viewers to influence societal norms and values. By presenting scientific issues within relatable contexts, documentaries can alter public discourse, catalyze community actions, and even shift policy directions~\cite{atakav2024impact}. Moreover, the reach and accessibility of documentaries have been vastly expanded by advances in digital technology and online streaming platforms, which allow for unprecedented dissemination and engagement across varied global audiences.

%Measuring the `impact'' of scientific documentaries requires multimodal efforts.There are many decisions that designers of cinematic treatments must contend with in order to make an effective CSV presentation \citep{woodward2015one,borkiewicz2022introducing,jensen2023evidence} and gauging audience attitudes and cognition can be difficult to do with precision, requiring one-on-one interviews and qualitative analysis \citep{chen2005top,cawthon2007effect,buck2013effect,fraser2012giant,smith2015aesthetics,smith2017capturing,jensen2023evidence,jenseninprep}. Thus, quantitative approaches can be used to complement in-depth qualitative analyses in order to better understand the impact of scientific media. This approach can also address some  shortcomings to prior work is that it often primarily relied on manual qualitative analysis, which is not only labor-intensive but also subject to bias and scalability issues~\cite{bieniek2019communicating,atakav2024impact,barrett2008assessing}.
Measuring the ``impact'' of scientific documentaries requires a multifaceted approach that integrates both qualitative and quantitative methods. 
%
Prior studies have demonstrated that filmmakers and designers of CSV presentations must navigate numerous creative and technical decisions to ensure their effectiveness \citep{woodward2015one,borkiewicz2022introducing,jensen2023evidence}.
%
Traditionally, assessing audience attitudes and cognition has relied on in-depth qualitative techniques such as interviews and thematic analysis, which, while insightful, can be labor-intensive and subject to bias \citep{chen2005top,cawthon2007effect,buck2013effect,fraser2012giant,smith2015aesthetics,smith2017capturing,jensen2023evidence,jenseninprep}. 
Additionally, the reliance on manual qualitative analysis in prior research has posed challenges related to efficiency and reproducibility \citep{bieniek2019communicating,atakav2024impact,barrett2008assessing}.

To address these limitations, quantitative approaches can serve as valuable complements, enhancing precision and scalability in impact assessments.
By integrating quantitative methods, researchers can overcome the constraints of qualitative-only studies, leading to a more comprehensive and rigorous evaluation of scientific media's influence.
With the emergence of digital platforms, viewers share their immediate reactions and detailed thoughts through online reviews, providing a rich dataset for analyzing public engagement and understanding~\cite [e.g.,][]{APPEL2016110}.
The interactivity provided by online platforms enhances the dissemination process, as viewers can discuss and share content, thereby amplifying the documentary's reach and impact. This engagement is measurable through the analysis of online reviews and social media commentary, which can serve as a feedback loop for content creators and scientists alike, indicating which aspects of their presentation meet the audience's needs and which do not~\cite{rezapour2017classification,bouzoubaa2024euphoria}.


%Previous work showed that while online engagements offer insights into viewer reception, they often lack a systematic approach to quantify the depth and spectrum of influence exerted by documentaries on scientific understanding and changes in cognition and attitude~\citep{rezapour2017classification}. 
% An additional shortcoming to prior work is that it often primarily relied on manual qualitative analysis, which is not only labor-intensive but also subject to bias and scalability issues~\cite{bieniek2019communicating,atakav2024impact,barrett2008assessing}.
We extend previous work by computationally analyzing the impact of scientific films using user-generated reviews and measuring the impact using a novel taxonomy of change and engagement in conjunction with a systematic approach to manual annotations. This approach employs natural language processing (NLP) techniques to systematically categorize and quantify viewer responses, thus providing a more scalable and objective measure of the documentaries' effectiveness in enhancing scientific literacy and influencing viewers' behavior. More specifically, we answer the following questions: 
 \setlist{nolistsep}
    \begin{itemize}[noitemsep]
    \item[-] {\textbf{RQ1:}} What are the most prevalent impacts of scientific documentaries on viewer cognition, attitudes, and interests? 
    \item[-] {\textbf{RQ2}:} What, if any, are some common themes in viewer cognition, attitude, and interests surrounding scientific documentaries?
\end{itemize}

To answer these questions, we extract reviews from Amazon for \nmoviesword\ prominent scientific documentaries and evaluate the tone, content, and thematic elements of viewer feedback. This data is further analyzed using our developed taxonomy, which categorizes impacts into ``Shift in Cognition (C)'', ``Attitudes Toward the Film (A)'', ``Interest with Science Topic (S)'', ``Impersonal report (I)'', and ``Not applicable (N)''.
Our data analysis shows that expressing different ``Attitudes Toward the Film'' is the most common \impact\ type reported in the reviews. 
The classification results suggest that large language model (LLM)-based classifiers, especially when the full context of the review is added to the prompt, are capable of successfully identifying \impact\ categories and their performance is generalizable to other datasets. Additionally, our thematic analysis of \impact\ groups demonstrates a diverse range of themes in the reviews. Viewers %express positive, negative, or neutral feedback 
used mixed sentiments to show their attitude toward the film, while expressing environmental concerns or celebrating cosmic science were some of the themes related to shifts in cognition. 

Our paper makes the following contributions: (1) providing annotated data on impact, offering a dataset that has been carefully categorized and labeled to reflect the various types of influence that scientific documentaries have on public perception and interest\footnote{\url{https://huggingface.co/datasets/scidoc/websci2025}}; (2) introducing a novel taxonomy of engagement, which enables a structured analysis of how documentaries affect cognitive, affective, and interest dimensions of their audiences; and (3) employ advanced computational techniques to analyze viewer responses in a scalable and unbiased manner. 
These contributions enhance our understanding of the educational and societal impacts of scientific documentaries. The paper also provides a framework for future research to build upon, improving the effectiveness and reach of science communication through visual media.

% \rt{This section needs to motivate why our dataset (and bench mark models if we do that too) are important. Some things could be:}
% \begin{easylist}[itemize]
%     & \rt{Understanding and quantifying the impact of science communication is vital in the age of misinformation (CITE?).}
%     & \rt{While in depth interviews of subjects with a wide variety of ages and backgrounds are imperative to gauge understanding of scientific topics, such interview campaigns are monitarily and temporally expensive (CITE?).}
%     & \rt{Impact and sentiment analysis (CITE) for the basis for ...}
% \end{easylist}
\section{Related Work}
Social media platforms are invaluable resources for understanding public sentiment and behavior, offering real-time access to users' expressed opinions and emotions. By analyzing vast amounts of social media data, researchers can uncover trends, gauge audience reactions, and explore complex phenomena like consumer behavior and societal attitudes~\cite{elalaoui2018sentiment,cortis2021socialopinion}. Platforms like Twitter, Reddit, and YouTube provide large-scale datasets that enable studies to capture real-time feedback and public sentiment at an unprecedented scale in domains such as politics~\cite{ANTYPAS2023100242, Yarchi15032021,rezapour2017identifying}, healthcare~\cite{10.1007/978-3-031-27409-1_41,9810923}, and education~\cite{electronics11050715,luo2020like}.
For instance, sentiment analysis of tweets related to significant global events or media productions, such as the Netflix documentary Our Planet, reveals insights into public sentiment and media impact. This analysis helps creators and researchers understand audience engagement and the potential societal influence of content~\citep{ieeexplore_10533989}. Similarly, Reddit's discourse, known for its depth and variety, serves as a valuable source for exploring nuanced discussions and opinions~\citep{asurveyof2012opinion,10.1145/3543873.3587324,10.3389/frai.2023.1163577,bouzoubaa2024euphoria}.

%However, these analyses are not without challenges. The inherent biases in social media data, such as demographic skew and echo chamber effects, can affect results. Furthermore, automated methods, while scalable, often face limitations in understanding contextual nuances compared to traditional qualitative methods like interviews and surveys \citep{wlv_openrepository, arxiv_2408.08694, arxiv_2407.13069}.

Sentiment analysis has been central to leveraging social media data for understanding public opinion~\cite{ijerph15112537, QIAN2022103098}. Advances in NLP and machine learning have significantly improved sentiment classification tasks. Traditional approaches like Support Vector Machines (SVM) and K-Nearest Neighbors (KNN) have given way to deep learning methods, which excel in handling large, complex datasets~\cite{BANSAL2022100071}. Models such as Long Short-Term Memory (LSTM), Bidirectional LSTM (BiLSTM), and transformers like BERT and RoBERTa have achieved state-of-the-art performance in sentiment analysis~\citep{ieeexplore_9716923, ieeexplore_8629198, kapur2022sentimentanalysis}.
Recent comparisons of machine learning models for social media sentiment analysis reveal that deep learning approaches consistently outperform traditional algorithms. RoBERTa-LSTM hybrids, for instance, demonstrate exceptional accuracy and robustness in Twitter sentiment classification~\citep{ieeexplore_10331232}. These models' ability to understand the context and handle large datasets has made them indispensable for analyzing public opinion across diverse media platforms~\citep{arxiv_2408.08694, ieeexplore_10533989, arxiv_2407.13069}.

With the emergence of LLMs such as GPT and BERT-based architectures, sentiment analysis has evolved further, leveraging pre-trained models to understand complex linguistic patterns and contextual subtleties\cite{sayeed2023bert,chang2024survey}. These models enable transfer learning, allowing fine-tuning on specific domains to achieve superior performance compared to traditional and earlier deep learning methods~\cite{10.1145/3543873.3587324, arxiv_2407.13069}. 
Studies emphasize the importance of domain-specific tuning of NLP models. For example, the application of sentiment analysis to social media reviews of environmental documentaries or films not only provides feedback on content reception but also sheds light on broader societal attitudes towards environmental issues~\citep{acerbi2023sentiment}.
Opinion mining goes a step further than sentiment analysis by focusing on specific aspects of text that inform user perspectives. This fine-grained analysis has been applied to social media and online discourse, uncovering detailed insights into user opinions on products, services, or cultural phenomena~\citep{asurveyof2012opinion, kapur2022sentimentanalysis, Gerard_Botzer_Weninger_2023}.
For instance, \citep{kapur2022sentimentanalysis} employed BiLSTM combined with a random forest classifier to identify sentiments across multiple platforms, achieving high accuracy. 

Film reviews provide a specifically rich ground for sentiment analysis and opinion mining \cite{malini2019opinion}. They are rich in emotion, context, and critique, making them ideal for studying user opinions and their motivations. Existing research has predominantly focused on sentiment polarity, identifying whether reviews are positive, negative, or neutral. While effective, this binary/ternary classification often misses the granular details of user opinions~\citep{Mrabti2024AnEM, 10.5120/ijca2017916005}.
Researchers have called for extending these efforts to include opinion mining, which examines the thematic and contextual dimensions of reviews. For instance, \citep{rezapour2017classification} introduced ``Impact'' categories to analyze how specific aspects of films influence people's cognition and attitude. This nuanced approach revealed critical factors driving engagement, such as emotional resonance and storytelling quality. In addition, \cite{yue2019survey} emphasizes the importance of incorporating fine-grained sentiment analysis to move beyond simple polarity classification. This involves analyzing not only whether reviews are positive or negative but also understanding the rich contextual and emotional dimensions that shape user opinions. By capturing these subtleties, researchers can uncover deeper insights into audience preferences and motivations, ultimately enabling more tailored and effective content strategies.

Building on this body of work, %our research aims to advance the application of NLP and machine learning techniques to film reviews, emphasizing both sentiment and impact analysis. B
by adopting a fine-grained approach, we seek to identify not only the overall sentiment but also the specific aspects of scientific films that shape public emotions and opinions. This holistic understanding will contribute to a deeper comprehension of user feedback, guiding content creators and industry stakeholders in optimizing their offerings.

%\subsection{NLP for Media Analysis}
%Gauging general public sentiment on media is a field with many recent contributions.
% Analyzing large volumes of online discourse with machine learning methods has become a goal for numerous researchers. 
% Sentiment analysis, in which statements are given tags of ``positive", ``negative" or ``neutral", of YouTube comments has been used to quantify public opinion on media content, though it has limitations compared to traditional methods like interviews and surveys \citep{wlv_openrepository, arxiv_2408.08694, arxiv_2407.13069}. Social media platforms offer a wealth of data for sentiment analysis, allowing insights into audience responses at scale \rtc{yue2019survey}.
% Large-scale analysis of over 2 million Twitter tweets about Our Planet \cite{our_planet_trailer} \rtc{need a reference for this -- what is Our Planet? is it a movie or something else?} shows how real-time public sentiment can be captured to study media impact, providing valuable feedback for content creators and conservationists \rtc{where is the conbio\_wiley reference?} \citep{(acerbi2023sentiment)conbio_wiley, ieeexplore_10533989}. 

% Machine learning models like SVM, KNN, and advanced techniques like RoBERTa-LSTM~\rtc{citation needed} have been compared for sentiment analysis, showing that deep learning approaches often outperform traditional methods in understanding social media sentiment \citep{ieeexplore_9716923, ieeexplore_8629198, ieeexplore_10331232}. These techniques are particularly effective in handling large datasets like Twitter reviews. 
% Sentiment analysis using natural language processing (NLP) tools has been applied to various media types across platforms, revealing trends and public reactions, with studies showing that both traditional and transformer-based models can be used to analyze sentiment effectively 
% \citep{arxiv_2408.08694, ieeexplore_10533989, arxiv_2407.13069}. 

% \rt{Other platforms, such as Reddit, also provide a rich source of data for media analysis. (how is this sentence related to what follows? is there a reference for it? What originally followed this sentence is commented out below as its been moved around.}

% % \begin{easylist}[itemize] 
 
% % & Machine learning models like SVM, KNN, and advanced techniques like RoBERTa-LSTM~\rtc{citation needed} have been compared for sentiment analysis, showing that deep learning approaches often outperform traditional methods in understanding social media sentiment \citep{ieeexplore_9716923, ieeexplore_8629198, ieeexplore_10331232}. These techniques are particularly effective in handling large datasets like Twitter reviews. 
% % & Sentiment analysis using natural language processing (NLP) tools has been applied to various media types across platforms, revealing trends and public reactions, with studies showing that both traditional and transformer-based models can be used to analyze sentiment effectively 
% % \citep{arxiv_2408.08694, ieeexplore_10533989, arxiv_2407.13069}. 

% % \end{easylist}

% Research in opinion mining, which can be used for both sentiment analysis and a finer-grained NLP analysis of words within statements, has been particularly robust in analyzing content on social media and other online discourse \citep{asurveyof2012opinion, kapur2022sentimentanalysis, Gerard_Botzer_Weninger_2023, Guerra_2023}. \citep{kapur2022sentimentanalysis} found particular success using Bidirectional Long Short-Term Memory (BiLSTM) and a random forest classifier to analyze sentiments in social media content across multiple platforms. 


% % \subsection{NLP for Review Analysis}

% In particular, sentiment analysis on film reviews has become an area of interest to many researchers \citep{bhola2022hyrbidframework, Mrabti2024AnEM, 10.5120/ijca2017916005, bodapati2019sentiment}. These studies, however, often focus on only identifying the reviews’ polarities (whether they are positive or negative) and do not further examine their content \citep{Mrabti2024AnEM, 10.5120/ijca2017916005}. Long Short-Term Memory (LSTM) has proven to be particularly effective in the completion of these review analysis tasks \citep{bhola2022hyrbidframework, bodapati2019sentiment}. 
% Opinion mining which examined the specific aspects of a film which may influence its effectiveness at engaging the public was conducted in  \cite{rezapour2017classification} with the addition of ``Impact" categories along with Sentiment categories.  In what follows, we build on this work for finer-grained analysis of scientific documentary reviews.
%Further opinion mining examining the specific aspects of a film influencing an author's choice to rate it as negative or positive would offer a more complete understanding of a consumer's perspectives. 


% \begin{easylist}[itemize]
%     & we'll start by looking at the cit`ations to \citep{rezapour2017classification}
%     && check out the URL \url{https://scholar.google.com/scholar?cites=17684968807680716583&as_sdt=400005&sciodt=0,14&hl=en}
%     && an example we might want to check out (this is only a test) is \citep{sherren2017digital}
%     && Ravinder Ahuja et al. 

%& \citep{kapur2022sentimentanalysis} use similar methodologies (SVM and LSTM) to analyze sentiment across social media platforms. 

%& \citep{bhola2022hyrbidframework} use similar methods (BERT and LSTM) and also consider a corpus of film reviews. 

%& \citep{Mrabti2024AnEM} also make use of machine learning methods to analyse online reviews, including a corpus of film reviews. 

%& \citep{bodapati2019sentiment} uses LSTM to classify film reviews as either positive or negative. 

% & \citep{asurveyof2012opinion} introduces methods for opinions mining, highlighting social media as key sources of opinionated content for sentiment analysis.

%& \citep{Gerard_Botzer_Weninger_2023} uses dataset from social media to analyze user engagement and the ways social platforms shape users' behavior.

%& \citep{Guerra_2023} uses word clouds and sentiment analysis approach to analyze news/posts sentiment from social media platform.

%& \citep{kapur2022comparativestudysentimentanalysis} uses similar methods (LSTM) and other machine learning techniques to analyze sentiment for social media platforms.


% & Some other useful references for folks to check out (if you haven't yet):
% && A SURVEY OF OPINION MINING AND SENTIMENT ANALYSIS - \url{https://doi.org/10.1007/978-1-4614-3223-4_13}
% && Dataset paper: Truth Social Dataset - \url{https://ojs.aaai.org/index.php/ICWSM/article/view/22211}
% && Sentiment Analysis for Measuring Hope and Fear from Reddit Posts During the 2022 Russo-Ukrainian Conflict - \url{http://arxiv.org/abs/2301.08347}
% && COMPARATIVE STUDY OF SENTIMENT ANALYSIS FOR MULTI-SOURCED SOCIAL MEDIA PLATFORMS - \url{http://arxiv.org/abs/2212.04688}
    
%\end{easylist}
%\usepackage[table,tabular,xcdraw]{xcolor}


\section{Data}

\subsection{Data Collection}

% \newcommand{\avllonger}{the Advanced Visualization Lab (AVL), housed at the National Center for Supercomputing Applications, a team of visualization designers who focus on cinematic presentations of scientific data}
\newcommand{\avllonger}{a team of visualization designers who focus on cinematic presentations of scientific data (CSVT)}

A dataset of \nsentences\ sentences was collected from a total of \nreviews\ Amazon reviews of the \nmoviesword\ movies produced by \avllonger. \rt{The mean word count per review is 11.2 with a standard deviation of 9.0 words, and the mean word length across all reviews is 4.6 characters with a standard deviation of 2.7 characters.}
% JPN: ``authors should also report the average word length and total word count per review''
% Mean word length in all reviews: 4.630138888888889
% STDDEV of word length in all reviews: 2.669375183371338
% Mean word count per review: 11.19751166407465
% STDDEV word count per review: 9.004829878946925
%
%AVL movies: 
Movies include: SuperTornado: Anatomy of a MegaDisaster~\citep{superTornado2015}, Birth of Planet Earth~\citep{bope2019}, Solar Superstorms~\citep{solarsuperstorms2013,solarsuperstorms2015}, Seeing the Beginning of Time~\citep{sbot2017}, Space Junk~\citep{spacejunk2012}, and The Jupiter Enigma~\citep{jupterenigma2018}.   
In addition to the \nsentences\ collected for annotation and model training, 400 sentences were collected from Amazon reviews of the Hubble documentary~\citep{hubble2010} for final model evaluation.
Movies were selected in collaboration with the \avlshort\ as this team of visual designers expressed interest in quantifying the impact of their works.
Full reviews were downloaded using the \textsf{selectorlib} software package\footnote{\url{https://pypi.org/project/selectorlib/}}. %and sentences are generated from full comments using \textsf{spaCY}'s \cite{spacy2} sentence tokenizer%\footnote{\url{https://spacy.io/api/sentencizer}}.
In addition to the full text, the date and rating of the review were also collected. 

\subsection{Data Annotation}

\subsubsection{Taxonomy of Scientific Documentary Impact}
In seeking evidence for constructivist learning~\citep[e.g.,][]{bada2015constructivism}, we aim to track written language that indicates a statement about cognition or a statement about engagement \rt{(\autoref{tab:impact_categories})}.
As such, our impact categories are modeled off those used in the analysis of issue-focused documentaries \citep{rezapour2017classification} with several updates for our set of scientific documentaries%, namely any change or affirmation categories for behavior cognition or emotion
\citep[in Table 1,][]{rezapour2017classification}. % have been replaced with the categories of ``Shift in Cognition'', ``Attitudes Toward the Film'', and ``Interest with Science Topic''.
More specifically, while internet comments are collected in too uncontrolled an environment to make any definitive conclusions about learning \citep[e.g.,][]{jensen_putting_2017}, we can identify trends among contributors in their comments associated with learning. Our impact category ``Shift in Cognition'' represents audiences' relative ability to identify terms, features, and concepts, up to and including metacognitive discussion of their own cognitive processes. % \rtc{citation? might not be needed since this is basically stating what we are doing with our category though?}. 
    
In our impact categories scheme, we identify engagement through interest and attitudes (impact categories ``Attitudes Toward the Film'' and ``Interest with Science Topic''). Some measures of interest in films can be measured from collected comments. Although they are not conclusive proof of the existence of, or the strength of, a viewer's interest, patterns across large numbers of viewers can provide useful evidence. 
We use the category ``Interest with Science Topic'' when viewers make statements that evaluate the documentary content in its immediate context. 
The impact category ``Impersonal report'' is used when a viewer describes events in the film, and ``Not Applicable''; is the category used for sentences that do not have a specific relationship to the film or science topic.

Sentiment categories follow the typical three classes of ``Positive'', ``Neutral'' and ``Negative'' \citep[e.g.,][]{Liu2012}.
See \autoref{tab:impact_categories} for an overview of these definitions and examples of the combination of sentiment and impact categories used in this work.



% \begin{easylist}[itemize]


% % From AJ:
% % I've just created this document with my thoughts and some text about the impact categories. I realize that some of my comments are probably not super useful at the current stage of the study, but I'm providing them (a) for future reference, and (b) in case someone asks questions about our rationale. I do strongly feel that we should rename the categories from knowledge->cognition and engagement->affect as these are the more appropriate terms in the learning science field, which I'm hoping doesn't create too much of a cascade of document updating. The next biggest concern I have is understanding how Opinion and Interest are differentiated in this scheme (or if they should be combined.) If we needed to combine them, it would require some effort to change, but it would mostly just be a process of deciding whether each opinion was positive or negative and adding it to the corresponding interest category. Which doesn't seem tooooo horrible.

% % Once I understand the answers to some of my questions, I'd like to revisit my text and better tailor it to the situation. It would also benefit from some citations that I would have to seek out.

% % Here's the document:
% % https://docs.google.com/document/d/1y8va9ozyOXdMJWYezjZzjsjmKih0aerhjP6lFbMJftA/edit?usp=sharing

% % FYI, I mostly used this doc as the basis for what I was describing (I also made comments in this doc). I'm assuming this is the most recent draft of the table that will be included in the paper?

% % Response from Shadi
% % Hi AJ,

% % Hope everything is going well. Sorry for the late reply, it has been a crazy summer with so many deadlines. Thanks so much for your input. I believe some of the decisions we made along the way address your concerns. Here is point-by-point response:

% % AJ: I do strongly feel that we should rename the categories from knowledge->cognition and engagement->affect as these are the more appropriate terms in the learning science field, which I'm hoping doesn't create too much of a cascade of document updating.

% % Shadi: This can be easily done.

% % AJ: The next biggest concern I have is understanding how Opinion and Interest are differentiated in this scheme (or if they should be combined.)

% % Shadi: We decided after the first round of annotation to remove “opinion” from our annotation categories. So, we only have the interest category now and no opinion.

% % Another point from your document:

% % AJ: I don’t love the terms “Positive Shift in Cognition” and “Negative Shift in Cognition”. I don’t think we can confidently say anything has shifted. Thinking is a process, not an achievement. Could we say something like … “Favorable Cognition” vs “Conflicting Cognition”???

% % Shadi: we decided to remove positive and negative from the impact categories when annotating the reviews; so, we did the sentiment annotation and impact as two separate tasks.

% % I think that addresses your concern with positive and negative shifts.

% % Our undergrad students prepared a presentation this summer, you can see the distribution of each label in the annotated data. Here is the link: https://docs.google.com/presentation/d/19O1xhd8ONxPum-TvlKyKYemU1KYYbkmdAJYpjd5sFwA/edit#slide=id.g2e8940d8594_0_3

% % I would be more than happy to set a zoom meeting to discuss these further.

% % Also, I am going to do my best to prepare the paper for ICWSM 2025 (submission deadline: Sep 15, 2024). Let me know if you have any issues or concerns. 
    
%     % && \rt{Here maybe we need some words from AJ discussing how/why these changes are made...}

%     & In seeking evidence for constructivist learning \citep[e.g.,][]{bada2015constructivism}, we are especially interested in tracking written language that indicates a statement about cognition or a statement about engagement.
%     && As such, our impact categories are modeled off those used in analysis of issue-focused documentaries \citep{rezapour2017classification} with several updates for our set of scientific documentaries, namely any change or affirmation categories for behavior cognition or emotion \citep[Table 1, ][]{rezapour2017classification} have been replaced with the categories of ``Shift in Cognition'', ``Attitudes Toward the Film'', and ``Interest with Science Topic''.

    
%     & While internet comments are collected in too uncontrolled an environment to make any definitive conclusions about learning \citep[e.g.,][]{jensen_putting_2017}, we can identify trends among contributors in their comments associated with learning. Our impact category ``Shift in Cognition'' represents audiences' relative ability to identify terms, features, and concepts, up to and including metacognitive discussion of their own cognitive processes \rtc{citation?}. %Comparing the relative frequencies of these two categories can provide evidence for the benefit of \rt{this trials off...}
    
%     & Not only is engagement talk more evidence of learning, but it is also a building block that can lead to behaviors with prosocial impact, like de-stigmatizing and sharing science information with community in informal environments, improving personal science identity which can lead to the growth of future science professionals, and advocating for more research funding, political support, and science-motivated decision making \citep[e.g.,][]{lee_robbins_affective_2022}. In our impact categories scheme, we identify engagement through interest and attitudes (impact categories ``Attitudes Toward the Film'' and ``Interest with Science Topic'').
%     && Some measures of interest are measured from internet comments are collected in this study. Although they are not conclusive proof of the existence of, or the strength of, a viewer's interest, patterns across large numbers of viewers can provide useful evidence. We use the category ``Interest with Science Topic'' when viewers make statements that evaluate the documentary content in its immediate context. 

%     & The impact category ``Impersonal report'' is used when a viewer provides description of events in the film, and ``Not Applicable'' is the category used for sentences which do not have a specific relationship to the film or science topic.
%     %&& Further, we identify statements of subjective opinion with the code ``Personal Opinion''. These are connection-making processes that suggest a viewer is engaged enough to branch out beyond the immediate context and recall previous experiences and beliefs and relate them to the documentary content.
%     %&& We can also look for learning talk in audience members' statements of affect. Measures of ``Positive Affect'' and ``Negative Affect'' show different emotional valences of viewers’ experiences as captured in their online comments.

%     & Sentiment categories follow the typical three classes of ``Positive'', ``Neutral'' and ``Negative'' \citep[e.g.,][]{Liu2012}.
%     && See \autoref{tab:impact_categories} for an overview of these definitions and examples of the combination of sentiment and impact categories used in this work.

% % references for the google docs that have the categories
% % * https://docs.google.com/document/d/1-1_FhXDceLlFXZOeAcB_FA_0XqE7k0l1lXZlePj4GDU/edit
% % * https://docs.google.com/document/d/10PQU97mPYnSHf4nph7_Er7tebvXGiWRU94QWdNcesaI/edit
% % * https://docs.google.com/document/d/1HOMGEQGD7-nJERBXGaBTRqMoGHC54Z_vHO8uZlBp9f8/edit
    
% \end{easylist}

%\setlength\cmidrulewidth{0.01\lightrulewidth}
%\begin{center}
\begin{table*}[t]
\resizebox{0.9\textwidth}{!}{
%\begin{tabular}{ |c|c|c|c|c| } {|p{1.5in}|p{1.5in}|}
\begin{tabular}{ @{}p{1.4in}p{1.6in}p{1.6in}p{1.6in}p{1.6in}@{} }
\toprule 
\multicolumn{1}{c}{Impact Category} & \multicolumn{3}{c}{Sentiment Category}
\\\cmidrule(l){2-4}
% \multicolumn{1}{c}{} &\multicolumn{1}{c}{} & \multicolumn{1}{c}{} & \multicolumn{1}{c}{}\\
 & \multicolumn{1}{c}{Positive (+)} & \multicolumn{1}{c}{Neutral (0)} & \multicolumn{1}{c}{Negative (-)} \\\midrule
%\headercell{Impact Category} & \multicolumn{4}{c@{}}{Sentiment Category}\\
%Impact Category & Positive & Neutral & Negative \\
%\multirow{3}{4em}{Shift in Knowledge} & Person gained knowledge & No change in knowledge & Person expressed disagreement with scientific concepts presented in film \\
% \multirow{3}{4em}{Multiple row} & cell2 & cell3 \\ 
% & cell5 & cell6 \\ 
% & cell8 & cell9 \\ 
%And I would especially be more descriptive where it says “Person gained knowledge”, maybe to something like “Person encountered a new idea or way of thinking”?
Shift in Cognition (C)\rt{: ability to identify terms, features, and concepts, up to and including
metacognitive elements} & Person encountered a new idea or way of thinking & No change in way of thinking & Person expressed disagreement with scientific concepts presented in film \\ 
 % & & & \\
 & ``very informative about the development of tornados and the research behind their formation'' & ``We need to reign in the junk orbiting our planet before we lose something more important than the satellite Iridium 23, like a human life!'' & ``Unfortunately, HUGE lack of actual facts.'' \\
\hline
Attitudes Toward the Film (A)\rt{: evaluations of film technical and artistic attributes} & Person indicates positive attributes about film attributes & Person simply acknowledges film attributes & Person expresses dislike of film attributes \\ 
 % & & & \\
 & ``Way cool illustrations of our solar friend and how she can be a bad girl.'' & ``More technical than I thought it would be.'' & ``How could something so AWESOME and real life be made to be so boring?'' \\
\hline
Interest with Science Topic (S)\rt{: evaluations and connections to film scientific content} & Person expresses positive interest in science topic presented in film & Person expresses a connection with the science topic & Person expresses disinterest in science topic \\ 
 % & & & \\
 & ``Space intrigues me greatly, and this was an amazing program.'' & ``I've followed supercomputing since inception (and currently run some of their software).'' &  ``used to intrigue me as a child, but knowing reality now, I seek real answers over lies.''%``used to intrigue me as a child, but knowing reality and lies are now to be believed over actual evidence leaves me wanting real answers.'' 
 \\
\hline
Impersonal report (I)\rt{: descriptions of events in a film without evaluation} & Person describes events in the film with positive connotations & Person describes events in the film & Person describes events in the film with negative connotations \\ 
 % & & & \\
 & ``This video reveals the cutting edge science gathered by the Juno Mission to Jupiter.'' & ``There's not much footage of the Joplin Tornado itself, most of this is footage of the wreckage.'' & ``When there were other scenes to show a specific class of Space Junk then what you saw was simply a different geometric shape and that is it.'' \\
\hline 
Not Applicable (N)\rt{: statements with no specific relation to film or science topic} & Positive comment, but no specific relation to film or topic & Comment has no specific relation to film or topic & Negative comment, but no specific relation to film or topic  \\ 
 % & & & \\
 & ``Interesting'' & ``This has never happened before.'' & ``Documentaries have tried to be too interesting lately.'' \\
\bottomrule
\end{tabular}}
\caption{Sentiment and impact categories with definitions and examples.}
\label{tab:impact_categories}
\end{table*}

\subsection{Annotation Procedure}
After reviewing a sample of reviews, we noticed that individual sentences within a review can address different impacts and may express opposite sentiments. As a result, we opted to perform annotations at the sentence level. We used the Zooniverse\footnote{\url{https://www.zooniverse.org/}} citizen science platform, one of the largest platforms for non-professionals to participate in the scientific process, for the annotation process. 
%Zooniverse boasts over one million active members collaborating on hundreds of projects ranging from the space sciences to transcription of historical documents%\footnote{\url{https://www.zooniverse.org/about}}.

Using Zooniverse's default interface for textual data, we displayed a single sentence from a review followed by the sentence within the full review for context.
This display and the first stage of the annotation process is depicted in \autoref{fig:first_step_annotation} in the Appendix in which the user is asked to select their first choice for the sentiment category (Positive, Negative, or Neutral).
% Here, ``first choice'' is defined as the strongest sentiment expressed in the statement. If there were two equally strong sentiments expressed, then the sentiment that appeared first was selected as the ``first choice'' (the analysis of ``second-choice'' selections is relegated to a future paper).
% Once the user made their selection, they could select ``Done'' to move to the next stage of the annotation (green button in \autoref{fig:first_step_annotation}) or ``Done \& Talk'' to ask questions in the main forum (blue button in \autoref{fig:first_step_annotation}).
% % Annotators can also click the ``Favorite'' button (heart symbol) to save the specific annotation to their personal collection for later discussion or can access more information about the annotation (URL and film title) with the ``Info'' button (circled ``i'' symbol).
% % After completion of the first choice sentiment annotation, the user is prompted to self report their confidence in their annotation (with levels of ``I am confident'', ``I am a little unsure'', and ``I am a lot unsure'').
% The annotator was then prompted to repeat the process with their first choice for impact category (categories listed in the first column of \autoref{tab:impact_categories}).

% Once the user reports their confidence, the first choice impact category (categories listed in first column of \autoref{tab:impact_categories}) is annotated.  Here, ``first choice'' has the same overall definition as with the sentiment category (strongest, or first present).
% After the impact category is selected and ``Done'' (or ``Done \& Talk'') is selected, the user is then prompted for their second-choice selections for sentiment and impact, with an option to select ``no second choice present'' if there is no apparent secondary sentiment or impact category.
% After each annotation prompt (first and second choice sentiment, first and second choice impact), the user is prompted to self report their confidence level.

Annotation of each of the \nsentences\ sentences occured in batches of $\approx$200 sentences, with each sentence in a batch ``retired'' once three users have completed annotations on the sentence.
% timing calculation doc: https://colab.research.google.com/drive/1vDz8tjlB0H9nnt5G1NKtJZCSyonXaoAx?authuser=1#scrollTo=HJNfDtrwSniY
%The median annotation time per sentence is $\approx$20 seconds, however the spread is large with an IQR of $\approx$20 seconds.  This is likely due to users performing annotations ``in the background'' while completing other tasks.
The breakdown of the number of annotated sentences per movie is shown in \autoref{tab:movie_info}.

% In what follows, we make use of only first choice sentiment and impact categories in our analysis, with an analysis using self reported confidence levels and secondary categories relegated to a subsequent paper.


\subsection{Data Cleaning and Analysis} \label{sec:datacleaning}
%Raw annotations are downloaded from the Zooniverse platform and processed to determine intercoder metrics and final annotation categories.
After completing the annotation on \sentiment\ and \impact, we used Cohen's Kappa \citep{cohen1960coefficient} to measure the degree of reliability and agreement between the annotators. For sentiment categories, the kappa score was relatively consistent across each category with 0.66 for ``Positive'' sentiments, 0.67 for ``Neutral'' sentiments, and 0.68 for ``Negative'' sentiments.
For impact categories, the lowest Kappa measurement was for ``Impersonal Report'' (0.61) with higher measurements for ``Interest in Science Topic'', ``Shift in Cognition'' and ``Not applicable'' (0.67) along with ``Attitudes Toward the Film'' (0.68).

Final sentiment and impact categories were decided by a majority vote. In $\approx$10\% of the sentences, where no majority consensus was possible, all annotators discussed each sentence in a group and either a majority consensus was reached, or the sentence was marked as ``no agreement''. This resulted in eight ``no agreement'' sentences, which we excluded from our analysis.
Finally, we excluded sentences that were only emojis or consisted of solely punctuation marks (e.g., ``:D'').
% The composition of the remaining \nsentencesNoTBDs\ sentences in sentiment and impact categories is shown as a percentages heatmap in \autoref{fig:sentiment_impact} with the per-media breakdown of sentiment and impact category shown in \autoref{fig:sentiment_by_film} and \autoref{fig:impact_by_film}, respectively.
The composition of the remaining \nsentencesNoTBDs\ sentences in sentiment and impact categories is shown as a percentages heatmap in \autoref{fig:sentiment_impact} with the per-media breakdown in raw numbers of sentiments of sentiment and impact category per film shown in \autoref{tab:movie_info}.

% \rt{Be sure to cite \autoref{fig:sentiment_by_film} and \autoref{fig:impact_by_film}.}
% \rt{Combine both solar superstorms! same material, slightly different presentations}


%%%% JPN: option -- table with heatmap
\begin{table*}[h]
\begin{center}
\begin{tabular} {@{}lcccccccccc@{}} 
\toprule
\multicolumn{1}{c}{Media} & \multicolumn{1}{c}{Year} & \multicolumn{1}{c}{\# Sentences} & \multicolumn{3}{c}{Sentiment [\%]} & \multicolumn{5}{c}{Impact [\%]} \\\cmidrule(lr){4-6} \cmidrule(l){7-11}
\multicolumn{1}{c}{} & \multicolumn{1}{c}{} & \multicolumn{1}{c}{} & \multicolumn{1}{c}{+} & \multicolumn{1}{c}{0} & \multicolumn{1}{c}{-} & \multicolumn{1}{c}{A} & \multicolumn{1}{c}{S} & \multicolumn{1}{c}{C} & \multicolumn{1}{c}{I} &  \multicolumn{1}{c}{N} \\\cmidrule(lr){1-3}\cmidrule(lr){4-6} \cmidrule(l){7-11}


Space Junk & 2012 & 137 & \cellcolor[rgb]{0.37673202614379087,0.6530718954248366,0.8224836601307189}53.28 & \cellcolor[rgb]{0.8672664359861592,0.9193540945790081,0.967520184544406}13.14 & \cellcolor[rgb]{0.6718954248366014,0.8143790849673203,0.9006535947712418}33.58 & \cellcolor[rgb]{0.8605767012687427,0.29554786620530565,0.011118800461361017}73.72 & \cellcolor[rgb]{0.9991387927720108,0.9478662053056517,0.8965936178392926}2.92 & \cellcolor[rgb]{0.9982775855440216,0.9349480968858132,0.8716186082276047}5.84 & \cellcolor[rgb]{0.9982775855440216,0.9349480968858132,0.8716186082276047}5.84 & \cellcolor[rgb]{0.9964321414840446,0.9072664359861592,0.8181007304882737}11.68 \\ 
Solar Superstorms$^\dagger$ & 2013(15) & 107 & \cellcolor[rgb]{0.28089196462898885,0.5876201460976547,0.7850826605151865}60.75 & \cellcolor[rgb]{0.8318339100346022,0.8957324106113033,0.9557093425605536}17.76 & \cellcolor[rgb]{0.8023068050749712,0.8760476739715494,0.9458669742406767}21.5 & \cellcolor[rgb]{0.9515570934256055,0.4311418685121107,0.09657823913879277}60.75 & \cellcolor[rgb]{0.9988927335640139,0.9441753171856978,0.8894579008073817}3.74 & \cellcolor[rgb]{0.996555171088043,0.9091118800461361,0.8216685890042291}11.21 & \cellcolor[rgb]{0.9982775855440216,0.9349480968858132,0.8716186082276047}5.61 & \cellcolor[rgb]{0.9942176086120723,0.8610226835832372,0.7259669357939253}18.69 \\ 
SuperTornado & 2015 & 555 & \cellcolor[rgb]{0.36159938485198,0.6427374086889658,0.8165782391387928}54.41 & \cellcolor[rgb]{0.8377393310265283,0.8996693579392541,0.9576778162245291}17.12 & \cellcolor[rgb]{0.7358708189158016,0.8415686274509804,0.923044982698962}28.47 & \cellcolor[rgb]{0.9515570934256055,0.4311418685121107,0.09657823913879277}60.72 & \cellcolor[rgb]{0.9979084967320262,0.9294117647058824,0.8609150326797386}7.03 & \cellcolor[rgb]{0.9961860822760477,0.9035755478662053,0.810965013456363}12.43 & \cellcolor[rgb]{0.9990157631680123,0.9460207612456748,0.8930257593233372}3.42 & \cellcolor[rgb]{0.9949557862360631,0.8772625913110342,0.7584467512495194}16.4 \\ 
Seeing the Beginning of Time & 2017 & 233 & \cellcolor[rgb]{0.35151095732410614,0.6358477508650519,0.812641291810842}55.36 & \cellcolor[rgb]{0.8554555940023069,0.9114801999231065,0.9635832372164552}14.59 & \cellcolor[rgb]{0.7161860822760477,0.8332026143790849,0.916155324875048}30.04 & \cellcolor[rgb]{0.898961937716263,0.3483275663206459,0.039907727797001157}68.67 & \cellcolor[rgb]{0.9966782006920415,0.9109573241061131,0.8252364475201845}10.73 & \cellcolor[rgb]{0.9976624375240293,0.9257208765859285,0.8537793156478277}7.73 & \cellcolor[rgb]{1.0,0.9607843137254902,0.9215686274509803}0.0 & \cellcolor[rgb]{0.9960630526720492,0.9016224529027297,0.8071664744329105}12.88 \\ 
The Jupiter Enigma & 2018 & 219 & \cellcolor[rgb]{0.491764705882353,0.7219684736639754,0.8547789311803152}45.66 & \cellcolor[rgb]{0.7703191080353711,0.8562091503267973,0.9351018838908113}25.57 & \cellcolor[rgb]{0.7309496347558632,0.8394771241830065,0.9213225682429834}28.77 & \cellcolor[rgb]{0.9855132641291812,0.5330103806228376,0.2125951557093427}51.6 & \cellcolor[rgb]{0.996555171088043,0.9091118800461361,0.8216685890042291}10.96 & \cellcolor[rgb]{0.9960630526720492,0.9016224529027297,0.8071664744329105}12.79 & \cellcolor[rgb]{0.9985236447520185,0.938638985005767,0.8787543252595156}5.02 & \cellcolor[rgb]{0.9938485198000769,0.8529027297193387,0.7097270280661284}19.63 \\ 
Birth of Planet Earth & 2019 & 35 & \cellcolor[rgb]{0.4019530949634756,0.6702960399846213,0.8323260284505959}51.43 & \cellcolor[rgb]{0.8584083044982699,0.9134486735870818,0.9645674740484429}14.29 & \cellcolor[rgb]{0.6620530565167244,0.8101960784313725,0.8972087658592849}34.29 & \cellcolor[rgb]{0.8782929642445213,0.31990772779700116,0.024405997693194924}71.43 & \cellcolor[rgb]{0.9991387927720108,0.9478662053056517,0.8965936178392926}2.86 & \cellcolor[rgb]{0.9955709342560554,0.8907958477508651,0.7855132641291811}14.29 & \cellcolor[rgb]{1.0,0.9607843137254902,0.9215686274509803}0.0 & \cellcolor[rgb]{0.9964321414840446,0.9072664359861592,0.8181007304882737}11.43 \\ 




\hline
\end{tabular}
\end{center}
\footnotesize{$^\dagger$ Solar Superstorms sentences come from two media presentations: Solar Superstorms - a TV Episode from ``Cosmic Journeys'' \citep{solarsuperstorms2013} and a full-length documentary Solar Superstorms: Journey to the Center of the Sun \citep{solarsuperstorms2015}.}\\
\caption{Total number of annotated sentences per media.  Single character sentiment and impact codes are defined in \autoref{tab:impact_categories}.}
\label{tab:movie_info}
\end{table*}



%\end{center}


% \begin{table*}
% \begin{center}
% \begin{tabular} {@{}lcccccccccc@{}} %{|p{1.5in}|p{1.5in}|}
% %\begin{tabular}{ |p{1.0in}|p{1.7in}|p{1.7in}|p{1.7in}|p{1.7in}| }

% % \multicolumn{1}{|c|}{Impact Category} & \multicolumn{3}{c|}{Sentiment Category} \\
% % \multicolumn{1}{|c|}{} &\multicolumn{1}{c}{} & \multicolumn{1}{c}{} & \multicolumn{1}{c|}{}\\
% %  & \multicolumn{1}{c}{Positive (+)} & \multicolumn{1}{c}{Neutral (0)} & \multicolumn{1}{c|}{Negative (-)} \\
% \toprule
% \multicolumn{1}{c}{Media} & \multicolumn{1}{c}{Year} & \multicolumn{1}{c}{\# Sentences} & \multicolumn{3}{c}{Sentiment [\%]} & \multicolumn{5}{c}{Impact [\%]} \\\cmidrule(lr){4-6} \cmidrule(l){7-11}
% \multicolumn{1}{c}{} & \multicolumn{1}{c}{} & \multicolumn{1}{c}{} & \multicolumn{1}{c}{+} & \multicolumn{1}{c}{0} & \multicolumn{1}{c}{-} & \multicolumn{1}{c}{A} & \multicolumn{1}{c}{S} & \multicolumn{1}{c}{C} & \multicolumn{1}{c}{I} &  \multicolumn{1}{c}{N} \\\cmidrule(lr){1-3}\cmidrule(lr){4-6} \cmidrule(l){7-11}


% % Space Junk & 2012 & 137 & 73 & 18 & 46 & 101 & 4 & 8 & 8 & 16 \\ 
% % Solar Superstorms$^\dagger$ & 2013(15) & 107 & 65 & 19 & 23 & 65 & 4 & 12 & 6 & 20 \\ 
% % SuperTornado & 2015 & 555 & 302 & 95 & 158 & 337 & 39 & 69 & 19 & 91 \\ 
% % Seeing the Beginning of Time & 2017 & 233 & 129 & 34 & 70 & 160 & 25 & 18 & 0 & 30 \\ 
% % The Jupiter Enigma & 2018 & 219 & 100 & 56 & 63 & 113 & 24 & 28 & 11 & 43 \\ 
% % Birth of Planet Earth & 2019 & 35 & 18 & 5 & 12 & 25 & 1 & 5 & 0 & 4 \\

% Space Junk & 2012 & 137 & 53.28 & 13.14 & 33.58 & 73.72 & 2.92 & 5.84 & 5.84 & 11.68 \\ 
% Solar Superstorms$^\dagger$ & 2013(15) & 107 & 60.75 & 17.76 & 21.5 & 60.75 & 3.74 & 11.21 & 5.61 & 18.69 \\ 
% SuperTornado & 2015 & 555 & 54.41 & 17.12 & 28.47 & 60.72 & 7.03 & 12.43 & 3.42 & 16.4 \\ 
% Seeing the Beginning of Time & 2017 & 233 & 55.36 & 14.59 & 30.04 & 68.67 & 10.73 & 7.73 & 0.0 & 12.88 \\ 
% The Jupiter Enigma & 2018 & 219 & 45.66 & 25.57 & 28.77 & 51.6 & 10.96 & 12.79 & 5.02 & 19.63 \\ 
% Birth of Planet Earth & 2019 & 35 & 51.43 & 14.29 & 34.29 & 71.43 & 2.86 & 14.29 & 0.0 & 11.43 \\ 

% \hline
% \end{tabular}
% \end{center}
% %\footnotetext[$^\dagger$]{The smallest spatial unit is county}
% \footnotesize{$^\dagger$ Solar Superstorms sentences come from two media presentations: Solar Superstorms - a TV Episode from ``Cosmic Journeys'' \citep{solarsuperstorms2013} and a full-length documentary Solar Superstorms: Journey to the Center of the Sun \citep{solarsuperstorms2015}.}\\
% \caption{\rt{OPTION 1 FOR THIS TABLE: }Total number of annotated sentences per media.  Single character sentiment and impact codes are defined in \autoref{tab:impact_categories}.}
% \label{tab:movie_info}
% \end{table*}




% %%%% JPN: option -- table with heatmap AND lognorm
% \begin{table*}
% \begin{center}
% \begin{tabular} {@{}lcccccccccc@{}} 
% \toprule
% \multicolumn{1}{c}{Media} & \multicolumn{1}{c}{Year} & \multicolumn{1}{c}{\# Sentences} & \multicolumn{3}{c}{Sentiment [\%]} & \multicolumn{5}{c}{Impact [\%]} \\\cmidrule(lr){4-6} \cmidrule(l){7-11}
% \multicolumn{1}{c}{} & \multicolumn{1}{c}{} & \multicolumn{1}{c}{} & \multicolumn{1}{c}{+} & \multicolumn{1}{c}{0} & \multicolumn{1}{c}{-} & \multicolumn{1}{c}{A} & \multicolumn{1}{c}{S} & \multicolumn{1}{c}{C} & \multicolumn{1}{c}{I} &  \multicolumn{1}{c}{N} \\\cmidrule(lr){1-3}\cmidrule(lr){4-6} \cmidrule(l){7-11}


% % Space Junk & 2012 & 137 & 73 & 18 & 46 & \cellcolor[rgb]{0.1,0.5,0.8} 73.72 & 2.92 & 5.84 & 5.84 & 11.68 \\ 
% % Solar Superstorms$^\dagger$ & 2013(15) & 107 & 65 & 19 & 23 & 60.75 & 3.74 & 11.21 & 5.61 & 18.69 \\ 
% % SuperTornado & 2015 & 555 & 302 & 95 & 158 & 60.72 & 7.03 & 12.43 & 3.42 & 16.4 \\ 
% % Seeing the Beginning of Time & 2017 & 233 & 129 & 34 & 70 & 68.67 & 10.73 & 7.73 & 0.0 & 12.88 \\ 
% % The Jupiter Enigma & 2018 & 219 & 100 & 56 & 63 & 51.6 & 10.96 & 12.79 & 5.02 & 19.63 \\ 
% % Birth of Planet Earth & 2019 & 35 & 18 & 5 & 12 & 71.43 & 2.86 & 14.29 & 0.0 & 11.43 \\ 

% Space Junk & 2012 & 137 & \cellcolor[rgb]{0.949151,0.948435,0.152178}53.28 & \cellcolor[rgb]{0.993032,0.692907,0.189084}13.14 & \cellcolor[rgb]{0.977995,0.861432,0.142808}33.58 & \cellcolor[rgb]{0.993248,0.906157,0.143936}73.72 & \cellcolor[rgb]{0.180653,0.701402,0.488189}2.92 & \cellcolor[rgb]{0.304148,0.764704,0.419943}5.84 & \cellcolor[rgb]{0.304148,0.764704,0.419943}5.84 & \cellcolor[rgb]{0.477504,0.821444,0.318195}11.68 \\ 
% Solar Superstorms$^\dagger$ & 2013(15) & 107 & \cellcolor[rgb]{0.940015,0.975158,0.131326}60.75 & \cellcolor[rgb]{0.994355,0.746995,0.163821}17.76 & \cellcolor[rgb]{0.992505,0.777967,0.152855}21.5 & \cellcolor[rgb]{0.945636,0.899815,0.112838}60.75 & \cellcolor[rgb]{0.220124,0.725509,0.466226}3.74 & \cellcolor[rgb]{0.458674,0.816363,0.329727}11.21 & \cellcolor[rgb]{0.296479,0.761561,0.424223}5.61 & \cellcolor[rgb]{0.606045,0.850733,0.236712}18.69 \\ 
% SuperTornado & 2015 & 555 & \cellcolor[rgb]{0.946602,0.95519,0.150328}54.41 & \cellcolor[rgb]{0.994495,0.74088,0.166335}17.12 & \cellcolor[rgb]{0.985314,0.828846,0.142945}28.47 & \cellcolor[rgb]{0.945636,0.899815,0.112838}60.72 & \cellcolor[rgb]{0.344074,0.780029,0.397381}7.03 & \cellcolor[rgb]{0.487026,0.823929,0.312321}12.43 & \cellcolor[rgb]{0.202219,0.715272,0.476084}3.42 & \cellcolor[rgb]{0.565498,0.84243,0.262877}16.4 \\ 
% Seeing the Beginning of Time & 2017 & 233 & \cellcolor[rgb]{0.946602,0.95519,0.150328}55.36 & \cellcolor[rgb]{0.994103,0.710698,0.180097}14.59 & \cellcolor[rgb]{0.982653,0.841812,0.142303}30.04 & \cellcolor[rgb]{0.974417,0.90359,0.130215}68.67 & \cellcolor[rgb]{0.449368,0.813768,0.335384}10.73 & \cellcolor[rgb]{0.369214,0.788888,0.382914}7.73 & \cellcolor[rgb]{0.267004,0.004874,0.329415}0.0 & \cellcolor[rgb]{0.496615,0.826376,0.306377}12.88 \\ 
% The Jupiter Enigma & 2018 & 219 & \cellcolor[rgb]{0.959276,0.921407,0.151566}45.66 & \cellcolor[rgb]{0.988648,0.809579,0.145357}25.57 & \cellcolor[rgb]{0.984031,0.835315,0.142528}28.77 & \cellcolor[rgb]{0.89632,0.893616,0.096335}51.6 & \cellcolor[rgb]{0.458674,0.816363,0.329727}10.96 & \cellcolor[rgb]{0.496615,0.826376,0.306377}12.79 & \cellcolor[rgb]{0.274149,0.751988,0.436601}5.02 & \cellcolor[rgb]{0.616293,0.852709,0.230052}19.63 \\ 
% Birth of Planet Earth & 2019 & 35 & \cellcolor[rgb]{0.951726,0.941671,0.152925}51.43 & \cellcolor[rgb]{0.994103,0.710698,0.180097}14.29 & \cellcolor[rgb]{0.976265,0.868016,0.143351}34.29 & \cellcolor[rgb]{0.983868,0.904867,0.136897}71.43 & \cellcolor[rgb]{0.175707,0.6979,0.491033}2.86 & \cellcolor[rgb]{0.525776,0.833491,0.288127}14.29 & \cellcolor[rgb]{0.267004,0.004874,0.329415}0.0 & \cellcolor[rgb]{0.468053,0.818921,0.323998}11.43 \\ 



% \hline
% \end{tabular}
% \end{center}
% \footnote{$^\dagger$ Solar Superstorms sentences come from two media presentations: Solar Superstorms - a TV Episode from ``Cosmic Journeys'' \citep{solarsuperstorms2013} and a full-length documentary Solar Superstorms: Journey to the Center of the Sun \citep{solarsuperstorms2015}.}\\
% \caption{\rt{OPTION 3 FOR THIS TABLE: }Total number of annotated sentences per media.  Single character sentiment and impact codes are defined in \autoref{tab:impact_categories}.}
% \label{tab:movie_info3}
% \end{table*}



\begin{figure}[t]
\centering
% \includegraphics[width=0.9\columnwidth]{sections/figures/annotation_sentiment_impacts.pdf} % Reduce the figure size so that it is slightly narrower than the column. Don't use precise values for figure width.This setup will avoid overfull boxes.
\includegraphics[width=0.7\columnwidth]{sections/figures/annotation_sentiment_impacts_c6_T.pdf} % Reduce the figure size so that it is slightly narrower than the column. Don't use precise values for figure width.This setup will avoid overfull boxes.
\caption{Breakdown of full \nsentencesNoTBDs\ sentences in the annotated dataset as described in the ``Data'' section.  Numbers are percentages of the total sentences in the dataset within each Sentiment and Impact combination.}
%\caption{\rt{This shows the breakdown of impact analysis and sentiment spreads for final annotations.  My guess is that we want to use just one or other other of these plots (and maybe make it only the size of a column instead of full page?  Maybe numbers?  I like percentages numerically though...}}.
\label{fig:sentiment_impact}
\end{figure}

% \begin{figure*}[t]
% \centering
% \includegraphics[width=0.9\textwidth]{sections/figures/sentiment_by_film.pdf} 
% \caption{Breakdown of sentiment by film with films ordered from oldest (Space Junk, 2012) to newest (Birth of Planet Earth 2019).  Percentages are calculated as totals over all sentences of a specific film.}
% \label{fig:sentiment_by_film}
% \end{figure*}

% \begin{figure*}[t]
% \centering
% \includegraphics[width=0.9\textwidth]{sections/figures/impact_by_film.pdf} 
% \caption{Breakdown of impact by film with films ordered from oldest (Space Junk, 2012) to newest (Birth of Planet Earth 2019).  Percentages are calculated as totals over all sentences of a specific film.}
% \label{fig:impact_by_film}
% \end{figure*}

\section{EAC: \underline{E}diting \underline{A}nchor \underline{C}ompression}
\label{method}

\begin{figure}[t]
  \centering
  \includegraphics[width=0.48\textwidth]{figures/EAC.pdf}
  \vspace{-4mm}
  \caption{Proposed method: EAC. We first identify the key dimensions of the editing anchors using a weighted-gradient saliency map, followed by retraining on these dimensions to achieve the final optimization.}
  \vspace{-3mm}
  \label{EAC}
\end{figure}

In Section~\ref{analysis}, an in-depth analysis is provided on the factors that lead to the decrease in the general abilities of the model. 
Besides, it is found that the deviation of the edited parameter matrix could be constrained
by reducing the norm of the update matrix at each edit.
This helps maintain the semantic similarity of the facts recalled by the model before and after editing, ultimately preserving the general abilities of the edited model.
As depicted in Figure \ref{EAC}, a framework called EAC is proposed to compress information only in certain dimensions, thereby reducing the norm of the edited matrix and further constraining its deviation.

\subsection{Definition of Editing Anchors}
ROME uses an update matrix to insert a new knowledge triple \( t = (subject, relation, object) \). 
As mentioned in Section~\ref{sec_visual}, ROME calculates the update matrix by multiplying the pair \( (\mathbf{k}_*, \mathbf{v}_*) \), where \( \mathbf{k}_* \) identifies patterns of the input at the specified layer\footnotemark{} and \( \mathbf{v}_* \) is the fact recalled by the model.
Readers can refer to Appendix~\ref{b} for the details of ROME.
When injecting a new knowledge triple \( t = (subject, relation, object^*) \) to replace an old one \( t = (subject, relation, object) \), the specific part we aim to modify is the new relation \((relation, object^*) \), which is a property of the subject. It is believed that \( \mathbf{v}_* \) gathers all the information about how the model understands the subject and how it will respond thus we think that the new relation is primarily encoded in \( \mathbf{v}_* \).
Thus, EAC chooses to reduce the norm of \( \mathbf{v}_* \) for compression. Using a weighted gradient saliency map, EAC identifies high-scoring editing anchors crucial for encoding new relations. A scored elastic net is then applied to retrain and compress the editing information in key dimensions.
\footnotetext{Found by causal tracing methods~\cite{DBLP:conf/nips/MengBAB22}.}

\begin{figure*}[ht]
  \centering
  \includegraphics[width=0.5\textwidth]{figures/legend_edit.pdf}
  \vspace{-4mm}
\end{figure*}

\begin{figure*}[t]
  \centering
  \subfigure{
  \includegraphics[width=0.23\textwidth]{figures/ROME-GPT2XL-ZsRE-edit.pdf}}
  \subfigure{
  \includegraphics[width=0.23\textwidth]{figures/ROME-LLaMA3-8B-ZsRE-edit.pdf}}
  \subfigure{
  \includegraphics[width=0.23\textwidth]{figures/MEMIT-GPT2XL-ZsRE-edit.pdf}}
  \subfigure{
  \includegraphics[width=0.23\textwidth]{figures/MEMIT-LLaMA3-8B-ZsRE-edit.pdf}}
  \vspace{-2mm}
  \caption{Edited on the ZsRE dataset, the sequential editing performance of ROME and MEMIT with GPT2-XL and LLaMA-3 (8B) before and after the introduction of EAC, as the number of edits increases.}
  \vspace{-3mm}
  \label{edit-performance}
\end{figure*}

\subsection{Weighted-gradient Saliency Map}\label{a}
To reduce the norm of \( \mathbf{v}_* \) while preserving as much editing information as possible about the new relation, the goal is to compress this information over the smallest possible dimension. Drawing inspiration from gradient-based input saliency maps~\cite{DBLP:journals/corr/SmilkovTKVW17, DBLP:conf/nips/AdebayoGMGHK18}, a question is posed that whether a \textit{weight saliency map} can be constructed to aid compression. 
In previous work, ROME set \( \mathbf{v}_* = \arg\min_z \mathcal{L}(\mathbf{z}) \)~\cite{DBLP:conf/nips/MengBAB22}. Similar to the input saliency map, the gradient of this loss function with respect to each feature is utilized, and the magnitude of the values of \( \mathbf{v}_* \) is weighted accordingly to serve as the score for each feature: 
\begin{equation}
\text{score} = \mathbf{v}_* \odot \nabla \mathcal{L}(\mathbf{z}),
\label{score}
\end{equation}
where \(\odot\) is element-wise product. For GPT2-XL, the vectors $\mathbf{v}_*$ and $\mathbf{z}$ have dimensions of $1600 \times $1, whereas for LLaMA-3 (8B), the dimensions of these vectors are $4096 \times $1.
Based on Eq.~(\ref{score}), the desired weighted-gradient saliency map is obtained by applying a hard threshold:
\begin{equation}
\mathbf{m}_S = \mathbf{1} \left( \left| \text{score} \right| \geq \gamma \right),
\label{map}
\end{equation}
where \(\mathbf{1}(\mathbf{g} \geq \gamma)\) is an element-wise indicator function, which yields a value of 1 for the \(i\)-th element if \(g_i \geq \gamma\) and 0 otherwise.
\(| \cdot |\) is an element-wise absolute value operation, and \(\gamma > 0\) is a hard threshold. In practice, the value of \(\gamma\) is chosen according to different models and methods.
Specifically, the value of \(\gamma\) is chosen to retain the editing performance of the model in single editing. For more details refer to Appendix \ref{threshhold}.

With the introduction of the weighted gradient saliency map, the dimension of \( \mathbf{v}_* \) is split into two parts: one part represents the dimensions where the important editing anchors are located and it will be retrained to encode new relation, while the other part is set to 0, thereby reducing the norm of \( \mathbf{v}_* \). Based on Eq.~(\ref{map}), the \( \mathbf{v'}_* \), can be expressed as:
\begin{equation}
\mathbf{v'}_* \leftarrow \mathbf{m}_S \odot (\Delta \mathbf{v}_* + \mathbf{v}_*) + (\mathbf{1} - \mathbf{m}_S) \odot \mathbf{0},
\label{v}
\end{equation}
where \(\mathbf{1}\) denotes an all-one vector and \(\mathbf{0}\) denotes an all-zero vector. \(\Delta  \mathbf{v}_* \) is the part of \( \mathbf{v}_* \) that requires updating during retraining. Eq.~(\ref{v}) demonstrates that during retraining, only the dimensions where these important anchors are located need to be retrained to compress the editing information.

\subsection{Retraining Based on Scored Elastic Net}
After choosing important anchors, retraining for $\mathbf{v}_*$ is performed in this section.
To further compress the editing information, inspired by~\citet{zou2005regularization}, a score-based elastic net is also introduced during retraining:
\begin{equation}
\mathcal{L}_{0}(\mathbf{z}) = \lambda \|\mathbf{z}\|_{1, \alpha} + \mu \|\mathbf{z}\|_2^2,
\label{net}
\end{equation}
where \(\lambda \) and \(\mu \) are the hyper-parameters that control the strength of regularization and \( \mathbf{z} \) is the vector that causes the network to predict the target object in response to the factual prompt. Detailed hyper-parameters can be referred to in Appendix~\ref{hy}. Considering that the score computed in Eq.~(\ref{score}) represents the importance of the anchors for encoding the new relation, a weighted L1 norm is utilized when computing the L1 norm:
\begin{equation}
\|\mathbf{z}\|_{1, \alpha} = \sum_{i=1}^n \alpha_i |z_i|.
\end{equation}

In practice, we set \(\mathbf{\alpha} = \frac{1}{\text{score} + \epsilon}\), a small positive number \(\epsilon \) is introduced to prevent the score from being zero. 
Applying an elastic network, we ultimately derived the loss function during the retraining process to get the $\mathbf{v'}_*$ in Eq.~(\ref{v}):
\begin{equation}
\mathcal{L}_{r}(\mathbf{z}) =  \mathcal{L}(\mathbf{z}) + \mathcal{L}_{0}(\mathbf{z}).
\end{equation}

It is worth noting that when we make optimization here, only the dimensions where the editing anchors identified in section~\ref{a} are modified.
By introducing the elastic net, L1 regularization enables refining the selection of the editing anchors identified through the weighted-gradient saliency map during the retraining process. Meanwhile, L2 regularization effectively prevents model overfitting and improves the model's stability.
Finally, we complete the optimization. For specific optimization details, we recommend interested readers to refer to Appendix~\ref{b}.
Furthermore, the scored elastic net can also be applied to the FT. Readers can refer to Appendix \ref{FT} for more details.

\section{Results}
\label{sec:res}
This section evaluates the performance of our proposed method, \jola{}, in comparison with various baselines.
Table~\ref{tab:main_res} presents the average performance for all methods across the three tasks, while Figure~\ref{fig:main_res} illustrates the results for individual subtasks.
More detailed numerical results can be found in Appendix~\ref{appendix:full_res}.

%% main res table
\definecolor{myblue}{RGB}{120,145,181}


\begin{table*}[]
\centering
\small
\caption{Performance comparison across different generators and benchmarks. We evaluate different configurations, with critique-revision representing an iterative process where a critic model provides feedback to guide solution improvement. Pass@1 shows the success rate, while $\Delta_\uparrow$ and $\Delta_\downarrow$ indicate the percentage of wrong solutions being correctly revised and correct solutions being revised to wrong solutions, respectively.
Results are averaged over 5 random seeds.}
\label{tab:main}
\vspace{3mm}



\begin{tabular}{lcccccccccc}
\toprule
\multirow{2}{*}{} & \multicolumn{3}{c}{\textbf{CodeContests}} & \multicolumn{3}{c}{\textbf{LiveCodeBench}} & \multicolumn{3}{c}{\textbf{MBPP+}} & \textbf{Average} \\
 & \multicolumn{1}{c}{\textbf{Pass@1}} & \multicolumn{1}{c}{\textbf{$\Delta_\uparrow$}} & \multicolumn{1}{c}{\textbf{$\Delta_\downarrow$}} & \multicolumn{1}{c}{\textbf{Pass@1}} & \multicolumn{1}{c}{\textbf{$\Delta_\uparrow$}} & \multicolumn{1}{c}{\textbf{$\Delta_\downarrow$}} & \multicolumn{1}{c}{\textbf{Pass@1}} & \multicolumn{1}{c}{\textbf{$\Delta_\uparrow$}} & \multicolumn{1}{c}{\textbf{$\Delta_\downarrow$}} & \textbf{Pass@1} \\ \midrule
\rowcolor{gray!10} \multicolumn{11}{c}{\textit{Qwen2.5-Coder as Generator}} \\
Zero-shot & 7.88 & 0.00 & 0.00 & 30.54 & 0.00 & 0.00 & 77.83 & 0.00 & 0.00  & 38.75 \\
\emph{Single-turn Critique-revision} \\
Critique w/ Qwen2.5-Coder & 8.36 & 2.30 & 1.82 & 32.14 & 2.50 & 0.89 & 77.83 & 3.49 & 3.49 & 39.45 \\
Critique w/ GPT-4o & 10.67 & \textbf{4.85} & 2.06 & 32.32 & 2.32 & \textbf{0.54} & 77.46 & \textbf{3.81} & 4.18 & 40.15 \\
\rowcolor{myblue!20} Critique w/ {\ours} & \textbf{11.76} & 4.73 & \textbf{0.85} & \textbf{33.21} & \textbf{3.39} & 0.71 & \textbf{78.84} & 2.43 & \textbf{1.43} & \textbf{41.27} \\
\emph{Multi-turn Critique-revision} \\
Critique$\times 5$ w/ Qwen2.5-Coder & 9.21 & 3.76 & 2.42 & 29.64 & 2.14 & 3.04 & 76.03 & 3.81 & 5.61 & 38.30 \\
Critique$\times 5$ w/ GPT-4o & 12.48 & 7.03 & 2.42 & 32.86 & \textbf{4.82} & 2.50 & 74.60 & \textbf{4.34} & 	\textbf{7.57} &	39.98 \\
\rowcolor{myblue!20} Critique$\times 5$ w/ {\ours} & \textbf{16.24} & \textbf{9.21} & \textbf{0.85} & \textbf{33.39} & 3.75 & \textbf{0.89} & \textbf{78.68} & 3.23 & 2.38 & \textbf{42.77} \\
\midrule
\rowcolor{gray!10} \multicolumn{11}{c}{\textit{GPT-4o as Generator}} \\
Zero-shot & 20.61 & 0.00 & 0.00 & 32.32 & 0.00 & 0.00 & 77.67 & 0.00 & 0.00 & 43.53 \\
\emph{Single-turn Critique-revision} \\
Critique w/ Qwen2.5-Coder & 20.24 & 3.52 & 3.88 & \textbf{35.36} & \textbf{3.93} & 0.89 & 76.67 & 0.85 & 1.85 & 44.09 \\
Critique w/ GPT-4o & 20.97 & 2.30 & \textbf{1.94} & 34.82 & 2.68 & \textbf{0.18} & 77.41 & \textbf{1.01} & 1.27 & 44.40 \\
\rowcolor{myblue!20} Critique w/ {\ours} & \textbf{23.03} & \textbf{4.97} & 2.55 & 33.39 & 2.14 & 1.07 & \textbf{77.83} & 0.53 & \textbf{0.37} & \textbf{44.75} \\
\emph{Multi-turn Critique-revision} \\
Critique$\times 5$ w/ Qwen2.5-Coder & 19.52 & 5.21 & 6.30 & \textbf{35.54} & \textbf{5.36} & 2.14 & 76.08 & 1.53 & 3.12 & 43.71 \\
Critique$\times 5$ w/ GPT-4o & 20.61 & 3.39 & 3.39 & 35.18 & 3.21 & \textbf{0.36} & 76.61 & \textbf{2.06} & 3.12 & 44.13 \\
\rowcolor{myblue!20} Critique$\times 5$ w/ {\ours} & \textbf{25.45} & \textbf{7.88} & \textbf{3.03} & 34.11 & 3.21 & 1.43 &  \textbf{77.94} & 0.79 & \textbf{0.53} & \textbf{45.83} \\ \bottomrule
\end{tabular}







\end{table*}

%%

\noindent\textbf{Performance of Activation-Based Baselines.}
Activation editing baselines exhibit varying levels of success across tasks, but their sensitivity to hyperparameter selection and layer intervention limits their consistency.
For example, BitFit~\cite{ben-zaken-etal-2022-bitfit} demonstrates notable sensitivity to the placement of bias terms within the model.
Adjusting bias terms in dropout layers or attention mechanisms results in performance fluctuations, particularly in low-data scenarios.
Similarly, RED~\cite{wu-etal-2024-advancing} depends on the specific positions where scaling and bias vectors are introduced, leading to inconsistent results.
REPE~\cite{zou2023representation} is highly sensitive to the quality of activation representations across tasks, making it challenging to generalize its performance.
ReFT~\cite{wu2024reft} achieves moderate success by intervening on selected layers but faces challenges in determining the optimal number and choice of layers.
LoFIT~\cite{yin2024lofit}, while effective in leveraging task-relevant attention heads, struggles to maintain consistency across tasks.

\noindent\textbf{Performance of LoRA.}
LoRA achieves noticeable improvements over zero-shot baselines and, somewhat surprisingly, outperforms previous activation editing methods across all tasks when its rank hyperparameter is appropriately tuned. 
In tasks such as natural language generation, LoRA achieves higher BLEU and ROUGE-L scores, highlighting its ability to generate coherent outputs.

\noindent\textbf{Performance of \jola{}}
Our proposed method, \jola{}, consistently outperforms all baselines across the three tasks by a significant margin.
This can be attributed to \jola{}'s dynamic gated attention mechanism, which allows for adaptive activation of attention heads.
Unlike LoFIT~\cite{yin2024lofit}, which requires manual selection of attention heads, \jola{}'s mechanism enables less relevant heads to gradually ``die off" during training, improving robustness and adaptability.
In commonsense reasoning, \jola{} achieves an average improvement of 3.97\% over the best-performing baseline (LoRA) in LLaMA-3, as shown in Table~\ref{tab:main_res}.
For natural language understanding, \jola{} demonstrates consistent performance across diverse domains in the MMLU-Pro benchmark~\cite{wang2024mmlu} across all 14 subtasks as illustrated in Figure~\ref{fig:main_res}.
In natural language generation tasks, \jola{} achieves higher BLEU, ROUGE-L and BERTScore scores compared to activation-based baselines and LoRA.
\section{Discussion}
\label{sec:discussion}
In this work, we propose to leverage few-shot learning to enable users to self-define personal undesirable actions for personalized intervention on smartwatches.
We developed a three-stage pipeline that began with a self-supervised, pre-trained IMU model for robust feature extraction, then fine-tuned it for accurate human activity recognition, and finally enhanced it with data augmentation and synthesis that enabled rapid customization of new user-defined actions using only a small number of examples. 
We implemented this pipeline on a smartwatch as a real-time intervention system, \projectname, and demonstrated its effectiveness and advantages over the rule-based method through a multi-hour user study.
In this section, we discuss some interesting takeaways from our study, together with our vision of how \projectname can be generally applied to other health domains. We also briefly summarize the limitations of our work.


\subsection{Distorted Perception with AI-powered Intervention}
\label{sub:discussion:distorted}
During the study, we observed an interesting phenomenon where some participants developed a distorted perception towards their own actions or the intervention (see Sec.~\ref{sub:intervention_evaluation:qualitative_results}).
For instance, several participants felt \projectname's vibrations were stronger than the baseline (yet the actual strength of vibration remained constant), and some felt they did the target actions more frequently with \projectname (yet the objective data indicated otherwise).
There are several potential interpretations of such interesting observations.
The distorted perception might be caused by participants' heightened awareness of the AI-guided interventions: because \projectname more accurately and promptly caught the target actions, users started to pay extra and prolonged attention to any intervention. This could leave a stronger impression on them, and subsequently, they found it stronger or more frequent.
Another potential explanation is that the participants, often associating their personal and idiosyncratic undesirable actions with ``wrong-doing'' and thus responding with negative emotions, might have subconsciously perceived their undesirable actions as being more frequent due to the \projectname's more precise and timely feedback eliciting stronger negative emotions. This, combined with an emotional interpretation of being 'corrected', may have amplified their perception of the intervention's intensity (vibration strength) and created the mistaken impression of performing these actions excessively.

Meanwhile, it is an interesting open question of how long such perception will last from a longitudinal intervention perspective. Depending on the cases, the growing self-awareness and/or reliability of AI could potentially assist users in building a long-term habit to reduce the target action, or on the contrary, the effect may fade away due to the AI intervention method no longer being novel or enticing.
Future work can explore the lasting effect of the intervention, the corresponding perception, as well as user engagement in a long-term, field-based intervention study.~\cite{middleton2013long, short2018measuring, wei2020design}.


\subsection{Towards Human-AI Collaborative Interventions}
\label{sub:discussion:collaboration}
Users' mental models of \projectname varied significantly. Some viewed it as a passive watchdog, and some viewed it as a playful interactive system, while others sought to take greater agency in the moment of intervention delivery.
Our findings show the potential for and benefit of developing a collaborative relationship between humans and AI for behavioral intervention.
An AI system can provide appropriate support to users and help them achieve effective intervention outcomes.
Such collaboration is closely relevant to the vision of just-in-time adaptive interventions (JITAIs)~\cite{nahum-shani_translating_2021, nahum2018just}, where the delivery timing and methods of intervention are designed to be dynamically adapting to an individual's internal state and surrounding context.

For instance, for users who see the system as a passive monitor, the system can provide the option for them to configure the frequency and style of intervention (\eg higher/lower-intensity vibrations or consolidated notifications), ensuring the AI remains in the background but still provides supportive nudges.
Taking one step further, the AI system may analyze user behavior over time and suggest new setups or goals for users with transparency (\eg transitioning from nail-biting to managing stress). Users can accept, modify, or reject these suggestions, creating a dialogue where AI acts as a coach or collaborator rather than a rigid enforcer of predefined behaviors.
Meanwhile, for those who see AI as a proactive system, one promising avenue is to incorporate user feedback into the AI's learning process~\cite{orzikulova2024time2stop}. Users can label the AI's predictions as accurate or not, which could serve as input for the model to further adapt to the user and improve performance over time (\eg through reinforcement learning).
Combined with contextual information that can potentially be inferred from sensors~\cite{xu2023globem}, such feedback can enable more precise, context-sensitive and personalized JITIs.
In addition, the system would periodically prompt users to reassess their goals and update intervention targets, ensuring long-term relevance and efficacy.

It is noteworthy that such a human-AI collaboration paradigm needs to follow the principles of transparency and ethical design.
Other than the options mentioned above, namely custom configurations and continuous feedback, users should have visibility into the system's functionality and action logic regardless of the specific collaboration setup. This is important to provide users with agency and build their trust in the system.

\subsection{Beyond Smartwatch and Broader Customization}
In this work, our real-time intervention was implemented on a smartwatch. However, our proposed idea of empowering users to define any personal action and achieve a personalized intervention system can be more broadly applied to other domains.
Instead of relying solely on a watch-based IMU, we can explore other body-based sensor arrays (\eg headbands, rings, or joint sensors) to capture a more diverse range of behaviors in real time.
This would enable the system to accommodate a wide variety of undesirable actions or habits, such as posture corrections and fidgeting management.
In addition, beyond physical interventions, future customization can also delve into psychological or mental health support.
For instance, individuals dealing with obsessive-compulsive disorder (OCD) or other habitual thought/action patterns could define personal triggers (\eg a particular repetitive motion or behavioral cue) and receive timely AI-driven interventions.
Such holistic approaches highlight the flexibility and scalability of our pipeline.
By enabling user-defined actions, we open up possibilities for long-term and effective management of both physical and psychological well-being using a multitude of wearable and sensor-based platforms.

\subsection{Limitations}

Despite \projectname's positive outcome and the promising insights generated, we recognize some limitations in our study design.
As mentioned above, our current model relies solely on accelerometer data for action recognition, which may limit its ability to capture the full range of motion characteristics or other physiology. Future work can explore additional sensing modalities, such as gyroscope, photoplethysmography (PPG), joint locations, to enhance the accuracy and robustness of action recognition. 
Besides, the study was conducted with a relatively small number of participants and a limited set of actions, which may not fully capture the variability and diversity of human activities in real-world scenarios \cite{trapp2015individual, narayanan2013behavioral}.
Additionally, although we tried to simulate real-life scenarios, our intervention study was conducted over a limited duration and in controlled experimental settings, which may not fully reflect the complexities and dynamics of real-life environments. 
Real-world contexts introduce factors such as environmental noise, varying sensor placements, and user behavior changes over time \cite{trapp2015individual,truong2015deployment,mejia2023enhancing,mills2022development}, which were not thoroughly simulated in this study. Future research should conduct longitudinal field experiments with real-world deployment of the system.





\section{Conclusion and Future Work}
This paper analyzes the impact of scientific films using user-generated reviews and measures the impact using a novel taxonomy of change and engagement. 
Beyond the immediate goals of \avlshort\ the results of our best model as outlined in \autoref{tab:Classification Result Sentiment} and \autoref{section:classification} indicate our model's efficiency in predicting impact and sentiment which can be useful in future productions assessment by the \avlshort\ and other teams to assess the impact of their produced works. Additionally, the results of our generalizability analysis presented in \autoref{tab:Hubble_Prediction} and \autoref{section:generalizability} indicate the model should be applicable to a wide range of scientific CSV-style documentaries.
%as discussed in \autoref{section:classification_models} and \autoref{section:classification}, 
Overall, our results underscore the multifaceted impact of scientific documentaries. They engage audiences emotionally, provoke intellectual curiosity, and influence perceptions, making them a powerful medium for science communication. Our findings contribute to a growing body of evidence on the effectiveness of media in bridging the gap between science and the public, emphasizing the value of leveraging documentaries to achieve broader educational and societal goals. Future research could explore the long-term impacts of such engagement, particularly in terms of sustained interest in science and changes in behavior or policy advocacy.

The \avlshort\ expressed interest in tying specific learning goals to sentiment and impact categories and/or impact themes.
As learning goals are often developed to guide the creation of these scientific documentaries and are integral to grant funding, our large-scale quantitative analysis of the impact of these documentaries would be a useful complement to the smaller-scale audience-based feedback the \avlshort\ currently collects.
While our model was trained and tested on Amazon comments, our collaboration also has access to YouTube comments for a subset of the analyzed films. Future work will include an analysis of these comments, however, there is evidence of a bias toward more negative reviews on YouTube \citep[e.g.][]{ardestani2024youtube}, making such a comparison a test of the inter-platform generalizability of our model.

 %While beyond the scope of the present work, future research with a larger dataset of cross-platform and multilingual input could benefit from this more detailed analysis.



\section{Limitations and Ethics Statement}
Our data collection is limited only to English reviews, impacting the generalizability of our findings.
We also limited the scope of our study to only one platform, i.e., Amazon. This can enable platform-specific biases to influence the findings, and classification performance and cause potential inaccuracies.
In addition, media impact reviews are self-reported and collected only once. Without access to follow-up reviews or more nuanced, contextualized reports, we cannot be certain of the persistence and durability of such reported impacts. 
Finally, we acknowledge that LLM-based classifiers often make mistakes in predicting the best label, resulting in misclassified reviews. Future improvements in these models and better prompt engineering can enhance the performance of LLM classifiers. 




%\section{Acknowledgments}
\begin{acks}

The authors thank the work of Alistair Nunn, Rishabh Sharma, and Shashwat Mann for their annotation and analysis consultations. The authors additionally thank the members of the Advanced Visualization Laboratory at the National Center for Supercomputing Applications for their collaboration on this work.  Finally, the authors thank the referees of this work for their valuable input.
\end{acks}

\bibliographystyle{ACM-Reference-Format}
\bibliography{aaai22}
%%
%% The "author" command and its associated commands are used to define
%% the authors and their affiliations.
%% Of note is the shared affiliation of the first two authors, and the
%% "authornote" and "authornotemark" commands
%% used to denote shared contribution to the research.

%%
%% If your work has an appendix, this is the place to put it.



\appendix

\appendix
\onecolumn

\section*{Appendix}

In and this following sections we include additional experiments, further insights on related work, proofs, and theoretical details. The sections are organized as follows:
\begin{itemize}
   \item In \cref{subsec:causal_lm}, we provide additional experiments on causal language modeling.
   \item In \cref{sec:detailrelatedwork}, we include an extension of related work.
   \item In \cref{app:proofs}, we present proofs and theoretical details.
   \item In \cref{sec:app_experiments}, we explore ablation studies and parameter configurations for LRA, masked language Modeling,  and Image Classification tasks.
\end{itemize}



\section{Causal Language Modelling}
\label{subsec:causal_lm}
Efficient causal language modelling with linearized attention was previously studied by \citet{trans_rnn} and \citet{retnet}. In this setup, our formulation becomes similar to the retentive network \citep{retnet}, with the difference that \citet{retnet} choose their selective parameters before training and keep them fixed, while our selective parameters  are trained jointly with the rest of the model. 

We evaluate the performance of our formulation against the GPT-2 architecture \citep{radford2019language} and its linearized version obtained by simply removing the softmax (LinAtt). Our architectures \lions{} is trained with a trainable linear layer to obtain input-dependent selectivity, i.e., $a_i = \log(\sigma(\mathbf{W}_{a}\mathbf{x}_i + b))$. Note that our model do not use absolute positional encodings. We train our models in the OpenWebText corpus \citep{Gokaslan2019OpenWeb}. We evaluate the architectures in the 124 M parameter setup. Our implementation is based on nanoGPT\footnote{\url{https://github.com/karpathy/nanoGPT}}. We use the default GPT-2 hyperparameters and train our models for $8$ days in $4$ NVIDIA A100 SXM4 40 GB GPUs.

\begin{figure}[h]
    \centering
    \subfloat[OpenWebText PPL]{
        \raisebox{60pt}{\begin{tabular}{c|c}
            \toprule
            Model & Perplexity \\
            \midrule
            GPT-2 & $17.42_{(\pm 1.11)}$\\
            LinAtt & $21.07_{(\pm 1.32)}$\\
            \midrule 
            \rowcolor{orange!17} \lions (1D) & $18.16_{(\pm 1.16)}$\\
            \bottomrule
        \end{tabular}}} %
    \subfloat[Perplexity vs. sequence length]{
        \includegraphics[width=0.5\textwidth]{figs/ppl_vs_sequence_length_owt.pdf}}
    \caption{\textit{Causal Language Modelling results in the GPT-2 128M size.} \textit{(a)} Perplexity in the OpenWebText dataset. \textit{(b)} Perplexity vs. sequence length in OpenWebText. Our models improve over the LinAtt baseline \citep{trans_rnn} while obtaining similar performance to the GPT baseline and being able to extrapolate to larger context lengths than the one used during training.}
    \label{fig:owt}
\end{figure}

\begin{figure}
    \subfloat[Latency]{\includegraphics[width=0.5\linewidth]{figs/latency_vs_sequence_length_owt.pdf}}
    \subfloat[Memory]{\includegraphics[width=0.5\linewidth]{figs/memory_vs_sequence_length_owt.pdf}}
    \caption{\textit{Efficiency of the \lions(1D){} framework in the next-token generation task.} In \textit{(a)} and \textit{(b)} we measure respectively the latency and memory to generate the next token in the sentence. We compare three generation modes: Attention, Attention with KV cache and the Recurrence formulation. While all three produce the same output, the Recurrence formulation is the most efficient, requiring constant memory and latency to generate the next token.}
    \label{fig:memory_latency_LM}
\end{figure}

In \cref{fig:owt} we can observe \lions(1D){} significantly improve over the LinAtt baseline, while obtain perplexity close to GPT-2. The lack of absolute positional encodings allows \lions(1D){} to scale to larger sequence lengths than the one used during training. 

In \cref{fig:memory_latency_LM} we evaluate the latency and memory of \lions(1D){} in three modes: Attention, Attention + KV cache and RNN. While the three modes have the same output, the RNN formulation allows to save computation from previous token generations to require constant memory and latency for generating the next token. Our results align with the findings of \citet{retnet}, showing that efficient models in training and inference, with a strong performance (up to a small degradation) can be obtained.

\section{Detailed related work}
\label{sec:detailrelatedwork}

\subsection{State Space Models and Transformers}

State Space Models, such as S4 \citep{gu2021efficiently} and S5 \citep{s5}, advanced efficient sequence modeling with linear complexity. 
Mamba \citep{mamba} and Mamba-2 \citep{mamba2} introduced selective mechanisms within SSMs, achieving strong language modeling performance. 
Recently, many recurrent models for language have been proposed, e.g., xLSTM \citep{xlstm}, RWKV \citep{peng2024eagle}.
While RNNs for autoregressive modelling are prevalent, bidirectional models are less explored.
Hydra \citep{hwang2024hydrabidirectionalstatespace} extends Mamba to bidirectional settings using quasiseparable matrix mixers.
VisionMamba \citep{zhu2024visionmambaefficientvisual} employs two separate SSMs to pass over images.
However, these works are not equivalent to bidirectional attention.
\lion adopts a different approach: instead of extending SSMs, we derive equivalence between bidirectional attention with learnable mask and bidirectional RNNs.

Since the pioneering works \citep{tsai2019transformer,trans_rnn}, many works have been proposed to enhance linearized attention, including learnable relative positional encoding \citep{dai2019transformer}, gate mechanisms \citep{peng2021random,han2024demystify,ma2022mega}, FFT for kernelized attention \citep{luo2021stable}, decay terms in RetNet \citep{retnet}, and variants with enhanced expressiveness \citep{arora2024simple,zhang2024hedgehogporcupineexpressive,deltanet}.
These works focus on causal attention and cannot be directly applied with bidirectionality, while we explicitly write bidirectional attention as bidirectional RNN combined with selectivity, enhancing performance and providing a principled framework for parallel training and linear-time inference in non-causal tasks.

\subsection{Linear Transformers Summary}
\label{overview_lrmsec}
\label{ap:1}
\begin{table}[t]
\footnotesize
\caption{\textit{Overview of recent Linear Transformers applied to autoregressive language modeling.} The $-$ mark indicates models without scaling, as they lack \(\mathbf{\alpha}_i\) and \(\mathbf{\beta}_i\) and do not scale attention scores. The \(\times\) denotes matrix multiplication, \(\odot\) represents the Hadamard product, and \(*\) signifies the scalar product. All these models are used for autoregressive language modeling.
}
\resizebox{\textwidth}{!}{
\begin{tabular}{l|ll|cc|c|l}
\toprule
   \multirow{2}{*}{Model}  & \multicolumn{2}{c}{Recurrence Parameters }& \multicolumn{2}{l}{Operations} & \multirow{2}{*}{Scaled}&  \multirow{2}{*}{$\rightleftarrows$}\\
          &$\boldsymbol{\Lambda_i}$ & $\boldsymbol{\gamma_i}$  &   $\tikz\draw[white,fill=black] (0,0) circle (.5ex);$ & $\star$  \\
   \midrule
    Linear Trans\ \citep{trans_rnn}  &$\mathbf{I}$ &1& $\times$ & $\times$ & \checkmark & \xmark \\
    DeltaNet \citep{secretlin} &$\mathbf{I} - \gamma_i\mathbf{k}_i\mathbf{k}_i^{\top}$  & $\mathbf{\gamma_i}$& $*$ & $\times$ & \xmark & \xmark\\
    S4/S5 \citep{gu2021efficiently,s5}  &$\mathbf{e^{(- ({\delta} \mathbf{1}^\intercal)\odot \exp(\boldsymbol{A}))}}$&$\boldsymbol{B}$ & $\odot $ & $\odot $ & \xmark & \xmark\\
    Gated RFA \citep{peng2021random} &$g_i$&$1-g_i$ & $*$  &$*$   & \checkmark & \xmark \\
    RetNet \citep{retnet} &a & 1 & $*$  & $*$  & \xmark & \xmark\\
    Mamba (S6) \citep{mamba}  &$\mathbf{e^{(- ({\delta}_i \mathbf{1}^\intercal)\odot \exp(\boldsymbol{A_i}))}}$ &$\boldsymbol{B}_i$ & $\odot $ & $\odot $ & \xmark & \xmark \\
    GLA \citep{yang2023gated} & \textsc{DIAG}($g_i$)& 1 & $*$  & $\times$ & \xmark & \xmark \\
    RWKV \citep{peng2024eagle} & \textsc{DIAG}($g_i$)& 1 & $*$  & $\times$ & \xmark & \xmark\\
    xLSTM \citep{xlstm} &${f}_i$ &${i}_i$ & $*$  &$*$   & \checkmark 
    & \xmark \\
    Mamba-2 \citep{mamba2} &${a}_i$& 1 & $*$  &$*$   & \xmark & \xmark \\
    \rowcolor{Green!10}
   \textbf{\lionlit (ours)} & 1 & 1& $\odot$  &$*$   & \checkmark & \checkmark \\
   \rowcolor{violet!17}
   \textbf{\lionretnet (ours)} & $\gamma$ & $\gamma$& $\odot$  &$*$   & \checkmark & \checkmark \\
    \rowcolor{orange!17}
   \textbf{\lions (ours)} &${e}^{-{a}_i}$& $1$ & $\odot$  &$*$   & \checkmark & \checkmark \\
    \bottomrule
\end{tabular}
}
\label{table:overview_lrm}
\end{table}






\subsection{{Parallel Training and Efficient Inference}}
For linear transformers, efficient training is ideally achieved either by employing attention parallelization similar to  $    \mathbf{Y} = \text{softmax}\left(\mathbf{Q} \mathbf{K}^\top\right) 
    \mathbf{V}$ 
or by using techniques like parallel scan, as utilized by many structured SSMs (e.g., Mamba, S5) \citep{scanalg}. We will cover both techniques in the following sections.

\textbf{Parallel training in transformers.} As illustrated in equation $    \mathbf{Y} = \text{softmax}\left(\mathbf{Q} \mathbf{K}^\top\right) 
    \mathbf{V}$ of the Transformer, vectorization over the sequence is crucial to avoid sequential operations, where the model iterates over the sequence, leading to extensive training times \citep{vaswani_attention_2017}.
Parallelizing the operations across the sequence length for linear transformers ideally should take a form similar to \citep{trans_rnn,retnet}:
\begin{equation}\label{eq:lrmvec}
    \mathbf{Y} = \mathbf{M} \ast  \left(\mathbf{\phi(Q)} \mathbf{\phi(K)}^\top\right) \mathbf{V} 
\end{equation}
Here, \(\mathbf{M}\) represents a mask generated from the interaction of recurrent model parameters (\(\textcolor{black}{\boldsymbol{\Lambda_i}}\) and \(\textcolor{black}{\boldsymbol{\gamma_i}}\). Attention scores can be scaled before or after applying the mask $\mathbf{M}$ and during inference the scaling can be done by using the scaling state $\boldsymbol{\mathbf{z}}_i$. The symbol \(\ast\) indicates the operation in-which mask is applied to the attention. Equation \eqref{eq:lrmvec} highlights the importance of carefully selecting operations and parameters to ensure parallelizability during training. The mask \(\mathbf{M}\) is a lower diagonal mask in case of autoregressive models \citep{ma2022mega}.


\textbf{Parallel Scan.} Most SSMs utilize the state matrix \(\boldsymbol{\Lambda_i}\) as a full matrix, with the \(\textcolor{black}{\star}\) operation defined as matrix multiplication. Consequently, the output of each layer cannot be represented as in \eqref{eq:lrmvec}. This limitation becomes evident when applying recurrence over the discrete sequence in leading to the output:

\begin{align}
\label{eq:ssm}
\mathbf{h}_i = \mathbf{A}_i\mathbf{h}_{i-1} +\mathbf{B}_i\mathbf{x}_i , \quad
\mathbf{y}_i = \mathbf{C}^\top_i\mathbf{h}_i, \quad
\mathbf{y}_\iter = \boldsymbol{\bar{C}}_{i}^\top \sum_{j=1}^\iter \left( \prod_{k=j+1}^\iter \boldsymbol{\bar{A}}_{k} \right) \boldsymbol{\bar{B}}_{j} \mathbf{x}_j, 
\end{align}

which requires matrix multiplications for \(\boldsymbol{\bar{A}}_{k}\) across all tokens between \(i\) and \(j\), resulting in substantial memory requirements during training.

To mitigate this issue, SSMs adopt the parallel scan approach \citep{scanalg, scan2}, which enables efficient parallelization over sequence length. Initially introduced in S5 \citep{s5}, this method has a time complexity of \(\cO(L \log L)\). However, Mamba \citep{mamba} improves upon this by dividing storage and computation across GPUs, achieving linear scaling of \(\cO(L)\) with respect to sequence length and enabling parallelization over the state dimension \(N\). 
Ideally, a model should achieve complete parallelization in training without sequential operations, maintain a memory requirement for inference independent of token count, and have linear complexity. Table \ref{tab:summ o(n)} summarizes various training and inference strategies, along with their complexity and memory demands. 

Linear Transformers \citep{retnet, trans_rnn} employ attention during training and recurrence during inference, placing them in the last category of Table \ref{tab:summ o(n)}. To our knowledge, an exact mapping between attention and bidirectional recurrence does not exist; thus, naive forward and backward recurrence cannot be theoretically equated to the attention formulations in \eqref{eq:lrmvec} and  $    \mathbf{Y} = \text{softmax}\left(\mathbf{Q} \mathbf{K}^\top\right) 
    \mathbf{V}$. 



\subsection{Different Training strategies for models} \label{trainstrag}
\begin{table}[h]
   \caption{\textit{Summary of training and inference strategies.}
   $\rightleftarrows$ represents bidirectionality of the method.
   \looseness=-1Complexity indicates the computational and memory requirements during inference for processing \(L\) tokens \rebuttal{and $d$ is the model dimension}. \textbf{\lion} (Theorem \textbf{\ref{sec:theor}}) is designed to parallelize training using masked attention while employing recurrence during inference, specifically for bidirectional sequence modeling.} \label{tab:summ o(n)}
    \resizebox{1\textwidth}{!}{
\begin{tabular}{lllllll}
\toprule
   \begin{tabular}{@{}l@{}}  Train \\ Strategy \end{tabular}  &  \begin{tabular}{@{}l@{}}  Inference \\ Strategy \end{tabular}  & \begin{tabular}{@{}l@{}}  Method \\ Instantiations \end{tabular}  & \begin{tabular}{@{}l@{}}  Train sequential \\ operations \end{tabular}  & Complexity  & \begin{tabular}{@{}l@{}}  Inference \\ Memory \end{tabular} & \textbf{$\rightleftarrows$} \\
   \midrule
          Recurrence  & Recurrence & LSTM, GRU & $\cO(L)$ & $\cO(L\rebuttal{d})$ & $\cO(\rebuttal{d})$ & \xmark \\
          Recurrence  & Recurrence & ELMO & $\cO(L)$ & $\cO(L\rebuttal{d})$ & $\cO(L\rebuttal{d})$ & \checkmark \\
          Full Attention  & Full Attention &  Vit, BERT & $\cO(1)$& $\cO(L^2\rebuttal{d^2})$ & $\cO(L^2\rebuttal{d^2})$ & \checkmark \\ 
          Causal Attention & KV Cache & GPT-x, Llama & $\cO(1)$& $\cO(L^2\rebuttal{d^2})$ & $\cO(L\rebuttal{d^2})$ & \xmark \\   
          Causal Attention & Recurrence & LinearTrans, RetNet & $\cO(1)$&  $\cO(L\rebuttal{d^2})$ & $\cO(\rebuttal{d^2})$ & \xmark \\        
          Parallel Scan & Recurrence & Mamba, S4, S5 & $\cO(1)$ & $\cO(L\rebuttal{d})$ & $\cO(\rebuttal{d})$ & \xmark \\ 
          Parallel Scan & Recurrence  & Vim & $\cO(1)$ & $\cO(L\rebuttal{d^2})$ & $\cO(L\rebuttal{d})$ & \checkmark \\ 
          \rowcolor{blue!10}
        \textbf{Full Attention}  & \textbf{Recurrence}(\lion \textbf{\ref{sec:theor}})& \lionlit, \lions, \lionretnet & $\cO(1)$&$\cO(L\rebuttal{d^2})$ & $\cO(L\rebuttal{d})$ & \checkmark  \\  
    \bottomrule
\end{tabular} }
\vspace{-5mm}

\end{table}


\subsection{{Architectural Differences in Autoregressive Linear Transformers}}

\textbf{Multi-head attention and state expansion. }Another difference between various Linear Transformers, particularly SSMs and transformers, is how they expand single-head attention or SSM recurrence to learn different features at each layer, akin to convolutional neurons in CNNs \citep{resnet}. Transformers achieve this through multi-head attention, while SSMs like Mamba and Mamba-2 \citep{mamba, mamba2} use state expansion also known as Single-Input Single-Output (\textit{SISO}) framework to enlarge the hidden state. In \textit{SISO} framework, the input \(\mathbf{x}_{i}\) in  \eqref{eq:ssm} is a scalar and recurrence is applied to all elements in the hidden state independently \citep{s5}, allowing for parallelization during inference and training.

In contrast, simplified SSMs like S5 employ a Multiple-Input Multiple-Output (\textit{MIMO}) approach, where \(\mathbf{x}_{i}\) is a vector, which aligns them more closely with RNN variants like LRU \citep{lru} that are successful in long-range modeling \citep{s5}. However, the \textit{SISO} framework continues to be effective in Mamba models for language modeling \citep{mamba2}.

\textbf{Rule of Positional Encoding.} The parameter \(\textcolor{black}{\boldsymbol{\Lambda_i}}\) serves as a gating mechanism \citep{yang2023gated,mamba} and can also be interpreted as relative positional encoding \citep{retnet}. For instance, in an autoregressive model, considering \(\textcolor{black}{\boldsymbol{\Lambda_i}}\) as scaler, the mask \(\mathbf{M}\) can be defined as follows:
\begin{minipage}[t]{.5\textwidth}
    \vspace{-5mm}
    \begin{align}
    \label{eq:masksel}
    \textsc{\textbf{Se}}&\textsc{\textbf{lective Mask}} \notag \\
     \mathbf{M}_{ij} = &
    \begin{cases} 
    \Pi_{k=i}^{j+1}\lambda_k & i \geq j  \\
    0 & i < j
\end{cases}
    \end{align}
\end{minipage}%
\begin{minipage}[t]{.5\textwidth}
    \vspace{-5mm}
      \begin{align}  
      \label{eq:maskfix}
        \textsc{\textbf{Fi}}&\textsc{\textbf{xed Mask}} \notag  \\
   \mathbf{M}_{ij} = &
    \begin{cases} 
    \lambda^{i-j} & i \geq j  \\
    0 & i < j
\end{cases}
    \end{align}

\end{minipage}

In this context, the selective mask (where \(\textcolor{black}{\boldsymbol{\Lambda_i}} = {\lambda_i}\) varies for each token) is used in architectures like Mamba \citep{mamba}, while the fixed mask (where \(\textcolor{black}{\boldsymbol{\Lambda_i}} = {\lambda}\) is constant across all tokens) is implemented in architectures like RetNet \citep{retnet}. 
In both cases, the mask \(\mathbf{M}_{ij}\) provides rich relative positional encoding between tokens \(i\) and \(j\). Its structure reinforces the multiplication of all \(\boldsymbol{\Lambda_k}\) elements for \(k \in [j, \ldots, i]\), while the selectivity allows the model to disregard noisy tokens, preventing their inclusion in the attention matrix for other tokens. 

In contrast, Linear Transformers such as Linear Transformer \citep{trans_rnn} set \(\boldsymbol{\Lambda_k} = 1\), resulting in \(\mathbf{M}\) functioning as a standard causal mask, similar to those used in generative transformers \citep{gpt}. This necessitates the injection of positional information into the sequence, which is achieved using the traditional positional encoding employed in transformers \citep{vaswani_attention_2017}. In this framework, each element of the input data sequence is represented as \(\mathbf{x}_i = \mathbf{f}_i + \mathbf{t}_i\), where \(\mathbf{f}_i\) denotes the features at time \(i\) and \(\mathbf{t}_i\) represents the positional embedding. However, this traditional positional encoding has been shown to be less informative compared to relative positional encoding \citep{roformer}, which is utilized in other Linear Transformers where \(\boldsymbol{\Lambda_k} \neq 1\).



\subsection{\lion Framework}

\begin{figure}[h]
    \centering
    \includegraphics[width=1\textwidth]{figs/frfr.png}
    \caption{(\textit{Left}) Standard Transformer block. (\textit{Middle}) Training mode of \lion{} with  full linear attention. (\textit{Right}) Inference mode of \lion{} in the equivalent bidirectional RNN. 
    \rebuttal{Norm refers to Layer normalization,
    Proj is the projection operation to calculate $\mathbf{Q},\mathbf{K},\mathbf{V}$ and $\boldsymbol{\lambda}$ values,
    Scale is the scaling operation in \cref{equ:linear_attention}, 
    Inv is element wise inverse, 
    $\mathbf{A}$ is the linear attention,
    $\mathbf{M}^{F/B}$ are the forward and backward recurrence masks, $\mathbf{y}^{F/B}$ the outputs, and $c^{F/B}$ the scaling coefficients. We also provide a memory and scalability trade-off at inference time with chunking. 
    }} \vspace{-4mm}
    \label{fig:model}
\end{figure}



\subsection{Memory allocation in \lion during Forward and Backward recurrences} \label{ap:memoryall}
\begin{figure}[h]
    \centering
    \includegraphics[width=1\textwidth]{figs/memory.png}
  \caption{\textit{Memory allocation in \lion during Forward and Backward recurrences.} The efficient way of re-using the memory during inference is explained. }
    \label{fig:mem}
\end{figure}

During the forward and backward recurrences, as illustrated in Figure \ref{fig:mem}, each recurrence saves its corresponding output vector for each token, along with the scaling factor \(c\), to generate the final output. Once the backward recurrence reaches a token that the forward recurrence has already passed, it can directly calculate the output \(\mathbf{y}_i\) for that token, as \(c^F_i\) and \(\mathbf{y}^F_i\) have already been computed during the forward pass. Furthermore, the backward recurrence can overwrite the final output in the same memory cell where \(\mathbf{y}^F_i\) was stored, since both outputs share the same dimensions. This approach keeps memory allocation consistent with the forward pass, and the time required to process the sequence remains similar to that of autoregressive models, as both recurrences can traverse the sequence in parallel.

 \subsection{\rebuttal{Zero-Order Hold Discretization} }
\label{ap:zoh}
\rebuttal{Below we explain the zero-order hold discretization derived by \cite{kalman1960new}. An LTI system can be represented with the equation:
\begin{equation}
{\mathbf{h}}'(t) = A\mathbf{h}(t) + B\mathbf{x}(t),
\end{equation}
which can be rearranged to isolate \(\mathbf{h}(t)\):
\begin{equation}
{\mathbf{h}}'(t) - A\mathbf{h}(t) = B\mathbf{x}(t).
\end{equation}
By multiplying the equation by \(e^{-At}\), we get
\begin{equation}
e^{-At} {\mathbf{h}}'(t) - e^{-At}A\mathbf{h}(t) = e^{-At} B \mathbf{x}(t)
\label{eq:zho1}
\end{equation}
Since $\frac{\partial}{\partial t} e^{At} = Ae^{At} = e^{At}A$, \cref{eq:zho1} can be written as:
\begin{equation}
\frac{\partial}{\partial t} \left( e^{-At} \mathbf{h}(t) \right) = e^{-At} B \mathbf{x}(t).
\end{equation}
After integrating both sides and simplifications, we get
\begin{equation}
e^{-At} \mathbf{h}(t) = \int_{0}^{t} e^{-A\tau} B \mathbf{x}(\tau) \, d\tau + \mathbf{h}(0).
\end{equation}
By multiplying both sides by \(e^{At}\) to isolate \(\mathbf{h}(t)\) and performing further simplifications, at the end we get
\begin{equation}
\mathbf{h}(t) = e^{At} \int_{0}^{t} e^{-A\tau} B \mathbf{x}(\tau) \, d\tau + e^{At} \mathbf{h}(0).
\label{eq:zho2}
\end{equation}
To discretize this solution,  we can assume sampling the system at even intervals, i.e. each sample is at $kT$ for some time step $T$, and that the input \textbf{x}(t) is constant between samples. To simplify the notation, we can define $\mathbf{h}_k$ in terms of $\mathbf{h}(kT)$ such that
\begin{equation}
\mathbf{h}_k = \mathbf{h}(kT).
\end{equation}
Using the new notation, \cref{eq:zho2} becomes 
\begin{equation}
\mathbf{h}_k = e^{\mathbf{A}kT} \mathbf{h}(0) + e^{\mathbf{A}kT} \int_0^{kT} e^{-\mathbf{A}\tau} \mathbf{B} \mathbf{x}(\tau) \, d\tau.
\end{equation}
Now we want to express the system in the form:
\begin{equation}
\mathbf{h}_{k+1} = \mathbf{\tilde{A}} \mathbf{h}_k + \mathbf{\tilde{B}} \mathbf{x}_k.
\end{equation}
To start, let’s write out the equation for \(\mathbf{x}_{k+1}\) as
\begin{equation}
\mathbf{h}_{k+1} = e^{\mathbf{A}(k+1)T} \mathbf{h}(0) + e^{\mathbf{A}(k+1)T} \int_0^{(k+1)T} e^{-\mathbf{A}\tau} \mathbf{B} \mathbf{x}(\tau) \, d\tau.
\label{eq:zho3}
\end{equation}
After multiplying by \(e^{\mathbf{A}T}\) and rearranging we get
\begin{equation}
e^{\mathbf{A}(k+1)T} \mathbf{h}(0) = e^{\mathbf{A}T} \mathbf{h}_k - e^{\mathbf{A}(k+1)T} \int_0^{kT} e^{-\mathbf{A}\tau} \mathbf{B}\mathbf{x}(\tau) \, d\tau.
\end{equation}
Plugging this expression for \(\mathbf{x}_{k+1}\) in \cref{eq:zho3} yields to
\begin{equation}
\mathbf{h}_{k+1} = e^{\mathbf{A}T} \mathbf{h}_k - e^{\mathbf{A}(k+1)T} \left( \int_0^{kT} e^{-\mathbf{A}\tau} \mathbf{B}\mathbf{x}(\tau) \, d\tau + \int_0^{(k+1)T} e^{-\mathbf{A}\tau} \mathbf{B}\mathbf{x}(\tau) \, d\tau \right),
\end{equation}
which can be further simplified to
\begin{equation}
\mathbf{h}_{k+1} = e^{\mathbf{A}T} \mathbf{h}_k - e^{\mathbf{A}(k+1)T} \int_{kT}^{(k+1)T} e^{-\mathbf{A}\tau} \mathbf{B}\mathbf{x}(\tau) \, d\tau.
\end{equation}
Now, assuming that \(\mathbf{x}(t)\) is constant on the interval \([kT, (k+1)T)\), which allows us to take \(\mathbf{B}\mathbf{x}(t)\) outside the integral. Moreover, by bringing the \(e^{\mathbf{A}(k+1)T}\) term inside the integral we have
\begin{equation}
\mathbf{h}_{k+1} = e^{\mathbf{A}T} \mathbf{h}_k - \int_{kT}^{(k+1)T} e^{\mathbf{A}((k+1)T - \tau)} \, d\tau \, \mathbf{B}\mathbf{x}_k.
\end{equation}
Using a change of variables \(v = (k+1)T - \tau\), with \(d\tau = -dv\), and reversing the integration bounds results in
\begin{equation}
\mathbf{h}_{k+1} = e^{\mathbf{A}T} \mathbf{h}_k + \int_0^T e^{\mathbf{A}v} \, dv \, \mathbf{B}\mathbf{x}_k.
\end{equation}
Finally, if we evaluate the integral by noting that \(\frac{d}{dt} e^{\mathbf{A}t} = \mathbf{A} e^{\mathbf{A}t}\) and assuming \(\mathbf{A}\) is invertible, we get
\begin{equation}
\mathbf{h}_{k+1} = e^{\mathbf{A}T} \mathbf{h}_k + \mathbf{A}^{-1} \left( e^{\mathbf{A}T} - \mathbf{I} \right) \mathbf{B}\mathbf{x}_k.
\end{equation}
Thus, we find the discrete-time state and input matrices:
\begin{equation}
\mathbf{\tilde{A}} = e^{\mathbf{A}T}
\end{equation}
\begin{equation}
\mathbf{\tilde{B}} = \mathbf{A}^{-1} \left( e^{\mathbf{A}T} - \mathbf{I} \right) \mathbf{B}.
\end{equation}
And the final desecrate state space representation is:
\begin{equation}
\mathbf{h_k} = e^{\mathbf{A}T}\mathbf{h}_{k-1} +  \mathbf{A}^{-1} \left( e^{\mathbf{A}T} - \mathbf{I} \right) \mathbf{B}_k\mathbf{x}_k.
\end{equation}
As in case of \lions (similar to choice of Mamba2 \cite{mamba2}) the matrix $\mathbf{A}$ is identity while the time step $T$ is selective and equal to $a_i$. And simply for \lions scenario the term $Bx(t)$ will change into $\mathbf{k}_i\mathbf{v}^\top_i$ therefor considering Linear Transformer as continuous system like:
\begin{align}
\label{eq:recurscale22}
    \mathbf{S}'_{(t)} &=  \mathbf{S}_{(t)} + \mathbf{k}_{(t)}\mathbf{v}_{(t)}^{\top}, \\
    \mathbf{z}_{(t)} & = \mathbf{z}_{(t)} + {\mathbf{k}_{(t)}}, \\
\end{align}
By applying the ZOH discritization the final descreate \lions will be equal to:
 \begin{align}  
    \hspace{4mm}  & \hspace{-0.4cm}  \textsc{Discrete } \notag \\
      & \hspace{-0.4cm}  \mathbf{S}_{i} = e^{a_i}\mathbf{S}_{i-1}+ (e^{a_i}-1)\mathbf{k}_{i}\mathbf{v}_{i}^{\top},  \\
     &\hspace{-0.4cm}  \mathbf{z}_{i} =  e^{a_i}\mathbf{z}_{i-1} + (e^{a_i}-1)\mathbf{k}_{i}, 
    \end{align}
    And it applies to both directions forward and backward.
}

\section{Proofs}
\label{app:proofs}

\subsection{Proof of \Cref{prop:ssd}: Duality between Linear Recurrence and Attention}
Considering the following recurrence: \label{sec:proofqtk}

\begin{align}
\label{eq:recurscale2}
    \mathbf{S}_i &= \textcolor{black}{{\lambda_i}} \mathbf{S}_{i-1} + \mathbf{k}_i \mathbf{v}_i^{\top}, \\
    \mathbf{z}_i & = \textcolor{black}{{\lambda_i}} \mathbf{z}_{i-1} + {\mathbf{k}_i}, \\
    \hspace{4mm} \textsc{S}&\textsc{caled}: \mathbf{y}_i= \frac{{{\mathbf{q}_i}}^{\top} \mathbf{S}_i}{{{\mathbf{q}_i}}^{\top} \mathbf{z_i}} 
\end{align}

We can calculate each output $\mathbf{y}_i$ recursively as below: 

\scalebox{0.82}{
\begin{minipage}[t]{1.2\textwidth}  
\begin{align}
    &\mathbf{S}_1 = \mathbf{k}_1 \mathbf{v}_1^{\top},  \hspace{2mm} \mathbf{z}_1 = \mathbf{k}_1, \hspace{2mm} \mathbf{y}_1 = \mathbf{v}_1 \\
    &\mathbf{S}_2 = \mathbf{k}_2 \mathbf{v}_2^{\top}+{\lambda}_1\mathbf{k}_1 \mathbf{v}_1^{\top},  \hspace{2mm} \mathbf{z}_2 = \mathbf{k}_2+{\lambda}_1\mathbf{k}_1, \hspace{2mm}  \mathbf{y}_2= \frac{{{\mathbf{q}_2}}^{\top} (\mathbf{k}_2 \mathbf{v}_2^{\top}+{\lambda}_1\mathbf{k}_1\mathbf{v}^{\top}_1)}{{{\mathbf{q}_2}}^{\top} (\mathbf{k}_2+{\lambda}_1\mathbf{k}_1)} \\
    &\mathbf{S}_3 = \mathbf{k}_3 \mathbf{v}_3^{\top}+{\lambda}_1\mathbf{k}_2 \mathbf{v}_2^{\top}+ {\lambda}_2{\lambda}_1\mathbf{k}_1 \mathbf{v}_1^{\top},  \hspace{2mm} \mathbf{z}_3 = \mathbf{k}_3+{\lambda}_1\mathbf{k}_2+{\lambda}_2{\lambda}_1\mathbf{k}_1, \hspace{2mm}  \mathbf{y}_3= \frac{\mathbf{q}^{\top}_3(\mathbf{k}_3 \mathbf{v}_3^{\top}+{\lambda}_1\mathbf{k}_2 \mathbf{v}_2^{\top}+ {\lambda}_2{\lambda}_1\mathbf{k}_1 \mathbf{v}_1^{\top})}{\mathbf{q}^{\top}_3(\mathbf{k}_3+{\lambda}_1\mathbf{k}_2+{\lambda}_2{\lambda}_1\mathbf{k}_1)} \\
    &\Rightarrow \mathbf{y}_i = \frac{\mathbf{q}^{\top}_i(\sum_{j=1}^i \mathbf{M}^C_{ij}\mathbf{k}_j\mathbf{v}^{\top}_j)}{\mathbf{q}^{\top}_i(\sum_{j=1}^i \mathbf{M}^C_{ij}\mathbf{k}_j)},  \hspace{5mm}   \mathbf{M}^C_{ij} = 
    \begin{cases} 
    \Pi_{k=i}^{j+1}{\lambda_k} & i \geq j  \\
    0 & i < j
\end{cases}
\end{align}
\end{minipage}
}

This can be shown in a vectorized form as:
\begin{align}
   \mathbf{Y} = \textsc{scale}(\mathbf{Q}\mathbf{K}^{\top} \odot \mathbf{M}^C) \mathbf{V}
\end{align}
Where \textsc{scale} is the scaling function which scaled the attention matrix with respect to each row or can also be written as:
\begin{align}
   \textsc{scale}(\mathbf{A})_{ij} = \frac{\mathbf{A}_{ij}}{\sum_{j=1}^L\mathbf{A}_{ij}}
\end{align}

Similarly if the $\textsc{scale}$ is applied before masking we have:
\begin{align}
\label{eq:scalecm}
   \mathbf{Y} = \big(\textsc{scale}(\mathbf{Q}\mathbf{K}^{\top} \odot \mathbf{M}_{\textsc{causal}})\odot \mathbf{M}\big) \mathbf{V}
\end{align}

With $\mathbf{M}_{\textsc{causal}}$ being the causal mask used in autoregressive models \citep{gpt}. This vectorized form is equivalent to:

\begin{minipage}[t]{1\textwidth}  
\begin{align}
\label{eq:yscale}
\mathbf{y}_i = \frac{\mathbf{q}^{\top}_i(\sum_{j=1}^i \mathbf{M}_{ij}\mathbf{k}_j\mathbf{v}^{\top}_j)}{\mathbf{q}^{\top}_i(\sum_{j=1}^i \mathbf{k}_j)},  \hspace{5mm}   \mathbf{M}_{ij} = 
    \begin{cases} 
    \Pi_{k=i}^{j+1}{\lambda_k} & i \geq j  \\
    0 & i < j
\end{cases}
\end{align}
\end{minipage}

And the recurrence for this vectorized form can be written as:

\begin{align}
\label{eq:recurscaleafter}
    \mathbf{S}_i &= \textcolor{black}{{\lambda_i}} \mathbf{S}_{i-1} + \mathbf{k}_i \mathbf{v}_i^{\top}, \\
    \mathbf{z}_i & =  \mathbf{z}_{i-1} + {\mathbf{k}_i}, \\
    \hspace{4mm} \textsc{S}&\textsc{caled}: \mathbf{y}_i= \frac{{{\mathbf{q}_i}}^{\top} \mathbf{S}_i}{{{\mathbf{q}_i}}^{\top} \mathbf{z_i}} \label{eq:scaling_zi}
\end{align}

\subsection{Forward and Backward Recurrences Theoretical Details}

Considering the following recurrence:

\begin{align}
    \mathbf{S}_i &= \textcolor{black}{{\lambda_i}} \mathbf{S}_{i-1} + \mathbf{k}_i \mathbf{v}_i^{\top}, \\
    \mathbf{z}_L & = \sum^{L}_{i=1} \mathbf{k}_i \\
    \mathbf{y}_i &= \frac{{{\mathbf{q}_i}}^{\top} \mathbf{S}_i}{{{\mathbf{q}_i}}^{\top} \mathbf{z_L}} 
    \label{eq:forzl}
\end{align}

This recurrence is the same as recurrence \eqref{eq:recurscaleafter} but with $\mathbf{z}_L$ being fixed to the summation of all keys in the sequence, therefor the output $\mathbf{y}_i$ can simply be written as:

\begin{minipage}[t]{1\textwidth}  
\begin{align}
\label{eq:dumyrec}
\mathbf{y}_i = \frac{\mathbf{q}^{\top}_i(\sum_{j=1}^i \mathbf{M}_{ij}\mathbf{k}_j\mathbf{v}^{\top}_j)}{\mathbf{q}^{\top}_i\mathbf{z}_L},  \hspace{5mm}   \mathbf{M}_{ij} = 
    \begin{cases} 
    \Pi_{k=i}^{j+1}{\lambda_k} & i \geq j  \\
    0 & i < j
\end{cases}
\end{align}
\end{minipage}

By replacing the $\mathbf{z}_i = \sum_{j=1}^i \mathbf{k}_j$ in the denominator of equation \eqref{eq:scaling_zi} with $\mathbf{z}_L$. Therefore in vectorized form, it will become:

\begin{align}
   \mathbf{Y} = (\mathbf{A}^{C}  \odot \mathbf{M}\big) \mathbf{V}
\end{align}

With $\mathbf{A}^{C}$ being:

\begin{center}
\scalebox{0.75}{
\begin{minipage}[t]{1.25\textwidth}  
\begin{align*}   
 \mathbf{A}^{C} = \left( \renewcommand*{\arraystretch}{1} \begin{array}{cccc}
       {\frac{\mathbf{q}_1^{\top}\mathbf{k}_1}{\mathbf{q}_1^{\top}\mathbf{z}_L}} &  &  &  \\
      {\frac{\mathbf{q}_2^{\top}\mathbf{k}_1}{\mathbf{q}_2^{\top}\mathbf{z}_L}} & {\frac{\mathbf{q}_2^{\top}\mathbf{k}_2}{\mathbf{q}_2^{\top}\mathbf{z}_L}} &  &  \\
      {\frac{\mathbf{q}_3^{\top}\mathbf{k}_1}{\mathbf{q}_3^{\top}\mathbf{z}_L}} & {\frac{\mathbf{q}_3^{\top}\mathbf{k}_2}{\mathbf{q}_3^{\top}\mathbf{z}_L}} &  {\frac{\mathbf{q}_3^{\top}\mathbf{k}_3}{\mathbf{q}_3^{\top}\mathbf{z}_L}} &  \\
      \vdots & \vdots & \vdots & \ddots \\
      {\frac{\mathbf{q}_L^{\top}\mathbf{k}_1}{\mathbf{q}_L^{\top}\mathbf{z}_L}} & {\frac{\mathbf{q}_L^{\top}\mathbf{k}_2}{\mathbf{q}_L^{\top}\mathbf{z}_L}} & \cdots & {\frac{\mathbf{q}_L^{\top}\mathbf{k}_L}{\mathbf{q}_L^{\top}\mathbf{z}_L}} \\
  \end{array} \right)
\end{align*} 
\end{minipage}
}
\end{center}

Importantly this equation can  be written as:

\begin{align}
   \mathbf{Y} = \big(\textsc{scale}(\mathbf{Q}\mathbf{K}^{\top} )\odot \mathbf{M}\big) \mathbf{V}
\end{align}

which despite equation \eqref{eq:scalecm} scaling is applied over the whole sequence not for the causal part of the sequence.
The matrix $\mathbf{A}^{C}$ is helpful for driving the recurrent version of \lion for Forward and Backward recurrences and the mask here $\mathbf{M}$ is equal to \lion's forward mask $\mathbf{M}^F$ in equation \eqref{eq:backpath}. As shown in \eqref{eq:backpath} the forward recurrence for the causal part of the attention can be presented as $\mathbf{Y}^{B} = \mathbf{A}^F \odot \mathbf{M}^F$ the matrix $\mathbf{A}^F$ can be created simply by using matrix $\mathbf{A}^{C}$ as bellow:


\begin{center}
\scalebox{0.75}{
\begin{minipage}[t]{1.25\textwidth}  
\begin{align*}   
    \underbrace{\left( \renewcommand*{\arraystretch}{1} \begin{array}{cccc}
       \blue{\frac{1}{2}\frac{\mathbf{q}_1^{\top}\mathbf{k}_1}{\mathbf{q}_1^{\top}\mathbf{z}_L}} &  &  &  \\
      \blue{\frac{\mathbf{q}_2^{\top}\mathbf{k}_1}{\mathbf{q}_2^{\top}\mathbf{z}_L}} & \blue{\frac{1}{2} \frac{\mathbf{q}_2^{\top}\mathbf{k}_2}{\mathbf{q}_2^{\top}\mathbf{z}_L}} &  &  \\
      \blue{\frac{\mathbf{q}_3^{\top}\mathbf{k}_1}{\mathbf{q}_3^{\top}\mathbf{z}_L}} & \blue{\frac{\mathbf{q}_3^{\top}\mathbf{k}_2}{\mathbf{q}_3^{\top}\mathbf{z}_L}} &  \blue{\frac{1}{2}\frac{\mathbf{q}_3^{\top}\mathbf{k}_3}{\mathbf{q}_3^{\top}\mathbf{z}_L}} &  \\
      \blue\vdots & \blue\vdots & \blue\vdots & \blue\ddots \\
      \blue{\frac{\mathbf{q}_L^{\top}\mathbf{k}_1}{\mathbf{q}_L^{\top}\mathbf{z}_L}} & \blue{\frac{\mathbf{q}_L^{\top}\mathbf{k}_2}{\mathbf{q}_L^{\top}\mathbf{z}_L}} & \blue\cdots & \blue{\frac{1}{2}\frac{\mathbf{q}_L^{\top}\mathbf{k}_L}{\mathbf{q}_L^{\top}\mathbf{z}_L}} \\
  \end{array} \right)}_{\mathbf{A}^F} =
  \underbrace{\left( \renewcommand*{\arraystretch}{1} \begin{array}{cccc}
       \blue{\frac{\mathbf{q}_1^{\top}\mathbf{k}_1}{\mathbf{q}_1^{\top}\mathbf{z}_L}} &  &  &  \\
      \blue{\frac{\mathbf{q}_2^{\top}\mathbf{k}_1}{\mathbf{q}_2^{\top}\mathbf{z}_L}} & \blue{\frac{\mathbf{q}_2^{\top}\mathbf{k}_2}{\mathbf{q}_2^{\top}\mathbf{z}_L}} &  &  \\
      \blue{\frac{\mathbf{q}_3^{\top}\mathbf{k}_1}{\mathbf{q}_3^{\top}\mathbf{z}_L}} & \blue{\frac{\mathbf{q}_3^{\top}\mathbf{k}_2}{\mathbf{q}_3^{\top}\mathbf{z}_L}} &  \blue{\frac{\mathbf{q}_3^{\top}\mathbf{k}_3}{\mathbf{q}_3^{\top}\mathbf{z}_L}} &  \\
      \blue\vdots & \blue\vdots & \blue\vdots & \blue\ddots \\
      \blue{\frac{\mathbf{q}_L^{\top}\mathbf{k}_1}{\mathbf{q}_L^{\top}\mathbf{z}_L}} & \blue{\frac{\mathbf{q}_L^{\top}\mathbf{k}_2}{\mathbf{q}_L^{\top}\mathbf{z}_L}} & \blue\cdots & \blue{\frac{\mathbf{q}_L^{\top}\mathbf{k}_L}{\mathbf{q}_L^{\top}\mathbf{z}_L}} \\
  \end{array} \right)}_{\mathbf{A}^{C}} -
     \underbrace{\left( \renewcommand*{\arraystretch}{1} \begin{array}{ccccc}
       \dcolor{\frac{1}{2}\frac{\mathbf{q}_1^{\top}\mathbf{k}_1}{\mathbf{q}_1^{\top}\mathbf{z}_L}} &  &  & & \\
      & \dcolor{\frac{1}{2} \frac{\mathbf{q}_2^{\top}\mathbf{k}_2}{\mathbf{q}_2^{\top}\mathbf{z}_L}} &  &  &\\
       & &  \dcolor{\frac{1}{2}\frac{\mathbf{q}_3^{\top}\mathbf{k}_3}{\mathbf{q}_3^{\top}\mathbf{z}_L}} &  &\\
       &  &  & \dcolor\ddots & \\
       &  &  & &\dcolor{\frac{1}{2}\frac{\mathbf{q}_L^{\top}\mathbf{k}_L}{\mathbf{q}_L^{\top}\mathbf{z}_L}} \\
  \end{array} \right)}_{\textcolor{red!80}{\mathbf{D}^F}}
\end{align*} 
\end{minipage}
}
\end{center}

Or equivalently:

\begin{equation}
    \mathbf{Y}^F = \mathbf{A}^F \odot \mathbf{M}^F = (\mathbf{A}^C - {\textcolor{red!80}{\mathbf{D}^F}} ) \odot \mathbf{M}^F
\end{equation}

Since the diagonal values of the mask \(\mathbf{M}^F\) are all ones and the matrix \({\textcolor{red!80}{\mathbf{D}^F}}\) is diagonal, we have:

\begin{equation}
\label{eq:dumcev2}
    \mathbf{Y}^F = (\mathbf{A}^C - {\textcolor{red!80}{\mathbf{D}^F}} ) \odot \mathbf{M}^F = \mathbf{A}^C  { \odot \mathbf{M}^F - {\mathbf{D}^F}}
\end{equation}



As $\mathbf{A}^C \odot \mathbf{M}^F$ corresponds to linear recurrence shown at \eqref{eq:dumyrec}. The vectorized form \eqref{eq:dumcev2} can be presented as linear recurrence:

\begin{minipage}[t]{1\textwidth}  
\begin{align}
\label{eq:recrec2}
\mathbf{y}_i = \frac{\mathbf{q}^{\top}_i(\sum_{j=1}^i \mathbf{M}_{ij}\mathbf{k}_j\mathbf{v}^{\top}_j)}{\mathbf{q}^{\top}_i\mathbf{z}_L} -\frac{1}{2}\frac{\mathbf{q}^{\top}_i\mathbf{k}_i}{\mathbf{q}^{\top}_i\mathbf{z}_L},  \hspace{5mm}   \mathbf{M}_{ij} = 
    \begin{cases} 
    \Pi_{k=i}^{j+1}{\lambda_k} & i \geq j  \\
    0 & i < j
\end{cases}
\end{align}
\end{minipage}

This is equivalent to the linear recurrence presented in equation \eqref{eq:forzl}. The same theoretical approach applies to the backward recurrence, leading to the following linear recurrence for both recurrences:  


\begin{minipage}[t]{.35\textwidth}
\begin{subequations}
\label{eq:rec11}
\begin{align}
    \mathbf{S}^{F}_i &= \textcolor{black}{{\lambda}_i} \mathbf{S}^F_{i-1} + \mathbf{k}_i \mathbf{v}_i^{\top}, \\
    \mathbf{y}^{F}_i &= \frac{{{\mathbf{q}_i}}^{\top} \mathbf{S}^F_i}{{{\mathbf{q}_i}}^{\top} \mathbf{z}_L}  - \frac{1}{2}\frac{{{\mathbf{q}_{i}}}^{\top} \mathbf{k}_{i}}{{{\mathbf{q}_{i}}}^{\top} \mathbf{z}_L}
\end{align}
\end{subequations}
\end{minipage}
\begin{minipage}[t]{.64\textwidth}
\begin{subequations}
\label{eq:rec12}
    \begin{align}
    &\mathbf{S}^{B}_i= \textcolor{black}{{\lambda}_{L-i}} \mathbf{S}^B_{i-1} + \mathbf{k}_{L-i+1} \mathbf{v}_{L-i+1}^{\top}, \\
    &\mathbf{y}^{B}_{L-i+1} = \frac{{{\mathbf{q}_{L-i+1}}}^{\top} \mathbf{S}^B_i}{{{\mathbf{q}_{L-i+1}}}^{\top} \mathbf{z}_L} - \frac{1}{2}\frac{{{\mathbf{q}_{L-i+1}}}^{\top} \mathbf{k}_{L-i+1}}{{{\mathbf{q}_{L-i+1}}}^{\top} \mathbf{z}_L}
\end{align}
\end{subequations}
\end{minipage}%

However, the above equation requires access to the summation of scaling values \(\mathbf{z}_L\). A naive approach would involve adding an additional scaling recurrence alongside the forward and backward recurrences to compute the summation of all keys in the sequence. This approach, however, is inefficient, as it complicates the process. While the forward and backward recurrences can traverse the sequence in parallel to obtain the forward and backward recurrences outputs \(\mathbf{Y}^F\) and \(\mathbf{Y}^B\), the scaling recurrence must be computed prior to these recurrences because both the forward and backward recurrences computations rely on the final scaling value \(\mathbf{z}_L\) to generate their outputs.

\subsection{Efficient and Simple Method for Scaling Attention During Inference}

As shown in previous section scaled attention matrix can be formulated as two recurrences \eqref{eq:rec11} and \eqref{eq:rec12} with an additional recurrence to sum all the keys ($\mathbf{z}_L$). This section we will proof how to avoid an extra scaling recurrence by simple modifications to equation \eqref{eq:rec11} and \eqref{eq:rec12}.

Considering having a scaling recurrence as part of forward and backward recurrence we will have:

\begin{minipage}[t]{.35\textwidth}
\begin{subequations}
\begin{align}
    \mathbf{S}^{F}_i &= \textcolor{black}{{\lambda}_i} \mathbf{S}^F_{i-1} + \mathbf{k}_i \mathbf{v}_i^{\top}, \\
    \mathbf{z}^F_i & =  \mathbf{z}^F_{i-1} +  \mathbf{k}_i \\
    c^F_i & =  {{\mathbf{q}_i}}^{\top} \mathbf{z}^F_{i} - \frac{1}{2}{{\mathbf{q}_i}}^{\top} \mathbf{k}_i \\
    \mathbf{y}^{F}_i &= {{{\mathbf{q}_i}}^{\top} \mathbf{S}^F_i}  - \frac{1}{2}{{\mathbf{q}_i}}^{\top} \mathbf{k}_i \mathbf{v}_i
\end{align}
\end{subequations}
\end{minipage}
\begin{minipage}[t]{.64\textwidth}
\begin{subequations}
    \begin{align}
    &\mathbf{S}^{B}_i= \textcolor{black}{{\lambda}_{L-i}} \mathbf{S}^B_{i-1} + \mathbf{k}_{L-i+1} \mathbf{v}_{L-i+1}^{\top}, \\
    & \mathbf{z}^B_i= \mathbf{z}^B_{i-1} +  \mathbf{k}_{L-i+1} \\
    & c^B_i = {\mathbf{q}_{L-i+1}^{\top}}\mathbf{z}^B_i -  \frac{1}{2}{\mathbf{q}_{L-i+1}^{\top}}\mathbf{k}_{L-i+1} \\
    &\mathbf{y}^{B}_{L-i+1} = {{{\mathbf{q}_{L-i+1}}}^{\top} \mathbf{S}^B_i} -  \frac{1}{2}{\mathbf{q}_{L-i+1}^{\top}}\mathbf{k}_{L-i+1}\mathbf{v}_{L-i+1}^{\top}
\end{align}
\end{subequations}
\end{minipage}

The equations above are similar to the previous ones, with the addition of scalar states \(c^F\) and \(c^B\) for the backward and forward recurrences, respectively. During each recurrence, the outputs \(\mathbf{y}^{F}_i\) and \(\mathbf{y}^{B}_i\), along with the scalars \(c^F_i\) and \(c^B_i\), are saved for each token to construct the final output of each layer. \textit{It is also important to note that there is no need to save \(\mathbf{z}^F\) and \(\mathbf{z}^B\) for each token; these states can simply be overwritten in memory.} The final output of each layer is equal to:
\begin{align}
    \mathbf{y}_i = \frac{\mathbf{y}^{F}_i + \mathbf{y}^{B}_i}{c^F_i + c^B_i }
\end{align}
Where $\mathbf{y}^{F}_i$ and  $\mathbf{y}^{B}_i$ can be written as:

\begin{align}
    \mathbf{y}^{F}_i = \mathbf{q}^{\top}_i(\sum_{j=1}^i \mathbf{M}^{F}_{ij}\mathbf{k}_j\mathbf{v}^{\top}_j)
    - \frac{1}{2}{{\mathbf{q}_i}}^{\top} \mathbf{k}_i \mathbf{v}_i
    \hspace{2mm}
    ,
    \hspace{2mm}
     \mathbf{y}^{B}_i = \mathbf{q}^{\top}_i(\sum_{j=i}^L \mathbf{M}^{B}_{ij}\mathbf{k}_j\mathbf{v}^{\top}_j)
     - \frac{1}{2}{{\mathbf{q}_i}}^{\top} \mathbf{k}_i \mathbf{v}_i
\end{align}

So the addition $\mathbf{y}^{F}_i + \mathbf{y}^{B}_i $ is equal to:
\begin{align}
\label{eq:54}
    & \mathbf{y}^{F}_i + \mathbf{y}^{B}_i = \mathbf{q}^{\top}_i(\sum_{j=1}^i \mathbf{M}^{F}_{ij}\mathbf{k}_j\mathbf{v}^{\top}_j)+\mathbf{q}^{\top}_i(\sum_{j=i}^L \mathbf{M}^{B}_{ij}\mathbf{k}_j\mathbf{v}^{\top}_j) - {{\mathbf{q}_i}}^{\top} \mathbf{k}_i \mathbf{v}_i \\
    & \Rightarrow \mathbf{y}^{F}_i + \mathbf{y}^{B}_i =
     \mathbf{q}^{\top}_i(\sum_{j=1}^i \mathbf{M}^{F}_{ij}\mathbf{k}_j\mathbf{v}^{\top}_j + \sum_{j=i}^L \mathbf{M}^{B}_{ij}\mathbf{k}_j\mathbf{v}^{\top}_j)  - {{\mathbf{q}_i}}^{\top} \mathbf{k}_i \mathbf{v}_i
\end{align}

Where by considering the mask $\mathbf{M}$ as bellow:

\scalebox{0.9}{
\begin{minipage}[t]{1.1\textwidth}  
\begin{align}
\label{fullatt}
\mathbf{M}_{ij} = \begin{cases} 
    \hblue {\Pi_{k=j}^{i+1}{\lambda_k}} & i > j  \\
    \horange {\Pi_{k=i+1}^{j}{\lambda_k}}  & i < j \\
    \hdiag{1}  & i = j
\end{cases} \hspace{1mm}  =  \hspace{1mm}
   \left(  \renewcommand*{\arraystretch}{2} \begin{array}{ccccc}
    \dcolor{\mathbf{1}}  & \orange{{\lambda}_2} & \orange{{\lambda}_2 {\lambda}_3}  & \orange{\cdots} & \orange{{\lambda}_2\cdots{\lambda}_L} \\
    \blue{{\lambda}_1} &  \dcolor{\mathbf{1}} & \orange{{\lambda}_3} & \orange{\cdots} & \orange{{\lambda}_3 \cdots {\lambda}_L} \\
    \blue{{\lambda}_1 {\lambda}_2} & \blue{{\lambda}_2} & \dcolor{\mathbf{1}} & \orange{\cdots} & \orange{{\lambda}_4 \cdots {\lambda}_L} \\
    \blue\vdots & \blue\vdots & \blue\vdots & \dcolor{\ddots} & \orange \vdots \\
    \blue{{{\lambda}_{L-1}\cdots {\lambda}_1}} & \blue{{{\lambda}_{L-1}\cdots {\lambda}_2}} & \blue{{{\lambda}_{L-1}\cdots {\lambda}_3}} & \blue{\cdots} &   \dcolor{\mathbf{1}} \\   
\end{array}  \right)  
\end{align}
\end{minipage} }

The above mask is equal to \(\mathbf{M}^F + \mathbf{M}^B -\mathbf{I}\), allowing equation \eqref{eq:54} to be rewritten as:

\begin{align}
    \mathbf{y}^{F}_i + \mathbf{y}^{B}_i &=
     \mathbf{q}^{\top}_i(\sum_{j=1}^i \mathbf{M}^{F}_{ij}\mathbf{k}_j\mathbf{v}^{\top}_j + \sum_{j=i}^L \mathbf{M}^{B}_{ij}\mathbf{k}_j\mathbf{v}^{\top}_j)  - {{\mathbf{q}_i}}^{\top} \mathbf{k}_i \mathbf{v}_i \\ 
    & = \mathbf{q}^{\top}_i(\sum_{j=1}^L \mathbf{M}_{ij}\mathbf{k}_j\mathbf{v}^{\top}_j) +  {{\mathbf{q}_i}}^{\top} \mathbf{k}_i \mathbf{v}_i - {{\mathbf{q}_i}}^{\top} \mathbf{k}_i \mathbf{v}_i \\
    & = \mathbf{q}^{\top}_i(\sum_{j=1}^L \mathbf{M}_{ij}\mathbf{k}_j\mathbf{v}^{\top}_j) 
    \label{eq:wrapup}
\end{align}

So we can finally find the output of each layer $\mathbf{y}_i $ as:
\begin{align}
\label{eq:dumm2}
    \mathbf{y}_i = \frac{\mathbf{y}^{F}_i + \mathbf{y}^{B}_i}{c^F_i + c^B_i }
    \xrightarrow{\text{Equation \eqref{eq:wrapup}}} \mathbf{y}_i = \frac{\mathbf{q}^{\top}_i(\sum_{j=1}^L \mathbf{M}_{ij}\mathbf{k}_j\mathbf{v}^{\top}_j)}{c^F_i + c^B_i } 
\end{align}

It can easily be shown that:

\begin{align}
    c^F_i = \mathbf{q}^{\top}_i & (\sum^i_{j=1} \mathbf{k}_j) - \frac{1}{2}\mathbf{q}^{\top}_i\mathbf{k}_i \hspace{2mm} , \hspace{2mm}  c^B_i = \mathbf{q}^{\top}_i (\sum^L_{j=i} \mathbf{k}_j) - \frac{1}{2}\mathbf{q}^{\top}_i\mathbf{k}_i \\
    \Rightarrow c^F_i + c^B_i &= \mathbf{q}^{\top}_i (\sum^L_{j=1} \mathbf{k}_j) +  \mathbf{q}^{\top}_i\mathbf{k}_i - \frac{1}{2}\mathbf{q}^{\top}_i\mathbf{k}_i - \frac{1}{2}\mathbf{q}^{\top}_i\mathbf{k}_i \\
     \Rightarrow c^F_i + c^B_i & = \mathbf{q}^{\top}_i (\sum^L_{j=1} \mathbf{k}_j) +  \mathbf{q}^{\top}_i\mathbf{k}_i - \mathbf{q}^{\top}_i\mathbf{k}_i =  \mathbf{q}^{\top}_i (\sum^L_{j=1} \mathbf{k}_j)
     = \mathbf{q}^{\top}_i \mathbf{z}_L
\end{align}

So the final output of the layer is:

\begin{align}
\label{eq:finalyay}
    \mathbf{y}_i = \frac{\mathbf{y}^{F}_i + \mathbf{y}^{B}_i}{c^F_i + c^B_i } = \frac{\mathbf{q}^{\top}_i(\sum_{j=1}^L \mathbf{M}_{ij}\mathbf{k}_j\mathbf{v}^{\top}_j)}{\mathbf{q}^{\top}_i (\sum^L_{j=1} \mathbf{k}_j)}
\end{align}

Alternatively, in vectorized form, it can be expressed as:

\begin{equation}
    \mathbf{Y} = \mathbf{Y}^{F}+ \mathbf{Y}^{B} = \big(\textsc{scale}(\mathbf{Q}\mathbf{K}^{\top} )\odot \mathbf{M}\big) \mathbf{V}
\end{equation}

with $\mathbf{M}$ being the attention mask created by ${\lambda}_i$s as in equation \ref{fullatt}.


\subsection{Flipping Operation in Backward recurrence}
\label{ap:flip}
Here we define the operation which flip the matrices $\mathbf{A}^B , \mathbf{M}^B$ for the reverse reccurence th goal is to find the $F(.)$ such that:

\scalebox{0.8}{
 \begin{minipage}[t]{1.2\textwidth}  
\begin{align}
\label{backattflip2}
\mathbf{A}^B = 
\left( \renewcommand*{\arraystretch}{2} \begin{array}{cccc}
      \frac{1}{2}{\mathbf{q}_1^{\top}\mathbf{k}_1} & {\mathbf{q}_1^{\top}\mathbf{k}_2} & \cdots & {\mathbf{q}_1^{\top}\mathbf{k}_L} \\
       &  \frac{1}{2}{\mathbf{q}_2^{\top}\mathbf{k}_2}  & \cdots & {\mathbf{q}_2^{\top}\mathbf{k}_L} \\
      & & \ddots & \vdots \\
       &  &  &  \frac{1}{2}{\mathbf{q}_L^{\top}\mathbf{k}_L}  \\   
  \end{array} \right) \rightarrow F(\mathbf{A}^B) =
   \left( \renewcommand*{\arraystretch}{1} \begin{array}{cccc}
      \frac{1}{2}\frac{\mathbf{q}_L^{\top}\mathbf{k}_L}{\mathbf{q}_L^{\top}\mathbf{z}_L} &  &  & \\
      \frac{\mathbf{q}_{L-1}^{\top}\mathbf{k}_{L}}{\mathbf{q}_2^{\top}\mathbf{z}_L} & \frac{1}{2}\frac{\mathbf{q}_{L-1}^{\top}\mathbf{k}_{L-1}}{\mathbf{q}_2^{\top}\mathbf{z}_L}  &  &  \\
      \vdots & \vdots & \ddots & \\
      \frac{\mathbf{q}_1^{\top}\mathbf{k}_L}{\mathbf{q}_1^{\top}\mathbf{z}_L} & \frac{\mathbf{q}_1^{\top}\mathbf{k}_{L-1}}{\mathbf{q}_1^{\top}\mathbf{z}_L} & \cdots & \frac{1}{2}\frac{\mathbf{q}_1^{\top}\mathbf{k}_1}{\mathbf{q}_1^{\top}\mathbf{z}_L}  \\   
  \end{array} \right)  
\end{align} 

\end{minipage}
}

\scalebox{0.8}{
\begin{minipage}[t]{1.2\textwidth}  
\begin{align}
\label{eq:maskdec}
\mathbf{M}^B = \left( \renewcommand*{\arraystretch}{1.5} \begin{array}{ccccc}
    {\mathbf{1}}  & {{\lambda}_2} & {{\lambda}_2 {\lambda}_3}  & {\cdots} & {{\lambda}_2\cdots{\lambda}_L} \\
     &  {\mathbf{1}} & {{\lambda}_3} & {\cdots} & {{\lambda}_3 \cdots {\lambda}_L} \\
     &  & {\mathbf{1}} & {\cdots} & {{\lambda}_4 \cdots {\lambda}_L} \\
     &  &  & {\ddots} &  \vdots \\
     &  &  &  &   {\mathbf{1}} \\   
\end{array} \right) \rightarrow F(\mathbf{M}^B) =
    \left( \renewcommand*{\arraystretch}{1} \begin{array}{ccccc}
    {\mathbf{1}}  &  &  &  & \\
     {{\lambda}_L} &  {\mathbf{1}} &  &  &  \\
     {{\lambda}_L} {{\lambda}_{L-1}}& {{\lambda}_{L-1}} & {\mathbf{1}} &  &  \\
       \vdots &  \vdots &  \vdots & {\ddots} &  \\
     {{\lambda}_{L} \cdots {\lambda}_2}& {{\lambda}_L \cdots {\lambda}_3} & {{\lambda}_L \cdots {\lambda}_4} &  \cdots &   {\mathbf{1}} \\   
\end{array} \right)
\end{align}
\end{minipage} 
}

The above can be achieved by:

\begin{minipage}[t]{1\textwidth}  
\begin{align}
    F(\mathbf{A}) = \mathbf{J}_L\mathbf{A}\mathbf{J}_L, \hspace{2mm},  \mathbf{J}_L = \left( \begin{array}{cccc} 
 &&&1\\
 &&1&\\
 &\iddots&&\\
 1&&&\\ 
\end{array}  \right)
\end{align}
\end{minipage}

\subsection{Mapping Existing autoregressive models into \lion} \label{sec:map}

As noted, other autoregressive recurrent models can also be integrated into our bidirectional framework, benefiting from parallelization during training and fast bidirectional inference. Here, we demonstrate how to map several well-known Linear Transformers into the bidirectional form of \lion, along with their corresponding masked attention matrix and inference linear recurrence.

\textbf{Linear Transformer (\textbf{\lionlit}).} According to \cite{trans_rnn} the linear transformer has a recurrence:

\begin{align}
{\mathbf{S}_i^F} &=   {\mathbf{S}_{i-1}^F} + \mathbf{k}_i \mathbf{v}_i^{\top}, \\
     {\mathbf{z}_i^F} & =   {\mathbf{z}_{i-1}^F} + {\mathbf{k}_i}, \\
    \hspace{4mm} \textsc{S}&\textsc{caled}:  {\mathbf{y}_i^F}= \frac{{{\mathbf{q}_i}}^{\top}  {\mathbf{S}_i^F}}{{{\mathbf{q}_i}}^{\top}  {\mathbf{z}_i^F}} \\
    \hspace{4mm} \textsc{No}&\textsc{n-scaled}:  {\mathbf{y}_i^F}= {{{\mathbf{q}_i}}^{\top}  {\mathbf{S}_i^F}}
\end{align}

As observed, this is a special case of our bidirectional recurrence defined in \eqref{eq:bestrnn} with \(\lambda_i = 1\), as \textbf{\lion} resembles the scaled masked attention. In the case of the linear transformer, we require attention without scaling for the recurrence. The vectorized form for the scaled version can then be derived easily as follows:

\begin{minipage}[t]{0.45\textwidth}
\begin{whitebox}
    \begin{align}
        \mathbf{S}^{F/B}_i &= \mathbf{S}^{F/B}_{i-1} + \mathbf{k}_i \mathbf{v}_i^{\top}, \\
    \mathbf{z}^{F/B}_i & =  \mathbf{z}^{F/B}_{i-1} +  \mathbf{k}_i \\
    c^{F/B}_i & =  {{{\mathbf{q}_i}}^{\top} \mathbf{z}^{F/B}_{i}} - \frac{1}{2}{{\mathbf{q}_i}}^{\top} \mathbf{k}_i, \\
    \mathbf{y}^{F/B}_i &= {{{\mathbf{q}_i}}^{\top} \mathbf{S}^{F/B}_i}  - \frac{1}{2}{{\mathbf{q}_i}}^{\top} \mathbf{k}_i \mathbf{v}_i 
    \end{align}
\end{whitebox}
\end{minipage}
\begin{minipage}[t]{0.1\textwidth}
\vspace{-2.3cm}
    \begin{align}
       \textbf{=} \notag
    \end{align}
\end{minipage}
\begin{minipage}[t]{0.4\textwidth}
\vspace{-2.3cm}
\begin{whitebox}
    \begin{align}
        \mathbf{Y} &= \textsc{scale}(\mathbf{Q}\mathbf{K}^{\top}\mathbf{V})
    \end{align}
\end{whitebox}
\end{minipage}

For the non-scaled variant, we simply remove the scaling state \(\mathbf{z}\) as well as the scaling parameter \(c\). Consequently, the bidirectional linear transformer, which is equivalent to and parallelizable with attention without scaling, can be expressed as follows:

\begin{minipage}[t]{0.45\textwidth}
\begin{whitebox}
    \begin{align}
        \mathbf{S}^{F/B}_i &= \mathbf{S}^{F/B}_{i-1} + \mathbf{k}_i \mathbf{v}_i^{\top}, \\
    \mathbf{y}^{F/B}_i &= {{{\mathbf{q}_i}}^{\top} \mathbf{S}^{F/B}_i}  - \frac{1}{2}{{\mathbf{q}_i}}^{\top} \mathbf{k}_i \mathbf{v}_i 
    \end{align}
\end{whitebox}
\end{minipage}
\begin{minipage}[t]{0.1\textwidth}
\vspace{-1.5cm}
    \begin{align}
       \textbf{=} \notag
    \end{align}
\end{minipage}
\begin{minipage}[t]{0.4\textwidth}
\vspace{-1.5cm}
\begin{whitebox}
    \begin{align}
        \mathbf{Y} &= \mathbf{Q}\mathbf{K}^{\top}\mathbf{V}
    \end{align}
\end{whitebox}
\end{minipage}

The final output for scaled version can be extracted as $\mathbf{y}_i = \frac{\mathbf{y}^{B}_i + \mathbf{y}^{B}_i}{c^B_i + c^B_i }$ for scaled and as $\mathbf{y}_i = {\mathbf{y}^{B}_i + \mathbf{y}^{B}_i}$ for non-scaled version. Variations of linear transformers, such as Performer \citep{performer}, which employ different non-linearities \(\phi(.)\) for keys and queries, can be adapted to a bidirectional format using the framework established for linear transformers.


\textbf{Retentive Network} (\lionretnet)\textbf{.} According to \cite{retnet} the forward equation for a retentive network can be written as:

\begin{align}
\label{eq:recurscale}
     \mathbf{S}_i^F &=  \lambda \mathbf{S}_{i-1}^F + \mathbf{k}_i \mathbf{v}_i^{\top}, \\
    \mathbf{y}_i^F &= {{{\mathbf{q}_i}}^{\top}  \mathbf{S}_i^F}
\end{align}

This architecture can also be expanded to bi-directional setting simply by not scaling the attention in our framework and only using the mask with non input-dependent $\lambda_i=\lambda$ values:

\begin{minipage}[t]{0.45\textwidth}
\begin{whitebox}
    \begin{align}
        \mathbf{S}^{F/B}_i &= \lambda\mathbf{S}^{F/B}_{i-1} + \mathbf{k}_i \mathbf{v}_i^{\top}, \\
    \mathbf{y}^{F/B}_i &= {{{\mathbf{q}_i}}^{\top} \mathbf{S}^{F/B}_i}  - \frac{1}{2}{{\mathbf{q}_i}}^{\top} \mathbf{k}_i \mathbf{v}_i 
    \end{align}
\end{whitebox}
\end{minipage}
\begin{minipage}[t]{0.1\textwidth}
\vspace{-1.5cm}
    \begin{align}
       \textbf{=} \notag
    \end{align}
\end{minipage}
\begin{minipage}[t]{0.4\textwidth}
\vspace{-1.5cm}
\begin{whitebox}
    \begin{align}
        \mathbf{Y} &= (\mathbf{Q}\mathbf{K}^{\top} \odot \mathbf{M}^R)\mathbf{V}
    \end{align}
\end{whitebox}
\end{minipage}

Note that:  $\mathbf{M}^R_{ij} = \lambda^{|i-j|}$.

\textbf{xLSTM (\lion-\textsc{LSTM}).} According to \cite{xlstm} the recurrence for forward recurrence of xLSTM can be written as:

\begin{align}
     \mathbf{S}_i^F &=   f_i\mathbf{S}_{i-1}^F + i_i\mathbf{k}_i \mathbf{v}_i^{\top}, \\
     \mathbf{z}_i^F & =   f_i\mathbf{z}_{i-1}^F + i_i{\mathbf{k}_i}, \\
     \mathbf{y}_i^F &= \frac{{{\mathbf{q}_i}}^{\top}  \mathbf{S}_i^F}{{{\mathbf{q}_i}}^{\top}  \mathbf{z_i}^F} 
\end{align}

The above recurrence is equivalent to \eqref{eq:recrec} by considering \(i_i \mathbf{k}_i\) as a new key. The term \(i_i \mathbf{k}_i\) can be easily vectorized by aggregating all \(i_i\) values for each token into a vector \(\mathbf{i}\). Thus, we can express the vectorized form of the bidirectional xLSTM and its equivalence to attention as follows:

\begin{minipage}[t]{0.46\textwidth}
\begin{whitebox}
    \begin{align}
        \mathbf{S}^{F/B}_i &= f_i\mathbf{S}^{F/B}_{i-1} + i_i\mathbf{k}_i \mathbf{v}_i^{\top}, \\
    \mathbf{z}^{F/B}_i & =  f_i\mathbf{z}^{F/B}_{i-1} +  i_i\mathbf{k}_i \\
    c^{F/B}_i & =  {{{\mathbf{q}_i}}^{\top} \mathbf{z}^{F/B}_{i}} - \frac{1}{2}{{\mathbf{q}_i}}^{\top} \mathbf{k}_i, \\
    \mathbf{y}^{F/B}_i &= {{{\mathbf{q}_i}}^{\top} \mathbf{S}^{F/B}_i}  - \frac{1}{2}{{\mathbf{q}_i}}^{\top} \mathbf{k}_i \mathbf{v}_i \\
    \text{Output:} \hspace{2mm} \mathbf{y}_i & = \frac{\mathbf{y}^F_i+\mathbf{y}^B_i}{\max(c^F_i+c^B_i,1)}
    \end{align}
\end{whitebox}
\end{minipage}
\begin{minipage}[t]{0.05\textwidth}
\vspace{-2.4cm}
    \begin{align}
       \textbf{=} \notag
    \end{align}
\end{minipage}
\begin{minipage}[t]{0.48\textwidth}
\vspace{-2.42cm}
\begin{whitebox}
    \begin{align}
        \mathbf{Y} &= \textsc{scale'}(\mathbf{Q}(\mathbf{i}\odot\mathbf{K}^{\top}))\odot \mathbf{M}^f)\mathbf{V}
    \label{xlstmvec1}
    \end{align}
\end{whitebox}
\end{minipage}

where the mask $\mathbf{M}^f$ is equal to the \lion mask \eqref{fullatt} just by replacing $\lambda_i = f_i$. And where operation $\textsc{scale'}$ consider the maximum of operation in the denominator as:
\begin{equation}
    \textsc{scale'}(\mathbf{A})_{ij} = \frac{\mathbf{A}_{ij}}{\max(\sum_{j=1}^L \mathbf{A}_{ij},1)}
\end{equation}


\textbf{Gated RFA (\lion-\textsc{GRFA}).} Gated RFA \citep{yang2023gated} in autoregressive mode exhibits a recurrence similar to that of xLSTM, with only minor differences:

\begin{align}
     \mathbf{S}_i^F &=   g_i\mathbf{S}_{i-1}^F + (1-g_i)\mathbf{k}_i \mathbf{v}_i^{\top}, \\
     \mathbf{z}_i^F & =   g_i\mathbf{z}_{i-1}^F + (1-g_i){\mathbf{k}_i}, \\
     \mathbf{y}_i^F &= \frac{{{\mathbf{q}_i}}^{\top}  \mathbf{S}_i^F}{{{\mathbf{q}_i}}^{\top}  \mathbf{z_i}^F} 
\end{align}

Thus, the bidirectional version of the model retains a similar output, achieved by replacing the vector \(\mathbf{i}\) in \eqref{xlstmvec1} with \(1 - \mathbf{g}\), where \(\mathbf{g}\) represents the vectorized form of all scalar values \(g_i\).

\begin{minipage}[t]{0.475\textwidth}
\begin{whitebox}
    \begin{align}
        \mathbf{S}^{F/B}_i &= g_i\mathbf{S}^{F/B}_{i-1} + (1-g_i)\mathbf{k}_i \mathbf{v}_i^{\top}, \\
    \mathbf{z}^{F/B}_i & =  g_i\mathbf{z}^{F/B}_{i-1} +  (1-g_i)\mathbf{k}_i \\
    c^{F/B}_i & =  {{{\mathbf{q}_i}}^{\top} \mathbf{z}^{F/B}_{i}} - \frac{1}{2}{{\mathbf{q}_i}}^{\top} \mathbf{k}_i, \\
    \mathbf{y}^{F/B}_i &= {{{\mathbf{q}_i}}^{\top} \mathbf{S}^{F/B}_i}  - \frac{1}{2}{{\mathbf{q}_i}}^{\top} \mathbf{k}_i \mathbf{v}_i 
    \end{align}
\end{whitebox}
\end{minipage}
\begin{minipage}[t]{0.01\textwidth}
\vspace{-2.3cm}
    \begin{align}
       \textbf{=} \notag
    \end{align}
\end{minipage}
\scalebox{0.9}{
\begin{minipage}[t]{0.55\textwidth}
\vspace{-2.7cm}
\begin{whitebox}
    \begin{align}
        \mathbf{Y} &= \textsc{scale}(\mathbf{Q}((1-\mathbf{g})\odot\mathbf{K}^{\top})\odot \mathbf{M})\mathbf{V}
    \end{align}
      \label{xlstmvec2}
\end{whitebox}
\end{minipage}}


\subsection{Mask $\mathbf{M}^F \& \mathbf{M}^B$ are Semiseperable with rank-1} \label{ap:rankmask}

For the lower triangular part of the selective mask \(\mathbf{M}^F\), the upper triangular part can be filled such that it creates a full matrix with rank 1, which aligns with the definition of a semi-separable matrix with rank 1, as below:

\setlength{\arrayrulewidth}{2.5pt}
\arrayrulecolor{azure!70} 
\scalebox{0.45}{
\begin{minipage}[t]{0\textwidth}  
\begin{align*}
\textbf{\huge{$\M^F$ = \hspace{1mm}}}
\renewcommand*{\arraystretch}{2} 
\begin{array}{ccccc}
\blue{\mathbf{1}}  &  &  & & \\ 
\blue{{\lambda}_1} &  \blue{\mathbf{1}} &  &  &  \\
\blue{{\lambda}_1 {\lambda}_2} & \blue{{\lambda}_2} & \blue{\mathbf{1}} &  &  \\
\blue\vdots & \blue\vdots & \blue\vdots & \blue{\ddots} &  \\
\blue{{{\lambda}_{L-1}\cdots {\lambda}_1}} & \blue{{{\lambda}_{L-1}\cdots {\lambda}_2}} & \blue{{{\lambda}_{L-1}\cdots {\lambda}_3}} & \blue{\cdots} &   \blue{\mathbf{1}} \\   
\end{array} 
\textbf{\huge{= \textsc{Tril}}} \left(
\renewcommand*{\arraystretch}{2} 
\begin{array}{|c|cccc}
\cline{1-1}\multicolumn{5}{c}{}\\[-5ex]
\blue{\mathbf{1}}  & \orange{{\lambda}^{-1}_1} & \orange{{\lambda}^{-1}_1 {\lambda}^{-1}_2}  & \orange{\cdots} & \orange{{\lambda}^{-1}_1\cdots{\lambda}^{-1}_{L-1}} \\ 
\blue{{\lambda}_1} &  \blue{\mathbf{1}} & \orange{{\lambda}^{-1}_2} & \orange{\cdots} & \orange{{\lambda}^{-1}_2 \cdots {\lambda}^{-1}_{L-1}} \\
\blue{{\lambda}_1 {\lambda}_2} & \blue{{\lambda}_2} & \blue{\mathbf{1}} & \orange{\cdots} & \orange{{\lambda}^{-1}_3 \cdots {\lambda}^{-1}_{L-1}} \\
\blue\vdots & \blue\vdots & \blue\vdots & \blue{\ddots} & \orange \vdots \\
\blue{{{\lambda}_{L-1}\cdots {\lambda}_1}} & \blue{{{\lambda}_{L-1}\cdots {\lambda}_2}} & \blue{{{\lambda}_{L-1}\cdots {\lambda}_3}} & \blue{\cdots} &   \blue{\mathbf{1}} \\   \cline{1-1}
\end{array} 
\right)
\textbf{ \hspace{1mm}  \huge{= \textsc{Tril}} \hspace{1mm}} \left(
\underbrace{
\begin{array}{|c|} 
\cline{1-1}\multicolumn{1}{c}{}\\[-5ex]
\blue{\mathbf{1}}  \\ 
\blue{{\lambda}_1} \\
\blue{{\lambda}_1 {\lambda}_2}  \\
\blue\vdots  \\
\blue{{{\lambda}_{L-1}\cdots {\lambda}_1}} \\  \cline{1-1}
\end{array} }_{\hspace{1mm}\scalemath{1.5}{ \textcolor{azure!100}{\lvec} }}
\underbrace{
\arrayrulecolor{red!60} \begin{array}{|ccccc|} 
\cline{1-5}\multicolumn{5}{c}{}\\[-5ex]
 \orange{\mathbf{1}}  & \orange{{\lambda}^{-1}_1} & \orange{{\lambda}^{-1}_1 {\lambda}^{-1}_2}  & \orange{\cdots} & \orange{{\lambda}^{-1}_1\cdots{\lambda}^{-1}_{L-1}} \\ \cline{1-5}
\end{array}  \arrayrulecolor{azure!70} }_{\hspace{1mm}\scalemath{1.5}{\textcolor{red!60}{\uu^\top = \einv (\lvec)^\top }}} \right) 
\textbf{\huge{$ \hspace{1mm} = \textsc{Tril} ( \lvec \uu^\top )$  }}
\end{align*}
\end{minipage}
}

\setlength{\arrayrulewidth}{0.7pt}
\arrayrulecolor{black} 

where $\textsc{Tril}$ is the function which masks the upper part of the matrix and set it to zero. Same is applied for the upper triangular part of the matrix $\mathbf{M}^B$ as well. Also since decay mask is the specific case of selective mask by setting $\lambda_i=\lambda$ all the proofs above also holds for the fixed decay mask used in RetNet. 

\subsection{Expanding the Dimension of $a_i$} \label{sec:expandai}

Similar to other recurrent models, particularly SSM variations, the dimension of \(a_i\) can be increased beyond a scalar. When \(a_i\) is a scalar, the same mask \(\mathbf{M}\) is applied to all elements of the value vector \(\mathbf{v}\). However, if we allow \(a_i\) to be a vector \(\mathbf{a}_i \in \mathbb{R}^d\), the mask matrix transforms into a tensor \(\bar{\mathbf{M}} \in \mathbb{R}^{L \times L \times d}\). This tensor can be computed in parallel for each individual value element along the last dimension. The last dimension will then be multiplied using the Hadamard product with the values, resulting in the following vectorized form:

\begin{align}
   \mathbf{Y} = \textsc{scale}(\mathbf{Q}\mathbf{K}^{\top} \odot \bar{\mathbf{M}}) * \mathbf{V}
\end{align}

In this equation, the operation \( * \) denotes the Hadamard product applied along the last dimension of the tensor mask \(\bar{\mathbf{M}}\) with the value vector \(\mathbf{V}\), while the first two dimensions are combined using a standard matrix product. The corresponding code is as follows:

\begin{figure}[t]
\hspace{5mm}\begin{minipage}[t]{0.95\columnwidth} 
    \centering
    \begin{python}[framerule=0.3
    mm , rulecolor=\color{black} ,frame=single]
attn = (Q @ K.transpose(-2, -1))
attn = torch.einsum("nhkmd,nhkm->nhkmd", M, attn)
attn = scale(attn)
x = torch.einsum("nhkmd,nhmd->nhkd", attn, V)
\end{python}
\end{minipage}
\end{figure}


\subsection{Generation of the Mask}
\label{subsec:code}
Below we present the Python code used for the creation of the bidirectional mask $\mathbf{M}$ as described in previous sections. 

\begin{figure}[t]
\hspace{5mm}\begin{minipage}[t]{0.95\columnwidth} 
    \centering
    \begin{python}[framerule=0.3
    mm , rulecolor=\color{black} ,frame=single]
#caption=Code for generation of the selective bidirectional mask of \lion , 
def create_matrix_from_tensor(tensor):
    cumpsum = torch.exp(torch.cumsum(tensor, dim=-1 ))
    A = torch.matmul(cump.unsqueeze(-1) , 1/ ( cump.unsqueeze(-1).transpose(-1,-2)))
    return torch.tril(A)

def Mask_selective(vec):
    vec_shape = vec.shape
    A_for = create_matrix_from_tensor(vec)
    A_back = create_matrix_from_tensor(flip(vec))
    return A_for + A_back - torch.eye(A_for.shape[-1])

def Mask_Decay(a_i , L):
    idx = torch.arange(L,device=a_i.device)
    I, J = torch.meshgrid(idx, idx, indexing='ij')
    E = (torch.abs((I-J)).float().view(1,1,L,L))
    M = torch.sigmoid(a_i).view(1,-1,1,1)**E
    return M

\end{python}
\end{minipage}
\end{figure}

\begin{figure}[t]
\hspace{5mm}\begin{minipage}[t]{0.95\columnwidth} 
    \centering
    \begin{python}[framerule=0.3
    mm , rulecolor=\color{black} ,frame=single]
#caption=Code for generation of the partial selective bidirectional mask of \lion for chunking, 
def Partial_Mask_selective(vec):
    B,H,L = vec.shape
    A_for = create_matrix_from_tensor_forward(vec[...,:-1]),chunk_index,chunk_len)
    A_back = create_matrix_from_tensor_backward(vec[...,1:]),chunk_index,chunk_len)
    I  = torch.diag_embed(torch.ones((B,H,L-chunk_index*chunk_len)),offset = -chunk_index*chunk_len)[...,:L]
    return A_for + A_back - I.to(A_for.device)

\end{python}
\end{minipage}
\end{figure}

\subsection{Details for \lion  of Chunkwise Parallel} \label{detailchunk}

As the full linear attention is written as:
\begin{align}
\label{now}
    \mathbf{Y} = \textsc{Scale}(\mathbf{Q}\mathbf{K}^\top \odot \mathbf{M})\mathbf{V}
\end{align}

by apply chunking to queries/keys/values and defining the $\mathbf{Q}_{[i]},\mathbf{K}_{[i]},\mathbf{V}_{[i]} = \mathbf{Q}_{iC+1:i(C+1)},\mathbf{K}_{iC+1:i(C+1)},\mathbf{V}_{iC+1:i(C+1)} \in \R^{C\times d}$ we can simply rewrite the \cref{now} in chunkwise form as:

\begin{align}
\label{nowresss}
     \mathbf{A}_{[ij]} & = \mathbf{Q}_{[i]}\mathbf{K}_{[j]}^\top \odot \mathbf{M}_{[ij]}, \quad \mathbf{C}_{[ij]} = \mathbf{C}_{[ij-1]} + \text{Sum} (\mathbf{A}_{[ij]}), \\
     \mathbf{S}_{[ij]} & =\mathbf{S}_{[ij-1]} + \mathbf{A}_{[ij]} \mathbf{V}_{[j]} , \quad \mathbf{Y}_{[i]} = \frac{\mathbf{S}_{[iN]}}{\mathbf{C}_{[iN]}}
\end{align}

where $N$ is the number of total chunks and $N=\frac{L}{C}$ and $\text{Sum}$ is the summation over the rows of the matrix. Since the full attention matrix needs to be scaled according to the full row of attention we need to update the scaling value for each chunk as stored in $\mathbf{C}_{ij}$ and the final output for chunk $i$ is computed by using the last chunkwise hidden state $\mathbf{S}_{[i]}$ divided by the scaling for that chunk $\mathbf{C}_{[i]}$ which are equal to ${\mathbf{S}_{[iN]}},{\mathbf{C}_{[iN]}}$.

To construct the chunkwise mask \(\mathbf{M}_{ij}\), we define the chunk-level selective parameters as:  

\[
\mathbf{L}^F_{[i]} = \cumprod(\mathbf{\lambda^F})_{iC+1:(i+1)C}, \quad \mathbf{L}^B_{[i]} = \cumprod(\mathbf{\lambda^B})_{iC+1:(i+1)C}.
\]

Since the full mask is composed of lower and upper triangular components:

\[
\mathbf{M}^F = \text{TRIL}(\mathbf{L}^F \frac{1}{\mathbf{L}^F}), \quad \mathbf{M}^B = \text{TRIU}(\mathbf{L}^B \frac{1}{\mathbf{L}^B}),
\]

we determine the appropriate chunkwise form based on relative chunk positions:  

\begin{itemize}
   \item If \(i > j\), the chunk falls entirely within the lower triangular part, requiring only \(\mathbf{M}^F\), which can be efficiently computed as \(\mathbf{L}^F_{[i]} \frac{1}{\mathbf{L}^F_{[j]}}\).  
\item If \(i < j\), the chunk is fully in the upper triangular region, needing only \(\mathbf{M}^B\), which follows from \(\mathbf{L}^B_{[j]} \frac{1}{\mathbf{L}^B_{[i]}}\).  
\item If \(i = j\), the chunk lies along the diagonal and requires both the lower triangular part of \(\mathbf{M}^F\) and the upper triangular part of \(\mathbf{M}^B\), expressed as:

\end{itemize}
    
The full matrix is: 

\[
\mathbf{M}_{[ij]} = 
\begin{cases} 
\mathbf{L}^F_{[i]} \frac{1}{\mathbf{L}^F_{[j]}}^\top & \text{if } i>j,  \\
\mathbf{L}^B_{[j]} \frac{1}{\mathbf{L}^B_{[i]}}^\top & \text{if } i<j,  \\
\text{Tril}\left(\mathbf{L}^F_{[i]} \frac{1}{\mathbf{L}^F_{[i]}}^\top\right) + \text{Triu}\left(\mathbf{L}^B_{[i]} \frac{1}{\mathbf{L}^B_{[i]}}^\top\right) - \mathbf{I} & \text{if } i = j. 
\end{cases} 
\]

More visual presentation is shown at Figure \cref{maskchunk}.

\begin{figure}[tb]
    \centering
    \includegraphics[width=0.3\textwidth]{figs/exexex.png}
  \caption{Chunkwise Parallel Mask $\mathbf{M}$ of \lion.}
    \label{maskchunk}
\end{figure}

For the fixed decay mask, a simpler case of the general selective mask, \(\mathbf{L}^F\) and \(\mathbf{L}^B\) are identical and simplify to \(\mathbf{L}_i = \lambda^i\). Since the full mask follows \(\mathbf{M}_{ij} = \lambda^{|i-j|}\), the chunkwise mask for \(i, j\) can be written as:

\[
\mathbf{M}_{[ij]} = \mathbf{L}_{[i]} \frac{1}{\mathbf{L}_{[j]}} = \lambda^{|i-j|} \mathbf{L}_{[0]} \frac{1}{\mathbf{L}_{[0]}}.
\]

Similarly, for the upper triangular part:

\[
\mathbf{M}_{[ij]} = \lambda^{|i-j|} \frac{1}{\mathbf{L}^\top_{[0]}} \mathbf{L}_{[0]}.
\]

For diagonal chunks, the mask remains a fixed matrix \(\mathbf{\Gamma} \in \mathbb{R}^{C \times C}\), where \(\mathbf{\Gamma}_{ij} = \lambda^{|i-j|}\), representing a smaller version of the full fixed decay mask \(\mathbf{M} \in \mathbb{R}^{L \times L}\) with \(\mathbf{M}_{ij} = \lambda^{|i-j|}\).


\subsection{Parallel Chunkwise Materialization for Parallel Output Computation} \label{parpar}

As mentioned in \cref{sec:chunk}, \cref{nowresss} can be further parallelized by processing all \(i\) tokens simultaneously, leading to the following matrix multiplications in the parallel chunkwise form:

\begin{align}
     \mathbf{A}_{[i]} & = \mathbf{Q}\mathbf{K}_{[j]}^\top \odot \mathbf{M}_{[j]}, \quad \mathbf{C}_{[j]} = \mathbf{C}_{[j-1]} + \text{Sum} (\mathbf{A}_{[j]}), \\
     \mathbf{S}_{[j]} & =\mathbf{S}_{[j-1]} + \mathbf{A}_{[ij]} \mathbf{V}_{[j]} , \quad \mathbf{Y} = \frac{\mathbf{S}_{[N]}}{\mathbf{C}_{[N]}}
\end{align}

and the mask $\mathbf{M}_{[j]}$ is created based on:

\begin{align}
\mathbf{M}_{[j]} = \text{Tril}(\mathbf{L}^F \frac{1}{{\mathbf{L}^F}^\top_{[j]}} , \textbf{diagonal}=-j) + \text{Tril}(\mathbf{L}^B \frac{1}{{\mathbf{L}^B}^\top_{[j]}} , \textbf{diagonal}=j)
\end{align}

Consider a real matrix \(X \in \mathbb{R}^{n \times n}\). The operator \(\mathrm{Tril}(\mathbf{X}, d)\) returns the lower-triangular region of \(X\), including all elements on and below the diagonal shifted by \(d\). In other words, \(\mathrm{tril}(\mathbf{X}, d)_{ij} = \mathbf{X}_{ij}\) whenever \(j \leq i + d\) and is \(0\) otherwise. Similarly, \(\mathrm{Triu}(\mathbf{X}, d)\) returns the upper-triangular region, keeping elements on and above the diagonal shifted by \(d\). Formally, \(\mathrm{Triu}(\mathbf{X}, d)_{ij} = \mathbf{X}_{ij}\) if \(j \geq i + d\) and \(0\) otherwise.





\subsection{\rebuttal{Changing the order of patches}}
\label{subsec:rotation}
\rebuttal{When processing images, both the spatial relationships among neighboring pixels and their positions are as critical as the pixel values themselves. Positional embeddings provide a way to incorporate these spatial relationships. A common approach in Transformers involves flattening the image, as illustrated in the left panel of \cref{fig:rot}. However, we argue that this method of flattening is suboptimal and can be enhanced to include additional contextual information.
\\
\\
Furthermore, in scenarios involving a fully masked setup or RNN-based inference, the sequence in which pixels are processed becomes increasingly important. To address this, we propose a new reordering scheme for pixel values. In the attention module, the pixel values are reordered following the patterns depicted in the center and right panels of \cref{fig:rot}. Forward and backward passes are then executed based on this new ordering, adhering to established procedures. The outputs from these two passes are subsequently averaged to generate the final result.
\\
\\
We refer to this method as \lionrot throughout the paper. This approach demonstrated a notable improvement in accuracy for image classification tasks while maintaining the efficiency and flexibility inherent to the method. A similar concept has been previously explored in Vision-LSTM \citep{alkin2024visionlstm}.}

\begin{figure}[tb]
    \centering
    \includegraphics[width=1\textwidth]{figs/rotational.png}
  \caption{\rebuttal{\textit{Reordering of patches.} Left is the naive approach to flatten images, also used in \lions. Center and right figures are the new approaches applied in \lionrot to consider further spatial information. }}
    \label{fig:rot}
\end{figure}

\section{Additional experimental validation}
\label{sec:app_experiments}

\subsection{Citations for LRA Benchmarks}
\label{sec:lra_bench}

The LRA baselines included in Table \ref{tab:lra_exp} correspond to Transformer \citep{vaswani_attention_2017}, MEGA and MEGA-chunk \citep{ma2022mega}, DSS \citep{gupta2022diagonalstatespaceseffective}, S4 \citep{gu2021efficiently}, S5 \citep{s5}, Mamba \citep{mamba}, Local Att. \citep{vaswani_attention_2017}, Sparse Transformer \citep{sparsetransformer}, Longformer \citep{Beltagy2020Longformer}, Linformer \citep{wang2020linformer}, Reformer \citep{kitaev2020reformerefficienttransformer}, Sinkhorn Transformer \citep{tay2020sinkhorn}, BigBird \citep{zaheer2020bigbirdtransformerslonger}, Linear Transformer \citep{trans_rnn}, Performer \citep{performer}, FNet \citep{leethorp2022fnetmixingtokensfourier}, Nyströmformer \citep{xiong2021nystromformernystrombasedalgorithmapproximating}, Luna-256 \citep{ma2021luna} and H-Transformer-1D \citep{zhu2021htransformer1dfastonedimensionalhierarchical}.

\subsection{LRA Configurations for \lion}
\label{lraconfig}
For the LRA task, we utilized the same model dimensions as specified in the S5 \citep{s5} paper, following the guidelines from the S5 GitHub repository\footnote{\url{https://github.com/lindermanlab/S5}}. Our state matrix was represented as a vector \(\boldsymbol{\Lambda}_i = \boldsymbol{\lambda}_i\), where each element contains a scalar non-input dependent value \(e^a\). The value \(a\) was initialized based on \textit{HIPPO} theory, alongside the input-dependent \(a_i\), as described in main body.

We employed the ADAMW optimizer with an initial learning rate of \(5 \times 10^{-4}\) and a cosine learning rate scheduler \citep{cossche}. The weights for the queries and keys, as well as the selective component of \(\Lambda\), were initialized using a Gaussian distribution with a standard deviation of 0.1. For the values \(\mathbf{v}\), we initialized \(W_\mathbf{v}\) using zero-order hold discretization, represented as \(W^{\text{init}}_\mathbf{v} = \left({\Lambda}^{-1} \cdot (\Lambda - \mathbf{I})\right)\). The non-selective parts of \(\Lambda\) were initialized based on the \textit{HIPPO} \citep{s5} matrix.


\subsection{Ablation Studies on LRA dataset} \label{laasjdhakjsdh}
We have did an ablation study for choosing the activation functions and using non-scalar decay factor for \lion.

\begin{table}[h] 
    \centering
        \caption{\textit{Effects of different parameter choices and non-linearities in \lions on LRA tasks.} Codes: $[1]$ Sigmoid non-linearity was applied to the $\mathbf{k}$ and $\mathbf{q}$ values with unscaled masked attention; $[2]$ ReLU non-linearity was utilized, and the masked attention was scaled; $[3]$ The parameter $a_i$ was selected as a scalar instead of a vector; $[4]$ \lions model parameters were used without scaling; $[5]$ The attention matrix of \lions was scaled, but attention values were adjusted without the factor of $\lambda_i$; $[6]$ The selective component of $a_i$ was removed; $[7]$ SoftPlus activation function was employed for the $a_i$ values. We used the HIPPO \cite{hippo} initialisation for LRA task since random initalisation of \lions and \lionretnet can not solve LRA.}
    \resizebox{1\textwidth}{!}{
\begin{tabular}{l|lcccccc}
\toprule
 Model & ListOps & Text & Retrieval	& Image	& Pathfinder & PathX & Avg. \\
 (input length) & 2048 & 2048 & 4000 & 1024 & 1024  & 16K & \\
 \bottomrule
$[1]$ $\phi(x) = \sigma(x)$ w.o scaling   & 61.02 & {88.02} & {89.10} & {86.2} & {91.06} & 97.1 & {85.41} \\
$[2]$ $\phi(x) = \textsc{Relu}(x)$ w. scaling   &  36.37 & 65.24 & 58.88 & 42.21 & 69.40 & \xmark &  54.42  \\
$[3]$ $a_i$ only scalar & 36.23 &  60.33 & 60.45 & 58.89 & 70.00 &\xmark & 57.17 \\
$[4]$ \lion w.o scaling & 58.76 & 67.22 & 59.90 & 60.0 & 65.51 & \xmark & 62.27 \\
$[5]$ scaled attention w.o mask  & 60.12 & 87.67 & 87.42 &88.01&89.23 & \xmark& 82.49  \\
$[6]$ $a_i$ From \textit{HIPPO} w.o selectivity & 60.12 & 88.00 & 89.22 & 83.21 &  91.0 & 96.30  & 84.64\\
$[7]$ $a_i=\textsc{SoftPlus}(x)$ &  16.23 & 59.90 & 60.00& 45.12 & 70.07 & \xmark &   50.26 \\
 \bottomrule
\rowcolor{orange!17}
 \textbf{\lions (w/ \textit{HIPPO})}  & \textbf{62.25} & \textbf{88.10} & \textbf{90.35} & \textbf{86.14} & \textbf{91.30} & \textbf{97.99} & \textbf{86.07} \\
         \bottomrule
\end{tabular} }
    \label{tab:lra}
\end{table} 

We have observed that bounding the keys and queries significantly enhances the model's ability to solve tasks. This finding is consistent with the observations in \cite{deltanet}. As demonstrated in variation \([1]\), it can successfully tackle the LRA task even without scaling, while the \textsc{ReLU} activation fails to do so. Additionally, we found that scaling plays a crucial role, particularly when it comes to scaling the masked attention. The approach used in \lion, which scales the attention before applying the mask expressed as \(\mathbf{Y} = \textsc{scale}(\mathbf{Q}\mathbf{K}^{\top}) \odot \mathbf{M}\) has proven ineffective in addressing the challenging PathX task, as shown in \([5]\). Furthermore, the modifications implemented in \lions have demonstrated superior performance compared to all other variations tested.


\subsection{LRA full Results}

We have evaluated \lion variants against benchmarks for long-range sequence modeling across different categories, including softmax-based Transformers, RNNs, SSMs, and Linear Transformers.

  \begin{table*}[h] 
    \centering
        \caption{\textit{Performance on Long Range Arena Tasks.} 
        For each column (dataset), the best and the second best results are highlighted with \textbf{bold} and \underline{underline} respectively. Note that the MEGA architecture has roughly 10$\times$ the number of parameters as  the other architectures.}
        \vspace{-2mm}
    \resizebox{1\textwidth}{!}{
\begin{tabular}{l|lcccccc|cc}
\toprule
Category & Model & ListOps & Text & Retrieval	& Image	& Pathfinder & PathX & Avg. \\
& (input length) & 2048 & \rebuttal{4096} & 4000 & 1024 & 1024  & 16K & \\
 \bottomrule
\multirow{3}{*}{{Transformer}} & Transformer    & 36.37 & 64.27 & 57.46 & 42.44 & 71.40& \xmark &  54.39  \\
& MEGA ($\cO(L^2)$)& \textbf{63.14} & \textbf{90.43} & \underline{91.25} & \textbf{90.44} & \underline{96.01} & 97.98 & \textbf{88.21} \\
& MEGA-chunk ($\cO(L)$) & 58.76 & \underline{90.19} & 90.97 & 85.80 & 94.41 & 93.81 & 85.66 \\
\bottomrule
\multirow{4}{*}{{SSM}} & DSS  & 57.60 & 76.60 & 87.60 & 85.80 & 84.10 & 85.00 & 79.45  \\
& S4 (original)   & 58.35 & {86.82} & {89.46} & 88.19 &  {93.06} & 96.30  & {85.36} \\
& \rebuttal{S5 } & \rebuttal{62.15} & \rebuttal{89.31} & \rebuttal{\textbf{91.40}} & \rebuttal{88.00} & \rebuttal{95.33} & \rebuttal{\textbf{98.58}} & \rebuttal{\underline{87.46}} \\
& \rebuttal{Mamba (From \cite{xlstm})} & \rebuttal{32.5} & \rebuttal{N/A} & \rebuttal{90.2} &\rebuttal{68.9} &\rebuttal{\textbf{99.2}}&  \rebuttal{N/A} & \rebuttal{N/A} \\
 \bottomrule
 \rebuttal{RNN} & \rebuttal{LRU} &\rebuttal{ 60.2}& \rebuttal{89.4}& \rebuttal{89.9} & \rebuttal{\underline{89.0}}& \rebuttal{95.1}& \rebuttal{94.2} & \rebuttal{86.3}\\
 & \rebuttal{xLSTM} (From \cite{xlstm}) & \rebuttal{41.1} & \rebuttal{N/A} & \rebuttal{90.6} &\rebuttal{69.5} &\rebuttal{91.9}&  \rebuttal{N/A} & \rebuttal{N/A} \\
  \bottomrule
\multirow{13}{*}{\begin{tabular}{l} Linear\\ Transformer \end{tabular}} & Local Att.   & 15.82 &	52.98 &	53.39 &	41.46 &	66.63& \xmark &	46.06 \\
& Sparse Transformer   & 17.07	& 63.58 &	59.59 &	44.24 &	71.71& \xmark		& 51.24 \\
& Longformer     & 35.63	& 62.85	& 56.89	& 42.22	& 69.71& \xmark	&  53.46 \\
& Linformer    & 16.13	& 65.90	& 53.09	& 42.34	& 75.30& \xmark	&  	50.55 \\
& Reformer   & 37.27 & 56.10 & 53.40 & 38.07 & 68.50 & \xmark &  50.67 \\
& Sinkhorn Trans.    & 33.67 & 61.20 & 53.83 & 41.23 & 67.45& \xmark & 51.48 \\
& BigBird   & 36.05	& 64.02	& 59.29	& 40.83	& 74.87& \xmark	& 	55.01 \\
& Linear Trans.   & 16.13 & 65.90 & 53.09 & 42.34 & 75.30& \xmark &  50.55 \\
& Performer     & 18.01 & 65.40 & 53.82 & 42.77 & 77.05 & \xmark & 51.41 \\
& FNet    & 35.33 & 65.11 & 59.61 & 38.67 & 77.80 & \xmark &  55.30 \\
& Nyströmformer    & 37.15 & 65.52 & 79.56 & 41.58 & 70.94& \xmark &  58.95 \\
& Luna-256  & 37.25 & 64.57 & 79.29 & 47.38 & 77.72& \xmark &  61.24 \\
& H-Transformer-1D   & 49.53 & 78.69 & 63.99 & 46.05 & 68.78& \xmark & 61.41 \\
\rowcolor{Green!10} 
&\rebuttal{\lionlit}&\rebuttal{16.78}&\rebuttal{65.21}&\rebuttal{54.00}&\rebuttal{43.29}&\rebuttal{72.78} & \rebuttal{\xmark }&\rebuttal{50.41}\\
\rowcolor{violet!20}
& \lionretnet (w/ \textit{HIPPO})  & 62.0 & 88.78& 90.12& 85.66 & 90.25& 97.28& 85.63\\
\rowcolor{orange!17}
& \lions (w/ \textit{HIPPO}) & \underline{62.25} & {88.10} & {90.35} & {86.14} & {91.30} & \underline{97.99} & 86.07 \\
         \bottomrule
    \end{tabular} }
    \label{tab:lra_exp}
\end{table*}

\subsection{\rebuttal{Experimental details for the MLM/GLUE tasks}}
\label{subsec:details_glue}
\rebuttal{
\textbf{Architectures} We train the BASE (110M parameters) and LARGE (334M parameters) model families from the original BERT paper \citep{devlin2019bert}. For the \lion models, we replace the standard self-attention blocks with \lionlit/\lionretnet/\lions blocks while keeping all hyperparameters the same. For \lionlit, we incorporate LayerNorm \citep{ba2016layernorm} after the attention block to enhance stability. For Hydra, we take the default hyperparameters in \citep{hwang2024hydrabidirectionalstatespace} and increase the number of layers to 45 to match the number of parameters in the LARGE scale. Our implementation is based on the M2 repository \citep{fu2023monarch}, i.e., \url{https://github.com/HazyResearch/m2}.
\\
\\
\textbf{Pretraining} All our pretraining hyperparameters follow \citet{fu2023monarch}: We employ the C4 dataset \citep{dodge2021c4}, a maximum sequence length during pretraining of $128$ and a masking probability of $0.3$ and $0.15$ for the training and validation sets respectively. We train our model for $70,000$ steps with a batch size of $4096$. We employ the decoupled AdamW optimizer with a learning rate of $8\cdot 10^{-4}$, $\beta_1=0.9$, $\beta_2 = 0.98$, $\epsilon = 10^{-6}$ and weight decay $10^{-5}$. As a scheduler, we perform a linear warm-up for $6\%$ of the training steps and a linear decay for the rest of training until reaching $20\%$ of the maximum learning rate.
\\
\\
Our only change in the pretraining hyperparameters is setting the learning rate to $2\cdot10^{-4}$ for the LARGE model family. In our preliminary experiments, we found that training diverged when using a learning rate of $8\cdot10^{-4}$ for BERT-LARGE.
\\
\\
For completeness, we present the results with the BERT pretraining\footnote{\url{https://github.com/HazyResearch/m2/blob/main/bert/yamls/pretrain/hf-transformer-pretrain-bert-base-uncased.yaml}} and BERT 24 finetuning\footnote{\url{https://github.com/HazyResearch/m2/blob/main/bert/yamls/finetune-glue/hf-transformer-finetune-glue-bert-base-uncased.yaml}} recipes available in the M2 repository.
\\
\\
\textbf{Finetuning} For the GLUE finetuning experiments, we employ five different configurations:
\begin{itemize}
    \item \textbf{BERT24}: Available in \citet{izsak2021train} and the file  \url{https://github.com/HazyResearch/m2/blob/main/bert/yamls/finetune-glue/hf-transformer-finetune-glue-bert-base-uncased.yaml}.
    \item \textbf{M2-BASE}: Available in \citet{fu2023monarch}, Section C.1 and the file \url{https://github.com/HazyResearch/m2/blob/main/bert/yamls/finetune-glue/monarch-mixer-finetune-glue-960dim-parameter-matched.yaml}.
    \item \textbf{M2-LARGE}: Available in \citet{fu2023monarch}, Section C.1 and the file \url{https://github.com/HazyResearch/m2/blob/main/bert/yamls/finetune-glue/monarch-mixer-large-finetune-glue-1792dim-341m-parameters.yaml}.
    \item \textbf{Hydra}: Available in \citet{hwang2024hydrabidirectionalstatespace}, Section D.2 and the file \url{https://github.com/goombalab/hydra/blob/main/hydra/bert/yamls/finetune/hydra.yaml}.
    \item \textbf{Modified}: Same as M2-LARGE but all learning rates are set to $10^{-5}$.
\end{itemize}
The recipes are summarized in \cref{table:glue_hyperparams}. The Modified hyperparameter set was devised as M2-LARGE was found to diverge for BERT-LARGE.
\begin{table}[]
    \rebuttal{
    \centering
    \caption{\rebuttal{GLUE finetuning recipes employed in this work. All recipes finetune on RTE, STSB and MRPC from the weights finetuned in MNLI and the rest from the C4-pretrained weights. All recipes use a sequence length of 128 tokens except BERT24 and Hydra, that use $256$. D. AdamW stands for decoupled AdamW.}}
    \resizebox{\textwidth}{!}{
    \begin{tabular}{lc|cccccccc}
        \toprule
        \multirow{2}{*}{Recipe} & \multirow{2}{*}{Param.} & \multicolumn{8}{c}{Dataset}\\
         & & MNLI & QNLI & QQP & RTE & SST2 & MRPC & COLA & STSB \\
         \midrule
         \multirow{3}{*}{\begin{minipage}{2.6cm}
             BERT24\\ 
             \citep{izsak2021train}
         \end{minipage}}& LR & $5\cdot10^{-5}$ & $1\cdot10^{-5}$ & $3\cdot10^{-5}$ & $1\cdot10^{-5}$ & $3\cdot10^{-5}$ & $8\cdot10^{-5}$ & $5\cdot10^{-5}$ & $3\cdot10^{-5}$\\
         & WD & $5\cdot10^{-6}$ & $1\cdot10^{-5}$ & $3\cdot10^{-6}$ & $1\cdot10^{-6}$ & $3\cdot10^{-6}$ & $8\cdot10^{-5}$ & $5\cdot10^{-6}$ & $3\cdot10^{-6}$  \\
         & Epochs & 3 & 10 & 5 & 3 & 3 & 10 & 10 & 10 \\
         & Optimizer & D. AdamW & D. AdamW & D. AdamW & D. AdamW & D. AdamW & D. AdamW & D. AdamW & D. AdamW \\
         \midrule
         \multirow{3}{*}{\begin{minipage}{2.2cm}
             M2-BASE\\ 
             \citep{fu2023monarch}
         \end{minipage}}& LR & $5\cdot10^{-5}$ & $5\cdot10^{-5}$ & $3\cdot10^{-5}$ & $1\cdot10^{-5}$ & $3\cdot10^{-5}$ & $8\cdot10^{-5}$ & $8\cdot10^{-5}$ & $8\cdot10^{-5}$\\
         & WD & $5\cdot10^{-6}$ & $1\cdot10^{-5}$ & $3\cdot10^{-6}$ & $1\cdot10^{-6}$ & $3\cdot10^{-6}$ & $8\cdot10^{-5}$ & $5\cdot10^{-6}$ & $3\cdot10^{-6}$  \\
         & Epochs & 3 & 10 & 5 & 3 & 3 & 10 & 10 & 10 \\
         & Optimizer & D. AdamW & D. AdamW & D. AdamW & D. AdamW & D. AdamW & D. AdamW & D. AdamW & AdamW \\
         \midrule
         \multirow{3}{*}{\begin{minipage}{2.2cm}
             M2-LARGE\\ 
             \citep{fu2023monarch}
         \end{minipage}}& LR & $5\cdot10^{-5}$ & $5\cdot10^{-5}$ & $3\cdot10^{-5}$ & $5\cdot10^{-5}$ & $3\cdot10^{-5}$ & $8\cdot10^{-5}$ & $5\cdot10^{-5}$ & $8\cdot10^{-5}$\\
         & WD & $5\cdot10^{-6}$ & $1\cdot10^{-6}$ & $3\cdot10^{-6}$ & $1\cdot10^{-6}$ & $3\cdot10^{-6}$ & $8\cdot10^{-6}$ & $1\cdot10^{-6}$ & $3\cdot10^{-5}$  \\
         & Epochs & 3 & 10 & 5 & 2 & 3 & 10 & 10 & 8 \\
         & Optimizer & D. AdamW & D. AdamW & D. AdamW & AdamW & D. AdamW & D. AdamW & D. AdamW & D. AdamW \\
         \midrule
         \multirow{3}{*}{\begin{minipage}{2.2cm}
             Hydra\\
             \citep{hwang2024hydrabidirectionalstatespace}
         \end{minipage}}& LR & $10^{-4}$ & $5\cdot10^{-5}$ & $5\cdot10^{-5}$ & $10^{-5}$ & $5\cdot10^{-5}$ & $8\cdot10^{-5}$ & $10^{-4}$ & $3\cdot10^{-5}$\\
         & WD & $5\cdot10^{-6}$ & $10^{-6}$ & $3\cdot10^{-6}$ & $10^{-6}$ & $3\cdot10^{-6}$ & $8\cdot10^{-6}$ & $8\cdot10^{-6}$ & $3\cdot10^{-6}$  \\
         & Epochs & 2 & 7 & 3 & 3 & 2 & 10 & 10 & 8 \\
         & Optimizer & D. AdamW & D. AdamW & D. AdamW & AdamW & D. AdamW & D. AdamW & D. AdamW & D. AdamW \\
         \midrule
         \multirow{3}{*}{\begin{minipage}{2.2cm}
             Modified\\
             (Ours)
         \end{minipage}}& LR & $10^{-5}$ & $10^{-5}$ & $10^{-5}$ & $10^{-5}$ & $10^{-5}$ & $10^{-5}$ & $10^{-5}$ & $10^{-5}$\\
         & WD & $5\cdot10^{-6}$ & $1\cdot10^{-6}$ & $3\cdot10^{-6}$ & $1\cdot10^{-6}$ & $3\cdot10^{-6}$ & $8\cdot10^{-6}$ & $1\cdot10^{-6}$ & $3\cdot10^{-5}$  \\
         & Epochs & 3 & 10 & 5 & 2 & 3 & 10 & 10 & 8 \\
         & Optimizer & D. AdamW & D. AdamW & D. AdamW & AdamW & D. AdamW & D. AdamW & D. AdamW & D. AdamW \\
         \bottomrule
    \end{tabular}}
    }
    \label{table:glue_hyperparams}
\end{table}
}

\subsection{\rebuttal{Ablation studies in the MLM/GLUE tasks}}
\label{subsec:ablations_glue}

\rebuttal{
\textbf{Combining positional embeddings with \lion.} We compare the GLUE performance of \lionretnet and \lions when including positional embeddings. We pretrain the BASE models and finetune them with the M2-BASE recipe.
}
\begin{table}[t]
    \caption{\rebuttal{Combining positional embeddings with \lionretnet and \lions. Both pretrained models improve in the validation MLM acc. when employing positional embeddings.}}
    \centering
    \rebuttal{
    \resizebox{\textwidth}{!}{
    \begin{tabular}{lc|c|cccccccc|c}
    \toprule
        Model & Pos. Emb. & MLM Acc. & MNLI & RTE & QQP & QNLI & SST2 & STSB & MRPC & COLA & Avg.\\
        \midrule
        \multirow{2}{*}{\lionretnet} & \textcolor{tabred}{\xmark} & 66.62 & 82.85 & 52.49 & 89.63 & 88.43 & 91.86 & 85.96 & 83.94 & 53.58 & 78.59 \\
        & \textcolor{tabgreen}{\cmark} & 66.97 & 83.37 & 54.08 & 89.52 & 88.32 & 92.35 & 83.58 & 79.40 & 54.53 & 78.15 \\
        \midrule
        \multirow{2}{*}{LION-s} & \textcolor{tabred}{\xmark} & 67.05 & 83.17 & 53.50 & 89.35 & 88.89 & 93.00 & 37.73 & 77.87 & 53.18 & 72.09\\ 
        & \textcolor{tabgreen}{\cmark} & 67.35 & 83.26 & 52.42 & 89.82 & 88.38 & 92.58 & 83.87 & 79.54 & 55.25 & 78.14\\
        \bottomrule
    \end{tabular}}
    }

    \label{tab:ablation_posemb}
\end{table}

\rebuttal{
In \cref{tab:ablation_posemb} we can observe that adding positional embeddings increased the MLM acc. in around $0.3$ percentage points. In the GLUE benchmark, we observe that for \lionretnet performance degraded in $0.44$ percentage points, while for \lions, performance improved in $6.05$ percentage points. We attribute this behavior in GLUE to the dependence on the finetuning recipe.
}

\rebuttal{
\textbf{Recipe selection.} In this section, we select the best finetuning recipe for each model family and size. For the BASE models, we test the M2-BASE and Modified recipes. For the LARGE models, we test the M2-LARGE and Modified recipes.
\\
\\
In \cref{tab:ablation_glue_recipe}, firstly, we observe that the M2-BASE recipe generally provides a higher GLUE score than the Modified recipe for the BASE models, e.g., $82.25$ v.s. $80.26$ for the BERT model. Secondly, we observe that for the LARGE model family, the M2-LARGE recipe fails, providing poor performances between $60.96$ and $72.41$ GLUE points. When reducing the learning rate to $10^{-5}$ (Modified recipe), training is more stable and performance reaches between $80.76$ and $82.95$ GLUE points. We find that small changes in the finetuning recipe have a large effect in the performance. Our results in standard recipes show that the \lion family of models can obtain a high performance without extensive tuning and closely follow the performance of the BERT family models, at $80.31$ v.s. $82.25$ for the BASE model size and $81.58$ v.s. $82.95$ for the LARGE model size.
}

\begin{table}[t]
    \caption{\rebuttal{Recipe selection for the GLUE benchmark.}}
    \centering    
    \resizebox{\textwidth}{!}{
    \rebuttal{
    \begin{tabular}{lcc|cccccccc|c}
    \toprule
        Model & MLM Acc. & Recipe & MNLI & RTE & QQP & QNLI & SST2 & STSB & MRPC & COLA & Avg.\\
        \midrule
        \multirow{2}{*}{BERT} & \multirow{2}{*}{67.70} & M2-BASE & 84.63 & 64.33 & 89.99 & 89.80 & 92.51 & 86.69 & 89.62 & 60.42 & 82.25\\
        & & Mod. & 83.09 & 58.27 & 89.35 & 89.88 & 92.16 & 86.56 & 87.78 & 55.02 & 80.26\\
        \midrule
        \multirow{2}{*}{\lionlit} & \multirow{2}{*}{65.47} & M2-BASE & 82.50 & 63.47 & 89.72 & 89.27 & 91.74 & 87.18 & 89.37 & 49.22 & 80.31\\
        & & Mod. & 80.88 & 54.95 & 88.80 & 88.83 & 91.32 & 85.42 & 87.07 & 46.98 & 78.03 \\
        \midrule
        \multirow{2}{*}{\lionretnet} & \multirow{2}{*}{66.62} & M2-BASE  & 82.85 & 52.49 & 89.63 & 88.43 & 91.86 & 85.96 & 83.94 & 53.58 & 78.59 \\
        & & Mod. & 80.52 & 52.85 & 88.93 & 88.36 & 91.55 & 82.05 & 84.48 & 49.13 & 77.23 \\
        \midrule
        \multirow{2}{*}{\lions} & \multirow{2}{*}{67.05} & M2-BASE & 83.17 & 53.50 & 89.35 & 88.89 & 93.00 & 37.73 & 77.87 & 53.18 & 72.09\\ 
        & & Mod. & 78.14 & 56.39 & 88.68 & 88.52 & 92.39 & 51.22 & 77.60 & 49.75 & 72.84 \\ 
        \midrule
        \multirow{2}{*}{BERT$_{\text{LARGE}}$} & \multirow{2}{*}{69.88} & M2-LARGE & 84.97 & 69.10 & 31.59 & 49.15 & 91.93 & 53.61 & 87.87 & 51.16 & 64.92\\
        & & Mod. & 85.68 & 67.44 & 89.90 & 91.89 & 93.04 & 88.63 & 90.89 & 56.14 & 82.95\\ 
        \midrule
        \multirow{2}{*}{Hydra$_{\text{LARGE}}$} & \multirow{2}{*}{71.18} & Hydra & 84.24 & 60.44 & 89.24 & 89.73 & 91.70 & 88.21 & 88.99 & 47.72 & 80.03\\
        & & Mod. & 84.39 & 59.42 & 90.38 & 91.31 & 93.43 & 87.19 & 88.57 & 59.46 & 81.77 \\
        \midrule
        \multirow{2}{*}{\lionlit$_{\text{LARGE}}$} & \multirow{2}{*}{67.11} & M2-LARGE & 83.20 & 54.51 & 89.08 & 84.90 & 90.44 & 68.57 & 85.25 & 23.35 & 72.41 \\
        & & Mod. & 83.73 & 57.18 & 89.85 & 89.93 & 91.86 & 88.02 & 90.18 & 55.36 & 80.76 \\
        \midrule
        \multirow{2}{*}{\lionretnet$_{\text{LARGE}}$} & \multirow{2}{*}{68.64} & M2-LARGE & 83.82 & 52.85 & 41.48 & 53.67 & 91.13 & 36.87 & 82.41 & 45.79 & 61.00 \\ 
        & & Mod. & 83.82 & 60.72 & 89.72 & 89.79 & 92.93 & 87.29 & 89.66 & 56.83 & 81.34 \\ 
        \midrule
        \multirow{2}{*}{\lions$_{\text{LARGE}}$} & \multirow{2}{*}{69.16} & M2-LARGE & 83.71 & 50.04 & 38.81 & 53.98 & 91.59 & 36.98 & 82.29 & 50.27 & 60.96 \\
        & & Mod. & 84.38 & 57.69 & 89.57 & 90.30 & 92.93 & 87.68 & 90.57 & 59.54 & 81.58 \\
        \bottomrule
    \end{tabular}}
    }
    \label{tab:ablation_glue_recipe}
\end{table}

\subsection{\rebuttal{Additional experimental results for the MLM/GLUE tasks}}
\label{subsec:glue_small}

In this section, we present our bidirectional MLM results in the BASE scale using the BERT pretraining recipe described in \cref{subsec:details_glue} and BERT24 \citep{izsak2021train} finetuning recipes (\cref{tab:MLM_small}), we present the per-task GLUE results omitted in the main text (\cref{tab:MLM_full}) and present the length scaling capabilities of the \lions model (\cref{fig:bert_memory}).

\begin{table}[ht]
\centering
\caption{\textit{C4 Masked Language Modelling and GLUE results for the BASE scale ($110$M).}}
\vspace{-2mm}
\begin{tabular}{l|cccccccccc}
\toprule
Model & MLM Acc. & MNLI & RTE & QQP & QNLI & SST2 & STSB & MRPC & COLA & Avg. \\
\midrule
BERT        & {67.23}  & 84.26         & {59.21}  & 89.87         & 90.24         & {92.35}  & {88.12}  & {90.24}  & 56.76         & 81.38         \\ 
Hydra$^{*}$    & {69.10}     & {84.50} & {57.20}  & {91.30} & {90.00} & {93.50}  & {91.20}  & {88.90}  & {77.50} & {84.30} \\
\rowcolor{Green!10}
\lionlit    & 65.08              & 82.37         & 55.81           & 89.49         & {89.57}  & 91.74         & 86.27         & 88.25         & 44.46         & 78.50         \\ 
\rowcolor{violet!20}
\lionretnet & 66.62              & 82.85         & 52.49           & {89.63}  & 88.43         & 91.86         & 85.96         & 83.94         & 53.58         & 78.59         \\
\rowcolor{orange!17}
\lions      & 66.19              & 82.50         & {57.47}  & 89.38         & 87.88         & {92.70}  & 82.42         & 82.46         & {53.39}  & 78.40         \\ 
\bottomrule
\multicolumn{11}{l}{\begin{footnotesize}
    * Results extracted from the original paper \citep{hwang2024hydrabidirectionalstatespace}.
\end{footnotesize}}
\end{tabular}
\label{tab:MLM_small}
\end{table}

\begin{table*}[t]
    \caption{\textit{C4 Masked Language Modelling and GLUE results for the LARGE scale ($334$M).} For each column (dataset), the best and the second best results are highlighted with \textbf{bold} and \underline{underline} respectively.}
    \centering
    \vspace{-2mm}

    \resizebox{\textwidth}{!}{
    \rebuttal{
    \begin{tabular}{l|c|cccccccc|c}
    \toprule
        Model & MLM Acc. & MNLI & RTE & QQP & QNLI & SST2 & STSB & MRPC & COLA & Avg.\\
        \midrule
        BERT  & $69.88$ & $\mathbf{85.68}$ & $\mathbf{67.44}$ & $\underline{89.90}$ & $\mathbf{91.89}$ & $\underline{93.04}$ & $\mathbf{88.63}$ & $\mathbf{90.89}$ & ${56.14}$ & $\mathbf{82.95}$\\ 
        Hydra  & $\mathbf{71.18}$ & \underline{84.39} & 59.42 & $\mathbf{90.38}$ & $\underline{91.31}$ & $\mathbf{93.43}$ & 87.19 & 88.57 & \underline{59.46} & \underline{81.77} \\
        \rowcolor{Green!10} \lionlit  & 67.11 & 83.73 & 57.18 & $89.85$ & 89.93 & 91.86 & $\underline{88.02}$ & 90.18 & 55.36 & 80.76 \\
        \rowcolor{ violet!20}
        \rebuttal{\lionretnet}  & \rebuttal{68.64}& \rebuttal{83.82} & \rebuttal{\underline{60.72}} & \rebuttal{89.72} & \rebuttal{89.79} & \rebuttal{92.93} & \rebuttal{87.29} & \rebuttal{89.66} & \rebuttal{56.83} & \rebuttal{81.34} \\ 
        \rowcolor{orange!17} \lions  & $\underline{69.16}$ & $84.38$ & ${57.69}$ & 89.57 & $90.30$ & $92.93$ & 87.68 & $\underline{90.57}$ & $\mathbf{59.54}$ & $81.58$ \\
        \bottomrule
    \end{tabular}}}
    \label{tab:MLM_full}

\end{table*}


\begin{figure}[t] 
    \centering
    \includegraphics[width=0.45\columnwidth]{figs/XSequenceLength_yAcc.pdf}
    \includegraphics[width=0.45\columnwidth]{figs/text_xRes_yMem.pdf}
    \vspace{-1mm}
  \caption{%
  \textit{(Left) MLM Acc. for different sequence lengths.} \lions is able to generalize to larger sequence lengths and does not require positional encoddings. \textit{(Right) GPU Memory for different sequence lengths.} Results for the LARGE (334M) scale. The memory employed by \lion and Hydra scales linearly with the sequence length, being able to process a sequence length of $\sim 16.000$ tokens with less than 20GB. Contrarily, the memory employed by BERT scales quadratically, going out of memory ($80$GB) at a sequence length of $\sim 12.000$ tokens.
  }
  \label{fig:bert_memory}
\end{figure}

\subsection{ImageNet classification results for Tiny scale}
\label{app:tiny}

In \cref{tab:imc_tiny}, we present the image classification results of \lion models on the tiny scale models an compare them against the baseline models. Results indicate that the high training speed and competitive performance of \lion models is also applicable in the tiny scaled models.

\begin{table}[t]
        \captionof{table}{\textit{Image classification task in Tiny scale.} We present the Top-1 accuracy on the validation data.* represents the changing in patch orders.
        For each scale, the best and the second best results for each model are highlighted with \textbf{bold} and \underline{underline} respectively.
        }
        \label{tab:imc_tiny}
        \begin{minipage}{\linewidth}
          \centering
                \begin{tabular}{clccc}
                \toprule
                    & Model    & $\#$Param & \begin{tabular}{@{}l@{}}  Imagenet \\ Top-1 Acc. \end{tabular}   & Train. time\\
                     \bottomrule
\parbox[t]{3mm}{\multirow{9}{*}{\rotatebox[origin=c]{90}{Tiny}}} & ViT      & 5M &  70.2 & $\times 1$\\
                    & DeiT      & 5M &   72.2 & $\times 1$\\
                    &\rebuttal{Vim  }  & 7M & 76.1 & $\times 9.48$\\%_{\pm 0.17}$\\
                    &\cellcolor{Green!10} \lionlit   & \cellcolor{Green!10} 5M & \cellcolor{Green!10} 68.9 & \cellcolor{Green!10}$\times 0.69$\\%_{\pm 0.01}$\\
                    &\cellcolor{ violet!20}
                    \rebuttal{\lionretnet}   & \cellcolor{ violet!20} 5M & \cellcolor{ violet!20} 72.4  & \cellcolor{ violet!20}$\times 1.48$\\%_{\pm 0.03}$\\ 
                    &\cellcolor{ violet!20}
                    \textbf{\lionrotd}   & \cellcolor{ violet!20} 5M & \cellcolor{ violet!20} 74.2 & \cellcolor{ violet!20}$\times 1.73$\\%_{\pm 0.04}$\\
                    &\cellcolor{orange!17}
                    \textbf{\lions}   & \cellcolor{orange!17} 5M & \cellcolor{orange!17} 72.4 & \cellcolor{orange!17}$\times 2.05$\\%_{\pm 0.21}$\\
                    &\cellcolor{orange!17}
                    \textbf{\lionrot}   & \cellcolor{orange!17} 5M & \cellcolor{orange!17} 73.5 & \cellcolor{orange!17}$\times 3.83$\\%_{\pm 0.17}$\\
                     \bottomrule
                     \end{tabular} 
        \end{minipage}
\end{table} 

\subsection{The inference time/memory trade off}
\label{subsec:chunks}

In \cref{fig:chunks}, we illustrate the trade-off between memory consumption and inference time. Across all models, RNN proves to be the most memory-efficient approach, while full attention is the most demanding. Both chunking and full attention goes out of memory sooner than RNN. Similar to \lionretnet, chunking achieves inference times comparable to, and occasionally better than, full attention. However, in the case of \lions, chunking is faster than RNN at lower resolutions but becomes slower at higher resolutions due to the computational cost of mask calculation. Consequently, while chunking is preferable for \lionlit and \lionretnet when memory permits, at high resolutions, RNN can be a better choice when dealing with more complex masks.

\begin{figure}
     \centering
     \begin{subfigure}[b]{0.45\textwidth}
         \centering
    \includegraphics[width=\columnwidth]{figs/linear_chunking.pdf}
         \caption{\lionlit}
     \end{subfigure}
     \hfill
     \begin{subfigure}[b]{0.45\textwidth}
         \centering
    \includegraphics[width=\columnwidth]{figs/selective_chunking.pdf}
         \caption{\lions}
     \end{subfigure}
    \caption{\textit{The memory-time trade off.}The inference memory and time plots of \lionlit and \lions models in three formats: RNN, chunking and full attention.}
    \label{fig:chunks}
\end{figure}




\subsection{Inference time comparison for image classification tasks}
\label{subsec:inf_time}

In \cref{fig:inf_time}, we compare the baseline models on the image classification task. At lower resolutions, all models exhibit similar inference times, with Vim and Hydra being slightly slower than ViT and \lionretnet chunking. However, at higher resolutions, vision SSMs (Vim and Hydra) become faster. This trend arises because vision SSMs leverage specialized kernels, whereas ViT and \lionretnet rely on plain Python implementations, leading to increasing overhead as resolution grows.

\begin{figure}[t] 
    \centering
    \includegraphics[width=0.5\columnwidth]{figs/inference_time.pdf}
    \vspace{-1mm}
  \caption{\looseness=-1 \textit{Inference times comparison.} The inference time of \lionretnet with chunking, Vim, Hydra and ViT models are presented for different resolutions.
   }
  \label{fig:inf_time}
\end{figure}

\subsection{Ablation studies with image classification}
\label{subsec:image_ablation}

\textbf{Resolution vs. Accuracy.} The most common practice in the literature on Vision Transformers is to resize images to $224\times224$ even though most of the images in the ImageNet dataset are larger. Since regular Transformers have positional embedding, it is not possible to use a larger resolution during inference than the training. However, since the \lions architecture does not include any positional embeddings, it can be used with different resolutions. In Figure \ref{fig:res_acc}, we present the accuracy of the architectures trained on $224\times224$ resolution on the ImageNet dataset at different inference resolutions. As the results illustrate, the abilities of \lions can be effectively transferred among different resolutions.


\begin{figure*}[!b]
    \vspace{-3mm}
    \centering
    \includegraphics[width=0.7\textwidth]{figs/XRes_yAcc.pdf}
  \caption{\textit{Top-1 accuracy on Imagenet of the models at different resolutions.} Images are resized at the corresponding resolution and fed into the model. Due to positional embeddings, ViT and \lionlit models cannot perform with sizes larger than the training size while \lions can preserve the accuracy for much higher resolutions. }
    \label{fig:res_acc}
    \vspace{-3mm}
    
\end{figure*}


\textbf{Choice of $\lambda_i$ values.}

In this section, we study the properties of the selectivity parameter $a_i$ on CIFAR-100 dataset. We tested, three cases: ($i$) fixed mask with scalars $a_i = a^i$, ($ii$) vector, input-dependent \(\mathbf{a}_i \in \mathbb{R}^d\) (\textit{cf.}, \cref{sec:expandai}) and iii) input dependent scalar \(\mathbf{a}_i \in \mathbb{R}\). The results, presented in Table~\ref{tab:ablation_lambda}, show that while the input dependency is beneficial, the expansion of \(\mathbf{a}_i \) is not necessary for image tasks. As a result, we employ option three in all image classification tasks, and the end model is called \lions.

\begin{table}[h]
 \caption{\textit{Ablation studies on image classification.} Additional ablations with CIFAR100 dataset to determine the size and input dependency of the selectivity parameter of the model \lions.}
     \centering

\begin{tabular}{l|c}
    \toprule
    Models & Top-1 Acc.\\
    \midrule
    Fixed mask $a_i = a^i$ & 75.66 \\
    Vector $\mathbf{a}_i \in \mathbb{R}^d$ & 67.55 \\
     \rowcolor{orange!17}
    Scalar, input dependent \(\mathbf{a}_i \in \mathbb{R}\) (\lions) &  \textbf{77.56} \\
    \bottomrule
\end{tabular} 
   \label{tab:ablation_lambda}
\end{table}

\textbf{Understanding the power of non-linearity, softmax, and positional embeddings.} In Table~\ref{tab:ablation_extensive}, we present additional ablations on certain design elements of a Vision Transformer. We perform these experiments on CIFAR-100 data using the same hyperparameters with \lions. We have observed that either nonlinearity or softmax is essential for the model to converge with a nice accuracy. Though positional embedding boosts the accuracy, a mask can easily replace it. 

\begin{table}[h]
 \caption{\textit{Ablation studies on image classification.} Additional ablations with the CIFAR-100 dataset to understand the contribution of softmax, nonlinearities in a model is presented. Soft., PosEmb and NonLin expresses if softmax, positional embedding, and non-linearity have been applied. \xmark ~means the model did not converge. The \legendsquare{green!10} symbol denotes the adaptation of recurrent models that achieve equivalence to attention during training while utilizing recurrence during inference, as established by our theorem.}
    \centering
\begin{tabular}{l|c}
    \toprule
    Models & Top-1 Acc.\\
    \midrule
    $[1]$ Soft. + PosEmb + NonLin & 73.88 \\
    $[2]$ Soft. + PosEmb (ViT-T)   & 77.33\\
    $[3]$ Soft. + NonLin           & \xmark \\
    $[4]$ Soft.                     & 73.15 \\
    \rowcolor{Green!10}$[5]$ PosEmb + Non.Lin (\lionlit) & 73.61\\
    \rowcolor{Green!10}$[6]$ PosEmb         & 68.54 \\
    \rowcolor{Green!10}$[7]$ NonLin        & 65.28\\
    \rowcolor{Green!10}$[8]$ Base       & \xmark \\
    \midrule
    \rowcolor{orange!17}
    Non.Lin + Mask (\lions) & \textbf{77.56} \\
    \bottomrule
\end{tabular} 
   \label{tab:ablation_extensive}
\end{table}



\subsection{Hyperparameters for Training Image Classifiers}
\label{subsec:image_hyper}

All experiments were conducted on a single machine for CIFAR-100 and multiple machines for ImageNet, using NVIDIA A100 SXM4 80GB GPUs. For \lion models and Hydra, the ViT architecture serves as the main structure, with Hydra following the Hydra training recipe and other models following the ViT recipe. The training and evaluation codes are adapted from ~\cite{Touvron2020Training} and ~\cite{rw2019timm}.

\subsection{\rebuttal{Calculation of Number of FLOPS}}
\label{subsec:flops}
\rebuttal{Below we present a theoretical number of FLOPS used in the attention of vision transformers and \lions during inference where $L$ is the resolution/context length and $D$ is the hidden dimension. Results show that while transformer has $\cO(L^2+LD^2)$ \lions has $\cO(LD^2)$. Note that in this calculation, the exponentials and other nonlinearities are considered as 1 FLOP whereas in reality, the Softmax introduces additional complexities. The same calculations should also apply to other bi-directional models. 
\\
\\
The number of FLOPs in the one head of the one layer attention for a vision transformer:
\begin{itemize}
    \item Calculating  $\mathbf{Q}, \mathbf{K}, \mathbf{V}$: $6 L D^2$,
    \item Attention $ A = \mathbf{Q} \mathbf{K}^T$ : $2L^2 D$
    \item Softmax (assuming 1 FLOP for exp): $2 L^2$
    \item Calculating $\mathbf{Y}$: $ 2 L^2 D$
    \item \textbf{TOTAL:} $L(6D^2 + 4LD+ 2L)$
\end{itemize}
The number of FLOPs in the attention module for \lion:
\begin{itemize}
    \item Calculating $\mathbf{Q}, \mathbf{K}, \mathbf{V}, \mathbf{\lambda}$: $6LD^2 + 2LD$,
    \item For each token in one forward/backward recurrence:
    \begin{itemize}
        \item Updating $\mathbf{S}_i^{F/B}$: $3 D^2$
        \item Updating $\mathbf{z}_i^{F/B}$: $2D$
        \item Calculating $c_i^{F/B}$: $4D + 2$
        \item Calculating $\mathbf{y}_i^{F/B}$: $2D^2 + 4D+1$
        \item Total: $5D^2 +10D+3$
    \end{itemize} 
    \item L forward + backward recurrences:  $2L(5D^2 +10D+3)$
    \item Calculating $\mathbf{Y}$: $2L(D+1)$
    \item \textbf{TOTAL:} $L(16D^2 + 24D+ 7) $
\end{itemize}
}



\subsection{\rebuttal{Distillation Results of LION-S}}
\label{app:distill}
\rebuttal{We have also used the same recipe from DeiT distillation \cite{deit} and distilled the RegNet network into LION-S. We observed that the distillation outperforms the original ViT-Tiny on the ImageNet dataset. The results are shown in the table below:}
\begin{table}[h]
 \caption{\textit{\rebuttal{Distillation results of \lions.}}}
    \centering
    \rebuttal{
\begin{tabular}{l|c}
    \toprule
    Models & Top-1 Acc.\\
    \hline
    \lions & 67.95 \\
    VIT-Tiny & 70.23 \\
    \rowcolor{orange!17}
    \lions (Distilled) &\textbf{ 70.44} \\
    \bottomrule
\end{tabular} }
   \label{tab:destill}
\end{table}






\subsection{\rebuttal{Ablation studies in mapping of autoregressive models to \lion framework}}
\label{app:direction2}
\rebuttal{ Building on the mapping of autoregressive models in \cref{sec:map}, we conducted additional experiments using \lionretnet and \lion-\textsc{GRFA}. Specifically, we modified the transformer block of the VIT-Tiny model according to the proposed mapping and evaluated its performance on the CIFAR-100 dataset, maintaining the same training recipes as \lions. The results, summarized in \cref{tab:grfa}, demonstrate that the \lion framework facilitates the seamless extension of other autoregressive models to a bi-directional setting, achieving strong performance without requiring additional hyperparameter tuning.}

\begin{table}[h]
    \centering
    \caption{\rebuttal{\textit{Mapping of autoregressive models to bidirectional setting with \lion framework.}. These models benefit from the expansion to the bi-directional setting using the \lion framework.}}
    \rebuttal{
    \begin{tabular}{l|c}
    \toprule
        \textbf{Model} & \textbf{Top-1 Acc.} \\
        \midrule
        \textsc{GRFA} (Uni-directional)       & 71.56 \\
        \lion-\textsc{GRFA}  (Bi-directional)   & 73.24 \\
        \hline
        \textsc{RetNet}  (Uni-directional)      & 72.24  \\
        \rowcolor{ violet!20} \lionretnet (Bi-directional) & 75.66  \\
        \hline
    \end{tabular}}
    \label{tab:grfa}
\end{table}

\subsection{\rebuttal{Ablation studies on importance of bi-directionality on image classification}}
\label{app:direction}

\rebuttal{To highlight the importance of bi-directionality and demonstrate the versatility of the \lion framework, we conducted additional experiments examining the processing directions of the blocks. We evaluated four settings: (i) all blocks process patches in the forward direction only (Forward), (ii) all blocks process patches in the backward direction only (Backward), (iii) odd-numbered blocks process patches in the forward direction while even-numbered blocks process them in the backward direction (Forward-Backward), and (iv) all blocks process patches in both directions (Bi-directional). The results reveal that incorporating both directions improves performance by approximately $4\%$, while full bi-directionality achieves a significant boost of up to $10\%$.}

\begin{table}[h]
\centering
\caption{\rebuttal{Results for \lions and \lionrot with different directional settings on CIFAR-100. Incorporating both directions improves performance by approximately $4\%$, while full bi-directionality achieves a significant boost of up to $10\%$.}}
\label{tab:forward}
    \rebuttal{
    \begin{tabular}{l|c}
    \toprule
    \textbf{Model}                             & \textbf{Top-1 Acc.}  \\ 
    \midrule
    \lions (Forward)                       & 71.08          \\ 
    \lions (Backward)                      & 69.61          \\ 
    \lions (Forward-backward)                   & \underline{73.93}\\ 
    \rowcolor{orange!17} \textbf{\lions (Bi-directional)}            & \textbf{77.56} \\ 
    \hline
    \lionrot (Forward)                  & 70.24          \\ 
    \lionrot (Backward)                 & \underline{70.42} \\ 
    \rowcolor{orange!17} \textbf{\lionrot (Bi-directional)}       & \textbf{80.07} \\ 
    \hline
    \end{tabular}}
\end{table}

\end{document}
\endinput
%%
%% End of file `sample-sigconf.tex'.
