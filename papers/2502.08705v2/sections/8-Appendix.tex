%\vspace{-0.5cm}
%\clearpage
\section{Annotation Platform}
We used the Zooniverse for the annotation process. 
Zooniverse boasts over one million active members collaborating on hundreds of projects ranging from the space sciences to transcription of historical documents\footnote{\url{https://www.zooniverse.org/about}}. 
Here, ``first choice'' is defined as the strongest sentiment expressed in the statement. If there were two equally strong sentiments expressed, then the sentiment that appeared first was selected as the ``first choice'' (the analysis of ``second-choice'' selections is relegated to a future paper).
Once the user made their selection, they could select ``Done'' to move to the next stage of the annotation (green button in \autoref{fig:first_step_annotation}) or ``Done \& Talk'' to ask questions in the main forum (blue button in \autoref{fig:first_step_annotation}).
% Annotators can also click the ``Favorite'' button (heart symbol) to save the specific annotation to their personal collection for later discussion or can access more information about the annotation (URL and film title) with the ``Info'' button (circled ``i'' symbol).
% After completion of the first choice sentiment annotation, the user is prompted to self report their confidence in their annotation (with levels of ``I am confident'', ``I am a little unsure'', and ``I am a lot unsure'').
The annotator was then prompted to repeat the process with their first choice for impact category (categories listed in the first column of \autoref{tab:impact_categories}).

\begin{figure*}[t]
\centering
\includegraphics[width=0.6\textwidth]{sections/figures/example3_image.png} 
\caption{First stage of annotation process in which users are instructed to select their first choice for the sentiment category for each sentence. Each sentence (top) is shown in the context of the full review (bottom). The annotator moves to the next stage of annotation with the green ``Done'' button (or blue ``Done \& Talk'' button to post comments in the project forum).  More information (comment URL and film title) is accessed with the circled ``i'' Info button and annotation can be saved in the user's personal collection for later reference with the heart Favorite button.}
\label{fig:first_step_annotation}
\end{figure*}


\section{Prompt design}\label{appendix}
% \captionsetup{width=\textwidth}
% \captionsetup{justification=centering, width=12cm}
\begin{figure*}[ht]
    % \vspace*{-0.4cm}  % Adjust the value as needed
    % \hspace{-0.7cm}
    \centering
    \includegraphics[width=0.7\textwidth] {sections/figures/Impact_Prompt.pdf}
    % \captionsetup{skip=0pt}
    \caption{Prompt used for prediction of impact categories}
    \label{Prompt}
    % \Description[<Prompt>]{<Prompt used for prediction of impact categories>}
\end{figure*}
Figure \ref{Prompt} illustrates the prompt we used to develop LLM classifiers for impact categories that are instructed to label a target sentence, using full reviews as context. The prompt begins with an instruction of the classification task, followed by the provided definitions of impact labels, and feeding both the context and target sentence to the prompt at the end.   






