\section{Results}



%https://docs.google.com/spreadsheets/d/10qeBrleRQxnq_SM6HEPA1pWhQKOgwK_U/edit?gid=1849672800#gid=1849672800
\subsection{Classification} \label{section:classification}

Table \ref{tab:Classification Result Sentiment} shows the performance of different classifiers on our dataset. Transformer- and LLM-based classifiers outperformed baseline models. GPT4 and GPT4o with context embedded in the prompts are performing the best compared to similar models, with F1 scores of 0.71 (GPT4o) for \textit{Impact} and 0.85 (GPT4) for \textit{Sentiment}, respectively. Our fine-tuned model of GPT4o resulted in the highest performance for both tasks, 0.76 and 0.88. This confirms the usefulness of fine-tuning LLMs with additional data. We also observed that while using full reviews as context is improving the performance of Sentiment classifiers, its helpfulness for Impact classification seems less significant. 

% Please add the following required packages to your document preamble:
% \usepackage{booktabs}
% \usepackage{multirow}
% \usepackage{graphicx}
\begin{table}[b]
\centering
\resizebox{0.9\columnwidth}{!}{%
\begin{tabular}{@{}cllll|lll@{}}
\toprule
\multicolumn{1}{l}{} &  & \multicolumn{3}{c}{Impact} & \multicolumn{3}{c}{Sentiment} \\ \cmidrule(lr){3-5} \cmidrule(l){6-8}
\multicolumn{1}{l}{} &  & P & R & F1 & P & R & F1 \\\cmidrule(lr){1-2}\cmidrule(lr){3-5} \cmidrule(l){6-8}
\multirow{3}{*}{} & SVM & 0.57 & 0.61 & 0.58 & 0.68 & 0.65 & 0.6 \\
 & Logistic Regression & 0.56 & 0.62 & 0.57 & 0.63 & 0.64 & 0.63 \\
 & Decision Tree & 0.47 & 0.61 & 0.51 & 0.49 & 0.55 & 0.49 \\\midrule
\multirow{2}{*}{} & BERT & 0.56 & 0.67 & 0.59 & 0.75 & 0.77 & 0.75 \\
 & RoBERTa & 0.62 & 0.72 & 0.66 & 0.76 & 0.76 & 0.76 \\\midrule
\multirow{5}{*}{\rotatebox{90}{w/o context}} & GPT3.5 & 0.73 & 0.65 & 0.67 & 0.76 & 0.77 & 0.76 \\
 & GPT4 & 0.75 & 0.68 & 0.69 & 0.82 & 73 & 0.75 \\
 & GPT4o & 0.76 & 0.68 & 0.7 & 0.82 & 0.7 & 0.72 \\
 & LLama & 0.68 & 0.45 & 0.47 & 0.79 & 0.71 & 0.73 \\
 & Mixtral & 0.73 & 0.52 & 0.57 & 0.81 & 0.74 & 0.76 \\\midrule
\multirow{5}{*}{\rotatebox{90}{with context}} & GPT3.5 & 0.68 & 0.64 & 0.61 & 0.77 & 0.8 & 0.77 \\
 & GPT4 & 0.73 & 0.68 & 0.69 & 0.86 & 0.84 & 0.85 \\
 & GPT4o & 0.74 & 0.71 & 0.71 & 0.84 & 0.78 & 0.8 \\
 & LLama & 0.67 & 0.57 & 0.6 & 0.8 & 0.8 & 0.8 \\
 & Mixtral & 0.73 & 0.63 & 0.62 & 0.82 & 0.8 & 0.81 \\\midrule
\multicolumn{1}{l}{} & GPT4o Fine-tuned & \textbf{0.77} & \textbf{0.77} & \textbf{0.76} & \textbf{0.89} & \textbf{0.87} & \textbf{0.88} \\ \bottomrule
\end{tabular}%
}
\caption{Classification results for sentiment and impact categories (P: Precision, R: Recall, F1: F1-Score)}
\label{tab:Classification Result Sentiment}
\end{table}



\subsection{Error Analysis}
To enhance our understanding of our top-performing models, we conducted a detailed analysis of the fine-tuned GPT4o model based on the \impact\ class, which is the central focus of our study. \autoref{tab:F1 per class} shows the F1 score per impact class \rt{(center column)}. ``Attitudes Toward the Film'' has the highest classification performance of 0.86 while the F1 score for ``Impersonal Report'' is lowest at 0.42. 
%% JPN:
% authors report the most common misclassification categories by the models. It would be valuable to examine the agreement levels between human annotators for these specific categories. This analysis could help determine whether the misclassification stems from ambiguities or limitations in the taxonomy itself, which might suggest the need for refinement. Alternatively, it could highlight other factors such as model bias or data-related issues.
%
\rt{These trends rightly align with the Cohen's Kappa measure of human annotator intercoder agreement \citep{mchugh2012interrater} shown in the right column of \autoref{tab:F1 per class}.  While most categories show substantial agreement between annotators ($\kappa > 0.6$), ``Impersonal Report'' is on the borderline between moderate and substantial agreement.}
%
\rt{The relatively larger} disparity \rt{in F1 score from the model's classifications when compared to differences in Cohen's Kappa} may be linked to the imbalance of impact categories in the dataset (as seen across the entire dataset in \autoref{fig:sentiment_impact} and \autoref{tab:movie_info}) during fine-tuning process or the more challenging nature of identifying this category. 

\autoref{tab:confusion matrix} demonstrates the confusion matrix of \impact\ classes, where we can see the most common type of misclassifications by the model. It is evident that ``Attitudes Toward the Film'' and ``Impersonal Report'' are most often misclassified as ``Shift in Cognition'', while ``Interest with Science Topic'', ``Not applicable'', and ``Shift in Cognition'' are mostly misclassified as ``Attitudes Toward the Film''. 
%It is 
\begin{table}[b]
\centering
\resizebox{\columnwidth}{!}{%
\begin{tabular}{@{}lccccc@{}}
\hline
\textbf{True Label} & \multicolumn{5}{c@{}}{\textbf{Predicted Label}}\\
\cmidrule(l){2-6}
& Attitudes &  Impersonal & Interest & NA & Shift  \\ 
\midrule
  Attitudes Toward the Film  & 202 & 1 & 4 & 5 & 17  \\
  Impersonal Report  & 3 & 4 & 0 & 1 & 4  \\
  Interest with Science Topic  & 10 & 0 & 11 & 1 & 4  \\
  Not applicable & 20 & 2 & 0 & 31 & 4  \\
  Shift in Cognition  & 8 & 0 & 1 & 1 & 32  \\
\hline
\end{tabular}}

\caption{Confusion Matrix for True and Predicted Impact Categories: Attitudes (Attitudes Toward the Film), Impersonal (Impersonal Report), Interest (Interest with Science Topic), NA (Not applicable), and Shift (Shift in Cognition)}
\label{tab:confusion matrix}

\end{table}
%\clearpage

\begin{table}[t]

\begin{tabular}{@{}lcc@{}}
\toprule
\rt{Impact Category} & \rt{F1} & \rt{Cohen's Kappa} \\
\hline
Attitudes Toward the Film & 0.86 & \rt{0.68} \\ 
Impersonal Report & 0.42 & \rt{0.61} \\ 
Interest with Science Topic & 0.52 & \rt{0.67} \\ 
Shift in Cognition & 0.62 & \rt{0.67} \\ 
Not applicable & 0.65 & \rt{0.67} \\ \bottomrule
\end{tabular}
\caption{F1 score \rt{(model) and Cohen's Kappa (human-annotations)} per class for Impact.}
\label{tab:F1 per class}
\end{table}

%\clearpage
\subsection{Generalizability Test} \label{section:generalizability}
To evaluate the generalizability of our models in successfully capturing the impact of movies, we applied the best model, with the highest classification performance (``GPT4o Fine-tuned'') to the Hubble dataset and predicted the labels for both \impact\ and \sentiment. 
\autoref{tab:Hubble_Prediction} shows the distribution of predicted classes for this dataset. While the majority of reviews show ``Positive'' sentiment toward the Hubble documentary, ``Attitudes Toward the Film'' is the most frequent type of impact predicted by our model in the movie review. 
%
This aligns with the prevalence of the ``Attitudes Toward the Film'' impact category among the other movies in %\autoref{fig:first_step_annotation} 
 \autoref{tab:movie_info}.

To assess the model's performance, we tasked two annotators with evaluating the accuracy of the predicted labels for both the \impact\ and \sentiment\ categories. The mean agreement for \impact\ is approximately 85.3\%. For \sentiment\ agreement is $\sim$87.6\%, demonstrating a substantial level of alignment between the annotators and the classification model.


\begin{table}[b]
\centering
\begin{tabular}{@{}lc|lc@{}}
\toprule
\textbf{Impact} &  & \textbf{Sentiment} & \\ \cmidrule(l){1-2}\cmidrule(l){3-4}
Attitudes Toward the Film & 307 & Positive & 250 \\
Not applicable & 50 & Neutral & 94 \\
Interest with Science Topic & 17 & Negative & 56 \\
Shift in Cognition & 15 &  &  \\
Impersonal Report & 11 &  & \\\bottomrule
\end{tabular}
\caption{Distribution of predicted classes for Hubble dataset}
\label{tab:Hubble_Prediction}

\end{table}

\subsection{Thematic Analysis}
\autoref{tab:Themes Impact} shows the themes of discussions in sentences, labeled with each impact category using LLooM. We exclude ``Not applicable'' from \autoref{tab:Themes Impact} as this category covers a wide variety of comments which are not relevant to the overall analysis of the movies' impacts.
Each \impact\ contains several related themes based on how reviewers have described the movie. While sentences labeled as ``Attitudes Toward the Film'' discuss various positive, negative, or mixed feedback on the movie, e.g., \textit{``Kind of boring bearly any action'' or ``The graphics are... OUT OF THIS WORLD''}, reviews tagged with ``Shift in Cognition'' cover themes such as space science and environmental concerns or information quality, e.g., \textit{``Jupiter is a stunningly gorgeous planet, and it's amazing to finally learn some of its secrets.''}

Context around ``Interest with Science Topic'' includes mentions of science and technical aspects, emotional and experiential reactions, content quality, and learning, e.g., \textit{``If observing and learning of the universe is your thing, you will love this''}. ``Impersonal Report'' often consists of safety issues or historical and informational aspects of the movies, e.g., \textit{``An 1859 type event will occur again and cause
1 to 2 Trillion dollars in damage.''} These themes collectively show how the viewers have interacted with scientific documentaries and how these movies engage with their cognition, attitudes, and interests.

\begin{table*}[t]
\centering
\resizebox{0.9\textwidth}{!}{%
\begin{tabular}{@{}lll@{}}
\toprule
Impact class & Themes & Examples \\
\midrule
% Attitudes Toward the Film
\multirow{4}{*}{Attitudes Toward the Film} & \begin{tabular}[c]{@{}l@{}}Positive Engagement and Impact e.g. Positive feedback, \\ Desire for more similar content, \\ Describing the movie as engaging \& informative\end{tabular} & Very interesting and informative, liked it a lot. \\ \cline{2-3}
& \begin{tabular}[c]{@{}l@{}}Negative Feedback and Criticism e.g. Poor narrative, \\ lack of excitement, poor production or unengaging experience\end{tabular} & Kind of boring bearly any action \\ \cline{2-3}
& \begin{tabular}[c]{@{}l@{}}Content Features e.g. Visuals and graphics, Technical \\ and scientific features, Historical or factual aspects of the movie\end{tabular} & The graphics are... OUT OF THIS WORLD. \\ \cline{2-3}
& Mixed Opinions or Doubt e.g. Skepticism or criticism & \begin{tabular}[c]{@{}l@{}}The quality of this film is very good, but someone \\ should have checked for misinformation in the script.\end{tabular} \\ \midrule

% Shift in Cognition
\multirow{3}{*}{Shift in Cognition} & \begin{tabular}[c]{@{}l@{}}Space and Cosmic Science e.g. Orbital Mechanics, \\ Cosmic Learning\end{tabular} & \begin{tabular}[c]{@{}l@{}}Jupiter is a stunningly gorgeous planet, and \\ it's amazing to finally learn some of its secrets.\end{tabular} \\ \cline{2-3}
& \begin{tabular}[c]{@{}l@{}}Weather and Environmental Concerns e.g. Structural Safety, \\ Weather Knowledge, Environmental Impact\end{tabular} & \begin{tabular}[c]{@{}l@{}}It's about a terrible disaster to a community, \\ but also how they came back together.\end{tabular} \\ \cline{2-3}
& \begin{tabular}[c]{@{}l@{}}Criticism and Information Quality e.g. Misleading or Incorrect,\\  Lack of Clarity or New Insights\end{tabular} & \begin{tabular}[c]{@{}l@{}}I gave this 2 stars because they had \\ some facts wrong.\end{tabular} \\ \midrule

% Interest with Science Topic
\multirow{4}{*}{Interest with Science Topic} & \begin{tabular}[c]{@{}l@{}}Science and Technical Aspects e.g. Technical Expertise, \\ Weather Science, Science Exploration, Space Interest\end{tabular} & \begin{tabular}[c]{@{}l@{}}If observing and learning of the universe \\ is your thing, you will love this.\end{tabular} \\ \cline{2-3}
& \begin{tabular}[c]{@{}l@{}}Emotional and Experiential Reactions e.g. Uncertainty or \\ Suspense, Emotional Experience, Relaxation or Fatigue\end{tabular} & \begin{tabular}[c]{@{}l@{}}The first documentary leaves us hanging, will \\ Juno be able to do what its designers want it to do\end{tabular} \\ \cline{2-3}
& \begin{tabular}[c]{@{}l@{}}Content Quality and Presentation e.g. Media Coverage, \\ Storytelling, Interesting Content, Critical Perspective\end{tabular} & Best of all, it is based upon a true story. \\ \cline{2-3}
& \begin{tabular}[c]{@{}l@{}}Learning and Engagement e.g. Informative Enjoyment, \\ Progress and Improvement\end{tabular} & \begin{tabular}[c]{@{}l@{}}It's an interesting movie subject wise but if it \\ had better 3D, I be drawn to see it again.\end{tabular} \\ \midrule

% Impersonal Report
\multirow{3}{*}{Impersonal Report} & Tornado and Safety e.g. Building Safety, Tornado Consequences & \begin{tabular}[c]{@{}l@{}}This documentary describes positive building \\ code changes to help avert total demolition in a \\ tornado like this.\end{tabular} \\ \cline{2-3}
& \begin{tabular}[c]{@{}l@{}}Space and Astronomy e.g. Space Debris, \\ Solar Phenomena, Planetary Systems\end{tabular} & \begin{tabular}[c]{@{}l@{}}An hour of video images of solar observatory \\ and computer simulations of solar process and flares.\end{tabular} \\ \cline{2-3}
& \begin{tabular}[c]{@{}l@{}}Historical and Informational Content e.g. \\ Historical Events, Interviews and Reports\end{tabular} & \begin{tabular}[c]{@{}l@{}}An 1859 type event will occur again and cause \\ 1 to 2 Trillion dollars in damage\end{tabular} \\ \bottomrule
 
\end{tabular}}
\caption{Themes related to each Impact category}
\label{tab:Themes Impact}

\end{table*}





% Please add the following required packages to your document preamble:
% \usepackage[table,xcdraw]{xcolor}
% Beamer presentation requires \usepackage{colortbl} instead of \usepackage[table,xcdraw]{xcolor}
% \begin{table}[]
% {\fontsize{7}{8}\selectfont 
% \begin{tabular}{|l|l|l|}
% \hline
% Impact class & Themes & Examples \\ \hline
% Attitudes Toward the Film & \begin{tabular}[c]{@{}l@{}}Positive Engagement and Impact e.g. Positive feedback, \\ Desire for more similar content, \\ Describing the movie as engaging \& informative\end{tabular} & Very interesting and informative, liked it a lot. \\ \hline
% Attitudes Toward the Film & \begin{tabular}[c]{@{}l@{}}Negative Feedback and Criticism e.g. Poor narrative, \\ lack of excitement,  poor production or unengaging experience\end{tabular} & Kind of boring bearly any action \\ \hline
% Attitudes Toward the Film & \begin{tabular}[c]{@{}l@{}}Content Features e.g. Visuals and graphics, Technical \\ and scientific features, Historical or factual aspects of the movie\end{tabular} & The graphics are... OUT OF THIS WORLD. \\ \hline
% Attitudes Toward the Film & Mixed Opinions or Doubt e.g. Skepticism or criticism & \begin{tabular}[c]{@{}l@{}}The quality of this film is very good, but someone\\ should have checked for misinformation in the script.\end{tabular} \\ \hline
% Shift in Cognition & \begin{tabular}[c]{@{}l@{}}Space and Cosmic Science e.g. Orbital Mechanics, \\ Cosmic Learning\end{tabular} & \begin{tabular}[c]{@{}l@{}}Jupiter is a stunningly gorgeous planet, and \\ it's amazing to finally learn some of its secrets.\end{tabular} \\ \hline
% Shift in Cognition & \begin{tabular}[c]{@{}l@{}}Weather and Environmental Concerns e.g. Structural Safety, \\ Weather Knowledge, Environmental Impact\end{tabular} & \begin{tabular}[c]{@{}l@{}}It's about a terrible disaster to a community, \\ but also how they came back together.\end{tabular} \\ \hline
% Shift in Cognition & \begin{tabular}[c]{@{}l@{}}Criticism and Information Quality e.g. Misleading or Incorrect,\\  Lack of Clarity or New Insights\end{tabular} & \begin{tabular}[c]{@{}l@{}}I gave this 2 stars because they had \\ some facts wrong.\end{tabular} \\ \hline
% Interest with Science Topic & \begin{tabular}[c]{@{}l@{}}Science and Technical Aspects e.g. Technical Expertise, \\ Weather Science, Science Exploration, Space Interest\end{tabular} & \begin{tabular}[c]{@{}l@{}}If observing and learning of the universe\\ is your thing, you will love this.\end{tabular} \\ \hline
% Interest with Science Topic & \begin{tabular}[c]{@{}l@{}}Emotional and Experiential Reactions e.g. Uncertainty or\\ Suspense, Emotional Experience, Relaxation or Fatigue\end{tabular} & \begin{tabular}[c]{@{}l@{}}The first documentary leaves us hanging, will\\  Juno be able to do what its designers want it to do\end{tabular} \\ \hline
% Interest with Science Topic & \begin{tabular}[c]{@{}l@{}}Content Quality and Presentation e.g. Media Coverage, \\ Storytelling, Interesting Content, Critical Perspective\end{tabular} & Best of all, it is based upon a true story. \\ \hline
% Interest with Science Topic & \begin{tabular}[c]{@{}l@{}}Learning and Engagement e.g. Informative Enjoyment, \\ Progress and Improvement\end{tabular} & \begin{tabular}[c]{@{}l@{}}It's an interesting movie subject wise but if it \\ had better 3D, I be drawn to see it again.\end{tabular} \\ \hline
% Impersonal Report & Tornado and Safety e.g. Building Safety, Tornado Consequences & \begin{tabular}[c]{@{}l@{}}This documentary describes positive building \\ code changes to help avert total demolition in a\\ tornado like this.\end{tabular} \\ \hline
% Impersonal Report & \begin{tabular}[c]{@{}l@{}}Space and Astronomy e.g. Space Debris, \\ Solar Phenomena, Planetary Systems\end{tabular} & \begin{tabular}[c]{@{}l@{}}An hour of video images of solar observatory\\ and computer simulations of solar process and flares.\end{tabular} \\ \hline
% Impersonal Report & \begin{tabular}[c]{@{}l@{}}Historical and Informational Content e.g. \\ Historical Events, Interviews and Reports\end{tabular} & \begin{tabular}[c]{@{}l@{}}An 1859 type event will occur again and cause\\  1 to 2 Trillion dollars in damage\end{tabular} \\ \hline
% \end{tabular}
% \caption{Themes related to each Impact category}
% \label{tab:Themes Impact}
% }
% \end{table}

 
% \begin{table*}[]
% \begin{tabular}{|l|l|r|r|r|r|r|}
% \hline
%  & \textbf{Predicted Label} & \multicolumn{1}{l|}{} & \multicolumn{1}{l|}{} & \multicolumn{1}{l|}{} & \multicolumn{1}{l|}{} & \multicolumn{1}{l|}{} \\ \hline
% \textbf{True Label} &  & \multicolumn{1}{l|}{Attitudes} & \multicolumn{1}{l|}{Impersonal} & \multicolumn{1}{l|}{Interest} & \multicolumn{1}{l|}{NA} & \multicolumn{1}{l|}{Shift} \\ \hline
%  & Attitudes & 202 & 1 & 4 & 5 & 17 \\ \hline
%  & Impersonal & 3 & 4 & 0 & 1 & 4 \\ \hline
%  & Interest & 10 & 0 & 11 & 1 & 4 \\ \hline
%  & NA & 20 & 2 & 0 & 31 & 4 \\ \hline
%  & Shift & 8 & 0 & 1 & 1 & 32 \\ \hline
% \end{tabular}
% \caption{Confusion Matrix for True and Predicted Impact Categories: Attitudes (Attitudes Toward the Film), Impersonal (Impersonal Report), Interest (Interest with Science Topic), NA (Not applicable), and Shift (Shift in Cognition)}
% \label{tab:confusion matrix}
% \end{table*}



