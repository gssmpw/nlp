\section{Related Work}
Social media platforms are invaluable resources for understanding public sentiment and behavior, offering real-time access to users' expressed opinions and emotions. By analyzing vast amounts of social media data, researchers can uncover trends, gauge audience reactions, and explore complex phenomena like consumer behavior and societal attitudes~\cite{elalaoui2018sentiment,cortis2021socialopinion}. Platforms like Twitter, Reddit, and YouTube provide large-scale datasets that enable studies to capture real-time feedback and public sentiment at an unprecedented scale in domains such as politics~\cite{ANTYPAS2023100242, Yarchi15032021,rezapour2017identifying}, healthcare~\cite{10.1007/978-3-031-27409-1_41,9810923}, and education~\cite{electronics11050715,luo2020like}.
For instance, sentiment analysis of tweets related to significant global events or media productions, such as the Netflix documentary Our Planet, reveals insights into public sentiment and media impact. This analysis helps creators and researchers understand audience engagement and the potential societal influence of content~\citep{ieeexplore_10533989}. Similarly, Reddit's discourse, known for its depth and variety, serves as a valuable source for exploring nuanced discussions and opinions~\citep{asurveyof2012opinion,10.1145/3543873.3587324,10.3389/frai.2023.1163577,bouzoubaa2024euphoria}.

%However, these analyses are not without challenges. The inherent biases in social media data, such as demographic skew and echo chamber effects, can affect results. Furthermore, automated methods, while scalable, often face limitations in understanding contextual nuances compared to traditional qualitative methods like interviews and surveys \citep{wlv_openrepository, arxiv_2408.08694, arxiv_2407.13069}.

Sentiment analysis has been central to leveraging social media data for understanding public opinion~\cite{ijerph15112537, QIAN2022103098}. Advances in NLP and machine learning have significantly improved sentiment classification tasks. Traditional approaches like Support Vector Machines (SVM) and K-Nearest Neighbors (KNN) have given way to deep learning methods, which excel in handling large, complex datasets~\cite{BANSAL2022100071}. Models such as Long Short-Term Memory (LSTM), Bidirectional LSTM (BiLSTM), and transformers like BERT and RoBERTa have achieved state-of-the-art performance in sentiment analysis~\citep{ieeexplore_9716923, ieeexplore_8629198, kapur2022sentimentanalysis}.
Recent comparisons of machine learning models for social media sentiment analysis reveal that deep learning approaches consistently outperform traditional algorithms. RoBERTa-LSTM hybrids, for instance, demonstrate exceptional accuracy and robustness in Twitter sentiment classification~\citep{ieeexplore_10331232}. These models' ability to understand the context and handle large datasets has made them indispensable for analyzing public opinion across diverse media platforms~\citep{arxiv_2408.08694, ieeexplore_10533989, arxiv_2407.13069}.

With the emergence of LLMs such as GPT and BERT-based architectures, sentiment analysis has evolved further, leveraging pre-trained models to understand complex linguistic patterns and contextual subtleties\cite{sayeed2023bert,chang2024survey}. These models enable transfer learning, allowing fine-tuning on specific domains to achieve superior performance compared to traditional and earlier deep learning methods~\cite{10.1145/3543873.3587324, arxiv_2407.13069}. 
Studies emphasize the importance of domain-specific tuning of NLP models. For example, the application of sentiment analysis to social media reviews of environmental documentaries or films not only provides feedback on content reception but also sheds light on broader societal attitudes towards environmental issues~\citep{acerbi2023sentiment}.
Opinion mining goes a step further than sentiment analysis by focusing on specific aspects of text that inform user perspectives. This fine-grained analysis has been applied to social media and online discourse, uncovering detailed insights into user opinions on products, services, or cultural phenomena~\citep{asurveyof2012opinion, kapur2022sentimentanalysis, Gerard_Botzer_Weninger_2023}.
For instance, \citep{kapur2022sentimentanalysis} employed BiLSTM combined with a random forest classifier to identify sentiments across multiple platforms, achieving high accuracy. 

Film reviews provide a specifically rich ground for sentiment analysis and opinion mining \cite{malini2019opinion}. They are rich in emotion, context, and critique, making them ideal for studying user opinions and their motivations. Existing research has predominantly focused on sentiment polarity, identifying whether reviews are positive, negative, or neutral. While effective, this binary/ternary classification often misses the granular details of user opinions~\citep{Mrabti2024AnEM, 10.5120/ijca2017916005}.
Researchers have called for extending these efforts to include opinion mining, which examines the thematic and contextual dimensions of reviews. For instance, \citep{rezapour2017classification} introduced ``Impact'' categories to analyze how specific aspects of films influence people's cognition and attitude. This nuanced approach revealed critical factors driving engagement, such as emotional resonance and storytelling quality. In addition, \cite{yue2019survey} emphasizes the importance of incorporating fine-grained sentiment analysis to move beyond simple polarity classification. This involves analyzing not only whether reviews are positive or negative but also understanding the rich contextual and emotional dimensions that shape user opinions. By capturing these subtleties, researchers can uncover deeper insights into audience preferences and motivations, ultimately enabling more tailored and effective content strategies.

Building on this body of work, %our research aims to advance the application of NLP and machine learning techniques to film reviews, emphasizing both sentiment and impact analysis. B
by adopting a fine-grained approach, we seek to identify not only the overall sentiment but also the specific aspects of scientific films that shape public emotions and opinions. This holistic understanding will contribute to a deeper comprehension of user feedback, guiding content creators and industry stakeholders in optimizing their offerings.

%\subsection{NLP for Media Analysis}
%Gauging general public sentiment on media is a field with many recent contributions.
% Analyzing large volumes of online discourse with machine learning methods has become a goal for numerous researchers. 
% Sentiment analysis, in which statements are given tags of ``positive", ``negative" or ``neutral", of YouTube comments has been used to quantify public opinion on media content, though it has limitations compared to traditional methods like interviews and surveys \citep{wlv_openrepository, arxiv_2408.08694, arxiv_2407.13069}. Social media platforms offer a wealth of data for sentiment analysis, allowing insights into audience responses at scale \rtc{yue2019survey}.
% Large-scale analysis of over 2 million Twitter tweets about Our Planet \cite{our_planet_trailer} \rtc{need a reference for this -- what is Our Planet? is it a movie or something else?} shows how real-time public sentiment can be captured to study media impact, providing valuable feedback for content creators and conservationists \rtc{where is the conbio\_wiley reference?} \citep{(acerbi2023sentiment)conbio_wiley, ieeexplore_10533989}. 

% Machine learning models like SVM, KNN, and advanced techniques like RoBERTa-LSTM~\rtc{citation needed} have been compared for sentiment analysis, showing that deep learning approaches often outperform traditional methods in understanding social media sentiment \citep{ieeexplore_9716923, ieeexplore_8629198, ieeexplore_10331232}. These techniques are particularly effective in handling large datasets like Twitter reviews. 
% Sentiment analysis using natural language processing (NLP) tools has been applied to various media types across platforms, revealing trends and public reactions, with studies showing that both traditional and transformer-based models can be used to analyze sentiment effectively 
% \citep{arxiv_2408.08694, ieeexplore_10533989, arxiv_2407.13069}. 

% \rt{Other platforms, such as Reddit, also provide a rich source of data for media analysis. (how is this sentence related to what follows? is there a reference for it? What originally followed this sentence is commented out below as its been moved around.}

% % \begin{easylist}[itemize] 
 
% % & Machine learning models like SVM, KNN, and advanced techniques like RoBERTa-LSTM~\rtc{citation needed} have been compared for sentiment analysis, showing that deep learning approaches often outperform traditional methods in understanding social media sentiment \citep{ieeexplore_9716923, ieeexplore_8629198, ieeexplore_10331232}. These techniques are particularly effective in handling large datasets like Twitter reviews. 
% % & Sentiment analysis using natural language processing (NLP) tools has been applied to various media types across platforms, revealing trends and public reactions, with studies showing that both traditional and transformer-based models can be used to analyze sentiment effectively 
% % \citep{arxiv_2408.08694, ieeexplore_10533989, arxiv_2407.13069}. 

% % \end{easylist}

% Research in opinion mining, which can be used for both sentiment analysis and a finer-grained NLP analysis of words within statements, has been particularly robust in analyzing content on social media and other online discourse \citep{asurveyof2012opinion, kapur2022sentimentanalysis, Gerard_Botzer_Weninger_2023, Guerra_2023}. \citep{kapur2022sentimentanalysis} found particular success using Bidirectional Long Short-Term Memory (BiLSTM) and a random forest classifier to analyze sentiments in social media content across multiple platforms. 


% % \subsection{NLP for Review Analysis}

% In particular, sentiment analysis on film reviews has become an area of interest to many researchers \citep{bhola2022hyrbidframework, Mrabti2024AnEM, 10.5120/ijca2017916005, bodapati2019sentiment}. These studies, however, often focus on only identifying the reviews’ polarities (whether they are positive or negative) and do not further examine their content \citep{Mrabti2024AnEM, 10.5120/ijca2017916005}. Long Short-Term Memory (LSTM) has proven to be particularly effective in the completion of these review analysis tasks \citep{bhola2022hyrbidframework, bodapati2019sentiment}. 
% Opinion mining which examined the specific aspects of a film which may influence its effectiveness at engaging the public was conducted in  \cite{rezapour2017classification} with the addition of ``Impact" categories along with Sentiment categories.  In what follows, we build on this work for finer-grained analysis of scientific documentary reviews.
%Further opinion mining examining the specific aspects of a film influencing an author's choice to rate it as negative or positive would offer a more complete understanding of a consumer's perspectives. 


% \begin{easylist}[itemize]
%     & we'll start by looking at the cit`ations to \citep{rezapour2017classification}
%     && check out the URL \url{https://scholar.google.com/scholar?cites=17684968807680716583&as_sdt=400005&sciodt=0,14&hl=en}
%     && an example we might want to check out (this is only a test) is \citep{sherren2017digital}
%     && Ravinder Ahuja et al. 

%& \citep{kapur2022sentimentanalysis} use similar methodologies (SVM and LSTM) to analyze sentiment across social media platforms. 

%& \citep{bhola2022hyrbidframework} use similar methods (BERT and LSTM) and also consider a corpus of film reviews. 

%& \citep{Mrabti2024AnEM} also make use of machine learning methods to analyse online reviews, including a corpus of film reviews. 

%& \citep{bodapati2019sentiment} uses LSTM to classify film reviews as either positive or negative. 

% & \citep{asurveyof2012opinion} introduces methods for opinions mining, highlighting social media as key sources of opinionated content for sentiment analysis.

%& \citep{Gerard_Botzer_Weninger_2023} uses dataset from social media to analyze user engagement and the ways social platforms shape users' behavior.

%& \citep{Guerra_2023} uses word clouds and sentiment analysis approach to analyze news/posts sentiment from social media platform.

%& \citep{kapur2022comparativestudysentimentanalysis} uses similar methods (LSTM) and other machine learning techniques to analyze sentiment for social media platforms.


% & Some other useful references for folks to check out (if you haven't yet):
% && A SURVEY OF OPINION MINING AND SENTIMENT ANALYSIS - \url{https://doi.org/10.1007/978-1-4614-3223-4_13}
% && Dataset paper: Truth Social Dataset - \url{https://ojs.aaai.org/index.php/ICWSM/article/view/22211}
% && Sentiment Analysis for Measuring Hope and Fear from Reddit Posts During the 2022 Russo-Ukrainian Conflict - \url{http://arxiv.org/abs/2301.08347}
% && COMPARATIVE STUDY OF SENTIMENT ANALYSIS FOR MULTI-SOURCED SOCIAL MEDIA PLATFORMS - \url{http://arxiv.org/abs/2212.04688}
    
%\end{easylist}