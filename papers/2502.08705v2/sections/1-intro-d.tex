\section{Introduction}
Communicating science is an important goal of many societies and institutions \citep[e.g.,][]{national2017communicating}. 
Scientific documentaries have long served as an important communication medium for disseminating scientific knowledge to the general public. They bridge the gap between complex scientific concepts and public understanding, often influencing public opinion and policy on critical issues such as climate change, public health, and technology~\citep{barnes2008,punzo2015,vogt2016,borkiewicz2019cinematic,ytini}. 
Cinematic treatments of data visualization in the field of ``Cinematic Scientific Visualization'' (CSV) \citep{borkiewicz2022introducing} such as documentaries and IMAX films can be especially effective in engaging the general public \citep{dubeck2004fantastic,arroio2010,franconeri2021science,lee2022affective}.
From a media studies perspective, the narrative techniques and cinematographic strategies employed in documentaries can significantly enhance viewer engagement and retention of information~\cite{barrett2008assessing,yeo2018inconvenient,saputra2022impactful}. 
Research shows that the storytelling aspect of documentaries can lead to greater empathetic understanding and cognitive retention of scientific facts~\cite{ginting2024effects,gaunkar2022exploring}. The emotional responses elicited by documentaries are seen as key drivers for behavioral change and advocacy, influencing how individuals and communities respond to scientific challenges like environmental conservation or disease prevention~\cite{bieniek2019communicating}.

The impact of media was studied from the perspective of sociology, examining how documentaries reinforce or challenge cultural norms and societal structures~\cite{rezapour2017classification,diesner2015social}. The results show that documentaries can act as catalysts for social change by highlighting underrepresented issues and providing a platform for marginalized voices. This fosters a more inclusive public dialogue around scientific and technological advancements~\cite{bouzoubaa2024euphoria,conrad2022breaking,diesner2016assessing}.
These impacts extend beyond individual viewers to influence societal norms and values. By presenting scientific issues within relatable contexts, documentaries can alter public discourse, catalyze community actions, and even shift policy directions~\cite{atakav2024impact}. Moreover, the reach and accessibility of documentaries have been vastly expanded by advances in digital technology and online streaming platforms, which allow for unprecedented dissemination and engagement across varied global audiences.

%Measuring the `impact'' of scientific documentaries requires multimodal efforts.There are many decisions that designers of cinematic treatments must contend with in order to make an effective CSV presentation \citep{woodward2015one,borkiewicz2022introducing,jensen2023evidence} and gauging audience attitudes and cognition can be difficult to do with precision, requiring one-on-one interviews and qualitative analysis \citep{chen2005top,cawthon2007effect,buck2013effect,fraser2012giant,smith2015aesthetics,smith2017capturing,jensen2023evidence,jenseninprep}. Thus, quantitative approaches can be used to complement in-depth qualitative analyses in order to better understand the impact of scientific media. This approach can also address some  shortcomings to prior work is that it often primarily relied on manual qualitative analysis, which is not only labor-intensive but also subject to bias and scalability issues~\cite{bieniek2019communicating,atakav2024impact,barrett2008assessing}.
Measuring the ``impact'' of scientific documentaries requires a multifaceted approach that integrates both qualitative and quantitative methods. 
%
Prior studies have demonstrated that filmmakers and designers of CSV presentations must navigate numerous creative and technical decisions to ensure their effectiveness \citep{woodward2015one,borkiewicz2022introducing,jensen2023evidence}.
%
Traditionally, assessing audience attitudes and cognition has relied on in-depth qualitative techniques such as interviews and thematic analysis, which, while insightful, can be labor-intensive and subject to bias \citep{chen2005top,cawthon2007effect,buck2013effect,fraser2012giant,smith2015aesthetics,smith2017capturing,jensen2023evidence,jenseninprep}. 
Additionally, the reliance on manual qualitative analysis in prior research has posed challenges related to efficiency and reproducibility \citep{bieniek2019communicating,atakav2024impact,barrett2008assessing}.

To address these limitations, quantitative approaches can serve as valuable complements, enhancing precision and scalability in impact assessments.
By integrating quantitative methods, researchers can overcome the constraints of qualitative-only studies, leading to a more comprehensive and rigorous evaluation of scientific media's influence.
With the emergence of digital platforms, viewers share their immediate reactions and detailed thoughts through online reviews, providing a rich dataset for analyzing public engagement and understanding~\cite [e.g.,][]{APPEL2016110}.
The interactivity provided by online platforms enhances the dissemination process, as viewers can discuss and share content, thereby amplifying the documentary's reach and impact. This engagement is measurable through the analysis of online reviews and social media commentary, which can serve as a feedback loop for content creators and scientists alike, indicating which aspects of their presentation meet the audience's needs and which do not~\cite{rezapour2017classification,bouzoubaa2024euphoria}.


%Previous work showed that while online engagements offer insights into viewer reception, they often lack a systematic approach to quantify the depth and spectrum of influence exerted by documentaries on scientific understanding and changes in cognition and attitude~\citep{rezapour2017classification}. 
% An additional shortcoming to prior work is that it often primarily relied on manual qualitative analysis, which is not only labor-intensive but also subject to bias and scalability issues~\cite{bieniek2019communicating,atakav2024impact,barrett2008assessing}.
We extend previous work by computationally analyzing the impact of scientific films using user-generated reviews and measuring the impact using a novel taxonomy of change and engagement in conjunction with a systematic approach to manual annotations. This approach employs natural language processing (NLP) techniques to systematically categorize and quantify viewer responses, thus providing a more scalable and objective measure of the documentaries' effectiveness in enhancing scientific literacy and influencing viewers' behavior. More specifically, we answer the following questions: 
 \setlist{nolistsep}
    \begin{itemize}[noitemsep]
    \item[-] {\textbf{RQ1:}} What are the most prevalent impacts of scientific documentaries on viewer cognition, attitudes, and interests? 
    \item[-] {\textbf{RQ2}:} What, if any, are some common themes in viewer cognition, attitude, and interests surrounding scientific documentaries?
\end{itemize}

To answer these questions, we extract reviews from Amazon for \nmoviesword\ prominent scientific documentaries and evaluate the tone, content, and thematic elements of viewer feedback. This data is further analyzed using our developed taxonomy, which categorizes impacts into ``Shift in Cognition (C)'', ``Attitudes Toward the Film (A)'', ``Interest with Science Topic (S)'', ``Impersonal report (I)'', and ``Not applicable (N)''.
Our data analysis shows that expressing different ``Attitudes Toward the Film'' is the most common \impact\ type reported in the reviews. 
The classification results suggest that large language model (LLM)-based classifiers, especially when the full context of the review is added to the prompt, are capable of successfully identifying \impact\ categories and their performance is generalizable to other datasets. Additionally, our thematic analysis of \impact\ groups demonstrates a diverse range of themes in the reviews. Viewers %express positive, negative, or neutral feedback 
used mixed sentiments to show their attitude toward the film, while expressing environmental concerns or celebrating cosmic science were some of the themes related to shifts in cognition. 

Our paper makes the following contributions: (1) providing annotated data on impact, offering a dataset that has been carefully categorized and labeled to reflect the various types of influence that scientific documentaries have on public perception and interest\footnote{\url{https://huggingface.co/datasets/scidoc/websci2025}}; (2) introducing a novel taxonomy of engagement, which enables a structured analysis of how documentaries affect cognitive, affective, and interest dimensions of their audiences; and (3) employ advanced computational techniques to analyze viewer responses in a scalable and unbiased manner. 
These contributions enhance our understanding of the educational and societal impacts of scientific documentaries. The paper also provides a framework for future research to build upon, improving the effectiveness and reach of science communication through visual media.

% \rt{This section needs to motivate why our dataset (and bench mark models if we do that too) are important. Some things could be:}
% \begin{easylist}[itemize]
%     & \rt{Understanding and quantifying the impact of science communication is vital in the age of misinformation (CITE?).}
%     & \rt{While in depth interviews of subjects with a wide variety of ages and backgrounds are imperative to gauge understanding of scientific topics, such interview campaigns are monitarily and temporally expensive (CITE?).}
%     & \rt{Impact and sentiment analysis (CITE) for the basis for ...}
% \end{easylist}