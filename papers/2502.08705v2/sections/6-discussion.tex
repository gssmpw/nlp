\section{Discussion}

In conversation with the \avlshort\, several of the interests the group emphasized were (1) a desire to see a breakdown between positive and negative views of their documentaries, (2) an understanding of what film attributes might drive a viewer's opinion of the film, and (3) that, as often engagement in and of itself is a goal, what measurements of audience engagement might be evident in comments.
These interests align with our two research questions.

\vspace{0.2cm}
\noindent\textbf{RQ1: The most prevalent impacts of scientific documentaries. } 
Our findings highlight the substantial role that scientific documentaries play in shaping audience engagement and attitudes. The predominance of positive sentiments, particularly those linked to ``Attitudes Toward the Film'' (37\% of all sentences), suggests that these films successfully resonate with their audiences. This is likely due to the compelling storytelling and production quality typical of impactful documentaries. This supports existing literature on the ability of documentary films to evoke emotional responses and create a sense of connection between viewers and experts \cite{nisbet2009documentary,corner2002performing}.

The disparity between ``Positive''/``Attitudes Toward the Film'' (37\%) and ``Negative''/``Attitudes Toward the Film'' (21.9\%) further underscores a general trend of audience favorability. This aligns with findings that highlight how carefully crafted documentaries can frame scientific content in ways that promote optimism, curiosity, and a sense of wonder about the natural or scientific world \cite{gregory1998science}. The approximately $\sim$1.7 greater prevalence of positive over negative sentiments reflects the potential of documentaries to act as persuasive and motivational tools, aligning with interests (1) and (3) of the \avlshort, focusing on emotional resonance and knowledge dissemination.
%
``Interest with Science Topic'' (5.2\%) and ``Shift in Cognition'' (6.9\%) prevalence, as significant secondary impacts, indicate that these films go beyond entertainment, fostering intellectual engagement and engaging with viewer perspectives. This mirrors findings from studies on science communication that emphasize the dual purpose of documentaries: sparking curiosity and promoting critical thinking \cite{dahlstrom2014using}.
The fact that these impact types, although not the most prevalent, still contribute to a notable portion of the dataset shows that documentaries can serve as informal educational tools, stimulating deeper engagement with scientific topics and potentially influencing public understanding of science.

% moving to conclusions
% Overall, our results underscore the multifaceted impact of scientific documentaries. They engage audiences emotionally, provoke intellectual curiosity, and influence perceptions, making them a powerful medium for science communication. Our findings contribute to a growing body of evidence on the effectiveness of media in bridging the gap between science and the public, emphasizing the value of leveraging documentaries to achieve broader educational and societal goals. Future research could explore the long-term impacts of such engagement, particularly in terms of sustained interest in science and changes in behavior or policy advocacy.


% As evidenced in \autoref{fig:sentiment_impact} and \autoref{tab:movie_info}, the most prevalent sentiment and impacts are ``Positive'' and ``Attitudes Toward the Film'' -- with 37\% of all sentences in this combination of categories.
%

%As shown in \autoref{tab:F1 per class},  the category ``Attitudes Toward the Film'' achieved the highest F1 score(0.86) among all impact categories, highlighting the prevalence of emotional and evaluative responses in viewer comments.

% This indicates comments lean toward engagement with the film material in a positive fashion, at a rate $\sim$1.7 times higher than the next sentiment/impact category combination of ``Negative''/``Attitudes Toward the Film'' at 21.9\% of all sentences.  These results align with interests (1) and (3) of the \avlshort.  

% The next largest sentiment/impact categories are ``Positive'' and ``Interest with Science Topic'' and ``Shift in Cognition'' (5.2\% and 6.9\%, respectively), suggesting a significant contribution from comments engaging with the scientific material within the documentaries (interest (1) and (3) of the \avlshort).


% Based on the results shown in the \autoref{tab:Hubble_Prediction}, ``Positive'' sentiment dominates with 250 predictions and ``Attituddes Toward the Film'' impact dominates 307 prediction, suggesting alignment with the observed prevalence of ``Positive'' and ``Attitudes Toward the Film'' in the previous discussion, further supporting the ability of the film materials to engage audiences emotionally. 

% These findings illustrate that scientific documentaries primarily impact viewers' attitudes along with the noticeable influence on cognitive interests and reflection. By waving emotional narratives into the scientific themes, the documentaries we investigated in succeed in transforming the professional topics engaging and understandable.


\vspace{0.2cm}
\noindent\textbf{RQ2: Common themes and discourse in viewers' feedback.}
As indicated in \autoref{tab:Themes Impact}, there are several themes that align with each impact category.  For example, attitudes toward the film which lean negative often mention poor narration or production quality or a lack of excitement, while positive attitudes mention a desire for similar content and that the content is engaging and informative.  These thematic trends align with interest (2) of the \avlshort. 

%moved from annotation:
Additionally, many themes listed in \autoref{tab:Themes Impact} involve engagement as well as the visualization/technical aspects of the films, aligning with the interest (3) of the \avlshort. 
Not only is engagement talk more evidence of learning, but it is also a building block that can lead to behaviors with prosocial impact. These behaviors include de-stigmatizing and sharing science information with the community in informal environments, improving personal science identity (which can lead to the growth of future science professionals), and advocating for more research funding, political support, and science-motivated decision-making \citep[e.g.,][]{lee_robbins_affective_2022}.

The themes identified further highlight the dual role of media and documentary films: as both educational tools and catalysts for societal discourse. Research has shown that media content with high production quality and strong narrative elements fosters emotional engagement, which can enhance message retention and stimulate critical thinking among viewers \citep{borkiewicz2019cinematic,arroio2010,franconeri2021science,lee2022affective}. This underscores the importance of persuasive storytelling and high-quality visuals, as they play a critical role in bridging the gap between scientific content and public understanding.
Moreover, the recurring themes of engagement and learning reflect the capacity of documentaries to create shared cultural moments that go beyond individual experiences. Studies suggest that collaborative viewing and subsequent discussions can amplify the influence of documentaries, encouraging viewers to reflect on social norms and consider alternative perspectives \cite{whiteman2009documentary,hirsch2007documentaries}.

Finally, the themes show the potential for documentaries to reinforce or challenge preconceived notions about science. For instance, when technical visualization aspects are well-executed, they not only enhance comprehension but also demystify complex scientific processes, making science appear more accessible and inclusive \cite{yang2020power}. These findings align with the broader literature emphasizing that media's impact is mediated not just by content but also by how that content resonates with and activates viewers' values, beliefs, and aspirations \cite{rezapour2017classification}. 


%RQ2: %How do the modes of presentation (narrative style and visual imagery) in scientific documentaries influence viewer reception and understanding?

% The high F1 score (0.86) of the ``Attitudes Toward the Film'' category highlights how narrative style plays a central role in shaping how viewers respond to documentaries. Insights from this category \autoref{tab:Themes Impact} show that viewers often comment on elements like the pacing, tone, and emotional impact of the storytelling. For example, comment, ``Very interesting and informative, liked it a lot'' suggests that a well-crafted narrative can effectively evoke positive emotions and enhance audiences' overall experience.

% The visual imagery also plays a vital role in impacting the audience feedback towards the scientific films. The comment ``It's an interesting movie subject wise but if it had better 3D, I be drawn to see it again'' in the ``Interest with Science Topic'' category indicates the high-quality visuals can attract audience and lead the viewers dive deeper in scientific topics. However, the F1 score (0.52) of this category suggests that while audience may value the visuals of the documentaries, their ability to influence viewers' reception and understanding is limited.