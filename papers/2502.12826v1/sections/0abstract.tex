\begin{abstract}

As the memory demands of individual mobile applications continue to grow and the number of concurrently running applications increases, available memory on mobile devices is becoming increasingly scarce. 
When memory pressure is high, current mobile systems use a RAM-based compressed swap scheme (called ZRAM) to compress unused execution-related data (called anonymous data in Linux) in main memory. 
This approach avoids swapping data to secondary storage (NAND flash memory) or terminating applications, thereby achieving shorter application relaunch latency.

In this paper, we observe that the state-of-the-art ZRAM scheme prolongs relaunch latency and wastes CPU time because it does \emph{not} differentiate between hot and cold data or leverage different compression chunk sizes and data locality.
We make three new observations. First, anonymous data has different levels of hotness. Hot data, used during application relaunch, is usually similar between consecutive relaunches.
Second, when compressing the same amount of anonymous data, small-size compression is very fast, while large-size compression achieves a better compression ratio.
Third, there is locality in data access during application relaunch.

Based on these observations, we propose a hotness-aware and size-adaptive compressed swap scheme, \proposal, for mobile devices to mitigate relaunch latency and reduce CPU usage. \proposal incorporates three key techniques.
First, a low-overhead hotness-aware data organization scheme aims to quickly identify the hotness of anonymous data without significant overhead.
Second, a size-adaptive compression scheme uses different compression chunk sizes based on the data's hotness level to ensure fast decompression of hot and warm data. 
Third, a proactive decompression scheme predicts the next set of data to be used and decompresses it in advance, reducing the impact of data swapping back into main memory during application relaunch.
 
We implement and evaluate \proposal on a commercial smartphone,
Google Pixel 7 with the latest Android 14.
Our experimental evaluation results show that, on average, \proposal reduces application relaunch latency by 50\% and decreases the CPU usage of compression and decompression procedures by 15\% compared to the state-of-the-art compressed swap scheme for mobile devices.

\end{abstract}