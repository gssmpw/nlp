\subsection{Compatibility of \proposal with Other Techniques}
\label{sec:implement}
We discuss the compatibility of \proposal with: i) different compression algorithms and ii) other memory management schemes (e.g., memory allocation and memory reclamation schemes).


\noindent\textbf{Compression algorithms.}
It is valuable for systems to support multiple compression algorithms to cater to different objectives. Therefore, ensuring compatibility with a range of compression algorithms is important.
\proposal is compatible with various compression algorithms, such as  LZO~\cite{LZO}, LZ4~\cite{LZ4}, and base-delta compression{\cite{pekhimenko2012base, pekhimenko2013linearly}.
\proposal naturally supports different compression algorithms, such as switching between LZO and LZ4, as it inherits \emph{ZRAM}'s interface.  Compression algorithms might require slight interface modifications to support \compress by adjusting the compression chunk size.



\noindent\textbf{Impact on other memory management schemes.}
\proposal is compatible with various baseline memory management algorithms, such as default memory allocation and memory reclamation techniques used in modern systems. \proposal does \emph{not} impact the memory allocation procedure because memory allocation operates only on available memory space, while \proposal focuses on organizing data.  However, \proposal affects the memory reclamation scheme. Specifically, in a mobile system, the memory reclamation scheme selects data to reclaim (called victim data) based on the LRU policy. In contrast, the memory reclamation scheme of \proposal uses our hotness-aware data organization scheme (\dataorg) to select victim data, enhancing reclamation efficiency by prioritizing the reclamation of cold data.
\proposal is also compatible with swap scheme optimizations, such as MARS~\cite{guo2015mars}, FlashVM~\cite{saxena2010flashvm}, Fleet~\cite{end2024more}, and SmartSwap~\cite{zhu2017smartswap}, as \proposal is complementary to them.




 