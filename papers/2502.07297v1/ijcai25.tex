%%%% ijcai25.tex

\typeout{IJCAI--25 Instructions for Authors}

% These are the instructions for authors for IJCAI-25.

\documentclass{article}
\pdfpagewidth=8.5in
\pdfpageheight=11in

% The file ijcai25.sty is a copy from ijcai22.sty
% The file ijcai22.sty is NOT the same as previous years'
\usepackage{ijcai25}

% Use the postscript times font!
\usepackage{times}
\usepackage{soul}
\usepackage{url}
\usepackage[hidelinks]{hyperref}
\usepackage[utf8]{inputenc}
\usepackage[small]{caption}
\usepackage{graphicx}
\usepackage{amsmath}
\usepackage{amsthm}
\usepackage{booktabs}
\usepackage{algorithm}
\usepackage{algorithmic}
\usepackage[switch]{lineno}
\usepackage{natbib}
\usepackage{multirow}
\usepackage{xcolor}
\usepackage{colortbl}
\usepackage{amssymb}
\usepackage{bbding}
\usepackage{siunitx}


% Comment out this line in the camera-ready submission
% \linenumbers

\urlstyle{same}

% the following package is optional:
%\usepackage{latexsym}

% See https://www.overleaf.com/learn/latex/theorems_and_proofs
% for a nice explanation of how to define new theorems, but keep
% in mind that the amsthm package is already included in this
% template and that you must *not* alter the styling.
\newtheorem{example}{Example}
\newtheorem{theorem}{Theorem}

% Following comment is from ijcai97-submit.tex:
% The preparation of these files was supported by Schlumberger Palo Alto
% Research, AT\&T Bell Laboratories, and Morgan Kaufmann Publishers.
% Shirley Jowell, of Morgan Kaufmann Publishers, and Peter F.
% Patel-Schneider, of AT\&T Bell Laboratories collaborated on their
% preparation.

% These instructions can be modified and used in other conferences as long
% as credit to the authors and supporting agencies is retained, this notice
% is not changed, and further modification or reuse is not restricted.
% Neither Shirley Jowell nor Peter F. Patel-Schneider can be listed as
% contacts for providing assistance without their prior permission.

% To use for other conferences, change references to files and the
% conference appropriate and use other authors, contacts, publishers, and
% organizations.
% Also change the deadline and address for returning papers and the length and
% page charge instructions.
% Put where the files are available in the appropriate places.


% PDF Info Is REQUIRED.

% Please leave this \pdfinfo block untouched both for the submission and
% Camera Ready Copy. Do not include Title and Author information in the pdfinfo section
\pdfinfo{
/TemplateVersion (IJCAI.2025.0)
}

% \title{DADM: Drug-Aware Diffusion Model for Simulating Cardiac Drug Reactions}
\title{Generation of Drug-Induced Cardiac Reactions towards Virtual Clinical Trials}


% % Single author syntax
% \author{
%     Author Name
%     \affiliations
%     Affiliation
%     \emails
%     email@example.com
% }

% % Multiple author syntax (remove the single-author syntax above and the \iffalse ... \fi here)
% \iffalse
\author{
Qian Shao$^1$\thanks{These authors contributed equally to this work.}
\and
Bang Du$^1$\footnotemark[1]
\and
Zepeng Li$^1$
\and
Qiyuan Chen$^1$
\and
Hongxia Xu$^1$\and\\
Jimeng Sun$^3$
\and
Jian Wu$^1$\thanks{Corresponding authors.}
\and
Jintai Chen$^2$\footnotemark[2]\\
\affiliations
$^1$Zhejiang University\\
$^2$The Hong Kong University of Science and Technology\\
$^3$University of Illinois Urbana-Champaign\\
\emails
\{qianshao, bangdu1994\}@zju.edu.cn, \{wujian2000, jtchen721\}@gmail.com
}
% \fi

\begin{document}

\maketitle

\begin{abstract}
Clinical trials are pivotal in cardiac drug development, yet they often fail due to inadequate efficacy and unexpected safety issues, leading to significant financial losses. Using in-silico trials to replace a part of physical clinical trials, \textit{e.g.}, leveraging advanced generative models to generate drug-influenced electrocardiograms (ECGs), seems an effective method to reduce financial risk and potential harm to trial participants. While existing generative models have demonstrated progress in ECG generation, they fall short in modeling drug reactions due to limited fidelity and inability to capture individualized drug response patterns. In this paper, we propose a Drug-Aware Diffusion Model (DADM), which could simulate individualized drug reactions while ensuring fidelity. To ensure fidelity, we construct a set of ordinary differential equations to provide external physical knowledge (EPK) of the realistic ECG morphology. The EPK is used to adaptively constrain the morphology of the generated ECGs through a dynamic cross-attention (DCA) mechanism. Furthermore, we propose an extension of ControlNet to incorporate demographic and drug data, simulating individual drug reactions. We compare DADM with the other eight state-of-the-art ECG generative models on two real-world databases covering $8$ types of drug regimens. The results demonstrate that DADM can more accurately simulate drug-induced changes in ECGs, improving the accuracy by at least $5.79\%$ and recall by $8\%$.
\end{abstract}

\section{Introduction}

Cardiac diseases impose a significant burden globally, with drug development in this domain facing disproportionate challenges compared to other therapeutic areas~\citep{fordyce2015cardiovascular}.
A key bottleneck lies in the exorbitant costs of clinical trials, compounded by high attrition rates across development phases~\citep{dimasi2016innovation}.
Previous analyses show that $79\%$ of failures stem from inadequate efficacy or unforeseen safety issues~\citep{dowden2019trends}, highlighting the critical need for improved predictive tools during early-stage development.

Leveraging generative models to simulate electrocardiogram (ECG) changes induced by drugs is an effective way to conduct virtual clinical trials.
As illustrated in Figure~\ref{i1}, this approach harnesses limited physical trial data to train generative models capable of simulating diverse pharmacological responses, thereby mitigating financial risks.


\begin{figure}[tbp]
    \centering
    \includegraphics[width=1\linewidth]{images/i1-cropped.pdf}
    \caption{Physical clinical trials v.s. virtual clinical trials. In realistic scenarios, we aim to train the DADM for virtual trials with limited physical trial data, thereby avoiding financial and safety risks associated with large-scale clinical trials.}
    \label{i1}
\end{figure}

While existing works have demonstrated progress in ECG signal synthesis~\citep{simone2023ecgan,chung2023text,li2024biodiffusion}, two challenges persist when adapting these methods to pharmacodynamic ECG modeling.
The first challenge lies in \textbf{ensuring that the generated ECGs exhibit realistic PQRST morphology}, a fundamental requirement in ECG generation tasks~\citep{kaplan1990simultaneous,davey1999new}.
Previous studies have introduced ordinary differential equation (ODE) systems to enhance the authenticity of generated signals, achieving notable advancements.
For example, \citet{mcsharry2003dynamical} represents a dynamical model by a system of three ODEs to generate ECGs with realistic PQRST morphology, as well as prescribed heart rate dynamics.
Building on this foundation, \citet{golany2020simgans,golany2021ecg} design SimGAN and ECG ODE-GAN, respectively.
They leverage ODEs as External Physical Knowledge (EPK) to constrain the generator's output, ensuring fidelity.
However, these methods cannot control the restricting strength of EPK on ECG generation as the time steps iterate, which may undermine the generation quality (see Section~\ref{as}).
The second challenge lies in \textbf{simulating cardiac drug reactions on ECGs from diverse subjects at various post-dose time points}, an area that remains underexplored.


In this paper, we design a Drug-Aware Diffusion Model (DADM), a novel deep-learning framework capable of generating post-dose ECGs based on pre-dose ECGs and clinical information.
To generate realistic ECGs, we first construct a set of ODEs to describe EPK~\citep{mcsharry2003dynamical}.
Then, we propose a dynamic cross-attention (DCA) mechanism for the denoising process, which can adaptively adjust the constraints imposed by the EPK according to the time steps.
In the initial generation stage, the model should rely more on EPK to generate basic waveforms. In the later stage, the external control should be reduced to generate more detailed data.
To accurately simulate the individual cardiac drug reactions on ECGs, we extend ControlNet~\citep{zhang2023adding}to enable it to incorporate both EPK and tabular clinical data including demographic and drug information, which is named Clinical Information ControlNet (CICN).
% Different from the original ControlNet, CICN not only inherits the capability to ensure authenticity from the backbone diffusion model, but can also generate ECGs affected by different drugs at various time points post-dose by changing the input of clinical data.

We conduct experiments on two publicly available databases~\citep{johannesen2014differentiating,johannesen2016late}, comprising a total of 9,443 12-lead ECGs, which cover the effects of $8$ drug regimens to evaluate the performance of our method.
Specifically, we evaluate the consistency of the generated ECGs and the real post-dose ECGs in terms of normality on three critical cardiac indicators.
Compared to eight state-of-the-art (SOTA) ECG generative models, our proposed DADM achieves the best performance.
Visualization results demonstrate that DADM can accurately simulate the effects of various drug reactions on ECGs across different subjects while maintaining fidelity.

In summary, the main contributions of this article are as follows:
\begin{itemize}
\item We propose the DADM for virtual trials of cardiac drugs by generating drug-induced ECGs.
\item We inject the EPK into the denoising steps via the DCA mechanism to ensure that the generated ECGs exhibit realistic PQRST waves.
\item We design the CICN to control the model to generate drug-induced ECGs based on demographic information and drug data.
\item We validate the superiority of DADM compared with eight SOTA generative models on two public databases, covering the evaluation of the effects of $8$ drug regimens.
\end{itemize}



\section{Related Works}

Generative methods have been widely used to generate ECGs, \textit{e.g.}, Generative Adversarial Networks (GANs)~\citep{simone2023ecgan}, Autoregressive Models (AMs)~\citep{chung2023text}, and Diffusion Models (DMs)~\citep{li2024biodiffusion}. Researchers design various GANs to generate realistic and personalized ECGs, for example, \citet{golany2020simgans,golany2021ecg} integrate physical knowledge into the generation process of GAN to produce realistic ECG signals by using a physically motivated ECG simulator, given by a set of ordinary differential equations~\citep{mcsharry2003dynamical}. \citet{golany2019pgans} uses the wave values and durations of ECG signals as constraints for GAN to generate personalized heartbeats. \citet{hu2024personalized} generates digital twins of healthy individuals’ anomalous ECGs and enhances the model sensitivity to the personalized symptoms.
Besides GANs, AMs and DMs have also been employed for ECG signal generation, achieving SOTA performance in aiding the diagnosis of cardiac disease. For instance, \citet{chung2023text} proposes Auto-TTE, an autoregressive generation model driven by clinical text reports. \citet{neifar2023diffecg} introduces a novel versatile approach based on DMs for ECG synthesis. \citet{alcaraz2023diffusion,zama2023ecg} design novel DMs coupled with state space models to generate 12-lead ECGs.

Most of the aforementioned work has primarily focused on aiding cardiac disease diagnosis and has achieved significant progress in this area. In contrast, our work is centered on generating ECGs that reflect drug reactions, thereby enabling virtual clinical trials.



\begin{figure*}[htbp]
    \centering
    \includegraphics[width=1\textwidth]{images/1-cropped.pdf}
    \caption{(a) Workflow of integrating EPK; (b) Structure of RBDCA. $\mathbf{x}_{ode}$ represents the external physical knowledge. The output of the previous residual block is the input of the next one. Each block is also connected through a skip connection; (c) Structure of DCA. $\mathbf{y}_{inter}$ and $\mathbf{A}$ represent the output of the Bi-DilConv and the cross-attention map, respectively.}
    \label{1}
\end{figure*}


\section{Methods}

\subsection{Overview}

In the following sections, we first describe the EPK construction using an ODE system (see Section~\ref{odes}).
Next, we introduce the workflow of denoising diffusion probabilistic models (see Section~\ref{ddpm}).
Then, we describe integrating EPK into the denoising process using the DCA mechanism (see Section~\ref{dca}).
Finally, we present the design of the CICN (see Section~\ref{cicn}).

\subsection{Ordinary Differential Equation System}
\label{odes}
We construct an ODE system~\citep{mcsharry2003dynamical} to describe the EPK, which includes specific heart rate statistics, such as the mean and standard deviation of heart rate, as well as the frequency-domain characteristics of heart rate variability~\citep{malik1990heart}.
The following two sections introduce the specific forms of the ODEs and the numerical solution method, respectively.


\paragraph{Ordinary Differential Equations.}

The EPK is defined by three coupled ODEs, giving rise to a trajectory $(x(t),y(t),z(t))$. The ODEs are as follows:
\begin{equation}
\label{ode}
\begin{aligned}
\frac{dx}{dt}&=\alpha x-\omega y \equiv f_x(x,y;\eta), \\
\frac{dy}{dt}&=\alpha y+\omega x \equiv f_y(x,y;\eta), \\
\frac{dz}{dt}&=-\sum\limits_{i \in \{P,Q,R,S,T\}}a_i\Delta\theta_i \exp\left(-\frac{\Delta\theta_i^2}{2b_i^2}\right)-(z-z_0(t)) \\
& \equiv f_z(x,y,z,t;\eta),
\end{aligned}
\end{equation}
where $\alpha=1-\sqrt{x^2+y^2}$, $\Delta\theta_i=(\theta-\theta_i) \bmod 2\pi$, $\theta=\mathrm{atan2}(y,x)$ (the four quadrant arctangent of the real parts of the elements of $x$ and $y$, with $-\pi \leq \mathrm{atan2}(y,x) \leq \pi$), and $\omega$ is the angular velocity of the trajectory as it moves around the limit cycle. Baseline wander is introduced by coupling the baseline value $z_0(t)$ in (\ref{ode}) to the respiratory frequency $f_2$ using:
\begin{equation*}
z_0(t)=A\sin(2\pi f_2t),
\end{equation*}
where $A=0.15\mathrm{mV}$.
Baseline wander is a type of low-frequency noise ($0.05 - 3 \mathrm{Hz}$ during stress testing)~\citep{sornmo2005bioelectrical}, caused by various factors: respiratory motion, patient movement, poor contact of electrode cables and ECG recording equipment, inadequate skin preparation at the electrode placement sites, and unclean electrodes~\citep{golany2020simgans}.
The solution of equation (\ref{ode}) is a trajectory in three-dimensional space with coordinates $(x(t), y(t), z(t))$. The generated ECG signal is only the third dimension, \textit{i.e.}, $z(t)$, which is the motion trajectory around a unit radius limit cycle on the $(x, y)$ plane. Each complete rotation of this circle corresponds to one heartbeat (or cardiac cycle). The locations of different points on the ECG signal (such as the $P$, $Q$, $R$, $S$, and $T$ waves) are determined by the parameters $\theta_i$, $a_i$, and $b_i$ in (\ref{ode}), where $i\in\{P, Q, R, S, T\}$.
Parameters $\theta$, $a$ and $b$ determine fixed angles on the unit circle, amplitudes and durations, respectively.
We denote the above 15 parameters by $\eta$ following \citep{mcsharry2003dynamical}.


\paragraph{Solving Method of ODEs.}

The equations given by (\ref{ode}) are integrated numerically using a fourth-order Runge-Kutta method~\citep{butcher1987numerical} with a fixed time step $\Delta t=1/f_s$, where $f_s=1000$ is the sampling frequency. Specifically, we adopt the simplest form of the Runge-Kutta family, the Euler method~\citep{atkinson1991introduction}.
The Euler method is based on the finite difference approximation~\citep{milne2000calculus}:
\begin{equation}
\label{dut}
\frac{du}{dt}(t) \approx \frac{u(t+\Delta t)-u(t)}{\Delta t}.
\end{equation}
Given an ODE of the form $du/dt=v$, substituting it into the formula (\ref{dut}) yields:
\begin{equation}
\label{e11}
u(t+\Delta t)=u(t)+v(t)\Delta t.
\end{equation}
We iterate equation (\ref{e11}) for $L$ time steps, where the $l$-th step corresponds to $t_l = l\Delta t$, and $L$ is the number of sample points in a single signal. We then obtain:
\begin{equation}
\label{e12}
u_{l+1}=u_l+v_l\Delta t.
\end{equation}
Substitute (\ref{ode}) into (\ref{e12}) yields:
\begin{equation*}
\begin{aligned}
t_l&=l\Delta t, \\
x_{l+1}&=x_l+f_x(x_l,y_l;\eta)\Delta t, \\
y_{l+1}&=y_l+f_y(x_l,y_l;\eta)\Delta t, \\
z_{l+1}&=z_l+f_z(x_l,y_l,z_l,t_l;\eta)\Delta t.
\end{aligned}
\end{equation*}




\subsection{Denoising Diffusion Probabilistic Model}
\label{ddpm}
Denoising Diffusion Probabilistic Models (DDPM)~\citep{sohl2015deep,ho2020denoising} are latent variable models that include a forward process or diffusion process where data is gradually diffused by adding Gaussian noise until it reaches a Gaussian distribution, and a reverse process where the noise is converted back into samples from the target distribution.

\paragraph{Diffusion Process.} Diffusion process gradually adds Gaussian noise to the data $\mathbf{x}_0 \sim q(\mathbf{x}_0)$ over $T$ iterations, which is set to a simple parameterization of a fixed Markov Chain as each step:
\begin{equation*}
q(\mathbf{x}_{1:T}|\mathbf{x}_0)=\prod \limits_{t=1}^T q(\mathbf{x}_t|\mathbf{x}_{t-1}),
\end{equation*}
\begin{equation*}
q(\mathbf{x}_t|\mathbf{x}_{t-1})=\mathcal{N}\left(\mathbf{x}_t; \sqrt{1-\beta_t}\mathbf{x}_{t-1},\beta_t\mathbf{I}\right),
\end{equation*}
where $\mathbf{x}_1, ..., \mathbf{x}_T$ are the latent variables with the same dimensionality as $\mathbf{x}_0$, $\beta_1, ..., \beta_T$ is a variance schedule (either learned or fixed) which ensures that $\mathbf{x}_T$ approximates a standard normal distribution for sufficiently large $T$, and $\mathcal{N}(\mathbf{x};\mu,\Sigma)$ is a Gaussian Probability Density Function (PDF) with parameters $\mu$ and $\Sigma$.
The sampling of $\mathbf{x}_t$ at an arbitrary timestep $t$ has the closed-form of $q(\mathbf{x}_t|\mathbf{x}_0)=\mathcal{N}(\mathbf{x}_t;\sqrt{\overline{\mathbf{\alpha}}_t}\mathbf{x}_0,(1-\overline{\mathbf{\alpha}}_t)\mathbf{I})$, where $\overline{\alpha}_t=\prod_{s=1}^t\alpha_s$ and $\alpha_s=1-\beta_s$. Thus $\mathbf{x}_t$ can be expressed directly as $\mathbf{x}_t=\sqrt{\alpha_t}\mathbf{x}_0+(1-\alpha_t)\boldsymbol{\epsilon}$, where $\boldsymbol{\epsilon} \sim \mathcal{N}(\mathbf{0},\mathbf{I})$.

\paragraph{Reverse Process.} Reverse process learns to invert the above diffusion process, restoring $\mathbf{x}_t$ to $\mathbf{x}_0$. Starting from pure Gaussian noise drawn from $p(\mathbf{x}_T) = \mathcal{N}(\mathbf{x}_T; \mathbf{0}, \mathbf{I})$, the reverse process is defined by the following Markov Chain:
\begin{equation*}
p_\theta(\mathbf{x}_{0:T})=p(\mathbf{x}_T)\prod \limits_{t=1}^T p_\theta(\mathbf{x}_{t-1}|\mathbf{x}_t), \mathbf{x}_T \sim \mathcal{N}(\mathbf{0},\mathbf{I}),
\end{equation*}
\begin{equation*}
p_\theta(\mathbf{x}_{t-1}|\mathbf{x}_t)=\mathcal{N}(\mathbf{x}_{t-1};\boldsymbol{\mu}_\theta(\mathbf{x}_t,t),\sigma_\theta(\mathbf{x}_t,t)\mathbf{I}).
\end{equation*}
In DDPM's setting, $\boldsymbol{\mu}_\theta(\mathbf{x}_t,t)$ and $\sigma_\theta(\mathbf{x}_t,t)$ are defined as follows:
\begin{equation*}
\boldsymbol{\mu}_\theta(\mathbf{x}_t,t)=\frac{1}{\alpha_t} \left(\mathbf{x}_t-\frac{\beta_t}{\sqrt{1-\alpha_t}}\boldsymbol{\epsilon}_\theta(\mathbf{x}_t,t)\right),
\end{equation*}
\begin{equation*}
\sigma_\theta(\mathbf{x}_t,t)=\sqrt{\tilde{\beta}_t}, \tilde{\beta}_t=\left\{
\begin{aligned}
& \frac{1-\alpha_{t-1}}{1-\alpha_t}\beta_t, & t>1 \\
& \beta_1, & t=1 
\end{aligned}
\right.,
\end{equation*}
where $\boldsymbol{\epsilon}_\theta(\cdot,\cdot)$ is a learnable denoising function used to predict the noise vector $\boldsymbol{\epsilon}$ added to $\mathbf{x}_t$.
Our objective is to optimize the following alternative loss function:
\begin{equation*}
Loss(\theta)=\mathbb{E}_{t,\mathbf{x}_0,\epsilon}\left[ \Vert\boldsymbol{\epsilon}-\boldsymbol{\epsilon}_\theta(\mathbf{x}_t,t) \Vert^2\right],
\end{equation*}
where $t$ is uniform between $1$ and $T$.



\subsection{Dynamic Cross-Attention Mechanism}
\label{dca}

We integrate the EPK into each block during the denoising process by utilizing the DCA mechanism, as shown in Figure~\ref{1} (a).
Given that the DCA mechanism is a plug-and-play method, we specifically combine DCA with Residual Block~\citep{kongdiffwave} due to its effective framework for waveform generation. And we name this combination RBDCA.
In each RBDCA, as shown in Figure~\ref{1} (b), we use a Bidirectional Dilated Convolution (Bi-DilConv) with kernel size $3$, where the dilation is doubled at each layer within each block, \textit{i.e.}, $[1,2,4,...,2^{n-1}]$.
The value of $n$ is related to the number of RBDCA.
Specifically, the model is composed of $N$ RBDCA, which are grouped into $m$ parts and each part has $n=N/m$ RBDCA. Here, $N=30$, $m=3$ and $n=10$.

During the Bi-DilConv process, we use an encoder $\boldsymbol{\tau}_\theta$ to encode signals $\mathbf{x}_{ode}$ synthesized by the ODE system (\textit{i.e.}, EPK).
Then, we can obtain $\mathbf{y}_{ode}=\boldsymbol{\tau}_\theta(\mathbf{x}_{ode})$ and an intermediate representation $\mathbf{y}_{inter}$ after Bi-DilConv, which are then fused by our proposed DCA mechanism, as shown in Figure~\ref{1} (c).
The process can be formulated as follows:
\begin{equation*}
\mathrm{DCA}=\left( \alpha (\mathrm{Attention_1})+(1-\alpha) \mathrm{Attention_2}\right) \cdot \mathbf{V},
\end{equation*}
where $\mathrm{Attention_i}=\mathrm{softmax}(\mathbf{Q}_i\mathbf{K}^T/ \sqrt{d})$, $i=\{1,2\}$.
Here, $\mathbf{Q}_1=\mathbf{Q}_{ode}=\mathbf{W}^{q_1}\cdot \mathbf{y}_{ode}$,
$\mathbf{Q}_2=\mathbf{Q}_{inter}=\mathbf{W}^{q_2} \cdot \mathbf{y}_{inter}$,
$\mathbf{K}=\mathbf{W}^k\cdot\mathbf{y}_{ode}$,
$\mathbf{V}=\mathbf{W}^v\cdot\mathbf{y}_{ode}$.
And $\mathbf{W}^{q_1}$, $\mathbf{W}^{q_2}$, $\mathbf{W}^k$ and $\mathbf{W}^v$ are learnable projection matrices~\citep{vaswani2017attention}.
$\alpha \in [0,1)$ is the result of a fully connected layer's prediction for time-step $t$, used to measure the weight of EPK in the fusion process. The larger the value of $\alpha$, the greater the proportion of external EPK, and vice versa.


\begin{figure}[tbp]
    \centering
    \includegraphics[width=1\linewidth]{images/2-cropped.pdf}
    \caption{The architecture of CICN.}
    \label{2}
\end{figure}



\begin{table*}[tbp]
\centering
\begin{tabular}{l|cc|cc|cc}
\toprule
\multicolumn{1}{c|}{}
& \multicolumn{2}{c|}{$QT_c$ Interval}   & \multicolumn{2}{c|}{$PR$ Interval}   & \multicolumn{2}{c}{$T_{peak}-T_{end}$ Interval} \\
\multicolumn{1}{l|}{\multirow{-2}{*}{Methods}}
& Accuracy ($\uparrow$) & Recall ($\uparrow$) & Accuracy ($\uparrow$) & Recall ($\uparrow$) & Accuracy ($\uparrow$) & Recall ($\uparrow$) \\
\hline
\multicolumn{7}{l}{\cellcolor[HTML]{C0C0C0}\textit{Generative Adversarial Networks}} \\
WGAN
& $74.96\%$   & $75.00\%$   & $74.96\%$   & $73.53\%$   & $78.53\%$   & $80.36\%$   \\
StyleGAN
& $73.59\%$   & $70.00\%$   & $75.13\%$   & $73.53\%$   & $77.34\%$   & $78.55\%$   \\
ECG ODE-GAN
& $63.20\%$   & $55.00\%$   & $62.69\%$   & $58.82\%$   & $74.11\%$   & $72.73\%$   \\
TTS-CGAN
& $70.53\%$   & $72.50\%$   & $70.53\%$   & $76.47\%$   & $83.48\%$   & $82.55\%$   \\
CECG‐GAN
& $73.59\%$   & $75.00\%$   & $72.91\%$   & $70.59\%$   & $78.19\%$   & $77.09\%$   \\
\hline
\multicolumn{7}{l}{\cellcolor[HTML]{C0C0C0}\textit{Diffusion Models}} \\
DiffECG
& $\underline{83.48\%}$   & $\underline{77.50\%}$   & $83.65\%$   & $\underline{82.35\%}$   & $\underline{85.86\%}$   & $\underline{85.45\%}$   \\
BioDiffusion
& $80.41\%$   & $72.50\%$   & $\underline{85.01\%}$   & $79.41\%$   & $83.82\%$   & $83.64\%$   \\
\hline
\multicolumn{7}{l}{\cellcolor[HTML]{C0C0C0}\textit{Autoregressive Models}} \\
Auto-TTE
& $62.35\%$   & $60.00\%$   & $72.40\%$   & $67.65\%$   & $71.55\%$   & $73.45\%$   \\
\cmidrule{1-7}
DADM (Ours)
& $\mathbf{89.95\%}$   & $\mathbf{87.50\%}$   & $\mathbf{90.80\%}$   & $\mathbf{91.18\%}$   & $\mathbf{92.16\%}$   & $\mathbf{93.45\%}$   \\
\bottomrule
\end{tabular}
\caption{Comparison with other methods. The best performance is bold, and the second-best performance is underlined.}\label{others}
\end{table*}





\subsection{Clinical Information ControlNet}
\label{cicn}
We design Clinical Information ControlNet (CICN) to accurately simulate the individual cardiac drug reactions on ECGs.
Different from the original ControlNet, CICN not only inherits the capability to ensure authenticity from the backbone diffusion model, but can also generate ECGs affected by different drugs at various time points post-dose by changing the input of clinical data.


CICN consists of three types of neural network blocks, which are $\mathbf{f}_{rbdca}^{\prime}(\cdot;\theta_{rbdca}^{\prime})$, $\mathbf{z}_1(\cdot;\theta_{z1})$, and $\mathbf{z}_2(\cdot;\theta_{z2})$, corresponding respectively to RBDCA (trainable copy), Zero Convolution\_1, and Zero Convolution\_2 in Figure~\ref{2}.
Following the setting of \citep{zhang2023adding}, $\mathbf{f}_{rbdca}^{\prime}(\cdot;\theta_{rbdca}^{\prime})$ is the trainable copy from the original block $\mathbf{f}_{rbdca}$ with parameters $\theta_{rbdca}$.
$\mathbf{z}(\cdot;\cdot)$ is a 1D convolution layer with both weight and bias initialized to zeros.
To build up CICN, we use two instances of zero convolutions with parameters $\theta_{z1}$ and $\theta_{z2}$.
$\mathbf{z}_1(\cdot;\theta_{z1})$ is used to process the input clinical information, while $\mathbf{z}_2(\cdot;\theta_{z2})$ is connected to the locked model.
The complete CICN then computes:
\begin{equation*}
\mathbf{y}_{ci}= \mathbf{z}_2(\mathbf{f}_{rbdca}^{\prime}(\mathbf{x}_t+\mathbf{z}_1(\mathbf{x}_{ci};\theta_{z1});\theta_{rbdca}^{\prime});\theta_{z2}),
\end{equation*}
where $\mathbf{x}_{ci}$ and $\mathbf{y}_{ci}$ are the input and output of CICN, respectively.
$\mathbf{x}_{ci}$ includes demographic information and drug data, which is a kind of tabular data.
During the training process of CICN, the parameters $\theta_{rbdca}$ of the original block are frozen. We only update the parameters of CICN. The total loss function is as follows:
\begin{equation*}
Loss(\theta^{\prime}_{rbdca},\theta_{z}) = \mathbb{E}_{t,\mathbf{x}_0,\mathbf{y},\boldsymbol{\epsilon}}\left[ \Vert\boldsymbol{\epsilon}-\boldsymbol{\epsilon}_\theta(\mathbf{x}_t,t,\mathbf{y}_{ode},\mathbf{y}_{ci}) \Vert^2\right],
\end{equation*}
where $\theta_{z}=\{\theta_{z1},\theta_{z2}\}$.






\section{Experiments}

\subsection{Dataset}

We utilize two publicly available ECG datasets, including the ECGRDVQ database~\citep{johannesen2014differentiating} and the ECGDMMLD database~\citep{johannesen2016late}. These two datasets collectively comprise 9,443 12-lead ECGs from 44 subjects, covering the evaluation of the effects of $8$ types of drug regimens.
The treatment regimens include: (1) Dofetilide; (2) Lidocaine + Dofetilide; (3) Mexiletine + Dofetilide; (4) Moxifloxacin + Diltiazem; (5) Placebo; (6) Quinidine; (7) Ranolazine; (8) Verapamil.
We partition the data into training and testing sets with a $4:1$ ratio, ensuring that the proportion of ECGs affected by each drug in the training and testing sets also follows a $4:1$ ratio. More details can be found in the Technical Appendix.



\subsection{Implementation Details}

We train a generative model for each lead using $8$ NVIDIA GeForce RTX 4090 GPUs. The model training process is divided into two steps: (1) training a backbone DDPM which is injected EPK via DCA; (2) freezing the parameters of the DDPM and training the CICN.
We set $T=1000$ and the forward process variances to constants increasing linearly from $\beta_1=10^{-4}$ to $\beta_T=0.02$. We use Adam optimizer~\citep{kingma2015adam} with a batch size of $16$ and a learning rate of $2\times10^{-4}$. We train all models for $50$ epochs.
During the validation phase, we generate the corresponding post-dose ECGs at specific time points by varying the clinical data input to CICN. The code is uploaded as Supplementary Material.

\subsection{Indicator Details}

To validate whether the generated ECGs accurately simulate the post-dose reactions, we evaluate the consistency of the generated ECGs and the real post-dose ECGs in terms of normality on three critical indicators.
The indicators include $QT_c$ interval, $PR$ interval, and $T_{peak}-T_{end}$ interval.
For example, if the real post-dose ECGs indicate that the $QT_c$ interval is normal (or abnormal), and the generated ECGs also show this indicator as normal (or abnormal), it signifies that the model accurately simulates the drug reactions. Conversely, if not, the opposite is true.

The calculation rules and normal ranges of each indicator are as follows:
$\mathbf{QT_c}$ \textbf{interval} is the corrected $QT$ interval, representing the time from ventricular depolarization to complete repolarization, adjusted for heart rate. Here, we use the Bazett formula for correction: $QT_c = QT / \sqrt{RR}$, where $RR$ represents the heart rate.
$QT_c\leq450\mathrm{ms}$ in men and $QT_c\leq470\mathrm{ms}$ in women are considered normal~\citep{gupta2007current}.
$\mathbf{PR}$ \textbf{interval} represents the time it takes for the electrical signal to travel from the atria to the ventricles, including atrial depolarization and the delay at the atrioventricular node. $120 \mathrm{ms} \leq PR \leq 200 \mathrm{ms}$ is normal~\citep{kwok2016prolonged}.
$\mathbf{T_{peak}-T_{end}}$ \textbf{interval} represents the late phase of ventricular repolarization and reflects the dispersion of ventricular repolarization.
$80 \mathrm{ms} \leq T_{peak}-T_{end} \leq 113 ms$ is considered normal~\citep{icli2015prognostic}.



\subsection{Comparison with Other Generated Methods}

The comparison methods include five GAN-based methods, two DM-based methods and one AM-based method, which are \textbf{WGAN}~\citep{arjovsky2017wasserstein}, \textbf{StyleGAN}~\citep{8953766}, \textbf{ECG ODE-GAN}~\citep{golany2021ecg}, \textbf{TTS-CGAN}~\citep{li2022tts}, \textbf{CECG‐GAN}~\citep{yang2024data}, \textbf{DiffECG}~\citep{neifar2023diffecg}, \textbf{BioDiffusion}~\citep{li2024biodiffusion}, and \textbf{Auto-TTE}~\citep{chung2023text}.
Considering that our focus is to make the generative model simulate the signals relevant to drugs as much as possible, in addition to accuracy, we also use recall as a metric. The higher the recall rate, the greater the ability of the generative model to simulate the signals caused by different drugs.



\paragraph{Single Indicator Evaluation.}
First, we compare the performance of different methods based on single indicators.
The experimental results are shown in Table~\ref{others}, from which we have several observations: (1) Our method achieves the highest accuracy and recall across all three indicators, demonstrating its superiority. For example, in terms of the $QT_c$ interval indicator, our method improves the accuracy by $6.47\%$ and the recall by $10\%$ compared to the second-best method;
(2) Although other DM-based methods perform worse than our approach, they still outperform GAN-based methods, indicating that DM-based methods are better suited for our generative task;
(3) While the ECG ODE-GAN model also incorporates EPK through ODEs during the generation process, its performance is inferior to our method. This highlights the critical importance of the EPK integration strategy and, indirectly, underscores the superiority of DCA mechanism.

\begin{table}[tbp]
\centering
\begin{tabular}{l|ccc}
\toprule
\multicolumn{1}{l|}{Methods} & \multicolumn{1}{c}{Lid+Dof} & \multicolumn{1}{c}{Mex+Dof} & \multicolumn{1}{c}{Mox+Dil} \\
\cmidrule{1-4}
WGAN                  & $39.22\%$ & $49.09\%$ & $62.75\%$\\
StyleGAN              & $31.37\%$ & $41.82\%$ & $70.59\%$\\
ECG ODE-GAN           & $27.45\%$ & $32.73\%$ & $47.06\%$\\
TTS-GAN               & $50.98\%$ & $30.91\%$ & $60.78\%$\\
CECG-GAN              & $29.41\%$ & $43.64\%$ & $45.10\%$\\
DiffECG               & $43.14\%$ & $\mathbf{61.82\%}$ & $\mathbf{80.39\%}$\\
BioDiffusion          & $\underline{62.75\%}$ & $\underline{60.00\%}$ & $64.71\%$\\
Auto-TTE              & $39.22\%$ & $36.36\%$ & $37.25\%$\\
\cmidrule{1-4}
DADM (Ours)           & $\mathbf{76.47\%}$ & $\mathbf{61.82\%}$ & $\underline{76.47\%}$\\
\bottomrule
\end{tabular}
\caption{Comparison with other methods in simulating composite drug reactions. Lid, Dof, Mex, Mox, and Dil respectively represent Lidocaine, Dofetilide, Mexiletine, Moxifloxacin, and Diltiazem.}\label{multi}
\end{table}

\begin{table}[tbp]
\centering
\begin{tabular}{l|cc}
\toprule
\multicolumn{1}{l|}{Methods} & \multicolumn{1}{c}{FRD ($\downarrow$)} & \multicolumn{1}{c}{RMSE ($\downarrow$)} \\
\cmidrule{1-3}
WGAN                  & $1.3466$                 & $0.2328$ \\
StyleGAN              & $1.7853$                 & $0.3021$ \\
ECG ODE-GAN           & $2.5639$                 & $0.3764$ \\
TTS-GAN               & $0.7880$                 & $0.0716$ \\
CECG-GAN              & $0.7336$                 & $0.0728$ \\
DiffECG               & $0.5322$                 & $0.0412$ \\
BioDiffusion          & $\underline{0.4832}$     & $\underline{0.0343}$ \\
Auto-TTE              & $2.5234$                 & $0.4162$ \\
\cmidrule{1-3}
DADM (Ours)           & $\mathbf{0.2431}$        & $\mathbf{0.0127}$ \\
\bottomrule
\end{tabular}
\caption{Evaluation of the fidelity of ECGs generated by different generative models. The best performance is bold, and the second-best performance is underlined.}\label{fid}
\end{table}



\paragraph{Multi-Indicator Evaluation.}
Furthermore, we compare the models' ability to simulate the composite effects of drug combinations on ECG, including \textbf{Lidocaine + Dofetilide}, \textbf{Mexiletine + Dofetilide}, and \textbf{Moxifloxacin + Diltiazem}. We jointly evaluate several indicators among the three to reflect the models' ability to simulate composite effects. For example, in the third combination, Moxifloxacin may prolong the $QT$ interval, and Diltiazem may prolong the $PR$ interval. Thus, the results are only considered correct if the model accurately simulates both indicators. We calculate the accuracy of different models based on this criterion.
The experimental results are shown in Table~\ref{multi}, from which we can observe that (1) DADM achieves the best or second-best results in all cases. Although DM-based DiffECG achieves the best performance in the second and third cases, the performance of DADM is more stable;
(2) Compared to single indicator evaluation tasks, the accuracy of all methods has dropped significantly, often falling below $50\%$, indicating that simulating composite drug reactions remains challenging.







\subsection{Evaluation of Generated ECG Fidelity}

The Fr\'echet Inception Distance (FID)~\citep{heusel2017gans} is a traditional metric used for assessing the fidelity of generated images. Following \citet{hu2024personalized}, we introduce the Fr\'echet ResNet Distance (FRD) as the metric to evaluate the generated ECG fidelity.
Specifically, we utilize a 1D ResNet model~\citep{hong2020holmes} pre-trained on Challenge 2017 data~\citep{clifford2017af} instead of the Inception V3~\citep{szegedy2016rethinking} as the backbone for feature extraction from ECG data. We use the output from the network's final fully connected layer as a feature vector to calculate the FRD on the test set.
In addition, we adopt Root Mean Squared Error (RMSE) as a metric to evaluate the similarity between the generated and real ECGs post-dose. Lower FRD scores and RMSE indicate that the distribution of generated ECGs is closer to that of real ECGs.

We compare the average FRD scores and RMSE of 12-lead ECGs generated by different methods, and the results are presented in Table~\ref{fid}. Our findings demonstrate that our method achieves the optimal FRD score of $0.2431$ and an RMSE of $0.0127$.


\begin{figure}[tbp]
    \centering
    \includegraphics[width=0.97\linewidth]{images/e1-cropped.pdf}
    \caption{(a) Ablation study with and without EPK, where \textit{w/o} represents generating ECGs without EPK, \textit{Fixed} represents fixed weight fusion, and \textit{Dyn} represents DCA fusion; (b) The changing trend of $\alpha$ over time-steps.}
    \label{e1}
\end{figure}



\subsection{Ablation Study}
\label{as}

We conduct ablation studies to validate the importance of incorporating EPK and the effectiveness of DCA.
We calculate the FRD scores of the generated ECGs under three conditions: without EPK, with fixed weight fusion, and with DCA.
In the experiment with fixed weight fusion, we set $\alpha$ to be fixed at 0.5.
For a fair comparison, the generative model with DCA does not have a CICN module.
The experimental results are presented in Figure~\ref{e1} (a), from which we have several observations:
(1) The FRD achieved by DCA is significantly lower than that achieved by fixed weight fusion and fusion without EPK, demonstrating the effectiveness of our proposed DCA.
(2) The FRD achieved by fixed weight fusion shows a slight improvement over fusion without EPK, indicating that merely introducing EPK does not substantially enhance the quality of ECG generation, which is consistent with the conclusion drawn in the third point of Table~\ref{others}.


Moreover, we analyze the trend of the dynamic parameter $\alpha$ as it varies with the time-step $t$ to validate the effectiveness of DCA, as shown in Figure~\ref{e1} (b). In the initial stage of the denoising period, the value of $\alpha$ is relatively high, indicating that EPK plays a crucial role in generating the basic ECG waveform. In the later stage, $\alpha$ gradually decreases, suggesting that the constraints imposed by EPK diminish, allowing the generated ECGs to exhibit more diverse details.

\begin{figure}[tbp]
    \centering
    \includegraphics[width=1\linewidth]{images/vis2-cropped.pdf}
    \caption{Visualization of generated ECGs with and without EPK. For clarity in comparison, we use the electrocardiograms of three leads: II, aVR, and V3.}
    \label{epk}
\end{figure}





\subsection{Visualization}

\paragraph{Visualization of ECGs With and Without EPK.}

To provide a more intuitive comparison of the generated ECGs with and without the introduction of EPK, we visualize the results, as shown in Figure~\ref{epk}.
Note that in this experiment, we use the DCA mechanism to introduce EPK.
Figure~\ref{epk} shows that the model incorporating EPK not only captures the fundamental characteristics of heartbeats but also significantly reduces noise, thus enhancing the realism of the ECGs.
In contrast, the model without EPK fails to synthesize clear waveforms, resulting in noisy signals.
More visualization results can be found in the Technical Appendix.




\paragraph{Visualization of Drug Reactions.}

To more intuitively verify that DADM can accurately simulate the drug effects, we visualize ECGs showing the prolongation of the $PR$ interval caused by Verapamil's effect on extending atrioventricular conduction time.
Specifically, we visualize the ECGs of a subject one hour after administering Verapamil.
The visualization results are presented in Figure~\ref{ar}. It is observed that after taking Verapamil, the subject's $PR$ interval increases from $167\mathrm{ms}$ pre-dose to $293\mathrm{ms}$ post-dose. In the ECGs generated by our method, the $PR$ interval is $250\mathrm{ms}$. Both values fall within the abnormal range, demonstrating that our method can effectively simulate the reaction of drugs on ECGs.
More visualization results can be found in the Technical Appendix.

\begin{figure}[tbp]
    \centering
    \includegraphics[width=1\linewidth]{images/eff2-cropped.pdf}
    \caption{Visualization of Verapamil's reactions on ECGs. Given that lead II is the most suitable for the detection of $PR$ interval, we visualize a heartbeat from the 10-second ECGs on lead II.}
    \label{ar}
\end{figure}



\section{Conclusions}


In this paper, we design a novel framework named DADM to simulate cardiac drug reactions on ECGs for virtual clinical trials. 
To generate realistic ECGs, we first construct an ODE system to model EPK.
Then, we propose a DCA mechanism that adaptively adjusts the constraints imposed by the EPK on ECG generation according to the time steps.
To accurately simulate cardiac drug reactions, we propose a ControlNet extension called CICN to control the effects of different drugs on the ECGs according to the input clinical information.
We conduct comparative and visualization experiments across two public datasets covering $8$ types of drug regimens, the results of which demonstrate superior performance over the other eight SOTA generative models in simulating cardiac drug reactions.
The ablation study demonstrates the effectiveness of DCA mechanism.



\section{Limitations}

While our proposed DADM demonstrates potential utility in mitigating clinical trial risks by simulating drug reactions on ECGs for virtual clinical trials, several limitations still exist.
First, model performance evaluation remains constrained to three fundamental cardiac indicators, which may insufficiently capture the multidimensional nature of drug toxicity.
Second, the model exhibits suboptimal performance in modeling composite drug interactions.
These limitations primarily stem from the limited availability of datasets, and current architectural constraints in modeling higher-order pharmacological synergies.
To address these challenges, future work will focus on incorporating more comprehensive indicators (\textit{e.g.}, Torsades de Pointes and ST-segment depression) and modeling composite drug interactions to better approximate real-world clinical complexity.







%% The file named.bst is a bibliography style file for BibTeX 0.99c
\bibliographystyle{named}
\bibliography{ijcai25}


\clearpage

\appendix

\section*{Technical Appendix}

\section{Details of Datasets}
\label{dod}
\paragraph{ECG Recording Settings.}
The ECGs from the two databases are recorded at $1000 \mathrm{Hz}$ with an amplitude resolution of $2.5$\si{\micro V}. In the ECGRDVQ database, triplicate $10$-second ECGs are extracted at 16 predefined time points: 1 point pre-dose ($-0.5 \mathrm{h}$) and 15 points post-dose ($0.5$, $1$, $1.5$, $2$, $2.5$, $3$, $3.5$, $4$, $5$, $6$, $7$, $8$, $12$, $14$, and $24 \mathrm{h}$), during which the subjects were resting in a supine position for $10$ minutes.
In the ECGDMMLD database, triplicate $10$-second ECGs are extracted at 14 predefined time points: 1 point pre-dose ($-0.5 \mathrm{h}$) and 13 points post-dose ($1.5$, $2$, $2.5$, $3$, $6.5$, $7$, $7.5$, $8$, $12$, $12.5$, $13$, $13.5$ and $24 \mathrm{h}$), during which the subjects were resting in a supine position for 10 minutes.
We randomly select one ECG signal from the three recorded at each time point for subsequent experiments.
We combine the pre-dose ECGs, each post-dose ECGs, and the corresponding clinical information into a single training data point. This means that for each subject, $N$ training data points can be generated, where $N$ represents the number of time points at which post-dose ECGs are recorded.

\paragraph{Clinical Information.} Clinical information includes sex, age, height, weight, baseline systolic blood pressure, baseline diastolic blood pressure, race, ethnicity, sequence of treatments, visit code (PERIOD-X-Dosing refers to the X'th dosing), treatment (types of drug regimens), nominal time-point. The nominal time-point represents the time at which the post-dose ECGs are recorded.

\paragraph{Drug Information.} The drugs in the datasets cause the following cardiac reactions:
\textbf{Dofetilide} is mainly used to maintain the sinus rhythm in patients with atrial fibrillation and atrial flutter. It selectively blocks the delayed rectifier potassium current, prolonging the repolarization process of myocardial cells;
\textbf{Lidocaine} and \textbf{Mexiletine} are mainly used for the treatment of ventricular arrhythmia. They may shorten the QT interval;
\textbf{Moxifloxacin} is mainly used for the treatment of bacterial infections rather than directly targeting heart diseases. It may cause a prolongation of the QT interval;
\textbf{Diltiazem} is mainly used for the treatment of arrhythmia, angina pectoris, and hypertension. It prolongs the PR interval;
\textbf{Placebo} is a substance or treatment that looks like real medicine but contains no active medical ingredients. Despite having no therapeutic properties, Placebos can sometimes produce real physical or psychological benefits, known as the Placebo Effect.
\textbf{Quinidine} is mainly used for the treatment of arrhythmia. It prolongs the QT interval and the width of the QRS complex;
\textbf{Ranolazine} is mainly used for the treatment of chronic angina pectoris. It has a certain impact on the repolarization process of myocardial cells, which may lead to a moderate prolongation of the QT interval;
\textbf{Verapamil} is mainly used for the treatment of arrhythmia, angina pectoris, and hypertension. It inhibits the influx of calcium ions, slowing down the conduction speed of the sinoatrial node and atrioventricular node, and prolonging the time it takes for the atrial impulse to be conducted to the ventricle.




\section{Additional Visualization Results}
\label{vis_dr}

Figures~\ref{app_drug} presents additional visual experimental results of drug reactions, from which we can observe that after taking Quinidine, the subject's $T_{peak}-T_{end}$ interval increases from $79\mathrm{ms}$ pre-dose to $142\mathrm{ms}$ post-dose. In our generated ECGs, the $T_{peak}-T_{end}$ interval is $133\mathrm{ms}$. Both values fall within the abnormal range, demonstrating that our method can effectively simulate the reaction of drugs on ECGs.

Figure~\ref{app_epk} presents additional visual experimental results with and without EPK and drug reactions, from which we can observe that the introduction of EPK makes the generation of the ECG signal more realistic in detail. In the absence of EPK, the generated ECG signal contains a lot of noise.

\begin{figure}[hbp]
    \centering
    \includegraphics[width=1\linewidth]{images/app_drug-cropped.pdf}
    \caption{Visualization of Quinidine’s reactions on ECGs. Given that lead II is the most suitable for the detection of $T_{peak}-T_{end}$ interval, we visualize a heartbeat from the 10-second ECGs on lead II.}
    \label{app_drug}
\end{figure}




\begin{figure*}[htbp]
    \centering
    \includegraphics[width=1\linewidth]{images/app_epk-cropped.pdf}
    \caption{Additional visualization results of generated ECGs with and without EPK.}
    \label{app_epk}
\end{figure*}






\end{document}