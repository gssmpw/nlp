\section{Related Work}
\textbf{Image augmentation} has been a prevalent technique in computer vision tasks, with traditional methods including rotation, cropping, color space transformation, Gaussian noise injection, and more \cite{shorten2019survey}. However, these techniques are not specifically designed for document images, nor can they effectively mimic different document styles. Augraphy, introduced by \cite{groleau2023augraphy}, is a Python library that addresses this gap by offering 26 unique augmentations specifically designed for document images.

\textbf{Layout Analysing} plays a vital role in understanding the structure of documents, as it involves detecting and recognizing semantic entities to transform unstructured documents into structured formats. Initially, vision-only frameworks like Faster R-CNN \cite{ren2015faster} and Mask R-CNN \cite{he2017mask} were commonly used as foundational models for this task. However, to gain a deeper and more nuanced understanding, multi-modal \cite{huang2022layoutlmv3} and relation-aware \cite{luo2022doc} frameworks have been developed, leading to significant improvements. These advanced models can enhance existing object detection frameworks by offering richer and more comprehensive representations, making them powerful backbones for document layout analysis.