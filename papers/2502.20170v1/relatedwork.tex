\section{Related Works}
\label{sec:related}

There is a rich body of literature studying rating methods with applications in Chess, Go, Tennis and video games. One family of probabilistic methods follows the Bradley-Terry model and predicts pairwise win probabilities from ratings. A widely used example is Elo \citep{elo1978rating} with extensions Bayes-Elo, mElo and Elo-MMR \citep{coulom2008whole, balduzzi2018re, ebtekar2021elo, vadori2024ordinal} capturing temporal variation, cyclicality and ordinal ranks in data. Elo ratings can typically be efficiently solved as regression problems, although their ratings are vulnerable to redundancy. A separate line of work draws from Social Choice (or \emph{Voting}) Theory (SCT, \citet{sen1977social, lanctot2023evaluating}),
which also studies {\em independence of clones}: rankings should be invariant to redundant candidates (e.g., LLM models) being added.
However, invariance to redundancy in votes (e.g., prompts) is in direct opposition to the spirit of social choice theory. In this sense, SCT provides partial (one-sided) clone invariance, which we argue is insufficient for open-ended, LMSYS-style evaluation. Finally, game-theoretic evaluation has been previously studied in \citet{balduzzi2018re} and \citet{marris2022_game_theoretic_rating} where full clone-invariance is guaranteed in the \tpzs setting. Our method generalises the approaches in these works to \npgs settings, with practical equilibrium solving and selection algorithms based on our novel affinity entropy definition. Other approaches have been concurrently developed that avoid the equilibrium selection dilemma, and hence obviate the use of entropy~\citep{marris2025deviationratings}.