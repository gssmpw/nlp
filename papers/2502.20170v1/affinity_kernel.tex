\section{Integrals over Simplex}


It is possible to derive a closed-form result for the dis-similarity kernel in~\eqref{eqn:dissimilarity} by appealing to known results of integrals of polynomial functions over the simplex.

Let $T^d = \{(x_1, \ldots, x_d): x_i \ge 0, \sum_{i=1}^d x_i \le 1\}$ be the standard simplex in $\mathbb{R}^d$. Let $\nu_i > 0$ for all $i$, then
\begin{align}
    \int_{T^d} x_1^{\nu_1 - 1} \ldots x_d^{\nu_d - 1} (1 - x_1 - \ldots - x_d)^{\nu_0 - 1} = \frac{\prod_{i=0}^d \Gamma(\nu_i)}{\Gamma(\sum_{i=0}^d \nu_i)}. \label{eqn:polyint}
\end{align}

\begin{proposition}
From player $i$'s perspective, the expected dis-similarity between two actions $p$ and $q$ under a uniform distribution over all opponent joint strategy profiles $x_{-i}$ is equal to
\begin{align}
    D^{(i)}_{pq} &= \frac{1}{(d_i+1)(d_i+2)} \Big[ ||U^{(i)}_p - U^{(i)}_{q}||^2 + \big(1^\top (U^{(i)}_p - U^{(i)}_{q})\big)^2 \Big]
\end{align}
where $U^{(i)}$ is a $\vert \mathcal{A}_i \vert \times \vert \mathcal{A}_{-i} \vert$ matrix where each entry $U^{(i)}_{a_i,a_{-i}}$ is the expected utility for player $i$ playing action $a_i$ against the background joint action $a_{-i}$. $U^{(i)}_{a_i}$ indicates an entire row of the matrix. The integer $d_i = \prod_{j \ne i} \vert \mathcal{A}_j \vert$.
\end{proposition}
\begin{proof}
Recall~\eqref{eqn:polyint} and $\Gamma(n) = (n-1)!$ for $n \in \mathbb{N}$. Let $r_p = \sum_w U_{pw} x_w$ be the rating for the $p$th action under an opponent strategy profile $x_{-i}$.

Then we want to compute $\mathbb{E}_{x_{-i} \sim Dir(\boldsymbol{1})}[(r_p - r_q)^2]$. Recall the volume of the simplex is $\frac{1}{d!}$. Then
\begin{align}
    \mathbb{E}_{x_{-i} \sim Dir(\boldsymbol{1})}[(r_p - r_q)^2] &= \frac{\int_{T^d} (r_p - r_q)^2 dx_{-i}}{\int_{T^d} dx_{-i}}
    \\ &= d! \int_{T^d} (r_p - r_q)^2 dx_{-i}
    \\ &= d! \int_{T^d} (\sum_w U^{(i)}_{pw} x_w - \sum_w U^{(i)}_{qw} x_w)^2 dx_{-i}
    \\ &= d! \int_{T^d} \Big[ (\sum_w \sum_y U^{(i)}_{pw} U^{(i)}_{py} x_w x_y) + (\sum_w \sum_y U^{(i)}_{qw} U^{(i)}_{qy} x_w x_y)
    \\ &\quad - 2 (\sum_w \sum_y U^{(i)}_{pw} U^{(i)}_{qy} x_w x_y) \Big] dx_{-i}
    \\ &= d! \sum_w \sum_y \Big[ \big( U^{(i)}_{pw} U^{(i)}_{py} \underbrace{\int_{T^d} x_w x_y dx_{-i}}_{\frac{2}{(d+2)!} \text{ if } w=y \text{ else } \frac{1}{(d+2)!}} \big)
    \\ &\quad + \big( U^{(i)}_{qw} U^{(i)}_{qy} \int_{T^d} x_w x_y dx_{-i} \big) - 2 \big( U^{(i)}_{pw} U^{(i)}_{qy} \int_{T^d} x_w x_y dx_{-i} \big) \Big]
    \\ &= \frac{d!}{(d+2)!} \sum_w \Big[ (U^{(i)2}_{pw} + U^{(i)2}_{qw} - 2U^{(i)}_{pw} U^{(i)}_{qw})
    \\ &+ \sum_y \big( U^{(i)}_{pw} U^{(i)}_{py} + U^{(i)}_{qw} U^{(i)}_{qy} - 2 U^{(i)}_{pw} U^{(i)}_{qy} \big) \Big]
    \\ &= \frac{1}{(d+1)(d+2)} \Big[ \sum_w (U^{(i)}_{pw} - U^{(i)}_{qw})^2 + (\sum_w U^{(i)}_{pw} - \sum_w U^{(i)}_{qw})^2 \Big]
    \\ &= \frac{1}{(d+1)(d+2)} \Big[ ||U^{(i)}_p - U^{(i)}_{q}||^2 + \big(1^\top (U^{(i)}_p - U^{(i)}_{q})\big)^2 \Big].
\end{align}
\end{proof}

\begin{proposition}
From player $i$'s perspective, the expected dis-similarity between two actions $p$ and $q$ under a uniform distribution over all factorize-able opponent strategy profiles $x_{-i} = \prod_{j \ne i} x_j$ is equal to
\begin{align}
    D^{(i)}_{pq} &= \prod_{j \ne i} \frac{1}{(d_j + 1)(d_j + 2)} \Big(
    \\ &\quad \sum_{a_{-i} \in \mathcal{A}_{-i}} \sum_{a'_{-i} \in \mathcal{A}_{-i}} \big( u_i(p, a_{-i}) - u_i(q, a_{-i}) \big) \big( u_i(p, a'_{-i}) - u_i(q, a'_{-i}) \big) \big( 2^{\#a=a'} \big) \Big)
\end{align}
where the integer $d_i = \vert \mathcal{A}_i \vert$ and ``$\#a$$=$$a'$'' $ = \sum_{j \ne i} \mathbf{1}[a_j = a_j']$ indicates the number of action matches between two opponent profiles.
\end{proposition}
\begin{proof}
Let $r_p = \sum_{a_{-i} \in \mathcal{A}_{-i}} u_i(p, a_{-i}) \prod_{j \ne i} x_{j, a_j}$ be the rating for the $p$th action under an opponent profile $x_{-i} = \prod_{j \ne i} x_j$. Let $dx_{-i}$ be a shorthand for $dx_{-i}$. Likewise, let $\int_{T^{d_{-i}}}$ be a shorthand for $\int_{T^{d_1}} \ldots \int_{T^{d_{i-1}}} \int_{T^{d_{i+1}}} \ldots \int_{T^{d_n}}$.

Then we want to compute $\mathbb{E}_{x_j \sim Dir(\boldsymbol{1}) \forall j \ne i}[(r_p - r_q)^2]$. Recall the volume of a simplex is $\frac{1}{d!}$. Then
\begin{align}
    &\mathbb{E}_{x_j \sim Dir(\boldsymbol{1}) \forall j \ne i}[(r_p - r_q)^2]
    \\ &= \frac{\int_{T^{d_{-i}}} (r_i - r_i')^2 dx_{-i}}{\int_{T^{d_{-i}}} dx_{-i}}
    \\ &= \Big( \prod_{j \ne i} d_j! \Big) \int_{T^{d_{-i}}} (r_i - r_i')^2 dx_{-i}
    \\ &= \Big( \prod_{j \ne i} d_j! \Big) \int_{T^{d_{-i}}} \Big( \sum_{a_{-i} \in \mathcal{A}_{-i}} u_i(p, a_{-i}) \prod_{j \ne i} x_{j, a_j} - \sum_{a_{-i} \in \mathcal{A}_{-i}} u_i(q, a_{-i}) \prod_{j \ne i} x_{j, a_j} \Big)^2 dx_{-i}
    \\ &= \Big( \prod_{j \ne i} d_j! \Big) \int_{T^{d_{-i}}} \Big( \sum_{a_{-i} \in \mathcal{A}_{-i}} \prod_{j \ne i} x_{j, a_j} \big( u_i(p, a_{-i}) - u_i(q, a_{-i}) \big) \Big)^2 dx_{-i}
    \\ &= \Big( \prod_{j \ne i} d_j! \Big) \int_{T^{d_{-i}}} \Big(
    \\ &\quad \sum_{a_{-i} \in \mathcal{A}_{-i}} \sum_{a'_{-i} \in \mathcal{A}_{-i}} (\prod_{j \ne i} x_{j, a_j}) (\prod_{j \ne i} x_{j, a'_j}) \big( u_i(p, a_{-i}) - u_i(q, a_{-i}) \big) \big( u_i(p, a'_{-i}) - u_i(q, a'_{-i}) \big) \Big) dx_{-i}
    \\ &= \Big( \prod_{j \ne i} d_j! \Big) \int_{T^{d_{-i}}} \Big( 
    \\ & \sum_{a_{-i} \in \mathcal{A}_{-i}} \sum_{a'_{-i} \in \mathcal{A}_{-i}} \big( u_i(p, a_{-i}) - u_i(q, a_{-i}) \big) \big( u_i(p, a'_{-i}) - u_i(q, a'_{-i}) \big) \big( \prod_{j \ne i} x_{j, a_j} x_{j, a'_j} \big) \Big) dx_{-i}
    \\ &= \Big( \prod_{j \ne i} d_j! \Big) \Big(
    \\ & \sum_{a_{-i} \in \mathcal{A}_{-i}} \sum_{a'_{-i} \in \mathcal{A}_{-i}} \big( u_i(p, a_{-i}) - u_i(q, a_{-i}) \big) \big( u_i(p, a'_{-i}) - u_i(q, a'_{-i}) \big) \big( \prod_{j \ne i} \underbrace{\int_{T^{d_{j}}} x_{j, a_j} x_{j, a'_j} dx_j}_{\frac{2}{(d_j+2)!} \text{ if } a_j=a_j' \text{ else } \frac{1}{(d_j+2)!}} \big) \Big)
    \\ &= \Big( \prod_{j \ne i} d_j! \Big) / \Big( \prod_{j \ne i} (d_j + 2)! \Big) \Big(
    \\ & \sum_{a_{-i} \in \mathcal{A}_{-i}} \sum_{a'_{-i} \in \mathcal{A}_{-i}} \big( u_i(p, a_{-i}) - u_i(q, a_{-i}) \big) \big( u_i(p, a'_{-i}) - u_i(q, a'_{-i}) \big) \big( 2^{\#a=a'} \big) \Big)
    \\ &= \prod_{j \ne i} \frac{1}{(d_j + 1)(d_j + 2)} \Big(
    \\ & \sum_{a_{-i} \in \mathcal{A}_{-i}} \sum_{a'_{-i} \in \mathcal{A}_{-i}} \big( u_i(p, a_{-i}) - u_i(q, a_{-i}) \big) \big( u_i(p, a'_{-i}) - u_i(q, a'_{-i}) \big) \big( 2^{\#a=a'} \big) \Big).
\end{align}
\end{proof}

\color{black}
\section{Warmup: Game-Theoretic Ranking of {\em rock-paper-scissors}}
\label{app:toy_games}

\begin{figure}
    \centering
    \includegraphics[width=0.9\textwidth]{figures/ne_rating_rrps.pdf}
    \caption{\textcolor{black}{We visualise the marginal NE rating contributions of each player 2 action to each player 1 action. We show that a) all actions receive zero ratings and b) the rating of each action is interpretable and corresponds to our intuition.}}
    \label{fig:rrps_marginal_contrib}
\end{figure}

We provide a demonstration of game-theoretic ranking on the classic $2$-player, $3$-action zero-sum Rock-Paper-Scissors game. \citet{balduzzi2018re} proposed rating actions under the max-entropy Nash equilibrium of the game. In that case, each action receives a rating of zero. If we duplicate the Rock action, for example, the ratings remain zero under the max-entropy NE. Our proposed LLE based approach returns the same ratings.


\def\sgtextcolor{black}%
\def\sglinecolor{black}%
\begin{figure}[htb]\hspace*{\fill}%
\begin{game}{3}{3}
& Rock & Paper & Scissors \\
Rock & $0,0$ & $-1,+1$ & $+1,-1$ \\
Paper & $+1,-1$ & $0,0$ & $-1,+1$ \\
Scissors & $-1,+1$ & $+1,-1$ & $0,0$
\end{game}\hspace*{\fill}%
\hspace*{\fill}%
\begin{game}{4}{4}
& Rock\-1 & Rock\-2 & Paper & Scissors \\
Rock\-1 & $0,0$ & $0,0$ & $-1,+1$ & $+1,-1$ \\
Rock\-2 & $0,0$ & $0,0$ & $-1,+1$ & $+1,-1$ \\
Paper & $+1,-1$ & $+1,-1$ & $0,0$ & $-1,+1$ \\
Scissors & $-1,+1$ & $-1,+1$ & $+1,-1$ & $0,0$
\end{game}\hspace*{\fill}%
\caption[]{\textcolor{black}{Rock-Paper-Scissors (RPS) Game and RPS Game with Duplicate Rock Action.}}
\end{figure}

In Figure~\ref{fig:rrps_marginal_contrib}, we show that the equilibrium underlying the scalar ratings reflects incentive structure of the game --- player 1 does not wish to deviate to the {\em Paper} action precisely because doing so would lead to losses against the {\em Scissors} action despite wins against the two {\em Rock} actions.

\color{black}

\section{Vulnerability of Standard Shannon Entropy}
\label{app:vulnerability}

\textcolor{black}{Prior work has shown max-entropy Nash equilibrium (equivalently max-entropy (C)CE) to be invariant to clones in \tpzs games~\citep{balduzzi2018re}. We include a simple experiment here to illustrate why max-entropy Nash equilibrium becomes vulnerable to redundancy in the \npgs setting.}


\paragraph{Chicken Game}

Consider the $2$-player $2$-action general-sum {\em Chicken} game. Let players receive $0$ if they both {\em swerve}. If one player swerves while the other goes straight, the one who swerves receives $-1$ and the other $+1$. If both go straight, then they both receive $-12$. This game has three NEs. Two are pure in which one player goes straight and the other swerves. The third is symmetric and the max-entropy NE of this game; each player swerves with probability $\sfrac{11}{12}$. Both \emph{straight} and \emph{swerve} have an expected payoff of $-\sfrac{1}{12}$ under this NE. If we duplicate the \emph{straight} action, the original max-entropy NE becomes the \emph{min}-entropy NE!
The other two NEs representing each player swerving while the other goes straight now have higher entropy. 
The player that swerves rates their swerve and straight actions as $-1$ and $-12$ respectively. The player that goes straight rates their swerve and straight actions as $0$ and $1$ respectively, demonstrating that the max-entropy NE solution concept is not invariant to clones in the general-sum setting.

The story in the max-entropy CCE setting is more nuanced. We find that although the CCE ratings change under the addition of clones, the ratio of the ratings of the two actions remains stable. Further investigation is necessary to understand whether max-entropy CCE ratings are equivariant (robust up to affine transformations of the ratings) to cloned actions.

\def\sgtextcolor{black}%
\def\sglinecolor{black}%
\begin{figure}[htb]\hspace*{\fill}%
\begin{game}{2}{2}
& Swerve & Straight \\
Swerve & $0,0$ & $-1,+1$ \\
Straight & $+1,-1$ & $-12,-12$
\end{game}\hspace*{\fill}%
\hspace*{\fill}%
\begin{game}{3}{3}
& Swerve & Straight & Straight \\
Swerve & $0,0$ & $-1,+1$ & $-1,+1$ \\
Straight & $+1,-1$ & $-12,-12$ & $-12,-12$ \\
Straight & $+1,-1$ & $-12,-12$ & $-12,-12$
\end{game}\hspace*{\fill}%
\caption[]{\textcolor{black}{Chicken Game and Chicken Game with Duplicate Straight Actions.}}
\end{figure}

\begin{figure}
    \centering
    \includegraphics[width=0.9\textwidth]{figures/ne_rating_chicken.pdf}
    \caption{\textcolor{black}{We visualise the marginal NE rating contributions of each player 2 action to each player 1 action. We show that a) all actions receive zero ratings and b) the rating of each action is interpretable and corresponds to our intuition.}}
    \label{fig:chicken_marginal_contrib}
\end{figure}

\textcolor{black}{By contrast, we show in Figure~\ref{fig:chicken_marginal_contrib} that all actions would receive zero ratings under our proposed equilibrium ratings. In other words, our equilibrium selection procedure continues to select the mixed-strategy NE in the original game, unaffected by the additional redundant ``straight'' action. Further, the widths of the bars are interpretable: suggesting that deviating to the {\em Swerve} action is a safe option without major risk or reward. Deviating to one of the {\em Straight} actions however, can lead to high rewards but also catastrophic losses.}
