\section{Discussion}

This study presents PathFinder, a multi-modal, multi-agent AI framework designed to emulate the multi-scale, iterative diagnostic approach of expert pathologists for histopathology whole slide images (WSIs). By integrating Triage, Navigation, Description, and Diagnosis Agents, PathFinder collaboratively gathers evidence to deliver accurate, interpretable diagnoses with natural language explanations. Notably, it surpasses state-of-the-art methods and the average performance of human experts in melanoma diagnosis, setting a new benchmark in AI-driven pathology.

PathFinder has the potential to accelerate diagnostic workflows, reducing the reliance on manual examination and enabling timely patient care in clinical settings. Its natural language descriptions provide interpretability, facilitating the validation of AI-generated diagnoses by pathologists. Moreover, its integration of vision-language models (VLMs) and large language models (LLMs) highlights the promise of multi-modal AI in delivering scalable, specialized diagnostic tools that could improve access to pathology expertise.

\noindent\textbf{Limitations.} Despite its strengths, PathFinder has limitations. The framework relies on pre-existing datasets and significant computational resources, posing challenges in resource-constrained environments. Additionally, the complexity of the Navigation Agent’s decision-making process and occasional hallucinations by the Description Agent could affect transparency and accuracy of the decision-making process. Future work should address these issues by enhancing dataset diversity, computational efficiency, and patch selection strategies, further advancing PathFinder's potential as a transformative tool in AI-assisted pathology.
