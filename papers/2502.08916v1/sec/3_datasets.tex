\section{Datasets}
To lay the groundwork for describing our agents, we first start by introducing the different datasets used for training and evaluating our system.

\label{data-mpath}
\noindent\textbf{M-Path Skin Biopsy WSIs.} The skin biopsy WSIs in this dataset originate from M-Path study \cite{elmore2017pathologists, carney2016achieving, onega2018accuracy}, consisting of 238 melanocytic lesion specimens stained with Hematoxylin and eosin (H\&E). A consensus reference panel of three dermatopathologists, each with internationally recognized expertise, independently interpreted all 238 cases and established a consensus diagnosis for each case through a series of review meetings. There are 4 diagnostic classes in this dataset: class 1 with 35 cases (mild and moderate dysplastic nevi); class 2 with 86 cases (severe dysplasia/melanoma in situ); class 3 with 70 cases (invasive melanoma stage pT1a); and class 4 with 47 cases (advanced invasive melanoma stage pT1b or more). For model development, the dataset is divided into training, validation, and test sets with a 168/35/35 case split, maintaining consistent class distribution across these sets.

\noindent\textbf{M-Path Pathologists’ Viewport Data.}
The M-Path study conducted viewport data collection, recruiting 87 pathologists from 10 U.S. states. Eligibility criteria included completion of residency and/or fellowship training and recent experience interpreting skin specimens in clinical practice. Pathologists’ viewport data was gathered through an online digital slide viewer developed using Microsoft’s open-source Silverlight-based HD View SL, a gigapixel image viewer. This viewer enabled pathologists to navigate each image by panning and zooming up to 60x magnification. During interpretation, the web-based viewer automatically logged viewport tracking data, capturing a rectangular image area displayed on the pathologist's screen at any given moment. For each interpretation (unique pathologist-case pair), the system recorded a list of viewport coordinates, magnification levels, and timestamps. Data from 32 pathologists who completed the M-Path study were included in the current study. Detailed methodology of the M-Path study is available in \cite{onega2018accuracy}.



