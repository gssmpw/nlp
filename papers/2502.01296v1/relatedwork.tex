\section{RELATED WORK}
Molecular odor prediction has been a key research focus in chemistry, neuroscience, and computer science. With the rise of machine learning techniques in recent years, many studies have turned to computational methods to predict the olfactory properties of molecules. Early research primarily investigated the relationship between molecular structure and odor through chemical parameters. For instance, PaDEL-Descriptor\cite{7} computes 797 molecular descriptors and 10 types of fingerprints, including electro-topological state descriptors and molecular volume, which are crucial for quantitative structure–activity relationship (QSAR) studies. However, despite its extensive descriptor library, PaDEL-Descriptor faces limitations in processing speed and the ability to handle large molecules. To address these challenges, Mordred\cite{8} introduced a more advanced descriptor calculation tool, capable of computing over 1,800 2D and 3D molecular descriptors. Mordred is at least twice as fast as PaDEL, and can compute large molecular descriptors that other software cannot handle. With its high performance, ease of use, and comprehensive descriptor library, Mordred has become a key tool in cheminformatics, particularly for structure–property relationship studies.

While descriptor-based feature extraction remains vital, machine learning approaches are increasingly leading molecular odor prediction research. Graph neural networks (GNNs)\cite{9,10} have shown significant potential in modeling the complex relationship between molecular structure and odor perception. For example,\cite{9} introduced a GNN-based method that performs end-to-end learning, automatically extracting relevant features from molecular graphs. This method has shown superior performance in classifying odors, such as fruity, floral, and woody scents. More recently,\cite{10} utilized GNNs to generate an odor mapping that maintains perceptual relationships and supports quality prediction for uncharacterized odor molecules. In prospective validation on 400 unseen odor samples, the odor profiles generated by the POM model were closer to the mean of the training group than the median, confirming its reliability as a prediction tool. The model outperformed traditional cheminformatics methods, demonstrating success in encoding the structure–odor relationship. Additionally, OWSum\cite{11} proposed the Odor Weighted Sum (OWSum) algorithm, a linear classifier that combines structural patterns with conditional probabilities and tf-idf values for odor prediction. This approach not only improves the understanding of odor prediction but also offers valuable insights into odor descriptors by quantifying the semantic overlap among them, further advancing molecular odor prediction.