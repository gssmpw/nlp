\section{Related Work}
\paragraph{Table-to-Text Generation} 
Table-to-text generation has advanced through datasets tailored to diverse domains and applications, as summarized in Table \ref{tab:datasets}. Early efforts, such as WikiTableT \cite{chen2021wikitabletlargescaledatatotextdataset}, focused on generating natural language descriptions from Wikipedia tables, while TabFact \cite{2019TabFactA} introduced fact-checking capabilities and ROTOWIRE \cite{wiseman2017challengesdatatodocumentgeneration} generated detailed sports summaries. However, these datasets are limited in their relevance to product-specific domains. Later datasets like LogicNLG \cite{chen2020logicalnaturallanguagegeneration} emphasized logical inference and reasoning, and ToTTo \cite{parikh2020tottocontrolledtabletotextgeneration} supported controlled text generation by focusing on specific table regions. HiTab \cite{cheng-etal-2022-hitab} extended these capabilities with hierarchical table structures and reasoning operators. Despite these advancements, none of these datasets provide the contextual and attribute-specific depth necessary for e-commerce applications, where generating meaningful descriptions requires reasoning across heterogeneous attributes, such as linking battery capacity to battery life or associating display size with user experience. 


\paragraph{Query-Focused Summarization (QFS)} 
Advances in text summarization have improved multi-document summarization through abstractive methods like paraphrastic fusion \cite{10.1145/3132847.3133106, nayeem-etal-2018-abstractive}, compression \cite{10.1007/978-3-030-15719-7_14, chowdhury-etal-2021-unsupervised}, and diverse fusion models \cite{FUAD2019216, nayeem2017methods}, among others \cite{nayeem-chali-2017-extract, chali-etal-2017-towards}. These approaches lay the groundwork for query-focused summarization (QFS), which tailors summaries to user-specific queries. Initially formulated as a document summarization task, QFS aims to generate summaries tailored to specific user queries \cite{dang-2006-duc}. Despite its potential real-world applications, QFS remains a challenging task due to the lack of datasets. In the textual domain, QFS has been explored in multi-document settings \cite{giorgi-etal-2023-open} and meeting summarization \cite{zhong-etal-2021-qmsum}. Recent datasets like QTSumm \cite{zhao2023qtsummqueryfocusedsummarizationtabular} extend QFS to a new modality, using tables as input. However, QTSumm's general-purpose nature limits its applicability to product reviews, which require nuanced reasoning over attributes and user-specific contexts. Additionally, its queries are often disconnected from real-world e-commerce scenarios. In contrast, our proposed dataset, \textbf{eC-Tab2Text}, bridges this gap by providing attribute-specific and query-driven summaries tailored to e-commerce product tables.