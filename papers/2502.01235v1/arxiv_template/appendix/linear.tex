\newpage
\section{Proofs for Linear Model}
\label{lora_linear}


In \cref{app:align_linear}, we deliver the proofs for alignment in \cref{sec:align_linear}.
In \cref{app:lrspec}, we present the proofs for the main results in \cref{sec:linear-spectral} under spectral initialization.
In \cref{app:precgdlr}, we give the proofs for precondition GD in \cref{sec:scaledgd}.

\subsection{Proofs for LoRA under Random Initialization}
\label{app:align_linear}

Let $\widetilde{\bm X}$ be the fine-tuned data with $\widetilde{\bm X} \in \mathbb{R}^{N \times d}$ and the multi-output $\widetilde{\bm Y} \in \mathbb{R}^{N \times k}$.
For simplicity, we define the initial residual error $\widetilde{\bm Y}_{\Delta} := \widetilde{\bm Y} - \widetilde{\bm X}\bm W^\natural = \widetilde{\bm X}\Delta$. Then, denote the negative gradient of Full Fine-tuning after the first step as
\begin{align*}
  {\bm G}^{\natural} & = -\nabla_{\bm W} {L}(\bm W^\natural) = -\frac{1}{N}\widetilde{\bm X}^{\!\top}(\widetilde{\bm X}\bm W^\natural-\widetilde{\bm Y}) = \frac{1}{N}\widetilde{\bm X}^{\!\top}\widetilde{\bm Y}_{\Delta} \in \mathbb{R}^{d \times k}\, .
\end{align*}

Recall the gradient update for LoRA
\begin{equation*}
    \bm A_{t+1} = \bm A_{t}-\frac{\eta_1}{N} \widetilde{\bm X}^{\!\top} \Bigl(\widetilde{\bm X} (\bm W^\natural+\bm A_t \bm B_t) - \widetilde{\bm Y}\Bigr) \bm B^{\!\top}_t\, ,
\end{equation*}
\begin{equation*}
    \bm B_{t+1} = \bm B_t -\frac{\eta_2}{N} \bm A^{\!\top}_t \widetilde{\bm X}^{\!\top} \Bigl(\widetilde{\bm X} (\bm W^\natural+\bm A_t \bm B_t) - \widetilde{\bm Y}\Bigr)\,,
\end{equation*}
we rewrite it in a compact form
\begin{equation}\label{eq:dynamics}
    \begin{split}
    \begin{bmatrix}
        \bm A_{t+1} \\ \bm B_{t+1}^{\!\top}
    \end{bmatrix} & = \begin{bmatrix}
        \bm A_t \\ \bm B_t^{\!\top}
    \end{bmatrix} + \underbrace{\begin{bmatrix}
        \bm 0 & \eta_1 {\bm G}^{\natural}\\
        \eta_2 {{\bm G}^{\natural}}^{\!\top} & \bm 0
    \end{bmatrix}}_{:=\bm H} \begin{bmatrix}
        \bm A_t \\ \bm B_t^{\!\top}
    \end{bmatrix} - \frac{1}{N} \begin{bmatrix}
        \bm 0 & \eta_1 \widetilde{\bm X}^{\!\top}\widetilde{\bm X}\bm A_t \bm B_t\\
        \eta_2 \bm B_t^{\!\top} \bm A_t^{\!\top}\widetilde{\bm X}^{\!\top}\widetilde{\bm X} & \bm 0
    \end{bmatrix}\begin{bmatrix}
        \bm A_t \\ \bm B_t^{\!\top}
    \end{bmatrix}\\
    & = \underbrace{\begin{bmatrix}
        \bm I_d & \eta_1 {\bm G}^{\natural}\\
        \eta_2 {{\bm G}^{\natural}}^{\!\top} & \bm I_k
    \end{bmatrix}}_{:=\bm H} \begin{bmatrix}
        \bm A_t \\ \bm B_t^{\!\top}
    \end{bmatrix} - \underbrace{\frac{1}{N} \begin{bmatrix}
        \bm 0 & \eta_1 \widetilde{\bm X}^{\!\top}\widetilde{\bm X}\bm A_t \bm B_t\\
        \eta_2 \bm B_t^{\!\top} \bm A_t^{\!\top}\widetilde{\bm X}^{\!\top}\widetilde{\bm X} & \bm 0
    \end{bmatrix}\begin{bmatrix}
        \bm A_t \\ \bm B_t^{\!\top}
    \end{bmatrix}}_{:=\widehat{\bm E}_{t+1}}\, .     
    \end{split}
\end{equation}
By defining a stack iterate
\begin{align}
    \bm Z_t & := \begin{bmatrix}
        \bm A_t \\ \bm B_t^{\!\top}
    \end{bmatrix}\, , \quad \mbox{and} \quad \bm Z_0 := \begin{bmatrix}
        \bm A_0 \\ \bm 0
    \end{bmatrix}\in \mathbb{R}^{(d+k)\times r}\, , \label{stack-Z}
\end{align}
we can formulate \cref{eq:dynamics} as a compact form of a nonlinear dynamical system  
\begin{align}\label{eq:nonlineardy}
    \bm Z_{t+1} & = \bm H \bm Z_t - \widehat{\bm E}_{t+1}\,,
\end{align}
where $\bm H$ is a time-independent matrix corresponding to the linear part, and $\widehat{\bm E}_{t+1}$ corresponds to the nonlinear part.

\subsubsection{SVD and Schur Decomposition}
\label{app:svd}

We recall the complete SVD of $\Delta \in \mathbb{R}^{d \times k}$ 
\begin{align*}
\Delta=\widetilde{\bm U} \widetilde{\bm S} \widetilde{\bm V}^{\!\top}=
    \begin{bmatrix}
        \bm U & \bm U_\perp
    \end{bmatrix}\begin{bmatrix}
       \bm S^* & \bm 0_{r^*\times (k-r^*)}\\
        \bm 0_{(d-r^*)\times r^*} & \bm 0_{(d-r^*)\times (k-r^*)}
    \end{bmatrix}\begin{bmatrix}
        \bm V^{\!\top} \\ \bm V_\perp^{\!\top}
    \end{bmatrix},\quad \text{where }\bm S^* = \operatorname{Diag}\left(\lambda_1^*\,, \cdots \,,\lambda_{r^*}^*\right)\,.
\end{align*}

Similarly, we recall the complete SVD of ${\bm G}^{\natural}$ as ${\bm G}^{\natural} =\widetilde{\bm U}_{\bm G^\natural} \widetilde{\bm S}_{\bm G^\natural} \widetilde{\bm V}_{\bm G^\natural}^{\!\top}$. We derive the Schur decomposition of $\bm H$ under the special case $d=k$ in \cref{H-schur} and then extend to $d\neq k$ in \cref{lemma:Hdnk} via zero padding on SVD in \cref{zero-block-svd}.
\begin{lemma}[Schur Decomposition of $\bm H$ under $d=k$]
\label{H-schur}
    Under assumptions in \cref{sec:assumptions} for the linear setting, given ${\bm G}^{\natural}\in\mathbb{R}^{d\times k}$ in \cref{eq:G} and its complete SVD $\widetilde{\bm U}_{\bm G^\natural}\widetilde{\bm S}_{\bm G^\natural}\widetilde{\bm V}_{\bm G^\natural}^{\!\top}$, if $d=k$, then the block matrix $\bm H$ admits the following Schur decomposition
    \begin{align*}
        \bm H = 
        \begin{bmatrix}
            \bm I_d & \eta_1 {\bm G}^{\natural}\\
            \eta_2 \left({\bm G}^{\natural}\right)^{\!\top} & \bm I_d
        \end{bmatrix} = \mathbf{C}\mathbf{T}\mathbf{C}^{\!\top}\,,
    \end{align*}
    where $\bm C$ is an orthogonal matrix and $\bm T$ is a block upper triangular matrix
    \begin{align*}
        \mathbf{C} & = \frac{1}{\sqrt{1+\frac{\eta_2}{\eta_1}}}\begin{bmatrix}
        \widetilde{\bm U}_{\bm G^\natural} & -\sqrt{\frac{\eta_2}{\eta_1}}\widetilde{\bm U}_{\bm G^\natural}\\
        \sqrt{\frac{\eta_2}{\eta_1}}\widetilde{\bm V}_{\bm G^\natural} & \widetilde{\bm V}_{\bm G^\natural}
    \end{bmatrix}\,,
   \quad \mbox{and} \quad
        \mathbf{T} = \begin{bmatrix}
        \bm I_d+\sqrt{\eta_1\eta_2}\widetilde{\bm S}_{\bm G^\natural} & (\eta_1 - \eta_2)\widetilde{\bm S}_{\bm G^\natural} \\
        \bm 0 & \bm I_d-\sqrt{\eta_1\eta_2}\widetilde{\bm S}_{\bm G^\natural}
        \end{bmatrix}\,.
    \end{align*}
\end{lemma}
\begin{proof}
    We prove by verifying the claim. Starting with
    \begin{align*}
        & \begin{bmatrix}
        \widetilde{\bm U}_{\bm G^\natural} & -\sqrt{\frac{\eta_2}{\eta_1}}\widetilde{\bm U}_{\bm G^\natural}\\
        \sqrt{\frac{\eta_2}{\eta_1}}\widetilde{\bm V}_{\bm G^\natural} & \widetilde{\bm V}_{\bm G^\natural}
        \end{bmatrix}\begin{bmatrix}
        \bm I_d+\sqrt{\eta_1\eta_2}\widetilde{\bm S}_{\bm G^\natural} & (\eta_1 - \eta_2)\widetilde{\bm S}_{\bm G^\natural} \\
        \bm 0 & \bm I_d-\sqrt{\eta_1\eta_2}\widetilde{\bm S}_{\bm G^\natural}
        \end{bmatrix}\\
        & =  \begin{bmatrix}
            \widetilde{\bm U}_{\bm G^\natural} + \sqrt{\eta_1\eta_2}\widetilde{\bm U}_{\bm G^\natural}\widetilde{\bm S}_{\bm G^\natural} & \eta_1\widetilde{\bm U}_{\bm G^\natural}\widetilde{\bm S}_{\bm G^\natural} - \sqrt{\frac{\eta_2}{\eta_1}}\widetilde{\bm U}_{\bm G^\natural}\\
            \sqrt{\frac{\eta_2}{\eta_1}}\widetilde{\bm V}_{\bm G^\natural}+\eta_2\widetilde{\bm V}_{\bm G^\natural}\widetilde{\bm S}_{\bm G^\natural} & \sqrt{\frac{\eta_2}{\eta_1}}(\eta_1 - \eta_2)\widetilde{\bm V}_{\bm G^\natural}\widetilde{\bm S}_{\bm G^\natural} + \widetilde{\bm V}_{\bm G^\natural} - \sqrt{\eta_1\eta_2}\widetilde{\bm V}_{\bm G^\natural}\widetilde{\bm S}_{\bm G^\natural}
        \end{bmatrix} =: \bm \Xi \,,
    \end{align*}
    then we can verify that
    \begin{align*}
    & \frac{\eta_1}{\eta_1 + \eta_2}\times \bm \Xi \times\begin{bmatrix}
        \widetilde{\bm U}_{\bm G^\natural}^{\!\top} & \sqrt{\frac{\eta_2}{\eta_1}}\widetilde{\bm V}_{\bm G^\natural}^{\!\top} \\
        -\sqrt{\frac{\eta_2}{\eta_1}}\widetilde{\bm U}_{\bm G^\natural}^{\!\top} & \widetilde{\bm V}_{\bm G^\natural}^{\!\top}
    \end{bmatrix}\\
    = & \frac{\eta_1}{\eta_1 + \eta_2}\times\begin{bmatrix}
        \left(1+\frac{\eta_2}{\eta_1}\right)\bm I_d & (\eta_1 + \eta_2) \widetilde{\bm U}_{\bm G^\natural}\widetilde{\bm S}_{\bm G^\natural}\widetilde{\bm V}_{\bm G^\natural}^{\!\top}\\
        \eta_2 \left(1+\frac{\eta_2}{\eta_1}\right)\widetilde{\bm V}_{\bm G^\natural}\widetilde{\bm S}_{\bm G^\natural}\widetilde{\bm U}_{\bm G^\natural}^{\!\top} & \left(1+\frac{\eta_2}{\eta_1}\right)\bm I_d
    \end{bmatrix}\\
    = & \begin{bmatrix}
        \bm I_d & \eta_1 \widetilde{\bm U}_{\bm G^\natural}\widetilde{\bm S}_{\bm G^\natural}\widetilde{\bm V}_{\bm G^\natural}^{\!\top}\\
        \eta_2 \widetilde{\bm V}_{\bm G^\natural}\widetilde{\bm S}_{\bm G^\natural}\widetilde{\bm U}_{\bm G^\natural}^{\!\top} & \bm I_d
    \end{bmatrix}=\bm H\,.
    \end{align*}
    Accordingly, we conclude the result.
\end{proof}
Next, we consider the case of $d\neq k$.
\paragraph{Case 1 ($d>k$):} by zero padding, ${\bm G}^{\natural}$ and related matrices are given by
\begin{align*}
    \underline{\mathbf{G}}^\natural & = \begin{bmatrix}
        {\bm G}^{\natural} & \bm 0_{d \times (d-k)}
    \end{bmatrix}\,,\quad \underline{\bm H} = \begin{bmatrix}
        \bm I_d & \eta_1 \underline{\mathbf{G}}^\natural\\
        \eta_2 \left(\underline{\mathbf{G}}^\natural\right)^{\!\top} & \bm I_d
    \end{bmatrix}\,,
\end{align*}
and for any $t\geq 0$, we have the following related matrices
\begin{align*}
    \underline{\bm B}_t & = \begin{bmatrix}
        \bm B_t & \bm 0_{r\times (d-k)}
    \end{bmatrix}\,,\quad \underline{\bm Z}_t = \begin{bmatrix}
        \bm A_t \\ \left(\underline{\bm B}_t\right)^{\!\top}
    \end{bmatrix}\,,\quad \underline{\widetilde{\bm Z}}_t = \begin{bmatrix}
        \bm A^{\tt lin}_t \\
        \left(\underline{\bm B^{\tt lin}}_t\right)^{\!\top}
    \end{bmatrix} = \underline{\bm H}^t \underline{\bm Z}_0\,.
\end{align*}
\paragraph{Case 2 ($d<k$):} Similarly, by zero padding, we define
\begin{align*}
    \underline{\mathbf{G}}^\natural & = \begin{bmatrix}
        {\bm G}^{\natural} \\ \bm 0_{(k-d) \times k}
    \end{bmatrix}\,,\quad \underline{\bm H} = \begin{bmatrix}
        \bm I_k & \eta_1 \underline{\mathbf{G}}^\natural\\
        \eta_2 \left(\underline{\mathbf{G}}^\natural\right)^{\!\top} & \bm I_k
    \end{bmatrix}\,,
\end{align*}
and for $\forall\,t\geq 0$, we define
\begin{align*}
    \underline{\bm A}_t & = \begin{bmatrix}
        \bm A_t \\ \bm 0_{(k-d)\times r}
    \end{bmatrix}\,,\quad \underline{\bm Z}_t = \begin{bmatrix}
        \underline{\bm A}_t \\ \left(\bm B_t\right)^{\!\top}
    \end{bmatrix}\,,\quad \underline{\widetilde{\bm Z}}_t = \begin{bmatrix}
        \underline{\bm A^{\tt lin}}_t\\
        \left(\bm B^{\tt lin}_t\right)^{\!\top}
    \end{bmatrix} = \underline{\bm H}^t \underline{\bm Z}_0\,.
\end{align*}
Then we have the following lemma on the SVD of $\underline{\mathbf{G}}^\natural$.
\begin{lemma}
\label{zero-block-svd}
    If $d>k$, then we have the following SVD of $\underline{\mathbf{G}}^\natural$
    \begin{align*}
        \underline{\mathbf{G}}^\natural & = \widetilde{\bm U}_{\bm G^\natural} \underline{\widetilde{\bm S}_{\bm G^\natural}} \underline{\widetilde{\bm V}_{\bm G^\natural}}^{\!\top}\,,
    \end{align*}
    where
    \begin{align*}
        \underline{\widetilde{\bm V}_{\bm G^\natural}} & = \begin{bmatrix}
            \widetilde{\bm V}_{\bm G^\natural} & \bm 0_{k\times (d-k)}\\
            \bm 0_{(d-k) \times k} & \bm I_{(d-k)}
        \end{bmatrix}\,, \quad \text{and} \quad \underline{\widetilde{\bm S}_{\bm G^\natural}} = \begin{bmatrix}
       \widetilde{\bm S}_{\bm G^\natural} & \bm 0_{d\times (d-k)}
       \end{bmatrix}\,.
    \end{align*}
    If $d<k$, then we have the following SVD of $\underline{\mathbf{G}}^\natural$
    \begin{align*}
        \underline{\mathbf{G}}^\natural & = \underline{\widetilde{\bm U}_{\bm G^\natural}} \underline{\widetilde{\bm S}_{\bm G^\natural}} \widetilde{\bm V}_{\bm G^\natural}^{\!\top}\,,
    \end{align*}
    where
    \begin{align*}
        \underline{\widetilde{\bm U}_{\bm G^\natural}} & = \begin{bmatrix}
            \widetilde{\bm U}_{\bm G^\natural} & \bm 0_{k\times (k-d)}\\
            \bm 0_{(k-d) \times k} & \bm I_{(k-d)}
        \end{bmatrix}\,, \quad \text{and} \quad \underline{\widetilde{\bm S}_{\bm G^\natural}} = \begin{bmatrix}
       \widetilde{\bm S}_{\bm G^\natural} \\ \bm 0_{(k-d)\times k}
       \end{bmatrix}\,.
    \end{align*}
\end{lemma}
\begin{proof}
    The block construction does not affect the original part of the SVD. It only appends zeros to the singular values and grows the corresponding orthonormal bases as partial identity matrices appropriately.
\end{proof}
Now we can apply Lemma~\ref{zero-block-svd} for Lemma~\ref{H-schur} to extend to $d\neq k$ via the following lemma.
The proof is direct and we omit it here.
\begin{lemma}[Schur decomposition of $\bm H$ under $d \neq k$]
\label{lemma:Hdnk}
    Given the defined block matrix $\underline{\bm H} \in \mathbb{R}^{2s \times 2s}$ with $s:=\max\{ d,k \}$, we have the following decomposition
    \begin{align*}
        \underline{\bm H} & = \mathbf{C}\mathbf{T}\mathbf{C}^{\!\top}\,,
    \end{align*}
    If $d>k$,
    \begin{align*}
        \mathbf{C} & = \frac{1}{\sqrt{1+\frac{\eta_2}{\eta_1}}}\begin{bmatrix}
        \widetilde{\bm U}_{\bm G^\natural} & -\sqrt{\frac{\eta_2}{\eta_1}}\widetilde{\bm U}_{\bm G^\natural}\\
        \sqrt{\frac{\eta_2}{\eta_1}}\underline{\widetilde{\bm V}_{\bm G^\natural}} & \underline{\widetilde{\bm V}_{\bm G^\natural}}
    \end{bmatrix}\,,\quad
        \mathbf{T} = \begin{bmatrix}
        \bm I_d+\sqrt{\eta_1\eta_2}\underline{\widetilde{\bm S}_{\bm G^\natural}} & (\eta_1 - \eta_2)\underline{\widetilde{\bm S}_{\bm G^\natural}} \\
        \bm 0 & \bm I_d-\sqrt{\eta_1\eta_2}\underline{\widetilde{\bm S}_{\bm G^\natural}}
        \end{bmatrix}\,.
    \end{align*}
    If $d<k$,
    \begin{align*}
        \mathbf{C} & = \frac{1}{\sqrt{1+\frac{\eta_2}{\eta_1}}}\begin{bmatrix}
        \underline{\widetilde{\bm U}_{\bm G^\natural}} & -\sqrt{\frac{\eta_2}{\eta_1}}\underline{\widetilde{\bm U}_{\bm G^\natural}}\\
        \sqrt{\frac{\eta_2}{\eta_1}}{\widetilde{\bm V}_{\bm G^\natural}} & {\widetilde{\bm V}_{\bm G^\natural}}
    \end{bmatrix}\,,\quad
        \mathbf{T} = \begin{bmatrix}
        \bm I_k+\sqrt{\eta_1\eta_2}\underline{\widetilde{\bm S}_{\bm G^\natural}} & (\eta_1 - \eta_2)\underline{\widetilde{\bm S}_{\bm G^\natural}} \\
        \bm 0 & \bm I_k-\sqrt{\eta_1\eta_2}\underline{\widetilde{\bm S}_{\bm G^\natural}}
        \end{bmatrix}\,.
    \end{align*}
\end{lemma}


\subsubsection{Dynamics of Linear Approximation}
The target of our proof is to demonstrate that $\widehat{\bm E}_{t+1}$ does not effect the dynamics too much such that the dynamics of $\bm Z_{t}$ is close to the following pseudo iterate
\begin{align}\label{eq:pseduo_iterate}
    \bm Z^{\tt lin}_t := \bm H^t \bm Z_0 =: \begin{bmatrix}
    \bm A^{\tt lin}_t \\ \left(\bm B^{\tt lin}_t\right)^{\!\top}
\end{bmatrix} \,.
\end{align}
The updates of the pseudo iterate follow the trajectory of Oja's Power Method \citep{oja1982simplified}. Therefore, we aim to prove that the error between the actual iterate $\bm Z_t$ and the pseudo iterate $\bm Z^{\tt lin}_t$ is sufficiently small, which is equivalent to that the actual iterate $\bm Z_t$ performs a power iteration during the early steps. First, 
we obtain the difference between $\bm Z_t$ and $\bm Z^{\tt lin}_t$ by the following lemma.
\begin{lemma}[Formulation of $\bm E_t$]
\label{induc}
Under assumptions in \cref{sec:assumptions} for the linear setting, given the nonlinear dynamical system \eqref{eq:nonlineardy} and its linear part \eqref{eq:pseduo_iterate}, their difference admits
    \begin{align}
      \bm E_t := \bm Z_t - \bm Z^{\tt lin}_t = - \sum_{i=1}^t \bm H^{t-i} \widehat{\bm E}_{i}\label{pseudo-error} \,, \quad \forall t \in \mathbb{N}^+\,,
    \end{align}
where $\widehat{\bm E}_{i}$ corresponds to the nonlinear part in \cref{eq:dynamics}.
\end{lemma}

\begin{proof}[Proof of \cref{induc}]
We prove it by induction.
Recall the formulation of the nonlinear dynamical system
$\bm Z_{t+1} = \bm H \bm Z_t - \widehat{\bm E}_{t+1}$, we start with the base case $t=1$ such that
    \begin{align*}
        \bm Z_1 & = \bm H \bm Z_0 - \widehat{\bm E}_{1} = \widetilde{\bm Z}_1 - \widehat{\bm E}_{1}\, ,
    \end{align*}
    which proves the claim. Next, we assume Eq.~\eqref{pseudo-error} holds for $t\geq 2$, then for $t+1$, we have
    \begin{align*}
        \bm Z_{t+1} & = \bm H \bm Z_t - \widehat{\bm E}_{t+1}\\
        & = \bm H \left(\bm Z^{\tt lin}_t - \sum_{i=1}^t \bm H^{t-i} \widehat{\bm E}_{i}\right)- \widehat{\bm E}_{t+1}\\
        & = \bm Z^{\tt lin}_{t+1} - \sum_{i=1}^{t} \bm H^{t+1-i} \widehat{\bm E}_{i} - \widehat{\bm E}_{t+1}\\
        & = \bm Z^{\tt lin}_{t+1} - \sum_{i=1}^{t+1} \bm H^{t+1-i} \widehat{\bm E}_{i}\, ,
    \end{align*}
    which proves the claim.
\end{proof}

If $\|\bm E_t\|_{op}$ is sufficiently small within a certain period, e.g., $t \leq T$, then we could approximate the early dynamics by
\begin{align*}
 \bm Z_{t+1} :=
    \begin{bmatrix}
       \bm A_{t+1} \\ \bm B_{t+1}^{\!\top}
    \end{bmatrix} & \approx \bm Z^{\tt lin}_t := \begin{bmatrix}
    \bm A^{\tt lin}_{t+1} \\ \left(\bm B^{\tt lin}_{t+1}\right)^{\!\top}
\end{bmatrix}=
\begin{bmatrix}
    \bm A^{\tt lin}_t \\ \left(\bm B^{\tt lin}_{t}\right)^{\!\top}
\end{bmatrix} + 
    \begin{bmatrix}
        \bm 0 & \eta_1{\bm G}^{\natural}\\
        \eta_2{{\bm G}^{\natural}}^{\!\top} & \bm 0
    \end{bmatrix} \begin{bmatrix}
    \bm A^{\tt lin}_t \\ \left(\bm B^{\tt lin}_{t}\right)^{\!\top}
\end{bmatrix}\,,
\end{align*}
via
\begin{align*}
    \left\|\begin{bmatrix}
        \bm A_t \\ \bm B_t^{\!\top}
    \end{bmatrix}-\begin{bmatrix}
        \bm A^{\tt lin}_t \\ \left(\bm B^{\tt lin}_t\right)^{\!\top}
    \end{bmatrix}\right\|_{op} & \leq \|\bm E_t\|_{op}\,.
\end{align*}
In this subsection, we will bound $\|\bm E_t\|_{op}$ to show that it is actually small up to the initialization.
To prove it, we first conduct the dynamical analysis of $\bm Z^{\tt lin}_t$ via the structure of $\bm H$.\\

\noindent
{\bf Part I: Dynamics of $\bm Z^{\tt lin}_t$}\\

With the algebra fact above, we can derive the precise spectral dynamics of $\bm Z^{\tt lin}_t$, i.e., $\bm A^{\tt lin}_t$ and $\bm B^{\tt lin}_t$ separately.
\begin{lemma}\label{psuedo-dynamics}
Under assumptions in \cref{sec:assumptions} for the linear setting, given the pseudo iterate \eqref{eq:pseduo_iterate} on $\bm Z^{\tt lin}_t$, where two components $\bm A^{\tt lin}_t$ and $\bm B^{\tt lin}_t$ admit the following recursion
    \begin{align*}
        \left\{\begin{aligned}
            \bm A^{\tt lin}_t & = \underbrace{\frac{1}{2}\widetilde{\bm U}_{\bm G^\natural}\bigg(\left(\bm I_d + \sqrt{\eta_1 \eta_2}\widetilde{\bm S}_{\bm G^\natural}\right)^t + \left(\bm I_d - \sqrt{\eta_1 \eta_2}\widetilde{\bm S}_{\bm G^\natural}\right)^t\bigg)\widetilde{\bm U}_{\bm G^\natural}^{\!\top}}_{:= \bm P_t^{\bm A}}\bm A_0\,,\\
            \left(\bm B^{\tt lin}_t\right)^{\!\top} & = \underbrace{\frac{1}{2}\sqrt{\frac{\eta_2}{\eta_1}}\widetilde{\bm V}_{\bm G^\natural}\bigg(\left(\bm I_d + \sqrt{\eta_1 \eta_2}\widetilde{\bm S}_{\bm G^\natural}\right)^t - \left(\bm I_d - \sqrt{\eta_1 \eta_2}\widetilde{\bm S}_{\bm G^\natural}\right)^t\bigg)\widetilde{\bm U}_{\bm G^\natural}^{\!\top}}_{:=\bm P_t^{\bm B}}\bm A_0\,.
        \end{aligned}\right.
    \end{align*}
    Furthermore, if $\widetilde{\bm X}^{\!\top}\widetilde{\bm X}$ is non-singular, $ \bm P_t^{\bm A}$ is a full rank matrix and singular values are 1 after the $r^*$-th order. $ \bm P_t^{\bm B}$ is a rank-$r^*$ matrix.
\end{lemma}

\begin{proof}
    We start with the special case $d=k$ and then discuss the case of $d \neq k$.
    For the case of $d = k$, we have
    \begin{align*}
        \bm Z^{\tt lin}_t = \bm H^t \bm Z_0 = (\mathbf{C}\mathbf{T}\mathbf{C}^{\!\top})^t\bm Z_0 = \mathbf{C}\mathbf{T}^t\mathbf{C}^{\!\top}\bm Z_0\,,
    \end{align*}
    where the last equality follows from the fact that $\mathbf{C}$ is an orthogonal matrix. Next, we compute $\mathbf{T}^t$
    \begin{align}
        \mathbf{T}^t & = \begin{bmatrix}
            \left(\bm I_d + \sqrt{\eta_1 \eta_2}\widetilde{\bm S}_{\bm G^\natural}\right)^t & \underbrace{(\eta_1 - \eta_2) \widetilde{\bm S}_{\bm G^\natural}\left(\sum_{j=0}^{t-1} \left(\bm I_d - \sqrt{\eta_1 \eta_2}\widetilde{\bm S}_{\bm G^\natural}\right)^{t-j-1}\left(\bm I_d + \sqrt{\eta_1 \eta_2}\widetilde{\bm S}_{\bm G^\natural}\right)^{j}\right)}_{:= \mathbf{D}^t}\\
            \bm 0_{d\times d} & \left(\bm I_d - \sqrt{\eta_1 \eta_2}\widetilde{\bm S}_{\bm G^\natural}\right)^t
        \end{bmatrix}\label{T^t}\,.
    \end{align}
    We first deal with the upper-right part $\mathbf{D}^t$.
    It is a mix of several diagonal matrices under addition and multiplications, leading to be a diagonal matrix again, i.e., $\mathbf{D}^t_{(i,j)}=0, \forall i \neq j$. 
   Note that the diagonal matrix $\widetilde{\bm S}_{\bm G^\natural}$ is a rank-$r^*$ matrix, we have
   $\mathbf{D}^t_{(i,i)}=0$ for $(r^*+1)\leq i \leq d$. Accordingly, we only need to handle $\mathbf{D}^t_{(i,i)}$ in the $1\leq i \leq r^*$ part
    \begin{align*}
        \mathbf{D}^t_{(i,i)} & = (\eta_1 - \eta_2) \sigma_i^* \left(\sum_{j=0}^{t-1} \left(1 - \sqrt{\eta_1 \eta_2}\,\sigma_i^*\right)^{t-j-1}\left(1 + \sqrt{\eta_1 \eta_2}\,\sigma_i^*\right)^{j}\right)\\
        & = (\eta_1 - \eta_2) \sigma_i^* \frac{\left(1 + \sqrt{\eta_1 \eta_2}\,\sigma_i^*\right)^t - \left(1 - \sqrt{\eta_1 \eta_2}\,\sigma_i^*\right)^t}{ 2\sqrt{\eta_1 \eta_2}\,\sigma_i^*}\\
        & = \frac{\eta_1 - \eta_2}{2\sqrt{\eta_1 \eta_2}}\Bigg(\left(1 + \sqrt{\eta_1 \eta_2}\,\sigma_i^*\right)^t - \left(1 - \sqrt{\eta_1 \eta_2}\,\sigma_i^*\right)^t\Bigg)\,,
    \end{align*}
    where we use $\sum_{j=0}^{t-1}x^{t-j-1}y^j=\frac{x^t-y^t}{x-y}$. Therefore, we can conclude 
    \begin{align*}
        \mathbf{T}^t & = \begin{bmatrix}
            \left(\bm I_d + \sqrt{\eta_1 \eta_2}\widetilde{\bm S}_{\bm G^\natural}\right)^t & \frac{\eta_1 - \eta_2}{2\sqrt{\eta_1 \eta_2}}\Bigg(\left(\bm I_d + \sqrt{\eta_1 \eta_2}\widetilde{\bm S}_{\bm G^\natural}\right)^t - \left(\bm I_d - \sqrt{\eta_1 \eta_2}\widetilde{\bm S}_{\bm G^\natural}\right)^t\Bigg)\\
            \bm 0_{d\times d} & \left(\bm I_d - \sqrt{\eta_1 \eta_2}\widetilde{\bm S}_{\bm G^\natural}\right)^t
        \end{bmatrix}\,.
    \end{align*}
    Finally, we can derive the following recursion
    {\begin{align*}
    & \bm Z^{\tt lin}_t
    =  \bm H^t\bm Z_0\\
    = & \frac{1}{\sqrt{1+\frac{\eta_2}{\eta_1}}}\begin{bmatrix}
        \widetilde{\bm U}_{\bm G^\natural} & -\sqrt{\frac{\eta_2}{\eta_1}}\widetilde{\bm U}_{\bm G^\natural}\\
        \sqrt{\frac{\eta_2}{\eta_1}}\widetilde{\bm V}_{\bm G^\natural} & \widetilde{\bm V}_{\bm G^\natural}
    \end{bmatrix}\\
    & \times \begin{bmatrix}
            \left(\bm I_d + \sqrt{\eta_1 \eta_2}\widetilde{\bm S}_{\bm G^\natural}\right)^t & \frac{\eta_1 - \eta_2}{2\sqrt{\eta_1 \eta_2}}\Bigg(\left(\bm I_d + \sqrt{\eta_1 \eta_2}\widetilde{\bm S}_{\bm G^\natural}\right)^t - \left(\bm I_d - \sqrt{\eta_1 \eta_2}\widetilde{\bm S}_{\bm G^\natural}\right)^t\Bigg)\\
            \bm 0_{d\times d} & \left(\bm I_d - \sqrt{\eta_1 \eta_2}\widetilde{\bm S}_{\bm G^\natural}\right)^t\,.
        \end{bmatrix}\mathbf{C}^{\!\top}\bm Z_0\\
    = & \begin{bmatrix}
            \widetilde{\bm U}_{\bm G^\natural}\left(\bm I_d + \sqrt{\eta_1 \eta_2}\widetilde{\bm S}_{\bm G^\natural}\right)^t & \frac{\eta_1 - \eta_2}{2\sqrt{\eta_1 \eta_2}}\widetilde{\bm U}_{\bm G^\natural}\left(\bm I_d + \sqrt{\eta_1 \eta_2}\widetilde{\bm S}_{\bm G^\natural}\right)^t - \frac{1}{2}\left(\sqrt{\frac{\eta_1}{\eta_2}}+\sqrt{\frac{\eta_2}{\eta_1}}\right)\widetilde{\bm U}_{\bm G^\natural}\left(\bm I_d - \sqrt{\eta_1 \eta_2}\widetilde{\bm S}_{\bm G^\natural}\right)^t\\
            \sqrt{\frac{\eta_2}{\eta_1}}\widetilde{\bm V}_{\bm G^\natural}\left(\bm I_d + \sqrt{\eta_1 \eta_2}\widetilde{\bm S}_{\bm G^\natural}\right)^t & \frac{1}{2}\left(1-\frac{\eta_2}{\eta_1}\right)\widetilde{\bm V}_{\bm G^\natural}\left(\bm I_d + \sqrt{\eta_1 \eta_2}\widetilde{\bm S}_{\bm G^\natural}\right)^t + \frac{1}{2}\left(1+\frac{\eta_2}{\eta_1}\right)\widetilde{\bm V}_{\bm G^\natural}\left(\bm I_d - \sqrt{\eta_1 \eta_2}\widetilde{\bm S}_{\bm G^\natural}\right)^t
        \end{bmatrix}\\
        & \times \frac{\mathbf{C}^{\!\top}\bm Z_0}{\sqrt{1+\frac{\eta_2}{\eta_1}}}\\
        = & \begin{bmatrix}
            \frac{1}{2}\widetilde{\bm U}_{\bm G^\natural}\bigg(\left(\bm I_d + \sqrt{\eta_1 \eta_2}\widetilde{\bm S}_{\bm G^\natural}\right)^t + \left(\bm I_d - \sqrt{\eta_1 \eta_2}\widetilde{\bm S}_{\bm G^\natural}\right)^t\bigg)\widetilde{\bm U}_{\bm G^\natural}^{\!\top} & * \quad\quad\\
            \frac{1}{2}\sqrt{\frac{\eta_2}{\eta_1}}\widetilde{\bm V}_{\bm G^\natural}\bigg(\left(\bm I_d + \sqrt{\eta_1 \eta_2}\widetilde{\bm S}_{\bm G^\natural}\right)^t - \left(\bm I_d - \sqrt{\eta_1 \eta_2}\widetilde{\bm S}_{\bm G^\natural}\right)^t\bigg)\widetilde{\bm U}_{\bm G^\natural}^{\!\top} & * \quad\quad
        \end{bmatrix}\begin{bmatrix}
            \bm A_0 \\ \bm 0
        \end{bmatrix}\\
        = & \begin{bmatrix}
            \frac{1}{2}\widetilde{\bm U}_{\bm G^\natural}\bigg(\left(\bm I_d + \sqrt{\eta_1 \eta_2}\widetilde{\bm S}_{\bm G^\natural}\right)^t + \left(\bm I_d - \sqrt{\eta_1 \eta_2}\widetilde{\bm S}_{\bm G^\natural}\right)^t\bigg)\widetilde{\bm U}_{\bm G^\natural}^{\!\top}\bm A_0\\
            \frac{1}{2}\sqrt{\frac{\eta_2}{\eta_1}}\widetilde{\bm V}_{\bm G^\natural}\bigg(\left(\bm I_d + \sqrt{\eta_1 \eta_2}\widetilde{\bm S}_{\bm G^\natural}\right)^t - \left(\bm I_d - \sqrt{\eta_1 \eta_2}\widetilde{\bm S}_{\bm G^\natural}\right)^t\bigg)\widetilde{\bm U}_{\bm G^\natural}^{\!\top}\bm A_0
        \end{bmatrix}\,.
    \end{align*}}
    Next, we extend the results above to $d\neq k$. Here we take $d>k$,
    \begin{align*}
        \underline{\bm B}^{\tt lin}_t & = \frac{1}{2}\sqrt{\frac{\eta_2}{\eta_1}}\underline{\widetilde{\bm V}_{\bm G^\natural}}\bigg(\left(\bm I_d + \sqrt{\eta_1 \eta_2}\underline{\widetilde{\bm S}_{\bm G^\natural}}\right)^t - \left(\bm I_d - \sqrt{\eta_1 \eta_2}\underline{\widetilde{\bm S}_{\bm G^\natural}}\right)^t\bigg)\widetilde{\bm U}_{\bm G^\natural}^{\!\top}\bm A_0\\
        & = \begin{bmatrix}
            \frac{1}{2}\sqrt{\frac{\eta_2}{\eta_1}}\widetilde{\bm V}_{\bm G^\natural}\bigg(\left(\bm I_d + \sqrt{\eta_1 \eta_2}\widetilde{\bm S}_{\bm G^\natural}\right)^t - \left(\bm I_d - \sqrt{\eta_1 \eta_2}\widetilde{\bm S}_{\bm G^\natural}\right)^t\bigg)\widetilde{\bm U}_{\bm G^\natural}^{\!\top}\bm A_0 & \bm 0_{r \times (d-k)}
        \end{bmatrix}\,,
    \end{align*}
    which proves the claim. Lastly, we take $d<k$,
    \begin{align*}
        \underline{\bm A}^{\tt lin}_t & = \frac{1}{2}\underline{\widetilde{\bm U}_{\bm G^\natural}}\bigg(\left(\bm I_k + \sqrt{\eta_1 \eta_2}\underline{\widetilde{\bm S}_{\bm G^\natural}}\right)^t + \left(\bm I_k - \sqrt{\eta_1 \eta_2}\underline{\widetilde{\bm S}_{\bm G^\natural}}\right)^t\bigg)\underline{\widetilde{\bm U}_{\bm G^\natural}}^{\!\top}\underline{\bm A_0}\\
        & \begin{bmatrix}
            \frac{1}{2}\widetilde{\bm U}_{\bm G^\natural}\bigg(\left(\bm I_d + \sqrt{\eta_1 \eta_2}\widetilde{\bm S}_{\bm G^\natural}\right)^t + \left(\bm I_d - \sqrt{\eta_1 \eta_2}\widetilde{\bm S}_{\bm G^\natural}\right)^t\bigg)\widetilde{\bm U}_{\bm G^\natural}^{\!\top}\bm A_0 \\
            \bm 0_{(k-d) \times r}
        \end{bmatrix}\,,
    \end{align*}
    which completes the proof.

    Besides, we discuss about some properties of $\bm P_t^{\bm A}$ and $\bm P_t^{\bm B}$. Recall $\operatorname{Rank}({\bm G}^{\natural}) = \operatorname{Rank}(\Delta) = r^*$, then we have
    \begin{align*}
        \lambda_{r^*+i}(\bm P_t^{\bm A}) & = \frac{1}{2}\lambda_{r^*+i}\left((\bm I_d+\sqrt{\eta_1 \eta_2}\widetilde{\bm S}_{\bm G^\natural})^t + (\bm I_d-\sqrt{\eta_1 \eta_2}\widetilde{\bm S}_{\bm G^\natural})^t\right)=1\,,\quad \text{for }\forall\,1\leq i \leq (d-r^*)\,.
    \end{align*}
   That means $ \bm P_t^{\bm A} \in \mathbb{R}^{d \times d}$ is a full rank matrix and the  singular values are 1 after the $r^*$-th order. However $ \bm P_t^{\bm B} \in \mathbb{R}^{k \times k} $ is a rank-$r^*$ matrix.\\
\end{proof}

\noindent
{\bf Part II: Control $\|\bm E_t\|_{op}$}\\

Based on the above results, we are ready to prove that $\|\bm E_t\|_{op}$ is small.
\begin{lemma}\label{E_t_A_0}
Under assumptions in \cref{sec:assumptions} for the linear setting, with LoRA initialization \eqref{eq:lorainit}, given $\| \bm A_0\|_{op}$ and $\bm G^{\natural}$ in \cref{eq:G} and its largest singular value $\lambda_1(\bm G^{\natural})$, 
consider the following time period
\begin{equation*}\label{eq:t*}
t \leq t^* : =\frac{\ln\left(\frac{\lambda_1(\bm G^{\natural})}{6\,\zeta(\eta_1\,,\eta_2) \|\bm A_0\|_{op}^2}\right)}{3\ln\left(1+\sqrt{\eta_1 \eta_2}\lambda_1({\bm G}^{\natural})\right)}\,,  
\end{equation*}
where $\zeta(\eta_1\,,\eta_2)$ is a function of $\eta_1\,,\eta_2$ defined as
\begin{equation}\label{eq:funczeta}
\zeta(\eta_1\,,\eta_2) := \max\left\{1,\,\frac{1}{2}\sqrt{\frac{\eta_2}{\eta_1}}\right\} \times \max\left\{\left(\sqrt{\frac{\eta_2}{\eta_1}} + \frac{1}{2}\right),\,\left(\sqrt{\frac{\eta_1}{\eta_2}}+\sqrt{\frac{\eta_2}{\eta_1}}\right)\right\}\,,  
\end{equation}
then the following statement holds with probability at least $1- 2C\exp(-N)$ for a universal constant $C$ over random Gaussian data
\begin{align}
\label{E_t_A0}
    \|\bm E_t\|_{op} \leq \|\bm A_0\|_{op}\,.
\end{align}
\end{lemma}

\noindent
{\bf Remark:} By choosing proper random initialization variance over $\bm A_0$, we can ensure $t^* > 1$ to avoid vacuous upper bound.

\begin{proof}
We will prove by induction. Starting from $t=0$, this is trivially true since $\bm Z_0 = \bm Z^{\tt lin}_0$. Next, we assume \cref{E_t_A0} holds for $t-1$ with $t\geq 1$ and prove $\|\bm E_t\|_{op} \leq \|\bm A_0\|_{op}$.
To deliver the proof, denote $a_0:=\|\bm A_0\|_{op}$, from \cref{psuedo-dynamics}, we know that 
\begin{equation}\label{eq:normABt}
 \|\bm A^{\tt lin}_{t-1}\|_{op} \leq \left(1+\sqrt{\eta_1 \eta_2}\lambda_1({\bm G}^{\natural})\right)^{t-1} a_0\,, \quad   \|\bm B^{\tt lin}_{t-1}\|_{op} \leq \frac{1}{2}\sqrt{\frac{\eta_1}{\eta_2}}\left(1+\sqrt{\eta_1 \eta_2}\lambda_1({\bm G}^{\natural})\right)^{t-1} a_0\,.
\end{equation}

Besides, since $(\bm A_t -\bm A^{\tt lin}_t)$ and $(\bm B_t -\bm B^{\tt lin}_t)$ are the sub-matrices of the error term $\bm E_t$, our condition $\|\bm E_{t-1}\|_{op} \leq \|\bm A_0\|_{op}$ we have 
\begin{align}\label{AB_t_diff}
    \left\|
        \bm A_{t-1} -\bm A^{\tt lin}_{t-1}
    \right\|_{op} \leq \|\bm E_{t-1}\|_{op}\,,\quad
    \left\|
        \bm B_{t-1} -\bm B^{\tt lin}_{t-1}
    \right\|_{op} \leq \|\bm E_{t-1}\|_{op}\, .
\end{align}
It implies that
\begin{equation}\label{A-B-pseudo-upper-bound}
\begin{split}
    & \|\bm A_{t-1}\|_{op} \leq \left(1+\sqrt{\eta_1 \eta_2}\lambda_1({\bm G}^{\natural})\right)^{t-1} a_0 + \|\bm E_{t-1}\|_{op}\,, \\ 
    & \|\bm B_{t-1}\|_{op} \leq \frac{1}{2}\sqrt{\frac{\eta_1}{\eta_2}}\left(1+\sqrt{\eta_1 \eta_2}\lambda_1({\bm G}^{\natural})\right)^{t-1} a_0 + \|\bm E_{t-1}\|_{op}\,.
\end{split}
\end{equation}
Besides, according to covariance matrix estimation in the operator norm in \cref{lem:conrg}, with probability at least $1-2C\exp(-N{\epsilon}^2)$ for a universal constant $C>0$, we have (taking $\epsilon=1$)
\begin{align}\label{eq:concenXX}
    \left\|\frac{1}{N}\widetilde{\bm X}^{\!\top}\widetilde{\bm X} - \bm I_d\right\|_{op} \leq \epsilon = 1\,.
\end{align} 
Accordingly, with probability at least $1-2C\exp(-N)$, $\|\widehat{\bm E}_t\|_{op}$ can be upper bounded by
\begin{align*}
    \|\widehat{\bm E}_t\|_{op} & \leq \eta_1 \left\|\frac{1}{N}\widetilde{\bm X}^{\!\top}\widetilde{\bm X}\bm A_{t-1} \bm B_{t-1} \bm B_{t-1}^{\!\top}\right\|_{op}+ \eta_2 \left\|\bm B_{t-1}^{\!\top}\bm A_{t-1}^{\!\top}\frac{1}{N}\widetilde{\bm X}^{\!\top}\widetilde{\bm X}\bm A_{t-1}\right\|_{op}\\
    & \leq \eta_1 (1+\epsilon) \|\bm A_{t-1}\|_{op} \|\bm B_{t-1}\|_{op}^2 + \eta_2 (1+\epsilon) \|\bm A_{t-1}\|_{op}^2 \|\bm B_{t-1}\|_{op} \quad \tag*{\color{teal}[using~\cref{eq:concenXX}]} \\
    & \leq (1+\epsilon) \left(\|\bm A^{\tt lin}_{t-1}\|_{op}+\|\bm E_{t-1}\|_{op}\right) \left(\|\bm B^{\tt lin}_{t-1}\|_{op}+\|\bm E_{t-1}\|_{op}\right)\times\\
    & \quad \left(\eta_1 \|\bm B^{\tt lin}_{t-1}\|_{op} + \eta_2 \|\bm A^{\tt lin}_{t-1}\|_{op}+\left(\eta_1+\eta_2\right)\|\bm E_{t-1}\|_{op}\right) \quad \tag*{\color{teal}[using~\cref{AB_t_diff}]} \\
    & \leq (1+\epsilon) \sqrt{\eta_1 \eta_2} \left(\left(1+\sqrt{\eta_1 \eta_2}\lambda_1({\bm G}^{\natural})\right)^{t-1} a_0+\|\bm E_{t-1}\|_{op}\right)\tag*{\color{teal}[using~\cref{A-B-pseudo-upper-bound}]}\\
    &\quad \times \left(\frac{1}{2}\sqrt{\frac{\eta_2}{\eta_1}}\left(1+\sqrt{\eta_1 \eta_2}\lambda_1({\bm G}^{\natural})\right)^{t-1} a_0+\|\bm E_{t-1}\|_{op}\right)\\
    & \quad \times \left(\left(\sqrt{\frac{\eta_2}{\eta_1}} + \frac{1}{2}\right)\left(1+\sqrt{\eta_1 \eta_2}\lambda_1({\bm G}^{\natural})\right)^{t-1} a_0 +\left(\sqrt{\frac{\eta_1}{\eta_2}}+\sqrt{\frac{\eta_2}{\eta_1}}\right)\|\bm E_{t-1}\|_{op}\right)\\
    & \leq (1+\epsilon) \underbrace{\max\left\{1,\,\frac{1}{2}\sqrt{\frac{\eta_2}{\eta_1}}\right\} \times \max\left\{\left(\sqrt{\frac{\eta_2}{\eta_1}} + \frac{1}{2}\right),\,\left(\sqrt{\frac{\eta_1}{\eta_2}}+\sqrt{\frac{\eta_2}{\eta_1}}\right)\right\}}_{:=\zeta(\eta_1\,,\eta_2)}\times \\ 
    & \left(\left(1+\sqrt{\eta_1 \eta_2}\lambda_1({\bm G}^{\natural})\right)^{t-1} a_0+\|\bm E_{t-1}\|_{op}\right)^3\\
    & \leq 4 (1+\epsilon) \sqrt{\eta_1 \eta_2} \,\zeta(\eta_1\,,\eta_2) \left(\left(1+\sqrt{\eta_1 \eta_2}\lambda_1({\bm G}^{\natural})\right)^{3t-3} a_0^3+\|\bm E_{t-1}\|_{op}^3\right)\\
    & \leq 12 \sqrt{\eta_1 \eta_2} \,\zeta(\eta_1\,,\eta_2) \left(1+\sqrt{\eta_1 \eta_2}\lambda_1({\bm G}^{\natural})\right)^{3t-3} a_0^3 \,. \quad \tag*{\color{teal}[from our inductive hypothesis]} 
\end{align*}
Then, by Lemma~\ref{induc}, we can conclude that
\begin{align}
    \|\bm E_t\|_{op} & = \left\|\sum_{i=1}^t \bm H^{t-i} \widehat{\bm E}_i \right\|_{op} \leq \sum_{i=1}^t \|\bm H\|_{op}^{t-i} \|\widehat{\bm E}_i\|_{op}\nonumber\\
    & \leq 12 \sqrt{\eta_1 \eta_2} \,\zeta(\eta_1\,,\eta_2) a_0^3 \times \sum_{i=1}^t  \left(1+\sqrt{\eta_1 \eta_2}\lambda_1({\bm G}^{\natural})\right)^{t+2i-3} \quad \tag*{\color{teal}[using~\cref{H-schur}]} \nonumber\\
    & = 12 \sqrt{\eta_1 \eta_2} \,\zeta(\eta_1\,,\eta_2) a_0^3 \times \left(1+\sqrt{\eta_1 \eta_2}\lambda_1({\bm G}^{\natural})\right)^{t-1}\sum_{i=1}^t  \left(1+\sqrt{\eta_1 \eta_2}\lambda_1({\bm G}^{\natural})\right)^{2i-2} \nonumber\\
    & = 12 \sqrt{\eta_1 \eta_2} \,\zeta(\eta_1\,,\eta_2) a_0^3 \times \left(1+\sqrt{\eta_1 \eta_2}\lambda_1({\bm G}^{\natural})\right)^{t-1} \frac{\left(1+\sqrt{\eta_1 \eta_2}\lambda_1({\bm G}^{\natural})\right)^{2t}-1}{\left(1+\sqrt{\eta_1 \eta_2}\lambda_1({\bm G}^{\natural})\right)^2-1} \quad \tag*{\color{teal}[geometric series]} \nonumber\\
    & \leq 12 \sqrt{\eta_1 \eta_2} \,\zeta(\eta_1\,,\eta_2) a_0^3 \times \left(1+\sqrt{\eta_1 \eta_2}\lambda_1({\bm G}^{\natural})\right)^{t-1} \frac{\left(1+\sqrt{\eta_1 \eta_2}\lambda_1({\bm G}^{\natural})\right)^{2t+1}}{2\sqrt{\eta_1 \eta_2}\lambda_1({\bm G}^{\natural})}\nonumber\\
    & \leq 6 \zeta(\eta_1\,,\eta_2) \left(1+\sqrt{\eta_1 \eta_2}\lambda_1({\bm G}^{\natural})\right)^{3t} \frac{a_0^3}{\lambda_1({\bm G}^{\natural})}\label{E_t_induction}\,.
\end{align}
Accordingly, when $t \leq t^* := \frac{\ln\left(\frac{\lambda_1({\bm G}^{\natural})}{6\,\zeta(\eta_1\,,\eta_2) \|\bm A_0\|_{op}^2}\right)}{3\ln\left(1+\sqrt{\eta_1 \eta_2}\lambda_1({\bm G}^{\natural})\right)}$, we have
\begin{align*}
    \|\bm E_t\|_{op} \leq \|\bm A_0\|_{op}\,,
\end{align*}
which proves the claim.
\end{proof}

\subsubsection{Alignment to Negative Gradient of Full Fine-tuning}

Now we can apply \cref{E_t_A_0} to obtain
\[
\left\|\bm A_t -\bm A^{\tt lin}_t\right\|_{op} \leq \|\bm A_0\|_{op}\,.
\]
Recall \cref{psuedo-dynamics}, we can observe that the dynamic of $\bm A^{\tt lin}_t$ also follows an Oja's Power Method \citep{oja1982simplified}, which aligns $\bm A^{\tt lin}_t$'s left singular subspace to the left subspace of the initial negative gradient step ${\bm G}^{\natural}$ of full fine-tuning. We anticipate that $\lambda_{r^*}\left(\bm A_t\right)\gg\lambda_{r^*+1}\left(\bm A_t\right)$ for sufficiently large $t$. Furthermore, if $\|\bm E_t\|_{op}$ remains small, then the top-$r^*$ left singular subspace of $\bm A_t$ can closely align to ${\bm G}^{\natural}$'s. To prove this alignment, we modify \citet[Lemma 8.3]{stoger2021small} to obtain the following results.
\begin{lemma}
\label{Mahdi}
Under assumptions in \cref{sec:assumptions} for the linear setting, recall
\begin{align*}
    \bm P_t^{\bm A}:=\frac{1}{2}\widetilde{\bm U}_{\bm G^\natural}\bigg(\left(\bm I_d + \sqrt{\eta_1 \eta_2}\widetilde{\bm S}_{\bm G^\natural}\right)^t + \left(\bm I_d - \sqrt{\eta_1 \eta_2}\widetilde{\bm S}_{\bm G^\natural}\right)^t\bigg)\widetilde{\bm U}_{\bm G^\natural}^{\!\top}
\end{align*}
as an $\mathbb{R}^{d\times d}$-valued symmetric matrix in \cref{psuedo-dynamics}, we assume that
    \begin{align*}
        \lambda_{r^*+1}(\bm P_t^{\bm A})\|\bm A_0\|_{op}+\|\bm E_t\|_{op} < \lambda_{r^*}(\bm P_t^{\bm A})\lambda_{\min}(\bm U^{\!\top}_{r^*}(\bm P_t^{\bm A}) \bm A_0)\, ,
    \end{align*}
    that can be satisfied under certain conditions (discussed later).
    Then the following three inequalities hold:
    \begin{align}
        \lambda_{r^*}(\bm P_t^{\bm A}\bm A_0+\bm E_t) & \geq \lambda_{r^*}(\bm P_t^{\bm A})\lambda_{\min}(\bm U^{\!\top}_{r^*}(\bm P_t^{\bm A}) \bm A_0)-\|\bm E_t\|_{op}\,, \label{eq:rP} \\
        \lambda_{r^*+1}(\bm P_t^{\bm A}\bm A_0+\bm E_t) & \leq \lambda_{r^*+1}(\bm P_t^{\bm A})\|\bm A_0\|_{op} + \|\bm E_t\|_{op}\,, \label{eq:r1P} \\
        \|\bm U^{\!\top}_{r^*,\perp}(\bm P_t^{\bm A})\bm U_{r^*}(\bm P_t^{\bm A}\bm A_0+\bm E_t)\|_{op} & \leq \frac{\lambda_{r^*+1}(\bm P_t^{\bm A})\|\bm A_0\|_{op} + \|\bm E_t\|_{op}}{\lambda_{r^*}(\bm P_t^{\bm A})\lambda_{\min}(\bm U^{\!\top}_{r^*}(\bm P_t^{\bm A}) \bm A_0) - \lambda_{r^*+1}(\bm P_t^{\bm A})\|\bm A_0\|_{op}-\|\bm E_t\|_{op}}\,, \label{eq:angle}
    \end{align}
    where $\bm U_k(\bm M)$ denotes the left singular subspace spanned by the $k$ largest singular values of the input matrix $\bm M$ and $\bm U_{k,\perp}(\bm M)$ denotes the left singular subspace orthogonal to $\bm U_{k}\left(\bm{M}\right)$.
\end{lemma}
This lemma can help us derive the principle angle of the left singular subspace between $\bm A^{\tt lin}_t$ and $\bm A_t$. Note that the assumption comes from the necessary condition of Wedin's $\sin \theta$ theorem \citep{wedin1972perturbation}. In the next lemma, we aim to derive the time threshold which can fulfill this assumption.
\begin{lemma}\label{lemma:aligntheta}
Under assumptions in \cref{sec:assumptions} for the linear setting, given $\| \bm A_0\|_{op}$, 
    for any $\theta \in (0,1)$, taking
    \[
    t \leq \frac{\ln\left(\frac{8\|\bm A_0\|_{op}}{\theta \lambda_{\min}(\bm U^{\!\top}_{r^*}(\bm P_t^{\bm A}) \bm A_0)}\right)}{\ln\left(1+\sqrt{\eta_1 \eta_2}\lambda_{r^*}\left({\bm G}^{\natural}\right)\right)}\,,
    \]
     then \cref{eq:angle} holds with probability at least $1- 2C\exp(- N)$ for a universal constant $C$ over random Gaussian data, i.e. 
    \begin{align*}
        \|\bm U^{\!\top}_{r^*,\perp}(\bm P_t^{\bm A})\bm U_{r^*}(\bm P_t^{\bm A}\bm A_0+\bm E_t)\|_{op} & \leq \theta\, .
    \end{align*}
\end{lemma}
\noindent {\bf Remark:} To ensure that the $\theta$-alignment phase still falls into the early phase in \cref{E_t_A_0} for $\| \bm E_t \|_{op} \leq \| \bm A_0 \|_{op}$, we need to choose proper initialization for $\bm A_0$.
We will detail this in \cref{thm:alignlinearA} later.
\begin{proof}
    First, $\lambda_{r^*}(\bm P_t^{\bm A})$ in \cref{psuedo-dynamics} can be lower bounded by
    \begin{equation}\label{eq:lambdarpta}
      \begin{split}
            \lambda_{r^*}(\bm P_t^{\bm A}) & = \frac{1}{2}\lambda_{r^*}\left((\bm I_d+\sqrt{\eta_1 \eta_2}\widetilde{\bm S}_{\bm G^\natural})^t + (\bm I_d-\sqrt{\eta_1 \eta_2}\widetilde{\bm S}_{\bm G^\natural})^t\right)\\ 
        & \geq \frac{1}{2}\lambda_{r^*}\left((\bm I_d+\sqrt{\eta_1 \eta_2}\widetilde{\bm S}_{\bm G^\natural})^t\right)\\
        & = \frac{1}{2}\left(1+\sqrt{\eta_1 \eta_2}\lambda_{r^*}\left({\bm G}^{\natural}\right)\right)^t\,.
      \end{split}  
    \end{equation}

    Recall \cref{psuedo-dynamics}, we have $\lambda_{r^*+1}(\bm P_t^{\bm A}) = 1$ and \cref{E_t_A_0} with $\|\bm E_t\|_{op} \leq \|\bm A_0\|_{op}$, we define the following threshold $\gamma$ and upper bound it
    \begin{align}
        \gamma & := \frac{\lambda_{r^*+1}(\bm P_t^{\bm A})\|\bm A_0\|_{op}+\|\bm E_t\|_{op}}{\lambda_{r^*}(\bm P_t^{\bm A})\lambda_{\min}(\bm U^{\!\top}_{r^*}(\bm P_t^{\bm A}) \bm A_0)}\nonumber\\
        & \leq \frac{2\|\bm A_0\|_{op}}{\frac{1}{2}\left(1+\sqrt{\eta_1 \eta_2}\lambda_{r^*}\left({\bm G}^{\natural}\right)\right)^t \lambda_{\min}(\bm U^{\!\top}_{r^*}(\bm P_t^{\bm A}) \bm A_0)} \quad \tag*{\color{teal}[using~\cref{psuedo-dynamics},~\ref{E_t_A_0}]} \nonumber\\
        & = \exp\left(-\ln\left(1+\sqrt{\eta_1 \eta_2}\lambda_{r^*}\left({\bm G}^{\natural}\right)\right)\cdot t\right)\cdot\frac{4\|\bm A_0\|_{op}}{\lambda_{\min}(\bm U^{\!\top}_{r^*}(\bm P_t^{\bm A}) \bm A_0)}\, .\label{gamma-upper-bound}
    \end{align}
    Set $\theta\in(0,1)$, let Eq.(\ref{gamma-upper-bound})$\leq \frac{\theta}{2}$, then we have that
    \begin{align*}
        \|\bm U^{\!\top}_{r^*,\perp}(\bm P_t^{\bm A})\bm U_{r^*}(\bm P_t^{\bm A}\bm A_0+\bm E_t)\|_{op} & \leq \theta\, .
    \end{align*}
    The time $t$ to achieve this angle $\theta$ can be upper bounded by
    \begin{align*}
        \exp\left(-\ln\left(1+\sqrt{\eta_1 \eta_2}\lambda_{r^*}\left({\bm G}^{\natural}\right)\right)\cdot t\right)\cdot\frac{4\|\bm A_0\|_{op}}{\lambda_{\min}(\bm U^{\!\top}_{r^*}(\bm P_t^{\bm A}) \bm A_0)} \leq \frac{\theta}{2}\,,
        \end{align*}
   which implies that     
        \begin{align*}
        t \leq \frac{\ln\left(\frac{8\|\bm A_0\|_{op}}{\theta \lambda_{\min}(\bm U^{\!\top}_{r^*}(\bm P_t^{\bm A}) \bm A_0)}\right)}{\ln\left(1+\sqrt{\eta_1 \eta_2}\lambda_{r^*}\left({\bm G}^{\natural}\right)\right)}\, .
    \end{align*}
    Finally we conclude the proof.
\end{proof}

\begin{theorem}\label{thm:alignlinearA:full}[Full version of \cref{thm:alignlinearA}]
    Under assumptions in \cref{sec:assumptions} for the linear setting, recall ${\bm G}^{\natural}$ defined in \cref{eq:G} with its condition number $\kappa^{\natural}$, we consider random Gaussian initialization $\bm A_0 \in \mathbb{R}^{d \times r}$ with $[\bm A_0]_{ij} \sim \mathcal{N}(0, \alpha^2)$ in \eqref{eq:lorainit}, for any $\theta \in (0,1)$, let $\xi = o(1)$ be chosen such that
\begin{equation*}
    \alpha \leq
\begin{cases} 
\left(\frac{\theta \xi}{24r\sqrt{d}}\right)^{\frac{3\kappa^\natural}{2}}\sqrt{\frac{\lambda_1({\bm G}^{\natural})}{54d\,\zeta(\eta_1, \eta_2)}} & \text{if } r^*\leq r < 2r^*, \\
\left(\frac{\theta}{24\sqrt{d}}\right)^{\frac{3\kappa^\natural}{2}}\sqrt{\frac{\lambda_1({\bm G}^{\natural})}{54d\,\zeta(\eta_1, \eta_2)}} & \text{if } r \geq 2r^*\,,
\end{cases}
\end{equation*}
where $\zeta(\eta_1, \eta_2)$ is defined in \cref{eq:funczeta} and satisfies $\zeta(\eta_1, \eta_2) = \Theta(1)$.
Then if we run gradient descent for $t^*$ steps with
\begin{align*}
    t^* \lesssim 
    \begin{cases}
        \frac{\ln\left(\frac{24r\sqrt{d}}{\theta \xi}\right)}{\ln\left(1+\sqrt{\eta_1 \eta_2}\lambda_{r^*}\left({\bm G}^{\natural}\right)\right)}  & \text{if } r^*\leq r < 2r^*, \\
        \frac{\ln\left(\frac{24\sqrt{d}}{\theta}\right)}{\ln\left(1+\sqrt{\eta_1 \eta_2}\lambda_{r^*}\left({\bm G}^{\natural}\right)\right)}  & \text{if } r \geq 2r^*\,,
    \end{cases}
\end{align*}
we have the following alignment on the left singular subspace between $\bm G^{\natural}$ and $\bm A_{t^*}$
    \begin{align*}
        &\left\|\bm U^{\!\top}_{r^*,\perp}(  \bm G^{\natural} )\bm U_{r^*}\left(\bm A_{t^*}\right)\right\|_{op} \lesssim \theta\,,\\
        &\mbox{with probability at least}~
        \begin{cases} 
1\!-\! C_1\exp(-d) \!-\! (C_2 \xi)^{r-r^*+1} \!-\! C_3\exp(-r) \!-\! C\exp(-N) & \text{if } r^*\leq r < 2r^*, \\
1 \!-\! C_4\exp(- d) -C_5\exp(- r) -C\exp(- N) & \text{if } r \geq 2r^*\,,
\end{cases}
    \end{align*}
for some positive constants $C\,,C_1\,,C_2\,,C_3\,,C_4\,,C_5$.
Here $\bm U_{r^*}(\bm A_{t^*})$ denotes the left singular subspace spanned by the $r^*$ largest singular values of $\bm A_{t^*}$ and $\bm U_{r^*,\perp}(\bm M)$ denotes the left singular subspace orthogonal to $\bm U_{r^*}\left(\bm{M}\right)$. Note that we can select any pair of stepsizes $(\eta_1\,,\eta_2)$ that satisfies the conditions $t^*>1$, $\eta_2 \geq \eta_1$, and $\zeta(\eta_1, \eta_2) = \Theta(1)$.
\end{theorem}
\begin{proof}
For ease of description, we denote $\bm A_0 := \alpha \bm T \in \mathbb{R}^{d \times r}$ where $\bm T$ is a standard random Gaussian matrix with zero-mean and unit variance.
Here we aim to choose a proper $\alpha$ to ensure that $\theta$-alignment phase in \cref{lemma:aligntheta} still falls into the early phase in \cref{E_t_A_0}, i.e.
    \begin{align*}
        & \frac{\ln\left(\frac{8\|\bm A_0\|_{op}}{\theta \lambda_{\min}(\bm U^{\!\top}_{r^*}(\bm P_t^{\bm A}) \bm A_0)}\right)}{\ln\left(1+\sqrt{\eta_1 \eta_2}\lambda_{r^*}\left({\bm G}^{\natural}\right)\right)} = \frac{\ln\left(\frac{\lambda_1({\bm G}^{\natural})}{6\,\zeta(\eta_1\,,\eta_2) \|\bm A_0\|_{op}^2}\right)}{3\ln\left(1+\sqrt{\eta_1 \eta_2}\lambda_1({\bm G}^{\natural})\right)}=t^*\\
        \Leftrightarrow\quad & \ln\left(\frac{8\|\bm A_0\|_{op}}{\theta \lambda_{\min}(\bm U^{\!\top}_{r^*}(\bm P_t^{\bm A}) \bm A_0)}\right) = \frac{\ln\left(1+\sqrt{\eta_1 \eta_2}\lambda_{r^*}\left({\bm G}^{\natural}\right)\right)}{3\ln\left(1+\sqrt{\eta_1 \eta_2}\lambda_1({\bm G}^{\natural})\right)}\ln\left(\frac{\lambda_1({\bm G}^{\natural})}{6\,\zeta(\eta_1\,,\eta_2) \|\bm A_0\|_{op}^2}\right)\\
        \Leftrightarrow\quad & \frac{8\|\bm A_0\|_{op}}{\theta \lambda_{\min}(\bm U^{\!\top}_{r^*}(\bm P_t^{\bm A}) \bm A_0)} = \left(\frac{\lambda_1({\bm G}^{\natural})}{6\,\zeta(\eta_1\,,\eta_2) \|\bm A_0\|_{op}^2}\right)^\frac{\ln\left(1+\sqrt{\eta_1 \eta_2}\lambda_{r^*}\left({\bm G}^{\natural}\right)\right)}{3\ln\left(1+\sqrt{\eta_1 \eta_2}\lambda_1({\bm G}^{\natural})\right)}\\
        \Leftrightarrow\quad & \theta = \frac{8\|\bm A_0\|_{op}}{\lambda_{\min}(\bm U^{\!\top}_{r^*}(\bm P_t^{\bm A}) \bm A_0)} \left(\frac{6\,\zeta(\eta_1\,,\eta_2) \|\bm A_0\|_{op}^2}{\lambda_1({\bm G}^{\natural})}\right)^{\frac{\ln\left(1+\sqrt{\eta_1 \eta_2}\lambda_{r^*}\left({\bm G}^{\natural}\right)\right)}{3\ln\left(1+\sqrt{\eta_1 \eta_2}\lambda_1({\bm G}^{\natural})\right)}}\\
        & = \frac{8\|\bm T\|_{op}}{\lambda_{\min}(\bm U^{\!\top}_{r^*}(\bm P_t^{\bm A}) \bm T)} \left(\frac{6\,\zeta(\eta_1\,,\eta_2) \|\bm T\|_{op}^2}{\lambda_1({\bm G}^{\natural})}\right)^\iota \alpha^{2\iota}\,. \tag*{\color{teal}$\left[\text{by setting }\iota:=\frac{\ln\left(1+\sqrt{\eta_1 \eta_2}\lambda_{r^*}\left({\bm G}^{\natural}\right)\right)}{3\ln\left(1+\sqrt{\eta_1 \eta_2}\lambda_1({\bm G}^{\natural})\right)}\right]$}
    \end{align*}
    In the next, we will discuss how to pick up $\alpha$.
     According to \cref{lem:min-singular-conct}, we need to consider the following two cases on the relationship between $r^*$ and $r$.
    
    {\bf Case 1.} $r^*\leq r < 2r^*$: by \cref{lem:init-op-conct} and \cref{lem:min-singular-conct}, with probability at least $1-C_1 \exp(-d)-(C_2 \xi)^{r-r^*+1}-C_3\exp(-r)$ for some positive constants $C_1\,,C_2\,,C_3$, we have
    \begin{align}\label{eq:r2r}
        \frac{\|\bm T\|_{op}}{3\sqrt{d}} \leq 1\,,\quad \frac{\xi}{r\lambda_{\min}(\bm U^{\!\top}_{r^*}(\bm P_t^{\bm A}) \bm T)} \lesssim 1\,.
    \end{align}
    Here we pick
    \begin{align*}
        \alpha & \leq \left(\frac{\theta \xi}{24r\sqrt{d}}\right)^{\frac{3\kappa^\natural}{2}}\sqrt{\frac{\lambda_1({\bm G}^{\natural})}{54\,\zeta(\eta_1\,,\eta_2)d}}\,,
    \end{align*}
    then recall \cref{lemma:aligntheta} on the alignment, we take $\alpha$ here
    \begin{align*}
        & \left\|\bm U^{\!\top}_{r^*,\perp}\left(-\nabla_{\bm W}\widetilde{L}(\bm W^\natural)\right)\bm U_{r^*}\left(\bm A_{t^*}\right)\right\|_{op}\\
        \leq & \frac{8\|\bm T\|_{op}}{\lambda_{\min}(\bm U^{\!\top}_{r^*}(\bm P_t^{\bm A}) \bm T)} \left(\frac{6\,\zeta(\eta_1\,,\eta_2) \|\bm S\|_{op}^2}{\lambda_1({\bm G}^{\natural})}\right)^\iota \alpha^{2\iota}\\
        = & \frac{8\|\bm T\|_{op}}{\lambda_{\min}(\bm U^{\!\top}_{r^*}(\bm P_t^{\bm A}) \bm T)} \left(\frac{6\,\zeta(\eta_1\,,\eta_2) \|\bm S\|_{op}^2}{\lambda_1({\bm G}^{\natural})}\right)^\iota \left(\frac{\theta \xi}{24r\sqrt{d}}\right)^{3\kappa^\natural\iota}\left(\frac{\lambda_1({\bm G}^{\natural})}{54\,\zeta(\eta_1\,,\eta_2)d}\right)^\iota\\
        = & \frac{8\|\bm T\|_{op}}{\lambda_{\min}(\bm U^{\!\top}_{r^*}(\bm P_t^{\bm A}) \bm T)} \left(\frac{\|\bm T\|_{op}^2}{9d}\right)^\iota \left(\frac{\theta \xi}{24r\sqrt{d}}\right)^{3\kappa^\natural\iota}\\
        \leq & \frac{\|\bm T\|_{op}\theta \xi}{3r\sqrt{d}\lambda_{\min}(\bm U^{\!\top}_{r^*}(\bm P_t^{\bm A}) \bm T)} \left(\frac{\|\bm T\|_{op}^2}{9d}\right)^\iota\,.\quad \tag*{\color{teal}$\left[\text{since }\iota \geq 1/3\kappa^\natural\text{ and }\frac{\theta \xi}{24r\sqrt{d}}\in(0,1)\right]$}\\
    \end{align*}
    Then using \cref{eq:r2r}, with probability at least $1-C_1 \exp(-d)-(C_2 \xi)^{r-r^*+1}-C_3\exp(-r)$ for some positive constants $C_1\,,C_2\,,C_3$, we have
    \begin{align*}
        \left\|\bm U^{\!\top}_{r^*,\perp}\left(-\nabla_{\bm W}\widetilde{L}(\bm W^\natural)\right)\bm U_{r^*}\left(\bm A_{t^*}\right)\right\|_{op} & \lesssim \theta\,.
    \end{align*}
    And we can compute the upper bound of $t^*$ as
    \begin{align*}
        t^* &
        = \frac{\ln\left(\frac{8\|\bm A\|_{op}}{\theta \lambda_{\min}(\bm U^{\!\top}_{r^*}(\bm P_t^{\bm A}) \bm A)}\right)}{\ln\left(1+\sqrt{\eta_1 \eta_2}\lambda_{r^*}\left({\bm G}^{\natural}\right)\right)}
        \lesssim \frac{\ln\left(\frac{24r\sqrt{d}}{\theta \xi}\right)}{\ln\left(1+\sqrt{\eta_1 \eta_2}\lambda_{r^*}\left({\bm G}^{\natural}\right)\right)}\,.
    \end{align*}
   {\bf Case 2.} $r \geq 2r^*$: by \cref{lem:init-op-conct} and \cref{lem:min-singular-conct}, with probability at least $1-C_4 \exp(-d)-C_5 \exp(- r)$ for some positive constants $C_4\,,C_5$, we have
    \begin{align*}
        \frac{\|\bm T \|_{op}}{3\sqrt{d}} \leq 1\,,\quad \frac{1}{\lambda_{\min}(\bm U^{\!\top}_{r^*}(\bm P_t^{\bm A}) \bm T)} \lesssim 1\,.
    \end{align*}
   Here we pick
    \begin{align*}
        \alpha & \leq \left(\frac{\theta}{24\sqrt{d}}\right)^{\frac{3\kappa^\natural}{2}}\sqrt{\frac{\lambda_1({\bm G}^{\natural})}{54d\,\zeta(\eta_1\,,\eta_2)}}\,.
    \end{align*}
    Similarly, we can obtain
    \begin{align*}
        \left\|\bm U^{\!\top}_{r^*,\perp}\left(-\nabla_{\bm W}\widetilde{L}(\bm W^\natural)\right)\bm U_{r^*}\left(\bm A_t\right)\right\|_{op} & \leq \frac{\|\bm T \|_{op}\theta}{3\sqrt{d}\lambda_{\min}(\bm U^{\!\top}_{r^*}(\bm P_t^{\bm A}) \bm S)} \left(\frac{\|\bm S\|_{op}^2}{9d}\right)^\iota \lesssim \theta\,.
    \end{align*}
    And we can compute the upper bound of $t^*$ as
    \begin{align*}
        t^* &
        \leq \frac{\ln\left(\frac{24\sqrt{d}}{\theta}\right)}{\ln\left(1+\sqrt{\eta_1 \eta_2}\lambda_{r^*}\left({\bm G}^{\natural}\right)\right)}\,.
    \end{align*}
\end{proof}


\begin{theorem}
\label{linear-align-Bt}
    Under assumptions in \cref{sec:assumptions} for the linear setting, using the LoRA initialization for $\bm B_0 = \bm 0$, then for any time-step $t \in \mathbb{N}_+$, we have
    \begin{align*}
        \left\|\bm V^{\!\top}_{r^*,\perp}\left(-\nabla_{\bm W}\widetilde{L}(\bm W^\natural)\right)\bm V_{r^*}\left(\bm B_t\right)\right\|_{op} & = 0\,.
    \end{align*}
\end{theorem}
\begin{proof}
    We prove by induction. Recall the complete SVD of $\Delta$ in \cref{Delta-SVD} as
    \begin{align*}
    \Delta=\widetilde{\bm U} \widetilde{\bm S}^* \widetilde{\bm V}^{\!\top}=
    \begin{bmatrix}
        \bm U & \bm U_\perp
    \end{bmatrix}\begin{bmatrix}
       \bm S^* & \bm 0_{r^*\times (d-r^*)}\\
        \bm 0_{(d-r^*)\times r^*} & \bm 0_{(d-r^*)\times (d-r^*)}
    \end{bmatrix}\begin{bmatrix}
        \bm V^{\!\top} \\ \bm V_\perp^{\!\top}
    \end{bmatrix}\,.
\end{align*}
    For $t=1$, recall ${\bm G}^{\natural} = \frac{1}{N}\widetilde{\bm X}^{\!\top} \widetilde{\bm X}\Delta$ in \cref{eq:G}, we have
    \begin{align*}
        \bm B_1\bm V_\perp & = \frac{\eta_2}{N}\bm A_0^{\!\top}{\bm G}^{\natural}\bm V_\perp  = \frac{\eta_2}{N}\bm A_0^{\!\top}\widetilde{\bm X}^{\!\top}\widetilde{\bm X}\Delta\bm V_\perp = \bm 0_{r\times (d-r^*)}\,.
    \end{align*}
    Assume $\bm B_t\bm V_\perp = \bm 0_{r\times (d-r^*)}$ holds for any $t \in \mathbb{N}_+$ and $t \geq 2$, then
    \begin{align*}
        \bm B_{t+1}\bm V_\perp & = \bm B_t\bm V_\perp - \frac{\eta_2}{N}\bm A_t^{\!\top}\widetilde{\bm X}^{\!\top}\widetilde{\bm X}\bm A_t\bm B_t\bm V_\perp+\frac{\eta_2}{N}\bm A_t^{\!\top}{\bm G}^{\natural}\bm V_\perp= \bm 0_{r\times (d-r^*)}\,,
    \end{align*}
    which completes the claim. 
\end{proof}

\subsection{Gradient Descent under Spectral Initialization}
\label{app:lrspec}
For notational simplicity, we denote $\widehat{\bm \Sigma} := \frac{1}{N}\widetilde{\bm X}^{\!\top}\widetilde{\bm X}$ in the following content. Recall the negative gradient of Full Fine-tuning at the first step in \cref{eq:G}, we write it here again
\begin{align}
\label{NGG}
  {\bm G}^{\natural} & = -\nabla_{\bm W} \widetilde{L}(\bm W^\natural) = \frac{1}{N}\widetilde{\bm X}^{\!\top}\widetilde{\bm Y}_{\Delta} = \widehat{\bm \Sigma}\Delta = \widetilde{\bm U}_{\bm G^\natural}\widetilde{\bm S}_{\bm G^\natural}\widetilde{\bm V}_{\bm G^\natural}^{\!\top}\,.
\end{align}
In this section, according to \cref{lem:conrg}, the following statement 
\begin{align}
\label{concentration-N}
    \left\|\widehat{\bm \Sigma} - \bm I_d\right\|_{op}=\epsilon \leq \min\left\{\frac{1}{2\kappa}\,,\frac{c}{\kappa^3}\right\} \leq \frac{1}{2} \,, \quad \mbox{for some small constant $c$}\,,
\end{align}
holds with probability at least $1- 2C\exp(-\epsilon^2 N)$ for a universal constant $C>0$. We propose the following initialization scheme \eqref{eq:spectral-init-linear}
\begin{align*}
    \bm A_0 = \left[\widetilde{\bm U}_{\bm G^\natural}\right]_{[:,1:r]}\left[\widetilde{\bm S}_{\bm G^\natural}^{1/2}\right]_{[1:r]}\,,\quad \bm B_0 = \left[\widetilde{\bm S}_{\bm G^\natural}^{1/2}\right]_{[1:r]}\left[\widetilde{\bm V}_{\bm G^\natural}\right]_{[:,1:r]}^{\!\top}\,.
\end{align*}
First, we have the following lemma.
\begin{lemma}
\label{linear-initial-risk}
Under assumptions in \cref{sec:assumptions} for the linear setting, with spectral initialization \eqref{eq:spectral-init-linear}, recall $\kappa := \lambda_1^*(\Delta) / \lambda_{r^*}^*(\Delta)$, then with probability at least with probability $1- 2C\exp(-\epsilon^2 N)$ for a universal constant $C>0$, we have
    \begin{align}
        \left\|\bm A_0 \bm B_0 - \Delta\right\|_{op} & \leq \epsilon \| \Delta \|_{op} \leq \frac{\lambda_{r^*}^*}{2} \label{spectral_linear_risk_initial}\,,
    \end{align}
    and
    \begin{align}
        \lambda_{r^*}\left(\bm A_0\right)\geq \frac{\sqrt{\lambda_{r^*}^*}}{2}\,,\quad \lambda_{r^*}\left(\bm B_0\right)\geq \frac{\sqrt{\lambda_{r^*}^*}}{2}\label{initial-smallest-singular-values-linear}\,.
    \end{align}
\end{lemma}
\begin{proof}
    Due to $\operatorname{rank}\left({\bm G}^{\natural}\right) = r^*$ and $r\geq r^*$, then $\bm A_0 \bm B_0 = {\bm G}^{\natural}$, so we have
    \begin{align*}
        \left\|\bm A_0 \bm B_0 - \Delta\right\|_{op} & \leq \left\|\bm A_0 \bm B_0 - {\bm G}^{\natural}\right\|_{op} + \left\|{\bm G}^{\natural} - \Delta\right\|_{op}\\
        & = \left\|{\bm G}^{\natural} - \Delta\right\|_{op}\\
        & = \left\|\left(\widehat{\bm \Sigma} - \bm I_d\right) \Delta\right\|_{op} \tag*{\color{teal}[using \cref{NGG}]} \\
        & \leq \left\|\widehat{\bm \Sigma} - \bm I_d\right\|_{op} 
        \left\|\Delta\right\|_{op}\,.
    \end{align*}
    Accordingly, by \cref{concentration-N}, with probability at least $1-2\exp(- c \epsilon^2 N )$, we have
    \begin{align*}
        \left\|\bm A_0 \bm B_0 - \Delta\right\|_{op} & \leq \epsilon \| \Delta \|_{op} \\
        & \leq \frac{1}{2\kappa}\left\|\Delta\right\|_{op} \tag*{\color{teal}[using \cref{concentration-N}]} \\
        & = \frac{\lambda_{r^*}^*}{2}\,.
    \end{align*}
    Then, using the above result and Weyl's inequality, we have the upper bound $\lambda_{r^*}\left(\bm A_0 \bm B_0\right) \leq \lambda_{1}\left(\bm A_0\right)\lambda_{r^*}\left(\bm B_0\right)$ and the lower bound
    \begin{align*}
        \lambda_{r^*}\left(\bm A_0 \bm B_0\right) = \lambda_{r^*}\left(\bm G^{\natural} \right) \geq \lambda_{r^*}\left(\Delta\right) - \left\|{\bm G}^{\natural}-\Delta\right\|_{op} = \lambda_{r^*}\left(\Delta\right) - \left\|\bm A_0 \bm B_0 -\Delta\right\|_{op} \geq \frac{\lambda_{r^*}^*}{2}\,.
    \end{align*}
    Now we are ready to give the lower bound of $\lambda_{r^*}\left(\bm B_0\right)$. 
    Because of $\bm A_0 \bm B_0 = \bm G^{\natural}$ under spectral initialization, we have 
    \begin{equation*}
        \lambda_{1}\left(\bm A_0\right) \leq \sqrt{\lambda_1({\bm G}^{\natural})}\leq \sqrt{\left\|\widehat{\bm \Sigma} - \bm I_d\right\|_{op}\lambda_1(\Delta)}\leq \sqrt{\epsilon \lambda_1(\Delta)}\,, \quad \mbox{with high probability at least}~1- 2C\exp(- \epsilon^2 N)\,.
    \end{equation*}
    where we use $\bm G^{\natural} = \widehat{\bm \Sigma}\Delta$ and the concentration results on $\widehat{\bm \Sigma}$. Then combining the above two inequalities, $\lambda_{r^*}\left(\bm B_0\right)$ is lower bounded by
    \begin{equation*}
        \lambda_{r^*}\left(\bm B_0\right) \geq \frac{\lambda_{r^*}\left(\bm A_0 \bm B_0\right)}{\lambda_{1}\left(\bm A_0\right)} \geq \frac{\lambda_{r^*}^*/2}{\lambda_{1}\left(\bm A_0\right)}\geq \frac{\sqrt{\lambda_{r^*}^*}}{2}\,,
    \end{equation*}
by taking $\epsilon \leq \frac{1}{2 \kappa}$.
The lower bound of $\lambda_{r^*}\left(\bm A_0\right)$ can be obtained similarly.
\end{proof}
The following lemma indicates $\bm B_t$'s GD dynamics stay in the low-dimensional target subspace under the spectral initialization.
\begin{lemma}
\label{linear-invariant-B2}
Under assumptions in \cref{sec:assumptions} for the linear setting, with spectral initialization \eqref{eq:spectral-init-linear}, during the iteration, for any $t\in\mathbb{N}^{+}$, we always have $\bm B_t \bm V_\perp = \bm 0_{d\times (d-r^*)}$, where $\bm V_\perp$ comes from the complete SVD of $\Delta$ in \cref{Delta-SVD}.
\end{lemma}
\begin{proof}
We prove it by induction.
First, recall the SVD of $\Delta$ in \cref{Delta-SVD}, we have
    \begin{align*}
        {\bm G}^{\natural} \bm V_\perp = \widetilde{\bm \Sigma}\Delta \bm V_\perp = \bm 0_{d\times (d-r^*)}\,,
    \end{align*}
    and
    \begin{align*}
        \bm B_0 \bm V_\perp & = \left[\widetilde{\bm S}_{\bm G^\natural}^{1/2}\right]_{[1:r]}\left[\widetilde{\bm V}_{\bm G^\natural}^{\!\top}\right]_{[:,1:r]}\bm V_\perp\\
        & = \left[\widetilde{\bm S}_{\bm G^\natural}^{-1/2}\right]_{[1:r]}\left[\widetilde{\bm U}_{\bm G^\natural}^{\!\top}\right]_{[:,1:r]}{\bm G}^{\natural} \bm V_\perp \\
        & = \left[\widetilde{\bm S}_{\bm G^\natural}^{-1/2}\right]_{[1:r]}\left[\widetilde{\bm U}_{\bm G^\natural}^{\!\top}\right]_{[:,1:r]}\widehat{\bm \Sigma}\Delta \bm V_\perp\\
        & = \bm 0_{d\times (d-r^*)}\,.
    \end{align*}
    Next, We prove by induction. Starting from $t = 1$, using the above two equations, we have
    \begin{align*}
        \bm B_1 \bm V_\perp 
        & = \bm B_0 \bm V_\perp -\frac{\eta_2}{N} \bm A^{\!\top}_0 \widetilde{\bm X}^{\!\top} \Bigl(\widetilde{\bm X} (\bm W^\natural+\bm A_0 \bm B_0) - \widetilde{\bm Y}\Bigr)\bm V_\perp\\
        & = \bm B_0 \bm V_\perp - \frac{\eta_2}{N} \bm A^{\!\top}_0 \widetilde{\bm X}^{\!\top} \widetilde{\bm X} \bm A_0 \bm B_0\bm V_\perp + \eta \bm A^{\!\top}_0 {\bm G}^{\natural} \bm V_\perp\\
        & = \bm 0_{d\times (d-r^*)}\,.
    \end{align*}
    Assume $\bm B_t {\bm V_\perp} = \bm 0_{d\times (d-r^*)}$ holds for any  $t= 2,3,\cdots$, then at $t+1$, we have
    \begin{align*}
         \bm B_{t+1} \bm V_\perp
        & = \bm B_t \bm V_\perp - \frac{\eta}{N} \bm A^{\!\top}_t \widetilde{\bm X}^{\!\top} \widetilde{\bm X} \bm A_t \bm B_t\bm V_\perp + \eta_2 \bm A^{\!\top}_t {\bm G}^{\natural} \bm V_\perp\\
        & = \bm 0_{d\times (d-r^*)}\,.
    \end{align*}
    Accordingly we finish the proof.
\end{proof}
Under spectral initialization, we have already demonstrated that $\bm A_0 \bm B_0$ is close to $\Delta$. In the following content, we aim to track how $\left\|\bm A_t \bm B_t - \Delta\right\|_{op}$ behaves (in a local sense), which is a critical ingredient to study both the loss and risk of LoRA training. In this regime, there is no significant difference on setting different step-size $\eta_1$ and $\eta_2$. For ease of description, we set $\eta_1=\eta_2 := \eta$.

Here we can characterize the operator norm of $\left(\bm A_t \bm B_t - \Delta\right)$ as
\begin{align}
    \left\|\bm A_t \bm B_t - \Delta\right\|_{op} & = \left\|\bigg(\bm A_t \bm B_t - \Delta\bigg)\begin{bmatrix}
        \bm V & \bm V_\perp
    \end{bmatrix}\right\|_{op} \quad \tag*{\color{teal}[by unitary invariance of operator norm]}\nonumber\\
    & = \left\|\bm A_t \bm B_t\bm V - \bm U\bm S^*\right\|_{op} \quad \tag*{\color{teal}[by \cref{linear-invariant-B2}]}\nonumber\\
    & = \left\|\bigg(\bm U \bm U^{\!\top}+\bm U_{\perp} \bm U^{\!\top}_\perp\bigg)\bigg(\bm A_t \bm B_t\bm V - \bm U\bm S^*\bigg)\right\|_{op}\nonumber\\
    & = \left\|\bm U \bigg(\bm U^{\!\top}\bm A_t \bm B_t\bm V - \bm S^*\bigg)\right\|_{op}+\left\|\bm U_{\perp} \bm U^{\!\top}_\perp\bm A_t \bm B_t\bm V\right\|_{op}\nonumber\\
    & \leq  \underbrace{\left\|\bm U^{\!\top}\bm A_t \bm B_t\bm V - \bm S^*\right\|_{op}}_{\mbox{signal space}}+ \underbrace{\left\|\bm U^{\!\top}_\perp\bm A_t \bm B_t\bm V\right\|_{op}}_{\mbox{complementary}}\,,\label{deco}
\end{align}
where the first term denotes the loss in the signal space $\left\|\bm U^{\!\top}\bm A \bm B\bm V - \bm S^*\right\|_{op}$ and the second term denotes the complementary space decay $\left\|\bm U^{\!\top}_\perp\bm A \bm B\bm V\right\|_{op}$. Next, we need a new parametrization to track the dynamics of these two terms. Recall the complete SVD of $\Delta$ in \cref{Delta-SVD} as
\begin{align*}
\Delta=\widetilde{\bm U} \widetilde{\bm S}^* \widetilde{\bm V}^{\!\top}=
    \begin{bmatrix}
        \bm U & \bm U_\perp
    \end{bmatrix}\begin{bmatrix}
       \bm S^* & \bm 0_{r^*\times (d-r^*)}\\
        \bm 0_{(d-r^*)\times r^*} & \bm 0_{(d-r^*)\times (d-r^*)}
    \end{bmatrix}\begin{bmatrix}
        \bm V^{\!\top} \\ \bm V_\perp^{\!\top}
    \end{bmatrix}\,.
\end{align*}
For notational simplicity, we denote 
\begin{align*}
    \bm A^{\bm U}_t:=\bm U^{\!\top}\bm A_t\,,\quad \bm A^{\bm U_\perp}_t:=\bm U_\perp^{\!\top}\bm A_t\,,\quad \bm B_t \bm V:=\bm B_t^{\bm V}\,,\quad \bm B_t \bm V_\perp:=\bm B_t^{\bm V_\perp}\,.
\end{align*}
and thus
\begin{align*}
    \bm R_t := (\bm A_t \bm B_t - \Delta) \bm V \,,\quad \bm R_t^*:=\bm A^{\bm U}_t\bm B_t^{\bm V}-\bm S^*\,,\quad \bm R_t^\perp := \bm A^{\bm U_\perp}_t\bm B_t^{\bm V}\,.
\end{align*}
Accordingly, \cref{deco} can be reformulated as $\left\|\bm R_t\right\|_{op}\leq \left\|\bm R_t^*\right\|_{op}+\left\|\bm R_t^\perp\right\|_{op}$. By \cref{linear-invariant-B2}, we have $\bm B^{\bm V_\perp}=\bm 0_{r\times(k-r^*)}$ for $\forall\,t \in \mathbb{N}^+$.
Next, we can track $\bm R^*_t$ and $\bm R^\perp_t$ via the following two lemmas.
\begin{lemma}
    \label{A^U_t-B_t^V}
Under assumptions in \cref{sec:assumptions} for the linear setting, with spectral initialization \eqref{eq:spectral-init-linear}, we have the following reparametrized iterates
    \begin{align}
        \bm A^{\bm U}_{t+1} & = \bm A^{\bm U}_t - \eta \bm R^*_t \left(\bm B_t^{\bm V}\right)^{\!\top} - \eta \bm U^{\!\top}\left(\widehat{\bm \Sigma}
        - \bm I_d\right) \bm R_t \left(\bm B_t^{\bm V}\right)^{\!\top}\,,\label{AtU}\\
        \bm A^{\bm U_\perp}_{t+1} & = \bm A^{\bm U_\perp}_t - \eta \bm R^\perp_t \left(\bm B_t^{\bm V}\right)^{\!\top} - \eta \bm U^{\!\top}_\perp\left(\widehat{\bm \Sigma}
        - \bm I_d\right) \bm R_t \left(\bm B_t^{\bm V}\right)^{\!\top}\,,\label{AUperp}\\
        \bm B_{t+1}^{\bm V} & = \bm B_t^{\bm V} - \eta \left(\bm A^{\bm U}_t\right)^{\!\top}\bm R^*_t - \eta \left(\bm A^{\bm U_\perp}_t\right)^{\!\top}\bm R^\perp_t\nonumber\\
        & - \eta \left(\bm A_t^{\bm U}\right)^{\!\top}\bm U^{\!\top}\left(\widehat{\bm \Sigma}-\bm I_d\right)\bm R_t  - \eta \left(\bm A_t^{\bm U_\perp}\right)^{\!\top}\bm U_\perp^{\!\top}\left(\widehat{\bm \Sigma}-\bm I_d\right)\bm R_t\label{BVt}\,.
    \end{align}
\end{lemma}
\begin{proof}
    Recall the gradient update for $\bm A_{t+1}$, we have
    \begin{align*}
        \bm A_{t+1} & = \bm A_t - \eta \widehat{\bm \Sigma}\left(\bm A_t \bm B_t - \Delta\right)\left(\bm B_t\right)^{\!\top}\\
        & = \bm A_t - \eta \left(\bm A_t \bm B_t - \Delta\right)\left(\bm B_t\right)^{\!\top} - \eta \left(\widehat{\bm \Sigma}-\bm I_d\right)\left(\bm A_t \bm B_t - \Delta\right)\left(\bm B_t\right)^{\!\top}\,.
    \end{align*}
   Recall $\bm R_t := (\bm A_t \bm B_t - \Delta)\bm V$ and $\Delta = \bm U \bm S^* \bm V^{\!\top}$, we have
    \begin{align*}
        \bm U^{\!\top}\bm A_{t+1} & = \bm U^{\!\top}\bm A_t - \eta \bm U^{\!\top}\left(\bm A_t \bm B_t - \Delta\right)\left(\bm V\bm V^{\!\top}+\bm V_\perp \bm V_\perp^{\!\top}\right)\left(\bm B_t\right)^{\!\top}\\
        &- \eta \bm U^{\!\top}\left(\widehat{\bm \Sigma}-\bm I_d\right)\left(\bm A_t \bm B_t - \Delta\right)\left(\bm V\bm V^{\!\top}+\bm V_\perp \bm V_\perp^{\!\top}\right)\left(\bm B_t\right)^{\!\top}\\
        & = \bm U^{\!\top}\bm A_t - \eta \bm U^{\!\top}\left(\bm A_t \bm B_t\bm V - \Delta\bm V\right)\left(\bm B_t\bm V\right)^{\!\top} - \eta \bm U^{\!\top}\left(\widehat{\bm \Sigma}
        - \bm I_d\right)\left(\bm A_t \bm B_t\bm V - \Delta\bm V\right)\left(\bm B_t\bm V\right)^{\!\top}\quad \tag*{\color{teal}[by \cref{linear-invariant-B2}]}\\
        & = \bm U^{\!\top}\bm A_t - \eta \left(\bm U^{\!\top}\bm A_t \bm B_t\bm V - \bm S^*\right)\left(\bm B_t\bm V\right)^{\!\top}
         - \eta \bm U^{\!\top}\left(\widehat{\bm \Sigma}
        - \bm I_d\right)\bm R_t \left(\bm B_t\bm V\right)^{\!\top}\,.
    \end{align*} 
   Accordingly, the recursion for $\bm A^{\bm U}_{t+1}$ is reformulated as
    \begin{align*}
        \bm A^{\bm U}_{t+1} & = \bm A^{\bm U}_t - \eta \bm R^*_t \left(\bm B_t^{\bm V}\right)^{\!\top} - \eta \bm U^{\!\top}\left(\widehat{\bm \Sigma}
        - \bm I_d\right) \bm R_t \left(\bm B_t^{\bm V}\right)^{\!\top}\,.
    \end{align*}
    Similarly, we can obtain
    \begin{align*}
        \bm A^{\bm U_\perp}_{t+1} & = \bm A^{\bm U_\perp}_t - \eta \bm R^\perp_t \left(\bm B_t^{\bm V}\right)^{\!\top} - \eta \bm U^{\!\top}_\perp\left(\widehat{\bm \Sigma}
        - \bm I_d\right) \bm R_t \left(\bm B_t^{\bm V}\right)^{\!\top}\,.
    \end{align*}
    Regarding the recursion for $ \bm B_{t+1}$, we can derive in a similar way
    \begin{align*}
        \bm B_{t+1}\bm V & = \bm B_t\bm V - \eta \left(\bm A_t\right)^{\!\top}\widehat{\bm \Sigma}\left(\bm A_t \bm B_t - \Delta\right)\bm V\\
        & = \bm B_t\bm V - \eta \left(\bm A_t\right)^{\!\top}\left(\bm U \bm U^{\!\top}+\bm U_\perp \bm U_\perp^{\!\top}\right)\left(\bm A_t \bm B_t - \Delta\right)\bm V\\
        & - \eta \left(\bm A_t\right)^{\!\top}\left(\bm U \bm U^{\!\top}+\bm U_\perp \bm U_\perp^{\!\top}\right)\left(\widehat{\bm \Sigma}-\bm I_d\right)\left(\bm A_t \bm B_t - \Delta\right)\bm V\,,
    \end{align*}
    which implies
    \begin{align*}
        \bm B_{t+1}^{\bm V} & = \bm B_t^{\bm V} - \eta \left(\bm A^{\bm U}_t\right)^{\!\top}\bm R^*_t - \eta \left(\bm A^{\bm U_\perp}_t\right)^{\!\top}\bm R^\perp_t  - \eta \left(\bm A_t^{\bm U}\right)^{\!\top}\bm U^{\!\top}\left(\widehat{\bm \Sigma}-\bm I_d\right)\bm R_t  - \eta \left(\bm A_t^{\bm U_\perp}\right)^{\!\top}\bm U_\perp^{\!\top}\left(\widehat{\bm \Sigma}-\bm I_d\right)\bm R_t\,.
    \end{align*}
\end{proof}
In the next, we are able to characterize the upper bound of $  \left\|\bm R^*_{t+1}\right\|_{op}$.
\begin{lemma}
\label{mathcalMt}
    Denote $\mathcal{M}_t:=\max \left\{\left\|\bm R^*_{t}\right\|_{op}\,,\left\|\bm R^\perp_{t}\right\|_{op}\right\}$, 
under assumptions in \cref{sec:assumptions} for the linear setting, with spectral initialization \eqref{eq:spectral-init-linear}, then we choose $\epsilon$ with probability at least $1- 2C\exp(-\epsilon^2 N)$ for a universal constant $C>0$, we have 
    \begin{align*}
        \left\|\bm R^*_{t+1}\right\|_{op} & \leq \bigg(1-\eta\left(\lambda_{r^*}^2\left(\bm A_t^{\bm U}\right)+\lambda_{r^*}^2\left(\bm B_t^{\bm V}\right)\right)\bigg)\mathcal{M}_t\\
        & + 2\eta \epsilon \left\|\bm B^{\bm V}_{t}\right\|_{op}^2 \mathcal{M}_t + \eta^2 \left\|\bm A^{\bm U}_{t}\right\|_{op}\left\|\bm B^{\bm V}_{t}\right\|_{op}\mathcal{M}_t^2 + 2\eta^2 \epsilon \left\|\bm A^{\bm U}_{t}\right\|_{op}\left\|\bm B^{\bm V}_{t}\right\|_{op}\mathcal{M}_t^2\\
        & + \eta \left\|\bm A^{\bm U}_{t}\right\|_{op}\left\|\bm A^{\bm U_\perp}_{t}\right\|_{op}\mathcal{M}_t+\eta^2 \left\|\bm B^{\bm V}_{t}\right\|_{op}\left\|\bm A^{\bm U_\perp}_{t}\right\|_{op}\mathcal{M}_t^2\\
        & +2\eta^2 \epsilon \left\|\bm B^{\bm V}_{t}\right\|_{op}\left\|\bm A^{\bm U_\perp}_{t}\right\|_{op}\mathcal{M}_t^2+2\eta \epsilon \left\|\bm A^{\bm U}_{t}\right\|_{op}^2 \mathcal{M}_t\\
        & + 2 \eta^2 \epsilon \left\|\bm A^{\bm U}_{t}\right\|_{op}\left\|\bm B^{\bm V}_{t}\right\|_{op} \mathcal{M}_t + 4 \eta^2 \epsilon^2 \left\|\bm A^{\bm U}_{t}\right\|_{op}\left\|\bm B^{\bm V}_{t}\right\|_{op} \mathcal{M}_t^2\\
        & + 2 \eta \epsilon \left\|\bm A^{\bm U}_{t}\right\|_{op}\left\|\bm A^{\bm U_\perp}_{t}\right\|_{op}\mathcal{M}_t+2\eta^2 \epsilon \left\|\bm A^{\bm U_\perp}_{t}\right\|_{op}\left\|\bm B^{\bm V}_{t}\right\|_{op}\mathcal{M}_t^2\\
        & + 4 \eta^2 \epsilon^2 \left\|\bm A^{\bm U_\perp}_{t}\right\|_{op}\left\|\bm B^{\bm V}_{t}\right\|_{op} \mathcal{M}_t^2\,,
    \end{align*}
    and
    \begin{align}
        \left\|\bm R^\perp_{t+1}\right\|_{op} & \leq \bigg(1-\eta\left(\lambda_{\min}^2\left(\bm A_t^{\bm U_\perp}\right)+\lambda_{r^*}^2\left(\bm B_t^{\bm V}\right)\right)\bigg)\mathcal{M}_t\label{drop}\\
        & + 2 \eta \epsilon \left\|\bm B^{\bm V}_{t}\right\|_{op}^2 \mathcal{M}_t + \eta^2 \left\|\bm A^{\bm U}_{t}\right\|_{op}\left\|\bm B^{\bm V}_{t}\right\|_{op}\mathcal{M}_t^2 + 2\eta^2 \epsilon \left\|\bm A^{\bm U}_{t}\right\|_{op}\left\|\bm B^{\bm V}_{t}\right\|_{op}\mathcal{M}_t^2\nonumber\\
        & + \eta \left\|\bm A^{\bm U}_{t}\right\|_{op}\left\|\bm A^{\bm U_\perp}_{t}\right\|_{op}\mathcal{M}_t+\eta^2 \left\|\bm B^{\bm V}_{t}\right\|_{op}\left\|\bm A^{\bm U_\perp}_{t}\right\|_{op}\mathcal{M}_t^2\nonumber\\
        & +2\eta^2 \epsilon \left\|\bm B^{\bm V}_{t}\right\|_{op}\left\|\bm A^{\bm U_\perp}_{t}\right\|_{op}\mathcal{M}_t^2+2\eta \epsilon \left\|\bm A^{\bm U}_{t}\right\|_{op}\left\|\bm A^{\bm U_\perp}_{t}\right\|_{op} \mathcal{M}_t\nonumber\\
        & + 2 \eta^2 \epsilon \left\|\bm A^{\bm U}_{t}\right\|_{op}\left\|\bm B^{\bm V}_{t}\right\|_{op} \mathcal{M}_t + 4 \eta^2 \epsilon^2 \left\|\bm A^{\bm U}_{t}\right\|_{op}\left\|\bm B^{\bm V}_{t}\right\|_{op} \mathcal{M}_t^2\nonumber\\
        & + 2 \eta \epsilon \left\|\bm A^{\bm U_\perp}_{t}\right\|_{op}^2\mathcal{M}_t+2\eta^2 \epsilon \left\|\bm A^{\bm U_\perp}_{t}\right\|_{op}\left\|\bm B^{\bm V}_{t}\right\|_{op}\mathcal{M}_t^2\nonumber\\
        & + 4 \eta^2 \epsilon^2 \left\|\bm A^{\bm U_\perp}_{t}\right\|_{op}\left\|\bm B^{\bm V}_{t}\right\|_{op} \mathcal{M}_t^2\,.\nonumber
    \end{align}
\end{lemma}
\begin{proof}
    Here we first track the dynamics of $\bm R^*_t$. We have
    \begin{align*}
        \bm R^*_{t+1} & = \bm A^{\bm U}_{t+1}\bm B_{t+1}^{\bm V} - \bm S^*\\
        & = \bm R^*_t
        - \eta \bm R^*_t \left(\bm B_t^{\bm V}\right)^{\!\top}\bm B_{t}^{\bm V}
        - \eta \bm U^{\!\top}\left(\widehat{\bm \Sigma}
        - \bm I_d\right) \bm R_t \left(\bm B_t^{\bm V}\right)^{\!\top}\bm B_{t}^{\bm V}\\
        & -\eta \bm A^{\bm U}_t\left(\bm A^{\bm U}_t\right)^{\!\top}\bm R^*_t
        + \eta^2 \bm R^*_t \left(\bm B_t^{\bm V}\right)^{\!\top}\left(\bm A^{\bm U}_t\right)^{\!\top}\bm R^*_t
        + \eta^2 \bm U^{\!\top}\left(\widehat{\bm \Sigma}
        - \bm I_d\right) \bm R_t \left(\bm B_t^{\bm V}\right)^{\!\top}\left(\bm A^{\bm U}_t\right)^{\!\top}\bm R^*_t\\
        & - \eta \bm A^{\bm U}_t \left(\bm A^{\bm U_\perp}_t\right)^{\!\top}\bm R^\perp_t
        + \eta^2 \bm R^*_t \left(\bm B_t^{\bm V}\right)^{\!\top}\left(\bm A^{\bm U_\perp}_t\right)^{\!\top}\bm R^\perp_t
        + \eta^2 \bm U^{\!\top}\left(\widehat{\bm \Sigma}
        - \bm I_d\right) \bm R_t \left(\bm B_t^{\bm V}\right)^{\!\top}\left(\bm A^{\bm U_\perp}_t\right)^{\!\top}\bm R^\perp_t\\
        & ------ \\
        & - \eta \bm A^{\bm U}_t \left(\bm A_t^{\bm U}\right)^{\!\top}\bm U^{\!\top}\left(\widehat{\bm \Sigma}-\bm I_d\right)\bm R_t\\
        & + \eta^2 \bm R^*_t \left(\bm B_t^{\bm V}\right)^{\!\top}\left(\bm A_t^{\bm U}\right)^{\!\top}\bm U^{\!\top}\left(\widehat{\bm \Sigma}-\bm I_d\right)\bm R_t\\
        & + \eta^2 \bm U^{\!\top}\left(\widehat{\bm \Sigma}
        - \bm I_d\right) \bm R_t \left(\bm B_t^{\bm V}\right)^{\!\top}\left(\bm A_t^{\bm U}\right)^{\!\top}\bm U^{\!\top}\left(\widehat{\bm \Sigma}-\bm I_d\right)\bm R_t\\
        & ------ \\
        & - \eta \bm A^{\bm U}_t \left(\bm A_t^{\bm U_\perp}\right)^{\!\top}\bm U_\perp^{\!\top}\left(\widehat{\bm \Sigma}-\bm I_d\right)\bm R_t \\
        & + \eta^2 \bm R^*_t \left(\bm B_t^{\bm V}\right)^{\!\top}\left(\bm A_t^{\bm U_\perp}\right)^{\!\top}\bm U_\perp^{\!\top}\left(\widehat{\bm \Sigma}-\bm I_d\right)\bm R_t\\
        & + \eta^2 \bm U^{\!\top}\left(\widehat{\bm \Sigma}
        - \bm I_d\right) \bm R_t \left(\bm B_t^{\bm V}\right)^{\!\top}\left(\bm A_t^{\bm U_\perp}\right)^{\!\top}\bm U_\perp^{\!\top}\left(\widehat{\bm \Sigma}-\bm I_d\right)\bm R_t\,.
    \end{align*}
    Then, we take operator norm over the above equation. Hence, with probability at least $1- 2C\exp(-\epsilon^2 N)$ for a universal constant $C>0$, we have
    \begin{align*}
        \left\|\bm R^*_{t+1}\right\|_{op} & \leq \bigg(1-\eta\left(\lambda_{r^*}^2\left(\bm A_t^{\bm U}\right)+\lambda_{r^*}^2\left(\bm B_t^{\bm V}\right)\right)\bigg)\left\|\bm R^*_{t}\right\|_{op}\\
        & + \eta \epsilon \left\|\bm B^{\bm V}_{t}\right\|_{op}^2 \left\|\bm R_{t}\right\|_{op} + \eta^2 \left\|\bm A^{\bm U}_{t}\right\|_{op}\left\|\bm B^{\bm V}_{t}\right\|_{op}\left\|\bm R^*_{t}\right\|_{op}^2 + \eta^2 \epsilon \left\|\bm A^{\bm U}_{t}\right\|_{op}\left\|\bm B^{\bm V}_{t}\right\|_{op}\left\|\bm R^*_{t}\right\|_{op}\left\|\bm R_{t}\right\|_{op}\\
        & + \eta \left\|\bm A^{\bm U}_{t}\right\|_{op}\left\|\bm A^{\bm U_\perp}_{t}\right\|_{op}\left\|\bm R^\perp_{t}\right\|_{op}+\eta^2 \left\|\bm B^{\bm V}_{t}\right\|_{op}\left\|\bm A^{\bm U_\perp}_{t}\right\|_{op}\left\|\bm R^\perp_{t}\right\|_{op}\left\|\bm R^*{t}\right\|_{op}\\
        & +\eta^2 \epsilon \left\|\bm B^{\bm V}_{t}\right\|_{op}\left\|\bm A^{\bm U_\perp}_{t}\right\|_{op}\left\|\bm R^\perp_{t}\right\|_{op}\left\|\bm R_{t}\right\|_{op}+\eta \epsilon \left\|\bm A^{\bm U}_{t}\right\|_{op}^2 \left\|\bm R_{t}\right\|_{op}\\
        & + \eta^2 \epsilon \left\|\bm A^{\bm U}_{t}\right\|_{op}\left\|\bm B^{\bm V}_{t}\right\|_{op} \left\|\bm R_{t}\right\|_{op} + \eta^2 \epsilon^2 \left\|\bm A^{\bm U}_{t}\right\|_{op}\left\|\bm B^{\bm V}_{t}\right\|_{op} \left\|\bm R_{t}\right\|_{op}^2\\
        & + \eta \epsilon \left\|\bm A^{\bm U}_{t}\right\|_{op}\left\|\bm A^{\bm U_\perp}_{t}\right\|_{op}\left\|\bm R_{t}\right\|_{op}+\eta^2 \epsilon \left\|\bm A^{\bm U_\perp}_{t}\right\|_{op}\left\|\bm B^{\bm V}_{t}\right\|_{op}\left\|\bm R^*_{t}\right\|_{op}\left\|\bm R_{t}\right\|_{op}\\
        & + \eta^2 \epsilon^2 \left\|\bm A^{\bm U_\perp}_{t}\right\|_{op}\left\|\bm B^{\bm V}_{t}\right\|_{op} \left\|\bm R_{t}\right\|_{op}^2\,.
    \end{align*}
    Next, we take maximum over $\left\|\bm R^*_{t}\right\|_{op}$  and $\left\|\bm R^\perp_{t}\right\|_{op}$ on the right hand side above. Recall $\mathcal{M}_t=\max \left\{\left\|\bm R^*_{t}\right\|_{op}\,,\left\|\bm R^\perp_{t}\right\|_{op}\right\}$, using the fact that $\left\|\bm R_{t}\right\|_{op}\leq 2 \mathcal{M}_t$, we have:
    \begin{align*}
        \left\|\bm R^*_{t+1}\right\|_{op} & \leq \bigg(1-\eta\left(\lambda_{r^*}^2\left(\bm A_t^{\bm U}\right)+\lambda_{r^*}^2\left(\bm B_t^{\bm V}\right)\right)\bigg)\mathcal{M}_t\\
        & + 2\eta \epsilon \left\|\bm B^{\bm V}_{t}\right\|_{op}^2 \mathcal{M}_t + \eta^2 \left\|\bm A^{\bm U}_{t}\right\|_{op}\left\|\bm B^{\bm V}_{t}\right\|_{op}\mathcal{M}_t^2 + 2\eta^2 \epsilon \left\|\bm A^{\bm U}_{t}\right\|_{op}\left\|\bm B^{\bm V}_{t}\right\|_{op}\mathcal{M}_t^2\\
        & + \eta \left\|\bm A^{\bm U}_{t}\right\|_{op}\left\|\bm A^{\bm U_\perp}_{t}\right\|_{op}\mathcal{M}_t+\eta^2 \left\|\bm B^{\bm V}_{t}\right\|_{op}\left\|\bm A^{\bm U_\perp}_{t}\right\|_{op}\mathcal{M}_t^2\\
        & +2\eta^2 \epsilon \left\|\bm B^{\bm V}_{t}\right\|_{op}\left\|\bm A^{\bm U_\perp}_{t}\right\|_{op}\mathcal{M}_t^2+2\eta \epsilon \left\|\bm A^{\bm U}_{t}\right\|_{op}^2 \mathcal{M}_t\\
        & + 2 \eta^2 \epsilon \left\|\bm A^{\bm U}_{t}\right\|_{op}\left\|\bm B^{\bm V}_{t}\right\|_{op} \mathcal{M}_t + 4 \eta^2 \epsilon^2 \left\|\bm A^{\bm U}_{t}\right\|_{op}\left\|\bm B^{\bm V}_{t}\right\|_{op} \mathcal{M}_t^2\\
        & + 2 \eta \epsilon \left\|\bm A^{\bm U}_{t}\right\|_{op}\left\|\bm A^{\bm U_\perp}_{t}\right\|_{op}\mathcal{M}_t+2\eta^2 \epsilon \left\|\bm A^{\bm U_\perp}_{t}\right\|_{op}\left\|\bm B^{\bm V}_{t}\right\|_{op}\mathcal{M}_t^2\\
        & + 4 \eta^2 \epsilon^2 \left\|\bm A^{\bm U_\perp}_{t}\right\|_{op}\left\|\bm B^{\bm V}_{t}\right\|_{op} \mathcal{M}_t^2\,.
    \end{align*}
    Next, we track the dynamics of $\bm R^\perp_t$. We have
    \begin{align*}
        \bm R^\perp_{t+1} & = \bm A^{\bm U_\perp}_{t+1}\bm B_{t+1}^{\bm V}\\
        & = \bm R^\perp_t
        - \eta \bm R^\perp_t \left(\bm B_t^{\bm V}\right)^{\!\top}\bm B_{t}^{\bm V}
        - \eta \bm U_\perp^{\!\top}\left(\widehat{\bm \Sigma}
        - \bm I_d\right) \bm R_t \left(\bm B_t^{\bm V}\right)^{\!\top}\bm B_{t}^{\bm V}\\
        & -\eta \bm A^{\bm U_\perp}_t\left(\bm A^{\bm U}_t\right)^{\!\top}\bm R^*_t
        + \eta^2 \bm R^\perp_t \left(\bm B_t^{\bm V}\right)^{\!\top}\left(\bm A^{\bm U}_t\right)^{\!\top}\bm R^*_t
        + \eta^2 \bm U_\perp^{\!\top}\left(\widehat{\bm \Sigma}
        - \bm I_d\right) \bm R_t \left(\bm B_t^{\bm V}\right)^{\!\top}\left(\bm A^{\bm U}_t\right)^{\!\top}\bm R^*_t\\
        & - \eta \bm A^{\bm U_\perp}_t \left(\bm A^{\bm U_\perp}_t\right)^{\!\top}\bm R^\perp_t
        + \eta^2 \bm R^\perp_t \left(\bm B_t^{\bm V}\right)^{\!\top}\left(\bm A^{\bm U_\perp}_t\right)^{\!\top}\bm R^\perp_t
        + \eta^2 \bm U_\perp^{\!\top}\left(\widehat{\bm \Sigma}
        - \bm I_d\right) \bm R_t \left(\bm B_t^{\bm V}\right)^{\!\top}\left(\bm A^{\bm U_\perp}_t\right)^{\!\top}\bm R^\perp_t\\
        & - \eta \bm A^{\bm U_\perp}_t \left(\bm A_t^{\bm U}\right)^{\!\top}\bm U^{\!\top}\left(\widehat{\bm \Sigma}-\bm I_d\right)\bm R_t\\
        & + \eta^2 \bm R^\perp_t \left(\bm B_t^{\bm V}\right)^{\!\top}\left(\bm A_t^{\bm U}\right)^{\!\top}\bm U^{\!\top}\left(\widehat{\bm \Sigma}-\bm I_d\right)\bm R_t\\
        & + \eta^2 \bm U_\perp^{\!\top}\left(\widehat{\bm \Sigma}
        - \bm I_d\right) \bm R_t \left(\bm B_t^{\bm V}\right)^{\!\top}\left(\bm A_t^{\bm U}\right)^{\!\top}\bm U^{\!\top}\left(\widehat{\bm \Sigma}-\bm I_d\right)\bm R_t\\
        & - \eta \bm A^{\bm U_\perp}_t \left(\bm A_t^{\bm U_\perp}\right)^{\!\top}\bm U_\perp^{\!\top}\left(\widehat{\bm \Sigma}-\bm I_d\right)\bm R_t \\
        & + \eta^2 \bm R^\perp_t \left(\bm B_t^{\bm V}\right)^{\!\top}\left(\bm A_t^{\bm U_\perp}\right)^{\!\top}\bm U_\perp^{\!\top}\left(\widehat{\bm \Sigma}-\bm I_d\right)\bm R_t\\
        & + \eta^2 \bm U_\perp^{\!\top}\left(\widehat{\bm \Sigma}
        - \bm I_d\right) \bm R_t \left(\bm B_t^{\bm V}\right)^{\!\top}\left(\bm A_t^{\bm U_\perp}\right)^{\!\top}\bm U_\perp^{\!\top}\left(\widehat{\bm \Sigma}-\bm I_d\right)\bm R_t\,.
    \end{align*}
    Then, we take operator norm over the above equation. With probability at least $1- 2C\exp(-\epsilon^2 N)$ for a universal constant $C>0$, we have
    \begin{align*}
        \left\|\bm R^\perp_{t+1}\right\|_{op} & \leq \bigg(1-\eta\left(\lambda_{\min}^2\left(\bm A_t^{\bm U_\perp}\right)+\lambda_{r^*}^2\left(\bm B_t^{\bm V}\right)\right)\bigg)\left\|\bm R^\perp_{t}\right\|_{op}\\
        & + \eta \epsilon \left\|\bm B^{\bm V}_{t}\right\|_{op}^2 \left\|\bm R_{t}\right\|_{op} + \eta^2 \left\|\bm A^{\bm U}_{t}\right\|_{op}\left\|\bm B^{\bm V}_{t}\right\|_{op}\left\|\bm R^*_{t}\right\|_{op}\left\|\bm R^\perp_{t}\right\|_{op} + \eta^2 \epsilon \left\|\bm A^{\bm U}_{t}\right\|_{op}\left\|\bm B^{\bm V}_{t}\right\|_{op}\left\|\bm R^*_{t}\right\|_{op}\left\|\bm R_{t}\right\|_{op}\\
        & + \eta \left\|\bm A^{\bm U}_{t}\right\|_{op}\left\|\bm A^{\bm U_\perp}_{t}\right\|_{op}\left\|\bm R^*_{t}\right\|_{op}+\eta^2 \left\|\bm B^{\bm V}_{t}\right\|_{op}\left\|\bm A^{\bm U_\perp}_{t}\right\|_{op}\left\|\bm R^\perp_{t}\right\|_{op}^2\\
        & +\eta^2 \epsilon \left\|\bm B^{\bm V}_{t}\right\|_{op}\left\|\bm A^{\bm U_\perp}_{t}\right\|_{op}\left\|\bm R^\perp_{t}\right\|_{op}\left\|\bm R_{t}\right\|_{op}+\eta \epsilon \left\|\bm A^{\bm U}_{t}\right\|_{op}\left\|\bm A^{\bm U_\perp}_{t}\right\|_{op} \left\|\bm R_{t}\right\|_{op}\\
        & + \eta^2 \epsilon \left\|\bm A^{\bm U}_{t}\right\|_{op}\left\|\bm B^{\bm V}_{t}\right\|_{op} \left\|\bm R_{t}\right\|_{op} + \eta^2 \epsilon^2 \left\|\bm A^{\bm U}_{t}\right\|_{op}\left\|\bm B^{\bm V}_{t}\right\|_{op} \left\|\bm R_{t}\right\|_{op}^2\\
        & + \eta \epsilon \left\|\bm A^{\bm U_\perp}_{t}\right\|_{op}^2\left\|\bm R_{t}\right\|_{op}+\eta^2 \epsilon \left\|\bm A^{\bm U_\perp}_{t}\right\|_{op}\left\|\bm B^{\bm V}_{t}\right\|_{op}\left\|\bm R^\perp_{t}\right\|_{op}\left\|\bm R_{t}\right\|_{op}\\
        & + \eta^2 \epsilon^2 \left\|\bm A^{\bm U_\perp}_{t}\right\|_{op}\left\|\bm B^{\bm V}_{t}\right\|_{op} \left\|\bm R_{t}\right\|_{op}^2\,.
    \end{align*}
    Next, we take maximum over $\left\|\bm R^*_{t}\right\|_{op}$  and $\left\|\bm R^\perp_{t}\right\|_{op}$ on the right hand side above. Recall $\mathcal{M}_t=\max \left\{\left\|\bm R^*_{t}\right\|_{op}\,,\left\|\bm R^\perp_{t}\right\|_{op}\right\}$, using the fact that $\left\|\bm R_{t}\right\|_{op}\leq 2 \mathcal{M}_t$, we have:
    \begin{align*}
        \left\|\bm R^\perp_{t+1}\right\|_{op} & \leq \bigg(1-\eta\left(\lambda_{\min}^2\left(\bm A_t^{\bm U_\perp}\right)+\lambda_{r^*}^2\left(\bm B_t^{\bm V}\right)\right)\bigg)\mathcal{M}_t\\
        & + 2 \eta \epsilon \left\|\bm B^{\bm V}_{t}\right\|_{op}^2 \mathcal{M}_t + \eta^2 \left\|\bm A^{\bm U}_{t}\right\|_{op}\left\|\bm B^{\bm V}_{t}\right\|_{op}\mathcal{M}_t^2 + 2\eta^2 \epsilon \left\|\bm A^{\bm U}_{t}\right\|_{op}\left\|\bm B^{\bm V}_{t}\right\|_{op}\mathcal{M}_t^2\\
        & + \eta \left\|\bm A^{\bm U}_{t}\right\|_{op}\left\|\bm A^{\bm U_\perp}_{t}\right\|_{op}\mathcal{M}_t+\eta^2 \left\|\bm B^{\bm V}_{t}\right\|_{op}\left\|\bm A^{\bm U_\perp}_{t}\right\|_{op}\mathcal{M}_t^2\\
        & +2\eta^2 \epsilon \left\|\bm B^{\bm V}_{t}\right\|_{op}\left\|\bm A^{\bm U_\perp}_{t}\right\|_{op}\mathcal{M}_t^2+2\eta \epsilon \left\|\bm A^{\bm U}_{t}\right\|_{op}\left\|\bm A^{\bm U_\perp}_{t}\right\|_{op} \mathcal{M}_t\\
        & + 2 \eta^2 \epsilon \left\|\bm A^{\bm U}_{t}\right\|_{op}\left\|\bm B^{\bm V}_{t}\right\|_{op} \mathcal{M}_t + 4 \eta^2 \epsilon^2 \left\|\bm A^{\bm U}_{t}\right\|_{op}\left\|\bm B^{\bm V}_{t}\right\|_{op} \mathcal{M}_t^2\\
        & + 2 \eta \epsilon \left\|\bm A^{\bm U_\perp}_{t}\right\|_{op}^2\mathcal{M}_t+2\eta^2 \epsilon \left\|\bm A^{\bm U_\perp}_{t}\right\|_{op}\left\|\bm B^{\bm V}_{t}\right\|_{op}\mathcal{M}_t^2\\
        & + 4 \eta^2 \epsilon^2 \left\|\bm A^{\bm U_\perp}_{t}\right\|_{op}\left\|\bm B^{\bm V}_{t}\right\|_{op} \mathcal{M}_t^2\,.
    \end{align*}
    Finally we conclude the proof.
\end{proof}
Before we move to the main proof, we need to establish a strict upper bound on $\bm A_t$ and $\bm B_t$.
\begin{lemma}
\label{mahdi-upper}
    Under assumptions in \cref{sec:assumptions} for the linear setting, suppose $\left\|\bm A_t^{\!\top}\bm A_t - \bm B_t^{\!\top}\bm B_t\right\|_{op} + \epsilon \left\|\bm R_t\right\|_{op} \leq \lambda_1^*$ and $\eta\leq \frac{1}{10\lambda_1^*}$, if $\left\|\bm A_{t}\right\|_{op}\leq 2\sqrt{\lambda_1^*}$ and $\left\|\bm B_{t}\right\|_{op}\leq 2\sqrt{\lambda_1^*}$, we choose $\epsilon$ satisfying \cref{concentration-N}, then with probability $1- 2C\exp(-\epsilon^2 N)$ for a universal constant $C>0$, we have
    \begin{align*}
        \left\|\bm A_{t+1}\right\|_{op}\leq 2\sqrt{\lambda_1^*}\,,\quad\left\|\bm B_{t+1}\right\|_{op}\leq 2\sqrt{\lambda_1^*}\,.
    \end{align*}
\end{lemma}
\begin{proof}
    Inspired by \cite{soltanolkotabi2023implicit}, we recall the stacked iterate $\bm Z_t$ defined in \cref{stack-Z} and construct an anti-iterate
    \begin{align*}
        \underline{\bm Z}_t:=\begin{bmatrix}
            \bm A_t \\ -\bm B_t^{\!\top}
        \end{bmatrix}\,.
    \end{align*}
Additionally, we define a perturbation matrix
    \begin{align*}
        \bm \Xi_t & := \begin{bmatrix}
            \bm 0_{d\times d} & \left(\widetilde{\bm \Sigma}-\bm I_d\right)\bm R_t \\
            \bm R_t^{\!\top}\left(\widetilde{\bm \Sigma}-\bm I_d\right) & \bm 0_{k\times k}
        \end{bmatrix}\,.
    \end{align*}
    Then, we can reformulate the recursion of $\bm Z_{t+1}$ as
    \begin{align*}
        \bm Z_{t+1} & = \bm Z_t - \eta \left(\bm Z_t\bm Z_t^{\!\top}-\underline{\bm Z}_t\underline{\bm Z}_t^{\!\top}-\bm \Gamma\right)\bm Z_t+\eta \bm \Xi_t \bm Z_t\\
        & = \left(\bm I_{2d}-\eta\bm Z_t\bm Z_t^{\!\top}\right)\bm Z_t+\eta\underline{\bm Z}_t\underline{\bm Z}_t^{\!\top}\bm Z_t-\eta \bm \Gamma \bm Z_t+\eta \bm \Xi_t \bm Z_t\,,
    \end{align*}
    where $\bm \Gamma$ is defined as
    \begin{align*}
        \bm \Gamma & := \begin{bmatrix}
            \bm 0_{d\times d} & \Delta \\
            \Delta^{\!\top} & \bm 0_{k\times k}
        \end{bmatrix}\,.
    \end{align*}
    Then, by the triangle inequality, with probability $1- 2C\exp(-\epsilon^2 N)$ for a universal constant $C>0$, we have
    \begin{align*}
        \left\|\bm Z_{t+1}\right\|_{op} & \leq \left\|\left(\bm I_{2d}-\eta\bm Z_t\bm Z_t^{\!\top}\right)\bm Z_t\right\|_{op}+\eta\left\|\underline{\bm Z}_t\underline{\bm Z}_t^{\!\top}\bm Z_t\right\|_{op}+\eta\left\|\bm \Gamma \bm Z_t\right\|_{op}+\eta \left\|\bm \Xi_t \bm Z_t\right\|_{op}\\
        & \leq \left(1-\eta \left\|\bm Z_{t}\right\|_{op}^2\right)\left\|\bm Z_{t}\right\|_{op}\quad\tag*{\color{teal}[by simultaneous diagonalization]}\\
        & +\eta\left\|\underline{\bm Z}_t\underline{\bm Z}_t^{\!\top}\bm Z_t\right\|_{op}+\eta\left\|\bm \Gamma \bm Z_t\right\|_{op}+\eta \left\|\bm \Xi_t \bm Z_t\right\|_{op}\\
        & \leq \left(1-\eta \left\|\bm Z_{t}\right\|_{op}^2\right)\left\|\bm Z_{t}\right\|_{op}
        + \eta \left\|\underline{\bm Z}_t^{\!\top}\bm Z_t\right\|_{op}\left\|\bm Z_{t}\right\|_{op} + \eta \lambda_1^* \left\|\bm Z_{t}\right\|_{op}+\eta \epsilon \left\|\bm R_t\right\|_{op} \left\|\bm Z_{t}\right\|_{op}\,,
    \end{align*}
    where the last inequality follows from the fact that
    \begin{align*}
        \left\|\underline{\bm Z}_t\right\|_{op}&=\left\|\bm Z_t\right\|_{op}\,,\\
        \left\|\bm \Gamma\right\|_{op}&=\lambda_1^*\,,\\
        \left\|\bm \Xi_t\right\|_{op}&=\left\|\left(\widetilde{\bm \Sigma}-\bm I_d\right)\bm R_t\right\|_{op}\leq \epsilon \left\|\bm R_t\right\|_{op}\,, \quad \mbox{w.h.p.}~1 - 2C\exp(-\epsilon^2 N)\,.
    \end{align*}
    Using the assumption
    \begin{align*}
        \left\|\underline{\bm Z}_t^{\!\top}\bm Z_t\right\|_{op}+ \epsilon \left\|\bm R_t\right\|_{op} & = \left\|\bm A_t^{\!\top}\bm A_t - \bm B_t^{\!\top}\bm B_t\right\|_{op} + \epsilon \left\|\bm R_t\right\|_{op} \leq \lambda_1^*\,,
    \end{align*}
    then $ \left\|\bm Z_{t+1}\right\|_{op}$ can be further bounded by
    \begin{align}
        \left\|\bm Z_{t+1}\right\|_{op} & \leq \left(1-\eta \left\|\bm Z_{t}\right\|_{op}^2 + 2 \eta \lambda_1^*\right)\left\|\bm Z_{t}\right\|_{op}\label{third-order-eq}\,.
    \end{align}
    Denote $x=\left\|\bm Z_{t}\right\|_{op}$ and $f(x)=\left(1-\eta x^2 + 2 \eta \lambda_1^*\right)x$, we have $f'(x)=1+2\eta\lambda_1^*-3\eta x^2$ and $f''(x)=-6\eta x$. Then, we know $f'(x^*)=0$ for $x>0$ attained at $x^*=\sqrt{\frac{1+2\eta\lambda_1^*}{3\eta}}=\sqrt{\frac{1}{3\eta}+\frac{2}{3}\lambda_1^*}$. As we pick $\eta \leq \frac{1}{10\lambda_1^*}$, then $x^*\geq 2\sqrt{\lambda_1^*}$, which implies the maximum of $f(x)$ attained at $x^*=2\sqrt{\lambda_1^*}$ over $x\in[0\,,2\lambda_1^*]$ since $\left\|\bm Z_{t}\right\|_{op}\leq 2\sqrt{\lambda_1^*}$ and
    \begin{align*}
        f(2\sqrt{\lambda_1^*})=2(1-4\eta\lambda_1^*+2\eta\lambda_1^*)\sqrt{\lambda_1^*}=2\sqrt{\lambda_1^*}-4\eta\lambda_1^*\leq 2\sqrt{\lambda_1^*}\,,
    \end{align*}
   which directly implies $\left\|\bm Z_{t+1}\right\|_{op}\leq 2\sqrt{\lambda_1^*}$. By consequence, $\left\|\bm A_{t+1}\right\|_{op}\,,\left\|\bm B_{t+1}\right\|_{op}\leq 2\sqrt{\lambda_1^*}$ if $\left\|\bm A_{t}\right\|_{op}\,,\left\|\bm B_{t}\right\|_{op}\leq 2\sqrt{\lambda_1^*}$, since $\bm A_{t+1}$ and $\bm B_{t+1}$ are sub-matrices of $\bm Z_{t+1}$.
\end{proof}

Based on the above results, we are ready to present the following intermediate results.
\begin{lemma}
\label{linear-induction-gd}
    Under assumptions in \cref{sec:assumptions} for the linear setting, with spectral initialization \eqref{eq:spectral-init-linear}, we take $\epsilon$ in data concentration as
    \begin{align*}
        \epsilon \leq \min\left\{\frac{1}{2\kappa}\,,\frac{\lambda^*_{r^*}}{32\kappa(32 \lambda_1^*+128 \kappa^2)}\right\}\,,
    \end{align*}
    and set the step-size as
    \begin{align*}
        \eta \leq \min\left\{\frac{1}{128\kappa\lambda_1^*}\,,\frac{(1-\epsilon/\kappa)}{1152\lambda^*_{1}}\right\}\,,
    \end{align*}
    then with probability at least with probability $1- 2C\exp(-\epsilon^2 N)$ for a universal constant $C>0$, we have that for $\forall\,t\geq 0$
    \begin{align}
        & \mathcal{M}_t \leq \frac{\lambda^*_{r^*}}{2} \label{M}\\
        & \max\left\{\left\|\bm A_{t}\right\|_{op}\,,\left\|\bm B_{t}\right\|_{op}\right\}\leq 2\sqrt{\lambda_1^*}\,, \label{upper}\\
        & \lambda_{r^*}^*\left(\bm A_t\right)\,,\lambda_{r^*}^*\left(\bm B_t\right) \geq \frac{\sqrt{\lambda_{r^*}^*}}{4\sqrt{\kappa}} \,,\label{lower}\\
        & \left\|\bm A^{\bm U_\perp}_{t}\right\|_{op} \leq \frac{32 \kappa \epsilon\sqrt{\lambda_1^*}}{\lambda^*_{r^*}}\,. \label{res-A}
    \end{align}
    Also, we can obtain
    \begin{align}
        \mathcal{M}_{t+1} \leq \bigg(1-\eta \frac{\lambda^*_{r^*}}{64\kappa}\bigg)\mathcal{M}_t \label{ML}\,.
    \end{align}
\end{lemma}
\begin{proof}
Inspired by the matrix sensing technique from \cite{xiong2023over}, we develop an inductive approach to prove the claims on our settings.
    At $t=0$, \cref{M}-\cref{res-A} can be adopted from \cref{linear-initial-risk}. We assume \cref{M}-\cref{res-A} hold at $t\geq 1$, recall \cref{AUperp}, we have
    \begin{align*}
        \left\|\bm A^{\bm U_\perp}_{t+1}\right\|_{op} & \leq \left(1-\eta \lambda_{r^*}^2\left(\bm B_t^{\bm V}\right)\right) \left\|\bm A^{\bm U_\perp}_t\right\|_{op} + \eta \epsilon \left\|\bm R_t\right\|_{op} \left\|\bm B_t^{\bm V}\right\|_{op}\\
        & \leq \left(1-\eta \lambda_{r^*}^2\left(\bm B_t^{\bm V}\right)\right) \left\|\bm A^{\bm U_\perp}_t\right\|_{op} + 4 \eta \epsilon \mathcal{M}_t \sqrt{\lambda_1^*}\\
        & \leq \bigg(1-\eta \frac{(\lambda^*_{r^*})^2}{16\kappa}\bigg)\left\|\bm A^{\bm U_\perp}_t\right\|_{op} + 2 \eta \epsilon \lambda^*_{r^*} \sqrt{\lambda_1^*}\\
        & \leq \frac{32 \kappa \epsilon\sqrt{\lambda_1^*}}{\lambda^*_{r^*}}\,, \tag*{\color{teal}$\left[\text{by }\left\|\bm A^{\bm U_\perp}_t\right\|_{op}\leq \frac{32 \kappa \epsilon\sqrt{\lambda_1^*}}{\lambda^*_{r^*}}\right]$}
    \end{align*}
    which proves the \cref{res-A} at $t+1$. Next, by \cref{mathcalMt}, we have
    \begin{align*}
        \left\|\bm R^*_{t+1}\right\|_{op} & \leq \bigg(1-\eta \frac{\lambda^*_{r^*}}{8\kappa}\bigg)\mathcal{M}_t\\
        & + 8 \eta \epsilon \lambda_1^* \mathcal{M}_t + 2\eta^2 \lambda_1^*\lambda_{r^*}^*\mathcal{M}_t + 4\eta^2 \epsilon \lambda_1^*\lambda_{r^*}^*\mathcal{M}_t
         + 64 \eta \epsilon \kappa^2 \mathcal{M}_t
        +32 \eta^2 \kappa^2 \epsilon\lambda^*_{r^*}\mathcal{M}_t+ 128 \eta^2 \epsilon^3 \kappa^2 \lambda_{r^*}^*\mathcal{M}_t\\
        & +64\eta^2 \epsilon^2  \kappa^2 \lambda_{r^*}^* \mathcal{M}_t+8\eta \epsilon \lambda_1^* \mathcal{M}_t
         + 8 \eta^2 \epsilon \lambda_1^* \mathcal{M}_t + 8 \eta^2 \epsilon^2 \lambda_1^* \lambda_{r^*}^* \mathcal{M}_t
         + 128 \eta \epsilon^2 \kappa^2 \mathcal{M}_t+64\eta^2 \epsilon^2  \kappa^2 \lambda_{r^*}^* \mathcal{M}_t\\
        & = \bigg(1-\eta \frac{\lambda^*_{r^*}}{8\kappa}\bigg)\mathcal{M}_t\\
        & \quad + \eta\Bigg\{16\epsilon \lambda_1^*+64\epsilon \kappa^2+2\eta \lambda_1^*\lambda_{r^*}^*+\eta\epsilon\left(4\lambda_1^*\lambda_{r^*}^*+32\kappa^2\lambda_{r^*}^*+8\lambda_1^*\right)
         + 128 \epsilon^2\kappa^2\\
         & \quad +\eta\left(128\eta\epsilon^2\kappa^2\lambda_{r^*}^*+8\eta\epsilon^2\lambda_1^*\lambda_{r^*}^*\right)+128\eta\epsilon^3\kappa^2\lambda_{r^*}^*\Bigg\}\mathcal{M}_t\\
         & \leq \bigg(1-\eta \frac{\lambda^*_{r^*}}{8\kappa}\bigg)\mathcal{M}_t
        + 2\eta\bigg(16\epsilon \lambda_1^*+64\epsilon \kappa^2+2\eta \lambda_1^*\lambda_{r^*}^*
         \bigg)\mathcal{M}_t \quad \tag*{\color{teal}[due to the order dominance]}\\
         & \leq \bigg(1-\eta \frac{\lambda^*_{r^*}}{16\kappa}\bigg)\mathcal{M}_t
        + 2\eta\bigg(16\epsilon \lambda_1^*+64\epsilon \kappa^2
         \bigg)\mathcal{M}_t\quad \tag*{\color{teal}$\left[\text{by }\eta \leq \frac{1}{64\kappa\lambda_1^*}\right]$}\\
         & \leq \bigg(1-\eta \frac{\lambda^*_{r^*}}{32\kappa}\bigg)\mathcal{M}_t\,,\quad \tag*{\color{teal}$\left[\text{by }\epsilon \leq \frac{\lambda^*_{r^*}}{16\kappa(32 \lambda_1^*+128 \kappa^2)}\right]$}
    \end{align*}
   where the order dominance from the second inequality follows from the fact that $\eta$ and $\epsilon$ are sufficiently small constant such that the terms in $\mathcal{O}(\eta \epsilon)\,,\mathcal{O}(\epsilon^2)\,,\mathcal{O}(\eta^2\epsilon^2)\,,\mathcal{O}(\eta \epsilon^3)$ are significantly smaller the terms in $\mathcal{O}(\eta)$ and $\mathcal{O}(\epsilon)$.
    
    Similarly, we can obtain
    \begin{align*}
        \left\|\bm R^\perp_{t+1}\right\|_{op} & \leq \bigg(1-\eta \frac{\lambda^*_{r^*}}{16\kappa}\bigg)\mathcal{M}_t\quad \tag*{\color{teal}$\left[\text{since }\lambda_{\min}\left(\bm A^{\bm U_\perp}_t\right)\geq 0\right]$}\\
        & + \eta \Bigg\{8\epsilon \lambda_1^*+2\eta \lambda_1^* \lambda_{r^*}^* + 4\eta \epsilon \lambda_1^* \lambda_{r^*}^* + 64 \epsilon \kappa^2 + 32 \eta \epsilon \kappa^2\lambda_{r^*}^*+64\eta \epsilon \kappa^2 \lambda_{r^*}^*+128\epsilon^2\kappa^2\\
        & + 8\eta \epsilon \lambda_1^* + 8\eta \epsilon^2 \lambda_1^* \lambda_{r^*}^* + 2048 \epsilon^3\frac{\kappa^3}{\lambda_{r^*}^*}+64\eta \epsilon^2 \kappa^2 + 128 \eta \epsilon^3\kappa^2
        \Bigg\}\mathcal{M}_t\\
        & \leq \bigg(1-\eta \frac{\lambda^*_{r^*}}{16\kappa}\bigg)\mathcal{M}_t
        + 2\eta \Bigg\{8\epsilon \lambda_1^*+2\eta \lambda_1^* \lambda_{r^*}^* + 64 \epsilon \kappa^2\Bigg\}\mathcal{M}_t\quad \tag*{\color{teal}[due to the order dominance]}\\
        & \leq \bigg(1-\eta \frac{\lambda^*_{r^*}}{32\kappa}\bigg)\mathcal{M}_t
        + 2\eta \Bigg\{8\epsilon \lambda_1^*+ 64 \epsilon \kappa^2\Bigg\}\mathcal{M}_t\quad \tag*{\color{teal}$\left[\text{by }\eta \leq \frac{1}{128\kappa\lambda_1^*}\right]$}\\
        & \leq \bigg(1-\eta \frac{\lambda^*_{r^*}}{64\kappa}\bigg)\mathcal{M}_t\,,\quad \tag*{\color{teal}$\left[\text{by }\epsilon \leq \frac{\lambda^*_{r^*}}{32\kappa(32 \lambda_1^*+128 \kappa^2)}\right]$}
    \end{align*}
    which proves the \cref{M} at $t+1$. 
    
    Therefore, we can conclude that
    \begin{align*}
        \mathcal{M}_{t+1} & \leq \bigg(1-\eta \frac{\lambda^*_{r^*}}{64\kappa}\bigg)\mathcal{M}_t\,.
    \end{align*}
    Next, assume \cref{M}-\cref{res-A} hold at $t\geq 1$, we have
    \begin{align*}
        \left(\bm A_{t+1}^{\!\top}\bm A_{t+1} - \bm B_{t+1}\bm B_{t+1}^{\!\top}\right)-\left(\bm A_{t}^{\!\top}\bm A_{t} - \bm B_{t}\bm B_{t}^{\!\top}\right) & = \eta^2 \bm B_t\left(\bm A_t \bm B_t - \Delta\right)^{\!\top}\widehat{\bm \Sigma}\widehat{\bm \Sigma}\left(\bm A_t \bm B_t - \Delta\right)\bm B_t^{\!\top}\\
        & + \eta^2 \bm A_t^{\!\top}\widehat{\bm \Sigma}\left(\bm A_t \bm B_t - \Delta\right)\left(\bm A_t \bm B_t - \Delta\right)^{\!\top}\widehat{\bm \Sigma}\bm A_t\,.
    \end{align*}
    Accordingly, we can derive
    \begin{align*}
        &\left\|\left(\bm A_{t+1}^{\!\top}\bm A_{t+1} - \bm B_{t+1}\bm B_{t+1}^{\!\top}\right)-\left(\bm A_{0}^{\!\top}\bm A_{0} - \bm B_{0}\bm B_{0}^{\!\top}\right)\right\|_{op}\\
        =&\sum_{i=1}^{t+1}\left\|\left(\bm A_{i}^{\!\top}\bm A_{i} - \bm B_{i}\bm B_{i}^{\!\top}\right)-\left(\bm A_{i-1}^{\!\top}\bm A_{i-1} - \bm B_{i-1}\bm B_{i-1}^{\!\top}\right)\right\|_{op}\\
        =&\sum_{i=1}^{t+1}2\eta^2 \|\widehat{\bm \Sigma}\|_{op}^2 \|\bm R_{i-1}\|_{op}^2 \max\left\{\|\bm A_{i-1}\|_{op}^2\,, \|\bm B_{i-1}\|_{op}^2\right\}\\
        =&\sum_{i=1}^{t+1}72\eta^2 \mathcal{M}_{i-1}^2 \lambda_1^*\quad \tag*{\color{teal}[by \cref{concentration-N}]}\\
        \leq &\sum_{i=1}^{t+1}18\eta^2 \bigg(1-\eta \frac{\lambda^*_{r^*}}{64\kappa}\bigg)^{2(i-1)}(\lambda_{r^*}^*)^2 \lambda_1^*\\
        \leq& 18\eta^2(\lambda_{r^*}^*)^2 \lambda_1^*\sum_{i=0}^{\infty}\bigg(1-\eta \frac{\lambda^*_{r^*}}{64\kappa}\bigg)^{2i}\\
        \leq& 18\eta^2(\lambda_{r^*}^*)^2 \lambda_1^* \frac{64\kappa}{\eta\lambda^*_{r^*}}\\
        =&1152\eta\lambda_1^*\lambda^*_{r^*}\kappa\\
        \leq& (1-\epsilon/\kappa)\lambda_1^*\,. \quad \tag*{\color{teal}$\left[\text{by }\eta\leq\frac{(1-\epsilon/\kappa)}{1152\lambda^*_{1}}\right]$}
    \end{align*}
    Since $\left\|\left(\bm A_{0}^{\!\top}\bm A_{0} - \bm B_{0}\bm B_{0}^{\!\top}\right)\right\|_{op}=0$ due to the spectral initialization \eqref{eq:spectral-init-linear}, by triangle inequality, $\left\|\left(\bm A_{t+1}^{\!\top}\bm A_{t+1} - \bm B_{t+1}\bm B_{t+1}^{\!\top}\right)\right\|_{op}\leq (1-\epsilon/\kappa)\lambda_1^*$. Next, by \cref{mahdi-upper}, we can obtain
    \begin{align*}
        \left\|\bm A_{t+1}\right\|_{op}\leq 2\sqrt{\lambda_1^*}\,,\quad\left\|\bm B_{t+1}\right\|_{op}\leq 2\sqrt{\lambda_1^*}\,,
    \end{align*}
    which proves the \cref{upper} at $t+1$. Lastly, assume \cref{M}-\cref{res-A} hold at $t\geq 1$, by Weyl's inequality, combine with $\mathcal{M}_{t+1}\leq \frac{\lambda_{r^*}^*}{2}$, we have
    \begin{align*}
        \frac{\lambda_{r^*}^*}{2} \geq \left\|\bm A_{t+1}^{\bm U} \bm B_{t+1}^{\bm V} - \bm S^*\right\|_{op} \geq \lambda_{r^*}^* - \lambda_{r^*}(\bm A_{t+1}^{\bm U} \bm B_{t+1}^{\bm V})\Rightarrow \lambda_{r^*}(\bm A_{t+1}^{\bm U} \bm B_{t+1}^{\bm V})\geq \frac{\lambda_{r^*}^*}{2}\,.
    \end{align*}
    Again by Weyl's inequality and the \cref{upper} at time $t+1$ we can get
    \begin{align*}
        2 \sqrt{\lambda_1^*}\cdot\lambda_{r^*}(\bm B_{t+1}^{\bm V})\geq \lambda_1(\bm A_{t+1}^{\bm U})\lambda_{r^*}(\bm B_{t+1}^{\bm V})\geq\lambda_{r^*}(\bm A_{t+1}^{\bm U} \bm B_{t+1}^{\bm V})\geq \frac{\lambda_{r^*}^*}{2}\Rightarrow \lambda_{r^*}(\bm B_{t+1}^{\bm V}) \geq \frac{\sqrt{\lambda_{r^*}^*}}{4 \sqrt{\kappa}}\,.
    \end{align*}
    Besides, $\lambda_{r^*}^*(\bm A_{t+1}^{\bm U})$ follows similar derivation. We prove all the claims.
\end{proof}
\begin{theorem}
\label{risk-conv-linear-vanilla-gd}
    Under assumptions in \cref{sec:assumptions} for the linear setting, with spectral initialization \eqref{eq:spectral-init-linear}, we take $\epsilon$ in data concentration as
    \begin{align*}
        \epsilon \leq \min\left\{\frac{1}{2\kappa}\,,\frac{\lambda^*_{r^*}}{32\kappa(32 \lambda_1^*+128 \kappa^2)}\right\}\,,
    \end{align*}
    and set the step-size as
    \begin{align}\label{lr-linear-gd}
        \eta \leq \min\left\{\frac{1}{128\kappa\lambda_1^*}\,,\frac{(1-\epsilon/\kappa)}{1152\lambda^*_{1}}\right\}\,,
    \end{align}
    then with probability at least with probability $1- 2C\exp(-\epsilon^2 N)$ for a universal constant $C>0$, we have that for $\forall\,t\geq 0$
    \begin{align*}
    \left\|\bm A_t \bm B_t - \Delta\right\|_F
    & \leq \sqrt{2 r^*} \left(1 - \eta \frac{\lambda_{r^*}^*}{64 \kappa}\right)^{t}\cdot\lambda_{r^*}^*\,.
\end{align*}
\end{theorem}
\begin{proof}
By \cref{linear-induction-gd}, with probability at least with probability $1- 2C\exp(-\epsilon^2 N)$ for a universal constant $C>0$, we can obtain the linear convergence of generalization risk
\begin{align*}
    \left\|\bm A_t \bm B_t - \Delta\right\|_F & \leq \sqrt{2 r^*} \left\|\bm A_t \bm B_t - \Delta\right\|_{op}\tag*{\color{teal}$\left[\operatorname{Rank}(\bm A_t \bm B_t)=r^*\text{ by \cref{linear-invariant-B2} and }\operatorname{Rank}(\Delta)=r^*\right]$}\\
    & \leq \sqrt{2 r^*} \left(1 - \eta \frac{\lambda_{r^*}^*}{64 \kappa}\right)^{t}\cdot\lambda_{r^*}^*\,,
\end{align*}
which is independent of the choice of LoRA rank $r$ if $r\geq r^*$.
\end{proof}
\subsection{Preconditioned Gradient Descent under Spectral Initialization}
\label{app:precgdlr}
Here we present the proof for preconditioned gradient descent.
\begin{equation}\tag{Prec-GD}\label{alg:prec-gd}
\begin{aligned}
    \bm A_{t+1} & = \bm A_t - \frac{\eta}{N}\widetilde{\bm X}^{\!\top}\left(\widetilde{\bm X}\left(\bm W^\natural+\bm A_t \bm B_t\right)-\widetilde{\bm Y}\right)\bm B_t^{\!\top}\left(\bm B_t \bm B_t^{\!\top}\right)^\dagger\,,\\
    \bm B_{t+1} & = \bm B_t - \frac{\eta}{N}\left(\bm A_t^{\!\top} \bm A_t\right)^\dagger\bm A_t^{\!\top}\widetilde{\bm X}^{\!\top}\left(\widetilde{\bm X}\left(\bm W^\natural+\bm A_t \bm B_t\right)-\widetilde{\bm Y}\right)\,.
\end{aligned}
\end{equation}

In the following proofs, we will prove that the LoRA fine-tuning can achieve faster linear convergence which is independent of condition number $\kappa$ under \eqref{eq:spectral-init-linear} and \eqref{alg:prec-gd}. Similar to \cref{linear-invariant-B2}, the dynamics of $\bm B_t$ are still limited to the $r^*$-dimensional singular subspace $\bm V$ of $\Delta$ under \eqref{eq:spectral-init-linear}. We can verify this fact by the following lemma.

\begin{lemma}
\label{BV-perp}
    For any natural number $t \geq 0$, under assumptions in \cref{sec:assumptions} for the linear setting, with \eqref{eq:spectral-init-linear} and \eqref{alg:prec-gd}, we have
    \begin{align*}
        \bm B_t \bm V_\perp & = \bm 0_{r\times(k-r^*)}\,.
    \end{align*}
\end{lemma}
\begin{proof}
    For $t=0$, recall the SVD of $\mathbf{G}^\natural$, i.e. $\widetilde{\bm U}_{\bm G^\natural}\widetilde{\bm S}_{\bm G^\natural}\widetilde{\bm V}_{\bm G^\natural}^{\!\top}$ in \cref{NGG}, we have
    \begin{align*}
        \bm B_0 \bm V_\perp & = \left[\widetilde{\bm S}_{\bm G^\natural}^{-1/2}\right]_{[1:r]}\left[\widetilde{\bm U}_{\bm G^\natural}^{\!\top}\right]_{[:,1:r]}\mathbf{G}^{\natural} \bm V_\perp = \left[\widetilde{\bm S}_{\bm G^\natural}^{-1/2}\right]_{[1:r]}\left[\widetilde{\bm U}_{\bm G^\natural}^{\!\top}\right]_{[:,1:r]} \widehat{\bm \Sigma} \Delta \bm V_\perp = \bm 0_{r\times(k-r^*)}\,.
    \end{align*}
    Assume $\bm B_t \bm V_\perp= \bm 0_{d\times (d-r^*)}$ holds for any natural number $t\geq 1$, then
    \begin{align*}
        \bm B_{t+1} \bm V_\perp & = \bm B_t \bm V_\perp - \frac{\eta}{N}\left(\bm A_t^{\!\top} \bm A_t\right)^\dagger\bm A_t^{\!\top}\widetilde{\bm X}^{\!\top}\left(\widetilde{\bm X}\left(\bm W^\natural+\bm A_t \bm B_t\right)-\widetilde{\bm Y}\right)\bm V_\perp \\
        & = \bm B_t \bm V_\perp - \eta\left(\bm A_t^{\!\top} \bm A_t\right)^\dagger\bm A_t^{\!\top}\widehat{\bm \Sigma}\left(\bm A_t \bm B_t-\Delta\right)\bm V_\perp\\
        & = \bm 0_{r\times(k-r^*)}\,, \quad \tag*{\color{teal}[by our inductive hypothesis]}
    \end{align*}
    which proves the claim.
\end{proof}
We can re-formulate \eqref{alg:prec-gd} to be
\begin{align}
    \bm A_{t+1} & = \bm A_t - \eta\widehat{\bm \Sigma}\left(\bm A_t \bm B_t-\Delta\right)\left(\bm B_t\right)^{\!\top}\left(\bm B_t \bm B_t^{\!\top}\right)^\dagger\,,\label{reparam-linear-scaled-gd-A}\\
    \bm B_{t+1} & = \bm B_t - \eta\left(\bm A_t^{\!\top} \bm A_t\right)^\dagger\bm A_t^{\!\top}\widehat{\bm \Sigma}\left(\bm A_t \bm B_t-\Delta\right)\label{reparam-linear-scaled-gd-B}\,.
\end{align}
Before we start our main proofs, we first define the following notations
\begin{itemize}
    \item SVD of product matrix $\bm A_t \bm B_t := \mathcal{U}_t \mathcal{S}_t \mathcal{V}_t^{\!\top}$, where $\mathcal{U}_t \in \mathbb{R}^{d\times r^*}$, $\mathcal{S}_t \in \mathbb{R}^{r^*\times r^*}$, and $\mathcal{V}_t\in \mathbb{R}^{k\times r^*}$. Notice that here we employ rank-$r^*$ SVD of $\bm A_t \bm B_t$ since $\operatorname{Rank}\left(\bm A_t\bm B_t\right)\leq r^*$ due to \cref{BV-perp} and $\lambda_{r^*}\left(\bm A_t \bm B_r\right)>0$ strictly which we will obtain from \cref{prec-gd-linear-conv}.
    \item The left compact singular matrix of $\bm A_t$ as $\bm U_{\bm A_t} \in \mathbb{R}^{d\times r}$.
    \item The right compact singular matrix of $\bm B_t$ as $\bm V_{\bm B_t} \in \mathbb{R}^{k\times r^*}$. Notice that here we take the top-$r^*$ right singular subspace of $\bm B_t$ due to \cref{BV-perp}.
\end{itemize}
By the pseudo inverse theorem and \citet[Lemma 14]{jia2024preconditioning}, we can obtain
\begin{align}
    &\bm A_t \left(\bm A_t^{\!\top} \bm A_t\right)^\dagger\bm A_t^{\!\top} = \bm U_{\bm A_t} \bm U_{\bm A_t}^{\!\top}\,,\label{proj-A}\\ 
    &\left(\bm B_t\right)^{\!\top}\left(\bm B_t \bm B_t^{\!\top}\right)^\dagger\bm B_t = \bm V_{\bm B_t}\bm V_{\bm B_t}^{\!\top}\,.\label{proj-B}\\
    &\left(\bm B_t\right)^{\!\top}\left(\bm B_t \bm B_t^{\!\top}\right)^\dagger\left(\bm A_t^{\!\top} \bm A_t\right)^\dagger\bm A_t^{\!\top} = \mathcal{V}_t \mathcal{S}^{-1}_t \mathcal{U}_t^{\!\top}\label{pseudo-inverse-AB}\,.
\end{align}

\begin{lemma}
\label{joint-scaled-gd-linear-dynamics}
    Denote $\bm R_t := \bm A_{t}\bm B_{t} - \Delta$, $\bm \Xi:=\widehat{\bm \Sigma}-\bm I_d$, under assumptions in \cref{sec:assumptions} for the linear setting, with \eqref{alg:prec-gd}, then we have
    \begin{align*}
        \bm R_{t+1} & = \bm R_t - \eta \bm U_{\bm A_t} \bm U_{\bm A_t}^{\!\top}\bm R_t - \eta \bm R_t\bm V_{\bm B_t} \bm V_{\bm B_t}^{\!\top} - \eta \bm U_{\bm A_t} \bm U_{\bm A_t}^{\!\top}\bm \Xi\bm R_t - \eta \bm \Xi\bm R_t\bm V_{\bm B_t} \bm V_{\bm B_t}^{\!\top}+ \eta^2 \widehat{\bm \Sigma}\bm R_t\mathcal{V}_t \mathcal{S}^{-1}_t \mathcal{U}_t^{\!\top}\widehat{\bm \Sigma}\bm R_t\,.
    \end{align*}
\end{lemma}
\begin{proof}
    With \cref{reparam-linear-scaled-gd-A} and \cref{reparam-linear-scaled-gd-B}, we can construct
    \begin{align*}
    \bm R_{t+1} &=\bm A_{t+1}\bm B_{t+1} - \Delta\\
    & = \bm A_{t}\bm B_{t} - \Delta\\
    & \quad - \eta \bm A_t\left(\bm A_t^{\!\top} \bm A_t\right)^\dagger\bm A_t^{\!\top}\widehat{\bm \Sigma}\left(\bm A_t \bm B_t-\Delta\right)\\
    & \quad - \eta \widehat{\bm \Sigma}\left(\bm A_t \bm B_t-\Delta\right)\left(\bm B_t\right)^{\!\top}\left(\bm B_t \bm B_t^{\!\top}\right)^\dagger\bm B_t\\
    & \quad + \eta^2 \widehat{\bm \Sigma}\left(\bm A_t \bm B_t-\Delta\right)\left(\bm B_t\right)^{\!\top}\left(\bm B_t \bm B_t^{\!\top}\right)^\dagger\left(\bm A_t^{\!\top} \bm A_t\right)^\dagger\bm A_t^{\!\top}\widehat{\bm \Sigma}\left(\bm A_t \bm B_t-\Delta\right)\\
    & = \bm R_t - \eta \bm U_{\bm A_t} \bm U_{\bm A_t}^{\!\top}\widehat{\bm \Sigma}\bm R_t - \eta \widehat{\bm \Sigma}\bm R_t\bm V_{\bm B_t} \bm V_{\bm B_t}^{\!\top}\quad \tag*{\color{teal}[by \cref{proj-A} and \cref{proj-B}]}\\
    & \quad + \eta^2 \widehat{\bm \Sigma}\bm R_t\mathcal{V}_t \mathcal{S}^{-1}_t \mathcal{U}_t^{\!\top}\widehat{\bm \Sigma}\bm R_t\quad \tag*{\color{teal}[by \cref{pseudo-inverse-AB}]}\\
    & = \bm R_t - \eta \bm U_{\bm A_t} \bm U_{\bm A_t}^{\!\top}\bm R_t - \eta \bm R_t\bm V_{\bm B_t} \bm V_{\bm B_t}^{\!\top} - \eta \bm U_{\bm A_t} \bm U_{\bm A_t}^{\!\top}\bm \Xi\bm R_t - \eta \bm \Xi\bm R_t\bm V_{\bm B_t} \bm V_{\bm B_t}^{\!\top}+ \eta^2 \widehat{\bm \Sigma}\bm R_t\mathcal{V}_t \mathcal{S}^{-1}_t \mathcal{U}_t^{\!\top}\widehat{\bm \Sigma}\bm R_t\,,
\end{align*}
which proves the claim.
\end{proof}

In the next, we aim to estimate the signal part $\bm R_t - \eta \bm U_{\bm A_t} \bm U_{\bm A_t}^{\!\top}\bm R_t - \eta \bm R_t\bm V_{\bm B_t} \bm V_{\bm B_t}^{\!\top}$.
\begin{lemma}
\label{strict-equal-F-norm}
    Recall $\bm R_t := \bm A_{t}\bm B_{t} - \Delta$, under assumptions in \cref{sec:assumptions} for the linear setting, with \eqref{alg:prec-gd}, then
    \begin{align*}
        \left\|\bm R_t - \eta \bm U_{\bm A_t} \bm U_{\bm A_t}^{\!\top}\bm R_t - \eta \bm R_t\bm V_{\bm B_t} \bm V_{\bm B_t}^{\!\top}\right\|_{\rm F} & \leq (1-\eta) \left\|\bm R_t\right\|_{\rm F}\,.
    \end{align*}
\end{lemma}
\begin{proof}
    \begin{align*}
        &\left\|\bm R_t - \eta \bm U_{\bm A_t} \bm U_{\bm A_t}^{\!\top}\bm R_t - \eta \bm R_t\bm V_{\bm B_t} \bm V_{\bm B_t}^{\!\top}\right\|_{\rm F}\\
        = & \left\|\bm R_t \left(\bm V_{\bm B_t} \bm V_{\bm B_t}^{\!\top}+\bm I_k - \bm V_{\bm B_t} \bm V_{\bm B_t}^{\!\top}\right) - \eta \bm U_{\bm A_t} \bm U_{\bm A_t}^{\!\top}\bm R_t \left(\bm V_{\bm B_t} \bm V_{\bm B_t}^{\!\top}+\bm I_k - \bm V_{\bm B_t} \bm V_{\bm B_t}^{\!\top}\right) - \eta \bm R_t\bm V_{\bm B_t} \bm V_{\bm B_t}^{\!\top}\right\|_{\rm F}\\
        = & \left\|\bm R_t \bm V_{\bm B_t} \bm V_{\bm B_t}^{\!\top} - \eta \bm U_{\bm A_t} \bm U_{\bm A_t}^{\!\top}\bm R_t \bm V_{\bm B_t} \bm V_{\bm B_t}^{\!\top} - \eta \bm R_t\bm V_{\bm B_t} \bm V_{\bm B_t}^{\!\top}\right\|_{\rm F} \quad \tag*{\color{teal}$\left[\text{since }\bm R_t\left(\bm I_k - \bm V_{\bm B_t} \bm V_{\bm B_t}^{\!\top}\right)=\bm 0\text{ by \cref{BV-perp}}\right]$}\\
        = & \left\|\left(\bm I_d - \eta \left(\bm I_d + \bm U_{\bm A_t} \bm U_{\bm A_t}^{\!\top}\right)\right)\bm R_t \bm V_{\bm B_t} \bm V_{\bm B_t}^{\!\top}\right\|_{\rm F}\\
        = & \left\|\bm I_d - \eta \left(\bm I_d + \bm U_{\bm A_t} \bm U_{\bm A_t}^{\!\top}\right)\right\|_{op}\left\|\bm R_t \bm V_{\bm B_t} \bm V_{\bm B_t}^{\!\top}\right\|_{\rm F}\\
        \leq & (1-\eta) \left\|\bm R_t\right\|_{\rm F}\,,\quad \tag*{\color{teal}$\left[\left\|\bm I_d - \eta \left(\bm I_d + \bm U_{\bm A_t} \bm U_{\bm A_t}^{\!\top}\right)\right\|_{op}\leq 1-\eta, \text{ since }\bm U_{\bm A_t} \bm U_{\bm A_t}^{\!\top}\text{ is a rank-}r\text{ projection matrix}\right]$}
    \end{align*}
    which concludes the proof.
\end{proof}

\begin{theorem}
    \label{prec-gd-linear-conv}
    Under assumptions in \cref{sec:assumptions} for the linear setting, with \eqref{eq:spectral-init-linear} and \eqref{alg:prec-gd}, we choose
    \begin{align*}
        \epsilon\leq\min\left\{\frac{1}{2\sqrt{r^*}\kappa}\,,\frac{1}{4}\right\}
    \end{align*}
    and set $ \eta \in \left(0, \frac{0.5-2\epsilon}{(1+\epsilon)^2}\right)$, then with probability at least $1- 2C\exp(-\epsilon^2 N)$ for a universal constant $C>0$, we have
    \begin{align*}
        \left\|\bm A_t \bm B_t - \Delta\right\|_{\rm F} & \leq \frac{1}{2}\left(1-\frac{\eta}{2}\right)^t\lambda_{r^*}^*\,.
    \end{align*}
\end{theorem}
\begin{proof}
We prove it by induction.
We suppose the following two inductive hypothesis
\begin{align}
    \lambda_{r^*}\left(\bm A_t \bm B_t\right) & \geq \frac{\lambda^*_{r^*}}{2}\,,\label{inductive-joint-rth-singular-value-lower}\\
    \left\|\bm A_0 \bm B_0 - \Delta\right\|_{\rm F} & \leq \frac{\lambda^*_{r^*}}{2}\,.\label{inductive-loss-hypothesis}
\end{align}
Starting from $t=0$, under \eqref{eq:spectral-init-linear}, with probability at least $1- 2C\exp(- \epsilon^2 N)$ for a universal constant $C>0$, we have
\begin{align*}
    \left\|\bm A_0 \bm B_0 - \Delta\right\|_{\rm F} & = \left\|\bm G^\natural - \Delta\right\|_{\rm F}\\
    & = \left\|\left(\widehat{\bm \Sigma}-\bm I_d\right)\Delta\right\|_{\rm F} \quad \tag*{\color{teal}[by \cref{NGG}]}\\
    & \leq \epsilon \|\Delta\|_{\rm F}\\
    & \leq \epsilon \sqrt{r^*} \|\Delta\|_{op}\tag*{\color{teal}$\left[\text{since }\operatorname{Rank}\left(\Delta\right)=r^*\right]$}\\
    & \leq \frac{\lambda^*_{r^*}}{2}\,. \tag*{\color{teal}$\left[\text{since }\epsilon\leq1/2\sqrt{r^*}\kappa\right]$}
\end{align*}
Then, by Weyl's inequality, we have
\begin{align*}
    \lambda_{r^*}\left(\Delta\right)-\lambda_{r^*}\left(\bm A_0 \bm B_0\right) & \leq \left\|\bm A_0 \bm B_0 - \Delta\right\|_{op} \leq \left\|\bm A_0 \bm B_0 - \Delta\right\|_{\rm F}\,,
\end{align*}
which implies
\begin{align}
\label{joint-rth-singular-value-lower}
    \lambda_{r^*}\left(\bm A_0 \bm B_0\right) & \geq \frac{\lambda^*_{r^*}}{2}\,.
\end{align}
Therefore, we verify \cref{inductive-joint-rth-singular-value-lower} and \cref{inductive-loss-hypothesis} at $t=0$. We assume \cref{inductive-joint-rth-singular-value-lower} and \cref{inductive-loss-hypothesis} hold at $t=2,3,...$, then by \cref{joint-scaled-gd-linear-dynamics}, with probability at least with probability $1- 2C\exp(-\epsilon^2 N)$ for a universal constant $C>0$, we have
    \begin{align*}
        \left\|\bm R_{t+1}\right\|_{\rm F} & \leq \left\|\bm R_t - \eta \bm U_{\bm A_t} \bm U_{\bm A_t}^{\!\top}\bm R_t - \eta \bm R_t\bm V_{\bm B_t} \bm V_{\bm B_t}^{\!\top}\right\|_{\rm F}\\
        & \quad + \eta \left\|\bm U_{\bm A_t} \bm U_{\bm A_t}^{\!\top}\bm \Xi\bm R_t\right\|_{\rm F} + \eta \left\|\bm \Xi\bm R_t\bm V_{\bm B_t} \bm V_{\bm B_t}^{\!\top}\right\|_{\rm F}
        + \eta^2 \left\|\widehat{\bm \Sigma}\bm R_t\mathcal{V}_t \mathcal{S}^{-1}_t \mathcal{U}_t^{\!\top}\widehat{\bm \Sigma}\bm R_t\right\|_{\rm F}\\
        & \leq (1-\eta)\left\|\bm R_t\right\|_{\rm F}\quad \tag*{\color{teal}[by \cref{strict-equal-F-norm}]}\\
        & \quad + \eta \epsilon \left\|\bm U_{\bm A_t} \bm U_{\bm A_t}^{\!\top}\bm R_t\right\|_{\rm F} + \eta \epsilon \left\|\bm R_t\bm V_{\bm B_t} \bm V_{\bm B_t}^{\!\top}\right\|_{\rm F}
        + \eta^2 (1+\epsilon)^2 \frac{\left\|\bm R_t\right\|^2_{\rm F}}{\lambda_{r^*}\left(\bm A_t \bm B_t\right)} \quad \tag*{\color{teal}$\left[\text{by }\|\bm \Xi\|_{op}\leq \epsilon\right]$}\\
        & \leq (1-\eta)\left\|\bm R_t\right\|_{\rm F}\\
        & \quad + \eta \epsilon \left\|\bm R_t\right\|_{\rm F} + \eta \epsilon \left\|\bm R_t\right\|_{\rm F}
        + \eta^2 (1+\epsilon)^2 \left\|\bm R_t\right\|_{\rm F}\quad \tag*{\color{teal}$\left[\text{since \cref{inductive-joint-rth-singular-value-lower} and \cref{inductive-loss-hypothesis} hold at }t\right]$}\\
        & = \left(1-(1-2\epsilon)\eta +\eta^2(1+\epsilon)^2\right)\left\|\bm R_t\right\|_{\rm F}\\
        & \leq \left(1-\frac{\eta}{2}\right)\left\|\bm R_t\right\|_{\rm F}\,.\quad \tag*{\color{teal}$\left[\text{taking}~\eta \leq \frac{0.5-2\epsilon}{(1+\epsilon)^2} \right]$}
    \end{align*}
    This implies \cref{inductive-loss-hypothesis} at time $t+1$. By consequence, we can obtain \cref{inductive-joint-rth-singular-value-lower} at time $t+1$ again by Weyl's inequality.
\end{proof}