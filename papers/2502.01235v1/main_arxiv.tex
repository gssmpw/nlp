\documentclass[11pt]{article}
\pdfoutput=1
\usepackage{microtype}
\usepackage{graphicx,wrapfig,lipsum}
\usepackage{subfigure}
\usepackage{booktabs} % for professional tables
\usepackage[shortlabels]{enumitem}
\usepackage{siunitx}  
\usepackage{hyperref}
\hypersetup{colorlinks, linkcolor=red, anchorcolor=blue, citecolor=blue}
\newcommand{\theHalgorithm}{\arabic{algorithm}}
\renewcommand{\baselinestretch}{1.05}
\usepackage{mylatexstyle}
\usepackage{setspace}
\usepackage[left=1in, right=1in, top=1in, bottom=1in]{geometry}
\usepackage{amsmath}
\usepackage{amssymb,bm}
\usepackage{mathrsfs}
\usepackage{graphicx}
\usepackage{mathtools}
\usepackage{caption}
\usepackage{subcaption,color}
\usepackage{indentfirst,threeparttable,multirow}
\usepackage{amsfonts,tcolorbox}
\usepackage{float,url}
\usepackage{bbm,cancel}
\usepackage{lipsum}
\usepackage{nicefrac} 


\newcommand{\disableaddcontentsline}{%
  \let\savedaddcontentsline\addcontentsline 
  \renewcommand{\addcontentsline}[3]{}
}
\newcommand{\enableaddcontentsline}{%
  \let\addcontentsline\savedaddcontentsline
}

\usepackage[textsize=tiny]{todonotes}
\definecolor{myblue}{RGB}{0 114 199}
\definecolor{mylightblue}{RGB}{77 191 241}
\definecolor{darkgray}{HTML}{878787}
\definecolor{myorange}{RGB}{217 83 25}
\newtcolorbox{myorangebox}{colframe = myorange}
\newtcolorbox{mygraybox}{colframe = gray}
\newtcolorbox{mybluebox}{colframe = myblue}

\usepackage{xcolor}
\definecolor{fhcolor}{rgb}{0.523, 0.235, 0.625}
\newcommand{\fh}[1]{\textcolor{fhcolor}{(SgrA: #1)}}
\newcommand{\yccomment}[1]{\textcolor{red}{(Yudong: #1)}}
\let\yc\yccomment
\let\yudong\yccomment
\newcommand{\yh}[1]{\textcolor{blue}{(Yuanhe: #1)}}

\title{\huge One-step full gradient suffices for low-rank fine-tuning, provably and efficiently}

\author
{
     Yuanhe Zhang\thanks{Department of Statistics, University of Warwick, United Kingdom; Email: {\tt yuanhe.zhang@warwick.ac.uk}} 
     ~~~
     Fanghui Liu\thanks{Department of Computer Science, also Centre for Discrete Mathematics and its Applications (DIMAP), University of Warwick, United Kingdom; Email: {\tt fanghui.liu@warwick.ac.uk} (Corresponding author)} 
     ~~~
     Yudong Chen\thanks{Department of Computer Sciences, University of Wisconsin-Madison, USA; e-mail: {\tt yudong.chen@wisc.edu}}
}


\begin{document}
\disableaddcontentsline
\maketitle

\begin{abstract}
This paper studies how to improve the performance of Low-Rank Adaption (LoRA) \citep{hu2022lora} as guided by our theoretical analysis. 
Our first set of theoretical results show that for random initialization and linear models, \textit{i)} LoRA will align to the certain singular subspace of one-step gradient of full fine-tuning; \textit{ii)} preconditioners improve convergence in the high-rank case.
These insights motivate us to focus on preconditioned LoRA using a specific spectral initialization strategy for aligning with certain subspaces.
For both linear and nonlinear models, we prove that alignment and generalization guarantees can be directly achieved at initialization, and the subsequent linear convergence can be also built.
Our analysis leads to the \emph{LoRA-One} algorithm (using \emph{One}-step gradient and preconditioning), a theoretically grounded algorithm that achieves significant empirical improvement over vanilla LoRA and its variants on several benchmarks.
Our theoretical analysis, based on decoupling the learning dynamics and characterizing how spectral initialization contributes to feature learning, may be of independent interest for understanding matrix sensing and deep learning theory.
The source code can be found in the \url{https://github.com/YuanheZ/LoRA-One}.

\end{abstract}

\section{Introduction}
\label{sec:intro}

\begin{figure*}[tb]
    \centering
    \includegraphics[width=0.848\linewidth]{figs/circuitnn.pdf} 
    \caption{Illustration of differentiable CircuitNN. CircuitNN is designed based on differentiable NAND gates. After DAS is guided by PI and PO pairs of the truth table, CircuitNN can get the precise circuit architecture logic equivalent to the truth table.}
    \label{fig:circuitnn}
\end{figure*}

% 1. Describe the importance of logic synthesis
% 2. Existing Problems
% (a) Neural Architecture Search: Unstable, Predefined Setting, etc.
% (b) Circuit Generation: Probabilistic Model, Logic Equivalence

With the rapid advancement of technology, the scale of integrated circuits (ICs) has expanded exponentially. 
This expansion has introduced significant challenges in chip manufacturing, particularly concerning power and area metrics.
A primary objective in IC design is achieving the same circuit function with fewer transistors, thereby reducing power usage and area occupancy.

Logic synthesis~\cite{hachtel2005logicsynth}, a critical step in electronic design automation (EDA), transforms behavioral-level circuit designs into optimized gate-level circuits, ultimately yielding the final IC layout. 
The primary goal of logic synthesis is to identify the physical implementation with the fewest gates for a given circuit function. 
This task constitutes a challenging NP-hard combinatorial optimization problem. 
Current logic synthesis tools~\cite{brayton2010abc, wolf2013yosys} rely on human-designed heuristics, often leading to sub-optimal outcomes.

Differentiable architecture search (DAS) techniques~\cite{liu2018darts, chu2020darts} offer novel perspectives on addressing challenges in this problem.
Circuit functions can be represented through truth tables, which map binary inputs to their corresponding outputs. 
Truth tables provide a precise representation of input-output relationships, ensuring the design of functionally equivalent circuits.
Inspired by this, researchers~\cite{deepmind2024ai4sys, wang2024tnet} have begun exploring the application of DAS to synthesize circuits directly from truth tables.
Specifically, \citet{deepmind2024ai4sys} proposed CircuitNN, a framework that learns differentiable connection structures with logic gates, enabling the automatic generation of logic circuits from truth tables.
This approach significantly reduces the complexity of traditional circuit generation. 
Building on this, \citet{wang2024tnet} introduced T-Net, a triangle-shaped variant of CircuitNN, incorporating regularization techniques to enhance the efficiency of DAS.

Despite these advancements, several challenges remain. 
The computational complexity of DAS grows quadratically with the number of gates, posing scalability issues.
Although triangle-shaped architecture~\cite{wang2024tnet} partially mitigates this problem, redundancy persists. 
%Additionally, DAS is susceptible to converging to local optima, limiting the ability to search architectures that satisfy the given truth tables~\cite{liu2018darts}. 
%Furthermore, hyperparameters (network depth and layer width) require extensive searches, introducing complexity and prolonging the synthesis process. 
Additionally, DAS is susceptible to converging to local optima~\cite{liu2018darts} and hyperparameters (network depth and layer width) require extensive searches. 
The challenges arise from the vast search space in DAS. 
% Even with predefined settings for CircuitNN, finding a configuration that meets the truth table requires extensive trial and error during the DAS process. 
Intuitively, limiting the search space through predefined parameters (network depth, gates per layer, and connection probabilities) can significantly reduce the complexity.

Recent advances~\cite{openai2023gpt4, abramson2024alphafold3, esser2024sd3, li2024mar} in conditional generative models have demonstrated remarkable performance across language, vision, and graph generation tasks. 
Motivated by these developments, we propose a novel approach to circuit generation that generates preliminary circuit structures to guide DAS in generating refined circuits matching specified truth tables. 
Firstly, we introduce CircuitVQ, a tokenizer with a discrete codebook for circuit tokenization. 
Built upon our Circuit AutoEncoder framework~\cite{hou2022graphmae,li2023maskgae,wu2025mgvga}, CircuitVQ is trained through a circuit reconstruction task. 
Specifically, the CircuitVQ encoder encodes input circuits into discrete tokens using a learnable codebook, while the decoder reconstructs the circuit adjacency matrix based on these tokens.
Subsequently, the CircuitVQ encoder serves as a circuit tokenizer for CircuitAR pretraining, which employs a masked autoregressive modeling paradigm~\cite{chang2022maskgit, li2023mage}. 
In this process, the discrete codes function as supervision signals. 
After training, CircuitAR can generate discrete tokens progressively, which can be decoded into initial circuit structures by the decoder of the CircuitVQ. 
These prior insights can guide DAS in producing refined circuits that match the target truth tables precisely.

Our key contributions can be summarized as follows:
\begin{itemize}
\item We introduce CircuitVQ, a circuit tokenizer that facilitates graph autoregressive modeling for circuit generation, based on our Circuit AutoEncoder framework;
\item Develop CircuitAR, a model trained using masked autoregressive modeling, which generates initial circuit structures conditioned on given truth tables;
\item Propose a refinement framework that integrates differentiable architecture search to produce functionally equivalent circuits guided by target truth tables;
\item Comprehensive experiments demonstrating the scalability and capability emergence of our CircuitAR and the superior performance of the proposed circuit generation approach.
\end{itemize}

% Motivation
% (a) Diffusion (Vision, Graph), Autoregressive (Language, Vision)
% (b) Circuit Generation for Predefined Setting
% (c) Neural Architecture Search for Strict Logic Equivalence

% Contribution
% (a) Circuit Tokenizer (new transformer arch, training strategy)
% (b) CircuitAR (train and gen strategies, post-ar strategy)
% (c) Extensive Evaluation including BitD (Bit Distance) for Scalability




\subsection{Plasticity in Neural Networks}
In recent years, various methods have been proposed to address plasticity loss.
Several works have focused on maintaining active units \cite{abbas2023loss, elsayed2024addressing} or re-initializing dead units \cite{sokar2023dormant, dohare2024loss}.
Other studies have explored limiting deviations from the initial statistics of model parameters \cite{kumar2023maintaining, lewandowski2023curvature, elsayed2024weight}.
Additionally, some methods rely on architectural modifications \cite{nikishin2024deep, lee2024slow, lewandowski2024plastic}.  
Plasticity loss also occurs in the reinforcement learning due to its inherent non-stationary. \citet{nikishin2022primacy} proposed resetting the model, while \citet{asadi2024resetting} suggested resetting the optimizer state. 

As noted by \citet{berariu2021study}, loss of plasticity can be divided into two distinct aspects: a decreased ability of networks to minimize training loss on new data (trainability) and a decreased ability to generalize to unseen data (generalizability).
While most previous works focused on trainability, \citet{lee2024slow} addressed generalizability loss.
They demonstrated that plasticity loss also occurs under a stationary distribution, as in a warm-start learning scenario where the model is pretrained on a subset of the training data and then fine-tuned on the full dataset.

Most existing studies have focused on only one of the following challenges: trainability, generalizability, or reinforcement learning.
However, in this study, we validate our AID method across all three aspects, demonstrating its effectiveness in each scenario.



\subsection{Activation Function}
Our AID method is a stochastic approach similar to Dropout while also functioning as an activation function.
Therefore, we aim to discuss previously proposed probabilistic activation functions.
Although the field of probabilistic activation functions has not seen extensive research, two noteworthy studies exist.
The first is the Randomized ReLU (RReLU) function, introduced in the Kaggle NDSB Competition \cite{xu2015empirical}.
The original ReLU function maps all negative values to zero, whereas RReLU maps negative values linearly based on a random slope.
During testing, negative values are mapped using the mean of the slope distribution.
Their experimental results suggest that RReLU effectively prevents overfitting.
Another example of a probabilistic activation function is DropReLU \cite{liang2021drop}.
DropReLU randomly determines whether a node's activation is processed through a ReLU function or a linear function.
The authors claim that DropReLU improves the generalization performance of neural networks.
The fundamental distinction between these probabilistic activation functions and our method lies in the generality of our approach.
Unlike simple probabilistic activation functions, our method encompasses techniques such as Dropout and ReLU, providing a more comprehensive framework.

Another related approach involves activation functions designed to address plasticity loss.
\citep{abbas2023loss} proposed the Concatenated Rectified Linear Units (CReLU), which concatenates the outputs of the standard ReLU applied to the input and its negation.
This structure prevents the occurrence of dead units, thereby improving plasticity.
Additionally, trainable activation functions have also been shown to effectively mitigate plasticity loss in reinforcement learning \citep{delfosseadaptive}.
Specifically, they introduced a trainable rational activation function that prevents value overfitting and overestimation in reinforcement learning.



\begin{figure*}[ht!]
    \centering
    \includegraphics[width=0.3\textwidth]{figures/sources/mainnet_pls_acc.pdf}
    \includegraphics[width=0.3\textwidth]{figures/sources/subnet_pls_acc.pdf}
    \includegraphics[width=0.3\textwidth]{figures/sources/warm_start_dropout.pdf}
    \caption{\textbf{Left.} Random label MNIST experiment using an 8-layer MLP. Higher dropout probabilities result in significant trainability loss. 
    \textbf{Middle.} Accuracy of the subnetworks trained on random target. Each subnetworks are sampled from original network after each epoch. Subnetworks of the Dropout also experience trainability loss. \textbf{Right.} Warm-start scenario of Resnet-18 model with CIFAR100 dataset. Dropout improves generalization performance; however, the reduction in accuracy compared to the cold-start scenario is nearly identical to that of the vanilla model.}
    \label{exp_dropout}
\end{figure*}




We study (stochastic) gradient descent on the empirical risk
\begin{equation*}
\cL(w) = \frac{1}{n}\sum_{i=1}^n l(p_i(w))\, ,
\end{equation*}
where the loss function $l$ and the functions  $(p_i)_{i=1}^n$  are specified in the following assumptions. Note that the empirical risk for binary classification from Equation~\eqref{def:emp_risk_intro} is a special case of the above objective.

\begin{assumption}\label{hyp:loss_exp_log}\phantom{=}
  \begin{enumerate}[label=\roman*)]
    \item The loss is either the exponential loss, $l(q) = e^{-q}$, or the logistic loss, $l(q) = \log(1{+}e^{-q})$.
    \item There exists an integer $L \in \mathbb{N}^*$  such that, for all $1 \leq i \leq n$, the function $p_i$ is $L$-homogeneous\footnote{We recall that a mapping $f : \mathbb{R}^d \rightarrow \mathbb{R}$ is positively $L$-homogeneous if $f(\lambda w) = \lambda^L f(w)$ for all $w \in \mathbb{R}^d$ and $\lambda >0$.}, locally Lipschitz continuous and semialgebraic.
  \end{enumerate}
\end{assumption}
If the $p_i$'s were differentiable with respect to $w$, the chain rule would guarantee that
\begin{align*}
\nabla \mathcal{L}(w) = \frac{1}{n}\sum_{i=1}^n  l'(p_i(w)) \nabla p_i(w)\enspace.
\end{align*}
However, we only assume that the $p_i$'s are semialgebraic. While we could consider Clarke subgradients, the Clarke subgradient of operations on functions (e.g., addition, composition, and minimum) is only contained within the composition of the respective Clarke subgradients. This, as noted in Section~\ref{sec:cons_field}, implies that the output of backpropagation is usually not an element of a Clarke subgradient but a selection of some conservative set-valued field.
Consequently, for $1\leq i \leq n$, we consider $D_i : \bbR^d \rightrightarrows\bbR^d$, a conservative set-valued field of $p_i$, and a function $\sa_i : \bbR^d \rightarrow \bbR^d$ such that for all $w \in \bbR^d$, $\sa_i(w) \in D_i(w)$. Given a step-size $\gamma >0$, gradient descent (GD)\footnote{More precisely, this refers to conservative gradient descent. We use the term GD for simplicity, as conservative gradients behave similarly to standard gradients.} is then expressed as
\begin{equation*}\label{eq:gd_new}\tag{GD}
  w_{k+1} = w_k - \frac{\gamma}{n} \sum_{i=1}^n l'(p_i(w_k))\sa_i(w_k)\,.
\end{equation*}
For its stochastic counterpart, stochastic gradient descent (SGD), we fix a batch-size $1\leq n_b \leq n$. At each iteration $k \in \bbN$, we randomly and uniformly draw a batch $B_k \subset \{1, \ldots, n \}$ of size $n_b$. The update rule is then given by 
\begin{equation*}\label{eq:sgd_new}\tag{SGD}
  w_{k+1} = w_k -  \frac{\gamma}{n_b}\sum_{i\in B_k} l'(p_i(w_k)) \sa_i(w_k)\, .
\end{equation*}
The considered conservative set-valued fields will satisfy an Euler lemma-type assumption.
%\nic{Smoother transition}
\begin{assumption}\phantom{=}\label{hyp:conserv}
  For every $i \leq n$, $\sa_i$ is measurable and $D_i$ is semialgebraic. Moreover, for every $w \in \bbR^d$ and $\lambda \geq 0$, $\sa_i(w)  \in D_i(w)$,
  \begin{equation*}
    D_i(\lambda w) = \lambda^{L-1} D_i(w)\, , \textrm{ and } \quad   L p_i(w) = \scalarp{\sa_i(w)}{w}\, .
  \end{equation*}
\end{assumption}
%\nic{Smoother transition}
Having in mind the binary classification setting, in which $p_i(w) = y_i \Phi(x_i, w)$, we define the margin
\begin{equation}\label{def:marg}
  \sm: \bbR^d \rightarrow \bbR, \quad \sm(w) = \min_{1\leq i \leq n} p_i(w)\, .
\end{equation}
It quantifies the quality of a prediction rule $\Phi(\cdot, w)$. In particular,  the training data is perfectly separated when $\sm(w) >0$. A binary prediction for $x$ is given by the sign of $\Phi(x, w)$, and under the homogeneity assumption, it depends only on the normalized direction $w / \norm{w}$. Consequently, we will focus on the sequence of directions $u_k := w_k / \norm{w_k}$. Our final assumption ensures that the normalized directions $(u_k)$ have stabilized in a region where the training data is correctly classified.

\begin{assumption}\label{hyp:marg_lowb}
  Almost surely, $\liminf \sm(u_k) >0$.
\end{assumption}
Before presenting our main result, we comment on our assumptions.

\paragraph{On Assumption~\ref{hyp:loss_exp_log}.} As discussed in the introduction, the primary example we consider is when $p_i(w) = y_i \Phi(x_i;w)$ is the signed prediction of a feedforward neural network without biases and with piecewise linear activation functions on a labeled dataset $((x_i,y_i))_{i \leq n}$. In this case,
\begin{equation}\label{eq:NN}
 p_i(w) = y_i \Phi(w;x_i) = y_i V_L(W_L) \sigma(V_{L-1}(W_{L-1}) \sigma(V_{L-1}(W_{L-2}) \ldots \sigma(V_{1}(W_1 x_i))))\, ,
\end{equation}
where $w = [W_1, \ldots, W_L]$, $W_i$ represents the weights of the $i$-th layer, $V_i$ is a linear function in the space of matrices (with $V_i$ being the identity for fully-connected layers) and $\sigma$ is a coordinate-wise activation function such as $z \mapsto \max(0,z)$ ($\ReLU$), $z \mapsto \max(az, z)$ for a small parameter $a>0$ (LeakyReLu) or $z \mapsto z$. Note that the mapping $w \mapsto p_i(w)$ is semialgebraic and $L$-homogeneous for any of these activation functions. Regarding the loss functions, the logistic and exponential losses are among the most commonly studied and widely used. In Appendix~\ref{app:gen_sett}, we extend our results to a broader class of losses, including $l(q) = e^{-q^a}$ and $l(q) = \ln (1 + e^{-q^a})$ for any $a \geq 1$.

\paragraph{On Assumption~\ref{hyp:conserv}.} Assumption~\ref{hyp:conserv} holds automatically  if $D_i$ is the Clarke subgradient of $p_i$. Indeed, at any vector $w \in \bbR^d$, where $p_i$ is differentiable it holds that $p_i(\lambda w) = \lambda^{L} p_i(w)$. Differentiating relatively to $w$ and $\lambda$ (noting that $p_i$ remains differentiable at $\lambda w$ due to homogeneity), we obtain $\lambda \nabla p_i(\lambda w) = \lambda^{L} \nabla p_i(w)$ and $\scalarp{\nabla p_i(\lambda w)}{w} = L \lambda^{L-1} p_i(w)$. The expression for any element of the Clarke subgradient then follows from~\eqref{eq:def_clarke}. 

However, for an arbitrary conservative set-valued field, Assumption~\ref{hyp:conserv} does not necessarily hold. For instance, $D(x) = \mathds{1}(x \in \mathbb{N})$ is a conservative set-valued field for $p \equiv 0$, which does not satisfy Assumption~\ref{hyp:conserv}. Nevertheless, in practice, conservative set-valued fields naturally arise from a formal application of the chain rule. For a non-smooth but homogeneous activation function $\sigma$, one selects an element $e \in \partial \sigma (0)$, and computes $\sa_i(w)$ via backpropagation. Whenever a gradient candidate of $\sigma$ at zero is required (i.e., in~\eqref{eq:NN}, for some $j$, $V_j(W_j)$ contains a zero entry), it is replaced by $e$. 
Since $V_j(W_j)$ and $V_j(\lambda W_j)$ have the same zero elements, it follows that for every such $w$, $
\sa_i(\lambda w) = \lambda^L \sa_i(w)$. The conservative set-valued field $D_i$ is then obtained by associating to each $w$ the set of all possible outcomes of the chain rule, with $e$ ranging over all elements of $\partial \sigma(0)$. Thus, for such fields, Assumption~\ref{hyp:conserv} holds.


\paragraph{On Assumption~\ref{hyp:marg_lowb}.} Training typically continues even after the training error reaches zero.
Assumption~\ref{hyp:marg_lowb} characterizes this late-training phase, where our result applies. 
As noted earlier, since $\sm$ is $L$-homogeneous, the classification rule is determined by the direction of the  iterates $u_k=w_k/\norm{w_k}$. Assumption~\ref{hyp:marg_lowb} then states that, beyond some iteration, the normalized margin remains positive. 
This assumption is natural in the context of studying the implicit bias of SGD: we \emph{assume} that we reached the phase in which the dataset is correctly classified and \emph{then} characterize the limit points. A similar perspective was taken in  \cite{nacson2019lexicographic}, where the implicit bias of GF was analyzed under the assumption that the sequence of directions and the loss converge. However, unlike their approach, ours does not require assuming such convergence a priori.

Earlier works such as \cite{ji2020directional,Lyu_Li_maxmargin}, which analyze subgradient flow or smooth GD, establish convergence by assuming the existence of a single iterate $w_{k_0}$ satisfying $\sm(w_{k_0}) > \varepsilon$ and then proving that $\lim \sm(u_{k}) > 0$. Their approach relies on constructing a smooth approximation of the margin, which increases during training, ensuring that $\sm(u_k) > 0$ for all iterates with $k \geq k_0$. This is feasible in their setting, as they study either subgradient flow or GD with smooth $p_i$’s, allowing them to leverage the descent lemma.

In contrast, our analysis considers a nonsmooth and stochastic setting, in which, even if an iterate $w_{k_0}$ satisfying $\sm(w_{k_0}) > \varepsilon$ exists, there is no a priori assurance that subsequent iterates remain in the region where Assumption~\ref{hyp:marg_lowb} holds. From this perspective, Assumption~\ref{hyp:marg_lowb} can be viewed as a stability assumption, ensuring that iterates continue to classify the dataset correctly. Establishing stability for stochastic and nonsmooth algorithms is notoriously hard, and only partial results in restrictive settings exist \cite{borkar2000ode,ramaswamy2017generalization,josz2024global}.

%Finally, note that Assumption~\ref{hyp:marg_lowb} only needs to hold almost surely. Specifically, with probability 1, there exist $k_0$ and $\varepsilon$ such that for all $k \geq k_0$, $\sm(u_k) \geq \varepsilon > 0$. In the case of~\eqref{eq:sgd_new}, $k_0$ and $\delta$ are random variables and may take different values across different realizations. 

%\paragraph{On constant stepsizes.}
%We allow the step size to be a constant of arbitrary magnitude, subject to the stability Assumption~\ref{hyp:marg_lowb}. This may seem surprising in a nonsmooth and stochastic setting, where a vanishing step size is typically required to ensure convergence (see, e.g., \cite{majewski2018analysis, dav-dru-kak-lee-19, bolte2023subgradient, le2024nonsmooth}).

\input{arxiv_template/main/Linear}

\section{Analysis of LoRA under Nonlinear Models}
\label{sec:nonlinear}


Now we focus on the nonlinear setting described in \cref{sec:problsemsetting}, where we consider the exact-rank case $r=r^*$ for delivery.
We will demonstrate the linear convergence rate in the linear setting can still hold for the nonlinear setting.

Following \cref{sec:scaledgd}, we continue to consider preconditioned GD on $(\bm A_t, \bm B_t)$ with the same step-size $\eta$ for convenience:
\begin{equation}\label{eq:ABiter_nonlinear}
\begin{split}
     \bm A_{t+1} & = \bm A_t - \eta \nabla_{\bm A} \widetilde{L}\left(\bm A_t\,,\bm B_t\right)\left(\bm B_t \bm B_t^{\!\top}\right)^{-1}\,, \\
     \bm B_{t+1}  & = \bm B_t - \eta \left(\bm A_t^{\!\top} \bm A_t\right)^{-1} \nabla_{\bm B} \widetilde{L}\left(\bm A_t\,,\bm B_t\right)\,.
\end{split}
\end{equation}
Notice that here we use standard matrix inversion since we can prove that $\bm A_t$ and $\bm B_t$ stay non-singular across all $t\geq 0$.
By denoting $\bm W_t := \bm W^{\natural} + \bm A_t \bm B_t$, we have the gradients
\begin{align*}
\nabla_{\bm A}\widetilde{L}\left(\bm A_t\,,\bm B_t\right) = -\bm J_{
\bm W_t} \bm B_t^{\!\top}, \quad 
\nabla_{\bm B}\widetilde{L}\left(\bm A_t\,,\bm B_t\right) = -\bm A_t^{\!\top} \bm J_{
\bm W_t} \,,
\end{align*}
where we define 
\begin{align*}
    \bm J_{
\bm W_t} := \frac{1}{N}\widetilde{\bm X}^{\!\top}\left[\sigma(\widetilde{\bm X}\widetilde{\bm W}^\natural) - \frac{1}{N}\widetilde{\bm X}^{\!\top}\sigma(\widetilde{\bm X}\bm W_t)\right]\odot \sigma'(\widetilde{\bm X}\bm W_t)\,.
\end{align*}

To deliver the proof, apart from the above-mentioned assumptions in \cref{sec:assumptions} for the the nonlinear setting, we also need the following additional assumptions.
\begin{assumption}\label{assum:nonlinear-orth}
  We assume that $\bm W^\natural$ has orthonormal columns and its row space is orthogonal to that of $\Delta$.
\end{assumption}

\begin{assumption}\label{assum:nonlinear-delta}
We assume that $\|\Delta\|_{op}\leq \frac{\sqrt{2}-1}{2}$ with $\operatorname{Rank}(\Delta)=r^*$ where $k+r^*\leq d$ and $r^* \ll \min\{d\,,k\}$.
\end{assumption}

\noindent
{\bf Remark:} 
\cref{assum:nonlinear-orth} ensures rich task diversity between pre-trained model and downstream tasks. We notice that such assumption is also considered in \cite{dayi2024gradientdynamicslowrankfinetuning}.  \cref{assum:nonlinear-delta} restricts the norm of downstream feature shift since the signal of adapted weight is generally smaller than the pre-trained weight. We can empirically assess the validity of this assumption in \cref{fig:SV}.


Here we can show that, for the nonlinear model, LoRA training can achieve global linear convergence under \eqref{eq:spectral-init-linear} via preconditioned GD in \cref{eq:ABiter_nonlinear}.
\begin{theorem}[Simplified version of \cref{LC}]\label{main:LC}
    Under assumptions in \cref{sec:assumptions} for the nonlinear setting, \cref{assum:nonlinear-orth}, and \ref{assum:nonlinear-delta}, with training conducted by \cref{eq:ABiter_nonlinear} and initialization via \eqref{eq:spectral-init-linear},  we take $\epsilon = \mathcal{O}\left(\frac{1}{r^*\kappa\sqrt{d}}\right)$ and $\rho\leq 0.01$ such that
    we set
    \begin{align}\label{main:selection-scale}
        \gamma\in\left[\frac{1}{c_{\rm H}}-\frac{\rho}{3c_{\rm H}\kappa\sqrt{2r^*}}\,,\frac{1}{c_{\rm H}}+\frac{\rho}{3c_{\rm H}\kappa\sqrt{2r^*}}\right]
    \end{align}
    with $c_{\rm H} :=\frac{1}{4} + \frac{1}{4\pi}\sum_{\substack{n\geq 1, \\ n \text{ odd}}} 2^{-n} n^{-2} (n!)^{-2}$.
    Then choosing $\eta \in \left(c_{\eta}\,,\frac{1}{2c_{\rm H}}\right)$ for a small constant $c_{\eta}>0$, with probability at least $1-2Cdk\operatorname{exp}\left(-\epsilon^2 N\right)$ for a universal constant $C>0$, we have
    \begin{align}\label{eq:atbtnon}
            \left\|\bm A_{t}\bm B_{t} - \Delta\right\|_{\rm F}  \leq \left(1-\frac{c_{\rm H}}{10}\eta\right)^t \rho\lambda^*_{r^*}\,, \forall t \geq 0\,.
        \end{align}
\end{theorem}

\noindent
{\bf Remark:} We make the following remarks:
\begin{itemize}
    \item This theorem is based on $\left\|\bm A_0 \bm B_0 - \Delta\right\|_{\rm F} \leq \rho\lambda^*_{r^*}$ at initialization (\cref{assum:nonlinear-delta} is not needed), see \cref{A0B0-init-risk} for details, which demonstrates the ability of one-step full gradient can improve feature learning.
    \item The final rate is independent of condition number $\kappa$ of downstream feature shift $\Delta$, which coincides with the results from linear model. This provide us evidence that adding preconditioners can also work for nonlinear model.
\end{itemize}

\noindent 
{\bf Proof Sketch:} The complete proof can be found in \cref{PGD-Nonlinear}. By Hermite decomposition, we can compute the expectation of $\bm J_{\bm W_t}$ (see \cref{expec-grad}) and decompose $\bm J_{\bm W_t}$ into $c_{\rm H}\left(\bm A_{t}\bm B_{t} - \Delta\right) + \bm \Xi_t$,
where $\bm \Xi_t$ is defined in \cref{Lip}. The first term is the signal term which can dominate the preconditioned GD dynamics. The second term $\bm \Xi_t := T1 +T2$ consists of two parts (details see \cref{err-concen-pop}): the first part $T1$ is the higher-order residual terms from $\mathbb{E}_{\widetilde{\bm x}}\left[\bm J_{\bm W_t}\right]$ which related to Hermite decomposition. Since the decay of Hermite coefficients of $\sigma$ is faster than polynomial decay, it can be well controlled. For the second term $T2$, it comes from the concentration error of $\bm J_{\bm W_t}$, which can also controlled by large sample size $N$.

To handle $\| \bm A_t \bm B_t - \Delta \|_{\rm F}$, we explore its recursion relationship in \cref{Lip}. The key part is to control $\left\|\left(\bm I_d - \bm U_{\bm A_t} \bm U_{\bm A_t}^{\!\top}\right)\Delta\left(\bm I_k - \bm V_{\bm B_t} \bm V_{\bm B_t}^{\!\top}\right)\right\|_{\rm F}$ as well as its complementary part in \cref{basis-alignment} and higher order term in \cref{err-cross}.  We deliver the complete proof to \cref{PGD-Nonlinear}. 

Note that the above two assumptions are not required if we modify the gradient update of \cref{eq:ABiter_nonlinear} by removing the mask matrix $\sigma'(\widetilde{\bm X}\bm W_t)$, a smoothing technique from \cite{kalai2009isotron,kakade2011efficient,wu2023finite}, i.e.,
\begin{align*}
    \bm J_{\bm W_t}^{\tt GLM} := \frac{1}{N}\widetilde{\bm X}^{\!\top}\bigg(\sigma\left(\widetilde{\bm X}\widetilde{\bm W}^\natural\right)-\sigma\left(\widetilde{\bm X}\bm W_t\right)\bigg)\,.
\end{align*}
In this case, we reformulate \cref{eq:ABiter_nonlinear} as
\begin{equation}\label{ABiter_GLM}
    \begin{split}
        \bm A_{t+1} & = \bm A_t + \eta \bm J_{\bm W_t}^{\tt GLM}\bm B_t^{\!\top}\left(\bm B_t\bm B_t^{\!\top}\right)^{-1}\,,\\
        \bm B_{t+1} & = \bm B_t + \eta \left(\bm A_t^{\!\top}\bm A_t\right)^{-1}\bm A_t^{\!\top}\bm J_{\bm W_t}^{\tt GLM}\,.
    \end{split}
\end{equation}
And we propose to use $\bm G^\natural := \bm J_{\bm W^\natural}^{\tt GLM}$ for \eqref{eq:spectral-init-linear} to initialize $\bm A_0$ and $\bm B_0$. The global linear convergence results are given as below.
\begin{theorem}[Simplified version of \cref{smo-LC}]\label{main:smo-LC}
    Under assumptions in \cref{sec:assumptions} for the nonlinear setting, with training conducted by \cref{ABiter_GLM} and initialization via \eqref{eq:spectral-init-linear} by taking $\bm G^\natural := \bm J_{\bm W^\natural}^{\tt GLM}$, suppose $\epsilon = \mathcal{O}\left(\frac{1}{r^* \kappa \sqrt{d}}\right)$ and $\rho\leq\frac{1}{20}$, we take 
    \begin{equation*}
       \gamma\in\left[2-\frac{2\rho}{3\kappa\sqrt{2r^*}}\,,2+\frac{2\rho}{3\kappa\sqrt{2r^*}}\right]\,,
    \end{equation*}
   and set $\eta \in \left(c^{\tt GLM}_{\eta}\,,1\right)$ where $c^{\tt GLM}_{\eta}>0$ is a small constant, then with probability at least $1-2Cdk\operatorname{exp}\left(-\epsilon^2 N\right)$ for a universal constant $C>0$, we have
    \begin{align*}
            \left\|\bm A_{t}\bm B_{t} - \Delta\right\|_{\rm F} & \leq \left(1-\frac{\eta}{4}\right)^t \rho\lambda^*_{r^*}\,.
        \end{align*}
\end{theorem}

\noindent
{\bf Remark:} Removing the mask matrix $\sigma'(\widetilde{\bm X}\bm W_t)$ in \cref{eq:ABiter_nonlinear} allows for better linear convergence performance with $(1-\eta/4)^t$ than that in \cref{eq:atbtnon}, albeit without \cref{assum:nonlinear-orth}, \ref{assum:nonlinear-delta}.\\

\noindent
{\bf Proof Sketch:} The proof strategy is similar to \cref{main:LC}. The key difference is that the corresponding $\bm \Xi_t$ under \cref{ABiter_GLM} does not have the residual terms from Hermite decomposition. We deliver the complete proof to \cref{prec-smooth-gd}.

\begin{figure*}[!h]
    \centering
    \begin{subfigure}[b]{0.8\linewidth}
        \centering
        \includegraphics[width=0.45\linewidth]{images/residual/text/CIReVL_Recall5.png}
        \hfil
        \includegraphics[width=0.45\linewidth]{images/residual/text/pic2word_recall5.png}
        \caption{\textbf{PDV-T}: Impact of $\alpha$ scaling on composed text embeddings}
        \label{fig:residual_text_sub}
    \end{subfigure}
    
    \begin{subfigure}[b]{0.8\linewidth}
        \centering
        \includegraphics[width=0.45\linewidth]{images/residual/image/CIReVL_Recall5.png}
        \hfil
        \includegraphics[width=0.45\linewidth]{images/residual/image/pic2word_recall5.png}
        \caption{\textbf{PDV-I}: Impact of $\alpha$ scaling on composed image embeddings}
        \label{fig:residual_image_sub}
    \end{subfigure}
    
    \begin{subfigure}[b]{0.8\linewidth}
        \centering
        \includegraphics[width=0.45\linewidth]{images/residual/fusion/CIReVL_Recall5.png}
        \hfil
        \includegraphics[width=0.45\linewidth]{images/residual/fusion/pic2word_recall5.png}
        \caption{\textbf{PDV-F}: Impact of varying $\beta$ with on composed fused embeddings}
        \label{fig:residual_fusion_sub}
    \end{subfigure}
    \caption{Impact of changing $\alpha$/$\beta$ on Recall@5 performance across different PDV applications. For each row, results are shown for the CIReVL (left) and Pic2Word (right) baseline methods.}
    \label{fig:residual_all}
\end{figure*}

\section{Experiments} 
\label{sec:exp}
\noindent\textbf{Implementation Details.} We utilize the official implementations of four ZS-CIR baseline methods: CIReVL\footnote{https://github.com/ExplainableML/Vision\_by\_Language} and LDRE \footnote{https://github.com/yzy-bupt/LDRE} as representative caption-based feature extraction approaches and Pic2Word\footnote{https://github.com/google-research/composed\_image\_retrieval} and SEARLE\footnote{https://github.com/miccunifi/SEARLE} as representative pseudo tokenization-based methods. All feature extraction processes follow the original implementations provided by these baseline methods. However, to calculate $\Delta_{PDV}$, we need text embeddings without prompts, which are not provided in the original implementations. For CIReVL and LDRE, we obtain these embeddings by passing the generated image captions directly to CLIP. For Pic2Word and SEARL, we construct the base text embedding by passing the phrase ``a photo of $\langle$token$\rangle$" to CLIP, where $\langle$token$\rangle$ represents the extracted image token obtained via text inversion.

\noindent\textbf{Datasets and Base Vision-Language Models.} Following previous work, we evaluated our method on a suite of datasets including Fashion-IQ \cite{wu2021fashion}, CIRR \cite{liu2021image} and CIRCO \cite{baldrati2023zero}. Our proposed method is a plug-and-play approach requiring no additional training, leveraging only pre-trained models. For feature extraction, we use three CLIP variants: ViT-B/32, ViT-L/14, and ViT-G/14, and used the same pre-trained weights as used by the baseline methods. For image tokenization, we employ the pre-trained Pic2Word model. 

\subsection{Effect of Using the PDV}
We now explore the impact of the three proposed uses of the PDV: Using the PDV to augment text queries (PDV-T, see Sec. \ref{sec:exp1}), using the PDV to augment image queries (PDV-I, see Sec. \ref{sec:exp2}), and using the PDV in queries that fuse image and text data (PDV-F, see Sec. \ref{sec:exp3}).

\begin{table*}
	\footnotesize
	\centering
	\begin{tabular}{l|l|c|c|c|cccccccc}
		\hline
		\textbf{Fashion-IQ} & & & & & \multicolumn{2}{c}{\textbf{Shirt}} & \multicolumn{2}{c}{\textbf{Dress}} & \multicolumn{2}{c}{\textbf{Toptee}} & \multicolumn{2}{c}{\textbf{Average}} \\ \hline
		Backbone & Method& $\beta$ & $\alpha_{I}$& $\alpha_{T}$ & R@10 & R@50 & R@10 & R@50 & R@10 & R@50 & R@10 & R@50 \\
		\hline
		\multirow{6}{*}{ViT-B/32} %
		& SEARLE & - & - & - & 24.14 & 41.81 & 18.39 & 38.08 & 25.91 & 47.02 & 22.81 & 42.30 \\
		& SEARLE + \textbf{PDV-F} & 0.9 & 1.1 & 0.9 & \hli{24.83} & 41.71 & \hli{20.13} & \hli{41.40} & \hli{25.96} & \hli{47.17}  & \hli{23.64} & \hli{43.43} \\
		& CIReVL \textdagger &- & -& -& 28.36 & 47.84 & 25.29 & 46.36 & 31.21 & 53.85 & 28.29 & 49.35 \\
		& CIReVL + \textbf{PDV-F} & 0.75 & 1.4 & 1.4 & \hlb{32.88} & \hlb{52.80} & \hlb{32.67} & \hlb{54.49} & \hlb{38.91} & \hlb{61.81} & \hlb{34.82} & \hlb{56.37} \\
		& LDRE \textdagger & - & - & - & 27.38 & 46.27 & 19.97 & 41.84 & 27.07 & 48.78 & 24.81 & 45.63 \\
		& SEIZE \textdagger & - & - & - & \underline{29.38} & \underline{47.97} & \underline{25.37} & \underline{46.84} & \underline{32.07} & \underline{54.78} & \underline{28.94} & \underline{49.86} \\
		\hline
		\multirow{8}{*}{ViT-L/14} & Pic2Word & & & & 25.96 & 43.52 & 19.63 & 40.90 & 27.28 & 47.83 & 24.29 & 44.08 \\
		& Pic2Word + \textbf{PV-F} & 0.8 & 1.0 & 1.0 & \hli{28.21} & \hli{44.55} & \hli{20.92} & \hli{42.24} & \hli{29.02} & \hli{48.90}& \hli{26.05} & \hli{45.23}\\
		& SEARLE & - & - & - & 26.84 & 45.19 & 20.08 & 42.19 & 28.40 & 49.62 & 25.11 & 45.67 \\
		& SEARLE +\textbf{PDV-F} & 0.8 & 1.2 & 1.0 & \hli{28.66} & \hli{46.76} & \hli{23.60} & \hli{46.41} & \hli{31.00} & \hli{52.32} & \hli{27.75} & \hli{48.50} \\
		& CIReVL \textdagger & & & & 29.49 & 47.40 & 24.79 & 44.76 & 31.36 & 53.65 & 28.55 & 48.57 \\
		
		& CIReVL + \textbf{PDV-F} & 0.55 & 1 & 1.3 & \hlb{37.78} & \hlb{54.22} & \hlb{33.61} & \hlb{56.07} & \hlb{41.61} & \hlb{62.16} & \hlb{37.67} & \hlb{57.48} \\
		& LinCIR & - & - & - & 29.10 & 46.81 & 20.92 & 42.44 & 28.81 & 50.18 & 26.82 & 46.49 \\
        & SEIZE & -& -& -& \underline{33.04} & \underline{53.22} & \underline{30.93} & \underline{50.76} & \underline{35.57} & \underline{58.64} & \underline{33.18} & \underline{54.21} \\
		\hline
        \multirow{6}{*}{ViT-G/14} & Pic2Word  & - & - & - & 33.17 & 50.39 & 25.43 & 47.65 & 35.24 & 57.62 & 31.28 & 51.89\\
         & SEARLE  & - & - & - & 36.46 & 55.35 & 28.16 & 50.32 & 39.83 & 61.45 & 34.81 & 55.71\\
		  & CIReVL \textdagger & -& -& -& 33.71 & 51.42 & 27.07 & 49.53 & 35.80 & 56.14 & 32.19 & 52.36 \\
		& CIReVL + \textbf{PV-F} & 0.6 & 1.4 & 1.4 & \hli{41.90} & \hli{58.19} & \hlb{40.70} & \hlb{62.82} & \underline{\hli{48.09}}& \hli{67.77}& \underline{\hli{43.56}}& \hli{62.93}\\
        & LinCIR & - & - & - & \textbf{46.76} & \underline{65.11} & 38.08& 60.88& \textbf{50.48}& \underline{71.09}& \textbf{45.11} & \underline{65.69}\\
        & SEIZE & - & - & - & \underline{43.60} & \textbf{65.42}& \underline{39.61} & \underline{61.02} & 45.94& \textbf{71.12}& 43.05& \textbf{65.85}\\
		\hline
	\end{tabular}
	\caption{Average recall for different methods on Fashion-IQ validation dataset. \textdagger~denotes that numbers are taken from the original paper.}
	\label{tab:fashion_iq_results}
\end{table*}


\begin{table*}
	\centering
	\footnotesize
	\setlength{\tabcolsep}{4pt}
	\begin{tabular}{ll|c|c|c|cccc|cccc|ccc}
		\hline
		\multicolumn{2}{c|}{\textbf{Dataset}} & & & &  \multicolumn{4}{c|}{\textbf{CIRCO}} & \multicolumn{7}{c}{\textbf{CIRR}} \\
		\hline
		\multicolumn{2}{c|}{Metric} & & & & \multicolumn{4}{c|}{mAP@k} & \multicolumn{4}{c|}{Recall@k} &\multicolumn{3}{c}{$R_s$@k} \\
		\cline{3-16}
		Arch & Method & $\beta$ & $\alpha_I$ & $\alpha_T$ & k=5 & k=10 & k=25 & k=50 & k=1 & k=5 & k=10 & k=50 & k=1 & k=2 & k=3 \\
		\hline
		\multirow{8}{*}{ViT-B/32} 
		& PALAVRA\cite{cohen2022my} \textdagger & -& -& -& 4.61 & 5.32 & 6.33 & 6.80 & 16.62 & 43.49 & 58.51 & 83.95 & 41.61 & 65.30 & 80.94 \\
		& SEARLE \textdagger & -& -&- & 9.35 & 9.94 & 11.13 & 11.84 & 24.00 & 53.42 & 66.82 
		& 89.78 & 54.89 & 76.60 & 88.19 \\
		& SEARLE + \textbf{PDV-F} & 0.9 & 1.4 & 1.2 & \hli{9.99} & \hli{10.50}  & \hli{11.70} & \hli{12.40} & \hli{24.53} & \hli{53.71} & \hli{67.33} & \hli{89.81} & \hli{56.94} & \hli{78.05} & \hli{88.99} \\
		&CIReVL \textdagger & - & - & -& 14.94 & 15.42 & 17.00 & 17.82 & 23.94 & 52.51 & 66.00 & 86.95 & 60.17 & 80.05 & 90.19 \\
		& CIReVL + \textbf{PDV-F} & 0.75 & 1.4 & 1.2 & \hlb{19.90} & \hlb{20.61} & \hlb{22.64} & \hlb{23.52} & \hlb{33.25} & \hlb{64.15} & \hlb{75.23} & \hlb{92.43} & \hlb{65.81} &\underline{\hli{83.76}} &\underline{\hli{92.10}} \\
		& LDRE & -& -& -& 17.81 & 18.04 & 19.73 & 20.67 & 25.69 & 55.52 & 68.77 & 89.86 & 60.10 & 80.58 & 91.04 \\
		& LDRE + \textbf{PDV-F} & 0.75 & 1.4 & 1.4 & \hli{17.80} & \hli{18.78} & \hli{20.61} & \hli{21.56} & \underline{\hli{29.30}} & \underline{\hli{60.39}} & \underline{\hli{72.51}} & \underline{\hli{91.42}} & \hli{63.06} & \hli{82.36} & \hli{91.54} \\
        & SEIZE & -&- &- & \underline{19.04} & \underline{19.64} & \underline{21.55}& \underline{22.49}& 27.47 & 57.42& 70.17 & - & \underline{65.59} & \textbf{84.48}& \textbf{92.77} \\
 		\hline
		\multirow{10}{*}{ViT-L/14}
		& Pic2Word & -& -& -& 6.81 & 7.49 & 8.51 & 9.07 & 23.69 & 51.32 & 63.66 & 86.21 & 53.61 & 74.34 & 87.28 \\
		& Pic2Word + \textbf{PDV-F} & 0.85 & 1.2 & 1.0 & \hli{7.74} &  \hli{8.67} & \hli{9.77} & \hli{10.37} & \hli{23.90} & \hli{51.95} & \hli{64.63} & \hli{87.04} & \hli{53.16}  & \hli{74.07} & \hli{87.08}\\
		& SEARLE \textdagger & - & - & - & 11.68 & 12.73 & 14.33 & 15.12 & 24.24 & 52.48 & 66.29 & 88.84 & 53.76 & 75.01 & 88.19 \\
		& SEARLE + \textbf{PDV-F} & 0.85 & 1.4 & 1.2 & \hli{12.58} & \hli{13.57} & \hli{15.30} & \hli{16.07} & \hli{25.64} & \hli{53.61} & \hli{66.58} & \hli{88.55} & \hli{55.83} & \hli{76.48} & \hli{88.53} \\
		& CIReVL \textdagger & -& -& -& 18.57 & 19.01 & 20.89 & 21.80 & 24.55 & 52.31 & 64.92 & 86.34 & 59.54 & 79.88 & 89.69 \\
		& CIReVL + \textbf{PDV-F} & 0.75 & 1.4 & 1.2 & \hlb{25.67} & \hlb{26.61} & \underline{\hli{28.81}} & \hlb{29.95} & \hlb{36.24} & \hlb{66.17} & \hlb{76.96} & \hlb{92.29} & \hlb{68.07} & \hlb{85.35} & \hlb{93.47} \\
		& LDRE & -& -& -& 22.32 & 23.75 & 25.97 & 27.03 & 26.68 &55.45  & 67.49 & 88.65 & 60.39 & 80.53 & 90.15 \\
		& LDRE + \textbf{PDV-F} & 0.75 & 1.4 & 1.4 & \hli{25.23} & \hli{26.52} & \hlb{28.94} & \hlb{29.95} & \underline{\hli{30.16}} & \underline{\hli{59.98}} & \underline{\hli{71.90}} & \underline{\hli{90.87}} & \hli{63.66} & \hli{82.87} & \hli{91.57} \\

        & LinCIR & - & - & - &12.59 &13.58 &15.00 &15.85 &25.04 &53.25 &66.68 & - &57.11 &77.37 &88.89\\
        & SEIZE & -& -& -& 24.98 & 25.82 &28.24 &\underline{29.35}& 28.65 &57.16& 69.23& - &\underline{66.22} &\underline{84.05} &\underline{92.34} \\
        

        
		\hline
		\multirow{7}{*}{ViT-G/14} & CIReVL \textdagger & -& -& -& 26.77 & 27.59 & 29.96 & 31.03 & 34.65 & 64.29 & 75.06 & 91.66 & 67.95 & 84.87 & 93.21 \\

		& CIReVL + \textbf{PDV-F} & 0.75 & 1.4 & 1.2 & \hli{30.02} & \hli{31.46} & \hli{34.01} & \hli{35.08} & \hli{38.15} &\hli{67.93} & \hli{77.90} & \hli{92.77} & \hli{69.37} & \hli{85.37} & \hli{93.45}  \\
		
		& LDRE & -& -& -& \underline{33.30} & \underline{34.32} & \underline{37.17} & \underline{38.27} & 37.40 & 66.96 & 78.17 & 93.66 & 68.84 & 85.64 & 93.90 \\
		& LDRE + \textbf{PDV-F} & 0.75 & 1.4 & 1.4 & \hlb{34.88} & \hlb{36.41} & \hlb{39.12} & \hlb{40.23} & \hlb{42.51} & \hlb{72.22} & \hlb{81.71} & \hlb{94.94} & \underline{\hli{72.39}} & \underline{\hli{88.34}} & \underline{\hli{94.80}} \\
        & SEARLE & - & - & - & 13.20 &13.85 &15.32 &16.04 & 34.80 & 64.07 & 75.11 &-&68.72 &84.70 &93.23 \\
        & LinCIR & - & - & - & 19.71 &21.01 &23.13 &24.18 &35.25 &64.72 &76.05 & - &63.35 &82.22 &91.98 \\
        & SEIZE & -& -& -& 32.46 & 33.77 &36.46 &37.55 &\underline{38.87} & \underline{69.42} & \underline{79.42} & -&\textbf{74.15} & \textbf{89.23} & \textbf{95.71} \\
		\hline
	\end{tabular}
	\caption{Performance comparison on CIRCO and CIRR test datasets. As in previous works, for CIRCO, mAP@k is reported, while for CIRR both Recall@k and $R_s$@k metrics are used. \textdagger~denotes that numbers are taken from the original paper.}
	\label{tab:circo_cirr_results}
\end{table*}

\noindent{\textbf{Analysing the PDV for Text (PDV-T)}}
\label{sec:exp1}
To investigate how scaling the prompt vector, $\Delta_{PDV}$, affects retrieval performance with composed text embeddings, we conducted experiments using two zero-shot approaches (CIReVL and Pic2Word) with different backbone networks across three datasets. We evaluated the performance by varying the scaling parameter, $\alpha$ (Eq. \ref{eqn:text_embedding}), from -0.5 to 3 by an interval of 0.1.

The results are presented in Figure \ref{fig:residual_text_sub}. To account for scale variations across different experiments, we report relative recall values, where a baseline of zero is established at $\alpha=1$. As shown in Figure \ref{fig:residual_text_sub}, varying $\alpha$ leads to significant changes in relative recall performance\footnote{See supplementary material for Recall@10 and Recall@50 figures}. Our analysis reveals method-specific patterns across datasets. With CIReVL, increasing $\alpha$ improves relative recall on both FashionIQ and CIRCO datasets. In contrast, Pic2Word shows no significant improvement on FashionIQ and CIRR when varying $\alpha$, while CIRCO's performance improves when $\alpha$ is reduced to 0.8-1.0. This divergent behavior is fundamentally linked to each method's ability to generate an accurate $\Delta_{PDV}$. As demonstrated in Tables \ref{tab:fashion_iq_results} and \ref{tab:circo_cirr_results}, CIReVL consistently outperforms Pic2Word across various benchmarks, indicating its superior ability to generate a more accuraute composed query, and thus a more accurate $\Delta_{PDV}$. Consequently, increasing $\alpha$ yields greater benefits for CIReVL compared to Pic2Word.

We visualize the top-5 retrieval results using CIReVL with a ViT-B-32 backbone across three datasets (one reference image from each) under varying $\alpha$ values, as shown in Figure \ref{fig:residual_qual}\red{a}. As $\alpha$ increases, the retrieved results show stronger alignment with the prompt. Conversely, when $\alpha$ exceeds 1, the results include semantically related but unseen variations, while $\alpha$ values below 0.5 yields results opposite to the prompt's intent. For instance, ``brighter blue and sleeveless" retrieves ``dark blue with sleeves," ``plain background" yields ``natural/dark background," and ``young boy" returns ``adult" images.





\noindent{\textbf{Analysing the PDV for Image (PDV-I)}}
\label{sec:exp2}
To evaluate whether $\Delta_{PDV}$ enhances the retrieval performance of image embeddings, we conducted experiments following the protocol described in Section~\ref{sec:exp1}. We modified image embeddings by adding $\Delta_{PDV}$ scaled with $\alpha$ values ranging from -0.5 to 2.0, where $\alpha=0$ represents the original image-only embeddings. As shown in Figure \ref{fig:residual_image_sub}, Recall@K exhibits a positive correlation with $\alpha$ for values below 1. This upward trend continues until $\alpha=2.0$ for CIReVL, while Pic2Word's performance peaks when $\alpha$ reaches 1.4.  The performance of PDV-I was evaluated on the CIRR and CIRCO datasets by comparing it with other visual embedding-based methods, as detailed in Table \ref{tab:circo_cirr_results_pdv-I}. The results reveal that PDV-I achieved marginal improvements over existing approaches.

Following the methodology in Section~\ref{sec:exp1}, we conduct similar visualizations, with results shown in Figure \ref{fig:residual_qual}\red{b}. As with PDV-T, increasing $\alpha$ leads to stronger alignment between retrieved results and the prompt. When $\alpha$ exceeds 0.5, the results exhibit semantic relationships to the query, while $\alpha$ values below 0.5 yield results opposing the prompt's intent.
Notably, PDV-I's top retrievals demonstrate higher visual similarity to reference images compared to PDV-F, as evidenced by the preserved design elements in the clothing item (left) and laptop (middle). This characteristic is particularly valuable for applications include fashion search \cite{wu2021fashion} and logo retrieval \cite{tursun2019component}, where visual similarity plays a crucial role.



\begin{figure*}[!tbh]
	\centering
	\includegraphics[width=0.825\linewidth]{images/qualitative/PV_qual_all_mini.pdf}
	\caption{Visualisation of the impact of $\alpha$/$\beta$ scaling on top-5 retrieval results. CIReVL with ViT-B-32 Clip model is the baseline method used. Representative examples with prompts from three datasets: FashionIQ (left), CIRR (middle), and CIRCO (right) are shown at the top. \textbf{\textcolor{boxgreen}{Green}} and \textbf{\textcolor{boxblue}{blue}} bounding boxes indicate true positives and near-true positives, respectively.}
	\label{fig:residual_qual}
	
\end{figure*}

\noindent{\textbf{Analysing PDV Fusion (PDV-F)}}
\label{sec:exp3}
Finally, we evaluate the effectiveness of fusing image and text-composed embeddings by varying the fusion parameter, $\beta$, from 0 to 1 while maintaining $\alpha=1$
for both PDV-I and PDV-F. At $\beta=0$, the model relies solely on composed image embeddings, while at $\beta=1$, it uses only composed text embeddings. As shown in Figure \ref{fig:residual_fusion_sub}, the fusion of both embeddings consistently outperforms using either embedding type alone. Optimal retrieval performance is typically achieved when $\beta$ is between 0.4 and 0.8.

We similarly visualize the top-5 retrieved results across different $\beta$ values. As shown in Figure \ref{fig:residual_qual}\red{c}, when $\beta$ is small, the retrieved results maintain high visual similarity to the reference image. Conversely, as $\beta$ exceeds 0.5, the results demonstrate stronger semantic alignment with the prompt.



\subsection{ZS-CIR Benchmark Comparison}






\begin{table*}
	\centering
	\footnotesize
	\setlength{\tabcolsep}{4pt}
	\begin{tabular}{l|l|c|cccc|cccc|ccc}
		\hline
		\multicolumn{2}{c|}{\textbf{Dataset}} & & \multicolumn{4}{c|}{\textbf{CIRCO}} & \multicolumn{7}{c}{\textbf{CIRR}} \\
		\hline
		& Metric & & \multicolumn{4}{c|}{mAP@k} & \multicolumn{4}{c|}{Recall@k} & \multicolumn{3}{c}{$R_s$@k} \\
		\cline{2-14}
		Arch & Method & $\alpha_I$ & k=5 & k=10 & k=25 & k=50 & k=1 & k=5 & k=10 & k=50 & k=1 & k=2 & k=3 \\
		\hline
		\multirow{6}{*}{ViT-B/32} 
		& Image-only \textdagger & - & 1.34 & 1.60 & 2.12 & 2.41 & 6.89 & 22.99 & 33.68 & 59.23 & 21.04 & 41.04 & 60.31 \\
		& Text-only \textdagger & - & 2.56 & 2.67 & 2.98 & 3.18 & 21.81 & 45.22 & 57.42 & 81.01 & 62.24 & 81.13 & 90.70 \\
		& Image + Text \textdagger & - & 2.65 & 3.25 & 4.14 & 4.54 & 11.71 & 35.06 & 48.94 & 77.49 & 32.77 & 56.89 & 74.96 \\
		& SEARLE + \textbf{PDV-I} & 1.5 & 4.77 & 5.23  & 6.31 & 6.82 & 16.65 & 42.53 & 55.16 & 81.42 & 44.68 & 67.78 & 82.94\\
		& CIReVL + \textbf{PDV-I} & 2.0 & \textbf{10.29 }& \textbf{10.80} & \textbf{12.23} & \textbf{12.93} & \textbf{27.18} & \textbf{56.53} & \textbf{67.76} & \textbf{87.64} & \textbf{59.81} & \textbf{79.59} & \textbf{90.15}\\
		& LDRE + \textbf{PDV-I} & 2.0 & 8.00 & 8.88 & 10.06 & 10.72 & 23.37 & 51.21 & 63.69 & 85.57 & 55.57 & 76.63 & 88.15\\
		\hline
	\end{tabular}
	\caption{PDV-I performance on CIRCO and CIRR test datasets. Note that the image-only approach utilizes the visual embedding of the reference image, whereas the text-only approach employs the text embedding of the prompt.}
	\label{tab:circo_cirr_results_pdv-I}
\end{table*}

We evaluated PDV-F alongside four baseline approaches (CIReVL, LDRE, Pic2Word, and SEARLE) across three benchmarks. Notably, CIReVL was tested with three different backbones on three datasets, as its models and intermediate results are publicly available. However, for the remaining methods, we conducted partial evaluations due to limited open-source availability or restricted support.

The numerical results are presented in Tables \ref{tab:fashion_iq_results} and \ref{tab:circo_cirr_results}.
On the FashionIQ benchmark, PDV-F yields substantial improvements for all baseline approaches, with CIReVL showing particularly strong gains that scale with backbone size. Similarly, all methods demonstrate significant performance improvements on CIRCO and CIRR datasets. Notably, CIReVL achieves larger improvements compared to other methods, with the most substantial gains observed when using small and medium backbone architectures. Our PDV-F implementation within the CIReVL framework consistently outperformed other state-of-the-art methods, including LinCIR and SEIZE, across most evaluation metrics. Similar to SEIZE, PDV-F offers the advantage of being entirely training-free; however, unlike SEIZE, it does not significantly increase feature extraction computational costs. While LinCIR demonstrates exceptional inference speed, it lacks the training-free nature of our approach, requiring dedicated model training before deployment.  






%\vspace{-0.2cm}
\section{Conclusion}
%\vspace{-0.1cm}
This paper theoretically demonstrates how LoRA can be improved from our theoretical analysis in both linear and nonlinear models: the alignment between LoRA's gradient update $(\bm A_t, \bm B_t)$ and the singular subspace of $\bm G^{\natural}$, and adding preconditioners.
Our theory clarifies some potential issues behind gradient alignment work and the theory-grounded algorithm, LoRA-One, obtains promising performance in practical fine-tuning benchmarks.

\section*{Acknowledgment}
F. Liu was supported by Royal Soceity KTP R1 241011 Kan Tong Po Visiting Fellowships. Y. Chen was supported in part by National Science Foundation grants CCF-2233152.

\bibliography{sample}
\bibliographystyle{ims}

\newpage
\appendix
\onecolumn
\enableaddcontentsline
\tableofcontents
\newpage
\section{Symbols and Notations}
\label{app:notation}
\begin{table}[h!]
\fontsize{8}{10}\selectfont
\centering
\renewcommand{\arraystretch}{1.2}
\begin{tabular}{c|c|l}
\hline
\textbf{Symbol} & \textbf{Dimension(s)} & \textbf{Definition} \\
\hline
$\mathcal{N}(\bm \mu, \bm \sigma)$ & - & Multivariate normal distribution with mean vector $\bm \mu$ and covariance matrix $\bm \sigma$ \\
$\mathcal{O}, o, \Omega, \Theta$ & - & Bachmann–Landau asymptotic notation \\
$\|\bm w\|_2$ & - & Euclidean norm of vector $\bm w$ \\
$\|\mathbf{M}\|_{op}$ & - & Operator norm of matrix $\mathbf{M}$ \\
$\|\mathbf{M}\|_{\rm F}$ & - & Frobenius norm of matrix $\mathbf{M}$ \\
$\langle \bm u, \bm v \rangle$ & - & Dot product of vectors $\bm u$ and $\bm v$ \\
$\mathbf{M}\odot \mathbf{N}$ & - & Hadamard product of matrix $\mathbf{M}$ and $\mathbf{N}$\\
\hline
$\bm W^\natural$ & $\mathbb{R}^{d\times k}$ & Pre-trained weight matrix\\
$\Delta$ & $\mathbb{R}^{d\times k}$ & Downstream feature shift matrix\\
$\widetilde{\bm W}^\natural$ & $\mathbb{R}^{d\times k}$ & Downstream weight matrix $\widetilde{\bm W}^\natural=\bm W^\natural+\Delta$\\
$\bm G^\natural$ & $\mathbb{R}^{d\times k}$ & The initial gradient matrix under full fine-tuning\\
$\bm A_t\,,\bm B_t$ & $\mathbb{R}^{d\times r}\,,\mathbb{R}^{r\times k}$ & Learnable low-rank adapters at step $t$\\
$\bm w^\natural_i$ & $\mathbb{R}^d$ & $i^\text{th}$ column of pre-trained weight matrix $\bm W^\natural$ \\
$\widetilde{\bm w}^\natural_i$ & $\mathbb{R}^d$ & $i^\text{th}$ column of downstream weight matrix $\widetilde{\bm W}^\natural$ \\
$\bm w_{t,i}$ & $\mathbb{R}^d$ & $i^\text{th}$ column of adapted weight matrix $\left(\bm W^\natural+\bm A_t \bm B_t\right)$ at step $t$ \\
$\Delta_{i}$ & $\mathbb{R}^d$ & $i^\text{th}$ column of downstream feature matrix $\Delta$\\
$[\bm A_t \bm B_t]_i$ & $\mathbb{R}^d$ & $i^\text{th}$ column of the product of adapters $\bm A_t \bm B_t$\\
$\widetilde{\bm X}$ & $\mathbb{R}^{N\times d}$ & Downstream data matrix\\
$\widetilde{\bm Y}$ & $\mathbb{R}^{N\times d}$ & Downstream label matrix\\
$\widetilde{\bm x}_n$ & $\mathbb{R}^d$ & $n^\text{th}$ downstream data point\\
\hline
$\mathbf{M}^{-1}$ & - & Inverse of matrix $\mathbf{M}$ \\
$\mathbf{M}^\dagger$ & - & Pseudo-inverse of matrix $\mathbf{M}$ \\
$\lambda_i\left(\mathbf{M}\right)$ & $\mathbb{R}$ & $i^\text{th}$ singular value of matrix $\mathbf{M}$ \\
$\lambda_i^*$ & $\mathbb{R}$ & $i^\text{th}$ singular value of downstream feature shift matrix $\Delta$ \\
$\kappa\left(\mathbf{M}\right)$ & $\mathbb{R}$ & The condition number of matrix $\mathbf{M}$ \\
$\kappa$ & $\mathbb{R}$ & The condition number of $\Delta$: $\kappa=\lambda_{\max}^*/\lambda^*_{\min}$ \\
$\kappa^{\natural}$ & $\mathbb{R}$ & The condition number of $\mathbf{G}^\natural$: $ \kappa^{\natural} =\lambda_{\max}\left(\mathbf{G}^\natural\right)/\lambda_{\min}\left(\mathbf{G}^\natural\right)$ \\
$\bm U_{m}\left(\mathbf{M}\right)$ & - & The left singular subspace spanned by the $m$ largest singular values of the input matrix $\mathbf{M}$\\
$\bm U_{m,\perp}\left(\mathbf{M}\right)$ & - & The left singular subspace orthogonal to $\bm U_{m}\left(\mathbf{M}\right)$\\
$\bm V_{m}\left(\mathbf{M}\right)$ & - & The right singular subspace spanned by the $m$ largest singular values of the input matrix $\mathbf{M}$\\
$\bm V_{m,\perp}\left(\mathbf{M}\right)$ & - & The right singular subspace orthogonal to $\bm V_{m}\left(\mathbf{M}\right)$\\
$\bm U_{\bm A}$ & - & The left singular matrix of the compact SVD of $\bm A$\\
$\bm U_{\bm A, \perp}$ & - & The corresponding orthogonal complement of $\bm U_{\bm A}$\\
$\bm V_{\bm A}$ & - & The right singular matrix of the compact SVD of $\bm A$\\
$\bm V_{\bm A, \perp}$ & - & The corresponding orthogonal complement of $\bm V_{\bm A}$\\
\hline
$\sigma(\,\cdot\,)$ & - & ReLU activation function \\
$\sigma'(\,\cdot\,)$ & - & The derivative of ReLU activation function \\
$c_j$ & $\mathbb{R}$ & $j^\text{th}$ Hermite coefficient of ReLU activation function \\
$\operatorname{He}_j(\,\cdot\,)$ & - & $j^\text{th}$ Hermite polynomial\\
\hline
$\nabla_{\mathbf{W}}f\left(\mathbf{W}\right)$ & - & The gradient matrix of function $f$ w.r.t. input matrix $\mathbf{W}$\\
$\widetilde{L}\left(\bm A\,,\bm B\right)$ & - & Loss function under LoRA fine-tuning\\
$L(\bm W)$ & - & Loss function under full fine-tuning \\
\hline
$N$ & - & Number of downstream data \\
$d$ & - & Input dimension of the data \\
$k$ & - & Output dimension of the label \\
$\eta\,,\eta_1\,,\eta_2$ & - & Learning rates \\
$\alpha$ & - & Random initialization scale of low-rank adapter $\bm A_0$ \\
\hline
\end{tabular}
\caption{Essential symbols and notations in this paper.}
\label{tab:notation}
\end{table}

\begin{table*}[t]
    \centering
    \small
    % \footnotesize	    
    \def\arraystretch{1.25}
    \setlength{\tabcolsep}{3pt}
    \begin{tabular}{lcr cc cccc}
        \toprule
         &  &  & \multicolumn{2}{c}{\textit{$L_2$ Loss} ($\downarrow$)} & \multicolumn{2}{c}{\textit{Accuracy} ($\uparrow$)} \\
         \cmidrule(lr){4-5}\cmidrule(lr){6-9}
         \textbf{Objective} & $q_\varphi$ & \textbf{Model} & \multicolumn{2}{c}{\textbf{Linear Regression}} & \multicolumn{4}{c}{\textbf{Linear Classification}} \\
         \cmidrule(lr){4-5}\cmidrule(lr){6-9}
         & & & \textit{2D} & \textit{100D} & \textit{2D-2cl} & \textit{2D-5cl} & \textit{100D-2cl} & \textit{100D-5cl} \\
         \midrule

\multirow{4}{*}{Baseline} & - & Random & $4.178$\sstd{$0.018$} & $202.601$\sstd{$0.321$} & $50.498$\sstd{$0.357$} & $19.891$\sstd{$0.028$} & $50.046$\sstd{$0.047$} & $20.054$\sstd{$0.053$} \\
& - & Optimization & $0.257$\sstd{$0.000$} & $25.083$\sstd{$0.006$} & $96.982$\sstd{$0.000$} & $93.449$\sstd{$0.002$} & $70.258$\sstd{$0.012$} & $41.338$\sstd{$0.012$} \\
& - & Langevin & $0.263$\sstd{$0.002$} & $23.340$\sstd{$0.689$} & $95.034$\sstd{$0.412$} & $88.277$\sstd{$0.290$} & $65.123$\sstd{$0.370$} & $32.544$\sstd{$0.422$} \\
& - & HMC & $0.263$\sstd{$0.001$} & $18.659$\sstd{$0.189$} & $92.659$\sstd{$0.344$} & $82.169$\sstd{$0.518$} & $62.145$\sstd{$0.245$} & $29.582$\sstd{$0.371$} \\
\cmidrule{2-9}

\multirow{3}{*}{Fwd-KL} & \multirow{6}{*}{\rotatebox[origin=c]{90}{Gaussian}} & GRU & \highlight{$0.264$\sstd{$0.001$}} & $124.823$\sstd{$0.135$} & $81.170$\sstd{$0.389$} & $71.170$\sstd{$0.275$} & $59.740$\sstd{$0.102$} & $23.042$\sstd{$0.246$} \\
& & DeepSets &$0.264$\sstd{$0.000$} & $123.133$\sstd{$1.080$} & $81.281$\sstd{$0.278$} & $70.993$\sstd{$0.191$} & $50.047$\sstd{$0.051$} & $20.053$\sstd{$0.045$} \\
& & Transformer &$0.264$\sstd{$0.000$} & $45.856$\sstd{$1.331$} & $80.960$\sstd{$0.285$} & $71.484$\sstd{$0.437$} & $62.954$\sstd{$0.062$} & $26.789$\sstd{$0.110$} \\
\cmidrule{3-9}

\multirow{3}{*}{Rev-KL} & & GRU & \highlight{$0.263$\sstd{$0.000$}} & $60.215$\sstd{$0.866$} & $94.258$\sstd{$0.034$} & $87.339$\sstd{$0.023$} & $63.465$\sstd{$0.307$} & $28.270$\sstd{$0.462$} \\
& & DeepSets & \highlight{$0.263$\sstd{$0.000$}} & $62.837$\sstd{$0.617$} & $94.285$\sstd{$0.116$} & $87.342$\sstd{$0.021$} & $60.867$\sstd{$0.265$} & $21.339$\sstd{$0.085$} \\
& & Transformer & \highlight{$0.264$\sstd{$0.001$}} & \highlight{$28.735$\sstd{$0.252$}} & $94.302$\sstd{$0.054$} & $87.540$\sstd{$0.117$} & $68.185$\sstd{$0.007$} & \highlight{$32.950$\sstd{$0.284$}} \\
\cmidrule{2-9}

\multirow{3}{*}{Fwd-KL} & \multirow{6}{*}{\rotatebox[origin=c]{90}{Flow}} & GRU & \highlight{$0.264$\sstd{$0.001$}} & $119.119$\sstd{$0.233$} & $96.305$\sstd{$0.008$} & $88.927$\sstd{$0.200$} & $59.920$\sstd{$0.221$} & $23.025$\sstd{$0.077$} \\
& & DeepSets & \highlight{$0.264$\sstd{$0.001$}} & $125.677$\sstd{$3.731$} & $96.191$\sstd{$0.021$} & $88.643$\sstd{$0.102$} & $50.061$\sstd{$0.021$} & $20.021$\sstd{$0.094$} \\
& & Transformer &$0.264$\sstd{$0.000$} & $43.272$\sstd{$2.700$} & \highlight{$96.344$\sstd{$0.059$}} & \highlight{$89.624$\sstd{$0.215$}} & $64.349$\sstd{$0.147$} & $26.952$\sstd{$0.203$} \\
\cmidrule{3-9}

\multirow{3}{*}{Rev-KL} & & GRU & \highlight{$0.263$\sstd{$0.000$}} & $61.295$\sstd{$1.008$} & $95.241$\sstd{$0.012$} & $88.429$\sstd{$0.024$} & $64.669$\sstd{$0.207$} & $28.409$\sstd{$1.167$} \\
& & DeepSets & \highlight{$0.263$\sstd{$0.001$}} & $76.412$\sstd{$2.038$} & $95.296$\sstd{$0.021$} & $88.464$\sstd{$0.061$} & $58.384$\sstd{$0.812$} & $21.569$\sstd{$0.117$} \\
& & Transformer & \highlight{$0.263$\sstd{$0.000$}} & \highlight{$29.358$\sstd{$1.569$}} & $95.339$\sstd{$0.063$} & $88.644$\sstd{$0.047$} & \highlight{$68.721$\sstd{$0.121$}} & \highlight{$33.107$\sstd{$0.333$}} \\
\bottomrule
    \end{tabular}
    \caption{\textbf{Fixed-Dimensional}. Results for estimating the parameters of linear regression (LR) and classification (LC) models with the expected $L_2$ loss and accuracy according to the posterior predictive as metrics.}
    \vspace{-4mm}
    \label{tab:}
\end{table*}

% \begin{table*}[t]
%     \centering
%     \small
%     % \footnotesize	    
%     \def\arraystretch{1.25}
%     \setlength{\tabcolsep}{5pt}
%     \begin{tabular}{lcr cc cccc}
%         \toprule
%          &  &  & \multicolumn{6 }{c}{\textit{Conditional Negative Log Likelihood} ($\downarrow$)} \\
%          \cmidrule(lr){4-9}
%          \textbf{Objective} & $q_\varphi$ & \textbf{Model} & \textit{2D} & \textit{100D} & \textit{2D-2cl} & \textit{2D-5cl} & \textit{100D-2cl} & \textit{100D-5cl} \\
%          \midrule

% \multirow{4}{*}{Baseline} & - & Random & $792.2$\sstd{$3.6$} & $37777.0$\sstd{$65.5$} & $109.7$\sstd{$0.7$} & $234.4$\sstd{$1.5$} & $532.4$\sstd{$0.5$} & $1097.9$\sstd{$2.8$} \\
% & - & Optimization & $69.2$\sstd{$0.0$} & $3980.9$\sstd{$0.9$} & $9.2$\sstd{$0.0$} & $26.2$\sstd{$0.0$} & $261.4$\sstd{$0.1$} & $442.3$\sstd{$0.0$} \\
% & - & Langevin & $70.3$\sstd{$0.3$} & $3671.3$\sstd{$110.7$} & $14.1$\sstd{$0.2$} & $36.2$\sstd{$0.2$} & $228.3$\sstd{$6.3$} & $585.8$\sstd{$5.3$} \\
% & - & HMC & $70.2$\sstd{$0.2$} & $3093.6$\sstd{$27.5$} & $27.7$\sstd{$0.5$} & $71.9$\sstd{$0.7$} & $105.3$\sstd{$0.2$} & $255.9$\sstd{$2.9$} \\
% \cmidrule{2-9}

% \multirow{3}{*}{Fwd-KL} & \multirow{6}{*}{\rotatebox[origin=c]{90}{Gaussian}} & GRU &$70.4$\sstd{$0.1$} & $22964.7$\sstd{$43.5$} & $38.5$\sstd{$0.5$} & $72.9$\sstd{$0.5$} & $397.1$\sstd{$1.0$} & $1013.3$\sstd{$5.4$} \\
%  & & DeepSets &$70.4$\sstd{$0.1$} & $22802.5$\sstd{$191.9$} & $38.4$\sstd{$0.5$} & $73.3$\sstd{$0.4$} & $532.4$\sstd{$0.5$} & $1098.0$\sstd{$2.8$} \\
%  & & Transformer &$70.4$\sstd{$0.0$} & $8002.2$\sstd{$250.1$} & $38.8$\sstd{$0.5$} & $72.0$\sstd{$0.8$} & $349.9$\sstd{$0.6$} & $912.3$\sstd{$2.7$} \\
% \cmidrule{3-9}

% \multirow{3}{*}{Rev-KL} & & GRU &$70.2$\sstd{$0.0$} & $11089.1$\sstd{$153.0$} & $14.1$\sstd{$0.0$} & $34.9$\sstd{$0.2$} & $265.1$\sstd{$1.9$} & $603.6$\sstd{$2.8$} \\
%  & & DeepSets &$70.2$\sstd{$0.1$} & $11604.3$\sstd{$110.4$} & $14.1$\sstd{$0.2$} & $35.1$\sstd{$0.2$} & $306.9$\sstd{$2.1$} & $497.8$\sstd{$22.8$} \\
%  & & Transformer &$70.3$\sstd{$0.1$} & $5111.2$\sstd{$49.1$} & $14.0$\sstd{$0.1$} & $34.3$\sstd{$0.3$} & $263.5$\sstd{$0.8$} & $664.5$\sstd{$8.1$} \\
% \cmidrule{2-9}

% \multirow{3}{*}{Fwd-KL} & \multirow{6}{*}{\rotatebox[origin=c]{90}{Flow}} & GRU &$70.3$\sstd{$0.1$} & $21890.2$\sstd{$58.5$} & $23.8$\sstd{$0.2$} & $57.6$\sstd{$0.5$} & $366.8$\sstd{$2.6$} & $1007.0$\sstd{$2.5$} \\
%  & & DeepSets &$70.4$\sstd{$0.2$} & $23320.5$\sstd{$691.7$} & $23.5$\sstd{$0.6$} & $58.0$\sstd{$0.7$} & $532.1$\sstd{$0.3$} & $1099.1$\sstd{$2.7$} \\
%  & & Transformer &$70.4$\sstd{$0.1$} & $7519.8$\sstd{$507.3$} & $23.5$\sstd{$0.1$} & $56.0$\sstd{$0.5$} & $297.1$\sstd{$1.0$} & $889.7$\sstd{$5.2$} \\
% \cmidrule{3-9}

% \multirow{3}{*}{Rev-KL} & & GRU &$70.3$\sstd{$0.1$} & $11306.4$\sstd{$209.1$} & $13.2$\sstd{$0.0$} & $33.5$\sstd{$0.2$} & $196.7$\sstd{$2.4$} & $556.5$\sstd{$10.3$} \\
%  & & DeepSets &$70.2$\sstd{$0.1$} & $14107.0$\sstd{$354.7$} & $13.1$\sstd{$0.1$} & $33.4$\sstd{$0.3$} & $181.8$\sstd{$10.9$} & $442.8$\sstd{$9.9$} \\
%  & & Transformer &$70.3$\sstd{$0.1$} & $5225.3$\sstd{$292.6$} & $13.0$\sstd{$0.1$} & $33.1$\sstd{$0.1$} & $209.2$\sstd{$1.5$} & $619.8$\sstd{$4.4$} \\
% \bottomrule
%     \end{tabular}
%     \caption{}
%     \vspace{-4mm}
%     \label{tab:}
% \end{table*}

\section{Analysis of LoRA under Nonlinear Models}
\label{sec:nonlinear}


Now we focus on the nonlinear setting described in \cref{sec:problsemsetting}, where we consider the exact-rank case $r=r^*$ for delivery.
We will demonstrate the linear convergence rate in the linear setting can still hold for the nonlinear setting.

Following \cref{sec:scaledgd}, we continue to consider preconditioned GD on $(\bm A_t, \bm B_t)$ with the same step-size $\eta$ for convenience:
\begin{equation}\label{eq:ABiter_nonlinear}
\begin{split}
     \bm A_{t+1} & = \bm A_t - \eta \nabla_{\bm A} \widetilde{L}\left(\bm A_t\,,\bm B_t\right)\left(\bm B_t \bm B_t^{\!\top}\right)^{-1}\,, \\
     \bm B_{t+1}  & = \bm B_t - \eta \left(\bm A_t^{\!\top} \bm A_t\right)^{-1} \nabla_{\bm B} \widetilde{L}\left(\bm A_t\,,\bm B_t\right)\,.
\end{split}
\end{equation}
Notice that here we use standard matrix inversion since we can prove that $\bm A_t$ and $\bm B_t$ stay non-singular across all $t\geq 0$.
By denoting $\bm W_t := \bm W^{\natural} + \bm A_t \bm B_t$, we have the gradients
\begin{align*}
\nabla_{\bm A}\widetilde{L}\left(\bm A_t\,,\bm B_t\right) = -\bm J_{
\bm W_t} \bm B_t^{\!\top}, \quad 
\nabla_{\bm B}\widetilde{L}\left(\bm A_t\,,\bm B_t\right) = -\bm A_t^{\!\top} \bm J_{
\bm W_t} \,,
\end{align*}
where we define 
\begin{align*}
    \bm J_{
\bm W_t} := \frac{1}{N}\widetilde{\bm X}^{\!\top}\left[\sigma(\widetilde{\bm X}\widetilde{\bm W}^\natural) - \frac{1}{N}\widetilde{\bm X}^{\!\top}\sigma(\widetilde{\bm X}\bm W_t)\right]\odot \sigma'(\widetilde{\bm X}\bm W_t)\,.
\end{align*}

To deliver the proof, apart from the above-mentioned assumptions in \cref{sec:assumptions} for the the nonlinear setting, we also need the following additional assumptions.
\begin{assumption}\label{assum:nonlinear-orth}
  We assume that $\bm W^\natural$ has orthonormal columns and its row space is orthogonal to that of $\Delta$.
\end{assumption}

\begin{assumption}\label{assum:nonlinear-delta}
We assume that $\|\Delta\|_{op}\leq \frac{\sqrt{2}-1}{2}$ with $\operatorname{Rank}(\Delta)=r^*$ where $k+r^*\leq d$ and $r^* \ll \min\{d\,,k\}$.
\end{assumption}

\noindent
{\bf Remark:} 
\cref{assum:nonlinear-orth} ensures rich task diversity between pre-trained model and downstream tasks. We notice that such assumption is also considered in \cite{dayi2024gradientdynamicslowrankfinetuning}.  \cref{assum:nonlinear-delta} restricts the norm of downstream feature shift since the signal of adapted weight is generally smaller than the pre-trained weight. We can empirically assess the validity of this assumption in \cref{fig:SV}.


Here we can show that, for the nonlinear model, LoRA training can achieve global linear convergence under \eqref{eq:spectral-init-linear} via preconditioned GD in \cref{eq:ABiter_nonlinear}.
\begin{theorem}[Simplified version of \cref{LC}]\label{main:LC}
    Under assumptions in \cref{sec:assumptions} for the nonlinear setting, \cref{assum:nonlinear-orth}, and \ref{assum:nonlinear-delta}, with training conducted by \cref{eq:ABiter_nonlinear} and initialization via \eqref{eq:spectral-init-linear},  we take $\epsilon = \mathcal{O}\left(\frac{1}{r^*\kappa\sqrt{d}}\right)$ and $\rho\leq 0.01$ such that
    we set
    \begin{align}\label{main:selection-scale}
        \gamma\in\left[\frac{1}{c_{\rm H}}-\frac{\rho}{3c_{\rm H}\kappa\sqrt{2r^*}}\,,\frac{1}{c_{\rm H}}+\frac{\rho}{3c_{\rm H}\kappa\sqrt{2r^*}}\right]
    \end{align}
    with $c_{\rm H} :=\frac{1}{4} + \frac{1}{4\pi}\sum_{\substack{n\geq 1, \\ n \text{ odd}}} 2^{-n} n^{-2} (n!)^{-2}$.
    Then choosing $\eta \in \left(c_{\eta}\,,\frac{1}{2c_{\rm H}}\right)$ for a small constant $c_{\eta}>0$, with probability at least $1-2Cdk\operatorname{exp}\left(-\epsilon^2 N\right)$ for a universal constant $C>0$, we have
    \begin{align}\label{eq:atbtnon}
            \left\|\bm A_{t}\bm B_{t} - \Delta\right\|_{\rm F}  \leq \left(1-\frac{c_{\rm H}}{10}\eta\right)^t \rho\lambda^*_{r^*}\,, \forall t \geq 0\,.
        \end{align}
\end{theorem}

\noindent
{\bf Remark:} We make the following remarks:
\begin{itemize}
    \item This theorem is based on $\left\|\bm A_0 \bm B_0 - \Delta\right\|_{\rm F} \leq \rho\lambda^*_{r^*}$ at initialization (\cref{assum:nonlinear-delta} is not needed), see \cref{A0B0-init-risk} for details, which demonstrates the ability of one-step full gradient can improve feature learning.
    \item The final rate is independent of condition number $\kappa$ of downstream feature shift $\Delta$, which coincides with the results from linear model. This provide us evidence that adding preconditioners can also work for nonlinear model.
\end{itemize}

\noindent 
{\bf Proof Sketch:} The complete proof can be found in \cref{PGD-Nonlinear}. By Hermite decomposition, we can compute the expectation of $\bm J_{\bm W_t}$ (see \cref{expec-grad}) and decompose $\bm J_{\bm W_t}$ into $c_{\rm H}\left(\bm A_{t}\bm B_{t} - \Delta\right) + \bm \Xi_t$,
where $\bm \Xi_t$ is defined in \cref{Lip}. The first term is the signal term which can dominate the preconditioned GD dynamics. The second term $\bm \Xi_t := T1 +T2$ consists of two parts (details see \cref{err-concen-pop}): the first part $T1$ is the higher-order residual terms from $\mathbb{E}_{\widetilde{\bm x}}\left[\bm J_{\bm W_t}\right]$ which related to Hermite decomposition. Since the decay of Hermite coefficients of $\sigma$ is faster than polynomial decay, it can be well controlled. For the second term $T2$, it comes from the concentration error of $\bm J_{\bm W_t}$, which can also controlled by large sample size $N$.

To handle $\| \bm A_t \bm B_t - \Delta \|_{\rm F}$, we explore its recursion relationship in \cref{Lip}. The key part is to control $\left\|\left(\bm I_d - \bm U_{\bm A_t} \bm U_{\bm A_t}^{\!\top}\right)\Delta\left(\bm I_k - \bm V_{\bm B_t} \bm V_{\bm B_t}^{\!\top}\right)\right\|_{\rm F}$ as well as its complementary part in \cref{basis-alignment} and higher order term in \cref{err-cross}.  We deliver the complete proof to \cref{PGD-Nonlinear}. 

Note that the above two assumptions are not required if we modify the gradient update of \cref{eq:ABiter_nonlinear} by removing the mask matrix $\sigma'(\widetilde{\bm X}\bm W_t)$, a smoothing technique from \cite{kalai2009isotron,kakade2011efficient,wu2023finite}, i.e.,
\begin{align*}
    \bm J_{\bm W_t}^{\tt GLM} := \frac{1}{N}\widetilde{\bm X}^{\!\top}\bigg(\sigma\left(\widetilde{\bm X}\widetilde{\bm W}^\natural\right)-\sigma\left(\widetilde{\bm X}\bm W_t\right)\bigg)\,.
\end{align*}
In this case, we reformulate \cref{eq:ABiter_nonlinear} as
\begin{equation}\label{ABiter_GLM}
    \begin{split}
        \bm A_{t+1} & = \bm A_t + \eta \bm J_{\bm W_t}^{\tt GLM}\bm B_t^{\!\top}\left(\bm B_t\bm B_t^{\!\top}\right)^{-1}\,,\\
        \bm B_{t+1} & = \bm B_t + \eta \left(\bm A_t^{\!\top}\bm A_t\right)^{-1}\bm A_t^{\!\top}\bm J_{\bm W_t}^{\tt GLM}\,.
    \end{split}
\end{equation}
And we propose to use $\bm G^\natural := \bm J_{\bm W^\natural}^{\tt GLM}$ for \eqref{eq:spectral-init-linear} to initialize $\bm A_0$ and $\bm B_0$. The global linear convergence results are given as below.
\begin{theorem}[Simplified version of \cref{smo-LC}]\label{main:smo-LC}
    Under assumptions in \cref{sec:assumptions} for the nonlinear setting, with training conducted by \cref{ABiter_GLM} and initialization via \eqref{eq:spectral-init-linear} by taking $\bm G^\natural := \bm J_{\bm W^\natural}^{\tt GLM}$, suppose $\epsilon = \mathcal{O}\left(\frac{1}{r^* \kappa \sqrt{d}}\right)$ and $\rho\leq\frac{1}{20}$, we take 
    \begin{equation*}
       \gamma\in\left[2-\frac{2\rho}{3\kappa\sqrt{2r^*}}\,,2+\frac{2\rho}{3\kappa\sqrt{2r^*}}\right]\,,
    \end{equation*}
   and set $\eta \in \left(c^{\tt GLM}_{\eta}\,,1\right)$ where $c^{\tt GLM}_{\eta}>0$ is a small constant, then with probability at least $1-2Cdk\operatorname{exp}\left(-\epsilon^2 N\right)$ for a universal constant $C>0$, we have
    \begin{align*}
            \left\|\bm A_{t}\bm B_{t} - \Delta\right\|_{\rm F} & \leq \left(1-\frac{\eta}{4}\right)^t \rho\lambda^*_{r^*}\,.
        \end{align*}
\end{theorem}

\noindent
{\bf Remark:} Removing the mask matrix $\sigma'(\widetilde{\bm X}\bm W_t)$ in \cref{eq:ABiter_nonlinear} allows for better linear convergence performance with $(1-\eta/4)^t$ than that in \cref{eq:atbtnon}, albeit without \cref{assum:nonlinear-orth}, \ref{assum:nonlinear-delta}.\\

\noindent
{\bf Proof Sketch:} The proof strategy is similar to \cref{main:LC}. The key difference is that the corresponding $\bm \Xi_t$ under \cref{ABiter_GLM} does not have the residual terms from Hermite decomposition. We deliver the complete proof to \cref{prec-smooth-gd}.

\section{Auxiliary Results for Proofs}
\label{auxiliary}
In this subsection, we present some auxiliary results that are needed for our proof.
First, we present the estimation of the spectral norm of random matrices.
It can be easily derived from \cite{vershynin2018high} and we put it here for the completeness.

\begin{lemma}\citep[Adapted from Theorem 4.6.1]{vershynin2018high}
\label{lem:conrg}
    For a random sub-Gaussian matrix $\widetilde{\bm X} \in \mathbb{R}^{N \times d}$ whose rows are i.i.d. isotropic sub-gaussian random vector with sub-Gaussian norm $K$, then we have the following statement
\[
\mathbb{P} \left(   \left\|\frac{1}{N}\widetilde{\bm X}^{\!\top}\widetilde{\bm X}-\bm I_d\right\|_{op}  > \delta \right) \leq 2 \exp \left( -C N \min\left(\delta^2, \delta\right) \right)\,.
\]
for a universal constant $C$ depending only on $K$.
\end{lemma}

\begin{lemma}\citep[Adapted from Corollary 5.35]{vershynin2010introduction}
\label{lem:init-op-conct}
    For a random standard Gaussian matrix $\bm S\in\mathbb{R}^{d\times r}$ with $[\bm S]_{ij} \sim \mathcal{N}(0, 1)$, if $d > 2r$, we have 
    \begin{align}
        \label{norm-A0}
        \frac{\sqrt{d}}{2} \leq \|\bm S\|_{op} \leq (2 \sqrt{d} + \sqrt{r})\,,
    \end{align}
    with probability at least $1-C \operatorname{exp}(-d)$ for some positive constants $C$.
\end{lemma}

The following results are modified from the proof of \citet[Lemma 8.7]{stoger2021small}.
\begin{lemma}
\label{lem:min-singular-conct}
    Suppose $\bm S\in\mathbb{R}^{d\times r}$ is a random standard Gaussian matrix with $[\bm S]_{ij} \sim \mathcal{N}(0, 1)$ and $\bm U\in\mathbb{R}^{d\times r^*}$ has orthonormal columns. If $r\geq 2r^*$, with probability at least $1-C\operatorname{exp}(-r)$ for some positive constants $C$, we have
    \begin{align*}
        \lambda_{\operatorname{min}}(\bm U^{\!\top}\bm S) & \gtrsim 1\,.
    \end{align*}
    If $r^*\leq r < 2r^*$, by choosing $\xi>0$ appropriately, with probability at least $1-(C \xi)^{r-r^*+1}-C'\operatorname{exp}(-r)$ for some positive constants $C\,,C'$, we have
    \begin{align*}
        \lambda_{\operatorname{min}}(\bm U^{\!\top}\bm S) & \gtrsim \frac{\xi}{r}\,.
    \end{align*}
\end{lemma}

Next, we give a short description of the Hermite expansion of ReLU function via Hermite polynomials. Details can be found in \citet[A.1.1]{damian2022neural} and \cite{arous2021online}.
To be specific, the Hermite expansion of ReLU function $\sigma(x)$ is
\begin{align}
\label{Hermite-sigma}
    \sigma(x)=\sum_{j=1}^\infty \frac{c_j}{j!}\operatorname{He}_j(x) =\frac{1}{\sqrt{2\pi}}+\frac{1}{2}x+\frac{1}{\sqrt{2\pi}}\sum_{j\geq 1}\frac{(-1)^{j-1}}{j!2^j(2j-1)}\operatorname{He}_{2j}(x)\,,
\end{align}
which implies that we can express the Hermite coefficients as
\begin{align}
\label{Hermite-coef}
    \left\{\begin{aligned}
        c_0 & = \frac{1}{\sqrt{2\pi}}\,,\\
        c_1 & = \frac{1}{2}\,,\\
        c_{2j} & = \frac{(-1)^{j-1}}{\sqrt{2\pi}2^j(2j-1)}\quad \text{for }j\geq 1\,.
    \end{aligned}\right.
\end{align}
Furthermore, the derivative of $\sigma(x)$ admits
\begin{align}
\label{Hermite-sigma'}
    \sigma'(x)=\frac{1}{2}+\frac{1}{\sqrt{2\pi}}\sum_{j\geq 0}\frac{(-1)^{j}}{j!2^j(2j+1)}\operatorname{He}_{2j+1}(x)\,.
\end{align}

\begin{lemma}\citep[Corollary 9]{oko2024pretrained}\label{differential}
$\mathbb{E}_{\widetilde{\bm x}}[\nabla^k \sigma(\langle \bm w\,, \widetilde{\bm x}\rangle)] = c_k \bm w^{\otimes k}$ for any $k$ such that $c_k\neq 0$.
\end{lemma}

\begin{lemma}\label{vec-ineq}
For any vectors $\bm u$ and $\bm v$, we have
    \begin{align*}
        \left|\langle \bm u\,, \bm u \rangle^j - \langle \bm u\,, \bm v \rangle^j\right| & \leq j\,\max\left\{\left\|\bm u\right\|_2\,,\left\|\bm v\right\|_2\right\}^{2j-1} \left\|\bm u - \bm v\right\|_2\,.
    \end{align*}
\end{lemma}
\begin{proof}
    First, we analyze the following two scalar variables case
    \begin{align*}
        \left|x^j-y^j\right|\,.
    \end{align*}
    By algebraic identity $\sum_{j=1}^{t-1}x^{t-j-1}y^j=\frac{x^t-y^t}{x-y}$ which is valid for $\forall\,j\in\mathbb{N}^+$, we have
    \begin{align*}
        \left|x^j-y^j\right|&=\left|(x-y)\sum_{i=0}^{j-1}x^{j-i-1}y^i\right|
        \leq |x-y|\sum_{i=0}^{j-1}\max\left\{|x|\,,|y|\right\}^{j-1}
        = j|x-y|\max\left\{|x|\,,|y|\right\}^{j-1}\,.
    \end{align*}
    Now we define $x:=\langle \bm u\,, \bm u \rangle$ and $y:=\langle \bm u\,, \bm v \rangle$, then we can obtain
    \begin{align*}
        \left|\langle \bm u\,, \bm u \rangle^j - \langle \bm u\,, \bm v \rangle^j\right| & \leq j\,\max\left\{\left|\langle \bm u\,, \bm u \rangle\right|\,,\left|\langle \bm u\,, \bm v \rangle\right|\right\}^{j-1}\left|\langle \bm u\,, \bm u \rangle - \langle \bm u\,, \bm v \rangle\right|\\
        & \leq j\,\max\left\{\left\|\bm u\right\|_2^2\,,\left\|\bm u\right\|_2 \left\|\bm v\right\|_2\right\}^{j-1}\left\|\bm u\right\|_2 \left\|\bm u - \bm v\right\|_2\quad \tag*{\color{teal}[by Cauchy-Schwartz inequality]}\\
        & = j\,\max\left\{\left\|\bm u\right\|_2\,,\left\|\bm v\right\|_2\right\}^{2j-1} \left\|\bm u - \bm v\right\|_2\,.
    \end{align*}
\end{proof}

\section{Discussion on Prior Work Based on Gradient Alignment}
\label{app:disGA}

Our initialization strategy in \cref{alg:lora_one_training} (line 4-6) shares some similarity with prior work on gradient alignment, e.g., LoRA-GA \citep{wang2024lora}, and LoRA-pro \citep{wang2024lorapro}.
However, the motivation behind these gradient alignment work differs significantly from ours. The above gradient alignment based algorithms are driven by how to approximate the full fine-tuning gradient by low-rank updates. Instead, our our work is motivated by which subspace $(\bm A_t, \bm B_t)$ will align with and then how to achieve this alignment efficiently so as to finally recover $\Delta$.

Here we take LoRA-GA as an example to explain the potential issue that the spirit of LoRA-GA might not help recover $\Delta$, both theoretically and empirically.
To be specific, LoRA-GA \citep{wang2024lora} also computes the SVD of $\nabla_{\bm W} {L}(\bm W^\natural)$. To ensure the pre-trained model remains unchanged at $t=0$, LoRA-GA the following strategy
\begin{equation}\tag{LoRA-GA}\label{LoRA-GA}
\begin{split}
    &\bm A_0 = -\sqrt{\gamma}\left[\widetilde{\bm U}_{\bm G^\natural}\right]_{[:,1:r]}\,,
    \bm B_0 = \sqrt{\gamma}\left[\widetilde{\bm V}_{\bm G^\natural}\right]_{[:,r+1:2r]}^{\!\top}\,,\\
    &\bm W_{\tt off}^\natural := \bm W^\natural - \frac{\alpha}{\sqrt{r}}\bm A_0 \bm B_0\,.
\end{split}
\end{equation}

Theoretically, LoRA-GA observes  $\operatorname{rank}(
\nabla_{\bm A}\widetilde{L}\left(\bm A_t\,,\bm B_t\right) + \nabla_{\bm B}\widetilde{L}\left(\bm A_t\,,\bm B_t\right)) \leq 2r$ and then proposes to find the best $2r$-rank approximation of one-step full gradient to the first step of LoRA. Accordingly, LoRA-GA chooses the first $r$ singular values for $\bm A_0$ and $(r+1)$th to $2r$th singular values for $\bm B_0$.
However, as pointed by our theory, $\bm B_t$ will also align to the right-side rank-$r^*$ singular subspace of $\bm G^{\natural}$ under random initilization. That means, due to the way LoRA-GA chooses the $(r+1)$th to $2r$th singular values for $\bm B_0$, the iterate $\bm B_t$ does not lie in the desired subspace and may not escape an undesirable subspace. 

Empirically, the mismatch of singular subspace induced by corresponding singular values in LoRA-GA might bring unfavorable performance even in a toy model. We consider the exact-ranked case ($r=r^*$) for fine-tuning task in the linear setting. 
We compare the generalization risk of three initialization strategies: \eqref{eq:spectral-init-linear}, \cref{alg:lora_one_training} without preconditioners, and LoRA-GA trained via vanilla GD. The results are shown in \cref{figs:GA-vs-Ours}. We can empirically observe that LoRA-GA fails to generalize and remain at a high-risk level throughout training. In contrast, \eqref{eq:spectral-init-linear} and \cref{alg:lora_one_training} both can generalize well. This empirically demonstrates the optimality of choosing top-$r$ singular subspace of $\bm G^\natural$.

Before the submission deadline we became aware of the concurrent work \cite{ponkshe2024initialization}, which uses the same initialization for $\bm B_0$ as in line 6 of our \cref{alg:lora_one_training}. However, the motivation, problem setting, and theoretical analysis are totally different between our work and theirs. Moreover, our \cref{alg:lora_one_training} also introduces the preconditioners and is able to efficiently handle ill-conditioned cases, and this is not available in \cite{ponkshe2024initialization}.

\section{Experimental Settings and Additional Results}
\label{exp-settings}

In \cref{exp:toy-setting}, we firstly provide the experimental details of small-scale experiments in our main text, e.g., \cref{figs:GA-vs-Ours} and \cref{fig:small-init}. 
Experimental settings of NLP tasks in the main text are given by \cref{app:expNLP}.
We also include the fine-tuning experiments on LLMs in \cref{sec:exp:llm}.
More ablation study is given by \cref{appx:ablation}. Finally, we visualize the singular values of both the pre-trained weights and the difference weights after fine-tuning in \cref{SV-figs}. All small-scale experiments were performed on AMD EPYC 7B12 CPU. All experiments for T5 base model and Llama 2-7B were performed on Nvidia A100 GPU (40GB).

\subsection{Small-Scale Experiments}
\label{exp:toy-setting}

Here we give the experimental details of \cref{figs:GA-vs-Ours} and \cref{fig:small-init}. Besides, we plot the GD trajectories under \eqref{eq:spectral-init-linear} and \eqref{eq:lorainit} for comparison.\\

\noindent
{\bf Details for \cref{figs:GA-vs-Ours}:} For the exact-ranked setting, we take $d=k=100$, $N=1600$, and $r=r^*=4$. We sample each element of $\bm W^\natural$ independently from $\mathcal{N}(0\,,1)$. We construct $\Delta:=\bm U \bm V^{\!\top}$ where $\bm U\in\mathbb{R}^{100\times 4}$ and $\bm V\in\mathbb{R}^{100\times 4}$ are obtained from the SVD of a matrix whose elements are independently sampled from $\mathcal{N}(0\,,1)$. For LoRA-One (-) and LoRA-GA (-), we use learning rate $\eta=\frac{1}{35}$ and stable parameter $s=2$. For \eqref{eq:spectral-init-linear} (-), we use learning rate $\eta=\frac{1}{10}$ and $\gamma=1$. 

For the ill-conditioned setting, we take $d=k=100$, $N=1600$, $r^*=4$, and $r=8$. We construct $\Delta:=\bm U \bm S^* \bm V^{\!\top}$ where $\bm U\in\mathbb{R}^{100\times 4}$ and $\bm V\in\mathbb{R}^{100\times 4}$ are obtained from the SVD of a matrix whose elements are independently sampled from $\mathcal{N}(0\,,1)$, and $\bm S^*=\operatorname{Diag}\left(1\,,0.75\,,0.5\,,0.25\right)$. For algorithms without preconditioners, we set the learning rate to be $\eta=\frac{1}{20}$. For algorithms with preconditioners, we set the learning rate to be $\eta=\frac{1}{2}$. For LoRA-One, LoRA-One (-), LoRA-GA (-), and LoRA-GA (+), we set the stable parameter $s=2$. For \eqref{eq:spectral-init-linear} (-) and \eqref{eq:spectral-init-linear} (+), we take $\gamma=1$. All damping parameters $\lambda$ for preconditioners are set to be 0.001.\\

\noindent
{\bf Details for \cref{fig:small-init}:} We examine for dimension $d=k=100$ and $d=k=1000$. We set $N=16d$, $r^*=4$, and $r=8$. We construct $\Delta:=\bm U \bm V^{\!\top}$ where $\bm U\in\mathbb{R}^{100\times 4}$ and $\bm V\in\mathbb{R}^{100\times 4}$ are obtained from the SVD of a matrix whose elements are independently sampled from $\mathcal{N}(0\,,1)$. We initialize $\bm A_0$ and $\bm B_0$ via \eqref{eq:lorainit} over variance $\alpha^2\in\{1\,,0.1\,,0.01\,,0.001\,,0.0001\}$. We set learning rate $\eta=\frac{1}{64}$. We run $1500$ GD steps for each case.\\

\noindent
{\bf Comparison on GD trajectories of \cref{fig:phase-transi}:}
Here we conduct a toy experiment to intuitively compare the GD trajectories under \eqref{eq:spectral-init-linear} and \eqref{eq:lorainit}. We fine-tune a simple pre-trained model $y=\bm x^{\!\top}\bm w^\natural$ on downstream data generated by $\widetilde{y}=\widetilde{\bm x}^{\!\top}(\bm w^\natural+\bm w)$, where $\bm x^{\!\top}\,,\widetilde{\bm x}\,,\bm w^\natural\,,\bm w\in\mathbb{R}^2$ and $y\,,\widetilde{y}\in\mathbb{R}$. We propose to use LoRA to fine-tune this model by $\widehat{y} = \widetilde{\bm x}^{\!\top}(\bm w^\natural+b \bm a)$ where $\bm a = [a_1\,a_2]^{\!\top}\in\mathbb{R}^2$ and $b\in\mathbb{R}$. Without loss of generality, we set $\bm w^\natural=\bm 0$ and $\bm w = [2\,\,1]^{\!\top}$. The set of global minimizers to this problem is $\{a_1^*=2/t\,,a_2^*=1/t\,,b^*=t\mid t \in \mathbb{R}\}$. We generate 4 data points $(\widetilde{\bm x}_1\,,\widetilde{\bm x}_2\,,\widetilde{\bm x}_3\,,\widetilde{\bm x}_4)$ whose elements are independently sampled from $\mathcal{N}(0\,,1)$ and calculate for $(\widetilde{y}_1\,,\widetilde{y}_2\,,\widetilde{y}_3\,,\widetilde{y}_4)$. We use the squared loss $\frac{1}{8}\sum_{i=1}^4 (\widetilde{y}_i-b\widetilde{\bm x}^{\!\top} \bm a)^2$. For \eqref{eq:lorainit}, we initialize each element of $\bm a_0$ from $\mathcal{N}(0\,,1)$ and $b_0=0$. Notice that the variance $1$ follows from the Kaiming initialization \citep{he2015delving}. For \eqref{eq:spectral-init-linear}, we first calculate the one-step full gradient, i.e. $\bm g^\natural := \frac{1}{4}\sum_{i=1}^4 \widetilde{y}_i^2 \widetilde{\bm x}_i$.
Accordingly, we initialize $\bm a_0 = \frac{\bm g^\natural}{\sqrt{\|\bm g^\natural\|_2}\,.}$ and $b_0 = \sqrt{\|\bm g^\natural\|_2}$. Next, we run GD to train $\bm a$ and $b$ for $1000$ steps with learning rate $\eta=0.1$. For each initialization strategy and data generation, we run for 3 different seeds. The starting points and stopping points with corresponding loss values are presented in \cref{tab:phase-spec} for \eqref{eq:spectral-init-linear} and \cref{tab:phase-random} for \eqref{eq:lorainit}.
Our experiments in \cref{fig:phase-transi} show that spectral initialization enables faster convergence to the global minimizer compared to LoRA initialization.
\begin{table}[h]
    \caption{The details of starting points with initial loss and stopping points with final loss under \eqref{eq:spectral-init-linear} over 3 runs.}
    \label{tab:phase-spec}
    \centering
    \begin{tabular}{ccccc}
    \toprule
         & Starting Point & Initial Loss & Stopping Point & Final Loss \\
         \midrule
       Run 1  & $\bm a=[0.26\,,0.55]^{\!\top}\,, b=0.61$ & $0.39$ & $\bm a=[1.34\,,0.67]^{\!\top}\,, b=1.49$ & \SI{5e-13}{} \\
       \midrule
       Run 2  & $\bm a=[1.10\,,-0.27]^{\!\top}\,, b=1.10$ & $0.38$ & $\bm a=[1.35\,,0.68]^{\!\top}\,, b=1.48$ & \SI{1e-13}{} \\
       \midrule
       Run 3 & $\bm a=[0.96\,,0.35]^{\!\top}\,, b=1.02$ & $0.34$ & $\bm a=[1.34\,,0.67]^{\!\top}\,, b=1.49$ & \SI{4e-13}{} \\
       \bottomrule
    \end{tabular}
\end{table}

\begin{table}[h]
    \caption{The details of starting points with initial loss and stopping points with final loss under \eqref{eq:lorainit} over 3 runs.}
    \label{tab:phase-random}
    \centering
    \begin{tabular}{ccccc}
    \toprule
         & Starting Point & Initial Loss & Stopping Point & Final Loss \\
         \midrule
       Run 1  & $\bm a=[-0.35\,,2.63]^{\!\top}\,, b=-0.03$ & $0.43$ & $\bm a=[-2.49\,,-1.24]^{\!\top}\,, b=-0.80$ & \SI{1e-10}{} \\
       \midrule
       Run 2  & $\bm a=[0.14\,,-1.68]^{\!\top}\,, b=0.10$ & $0.82$ & $\bm a=[1.81\,,0.91]^{\!\top}\,, b=1.10$ & \SI{1e-13}{} \\
       \midrule
       Run 3 & $\bm a=[-1.44\,,0.98]^{\!\top}\,, b=0.03$ & $0.97$ & $\bm a=[1.84\,,0.92]^{\!\top}\,, b=1.08$ & \SI{6e-13}{} \\
       \bottomrule
    \end{tabular}
\end{table}

\subsection{Natural Language Understanding}
\label{app:expNLP}

In the main text of \cref{sec:algoexp}, we have presented the experimental comparisons between \cref{alg:lora_one_training} and typical LoRA based algorithms. 
For experimental details, we follow the configuration of prompt tuning as \cite{wang2024lora}. The general hyperparameter settings are provides in \cref{tab:nlu-general}. Also, we employ the scaling parameter $\frac{\sqrt{d_\text{out}}}{s}$ for LoRA-One (\cref{alg:lora_one_training}) derived in \cite{wang2024lora} which is proven to be numerically stable. To ensure a fair comparison, we tune the learning rate via grid search over $\{ \SI{1e-3}{} , \SI{5e-4}{} , \SI{2e-4}{} , \SI{1e-4}{} , \SI{5e-5}{} , \SI{2e-5}{} , \SI{1e-5}{} \}$.

Furthermore, we fine-tune the model using one step update from full-batch gradient descent under full fine-tuning. To optimize GPU memory usage, we adopt the averaged gradient computation method from \cite{lv2023full, wang2024lora} to compute the full gradient, which is then manually added to the pre-trained weights, scaled by the learning rate. 

Besides, we notice that the test accuracy on the MNLI dataset remains $0.0\%$ for the first dozen steps in both full fine-tuning and LoRA fine-tuning. So we omit results on this dataset.
We conjecture that this is due to the significant discrepancy between pre-trained tasks and downstream tasks. For SST-2, CoLA, QNLI, and MRPC, the learning rates are set to be $\{ \SI{5e-4}{} , \SI{1e-2}{} , \SI{2e-2}{} , \num{0.5} \}$.

\begin{table}[h]
    \centering
     \caption{Hyperparameters for LoRA fine-tuning on T5-base model.}
    \label{tab:nlu-general}
    \begin{tabular}{ccccccc}
        \toprule
        Epoch & Optimizer & $(\beta_1, \beta_2)$ & $\epsilon$ & Batch Size \\
        \midrule
        1 & AdamW & (0.9, 0.999) & $\SI{1e-8}{}$ & 32 \\
        \midrule
        Warm-up Ratio & LoRA Alpha & $s$ (if needed) & $\lambda$ (if needed) & \#Runs \\
        \midrule
        0.03 & 16 & 16 & $\SI{1e-6}{}$ & 3 \\
        \midrule
        Weight Decay & LR Scheduler  & Sequence Length & Precision & \\
        \midrule
        0 & cosine & 128 & FP32 & \\
        \bottomrule
    \end{tabular}
\end{table}

\subsection{Experiments on LLM}
\label{sec:exp:llm}
We use a stronger baseline for full fine-tuning, as provided in \cite{wang2024lorapro}, compared to those in \cite{wang2024lora}. For vanilla LoRA, due to the limitation of computational resources, we use the results of LoRA with rank $8$ from \cite{wang2024lora}. For LoRA-GA, we pick the best results from \citep{wang2024lora}. We align our generation configuration and stable parameter $s$ with LoRA-GA \cite{wang2024lora} to ensure a fair comparison. The hyperparameter settings are provided in \cref{tab:llama-general}. For the learning rates of LoRA-One, we conduct a grid search over $\{ \SI{5e-5}{} , \SI{2e-5}{} , \SI{1e-5}{} \}$, following the configuration used in \cite{wang2024lora}.

\begin{table}[h]
    \centering
     \caption{Hyperparameters for LoRA fine-tuning on Llama 2-7B model.}
    \label{tab:llama-general}
    \begin{tabular}{ccccc}
        \toprule
        Epoch & Optimizer & $(\beta_1, \beta_2)$ & $\epsilon$ & Batch Size \\
        \midrule
        1 & AdamW & (0.9, 0.999) & $\SI{1e-8}{}$ & 32\\
        \midrule
        Warm-up Ratio & LoRA Alpha & $s$ (if needed) & $\lambda$ (if needed) & \#Runs\\
        \midrule
        0.03 & 16 & 64 & $\SI{1e-6}{}$ & 3 \\
        \midrule
        Weight Decay & LR Scheduler & Sequence Length & Precision & \\
        \midrule
        0 & cosine & 1024 & FP32 & \\
        \bottomrule
    \end{tabular}
\end{table}


\subsection{Ablation Study}
\label{appx:ablation}
In this subsection, we compare $5$ algorithms to provide insights for practical algorithm design. The details of $5$ algorithms are summarized in \cref{tab:descrip-alg}. The details of means and standard deviations over 3 runs are shown in \cref{tab:ab-cola} for CoLA and \cref{tab:ab-mrpc} for MRPC. The hyperparameter settings for LoRA-One, LoRA-One (-), LoRA-GA (-), and LoRA-GA (+) are same as the settings used in \cref{app:expNLP}. We tune the learning rates via grid search over $\{  \SI{1e-3}{}, \SI{5e-4}{}, \SI{2e-4}{},\SI{1e-4}{}, \SI{5e-5}{},\SI{2e-5}{},\SI{1e-5}{} \}$ to ensure a fair comparison. The implement details of Spectral (-) are provided in \cref{alg:spec}, which is a scaled version of \eqref{eq:spectral-init-linear} without preconditioning. We notice that Spectral (-) is highly sensitive to hyperparameters which makes it hard to tune. The general hyperparameters of Spectral (-) is same as the settings used in \cref{app:expNLP}. Here we provide the LoRA alpha and learning rates for Spectral (-) in \cref{tab:spec-hyp}.

\begin{table}[ht]
    \centering
    \caption{Initialization strategies and corresponding optimizers for ablation study.}
    \begin{tabular}{ccc}
      \toprule
         & Initialization & Optimizer \\
      \midrule
      LoRA-One   &\cref{alg:lora_one_training} (1-8) & Prec-AdamW\\
      LoRA-One (-) &\cref{alg:lora_one_training} (1-8) & AdamW\\
      Spectral (-) &\cref{alg:spec} (1-5) & AdamW\\
      LoRA-GA (-) &\eqref{LoRA-GA} & AdamW\\
      LoRA-GA (+) &\eqref{LoRA-GA} & Prec-AdamW\\
      \bottomrule
    \end{tabular}
    \label{tab:descrip-alg}
\end{table}

\begin{table}[h!]
\centering
\caption{Accuracy comparison across different methods on CoLA under three ranks, i.e. $r=8\,,32\,,128$. LoRA-One (-) stands for training with AdamW without preconditioning under initialization by line 1-8 in \cref{alg:lora_one_training}.}
\label{tab:ab-cola}
\begin{tabular}{cccccc}
\toprule
\textbf{Rank} & LoRA-One & LoRA-One (-) & Spectral (-) & LoRA-GA (-) & LoRA-GA (+) \\
\midrule
8 & {81.08}$_{\pm 0.36}$ & 80.83$_{\pm 0.54}$ &\textbf{81.40}$_{\pm 0.31}$  & 80.57$_{\pm 0.20}$ & 80.57$_{\pm 0.12}$\\
32 & \textbf{81.34}$_{\pm 0.51}$ & 81.30$_{\pm 0.16}$ &81.18$_{\pm 0.30}$& 80.86$_{\pm 0.23}$ & 80.92$_{\pm 0.34}$\\
128 & {81.53}$_{\pm 0.36}$ & 81.34$_{\pm 0.12}$ &\textbf{81.62}$_{\pm 0.48}$& 80.95$_{\pm 0.35}$ & 80.02$_{\pm 0.64}$\\
\bottomrule
\end{tabular}
\end{table}

\begin{table}[h!]
\centering
\caption{Accuracy comparison across different methods on MRPC under three ranks, i.e. $r=8\,,32\,,128$. LoRA-One (-) stands for training with AdamW without preconditioning under initialization by line 1-8 in \cref{alg:lora_one_training}.}
\begin{tabular}{cccccc}
\toprule
\textbf{Rank} & LoRA-One & LoRA-One (-) & Spectral (-) & LoRA-GA (-) & LoRA-GA (+) \\
\midrule
8 & 86.77$_{\pm 0.53}$ & \textbf{87.50}$_{\pm 0.60}$ &86.19$_{\pm 0.42}$& 85.29$_{\pm 0.24}$& 85.87$_{\pm 0.31}$\\
32 & \textbf{87.34}$_{\pm 0.31}$ & 87.34$_{\pm 0.42}$ &86.02$_{\pm 0.20}$& 86.36$_{\pm 0.42}$ &85.78$_{\pm 0.20}$\\
128 & \textbf{88.40}$_{\pm 0.70}$ & 87.26$_{\pm 0.20}$ &86.03$_{\pm 0.20}$& 85.46$_{\pm 0.23}$ &87.01$_{\pm 0.35}$\\
\bottomrule
\end{tabular}
\label{tab:ab-mrpc}
\end{table}

\begin{algorithm}[!h]
\caption{\eqref{eq:spectral-init-linear} training for a specific layer}
\label{alg:spec}
\begin{algorithmic}[1]
\Input Pre-trained weight $\bm W^\natural$, batched data $\{\mathcal{D}_t\}_{t=1}^{T}$, LoRA rank $r$, LoRA alpha $\alpha$, loss function $L$, scaling parameter $\gamma$
\Initialize
\STATE Compute $\bm G^\natural \gets -\nabla_{\bm W} L(\bm W^\natural)$
\STATE $\bm U, \bm S, \bm V \gets \text{SVD}\left(\bm G^\natural\right)$
\STATE $\bm A_0 \gets \sqrt{\gamma}\cdot\bm U_{[:,1:r]}\bm S^{1/2}_{[1:r]}$
\STATE $\bm B_0 \gets \sqrt{\gamma}\cdot \bm S^{1/2}_{[1:r]}\bm V^{\!\top}_{[:,1:r]}$
\STATE Clear $\bm G^\natural$
\Train
\FOR{$t=1\,,...\,,T$}
\STATE Update parameters $\bm A_t$ and $\bm B_t$ by AdamW given $\mathcal{D}_t$
\ENDFOR
\Return $\bm W^\natural + \frac{\alpha}{\sqrt{r}} \bm A_{T} \bm B_{T}$
\end{algorithmic}
\end{algorithm}

\begin{table}[h!]
    \centering
    \caption{Specific hyperparameter settings for Spectral (-) (see details in \cref{alg:spec}) used in \cref{appx:ablation}.}
    \begin{tabular}{cccc|ccc}
    \toprule
        Rank & \multicolumn{3}{c|}{CoLA} & \multicolumn{3}{c}{MRPC} \\
        \midrule
         & LR & LoRA Alpha & $\gamma$ & LR & LoRA Alpha & $\gamma$ \\
         \midrule
        8 & $\SI{2e-3}{}$ & $\sqrt{8}$ & $0.01$ & $\SI{6e-4}{}$ & $1$ & $0.01$ \\
        32 & $\SI{2e-3}{}$ & $\sqrt{32}$ & $0.01$ & $\SI{2e-3}{}$ & $16$ & $0.01$  \\
        128 & $\SI{2e-3}{}$ & $1$ & $0.01$  & $\SI{9e-4}{}$ & $1$ & $0.01$  \\
        \bottomrule
    \end{tabular}
    \label{tab:spec-hyp}
\end{table}


\subsection{Comparison of Singular Values}
\label{SV-figs}
First, we collect top-$32$ singular values for each pre-trained layer $\mathbf{W}^\natural$ of pre-trained T5-base model \citep{raffel2020exploring}. Next, we perform full fine-tuning to the pre-trained model on SST-2 dataset from GLUE. To ensure better convergence, we take the hyperparameter settings which are presented in \cref{tab:sv_config}. After training, we collect top-$32$ singular values for each difference weights, i.e. $\Delta \mathbf{W} = \mathbf{W}_\text{fine-tuned} - \mathbf{W}^\natural$. The results are shown in \cref{fig:SV}. The hyperparameter settings for full fine-tuning are provided in \cref{tab:sv_config}.

We observe that, across all layers, the singular values of the pre-trained weights are significantly larger than those of the difference weights.
\begin{table}[h!]
    \caption{Hyperparameters for full fine-tuning on T5-base model used for \cref{SV-figs}.}
    \label{tab:sv_config}
    \centering
    \begin{tabular}{ccccc}
    \toprule
    Epoch & Optimizer & $(\beta_1, \beta_2)$ & $\epsilon$ & Batchsize \\
    \midrule
    10 & AdamW & $(0.9, 0.999)$ & $\SI{1e-8}{}$ & 32 \\
    \midrule
    Weight Decay & LR & LR Scheduler & Warm-up Ratio & \\
    \midrule
    0.1 & $\SI{1e-4}{}$ & cosine & 0.03 & \\
    \bottomrule
    \end{tabular}
\end{table}
\end{document}

