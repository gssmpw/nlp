\documentclass[11pt]{article}
\pdfoutput=1
\usepackage{microtype}
\usepackage{graphicx,wrapfig,lipsum}
\usepackage{subfigure}
\usepackage{booktabs} % for professional tables
\usepackage[shortlabels]{enumitem}
\usepackage{siunitx}  
\usepackage{hyperref}
\hypersetup{colorlinks, linkcolor=red, anchorcolor=blue, citecolor=blue}
\newcommand{\theHalgorithm}{\arabic{algorithm}}
\renewcommand{\baselinestretch}{1.05}
\usepackage{mylatexstyle}
\usepackage{setspace}
\usepackage[left=1in, right=1in, top=1in, bottom=1in]{geometry}
\usepackage{amsmath}
\usepackage{amssymb,bm}
\usepackage{mathrsfs}
\usepackage{graphicx}
\usepackage{mathtools}
\usepackage{caption}
\usepackage{subcaption,color}
\usepackage{indentfirst,threeparttable,multirow}
\usepackage{amsfonts,tcolorbox}
\usepackage{float,url}
\usepackage{bbm,cancel}
\usepackage{lipsum}
\usepackage{nicefrac} 


\newcommand{\disableaddcontentsline}{%
  \let\savedaddcontentsline\addcontentsline 
  \renewcommand{\addcontentsline}[3]{}
}
\newcommand{\enableaddcontentsline}{%
  \let\addcontentsline\savedaddcontentsline
}

\usepackage[textsize=tiny]{todonotes}
\definecolor{myblue}{RGB}{0 114 199}
\definecolor{mylightblue}{RGB}{77 191 241}
\definecolor{darkgray}{HTML}{878787}
\definecolor{myorange}{RGB}{217 83 25}
\newtcolorbox{myorangebox}{colframe = myorange}
\newtcolorbox{mygraybox}{colframe = gray}
\newtcolorbox{mybluebox}{colframe = myblue}

\usepackage{xcolor}
\definecolor{fhcolor}{rgb}{0.523, 0.235, 0.625}
\newcommand{\fh}[1]{\textcolor{fhcolor}{(SgrA: #1)}}
\newcommand{\yccomment}[1]{\textcolor{red}{(Yudong: #1)}}
\let\yc\yccomment
\let\yudong\yccomment
\newcommand{\yh}[1]{\textcolor{blue}{(Yuanhe: #1)}}

\title{\huge One-step full gradient suffices for low-rank fine-tuning, provably and efficiently}

\author
{
     Yuanhe Zhang\thanks{Department of Statistics, University of Warwick, United Kingdom; Email: {\tt yuanhe.zhang@warwick.ac.uk}} 
     ~~~
     Fanghui Liu\thanks{Department of Computer Science, also Centre for Discrete Mathematics and its Applications (DIMAP), University of Warwick, United Kingdom; Email: {\tt fanghui.liu@warwick.ac.uk} (Corresponding author)} 
     ~~~
     Yudong Chen\thanks{Department of Computer Sciences, University of Wisconsin-Madison, USA; e-mail: {\tt yudong.chen@wisc.edu}}
}


\begin{document}
\disableaddcontentsline
\maketitle

\begin{abstract}
This paper studies how to improve the performance of Low-Rank Adaption (LoRA) \citep{hu2022lora} as guided by our theoretical analysis. 
Our first set of theoretical results show that for random initialization and linear models, \textit{i)} LoRA will align to the certain singular subspace of one-step gradient of full fine-tuning; \textit{ii)} preconditioners improve convergence in the high-rank case.
These insights motivate us to focus on preconditioned LoRA using a specific spectral initialization strategy for aligning with certain subspaces.
For both linear and nonlinear models, we prove that alignment and generalization guarantees can be directly achieved at initialization, and the subsequent linear convergence can be also built.
Our analysis leads to the \emph{LoRA-One} algorithm (using \emph{One}-step gradient and preconditioning), a theoretically grounded algorithm that achieves significant empirical improvement over vanilla LoRA and its variants on several benchmarks.
Our theoretical analysis, based on decoupling the learning dynamics and characterizing how spectral initialization contributes to feature learning, may be of independent interest for understanding matrix sensing and deep learning theory.
The source code can be found in the \url{https://github.com/YuanheZ/LoRA-One}.

\end{abstract}

\section{Introduction}


\begin{figure}[t]
\centering
\includegraphics[width=0.6\columnwidth]{figures/evaluation_desiderata_V5.pdf}
\vspace{-0.5cm}
\caption{\systemName is a platform for conducting realistic evaluations of code LLMs, collecting human preferences of coding models with real users, real tasks, and in realistic environments, aimed at addressing the limitations of existing evaluations.
}
\label{fig:motivation}
\end{figure}

\begin{figure*}[t]
\centering
\includegraphics[width=\textwidth]{figures/system_design_v2.png}
\caption{We introduce \systemName, a VSCode extension to collect human preferences of code directly in a developer's IDE. \systemName enables developers to use code completions from various models. The system comprises a) the interface in the user's IDE which presents paired completions to users (left), b) a sampling strategy that picks model pairs to reduce latency (right, top), and c) a prompting scheme that allows diverse LLMs to perform code completions with high fidelity.
Users can select between the top completion (green box) using \texttt{tab} or the bottom completion (blue box) using \texttt{shift+tab}.}
\label{fig:overview}
\end{figure*}

As model capabilities improve, large language models (LLMs) are increasingly integrated into user environments and workflows.
For example, software developers code with AI in integrated developer environments (IDEs)~\citep{peng2023impact}, doctors rely on notes generated through ambient listening~\citep{oberst2024science}, and lawyers consider case evidence identified by electronic discovery systems~\citep{yang2024beyond}.
Increasing deployment of models in productivity tools demands evaluation that more closely reflects real-world circumstances~\citep{hutchinson2022evaluation, saxon2024benchmarks, kapoor2024ai}.
While newer benchmarks and live platforms incorporate human feedback to capture real-world usage, they almost exclusively focus on evaluating LLMs in chat conversations~\citep{zheng2023judging,dubois2023alpacafarm,chiang2024chatbot, kirk2024the}.
Model evaluation must move beyond chat-based interactions and into specialized user environments.



 

In this work, we focus on evaluating LLM-based coding assistants. 
Despite the popularity of these tools---millions of developers use Github Copilot~\citep{Copilot}---existing
evaluations of the coding capabilities of new models exhibit multiple limitations (Figure~\ref{fig:motivation}, bottom).
Traditional ML benchmarks evaluate LLM capabilities by measuring how well a model can complete static, interview-style coding tasks~\citep{chen2021evaluating,austin2021program,jain2024livecodebench, white2024livebench} and lack \emph{real users}. 
User studies recruit real users to evaluate the effectiveness of LLMs as coding assistants, but are often limited to simple programming tasks as opposed to \emph{real tasks}~\citep{vaithilingam2022expectation,ross2023programmer, mozannar2024realhumaneval}.
Recent efforts to collect human feedback such as Chatbot Arena~\citep{chiang2024chatbot} are still removed from a \emph{realistic environment}, resulting in users and data that deviate from typical software development processes.
We introduce \systemName to address these limitations (Figure~\ref{fig:motivation}, top), and we describe our three main contributions below.


\textbf{We deploy \systemName in-the-wild to collect human preferences on code.} 
\systemName is a Visual Studio Code extension, collecting preferences directly in a developer's IDE within their actual workflow (Figure~\ref{fig:overview}).
\systemName provides developers with code completions, akin to the type of support provided by Github Copilot~\citep{Copilot}. 
Over the past 3 months, \systemName has served over~\completions suggestions from 10 state-of-the-art LLMs, 
gathering \sampleCount~votes from \userCount~users.
To collect user preferences,
\systemName presents a novel interface that shows users paired code completions from two different LLMs, which are determined based on a sampling strategy that aims to 
mitigate latency while preserving coverage across model comparisons.
Additionally, we devise a prompting scheme that allows a diverse set of models to perform code completions with high fidelity.
See Section~\ref{sec:system} and Section~\ref{sec:deployment} for details about system design and deployment respectively.



\textbf{We construct a leaderboard of user preferences and find notable differences from existing static benchmarks and human preference leaderboards.}
In general, we observe that smaller models seem to overperform in static benchmarks compared to our leaderboard, while performance among larger models is mixed (Section~\ref{sec:leaderboard_calculation}).
We attribute these differences to the fact that \systemName is exposed to users and tasks that differ drastically from code evaluations in the past. 
Our data spans 103 programming languages and 24 natural languages as well as a variety of real-world applications and code structures, while static benchmarks tend to focus on a specific programming and natural language and task (e.g. coding competition problems).
Additionally, while all of \systemName interactions contain code contexts and the majority involve infilling tasks, a much smaller fraction of Chatbot Arena's coding tasks contain code context, with infilling tasks appearing even more rarely. 
We analyze our data in depth in Section~\ref{subsec:comparison}.



\textbf{We derive new insights into user preferences of code by analyzing \systemName's diverse and distinct data distribution.}
We compare user preferences across different stratifications of input data (e.g., common versus rare languages) and observe which affect observed preferences most (Section~\ref{sec:analysis}).
For example, while user preferences stay relatively consistent across various programming languages, they differ drastically between different task categories (e.g. frontend/backend versus algorithm design).
We also observe variations in user preference due to different features related to code structure 
(e.g., context length and completion patterns).
We open-source \systemName and release a curated subset of code contexts.
Altogether, our results highlight the necessity of model evaluation in realistic and domain-specific settings.






\section{Related Works}
\label{sec:related_works}


\noindent\textbf{Diffusion-based Video Generation. }
The advancement of diffusion models \cite{rombach2022high, ramesh2022hierarchical, zheng2022entropy} has led to significant progress in video generation. Due to the scarcity of high-quality video-text datasets \cite{Blattmann2023, Blattmann2023a}, researchers have adapted existing text-to-image (T2I) models to facilitate text-to-video (T2V) generation. Notable examples include AnimateDiff \cite{Guo2023}, Align your Latents \cite{Blattmann2023a}, PYoCo \cite{ge2023preserve}, and Emu Video \cite{girdhar2023emu}. Further advancements, such as LVDM \cite{he2022latent}, VideoCrafter \cite{chen2023videocrafter1, chen2024videocrafter2}, ModelScope \cite{wang2023modelscope}, LAVIE \cite{wang2023lavie}, and VideoFactory \cite{wang2024videofactory}, have refined these approaches by fine-tuning both spatial and temporal blocks, leveraging T2I models for initialization to improve video quality.
Recently, Sora \cite{brooks2024video} and CogVideoX \cite{yang2024cogvideox} enhance video generation by introducing Transformer-based diffusion backbones \cite{Peebles2023, Ma2024, Yu2024} and utilizing 3D-VAE, unlocking the potential for realistic world simulators. Additionally, SVD \cite{Blattmann2023}, SEINE \cite{chen2023seine}, PixelDance \cite{zeng2024make} and PIA \cite{zhang2024pia} have made significant strides in image-to-video generation, achieving notable improvements in quality and flexibility.
Further, I2VGen-XL \cite{zhang2023i2vgen}, DynamicCrafter \cite{Xing2023}, and Moonshot \cite{zhang2024moonshot} incorporate additional cross-attention layers to strengthen conditional signals during generation.



\noindent\textbf{Controllable Generation.}
Controllable generation has become a central focus in both image \citep{Zhang2023,jiang2024survey, Mou2024, Zheng2023, peng2024controlnext, ye2023ip, wu2024spherediffusion, song2024moma, wu2024ifadapter} and video \citep{gong2024atomovideo, zhang2024moonshot, guo2025sparsectrl, jiang2024videobooth} generation, enabling users to direct the output through various types of control. A wide range of controllable inputs has been explored, including text descriptions, pose \citep{ma2024follow,wang2023disco,hu2024animate,xu2024magicanimate}, audio \citep{tang2023anytoany,tian2024emo,he2024co}, identity representations \citep{chefer2024still,wang2024customvideo,wu2024customcrafter}, trajectory \citep{yin2023dragnuwa,chen2024motion,li2024generative,wu2024motionbooth, namekata2024sg}.


\noindent\textbf{Text-based Camera Control.}
Text-based camera control methods use natural language descriptions to guide camera motion in video generation. AnimateDiff \cite{Guo2023} and SVD \cite{Blattmann2023} fine-tune LoRAs \cite{hu2021lora} for specific camera movements based on text input. 
Image conductor\cite{li2024image} proposed to separate different camera and object motions through camera LoRA weight and object LoRA weight to achieve more precise motion control.
In contrast, MotionMaster \cite{hu2024motionmaster} and Peekaboo \cite{jain2024peekaboo} offer training-free approaches for generating coarse-grained camera motions, though with limited precision. VideoComposer \cite{wang2024videocomposer} adjusts pixel-level motion vectors to provide finer control, but challenges remain in achieving precise camera control.

\noindent\textbf{Trajectory-based Camera Control.}
MotionCtrl \cite{Wang2024Motionctrl}, CameraCtrl \cite{He2024Cameractrl}, and Direct-a-Video \cite{yang2024direct} use camera pose as input to enhance control, while CVD \cite{kuang2024collaborative} extends CameraCtrl for multi-view generation, though still limited by motion complexity. To improve geometric consistency, Pose-guided diffusion \cite{tseng2023consistent}, CamCo \cite{Xu2024}, and CamI2V \cite{zheng2024cami2v} apply epipolar constraints for consistent viewpoints. VD3D \cite{bahmani2024vd3d} introduces a ControlNet\cite{Zhang2023}-like conditioning mechanism with spatiotemporal camera embeddings, enabling more precise control.
CamTrol \cite{hou2024training} offers a training-free approach that renders static point clouds into multi-view frames for video generation. Cavia \cite{xu2024cavia} introduces view-integrated attention mechanisms to improve viewpoint and temporal consistency, while I2VControl-Camera \cite{feng2024i2vcontrol} refines camera movement by employing point trajectories in the camera coordinate system. Despite these advancements, challenges in maintaining camera control and scene-scale consistency remain, which our method seeks to address. It is noted that 4Dim~\cite{watson2024controlling} introduces absolute scale but in  4D novel view synthesis (NVS) of scenes.




We study (stochastic) gradient descent on the empirical risk
\begin{equation*}
\cL(w) = \frac{1}{n}\sum_{i=1}^n l(p_i(w))\, ,
\end{equation*}
where the loss function $l$ and the functions  $(p_i)_{i=1}^n$  are specified in the following assumptions. Note that the empirical risk for binary classification from Equation~\eqref{def:emp_risk_intro} is a special case of the above objective.

\begin{assumption}\label{hyp:loss_exp_log}\phantom{=}
  \begin{enumerate}[label=\roman*)]
    \item The loss is either the exponential loss, $l(q) = e^{-q}$, or the logistic loss, $l(q) = \log(1{+}e^{-q})$.
    \item There exists an integer $L \in \mathbb{N}^*$  such that, for all $1 \leq i \leq n$, the function $p_i$ is $L$-homogeneous\footnote{We recall that a mapping $f : \mathbb{R}^d \rightarrow \mathbb{R}$ is positively $L$-homogeneous if $f(\lambda w) = \lambda^L f(w)$ for all $w \in \mathbb{R}^d$ and $\lambda >0$.}, locally Lipschitz continuous and semialgebraic.
  \end{enumerate}
\end{assumption}
If the $p_i$'s were differentiable with respect to $w$, the chain rule would guarantee that
\begin{align*}
\nabla \mathcal{L}(w) = \frac{1}{n}\sum_{i=1}^n  l'(p_i(w)) \nabla p_i(w)\enspace.
\end{align*}
However, we only assume that the $p_i$'s are semialgebraic. While we could consider Clarke subgradients, the Clarke subgradient of operations on functions (e.g., addition, composition, and minimum) is only contained within the composition of the respective Clarke subgradients. This, as noted in Section~\ref{sec:cons_field}, implies that the output of backpropagation is usually not an element of a Clarke subgradient but a selection of some conservative set-valued field.
Consequently, for $1\leq i \leq n$, we consider $D_i : \bbR^d \rightrightarrows\bbR^d$, a conservative set-valued field of $p_i$, and a function $\sa_i : \bbR^d \rightarrow \bbR^d$ such that for all $w \in \bbR^d$, $\sa_i(w) \in D_i(w)$. Given a step-size $\gamma >0$, gradient descent (GD)\footnote{More precisely, this refers to conservative gradient descent. We use the term GD for simplicity, as conservative gradients behave similarly to standard gradients.} is then expressed as
\begin{equation*}\label{eq:gd_new}\tag{GD}
  w_{k+1} = w_k - \frac{\gamma}{n} \sum_{i=1}^n l'(p_i(w_k))\sa_i(w_k)\,.
\end{equation*}
For its stochastic counterpart, stochastic gradient descent (SGD), we fix a batch-size $1\leq n_b \leq n$. At each iteration $k \in \bbN$, we randomly and uniformly draw a batch $B_k \subset \{1, \ldots, n \}$ of size $n_b$. The update rule is then given by 
\begin{equation*}\label{eq:sgd_new}\tag{SGD}
  w_{k+1} = w_k -  \frac{\gamma}{n_b}\sum_{i\in B_k} l'(p_i(w_k)) \sa_i(w_k)\, .
\end{equation*}
The considered conservative set-valued fields will satisfy an Euler lemma-type assumption.
%\nic{Smoother transition}
\begin{assumption}\phantom{=}\label{hyp:conserv}
  For every $i \leq n$, $\sa_i$ is measurable and $D_i$ is semialgebraic. Moreover, for every $w \in \bbR^d$ and $\lambda \geq 0$, $\sa_i(w)  \in D_i(w)$,
  \begin{equation*}
    D_i(\lambda w) = \lambda^{L-1} D_i(w)\, , \textrm{ and } \quad   L p_i(w) = \scalarp{\sa_i(w)}{w}\, .
  \end{equation*}
\end{assumption}
%\nic{Smoother transition}
Having in mind the binary classification setting, in which $p_i(w) = y_i \Phi(x_i, w)$, we define the margin
\begin{equation}\label{def:marg}
  \sm: \bbR^d \rightarrow \bbR, \quad \sm(w) = \min_{1\leq i \leq n} p_i(w)\, .
\end{equation}
It quantifies the quality of a prediction rule $\Phi(\cdot, w)$. In particular,  the training data is perfectly separated when $\sm(w) >0$. A binary prediction for $x$ is given by the sign of $\Phi(x, w)$, and under the homogeneity assumption, it depends only on the normalized direction $w / \norm{w}$. Consequently, we will focus on the sequence of directions $u_k := w_k / \norm{w_k}$. Our final assumption ensures that the normalized directions $(u_k)$ have stabilized in a region where the training data is correctly classified.

\begin{assumption}\label{hyp:marg_lowb}
  Almost surely, $\liminf \sm(u_k) >0$.
\end{assumption}
Before presenting our main result, we comment on our assumptions.

\paragraph{On Assumption~\ref{hyp:loss_exp_log}.} As discussed in the introduction, the primary example we consider is when $p_i(w) = y_i \Phi(x_i;w)$ is the signed prediction of a feedforward neural network without biases and with piecewise linear activation functions on a labeled dataset $((x_i,y_i))_{i \leq n}$. In this case,
\begin{equation}\label{eq:NN}
 p_i(w) = y_i \Phi(w;x_i) = y_i V_L(W_L) \sigma(V_{L-1}(W_{L-1}) \sigma(V_{L-1}(W_{L-2}) \ldots \sigma(V_{1}(W_1 x_i))))\, ,
\end{equation}
where $w = [W_1, \ldots, W_L]$, $W_i$ represents the weights of the $i$-th layer, $V_i$ is a linear function in the space of matrices (with $V_i$ being the identity for fully-connected layers) and $\sigma$ is a coordinate-wise activation function such as $z \mapsto \max(0,z)$ ($\ReLU$), $z \mapsto \max(az, z)$ for a small parameter $a>0$ (LeakyReLu) or $z \mapsto z$. Note that the mapping $w \mapsto p_i(w)$ is semialgebraic and $L$-homogeneous for any of these activation functions. Regarding the loss functions, the logistic and exponential losses are among the most commonly studied and widely used. In Appendix~\ref{app:gen_sett}, we extend our results to a broader class of losses, including $l(q) = e^{-q^a}$ and $l(q) = \ln (1 + e^{-q^a})$ for any $a \geq 1$.

\paragraph{On Assumption~\ref{hyp:conserv}.} Assumption~\ref{hyp:conserv} holds automatically  if $D_i$ is the Clarke subgradient of $p_i$. Indeed, at any vector $w \in \bbR^d$, where $p_i$ is differentiable it holds that $p_i(\lambda w) = \lambda^{L} p_i(w)$. Differentiating relatively to $w$ and $\lambda$ (noting that $p_i$ remains differentiable at $\lambda w$ due to homogeneity), we obtain $\lambda \nabla p_i(\lambda w) = \lambda^{L} \nabla p_i(w)$ and $\scalarp{\nabla p_i(\lambda w)}{w} = L \lambda^{L-1} p_i(w)$. The expression for any element of the Clarke subgradient then follows from~\eqref{eq:def_clarke}. 

However, for an arbitrary conservative set-valued field, Assumption~\ref{hyp:conserv} does not necessarily hold. For instance, $D(x) = \mathds{1}(x \in \mathbb{N})$ is a conservative set-valued field for $p \equiv 0$, which does not satisfy Assumption~\ref{hyp:conserv}. Nevertheless, in practice, conservative set-valued fields naturally arise from a formal application of the chain rule. For a non-smooth but homogeneous activation function $\sigma$, one selects an element $e \in \partial \sigma (0)$, and computes $\sa_i(w)$ via backpropagation. Whenever a gradient candidate of $\sigma$ at zero is required (i.e., in~\eqref{eq:NN}, for some $j$, $V_j(W_j)$ contains a zero entry), it is replaced by $e$. 
Since $V_j(W_j)$ and $V_j(\lambda W_j)$ have the same zero elements, it follows that for every such $w$, $
\sa_i(\lambda w) = \lambda^L \sa_i(w)$. The conservative set-valued field $D_i$ is then obtained by associating to each $w$ the set of all possible outcomes of the chain rule, with $e$ ranging over all elements of $\partial \sigma(0)$. Thus, for such fields, Assumption~\ref{hyp:conserv} holds.


\paragraph{On Assumption~\ref{hyp:marg_lowb}.} Training typically continues even after the training error reaches zero.
Assumption~\ref{hyp:marg_lowb} characterizes this late-training phase, where our result applies. 
As noted earlier, since $\sm$ is $L$-homogeneous, the classification rule is determined by the direction of the  iterates $u_k=w_k/\norm{w_k}$. Assumption~\ref{hyp:marg_lowb} then states that, beyond some iteration, the normalized margin remains positive. 
This assumption is natural in the context of studying the implicit bias of SGD: we \emph{assume} that we reached the phase in which the dataset is correctly classified and \emph{then} characterize the limit points. A similar perspective was taken in  \cite{nacson2019lexicographic}, where the implicit bias of GF was analyzed under the assumption that the sequence of directions and the loss converge. However, unlike their approach, ours does not require assuming such convergence a priori.

Earlier works such as \cite{ji2020directional,Lyu_Li_maxmargin}, which analyze subgradient flow or smooth GD, establish convergence by assuming the existence of a single iterate $w_{k_0}$ satisfying $\sm(w_{k_0}) > \varepsilon$ and then proving that $\lim \sm(u_{k}) > 0$. Their approach relies on constructing a smooth approximation of the margin, which increases during training, ensuring that $\sm(u_k) > 0$ for all iterates with $k \geq k_0$. This is feasible in their setting, as they study either subgradient flow or GD with smooth $p_i$’s, allowing them to leverage the descent lemma.

In contrast, our analysis considers a nonsmooth and stochastic setting, in which, even if an iterate $w_{k_0}$ satisfying $\sm(w_{k_0}) > \varepsilon$ exists, there is no a priori assurance that subsequent iterates remain in the region where Assumption~\ref{hyp:marg_lowb} holds. From this perspective, Assumption~\ref{hyp:marg_lowb} can be viewed as a stability assumption, ensuring that iterates continue to classify the dataset correctly. Establishing stability for stochastic and nonsmooth algorithms is notoriously hard, and only partial results in restrictive settings exist \cite{borkar2000ode,ramaswamy2017generalization,josz2024global}.

%Finally, note that Assumption~\ref{hyp:marg_lowb} only needs to hold almost surely. Specifically, with probability 1, there exist $k_0$ and $\varepsilon$ such that for all $k \geq k_0$, $\sm(u_k) \geq \varepsilon > 0$. In the case of~\eqref{eq:sgd_new}, $k_0$ and $\delta$ are random variables and may take different values across different realizations. 

%\paragraph{On constant stepsizes.}
%We allow the step size to be a constant of arbitrary magnitude, subject to the stability Assumption~\ref{hyp:marg_lowb}. This may seem surprising in a nonsmooth and stochastic setting, where a vanishing step size is typically required to ensure convergence (see, e.g., \cite{majewski2018analysis, dav-dru-kak-lee-19, bolte2023subgradient, le2024nonsmooth}).

\input{arxiv_template/main/Linear}

\section{Balancing Patients}
\subsection{Defining Balance}
Patient-provider matching frequently involves objectives beyond match quality due to the complex interactions on both sides of the market. 
We focus on a general class of objectives here: the need to balance patient ``workload'' between different providers. 
Formally, we let each patient correspond to a workload quantity, which we denote $\beta_{i}$. 
This workload can refer to quantities such as their age, level of commodities, and estimated visits per month. 
If we then define each provider to have a workload, $\hat{\beta}_{j}$, we aim to ensure equal distribution of new patients across providers. 
That is, we aim to minimize the variance in workload across providers: 
\begin{equation}
    \min\limits_{X_{i,j}} \frac{1}{P} \sum_{j=1}^{P} |\sum_{i=1}^{N} X_{i,j} \beta_{i} + \hat{\beta}_{j} - \bar{\beta}| 
\end{equation}
Here, $\bar{\beta}$ is the average workload across providers: $\bar{\beta} = \frac{1}{P} \sum_{j=1}^P \sum_{i=1}^{N} X_{i,j} \beta_{i} + \hat{\beta_{j}}$. 
We note that multiple values of $\beta$ are possible, which corresponds to the different notions of workload references earlier. 

We note that such an objective can be written through a linear program as follows: 
\begin{equation}
    \min\limits_{v_{j}} \frac{1}{P} \sum_{j=1}^{P} v_{j}
\end{equation}
So that $-v_{j} \leq \sum_{i=1}^{N} X_{i,j} \beta_{i} + \hat{\beta}_{j} - \bar{\beta}\leq v_{j}$ and $\bar{\beta}$ is a linear function in $X_{i,j}$ as defined earlier. 
Such a formulation allows for easy incorporation into our model.

\subsection{Optimizing for Balance}
By incorporating the need for patient balance across providers, our overall objective function becomes
\begin{equation}
    \max  \sum_{i=1}^{N} \sum_{j=1}^{P} \theta_{i,j} X_{i,j} - \frac{\lambda}{P} \sum_{j=1}^{P} v_{j}
\end{equation}
Here, $\lambda$ is a hyperparameter which balances the need for balancing patients with improving match quality. 
We note that optimizing such a problem with approximation guarantees becomes difficult when $\lambda >> N$, because matching any patients worsens the balance. 
We prove this formally as follows: 
\begin{lemma}
Consider the following Linear Program: 
\begin{equation}
    \max \sum_{i=1}^{N} \sum_{j=1}^{P} \theta_{i,j} X_{i,j} - \frac{\lambda}{P} \sum_{j=1}^{P} v_{j}
\end{equation}
When $\lambda >> N$, any solution with $X_{i,j} = 1$ for some $i,j$ results in an unbounded approximation ratio compared to $\mathrm{OPT}$. 
\end{lemma}

Motivated by this, we instead focus on scenarios where $\lambda$ is small. 
We display results in one-shot assortment decision scenarios with known and random ordering. 
We first show that, with known ordering, our previous algorithms can be used, with little impact on the approximation guarantees, assuming that $\lambda$ is small. 
\begin{lemma}
Suppose we select $X_{i,j}$ to optimize the following Linear Program: 
\begin{equation}
    \max \sum_{i=1}^{N} \sum_{j=1}^{P} \theta_{i,j} X_{i,j} - \frac{\lambda}{P} \sum_{j=1}^{P} v_{j}
\end{equation}
Where $-v_{j} \leq \sum_{i=1}^{N} X_{i,j} \beta_{i} + \hat{\beta}_{j} - \bar{\beta}\leq v_{j}$ for some known $\beta_{i}, \hat{\beta}_{j}$. 
Suppose the Linear Programming solution to this allocates provider $\pi_{i}$ to patient $i$. 
If $\lambda \leq \frac{\theta_{i,j}-\theta_{i,\pi_{i}}}{\beta_{i}}$ for all $i,j$ such that $\theta_{i,j}-\theta_{i,\pi_{i}} \geq 0$, then the Stacked Linear Programming solution is optimal. 
\end{lemma}
We informally note that, in scenarios where such a condition does not hold, a strategy is to only allocate provider $j$, after all patients with $\pi_{i}=j$ have been matched or declined. 
Such a scenario differs from the traditional Stacked Linear Programming scenario, because providers can be matched with multiple patients. 

We next detail two approxation bounds for the known-ordering and random-ordering scenarios: 
\begin{lemma}
Suppose we select $X_{i,j}$ to optimize the following Linear Program: 
\begin{equation}
    \max \sum_{i=1}^{N} \sum_{j=1}^{P} \theta_{i,j} X_{i,j} - \frac{\lambda}{P} \sum_{j=1}^{P} v_{j}
\end{equation}
Where $-v_{j} \leq \sum_{i=1}^{N} X_{i,j} \beta_{i} + \hat{\beta}_{j} - \bar{\beta}\leq v_{j}$ for some known $\beta_{i}, \hat{\beta}_{j}$. 
Suppose that $\beta_{i} \leq \frac{1}{P}$, and let the order of patient responses be known.  
Let the LP solution to this be denoted $\mathrm{ALG}$, and note that prior results showed $\mathrm{ALG} = \mathrm{OPT}$. 
Then using Stacked Linear Programming results in an approximation ratio of $\frac{\mathrm{ALG}}{\mathrm{ALG} + \lambda}$
\end{lemma}

\begin{lemma}
Suppose we select $X_{i,j}$ to optimize the following Linear Program: 
\begin{equation}
    \max \sum_{i=1}^{N} \sum_{j=1}^{P} \theta_{i,j} X_{i,j} - \frac{\lambda}{P} \sum_{j=1}^{P} v_{j}
\end{equation}
Where $-v_{j} \leq \sum_{i=1}^{N} X_{i,j} \beta_{i} + \hat{\beta}_{j} - \bar{\beta}\leq v_{j}$ for some known $\beta_{i}, \hat{\beta}_{j}$. 
Suppose that $\beta_{i} \leq \frac{1}{P}$, and let the order of patient responses be random uniform.  
Let the LP solution to this be denoted $\mathrm{OPT}$. 
Then allocating only the providers from the LP solution to each patient results in an approximation ratio of $\frac{p \mathrm{OPT}}{\mathrm{OPT} + \lambda}$
\end{lemma}
These demonstrate the ability to re-use our previous solutions, while adapting to the varying constraints placed by provider balancing. 



%%%%%%%%%%%%%%%%%%%%%%%%%%%%%%%%%%%%%%%%%%%%%%%%%%%%%%%%%%%%%%%%%%%%%%%%%%%%%%%%%%%%%%%%%%%%%%%%%%%%%%

%%%%%%%%%%%%%%%%%%%%%%%%%%%%%%%%%%%%%%%%%%%%%%
\begin{table*}[t]
\setlength{\tabcolsep}{3pt}
\centering
\renewcommand{\arraystretch}{1.1}
\tabcolsep=0.2cm
\begin{adjustbox}{max width=\textwidth}  % Set the maximum width to text width
\begin{tabular}{c| cccc ||  c| cc cc}
\toprule
General & \multicolumn{3}{c}{Preference} & Accuracy & Supervised & \multicolumn{3}{c}{Preference} & Accuracy \\ 
LLMs & PrefHit & PrefRecall & Reward & BLEU & Alignment & PrefHit & PrefRecall & Reward & BLEU \\ 
\midrule
GPT-J & 0.2572 & 0.6268 & 0.2410 & 0.0923 & Llama2-7B & 0.2029 & 0.803 & 0.0933 & 0.0947 \\
Pythia-2.8B & 0.3370 & 0.6449 & 0.1716 & 0.1355 & SFT & 0.2428 & 0.8125 & 0.1738 & 0.1364 \\
Qwen2-7B & 0.2790 & 0.8179 & 0.1593 & 0.2530 & Slic & 0.2464 & 0.6171 & 0.1700 & 0.1400 \\
Qwen2-57B & 0.3086 & 0.6481 & 0.6854 & 0.2568 & RRHF & 0.3297 & 0.8234 & 0.2263 & 0.1504 \\
Qwen2-72B & 0.3212 & 0.5555 & 0.6901 & 0.2286 & DPO-BT & 0.2500 & 0.8125 & 0.1728 & 0.1363 \\ 
StarCoder2-15B & 0.2464 & 0.6292 & 0.2962 & 0.1159 & DPO-PT & 0.2572 & 0.8067 & 0.1700 & 0.1348 \\
ChatGLM4-9B & 0.2246 & 0.6099 & 0.1686 & 0.1529 & PRO & 0.3025 & 0.6605 & 0.1802 & 0.1197 \\ 
Llama3-8B & 0.2826 & 0.6425 & 0.2458 & 0.1723 & \textbf{\shortname}* & \textbf{0.3659} & \textbf{0.8279} & \textbf{0.2301} & \textbf{0.1412} \\ 
\bottomrule
\end{tabular}
\end{adjustbox}
\caption{Main results on the StaCoCoQA. The left shows the performance of general LLMs, while the right presents the performance of the fine-tuned LLaMA2-7B across various strong benchmarks for preference alignment. Our method SeAdpra is highlighted in \textbf{bold}.}
\label{main}
\vspace{-0.2cm}
\end{table*}
%%%%%%%%%%%%%%%%%%%%%%%%%%%%%%%%%%%%%%%%%%%%%%%%%%%%%%%%%%%%%%%%%%%%%%%%%%%%%%%%%%%%%%%%%%%%%%%%%%%%
\begin{table}[h]
\centering
\renewcommand{\arraystretch}{1.02}
% \tabcolsep=0.1cm
\begin{adjustbox}{width=0.48\textwidth} % Adjust table width
\begin{tabularx}{0.495\textwidth}{p{1.2cm} p{0.7cm} p{0.95cm}p{0.95cm}p{0.7cm}p{0.7cm}}
     \toprule
    \multirow{2}{*}{\small \textbf{Dataset}} & \multirow{2}{*}{\small Model} & \multicolumn{2}{c}{\small Preference} & \multicolumn{2}{c}{\small Acc } \\ 
    & & \small \textit{PrefHit} & \small \textit{PrefRec} & \small \textit{Reward} & \small \textit{Rouge} \\ 
    \midrule
    \multirow{2}{*}{\small \textbf{Academia}}   & \small PRO & 33.78 & 59.56 & 69.94 & 9.84 \\ 
                                & \small \textbf{Ours} & 36.44 & 60.89 & 70.17 & 10.69 \\ 
    \midrule
    \multirow{2}{*}{\small \textbf{Chemistry}}  & \small PRO & 36.31 & 63.39 & 69.15 & 11.16 \\ 
                                & \small \textbf{Ours} & 38.69 & 64.68 & 69.31 & 12.27 \\ 
    \midrule
    \multirow{2}{*}{\small \textbf{Cooking}}    & \small PRO & 35.29 & 58.32 & 69.87 & 12.13 \\ 
                                & \small \textbf{Ours} & 38.50 & 60.01 & 69.93 & 13.73 \\ 
    \midrule
    \multirow{2}{*}{\small \textbf{Math}}       & \small PRO & 30.00 & 56.50 & 69.06 & 13.50 \\ 
                                & \small \textbf{Ours} & 32.00 & 58.54 & 69.21 & 14.45 \\ 
    \midrule
    \multirow{2}{*}{\small \textbf{Music}}      & \small PRO & 34.33 & 60.22 & 70.29 & 13.05 \\ 
                                & \small \textbf{Ours} & 37.00 & 60.61 & 70.84 & 13.82 \\ 
    \midrule
    \multirow{2}{*}{\small \textbf{Politics}}   & \small PRO & 41.77 & 66.10 & 69.52 & 9.31 \\ 
                                & \small \textbf{Ours} & 42.19 & 66.03 & 69.74 & 9.38 \\ 
    \midrule
    \multirow{2}{*}{\small \textbf{Code}} & \small PRO & 26.00 & 51.13 & 69.17 & 12.44 \\ 
                                & \small \textbf{Ours} & 27.00 & 51.77 & 69.46 & 13.33 \\ 
    \midrule
    \multirow{2}{*}{\small \textbf{Security}}   & \small PRO & 23.62 & 49.23 & 70.13 & 10.63 \\ 
                                & \small \textbf{Ours} & 25.20 & 49.24 & 70.92 & 10.98 \\ 
    \midrule
    \multirow{2}{*}{\small \textbf{Mean}}       & \small PRO & 32.64 & 58.05 & 69.64 & 11.51 \\ 
                                & \small \textbf{Ours} & \textbf{34.25} & \textbf{58.98} & \textbf{69.88} & \textbf{12.33} \\ 
    \bottomrule
\end{tabularx}
\end{adjustbox}
\caption{Main results (\%) on eight publicly available and popular CoQA datasets, comparing the strong list-wise benchmark PRO and \textbf{ours with bold}.}
\label{public}
\end{table}



%%%%%%%%%%%%%%%%%%%%%%%%%%%%%%%%%%%%%%%%%%%%%%%%%%%%%
\begin{table}[h]
\centering
\renewcommand{\arraystretch}{1.02}
\begin{tabularx}{0.48\textwidth}{p{1.45cm} p{0.56cm} p{0.6cm} p{0.6cm} p{0.50cm} p{0.45cm} X}
\toprule
\multirow{2}{*}{Method} & \multicolumn{3}{c}{Preference \((\uparrow)\)} & \multicolumn{3}{c}{Accuracy \((\uparrow)\)} \\ \cmidrule{2-4} \cmidrule{5-7}
& \small PrefHit & \small PrefRec & \small Reward & \small CoSim & \small BLEU & \small Rouge \\ \midrule
\small{SeAdpra} & \textbf{34.8} & \textbf{82.5} & \textbf{22.3} & \textbf{69.1} & \textbf{17.4} & \textbf{21.8} \\ 
\small{-w/o PerAl} & \underline{30.4} & 83.0 & 18.7 & 68.8 & \underline{12.6} & 21.0 \\
\small{-w/o PerCo} & 32.6 & 82.3 & \underline{24.2} & 69.3 & 16.4 & 21.0 \\
\small{-w/o \(\Delta_{Se}\)} & 31.2 & 82.8 & 18.6 & 68.3 & \underline{12.4} & 20.9 \\
\small{-w/o \(\Delta_{Po}\)} & \underline{29.4} & 82.2 & 22.1 & 69.0 & 16.6 & 21.4 \\
\small{\(PerCo_{Se}\)} & 30.9 & 83.5 & 15.6 & 67.6 & \underline{9.9} & 19.6 \\
\small{\(PerCo_{Po}\)} & \underline{30.3} & 82.7 & 20.5 & 68.9 & 14.4 & 20.1 \\ 
\bottomrule
\end{tabularx}
\caption{Ablation Results (\%). \(PerCo_{Se}\) or \(PerCo_{Po}\) only employs Single-APDF in Perceptual Comparison, replacing \(\Delta_{M}\) with \(\Delta_{Se}\) or \(\Delta_{Po}\). The bold represents the overall effect. The underlining highlights the most significant metric for each component's impact.}
\label{ablation}
% \vspace{-0.2cm}
\end{table}

\subsection{Dataset}

% These CoQA datasets contain questions and answers from the Stack Overflow data dump\footnote{https://archive.org/details/stackexchange}, intended for training preference models. 

Due to the additional challenges that programming QA presents for LLMs and the lack of high-quality, authentic multi-answer code preference datasets, we turned to StackExchange \footnote{https://archive.org/details/stackexchange}, a platform with forums that are accompanied by rich question-answering metadata. Based on this, we constructed a large-scale programming QA dataset in real-time (as of May 2024), called StaCoCoQA. It contains over 60,738 programming directories, as shown in Table~\ref{tab:stacocoqa_tags}, and 9,978,474 entries, with partial data statistics displayed in Figure~\ref{fig:dataset}. The data format of StaCoCoQA is presented in Table~\ref{fig::stacocoqa}.

The initial dataset \(D_I\) contains 24,101,803 entries, and is processed by the following steps:
(1) Select entries with "Questioner-picked answer" pairs to represent the preferences of the questioners, resulting in 12,260,106 entries in the \(D_Q\).
(2) Select data where the question includes at least one code block to focus on specific-domain programming QA, resulting in 9,978,474 entries in the dataset \(D_C\).
(3) All HTML tags were cleaned using BeautifulSoup \footnote{https://beautiful-soup-4.readthedocs.io/en/latest/} to ensure that the model is not affected by overly complex and meaningless content.
(4) Control the quality of the dataset by considering factors such as the time the question was posted, the size of the response pool, the difference between the highest and lowest votes within a pool, the votes for each response, the token-level length of the question and the answers, which yields varying sizes: 3K, 8K, 18K, 29K, and 64K. 
The controlled creation time variable and the data details after each processing step are shown in Table~\ref{tab:statistics}.

To further validate the effectiveness of SeAdpra, we also select eight popular topic CoQA datasets\footnote{https://huggingface.co/datasets/HuggingFaceH4/stack-exchange-preferences}, which have been filtered to meet specific criteria for preference models \cite{askell2021general}. Their detailed data information is provided in Table~\ref{domain}.
% Examples of some control variables are shown in Table~\ref{tab:statistics}.
% \noindent\textbf{Baselines}. 
% Following the DPO \cite{rafailov2024direct}, we evaluated several existing approaches aligned with human preference, including GPT-J \cite{gpt-j} and Pythia-2.8B \cite{biderman2023pythia}.  
% Next, we assessed StarCoder2 \cite{lozhkov2024starcoder}, which has demonstrated strong performance in code generation, alongside several general-purpose LLMs: Qwen2 \cite{qwen2}, ChatGLM4 \cite{wang2023cogvlm, glm2024chatglm} and LLaMA serials \cite{touvron2023llama,llama3modelcard}.
% Finally, we fine-tuned LLaMA2-7B on the StaCoCoQA and compared its performance with other strong baselines for supervised learning in preference alignment, including SFT, RRHF \cite{yuan2024rrhf}, Silc \cite{zhao2023slic}, DPO, and PRO \cite{song2024preference}.
%%%%%%%%%%%%%%%%%%%%%%%%%%%%%%%%%%%%%%%%%%%%%%%%%%%%%%%%%%%%%%%%%%%%%%%%%%%%%%%%%%%%%%%%%%%%%%%%%%%%%%%%%%%%%%%%%%%%%%%%%%%%%%%%%%

% For preference evaluation, traditional win-rate assessments are costly and not scalable. For instance, when an existing model \(M_A\) is evaluated against comparison methods \((M_B, M_C, M_D)\) in terms of win rates, upgrading model \(M_A\) would necessitate a reevaluation of its win rates against other models. Furthermore, if a new comparison method \(M_E\) is introduced, the win rates of model \(M_A\) against \(M_E\) would also need to be reassessed. Whether AI or humans are employed as evaluation mediators, binary preference between preferred and non-preferred choices or to score the inference results of the modified model, the costs of this process are substantial. 
% Therefore, from an economic perspective, we propose a novel list preference evaluation method. We utilize manually ranking results as the gold standard for assessing human preferences, to calculate the Hit and Recall, referred to as PrefHit and PrefRecall, respectively. Regardless of whether improving model \(M_A\) or expanding comparison method \(M_E\), only the calculation of PrefHit and PrefRecall for the modified model is required, eliminating the need for human evaluation. 
% We also employ a professional reward model\footnote{https://huggingface.co/OpenAssistant/reward-model-deberta-v3-large}
% for evaluation, denoted as the Reward metric.

% \subsection{Baseline} 
% Following the DPO \cite{rafailov2024direct}, we evaluated several existing approaches aligned with human preference, including GPT-J \cite{gpt-j} and Pythia-2.8B \cite{biderman2023pythia}.  
% Next, we assessed StarCoder2 \cite{lozhkov2024starcoder}, which has demonstrated strong performance in code generation, alongside several general-purpose LLMs: Qwen2 \cite{qwen2}, ChatGLM4 \cite{wang2023cogvlm, glm2024chatglm} and LLaMA serials \cite{touvron2023llama,llama3modelcard}.
% Finally, we fine-tuned LLaMA2-7B on the StaCoCoQA and compared its performance with other strong baselines for supervised learning in preference alignment, including SFT, RRHF \cite{yuan2024rrhf}, Silc \cite{zhao2023slic}, DPO, and PRO \cite{song2024preference}.
\subsection{Evaluation Metrics}
\label{sec: metric}
For preference evaluation, we design PrefHit and PrefRecall, adhering to the "CSTC" criterion outlined in Appendix \ref{sec::cstc}, which overcome the limitations of existing evaluation methods, as detailed in Appendix \ref{metric::mot}.
In addition, we demonstrate the effectiveness of thees new evaluation from two main aspects: 1) consistency with traditional metrics, and 2) applicability in different application scenarios in Appendix \ref{metric::ana}.
Following the previous \cite{song2024preference}, we also employ a professional reward.
% Following the previous \cite{song2024preference}, we also employ a professional reward model\footnote{https://huggingface.co/OpenAssistant/reward-model-deberta-v3-large} \cite{song2024preference}, denoted as the Reward.

For accuracy evaluation, we alternately employ BLEU \cite{papineni2002bleu}, RougeL \cite{lin2004rouge}, and CoSim. Similar to codebertscore \cite{zhou2023codebertscore}, CoSim not only focuses on the semantics of the code but also considers structural matching.
Additionally, the implementation details of SeAdpra are described in detail in the Appendix \ref{sec::imp}.
\subsection{Main Results}
We compared the performance of \shortname with general LLMs and strong preference alignment benchmarks on the StaCoCoQA dataset, as shown in Table~\ref{main}. Additionally, we compared SeAdpra with the strongly supervised alignment model PRO \cite{song2024preference} on eight publicly available CoQA datasets, as presented in Table~\ref{public} and Figure~\ref{fig::public}.

\textbf{Larger Model Parameters, Higher Preference.}
Firstly, the Qwen2 series has adopted DPO \cite{rafailov2024direct} in post-training, resulting in a significant enhancement in Reward.
In a horizontal comparison, the performance of Qwen2-7B and LLaMA2-7B in terms of PrefHit is comparable.
Gradually increasing the parameter size of Qwen2 \cite{qwen2} and LLaMA leads to higher PrefHit and Reward.
Additionally, general LLMs continue to demonstrate strong capabilities of programming understanding and generation preference datasets, contributing to high BLEU scores.
These findings indicate that increasing parameter size can significantly improve alignment.

\textbf{List-wise Ranking Outperforms Pair-wise Comparison.}
Intuitively, list-wise DPO-PT surpasses pair-wise DPO-{BT} on PrefHit. Other list-wise methods, such as RRHF, PRO, and our \shortname, also undoubtedly surpass the pair-wise Slic.

\textbf{Both Parameter Size and Alignment Strategies are Effective.}
Compared to other models, Pythia-2.8B achieved impressive results with significantly fewer parameters .
Effective alignment strategies can balance the performance differences brought by parameter size. For example, LLaMA2-7B with PRO achieves results close to Qwen2-57B in PrefHit. Moreover, LLaMA2-7B combined with our method SeAdpra has already far exceeded the PrefHit of Qwen2-57B.

\textbf{Rather not Higher Reward, Higher PrefHit.}
It is evident that Reward and PrefHit are not always positively correlated, indicating that models do not always accurately learn human preferences and cannot fully replace real human evaluation. Therefore, relying solely on a single public reward model is not sufficiently comprehensive when assessing preference alignment.

% In conclusion, during ensuring precise alignment, SeAdpra will focuse on PrefHit@1, even though the trade-off between PrefHit and PrefRecall is a common issue and increasing recall may sometimes lead to a decrease in hit rate. The positive correlation between Reward and BLEU, indicates that improving the quality of the generated text typically enhances the Reward. 
% Most importantly, evaluating preferences solely based on reward is clearly insufficient, as a high reward does not necessarily correspond to a high PrefHit or PrefRecall.
%%%%%%%%%%%%%%%%%%%%%%%%%%%%%%%%%%%%%%%%%%%
%%%%%%%%%%%%
\begin{figure}
  \centering
  \begin{subfigure}{0.49\linewidth}
    \includegraphics[width=\linewidth]{latex/pic/hit.png}
    \caption{The PrefHit}
    \label{scale:hit}
  \end{subfigure}
  \begin{subfigure}{0.49\linewidth}
    \includegraphics[width=\linewidth]{latex/pic/Recall.png}
    \caption{The PrefRecall}
    \label{scale:recall}
  \end{subfigure}
  \medskip
  \begin{subfigure}{0.48\linewidth}
    \includegraphics[width=\linewidth]{latex/pic/reward.png}
    \caption{The Reward}
    \label{scale:reward}
  \end{subfigure}
  \begin{subfigure}{0.48\linewidth}
    \includegraphics[width=\linewidth]{latex/pic/bleu.png}
    \caption{The BLEU}
    \label{scale:bleu}
  \end{subfigure}
  \caption{The performance with Confidence Interval (CI) of our SeAdpra and PRO at different data scales.}
  \label{fig:scale}
  % \vspace{-0.2cm}
\end{figure}
%%%%%%%%%%%%%%%%%%%%%%%%%%%%%%%%%%%%%%%%%%%%%%%%%%%%%%%%%%%%%%%%%%%%%%%%%%%%%%%%%%%%%%%%%%%%%%%%%%%%%%%%%%%%%%%%

\subsection{Ablation Study}

In this section, we discuss the effectiveness of each component of SeAdpra and its impact on various metrics. The results are presented in Table \ref{ablation}.

\textbf{Perceptual Comparison} aims to prevent the model from relying solely on linguistic probability ordering while neglecting the significance of APDF. Removing this Reward will significantly increase the margin, but PrefHit will decrease, which may hinder the model's ability to compare and learn the preference differences between responses.

\textbf{Perceptual Alignment} seeks to align with the optimal responses; removing it will lead to a significant decrease in PrefHit, while the Reward and accuracy metrics like CoSim will significantly increase, as it tends to favor preference over accuracy.

\textbf{Semantic Perceptual Distance} plays a crucial role in maintaining semantic accuracy in alignment learning. Removing it leads to a significant decrease in BLEU and Rouge. Since sacrificing accuracy recalls more possibilities, PrefHit decreases while PrefRecall increases. Moreover, eliminating both Semantic Perceptual Distance and Perceptual Alignment in \(PerCo_{Po}\) further increases PrefRecall, while the other metrics decline again, consistent with previous observations.


\textbf{Popularity Perceptual Distance} is most closely associated with PrefHit. Eliminating it causes PrefHit to drop to its lowest value, indicating that the popularity attribute is an extremely important factor in code communities.

% In summary, each module has a varying impact on preference and accuracy, but all outperform their respective foundation models and other baselines, as shown in Table \ref{main}, proving their effectiveness.


\subsection{Analysis and Discussion}

\textbf{SeAdpra adept at high-quality data rather than large-scale data.}
In StaCoCoQA, we tested PRO and SeAdpra across different data scales, and the results are shown in Figure~\ref{fig:scale}.
Since we rely on the popularity and clarity of questions and answers to filter data, a larger data scale often results in more pronounced deterioration in data quality. In Figure~\ref{scale:hit}, SeAdpra is highly sensitive to data quality in PrefHit, whereas PRO demonstrates improved performance with larger-scale data. Their performance on Prefrecall is consistent. In the native reward model of PRO, as depicted in Figure~\ref{scale:reward}, the reward fluctuations are minimal, while SeAdpra shows remarkable improvement.

\textbf{SeAdpra is relatively insensitive to ranking length.} 
We assessed SeAdpra's performance on different ranking lengths, as shown in Figure 6a. Unlike PRO, which varied with increasing ranking length, SeAdpra shows no significant differences across different lengths. There is a slight increase in performance on PrefHit and PrefRecall. Additionally, SeAdpra performs better at odd lengths compared to even lengths, which is an interesting phenomenon warranting further investigation.


\textbf{Balance Preference and Accuracy.} 
We analyzed the effect of control weights for Perceptual Comparisons in the optimization objective on preference and accuracy, with the findings presented in Figure~\ref{para:weight}.
When \( \alpha \) is greater than 0.05, the trends in PrefHit and BLEU are consistent, indicating that preference and accuracy can be optimized in tandem. However, when \( \alpha \) is 0.01, PrefHit is highest, but BLEU drops sharply.
Additionally, as \( \alpha \) changes, the variations in PrefHit and Reward, which are related to preference, are consistent with each other, reflecting their unified relationship in the optimization. Similarly, the variations in Recall and BLEU, which are related to accuracy, are also consistent, indicating a strong correlation between generation quality and comprehensiveness. 

%%%%%%%%%%%%%%%%%%%%%%%%%%%%%%%%%%%%%%%%%%%%%%%%%%%%%%%%%%%%%%%%%%%%%%%%%%%%%%%%%
\begin{figure}
  \centering
  \begin{subfigure}{0.475\linewidth}
    \includegraphics[width=\linewidth]{latex/pic/Rank1.png}
    \caption{Ranking length}
    \label{para:rank}
  \end{subfigure}
  \begin{subfigure}{0.475\linewidth}
    \includegraphics[width=\linewidth]{latex/pic/weights1.png}
    \caption{The \(\alpha\) in \(Loss\)}
    \label{para:weight}
  \end{subfigure}
  \caption{Parameters Analysis. Results of experiments on different ranking lengths and the weight \(\alpha\) in \(Loss\).}
  \label{fig:para}
  % \vspace{-0.2cm}
\end{figure}
%%%%%%%%%%%%%%%%%%%%%%%%%%%%%%%%%%%%%%%%%%%%
\begin{figure*}
  \centering
  \begin{subfigure}{0.305\linewidth}
    \includegraphics[width=\linewidth]{latex/pic/se2.pdf}
    \caption{The \(\Delta_{Se}\)}
    \label{visual:se}
  \end{subfigure}
  \begin{subfigure}{0.305\linewidth}
    \includegraphics[width=\linewidth]{latex/pic/po2.pdf}
    \caption{The \(\Delta_{Po}\)}
    \label{visual:po}
  \end{subfigure}
  \begin{subfigure}{0.305\linewidth}
    \includegraphics[width=\linewidth]{latex/pic/sv2.pdf}
    \caption{The \(\Delta_{M}\)}
    \label{visual:sv}
  \end{subfigure}
  \caption{The Visualization of Attribute-Perceptual Distance Factors (APDF) matrix of five responses. The blue represents the response with the highest APDF, and SeAdpra aligns with the fifth response corresponding to the maximum Multi-APDF in (c). The green represents the second response that is next best to the red one.}
  \label{visual}
  % \vspace{-0.2cm}
\end{figure*}
%%%%%%%%%%%%%%%%%%%%%%%%%%%%%%%%%%%%%%%%%
\textbf{Single-APDF Matrix Cannot Predict the Optimal Response.} We randomly selected a pair with a golden label and visualized its specific iteration in Figure~\ref{visual}.
It can be observed that the optimal response in a Single-APDF matrix is not necessarily the same as that in the Multi-APDF matrix.
Specifically, the optimal response in the Semantic Perceptual Factor matrix \(\Delta_{Se}\) is the fifth response in Figure~\ref{visual:se}, while in the Popularity Perceptual Factor matrix \(\Delta_{Po}\) (Figure~\ref{visual:po}), it is the third response. Ultimately, in the Multiple Perceptual Distance Factor matrix \(\Delta_{M}\), the third response is slightly inferior to the fifth response (0.037 vs. 0.038) in Figure~\ref{visual:sv}, and this result aligns with the golden label.
More key findings regarding the ADPF are described in Figure \ref{fig::hot1} and Figure \ref{fig::hot2}.

%\vspace{-0.2cm}
\section{Conclusion}
%\vspace{-0.1cm}
This paper theoretically demonstrates how LoRA can be improved from our theoretical analysis in both linear and nonlinear models: the alignment between LoRA's gradient update $(\bm A_t, \bm B_t)$ and the singular subspace of $\bm G^{\natural}$, and adding preconditioners.
Our theory clarifies some potential issues behind gradient alignment work and the theory-grounded algorithm, LoRA-One, obtains promising performance in practical fine-tuning benchmarks.

\section*{Acknowledgment}
F. Liu was supported by Royal Soceity KTP R1 241011 Kan Tong Po Visiting Fellowships. Y. Chen was supported in part by National Science Foundation grants CCF-2233152.

\bibliography{sample}
\bibliographystyle{ims}

\newpage
\appendix
\onecolumn
\enableaddcontentsline
\tableofcontents
\newpage
\section{Symbols and Notations}
\label{app:notation}
\begin{table}[h!]
\fontsize{8}{10}\selectfont
\centering
\renewcommand{\arraystretch}{1.2}
\begin{tabular}{c|c|l}
\hline
\textbf{Symbol} & \textbf{Dimension(s)} & \textbf{Definition} \\
\hline
$\mathcal{N}(\bm \mu, \bm \sigma)$ & - & Multivariate normal distribution with mean vector $\bm \mu$ and covariance matrix $\bm \sigma$ \\
$\mathcal{O}, o, \Omega, \Theta$ & - & Bachmann–Landau asymptotic notation \\
$\|\bm w\|_2$ & - & Euclidean norm of vector $\bm w$ \\
$\|\mathbf{M}\|_{op}$ & - & Operator norm of matrix $\mathbf{M}$ \\
$\|\mathbf{M}\|_{\rm F}$ & - & Frobenius norm of matrix $\mathbf{M}$ \\
$\langle \bm u, \bm v \rangle$ & - & Dot product of vectors $\bm u$ and $\bm v$ \\
$\mathbf{M}\odot \mathbf{N}$ & - & Hadamard product of matrix $\mathbf{M}$ and $\mathbf{N}$\\
\hline
$\bm W^\natural$ & $\mathbb{R}^{d\times k}$ & Pre-trained weight matrix\\
$\Delta$ & $\mathbb{R}^{d\times k}$ & Downstream feature shift matrix\\
$\widetilde{\bm W}^\natural$ & $\mathbb{R}^{d\times k}$ & Downstream weight matrix $\widetilde{\bm W}^\natural=\bm W^\natural+\Delta$\\
$\bm G^\natural$ & $\mathbb{R}^{d\times k}$ & The initial gradient matrix under full fine-tuning\\
$\bm A_t\,,\bm B_t$ & $\mathbb{R}^{d\times r}\,,\mathbb{R}^{r\times k}$ & Learnable low-rank adapters at step $t$\\
$\bm w^\natural_i$ & $\mathbb{R}^d$ & $i^\text{th}$ column of pre-trained weight matrix $\bm W^\natural$ \\
$\widetilde{\bm w}^\natural_i$ & $\mathbb{R}^d$ & $i^\text{th}$ column of downstream weight matrix $\widetilde{\bm W}^\natural$ \\
$\bm w_{t,i}$ & $\mathbb{R}^d$ & $i^\text{th}$ column of adapted weight matrix $\left(\bm W^\natural+\bm A_t \bm B_t\right)$ at step $t$ \\
$\Delta_{i}$ & $\mathbb{R}^d$ & $i^\text{th}$ column of downstream feature matrix $\Delta$\\
$[\bm A_t \bm B_t]_i$ & $\mathbb{R}^d$ & $i^\text{th}$ column of the product of adapters $\bm A_t \bm B_t$\\
$\widetilde{\bm X}$ & $\mathbb{R}^{N\times d}$ & Downstream data matrix\\
$\widetilde{\bm Y}$ & $\mathbb{R}^{N\times d}$ & Downstream label matrix\\
$\widetilde{\bm x}_n$ & $\mathbb{R}^d$ & $n^\text{th}$ downstream data point\\
\hline
$\mathbf{M}^{-1}$ & - & Inverse of matrix $\mathbf{M}$ \\
$\mathbf{M}^\dagger$ & - & Pseudo-inverse of matrix $\mathbf{M}$ \\
$\lambda_i\left(\mathbf{M}\right)$ & $\mathbb{R}$ & $i^\text{th}$ singular value of matrix $\mathbf{M}$ \\
$\lambda_i^*$ & $\mathbb{R}$ & $i^\text{th}$ singular value of downstream feature shift matrix $\Delta$ \\
$\kappa\left(\mathbf{M}\right)$ & $\mathbb{R}$ & The condition number of matrix $\mathbf{M}$ \\
$\kappa$ & $\mathbb{R}$ & The condition number of $\Delta$: $\kappa=\lambda_{\max}^*/\lambda^*_{\min}$ \\
$\kappa^{\natural}$ & $\mathbb{R}$ & The condition number of $\mathbf{G}^\natural$: $ \kappa^{\natural} =\lambda_{\max}\left(\mathbf{G}^\natural\right)/\lambda_{\min}\left(\mathbf{G}^\natural\right)$ \\
$\bm U_{m}\left(\mathbf{M}\right)$ & - & The left singular subspace spanned by the $m$ largest singular values of the input matrix $\mathbf{M}$\\
$\bm U_{m,\perp}\left(\mathbf{M}\right)$ & - & The left singular subspace orthogonal to $\bm U_{m}\left(\mathbf{M}\right)$\\
$\bm V_{m}\left(\mathbf{M}\right)$ & - & The right singular subspace spanned by the $m$ largest singular values of the input matrix $\mathbf{M}$\\
$\bm V_{m,\perp}\left(\mathbf{M}\right)$ & - & The right singular subspace orthogonal to $\bm V_{m}\left(\mathbf{M}\right)$\\
$\bm U_{\bm A}$ & - & The left singular matrix of the compact SVD of $\bm A$\\
$\bm U_{\bm A, \perp}$ & - & The corresponding orthogonal complement of $\bm U_{\bm A}$\\
$\bm V_{\bm A}$ & - & The right singular matrix of the compact SVD of $\bm A$\\
$\bm V_{\bm A, \perp}$ & - & The corresponding orthogonal complement of $\bm V_{\bm A}$\\
\hline
$\sigma(\,\cdot\,)$ & - & ReLU activation function \\
$\sigma'(\,\cdot\,)$ & - & The derivative of ReLU activation function \\
$c_j$ & $\mathbb{R}$ & $j^\text{th}$ Hermite coefficient of ReLU activation function \\
$\operatorname{He}_j(\,\cdot\,)$ & - & $j^\text{th}$ Hermite polynomial\\
\hline
$\nabla_{\mathbf{W}}f\left(\mathbf{W}\right)$ & - & The gradient matrix of function $f$ w.r.t. input matrix $\mathbf{W}$\\
$\widetilde{L}\left(\bm A\,,\bm B\right)$ & - & Loss function under LoRA fine-tuning\\
$L(\bm W)$ & - & Loss function under full fine-tuning \\
\hline
$N$ & - & Number of downstream data \\
$d$ & - & Input dimension of the data \\
$k$ & - & Output dimension of the label \\
$\eta\,,\eta_1\,,\eta_2$ & - & Learning rates \\
$\alpha$ & - & Random initialization scale of low-rank adapter $\bm A_0$ \\
\hline
\end{tabular}
\caption{Essential symbols and notations in this paper.}
\label{tab:notation}
\end{table}

\begin{table}[t]
  \centering
    \resizebox{\linewidth}{!}
    % \scalebox{0.8}
    {
    \begin{tabular}{l|c|cccccc}
    \toprule
    \multirow{2}[2]{*}{Method} & \multirow{2}[2]{*}{Proj.} & MMMU & \multicolumn{2}{c}{MMBench} & \multirow{2}[2]{*}{POPE}  & SQA & OK-  \\
        &  & VAL  & EN & CN     &             & IMG & VQA \\
    \midrule
    LLaVA-1.5 & MLP & 35.7  & 64.3 & 58.3  & 86.8  & 66.8 & 53.4  \\
    SAISA & Linear & 35.7 & 65.3 & 56.6   & 85.8  & 69.2 & 53.6  \\
    \rowcolor{cyan!20} SAISA & MLP & 36.9 & 65.7 & 59.0 & 87.2  & 70.1 & 56.8  \\
    \bottomrule
    \end{tabular}
    }
    \caption{\textbf{Ablation on Projector Designs.} ``Proj." denotes projector type.
    The SAISA model that uses MLPs outperforms the model that uses linear layers.
    Notably, SAISA with linear layers achieves comparable performance to LLaVA-1.5 with MLP.
    }
  \label{tab:linear}
\end{table}


\section{Balancing Patients}
\subsection{Defining Balance}
Patient-provider matching frequently involves objectives beyond match quality due to the complex interactions on both sides of the market. 
We focus on a general class of objectives here: the need to balance patient ``workload'' between different providers. 
Formally, we let each patient correspond to a workload quantity, which we denote $\beta_{i}$. 
This workload can refer to quantities such as their age, level of commodities, and estimated visits per month. 
If we then define each provider to have a workload, $\hat{\beta}_{j}$, we aim to ensure equal distribution of new patients across providers. 
That is, we aim to minimize the variance in workload across providers: 
\begin{equation}
    \min\limits_{X_{i,j}} \frac{1}{P} \sum_{j=1}^{P} |\sum_{i=1}^{N} X_{i,j} \beta_{i} + \hat{\beta}_{j} - \bar{\beta}| 
\end{equation}
Here, $\bar{\beta}$ is the average workload across providers: $\bar{\beta} = \frac{1}{P} \sum_{j=1}^P \sum_{i=1}^{N} X_{i,j} \beta_{i} + \hat{\beta_{j}}$. 
We note that multiple values of $\beta$ are possible, which corresponds to the different notions of workload references earlier. 

We note that such an objective can be written through a linear program as follows: 
\begin{equation}
    \min\limits_{v_{j}} \frac{1}{P} \sum_{j=1}^{P} v_{j}
\end{equation}
So that $-v_{j} \leq \sum_{i=1}^{N} X_{i,j} \beta_{i} + \hat{\beta}_{j} - \bar{\beta}\leq v_{j}$ and $\bar{\beta}$ is a linear function in $X_{i,j}$ as defined earlier. 
Such a formulation allows for easy incorporation into our model.

\subsection{Optimizing for Balance}
By incorporating the need for patient balance across providers, our overall objective function becomes
\begin{equation}
    \max  \sum_{i=1}^{N} \sum_{j=1}^{P} \theta_{i,j} X_{i,j} - \frac{\lambda}{P} \sum_{j=1}^{P} v_{j}
\end{equation}
Here, $\lambda$ is a hyperparameter which balances the need for balancing patients with improving match quality. 
We note that optimizing such a problem with approximation guarantees becomes difficult when $\lambda >> N$, because matching any patients worsens the balance. 
We prove this formally as follows: 
\begin{lemma}
Consider the following Linear Program: 
\begin{equation}
    \max \sum_{i=1}^{N} \sum_{j=1}^{P} \theta_{i,j} X_{i,j} - \frac{\lambda}{P} \sum_{j=1}^{P} v_{j}
\end{equation}
When $\lambda >> N$, any solution with $X_{i,j} = 1$ for some $i,j$ results in an unbounded approximation ratio compared to $\mathrm{OPT}$. 
\end{lemma}

Motivated by this, we instead focus on scenarios where $\lambda$ is small. 
We display results in one-shot assortment decision scenarios with known and random ordering. 
We first show that, with known ordering, our previous algorithms can be used, with little impact on the approximation guarantees, assuming that $\lambda$ is small. 
\begin{lemma}
Suppose we select $X_{i,j}$ to optimize the following Linear Program: 
\begin{equation}
    \max \sum_{i=1}^{N} \sum_{j=1}^{P} \theta_{i,j} X_{i,j} - \frac{\lambda}{P} \sum_{j=1}^{P} v_{j}
\end{equation}
Where $-v_{j} \leq \sum_{i=1}^{N} X_{i,j} \beta_{i} + \hat{\beta}_{j} - \bar{\beta}\leq v_{j}$ for some known $\beta_{i}, \hat{\beta}_{j}$. 
Suppose the Linear Programming solution to this allocates provider $\pi_{i}$ to patient $i$. 
If $\lambda \leq \frac{\theta_{i,j}-\theta_{i,\pi_{i}}}{\beta_{i}}$ for all $i,j$ such that $\theta_{i,j}-\theta_{i,\pi_{i}} \geq 0$, then the Stacked Linear Programming solution is optimal. 
\end{lemma}
We informally note that, in scenarios where such a condition does not hold, a strategy is to only allocate provider $j$, after all patients with $\pi_{i}=j$ have been matched or declined. 
Such a scenario differs from the traditional Stacked Linear Programming scenario, because providers can be matched with multiple patients. 

We next detail two approxation bounds for the known-ordering and random-ordering scenarios: 
\begin{lemma}
Suppose we select $X_{i,j}$ to optimize the following Linear Program: 
\begin{equation}
    \max \sum_{i=1}^{N} \sum_{j=1}^{P} \theta_{i,j} X_{i,j} - \frac{\lambda}{P} \sum_{j=1}^{P} v_{j}
\end{equation}
Where $-v_{j} \leq \sum_{i=1}^{N} X_{i,j} \beta_{i} + \hat{\beta}_{j} - \bar{\beta}\leq v_{j}$ for some known $\beta_{i}, \hat{\beta}_{j}$. 
Suppose that $\beta_{i} \leq \frac{1}{P}$, and let the order of patient responses be known.  
Let the LP solution to this be denoted $\mathrm{ALG}$, and note that prior results showed $\mathrm{ALG} = \mathrm{OPT}$. 
Then using Stacked Linear Programming results in an approximation ratio of $\frac{\mathrm{ALG}}{\mathrm{ALG} + \lambda}$
\end{lemma}

\begin{lemma}
Suppose we select $X_{i,j}$ to optimize the following Linear Program: 
\begin{equation}
    \max \sum_{i=1}^{N} \sum_{j=1}^{P} \theta_{i,j} X_{i,j} - \frac{\lambda}{P} \sum_{j=1}^{P} v_{j}
\end{equation}
Where $-v_{j} \leq \sum_{i=1}^{N} X_{i,j} \beta_{i} + \hat{\beta}_{j} - \bar{\beta}\leq v_{j}$ for some known $\beta_{i}, \hat{\beta}_{j}$. 
Suppose that $\beta_{i} \leq \frac{1}{P}$, and let the order of patient responses be random uniform.  
Let the LP solution to this be denoted $\mathrm{OPT}$. 
Then allocating only the providers from the LP solution to each patient results in an approximation ratio of $\frac{p \mathrm{OPT}}{\mathrm{OPT} + \lambda}$
\end{lemma}
These demonstrate the ability to re-use our previous solutions, while adapting to the varying constraints placed by provider balancing. 


\section{Auxiliary Results for Proofs}
\label{auxiliary}
In this subsection, we present some auxiliary results that are needed for our proof.
First, we present the estimation of the spectral norm of random matrices.
It can be easily derived from \cite{vershynin2018high} and we put it here for the completeness.

\begin{lemma}\citep[Adapted from Theorem 4.6.1]{vershynin2018high}
\label{lem:conrg}
    For a random sub-Gaussian matrix $\widetilde{\bm X} \in \mathbb{R}^{N \times d}$ whose rows are i.i.d. isotropic sub-gaussian random vector with sub-Gaussian norm $K$, then we have the following statement
\[
\mathbb{P} \left(   \left\|\frac{1}{N}\widetilde{\bm X}^{\!\top}\widetilde{\bm X}-\bm I_d\right\|_{op}  > \delta \right) \leq 2 \exp \left( -C N \min\left(\delta^2, \delta\right) \right)\,.
\]
for a universal constant $C$ depending only on $K$.
\end{lemma}

\begin{lemma}\citep[Adapted from Corollary 5.35]{vershynin2010introduction}
\label{lem:init-op-conct}
    For a random standard Gaussian matrix $\bm S\in\mathbb{R}^{d\times r}$ with $[\bm S]_{ij} \sim \mathcal{N}(0, 1)$, if $d > 2r$, we have 
    \begin{align}
        \label{norm-A0}
        \frac{\sqrt{d}}{2} \leq \|\bm S\|_{op} \leq (2 \sqrt{d} + \sqrt{r})\,,
    \end{align}
    with probability at least $1-C \operatorname{exp}(-d)$ for some positive constants $C$.
\end{lemma}

The following results are modified from the proof of \citet[Lemma 8.7]{stoger2021small}.
\begin{lemma}
\label{lem:min-singular-conct}
    Suppose $\bm S\in\mathbb{R}^{d\times r}$ is a random standard Gaussian matrix with $[\bm S]_{ij} \sim \mathcal{N}(0, 1)$ and $\bm U\in\mathbb{R}^{d\times r^*}$ has orthonormal columns. If $r\geq 2r^*$, with probability at least $1-C\operatorname{exp}(-r)$ for some positive constants $C$, we have
    \begin{align*}
        \lambda_{\operatorname{min}}(\bm U^{\!\top}\bm S) & \gtrsim 1\,.
    \end{align*}
    If $r^*\leq r < 2r^*$, by choosing $\xi>0$ appropriately, with probability at least $1-(C \xi)^{r-r^*+1}-C'\operatorname{exp}(-r)$ for some positive constants $C\,,C'$, we have
    \begin{align*}
        \lambda_{\operatorname{min}}(\bm U^{\!\top}\bm S) & \gtrsim \frac{\xi}{r}\,.
    \end{align*}
\end{lemma}

Next, we give a short description of the Hermite expansion of ReLU function via Hermite polynomials. Details can be found in \citet[A.1.1]{damian2022neural} and \cite{arous2021online}.
To be specific, the Hermite expansion of ReLU function $\sigma(x)$ is
\begin{align}
\label{Hermite-sigma}
    \sigma(x)=\sum_{j=1}^\infty \frac{c_j}{j!}\operatorname{He}_j(x) =\frac{1}{\sqrt{2\pi}}+\frac{1}{2}x+\frac{1}{\sqrt{2\pi}}\sum_{j\geq 1}\frac{(-1)^{j-1}}{j!2^j(2j-1)}\operatorname{He}_{2j}(x)\,,
\end{align}
which implies that we can express the Hermite coefficients as
\begin{align}
\label{Hermite-coef}
    \left\{\begin{aligned}
        c_0 & = \frac{1}{\sqrt{2\pi}}\,,\\
        c_1 & = \frac{1}{2}\,,\\
        c_{2j} & = \frac{(-1)^{j-1}}{\sqrt{2\pi}2^j(2j-1)}\quad \text{for }j\geq 1\,.
    \end{aligned}\right.
\end{align}
Furthermore, the derivative of $\sigma(x)$ admits
\begin{align}
\label{Hermite-sigma'}
    \sigma'(x)=\frac{1}{2}+\frac{1}{\sqrt{2\pi}}\sum_{j\geq 0}\frac{(-1)^{j}}{j!2^j(2j+1)}\operatorname{He}_{2j+1}(x)\,.
\end{align}

\begin{lemma}\citep[Corollary 9]{oko2024pretrained}\label{differential}
$\mathbb{E}_{\widetilde{\bm x}}[\nabla^k \sigma(\langle \bm w\,, \widetilde{\bm x}\rangle)] = c_k \bm w^{\otimes k}$ for any $k$ such that $c_k\neq 0$.
\end{lemma}

\begin{lemma}\label{vec-ineq}
For any vectors $\bm u$ and $\bm v$, we have
    \begin{align*}
        \left|\langle \bm u\,, \bm u \rangle^j - \langle \bm u\,, \bm v \rangle^j\right| & \leq j\,\max\left\{\left\|\bm u\right\|_2\,,\left\|\bm v\right\|_2\right\}^{2j-1} \left\|\bm u - \bm v\right\|_2\,.
    \end{align*}
\end{lemma}
\begin{proof}
    First, we analyze the following two scalar variables case
    \begin{align*}
        \left|x^j-y^j\right|\,.
    \end{align*}
    By algebraic identity $\sum_{j=1}^{t-1}x^{t-j-1}y^j=\frac{x^t-y^t}{x-y}$ which is valid for $\forall\,j\in\mathbb{N}^+$, we have
    \begin{align*}
        \left|x^j-y^j\right|&=\left|(x-y)\sum_{i=0}^{j-1}x^{j-i-1}y^i\right|
        \leq |x-y|\sum_{i=0}^{j-1}\max\left\{|x|\,,|y|\right\}^{j-1}
        = j|x-y|\max\left\{|x|\,,|y|\right\}^{j-1}\,.
    \end{align*}
    Now we define $x:=\langle \bm u\,, \bm u \rangle$ and $y:=\langle \bm u\,, \bm v \rangle$, then we can obtain
    \begin{align*}
        \left|\langle \bm u\,, \bm u \rangle^j - \langle \bm u\,, \bm v \rangle^j\right| & \leq j\,\max\left\{\left|\langle \bm u\,, \bm u \rangle\right|\,,\left|\langle \bm u\,, \bm v \rangle\right|\right\}^{j-1}\left|\langle \bm u\,, \bm u \rangle - \langle \bm u\,, \bm v \rangle\right|\\
        & \leq j\,\max\left\{\left\|\bm u\right\|_2^2\,,\left\|\bm u\right\|_2 \left\|\bm v\right\|_2\right\}^{j-1}\left\|\bm u\right\|_2 \left\|\bm u - \bm v\right\|_2\quad \tag*{\color{teal}[by Cauchy-Schwartz inequality]}\\
        & = j\,\max\left\{\left\|\bm u\right\|_2\,,\left\|\bm v\right\|_2\right\}^{2j-1} \left\|\bm u - \bm v\right\|_2\,.
    \end{align*}
\end{proof}

\section{Discussion on Prior Work Based on Gradient Alignment}
\label{app:disGA}

Our initialization strategy in \cref{alg:lora_one_training} (line 4-6) shares some similarity with prior work on gradient alignment, e.g., LoRA-GA \citep{wang2024lora}, and LoRA-pro \citep{wang2024lorapro}.
However, the motivation behind these gradient alignment work differs significantly from ours. The above gradient alignment based algorithms are driven by how to approximate the full fine-tuning gradient by low-rank updates. Instead, our our work is motivated by which subspace $(\bm A_t, \bm B_t)$ will align with and then how to achieve this alignment efficiently so as to finally recover $\Delta$.

Here we take LoRA-GA as an example to explain the potential issue that the spirit of LoRA-GA might not help recover $\Delta$, both theoretically and empirically.
To be specific, LoRA-GA \citep{wang2024lora} also computes the SVD of $\nabla_{\bm W} {L}(\bm W^\natural)$. To ensure the pre-trained model remains unchanged at $t=0$, LoRA-GA the following strategy
\begin{equation}\tag{LoRA-GA}\label{LoRA-GA}
\begin{split}
    &\bm A_0 = -\sqrt{\gamma}\left[\widetilde{\bm U}_{\bm G^\natural}\right]_{[:,1:r]}\,,
    \bm B_0 = \sqrt{\gamma}\left[\widetilde{\bm V}_{\bm G^\natural}\right]_{[:,r+1:2r]}^{\!\top}\,,\\
    &\bm W_{\tt off}^\natural := \bm W^\natural - \frac{\alpha}{\sqrt{r}}\bm A_0 \bm B_0\,.
\end{split}
\end{equation}

Theoretically, LoRA-GA observes  $\operatorname{rank}(
\nabla_{\bm A}\widetilde{L}\left(\bm A_t\,,\bm B_t\right) + \nabla_{\bm B}\widetilde{L}\left(\bm A_t\,,\bm B_t\right)) \leq 2r$ and then proposes to find the best $2r$-rank approximation of one-step full gradient to the first step of LoRA. Accordingly, LoRA-GA chooses the first $r$ singular values for $\bm A_0$ and $(r+1)$th to $2r$th singular values for $\bm B_0$.
However, as pointed by our theory, $\bm B_t$ will also align to the right-side rank-$r^*$ singular subspace of $\bm G^{\natural}$ under random initilization. That means, due to the way LoRA-GA chooses the $(r+1)$th to $2r$th singular values for $\bm B_0$, the iterate $\bm B_t$ does not lie in the desired subspace and may not escape an undesirable subspace. 

Empirically, the mismatch of singular subspace induced by corresponding singular values in LoRA-GA might bring unfavorable performance even in a toy model. We consider the exact-ranked case ($r=r^*$) for fine-tuning task in the linear setting. 
We compare the generalization risk of three initialization strategies: \eqref{eq:spectral-init-linear}, \cref{alg:lora_one_training} without preconditioners, and LoRA-GA trained via vanilla GD. The results are shown in \cref{figs:GA-vs-Ours}. We can empirically observe that LoRA-GA fails to generalize and remain at a high-risk level throughout training. In contrast, \eqref{eq:spectral-init-linear} and \cref{alg:lora_one_training} both can generalize well. This empirically demonstrates the optimality of choosing top-$r$ singular subspace of $\bm G^\natural$.

Before the submission deadline we became aware of the concurrent work \cite{ponkshe2024initialization}, which uses the same initialization for $\bm B_0$ as in line 6 of our \cref{alg:lora_one_training}. However, the motivation, problem setting, and theoretical analysis are totally different between our work and theirs. Moreover, our \cref{alg:lora_one_training} also introduces the preconditioners and is able to efficiently handle ill-conditioned cases, and this is not available in \cite{ponkshe2024initialization}.

\section{Experimental Settings and Additional Results}
\label{exp-settings}

In \cref{exp:toy-setting}, we firstly provide the experimental details of small-scale experiments in our main text, e.g., \cref{figs:GA-vs-Ours} and \cref{fig:small-init}. 
Experimental settings of NLP tasks in the main text are given by \cref{app:expNLP}.
We also include the fine-tuning experiments on LLMs in \cref{sec:exp:llm}.
More ablation study is given by \cref{appx:ablation}. Finally, we visualize the singular values of both the pre-trained weights and the difference weights after fine-tuning in \cref{SV-figs}. All small-scale experiments were performed on AMD EPYC 7B12 CPU. All experiments for T5 base model and Llama 2-7B were performed on Nvidia A100 GPU (40GB).

\subsection{Small-Scale Experiments}
\label{exp:toy-setting}

Here we give the experimental details of \cref{figs:GA-vs-Ours} and \cref{fig:small-init}. Besides, we plot the GD trajectories under \eqref{eq:spectral-init-linear} and \eqref{eq:lorainit} for comparison.\\

\noindent
{\bf Details for \cref{figs:GA-vs-Ours}:} For the exact-ranked setting, we take $d=k=100$, $N=1600$, and $r=r^*=4$. We sample each element of $\bm W^\natural$ independently from $\mathcal{N}(0\,,1)$. We construct $\Delta:=\bm U \bm V^{\!\top}$ where $\bm U\in\mathbb{R}^{100\times 4}$ and $\bm V\in\mathbb{R}^{100\times 4}$ are obtained from the SVD of a matrix whose elements are independently sampled from $\mathcal{N}(0\,,1)$. For LoRA-One (-) and LoRA-GA (-), we use learning rate $\eta=\frac{1}{35}$ and stable parameter $s=2$. For \eqref{eq:spectral-init-linear} (-), we use learning rate $\eta=\frac{1}{10}$ and $\gamma=1$. 

For the ill-conditioned setting, we take $d=k=100$, $N=1600$, $r^*=4$, and $r=8$. We construct $\Delta:=\bm U \bm S^* \bm V^{\!\top}$ where $\bm U\in\mathbb{R}^{100\times 4}$ and $\bm V\in\mathbb{R}^{100\times 4}$ are obtained from the SVD of a matrix whose elements are independently sampled from $\mathcal{N}(0\,,1)$, and $\bm S^*=\operatorname{Diag}\left(1\,,0.75\,,0.5\,,0.25\right)$. For algorithms without preconditioners, we set the learning rate to be $\eta=\frac{1}{20}$. For algorithms with preconditioners, we set the learning rate to be $\eta=\frac{1}{2}$. For LoRA-One, LoRA-One (-), LoRA-GA (-), and LoRA-GA (+), we set the stable parameter $s=2$. For \eqref{eq:spectral-init-linear} (-) and \eqref{eq:spectral-init-linear} (+), we take $\gamma=1$. All damping parameters $\lambda$ for preconditioners are set to be 0.001.\\

\noindent
{\bf Details for \cref{fig:small-init}:} We examine for dimension $d=k=100$ and $d=k=1000$. We set $N=16d$, $r^*=4$, and $r=8$. We construct $\Delta:=\bm U \bm V^{\!\top}$ where $\bm U\in\mathbb{R}^{100\times 4}$ and $\bm V\in\mathbb{R}^{100\times 4}$ are obtained from the SVD of a matrix whose elements are independently sampled from $\mathcal{N}(0\,,1)$. We initialize $\bm A_0$ and $\bm B_0$ via \eqref{eq:lorainit} over variance $\alpha^2\in\{1\,,0.1\,,0.01\,,0.001\,,0.0001\}$. We set learning rate $\eta=\frac{1}{64}$. We run $1500$ GD steps for each case.\\

\noindent
{\bf Comparison on GD trajectories of \cref{fig:phase-transi}:}
Here we conduct a toy experiment to intuitively compare the GD trajectories under \eqref{eq:spectral-init-linear} and \eqref{eq:lorainit}. We fine-tune a simple pre-trained model $y=\bm x^{\!\top}\bm w^\natural$ on downstream data generated by $\widetilde{y}=\widetilde{\bm x}^{\!\top}(\bm w^\natural+\bm w)$, where $\bm x^{\!\top}\,,\widetilde{\bm x}\,,\bm w^\natural\,,\bm w\in\mathbb{R}^2$ and $y\,,\widetilde{y}\in\mathbb{R}$. We propose to use LoRA to fine-tune this model by $\widehat{y} = \widetilde{\bm x}^{\!\top}(\bm w^\natural+b \bm a)$ where $\bm a = [a_1\,a_2]^{\!\top}\in\mathbb{R}^2$ and $b\in\mathbb{R}$. Without loss of generality, we set $\bm w^\natural=\bm 0$ and $\bm w = [2\,\,1]^{\!\top}$. The set of global minimizers to this problem is $\{a_1^*=2/t\,,a_2^*=1/t\,,b^*=t\mid t \in \mathbb{R}\}$. We generate 4 data points $(\widetilde{\bm x}_1\,,\widetilde{\bm x}_2\,,\widetilde{\bm x}_3\,,\widetilde{\bm x}_4)$ whose elements are independently sampled from $\mathcal{N}(0\,,1)$ and calculate for $(\widetilde{y}_1\,,\widetilde{y}_2\,,\widetilde{y}_3\,,\widetilde{y}_4)$. We use the squared loss $\frac{1}{8}\sum_{i=1}^4 (\widetilde{y}_i-b\widetilde{\bm x}^{\!\top} \bm a)^2$. For \eqref{eq:lorainit}, we initialize each element of $\bm a_0$ from $\mathcal{N}(0\,,1)$ and $b_0=0$. Notice that the variance $1$ follows from the Kaiming initialization \citep{he2015delving}. For \eqref{eq:spectral-init-linear}, we first calculate the one-step full gradient, i.e. $\bm g^\natural := \frac{1}{4}\sum_{i=1}^4 \widetilde{y}_i^2 \widetilde{\bm x}_i$.
Accordingly, we initialize $\bm a_0 = \frac{\bm g^\natural}{\sqrt{\|\bm g^\natural\|_2}\,.}$ and $b_0 = \sqrt{\|\bm g^\natural\|_2}$. Next, we run GD to train $\bm a$ and $b$ for $1000$ steps with learning rate $\eta=0.1$. For each initialization strategy and data generation, we run for 3 different seeds. The starting points and stopping points with corresponding loss values are presented in \cref{tab:phase-spec} for \eqref{eq:spectral-init-linear} and \cref{tab:phase-random} for \eqref{eq:lorainit}.
Our experiments in \cref{fig:phase-transi} show that spectral initialization enables faster convergence to the global minimizer compared to LoRA initialization.
\begin{table}[h]
    \caption{The details of starting points with initial loss and stopping points with final loss under \eqref{eq:spectral-init-linear} over 3 runs.}
    \label{tab:phase-spec}
    \centering
    \begin{tabular}{ccccc}
    \toprule
         & Starting Point & Initial Loss & Stopping Point & Final Loss \\
         \midrule
       Run 1  & $\bm a=[0.26\,,0.55]^{\!\top}\,, b=0.61$ & $0.39$ & $\bm a=[1.34\,,0.67]^{\!\top}\,, b=1.49$ & \SI{5e-13}{} \\
       \midrule
       Run 2  & $\bm a=[1.10\,,-0.27]^{\!\top}\,, b=1.10$ & $0.38$ & $\bm a=[1.35\,,0.68]^{\!\top}\,, b=1.48$ & \SI{1e-13}{} \\
       \midrule
       Run 3 & $\bm a=[0.96\,,0.35]^{\!\top}\,, b=1.02$ & $0.34$ & $\bm a=[1.34\,,0.67]^{\!\top}\,, b=1.49$ & \SI{4e-13}{} \\
       \bottomrule
    \end{tabular}
\end{table}

\begin{table}[h]
    \caption{The details of starting points with initial loss and stopping points with final loss under \eqref{eq:lorainit} over 3 runs.}
    \label{tab:phase-random}
    \centering
    \begin{tabular}{ccccc}
    \toprule
         & Starting Point & Initial Loss & Stopping Point & Final Loss \\
         \midrule
       Run 1  & $\bm a=[-0.35\,,2.63]^{\!\top}\,, b=-0.03$ & $0.43$ & $\bm a=[-2.49\,,-1.24]^{\!\top}\,, b=-0.80$ & \SI{1e-10}{} \\
       \midrule
       Run 2  & $\bm a=[0.14\,,-1.68]^{\!\top}\,, b=0.10$ & $0.82$ & $\bm a=[1.81\,,0.91]^{\!\top}\,, b=1.10$ & \SI{1e-13}{} \\
       \midrule
       Run 3 & $\bm a=[-1.44\,,0.98]^{\!\top}\,, b=0.03$ & $0.97$ & $\bm a=[1.84\,,0.92]^{\!\top}\,, b=1.08$ & \SI{6e-13}{} \\
       \bottomrule
    \end{tabular}
\end{table}

\subsection{Natural Language Understanding}
\label{app:expNLP}

In the main text of \cref{sec:algoexp}, we have presented the experimental comparisons between \cref{alg:lora_one_training} and typical LoRA based algorithms. 
For experimental details, we follow the configuration of prompt tuning as \cite{wang2024lora}. The general hyperparameter settings are provides in \cref{tab:nlu-general}. Also, we employ the scaling parameter $\frac{\sqrt{d_\text{out}}}{s}$ for LoRA-One (\cref{alg:lora_one_training}) derived in \cite{wang2024lora} which is proven to be numerically stable. To ensure a fair comparison, we tune the learning rate via grid search over $\{ \SI{1e-3}{} , \SI{5e-4}{} , \SI{2e-4}{} , \SI{1e-4}{} , \SI{5e-5}{} , \SI{2e-5}{} , \SI{1e-5}{} \}$.

Furthermore, we fine-tune the model using one step update from full-batch gradient descent under full fine-tuning. To optimize GPU memory usage, we adopt the averaged gradient computation method from \cite{lv2023full, wang2024lora} to compute the full gradient, which is then manually added to the pre-trained weights, scaled by the learning rate. 

Besides, we notice that the test accuracy on the MNLI dataset remains $0.0\%$ for the first dozen steps in both full fine-tuning and LoRA fine-tuning. So we omit results on this dataset.
We conjecture that this is due to the significant discrepancy between pre-trained tasks and downstream tasks. For SST-2, CoLA, QNLI, and MRPC, the learning rates are set to be $\{ \SI{5e-4}{} , \SI{1e-2}{} , \SI{2e-2}{} , \num{0.5} \}$.

\begin{table}[h]
    \centering
     \caption{Hyperparameters for LoRA fine-tuning on T5-base model.}
    \label{tab:nlu-general}
    \begin{tabular}{ccccccc}
        \toprule
        Epoch & Optimizer & $(\beta_1, \beta_2)$ & $\epsilon$ & Batch Size \\
        \midrule
        1 & AdamW & (0.9, 0.999) & $\SI{1e-8}{}$ & 32 \\
        \midrule
        Warm-up Ratio & LoRA Alpha & $s$ (if needed) & $\lambda$ (if needed) & \#Runs \\
        \midrule
        0.03 & 16 & 16 & $\SI{1e-6}{}$ & 3 \\
        \midrule
        Weight Decay & LR Scheduler  & Sequence Length & Precision & \\
        \midrule
        0 & cosine & 128 & FP32 & \\
        \bottomrule
    \end{tabular}
\end{table}

\subsection{Experiments on LLM}
\label{sec:exp:llm}
We use a stronger baseline for full fine-tuning, as provided in \cite{wang2024lorapro}, compared to those in \cite{wang2024lora}. For vanilla LoRA, due to the limitation of computational resources, we use the results of LoRA with rank $8$ from \cite{wang2024lora}. For LoRA-GA, we pick the best results from \citep{wang2024lora}. We align our generation configuration and stable parameter $s$ with LoRA-GA \cite{wang2024lora} to ensure a fair comparison. The hyperparameter settings are provided in \cref{tab:llama-general}. For the learning rates of LoRA-One, we conduct a grid search over $\{ \SI{5e-5}{} , \SI{2e-5}{} , \SI{1e-5}{} \}$, following the configuration used in \cite{wang2024lora}.

\begin{table}[h]
    \centering
     \caption{Hyperparameters for LoRA fine-tuning on Llama 2-7B model.}
    \label{tab:llama-general}
    \begin{tabular}{ccccc}
        \toprule
        Epoch & Optimizer & $(\beta_1, \beta_2)$ & $\epsilon$ & Batch Size \\
        \midrule
        1 & AdamW & (0.9, 0.999) & $\SI{1e-8}{}$ & 32\\
        \midrule
        Warm-up Ratio & LoRA Alpha & $s$ (if needed) & $\lambda$ (if needed) & \#Runs\\
        \midrule
        0.03 & 16 & 64 & $\SI{1e-6}{}$ & 3 \\
        \midrule
        Weight Decay & LR Scheduler & Sequence Length & Precision & \\
        \midrule
        0 & cosine & 1024 & FP32 & \\
        \bottomrule
    \end{tabular}
\end{table}


\subsection{Ablation Study}
\label{appx:ablation}
In this subsection, we compare $5$ algorithms to provide insights for practical algorithm design. The details of $5$ algorithms are summarized in \cref{tab:descrip-alg}. The details of means and standard deviations over 3 runs are shown in \cref{tab:ab-cola} for CoLA and \cref{tab:ab-mrpc} for MRPC. The hyperparameter settings for LoRA-One, LoRA-One (-), LoRA-GA (-), and LoRA-GA (+) are same as the settings used in \cref{app:expNLP}. We tune the learning rates via grid search over $\{  \SI{1e-3}{}, \SI{5e-4}{}, \SI{2e-4}{},\SI{1e-4}{}, \SI{5e-5}{},\SI{2e-5}{},\SI{1e-5}{} \}$ to ensure a fair comparison. The implement details of Spectral (-) are provided in \cref{alg:spec}, which is a scaled version of \eqref{eq:spectral-init-linear} without preconditioning. We notice that Spectral (-) is highly sensitive to hyperparameters which makes it hard to tune. The general hyperparameters of Spectral (-) is same as the settings used in \cref{app:expNLP}. Here we provide the LoRA alpha and learning rates for Spectral (-) in \cref{tab:spec-hyp}.

\begin{table}[ht]
    \centering
    \caption{Initialization strategies and corresponding optimizers for ablation study.}
    \begin{tabular}{ccc}
      \toprule
         & Initialization & Optimizer \\
      \midrule
      LoRA-One   &\cref{alg:lora_one_training} (1-8) & Prec-AdamW\\
      LoRA-One (-) &\cref{alg:lora_one_training} (1-8) & AdamW\\
      Spectral (-) &\cref{alg:spec} (1-5) & AdamW\\
      LoRA-GA (-) &\eqref{LoRA-GA} & AdamW\\
      LoRA-GA (+) &\eqref{LoRA-GA} & Prec-AdamW\\
      \bottomrule
    \end{tabular}
    \label{tab:descrip-alg}
\end{table}

\begin{table}[h!]
\centering
\caption{Accuracy comparison across different methods on CoLA under three ranks, i.e. $r=8\,,32\,,128$. LoRA-One (-) stands for training with AdamW without preconditioning under initialization by line 1-8 in \cref{alg:lora_one_training}.}
\label{tab:ab-cola}
\begin{tabular}{cccccc}
\toprule
\textbf{Rank} & LoRA-One & LoRA-One (-) & Spectral (-) & LoRA-GA (-) & LoRA-GA (+) \\
\midrule
8 & {81.08}$_{\pm 0.36}$ & 80.83$_{\pm 0.54}$ &\textbf{81.40}$_{\pm 0.31}$  & 80.57$_{\pm 0.20}$ & 80.57$_{\pm 0.12}$\\
32 & \textbf{81.34}$_{\pm 0.51}$ & 81.30$_{\pm 0.16}$ &81.18$_{\pm 0.30}$& 80.86$_{\pm 0.23}$ & 80.92$_{\pm 0.34}$\\
128 & {81.53}$_{\pm 0.36}$ & 81.34$_{\pm 0.12}$ &\textbf{81.62}$_{\pm 0.48}$& 80.95$_{\pm 0.35}$ & 80.02$_{\pm 0.64}$\\
\bottomrule
\end{tabular}
\end{table}

\begin{table}[h!]
\centering
\caption{Accuracy comparison across different methods on MRPC under three ranks, i.e. $r=8\,,32\,,128$. LoRA-One (-) stands for training with AdamW without preconditioning under initialization by line 1-8 in \cref{alg:lora_one_training}.}
\begin{tabular}{cccccc}
\toprule
\textbf{Rank} & LoRA-One & LoRA-One (-) & Spectral (-) & LoRA-GA (-) & LoRA-GA (+) \\
\midrule
8 & 86.77$_{\pm 0.53}$ & \textbf{87.50}$_{\pm 0.60}$ &86.19$_{\pm 0.42}$& 85.29$_{\pm 0.24}$& 85.87$_{\pm 0.31}$\\
32 & \textbf{87.34}$_{\pm 0.31}$ & 87.34$_{\pm 0.42}$ &86.02$_{\pm 0.20}$& 86.36$_{\pm 0.42}$ &85.78$_{\pm 0.20}$\\
128 & \textbf{88.40}$_{\pm 0.70}$ & 87.26$_{\pm 0.20}$ &86.03$_{\pm 0.20}$& 85.46$_{\pm 0.23}$ &87.01$_{\pm 0.35}$\\
\bottomrule
\end{tabular}
\label{tab:ab-mrpc}
\end{table}

\begin{algorithm}[!h]
\caption{\eqref{eq:spectral-init-linear} training for a specific layer}
\label{alg:spec}
\begin{algorithmic}[1]
\Input Pre-trained weight $\bm W^\natural$, batched data $\{\mathcal{D}_t\}_{t=1}^{T}$, LoRA rank $r$, LoRA alpha $\alpha$, loss function $L$, scaling parameter $\gamma$
\Initialize
\STATE Compute $\bm G^\natural \gets -\nabla_{\bm W} L(\bm W^\natural)$
\STATE $\bm U, \bm S, \bm V \gets \text{SVD}\left(\bm G^\natural\right)$
\STATE $\bm A_0 \gets \sqrt{\gamma}\cdot\bm U_{[:,1:r]}\bm S^{1/2}_{[1:r]}$
\STATE $\bm B_0 \gets \sqrt{\gamma}\cdot \bm S^{1/2}_{[1:r]}\bm V^{\!\top}_{[:,1:r]}$
\STATE Clear $\bm G^\natural$
\Train
\FOR{$t=1\,,...\,,T$}
\STATE Update parameters $\bm A_t$ and $\bm B_t$ by AdamW given $\mathcal{D}_t$
\ENDFOR
\Return $\bm W^\natural + \frac{\alpha}{\sqrt{r}} \bm A_{T} \bm B_{T}$
\end{algorithmic}
\end{algorithm}

\begin{table}[h!]
    \centering
    \caption{Specific hyperparameter settings for Spectral (-) (see details in \cref{alg:spec}) used in \cref{appx:ablation}.}
    \begin{tabular}{cccc|ccc}
    \toprule
        Rank & \multicolumn{3}{c|}{CoLA} & \multicolumn{3}{c}{MRPC} \\
        \midrule
         & LR & LoRA Alpha & $\gamma$ & LR & LoRA Alpha & $\gamma$ \\
         \midrule
        8 & $\SI{2e-3}{}$ & $\sqrt{8}$ & $0.01$ & $\SI{6e-4}{}$ & $1$ & $0.01$ \\
        32 & $\SI{2e-3}{}$ & $\sqrt{32}$ & $0.01$ & $\SI{2e-3}{}$ & $16$ & $0.01$  \\
        128 & $\SI{2e-3}{}$ & $1$ & $0.01$  & $\SI{9e-4}{}$ & $1$ & $0.01$  \\
        \bottomrule
    \end{tabular}
    \label{tab:spec-hyp}
\end{table}


\subsection{Comparison of Singular Values}
\label{SV-figs}
First, we collect top-$32$ singular values for each pre-trained layer $\mathbf{W}^\natural$ of pre-trained T5-base model \citep{raffel2020exploring}. Next, we perform full fine-tuning to the pre-trained model on SST-2 dataset from GLUE. To ensure better convergence, we take the hyperparameter settings which are presented in \cref{tab:sv_config}. After training, we collect top-$32$ singular values for each difference weights, i.e. $\Delta \mathbf{W} = \mathbf{W}_\text{fine-tuned} - \mathbf{W}^\natural$. The results are shown in \cref{fig:SV}. The hyperparameter settings for full fine-tuning are provided in \cref{tab:sv_config}.

We observe that, across all layers, the singular values of the pre-trained weights are significantly larger than those of the difference weights.
\begin{table}[h!]
    \caption{Hyperparameters for full fine-tuning on T5-base model used for \cref{SV-figs}.}
    \label{tab:sv_config}
    \centering
    \begin{tabular}{ccccc}
    \toprule
    Epoch & Optimizer & $(\beta_1, \beta_2)$ & $\epsilon$ & Batchsize \\
    \midrule
    10 & AdamW & $(0.9, 0.999)$ & $\SI{1e-8}{}$ & 32 \\
    \midrule
    Weight Decay & LR & LR Scheduler & Warm-up Ratio & \\
    \midrule
    0.1 & $\SI{1e-4}{}$ & cosine & 0.03 & \\
    \bottomrule
    \end{tabular}
\end{table}
\end{document}

