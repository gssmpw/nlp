\section{Conclusions}
We introduced the generative seismic foundation model (GSFM), a novel framework built upon generative diffusion models (GDMs) to address multi-task seismic processing. GSFM leverages a generative approach to learn and capture the underlying joint distribution of seismic data, aiming to represent clean, complete, and broadband characteristics. By encoding tasks with class labels and integrating synthetic pre-training with iterative fine-tuning on field data, GSFM achieves a unified framework for seismic denoising, backscattered noise attenuation, interpolation, and low-frequency extrapolation. 

On synthetic data, our pre-trained GSFM achieved performance comparable to traditional pre-training strategies followed by extensive fine-tuning, while significantly outperforming a benchmark model with the same architecture across all tasks. The results demonstrated that GSFM, even without task-specific adjustments, delivers robust and high-quality processing outputs. On field data, the iterative fine-tuning strategy we proposed effectively addressed the generalization challenges inherent in traditional pre-training and fine-tuning paradigms. The fine-tuned GSFM consistently outperformed both benchmarks, establishing its ability to adapt to field data distributions while preserving computational efficiency. 

Through comparative experiments, we demonstrated that our iterative fine-tuning strategy is optimal for refining GSFM on field data. This strategy not only improved processing performance but also provided a clear guideline for applying our pre-trained GSFM to real-world scenarios. Furthermore, the uncertainty quantification capability of our GSFM highlighted its potential for evaluating the reliability of processing results, adding a layer of interpretability that is critical for decision-making in seismic workflows. Also, we can use uncertainty quantification to gudie our fine-tuning stage and, thus, to evaluate whether our GSFM is properly trained. 

In summary, GSFM represents a significant step forward in seismic processing by unifying multiple tasks under a single generative framework. Its ability to generalize across synthetic and field data, coupled with its efficiency and versatility, demonstrates the value of incorporating GDMs into geophysical applications. While challenges remain, such as addressing different input-target distributions, GSFM establishes a strong foundation for future research, offering practical solutions for seismic processing.