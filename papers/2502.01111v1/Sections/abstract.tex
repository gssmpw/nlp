\begin{abstract}
Seismic data often face challenges in their utilization due to noise contamination, incomplete acquisition, and limited low-frequency information, which hinder accurate subsurface imaging and interpretation. Traditional processing methods rely heavily on task-specific designs to address these challenges and fail to account for the variability of data. To address these limitations, we present a generative seismic foundation model (GSFM), a unified framework based on generative diffusion models (GDMs), designed to tackle multi-task seismic processing challenges, including denoising, backscattered noise attenuation, interpolation, and low-frequency extrapolation. GSFM leverages a pre-training stage on synthetic data to capture the features of clean, complete, and broadband seismic data distributions and applies an iterative fine-tuning strategy to adapt the model to field data. By adopting a target-oriented diffusion process prediction, GSFM improves computational efficiency without compromising accuracy. Synthetic data tests demonstrate GSFM surpasses benchmarks with equivalent architectures in all tasks and achieves performance comparable to traditional pre-training strategies, even after their fine-tuning. Also, field data tests suggest that our iterative fine-tuning approach addresses the generalization limitations of conventional pre-training and fine-tuning paradigms, delivering significantly enhanced performance across diverse tasks. Furthermore, GSFM’s inherent probabilistic nature enables effective uncertainty quantification, offering valuable insights into the reliability of processing results.
\end{abstract}

\keywords{Generative foundation model \and Generative diffusion models \and Multi-task seismic processing}