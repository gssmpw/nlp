\section{Conclusion}

In this work, we introduced OpenTAD, a unified framework that standardizes and modularizes temporal action detection across diverse methods and datasets, providing the most extensive TAD codebase available, to the best of our knowledge. 
OpenTAD facilitates fair comparisons across different methods and enables seamless plug-in implementations for new innovations within its modular components. 
Leveraging OpenTAD, we conducted a comprehensive analysis and identified key factors that significantly impact TAD performance, such as second-stage processing and neck architectures. Based on these insights, we developed a new state-of-the-art TAD method. 
We invite the research community to engage with OpenTAD, which is publicly available along with all associated resources, to drive further innovation in video understanding. Through collaborative efforts, we envision OpenTAD advancing not only action detection but also broader video understanding tasks.
