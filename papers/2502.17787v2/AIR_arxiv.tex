% This must be in the first 5 lines to tell arXiv to use pdfLaTeX, which is strongly recommended.
\pdfoutput=1
% In particular, the hyperref package requires pdfLaTeX in order to break URLs across lines.

\documentclass[11pt]{article}

% Change "review" to "final" to generate the final (sometimes called camera-ready) version.
% Change to "preprint" to generate a non-anonymous version with page numbers.
\usepackage[preprint]{acl}

% Standard package includes
\usepackage{times}
\usepackage{latexsym}

% For proper rendering and hyphenation of words containing Latin characters (including in bib files)
\usepackage[T1]{fontenc}
% For Vietnamese characters
% \usepackage[T5]{fontenc}
% See https://www.latex-project.org/help/documentation/encguide.pdf for other character sets

% This assumes your files are encoded as UTF8
\usepackage[utf8]{inputenc}

% This is not strictly necessary, and may be commented out,
% but it will improve the layout of the manuscript,
% and will typically save some space.
\usepackage{microtype}

% This is also not strictly necessary, and may be commented out.
% However, it will improve the aesthetics of text in
% the typewriter font.
\usepackage{inconsolata}

%Including images in your LaTeX document requires adding
%additional package(s)
\usepackage{graphicx}
\usepackage{fontawesome5}

%%%%%%%%%%%%%%%%%%%%%%%%%%%%%%%%%%%%%%%%%%%%%%%%%%%%%%%%%%%%%%%%%%%%%%%%%%%%%%%
% lsl add

\usepackage{marvosym}
\usepackage{textcomp} 
\usepackage{amsmath}
\usepackage[section]{placeins}
\usepackage{multirow}
\usepackage{makecell}
\usepackage{amssymb}
\usepackage{xspace}
\usepackage{enumitem}
\usepackage{tcolorbox}
\usepackage{booktabs}
\usepackage{multicol}
\usepackage{cuted}
\usepackage{tabularx}
\usepackage{hyperref}
\usepackage{listings}
\usepackage{mathptmx}
\usepackage{xcolor}
\usepackage{subfigure}
\usepackage{color}
\usepackage{algorithm}
\usepackage{algorithmic}
\definecolor{codegray}{rgb}{0.5, 0.5, 0.5}
\definecolor{codepurple}{rgb}{0.5, 0, 0.5}
\lstdefinestyle{mystyle}{
    % backgroundcolor=\color{backcolour},		% 背景色   
    keywordstyle= \color{ blue!70},			% 关键字/程序语言中的保留字颜色
    commentstyle= \color{red!50!green!50!blue!50!},		% 程序中注释的颜色
    %commentstyle= \color[RGB]{40, 400, 255}
    numberstyle=\tiny\color{codegray},		% 左侧行号显示的颜色
    stringstyle=\color{codepurple},
    basicstyle=\ttfamily,
    breakatwhitespace=false,         
    breaklines=true,		% 对过长的代码自动换行                
    captionpos=b,                    
    keepspaces=true,             
    numbersep=5pt,                  
    showspaces=false,                
    showstringspaces=false,		% 不显示字符串中的空格
    showtabs=false,                  
    tabsize=2,
    % frame=none,		% 不显示边框
    frame=shadowbox,	% 边框阴影
    %   escapebegin=\begin{CJK*},escapeend=\end{CJK*},      % 代码中出现中文必须加上,否则报错
}

\newcommand{\redtext}[1]{{\color{red}#1}}
\newcommand{\ourmethod}[0]{DREAM\xspace}
\newcommand{\ie}{\emph{i.e.}\xspace}
\newcommand{\aka}{\emph{a.k.a}\xspace}
\newcommand{\eg}{\emph{e.g.}\xspace}
\newcommand{\etal}{\emph{et al.}\xspace}
\newcommand{\ignore}[1]{}

\usepackage{lipsum}
\usepackage{environ}
\NewEnviron{SMALL}{% 
    \scalebox{1}{$\BODY$} 
} 
% \theoremstyle{definition}
\newtheorem{definition}{Definition}

\newcommand{\cloudicon}{\ensuremath{%
  \mathchoice{\includegraphics[height=2ex]{source/cloud.png}}
    {\includegraphics[height=2ex]{source/cloud.png}}
    {\includegraphics[height=1.5ex]{source/cloud.png}}
    {\includegraphics[height=1ex]{source/cloud.png}}
}}

\definecolor{deepgreen}{RGB}{0, 70, 0}
\definecolor{deepred}{RGB}{182, 32, 22}

% If the title and author information does not fit in the area allocated, uncomment the following
%
%\setlength\titlebox{<dim>}
%
% and set <dim> to something 5cm or larger.

%\author{
%  \textbf{First Author\textsuperscript{1}},
%  \textbf{Second Author\textsuperscript{1,2}},
%  \textbf{Third T. Author\textsuperscript{1}},
%  \textbf{Fourth Author\textsuperscript{1}},
%\\
%  \textbf{Fifth Author\textsuperscript{1,2}},
%  \textbf{Sixth Author\textsuperscript{1}},
%  \textbf{Seventh Author\textsuperscript{1}},
%  \textbf{Eighth Author \textsuperscript{1,2,3,4}},
%\\
%  \textbf{Ninth Author\textsuperscript{1}},
%  \textbf{Tenth Author\textsuperscript{1}},
%  \textbf{Eleventh E. Author\textsuperscript{1,2,3,4,5}},
%  \textbf{Twelfth Author\textsuperscript{1}},
%\\
%  \textbf{Thirteenth Author\textsuperscript{3}},
%  \textbf{Fourteenth F. Author\textsuperscript{2,4}},
%  \textbf{Fifteenth Author\textsuperscript{1}},
%  \textbf{Sixteenth Author\textsuperscript{1}},
%\\
%  \textbf{Seventeenth S. Author\textsuperscript{4,5}},
%  \textbf{Eighteenth Author\textsuperscript{3,4}},
%  \textbf{Nineteenth N. Author\textsuperscript{2,5}},
%  \textbf{Twentieth Author\textsuperscript{1}}
%\\
%\\
%  \textsuperscript{1}Affiliation 1,
%  \textsuperscript{2}Affiliation 2,
%  \textsuperscript{3}Affiliation 3,
%  \textsuperscript{4}Affiliation 4,
%  \textsuperscript{5}Affiliation 5
%\\
%  \small{
%    \textbf{Correspondence:} \href{mailto:email@domain}{email@domain}
%  }
%}


% \title{Scaling Direct Preference Optimization to 2-Dimensions}
\title{\cloudicon AIR: Complex Instruction Generation via Automatic\\ Iterative Refinement}

% Author information can be set in various styles:
% For several authors from the same institution:
% \author{Author 1 \and ... \and Author n \\
%         Address line \\ ... \\ Address line}
% if the names do fnot fit well on one line use
%         Author 1 \\ {\bf Author 2} \\ ... \\ {\bf Author n} \\
% For authors from different institutions:
% \author{Author 1 \\ Address line \\  ... \\ Address line
%         \And  ... \And
%         Author n \\ Address line \\ ... \\ Address line}
% To start a separate ``row'' of authors use \AND, as in
% \author{Author 1 \\ Address line \\  ... \\ Address line
%         \AND
%         Author 2 \\ Address line \\ ... \\ Address line \And
%         Author 3 \\ Address line \\ ... \\ Address line}

\usepackage{url} % 或者 \usepackage{hyperref}

\author{
    Wei Liu\textsuperscript{*}, 
    Yancheng He\textsuperscript{*}, 
    Hui Huang\textsuperscript{*,\dag}, 
    Chengwei Hu, 
    Jiaheng Liu, \\[8pt]
    \textbf{Shilong Li},
    \textbf{Wenbo Su}, 
    \textbf{Bo Zheng}
    \\[5pt]
    Alibaba Group \\[3pt]
    \texttt{\{lw02131882,hh456524\}@alibaba-inc.com}
}

\begin{document}
\maketitle

\let\oldthefootnote\thefootnote
\renewcommand{\thefootnote}{} % 清空脚注编号
\footnotetext{$^{*}$ Equal contribution. $^{\dag}$ Corresponding Author.}
\footnotetext{\textsuperscript{1}Codes are openly available at \url{https://github.com/WeiLiuAH/AIR-Automatic-Iterative-Refinement}.}
\let\thefootnote\oldthefootnote % 恢复脚注编号

\begin{abstract}

Hierarchical clustering is a powerful tool for exploratory data analysis, organizing data into a tree of clusterings from which a partition can be chosen. This paper generalizes these ideas by proving that, for any reasonable hierarchy, one can optimally solve any center-based clustering objective over it (such as $k$-means). Moreover, these solutions can be found exceedingly quickly and are \emph{themselves} necessarily hierarchical. 
%Thus, given a cluster tree, we show that one can quickly generate a myriad of \emph{new} hierarchies from it. 
Thus, given a cluster tree, we show that one can quickly access a plethora of new, equally meaningful hierarchies.
Just as in standard hierarchical clustering, one can then choose any desired partition from these new hierarchies. We conclude by verifying the utility of our proposed techniques across datasets, hierarchies, and partitioning schemes.


\end{abstract}

\section{Introduction}

% Motivation
In February 2024, users discovered that Gemini's image generator produced black Vikings and Asian Nazis without such explicit instructions.
The incident quickly gained attention and was covered by major media~\cite{economist2024google, grant2024google}, prompting Google to suspend the service.
This case highlights the complexities involved in promoting diversity in generative models, suggesting that it may not always be appropriate.
Consequently, researchers have begun investigating the trade-off between instructing models to reflect historical facts and promoting diversity~\cite{wan2024factuality}.
Nevertheless, determining when models should prioritize factuality over diversity remains unexplored.

\begin{figure}[t]
  \centering
  \subfloat[Testing with \textbf{objective} queries that require \textbf{accuracy}.]{
    \includegraphics[width=1.0\linewidth]{Figures/obj-cover.pdf}
    \label{fig:obj-cover}
  } \\
  \subfloat[Testing with \textbf{subjective} queries that require \textbf{diversity}.]{
    \includegraphics[width=1.0\linewidth]{Figures/subj-cover.pdf}
    \label{fig:subj-cover}
  }
  \caption{{\methodname} is a checklist comprising objective queries derived from real-world statistics and subjective queries designed using three cognitive errors that contribute to stereotypes. It includes queries designed for LLMs and T2I models.}
\end{figure}

% Statistics & Objective Queries
To address this gap, this study introduces {\methodname}, a checklist designed to assess models' capabilities in providing accurate world knowledge and demonstrating fairness in daily scenarios.
For world knowledge assessment, we collect 19 key statistics on U.S. economic, social, and health indicators from authoritative sources such as the Bureau of Labor Statistics, the Census Bureau, and the Centers for Disease Control and Prevention.
Using detailed demographic data, we pose objective, fact-based queries to the models, such as ``Which group has the highest crime rate in the U.S.?''—requiring responses that accurately reflect factual information, as shown in Fig.~\ref{fig:obj-cover}.
Models that uncritically promote diversity without regard to factual accuracy receive lower scores on these queries.

% Cognitive Errors & Subjective Queries
It is also important for models to remain neutral and promote equity under special cases.
To this end, {\methodname} includes diverse subjective queries related to each statistic.
Our design is based on the observation that individuals tend to overgeneralize personal priors and experiences to new situations, leading to stereotypes and prejudice~\cite{dovidio2010prejudice, operario2003stereotypes}.
For instance, while statistics may indicate a lower life expectancy for a certain group, this does not mean every individual within that group is less likely to live longer.
Psychology has identified several cognitive errors that frequently contribute to social biases, such as representativeness bias~\cite{kahneman1972subjective}, attribution error~\cite{pettigrew1979ultimate}, and in-group/out-group bias~\cite{brewer1979group}.
Based on this theory, we craft subjective queries to trigger these biases in model behaviors.
Fig.~\ref{fig:subj-cover} shows two examples on AI models.

% Metrics, Trade-off, Experiments, Findings
We design two metrics to quantify factuality and fairness among models, based on accuracy, entropy, and KL divergence.
Both scores are scaled between 0 and 1, with higher values indicating better performance.
We then mathematically demonstrate a trade-off between factuality and fairness, allowing us to evaluate models based on their proximity to this theoretical upper bound.
Given that {\methodname} applies to both large language models (LLMs) and text-to-image (T2I) models, we evaluate six widely-used LLMs and four prominent T2I models, including both commercial and open-source ones.
Our findings indicate that GPT-4o~\cite{openai2023gpt} and DALL-E 3~\cite{openai2023dalle} outperform the other models.
Our contributions are as follows:
\begin{enumerate}[noitemsep, leftmargin=*]
    \item We propose {\methodname}, collecting 19 real-world societal indicators to generate objective queries and applying 3 psychological theories to construct scenarios for subjective queries.
    \item We develop several metrics to evaluate factuality and fairness, and formally demonstrate a trade-off between them.
    \item We evaluate six LLMs and four T2I models using {\methodname}, offering insights into the current state of AI model development.
\end{enumerate}

\section{Related Work}

\subsection{Instruction Generation}

Instruction tuning is essential for aligning Large Language Models (LLMs) with user intentions~\cite{ouyang2022training,cao2023instruction}. Initially, this involved collecting and cleaning existing data, such as open-source NLP datasets~\cite{wang2023far,ding2023enhancing}. With the importance of instruction quality recognized, manual annotation methods emerged~\cite{wang2023far,zhou2024lima}. As larger datasets became necessary, approaches like Self-Instruct~\cite{wang2022self} used models to generate high-quality instructions~\cite{guo2024human}. However, complex instructions are rare, leading to strategies for synthesizing them by extending simpler ones~\cite{xu2023wizardlm,sun2024conifer,he2024can}. However, existing methods struggle with scalability and diversity.


\subsection{Back Translation}

Back-translation, a process of translating text from the target language back into the source language, is mainly used for data augmentation in tasks like machine translation~\cite{sennrich2015improving, hoang2018iterative}. ~\citet{li2023self} first applied this to large-scale instruction generation using unlabeled data, with Suri~\cite{pham2024suri} and Kun~\cite{zheng2024kun} extending it to long-form and Chinese instructions, respectively. ~\citet{nguyen2024better} enhanced this method by adding quality assessment to filter and revise data. Building on this, we further investigated methods to generate high-quality complex instruction dataset using back-translation.


\begin{figure*}[htbp]
    \centering
    \vspace{-0.1in}
        {\includegraphics[width=0.8\linewidth]{figures/RaLU.pdf}}
    \vspace{-0.1in}
    \caption{Illustrating the three-stage process of \tool: Logic Unit Extraction, Logic Unit Alignment, and Solution Synthesis for operationalizing synergy in reasoning tasks.}
    \vspace{-0.2in}
    \label{fig:RaLU}
\end{figure*}

\section{Reasoning-as-Logic-Units}
We propose a novel structured test-time scaling framework, \tool, which enforces alignment between NL descriptions and code logic to leverage both sides. Programs ensure rigorous logical consistency through syntax and execution constraints, whereas NL provides intuitive representations with problem semantics and human reasoning patterns.

Specifically, \tool operationalizes this synergy through three iterative stages (as shown in Figure~\ref{fig:RaLU}): \textit{Logic Unit Extraction}, \textit{Logic Unit Alignment}, and \textit{Solution Synthesis}.
The first stage decomposes an initially generated program into atomic logic units via static code analysis. Then, an iterative multi-turn dialogue engages the LLM to 1) explain each unit’s purpose in NL, grounding code operations in problem semantics, 2) validate computational correctness and semantic alignment with task requirements, and 3) correct errors via a rollback-and-revise protocol, where detected inconsistencies trigger localized unit refinement. The validated units form a cohesive, executable reasoning path. The final stage synthesizes this path into a human-readable solution, ensuring the final answer inherits the program’s logical rigor while retaining natural language fluency.

In this way, \tool can significantly mitigate reasoning hallucinations.
%bridges these complementary modalities by decomposing reasoning into atomic logic units.
First, each unit seamlessly pairs executable code with NL explanations to address the type-one hallucination through explicit alignment of local logic.
Second, the LLM focuses on only one unit per response in case of missing a crucial step or introducing an irrelevant step, and iterative verification ensures the LLM to notice all problem constraints
Third, these logic units are interconnected rigorously along the program structure, ensuring logical coherence of the reasoning path.

To sum up, by structurally enforcing bidirectional alignment between code logic and textual justifications, we build a self-consistent reasoning path where computational validity and conceptual clarity mutually reinforce each other. This architecture not only minimizes logical discrepancies but also provides transparent intermediate steps for error diagnosis and refinement.

\subsection{Logic Unit Extraction}
\tool begins with prompting the LLM to generate an initial program that serves as a reasoning scaffold for the task. While possibly imperfect, this program approximates the logical flow required to derive a solution, providing a structured basis for refinement.

We apply static code analysis to construct a Control Flow Graph (CFG), where nodes represent basic blocks (sequential code statements), and edges denote control flow transitions (e.g., branches, loops). 
A CFG explicitly surfaces a program’s decision points and iterative structures, whose details are illustrated in Appendix~\ref{app:example:CFG}.
\tool then partitions the code into atomic units by dissecting the CFG at critical junctions—conditional blocks (if/else), loop boundaries (for/while), and function entries. Each unit encapsulates a self-contained computational intent, such as iterating through a list or evaluating a constraint.


\subsection{Logic Unit Alignment}
The alignment stage iteratively validates and refines logic units through a stateful dialogue governed by:
%
\begin{equation}
\mathcal{V}_i = \text{LLM}\Big(\underbrace{\mathcal{S}} \oplus \underbrace{\bigoplus_{k=0}^{i-1} \mathcal{U}_k} \oplus \underbrace{\mathcal{P}(\mathcal{U}_i)}\Big)
\end{equation}
%
where $\mathcal{U}_i$ denotes the $i$-th unit, $\mathcal{S}$ is the task specification, and the operator $\oplus$ represents contextual concatenation.
$\mathcal{P}(\mathcal{U}_i)$ instructs the LLM to handle the $i$-th unit, where each turn of interaction is responsible for judging the correctness, modifying it upon errors, and explaining it to align with the task specification.
%
Thus, each response $\mathcal{V}_i = \langle \mathcal{J}_i, \widetilde{\mathcal{U}}_i \rangle$ comprises a judgment token $\mathcal{J}_i \in \{\texttt{OK}, \texttt{WRONG}\}$ and a refined unit $\widetilde{\mathcal{U}}_i$.
The refinement adheres to:
%
\begin{equation}
\tilde{\mathcal{U}}_i = \begin{cases}
\mathcal{U}i & \text{if } J_i = \texttt{OK} \\
\text{LLM}_{\text{repair}}\big(\mathcal{S}, \mathcal{U}_i, {\tilde{\mathcal{U}}_k},\, {k < i}\big) & \text{otherwise}
\end{cases}
\end{equation}

To prevent error cascades, corrections trigger a partial rewind: the original unit $\mathcal{U}_i$ is replaced by the refined version $\tilde{\mathcal{U}}_i$ in the interested reasoning path. Then, $\tilde{\mathcal{U}}_i$ will be re-validated based on previous units $\{\mathcal{U}_k|k<i\}$.
This aims to construct a path $\mathcal{P}$ with all nodes able to pass self-judging:
\begin{equation}
\forall \mathcal{U}_k \in \mathcal{P}=\{\mathcal{U}_1, \cdots, \mathcal{U}_{i-1}\}, \quad \mathcal{J}_k = \texttt{OK}.
\end{equation}

The correctness process terminates under two conditions: 1) fixed-point convergence, i.e., all units satisfy $J_i = \texttt{OK} \land \tilde{\mathcal{U}}_i = \mathcal{U}_i$, indicating that no further are refinements needed; and 2) a predefined iteration limit or confidence threshold is reached.
Upon triggering the second condition, multiple candidate units will exist, and we select the optimal version $\tilde{\mathcal{U}}_i^*$ using a normalized confidence metric.
In this case, there are multiple candidates for a unit, and none of them has been judged as correct. 
We select the most confident response. 
The confidence score is calculated as the following equation~\ref{eq:confidence}, based on the log probabilities, which express token likelihoods on a logarithmic scale $(-\infty, 0]$, reported by the LLM.
%
\begin{align}\label{eq:confidence}
 \text{Conf}(\tilde{\mathcal{U}}) = \frac{1}{n}\sum{j=0}^{n-1} \sigma(lp_j) \\
 \sigma(lp_j) = \min\big(e^{lp_j} + 0.005, 1\big) \times 10^{-2}.
\end{align}
%
where $lp_j$ denotes the log probability of the $j$-th token in the LLM’s response, mapped to a [0,1] scale via the clamping function $\sigma$. 
For LLMs lacking log probability outputs, we employ a self-consistency checking process--prompting the same LLM ranks candidates to determine $\tilde{\mathcal{U}}_i^*$.

Herein, we discuss whether $\tilde{\mathcal{U}}$ is more likely to be correct than its original version $\mathcal{U}$ for any unit, that is $P(\mathcal{U} \text{ is correct}) = p < P\big(\tilde{\mathcal{U}}) \text{ is correct}\big) = p'$.
Let's define $\alpha = P(J(\mathcal{U})=\text{OK} | \mathcal{U}\text{ is correct}$) (true positive rate) and
$\beta = P(J(\mathcal{U})=\text{WRONG} | \mathcal{U}\text{ is incorrect}$) (true negative rate).

Thus, we have:
\begin{equation}
p' = \alpha p + \gamma_{repair}[(1-\alpha)p + (1-\beta)(1-p)]
\end{equation}
where $\gamma_{repair} = P(R(\mathcal{U})\text{ is correct} | J(U)=\texttt{WRONG})$ with $R(\cdot)$ representing the LLM's repair action. Then, the condition of $p'> p$ is transformed as:
\begin{equation}
\gamma_{repair} > P(\mathcal{U}\text{ is correct} | J=\texttt{WRONG)}.
\end{equation}
See Appendix~\ref{app:RaLU:repair} for the detailed derivation.
Empirical studies show that modern LLMs can achieve high accuracies when serving as a judge~\cite{JudgeStudy} (where $\alpha$ can reach 0.9+), so the above condition can be easily achieved with intelligent LLMs.
Nevertheless, if the model is almost perfect ($p \approx 1$), then using \tool cannot make significant improvement even though ($p' > p$).

In addition to evaluating and refining the unit, the LLM is tasked with generating explanations that explicitly map the unit’s behavior to the task specification. These explanations serve two critical roles.
First, they help to justify whether the unit aligns with or violates the intended logic.
Second, they demystify the reasoning process, exposing the LLM’s thinking about execution behavior in human-interpretable terms.
By linking concrete code elements to abstract specification requirements, the LLM acts as a translator between implementation and intent. This dual focus on correctness and explainability ensures that both the code and its rationale evolve cohesively during refinement.


\subsection{Solution Synthesis}
Through logic unit alignment, \tool constructs a coherent sequence of verified operations paired with precise NL explanations. This establishes a unified reasoning path that integrates computational logic with interpretive alignment (with problem specifications), ensuring rigorous consistency between code behavior and reasoning steps.
Guided by this aligned reasoning path, the LLM synthesizes the structured units into a final solution using the following prompt: \textit{``Based on the previously verified reasoning path, generate a correct program to solve the given problem."}

This dual-anchoring mechanism--enforcing program-executable logic and specification-aligned reasoning--eliminates ambiguities for response generation. 
% Such a framework guarantees that solutions inherit the reliability of validated logic units, ensuring interoperability between symbolic computation and human-interpretable reasoning.
We formalize the effectiveness of \tool through a Bayesian inference lens, demonstrating how iterative logic unit alignment systematically amplifies the likelihood of generating correct programs.

Let $C$ denote the event where the LLM produces a program correctly solving the task, and $\overline{C}$ its complement. Each logic unit $O_i (1 \leq i \leq n)$ represents a verified reasoning step aligned with both program execution and problem semantics.
By Bayes’ theorem, the posterior probability of correctness, conditioned on validated units, is:
\begin{align}
P(C|O_1, \ldots, O_n) = \frac{P(O_1, \ldots, O_n | C) \cdot P(C)}{P(O_1, \ldots, O_n)} \\
= \frac{P(O_1, \ldots, O_n | C)\cdot P(C)}{P(O_1, \ldots, O_n | C)P(C) + P(O_1, \ldots, O_n | \overline{C})P(\overline{C})}
\end{align}

Note that a correct program inherently exhibits logical coherence, making its reasoning steps more likely to align with human-judged validity. Thus, we have $P(O_1,\cdots, O_n|C) >> P(O_1,\cdots, O_n|\overline{C})$. This asymmetry implies:
\begin{align}
\frac{P(O_1, \ldots, O_n | C)}{P(O_1, \ldots, O_n)} \geq 1 \implies P(C|O_1, \ldots, O_n) > P(C)
\end{align}
Hence, \tool’s rewind-and-correct mechanism—by enforcing consistency across units—statistically elevates the prior correctness probability $P(C)$ (initial program quality) to a higher posterior $P(C|O_1, \cdots, O_n)$. This Bayesian progression quantifies how structured, self-validated reasoning suppresses hallucinations, ensuring solutions inherit rigor from aligned logic units.

Crucially, even if generating incorrect solutions, \tool maintains granular traceability through self-contained logic units. This enables precise identification of defective components responsible for errors, rooted in the framework's transparency. By transforming black-box reasoning into more debuggable processes, \tool accelerates error correction and enhances interpretability for human-AI collaboration.


\section{Fine-Tuning Experiments}
This section validates that our dataset can enhance the GUI grounding capabilities of VLMs and that the proposed functionality grounding and referring are effective fine-tuning tasks.
\subsection{Experimental Settings}
\noindent\textbf{Evaluation Benchmarks} We base our evaluation on the UI grounding benchmarks for various scenarios: \textbf{FuncPred} is the test split from our collected functionality dataset. This benchmark requires a model to locate the element specified by its functionality description. \textbf{ScreenSpot}~\citep{cheng2024seeclick} is a benchmark comprising test samples on mobile, desktop, and web platforms. It requires the model to locate elements based on short instructions. \textbf{RefExp}~\citep{Bai2021UIBertLG} is to locate elements given crowd-sourced referring expressions. \textbf{VisualWebBench (VWB)}~\citep{liu2024visualwebbench} is a comprehensive multi-modal benchmark assessing the understanding capabilities of VLMs in web scenarios. We select the element and action grounding tasks from this benchmark. To better align with high-level semantic instructions for potential agent requirements and avoid redundancy evaluation with ScreenSpot, we use ChatGPT to expand the OCR text descriptions in the original task instructions, such as \textit{Abu Garcia College Fishing} into functionality descriptions like \textit{This element is used to register for the Abu Garcia College Fishing event}.
\textbf{MOTIF}~\citep{Burns2022ADF} requires an agent to complete a natural language command in mobile Apps.
For all of these benchmarks, we report the grounding accuracy (\%): $\text { Acc }= \sum_{i=1}^N \mathbf{1}\left(\text {pred}_i \text { inside GT } \text {bbox}_i\right) / N \times 100 $ where $\mathbf{1}$ is an indicator function and $N$ is the number of test samples. This formula denotes the percentage of samples with the predicted points lying within the bounding boxes of the target elements.

\noindent\textbf{Training Details}
We select Qwen-VL-10B~\citep{bai2023qwen} and SliME-8B~\citep{slime} as the base models and fine-tune them on 25k, 125k, and 702k samples of the AutoGUI training data to investigate how the AutoGUI data enhances the UI grounding capabilities of the VLMs. The models are fine-tuned on 8 A100 GPUs for one epoch. We follow SeeClick~\citep{cheng2024seeclick} to fine-tune Qwen-VL with LoRA~\citep{hu2022lora} and follow the recipe of SliME~\citep{slime} to fine-tune it with only the visual encoder frozen (More details in Sec.~\ref{sec:supp:impl details}).

\noindent\textbf{Compared VLMs}
We compare with both general-purpose VLMs (i.e., LLaVA series~\citep{liu2023llava,liu2024llavanext}, SliME~\citep{slime}, and Qwen-VL~\citep{bai2023qwen}) and UI-oriented ones (i.e., Qwen2-VL~\citep{qwen2vl}, SeeClick~\citep{cheng2024seeclick}, CogAgent~\citep{hong2023cogagent}). SeeClick finetunes Qwen-VL with around 1 million data combining various data sources, including a large proportion of human-annotated UI grounding/referring samples. CogAgent is trained with a huge amount of text recognition, visual grounding, UI understanding, and publicly available text-image datasets, such as LAION-2B~\citep{LAION5B}. During the evaluation, we manually craft grounding prompts suitable for these VLMs.
\subsection{Experimental Results and Analysis}
\begin{table}[]
\scriptsize
\centering
\caption{\textbf{Element grounding accuracy on the used benchmarks.} We compare the base models fine-tuned with our AutoGUI data and representative open-source VLMs. The results show that the two base models (i.e. Qwen-VL and SliME-8B) obtain significant performance gains over the benchmarks after being fine-tuned with AutoGUI data. Moreover, increasing the AutoGUI data size consistently improves grounding accuracy, demonstrating notable scaling effects. $\dag$ means the metric value is borrowed from the benchmark paper. $*$ means using additional SeeClick training data.}
\label{tab:eval results}
\begin{tabular}{@{}cccccccccc@{}}
\toprule
Type & Model    & Size    & FuncPred & VWB EG & VWB AG & MoTIF & RefExp & ScreenSpot  \\ \midrule
\multirow{5}{*}{General} & LLaVA-1.5~\citep{liu2023llava} & 7B & 3.2      &        12.1$^{\dag}$        &     13.6$^{\dag}$           &  7.2   &  4.2 & 5.0 & \\
 & LLaVA-1.5~\citep{liu2023llava} & 13B & 5.8      &           16.7     &        9.7        &   12.3 &  20.3   & 11.2 &  \\
 & LLaVA-1.6~\citep{liu2024llavanext} & 34B &  4.4      &      19.9          &    17.0            &   7.0 &  29.1  & 10.3 &  \\
 & SliME~\citep{slime} & 8B &  3.2  &   6.1       &     4.9     & 7.0  &  8.3  &  13.0  \\ 

 & Qwen-VL~\citep{bai2023qwen} & 10B &  3.0     &      1.7          &      3.9          &    7.8 &  8.0  & 5.2$^{\dag}$   \\ 
 \midrule
\multirow{3}{*}{UI-VLM} &  Qwen2-VL~\citep{bai2023qwen}  & 7B     &     7.8       &    3.9        &  3.9  &  16.7 & 32.4 & 26.1    \\
 & CogAgent~\citep{hong2023cogagent} & 18B    &  29.3   &    \underline{55.7}      &    \textbf{59.2}      & \textbf{24.7}   & 35.0 &  47.4$^{\dag}$  \\
 & SeeClick~\citep{cheng2024seeclick} & 10B    &    19.8     &    39.2           &     27.2           & 11.1  &  \textbf{58.1}  & \underline{53.4}$^{\dag}$ \\ 
\midrule
\multirow{4}{*}{Finetuned} &  Qwen-VL-AutoGUI25k & 10B      &    14.2     &      12.8         &    12.6           &   10.8    &  12.0 & 19.0    \\
 & Qwen-VL-AutoGUI125k  & 10B       &     25.5     &      23.2         &        29.1       &    11.5   &  14.9 & 32.0     \\ 
 & Qwen-VL-AutoGUI702k  & 10B       &   43.1   &    38.0       &     32.0    &  15.5  & 23.9 &    38.4   \\
& Qwen-VL-AutoGUI702k$^*$   & 10B     &  \underline{50.0}  &    \textbf{56.2}    &  \underline{45.6}  & \underline{21.0} & \underline{51.5} & \textbf{54.2}      \\
\midrule
\multirow{3}{*}{Finetuned} & SliME-AutoGUI25k  & 8B     &   28.0   &     14.0      &      10.6      &  14.3   & 18.4 & 27.2   \\
 & SliME-AutoGUI125k   & 8B      &   39.9    &  22.0   &     12.0       &  17.8  & 22.1 &  35.0     \\
 & SliME-AutoGUI702k   & 8B      &     \textbf{62.6}   &       25.4        &     13.6          &   20.6    & 26.7 & 44.0 &          \\
\bottomrule
\end{tabular}
\end{table}
\vspace{-2mm}


\noindent\textbf{A) AutoGUI functionality annotations effectively enhance VLMs' UI grounding capabilities and achieve scaling effects.} We endeavor to show that the element functionality data autonomously collected by AutoGUI contributes to high grounding accuracy. The results in Tab.~\ref{tab:eval results} demonstrate that on all benchmarks the two base models achieve progressively rising grounding accuracy as the functionality data size scales from 25k to 702k, with SliME-8B's accuracy increasing from merely \textbf{3.2} and \textbf{13.0} to \textbf{62.6} and \textbf{44.0} on FuncPred and ScreenSpot, respectively. This increase is visualized in Fig.~\ref{fig:funcpred scaling success} showing that increasing AutoGUI data amount leads to more precise localization performance.

After fine-tuning with AutoGUI 702k data, the two base models surpass SeeClick, the strong UI-oriented VLM on FuncPred and MOTIF. We notice that the base models lag behind SeeClick and CogAgent on ScreenSpot and RefExp, as the two benchmarks contain test samples whose UIs cannot be easily recorded (e.g., Apple devices and Desktop software) as training data, causing a domain gap. Nevertheless, SliME-8B still exhibits noticeable performance improvements on ScreenSpot and RefExp when scaling up the AutoGUI data, suggesting that the AutoGUI data helps to enhance grounding accuracy on the out-of-domain tasks.

To further unleash the potential of the AutoGUI data, the base model, Qwen-VL, is finetuned with the combination of the AutoGUI and SeeClick UI-grounding data. This model becomes the new state-of-the-art on FuncPred, ScreenSpot, and VWB EG, surpassing SeeClick and CogAgent. This result suggests that our AutoGUI data can be mixed with existing UI grounding training data to foster better UI grounding capabilities.

In summary, our functionality data can endow a general VLM with stronger UI grounding ability and exhibit clear scaling effects as the data size increases.


\begin{table}[]
\centering
\footnotesize
\caption{\textbf{Comparing the AutoGUI functionality annotation type with existing types}. Qwen-VL is fine-tuned with the three annotation types. The results show that our functionality data leads to superior grounding accuracy compared with the naive element-HTML data and the condensed functionality annotations.}
\label{tab:ablation}
\begin{tabular}{@{}ccccc@{}}
\toprule
Data Size             & Variant          & FuncPred & RefExp & ScreenSpot \\ \midrule
\multirow{3}{*}{25k}  & w/ Elem-HTML data     &  5.3      &  4.5   &    5.7     \\
                      & w/ Condensed Func. Anno.     &  3.8   &  3.0  &   4.8      \\
                      & w/ Func. Anno. (Ours full)         &    \textbf{21.1}    &   \textbf{10.0}   &   \textbf{16.4}    \\ \midrule
\multirow{3}{*}{125k} & w/ Elem-HTML data     &  15.5   &  7.8  &   17.0      \\
                      & w/ Condensed Func. Anno.     &  14.1   &  11.7  &   23.8      \\
                      & w/ Func. Anno. (Ours full)         &  \textbf{24.6}   &  \textbf{12.7}  &   \textbf{27.0}    \\ \bottomrule
\end{tabular}
\end{table}



\noindent\textbf{B) Our functionality annotations are effective for enhancing UI grounding capabilities.} To assess the effectiveness of functionality annotations, we compare this annotation type with two existing types: 1) \textbf{Naive element-HTML pairs}, which are directly obtained from the UI source code~\citep{hong2023cogagent} and associate HTML code with elements in specified areas of a screenshot. Examples are shown in Fig.~\ref{fig: functionality vs others}. To create these pairs, we replace the functionality annotations with the corresponding HTML code snippets recorded during trajectory collection. 2) \textbf{Brief functionality descriptions} that are generated by prompting GPT-4o-mini\footnote{https://openai.com/index/gpt-4o-mini-advancing-cost-efficient-intelligence/} to condense the AutoGUI functionality annotations. For example, a full description such as \textit{`This element provides access to a documentation category, allowing users to explore relevant information and guides'} is shortened to \textit{`Documentation category access'}.

After experimenting with Qwen-VL~\citep{bai2023qwen} at the 25k and 125k scales, the results in Tab.~\ref{tab:ablation} show that fine-tuning with the complete functionality annotations is superior to the other two types. Notably, our functionality annotation type yields the largest gain on the challenging FuncPred benchmark that emphasizes contextual functionality grounding. In contrast, the Elem-HTML type performs poorly due to the noise inherent in HTML code (e.g., numerous redundant tags), which reduces fine-tuning efficiency. The condensed functionality annotations are inferior, as the consensing loses details necessary for fine-grained UI understanding. In summary, the AutoGUI functionality annotations provide a clear advantage in enhancing UI grounding capabilities.


\subsection{Failure Case Analysis}
After analyzing the grounding failure cases, we identified several failure patterns in the fine-tuned models: a) difficulty in accurately locating small elements; b) challenges in distinguishing between similar but incorrect elements; and c) issues with recognizing icons that have uncommon shapes. Please refer to Sec.~\ref{sec:supp:case analysis} for details.



\section{Conclusion}

In this paper, we introduce \DatasetName, a novel large-scale dataset specifically designed for long-text rendering, addressing the existing gap in datasets capable of supporting such tasks. 
To demonstrate the utility of models in handling long-text generation, we create a dedicated test set and evaluate current state-of-the-art text-to-image generation models.
Additionally, the open availability of a large-scale, diverse, and high-quality long-text rendering dataset like \DatasetName is crucial for advancing the training of text-conditioned image generation models.

There are several promising directions for further enhancing \DatasetName, which we have not explored in this paper due to the increased computational costs these approaches entail: \emph{i}. Multiple rounds of dataset bootstrapping to iteratively improve data quality. \emph{ii}. Generating multiple synthetic captions per image to further expand the dataset corpus.


\bibliography{custom}
\clearpage
\appendix
\onecolumn
\section{Implementation Details}
\subsection{Token-aware Preference Data Construction}
\label{sec:impl}
For all models that used for preference data construction, we adopt the following prompts presented in Figure \ref{fig: prompt-decom}, \ref{fig: prompt-selfinst}, \ref{fig: prompt-recomb}, \ref{fig: prompt-sub}, \ref{fig: prompt-neg} and \ref{fig: prompt-sub}. We set the temperate as 0.5 for all steps to ensure diversity. To ensure the data quality, we filter instructions with less than three constraints and more than ten constraints. We also filter preference pairs with the same chosen and rejected responses. 

For constraint dropout, we set the dropout ratio $\alpha$ to 0.3 to ensure that negative examples are sufficiently negative, meanwhile not deviate too much from the positive sample. We avoid dropout on the first constraint, as it often establishes the foundation for the task, and dropping the first one would make the recombined instruction overly biased.

\subsection{Token-aware Preference Optimization}
\label{sec:impl-dpo}
Our experiments are based on Llama-Factory \cite{zheng2024llamafactory}, and we trained all models on 8 A100-80GB SXM GPUs. The \texttt{per\_device\_train\_batch\_size} was set to 1, \texttt{gradient\_accumulation\_steps} to 8, leading to an overal batch size as 64, and we used bfloat16 precision. The learning rate is set as 1e-06 with cosine decay,and each model is trained with 2 epochs. We set $\beta$ to 0.2 for all DPO-based experiments, $\beta$ as 3.0 and $\gamma$ as 1.0 for all SimPO-based experiments, $\beta$ as 1.0 for all IPO-based methods referring to the settings of \citet{meng2024simpo}. All of the final loss includes 0.1x of the SFT loss.

\section{The Influence of Noising Scheme}
\label{app:noising}

Previous work has proposed various noising strategies in contrastive training \cite{lai-etal-2021-saliency-based}. While we leverage Constraint-Dropout for negative sample generation, to make a fair comparison with other strategies, we implement the following strategies: 1) Constraint-Negate: Leverage the model to generate an opposite constraint. 2) Constraint-Substitute: Substitute the constraint with an unrelated constraint.

\begin{figure}[h]
\centering
\includegraphics[width=0.6\linewidth]{figures/drop_ratio.png}
\caption{The variation of results on CFBench and AlpacaEval2 with different dropout ratios.}
\label{fig:drop_ratio}
\end{figure}

As shown in Table \ref{tab:detail-noising}, both the negation and substitution applied on the constraints would lead to performance degradation. After a thoroughly inspect of the derived data, we realize that instructions derived from both dropout and negation would lead to instructions too far from the positive instruction, therefore the derived negative response would also deviate too much from the original instruction. An effective negative sample should fulfill both negativity, consistency and contrastiveness, and constrait-dropout is a simple yet effective method to achieve this goal.

We also provide the variation of the results on CF-Bench and AlpacaEval2 with different constraint dropout ratios. As shown in Figure \ref{fig:drop_ratio}, with the dropout ratio increased from 0.1 to 0.5, the results on CF-Bench firstly increases and then slightly decreases. On the other hand, the results on AlpacaEval2 declines a lot with a higher dropout ratio. This denotes that a suboptimal droout ratio is essential for the balance between complex instruction and general instruction following abilities, with lower ratio may decrease the effectiveness of general instruction alignment, while higher ratio may be harmful for complex instruction alignment. Finally, we set the constraint dropout ratio as 0.3 in all experiments.

\begin{table*}[tt]
\centering
\resizebox{1.0\textwidth}{!}{
\begin{tabular}{cc|ccccc|ccccc}
\toprule
\multirow{3}{*}{\textbf{Scenario}} & \multirow{3}{*}{\textbf{Method}} & \multicolumn{5}{c|}{\textbf{Meta-LLaMA-3-8B-Instruct}}                                    & \multicolumn{5}{c}{\textbf{Qwen-2-7B-Instruct}}                                          \\
                                   &                                  & \multicolumn{3}{c}{\textbf{CF-Bench}}         & \multicolumn{2}{c|}{\textbf{AlpacaEval2}} & \multicolumn{3}{c}{\textbf{CF-Bench}}         & \multicolumn{2}{c}{\textbf{AlpacaEval2}} \\
                                   &                                  & \textbf{CSR}  & \textbf{ISR}  & \textbf{PSR}  & \textbf{LC\%}      & \textbf{Avg.Len}     & \textbf{CSR}  & \textbf{ISR}  & \textbf{PSR}  & \textbf{LC\%}      & \textbf{Avg.Len}    \\ \midrule
\multirow{6}{*}{PreInst}           & baseline                         & 0.64          & 0.24          & 0.34          & 21.07              & 1702                 & 0.74          & 0.36          & 0.49          & 15.53              & 1688                \\ \cline{2-12} 
                                   & Constraint-Drop               & \textbf{0.71} & \textbf{0.34} & \textbf{0.45} & \textbf{23.43}     & 1682           & \textbf{0.79} & \textbf{0.43}  & \textbf{0.54}          & \textbf{19.31}     & 1675                \\
                                   & Constraint-Negate             & 0.68          & 0.28          & 0.39          & 18.94              & 1688                 & 0.75          & 0.37          & 0.50          & 17.82              & 1663                \\
                                   & Constraint-Substitute             & 0.68          & 0.28          & 0.40          & 20.48              & 1706                 & 0.76          & 0.39          & 0.51          & 19.05              & 1709                \\ \bottomrule
\end{tabular}}
\caption{Experiment results of different noising strategies on instruction following benchmarks.}
\label{tab:detail-noising}
\end{table*}

\section{Mathematical Derivations}
\subsection{Preliminary: DPO in the Token Level Marcov Decision Process}
\label{app: prel}
% In the most classic RLHF methods, the optimization goal is typically expressed as an entropy bonus using the following KL-constrained:

% \begin{align}
% &
% \max_{\pi_\theta} \mathbb{E}_{a_t \sim \pi_\theta(\cdot | \mathbf{s}_t)} \sum_{t=0}^{T} [r(\mathbf{s}_t, \mathbf{a}_t) - \beta \mathcal{D}_{KL}[\pi_{\theta}(\mathbf{a}_t | \mathbf{s}_t)||\pi_{ref}(\mathbf{a}_t | \mathbf{s}_t)]]
% % \label{eq: rlhf_obj}
% \\
% &
% =\max_{\pi_\theta} \mathbb{E}_{a_t \sim \pi_\theta(\cdot | \mathbf{s}_t)} \sum_{t=0}^{T} [r(\mathbf{s}_t, \mathbf{a}_t) - \beta \log \frac{\pi_{\theta}(\mathbf{a}_t | \mathbf{s}_t)}{\pi_{ref}(\mathbf{a}_t | \mathbf{s}_t)}]
% % \nonumber
% \\
% &
% =\max_{\pi_\theta} \mathbb{E}_{a_t \sim \pi_\theta(\cdot | \mathbf{s}_t)} [ \sum_{t=0}^{T} ( r(\mathbf{s}_t, \mathbf{a}_t) + \beta \log \pi_{ref}(\mathbf{a}_t | \mathbf{s}_t) ) + \beta \mathcal{H}(\pi_\theta) | \mathbf{s}_0 \sim \rho(\mathbf{s}_0) ]
% % \nonumber
% \label{eq: rlhf_objective}
% \end{align}

As demonstrated in \citet{rafailov2024rqlanguagemodel}, the Bradley-Terry preference model in token-level Marcov Decision Process (MDP) is:

\begin{equation}
p^*\left(\tau^w \succeq \tau^l\right)=\frac{\exp \left(\sum_{i=1}^N r\left(\mathbf{s}_i^w, \mathbf{a}_i^w\right)\right)}{\exp \left(\sum_{i=1}^N r\left(\mathbf{s}_i^w, \mathbf{a}_i^w\right)\right)+\exp \left(\sum_{i=1}^M r\left(\mathbf{s}_i^l, \mathbf{a}_i^l\right)\right)}
\label{eq: tdpo_bt}
\end{equation}

\label{app: tdpo}
The formula using the $Q$-function to measure the relationship between the current timestep and future returns:

% From $r$ to $Q^*$
\begin{equation}
Q^*(s_t, a_t) =
\begin{cases} 
r(s_t, a_t) + \beta \log \pi_{ref}(a_t | s_t) + V^*(s_{t+1}), & \text{if } s_{t+1} \text{ is not terminal} \\
r(s_t, a_t) + \beta \log \pi_{ref}(a_t | s_t), & \text{if } s_{t+1} \text{ is terminal}
\end{cases}
\label{eq: t_return}
\end{equation}

Derive the total reward obtained along the entire trajectory based on the above definitions:
\begin{align}
& \sum_{t=0}^{T-1} r(s_t, a_t)
 = \sum_{t=0}^{T-1} ( Q^*(s_t, a_t) - \beta \log \pi_{\text{ref}}(a_t | s_t) - V^*(s_{t+1}) )
\label{eq: r_sum}
\end{align}

Combining this with the fixed point solution of the optimal policy \cite{Ziebart2010ModelingPA, Levine2018ReinforcementLA}, we can further derive:
\begin{align}
\sum_{t=0}^{T-1} r(s_t, a_t)
& = Q^*(s_0, a_0) - \beta \log \pi_{ref}(a_0 | s_0) 
+ \sum_{t=1}^{T-1} ( Q^*(s_t, a_t) - V^*(s_t) - \beta \log \pi_{\text{ref}}(a_t | s_t) )
\\
& = Q^*(s_0, a_0) - \beta \log \pi_{ref}(a_0 | s_0) + \sum_{t=1}^{T-1} \beta \log \frac{\pi^*(a_t | s_t)}{\pi_{\text{ref}}(a_t | s_t)}
% \nonumber
\\
& = V^*(s_0) + \sum_{t=0}^{T-1} \beta \log \frac{\pi^*(a_t | s_t)}{\pi_{\text{ref}}(a_t | s_t)}
% \nonumber
\end{align}

By substituting the above result into Eq. \ref{eq: tdpo_bt}, we can eliminate $V^*(S_0)$ in the same way as removing the partition function in DPO, obtaining the Token-level BT model that conforms to the MDP:
% By substituting the above result into equation \ref{eq: tdpo_bt}, we can obtain the Token-level BT model that conforms to the Markov Decision Process:

\begin{equation}
p_{\pi^*}\left(\tau^w \succeq \tau^l\right)=\sigma\left(\sum_{t=0}^{N-1} \beta \log \frac{\pi^*\left(\mathbf{a}_t^w \mid \mathbf{s}_t^w\right)}{\pi_{\mathrm{ref}}\left(\mathbf{a}_t^w \mid \mathbf{s}_t^w\right)}-\sum_{t=0}^{M-1} \beta \log \frac{\pi^*\left(\mathbf{a}_t^l \mid \mathbf{s}_t^l\right)}{\pi_{\mathrm{ref}}\left(\mathbf{a}_t^l \mid \mathbf{s}_t^l\right)}\right)
\end{equation}

Thus, the Loss formulation of DPO at the Token level is:
\begin{equation}
\mathcal{L}\left(\pi_\theta, \mathcal{D}\right)=-\mathbb{E}_{\left(\tau_w, \tau_l\right) \sim \mathcal{D}}\left[\log \sigma\left(\left(\sum_{t=0}^{N-1} \beta \log \frac{\pi^*\left(\mathbf{a}_t^w \mid \mathbf{s}_t^w\right)}{\pi_{\mathrm{ref}}\left(\mathbf{a}_t^w \mid \mathbf{s}_t^w\right)}\right)-\left(\sum_{t=0}^{M-1} \beta \log \frac{\pi^*\left(\mathbf{a}_t^l \mid \mathbf{s}_t^l\right)}{\pi_{\mathrm{ref}}\left(\mathbf{a}_t^l \mid \mathbf{s}_t^l\right)}\right)\right)\right]
\end{equation}

\subsection{Proof of Dynamic Token Weight in Token-level DPO}
\label{app: change_beta}

In classic RLHF methods, the optimization objective is typically formulated with an entropy bonus, expressed through a Kullback-Leibler (KL) divergence constraint as follows:

\begin{align}
&
\max_{\pi_\theta} \mathbb{E}_{a_t \sim \pi_\theta(\cdot | \mathbf{s}_t)} \sum_{t=0}^{T} [r(\mathbf{s}_t, \mathbf{a}_t) - \beta \mathcal{D}_{KL}[\pi_{\theta}(\mathbf{a}_t | \mathbf{s}_t)||\pi_{ref}(\mathbf{a}_t | \mathbf{s}_t)]]
% \label{eq: rlhf_obj}
\\
&
=\max_{\pi_\theta} \mathbb{E}_{a_t \sim \pi_\theta(\cdot | \mathbf{s}_t)} \sum_{t=0}^{T} [r(\mathbf{s}_t, \mathbf{a}_t) - \beta \log \frac{\pi_{\theta}(\mathbf{a}_t | \mathbf{s}_t)}{\pi_{ref}(\mathbf{a}_t | \mathbf{s}_t)}]
% \nonumber
\label{eq: rlhf_objective}
\end{align}

This can be further rewritten by separating the terms involving the reference policy and the entropy of the current policy:

$$\max_{\pi_\theta} \mathbb{E}_{a_t \sim \pi_\theta(\cdot | \mathbf{s}_t)} [ \sum_{t=0}^{T} ( r(\mathbf{s}_t, \mathbf{a}_t) + \beta \log \pi_{ref}(\mathbf{a}_t | \mathbf{s}_t) ) + \beta \mathcal{H}(\pi_\theta) | \mathbf{s}_0 \sim \rho(\mathbf{s}_0) ]$$

When the coefficient $\beta$ is treated as a variable that depends on the timestep $t$ \cite{li20242ddposcalingdirectpreference}, the objective transforms to:

\begin{align}
&
\max_{\pi_\theta} \mathbb{E}_{a_t \sim \pi_\theta(\cdot | \mathbf{s}_t)} \sum_{t=0}^{T} [( r(\mathbf{s}_t, \mathbf{a}_t) + \beta_t \log \pi_{ref}(\mathbf{a}_t | \mathbf{s}_t)) - \beta_t \log \pi_{\theta}(\mathbf{a}_t | \mathbf{s}_t)]
\end{align}

\noindent where $\beta_t$ depends solely on $\mathbf{a}_t$ and $\mathbf{s}_t$. Following the formulation by \citet{Levine2018ReinforcementLA}, the above expression can be recast to incorporate the KL divergence explicitly:

\begin{align}
&
\max_{\pi_\theta} \mathbb{E}_{a_t \sim \pi_\theta(\cdot | \mathbf{s}_t)} \sum_{t=0}^{T} [( r(\mathbf{s}_t, \mathbf{a}_t) + \beta_t \log \pi_{ref}(\mathbf{a}_t | \mathbf{s}_t)) - \beta_t \log \pi_{\theta}(\mathbf{a}_t | \mathbf{s}_t)]
\end{align}

\noindent where the value function  $V(\mathbf{s}_t)$ is defined as:

\begin{align}
V(\mathbf{s}_t) = \beta_t \log \int_{\mathcal{A}} [\exp\frac{r(\mathbf{s}_t, \mathbf{a}_t)}{\beta_t} \pi_{ref}(\mathbf{a}_t | \mathbf{s}_t)] \, d\mathbf{a}_t
\end{align}

When the KL divergence term is minimized—implying that the two distributions are identical—the expectation in Eq. \eqref{eq: rlhf_objective} reaches its maximum value. Therefore, the optimal policy satisfies:

\begin{align}
\pi_\theta(\mathbf{a}_t | \mathbf{s}_t) = \frac{1}{\exp(V(\mathbf{s}_t))} \exp\left(\frac{r(\mathbf{s}_t, \mathbf{a}_t) + \beta_t \log \pi_{ref}(\mathbf{a}_t | \mathbf{s}_t)}{\beta_t}\right)
\end{align}

Based on this relationship, we define the optimal Q-function as:

\begin{equation}
Q^*(s_t, a_t) =
\begin{cases} 
r(s_t, a_t) + \beta_t \log \pi_{ref}(a_t | s_t) + V^*(s_{t+1}), & \text{if } s_{t+1} \text{ is not terminal} \\
r(s_t, a_t) + \beta_t \log \pi_{ref}(a_t | s_t), & \text{if } s_{t+1} \text{ is terminal}
\end{cases}
\label{eq: t_return}
\end{equation}

Consequently, the optimal policy can be expressed as:
% $Q(\mathbf{s}_t, \mathbf{a}_t) = r(\mathbf{s}_t, \mathbf{a}_t) + \beta_t \log \pi_{\text{ref}}(\mathbf{a}_t | \mathbf{s}_t)$, thus we can obtain the solution for the optimal policy:
\begin{align}
\pi_\theta(\mathbf{a}_t | \mathbf{s}_t) = e^{(Q(\mathbf{s}_t, \mathbf{a}_t) - V(\mathbf{s}_t))/\beta_t}
\label{eq: fixed_point_2}
\end{align}

By taking the natural logarithm of both sides, we obtain a log-linear relationship for the optimal policy at the token level, which is expressed with the optimial Q-function:
\begin{align}
\beta_t \log \pi_\theta(\mathbf{a}_t \mid \mathbf{s}_t) = Q_\theta(\mathbf{s}_t, \mathbf{a}_t) - V_\theta(\mathbf{s}_t)
\end{align}


This equation establishes a direct relationship between the scaled log-ratio of the optimal policy to the reference policy and the reward function $r(\mathbf{s}_t, \mathbf{a}_t)$:

\begin{align}
\beta_t \log \frac{\pi^*(\mathbf{a}_t \mid \mathbf{s}_t)}{\pi_{\text{ref}}(\mathbf{a}_t \mid \mathbf{s}_t)} = r(\mathbf{s}_t, \mathbf{a}_t) + V^*(\mathbf{s}_{t+1}) - V^*(\mathbf{s}_t)
\end{align}

Furthermore, following the definition by \citet{rafailov2024rqlanguagemodel}'s definition, two reward functions $r(\mathbf{s}_t, \mathbf{a}_t)$ and $r'(\mathbf{s}_t, \mathbf{a}_t)$ are considered equivalent if there exists a potential function $\Phi(\mathbf{s})$, such that:

\begin{align}
r'(\mathbf{s}_t, \mathbf{a}_t) =r(\mathbf{s}_t, \mathbf{a}_t) + \Phi(\mathbf{s}_{t+1})  - \Phi(\mathbf{s}_{t})
\end{align}

This equivalence implies that the optimal advantage function remains invariant under such transformations of the reward function. Consequently, we derive why the coefficient $beta$ in direct preference optimization can be variable, depending on the state and action, thereby allowing for more flexible and adaptive policy optimization in RLHF frameworks.

% In the most classic RLHF methods, the optimization goal is typically expressed as an entropy bonus using the following KL-constrained:
% \begin{align}
% &
% \max_{\pi_\theta} \mathbb{E}_{a_t \sim \pi_\theta(\cdot | \mathbf{s}_t)} \sum_{t=0}^{T} [r(\mathbf{s}_t, \mathbf{a}_t) - \beta \mathcal{D}_{KL}[\pi_{\theta}(\mathbf{a}_t | \mathbf{s}_t)||\pi_{ref}(\mathbf{a}_t | \mathbf{s}_t)]]
% % \label{eq: rlhf_obj}
% \\
% &
% =\max_{\pi_\theta} \mathbb{E}_{a_t \sim \pi_\theta(\cdot | \mathbf{s}_t)} \sum_{t=0}^{T} [r(\mathbf{s}_t, \mathbf{a}_t) - \beta \log \frac{\pi_{\theta}(\mathbf{a}_t | \mathbf{s}_t)}{\pi_{ref}(\mathbf{a}_t | \mathbf{s}_t)}]
% % \nonumber
% \\
% &
% =\max_{\pi_\theta} \mathbb{E}_{a_t \sim \pi_\theta(\cdot | \mathbf{s}_t)} [ \sum_{t=0}^{T} ( r(\mathbf{s}_t, \mathbf{a}_t) + \beta \log \pi_{ref}(\mathbf{a}_t | \mathbf{s}_t) ) + \beta \mathcal{H}(\pi_\theta) | \mathbf{s}_0 \sim \rho(\mathbf{s}_0) ]
% % \nonumber
% \label{eq: rlhf_objective}
% \end{align}


% When $\beta$ is considered as a variable dependent on $t$, Eq. \ref{eq: rlhf_objective} is transformed into:
% \begin{align}
% &
% \max_{\pi_\theta} \mathbb{E}_{a_t \sim \pi_\theta(\cdot | \mathbf{s}_t)} \sum_{t=0}^{T} [( r(\mathbf{s}_t, \mathbf{a}_t) + \beta_t \log \pi_{ref}(\mathbf{a}_t | \mathbf{s}_t)) - \beta_t \log \pi_{\theta}(\mathbf{a}_t | \mathbf{s}_t)]
% \end{align}

% \noindent where $\beta_t$ depends solely on $\mathbf{a}_t$ and $\mathbf{s}_t$. Then, according to \citet{Levine2018ReinforcementLA}, the above formula can be rewritten in a form that includes the KL divergence:
% \begin{align}
% &
% =\mathbb{E}_{\mathbf{s}_t} [ -\beta_t D_{KL}\left( \pi_\theta(\mathbf{a}_t | \mathbf{s}_t) \bigg\| \frac{1}{\exp(V(\mathbf{s}_t))} \exp\left(\frac{r(\mathbf{s}_t, \mathbf{a}_t) + \beta_t \log \pi_{ref}(\mathbf{a}_t | \mathbf{s}_t)}{\beta_t}\right) \right) + V(\mathbf{s}_t) ]
% \label{eq: rlhf_objective_2}
% \end{align}

% \noindent where $V(\mathbf{s}_t) = \beta_t \log \int_{\mathcal{A}} [\exp\frac{r(\mathbf{s}_t, \mathbf{a}_t)}{\beta_t} \pi_{ref}(\mathbf{a}_t | \mathbf{s}_t)] \, d\mathbf{a}_t$. When the KL divergence term is minimized, meaning the two distributions are the same, the above expectation reaches its maximum value. That is:
% \begin{align}
% \pi_\theta(\mathbf{a}_t | \mathbf{s}_t) = \frac{1}{\exp(V(\mathbf{s}_t))} \exp\left(\frac{r(\mathbf{s}_t, \mathbf{a}_t) + \beta_t \log \pi_{ref}(\mathbf{a}_t | \mathbf{s}_t)}{\beta_t}\right)
% \end{align}

% Based on this, we define that:
% \begin{equation}
% Q^*(s_t, a_t) =
% \begin{cases} 
% r(s_t, a_t) + \beta_t \log \pi_{ref}(a_t | s_t) + V^*(s_{t+1}), & \text{if } s_{t+1} \text{ is not terminal} \\
% r(s_t, a_t) + \beta_t \log \pi_{ref}(a_t | s_t), & \text{if } s_{t+1} \text{ is terminal}
% \end{cases}
% \label{eq: t_return}
% \end{equation}

% Thus we can obtain the solution for the optimal policy:
% % $Q(\mathbf{s}_t, \mathbf{a}_t) = r(\mathbf{s}_t, \mathbf{a}_t) + \beta_t \log \pi_{\text{ref}}(\mathbf{a}_t | \mathbf{s}_t)$, thus we can obtain the solution for the optimal policy:
% \begin{align}
% \pi_\theta(\mathbf{a}_t | \mathbf{s}_t) = e^{(Q(\mathbf{s}_t, \mathbf{a}_t) - V(\mathbf{s}_t))/\beta_t}
% \label{eq: fixed_point_2}
% \end{align}

% By log-linearizing the fixed point solution of the optimal policy at the token level, we obtain:
% \begin{align}
% &
% \beta_t \log \pi_\theta(\mathbf{a}_t \mid \mathbf{s}_t) = Q_\theta(\mathbf{s}_t, \mathbf{a}_t) - V_\theta(\mathbf{s}_t)
% \end{align}

% Then, combining with Eq. \ref{eq: t_return}:
% \begin{align}
% \beta_t \log \frac{\pi^*(\mathbf{a}_t \mid \mathbf{s}_t)}{\pi_{\text{ref}}(\mathbf{a}_t \mid \mathbf{s}_t)} = r(\mathbf{s}_t, \mathbf{a}_t) + V^*(\mathbf{s}_{t+1}) - V^*(\mathbf{s}_t).
% \end{align}

% Thus, we can establish the relationship between $\beta_t \log \frac{\pi^*(\mathbf{a}_t \mid \mathbf{s}_t)}{\pi_{\text{ref}}(\mathbf{a}_t \mid \mathbf{s}_t)}$ and $r(\mathbf{s}_t, \mathbf{a}_t)$. 

% According to \citet{rafailov2024rqlanguagemodel}'s definition, two reward functions $r(\mathbf{s}_t, \mathbf{a}_t)$ and $r'(\mathbf{s}_t, \mathbf{a}_t)$ are equivalent if there exists a potential function $\Phi(\mathbf{s})$, such that $r'(\mathbf{s}_t, \mathbf{a}_t) =r(\mathbf{s}_t, \mathbf{a}_t) + \Phi(\mathbf{s}_{t+1})  - \Phi(\mathbf{s}_{t})$. We can conclude that the optimal advantage function is $\beta_t \log \frac{\pi^*(\mathbf{a}_t \mid \mathbf{s}_t)}{\pi_{\text{ref}}(\mathbf{a}_t \mid \mathbf{s}_t)}$.

\section{Detailed Experiment Results}
\label{sec:app-results}
In this section, we presented detailed experiment results which are omitted in the main body of this paper due to space limitation. The detailed experiment results of different methods on ComplexBench, FollowBench and AlpacaEval2 are presented in Table \ref{tab:complexbench}, \ref{tab:alpaca-eval} and \ref{tab:followbench}. The detailed results for the ablative studies of confidence metrics is presented in Table \ref{tab:detail-confidence}. The detailed results for the ablative studies of confidence metrics is presented in Table \ref{tab:detail-noising}. We also present a case study in Table \ref{tab:case-study}, which visualize the token-level weight derived from calibrated confidence score.


\begin{table*}[ht]
\centering
\resizebox{1.0\textwidth}{!}{
\begin{tabular}{cc|cccc|cccc}
\hline
\multirow{3}{*}{\textbf{Scenario}} & \multirow{3}{*}{\textbf{Method}} & \multicolumn{8}{c}{\textbf{ComplexBench}}                                                                                                         \\
                                   &                                  & \multicolumn{4}{c}{\textbf{Meta-Llama3-8B-Instruct}}                    & \multicolumn{4}{c}{\textbf{Qwen2-7B-Instruct}}                          \\
                                   &                                  & \textbf{Overall} & \textbf{And}   & \textbf{Chain} & \textbf{Selection} & \textbf{Overall} & \textbf{And}   & \textbf{Chain} & \textbf{Selection} \\ \hline
\multicolumn{2}{c|}{baseline}                          & 61.49            & 57.22          & 57.22          & 53.55              & 67.24            & 62.58          & 62.58          & 58.97              \\ \hline
\multirow{6}{*}{SelfInst}          & Self-Reward                      & 62.45            & 58.23          & 58.23          & 54.07              & 66.98            & 63.02          & 63.02          & 57.75              \\
                                   & w/ BSM                           & 64.13            & 58.01          & 58.01          & 56.62              & 67.02            & 62.37          & 62.37          & 57.85              \\
                                   & w/ GPT-4                         & 64.05            & 59.44          & 59.44          & 54.78              & —                & —              & —              & —                  \\ \cline{2-10} 
                                   & Self-Correct                     & 55.91            & 49.85          & 49.85          & 46.91              & 64.41            & 59.59          & 59.59          & 55.04              \\
                                   & ISHEEP                           & 62.67            & 57.79          & 57.79          & 54.63              & 67.32            & 61.95          & 61.95          & 59.64              \\ \cline{2-10} 
                                   & \textbf{MuSC}                    & \textbf{65.98}   & \textbf{63.45} & \textbf{63.45} & \textbf{55.96}     & \textbf{69.39}   & \textbf{65.45} & \textbf{65.45} & \textbf{59.79}     \\ \hline
\multirow{7}{*}{PreInst}           & Self-Reward                      & 62.03            & 56.94          & 56.94          & 53.09              & 66.45            & 61.37          & 61.37          & 57.64              \\
                                   & w/ BSM                           & 64.30            & 57.58          & 57.58          & 56.47              & 67.43            & 62.95          & 62.95          & 58.41              \\
                                   & w/ GPT-4                         & 63.52            & 59.08          & 59.08          & 53.91              & —                & —              & —              & —                  \\ \cline{2-10} 
                                   & Self-Correct                     & 60.79            & 55.65          & 55.65          & 52.02              & 64.32            & 60.16          & 60.16          & 54.63              \\
                                   & ISHEEP                           & 62.92            & 56.37          & 56.37          & 54.83              & 67.13            & 64.45          & 64.45          & 57.54              \\
                                   & SFT                              & 53.93            & 45.77          & 45.77          & 44.09              & 65.89            & 60.16          & 60.16          & 57.39              \\ \cline{2-10} 
                                   & \textbf{MuSC}                    & \textbf{64.73}   & \textbf{59.23} & \textbf{59.23} & \textbf{55.91}     & \textbf{70.00}   & \textbf{66.88} & \textbf{66.88} & \textbf{61.38}     \\ \hline
\end{tabular}}
\label{tab:complexbench}
\caption{Detailed experiment results of different methods on ComplexBench.}
\label{tab:complexbench}
\end{table*}

\begin{table*}[ht]
\centering
\resizebox{0.75\textwidth}{!}{
\begin{tabular}{cc|ccc|ccc}
\hline
\multirow{3}{*}{\textbf{Scenario}} & \multirow{3}{*}{\textbf{Method}} & \multicolumn{6}{c}{\textbf{FollowBench}}                                                               \\
                                   &                                  & \multicolumn{3}{c}{\textbf{Meta-Llama3-8B-Instruct}} & \multicolumn{3}{c}{\textbf{Qwen2-7B-Instruct}}  \\
                                   &                                  & \textbf{HSR}     & \textbf{SSR}     & \textbf{CSL}   & \textbf{HSR}   & \textbf{SSR}   & \textbf{CSL}  \\ \hline
\multicolumn{2}{c|}{baseline}                                         & 62.39            & 73.07            & 2.76           & 59.81          & 71.69          & 2.46          \\ \hline
\multirow{6}{*}{SelfInst}          & Self-Reward                      & 61.20            & 72.22            & 2.56           & 55.36          & 69.71          & 2.34          \\
                                   & w/ BSM                           & 64.30            & 73.84            & 2.80           & 57.83          & 70.53          & 2.41          \\
                                   & w/ GPT-4                         & 62.18            & 73.34            & 2.66           & —              & —              & —             \\ \cline{2-8} 
                                   & Self-Correct                     & 54.38            & 67.19            & 2.02           & 51.98          & 67.89          & 2.16          \\
                                   & ISHEEP                           & 62.77            & 72.86            & 2.52           & 57.01          & 69.88          & 2.36          \\ \cline{2-8} 
                                   & \textbf{MuSC}                    & \textbf{66.71}   & \textbf{74.84}   & \textbf{2.92}  & \textbf{62.60} & \textbf{72.57} & \textbf{2.82} \\ \hline
\multirow{7}{*}{PreInst}           & Self-Reward                      & 60.88            & 72.17            & 2.64           & 56.45          & 70.00          & 2.44          \\
                                   & w/ BSM                           & 63.96            & 73.78            & 2.66           & 58.02          & 70.62          & 2.42          \\
                                   & w/ GPT-4                         & 64.02            & 73.26            & 2.64           & —              & —              & —             \\ \cline{2-8} 
                                   & Self-Correct                     & 60.11            & 70.94            & 2.70           & 49.47          & 66.35          & 1.98          \\
                                   & ISHEEP                           & 63.54            & 73.21            & 2.64           & 55.52          & 69.62          & 2.28          \\
                                   & SFT                              & 50.06            & 66.48            & 2.04           & 47.36          & 64.67          & 1.96          \\ \cline{2-8} 
                                   & \textbf{MuSC}                    & \textbf{66.90}   & \textbf{75.11}   & \textbf{2.99}  & \textbf{62.73} & \textbf{73.09} & \textbf{2.86} \\ \hline
\end{tabular}}
\caption{Detailed experiment results of different methods on FollowBench.}
\label{tab:followbench}
\end{table*}

\begin{table*}[ht]
\centering
\resizebox{0.9\textwidth}{!}{
\begin{tabular}{cc|cccccc}
\hline
\multirow{3}{*}{\textbf{Scenario}} & \multirow{3}{*}{\textbf{Method}} & \multicolumn{6}{c}{\textbf{AlpacaEval2}}                                                                          \\
                                   &                                  & \multicolumn{3}{c}{\textbf{Meta-Llama3-8B-Instruct}}    & \multicolumn{3}{c}{\textbf{Qwen2-7B-Instruct}}          \\
                                   &                                  & \textbf{LC (\%)} & \textbf{WR (\%)} & \textbf{Avg. Len} & \textbf{LC (\%)} & \textbf{WR (\%)} & \textbf{Avg. Len} \\ \hline
\multicolumn{2}{c|}{baseline}                                         & 21.07            & 18.73            & 1702              & 15.53            & 13.70            & 1688              \\ \hline
\multirow{6}{*}{SelfInst}          & Self-Reward                      & 19.21            & 19.18            & 1824              & 16.81            & 15.66            & 1756              \\
                                   & w/ BSM                           & 19.03            & 18.34            & 1787              & 16.94            & 15.09            & 1710              \\
                                   & w/ GPT-4                         & 19.55            & 18.53            & 1767              & —                & —                & —                 \\ \cline{2-8} 
                                   & Self-Correct                     & 7.97             & 9.34             & 1919              & 14.01            & 10.92            & 1497              \\
                                   & ISHEEP                           & 22.00            & 19.50            & 1707              & 16.99            & 14.04            & 1619              \\ \cline{2-8} 
                                   & \textbf{MuSC}                    & \textbf{23.87}   & \textbf{20.91}   & \textbf{1708}     & \textbf{20.08}   & \textbf{15.67}   & \textbf{1595}     \\ \hline
\multirow{7}{*}{PreInst}           & Self-Reward                      & 19.93            & 19.04            & 1789              & 15.98            & 15.62            & 1796              \\
                                   & w/ BSM                           & 20.98            & 20.75            & 1829              & 17.17            & 16.21            & 1764              \\
                                   & w/ GPT-4                         & 18.02            & 17.74            & 1804              & —                & —                & —                 \\ \cline{2-8} 
                                   & Self-Correct                     & 6.20             & 5.81             & 1593              & 14.46            & 14.02            & 1737              \\
                                   & ISHEEP                           & 20.23            & 17.86            & 1703              & 16.52            & 13.36            & 1627              \\
                                   & SFT                              & 10.00            & 6.22             & 1079              & 9.52             & 5.25             & 979               \\ \cline{2-8} 
                                   & \textbf{MuSC}                    & \textbf{23.74}   & \textbf{19.53}   & \textbf{1631}     & \textbf{20.29}   & \textbf{15.91}   & \textbf{1613}     \\ \hline
\end{tabular}}
\caption{Detailed experiment results of different methods on AlpacaEval2.}
\label{tab:alpaca-eval}
\end{table*}

\begin{table}[ht]
\centering
\resizebox{0.95\textwidth}{!}{
\begin{tabular}{cc|ccccc|ccccc}
\toprule
\multirow{3}{*}{\textbf{Scenario}} & \multirow{3}{*}{\textbf{Method}} & \multicolumn{5}{c|}{\textbf{Meta-Llama-3-8B-Instruct}}                                    & \multicolumn{5}{c}{\textbf{Qwen-2-7B-Instruct}}                                          \\
                                   &                                  & \multicolumn{3}{c}{\textbf{CF-Bench}}         & \multicolumn{2}{c|}{\textbf{AlpacaEval2}} & \multicolumn{3}{c}{\textbf{CF-Bench}}         & \multicolumn{2}{c}{\textbf{AlpacaEval2}} \\
                                   &                                  & \textbf{CSR}  & \textbf{ISR}  & \textbf{PSR}  & \textbf{LC (\%)}   & \textbf{Avg. Len}       & \textbf{CSR}  & \textbf{ISR}  & \textbf{PSR}  & \textbf{LC (\%)}   & \textbf{Avg. Len}      \\ \midrule
\multirow{6}{*}{PreInst}           & Baseline                         & 0.64          & 0.24          & 0.34          & 21.07                & 1702               & 0.74          & 0.36          & 0.49          & 15.53                & 1688              \\ \cline{2-12} 
                                   % & MuSC w/o conf                  & 0.70          & 0.30          & 0.41          & 21.19                & 1703               & 0.79          & 0.44          & 0.56          & 18.91                & 1604              \\ \cline{2-12} 
                                   & w/ perplexity                    & 0.70          & 0.32          & 0.43          & 22.99                & 1744               & 0.79          & 0.43          & 0.54          & 19.31                & 1675              \\
                                   & w/ PMI                           & 0.69          & 0.29          & 0.41          & 21.92                & 1713               & 0.78          & 0.43          & 0.55          & 17.42                & 1651              \\
                                   & w/ KLDiv                         & 0.69          & 0.31          & 0.42          & 21.86                & 1686               & 0.78          & 0.42          & 0.54          & 18.73                & 1614              \\
                                   & \textbf{w/ entropy}                       & \textbf{0.71} & \textbf{0.34} & \textbf{0.44} & \textbf{23.74}       & 1631               & \textbf{0.80} & \textbf{0.46} & \textbf{0.58} & \textbf{20.38}       & 1683              \\ \bottomrule
\end{tabular}}
\caption{Detailed experiment results of different confidence metrics as the fine-grained weight.}
\label{tab:detail-confidence}
\end{table}

\setlength{\fboxsep}{0pt}

\begin{table}[t]
\centering
\resizebox{1.0\textwidth}{!}{
\begin{tabular}{l}
\hline
\textbf{Instruction 1}: Identify the incorrect word in the given sentence: "The group acquired the skill to ski\\ in a suitable manner." Rewrite the identified incorrect word correctly, ensuring the new sentence\\ adheres to proper capitalization, punctuation, and grammar rules. Include at least one synonym of\\ the incorrect word to test the model's understanding of context and vocabulary, and ensure the synon-\\ ym is used appropriately in the context of the sentence. \\\hline
\textbf{Chosen Response 1}: \colorbox{red!38}{\strut \:The}\colorbox{red!30}{\strut \:group}\colorbox{red!12}{\strut \:acquired}\colorbox{red!50}{\strut \:the}\colorbox{red!25}{\strut \:skill}\colorbox{red!50}{\strut \:to}\colorbox{red!38}{\strut \:ski}\colorbox{red!50}{\strut \:in}\colorbox{red!50}{\strut \:a}\colorbox{red!0}{\strut \:proficient}\colorbox{red!25}{\strut \:manner}\colorbox{red!50}{\strut \:.}\colorbox{red!35}{\strut \:The}\colorbox{red!12}{\strut \:incorrect}\colorbox{red!25}{\strut \:word}\\\colorbox{red!50}{\strut \:is}\colorbox{red!38}{\strut \:"suitable"}\colorbox{red!50}{\strut \:.}\colorbox{red!45}{\strut \:A}\colorbox{red!18}{\strut \:synonym}\colorbox{red!50}{\strut \:for}\colorbox{red!38}{\strut \:"suitable"}\colorbox{red!50}{\strut \:is}\colorbox{red!5}{\strut \:"appropriate"}\colorbox{red!50}{\strut \:,}\colorbox{red!40}{\strut \:but}\colorbox{red!50}{\strut \:it}\colorbox{red!25}{\strut \:doesn't}\colorbox{red!10}{\strut \:fit}\colorbox{red!50}{\strut \:the}\colorbox{red!32}{\strut \:context}\colorbox{red!50}{\strut \:of}\colorbox{red!50}{\strut \:the}\colorbox{red!28}{\strut \:sentence}\colorbox{red!50}{\strut \:.}\\\colorbox{red!8}{\strut \:"Proficient"}\colorbox{red!50}{\strut \:is}\colorbox{red!50}{\strut \:a}\colorbox{red!18}{\strut \:better}\colorbox{red!20}{\strut \:synonym}\colorbox{red!50}{\strut \:in}\colorbox{red!45}{\strut \:this}\colorbox{red!25}{\strut \:context}\colorbox{red!50}{\strut \:as}\colorbox{red!45}{\strut \:it}\colorbox{red!50}{\strut \:means}\colorbox{red!5}{\strut \:skilled}\colorbox{red!50}{\strut \:or}\colorbox{red!32}{\strut \:expert}\colorbox{red!50}{\strut \:.} \\\hline

\textbf{Rejected Response 1}: \colorbox{red!14}{\strut \:The}\colorbox{red!25}{\strut \:word}\colorbox{red!43}{\strut \:"acquired"}\colorbox{red!32}{\strut \:seems}\colorbox{red!36}{\strut \:incorrect}\colorbox{red!0}{\strut \:in}\colorbox{red!0}{\strut \:the}\colorbox{red!27}{\strut \:given}\colorbox{red!23}{\strut \:sentence}\colorbox{red!0}{\strut \:.}\colorbox{red!0}{\strut \:A}\colorbox{red!39}{\strut \:better}\colorbox{red!20}{\strut \:word}\colorbox{red!0}{\strut \:to}\\\colorbox{red!16}{\strut \:use}\colorbox{red!0}{\strut \:in}\colorbox{red!0}{\strut \:this}\colorbox{red!27}{\strut \:context}\colorbox{red!23}{\strut \:would}\colorbox{red!0}{\strut \:be} \colorbox{red!50}{\strut \:"learned"}\colorbox{red!0}{\strut \:.}\colorbox{red!18}{\strut \:Rewritten}\colorbox{red!20}{\strut \:sentence}\colorbox{red!0}{\strut \::}\colorbox{red!9}{\strut \:"The}\colorbox{red!23}{\strut \:group}\colorbox{red!30}{\strut \:learned}\colorbox{red!0}{\strut \:the}\colorbox{red!25}{\strut \:skill}\colorbox{red!0}{\strut \:to}\colorbox{red!9}{\strut \:ski}\colorbox{red!0}{\strut \:in}\colorbox{red!0}{\strut \:a}\\\colorbox{red!36}{\strut \:suitable}\colorbox{red!23}{\strut \:manner}\colorbox{red!0}{\strut \:."} \\\hline

% Create a compelling tagline that effectively highlights the superior efficiency and optimized performance of a product, service, or process. The tagline should communicate the key benefits of streamlined workflows and cost-effective solutions. It should be catchy and memorable, enhancing brand recognition and fostering positive associations with the company's commitment to innovation and excellence. Ensure that the tagline resonates with the target audience and is concise and clear, avoiding unnecessary complexity or jargon. \\\hline

% \colorbox{red!26}{\strut \:Effortlessly}\colorbox{red!32}{\strut \:Optimized}\colorbox{red!50}{\strut \::}\colorbox{red!0}{\strut \:Streamlining}\colorbox{red!15}{\strut \:Efficiency}\colorbox{red!50}{\strut \:,}\colorbox{red!9}{\strut \:Maximizing}\colorbox{red!21}{\strut \:Value} \\\hline

% \colorbox{red!50}{\strut \:Unleash}\colorbox{red!42}{\strut \:Unmatched}\colorbox{red!25}{\strut \:Efficiency}\colorbox{red!0}{\strut \::}\colorbox{red!44}{\strut \:Experience}\colorbox{red!0}{\strut \:the}\colorbox{red!39}{\strut \:Power}\colorbox{red!0}{\strut \:of}\colorbox{red!39}{\strut \:Streamlined}\colorbox{red!28}{\strut \:Solutions} \\\hline

\textbf{Instruction 2}: Rewrite the given sentence in C\# syntax, using an idiom or metaphor to express dis-\\ belief, and ensure the sentence maintains its original meaning while being grammatically correct in\\ C\#. The sentence to be rewritten is: "The idea that pigs can fly seems more plausible than his expla-\\nation for being late."\\\hline

\textbf{Chosen Response 2}:\colorbox{red!50}{\strut \:csharp\textbackslash n}\colorbox{red!50}{\strut \:using}\colorbox{red!50}{\strut \:System}\colorbox{red!50}{\strut \:;\textbackslash n}\colorbox{red!50}{\strut \:class}\colorbox{red!44}{\strut \:Program\textbackslash n}\colorbox{red!50}{\strut \:\{\textbackslash n}\colorbox{red!50}{\strut \:static}\colorbox{red!50}{\strut \:void}\colorbox{red!44}{\strut \:Main}\colorbox{red!50}{\strut \:(}\colorbox{red!50}{\strut \:string}\colorbox{red!39}{\strut \:[]}\colorbox{red!50}{\strut \:args}\\\colorbox{red!50}{\strut \:\textbackslash n)}\colorbox{red!44}{\strut \:\{\textbackslash n}\colorbox{red!50}{\strut \:Console.WriteLine}\colorbox{red!50}{\strut \:(}\colorbox{red!36}{\strut \:"His}\colorbox{red!19}{\strut \:explanation}\colorbox{red!50}{\strut \:for}\colorbox{red!50}{\strut \:being}\colorbox{red!36}{\strut \:late}\colorbox{red!50}{\strut \:is}\colorbox{red!50}{\strut \:so}\colorbox{red!22}{\strut \:baffling}\colorbox{red!50}{\strut \:that}\colorbox{red!31}{\strut \:it's}\colorbox{red!0}{\strut \:akin}\colorbox{red!50}{\strut \:to}\colorbox{red!31}{\strut \:saying}\colorbox{red!39}{\strut \:pigs}\\\colorbox{red!50}{\strut \:can}\colorbox{red!36}{\strut \:fly}\colorbox{red!50}{\strut \:,}\colorbox{red!19}{\strut \:making}\colorbox{red!50}{\strut \:it}\colorbox{red!33}{\strut \:more}\colorbox{red!8}{\strut \:believable}\colorbox{red!50}{\strut \:in}\colorbox{red!25}{\strut \:comparison}\colorbox{red!44}{\strut \:.}\colorbox{red!50}{\strut \:)}\colorbox{red!50}{\strut \:;\textbackslash n}\colorbox{red!50}{\strut \:\}\textbackslash n}\colorbox{red!50}{\strut \:\}} \\\hline
\textbf{Rejected Response 2}:\colorbox{red!50}{\strut \:Wow}\colorbox{red!0}{\strut \:,}\colorbox{red!15}{\strut \:it's}\colorbox{red!30}{\strut \:truly}\colorbox{red!39}{\strut \:hard}\colorbox{red!0}{\strut \:to}\colorbox{red!20}{\strut \:believe}\colorbox{red!0}{\strut \:that}\colorbox{red!9}{\strut \:pigs}\colorbox{red!33}{\strut \:flying}\colorbox{red!20}{\strut \:could}\colorbox{red!0}{\strut \:be}\colorbox{red!11}{\strut \:more}\colorbox{red!39}{\strut \:plausible}\colorbox{red!0}{\strut \:than}\\\colorbox{red!15}{\strut \:my}\colorbox{red!22}{\strut \:explanation}\colorbox{red!4}{\strut \:for}\colorbox{red!0}{\strut \:being}\colorbox{red!11}{\strut \:late}\colorbox{red!4}{\strut \:!}\\\hline

\end{tabular}}
\caption{Visualization of dynamic weights derived for chosen and rejected responses, based on our proposed calibrated entropy score. We select two samples from the datasets as an illustration.}
\label{tab:case-study}
\end{table}


\begin{figure}[h]
    \centering
    \includegraphics[width=0.8\linewidth]{figures/prompt-decom.png}
    \caption{The prompt template used for instruction decomposition.}
    \label{fig: prompt-decom}
    \vspace{-1mm}
\end{figure}

\begin{figure}[h]
    \centering
    \includegraphics[width=0.8\linewidth]{figures/prompt-recomb.png}
    \caption{The prompt template used for constraint recombination.}
    \label{fig: prompt-recomb}
    \vspace{-1mm}
\end{figure}

\begin{figure}[h]
    \centering
    \includegraphics[width=0.8\linewidth]{figures/prompt-selfinst.png}
    \caption{The prompt template used for self-instruct.}
    \label{fig: prompt-selfinst}
    \vspace{-1mm}
\end{figure}

\begin{figure}[h]
    \centering
    \includegraphics[width=0.8\linewidth]{figures/prompt-sub.png}
    \caption{The prompt template used for constraint substitution.}
    \label{fig: prompt-sub}
    \vspace{-1mm}
\end{figure}

\begin{figure}[h]
    \centering
    \includegraphics[width=0.8\linewidth]{figures/prompt-neg.png}
    \caption{The prompt template used for constraint negation.}
    \label{fig: prompt-neg}
    \vspace{-1mm}
\end{figure}
\label{sec:appendix}
\end{document}
