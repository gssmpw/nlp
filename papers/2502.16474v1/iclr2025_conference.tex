\documentclass[sigconf]{acmart}
% \usepackage{iclr2025_conference,times}

% Optional math commands from https://github.com/goodfeli/dlbook_notation.
% %%%%% NEW MATH DEFINITIONS %%%%%

\usepackage{amsmath,amsfonts,bm}
\usepackage{derivative}
% Mark sections of captions for referring to divisions of figures
\newcommand{\figleft}{{\em (Left)}}
\newcommand{\figcenter}{{\em (Center)}}
\newcommand{\figright}{{\em (Right)}}
\newcommand{\figtop}{{\em (Top)}}
\newcommand{\figbottom}{{\em (Bottom)}}
\newcommand{\captiona}{{\em (a)}}
\newcommand{\captionb}{{\em (b)}}
\newcommand{\captionc}{{\em (c)}}
\newcommand{\captiond}{{\em (d)}}

% Highlight a newly defined term
\newcommand{\newterm}[1]{{\bf #1}}

% Derivative d 
\newcommand{\deriv}{{\mathrm{d}}}

% Figure reference, lower-case.
\def\figref#1{figure~\ref{#1}}
% Figure reference, capital. For start of sentence
\def\Figref#1{Figure~\ref{#1}}
\def\twofigref#1#2{figures \ref{#1} and \ref{#2}}
\def\quadfigref#1#2#3#4{figures \ref{#1}, \ref{#2}, \ref{#3} and \ref{#4}}
% Section reference, lower-case.
\def\secref#1{section~\ref{#1}}
% Section reference, capital.
\def\Secref#1{Section~\ref{#1}}
% Reference to two sections.
\def\twosecrefs#1#2{sections \ref{#1} and \ref{#2}}
% Reference to three sections.
\def\secrefs#1#2#3{sections \ref{#1}, \ref{#2} and \ref{#3}}
% Reference to an equation, lower-case.
\def\eqref#1{equation~\ref{#1}}
% Reference to an equation, upper case
\def\Eqref#1{Equation~\ref{#1}}
% A raw reference to an equation---avoid using if possible
\def\plaineqref#1{\ref{#1}}
% Reference to a chapter, lower-case.
\def\chapref#1{chapter~\ref{#1}}
% Reference to an equation, upper case.
\def\Chapref#1{Chapter~\ref{#1}}
% Reference to a range of chapters
\def\rangechapref#1#2{chapters\ref{#1}--\ref{#2}}
% Reference to an algorithm, lower-case.
\def\algref#1{algorithm~\ref{#1}}
% Reference to an algorithm, upper case.
\def\Algref#1{Algorithm~\ref{#1}}
\def\twoalgref#1#2{algorithms \ref{#1} and \ref{#2}}
\def\Twoalgref#1#2{Algorithms \ref{#1} and \ref{#2}}
% Reference to a part, lower case
\def\partref#1{part~\ref{#1}}
% Reference to a part, upper case
\def\Partref#1{Part~\ref{#1}}
\def\twopartref#1#2{parts \ref{#1} and \ref{#2}}

\def\ceil#1{\lceil #1 \rceil}
\def\floor#1{\lfloor #1 \rfloor}
\def\1{\bm{1}}
\newcommand{\train}{\mathcal{D}}
\newcommand{\valid}{\mathcal{D_{\mathrm{valid}}}}
\newcommand{\test}{\mathcal{D_{\mathrm{test}}}}

\def\eps{{\epsilon}}


% Random variables
\def\reta{{\textnormal{$\eta$}}}
\def\ra{{\textnormal{a}}}
\def\rb{{\textnormal{b}}}
\def\rc{{\textnormal{c}}}
\def\rd{{\textnormal{d}}}
\def\re{{\textnormal{e}}}
\def\rf{{\textnormal{f}}}
\def\rg{{\textnormal{g}}}
\def\rh{{\textnormal{h}}}
\def\ri{{\textnormal{i}}}
\def\rj{{\textnormal{j}}}
\def\rk{{\textnormal{k}}}
\def\rl{{\textnormal{l}}}
% rm is already a command, just don't name any random variables m
\def\rn{{\textnormal{n}}}
\def\ro{{\textnormal{o}}}
\def\rp{{\textnormal{p}}}
\def\rq{{\textnormal{q}}}
\def\rr{{\textnormal{r}}}
\def\rs{{\textnormal{s}}}
\def\rt{{\textnormal{t}}}
\def\ru{{\textnormal{u}}}
\def\rv{{\textnormal{v}}}
\def\rw{{\textnormal{w}}}
\def\rx{{\textnormal{x}}}
\def\ry{{\textnormal{y}}}
\def\rz{{\textnormal{z}}}

% Random vectors
\def\rvepsilon{{\mathbf{\epsilon}}}
\def\rvphi{{\mathbf{\phi}}}
\def\rvtheta{{\mathbf{\theta}}}
\def\rva{{\mathbf{a}}}
\def\rvb{{\mathbf{b}}}
\def\rvc{{\mathbf{c}}}
\def\rvd{{\mathbf{d}}}
\def\rve{{\mathbf{e}}}
\def\rvf{{\mathbf{f}}}
\def\rvg{{\mathbf{g}}}
\def\rvh{{\mathbf{h}}}
\def\rvu{{\mathbf{i}}}
\def\rvj{{\mathbf{j}}}
\def\rvk{{\mathbf{k}}}
\def\rvl{{\mathbf{l}}}
\def\rvm{{\mathbf{m}}}
\def\rvn{{\mathbf{n}}}
\def\rvo{{\mathbf{o}}}
\def\rvp{{\mathbf{p}}}
\def\rvq{{\mathbf{q}}}
\def\rvr{{\mathbf{r}}}
\def\rvs{{\mathbf{s}}}
\def\rvt{{\mathbf{t}}}
\def\rvu{{\mathbf{u}}}
\def\rvv{{\mathbf{v}}}
\def\rvw{{\mathbf{w}}}
\def\rvx{{\mathbf{x}}}
\def\rvy{{\mathbf{y}}}
\def\rvz{{\mathbf{z}}}

% Elements of random vectors
\def\erva{{\textnormal{a}}}
\def\ervb{{\textnormal{b}}}
\def\ervc{{\textnormal{c}}}
\def\ervd{{\textnormal{d}}}
\def\erve{{\textnormal{e}}}
\def\ervf{{\textnormal{f}}}
\def\ervg{{\textnormal{g}}}
\def\ervh{{\textnormal{h}}}
\def\ervi{{\textnormal{i}}}
\def\ervj{{\textnormal{j}}}
\def\ervk{{\textnormal{k}}}
\def\ervl{{\textnormal{l}}}
\def\ervm{{\textnormal{m}}}
\def\ervn{{\textnormal{n}}}
\def\ervo{{\textnormal{o}}}
\def\ervp{{\textnormal{p}}}
\def\ervq{{\textnormal{q}}}
\def\ervr{{\textnormal{r}}}
\def\ervs{{\textnormal{s}}}
\def\ervt{{\textnormal{t}}}
\def\ervu{{\textnormal{u}}}
\def\ervv{{\textnormal{v}}}
\def\ervw{{\textnormal{w}}}
\def\ervx{{\textnormal{x}}}
\def\ervy{{\textnormal{y}}}
\def\ervz{{\textnormal{z}}}

% Random matrices
\def\rmA{{\mathbf{A}}}
\def\rmB{{\mathbf{B}}}
\def\rmC{{\mathbf{C}}}
\def\rmD{{\mathbf{D}}}
\def\rmE{{\mathbf{E}}}
\def\rmF{{\mathbf{F}}}
\def\rmG{{\mathbf{G}}}
\def\rmH{{\mathbf{H}}}
\def\rmI{{\mathbf{I}}}
\def\rmJ{{\mathbf{J}}}
\def\rmK{{\mathbf{K}}}
\def\rmL{{\mathbf{L}}}
\def\rmM{{\mathbf{M}}}
\def\rmN{{\mathbf{N}}}
\def\rmO{{\mathbf{O}}}
\def\rmP{{\mathbf{P}}}
\def\rmQ{{\mathbf{Q}}}
\def\rmR{{\mathbf{R}}}
\def\rmS{{\mathbf{S}}}
\def\rmT{{\mathbf{T}}}
\def\rmU{{\mathbf{U}}}
\def\rmV{{\mathbf{V}}}
\def\rmW{{\mathbf{W}}}
\def\rmX{{\mathbf{X}}}
\def\rmY{{\mathbf{Y}}}
\def\rmZ{{\mathbf{Z}}}

% Elements of random matrices
\def\ermA{{\textnormal{A}}}
\def\ermB{{\textnormal{B}}}
\def\ermC{{\textnormal{C}}}
\def\ermD{{\textnormal{D}}}
\def\ermE{{\textnormal{E}}}
\def\ermF{{\textnormal{F}}}
\def\ermG{{\textnormal{G}}}
\def\ermH{{\textnormal{H}}}
\def\ermI{{\textnormal{I}}}
\def\ermJ{{\textnormal{J}}}
\def\ermK{{\textnormal{K}}}
\def\ermL{{\textnormal{L}}}
\def\ermM{{\textnormal{M}}}
\def\ermN{{\textnormal{N}}}
\def\ermO{{\textnormal{O}}}
\def\ermP{{\textnormal{P}}}
\def\ermQ{{\textnormal{Q}}}
\def\ermR{{\textnormal{R}}}
\def\ermS{{\textnormal{S}}}
\def\ermT{{\textnormal{T}}}
\def\ermU{{\textnormal{U}}}
\def\ermV{{\textnormal{V}}}
\def\ermW{{\textnormal{W}}}
\def\ermX{{\textnormal{X}}}
\def\ermY{{\textnormal{Y}}}
\def\ermZ{{\textnormal{Z}}}

% Vectors
\def\vzero{{\bm{0}}}
\def\vone{{\bm{1}}}
\def\vmu{{\bm{\mu}}}
\def\vtheta{{\bm{\theta}}}
\def\vphi{{\bm{\phi}}}
\def\va{{\bm{a}}}
\def\vb{{\bm{b}}}
\def\vc{{\bm{c}}}
\def\vd{{\bm{d}}}
\def\ve{{\bm{e}}}
\def\vf{{\bm{f}}}
\def\vg{{\bm{g}}}
\def\vh{{\bm{h}}}
\def\vi{{\bm{i}}}
\def\vj{{\bm{j}}}
\def\vk{{\bm{k}}}
\def\vl{{\bm{l}}}
\def\vm{{\bm{m}}}
\def\vn{{\bm{n}}}
\def\vo{{\bm{o}}}
\def\vp{{\bm{p}}}
\def\vq{{\bm{q}}}
\def\vr{{\bm{r}}}
\def\vs{{\bm{s}}}
\def\vt{{\bm{t}}}
\def\vu{{\bm{u}}}
\def\vv{{\bm{v}}}
\def\vw{{\bm{w}}}
\def\vx{{\bm{x}}}
\def\vy{{\bm{y}}}
\def\vz{{\bm{z}}}

% Elements of vectors
\def\evalpha{{\alpha}}
\def\evbeta{{\beta}}
\def\evepsilon{{\epsilon}}
\def\evlambda{{\lambda}}
\def\evomega{{\omega}}
\def\evmu{{\mu}}
\def\evpsi{{\psi}}
\def\evsigma{{\sigma}}
\def\evtheta{{\theta}}
\def\eva{{a}}
\def\evb{{b}}
\def\evc{{c}}
\def\evd{{d}}
\def\eve{{e}}
\def\evf{{f}}
\def\evg{{g}}
\def\evh{{h}}
\def\evi{{i}}
\def\evj{{j}}
\def\evk{{k}}
\def\evl{{l}}
\def\evm{{m}}
\def\evn{{n}}
\def\evo{{o}}
\def\evp{{p}}
\def\evq{{q}}
\def\evr{{r}}
\def\evs{{s}}
\def\evt{{t}}
\def\evu{{u}}
\def\evv{{v}}
\def\evw{{w}}
\def\evx{{x}}
\def\evy{{y}}
\def\evz{{z}}

% Matrix
\def\mA{{\bm{A}}}
\def\mB{{\bm{B}}}
\def\mC{{\bm{C}}}
\def\mD{{\bm{D}}}
\def\mE{{\bm{E}}}
\def\mF{{\bm{F}}}
\def\mG{{\bm{G}}}
\def\mH{{\bm{H}}}
\def\mI{{\bm{I}}}
\def\mJ{{\bm{J}}}
\def\mK{{\bm{K}}}
\def\mL{{\bm{L}}}
\def\mM{{\bm{M}}}
\def\mN{{\bm{N}}}
\def\mO{{\bm{O}}}
\def\mP{{\bm{P}}}
\def\mQ{{\bm{Q}}}
\def\mR{{\bm{R}}}
\def\mS{{\bm{S}}}
\def\mT{{\bm{T}}}
\def\mU{{\bm{U}}}
\def\mV{{\bm{V}}}
\def\mW{{\bm{W}}}
\def\mX{{\bm{X}}}
\def\mY{{\bm{Y}}}
\def\mZ{{\bm{Z}}}
\def\mBeta{{\bm{\beta}}}
\def\mPhi{{\bm{\Phi}}}
\def\mLambda{{\bm{\Lambda}}}
\def\mSigma{{\bm{\Sigma}}}

% Tensor
\DeclareMathAlphabet{\mathsfit}{\encodingdefault}{\sfdefault}{m}{sl}
\SetMathAlphabet{\mathsfit}{bold}{\encodingdefault}{\sfdefault}{bx}{n}
\newcommand{\tens}[1]{\bm{\mathsfit{#1}}}
\def\tA{{\tens{A}}}
\def\tB{{\tens{B}}}
\def\tC{{\tens{C}}}
\def\tD{{\tens{D}}}
\def\tE{{\tens{E}}}
\def\tF{{\tens{F}}}
\def\tG{{\tens{G}}}
\def\tH{{\tens{H}}}
\def\tI{{\tens{I}}}
\def\tJ{{\tens{J}}}
\def\tK{{\tens{K}}}
\def\tL{{\tens{L}}}
\def\tM{{\tens{M}}}
\def\tN{{\tens{N}}}
\def\tO{{\tens{O}}}
\def\tP{{\tens{P}}}
\def\tQ{{\tens{Q}}}
\def\tR{{\tens{R}}}
\def\tS{{\tens{S}}}
\def\tT{{\tens{T}}}
\def\tU{{\tens{U}}}
\def\tV{{\tens{V}}}
\def\tW{{\tens{W}}}
\def\tX{{\tens{X}}}
\def\tY{{\tens{Y}}}
\def\tZ{{\tens{Z}}}


% Graph
\def\gA{{\mathcal{A}}}
\def\gB{{\mathcal{B}}}
\def\gC{{\mathcal{C}}}
\def\gD{{\mathcal{D}}}
\def\gE{{\mathcal{E}}}
\def\gF{{\mathcal{F}}}
\def\gG{{\mathcal{G}}}
\def\gH{{\mathcal{H}}}
\def\gI{{\mathcal{I}}}
\def\gJ{{\mathcal{J}}}
\def\gK{{\mathcal{K}}}
\def\gL{{\mathcal{L}}}
\def\gM{{\mathcal{M}}}
\def\gN{{\mathcal{N}}}
\def\gO{{\mathcal{O}}}
\def\gP{{\mathcal{P}}}
\def\gQ{{\mathcal{Q}}}
\def\gR{{\mathcal{R}}}
\def\gS{{\mathcal{S}}}
\def\gT{{\mathcal{T}}}
\def\gU{{\mathcal{U}}}
\def\gV{{\mathcal{V}}}
\def\gW{{\mathcal{W}}}
\def\gX{{\mathcal{X}}}
\def\gY{{\mathcal{Y}}}
\def\gZ{{\mathcal{Z}}}

% Sets
\def\sA{{\mathbb{A}}}
\def\sB{{\mathbb{B}}}
\def\sC{{\mathbb{C}}}
\def\sD{{\mathbb{D}}}
% Don't use a set called E, because this would be the same as our symbol
% for expectation.
\def\sF{{\mathbb{F}}}
\def\sG{{\mathbb{G}}}
\def\sH{{\mathbb{H}}}
\def\sI{{\mathbb{I}}}
\def\sJ{{\mathbb{J}}}
\def\sK{{\mathbb{K}}}
\def\sL{{\mathbb{L}}}
\def\sM{{\mathbb{M}}}
\def\sN{{\mathbb{N}}}
\def\sO{{\mathbb{O}}}
\def\sP{{\mathbb{P}}}
\def\sQ{{\mathbb{Q}}}
\def\sR{{\mathbb{R}}}
\def\sS{{\mathbb{S}}}
\def\sT{{\mathbb{T}}}
\def\sU{{\mathbb{U}}}
\def\sV{{\mathbb{V}}}
\def\sW{{\mathbb{W}}}
\def\sX{{\mathbb{X}}}
\def\sY{{\mathbb{Y}}}
\def\sZ{{\mathbb{Z}}}

% Entries of a matrix
\def\emLambda{{\Lambda}}
\def\emA{{A}}
\def\emB{{B}}
\def\emC{{C}}
\def\emD{{D}}
\def\emE{{E}}
\def\emF{{F}}
\def\emG{{G}}
\def\emH{{H}}
\def\emI{{I}}
\def\emJ{{J}}
\def\emK{{K}}
\def\emL{{L}}
\def\emM{{M}}
\def\emN{{N}}
\def\emO{{O}}
\def\emP{{P}}
\def\emQ{{Q}}
\def\emR{{R}}
\def\emS{{S}}
\def\emT{{T}}
\def\emU{{U}}
\def\emV{{V}}
\def\emW{{W}}
\def\emX{{X}}
\def\emY{{Y}}
\def\emZ{{Z}}
\def\emSigma{{\Sigma}}

% entries of a tensor
% Same font as tensor, without \bm wrapper
\newcommand{\etens}[1]{\mathsfit{#1}}
\def\etLambda{{\etens{\Lambda}}}
\def\etA{{\etens{A}}}
\def\etB{{\etens{B}}}
\def\etC{{\etens{C}}}
\def\etD{{\etens{D}}}
\def\etE{{\etens{E}}}
\def\etF{{\etens{F}}}
\def\etG{{\etens{G}}}
\def\etH{{\etens{H}}}
\def\etI{{\etens{I}}}
\def\etJ{{\etens{J}}}
\def\etK{{\etens{K}}}
\def\etL{{\etens{L}}}
\def\etM{{\etens{M}}}
\def\etN{{\etens{N}}}
\def\etO{{\etens{O}}}
\def\etP{{\etens{P}}}
\def\etQ{{\etens{Q}}}
\def\etR{{\etens{R}}}
\def\etS{{\etens{S}}}
\def\etT{{\etens{T}}}
\def\etU{{\etens{U}}}
\def\etV{{\etens{V}}}
\def\etW{{\etens{W}}}
\def\etX{{\etens{X}}}
\def\etY{{\etens{Y}}}
\def\etZ{{\etens{Z}}}

% The true underlying data generating distribution
\newcommand{\pdata}{p_{\rm{data}}}
\newcommand{\ptarget}{p_{\rm{target}}}
\newcommand{\pprior}{p_{\rm{prior}}}
\newcommand{\pbase}{p_{\rm{base}}}
\newcommand{\pref}{p_{\rm{ref}}}

% The empirical distribution defined by the training set
\newcommand{\ptrain}{\hat{p}_{\rm{data}}}
\newcommand{\Ptrain}{\hat{P}_{\rm{data}}}
% The model distribution
\newcommand{\pmodel}{p_{\rm{model}}}
\newcommand{\Pmodel}{P_{\rm{model}}}
\newcommand{\ptildemodel}{\tilde{p}_{\rm{model}}}
% Stochastic autoencoder distributions
\newcommand{\pencode}{p_{\rm{encoder}}}
\newcommand{\pdecode}{p_{\rm{decoder}}}
\newcommand{\precons}{p_{\rm{reconstruct}}}

\newcommand{\laplace}{\mathrm{Laplace}} % Laplace distribution

\newcommand{\E}{\mathbb{E}}
\newcommand{\Ls}{\mathcal{L}}
\newcommand{\R}{\mathbb{R}}
\newcommand{\emp}{\tilde{p}}
\newcommand{\lr}{\alpha}
\newcommand{\reg}{\lambda}
\newcommand{\rect}{\mathrm{rectifier}}
\newcommand{\softmax}{\mathrm{softmax}}
\newcommand{\sigmoid}{\sigma}
\newcommand{\softplus}{\zeta}
\newcommand{\KL}{D_{\mathrm{KL}}}
\newcommand{\Var}{\mathrm{Var}}
\newcommand{\standarderror}{\mathrm{SE}}
\newcommand{\Cov}{\mathrm{Cov}}
% Wolfram Mathworld says $L^2$ is for function spaces and $\ell^2$ is for vectors
% But then they seem to use $L^2$ for vectors throughout the site, and so does
% wikipedia.
\newcommand{\normlzero}{L^0}
\newcommand{\normlone}{L^1}
\newcommand{\normltwo}{L^2}
\newcommand{\normlp}{L^p}
\newcommand{\normmax}{L^\infty}

\newcommand{\parents}{Pa} % See usage in notation.tex. Chosen to match Daphne's book.

\DeclareMathOperator*{\argmax}{arg\,max}
\DeclareMathOperator*{\argmin}{arg\,min}

\DeclareMathOperator{\sign}{sign}
\DeclareMathOperator{\Tr}{Tr}
\let\ab\allowbreak

\usepackage{wrapfig}
\usepackage{hyperref}
\usepackage{url}
\usepackage{caption}

\usepackage{graphicx}
\newtheorem{definition}{Definition}
\newtheorem{proposition}{Proposition}
\newtheorem{assumption}{Assumption}
\newtheorem{theorem}{Theorem}
\usepackage{multirow}
\usepackage{algorithmic}
\usepackage{algorithm}
\usepackage{enumitem}
\setenumerate[1]{itemsep=0pt,partopsep=0pt,parsep=\parskip,topsep=5pt}
\setitemize[1]{itemsep=0pt,partopsep=0pt,parsep=\parskip,topsep=5pt}
\setdescription{itemsep=0pt,partopsep=0pt,parsep=\parskip,topsep=5pt}
\usepackage{booktabs}
\usepackage{multirow}
\usepackage[normalem]{ulem}
\useunder{\uline}{\ul}{}


% \AtBeginDocument{%
%   \providecommand\BibTeX{{%
%     Bib\TeX}}}

%% Rights management information.  This information is sent to you
%% when you complete the rights form.  These commands have SAMPLE
%% values in them; it is your responsibility as an author to replace
%% the commands and values with those provided to you when you
%% complete the rights form.
% \setcopyright{acmlicensed}
% \copyrightyear{2018}
% \acmYear{2018}
% \acmDOI{XXXXXXX.XXXXXXX}

%% These commands are for a PROCEEDINGS abstract or paper.
% \acmConference[Conference acronym 'XX]{Make sure to enter the correct
%   conference title from your rights confirmation emai}{June 03--05,
%   2018}{Woodstock, NY}
%%
%%  Uncomment \acmBooktitle if the title of the proceedings is different
%%  from ``Proceedings of ...''!
%%
%%\acmBooktitle{Woodstock '18: ACM Symposium on Neural Gaze Detection,
%%  June 03--05, 2018, Woodstock, NY}
% \acmISBN{978-1-4503-XXXX-X/18/06}

% Authors must not appear in the submitted version. They should be hidden
% as long as the \iclrfinalcopy macro remains commented out below.
% Non-anonymous submissions will be rejected without review.
\author{
Guanyu Lin\textsuperscript{1 2}, Zhigang Hua\textsuperscript{2}, Tao Feng\textsuperscript{1}, Shuang Yang\textsuperscript{2}, Bo Long\textsuperscript{2}, Jiaxuan You\textsuperscript{1}\\
\textsuperscript{1}University of Illinois at Urbana-Champaign,
\textsuperscript{2}Meta AI
}



% \author{Antiquus S.~Hippocampus, Natalia Cerebro \& Amelie P. Amygdale \thanks{ Use footnote for providing further information
% about author (webpage, alternative address)---\emph{not} for acknowledging
% funding agencies.  Funding acknowledgements go at the end of the paper.} \\
% Department of Computer Science\\
% Cranberry-Lemon University\\
% Pittsburgh, PA 15213, USA \\
% \texttt{\{hippo,brain,jen\}@cs.cranberry-lemon.edu} \\
% \And
% Ji Q. Ren \& Yevgeny LeNet \\
% Department of Computational Neuroscience \\
% University of the Witwatersrand \\
% Joburg, South Africa \\
% \texttt{\{robot,net\}@wits.ac.za} \\
% \AND
% Coauthor \\
% Affiliation \\
% Address \\
% \texttt{email}


% The \author macro works with any number of authors. There are two commands
% used to separate the names and addresses of multiple authors: \And and \AND.
%
% Using \And between authors leaves it to \LaTeX{} to determine where to break
% the lines. Using \AND forces a linebreak at that point. So, if \LaTeX{}
% puts 3 of 4 authors names on the first line, and the last on the second
% line, try using \AND instead of \And before the third author name.

\newcommand{\fix}{\marginpar{FIX}}
\newcommand{\new}{\marginpar{NEW}}
\newcommand{\lgy}[1]{{{\textcolor{black}{#1}}}}

%\iclrfinalcopy % Uncomment for camera-ready version, but NOT for submission.
\begin{document}

% \title{Unified Semantic and ID Representation Learning in Recommendation}

% how about this?
\title{Unified Semantic and ID Representation Learning for Deep Recommenders}



%%
%% By default, the full list of authors will be used in the page
%% headers. Often, this list is too long, and will overlap
%% other information printed in the page headers. This command allows
%% the author to define a more concise list
%% of authors' names for this purpose.
\renewcommand{\shortauthors}{Lin et al.}

%%
%% The abstract is a short summary of the work to be presented in the
%% article.
\begin{abstract}

Effective recommendation is crucial for large-scale online platforms. Traditional recommendation systems primarily rely on ID tokens to uniquely identify items, which can effectively capture specific item relationships but suffer from issues such as redundancy and poor performance in cold-start scenarios. Recent approaches have explored using semantic tokens as an alternative, yet they face challenges, including item duplication and inconsistent performance gains, leaving the potential advantages of semantic tokens inadequately examined. To address these limitations, we propose a Unified Semantic and ID Representation Learning framework that leverages the complementary strengths of both token types. In our framework, ID tokens capture unique item attributes, while semantic tokens represent shared, transferable characteristics. Additionally, we analyze the role of cosine similarity and Euclidean distance in embedding search, revealing that cosine similarity is more effective in decoupling accumulated embeddings, while Euclidean distance excels in distinguishing unique items. Our framework integrates cosine similarity in earlier layers and Euclidean distance in the final layer to optimize representation learning. Experiments on three benchmark datasets show that our method significantly outperforms state-of-the-art baselines, with improvements ranging from 6\% to 17\% and a reduction in token size by over 80\%. These results demonstrate the effectiveness of combining ID and semantic tokenization to enhance the generalization ability of recommender systems.

\end{abstract}

%%
%% The code below is generated by the tool at http://dl.acm.org/ccs.cfm.
%% Please copy and paste the code instead of the example below.
%%
% \begin{CCSXML}
% <ccs2012>
%  <concept>
%   <concept_id>00000000.0000000.0000000</concept_id>
%   <concept_desc>Do Not Use This Code, Generate the Correct Terms for Your Paper</concept_desc>
%   <concept_significance>500</concept_significance>
%  </concept>
%  <concept>
%   <concept_id>00000000.00000000.00000000</concept_id>
%   <concept_desc>Do Not Use This Code, Generate the Correct Terms for Your Paper</concept_desc>
%   <concept_significance>300</concept_significance>
%  </concept>
%  <concept>
%   <concept_id>00000000.00000000.00000000</concept_id>
%   <concept_desc>Do Not Use This Code, Generate the Correct Terms for Your Paper</concept_desc>
%   <concept_significance>100</concept_significance>
%  </concept>
%  <concept>
%   <concept_id>00000000.00000000.00000000</concept_id>
%   <concept_desc>Do Not Use This Code, Generate the Correct Terms for Your Paper</concept_desc>
%   <concept_significance>100</concept_significance>
%  </concept>
% </ccs2012>
% \end{CCSXML}

% \ccsdesc[500]{Do Not Use This Code~Generate the Correct Terms for Your Paper}
% \ccsdesc[300]{Do Not Use This Code~Generate the Correct Terms for Your Paper}
% \ccsdesc{Do Not Use This Code~Generate the Correct Terms for Your Paper}
% \ccsdesc[100]{Do Not Use This Code~Generate the Correct Terms for Your Paper}

%%
%% Keywords. The author(s) should pick words that accurately describe
%% the work being presented. Separate the keywords with commas.
% \keywords{Do, Not, Us, This, Code, Put, the, Correct, Terms, for,
%   Your, Paper}
%% A "teaser" image appears between the author and affiliation
%% information and the body of the document, and typically spans the
%% page.


% \received{20 February 2007}
% \received[revised]{12 March 2009}
% \received[accepted]{5 June 2009}

\maketitle

% \begin{abstract}

% \end{abstract}




\section{Introduction}
\begin{figure}[!htb]
		\begin{tabular}{c}
		    	\includegraphics[width=.6\columnwidth]{beauty.png}
		\end{tabular}
	\caption{Visualization of ID tokens on Amazon Beauty dataset. Here some ID tokens with the same color share a close embedding space, which means they can be compressed and represented with shared semantic tokens.}	\label{fig:token_dist}
\end{figure} 

In large-scale online platforms such as YouTube~\citep{covington2016deep}, TikTok~\citep{dfar}, and Amazon~\citep{he2016ups}, effectively recommending items that align with users’ preferences while filtering out irrelevant content is crucial. Traditional recommendation systems predominantly rely on ID tokens, wherein each item is uniquely identified by a distinct token~\citep{fm, sasrec}. However, as the number of items expands, this approach becomes increasingly cumbersome due to the redundancy and sheer scale of the token space. 

To overcome these limitations, recent research~\citep{rajput2024recommender, singh2023better} has explored the use of semantic tokens as an alternative to ID tokens. Nevertheless, existing works face challenges, such as inconsistent performance improvements and issues with item duplication. For instance, TIGER~\citep{rajput2024recommender} introduces semantic tokens within a deeper and more complex model architecture, making it difficult to isolate the benefits of semantic tokens themselves. Consequently, the true advantages of semantic tokens over ID tokens remain underexplored. Another study~\citep{singh2023better} demonstrates that semantic tokens offer notable improvements primarily in cold-start scenarios, yet both studies report that semantic tokens can map multiple items to the same token, leading to duplication. These open questions invite further investigation into the comparative effectiveness of semantic and ID tokens: \textit{Are semantic tokens inherently superior to ID tokens in recommendation tasks?}

In reality, semantic tokens and ID tokens complement each other. ID tokens have two primary advantages: (1) they can uncover unique, implicit relationships between items, such as the well-known association between beer and diapers, and (2) they facilitate the distinction between different items. However, ID tokens struggle to capture shared attributes across similar items and often suffer from redundancy at scale. This is analogous to whole-word tokenization in Natural Language Processing (NLP)\citep{NIPS2000_728f206c, collobert2008unified, mikolov2013distributed}, which tends to fail with unknown or out-of-vocabulary words\citep{mielke2021between}, making ID tokens less effective in cold-start situations. On the other hand, semantic tokens resemble sub-word tokenization in NLP~\citep{mikolov2012subword, wang2020neural}, where combinations of existing semantic tokens can represent new or unknown items. However, the drawback of semantic tokens lies in their tendency to map multiple, similar items to identical representations, thus failing to distinguish between them~\citep{singh2023better}. In summary, while semantic tokens excel in generalizing to unknown items, they are less effective in memorizing unique ones, suggesting that neither approach is universally superior.

To harness the complementary strengths of both token types, we propose a hybrid framework that unifies ID and semantic tokens. As illustrated in Figure~\ref{fig:token_dist}, our approach begins by visualizing the distribution of ID tokens, revealing that certain items cluster closely together in the embedding space. From this, we hypothesize that only a few dimensions of the ID token space are needed to capture unique item characteristics, while the remaining dimensions can be replaced by semantic tokens to represent shared features. Based on this hypothesis, we introduce a Unified Semantic and ID Representation Learning framework, which incorporates two key components: \textit{unified ID and semantic tokenization} and \textit{unified cosine similarity and Euclidean distance}. First, in unified tokenization, we quantize item content embeddings into a semantic codebook to capture shared characteristics, while assigning each item a low-dimensional ID token to capture unique attributes. Second, in the unified similarity and distance metric, we observe that cosine similarity is effective at disentangling densely clustered embeddings, yet struggles with distinguishing unique items, while Euclidean distance excels at the latter. Consequently, we apply cosine similarity in the earlier layers to decouple dense embeddings and Euclidean distance in the final layer to distinguish unique items. Experimental results on three benchmark datasets demonstrate that our method outperforms existing baselines by 6\% to 17\%, while reducing token size by over 80\%. Ablation studies further validate our hypothesis, showing that many ID tokens are redundant and can be effectively replaced by semantic tokens to enhance generalization.

In summary, the key contributions of this work are as follows: \begin{itemize}[leftmargin=*] \item We present the first comprehensive investigation into the complementary relationship between semantic and ID tokens in recommendation systems. \item We propose a novel \textit{unified ID and semantic tokenization} framework that captures both unique and shared item characteristics, alongside a \textit{unified similarity and distance} approach that balances embedding decoupling and item distinction. \item Our method achieves significant performance improvements on three benchmark datasets, outperforming baselines by 6\% to 17\% while reducing token size by over 80\%, thereby enhancing the system's generalization capability. \end{itemize}



\section{Preliminary}
\paragraph{\textbf{Problem Definition}}
Suppose there are \( m \) items, and each item \( i \) is represented by an encoded sentence embedding \( \boldsymbol{x}_i \). Let \( i_{t} \) denote user \( u \)'s \( t \)-th interacted item. If user \( u \) has interacted with a sequence of items \( \mathcal{I}_{u} = (i_{1}, i_{2}, \ldots, i_{t}) \), with corresponding sentence embeddings \( \mathcal{X}_{u} = (\boldsymbol{x}_{i_{1}}, \boldsymbol{x}_{i_{2}}, \ldots, \boldsymbol{x}_{i_{t}}) \), the objective of sequential recommendation is to accurately predict the next item that user \( u \) will interact with, based on their previous interaction history. Formally, the problem can be defined as follows:

\noindent \textbf{Input}: A sequence of items \( \mathcal{I}_{u} = (i_{1}, i_{2}, \ldots, i_{t}) \) that user \( u \) has interacted with, along with their corresponding sentence embeddings \( \mathcal{X}_{u} = (\boldsymbol{x}_{i_{1}}, \boldsymbol{x}_{i_{2}}, \ldots, \boldsymbol{x}_{i_{t}}) \).

\noindent \textbf{Output}: The estimated probability \( \hat{y}_{u, t+1} \) of the next item that user \( u \) will interact with at time step \( t + 1 \).


\paragraph{\textbf{ID Tokenization}} 
Traditional recommender systems often rely on ID tokenization to capture the unique characteristics of each item. In this approach, an item embedding matrix 
$\left\{\boldsymbol{e}_{i}\right\}_{i=1}^m$ is constructed, where each item \( i \) is associated with an embedding vector 
$\boldsymbol{e}_i \in \mathbb{R}^{1 \times D}$, and \( D \) represents the ID embedding dimension. The total embedding size for ID tokenization is thus $m \times D$, where \( m \) is the number of items. For a user \( u \) with an interaction sequence of items $\mathcal{I}_{u} = (i_{1}, i_{2}, \ldots, i_{t})$, we can retrieve the corresponding ID embeddings $(\boldsymbol{e}_{i_{1}}, \boldsymbol{e}_{i_{2}}, \ldots, \boldsymbol{e}_{i_{t}})$ through simple lookup operations in the embedding matrix.
\paragraph{\textbf{Semantic Tokenization}} 
To capture the semantic information of items, recent works have leveraged techniques like RQ-VAE~\citep{rajput2024recommender} to quantize content embeddings. Specifically, semantic tokenization builds $L$ layers of codebook embeddings, where each layer contains a set of embedding vectors $\left\{\boldsymbol{e}^c_{k}\right\}_{k=1}^K$, with $\boldsymbol{e}^c_{k} \in \mathbb{R}^{1 \times D'}$. Here, $D'$ denotes the semantic embedding dimension, and the total embedding size for semantic tokenization is $L \times K \times D'$. Since $L \times K \ll m$, semantic tokenization can significantly reduce the embedding size by replacing ID-based embeddings with semantically informed ones. As detailed in Algorithm~\ref{alg:rq} of Appendix~\ref{sec:semantic_token}, the RQ-VAE model quantizes the input sentence embedding $\boldsymbol{x}_{i_{t}}$ and returns the corresponding semantic embedding $\boldsymbol{z}_{i_{t}}$ for each item in user \( u \)'s interaction history. It is important to note that the stop-gradient operation, denoted as $\operatorname{sg}$, is applied during the quantization process.



% \paragraph{\textbf{ID Tokenization}}
% Traditional recommender systems are often based on ID tokenization. Specifically, to learn the unique information of each item, Traditional recommender systems often build an 
% item embedding matrix $\left\{\boldsymbol{e}_{i}\right\}_{i=1}^m, 
% \boldsymbol{e}_i \in \mathbb{R}^{1 \times D}$. Here $D$ is the ID embedding dimension and the total ID embedding size here is $m \times D$.
% As the item number $m$ can be very large, we set the dimension $D$ relatively smaller than the dimension $D'$ of the traditional recommender systems which rely on ID tokenization. Given item sequence $\mathcal{I}_{u} = (i_{1}, i_{2}, \ldots, i_{t})$ for user $u$, we can lookup ID embeddings $(\boldsymbol{e}_{i_{1}}, \boldsymbol{e}_{i_{2}}, \ldots, \boldsymbol{e}_{i_{t}})$.




% \paragraph{\textbf{Semantic Tokenization}} Recently, to learn the semantic information, some works leverage RQ-VAE~\citep{rajput2024recommender} to quantize the content embedding. Specifically, such method builds $L$ layers of codebook embeddings, each layer with $\left\{\boldsymbol{e}^c_{k}\right\}_{k=1}^K, \boldsymbol{e}^c_{k} \in \mathbb{R}^{1 \times D'}$. Here $D'$ is the semantic embedding dimension, and semantic tokenization has $L \times K \times D'$ of total embedding size. As $L \times K \ll m$, semantic tokenization can reduce the embedding size by replacing ID embedding with semantic embedding. As shown in Algorithm~\ref{alg:rq}, RQ-VAE will quantize the input sentence embedding $\boldsymbol{x}_{i_{t}}$ and return the semantic embedding $\boldsymbol{z}_{i_{t}}$ of each item in a given user $u$'s historical sequence. Note that here $\operatorname{sg}$ is the stop gradient operation.




\section{Problem Statement \& Methodology}
\subsection{Estimating a confidence interval for the ATE}
Our aim is to leverage online controlled experiments to assess the ATE of an intervention on some outcome of interest $Y$.
Without loss of generality, we assume that this random variable indicates a logged user-level event (e.g. a user opens the app, clicks, converts, renews, churns, et cetera).
We denote the intervention by superscript, for treatment $Y^{\rm T}$ and control $Y^{\rm C}$.
Our estimand is then, with an expectation over experiment randomisation units (i.e. users):
\begin{equation}
    \mathop{\rm ATE}\limits_{{\rm C} \to {\rm T}}(Y) = \mathbb{E}[Y^{\rm T}-Y^{\rm  C}].
\end{equation}

A straightforward estimator for the ATE is given by the difference in sample means.
For a set of users belonging to a \emph{group} (i.e. control C, treatment T, or a general A/B-testing group A) and their observed outcomes $Y$, we have:
\begin{equation}
    \mu_{\rm A}(Y) = \frac{1}{|\mathcal{U}_{\rm A}|} \sum_{i \in \mathcal{U}_{\rm A}} Y_i,\, \, \, \, \enskip \text{and} \, \, \, \, \enskip \widehat{\mathop{\rm ATE}\limits_{{\rm C} \to {\rm T}}}(Y)  =  \mu_{\rm T}(Y) - \mu_{\rm C}(Y).
\end{equation}

To quantify the uncertainty in the estimate, we wish to construct a confidence interval.
The CLT tells us that the distribution of $\mathbb{E}[Y]$ converges to a normal distribution, and hence, so does the distribution for the ATE.
This implies that we can compute:
\begin{align}
    \sigma^{2}_{\rm A}(Y) &= \frac{1}{|\mathcal{U}_{\rm A}|}\sum_{i \in \mathcal{U}_{\rm A}}(Y_i - \mu_{\rm A}(Y))^{2}, \\
   \text{SE}\left(\widehat{\mathop{\rm ATE}\limits_{{\rm C} \to {\rm T}}}(Y)\right) &= \sqrt{\frac{\sigma^{2}_{\rm C}(Y)}{|\mathcal{U}_{\rm C}|} + \frac{\sigma^{2}_{\rm T}(\rm Y)}{|\mathcal{U}_{\rm T}|}}.
\end{align}
A $100\cdot(1-\alpha)\%$ confidence interval can then be obtained as:
\begin{equation}
    \widehat{\mathop{\rm ATE}\limits_{{\rm C} \to {\rm T}}}(Y) \pm  \Phi^{-1}\left(1-\frac{\alpha}{2}\right)\cdot\text{SE}\left(\widehat{\mathop{\rm ATE}\limits_{{\rm C} \to {\rm T}}}(Y)\right),
\end{equation}
where the inverse cumulative distribution function for the standard normal distribution $\Phi^{-1}$ gives the critical value for confidence level $\alpha$.
It should include the ground truth ATE in $100\cdot(1-\alpha)\%$ of cases.
A confidence interval around the ATE is a crucial component to consider when properly interpreting A/B-testing results.

In  a statistical hypothesis testing framework, when zero is not contained by this interval, the null hypothesis is rejected and the result is deemed significant at level $\alpha$.
Alternatively, we can construct a two-tailed $p$-value as:
\begin{equation}\label{eq:pvalue}
    p = 2 \left(1- \Phi\left( \left| \underbrace{\frac{\widehat{\mathop{\rm ATE}\limits_{{\rm C} \to {\rm T}}}(Y)}{\text{SE}\left(\widehat{\mathop{\rm ATE}\limits_{{\rm C} \to {\rm T}}}(Y)\right)}}_{z\text{-score}} \right| \right)\right),
\end{equation}
and reject the null hypothesis when $p < \alpha$.
The $p$-value can be described as the probability of observing results at least as extreme as what is observed, given that the null hypothesis holds true.

The meaning that is ascribed to both confidence intervals and $p$-values relies heavily on the assumption that the distribution of the estimand has approached normality. 
Whilst the CLT guarantees this property to hold asymptotically, finite sample scenarios require us to empirically validate that the above procedure is appropriate. 

\subsection{Empirically validating confidence intervals}
Naturally, directly validating whether the obtained CI includes the true ATE would require knowledge of the latter, which is prohibitive.
Alternatively, A/A-tests allow us to emulate experiments where we know the true ATE by design (i.e. $0$, as the null hypothesis holds).
This enables us to estimate the empirical coverage of the obtained CIs, through repeated resampling of A/A-groups.
When the distribution of the ATE has approached normality, the distribution of the $p$-values that we obtain over resampled A/A-tests should resemble a uniform distribution.
The Kolmogorov-Smirnov test provides a rigorous statistical framework to flag cases where it does not.
This allows us to, for a set of users $\mathcal{U}$ and outcomes $Y$, assess how amenable the data is to reliable estimation of CIs on ATE($Y$) using the above-mentioned standard methods.

As such, we repeatedly resample groups ${\rm A}_{i}$,${\rm A}_{i}^\prime$ for $n$ iterations, obtain $n$ confidence intervals for $\widehat{\mathop{\rm ATE}\limits_{{\rm A}_i \to {\rm A}_{i}^{\prime}}}(Y)$ and obtain a set of $p$-values $\{p_{1},\ldots,p_{n}\}$.
Given the empirical Cumulative Distribution Function (eCDF) of $p$-values $F_{\rm emp}(p)$, we wish to assess how it deviates from the uniform distribution with CDF $F_{\rm uni}(p)=p$ for $p\in[0,1]$.
The test statistic leveraged by Kolmogorov-Smirnov, known as the $D$-statistic, measures the $\infty$-norm over the observed differences between the two CDFs as:
\begin{equation}
    D = \sup_{p\in[0,1]} \left|F_{\rm emp}(p) - F_{\rm uni}(p)\right|.
\end{equation}
Under the null hypothesis that the distributions are equivalent, $D$ follows a Kolmogorov distribution.
As such, we can obtain a $p$-value that is used to reject the null hypothesis that the $p$-values obtained from the A/A-tests are uniformly distributed, or, that we have sufficient samples to reliably estimate treatment effects on $Y$.
Note that this is equivalent to testing whether the $z$-scores in Equation~\ref{eq:pvalue} follow a standard normal distribution.
Arguments against the statistical hypothesis testing framework apply here as well.
We suggest to pay special attention to cases where the $D$-statistic is high, or conversely, the Kolmogorov Smirnov $p$-value is low, rather than assigning binary (non-)significant labels to metrics.

In cases where an outcome $Y$ is flagged through this procedure, CIs and $p$-values obtained through the $t$- or $z$-test should be considered unreliable.
We can rely on alternative non-parametric methods to estimate CIs on ATE($Y$) in those situations, e.g. based on permutation sampling or bootstrapping.
We note that other commonly used methods like the Mann-Whitney U-test formulate a different null hypothesis, and cannot be used as a drop-in replacement without careful consideration~\cite{Fay2010}.
Furthermore, as the above-mentioned alternatives typically come with a considerable computational cost and specialised engineering solutions, they are significantly less desirable as a default approach.
A deeper exploration of their applicability falls outside of the scope for this work, but provides an interesting avenue for future research.

\section{Experiments}
In our experiments, we evaluate the proposed method on three real-world benchmark datasets, focusing on the following key research questions (RQs): \textbf{RQ1}: Does the proposed unified representation learning method outperform state-of-the-art sequential recommendation models in terms of prediction accuracy? \textbf{RQ2}: What is the impact of our unified semantic and ID tokenization method on recommendation performance? Additionally, is the integration of cosine similarity and Euclidean distance effective in improving the final recommendation performance?
\textbf{RQ3}: To what extent can we reduce the dimensionality of ID tokens without compromising performance? Specifically, how does the model’s performance vary with different ID token dimensions?
\textbf{RQ4}: What patterns do the semantic and ID tokens learn, and how do these tokens contribute to the overall representation of items?

Additionally, in Appendix~\ref{appendix:codebooksize}, we explore the effects of varying the codebook size on the patterns learned by the semantic tokens.

\subsection{Experimental Setup}

\paragraph{\textbf{Datasets}} We evaluate the recommendation performance on Amazon product review datasets~\citep{he2016ups}. The statistics of these three benchmark datasets after applying 5-core filtering are presented in Appendix~\ref{appendix:data}.

\paragraph{\textbf{Evaluation Metrics}} We follow the approach used in prior work~\citep{zhou2020s3}, using Hit Ratio (HIT@k), Normalized Discounted Cumulative Gain (NDCG@k), and Mean Reciprocal Rank (MRR) as evaluation metrics, where $k$ is the number of top ranked items. Consistent with previous studies~\citep{zhou2020s3, dcn}, given a user behavior sequence, we use the last item for testing, the second-to-last item for validation, and the rest for training. Given the large item set, ranking against all possible items is computationally expensive. Therefore, following a commonly used approach~\citep{sasrec, man}, we evaluate the model by sampling 99 negative items along with the ground-truth item. All metrics are calculated based on the ranking of sampled and ground-truth items, and we present the mean scores across users.

\paragraph{\textbf{Baselines}}
To evaluate the pure impact of semantic tokenization, we compare our proposed method against several competitive recommendation baselines, including FM~\citep{fm}, GRU4Rec~\citep{gru4rec}, Caser~\citep{caser}, SASRec~\citep{sasrec}, BERT4Rec~\citep{bert4rec}, and HGN~\citep{hgn}. It is important to note that we do not compare our method with existing work~\citep{rajput2024recommender} that utilizes a different model architecture with a deeper network when incorporating RQ-VAE. The primary focus here is to examine the effects of semantic tokenization within the context of the same sequential recommendation model to ensure a fair and consistent comparison. Besides, we directly use the results of all baseline from prior work~\citep{zhou2020s3} and implement our method based on SASRec under its framework for a fair comparison. Besides, we show the detailed description of these baselines in Appendix~\ref{appendix:baseline}.

\paragraph{\textbf{Hyper-parameter Settings}} We directly use the results of all baseline from prior work~\citep{zhou2020s3} and implement our method based on its framework for a fair comparison. Besides, we set some new hyper-parameters of RQ-VAE following prior work~\citep{rajput2024recommender} with $L=3$ layers of codebook. We search the codebook size $K$ from 64 to 1024 and select 256 for both Beauty and Toys dataset, while 128 for Sports dataset. Besides, we set the dimension of codebook $D' = 64$ to align with the ID token only method. All other parameters like recommendation model layer and hidden size are set strictly the same as baselines. Additionally, we put more implementation details in Appendix~\ref{appendix:implementation}.

\subsection{Overall Performance}

\begin{table*}[t!]
\centering
\caption{Our method improves baseline significantly by 6\% to around 18\% on three benchmark datasets.}
\label{tab:overall}
% \tabcolsep=1mm
\begin{tabular}{cc|cccccc|c|c}
\toprule
Datasets & Metric  & FM     & GRU4Rec & Caser  & SASRec       & BERT4Rec     & HGN    & Ours            & Improv. \\ \midrule
\multirow{5}{*}{Beauty} & HIT@5 & 0.1461 & 0.3125 & 0.3032 & {\ul 0.3741} & 0.3640 & 0.3544 & \textbf{0.4201} & 12.30\% \\  
         & NDCG@5  & 0.0934 & 0.2268  & 0.2219 & {\ul 0.2848} & 0.2622       & 0.2656 & \textbf{0.3079} & 8.11\%  \\  
         & HIT@10   & 0.2311 & 0.4106  & 0.3942 & 0.4696       & {\ul 0.4739} & 0.4503 & \textbf{0.5318} & 12.22\% \\  
         & NDCG@10 & 0.1207 & 0.2584  & 0.2512 & {\ul 0.3156} & 0.2975       & 0.2965 & \textbf{0.3440} & 9.00\%  \\  
         & MRR     & 0.1096 & 0.2308  & 0.2263 & {\ul 0.2852} & 0.2614       & 0.2669 & \textbf{0.3025} & 6.07\%  \\ \midrule
\multirow{5}{*}{Sports} & HIT@5    & 0.1603 & 0.3055  & 0.2866 & {\ul 0.3466} & 0.3375       & 0.3349 & \textbf{0.3849} & 11.05\% \\
                        & NDCG@5  & 0.1048 & 0.2126  & 0.2020 & {\ul 0.2497} & 0.2341       & 0.2420 & \textbf{0.2717} & 8.81\%  \\
                        & HIT@10   & 0.2491 & 0.4299  & 0.4014 & 0.4622       & {\ul 0.4722} & 0.4551 & \textbf{0.5247} & 11.12\% \\
                        & NDCG@10 & 0.1334 & 0.2527  & 0.2390 & {\ul 0.2869} & 0.2775       & 0.2806 & \textbf{0.3168} & 10.42\% \\
                        & MRR     & 0.1202 & 0.2191  & 0.2100 & {\ul 0.2520} & 0.2378       & 0.2469 & \textbf{0.2722} & 8.02\%  \\ \midrule
\multirow{5}{*}{Toys}   & HIT@5 & 0.0978 & 0.2795 & 0.2614 & {\ul 0.3682} & 0.3344 & 0.3276 & \textbf{0.4340} & 17.87\% \\  
         & NDCG@5  & 0.0614 & 0.1919  & 0.1885 & {\ul 0.2820} & 0.2327       & 0.2423 & \textbf{0.3141} & 11.38\% \\  
         & HIT@10   & 0.1715 & 0.3896  & 0.3540 & {\ul 0.4663} & 0.4493       & 0.4211 & \textbf{0.5456} & 17.01\% \\  
         & NDCG@10 & 0.0850 & 0.2274  & 0.2183 & {\ul 0.3136} & 0.2698       & 0.2724 & \textbf{0.3501} & 11.64\% \\  
         & MRR     & 0.0819 & 0.1973  & 0.1967 & {\ul 0.2842} & 0.2338       & 0.2454 & \textbf{0.3064} & 7.81\%  \\ \bottomrule
\end{tabular}
\end{table*}
To compare the performance of our method with existing sequential recommenders, as shown in Table~\ref{tab:overall}, we evaluate them in three benchmark datasets under five metrics. From the table, we can have the following observation: \textbf{Our method achieves significant improvement.} The improvement of our method towards baselines ranges from 6.07\% to 17.87\%, which is very significant in sequential recommendation task~\citep{sasrec, zhou2020s3}. Besides, our method improves more on HIT metric than NDCG metric and MRR metric. This may be because semantic embedding is naturally less insensitive at ranking position due to duplicate tokenization, though we have added unique ID embedding.


\subsection{Ablation Study}
% Please add the following required packages to your document preamble:
% \usepackage{multirow}
% \usepackage[normalem]{ulem}
% \useunder{\uline}{\ul}{}
\begin{table*}[!htb]
% \tabcolsep=0.5mm
\small
\centering
\caption{Unified tokenization outperforms ID-only and semantic-only tokenizations with significant reduction of token size. Besides, the semantic tokenization outperforms ID tokenization in position-insensitive metric.}
\label{tab:token}
\begin{tabular}{cc|c|c|c|c|c|c|c}
\toprule
\multirow{2}{*}{Dataset} &
  \multirow{2}{*}{Method} &
  \multicolumn{3}{c|}{Metric} &
  \multicolumn{3}{c|}{Token Size} &
  \multirow{2}{*}{\begin{tabular}[c]{@{}c@{}}Token\\ Reduction\end{tabular}} \\ 
 &
   &
  HIT@10 &
  NDCG@10 &
  MRR &
  ID &
  Semantic &
  Total &
   \\ \midrule
\multirow{3}{*}{Beauty} &
  ID &
  0.4654 &
  {\ul 0.3121} &
  {\ul 0.282} &
  12,101 $\times$ 64 &
  0 &
  774,464 &
  \textbackslash{} \\  
 &
  Semantic &
  {\ul 0.4956} &
  0.2914 &
  0.2476 &
  0 &
  3 $\times$ 256 $\times$ 64 &
  49,152 &
  93.65\% \\  
 &
  Unified &
  \textbf{0.5318} &
  \textbf{0.344} &
  \textbf{0.3025} &
  12,101 $\times$ 8 &
  3 $\times$ 256 $\times$ 64 &
  145,960 &
  81.15\% \\ \midrule
\multirow{3}{*}{Sports} &
  ID &
  0.4582 &
  {\ul 0.2826} &
  {\ul 0.2482} &
  18,357 $\times$ 64 &
  0 &
  1,174,848 &
  \textbackslash{} \\  
 &
  Semantic &
  {\ul 0.4704} &
  0.2554 &
  0.2131 &
  0 &
  3 $\times$ 128 $\times$ 64 &
  24,576 &
  97.91\% \\  
 &
  Unified &
  \textbf{0.5247} &
  \textbf{0.3168} &
  \textbf{0.2722} &
  18,357 $\times$ 8 &
  3 $\times$ 128 $\times$ 64 &
  171,432 &
  85.41\% \\ \midrule
\multirow{3}{*}{Toys} &
  ID &
  0.4603 &
  {\ul 0.3092} &
  {\ul 0.2804} &
  11,924 $\times$ 64 &
  0 &
  763,136 &
  \textbackslash{} \\  
 &
  Semantic &
  {\ul 0.4644} &
  0.2741 &
  0.236 &
  0 &
  3 $\times$ 256 $\times$ 64 &
  49,152 &
  93.56\% \\  
 &
  Unified &
  \textbf{0.5456} &
  \textbf{0.3501} &
  \textbf{0.3064} &
  11,924 $\times$ 8 &
  3 $\times$ 256 $\times$ 64 &
  144,544 &
  81.06\% \\ \bottomrule
\end{tabular}
\end{table*}
To further study the performance of different tokenization methods, we compare our method with the ID tokenization only method and semantic tokenization only method as Table~\ref{tab:token}. From the table, we can have the following observations: (1) \textbf{Unified tokenization performs best with significant reduction of token.} In all these three benchmark datasets, our proposed method is significantly superior to solely ID tokenization and semantic tokenization methods. More importantly, compared with the traditional ID tokenization method, our method reduces by at least 80\% and even 85\% of tokens on Sports dataset. Here we reduce the tokens by replacing 56 dimensions of ID tokens with a small amount of semantic tokens, which supports our previous analysis that most ID tokens are redundant. (2) \textbf{Semantic tokenization outperforms ID tokenization in position-insensitive metric with significant reduction of token.} In three datasets, it is obvious that the semantic tokenization only method even outperforms ID tokenization only method on HIT metric with less than 10\% of tokens. This result also supports our previous analysis that semantic tokenization is effective at generalization and capturing high-level semantic information. However, semantic tokenization only method often performs poor at NDCG and MRR metrics which are sensitive to position. This is because the position of duplicate tokenized items from semantic tokenization only method are hard to distinguish in ranking.
           

% Please add the following required packages to your document preamble:
% \usepackage{multirow}
\begin{table*}[!htb]
% \tabcolsep=0.5mm
\centering
\caption{The integration of Euclidean distance into cosine similarity can improve the recommendation performance.}
\label{tab:abl_distance}
\begin{tabular}{c|ccc|ccc|ccc}
\toprule
\multirow{2}{*}{Method} & \multicolumn{3}{c|}{Beauty} & \multicolumn{3}{c|}{Sports} & \multicolumn{3}{c}{Toys}  \\  
                        & HIT@10  & NDCG@10 & MRR    & HIT@10  & NDCG@10 & MRR    & HIT@10 & NDCG@10 & MRR    \\ \midrule
Cosine       & 0.5212  & 0.3334  & 0.2921 & 0.5129  & 0.3081  & 0.2649 & 0.5252 & 0.3309  & 0.2879 \\ 
Unified & \textbf{0.5318} & \textbf{0.3440} & \textbf{0.3025} & \textbf{0.5247} & \textbf{0.3168} & \textbf{0.2722} & \textbf{0.5456} & \textbf{0.3501} & \textbf{0.3064} \\ \bottomrule
\end{tabular}
\end{table*}
Besides, we also compare our method with cosine similarity only method when searching the codebook of RQ-VAE, as shown in Table~\ref{tab:abl_distance}. From the table, we can observe that: \textbf{Our unified method outperforms cosine similarity.} The unified method which integrates cosine similarity with Euclidean distance proposed in Section~\ref{sec:unified_distance} outperforms the solely cosine similarity method on three benchmark datasets. This means our unified cosine similarity and Euclidean distance not only can improve the percentage of activated codebook and coverage of unique items, but also can really improve the final recommendation performance.


\subsection{Hyper-parameter Study}
\begin{figure*}[htb!]
		\centering
		\begin{tabular}{ccc}
		    	\includegraphics[width=0.24\linewidth]{fig/hit.pdf} &  \includegraphics[width=0.24\linewidth]{fig/NDCG.pdf} &
       \includegraphics[width=0.24\linewidth]{fig/mrr.pdf} 
		\end{tabular}
	\caption{The performance improvement shrinks when scaling up dimension of ID token, which means a small proportion of ID tokens is sufficient for capturing the item's unique characteristic.}	\label{fig:hyper_id}
\end{figure*} 
To further verify that we only need a small proportion of ID tokens, we further vary the ID dimension from $\{0, 4, 8, 16\}$ and study the performance under three key metrics as Figure~\ref{fig:hyper_id}. From the figure, we discvoer that: \textbf{The performance improvement shrinks when scaling up dimension of ID token.} It is obvious that the performance improvement becomes less and less with the growing of ID token dimension, and the performance even drops when dimension is greater than 8. This means a small proportion of ID tokens is sufficient for learning the unique information, and others are indeed redundant and can be saved.

\subsection{Token Visualization}
\begin{figure*}[htb!]
		\centering
		\begin{tabular}{cccc}
\includegraphics[width=0.2\linewidth]{fig/first_layerbeauty_4.png} &
       \includegraphics[width=0.2\linewidth]{fig/second_layerbeauty_4.png}  & \includegraphics[width=0.2\linewidth]{fig/third_layerbeauty_3.png}  & \includegraphics[width=0.2\linewidth]{fig/uniquebeauty_3.png}
		     \\ First Codebook & Second Codebook & Third Codebook & Unique Tokens
		\end{tabular}
	\caption{The patterns of codebooks are various across different layers and unique tokens are uniform for different items on Beauty dataset.}	\label{fig:vis_beauty}
\end{figure*} 




% \begin{figure}[htb!]
% 		\centering
% 		\begin{tabular}{c|c|c}
% \includegraphics[width=0.33\linewidth]{fig/uniquebeauty_3.png} &
%        \includegraphics[width=0.33\linewidth]{fig/uniqueSports_and_Outdoors.png}  & \includegraphics[width=0.33\linewidth]{fig/uniqueToys_and_Games.png}  
% 		     \\ (a) Beauty & (b) Sports & (c) Toys 
% 		\end{tabular}
% 	\caption{Visualization of ID tokens on three datasets.}	\label{fig:vis_id}
% \end{figure} 
To study the learned semantic and ID tokens, we further visualize these tokens on Beauty dataset using t-SNE, as shown in Figure~\ref{fig:vis_beauty}. Besides, we also visualize the tokens on Sports and Toys datasets in \ref{fig:vis_sport} and \ref{fig:vis_toys} of Appendix~\ref{sec:visual_token}. Here we label each semantic token with a unique color and thus these are totally $K$ types of color. In ID tokens, we also label them with $K$ types of color to show the distribution when they are assigned with one of the codebooks. Based on the visualized results, we can discover that: (1) \textbf{Semantic tokens vary across different layers.} It is obvious that the semantic tokens vary across different layer on all datasets, which means different layers of semantic codebooks can capture various shared patterns. With the combination of these shared patterns, we can better represent each item's semantic information. (2) \textbf{ID tokens distribute uniformly.} The unique ID tokens are uniform on all datasets. This means the ID token successfully capture the unique characteristic of each item and thus they will not accumulate together.


\section{Related Work}


\paragraph{\textbf{Sequential Recommendation}} The use of deep learning in sequential recommendation has evolved into a well-established area of research. GRU4REC~\citep{gru4rec} pioneered the application of Gated Recurrent Unit (GRU)-based Recurrent Neural Networks (RNNs) for sequential recommender. Then SASRec~\citep{sasrec} utilized self-attention mechanisms~\cite{vaswani2017attention} of Transformer to capture the context relation of whole sequence. Building on the success of masked self-supervised learning in natural language processing, subsequent works such as BERT4Rec~\citep{bert4rec} leveraged self-supervised learning to randomly mask the historical items and improved the robustness. Apart from the popular self-attention and Transformer architecture, researchers have also explored the use of Convolution Neural Networks (CNNs)~\citep{CNN} in sequential recommender~\citep{caser}. In this paper, we focus on improving the sequential recommendation using semantic tokens.

\paragraph{\textbf{Quantized Representation Learning}} Vector-quantized learning has grabbed researchers' attention with its discrete latents to reduce the model variance. In recommender systems, VQ-Rec~\citep{vqrec} proposes a transferable method to quantize item content embedding as item representation. When VQ-Rec utilizes product quantization~\citep{jegou2010product} for the generation of semantic codes, TIGER~\citep{rajput2024recommender} further leverages RQ-VAE to produce hierarchical semantic IDs as item representation. In parallel to TIGER, another work~\citep{singh2023better} demonstrated that semantic IDs can improve the generalization of recommendation ranking compared with traditional item IDs. Different from existing works aiming to replace item IDs with semantic IDs, we further consider the complementary strengths of them.

 





\section{Conclusion}
In conclusion, this work provides a comprehensive exploration of the complementary relationship between ID tokens and semantic tokens in recommendation systems, addressing the limitations of using either method in isolation. We introduced a novel framework that unifies ID and semantic tokenization, effectively capturing both unique and shared item characteristics while significantly reducing token redundancy. By leveraging a combination of cosine similarity and Euclidean distance, our approach successfully decouples accumulated embeddings and distinguishes unique items. Experimental results on three benchmark datasets demonstrate that our proposed method consistently outperforms the baselines, achieving notable improvements in performance (6\% to 17\%) while reducing token size by over 80\%. The results also validated our hypothesis that most ID tokens are redundant and can be substituted with semantic tokens to enhance generalization. Our work sets the foundation for a more efficient and effective representation strategy in recommendation systems, combining the strengths of both ID and semantic tokens for improved user experience.



\bibliography{iclr2025_conference}
\bibliographystyle{ACM-Reference-Format}

\newpage
\appendix
\section{Appendix}
\subsection{Algorithm for Semantic Tokenization}\label{sec:semantic_token}
As shown in Algorithm~\ref{alg:rq}, we present RQ-VAE for semantic tokenization.
\begin{figure}[!htb]
\vspace{-1em}
\centering
\small
\begin{algorithm}[H]
\caption{RQ-VAE for Semantic Tokenization}\label{alg:rq}
\textbf{Input:} Sentence embedding $\mathcal{X}_{u} = (\boldsymbol{x}_{i_{1}}, \boldsymbol{x}_{i_{2}}, \ldots, \boldsymbol{x}_{i_{T}})$ of user $u$\\
\textbf{Output:} Semantic representation $\hat{\mathcal{Z}}_{u} = (\hat{\boldsymbol{z}}_{i_{1}}, \hat{\boldsymbol{z}}_{i_{2}}, \ldots, \hat{\boldsymbol{z}}_{i_{T}})$ of user $u$\\
\begin{algorithmic}[1]
\FOR{$t = 1 \rightarrow T$ in parallel} 
 \STATE $\boldsymbol{z}_{i_t} = \textbf{Encoder} ({\boldsymbol{x}}_{i_t})$  \# encode the text embedding
\STATE $\boldsymbol{r}_1 = \boldsymbol{z}_{i_t}$, $\hat{\boldsymbol{{z}}}_{i_t} = 0$
    \FOR{$l = 1 \rightarrow L$}
            \STATE $\left\{\boldsymbol{e}^c_{k}\right\}_{k=1}^K, \boldsymbol{e}^c_{k} \in \mathbb{R}^{1 \times D'}$ \# codebook embedding of each layer 
        \STATE $k=\arg \min_k\left\|\boldsymbol{r}_{l}-\boldsymbol{e}^c_{k}\right\|$ \# search the index of closest codebook
        \STATE $\boldsymbol{r}_{l + 1} = \boldsymbol{r}_l-\boldsymbol{e}^c_{k}$ 
 \STATE $\hat{\boldsymbol{{z}}}_{i_t} += \boldsymbol{e}^c_{k}$ \# accumulate the quantized embedding
 \STATE $\mathcal{L}_{\text {rqvae }} += \left\|\operatorname{sg}\left[\boldsymbol{r}_l\right]-\boldsymbol{e}^c_{k}\right\|^2+\beta\left\|\boldsymbol{r}_l-\operatorname{sg}\left[\boldsymbol{e}^c_{k}\right]\right\|^2$ \# $\operatorname{sg}$ means stop gradient
    \ENDFOR
 \STATE $\hat{\boldsymbol{x}}_{i_t} = \textbf{Decoder}(\hat{\boldsymbol{z}}_{i_t})$  \# decode the quantized semantic embedding
  \STATE $\mathcal{L}_{\text {recon}} += \left\|\boldsymbol{x}_{i_t} - \hat{\boldsymbol{x}}_{i_t}\right\|^2$ \# reconstruction loss
    \ENDFOR
    \STATE \textbf{return} $\hat{\mathcal{Z}}_{u}$
\end{algorithmic}
\end{algorithm}
\vspace{-1em}
\end{figure}
\subsection{Implementation Details}\label{appendix:implementation}
Following TIGER~\citep{rajput2024recommender}, to obtain the semantic tokens, we utilize the pre-trained Sentence-T5~\citep{ni2021sentence}. Specifically, we construct item's sentence description using its content features, including title, brand, category and price. This constructed sentence is then fed into Sentence-T5, which outputs a 768-dimensional text embedding for each item as the input in our task. Besides, the RQ-VAE model includes a DNN encoder, a residual quantizer, and a DNN decoder. The DNN encoder takes the input text embedding and transforms the dimension to be aligned with codebook embedding. This encoder is activated by ReLU with layer sizes 512, 256, and 128, which ultimately produces a 64-dimensional latent representation. With the 64-dimensional latent representation from encoder, the residual quantizer then performs three levels of residual quantization. At each level, a codebook with size $K$ is used, where each token within the codebook has a dimension of 64. The output semantic token quantized by residual quantizer is then fed into the DNN decoder, which decodes it back to the original text embedding space. Note different from TIGER, we set the dimension of semantic token as 64 for alignment with ID token in our sequential recommendation setting. 

As for the implementation of sequential recommendation, we directly use the framework of $\text{S}^3\text{-Rec}$~\citep{zhou2020s3}. But as we train the model in an end-to-end manner, we just use the fine-tuning setting and do not use the pre-training setting of their framework. In our setting, we employ the Adam optimizer~\citep{kingma2014adam} with a learning rate of 0.001 and the batch size is set as 256.

\subsection{Baselines}\label{appendix:baseline}
In this section, we provide a brief overview of the baseline models employed for comparison:
\begin{itemize}[leftmargin=*]
\item \textbf{FM}~\citep{fm}: The Factorization Machine (FM) model characterizes pairwise interactions among variables through a factorized representation.
    \item \textbf{GRU4Rec}~\citep{gru4rec}: This model represents the pioneering application of recurrent neural networks (RNNs) for sequential recommendation, specifically utilizing a customized Gated Recurrent Unit (GRU).
    \item \textbf{Caser}~\citep{caser}: Caser introduces a convolution neural network (CNN) architecture designed to capture high-order Markov Chains. It achieves this through the implementation of both horizontal and vertical convolution operations tailored for sequential recommendation.
\item \textbf{HGN}~\citep{hgn}: The Hierarchical Gating Network (HGN) effectively models long-short-term user preference through an innovative gating mechanism.
\item \textbf{SASRec}~\citep{sasrec}: Self-Attentive Sequential Recommendation (SASRec) employs a causal masked self attention to model user’s historical behavior sequence.
\item \textbf{BERT4Rec}~\citep{bert4rec}: This model applies the bi-directional Transformer BERT for enhanced sequential recommender.
\end{itemize}

\subsection{Data Description}\label{appendix:data}
\begin{table}[htb!]
% \vspace{-1.6cm}
\centering
\caption{Data statistics for benchmark datasets after 5-core filtering. Here Sports and Toys are the `Sports and Outdoors' and `Toys and Games', respectively, from Amazon review datasets.}
\label{tab:data}
\begin{tabular}{ccccc}
\toprule
Dataset & \# Users &  \# Items & {Average Len.}\\
\midrule
Beauty & 22,363 & 12,101 & 8.87 \\
Sports & 35,598 & 18,357 & 8.32\\
Toys & 19,412 & 11,924 & 8.63 \\
\bottomrule
\end{tabular}
\end{table}
We utilize three real-world benchmark datasets derived from the Amazon Product Reviews dataset~\citep{he2016ups}, which includes user reviews and item metadata spanning from May 1996 to July 2014. In our task, we focus on three specific categories within this dataset: "Beauty," "Sports and Outdoors," and "Toys and Games." Table~\ref{tab:data} presents a summary of the statistics associated with these datasets, where "Average Len." represents the average length of all users' item sequences. To construct item sequences, we organize users' review histories chronologically by timestamp, ensuring that only users with a minimum of five reviews are retained in our analysis.


\subsection{Codebook Size Study}\label{appendix:codebooksize}

\begin{table*}[!htb]
\centering
\caption{Increasing codebook size does not improve the performance too much on Sports dataset.}
\label{tab:codebook_size}
\begin{tabular}{cccccc}
\hline
Codebook   Size & HR@5            & NDCG@5          & HR@10           & NDCG@10         & MRR             \\ \hline
64              & 0.3792          & 0.2675          & 0.5138          & 0.3109          & 0.2675          \\ \hline
128             & \textbf{0.3849} & \textbf{0.2717} & \textbf{0.5247} & \textbf{0.3168} & \textbf{0.2722} \\ \hline
256             & 0.3786          & 0.2672          & 0.5184          & 0.3123          & 0.2688          \\ \hline
521             & 0.3842          & 0.2719          & 0.5218          & 0.3163          & 0.2720          \\ \hline
1024            & 0.3809          & 0.2691          & 0.5202          & 0.3140          & 0.2696          \\ \hline
\end{tabular}
\end{table*}

% \begin{table}[!htb]
% \centering
% \caption{Increasing codebook size does not improve the performance too much on Sports dataset.}
% \label{tab:codebook_size}
% \begin{tabular}{cccccc}
% \hline
% Codebook   Size & HR@5            & NDCG@5          & HR@10           & NDCG@10         & MRR            \\ \hline
% 256             & 0.3786          & 0.2672          & 0.5184          & 0.3123          & 0.2688         \\ \hline
% 512             & \textbf{0.3842} & \textbf{0.2719} & \textbf{0.5218} & \textbf{0.3163} & \textbf{0.272} \\ \hline
% 1024            & 0.3809          & 0.2691          & 0.5202          & 0.314           & 0.2696         \\ \hline
% \end{tabular}
% \end{table}
As the first and third codebook in Amazon Sports dataset degenerate in Figure~\ref{fig:vis_sport}, we want to study whether the size of codebook $K$ has significant impact on this degeneration problem. Thus we vary the codebook size $K$ from 64 to 1024 as Table~\ref{tab:codebook_size}, and have the following discovery.
\begin{itemize}[leftmargin=*]
\item \textbf{Increasing codebook size does not improve the performance too much.} The performance reaches peak when codebook size is 128, but the performance fluctuates when codebook size grows to 256 and over.
\end{itemize}


\begin{figure*}[htb!]
		\centering
		\begin{tabular}{cccc}
\includegraphics[width=0.22\linewidth]{fig/first_layer64Sports_and_Outdoors.png} &
       \includegraphics[width=0.22\linewidth]{fig/second_layer64Sports_and_Outdoors.png}  & \includegraphics[width=0.22\linewidth]{fig/third_layer64Sports_and_Outdoors.png}  &
       \includegraphics[width=0.22\linewidth]{fig/unique64Sports_and_Outdoors.png}
	     \\ First Codebook & Second Codebook & Third Codebook & Unique Tokens
		\end{tabular}
	\caption{The patterns of codebooks are various across different layers but kind of sparse on Sports dataset with codebook size 64.}	\label{fig:vis_sports_64}
\end{figure*} 



\begin{figure*}[htb!]
		\centering
		\begin{tabular}{cccc}
\includegraphics[width=0.22\linewidth]{fig/first_layer128Sports_and_Outdoors.png} &
       \includegraphics[width=0.22\linewidth]{fig/second_layer128Sports_and_Outdoors.png}  & \includegraphics[width=0.22\linewidth]{fig/third_layer128Sports_and_Outdoors.png}  &
       \includegraphics[width=0.22\linewidth]{fig/unique128Sports_and_Outdoors.png}
	     \\ First Codebook & Second Codebook & Third Codebook & Unique Tokens
		\end{tabular}
	\caption{The patterns of codebooks are various across different layers on Sports dataset with codebook size 128.}	\label{fig:vis_sports_128}
\end{figure*} 

\begin{figure*}[htb!]
		\centering
		\begin{tabular}{cccc}
\includegraphics[width=0.22\linewidth]{fig/first_layerSports_and_Outdoors.png} &
       \includegraphics[width=0.22\linewidth]{fig/second_layerSports_and_Outdoors.png}  & \includegraphics[width=0.22\linewidth]{fig/third_layerSports_and_Outdoors.png}  &
       \includegraphics[width=0.22\linewidth]{fig/uniqueSports_and_Outdoors.png}
	     \\ First Codebook & Second Codebook & Third Codebook & Unique Tokens
      \end{tabular}
	\caption{The first and third codebooks start to degenerate on Sports dataset with codebook size 256.}	\label{fig:vis_sport_256}
\end{figure*} 


\begin{figure*}[htb!]
		\centering
		\begin{tabular}{cccc}
\includegraphics[width=0.22\linewidth]{fig/first_layer512Sports_and_Outdoors.png} &
       \includegraphics[width=0.22\linewidth]{fig/second_layer512Sports_and_Outdoors.png}  & \includegraphics[width=0.22\linewidth]{fig/third_layer512Sports_and_Outdoors.png}  &
       \includegraphics[width=0.22\linewidth]{fig/unique512Sports_and_Outdoors.png}
	     \\ First Codebook & Second Codebook & Third Codebook & Unique Tokens
		\end{tabular}
	\caption{The first and third codebooks still degenerate on Sports dataset with codebook size 512. And the second codebook also begin to degenerate.}	\label{fig:vis_sports_512}
\end{figure*} 

\begin{figure*}[htb!]
		\centering
		\begin{tabular}{cccc}
\includegraphics[width=0.22\linewidth]{fig/first_layer1024Sports_and_Outdoors.png} &
       \includegraphics[width=0.22\linewidth]{fig/second_layer1024Sports_and_Outdoors.png}  & \includegraphics[width=0.22\linewidth]{fig/third_layer1024Sports_and_Outdoors.png}  &
       \includegraphics[width=0.22\linewidth]{fig/unique1024Sports_and_Outdoors.png}
	     \\ First Codebook & Second Codebook & Third Codebook & Unique Tokens
		\end{tabular}
	\caption{Almost all codebooks degenerate on Sports dataset with codebook size 1024. In particular, the first and second codebooks degenerate extremely.}	\label{fig:vis_sports_1024}
\end{figure*} 

Besides, we also visualize the token distribution when codebook sizes are 64, 256, 512 and 1024 as Figure~\ref{fig:vis_sports_64} to \ref{fig:vis_sports_1024}. From the figure we can discover that:
\begin{itemize}[leftmargin=*]
\item \textbf{The codebooks begin to degenerate and be redundant when codebook size is greater than 256.} The first layer and second layer of codebooks begin to degenerate when codebook size is 256. With the increase of codebook size, the degeneration problem becomes more serious.
\item \textbf{The unique tokens are not influenced by codebook size too much.} With the growth of codebook size, the distribution of unqiue tokens almost keep unchange.

\end{itemize}

\begin{figure*}[htb!]
		\centering
		\begin{tabular}{cccc}
\includegraphics[width=0.22\linewidth]{fig/first_layer128Sports_and_Outdoors.png} &
       \includegraphics[width=0.22\linewidth]{fig/second_layer128Sports_and_Outdoors.png}  & \includegraphics[width=0.22\linewidth]{fig/third_layer128Sports_and_Outdoors.png}  &
       \includegraphics[width=0.22\linewidth]{fig/unique128Sports_and_Outdoors.png}
	     \\ First Codebook & Second Codebook & Third Codebook & Unique Tokens
      \end{tabular}
	\caption{The patterns of codebooks are various across different layers and unique tokens are uniform for different items on Sports dataset.}	\label{fig:vis_sport}
\end{figure*} 

\begin{figure*}[htb!]
		\centering
		\begin{tabular}{cccc}
\includegraphics[width=0.22\linewidth]{fig/first_layerToys_and_Games.png} &
       \includegraphics[width=0.22\linewidth]{fig/second_layerToys_and_Games.png}  & \includegraphics[width=0.22\linewidth]{fig/third_layerToys_and_Games.png}  &
       \includegraphics[width=0.22\linewidth]{fig/uniqueToys_and_Games.png}
	     \\ First Codebook & Second Codebook & Third Codebook & Unique Tokens
		\end{tabular}
	\caption{The patterns of codebooks are various across different layers and unique tokens are uniform for different items on Toys dataset.}	\label{fig:vis_toys}
\end{figure*} 

\subsection{Token Visualization on More Datasets}\label{sec:visual_token}
As shown in Figure~\ref{fig:vis_sport} and \ref{fig:vis_toys}, we visualize the patterns of codebooks on Sport and Toys datasets.


\end{document}
