\documentclass[sigconf]{acmart}
% \usepackage{iclr2025_conference,times}

% Optional math commands from https://github.com/goodfeli/dlbook_notation.
% %%%%% NEW MATH DEFINITIONS %%%%%

% \usepackage{amsmath,amsfonts,bm}
\usepackage{amsmath,amsfonts}

\usepackage{pifont}


\newcommand{\R}{\mathbb{R}}


\def\va{{\mathbf{a}}}
\def\vg{{\mathbf{g}}}

% Sets
\def\sR{\mathbb{R}}
\def\sC{\mathbb{C}}
\def\sZ{\mathbb{Z}}
\def\sN{\mathbb{N}}
\def\sQ{\mathbb{Q}}

\def\sS{\mathcal{S}}



% Vectors
\def\vzero{{\mathbf{0}}}
\def\vone{{\mathbf{1}}}
\def\vmu{{\mathbf{\mu}}}
\def\vtheta{{\mathbf{\theta}}}
\def\va{{\mathbf{a}}}
\def\vb{{\mathbf{b}}}
\def\vc{{\mathbf{c}}}
\def\vd{{\mathbf{d}}}
\def\ve{{\mathbf{e}}}
\def\vf{{\mathbf{f}}}
\def\vg{{\mathbf{g}}}
\def\vh{{\mathbf{h}}}
\def\vi{{\mathbf{i}}}
\def\vj{{\mathbf{j}}}
\def\vk{{\mathbf{k}}}
\def\vl{{\mathbf{l}}}
\def\vm{{\mathbf{m}}}
\def\vn{{\mathbf{n}}}
\def\vo{{\mathbf{o}}}
\def\vp{{\mathbf{p}}}
\def\vq{{\mathbf{q}}}
\def\vr{{\mathbf{r}}}
\def\vs{{\mathbf{s}}}
\def\vt{{\mathbf{t}}}
\def\vu{{\mathbf{u}}}
\def\vv{{\mathbf{v}}}
\def\vw{{\mathbf{w}}}
\def\vx{{\mathbf{x}}}
\def\vy{{\mathbf{y}}}
\def\vz{{\mathbf{z}}}
\def\vzeta{{\mathbf{\zeta}}}

% Matrix
\def\mA{{\mathbf{A}}}
\def\mB{{\mathbf{B}}}
\def\mC{{\mathbf{C}}}
\def\mD{{\mathbf{D}}}
\def\mE{{\mathbf{E}}}
\def\mF{{\mathbf{F}}}
\def\mG{{\mathbf{G}}}
\def\mH{{\mathbf{H}}}
\def\mI{{\mathbf{I}}}
\def\mJ{{\mathbf{J}}}
\def\mK{{\mathbf{K}}}
\def\mL{{\mathbf{L}}}
\def\mM{{\mathbf{M}}}
\def\mN{{\mathbf{N}}}
\def\mO{{\mathbf{O}}}
\def\mP{{\mathbf{P}}}
\def\mQ{{\mathbf{Q}}}
\def\mR{{\mathbf{R}}}
\def\mS{{\mathbf{S}}}
\def\mT{{\mathbf{T}}}
\def\mU{{\mathbf{U}}}
\def\mV{{\mathbf{V}}}
\def\mW{{\mathbf{W}}}
\def\mX{{\mathbf{X}}}
\def\mY{{\mathbf{Y}}}
\def\mZ{{\mathbf{Z}}}
\def\mBeta{{\mathbf{\beta}}}
\def\mPhi{{\mathbf{\Phi}}}
\def\mLambda{{\mathbf{\Lambda}}}
\def\mSigma{{\mathbf{\Sigma}}}


% Expectation
% \def\eE{\mathop{\mathbb{E}}\limits}
\def\eE{\mathbb{E}}

% Probability
\def\pP{\mathbb{P}}

% Tilde
\def\tf{\tilde{f}}
\def\tS{\tilde{S}}
\def\wtF{\widetilde{\mathcal{F}}}
\def\whR{\widehat{R}}
\def\tvx{\tilde{\mathbf{x}}}
\def\ty{\tilde{y}}


\def\defeq{\overset{\textup{def}}{=}}
% \def\defeq{\overset{.}{=}}
\def\defone{\overset{\text{\ding{172}}}{=}}
\def\deftwo{\overset{\text{\ding{173}}}{=}}
\def\leqone{\overset{\text{\ding{172}}}{\leq}}
\def\leqtwo{\overset{\text{\ding{173}}}{\leq}}
\def\leqthree{\overset{\text{\ding{174}}}{\leq}}
\def\leqfour{\overset{\text{\ding{175}}}{\leq}}
\def\eqone{\overset{\text{\ding{172}}}{=}}
\def\eqtwo{\overset{\text{\ding{173}}}{=}}
\def\eqthree{\overset{\text{\ding{174}}}{=}}
\def\eqfour{\overset{\text{\ding{175}}}{=}}
\def\geqfive{\overset{\text{\ding{176}}}{\geq}}
\usepackage{wrapfig}
\usepackage{hyperref}
\usepackage{url}
\usepackage{caption}

\usepackage{graphicx}
\newtheorem{definition}{Definition}
\newtheorem{proposition}{Proposition}
\newtheorem{assumption}{Assumption}
\newtheorem{theorem}{Theorem}
\usepackage{multirow}
\usepackage{algorithmic}
\usepackage{algorithm}
\usepackage{enumitem}
\setenumerate[1]{itemsep=0pt,partopsep=0pt,parsep=\parskip,topsep=5pt}
\setitemize[1]{itemsep=0pt,partopsep=0pt,parsep=\parskip,topsep=5pt}
\setdescription{itemsep=0pt,partopsep=0pt,parsep=\parskip,topsep=5pt}
\usepackage{booktabs}
\usepackage{multirow}
\usepackage[normalem]{ulem}
\useunder{\uline}{\ul}{}


% \AtBeginDocument{%
%   \providecommand\BibTeX{{%
%     Bib\TeX}}}

%% Rights management information.  This information is sent to you
%% when you complete the rights form.  These commands have SAMPLE
%% values in them; it is your responsibility as an author to replace
%% the commands and values with those provided to you when you
%% complete the rights form.
% \setcopyright{acmlicensed}
% \copyrightyear{2018}
% \acmYear{2018}
% \acmDOI{XXXXXXX.XXXXXXX}

%% These commands are for a PROCEEDINGS abstract or paper.
% \acmConference[Conference acronym 'XX]{Make sure to enter the correct
%   conference title from your rights confirmation emai}{June 03--05,
%   2018}{Woodstock, NY}
%%
%%  Uncomment \acmBooktitle if the title of the proceedings is different
%%  from ``Proceedings of ...''!
%%
%%\acmBooktitle{Woodstock '18: ACM Symposium on Neural Gaze Detection,
%%  June 03--05, 2018, Woodstock, NY}
% \acmISBN{978-1-4503-XXXX-X/18/06}

% Authors must not appear in the submitted version. They should be hidden
% as long as the \iclrfinalcopy macro remains commented out below.
% Non-anonymous submissions will be rejected without review.
\author{
Guanyu Lin\textsuperscript{1 2}, Zhigang Hua\textsuperscript{2}, Tao Feng\textsuperscript{1}, Shuang Yang\textsuperscript{2}, Bo Long\textsuperscript{2}, Jiaxuan You\textsuperscript{1}\\
\textsuperscript{1}University of Illinois at Urbana-Champaign,
\textsuperscript{2}Meta AI
}



% \author{Antiquus S.~Hippocampus, Natalia Cerebro \& Amelie P. Amygdale \thanks{ Use footnote for providing further information
% about author (webpage, alternative address)---\emph{not} for acknowledging
% funding agencies.  Funding acknowledgements go at the end of the paper.} \\
% Department of Computer Science\\
% Cranberry-Lemon University\\
% Pittsburgh, PA 15213, USA \\
% \texttt{\{hippo,brain,jen\}@cs.cranberry-lemon.edu} \\
% \And
% Ji Q. Ren \& Yevgeny LeNet \\
% Department of Computational Neuroscience \\
% University of the Witwatersrand \\
% Joburg, South Africa \\
% \texttt{\{robot,net\}@wits.ac.za} \\
% \AND
% Coauthor \\
% Affiliation \\
% Address \\
% \texttt{email}


% The \author macro works with any number of authors. There are two commands
% used to separate the names and addresses of multiple authors: \And and \AND.
%
% Using \And between authors leaves it to \LaTeX{} to determine where to break
% the lines. Using \AND forces a linebreak at that point. So, if \LaTeX{}
% puts 3 of 4 authors names on the first line, and the last on the second
% line, try using \AND instead of \And before the third author name.

\newcommand{\fix}{\marginpar{FIX}}
\newcommand{\new}{\marginpar{NEW}}
\newcommand{\lgy}[1]{{{\textcolor{black}{#1}}}}

%\iclrfinalcopy % Uncomment for camera-ready version, but NOT for submission.
\begin{document}

% \title{Unified Semantic and ID Representation Learning in Recommendation}

% how about this?
\title{Unified Semantic and ID Representation Learning for Deep Recommenders}



%%
%% By default, the full list of authors will be used in the page
%% headers. Often, this list is too long, and will overlap
%% other information printed in the page headers. This command allows
%% the author to define a more concise list
%% of authors' names for this purpose.
\renewcommand{\shortauthors}{Lin et al.}

%%
%% The abstract is a short summary of the work to be presented in the
%% article.
\begin{abstract}

Effective recommendation is crucial for large-scale online platforms. Traditional recommendation systems primarily rely on ID tokens to uniquely identify items, which can effectively capture specific item relationships but suffer from issues such as redundancy and poor performance in cold-start scenarios. Recent approaches have explored using semantic tokens as an alternative, yet they face challenges, including item duplication and inconsistent performance gains, leaving the potential advantages of semantic tokens inadequately examined. To address these limitations, we propose a Unified Semantic and ID Representation Learning framework that leverages the complementary strengths of both token types. In our framework, ID tokens capture unique item attributes, while semantic tokens represent shared, transferable characteristics. Additionally, we analyze the role of cosine similarity and Euclidean distance in embedding search, revealing that cosine similarity is more effective in decoupling accumulated embeddings, while Euclidean distance excels in distinguishing unique items. Our framework integrates cosine similarity in earlier layers and Euclidean distance in the final layer to optimize representation learning. Experiments on three benchmark datasets show that our method significantly outperforms state-of-the-art baselines, with improvements ranging from 6\% to 17\% and a reduction in token size by over 80\%. These results demonstrate the effectiveness of combining ID and semantic tokenization to enhance the generalization ability of recommender systems.

\end{abstract}

%%
%% The code below is generated by the tool at http://dl.acm.org/ccs.cfm.
%% Please copy and paste the code instead of the example below.
%%
% \begin{CCSXML}
% <ccs2012>
%  <concept>
%   <concept_id>00000000.0000000.0000000</concept_id>
%   <concept_desc>Do Not Use This Code, Generate the Correct Terms for Your Paper</concept_desc>
%   <concept_significance>500</concept_significance>
%  </concept>
%  <concept>
%   <concept_id>00000000.00000000.00000000</concept_id>
%   <concept_desc>Do Not Use This Code, Generate the Correct Terms for Your Paper</concept_desc>
%   <concept_significance>300</concept_significance>
%  </concept>
%  <concept>
%   <concept_id>00000000.00000000.00000000</concept_id>
%   <concept_desc>Do Not Use This Code, Generate the Correct Terms for Your Paper</concept_desc>
%   <concept_significance>100</concept_significance>
%  </concept>
%  <concept>
%   <concept_id>00000000.00000000.00000000</concept_id>
%   <concept_desc>Do Not Use This Code, Generate the Correct Terms for Your Paper</concept_desc>
%   <concept_significance>100</concept_significance>
%  </concept>
% </ccs2012>
% \end{CCSXML}

% \ccsdesc[500]{Do Not Use This Code~Generate the Correct Terms for Your Paper}
% \ccsdesc[300]{Do Not Use This Code~Generate the Correct Terms for Your Paper}
% \ccsdesc{Do Not Use This Code~Generate the Correct Terms for Your Paper}
% \ccsdesc[100]{Do Not Use This Code~Generate the Correct Terms for Your Paper}

%%
%% Keywords. The author(s) should pick words that accurately describe
%% the work being presented. Separate the keywords with commas.
% \keywords{Do, Not, Us, This, Code, Put, the, Correct, Terms, for,
%   Your, Paper}
%% A "teaser" image appears between the author and affiliation
%% information and the body of the document, and typically spans the
%% page.


% \received{20 February 2007}
% \received[revised]{12 March 2009}
% \received[accepted]{5 June 2009}

\maketitle

% \begin{abstract}

% \end{abstract}




\section{Introduction}
\begin{figure}[!htb]
		\begin{tabular}{c}
		    	\includegraphics[width=.6\columnwidth]{beauty.png}
		\end{tabular}
	\caption{Visualization of ID tokens on Amazon Beauty dataset. Here some ID tokens with the same color share a close embedding space, which means they can be compressed and represented with shared semantic tokens.}	\label{fig:token_dist}
\end{figure} 

In large-scale online platforms such as YouTube~\citep{covington2016deep}, TikTok~\citep{dfar}, and Amazon~\citep{he2016ups}, effectively recommending items that align with users’ preferences while filtering out irrelevant content is crucial. Traditional recommendation systems predominantly rely on ID tokens, wherein each item is uniquely identified by a distinct token~\citep{fm, sasrec}. However, as the number of items expands, this approach becomes increasingly cumbersome due to the redundancy and sheer scale of the token space. 

To overcome these limitations, recent research~\citep{rajput2024recommender, singh2023better} has explored the use of semantic tokens as an alternative to ID tokens. Nevertheless, existing works face challenges, such as inconsistent performance improvements and issues with item duplication. For instance, TIGER~\citep{rajput2024recommender} introduces semantic tokens within a deeper and more complex model architecture, making it difficult to isolate the benefits of semantic tokens themselves. Consequently, the true advantages of semantic tokens over ID tokens remain underexplored. Another study~\citep{singh2023better} demonstrates that semantic tokens offer notable improvements primarily in cold-start scenarios, yet both studies report that semantic tokens can map multiple items to the same token, leading to duplication. These open questions invite further investigation into the comparative effectiveness of semantic and ID tokens: \textit{Are semantic tokens inherently superior to ID tokens in recommendation tasks?}

In reality, semantic tokens and ID tokens complement each other. ID tokens have two primary advantages: (1) they can uncover unique, implicit relationships between items, such as the well-known association between beer and diapers, and (2) they facilitate the distinction between different items. However, ID tokens struggle to capture shared attributes across similar items and often suffer from redundancy at scale. This is analogous to whole-word tokenization in Natural Language Processing (NLP)\citep{NIPS2000_728f206c, collobert2008unified, mikolov2013distributed}, which tends to fail with unknown or out-of-vocabulary words\citep{mielke2021between}, making ID tokens less effective in cold-start situations. On the other hand, semantic tokens resemble sub-word tokenization in NLP~\citep{mikolov2012subword, wang2020neural}, where combinations of existing semantic tokens can represent new or unknown items. However, the drawback of semantic tokens lies in their tendency to map multiple, similar items to identical representations, thus failing to distinguish between them~\citep{singh2023better}. In summary, while semantic tokens excel in generalizing to unknown items, they are less effective in memorizing unique ones, suggesting that neither approach is universally superior.

To harness the complementary strengths of both token types, we propose a hybrid framework that unifies ID and semantic tokens. As illustrated in Figure~\ref{fig:token_dist}, our approach begins by visualizing the distribution of ID tokens, revealing that certain items cluster closely together in the embedding space. From this, we hypothesize that only a few dimensions of the ID token space are needed to capture unique item characteristics, while the remaining dimensions can be replaced by semantic tokens to represent shared features. Based on this hypothesis, we introduce a Unified Semantic and ID Representation Learning framework, which incorporates two key components: \textit{unified ID and semantic tokenization} and \textit{unified cosine similarity and Euclidean distance}. First, in unified tokenization, we quantize item content embeddings into a semantic codebook to capture shared characteristics, while assigning each item a low-dimensional ID token to capture unique attributes. Second, in the unified similarity and distance metric, we observe that cosine similarity is effective at disentangling densely clustered embeddings, yet struggles with distinguishing unique items, while Euclidean distance excels at the latter. Consequently, we apply cosine similarity in the earlier layers to decouple dense embeddings and Euclidean distance in the final layer to distinguish unique items. Experimental results on three benchmark datasets demonstrate that our method outperforms existing baselines by 6\% to 17\%, while reducing token size by over 80\%. Ablation studies further validate our hypothesis, showing that many ID tokens are redundant and can be effectively replaced by semantic tokens to enhance generalization.

In summary, the key contributions of this work are as follows: \begin{itemize}[leftmargin=*] \item We present the first comprehensive investigation into the complementary relationship between semantic and ID tokens in recommendation systems. \item We propose a novel \textit{unified ID and semantic tokenization} framework that captures both unique and shared item characteristics, alongside a \textit{unified similarity and distance} approach that balances embedding decoupling and item distinction. \item Our method achieves significant performance improvements on three benchmark datasets, outperforming baselines by 6\% to 17\% while reducing token size by over 80\%, thereby enhancing the system's generalization capability. \end{itemize}



\section{Preliminary}
\paragraph{\textbf{Problem Definition}}
Suppose there are \( m \) items, and each item \( i \) is represented by an encoded sentence embedding \( \boldsymbol{x}_i \). Let \( i_{t} \) denote user \( u \)'s \( t \)-th interacted item. If user \( u \) has interacted with a sequence of items \( \mathcal{I}_{u} = (i_{1}, i_{2}, \ldots, i_{t}) \), with corresponding sentence embeddings \( \mathcal{X}_{u} = (\boldsymbol{x}_{i_{1}}, \boldsymbol{x}_{i_{2}}, \ldots, \boldsymbol{x}_{i_{t}}) \), the objective of sequential recommendation is to accurately predict the next item that user \( u \) will interact with, based on their previous interaction history. Formally, the problem can be defined as follows:

\noindent \textbf{Input}: A sequence of items \( \mathcal{I}_{u} = (i_{1}, i_{2}, \ldots, i_{t}) \) that user \( u \) has interacted with, along with their corresponding sentence embeddings \( \mathcal{X}_{u} = (\boldsymbol{x}_{i_{1}}, \boldsymbol{x}_{i_{2}}, \ldots, \boldsymbol{x}_{i_{t}}) \).

\noindent \textbf{Output}: The estimated probability \( \hat{y}_{u, t+1} \) of the next item that user \( u \) will interact with at time step \( t + 1 \).


\paragraph{\textbf{ID Tokenization}} 
Traditional recommender systems often rely on ID tokenization to capture the unique characteristics of each item. In this approach, an item embedding matrix 
$\left\{\boldsymbol{e}_{i}\right\}_{i=1}^m$ is constructed, where each item \( i \) is associated with an embedding vector 
$\boldsymbol{e}_i \in \mathbb{R}^{1 \times D}$, and \( D \) represents the ID embedding dimension. The total embedding size for ID tokenization is thus $m \times D$, where \( m \) is the number of items. For a user \( u \) with an interaction sequence of items $\mathcal{I}_{u} = (i_{1}, i_{2}, \ldots, i_{t})$, we can retrieve the corresponding ID embeddings $(\boldsymbol{e}_{i_{1}}, \boldsymbol{e}_{i_{2}}, \ldots, \boldsymbol{e}_{i_{t}})$ through simple lookup operations in the embedding matrix.
\paragraph{\textbf{Semantic Tokenization}} 
To capture the semantic information of items, recent works have leveraged techniques like RQ-VAE~\citep{rajput2024recommender} to quantize content embeddings. Specifically, semantic tokenization builds $L$ layers of codebook embeddings, where each layer contains a set of embedding vectors $\left\{\boldsymbol{e}^c_{k}\right\}_{k=1}^K$, with $\boldsymbol{e}^c_{k} \in \mathbb{R}^{1 \times D'}$. Here, $D'$ denotes the semantic embedding dimension, and the total embedding size for semantic tokenization is $L \times K \times D'$. Since $L \times K \ll m$, semantic tokenization can significantly reduce the embedding size by replacing ID-based embeddings with semantically informed ones. As detailed in Algorithm~\ref{alg:rq} of Appendix~\ref{sec:semantic_token}, the RQ-VAE model quantizes the input sentence embedding $\boldsymbol{x}_{i_{t}}$ and returns the corresponding semantic embedding $\boldsymbol{z}_{i_{t}}$ for each item in user \( u \)'s interaction history. It is important to note that the stop-gradient operation, denoted as $\operatorname{sg}$, is applied during the quantization process.



% \paragraph{\textbf{ID Tokenization}}
% Traditional recommender systems are often based on ID tokenization. Specifically, to learn the unique information of each item, Traditional recommender systems often build an 
% item embedding matrix $\left\{\boldsymbol{e}_{i}\right\}_{i=1}^m, 
% \boldsymbol{e}_i \in \mathbb{R}^{1 \times D}$. Here $D$ is the ID embedding dimension and the total ID embedding size here is $m \times D$.
% As the item number $m$ can be very large, we set the dimension $D$ relatively smaller than the dimension $D'$ of the traditional recommender systems which rely on ID tokenization. Given item sequence $\mathcal{I}_{u} = (i_{1}, i_{2}, \ldots, i_{t})$ for user $u$, we can lookup ID embeddings $(\boldsymbol{e}_{i_{1}}, \boldsymbol{e}_{i_{2}}, \ldots, \boldsymbol{e}_{i_{t}})$.




% \paragraph{\textbf{Semantic Tokenization}} Recently, to learn the semantic information, some works leverage RQ-VAE~\citep{rajput2024recommender} to quantize the content embedding. Specifically, such method builds $L$ layers of codebook embeddings, each layer with $\left\{\boldsymbol{e}^c_{k}\right\}_{k=1}^K, \boldsymbol{e}^c_{k} \in \mathbb{R}^{1 \times D'}$. Here $D'$ is the semantic embedding dimension, and semantic tokenization has $L \times K \times D'$ of total embedding size. As $L \times K \ll m$, semantic tokenization can reduce the embedding size by replacing ID embedding with semantic embedding. As shown in Algorithm~\ref{alg:rq}, RQ-VAE will quantize the input sentence embedding $\boldsymbol{x}_{i_{t}}$ and return the semantic embedding $\boldsymbol{z}_{i_{t}}$ of each item in a given user $u$'s historical sequence. Note that here $\operatorname{sg}$ is the stop gradient operation.




\begin{figure*}[htb!]
		\centering
		\begin{tabular}{c}
		    	\includegraphics[width=0.75\linewidth]{fig/framework.pdf}
		\end{tabular}
	\caption{Framework of the unified semantic and ID representation learning. Firstly, the model integrates both semantic tokens, learned through RQ-VAE, and ID tokens for the recommendation task. Secondly, cosine similarity is applied in the first two layers to decouple accumulated embeddings, while Euclidean distance is utilized in the final layer to effectively distinguish unique items. Finally, the overall model is optimized in an end-to-end manner, combining the recommendation loss, RQ-VAE quantization loss, and text reconstruction loss.}	\label{fig:framework}
\end{figure*}

\section{Unified Representation Learning}
In this section, as illustrated in Figure~\ref{fig:framework}, we introduce a unified semantic and ID representation learning framework. Our method is designed to fully exploit the complementary strengths of semantic and ID tokens, integrate cosine similarity and Euclidean distance, and jointly optimize both the quantization and recommendation tasks. The key components of the framework are described as follows:

\begin{itemize}[leftmargin=*]
    \item \textbf{Unified Semantic and ID Tokenization}: To balance capturing unique and shared item characteristics, we retain only a small proportion of ID token dimensions to represent the unique attributes of items. Meanwhile, the semantic tokens, learned through RQ-VAE, are employed to capture the shared, transferable characteristics across items. This hybrid approach reduces redundancy in the ID space while enhancing generalization.
    
    \item \textbf{Unified Cosine Similarity and Euclidean Distance}: We leverage the strengths of cosine similarity and Euclidean distance in different layers of our model. Specifically, cosine similarity is applied in the earlier layers to effectively decouple accumulated embeddings, while Euclidean distance is employed in the final layer to distinguish unique items. This design maximizes the benefits of both metrics during codebook searching, enhancing the accuracy of item representation.
    
    \item \textbf{End-to-End Joint Optimization}: Our framework is trained in an end-to-end manner, jointly optimizing three key objectives: (1) the recommendation loss to ensure accurate predictions, (2) the RQ-VAE loss for effective codebook assignment, and (3) the text reconstruction loss to maintain the quality of semantic representation. This joint optimization strategy ensures that all components of the model are fine-tuned for optimal performance in both quantization and recommendation tasks.
\end{itemize}


\subsection{Unified Semantic and ID Tokenization}
\begin{figure}[!htb]
		\centering
		\begin{tabular}{c}
		    	\includegraphics[width=0.65\linewidth]{fig/architecture.pdf}
		\end{tabular}
	\caption{Illustration of unified semantic and ID tokenization. Specifically, we replace ID tokens with low-dimension ID tokens and semantic tokens.}	\label{fig:token}
\end{figure}

While ID tokenization is effective at capturing unique, item-specific information, it tends to suffer from redundancy and poor generalization, particularly in cold-start scenarios. In contrast, semantic tokenization excels at generalization by capturing shared, transferable features but may introduce item duplication when similar items are mapped to the same token. Therefore, these two approaches are complementary, and combining their strengths can address their respective limitations.

To this end, we propose a unified tokenization strategy that integrates both ID and semantic tokenization. Given that the number of items \( m \) can be very large, we reduce the dimensionality of the ID embeddings by setting \( D \) smaller than the dimension \( D' \) used for semantic embeddings. As shown in Figure~\ref{fig:token}, our method replaces most dimensions of the ID token with the more generalizable semantic token to reduce redundancy while retaining the ability to capture unique item characteristics. Specifically, for each item \( i_t \) in the user’s interaction history, we concatenate the semantic embedding \( \hat{\boldsymbol{z}}_{i_t} \) and the reduced ID embedding \( \boldsymbol{e}_{i_t} \) to form a unified representation, defined as:
\(
\boldsymbol{s}_{i_t} = [\hat{\boldsymbol{z}}_{i_t}, \boldsymbol{e}_{i_t}],
\)
which results in a sequence of unified embeddings for user \( u \), denoted as:
\(
\hat{\mathcal{S}}_{u} = (\hat{\boldsymbol{s}}_{i_{1}}, \hat{\boldsymbol{s}}_{i_{2}}, \ldots, \hat{\boldsymbol{s}}_{i_{t}})
\)

By combining ID and semantic embeddings, the unified tokenization approach retains the unique characteristics of each item while leveraging the semantic embedding's ability to generalize across similar items. This hybrid representation aims to improve both the efficiency and accuracy of recommendation by reducing redundancy in the ID space and enhancing the model's capacity to generalize to cold-start items.


% As shown in Figure~\ref{fig:token}, in this section, we replace most dimensions of ID tokens with semantic tokens, aiming to reduce the redundancy and improve the generalization ability of representation learning.

% \paragraph{\textbf{Unified Tokenization}} As ID tokenization is good at capturing unique information but falls short in generation and is redundant,  
% semantic tokenization is good at generation but has duplicate problem. That is to say, they are complimentary to each other, and we further make combination of them here. Firstly, given that the number of items \( m \) can be very large, we set the dimension \( D \) typically smaller than the dimension \( D' \) of ID tokenization. Further, we concatenate the ID embedding and semantic embedding for each item together as $\boldsymbol{s}_{i_t}$ = [$\hat{\boldsymbol{{z}}}_{i_t}$, $\boldsymbol{{e}}_{i_t}$]. Then we obtain a sequence of unified embeddings $\hat{\mathcal{S}}_{u} = (\hat{\boldsymbol{s}}_{i_{1}}, \hat{\boldsymbol{s}}_{i_{2}}, \ldots, \hat{\boldsymbol{s}}_{i_{t}})$ for user $u$.
 % $\boldsymbol{M}^c \in \mathbb{R}^{L \times K \times D}$



% \begin{equation}
% \mathcal{C}_l:=\left\{\boldsymbol{e}_{k}\right\}_{k=1}^K
% \end{equation}

% \begin{equation}
% \boldsymbol{z}_i=\textbf{Encoder}({x}_{i})
% \end{equation}
%  \begin{equation}
%  \begin{aligned}
% k_l=\arg \min_k\left\|\boldsymbol{r}_{l}-\boldsymbol{e}^c_{l, k}\right\|, \boldsymbol{r}_{l + 1} = \boldsymbol{r}_l-\boldsymbol{e}^c_{l, k}, \boldsymbol{r}_0 = \boldsymbol{z}_t,\\
%  \hat{\boldsymbol{{z}}}_t = \sum_{l = 1}^{L} \boldsymbol{e}^c_{l, k_l}
% % \\
% % k_{1,t}=\arg \min_k\left\|\boldsymbol{z}_t-\boldsymbol{e}^c_{1, k}\right\|, r_{1, t} = r_{0, t}-\boldsymbol{e}^c_{1, k_{1, t}, 
%  \end{aligned}
% \end{equation}


\subsection{Unified Distance Function}\label{sec:unified_distance}
\begin{table}[!htb]
\centering
\begin{tabular}{|l|c|c|}
\hline
\textbf{Type}         & \textbf{Cosine} & \textbf{Euclidean} \\ \hline
First layer           & 97.66\%         & 5.86\%             \\ \hline
Second layer          & 98.44\%         & 100.00\%           \\ \hline
Third layer           & 97.66\%         & 100.00\%           \\ \hline
Total coverage        & 70.13\%         & 92.67\%            \\ \hline
\end{tabular}

\caption{Comparison of cosine similarity and Euclidean distance in terms of the percentage of activated codebook across three layers and total coverage of unique items. Cosine similarity shows a high percentage of activated codebooks in all layers but lower overall coverage of unique items. In contrast, Euclidean distance exhibits high coverage of unique items, but struggles with a significantly lower percentage of activated codebooks in the first layer.}
\label{tab:distance}
\end{table}

To enhance the accuracy of codebook selection in our framework, we aim to improve the distance function used for identifying the closest codebook in $k=\arg \min_k\left\|\boldsymbol{r}_{l}-\boldsymbol{e}^c_{k}\right\|$, as defined in Algorithm~\ref{alg:rq} of Appendix~\ref{sec:semantic_token}.

\begin{figure*}[t!]
		\centering
		\begin{tabular}{ccc}
		    	\includegraphics[width=0.31\linewidth]{fig/cos_categories_first.pdf} &  \includegraphics[width=0.31\linewidth]{fig/cos_categories_second.pdf} &
       \includegraphics[width=0.31\linewidth]{fig/cos_categories_third.pdf} 
		     \\ First Codebook & Second Codebook & Third Codebook
		\end{tabular}
    \caption{Visualization of the codebook selection using cosine similarity across three layers. This figure shows the count of items from various categories assigned to specific token indices, with a focus on the top-3 codebook indices that contain the highest number of items. The distinct distribution of items across different indices suggests that cosine similarity effectively captures category-specific information and helps in distinguishing between categories.}\label{fig:cosine}
\end{figure*} 

\begin{figure*}[!htb]
		\centering
		\begin{tabular}{ccc}
		    	\includegraphics[height=0.38\linewidth]{fig/elu_categories_first.pdf} &  \includegraphics[height=0.38\linewidth]{fig/elu_categories_second.pdf} &
       \includegraphics[height=0.38\linewidth]{fig/elu_categories_third.pdf} 
		     \\ First Codebook & Second Codebook & Third Codebook
		\end{tabular}
 \caption{Visualization of the codebook selection using Euclidean distance across three layers. The uniform distribution of items across categories in the first layer indicates that Euclidean distance struggles to effectively capture category-specific information at this stage, making it less capable of distinguishing between categories compared to later layers.}
\label{fig:elu}
\end{figure*} 

\paragraph{\textbf{Statistical Analysis}} 
Our initial analysis, summarized in Table~\ref{tab:distance}, reveals that cosine similarity activates a high percentage of the codebook but struggles to cover unique items effectively. In contrast, Euclidean distance provides high coverage of unique items but activates a much lower percentage of the codebook, with only 5.86\% activation in the first layer. The limited activation of Euclidean distance in the early layers may result from its difficulty in decoupling accumulated embeddings, as these embeddings tend to cluster tightly at the beginning. Cosine similarity, on the other hand, excels in decoupling these embeddings, possibly due to its ability to handle orthogonal relationships between embeddings. However, cosine similarity’s limited ability to distinguish between distinct embeddings may be attributed to the bounded angular range of 0 to 360$^{\circ}$, while Euclidean distance, grounded in the Cartesian coordinate system, provides a more precise measure for distinguishing embeddings based on distance in \( \mathbb{R} \).





\begin{figure*}[t!]
		\centering
		\begin{tabular}{ccc}
		    	\includegraphics[width=0.31\linewidth]{fig/mix_categories_first.pdf} &  \includegraphics[width=0.31\linewidth]{fig/mix_categories_second.pdf} &
       \includegraphics[width=0.31\linewidth]{fig/mix_categories_third.pdf} 
		     \\ First Codebook & Second Codebook & Third Codebook
		\end{tabular}
  \caption{Visualization of codebook selection using the hybrid approach that combines cosine similarity and Euclidean distance. The variation in the counts of items assigned to different codebook tokens across categories demonstrates the effectiveness of this combined method in capturing category-specific information. The integration of both distance measures enhances the ability of Euclidean distance to distinguish between different categories, leading to more accurate item categorization.}\label{fig:hybrid}
\end{figure*} 

\paragraph{\textbf{Visualized Analysis}} 
To further investigate the performance of cosine similarity and Euclidean distance in codebook selection, we visualized the counts of the top-learned codebooks across different categories using both methods, as shown in Figures~\ref{fig:cosine} and \ref{fig:elu}, respectively. These visualizations demonstrate that cosine similarity can effectively capture category-specific information across layers, while Euclidean distance struggles to do so in the first layer. Specifically, in the first layer, the codebook entries selected by Euclidean distance appear uniformly distributed across categories, indicating that it fails to differentiate between them.

Based on these observations, we propose the following assumption: \textit{Cosine similarity is more effective at minimizing interference within accumulated embeddings but less capable of distinguishing distinct embeddings, whereas Euclidean distance excels at distinguishing unique embeddings but struggles to decouple accumulated ones.}

\begin{table}[!htb]
\centering
\caption{Effectiveness of the hybrid approach combining cosine similarity and Euclidean distance. The integration of Euclidean distance into cosine similarity results in a 100\% activation of the codebook across layers, while also improving the coverage of unique items. This demonstrates the advantage of leveraging both distance measures for more comprehensive and accurate item representation.}
\label{tab:hybrid}
\begin{tabular}{ccc}
\toprule
\multirow{3}{*}{\begin{tabular}[c]{@{}c@{}}Activated\\ codebook\end{tabular}} & First layer & 100.00\% \\ \cline{2-3} 
                & Second layer               & 100.00\% \\ \cline{2-3} 
                & Third layer                & 100.00\% \\ \midrule
\multicolumn{2}{c}{Coverage of unique items} & 83.27\%  \\ \bottomrule
\end{tabular}
\end{table}
\paragraph{\textbf{Proposed Method and Experimental Validation}} 
Building on this assumption, we propose a unified approach that combines cosine similarity and Euclidean distance. In the initial layers, cosine similarity is employed to decouple accumulated embeddings, while Euclidean distance is applied in the final layer to better distinguish unique items. To validate the effectiveness of this hybrid approach, we visualize the codebook selection counts across categories in Figure~\ref{fig:hybrid}. The results show that the combination of cosine similarity and Euclidean distance successfully captures category-specific information. Moreover, as shown in Table~\ref{tab:hybrid}, the percentage of activated codebook entries reaches 100\%, and the coverage of unique items improves significantly compared to using cosine similarity alone.

\paragraph{\textbf{Limitations}} 
Despite the improvements, our proposed method still results in approximately 17\% duplicate items, as observed in Table~\ref{tab:hybrid}. This issue arises when sentence embeddings for certain items are too similar to be distinguished. While this challenge is difficult to completely eliminate, it can be mitigated by assigning a unique, low-dimensional ID token to each item, helping to further differentiate items with highly similar embeddings.







\subsection{End-to-end Joint Optimization}
After unified tokenization of input item sequence for given user $u$, we then can predict the probability of next item as below.
\begin{equation}
    \hat{y}_{u, t } = \Phi (\boldsymbol{s}_{i_1}, \boldsymbol{s}_{i_2}, \cdots \boldsymbol{s}_{i_{t - 1}})
\end{equation}
where $\Phi$ is the sequential recommendation model to predict the probability $\hat{y}_{u, t }$ of next item. Here $\Phi$ can be any type of sequential recommendation models and we use SASRec~\citep{sasrec} here.
Based on the popular logloss~\citep{sasrec,dcn}, we then can optimize the recommendation model as: 
\begin{equation}\label{eq:loss}
\mathcal{L}_{recom}=-\frac{1}{|\mathcal{R}|} \sum_{(u, \mathcal{I}_{u}) \in \mathcal{R}}\left(y_{u, t} \log \hat{y}_{u, t}+\left(1-y_{u, t}\right) \log \left(1-\hat{y}_{u, t}\right)\right)  + \lambda\|\Theta\|,
\end{equation}
where $\mathcal{R}$ represents the training set, $\Theta$ denotes the learnable model parameters, and $\lambda$ denotes the regularization hyper-parameter. Finally, we jointly optimize the loss of recommendation, the loss of RQ-VAE, and the loss of reconstruction for text embedding as $\mathcal{L} = \mathcal{L}_{recom} + \mathcal{L}_{rqvae} + \mathcal{L}_{recon}$ (please refer to the algorithm in Appendix~\ref{sec:semantic_token}).

% \section{Complexity}


\section{Experimental Results \& Discussion}
Until now, we have discussed the theoretical aspects and assumptions of treatment effect estimation and uncertainty quantification in the context of A/B-testing.
To assess the practical utility of the aforementioned methods and their potential, we wish to empirically answer the following research questions:

\begin{description}
    \item[\textbf{RQ1}] \textit{Can the Kolmogorov-Smirnov test on resampled A/A-tests uncover outcomes $Y$ for which normal CIs are not appropriate?}
    \item[\textbf{RQ2}] \textit{Does the $D$-statistic provide additional information as a diagnostic over the number of observations alone?}
    \item[\textbf{RQ3}] \textit{Can we leverage other information about the distribution of $Y$?}
\end{description}

To provide empirical answers to these questions, we resort to a subet of a proprietary log of user activity data on a consumer-facing application. 
As our discussions and insights are general and agnostic to the use-cases, we expect our findings to translate to other applications.
The data consists of $|\mathcal{U}|\approx2$  million users, and approximately 17 million logged instances across 50 event types.

\paragraph{\textbf{RQ1}: Utility of the approach.}
For every possible outcome $Y$ to measure, we construct $n=5\,000$ synthetic A/A comparisons and collect $D$-statistics and Kolmogorov-Smirnov $p$-values w.r.t. the expected uniform distribution.
Figure~\ref{fig:1} visualises their distribution over event types.
As expected, the null hypothesis that the CLT has sufficiently kicked in cannot be refuted for the majority of outcomes $Y$, and normal CIs are appropriate to reflect the uncertainty in the ATE estimate.
Nevertheless, the procedure succeeds in highlighting several events that require further investigation.\footnote{Note that a direct interpretation of Kolmogorov-Smirnov $p$-values would require a multiple testing correction to be applied~\cite{Shaffer1995}. Even with a crude Bonferroni correction, the ATE($Y$) distribution of several events still violates normality significantly.}
\begin{figure}[!t]
    \centering
    \includegraphics[width=\linewidth]{img/SIRIP_Fig1.pdf}
    \caption{Visualising the Kolmogorov-Smirnov $D$-statistic and resulting $p$-value per user-event we measure. Whilst the majority of $p$-value distributions cannot be distinguished from uniform, we reject the null hypothesis for several.}
    \label{fig:1}
\end{figure}

\paragraph{\textbf{RQ2}: Considering event frequency.}
A natural question to consider is whether the sample size is the deciding factor in determining whether the sampling distribution of the ATE has approached normality sufficiently well enough for the $t$-test to be valid.
Since all comparisons use the full dataset (i.e. roughly 1 million users per A/A group), and the sample size is thus constant, this is clearly not the case.
Instead, we might then consider event frequency, as for rare events the majority of users will not contribute to the ATE.
Figure~\ref{fig:2} visualises the number of event observations on a logarithmic scale, ranging from approximately 200 to 8 million, in relation to the $D$-statistic on the y-axis. 
Whilst a clear correlation is visible (Spearman's $\rho\approx0.45$), there is no monotonic relationship.
This suggests that while event frequency is informative, the $D$-statistic brings additional diagnostic value when assessing $t$-test validity.

\begin{figure}[!t]
    \centering
    \includegraphics[width=\linewidth]{img/SIRIP_Fig2.pdf}
    \caption{Visualising the number of event observations overall to their $D$-statistic, with a log-linear trendline. Whilst rare events lead to an increase in distribution divergence, the relationship is not monotonic (Spearman's $\rho\approx0.45$).}
    \label{fig:2}
\end{figure}

\paragraph{\textbf{RQ3}: Exploring other summary statistics for $\mathsf{P}(Y)$.}
Aside from event frequency (i.e. $\mathbb{E}[Y]$), we might be interested in other moments of the outcome distribution $\mathsf{P}(Y)$.
\citet{Kohavi2014} and \cite{Kohavi2022} discuss the skewness as an important diagnostic for CLT appropriateness.
We consider the sample skewness of $Y$, and report it alongside histograms for four events in Figure~\ref{fig:3}.
We visualise, in order of the legend:
\begin{enumerate*}[label=(\roman*)]
  \item the most frequent event,
  \item the rarest event for which normality cannot be rejected,
  \item an event with similar frequency to (iv) but low $D$-statistic, and
  \item the most frequent event with $p$-value < $1e-4$.
\end{enumerate*}
Sample skewness estimates are shown in the legend, suggesting that a higher skewness implies slower CLT convergence.
Nevertheless, there is no monotonic relationship: Spearman's $\rho\approx0.43$ suggests a significant rank-correlation with the $D$-statistic, but confirms the independent informational value of the Kolmogorov-Smirnov test as a diagnostic tool.

\begin{figure}[!t]
    \centering
    \includegraphics[width=\linewidth]{img/SIRIP_Fig3.pdf}
    \caption{The empirical density function for various events intuitively shows that the sample skewness of the empirical event distribution per user is an indicator of the required sample size for the CLT to kick in, and the mean event distribution to approach normality (Spearman's $\rho\approx0.43$).}
    \label{fig:3}
\end{figure}


\section{Related Work}


\paragraph{\textbf{Sequential Recommendation}} The use of deep learning in sequential recommendation has evolved into a well-established area of research. GRU4REC~\citep{gru4rec} pioneered the application of Gated Recurrent Unit (GRU)-based Recurrent Neural Networks (RNNs) for sequential recommender. Then SASRec~\citep{sasrec} utilized self-attention mechanisms~\cite{vaswani2017attention} of Transformer to capture the context relation of whole sequence. Building on the success of masked self-supervised learning in natural language processing, subsequent works such as BERT4Rec~\citep{bert4rec} leveraged self-supervised learning to randomly mask the historical items and improved the robustness. Apart from the popular self-attention and Transformer architecture, researchers have also explored the use of Convolution Neural Networks (CNNs)~\citep{CNN} in sequential recommender~\citep{caser}. In this paper, we focus on improving the sequential recommendation using semantic tokens.

\paragraph{\textbf{Quantized Representation Learning}} Vector-quantized learning has grabbed researchers' attention with its discrete latents to reduce the model variance. In recommender systems, VQ-Rec~\citep{vqrec} proposes a transferable method to quantize item content embedding as item representation. When VQ-Rec utilizes product quantization~\citep{jegou2010product} for the generation of semantic codes, TIGER~\citep{rajput2024recommender} further leverages RQ-VAE to produce hierarchical semantic IDs as item representation. In parallel to TIGER, another work~\citep{singh2023better} demonstrated that semantic IDs can improve the generalization of recommendation ranking compared with traditional item IDs. Different from existing works aiming to replace item IDs with semantic IDs, we further consider the complementary strengths of them.

 





\section{Conclusion}
In conclusion, this work provides a comprehensive exploration of the complementary relationship between ID tokens and semantic tokens in recommendation systems, addressing the limitations of using either method in isolation. We introduced a novel framework that unifies ID and semantic tokenization, effectively capturing both unique and shared item characteristics while significantly reducing token redundancy. By leveraging a combination of cosine similarity and Euclidean distance, our approach successfully decouples accumulated embeddings and distinguishes unique items. Experimental results on three benchmark datasets demonstrate that our proposed method consistently outperforms the baselines, achieving notable improvements in performance (6\% to 17\%) while reducing token size by over 80\%. The results also validated our hypothesis that most ID tokens are redundant and can be substituted with semantic tokens to enhance generalization. Our work sets the foundation for a more efficient and effective representation strategy in recommendation systems, combining the strengths of both ID and semantic tokens for improved user experience.



\bibliography{iclr2025_conference}
\bibliographystyle{ACM-Reference-Format}

\newpage
\appendix
\clearpage
\setcounter{page}{1}
\maketitlesupplementary


\renewcommand{\thetable}{S\arabic{table}}
\renewcommand{\thefigure}{S\arabic{figure}}

\clearpage
\appendix

\onecolumn 

\section{Appendix}
\tableofcontents 
\clearpage


\begin{table*}[t]
\centering
\small
\renewcommand{\arraystretch}{1.2}
\setlength{\tabcolsep}{6pt}
\begin{tabular*}{\textwidth}{@{\extracolsep{\fill}} 
  >{\centering\arraybackslash}m{3cm}  % Sampling Mode
  >{\centering\arraybackslash}m{1.5cm} % Stride
  >{\centering\arraybackslash}m{3cm}   % Extrapolation Factor
  >{\centering\arraybackslash}m{2cm}   % Total NFE
  >{\centering\arraybackslash}m{3cm}   % Sampling Time [s]
  >{\centering\arraybackslash}m{1cm}   % FVD subcolumn 1
  >{\centering\arraybackslash}m{1cm}}  % FVD subcolumn 2
\toprule
\makecell{\textbf{Sampling}\\\textbf{Mode}} & 
\makecell{\textbf{Stride}} & 
\makecell{\textbf{Extrapolation}\\\textbf{Factor}} & 
\makecell{\textbf{Total}\\\textbf{NFE}} & 
\makecell{\textbf{Sampling}\\\textbf{Time [s]}} & 
\multicolumn{2}{c}{\textbf{FVD$\downarrow$}} \\
\cmidrule(rr){6-7}
 & & & & & \textbf{DMLab} & \textbf{FFS} \\
 \midrule
\rowcolor{gray!8}Diffusion Forcing~\cite{chen2024diffusionforcing} & $s=k-m$ & $1\times$ & $286 / 266$ & $45.32$ / $52.26$ & $60.30$ & $51.90$ \\
Rolling Diffusion~\cite{ruhe2024rollingdiffusionmodels} & $s=k-m$ & $1\times$ & $500$ / $500$ & $79.24$ / $98.23$ & \textbf{52.43} & \textbf{45.51} \\
\rowcolor{gray!8}\textit{MaskFlow} (MGM-Style) & $s=k-m$ & $1\times$ & \textbf{20} / \textbf{20} & \textbf{3.17 / 3.93} & $53.17$ & $45.92$ \\
\midrule
Diffusion Forcing~\cite{chen2024diffusionforcing} & $s=k-m$ & $2\times$ & $858$ / $798$ & $135.97$ / $156.78$ & $175.01$ & $144.43$ \\
\rowcolor{gray!8}Rolling Diffusion~\cite{ruhe2024rollingdiffusionmodels} & $s=k-m$ & $2\times$ & $896$ / $788$ & $141.99 / 154.81$ & 201.70 & 72.49 \\
\textit{MaskFlow} (MGM-Style) & $s=k-m$ & $2\times$ & \textbf{60} / \textbf{60} & \textbf{9.51} / \textbf{9.30} & $188.02$ &  $59.93$ \\
\rowcolor{gray!8}\textit{MaskFlow} (MGM-Style) & $s=1$ & $2\times$ & $740$ / $340$ & 117.27 / 66.80 & \textbf{50.87} & \textbf{30.43} \\
\midrule
Diffusion Forcing~\cite{chen2024diffusionforcing} & $s=k-m$ & $5\times$ & $2{,}002$ / $1{,}596$ & $317.27$ / $313.56$ & $232.89$ & $272.14$ \\
\rowcolor{gray!8}Rolling Diffusion~\cite{ruhe2024rollingdiffusionmodels} & $s=k-m$ & $5\times$ & $2{,}084$ / $1{,}652$ & $330.27$ / $324.56$ & $338.34$ & $248.13$ \\
\textit{MaskFlow} (MGM-Style) & $s=k-m$ & $5\times$ & \textbf{140} / \textbf{120} & \textbf{22.19 / 23.58} & $334.15$ & $108.74$ \\
\rowcolor{gray!8}\textit{MaskFlow} (MGM-Style) & $s=1$ & $5\times$ & $2{,}900$ / $1{,}300$ & 100.09/379.91 & \textbf{181.11} & \textbf{103.69} \\
\bottomrule
\end{tabular*}
\caption{\textbf{MGM Style sampling is much faster without sacrificing quality.} We report the total number of function evaluations (NFE), sampling time (in seconds), and FVD for various sampling methods and extrapolation factors across both datasets.}
\label{tab:speed_comparison}
\end{table*}



\subsection{Additional Related Work}

\paragraph{Masked Diffusion Models.} 
Limitations of autoregressive models for probabilistic language modeling have recently sparked increasing interest in masked diffusion models. Recent works like \cite{shi2024simplifiedgeneralizedmaskeddiffusion} and \cite{sahoo2024simpleeffectivemaskeddiffusion} have aligned masked generative models with the design space of diffusion models by formulating continuous-time forward and sampling processes. Works like \cite{nie2024scalingmaskeddiffusionmodels} and \cite{gong2024scalingdiffusionlanguagemodels} also demonstrate the significant scaling potential of MDM for language tasks, indicating that this masked modeling paradigm can rival autoregressive approaches for modalities beyond language such as protein co-design \cite{campbell2024generative} and vision.

\subsection{Computation of NFE for Different Sampling Methods}

Our sampling speed evaluations are determined by computing the required number of chunks 
\[
\ell = \left\lceil \frac{L - k}{s} \right\rceil + 1,
\]
to generate a video of total length \(L\), where \(k\) is the chunk size and \(s\) is the stride with which the chunk start is shifted. The overall number of function evaluations (NFEs) is then obtained by multiplying \(\ell\) with the number of sampling steps required to generate one chunk. We apply this methodology for all chunkwise-autoregressive approaches.

\begin{itemize}
    \item \textbf{MGM-Style Sampling:} In this method each chunk is generated in $20$ forward passes, so that the total NFE is
    \[
    \text{NFE}_{\mathrm{MGM}} = \ell \times 20.
    \]

    \item \textbf{FM-Style Sampling:} Here we generate each chunk in $250$ forward passes:
    \[
    \text{NFE}_{\mathrm{FM}} = \ell \times 250.
    \]
    
    \item \textbf{Diffusion Forcing with Pyramid Scheduling:} Here, we apply $250$ sampling timesteps per frame but begin unmasking earlier frames as the denoising process proceeds. For a chunk of \(k\) frames, we generate a scheduling matrix with 
    \[
    H = 250 + (k-1) + 1 = k + 250
    \]
    rows and \(k\) columns. Each entry in the scheduling matrix is computed as
    \[
    \text{scheduling\_matrix}[i,j] = 250 + j - i,\quad \text{for } i=0,\ldots,H-1 \text{ and } j=0,\ldots,k-1,
    \]
    and then clipped to the interval \([0,249]\). Since we iterate through each of the $H$ rows of the denoising matrix in each chunk we effectively compute
    \[
    \text{NFE}_{\text{DiffusionForcing}} = k + 250.
    \] 
    
    \item \textbf{RDM Sampling:} This approach proceeds in three stages:
    \begin{enumerate}
        \item \textit{Initialization (Init-Schedule):} The initial window of \(k\) frames is processed using a fixed schedule that applies $T=250$ forward passes to bring the window to its rolling state.
        
        \item \textit{Sliding Window Handling:} After initialization, the window is shifted by one frame at a time. For each shift, an inner loop is executed that updates the denoising levels until the first non-context frame (i.e., the frame immediately following the \(m\) context frames) is fully denoised (i.e., reaches a value of 1). This inner loop requires $\left\lceil \frac{T}{k-m} \right\rceil$ forward passes per window shift. As the window is shifted \((L - k)\) times, this stage contributes roughly \((L - k) \times \left\lceil \frac{T}{k-m} \right\rceil\) forward passes.
        
        \item \textit{Final Window Processing:} Once the sliding window stage is complete, the final (partial) window is further refined until all frames are fully denoised. This final stage requires additional $250$ forward passes.
    \end{enumerate}
    
    Thus, the total NFE for RDM is given by
    \[
    \text{NFE}_{\mathrm{Rolling}} = 250 \; (\text{init-schedule}) + (L - k) \times \left\lceil \frac{T}{k-m} \right\rceil\ \; (\text{sliding}) + 250 \; (\text{final window}).
    \]
\end{itemize}




\subsection{Training \& Implementation Details}

All FFS models were trained on 4 H100 GPUs with a local batch size of $4$. We run training for a total of $200{,}000$ steps and use a sigmoid scheduler that determines the per-frame masking ratio for a sampled masking level $t^k$. We use an AdamW optimizer with a learning rate of $1e-4$ and $\beta_1 = 0.9$ and $\beta_2 = 0.999$. We additionally incorporate a frame-level loss weighting mechanism based that is also based on \(t^k\). We adopt \emph{fused}-SNR loss weighting from \cite{hang2023efficient,chen2024diffusionforcing} and derive it for discrete flow matching. Let

\[
\text{SNR}(t) \;=\; \frac{\kappa(t)^2}{\,1 - \kappa(t)^2\,},
\]

where \(\kappa(t)\) is the masking schedule. The \emph{fused}-SNR mechanism smoothes SNR values across time steps in a video by computing an exponentially decaying SNR from previous frames (or tokens). We refer the reader to~\cite{chen2024diffusionforcing} for full details.


\begin{algorithm}[!ht]
\caption{\textbf{FM-Style Sampling with Context Frames for a Single Chunk}}
\label{alg:fmsampling}
\begin{algorithmic}[1]
\REQUIRE 
   $p(\mathbf{x}_1 | \mathbf{x}_t, \mathbf{t};\theta)$, 
   $t$, 
   context frames $\mathbf{c} = (c^1,\dots,c^m)$, 
   fully masked frame \([M]\) (i.e., a frame where every token equals the mask token \(M\)),
   $t \in [0,1]$, 
   $\Delta t$

\STATE $\mathbf{x}_t \,\gets\, (\,c^1,\dots,c^m,\,[M],\dots,[M])$
\STATE $t \,\gets\, 0$
\STATE $\mathbf{t} \gets (1,\dots,1,0,\dots0)$

\WHILE{$t \,\le\, 1 - \Delta t$}
    \STATE $u_t(\mathbf{x}_t) 
        \;=\; 
        \frac{t}{1-t}
        \Bigl[
          p_\theta(\mathbf{x}_1 \mid \mathbf{x}_t,\,\mathbf{t}) 
          \;-\; 
          \delta_{\mathbf{x}_t}
        \Bigr]$
    \STATE $p_\theta\!\bigl(\mathbf{x}_1 \mid \mathbf{x}_{t+\Delta t},\,\mathbf{t}+\Delta t\bigr)
        \;=\;
        \mathrm{Cat}\!\Bigl[\,
          \delta_{\mathbf{x}_t}
          \;+\; 
          u_t(\mathbf{x}_t)\,\Delta t
        \Bigr]$
    \STATE \textbf{For each token} $n$ in $\mathbf{x}_t$: 
    \STATE \quad 
    $
       x_{t+\Delta t}^{n} \gets
       \begin{cases}
          x_t^{n}, & \text{if } x_t^{n} \neq M,\\
          p(\cdot | \mathbf{x}_{t+\Delta t},\,\mathbf{t}+\Delta t; \theta), & \text{if } x_t^{n} = M.
       \end{cases}
    $
\STATE $t \gets t + \Delta t$
\STATE $\mathbf{t} \gets \mathbf{t} + \Delta t$

\ENDWHILE
\STATE \textbf{return} $\mathbf{x}_t$
\end{algorithmic}
\end{algorithm}

\begin{algorithm}[ht]
\caption{\textbf{MGM-Style Sampling for a Single Chunk}}
\label{alg:mgm_chunk_unmasking_revised}
\begin{algorithmic}[1]
\REQUIRE 
  Network $p(\mathbf{x}_1 \mid \mathbf{x}_t, \mathbf{t}; \theta)$,  
  context frames $\mathbf{c} = (c^1,\dots,c^m)$,  
  masked frame $[M]$ (i.e., every token equals $M$),    
  total unmasking steps $T$
\STATE \textbf{Initialize:}\\
$\mathbf{x}_t \;\leftarrow\; (\mathbf{c},\, [M],\dots,[M])$\\
$\mathbf{t} \;\leftarrow\; (\underbrace{1,\dots,1}_{m},\, \underbrace{0,\dots,0}_{k-m})$
\STATE Define the set of masked token indices in $\mathbf{x}_t$:\\
$\mathcal{M} \;\triangleq\; \{\, n \mid x_t^n = M \,\}.$
\FOR{$i=1$ \textbf{to} $T$}
    \STATE Compute token-wise logits:\\
    $\boldsymbol{\lambda} \;\leftarrow\; p(\mathbf{x}_1 \mid \mathbf{x}_t, \mathbf{t}; \theta).$
    \STATE \textbf{For each token} $n \in \mathcal{M}$: \\
    sample $\hat{x}_t^n \sim \mathrm{Cat}\Bigl(\mathrm{Softmax}\bigl(\boldsymbol{\lambda}^n\bigr)\Bigr)$ \\
    and compute the confidence score 
    $C_n \;=\; \mathrm{Softmax}\bigl(\boldsymbol{\lambda}^n\bigr)_{\hat{x}_t^n}.$ \\
    \STATE \textbf{Define the confidence threshold:}\\
    Let $\alpha$ denote the desired fraction of masked tokens to update in each iteration (e.g. $\alpha = 1/T$). \\
    
    Then set 
    $\tau_c \;=\; \min\Bigl\{ c \in [0,1] \;\Bigm|\; \Bigl|\{ j \in \mathcal{M} \mid C_j \ge c \}\Bigr| \ge \Bigl\lceil \alpha\,|\mathcal{M}| \Bigr\rceil \Bigr\}.$ \\
    
    (That is, $\tau_c$ is chosen as the minimum confidence such that at least $\lceil \alpha\,|\mathcal{M}| \rceil$ tokens have confidence scores at or above $\tau_c$, thereby selecting the top $\lceil \alpha\,|\mathcal{M}| \rceil$ tokens.)
    \STATE \textbf{For each token} $n \in \mathcal{M}$ with $C_n \ge \tau_c$, update:\\
    $x_t^n \;\leftarrow\; \hat{x}_t^n.$
    \STATE Update the set of masked indices:\\
    $\mathcal{M} \;\leftarrow\; \{\, n \mid x_t^n = M \,\}.$
    \IF{$\mathcal{M} = \varnothing$}
         \STATE \textbf{break}
    \ENDIF
\ENDFOR
\STATE \textbf{return} $\mathbf{x}_t$.
\end{algorithmic}
\end{algorithm}


\subsection{Baseline Details}

The two most comparable works to our method are \citet{chen2024diffusionforcing} and \citet{ruhe2024rollingdiffusionmodels}. Both of these techniques propose novel sampling methods that can be rolled out to long video lengths, and also apply frame-specific noise levels. Both of these approaches are diffusion-based and operate on continuous representations, whereas we operate on discrete tokens and use masking. We re-implement both the pyramid sampling scheme proposed in Diffusion Forcing and the Rolling Diffusion sampling method in our discrete setting. This allows us to compare the baseline sampling methods to MaskFlow on the same model backbones. 
%
To isolate the effect of our chunkwise autoregressive sampling methodology on performance from the effects of tokenization, we reimplement both the pyramid sampling scheme proposed in Diffusion Forcing and the Rolling Diffusion sampling method for our discrete setting. This allows us to compare the baseline sampling methods on the same timestep-dependent model backbone. 
%
Although it is conceivable that Rolling Diffusion sampling may perform better when applied to a model explicitly trained using the progressive noise schedule suggested in \citet{ruhe2024rollingdiffusionmodels}, we believe this comparison is still fair. Our training methodology does not inject any inductive bias by way of the masking level into the model, so there is no obvious advantage that our sampling should have over other methods. 
We provide a comprehensive evaluation of performance and sampling efficiency across both datasets and different sampling modes.


\subsection{Dataset Details}

\paragraph{Deepmind Lab.} The Deepmind Lab (DMLab) navigation dataset contains $64 \times 64$ resolution videos of random walks in a 3D maze environment. We use the total 625 videos with frame length 300 frames, and randomly sample sequences of 36 consecutive frames from each video during training. We upscale video frames to a resolution of $256 \times 256$ before tokenizing them similar to our approach for FaceForensics. We disregard the provided actions, focusing on action-unconditional video generation. We use $m=12$ and $s=24$ for the DMLab full sequence generation experiments unless stated otherwise.

\paragraph{FaceForensics.} FaceForensics (FFS) is a dataset that contains $150\times150$ images of deepfake faces, totaling 704 videos with varying number of frames at 8 frames-per-second. We upsample the resolution to $256 \times 256$, before encoding individual frames using the image-based tokenizer SD-VQGAN \cite{rombach2022high_latentdiffusion_ldm}. While image-based tokenizers have shown to lead to flickering issues, we observe high-reconstruction quality (reconstruction FVD $\approx 8$ on FFS) on our datasets and thus leave work on video tokenization to other works. After tokenization, we train on encoded frame sequences of 16 frames, each consisting of token grids with dimensionality $32 \times 32$. We generally use $m=2$ ground-truth context frames for conditioning, and $s=14$.

\subsection{Further Quantitative Results}

\paragraph{Our chunkwise autoregressive MGM-style sampling is preferable to full sequence training in settings with limited hardware.} To evaluate our method for long video generation against a longer training window baseline, we compare the performance of a frame-level masking model trained on $16$ frames with full sequence generation of a constant-masking level model trained on $32$ frames with similar batch size and on similar hardware. In Table ~\ref{tab:longer_train_window_baseline} we show that iterative rollout of our MGM-style sampling outperforms full sequence generation even when the full sequence model is trained on a longer window.

\begin{table}[ht]
    \centering
    \normalsize
    \resizebox{0.48\textwidth}{!}{%
    \begin{tabular}{l|cccc}
    \toprule
    \makecell{\textbf{Sampling} \\ \textbf{Mode}} 
    & \makecell{\textbf{Training} \\ \textbf{Window}} 
    & \makecell{\textbf{Sampling} \\ \textbf{Window}} 
    & \makecell{\textbf{Total} \\ \textbf{NFE}}
    & \makecell{\textbf{FVD} $\downarrow$} \\
    \midrule
    FM-Style (bs=2) & 32 & 32 & 250 & 253.08 \\
    \midrule
    \textit{MaskFlow} (MGM-Style) (bs=2) & 16 & 32 & 60 & 192.76 \\
    \rowcolor{gray!8}\textit{MaskFlow} (MGM-Style) (bs=4) & 16 & 32 & 60 & \textbf{59.93} \\
    \bottomrule
    \end{tabular}
    }
    \caption{\textbf{Our MGM-style sampling is more efficient and generates better results over baseline for larger training windows}. We train a constant masking ratio model on larger window sizes with similar batch size on similar hardware, and compare full sequence generation to generating the same length using our chunkwise MGM-style sampling.}
    \label{tab:longer_train_window_baseline}
\end{table}

\begin{table}[ht]
    \centering
    \normalsize
    \resizebox{0.48\textwidth}{!}{%
    \begin{tabular}{l|ccrr}
        \toprule
        & \makecell{\textbf{Extrapolation} \\ \textbf{Factor}}
        & \makecell{\textbf{Sampling} \\ \textbf{Stride}}
        & \makecell{\textbf{Total} \\ \textbf{NFE}}
        & \makecell{\textbf{FVD} $\downarrow$} \\
        \midrule
        FaceForensics   & $2\times$  & $s=14$ (\textit{full sequence}) & \textbf{60} & 59.93 \\
        \rowcolor{gray!8}FaceForensics   & $2\times$  & $s=1$ (\textit{autoregressive})  & 340 & \textbf{30.43} \\
        \midrule
        FaceForensics   & $5\times$  & $s=14$ (\textit{full sequence}) & \textbf{120} & 108.74 \\
        \rowcolor{gray!8}FaceForensics & $5\times$  & $s=1$ (\textit{autoregressive})  & 1,300 & \textbf{103.69} \\
        \midrule
        FaceForensics   & $10\times$ & $s=14$ (\textit{full sequence}) & \textbf{240} & 214.39 \\
       \rowcolor{gray!8} FaceForensics   & $10\times$ & $s=1$ (\textit{autoregressive})  & 2,900 & \textbf{165.02} \\
        \midrule
        \midrule
        DMLab & $2\times$  & $s=24$ (\textit{full sequence}) & \textbf{60} & 188.22 \\
        \rowcolor{gray!8}DMLab & $2\times$  & $s=1$ (\textit{autoregressive})  & 740  & \textbf{50.87} \\
        \midrule
        DMLab & $5\times$  & $s=24$ (\textit{full sequence}) & \textbf{140}  & 334.15 \\
       \rowcolor{gray!8} DMLab & $5\times$  & $s=1$ (\textit{autoregressive})  & 2,900 & \textbf{181.11} \\
        \bottomrule
    \end{tabular}
    }
    \caption{\textbf{Autoregressive sampling outperforms full sequence sampling on timestep-dependent models at the cost of higher NFE.}}
    \label{tab:autoregression_dependent}
\end{table}


\begin{figure*}[ht!]
    \centering
    \includegraphics[width=0.7\textwidth]{figpaper/realestate.pdf}\hfill
    \caption{\textbf{Further visualizations on the Realestate10K \cite{zhou2018stereo} dataset.} Models trained on chunk size $k = 16$ with $4$ H100 GPUs. Due to computational limitations, we cannot  provide further analyses on this larger, more compute intensive dataset.}
    \label{fig:faces_comparison}
\end{figure*}




\begin{table}[ht]
    \centering
    \normalsize
    \resizebox{0.58\textwidth}{!}{%
    \begin{tabular}{l|ccrr}
        \toprule
        & \makecell{\textbf{Extrapolation} \\ \textbf{Factor}}
        & \makecell{\textbf{Sampling} \\ \textbf{Stride}}
        & \makecell{\textbf{Total} \\ \textbf{NFE}}
        & \makecell{\textbf{FVD} $\downarrow$} \\
        \midrule
        FaceForensics   & $2\times$  & $s=14$ (\textit{full sequence}) & \textbf{60} & 109.96 \\
        FaceForensics   & $2\times$  & $s=1$ (\textit{autoregressive}) & 340 & \textbf{43.91} \\
        \midrule
        FaceForensics   & $5\times$  & $s=14$ (\textit{full sequence}) & \textbf{120} & \textbf{137.66} \\
        FaceForensics & $5\times$  & $s=1$ (\textit{autoregressive})  & 1,300 & 193.90 \\
        \midrule
        FaceForensics   & $10\times$ & $s=14$ (\textit{full sequence}) & \textbf{240} & \textbf{174.92} \\
        FaceForensics   & $10\times$ & $s=1$ (\textit{autoregressive})  & 2,900 & 293.16 \\
        \midrule
        \midrule
        DMLab & $2\times$  & $s=24$ (\textit{full sequence}) & \textbf{60} & 219.33 \\
        DMLab & $2\times$  & $s=1$ (\textit{autoregressive})  & 740  & \textbf{42.53} \\
        \midrule
        DMLab & $5\times$  & $s=24$ (\textit{full sequence}) & \textbf{140}  & 402.73 \\
        DMLab & $5\times$  & $s=1$ (\textit{autoregressive})  & 2,900 & \textbf{80.56} \\
        \bottomrule
    \end{tabular}
    }
    \caption{\textbf{Autoregressive sampling outperforms full sequence sampling on timestep-independent models at the cost of higher NFE.} Performance improvement on DMLab is substantial.}
    \label{tab:autoregression_independent}
\end{table}

\newpage

\subsection{Further Qualitative Results}

\begin{figure*}[ht!]
    \centering
    \includegraphics[width=0.48\textwidth]{figpaper/faces1.pdf}\hfill
    \includegraphics[width=0.48\textwidth]{figpaper/faces2.pdf}
    \caption{\textbf{Visualizations of FaceForensics generation results with different context frames.}}
    \label{fig:faces_comparison}
\end{figure*}
















\end{document}
