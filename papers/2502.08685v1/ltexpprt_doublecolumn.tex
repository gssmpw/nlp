% This is ltexpprt_twocolumn.tex, an example file for use with the SIAM LaTeX2E
% Preprint Series macros. It is designed to provide two-column output.
% Please take the time to read the following comments, as they document
% how to use these macros. This file can be composed and printed out for
% use as sample output.

% Any comments or questions regarding these macros should be directed to:
%
%                 Rachel Ginder
%                 SIAM
%                 3600 University City Science Center
%                 Philadelphia, PA 19104-2688
%                 USA
%                 Telephone: (215) 382-9800
%                 Fax: (215) 386-7999
%                 e-mail: rginder@siam.org


% This file is to be used as an example for style only. It should not be read
% for content.

%%%%%%%%%%%%%%% PLEASE NOTE THE FOLLOWING STYLE RESTRICTIONS %%%%%%%%%%%%%%%

%%  1. There are no new tags.  Existing LaTeX tags have been formatted to match
%%     the Preprint series style.
%%
%%  2. Do not change the margins or page size!  Do not change from the default
%%     text font!
%%
%%  3. You must use \cite in the text to mark your reference citations and
%%     \bibitem in the listing of references at the end of your chapter. See
%%     the examples in the following file. If you are using BibTeX, please
%%     supply the bst file with the manuscript file.
%%
%%  4. This macro is set up for two levels of headings (\section and
%%     \subsection). The macro will automatically number the headings for you.
%%
%%  5. No running heads are to be used for this volume.
%%
%%  6. Theorems, Lemmas, Definitions, Equations, etc. are to be double numbered,
%%     indicating the section and the occurrence of that element
%%     within that section. (For example, the first theorem in the second
%%     section would be numbered 2.1. The macro will
%%     automatically do the numbering for you.
%%
%%  7. Figures and Tables must be single-numbered.
%%     Use existing LaTeX tags for these elements.
%%     Numbering will be done automatically.
%%
%%  8. Page numbering is no longer included in this macro.
%%     Pagination will be set by the program committee.
%%
%%
%%%%%%%%%%%%%%%%%%%%%%%%%%%%%%%%%%%%%%%%%%%%%%%%%%%%%%%%%%%%%%%%%%%%%%%%%%%%%%%



\documentclass[twoside,leqno,twocolumn]{article}

% Comment out the line below if using A4 paper size
\usepackage[letterpaper]{geometry}

\usepackage{ltexpprt}
\usepackage{hyperref}

\usepackage{newfloat}
\usepackage{listings}
\usepackage{xcolor}
\usepackage{algorithm}
\usepackage{algorithmic}
\usepackage{multicol}
\usepackage{multirow}
\usepackage{amsmath, amssymb, amsfonts}
\usepackage{booktabs}
\usepackage{graphicx} 
\usepackage{bbm}
\bibliographystyle{siam}  

\usepackage[switch]{lineno}

\begin{document}
% \linenumbers
%
\newcommand\relatedversion{}
% \renewcommand\relatedversion{\thanks{The full version of the paper can be accessed at \protect\url{https://arxiv.org/abs/1902.09310}}} % Replace URL with link to full paper or comment out this line


%\setcounter{chapter}{2} % If you are doing your chapter as chapter one,
%\setcounter{section}{3} % comment these two lines out.

\title{\Large Beyond Models! Explainable Data Valuation and Metric Adaption for Recommendation}
% \author{Renqi Jia$^{1}$\thanks{Department of Computer Science, City University of Hong Kong, Hong Kong SAR.}
% \and Xiaokun Zhang\thanks{Department of Computer Science, City University of Hong Kong, Hong Kong SAR.}
% \and Bowei He \thanks{Department of Computer Science, City University of Hong Kong, Hong Kong SAR.}
% \and Qiannan Zhu \thanks{School of Artificial Intelligence, Beijing Normal University, China.}
% \and Weitao Xu \thanks{Department of Computer Science, City University of Hong Kong, Hong Kong SAR.}
% \and Jiehao Chen \thanks{China Academy of Industrial Internet, China.}
% \and Chen Ma \thanks{Department of Computer Science, City University of Hong Kong, Hong Kong SAR.}
% }
\author{Renqi Jia\thanks{Department of Computer Science, City University of Hong Kong, Hong Kong SAR. Email: renqijia2-c@my.cityu.edu.hk, dawnkun1993@gmail.com, boweihe2-c@my.cityu.edu.hk, \{weitaoxu, chenma\}@cityu.edu.hk} \and 
Xiaokun Zhang\footnotemark[1] \footnotemark[4] \and 
Bowei He\footnotemark[1] \and 
Qiannan Zhu\thanks{School of Artificial Intelligence, Beijing Normal University, China. Email: zhuqiannan@bnu.edu.cn} \and 
Weitao Xu\footnotemark[1] \and 
Jiehao Chen \thanks{China Academy of Industrial Internet, China. Email: chenjiehao@china-aii.com} \and
Chen Ma\footnotemark[1] \thanks{Corresponding author.}}

\date{}

\maketitle

% Copyright Statement
% When submitting your final paper to a SIAM proceedings, it is requested that you include
% the appropriate copyright in the footer of the paper.  The copyright added should be
% consistent with the copyright selected on the copyright form submitted with the paper.
% Please note that "2025" should be changed to the year of the meeting.

% Default Copyright Statement
\fancyfoot[R]{\scriptsize{Copyright \textcopyright\ 2025 by SIAM\\
Unauthorized reproduction of this article is prohibited}}

% Depending on which copyright you agree to when you sign the copyright form, the copyright
% can be changed to one of the following after commenting out the default copyright statement
% above.

%\fancyfoot[R]{\scriptsize{Copyright \textcopyright\ 20XX\\
%Copyright for this paper is retained by authors}}

%\fancyfoot[R]{\scriptsize{Copyright \textcopyright\ 20XX\\
%Copyright retained by principal author's organization}}

%\pagenumbering{arabic}
%\setcounter{page}{1}%Leave this line commented out.

% \begin{abstract} 
%   \small\baselineskip=9pt 
%   This is the text of my abstract. It is a brief
% description of my
% paper, outlining the purposes and goals I am trying to address.\end{abstract}

\begin{abstract}

% Recent works to jointly reconstruct 3D human and object from a single RGB image, are mostly model-based, that fail to capture the fine details of the clothed human body and object surface. In this paper, we introduce ReCHOR, a novel, model-free, first-method to produce realistic clothed human-object reconstructions from a monocular view. This is extremely challenging due to human-object occlusions, diverse interactions and depth ambiguity, as it needs to infer both 3D spatial awareness and high resolution details. Our core idea is based on estimating neural implicit representations for human and object respectively by an attention-based neural implicit model that attends to pixel-aligned features from both the global human-object image for spatial awareness and  the local separate view of human and object images for high quality details. Additionally, the network is conditioned on semantic features from an initial estimated human-object pose prior and a generative diffusion model that inpaints occluded regions, thus enabling the retrieval of details from them.
% We also propose a synthetic dataset with rendered scenes of diverse, inter-occluded 3D human and object scans, to train our network. We evaluate our method on the synthetic and real world BEHAVE dataset. Our experiments show that our method outperforms the SOTA in achieving realistic clothed human-object reconstructions.
Recent approaches to jointly reconstruct 3D humans and objects from a single RGB image represent 3D shapes with template-based or coarse models, which fail to capture details of loose clothing on human bodies. In this paper, we introduce a novel implicit approach for jointly reconstructing realistic 3D clothed humans and objects from a monocular view. For the first time, we model both the human and the object with an implicit representation, allowing to capture more realistic details such as clothing. This task is extremely challenging due to human-object occlusions and the lack of 3D information in 2D images, often leading to poor detail reconstruction and depth ambiguity. To address these problems, we propose a novel attention-based neural implicit model that leverages image pixel alignment from both the input human-object image for a global understanding of the human-object scene and from local separate views of the human and object images to improve realism with, for example, clothing details. Additionally, the network is conditioned on semantic features derived from an estimated human-object pose prior, which provides 3D spatial information about the shared space of humans and objects. To handle human occlusion caused by objects, we use a generative diffusion model that inpaints the occluded regions, recovering otherwise lost details. For training and evaluation, we introduce a synthetic dataset featuring rendered scenes of inter-occluded 3D human scans and diverse objects. Extensive evaluation on both synthetic and real-world datasets demonstrates the superior quality of the proposed human-object reconstructions over competitive methods.
\end{abstract}
\section{Introduction}\label{sec:intro}

In computational finance, Monte Carlo simulations are used extensively to estimate the expected value of financial payoffs based on the solution of stochastic differential equations (SDEs) which model the evolution of stock prices, interest rates, exchange rates and other quantities \cite{glasserman04}.  Monte Carlo methods are very general and flexible, but for high accuracy it requires generating a large number of costly SDE path approximations, which has motivated research into a number of variance reduction or, equivalently, cost reduction techniques. One such method is
Multilevel Monte Carlo (MLMC), which was proposed in \cite{GILES2008} and was adapted for various applications that are summarised in \cite{Giles_overview17} and successfully combined with other methods such as quasi-Monte Carlo methods. The main idea of MLMC is to approximate the payoff using different time stepping resolutions when numerically solving the underlying SDE and to generate an optimal number of samples on each level, such that the overall computational cost is minimised subject to the desired bound on the variance. %, such that the total computational cost is minimised. 
The computational savings come from the fact that most samples are computed on the coarser levels and hence are less expensive while only a few samples from the finest levels are required \cite{GILES2008}.


Among the directions in which the computational cost 
of MLMC methods could further be reduced, an important avenue is the use of lower precision calculations, especially for the first Monte Carlo levels where the targeted accuracy is relatively low. 
 An overview of the research on mixed precision for the standard Monte Carlo (MC) framework is provided in \cite{ChowMixedPrecisionStandardMC} but only a few references study the potential of low precision computation in the MLMC framework \cite{Rounding_error_oliver}. To the best of our knowledge, the only MLMC framework with customised precision in the literature is \cite{brugger2014mixed}, but they use a uniform precision for all operations on each Monte Carlo level instead of optimising 
 the precision of each intermediary variable to reduce as much as possible the cost of path generation.
 
An important motivation for an MLMC framework with variable precision would be performing the low precision computations on reconfigurable hardware devices such as Field Programmable Gate Arrays (FPGAs). FPGAs contain customizable logic blocks and connectors that make it easy to adapt the digital circuit architecture for a specific application, leading to a highly parallel and optimised implementation. Therefore they are successfully exploited in applications that require high speed and have high computational workload, such as signal processing \cite{woods2008fpga}, and real time applications like high frequency trading \cite{HFT1,HFT2}. That is why a number of previous works in hardware architecture design implemented the MLMC algorithm to price financial options using FPGAs as accelerators, which resulted in improved speed and power efficiency compared to full CPU architectures \cite{Schryver2013AMM}. The paper \cite{lindsey2016domain} also proposed 
a Domain Specific Language to automate the configuration of FPGAs for this specific application. However, only \cite{brugger2014mixed} proposed a heuristic to reduce the precision in calculations.

In addition, all aforementioned works considered that the random number generation (RNG) is performed in single or double precision. Yet in most cases an important portion of the workload in the overall MLMC simulation comes from the RNG and in \cite{brugger2014mixed} this limited the total computational savings.
To reduce the cost of MLMC simulations in particular those based on the Geometric Brownian Motion (GBM), \cite{approximateICDF_Oliver, NestedOliver} have proposed to use approximate random numbers that are generated by applying an approximation of the inverse CDF to uniform random numbers. In \cite{NestedOliver}, the authors proposed a way to integrate these lower precision random variables into a \textit{nested} MLMC framework and completed a numerical analysis to bound the resulting error at each MC level by a product of the time step and the error in the random number approximation. The same authors show in \cite{approximateICDF_Oliver} that using approximate random variables reduces the cost of path generation by a factor 7.


In this paper we propose a nested MLMC framework that combines the use of approximate random normal variables and lower precision calculations to reduce the computational cost of MLMC even further than \cite{brugger2014mixed,NestedOliver}. We illustrate the efficiency of our framework in Matlab, after making several assumptions on the cost of operations and size of the errors that we carefully justify. We focus on the case of GBM and use the approximate RNG methods presented in \cite{approximateICDF_Oliver} as well as a new slightly modified method that combines CDF inversion and the central limit theorem. To choose the precision of the variables in the low precision path generation, we introduce a novel method to optimise the bit-widths. This optimisation is performed before the main path generation loop is executed and is based on a linear model of the payoff error  
due to rounding when computing in low precision. The error model relies on algorithmic differentiation in a similar manner to \cite{unifying-bwoptim,bitwidth-AD,ADAPT}. The bit-width optimisation procedure can be performed off-line, so this stage can be excluded from the on-line time complexity of our framework. The user specified desired accuracy is then enforced by calculating on-line the number of samples that need to be generated.

In terms of hardware design, we suggest implementing the low precision path generation on FPGAs and the full-precision ones on a CPU or GPU. 
The FPGA offers enough flexibility to define a separate bit-width for every variable in the low precision path generation, and can be reconfigured periodically to update the bit-widths when the market parameters have changed considerably. 


The paper is organized as follows : \Cref{sec:MLMC} introduces MLMC and nested MLMC to make clear the estimator that is implemented in our framework. Then in \Cref{sec:RNG} we detail the methods that could be used to obtain approximate random normally distributed numbers very cheaply for the low precision path generation. In \Cref{sec:error_model} and \Cref{sec:costModel} we propose an error model and a cost model (resp.) that we then use to formulate the optimisation problem that is solved to obtain the optimal bit-widths of fixed point variables in \Cref{sec:optimisation}. Finally we summarise our results and future directions in \Cref{sec:conclusion}.



\section{Related Work}
\label{sec:related_work}

The original investigation \cite{gibson1979ecological} on the relationship between visual perception and human action defines \emph{affordance} as the opportunities for interaction with the surrounding environment. Behavioral studies on regular and cognitively impaired persons have shown evidence that perception results in both visual and motor signals in the human brain. An extended study \cite{anderson2002attentional} shows that visual attention to the spatial characteristics of the perceived objects initiates automatic motor signals for different actions. In computer vision, human affordance learning involves novel pose prediction such that the estimated pose represents a valid human action within the scene context. The task is fundamental to many problems requiring robust semantic reasoning about the environment, such as human motion synthesis \cite{wang2021scene} and scene-aware human pose generation \cite{wang2017binge, roy2016multi, zhang2022inpaint, yao2023scene}.

Earlier methods of affordance learning have explored knowledge mining \cite{zhu2014reasoning} and multimodal feature cues \cite{roy2016multi} to address the problem. In \cite{zhu2014reasoning}, the authors use a Markov Logic Network for constructing a knowledge base by extracting several object attributes from different image and metadata sources, which can perform various downstream visual inference tasks without any additional classifier, including zero-shot affordance prediction. In \cite{roy2016multi}, the authors use depth map, surface normals, and segmentation map as multimodal cues to train a multi-scale convolutional neural network (CNN) for scene-level semantic label assignment associated with specific human actions. In \cite{do2018affordancenet}, the authors design a multi-branch end-to-end CNN with two separate pathways for object detection and affordance label assignment to achieve high real-time inference throughput. Researchers \cite{chuang2018learning} have also explored socially imposed constraints for affordance learning. In \cite{chuang2018learning}, the authors propose a graph neural network (GNN) to propagate contextual scene information from egocentric views for action-object affordance reasoning.

Probabilistic modeling of scene-aware human motion generation also involves semantic reasoning of human interaction with the environment. Initial works on human motion synthesis have taken different architectural approaches, such as sequence-to-sequence models \cite{barsoum2018hp}, generative adversarial networks (GAN) \cite{barsoum2018hp, cai2018deep, yang2018pose}, graph convolutional networks (GCN) \cite{yan2019convolutional}, and variational autoencoders (VAE) \cite{guo2020action2motion}. However, these methods have mostly ignored the role of environmental semantics. Due to potential uncertainty in human motion, in a recent approach \cite{wang2021scene}, the authors address such motion synthesis with a GAN conditioned on scene attributes and motion trajectory to predict probable body pose dynamics.

One key challenge of human affordance generation in 2D scenes is the lack of large-scale datasets with rich pose annotations. In \cite{wang2017binge}, the authors compile the only public dataset of annotated human body poses in complex 2D indoor scenes by extracting frames from sitcom videos. Aiming to generate a contextually valid human affordance at a user-defined location, the authors propose sampling the scale and deformation parameters for an existing human pose template using a VAE conditioned on the localized image patches as scene context. In \cite{zhang2022inpaint}, the authors introduce a two-stage GAN architecture for achieving a similar goal by estimating the affine bounding box parameters to localize a probable human in the scene and then generating a potential body pose at that location. The method uses the input scene, corresponding depth, and segmentation maps as semantic guidance. In \cite{yao2023scene}, the authors propose a transformer-based approach with knowledge distillation for generating human affordances in 2D indoor scenes.


\section{Problem definition}
Given a set $W$ of n websites. Each website $W_i$, where $1 \leq i \leq$ n, is represented by a certain number of pages $m_i$, a website may consist of just one page, i.e. $m_i = 1$. 
Each page $W_i^j$ on website $W_i$, where $j$ is the page number, with $1 \leq j \leq m_i$, is a multi-record page, meaning it contains $k$ records where $2 \leq k$. 
We defined a record as an entity with a predefined set of characteristics, e.g. in the news domain these characteristics might include date, author, title, tag, etc. So the set of record's characteristics is a vector $H = \{h_0, h_1, ..., h_t\}$, where $t$ the number of characteristics in a given record. It is important to note that a record can have multiple characteristics of the same type, such as multiple tags.

We formulate the task of extracting information from multi-record pages as identifying all vectors $H$ within the given set $W$, with the following constraints:
\begin{enumerate}
  \item The proposed methods must not rely on visual information based on page rendering.
  \item The proposed methods must be applicable to all multi-record pages in the domain, even if the website was not included in the training dataset.
  \item The proposed methods should generalize beyond the news domain to other fields.
\end{enumerate}

The output of each method should be a structured representation of the records for each page, for example in JSON format.

\section{Preliminary}\label{sec:preliminary}
Given a set of documents $\gD=\{D_1,D_2,\ldots,D_{|\gD|}\}$, we can construct a knowledge graph $\gG=\{(e,r,e')\in \gE\times\gR\times\gE\}$, where $e\in\gE$ and $r\in\gR$ denote the set of entities and relations extracted from $\gD$, respectively. 

\noindent\textbf{GraphRAG Framework.} 
Given a user query $q$, we aim to design a graph-enhanced retriever to obtain relevant documents from $\gD$ by leveraging the knowledge graph $\gG$. The whole GraphRAG process  can be formulated as:
\begin{gather}
    \gG = \text{Construct}(\gD), \\ \gD^K = \text{Graph-retriever}(q,\gD,\gG),\\
    a = \text{LLM}(q,\gD^K),
\end{gather}
where $\gD^K$ denotes the top-$K$ retrieved documents and $a$ denotes the final answer generated by large language models (LLMs). In this paper, we design a novel graph retriever, a GFM retriever, which will be described in detail next.
\section{Methodology}
\label{sec:method}
\begin{figure}[!ht]
% \vspace{-1em}
\centering
    \includegraphics[width=0.90\columnwidth]{figures/HIM-framework.pdf}
    % \vspace{-1em}
    \caption{The overall framework of HIM.}
    \label{fig:HIM}
\end{figure}
% \vspace{-2em}
\subsection{Overview}
This work aims to address the IM problem from a new perspective.
We encode potential influence spread trends into hyperbolic representations for the effective selection of highly influential seed users.
Our motivation has two key points.
(1) We aim for a diffusion model agnostic method that solves the IM problem without relying on any assumptions on diffusion parameters.
(2) The influenc trend of users can be efficiently approximated by directly utilizing the properties of learned representations.
To this end, we leverage the benefits of hyperbolic geometry to propose a novel method for IM.

We use the social network and the graph set of influence propagation instances as learning data and apply hyperbolic network embedding to construct user representations.
Instead of explicitly computing users' influence spread, we implicitly estimate their influence spread with the learned representations.
The distance information of the representations can effectively measure the influence spread of seed user nodes.

Specifically, a novel hyperbolic spread learning method HIM is proposed, as is shown in Figure~\ref{fig:HIM}. HIM mainly consists of two modules: (1) \textit{Hyperbolic Influence Representation} aims to learn user representations in the hyperbolic space. (2) \textit{Adaptive Seed Selection} selects target seed users based on learned hyperbolic representations via an adaptive algorithm. 

\subsection{Hyperbolic Influence Representation}
We first encode influence spread features from social influence data to construct user representations in hyperbolic space. The social influence data includes social networks and influence propagation instances, as mentioned in Section~\ref{sec:assume}.
The structural information of the network and the historical spread patterns of propagation instances are crucial for estimating the influence spread of seed users. Both should be effectively integrated into user representations.

% Social connections can be easily obtained from a given social network.
% However, the propagation relations are relatively complicated. Instead of relying on specific diffusion models, we attempt to learn the influence propagation information from the observed data. Given the historical diffusion cascades, we can obtain their propagation instances~\cite{ICDE_feng2018inf2vec}. 
% As mentioned, each instance can be viewed as a directed subgraph $G_D$ of the social network $G$.
% Each instance can be viewed as a directed propagation subgraph of the social network, where an edge $(u \to v)$ denotes that user $u$ influences user $v$. 

The learning process follows a shallow embedding approach.
Both types of data naturally form a graph, enabling effective representation learning on edge sets.
We do not adopt more complex embedding methods as~\cite{KDD2016_grover_node2vec, KDD2017_ribeiro_struc2vec} as we aim to intuitively demonstrate that influence spread can be estimated based on hyperbolic representations learned from social influence data, which is previously unexplored.
Meanwhile, this approach maintains computational efficiency, making it scalable for large-scale social networks.

Given social influence data, we propose a rotation-based Lorentz model to learn hyperbolic user representations. 
Note that this preprocess is model-agnostic, making it adaptable to various diffusion models and practical applications. 

At first, given a social network $G = (V, E)$, we assign each user $u \in V$ an initial representation $\mathbf{x}_u \in \mathbb{L}^{n}_{\gamma}$, initialized via hyperbolic Gaussian sampling as in work~\cite{sun2021hgcf}.

\subsubsection{Rotation Operation.}
We apply hyperbolic rotation operation~\cite{ICLR19rotate, ACL20_chami2020low} to assist in integrating structure and influence spread information for effective representation learning. 
By adjusting angles, various rotation operations capture different types of information, ensuring seamless integration into unified user representations.

In detail, we use two sets of rotation matrices $(\mathbf{Rot}^{S}_{s}, \mathbf{Rot}^{T}_{s})$ and $(\mathbf{Rot}^{S}_{d}, \mathbf{Rot}^{T}_{d})$ to assist in representation learning. Here, $s$ denotes the social relation, while $d$ denotes the propagation relation. $S$ and $T$ denote the rotation operations applied to head nodes and tail nodes, respectively.
The rotation operation further brings extra benefits for IM.
% Employing rotation transformation offers several benefits to learning influence representations for the IM problem.
% First, rotations can capture various symmetric and asymmetric relations among users~\cite{ICLR19rotate,ACL20_chami2020low}.
The rotation operation in representation learning adjusts vectors' angles to bring related user representations closer while preserving their distances, therefore maintaining hierarchical information.
Besides, It is also efficient and easy to implement.

% -------------------------------------------------------------------------------------
% Learn Static Influence
% -------------------------------------------------------------------------------------

\subsubsection{Network Structure Learning.}

In this part, we deduce structure influence from the social connections present in the social network by modeling the edges within the given graph $G=(V, E)$. 
The core idea is to maximize the joint probability of observing all edges in the graph to learn node embeddings.

Specifically, given an observed edge $(u \rightarrow v) \in E$, the probability $\Pr(v|u)$ can be estimated by a score function based on the squared Lorentzian distance:
% $\small \Pr(v|u) = \frac{ \exp(\mathcal{V}^{S}_{uv}) } { Z(u) }$,
$\small \Pr(v|u) = \exp(\mathcal{V}^{S}_{uv})  /  Z(u) $,
% \begin{equation} \small \Pr(v|u) = \frac{ \exp(\mathcal{V}^{S}_{uv}) } { Z(u) }, \label{eq:prob-u-v} \end{equation}
where $Z(u) = \sum_{ o \in V } \exp(\mathcal{V}^{S}_{uo})$, and
edge score $\mathcal{V}^{S}_{uv} $ is defined as:
\begin{equation} 
\small \mathcal{V}^{S}_{uv} = - w_{uv} \cdot d^2_{\mathcal{L}}\left(\mathbf{x}^S_u, \mathbf{x}^T_v\right) + b_u + b_v,
\label{eq:relation-score}
\end{equation}
where $ w_{uv} > 0 $ is the coefficient associated with the edge $(u \rightarrow v)$.
Generally, we set $w_{uv} = 1/d_{u}$.
$b_u$ and $b_v$ represent biases of node $u$ and node $v$, respectively.
$\mathbf{x}^S_u = \mathbf{Rot}_{s}^S(\mathbf{x}_u)$ and $\mathbf{x}^T_v = \mathbf{Rot}_{s}^T(\mathbf{x}_v)$ are the rotated representations. 
Since the normalization term $Z(u)$ is expensive to compute, we approximate it via a negative sampling strategy~\cite{mikolov2013neg-sampling}.
% Note that the normalization term $Z(u)$ is expensive to compute, we approximate it via a new negative sampling strategy: We first divide all nodes into $L$ ranges according to their degrees. 
% When sampling negative nodes for given users, we carry out sample selection in the corresponding range according to their degrees, making refined distinctions among users with similar degrees. 
Therefore, we estimate $\Pr(v|u)$ in the log form as:
\begin{equation}
\small \log P(v|u) \approx \log \varphi \left(\mathcal{V}_{uv} \right) + \sum_{o \in \mathcal{N}_u} \log \varphi \left( - \mathcal{V}_{uo}\right),
\label{eq:log_p_u_v}
\end{equation}
where $\varphi(x) = 1/(1+e^{-x})$ is the Sigmoid function and $\mathcal{N}_u$ is the set of sampled negative nodes.
Assuming they are independent of each other, the joint probability of all social connections can be calculated as:
\begin{equation} \small \mathcal{P} = \sum_{(u,v)\in E} \log P(v|u). \end{equation}
By maximizing this joint probability, we encode the structure information of the social network into user representations.
% Accordingly, our goal is to capture the static influence of all users by maximizing this joint probability.

% -------------------------------------------------------------------------------------
% Learn Dynamic Influence
% -------------------------------------------------------------------------------------

\subsubsection{Influence Propagation Learning.}
Here, we extract historical influence spread patterns from the propagation instance graph sets $\mathcal{G}_D$.
Similarly, given any propagation graph $G^i_D \in \mathcal{G}_D$, we maximize the joint probability of observing influence activations in the $G^i_D$ to encode spread patterns into user embeddings.

In detail, given $G^i_D = (V^i_D, E^i_D)$, the edge probability of $(u \rightarrow v) \in E^i_D$ can be calculated similar to Eq. (\ref{eq:log_p_u_v}) as:
\begin{equation}
\small \log P(v|u) \approx \log \varphi \left(\mathcal{V}^{D}_{uv}\right) + \sum_{o \in \mathcal{N}_u} \log \varphi \left( - \mathcal{V}^{D}_{uo}\right),
\end{equation}
% \;\:
\begin{equation}
\small \mathcal{V}^{D}_{uv} = - w_{uv} \cdot d^2_{\mathcal{L}}\left(\mathbf{x}^S_u, \mathbf{x}^T_v\right) + b_u + b_v,
\label{eq:propagation_score}
\end{equation}
where $ w_{uv} = 1/d_u $ is the coefficient, $\mathbf{x}^S_u = \mathbf{Rot}_{d}^S(\mathbf{x}_u)$ and $\mathbf{x}^T_v = \mathbf{Rot}_{d}^T(\mathbf{x}_v)$ are the rotated user representations. 
The joint probability of all edges in $G^{i}_D$ can be calculated as:
\begin{equation} \small \mathcal{P}_{G^i_D} = \sum_{(u,v)\in E^{i}_D} \log P(v|u). \end{equation}

During the propagation process, once a user $u$ triggers influence activation, we want to assign a bonus to highlight this user’s tendency to positively influence others. 
Inspired by the approach in~\cite{ICML2023_Yang}, we address this intuitively by reducing the hyperbolic distance of the related user representations from the origin in the embedding space.
Thus, for all influence activations in $G^i_D$, we propose a proactive influence regularization term:
\begin{equation}
 \mathcal{I}_{G^i_D} = \sum_{(u,v)\in G^i_D} \alpha_u \cdot \log \varphi \left(d^2_{\mathcal{L}}(\mathbf{x_u}, \mathbf{o}_{\mathcal{L}})\right).
\end{equation}
where $\mathbf{o}_{\mathcal{L}}$ is the origin of the Lorentz model and $\alpha_u$ is calculated as $\sqrt{d_u/d_{\text{max}}}$.
This term further pulls high-influence users closer to the origin in the representation space.
The illustration of learning an observed influence instance $(u \rightarrow v)$ is shown in Figure~\ref{fig:emb}.
\begin{figure}[h]
  \centering
  \includegraphics[width=0.90\columnwidth]{figures/method/do_emb.pdf}
  \caption{ Illustration of the influence propagation learning. The propagation relation between user $u$ and $v$ is depicted by the distance $d^2_{\mathcal{L}}$ between their rotated embeddings. }
  \label{fig:emb}
\end{figure}

For simplicity, we define $LDO$ as the squared Lorentzian distance from a given representation to the origin. Specifically, for user $u$, the $LDO_u$ is defined as $LDO_u = d^2_{\mathcal{L}}(\mathbf{x}_u, \mathbf{o}_{\mathcal{L}})$.
Previous studies~\cite{nickel2017poincare, ICML2023_Yang, feng2022role} have shown that hierarchical information can be effectively inferred from $LDO$s. In our method, user nodes with smaller $LDO$ values are more likely to be influential in social networks. 
We will later design seed selection strategies based on $LDO$.

% -------------------------------------------------------------------------------------
% Objective Function
% -------------------------------------------------------------------------------------

\subsubsection{Objective Function.}

Combining above two parts, the overall loss function is calculated as:
\begin{equation}
\small
\mathcal{L} = - \left( \mathcal{P} + \sum_{G^i_D \in \mathcal{G}_D}\left(\mathcal{P}_{G^{i}_D} + \mathcal{I}_{G^i_D}\right)\right). 
\label{eq:over_loss}
\end{equation}
Optimizing Eq. (\ref{eq:over_loss}) brings relevant nodes closer together while keeping irrelevant nodes as far apart as possible. Meanwhile, users involved in more influence activations tend to have their representations move closer to the origin, indicating potential higher influence spread.
The time complexity can be found in Appendix.

Once the learning process is complete, users with strong spread relations will be clustered together in the embedding space, and highly influential users tend to be located near the origin, which helps to identify seed users for the IM problem.

\subsection{Adaptive Seed Selection} 
\begin{algorithm}[H]
% \small
\caption{Adaptive Sliding Window (ASW)}\label{alg:ASW}
\begin{algorithmic}[1]
\Statex \textbf{Input:} social graph $G$, user representations $\mathbf{X}$, seed number $k$ and window size coefficient $\beta$ 
\Statex \textbf{Output:} $S^*$ with $k$ seed users
\State $S^* \gets$ an empty set, window size $w \gets \beta \cdot k$
\State $D \gets$ compute $ LDO_u = d^2_{\mathcal{L}}(\mathbf{x}_u, \mathbf{o}_{\mathcal{L}}) \text{ for each } u \text{ in } V$ 
\State $\mathcal{Z} \gets \text{sort } D  \text{ in ascending order} $
\State $c \gets$ select the $u$ with minimum $\mathcal{Z}_u$
\State $Q \gets$ a priority queue initialized with key-value pairs $(u, \mathcal{Z}_u)$ for the next $w$ users in $\mathcal{Z}$.
\While{$|S^*| < k$}
\State add $c$ to $S^*$ and find $N_c$ the neighbors of $c$ from $G$
\State $\mathcal{C} = N_c \cap Q_{keys}$
\If{$ \mathcal{C} = \emptyset $}
\State $c = Q.$pop and add the next $(u, \mathcal{Z}_u)$ in $\mathcal{Z}$ to $Q$ 
\Else
\State compute $\mathcal{Z}'_v$ according to Eq. (\ref{eq:update_score}) 
\State update $Q$ with $(v, \mathcal{Z}'_v)$ for each $v$ in $\mathcal{C}$
\State $c = Q.$pop and add the next $(u, \mathcal{Z}_u)$ in $\mathcal{Z}$ to $Q$ 
\EndIf
\EndWhile \textbf{ and return $S^*$} 
\end{algorithmic}
\end{algorithm}

After integrating social influence information into the hyperbolic representations, the next step involves designing strategies to select target seed users based on these learned representations. Specifically, we propose adaptive seed selection, which aims to leverage the geometric properties of the hyperbolic representations to effectively find seed users who possess large influence spread.

In practice, users with high influence might have overlapping areas of influence. Independently selecting highly influential users may not result in optimal overall performance due to the submodularity of social influence~\cite{kempe2003im}. Additionally, the submodularity property of the IM problem implies diminishing marginal gains from seed users~\cite{TKDE18_li2018influence_survey}, particularly for users who are close to the already selected seed users. Therefore, it is crucial to consider these spread relations among users when selecting seed nodes. Previous methods required traversing all nodes, leading to high computational costs. Given that influence strength can be estimated by the distance of representations from the origin and that the spread relations among users can be measured by the distance between their representation vectors, we have designed a new algorithm for seed set selection, which is shown in Algorithm~\ref{alg:ASW}.

The key idea of our strategy is to assign each user an initial score and dynamically adjust these scores during the selection process. Therefore, we could determine the final seed set by considering the spread relation among users. Specifically, we first assign each user with a score $\mathcal{Z}_u = LDO_u = d^2_{\mathcal{L}}(\mathbf{x}_u, \mathbf{o}_{\mathcal{L}})$. We sort all scores $\mathcal{Z}$ and select the node $c$ with the smallest $\mathcal{Z}_c$ as the first seed user. Instead of directly choosing the node with the second lowest $LDO$, the next $w$ nodes in the sorted list are viewed as candidate nodes, where $w$ is the size of a sliding window $W$. The $W$ is used to explore a wider range of candidate nodes while maintaining computational efficiency. We determine the window size $w$ based on $k$ as $w = \beta \cdot k$, allowing it to adaptively adjust its size for different data scales.
In Algorithm~\ref{alg:ASW}, the sliding window $W$ is implemented by a priority queue $Q$.
Next, we find the intersection $\mathcal{C}$ of the current seed node's neighbors with the candidate nodes.
Accordingly, we update the scores of the nodes in $\mathcal{C}$. 
For a user $u \in \mathcal{C}$, the updated score $\mathcal{Z}'_u$ is calculated as:
\begin{equation}
\small
\mathcal{Z}'_u = \mathcal{Z}_u + \frac{w_{c,u}}{d_c} \cdot  \mathcal{Z}_{c},
\label{eq:update_score}
\end{equation}
\begin{equation}
% \;\;
\small
w_{c,u} = \frac{
\exp(1/d^2_{\mathcal{L}}(\mathbf{x}_c, \mathbf{x}_u))
}{ \sum_{v \in \mathcal{C}} \exp(1/d^2_{\mathcal{L}}(\mathbf{x}_c, \mathbf{x}_v))}.
\label{eq:update_score_2}
\end{equation}
Here, $c$ denotes the recently selected node, $d_c$ is the degree of node $c$, and $w_{c,u}$ means the weight between them. Intuitively, a node closer to node $c$ may have a larger spread overlap with $c$, leading to a larger penalty from node $c$ and thus increasing its score.
In this way, the candidates' scores in the sliding window will be updated. After that, we select the node with the lowest score. At each iteration, the chosen node is removed from the window, and the next node from the sorted $LDO$ list is added to the window. This process is repeated until $k$ seed users are selected. 
Due to space limitations, the time complexity analysis can be found in Appendix.

% \subsubsection{Discussion.} 

% The influence strength of user nodes can be effectively measured by the distance of their representations from the space's origin. Meanwhile, the relationship between two users can be efficiently measured by the distance between their representations. Compared to other methods that utilize graph properties, such as the shortest path, to measure the relationship between two nodes, calculating the distance between representation vectors can greatly enhance computational efficiency. Representations in hyperbolic space can effectively measure both the influence of individual users and the social relations among them, enabling the design of efficient algorithms for classic IM problems. 
% Indeed, how to effectively select seed nodes based on the learned hyperbolic representations remains an open question worth further exploration.

% The influence strength of user nodes can be effectively measured by the distance of their representations from the origin in hyperbolic space. Similarly, the relationship between two users can be efficiently assessed by the distance between their respective representations. Compared to traditional methods that rely on graph properties, such as the shortest path, calculating the distance between representation vectors significantly enhances computational efficiency. Thus, we argue that hyperbolic space representations are particularly well-suited for measuring both individual user influence and social relationships, thereby facilitating the design of efficient algorithms for the IM problem.

% Applying two proposed strategies to HIM, we obtain two specific IM methods: HIM-MD and HIM-ASW.
% Later, in the experimental section, we will evaluate the performance of two methods.

% Section Transition
\section{Experiments}
\label{sec:experiments}

\subsection{Experiment Setting}
\label{sec:exp_setting}

\textbf{Hyperparameters.}
We described details in~\cref{sec:appendix_hyperparameter}.

\textbf{Baselines.}
We compare the performance of \ours against the following baselines, mostly chosen for their long-context capabilities.
(1) \textbf{Truncated FA2}: The input context is truncated in the middle to fit in each model's pre-trained limit, and we perform dense attention with FlashAttention2 (FA2)~\citep{dao_flashattention_2022}.
(2) \textbf{DynamicNTK}~\citep{bloc97_ntk-aware_2023} and (3) \textbf{Self-Extend}~\citep{jin_llm_2024} adjust the RoPE for OOL generalization. We perform dense attention with FA2 without truncating the input context for these baselines.
Both (4) \textbf{LM-Infinite}~\citep{han_lm-infinite_2024} and (5) \textbf{StreamingLLM}~\citep{xiao_efficient_2024} use a combination of sink and streaming tokens while also adjusting the RoPE for OOL generalization.
(6) \textbf{H2O}~\citep{zhang_h_2o_2023} is a KV cache eviction strategy which retains the top-$k$ KV tokens at each decoding step. 
(7) \textbf{InfLLM}~\citep{xiao_infllm_2024} selects a set of representative tokens for each chunk of the context, and uses them for top-$k$ context selection.
%(8) \textbf{Double Sparse Attention}~\citep{yang_post-training_2024} estimates the top-$k$ tokens by sampling few channels of the key vectors.
(8) \textbf{HiP Attention}~\citep{lee_training-free_2024} uses a hierarchical top-$k$ token selection algorithm based on attention locality.

\textbf{Benchmarks.}
We evaluate the performance of \ours on mainstream long-context benchmarks. 
(1) LongBench~\citep{bai_longbench_2023}, whose sequence length averages at around 32K tokens, 
and (2) $\infty$Bench~\citep{zhang_inftybench_2024} with a sequence length of over 100K tokens. 
Both benchmarks feature a diverse range of tasks, such as long document QA, summarization, multi-shot learning, and information retrieval.
We apply our method to the instruction-tuned Llama 3 8B model~\citep{grattafiori_llama_2024} and the instruction-tuned Mistral 0.2 7B model~\citep{jiang_mistral_2023}. As our framework is training-free, applying our method to these models incurs zero extra cost.

\begin{figure}
\centering
\vspace{0.5em}
\includegraphics[width=0.485\linewidth]{figures/images/plot_sglang_decoding_RTX4090.pdf}
\includegraphics[width=0.485\linewidth]{figures/images/plot_sglang_decoding_L40s.pdf}
\vspace{0.5em}
\includegraphics[width=\linewidth]{figures/images/plot_sglang_decoding_legend.pdf}
\vspace{-2em}
\caption{\textbf{SGlang Decoding Throughput Benchmark.} Dashed lines are estimated values. RTX4090 has 24GB and L40s has 48GB of VRAM. We used is AWQ Llama3.1 with FP8 KV cache.}
\label{fig:sglang_decoding}
\vspace{-1.8em}
\end{figure}
\subsection{Results}
\textbf{LongBench.}
In \Cref{tab:longbench}, our method achieves about 7.17\%p better relative score using Llama 3 and 3.19\%p better using Mistral 0.2 compared to the best-performing baseline InfLLM.
What makes this significant is that our method processes 4$\times$ fewer key tokens through sparse attention in both models compared to InfLLM, leading to better decoding latency as shown in \cref{tab:latency}.

\textbf{$\infty$Bench.}
We show our results on $\infty$Bench in \Cref{tab:infbench}. The \textit{3K-fast and 3K-flash} window option of ours uses the same setting as \textit{3K} except using a longer mask refreshing interval as detailed in \Cref{sec:exp_setting}.
Our method achieves 9.99\%p better relative score using Llama 3 and 4.32\%p better using Mistral 0.2 compared to InfLLM. The performance gain is larger than in LongBench, which has a fourfold shorter context. This suggests that our method is able to better utilize longer contexts than the baselines.

To further demonstrate our method's superior OOL generalization ability, we compare $\infty$Bench's En.MC score in various context lengths with Llama 3.1 8B in \cref{fig:infbench_llama3.1}.
While \ours keeps gaining performance as the context length gets longer, baselines with no OOL generalization capability degrade significantly beyond the pretrained context length (128K).
In \cref{fig:infbench_gemma_exaone}, we experiment with other short-context LLMs: Exaone 3 (4K)~\citep{research_exaone3_2024}, Exaone 3.5 (32K)~\citep{research_exaone_2024} and Gemma2 (8K)~\citep{team_gemma_2024}.
We observe the most performance gain in an extended context with these short-context models. For instance, with Gemma2, we gain an impressive +24.45\%p in En.MC and +22.03\%p in En.QA compared to FA2.

\subsection{Analysis}
In this section, we analyze the latency and the effect of each of the components of our method.


\textbf{Latency.}
We analyze the latency of our method on a 1-million-token context and compare it against baselines with settings that yield similar benchmark scores. In \cref{tab:latency}, we measure the latencies of attention methods.
%InfLLM uses a 12K context window, HiP uses a 1K window, and ours uses the 3K window setting.
During a 1M token prefill, our method is 20.29$\times$ faster than FlashAttention2 (FA2), 6\% faster than InfLLM, and achieves similar latency with the baseline HiP.
During decoding with a 1M token context, our method significantly outperforms FA2 by 19.85$\times$, InfLLM by 4.98$\times$, and HiP by 92\%.
With context extension (dynamic RoPE) enabled, our method slows down about 1.6$\times$ in prefill and 5\% in decoding due to overheads incurred by additional memory reads of precomputed cos and sin vectors.
Therefore, our method is 50\% slower than InfLLM in context extension-enabled prefill, but it is significantly faster in decoding because decoding is memory-bound:
Our method with a 3K token context window reads fewer context tokens than InfLLM with a 12K token context window.

\textbf{Latency with KV Offloading.} In \cref{tab:latency_offload}, we measure the decoding latency with KV cache offloading enabled on a Passkey retrieval task sample.
We keep FA2 in the table for reference, even though FA2 with UVM offloading is 472$\times$ slower than the baseline HiP.
Among the baseline methods, only InfLLM achieves KV cache offloading in a practical way.
In 256K context decoding, we outperform InfLLM by 3.64$\times$.
With KV cache offloading, the attention mechanism is extremely memory-bound, because accessing the CPU memory over PCIe is 31.5$\times$ more expensive in terms of latency than accessing VRAM.
InfLLM chooses not to access the CPU memory while executing its attention kernel, so it has to sacrifice the precision of its top-k estimation algorithm. This makes larger block and context window sizes necessary to maintain the model's performance on downstream tasks.
In contrast, we choose to access the CPU memory during attention kernel execution like baseline HiP.
This allows more flexibility for the algorithm design, performing better in downstream NLU tasks.
Moreover, our UVM implementation makes the KV cache offloaded attention mechanism a graph-capturable operation, which allows us to avoid CPU overheads, unlike InfLLM.
In contrast to the offloading framework proposed by \citet{lee_training-free_2024}, we cache the sparse attention mask separately for each pruning stage. 
This enables us to reduce the frequency of calling the costly initial pruning stage, which scales linearly.

\textbf{Throughput.} In \cref{fig:sglang_decoding}, we present the decoding throughput of our method using RTX 4090 (24GB) and L40S (48GB) GPUs. On the 4090, our method achieves a throughput of 3.20$\times$ higher at a 1M context length compared to the estimated decoding throughput of SRT (SGlang Runtime with FlashInfer). Similarly, on the L40S, our method surpasses SRT by 7.25$\times$ at a 3M context length.
Due to hardware limitations, we estimated the decoding performance since a 1M and 3M context requires approximately 64GB and 192GB of KV cache, respectively, which exceeds the memory capacities of 24GB and 48GB GPUs.
We further demonstrate that adjusting the mask refreshing interval significantly enhances decoding throughput without substantially affecting performance. The \textit{Flash} configuration improves decoding throughput by approximately 3.14$\times$ in a 3M context compared to the \textit{Fast} configuration.

%\begin{wrapfigure}[9]{l}{0.35\linewidth}
\begin{figure}[t]
\vspace{0.5em}
\centering
\begin{subfigure}[b]{0.50\linewidth}
\includegraphics[width=\linewidth]{figures/images/plot_topk_recall.pdf}
\vspace{-1.5em}
\caption{\textbf{Recall.}}
\label{fig:topk_recall}
\end{subfigure}
% \hfill
% \hspace{0.2em}
\raisebox{0.2in}{
\begin{subtable}[b]{0.46\linewidth}
\vspace{0.5em}
\centering
\resizebox{0.8\linewidth}{!}{%
\begin{minipage}{\linewidth}
\centering
\begin{tabular}{lr}\toprule%
&En-MC \\\midrule%
FA2 (128K) &67.25 \\%
Ours ($N=2$) &70.31 \\%
Ours ($N=3$) &\textbf{74.24} \\%
\bottomrule%
\end{tabular}%
\end{minipage}%
}
\caption{\textbf{Pruning Stage Ablation Study in $\infty$Bench En.MC.}}
\label{tab:stage_ablation}%
\end{subtable}
}
\vspace{-2.2em}
\caption{\textbf{Analysis}}
\vspace{-1.8em}
\end{figure}
%\end{wrapfigure}
\textbf{Accuracy of top-$k$ estimation.}
In \cref{fig:topk_recall}, we demonstrate our method has better coverage of important tokens, which means higher recall of attention probabilities of selected key tokens. 
Our method performs 1.57\%p better than InfLLM and 4.72\%p better than baseline HiP.
The better recall indicates our method follows pretrained attention patterns more closely than the baselines. 

%\begin{wraptable}[8]{r}{3.2cm}
\vspace{-4.5ex}
\caption{\textbf{Pruning Stage Ablation Study in InfiniteBench En.MC.}}
\centering
\small
\vspace{1.0em}
\label{tab:stage_ablation}
\begin{tabular}{l@{\hskip-2pt}r}\toprule
&En-MC \\\midrule
FA2 (128K) &67.25 \\
Ours ($N=2$) &70.31 \\
Ours ($N=3$) &\textbf{74.24} \\
\bottomrule
\end{tabular}
\end{wraptable}
\textbf{Ablation on Depth of Stage Modules.}
In \Cref{tab:stage_ablation}, we perform an ablation study on a number of stages ($N$) that are used in ours. The latency-performance optimal pruning module combination for each setting is found empirically.

\begin{table}[t]
\caption{\textbf{RoPE Ablation Study in Context Pruning and Sparse Attention.} We measure the accuracy of $\infty$Bench En.MC subset truncated with $T$=128K with various combinations of RoPE extends style in context pruning and sparse attention kernels. Each row represents a single RoPE extend style in the context pruning procedure, and each column represents the RoPE extend style in block sparse attention. \textit{SA} stands for sparse attention, \textit{DE} stands for dynamic RoPE extend (SelfExtend variant), \textit{IL} stands for InfLLM style RoPE, \textit{ST} stands for StreamingLLM style RoPE, \textit{RT} stands for relative RoPE in hierarchical representative token selection.}
\label{tab:rope_ablation}
\vspace{1.0em}
\centering
% \resizebox{8\linewidth}{!}{
\small
\begin{tabular}{lrrrrr}\toprule
\makecell[r]{RoPE Style in\\Pruning \textbackslash\ SA} &DE &IL &ST &AVG. \\\midrule
DE (Dynamic) 
    &\cellcolor[HTML]{ff8c6c}52.40 
    &\cellcolor[HTML]{ff9c64}54.59 
    &\cellcolor[HTML]{ff8370}51.09 
    &\cellcolor[HTML]{ff8370}52.69 \\
IL (InfLLM)
    &\cellcolor[HTML]{3ab799}68.12
    &\cellcolor[HTML]{5dba8b}66.81 
    &\cellcolor[HTML]{00b1b0}\textbf{70.31} 
    &\cellcolor[HTML]{05b2ae}68.41 \\
CI (Chunk-Indexed)
    &\cellcolor[HTML]{46b895}67.69 
    &\cellcolor[HTML]{5dba8b}66.81 
    &\cellcolor[HTML]{46b894}67.69 
    &\cellcolor[HTML]{26b5a1}67.39 \\
RT (Relative)
    &\cellcolor[HTML]{5dba8b}66.81 
    &\cellcolor[HTML]{2fb69d}68.56 
    &\cellcolor[HTML]{01b2af}\textbf{70.31} 
    &\cellcolor[HTML]{00b1b0}\textbf{68.56} \\
AVG.
    &\cellcolor[HTML]{ff8370}63.76 
    &\cellcolor[HTML]{ffba54}64.19 
    &\cellcolor[HTML]{00b1b0}\textbf{64.85} 
    &- \\
\bottomrule
\end{tabular}
% }
\vspace{-1em}
\end{table}
\textbf{Ablation on RoPE interpolation strategies.}
In \cref{tab:rope_ablation}, we perform an ablation study on the dynamic RoPE extrapolation strategy in masking and sparse attention.
We choose the best-performing RT/ST combination for our method.
\section{Conclusion and future directions} \label{sec:conclusion}

In this paper we proposed a nested MLMC framework that offers important computational savings by performing most calculations in low precision and exploiting approximate random normal variables for the low precision path calculations. The low precision calculations could be performed in fixed precision on an FPGA for greater efficiency, and we suggested a procedure to optimise the bit-widths of every variable at each Monte Carlo level. This is an important improvement over previous mixed precision MLMC frameworks which held the lower precision fixed \cite{Rounding_error_oliver} or defined uniform bit-width at every level heuristically \cite{brugger2014mixed}. Our numerical results suggest that for the first levels our procedure reduces the cost at these levels by a factor 5 or 7. Hence the overall savings are significant since most paths are calculated on the first levels. Our approach would be even more efficient for the Milstein scheme because its higher order strong convergence leads to a greater proportion of the computational costs being on the coarsest levels.

The next stage of the research project will be to implement the RNG methods and the nested framework on FPGAs to determine the hardware requirements and confirm the extent of the computational savings. It would also be good to compare the performance benefits to using half-precision floating point arithmetic on GPUs or CPUs for the low-accuracy computations.




\bibliography{ltexpprt_doublecolumn}
% \newpage
\appendix
\section{Appendix}
\subsection{Metric Optimization}  \label{app:pg}
We utilize the REINFORCE algorithm to optimize the performance metric. The detailed optimization process is proved in the following equations:
    \begin{equation}
        \small
        \begin{aligned}
            &\nabla_{\Lambda}\hat{l}(\Lambda)
            =\nabla_{\Lambda} \mathbb{E}_{s\sim \pi{(\mathcal{B},\cdot;{\Lambda})}} \mathcal{R}(\hat{\mathcal{D}}, f(\Theta^*(\Lambda)))\\
            &=\nabla_{\Lambda}\sum_{s\in[0,1]^{|\mathcal{B}|}} \mathcal{R}(\hat{\mathcal{D}}, f(\Theta^*(\Lambda))) \cdot \pi(\mathcal{B},s;{\Lambda}) \\
            &=\sum_{s\in[0,1]^{|\mathcal{B}|}} \mathcal{R}(\hat{\mathcal{D}}, f(\Theta^*(\Lambda))) \cdot 
            \frac{\nabla_{\Lambda}\pi(\mathcal{B},s;{\Lambda})}{\pi(\mathcal{B},s;{\Lambda})}\cdot \pi(\mathcal{B},s;{\Lambda})\\
            &= \sum_{s\in[0,1]^{|\mathcal{B}|}} \mathcal{R}(\hat{\mathcal{D}}, f(\Theta^*(\Lambda))) \cdot \nabla_{\Lambda}log(\pi(\mathcal{B},s;{\Lambda}))\cdot \pi(\mathcal{B},s;{\Lambda})\\
            &=\mathbb{E}_{s\sim \pi(\mathcal{B},\cdot;{\Lambda})}[\mathcal{R}(\hat{\mathcal{D}}, f(\Theta^*(\Lambda)))\cdot \nabla_{\Lambda}log(\pi(\mathcal{B},s;{\Lambda}))],
        \end{aligned}
    \end{equation}

\subsection{Learning Algorithm}  \label{app:learning_algorithm}
The detailed optimization steps of the proposed framework are given in Algorithm \ref{al:method}.

\subsection{Detail of Studied Methods} \label{app:studied method}
To show the compatibility of our method, we apply the DVR framework on four recommendation backbones, i.e., BRPMF~\cite{koren2009matrix}, NeuMF~\cite{he2017neural}, MGCF~\cite{wang2019neural}, and LightGCN~\cite{he2020lightgcn}. We select BPRMF due to its widespread adoption in recommendation systems and proven effectiveness in practical applications. NeuMF, an MLP-based approach, extends the capabilities of BPRMF by capturing intricate user-item relationships. We leverage GNN-based models, such as MGCF and LightGCN, known for their state-of-the-art performance and competitive outcomes across various techniques, to serve as the recommendation backbone. 

Based on these backbones, different versions of the DVR model are tailored to optimize diverse metrics. For simplicity, we designate models optimized for ranking accuracy as DVR-Loss, DVR-Recall, and DVR-NDCG. Likewise, models focused on diversity and fairness metrics are labeled as DVR-CC, DVR-ILD, and DVR-Gini. 


We compare our framework with various data valuation methods for recommendations. BPR~\cite{10.5555/1795114.1795167} uniformly samples negative items and treats all training data equally in constructing the training objective. TCE-BPR and RCE-BPR are extensions of the TCE and RCE techniques \cite{10.1145/3437963.3441800}, aimed at dynamically filtering out noisy positive interactions during training based on loss values. In our implementation, we replace the original point-wise loss with a pair-wise ranking loss objective to ensure a fair comparison with these methods. AOBPR \cite{10.1145/2556195.2556248} enhances the BPR algorithm by incorporating adaptive sampling techniques that prioritize popular negative items. WBPR \cite{gantner2012personalized} assumes that unexplored popular items by a user are more likely to be true negatives. PRIS \cite{10.1145/3366423.3380187} assigns higher weights to informative negative samples using importance sampling. TIL-UI and TIL-MI \cite{wu2022adapting} learn the data value of training triplets through two aggregation strategies by optimizing the BPR loss within the training batch.

\subsection{Implementation Details} \label{app:implenmentation}
We optimize all models using the Adam optimizer with Xavier initialization \cite{glorot2010understanding} and maintain a fixed embedding size of 64 across all methods. When constructing the ranking loss objective, every positive item is associated with one sampled negative item for an efficient computation. Grid search is applied to choose learning rate and weight decay from $\left\{1e^{-4}, 1e^{-3}, 1e^{-2}, 1e^{-1}\right\}$ and $\left\{1e^{-6}, 1e^{-5}, 1e^{-4}, 1e^{-3}\right\}$. The backbone models NeuMF, MGCCF, and LightGCN utilize the provided implementations, with MGCF and LightGCN featuring two graph convolution layers. The total number of training epochs is set to 2000 for all models with an early stopping design. Given the initial training stages' limited information, we pre-train the recommendation model without data valuator for 1000 epochs to get meaningful embeddings. We set the number of the pre-training epochs to 1000. All experiments are conducted on GPU machines (NVIDIA GeForce RTX 3090).

\begin{algorithm}[H]
    \caption{The Proposed Method}
    \label{al:method}
    
    \textbf{Input:} Learning rates $\alpha$ and $\beta$, outer mini-batch size $B_1$, inner mini-batch size $B_2$, outer iteration count $T_1$, inner iteration count $T_2$, moving average window $W$, training pairs $\mathcal{D}_{1}=\{(u,i)\}_{k=1}^{L_1}$, validation pairs $\mathcal{D}_{2}=\{(u,i)\}_{k=1}^{L_2}$
    
    \textbf{Initialize:} parameters $\Theta$ and $\Lambda$, moving average $\delta=0$
    
    \begin{algorithmic}[1]
    \FOR{outer iteration $t_1=1,2,...,T_1$}
        \STATE Sample a mini-batch from the entire training dataset: $\mathcal{\hat{B}}=(u,i)_{k=1}^{B_1}\sim \mathcal{D}_1$
        \STATE Uniformly sample negative items $(j)_{k=1}^{B_1}$ for training pairs $(u,i)_{k=1}^{B_1}$ to get training data $\mathcal{B}=(u,i,j)_{k=1}^{B_1}$ for the recommendation model
        \FOR{$(u,i,j) \in \mathcal{B}$}
        \STATE Calculate the Shapley value by Eq. (\ref{eq:svcal}) and assign it to $w_{uij}$
        \ENDFOR
        \STATE Normalize the Shapley value $w_{uij}$ within the batch $\mathcal{B}$ as $\hat{w}_{uij}$ 
        \STATE Compute sample selection vector $s_{uij}=\text{Ber}(\hat{w}_{uij})$ from Bernoulli distribution
        \STATE Update the data valuator model by \ref{eq:mse}
        \FOR{inner iteration $t_2=1,2,...,T_2$}
        \STATE Sample a mini-batch $(u,i,j)_{m=1}^{B_2}\sim \mathcal{B}$
        \STATE Update the recommendation model:
    $$\Theta \leftarrow \Theta-\frac{\alpha}{B_2} \sum_{m=1}^{B_2} s_{uij} \cdot \nabla_\Theta \mathcal{L}_{\text{BPR}}(u,i,j;\Theta)$$
        \ENDFOR
    \STATE Update the data valuator:
    $$ \begin{array}{r}
    \Lambda \longleftarrow \Lambda - \beta [\mathcal{R}(\mathcal{D}_2, f(\Theta^*(\Lambda)))-\delta ]\\ \cdot \nabla_{\Lambda}log(\pi(\mathcal{B},(s_{uij})_{k=1}^{B_1};{\Lambda})
    \end{array}
    $$
    \STATE Update the moving average reward:
    $$
    \delta \leftarrow \frac{W-1}{W} \delta+\frac{1}{W} \mathcal{R}(\mathcal{D}_2, f(\Theta^*(\Lambda)))
    $$                               
    \ENDFOR
    \end{algorithmic}
\end{algorithm}






\end{document}

% End of ltexpprt.tex 