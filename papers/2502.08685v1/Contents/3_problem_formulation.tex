\section{Problem Formulation}
In this paper, the recommendation task takes the user behavior data as input. 
Let $\mathcal{U}$ and $\mathcal{V}$ be sets of users and items respectively, where $|\mathcal{U}|= m$, and $|\mathcal{V}|= n$. We use the index $u \in \mathcal{U}$ to denote a user and $i\in \mathcal{V}$ to denote an item. 
The user-item rating matrix is denoted as $\mathbf{R} = [r_i^u]^{m\times n}\in \mathbb{R}^{m\times n}$ to indicate whether user $u$ has interacted with item $i$, where $r_i^u=1$ represents user $u$ has interacted with item $i$,  whereas $r_i^u=0$ represents user $u$ has not interacted with item $i$. 
We use $\mathcal{V}^{+}_{u}=\{i\in\mathcal{V}|r_i^u=1\}$ to represent a set of items that user $u$ has interacted with.
$\mathcal{V}^{+}_{u}$ can be splited into a training set $\mathcal{S}_{u}^{+}$ and a testing set $\mathcal{T}_{u}$, requiring that $\mathcal{S}_{u}^{+} \cup \mathcal{T}_{u} = \mathcal{V}^{+}_{u}$ and $\mathcal{S}_{u}^{+} \cap \mathcal{T}_{u} = \emptyset$. It worth noted that $\mathcal{S}_u^{-}=\{i|r_i^u=0,i\in \mathcal{I}\}$, which means $\mathcal{S}_u^{-}$ consists of the negative items that user $u$ have not interacted with. The training set is denoted as $\mathcal{D}=\{(u,i, j)|u\in \mathcal{U}, i \in \mathcal{S}_{u}^{+}, j \in \mathcal{S}_{u}^{-}\}$. The testing set is denoted as $\mathcal{\hat{D}}=\{(u,i)|u\in \mathcal{U}, i \in \mathcal{T}_{u}\}$.

In the recommendation task, the model aims to recommend a list of $k$ items $\mathcal{X}_u$ for the user $u$, which matches the condition $\mathcal{X}_u\cap \mathcal{S}_u^{+}=\emptyset$. 
By comparing the recommendation list $\mathcal{X}_u$ with the testing set $\mathcal{T}_u$, we evaluate the recommendation quality from various perspectives, including accuracy, diversity, and fairness. 
