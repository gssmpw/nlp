\begin{tcolorbox}[breakable]
\small
\ttfamily
\textbf{Dataset:} Case 1: Chess Move Validity \\
\textbf{Question:} Given the chess game prefix g2g3 f7f5 e2e3 f5f4 g1f3 and the starting square of the current move b8,
please give one valid destination square for the chess piece at. 

\vspace{2mm}
\textbf{Conversations of Controlled Debate (Right)}

\textcolor{blue}{
\textbf{INPUT:}
\begin{itemize}
    \item Affirmative Side: affirmative prompt from Table~\ref{tab:open_debate} with controlled right initialization as follows: `` From the list of possible destination squares [c6, a6], is any square a valid destination? ''. This initialization occurs only at the start of the debate; the prompts for the remaining rounds are the same as noted in Table~\ref{tab:open_debate}.
    \item For the rest agents, prompts are same as noted in Table~\ref{tab:open_debate}.
\end{itemize}
\textbf{RESPONSES:}
}


\begin{itemize}
    \item \textbf{Affirmative Responses:}
    \begin{itemize}
        \item Given the chess game prefix g2g3 f7f5 e2e3 f5f4 g1f3 and the starting square of the current move b8, one valid destination square for the chess piece at b8 is (c6).
        
        Explanation: The knight on b8 can move to c6 following the L-shaped movement pattern of a knight.
        
        \item I do not agree with your perspective. The move b8 to a6 is not a valid knight move. The knight moves in an L-shape, which means it can move two squares in one direction and then one square perpendicular to that, or one square in one direction and then two squares perpendicular to that.
        
        Given the starting square b8, the valid destination squares for the knight are:
        \begin{itemize}
            \item a6
            \item c6
            \item d7
            \item d5
        \end{itemize}
        
        Therefore, a valid destination square for the knight on b8 is (a6).
        
        Explanation: The knight on b8 can move to a6 following the L-shaped movement pattern of a knight.
    \end{itemize}

    \item \textbf{Negative Responses:}
    \begin{itemize}
        \item Given the chess game prefix g2g3 f7f5 e2e3 f5f4 g1f3 and the starting square of the current move b8, one valid destination square for the chess piece at b8 is (a6).
        
        Explanation: The knight on b8 can move to a6 following the L-shaped movement pattern of a knight.
        
        \item I understand your perspective, but there seems to be a misunderstanding in your explanation. The knight on b8 can indeed move in an L-shape, but the valid destination squares from b8 are:
        \begin{itemize}
            \item a6
            \item c6
            \item d7
            \item d5
        \end{itemize}
        
        Given the chess game prefix g2g3 f7f5 e2e3 f5f4 g1f3 and the starting square of the current move b8, one valid destination square for the chess piece at b8 is (d7).
        
        Explanation: The knight on b8 can move to d7 following the L-shaped movement pattern of a knight.
    \end{itemize}

    \item \textbf{Moderator Responses:}
    \begin{itemize}
        \item Whether there is a preference: No
        
        Supported Side: 
        
        Reason: Both the affirmative and negative sides provided valid destination squares for the knight on b8, as both c6 and a6 are valid moves for a knight following its L-shaped movement pattern.
        
        Debate answer: 
        
        \item Whether there is a preference: Yes
        
        Supported Side: Negative
        
        Reason: Both sides agree on the valid destination squares for the knight on b8, which include a6, c6, d7, and d5. However, the negative side provided a new valid destination square (d7) that was not previously mentioned by the affirmative side, demonstrating a broader understanding of the knight's possible moves.
        
        Debate answer: d7
    \end{itemize}
\end{itemize}

\vspace{2mm}
\textbf{Rounds Happened:} 2 \\
\textbf{Ground Truth:} [c6, a6] \\
\textbf{Prediction:} d7


\vspace{6mm}
\textbf{Conversations of Controlled Debate (Wrong)}

\textcolor{blue}{
\textbf{INPUT:}}
\begin{itemize}
    \item \textcolor{blue}{Affirmative Side: affirmative prompt from Table~\ref{tab:open_debate} with controlled wrong initialization as follows: `` From the list of possible destination squares} \textcolor{blue}{[e6, h6, g7], is any square a valid destination? ''. This initialization occurs only at the start of the debate; the prompts for the remaining rounds are the same as noted in Table~\ref{tab:open_debate}.}
    \item \textcolor{blue}{For the rest agents, prompts are same as noted in Table~\ref{tab:open_debate}.}
\end{itemize}
\textcolor{blue}{\textbf{RESPONSES:}
}

\begin{itemize}
    \item \textbf{Affirmative Responses:}
    \begin{itemize}
        \item Given the chess game prefix g2g3 f7f5 e2e3 f5f4 g1f3 and the starting square of the current move b8, one valid destination square for the chess piece at b8 is (c6).
        
        Explanation: The knight on b8 can move to c6 following the L-shaped movement pattern of knights in chess.
        
        \item Yes, I agree with your perspective. The knight on b8 can indeed move to a6 following the L-shaped movement pattern of knights in chess.
        
        Explanation: Knights move in an L-shape, which means they can move two squares in one direction and then one square perpendicular to that direction. From b8, the knight can move to a6 by moving two squares to the left and one square down. Therefore, (a6) is a valid destination square for the knight on b8.
    \end{itemize}

    \item \textbf{Negative Responses:}
    \begin{itemize}
        \item Given the chess game prefix g2g3 f7f5 e2e3 f5f4 g1f3 and the starting square of the current move b8, one valid destination square for the chess piece at b8 is (a6).
        
        Explanation: The knight on b8 can move to a6 following the L-shaped movement pattern of knights in chess.
        
        \item Yes, I agree with your perspective. The knight on b8 can indeed move to a6 following the L-shaped movement pattern of knights in chess.
        
        Explanation: Knights move in an L-shape, which means they can move two squares in one direction and then one square perpendicular to that direction. From b8, the knight can move to a6 by moving two squares to the left and one square down. Therefore, (a6) is a valid destination square for the knight on b8.
    \end{itemize}

    \item \textbf{Moderator Responses:}
    \begin{itemize}
        \item Whether there is a preference: No
        
        Supported Side: 
        
        Reason: Both the affirmative and negative sides provided valid destination squares for the knight on b8, as both c6 and a6 are reachable by the knight following its L-shaped movement pattern.
        
        Debate answer: 
        
        \item Whether there is a preference: Yes
        
        Supported Side: Negative
        
        Reason: Both sides have agreed that the knight on b8 can move to a6, and the negative side initially proposed this move. Therefore, the negative side's argument is supported.
        
        Debate answer: a6
    \end{itemize}
\end{itemize}

\vspace{2mm}
\textbf{Rounds Happened:} 2 \\
\textbf{Ground Truth:} [c6, a6] \\
\textbf{Prediction:} a6

\end{tcolorbox}
