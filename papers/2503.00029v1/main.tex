\documentclass{article}
\usepackage{tencent_ailab_tech_report}


% Language setting
% Replace `english' with e.g. `spanish' to change the document language
% \usepackage[english]{babel}

% Set page size and margins
% Replace `letterpaper' with `a4paper' for UK/EU standard size
% \usepackage[letterpaper,top=2cm,bottom=2cm,left=3cm,right=3cm,marginparwidth=1.75cm]{geometry}

% Useful packages
\usepackage{amsmath}
\usepackage{algorithm}
\usepackage{algorithmic}

\usepackage{graphicx}
\usepackage[colorlinks=true, allcolors=blue]{hyperref}
\newcommand{\fix}{\marginpar{FIX}}
\newcommand{\new}{\marginpar{NEW}}
\usepackage{subcaption}  % 支持子图
\usepackage[utf8]{inputenc} % allow utf-8 input
\usepackage[T1]{fontenc}    % use 8-bit T1 fonts
\usepackage{hyperref}       % hyperlinks
\usepackage{url}            % simple URL typesetting
\usepackage{booktabs}       % professional-quality tables
\usepackage{amsfonts}       % blackboard math symbols
\usepackage{nicefrac}       % compact symbols for 1/2, etc.
\usepackage{microtype}      % microtypography
\usepackage{soul}
\usepackage{multirow}
\usepackage{graphicx}       % graphs
\usepackage{makecell}
\usepackage{hyperref}
\usepackage{wrapfig}
\usepackage{marvosym}
\usepackage[shortlabels]{enumitem}
\definecolor{myblue}{RGB}{215,238,247}
\definecolor{mygreen}{RGB}{230,241,221}
\definecolor{mygrey}{RGB}{242,242,242}
\definecolor{myorange}{RGB}{235,142,71}
\definecolor{textgreen}{RGB}{135,201,195}
\definecolor{mypurple}{RGB}{222,213,255}
\definecolor{darkblue}{RGB}{66,141,191}
\definecolor{greyblue}{RGB}{208,220,232}

\newcommand{\red}[1]{\textcolor{red}{#1}}
\newcommand{\green}[1]{\textcolor{green}{#1}}



\def\AM{{\mathcal A}}
\def\BM{{\mathcal B}}
\def\CM{{\mathcal C}}
\def\DM{{\mathcal D}}
\def\EM{{\mathcal E}}
\def\FM{{\mathcal F}}
\def\GM{{\mathcal G}}
\def\HM{{\mathcal H}}
\def\IM{{\mathcal I}}
\def\JM{{\mathcal J}}
\def\KM{{\mathcal K}}
\def\LM{{\mathcal L}}
\def\MM{{\mathcal M}}
\def\NM{{\mathcal N}}
\def\OM{{\mathcal O}}
\def\PM{{\mathcal P}}
\def\SM{{\mathcal S}}
\def\RM{{\mathcal R}}
\def\TM{{\mathcal T}}
\def\UM{{\mathcal U}}
\def\VM{{\mathcal V}}
\def\WM{{\mathcal W}}
\def\XM{{\mathcal X}}
\def\YM{{\mathcal Y}}
\def\ZM{{\mathcal Z}}
\def\ZB{{\mathbb Z}}
\def\RB{{\mathbb R}}

\def\Name{Cognitive Kernel}
\newcommand{\hmc}[1]{\textcolor{blue}{[HM:#1]}}
\newcommand{\hm}[1]{\textcolor{blue}{#1}}
\newcommand{\xmc}[1]{\textcolor{orange}{[XM:#1]}}
\newcommand{\km}[1]{\textcolor{green}{KX:#1]}}
\newcommand{\wh}[1]{\textcolor{red}{WH:#1]}}

% \title{Cognitive \ Kernel: \ A \ Dockerized \ ``Autopilot'' \ System \ that \\ Everyone \ Can \ Deploy \ Privately}

\title{Streaming Looking Ahead with Token-level Self-reward}


% Your Open-source AI Employee for Completing Real-World Tasks -- Everyone can Benefit!

\author{Hongming Zhang$^1$, Ruixin Hong$^{1,2}$, Dong Yu$^1$\\
$^1$Tencent AI Lab, Seattle, $^2$Tsinghua University \\
\textit{hongmzhang@global.tencent.com, hrx20@mails.tsinghua.edu.cn, dyu@global.tencent.com}}

\begin{document}
\maketitle

\begin{abstract}

Autoregressive decoding algorithms that use only past information often cannot guarantee the best performance. 
Recently, people discovered that looking-ahead algorithms such as Monte Carlo Tree Search (MCTS) with external reward models (RMs) can significantly improve models' output by allowing them to think ahead and leverage future outputs and associated rewards to guide the current generation.
Such techniques can help the reinforcement fine-tuning phase by sampling better trajectories and the inference phase by selecting the better output.
However, their high computational cost limits their applications, especially in streaming scenarios.
To address this issue, we propose equipping the policy model with token-level self-reward modeling (TRM) capability to eliminate the need for external models and extra communication.
We name the new architecture as \textit{Reward Transformer}.
In addition, we propose a streaming-looking-ahead (\textit{SLA}) algorithm to further boost search efficiency with better parallelization.
Experiments show that \textit{SLA} achieves an overall win rate of 79.7\% against the baseline greedy decoding algorithm on three general-domain datasets with a frozen policy model while maintaining streaming efficiency.
If we combine \textit{SLA} with reinforcement fine-tuning techniques such as \textit{DPO}, \textit{SLA} achieves an overall win rate of 89.4\%. We release the experiment code at: \url{https://github.com/CognitiveKernel/SLA}.
% We propose a novel architecture reward transformer and an associated token-level Bradley-Terry loss to achieve TRM from sequence-level distant reward annotation to achieve this goal.
% In addition, we propose a streaming-looking-ahead (SLA) algorithm to help the model ``look ahead,'' evaluate future rewards, and aggregate future rewards, enabling better decision-making during output generation in a streaming paradigm.
% Experiments show that SLA achieves an overall 62.5\% win rate against the baseline greedy decoding algorithm on three representative datasets with a freezer policy model while maintaining streaming efficiency.
% If we combine SLA with DPO, SLA could achieve an overall 82.9\% win rate.
% We release the experiment code at \red{xxx}.


\end{abstract}

\begin{figure}[ht]
    \centering
    \includegraphics[width=0.8\linewidth]{graphs/greater_than_naive.pdf}
    \vspace{0.5cm}
    \includegraphics[width=0.8\linewidth]{graphs/p1_bottom.png}
    \vspace{-5pt}
    \caption{\textcolor{positional}{Positional} vs.\ \textcolor{nonpositional}{non-positional} circuits. In a \textcolor{nonpositional}{non-positional} circuit, the same edges must be included at all positions. A \textcolor{positional}{positional} circuit can distinguish between the same edge at different positions. This specificity yields better trade-offs between circuit size and faithfulness. It can also increase both precision and recall.}
    \label{fig:p1}
    \vspace{-5pt}
\end{figure}

\section{Introduction}

\looseness=-1
A primary goal of interpretability research is to characterize the internal mechanisms in language models (LMs) and other NLP models. 
A core approach in this area is \textbf{circuit discovery}---identifying the minimal subgraph within the model's computation graph that performs a specific task \citep{olah2021framework,olah-mech}.
Typically, the nodes of a circuit represent model components (e.g., attention heads, neurons, or layers).
While manual circuit discovery methods can yield position-specific insights \citep{wanginterpretability,goldowskydill2023localizingmodelbehaviorpath}, \emph{automatic methods often overlook positional information}, treating components as uniformly relevant across all input token positions \citep{conmytowards,syed2023attribution}. 
For instance, if an attention head is included in a circuit, it is assumed to contribute equally to the computation for every position in the input sequence.
The assumption that circuits are position-invariant ignores the fact that different positions often require distinct computations.
By ignoring positions, current methods limit their ability to capture mechanisms that operate across positions, such as interactions between attention heads across positions.

In this study, we start by demonstrating that positional agnosticism is a significant limitation (\S\ref{sec:motivating}). Then, to address these limitations, we introduce a new approach: position-aware edge attribution patching (PEAP; \S\ref{sec:full_circ_discovery}; Figure~\ref{fig:p1}). Current approaches  assume that if an edge is in a circuit, then the same edge will be in the circuit at all positions, thus leading to low precision. It is also assumed that an edge's importance should be aggregated across positions before deciding whether it should be included in the circuit; this can lead to cancellation effects, and thus low recall. PEAP instead allows us to compute the importance of cross-positional edges, and separately evaluates edge importance at each position. We show that this leads to smaller and more accurate circuits; see Figure~\ref{fig:p1}.

Incorporating positional information into circuit discovery is straightforward when inputs have the same length and structure across examples.

However, realistic datasets are not nearly this templatic.
How, then, can we incorporate positional information into automatic circuit discovery?
To address this challenge, we propose \textbf{schemas} (\S\ref{sec:schema}). 
Schemas assign semantic labels to spans of tokens, enabling information aggregation across examples even when the spans differ in length.

For example, in the input ``The \textcolor{positional}{war} lasted from 1453 to 14\underline{\hspace{1em}},'' the span ``\textcolor{positional}{war}'' could be labeled as ``\emph{Subject}''.
This enables handling spans with varying lengths: the phrase ``\textcolor{positional}{Black Plague}'' in another example can be treated as a single positional span with the same role as ``\textcolor{positional}{war}''.
In experiments with two LMs and three tasks, we find that circuits discovered using schemas achieve a better trade-off between circuit size and faithfulness to the model's behavior than position-agnostic circuits.
Importantly, position-aware circuits offer a more precise representation of the underlying mechanisms, providing a more concise foundation for mechanistic explanations.

We also present a fully automated pipeline for schema generation and application (\S\ref{sec:schema-generation}) using large language models (LLMs). 
We evaluate the quality of the generated schemas and their utility in discovering position-aware circuits (\S\ref{sec:schema-eval}).
Notably, circuits derived using automatically generated and applied schemas achieve comparable faithfulness scores to circuits discovered with human-designed and manually applied schemas.

We summarize our contributions as follows:
\begin{itemize}[noitemsep,leftmargin=*,topsep=1pt,parsep=1pt]
    \item Introduce a position-aware circuit discovery method, which obtains better faithfulness than position-agnostic discovery.  
    \item Introduce dataset schemas,  facilitating positional circuit discovery in more naturalistic settings. 
    \item Develop an automated schema generation and application pipeline with LLMs, yielding schemas that are comparable to manually-annotated ones.
\end{itemize}

\section{Preliminaries}\label{sec:problem_formulation}

% \begin{table*}[h!]
% \centering
% \caption{Comparison of Algorithms}
% \label{tab:algorithm_comparison}
% \begin{tabular}{l|ccccc}
% \toprule
% \textbf{Algorithm Name} & \textbf{Performance} & \textbf{Diversity} & \textbf{Generalization} & \textbf{Efficiency} & \textbf{Streaming}\\
% \midrule
% Greedy-decoding & Moderate  &   Low  & High          & High      & Yes \\
% Decoding with Temperature & Low & High & High              & High & Yes   \\
% Top-K Sampling & Moderate      & High & High               & High   & Yes     \\
% Top-P Sampling & Moderate     & High & High              & High   & Yes      \\
% Beam-Search & Moderate     & Moderate & High              & Moderate    & Yes     \\
% Majority Voting & Moderate     & High & Low              & Moderate     & No    \\
% RM Selection & Moderate     & High & High              & Moderate    & No     \\
% RM-guided Tree Search & High     & High & Low              & Low    & No    \\
% \bottomrule
% \end{tabular}
% \end{table*}
% The language model generation process generally selects a sequence of tokens following certain algorithms (e.g., greedy or sampling methods) until a stopping criterion, such as an end-of-sequence token or maximum sequence length, is reached.

In this section, we first introduce how the LM generation process can be formulated as a token-level Markov Decision Process (MDP) and then explain how existing sampling algorithms relate to it.

\subsection{LLM Decoding as Token-level MDP}

The language model generation process takes a sequence of tokens as inputs and generates a sequence of tokens as outputs.
Mainstream transformer-based language models generate the output tokens one by one until the stopping criteria (e.g., an end-of-sequence token or maximum sequence length) are met.
The traditional MDP is usually formulated as a tuple $\mathcal{M} = (\mathcal{S}, \mathcal{A}, F, R, \gamma)$, where $\mathcal{S}$ is the set of all possible states, $\mathcal{A}$ is the set of actions, $F$ is the transition function, $R$ is the reward function, and $\gamma$ is the decay parameter.
In the language model scenarios, each state in $\mathcal{S}$ is a trajectory that can be denoted as $\tau$.
Each action in $\mathcal{A}$ is selecting a token $x$ from the vocabulary set.
$F$ is the deterministic transition of concatenating the selected action (i.e., a token) with the existing state (i.e., a trajectory) to become a new one.
Traditionally, rewards $r_t$ are defined at every step $t$ and contribute to the return $G_t=\sum_{k=0}^{\infty} \gamma^k R_{t+k}$ through the decay factor $\gamma$.
However, in the LLM scenario, we are only concerned with the quality of the complete trajectory generated, meaning that the reward function $R(s)$ evaluates the final trajectory rather than providing step-by-step feedback.
Thus, the return becomes $G=R_T$, where $R_T$ is the reward associated with the final sequence at step $T$, and $\gamma$ is irrelevant because intermediate rewards are not accumulated. 

\subsection{The Classical Decoding Algorithms}

Formally, given an input \( \mathbf{x}  = (x_1, x_2, \ldots, x_T) \), a reward function $R$ that provides a scalar reward for a trajectory $\tau$, and a language model \( p_\theta(x) \) parameterized by \( \theta \), the goal of decoding algorithms is to find the optimal trajectory $x^\star$ sampled from \( p_\theta(x) \) that could maximize the reward:

\[
\tau^* = \arg\max_{\tau \sim p_\theta(\mathbf{x} )} R(\tau).
\]

% Assumes that the input is a token sequence with $a$ and the foundation language model is a probabilistic model of predicting the likelihood of the next token given previous ones, which is usually formulated as $P(x \mid x_{<i})$, the goal of language model decoding is to utilize this likelihood to get the output trajectory of length $T$ that achieves the highest $R_T$.
This section covers representative decoding algorithms and explains how they are connected.

\textbf{Greedy Decoding}: The naive but most widely used algorithm is \textit{Greedy Decoding}, which uses the language modeling likelihood at each step as guidance.
At each step $i$, this algorithm selects the action token $x_i \in \mathcal{A}$ following:
\begin{equation}
x_i = \arg\max_{x} P(x \mid x_{<i}).
\end{equation}
From the angle of MDP, this method uses the accumulative likelihood predicted by the language as the final reward:
\begin{equation}
 R(\tau) \gets \Pi_{i}^{T}P(x_i \mid x_{<i}),
\end{equation}
where $T$ is the length of $\tau$, and takes a greedy solution to approach this goal.


\textbf{Sampling-based Decoding:} 
On top of greedy decoding, people also try to incorporate diversity in the final output.
For example, the temperature-based method introduces an additional parameter $\lambda$ to control the greedy sampling process by reshaping the likelihood distribution as:
\begin{equation}
x_i \sim P(x \mid x_{<i})^{1/\lambda}.
\end{equation}
From the angle of token-level MDP, we can reinterpret this process as introducing an additional diversity objective:

\begin{equation}
 R(\tau) \gets \Pi_{i}^{T}P(x_i \mid x_{<i}) \cdot D(x_i, x_{<i}),
\end{equation}
where 
\begin{equation}
    D(x_i , x_{<i}) = P(x_i \mid x_{<i})^{\frac{1}{\lambda}-1}.
\end{equation}


To avoid sampling rare tokens and achieve a balance between performance and diversity, researchers have investigated how to dynamically adjust the candidate token pool~\citep{holtzman2019curious,zarriess2021decoding}. 
For example, the \textit{Top-$k$ Sampling} algorithm only considers the top $k$ tokens with the highest probabilities as candidates instead of the whole vocabulary.
Similarly, the \textit{Nucleus Sampling}, which is also known as \textit{Top-p Sampling}, only selects from the smallest possible set $\mathcal{V}_p \subseteq \mathcal{V}$, where the cumulative probability mass exceeds a threshold $p$.

% \paragraph{Nucleus Sampling:} This strategy samples tokens from the smallest possible set $\mathcal{V}_p \subseteq \mathcal{V}$, where the cumulative probability mass exceeds a threshold $p$. Formally,
% \begin{equation}
%     \mathcal{V}_p = \{x_i \mid \sum_{x_j \in \mathcal{V}_p} p(x_j \mid \mathbf{x}_{<t}) \geq p\}.
% \end{equation}

\textbf{Trajectory-level Decoding:} Although these token-level decoding algorithms are efficient, they tend to generate locally coherent outputs that may lack global quality.
To solve this problem, people also developed decoding algorithms that consider partial or whole trajectories.
For example, the \textit{Beam Search Decoding} algorithm keeps track of the top $B$ partial trajectories, expanding them at each step and retaining only the ones with the highest joint likelihood.
Similar to the \textit{Greedy Decoding}, this method also uses the joint likelihood as the trajectory reward function.

\textbf{Advanced Reward Modeling Algorithms:}
A common limitation of the aforementioned algorithms is their fundamental assumption that the joint likelihood could represent $R(\tau)$ might not always hold.
People have been interested in introducing better reward signals as guidance to address this.
For example, in the QA scenario, the \textit{Majority-voting algorithm} assumes that the more frequent answer aligns better with the grounding reward function (i.e., accuracy) and thus selects candidate trajectories following this guidance.
Though this intuitive approach has been shown to be effective on tasks such as QA and math problems, it is restricted to tasks with structured output for voting. It cannot be generalized to more general-purpose applications.
To address this issue, researchers also include an external model $R^\prime$, which is often another transformer-based model, to approximate the ground truth reward model $R$. 
With that, we could sample $K$ trajectories $\mathcal{T}_K$ with sampling-based decoding algorithms and then use $R$ to select the trajectory with the maximum reward:
\begin{equation}
    \tau^\star = \arg \max_{\tau \in \mathcal{T}} R^\prime(\tau).
\end{equation}
Employing an external model to model the reward offers greater flexibility than heuristic rewards. This approach is not constrained by the structured answer format, which improves generality and adaptability in various scenarios. 








\section{Looking Ahead Search Algorithms}
\label{sec:MCTS}


A critical limitation of the aforementioned approaches is that they rely solely on past information, which may not be sufficiently informative for making a wise decision. 
To address this, researchers try to enable the model to look ahead and revisit its choices to make a more informed decision. 
This methodology mirrors how humans plan ahead before making a decision.
Such ideas were widely used in the traditional RL tasks such as the GO game~\citep{silver2016mastering,silver2017mastering} and one of the most widely used algorithms is the Monte Carlo Tree Search (MCTS)~\citep{metropolis1949monte}.

As shown in Algorithm \autoref{algorithm:MCTS}, when making a move, a typical MCTS algorithm typically involves the following steps: (1) selection: following the UCB policy to find a leaf node; (2) expansion: expand it if the located node is not the final state; (3) simulation: calculate the reward for the current state; (4) backpropagation: update the Q-value and visit count for all previous nodes.
This expand-simulation-backpropagation procedure is essentially a way of looking ahead and using future information for the current decision.
Typically, we repeat this procedure for $N$ times/rollouts and then decide based on the future rewards collected.


Recently, MCTS has been introduced into the LLM scenario for both the training and inference stages~\citep{zhang2022efficient,xie2024monte,zhang2024rest,wang2024towards,liu2024don}.
During training, MCTS was known as a power algorithm for sampling good responses, which can be used to optimize the model.
On the other hand, search methods like MCTS have also been proven to be a powerful inference algorithm for improving the model's performance on complex tasks~\citep{feng2023alphazero,lightman2023let}.
However, as the Deepseek technical report~\citep{deepseekai2025deepseekr1incentivizingreasoningcapability} discussed, efficiency is still the Achilles' heel of applying MCTS in LLM.
% The main time cost comes from the reward modeling part. 

\begin{algorithm}[t]
\small
\caption{Monte-Carlo Tree Search (Single Step)}
\label{algorithm:MCTS}
\textbf{INPUT}: Policy Model $P$, Reward Model $R$, Root Node $s_0$, Max Iterations $N$ \\
\textbf{OUTPUT}: Optimal Action $a^*$ \\
\begin{algorithmic}[1]
    \STATE \textbf{INITIALIZE}: Create a search tree with root node $s_0$ and initialize $Q(s, a) \leftarrow 0$, $N(s, a) \leftarrow 0$ for all states and actions.
    \FOR{$i = 1$ TO $N$}
        \STATE \textbf{SELECTION}: Start at $s_0$, traverse the tree by choosing child nodes using the UCB policy until a leaf node $s_L$ is reached.

        \STATE \textbf{EXPANSION}: if $s_L$ is not a terminal state, add a child node $s_{L+1}$ for each possible action $a$ and estimate the prior probability $P(s_{L+1}, a)$.

        \STATE \textbf{SIMULATION (ROLLOUT)}: Calculate the cumulative reward $r$ over the trajectory: $r = \sum_{t=0}^{T} R(s_t, a_t)$.
        
        \STATE \textbf{BACKPROPAGATION}: For each node $(s_t, a_t)$ along the path from $s_L$ to $s_0$, Update $Q$-value and visit count.
        % \FOR{each node $(s_t, a_t)$ along the path from $s_L$ to $s_0$}
        %     \STATE Update $Q$-value and visit counts:
        %     \[
        %     Q(s_t, a_t) \leftarrow \frac{N(s_t, a_t) \cdot Q(s_t, a_t) + r}{N(s_t, a_t) + 1}
        %     \]
        %     \STATE Increment visit count:
        %     \[
        %     N(s_t, a_t) \leftarrow N(s_t, a_t) + 1
        %     \]
        % \ENDFOR
    \ENDFOR

    \STATE \textbf{RETURN}: $a^* = \arg\max_{a} Q(s_0, a)$
    % \[
    % a^* = \arg\max_{a} Q(s_0, a)
    % \]
\end{algorithmic}


\end{algorithm}

If we use $N$ and $n$ to indicate the number of sampled trajectories and number of tokens per action.\footnote{We use ``action'' because different algorithms might use different granularities such as tokens and sentences.}
To select each action, the algorithm first expands trajectories, collects rewards for each sampled trajectory, backpropagates, and selects the action.
Since the main time costs in the large language model scenario are related to language model computing and communication, we ignore another time cost for simplicity.
For each action, the computation time complexity is
\begin{equation}
   O( N \cdot (n \cdot t_{d} + 2 \cdot t_{c} + t_{r})),
\end{equation}
where $t_d$, $t_c$, and $t_r$ are the time cost for the policy model to decode a token, communication between two models, and the reward model to generate a numerical score.
And then, if we use $t_p$ to indicate the prefilling time and $T_{max}$ as the maximum trajectory length, the total time cost will become:
\begin{equation}
   O( t_p + \frac{T_{max}}{n} \cdot N \cdot (n \cdot t_{d} + 2 \cdot t_{c} + t_{r})).
\end{equation}
% With the current infrastructure, since $t_d$ is usually way smaller than $t_c$ and $t_r$, the overall compute is bounded by the reward part.
% \begin{equation}
%    O( t_p + \frac{T_{max}}{n} \cdot N \cdot (2 \cdot t_{c} + t_{r})).
% \end{equation}
Given that people often use another LM with the same or larger size as the reward model, $t_r$ is often large.
In actual applications, people reduce this complexity by choosing a relatively larger $n$ and defining each action at coarser granularity, such as a sentence.
This trick makes the MCTS search slightly more affordable but also restricts the generalization capability.
 \section{Method}
\label{sec:method}











Given a set $\{x_{1_i},c_i\}_{i=1}^m$ of input samples and their corresponding conditioning states, our goal is to construct a flow-matching model that samples from $q(x_1|c)$ that start from our conditional prior distribution (CPD). 

\subsection{Flow Matching from Conditional Prior Distribution}
\label{sec:conditional_fm_joint}

We generalize the framework of  Sec.~\ref{sec:flow_matching} to a construction that uses an arbitrary conditional joint distribution of $\rho(x_0, x_1, c)$ which satisfy the marginal constraints:
\begin{equation*}
\label{eq:conditional_marginal}
    \int \rho(x_0, x_1, c)dx_0 = q(x_1, c),  \int \rho(x_0, x_1, c)dx_1dc = p(x_0)
\end{equation*}
Then, building on flow matching, we propose to modify the conditional probability path so that at $t=0$, we define:
\begin{equation}
    \rho_0(x_0|x_1, c) = p(x_0|x_1, c) 
\end{equation}
where $p(x_0|x_1, c)$ is the conditional distribution $\frac{\rho(x_0, x_1, c)}{q(x_1, c)}$. 
Using this construction, we satisfy the boundary condition of Eq.~\ref{eq:boundary_conditions}: 
\begin{align}
    \rho_0(x_0) &= \int\rho_0(x_0|x_1, c)q(x_1, c)dx_1dc  \\
                &=  \int p(x_0|x_1, c)dx_1dc = p(x_0)
\end{align}




The conditional probability path $\rho_t(x|x_1, c)$ does not need to be explicitly formulated. Instead, only its corresponding conditional vector field $u_t(x|x_1, c)$ needs to be defined such that points $x_0$ drawn from the conditional prior distribution $\rho_0(x_0|x_1, c) $, reach $x_1$ at $t=1$, i.e., reach distribution $\rho_1(x|x_1, c) = \delta(x - x_1)$.  We thus purpose the \emph{Conditional Generation Joint FM} $\gL_{\rm cgjfm}(\theta)$ objective:
\begin{equation}\label{eq:conditionl_joint_cfm}
    \mathbb{E}_{t\sim \mathcal{U}(0,1), q(x_0,x_1,c)} \|v_\theta(t, x, c) - u_t(x | x_1, c)\|^2
\end{equation}
where $x = \psi_t(x_0|x_1,c)$.
Training only involves sampling from $q(x_0,x_1,c)$ and does not require explicitly defining the densities $q(x_0,x_1,c)$ and $\rho_t(x|x_1,c)$.
We note that this objective is reduced to the CGFM objective Eq.~\ref{eq_conditional_generative_fm_objective} when $q(x_0,x_1,c) = q(x_1, c)p(x_0)$.

\subsection{Conditional Prior Distribution}
\label{sec:prior_distribution}

We now describe our choice of a condition-specific prior distribution. 
When choosing a conditional prior distribution we want to adhere to the following design principles:
(i) \emph{Easy to sample}: can be efficiently sampled from.
(ii) Well represents the target conditional modes. 
We design a condition-specific prior distribution based on a parametric \emph{Mixture Model} where each mode of the mixture is correlated to a specific conditional distribution $p(x_1|c)$. 
Specifically, we choose the prior distribution to be the following, \emph{easy to sample}, \emph{Gaussian Mixture Model} (GMM):
\begin{equation}\label{eq:gmm_formula}
    p_0 = \mathrm{GMM}(\gN(\mu_i, \Sigma_i)_{i=1}^n, \pi)
\end{equation}

where $\pi\in\R^n$ is a probability vector associated with the number of conditions $n$ (could be $\infty$) and $\mu_i, \Sigma_i$ are parameters determined by the conditional distribution $q(x_1|c_i)$ statistics, \emph{i.e.} 
 \begin{equation}\label{eq:gmm_parameters}
     \mu_i = \E[x_1|c_i], \quad \Sigma_i = \mathrm{cov}[x_1|c_i]
 \end{equation}
To sample from the marginal distribution $p(x_0|x_1, c_i)$, we sample from the cluster $\gN(\mu_i, \Sigma_i)$ associated with the condition $c_i$.

\noindent \textbf{Obtaining a Lower Global Truncation Error.} \quad 
Our CPD fits a GMM to the data distribution in a favorable setting, where the association between samples and clusters is given. 
\begin{equation}\label{eq:wasserstein_definition}
    d_1 \left(X, Y \right) \coloneqq \sup_{h \in \mathrm{Lip_1}} \mathbb{E}[h(X) - h(Y)] .
\end{equation}

In this process, we fit a dedicated Gaussian distribution to data points with the same condition. If the latter are close to being unimodal, this approximation is expected to be tight, in terms of the average distances between samples from the condition data mode and the fitted Gaussian. 
Tab.~\ref{tab:wasserstein_table} provides the average distances between pairs of samples from the prior and data distributions (i.e. the \emph{transport cost}) of CondOT~\cite{lipman2022flow}, BatchOT~\cite{pooladian2023multisample} and our CPD over the ImageNet-64~\cite{deng2009imagenet} and MS-COCO~\cite{lin2014microsoft} datasets. 
As expected, BatchOT which minimizes this exact measure within mini-batches, obtains better scores than the naïve pairing used in CondOT, while our CPD, which approximates the data using a GMM exploits the conditioning available in these datasets, and obtains considerably lower average distances.

As noted in \cite{pooladian2023multisample}, lower transport cost is generally associated with straighter flow trajectories, more efficient sampling and lower training time. We want to substantiate this claim from the viewpoint of cumulative errors in numerical integration.
Sampling from flow-based models consists of solving a time-dependent ODE of the form $\dot{x}_t =u_t(x_t)$, where $u_t$ is the velocity field. This equation is solved by the following integral $x_t = \int_{0}^t u_s(x_s)ds$, where the initial condition $x_0 $ is sampled from the prior distribution. Numerical integration over discrete time steps accumulate an error at each step $n$ which is known as the \emph{local truncation error $\tau_n$}, which accumulates into what is known as the \emph{global truncation error $e_n$}.  This error is bounded by ~\cite{suli2003introduction}
\begin{equation}
    |e_n| \leq \frac{max_j\tau_j}{hL}\big(e^{L(t_n-t_0)} - 1\big)
\end{equation}\label{eq:truncation_error_bound} 
where $h$ is the step size and $L$ is the Lipschitz constant of the velocity $u_t$. 
Accordingly, the distance between the endpoints of a path $\Delta = |x_1  - x_0|$  is given by $|\int_0^1 u_s(x_s)ds|$ which can be interpreted as the magnitude of the average velocity along the path $x_t$. Hence, the longer the path $\Delta$ is, the larger the integrated flow vector field $u_t$ is.
For example, if we scale a path uniformly by a factor $C>1$, i.e., $x_t \rightarrow C(x_t)$, we get,  $\frac{d}{dt}C(x_t) = C(u_t)$ in which case the Lipschitz constant $L$ is also multiplied by $C$.

By shortening the distance between the prior and and data distribution, as our CPD does, we lower the integration errors which permits the use of coarser integration steps, which in turn yield smaller global errors. Thus, our construction allows for fewer integration steps during sampling.

\subsubsection{Construction}


Next, we explain how we construct $p_0$ (Eq.~\ref{eq:gmm_formula}) for both the discrete case (e.g., class conditional generation) and continuous case (e.g., text conditional generation). 

\noindent \textbf{Discrete Condition.} \quad
In the setup of discrete conditional generation, we are given data $\{x_{1_i}, c_i\}_{i=1}^m$ where there are a finite set of conditions $c_i$.
We approximate the statistics of Eq.~\ref{eq:gmm_parameters} using the training data statistics. That is, we compute the mean and covariance matrix of each class (potentially in some latent represntation of a pretrained auto-encoder).  Since the classes at inference time are the same as in training, we use the same statistics at inference. 

\noindent \textbf{Continuous Condition.} \quad
While in the discrete case we can directly approximate the statistics in Eq.~\ref{eq:gmm_parameters} from the training data, in the continuous case (\emph{e.g.} text-conditional) we need to find those statistics also for conditions that were not seen during training. To this end, we first consider a joint representation space for training samples $\{x_{1_i}, c_i\}_{i=1}^m$, which represents the semantic distances between the conditions $c_i$ and the samples $x_{1_i}$. In the setting where $c_i$ is text, we choose a pretrained CLIP embedding. 
$c_i$ is then mapped to this representation space, and then mapped to the 
data space (which could be a latent representation of an auto-encoder), using a learned mapper $\gP_\theta$. 
Specifically, $\gP_\theta$ is trained to minimize the objective:
\begin{equation}
    \gL_{\rm prior}(\theta) = \mathbb{E}_{q(x_1,c)} \|\gP_\theta(E(c)) - x_1\|^2_2.
\end{equation}
where $E$ is the pre-trained mapping to the joint condition-sample space (e.g. CLIP). $\gP_\theta$ can be seen as approximating $\E[x_1|c]$, which is used as the mean for the condition specific Gaussian.  
At inference, where new conditions (e.g., texts) may appear, we first encode the condition $c_i$ to the joint representation space (e.g., CLIP) followed by $\gP_\theta$. This mapping provides us with the center $\mu_i$ of each Gaussian. %
We also define $\Sigma_i = \sigma_i^2\mathrm{I}$ where $\sigma_i$ is a hyper-parameter, ablated in Sec.~\ref{sec:results_quantitative} 

\subsection{Training and Inference}

Given the prior $p_0$ (either using the data statistics or by training $\gP_\theta$), for each condition $c$, we have its associated Gaussian parameters $\mu_c$ and $\Sigma_c$. The map $\psi_t(x|x_1,c)$ must be defined in order to minimize Eq.~\ref{eq:conditionl_joint_cfm} above. This corresponds to the interpolating maps between this Gaussian at $t=0$ and a small Gaussian around $x_1$ at $t=1$, defined by:
\begin{align}
    \psi_{t}(x|x_1,c) &= \sigma_t(x_1,c)x + \mu_t(x_1,c), \\ 
    \sigma_t(x_1,c) &= t (\sigma_{\min} \mathrm{I}) + (1-t)\Sigma_{c}^{1/2}, \quad \text{and} \\
    \mu_t(x_1,c) &= t x_1 + (1-t) \mu_c.
\end{align}
This results in the following target flow vector field 
\begin{equation*}
    u_t(\psi_{t}(x|x_1,c)) = \frac{d}{dt}\psi_t (x|x_1,c)  =   \big(\sigma_{\min}  \mathrm{I} - \Sigma_c^{1/2}\big)x +  x_1 - \mu_c.
\end{equation*}

During inference we are given a condition $c$ and want to sample from $q(x_1|c)$. Similarly to the training, we sample $x_0\sim p(x_0|c)$ and solve the ODE 
\begin{equation}
    \frac{d}{dt} \psi_t(x) = v_\theta \left(t, \psi_t(x), c \right), \quad \psi_0(x) = x_0
\end{equation}
Training and implementation details are in the appendix.








\section{Experiments}
\label{sec:experiment}

\subsection{Experimental Setup}
\label{sec:exp_setup}
The experiments are mainly conducted on SD1.5 \cite{sd1} and SDXL \cite{sdxl} without refiner. The LRM is first trained on Pick-a-Pic and then used to fine-tune diffusion models through LPO. Unless otherwise specified, we employ \textit{homogeneous optimization}.

\textbf{LRM Training.} We denote the LRM based on SD1.5 and SDXL as LRM-1.5 and LRM-XL, respectively. They are trained on the filtered Pick-a-Pic v1 \cite{pickscore} as clarified in Sec.\;\ref{sec:lrm_train}. The $gs$ in the VFE module is set to 7.5. 
More details are in \cref{sec:experimental_detail}.

\textbf{LPO Training.} The same 4k prompts in SPO are used for the LPO training, randomly sampled from the training set of Pick-a-Pic v1. The DDIM scheduler \cite{ddim} with 20 inference steps is employed. We use all steps for sampling and training, \ie $t\in[0,50,...,900,950]$. The dynamic threshold range $[th_{min}, th_{max}]$ is set to $[0.35, 0.5]$ for SD1.5 and $[0.45, 0.6]$ for SDXL. The $\beta$ in Eqn.\;(\ref{eq:spo_loss}) is set to 500 and the $K$ in the sampling process is set to 4. Further details can be found in \cref{sec:experimental_detail}.

\begin{table}[t]
    \centering
    \vspace{-2.5mm}
    \caption{General and aesthetic preference scores on Pick-a-Pic validation unique set. $^*$ denotes the metrics are copied from \cite{spo}. Others are evaluated using the official model.}
    \vskip 0.05in
    \label{tab:preferenece_eval}
    \scriptsize
    \setlength{\tabcolsep}{1.0mm}{
    \scalebox{1.1}{
    \begin{tabular}{l c c c c c}
         \toprule
         Method & PickScore & ImageReward & HPSv2 & HPSv2.1 & Aesthetic \\
         \midrule
         \textcolor{gray}{SD1.5} & & & & & \\
         \hspace{1pt} Original & 20.56 & 0.0076 & 26.46 & 24.05 & 5.468 \\
         \hspace{1pt} $^*$DDPO & 21.06 & 0.0817 & - & 24.91 & 5.591 \\
         \hspace{1pt} $^*$D3PO & 20.76 & -0.1235 & - & 23.97 & 5.527 \\
         \hspace{1pt} Diff.-DPO & 20.99 & 0.3020 & 27.03 & 25.54 & 5.595 \\
         \hspace{1pt} SPO & 21.22 & 0.1678 & 26.73 & 25.83 & 5.927 \\
         \rowcolor{cyan!15}\hspace{1pt} LPO & \textbf{21.69} & \textbf{0.6588} & \textbf{27.64} & \textbf{27.86} & \textbf{5.945} \\
         \midrule
         \textcolor{gray}{SDXL} & & & & & \\
         \hspace{1pt} Original & 21.65 & 0.4780 & 27.06 & 26.05 & 5.920 \\
         \hspace{1pt} Diff.-DPO & 22.22 & 0.8527 & 28.10 & 28.47 & 5.939 \\
         \hspace{1pt} MaPO & 21.89 & 0.7660 & 27.61 & 27.44 & 6.095 \\
         \hspace{1pt} SPO & 22.70 & 0.9951 & 28.42 & 31.15 & 6.343 \\
         \rowcolor{cyan!15}\hspace{1pt} LPO & \textbf{22.86} & \textbf{1.2166} & \textbf{28.96} & \textbf{31.89} & \textbf{6.360} \\
         \bottomrule
    \end{tabular}}}
    % \vspace{-3mm}
    \vskip -0.15in
\end{table}


\begin{table*}[t]
    \vspace{-2.5mm}
    \caption{Quantitative results on T2I-CompBench++ \cite{t2i_compbench}.}
    \vskip 0.05in
    \label{tab:t2i_eval}
    \centering
    \scriptsize
    \setlength{\tabcolsep}{1.8mm}{
    \scalebox{1.1}{
    \begin{tabular}{c l c c c c c c c c}
         \toprule
         Model & Method & Color & Shape & Texture & 2D-Spatial & 3D-Spatial & Numeracy & Non-Spatial & Complex \\
         \midrule
         \multirow{4}{*}{SD1.5} & Original \cite{sd1} & 0.3783 & 0.3616 & 0.4172 & 0.1230 & 0.2967 & 0.4485 & 0.3104 & 0.2999 \\
         & Diff.-DPO \cite{diffusion_dpo} & 0.4090 & 0.3664 & 0.4253 & 0.1336 & 0.3124 & 0.4543 & \textbf{0.3115} & 0.3042 \\
         & SPO \cite{spo} & 0.4112 & 0.4019 & 0.4044 & 0.1301 & 0.2909 & 0.4372 & 0.3008 & 0.2988 \\
         & \cellcolor{cyan!15}LPO & 
         \cellcolor{cyan!15}\textbf{0.5042} &
         \cellcolor{cyan!15}\textbf{0.4522} & 
         \cellcolor{cyan!15}\textbf{0.5259} & 
         \cellcolor{cyan!15}\textbf{0.1928} & 
         \cellcolor{cyan!15}\textbf{0.3562} & 
         \cellcolor{cyan!15}\textbf{0.4845} & 
         \cellcolor{cyan!15}0.3110 &
         \cellcolor{cyan!15}\textbf{0.3308}\\
         \midrule
         \multirow{5}{*}{SDXL} & Original \cite{sdxl} & 0.5833 & 0.4782 & 0.5211 & 0.1936 & 0.3319 & 0.4874 & 0.3137 & 0.3327 \\
         & Diff.-DPO \cite{diffusion_dpo} & 0.6941 & 0.5311 & 0.6127 & 0.2153 & 0.3686 & 0.5304 & \textbf{0.3178} & 0.3525 \\
         & MaPO \cite{mapo} & 0.6090 & 0.5043 & 0.5485 & 0.1964 & 0.3473 & 0.5015 & 0.3154 & 0.3229 \\
         & SPO \cite{spo} & 0.6410 & 0.4999 & 0.5551 & 0.2096 & 0.3629 & 0.4931 & 0.3098 & 0.3467 \\
         & \cellcolor{cyan!15}LPO & 
         \cellcolor{cyan!15}\textbf{0.7351} & 
         \cellcolor{cyan!15}\textbf{0.5463} & \cellcolor{cyan!15}\textbf{0.6606} &
         \cellcolor{cyan!15}\textbf{0.2414} &
         \cellcolor{cyan!15}\textbf{0.4075} &
         \cellcolor{cyan!15}\textbf{0.5493} &
         \cellcolor{cyan!15}0.3152 &
         \cellcolor{cyan!15}\textbf{0.3801}\\
         \bottomrule
    \end{tabular}}}
    \vspace{-2mm}
    % \vskip -0.1in
\end{table*}


\begin{table*}[t]
    \begin{minipage}{0.63\linewidth}
        \vspace{-2mm}
        \caption{Quantitative results on GenEval \cite{geneval}.}
        \vskip 0.05in
        \label{tab:geneval}
        \centering
        \scriptsize
        \setlength{\tabcolsep}{1.1mm}{
        \scalebox{1.1}{
        \begin{tabular}{l l c c c c c c c}
             \toprule
             Model & Method & \makecell[c]{Single \\ Object} & \makecell[c]{Two \\ Object} & Counting & Colors & Position & \makecell[c]{Color \\ Attribution} & Overall \\
             \midrule
             \multirow{4}{*}{SD1.5} & Original & 97.50 & 37.12 & 34.69 & 75.53 & 3.75 & 6.75 & 42.56 \\
             & Diff.-DPO & \textbf{98.44} & 38.38 & 36.25 & 77.93 & 4.50 & 7.25 & 43.79 \\
             & SPO & 95.00 & 33.84 & 32.50 & 69.95 & 4.25 & 7.25 & 40.46 \\
             & \cellcolor{cyan!15}LPO & \cellcolor{cyan!15}97.81 &
             \cellcolor{cyan!15}\textbf{54.80}&
             \cellcolor{cyan!15}\textbf{40.94}&
             \cellcolor{cyan!15}\textbf{79.52}&
             \cellcolor{cyan!15}\textbf{7.00}& 
             \cellcolor{cyan!15}\textbf{10.25}&
             \cellcolor{cyan!15}\textbf{48.39}\\
             \midrule
             \multirow{5}{*}{SDXL} & Original & 93.75 & 63.38 & 30.94 & 80.05 & 9.25 & 19.00 & 49.40  \\
             & Diff.-DPO & 99.06 & 76.52 & \textbf{45.00} & 88.83 & 11.50 & 25.75 & 57.78 \\
             & MaPO & 95.63 & 68.94 & 32.19 & 83.51 & 11.50 & 17.75 & 51.59 \\
             & SPO & 94.38 & 69.44 & 31.88 & 81.65 & 10.25 & 15.50 & 50.52  \\
             & \cellcolor{cyan!15}LPO & \cellcolor{cyan!15}\textbf{99.69} &
             \cellcolor{cyan!15}\textbf{81.57} &
             \cellcolor{cyan!15}43.75 &
             \cellcolor{cyan!15}\textbf{89.10} &
             \cellcolor{cyan!15}\textbf{14.00} &
             \cellcolor{cyan!15}\textbf{27.50} & 
             \cellcolor{cyan!15}\textbf{59.27}\\
             \bottomrule
        \end{tabular}}}
        \vskip -0.1in
    \end{minipage}
    \hfill
    \begin{minipage}{0.35\linewidth}
        \vspace{-2mm}
        \caption{Comparisons of training speed.}
        \vskip 0.05in
        \label{tab:speed}
        \centering
        % \footnotesize
        \scriptsize
        \setlength{\tabcolsep}{1.1mm}{
        \scalebox{1.1}{
        \begin{tabular}{l c c c}
             \toprule
             Method & \makecell[c]{Reward \\ Modeling} & \makecell[c]{Preference \\ Optimization} & \makecell[c]{Total $\downarrow$ \\ (A100 h)} \\
             \midrule
             \textcolor{gray}{SD1.5} \\
             \hspace{1pt} Diff.-DPO & 0 & 240 & 240 \\
             \hspace{1pt} SPO & 32 & 48 & 80 \\
             \hspace{1pt} \cellcolor{cyan!15}LPO & \cellcolor{cyan!15}\textbf{15} & \cellcolor{cyan!15}\textbf{8} & \cellcolor{cyan!15}\textbf{23} \\
             \midrule
             \textcolor{gray}{SDXL} \\
             \hspace{1pt} Diff.-DPO & 0 & 2,560 & 2,560 \\
             \hspace{1pt} SPO & 116 & 118 & 234 \\
             \hspace{1pt} \cellcolor{cyan!15}LPO & \cellcolor{cyan!15}\textbf{52} & \cellcolor{cyan!15}\textbf{40} & \cellcolor{cyan!15}\textbf{92} \\
             \bottomrule
        \end{tabular}}}
        \vskip -0.1in
    \end{minipage}
    \vspace{-0.8mm}
\end{table*}

\begin{table}[t]
    \centering
    \vspace{-2mm}
    \caption{Heterogeneous optimization based on LRM-SD1.5. P-S and I-R denote the PickScore and ImageReward metrics.}
    \vskip 0.05in
    \label{tab:sd15_for_sd21}
    \scriptsize
    \setlength{\tabcolsep}{1.0mm}{
    \scalebox{1.0}{
    \begin{tabular}{c c c c c c c c}
         \toprule
         Model & Method & Aesthetic & GenEval & P-S & I-R & HPSv2 & HPSv2.1\\
         \midrule
         SD2.1 & Original & 5.673 & 48.59 & 20.92 & 0.3063 & 27.05 & 25.49 \\
         \tiny(Same VAE) & \cellcolor{cyan!15}LPO & \cellcolor{cyan!15}\textbf{5.969} & \cellcolor{cyan!15}\textbf{56.01}  & \cellcolor{cyan!15}\textbf{21.76} & \cellcolor{cyan!15}\textbf{0.7978} & \cellcolor{cyan!15}\textbf{28.05} & \cellcolor{cyan!15}\textbf{28.61} \\
         \midrule
         SDXL & Original & 5.920 & \textbf{49.40} & \textbf{21.65} & \textbf{0.4780} & 27.06 & 26.05\\
         \tiny(Diff. VAE) & \cellcolor{cyan!15}LPO & \cellcolor{cyan!15}\textbf{5.953} & \cellcolor{cyan!15}40.85 & \cellcolor{cyan!15}20.82 & \cellcolor{cyan!15}0.3919 & \cellcolor{cyan!15}\textbf{27.10} & \cellcolor{cyan!15}\textbf{26.69} \\
         \bottomrule
    \end{tabular}}}
    % \vspace{-2mm}
    \vskip -0.15in
\end{table}


\textbf{Baseline Methods.} We compare LPO with DDPO \cite{ddpo}, D3PO \cite{d3po}, Diffusion-DPO \cite{diffusion_dpo}, MaPO \cite{mapo}, and SPO \cite{spo}. These methods are trained on similar datasets, such as Pick-a-Pic v1 and v2, to ensure a fair comparison. Details are provided in \cref{sec:experimental_detail}.


\textbf{Evaluation Protocol.} We evaluate various diffusion models across three dimensions: general preference, aesthetic preference, and text-image alignment. The PickScore \cite{pickscore}, HPSv2 \cite{hpsv2}, HPSv2.1 \cite{hpsv2}, and ImageReward \cite{imagereward} are utilized to assess the general preference. The aesthetic preference is evaluated using the Aesthetic Score \cite{aesthetic}. Consistent with \cite{spo}, both general and aesthetic preferences are assessed on the validation unique split of Pick-a-Pic v1, which has 500 different prompts. For text-image alignment, we employ the GenEval \cite{geneval} and T2I-CompBench++ \cite{t2i_compbench} metrics. All images are generated using the DDIM scheduler with 20 steps. Additionally, to assess the correlations between the LRM and aesthetics as well as text-image alignment, we propose two corresponding metrics. Specifically, we calculate the score gaps $G_*,*\in\{A,C,L\}$ between winning and losing images, where $A$, $C$, $L$ represent Aesthetic, CLIP, and LRM. For LRM, the score is taken at $t=0$. Then the Pearson Correlation Coefficient \cite{pearson} between $G_L$ and $G_A$ is referred to as \textit{Aes-Corr} while that between $G_L$ and $G_C$ is termed \textit{CLIP-Corr}. They are evaluated on the validation unique and test unique splits of Pick-a-Pic v1.

\subsection{Main Results}


\textbf{Quantitative Comparison.} As indicated in Tab.\;\ref{tab:preferenece_eval}, Tab.\;\ref{tab:t2i_eval}, and Tab.\;\ref{tab:geneval}, Diffusion-DPO excels in enhancing the text-image alignment, while SPO focuses more on aesthetics. LPO outperforms both methods across three dimensions, achieving higher Aesthetic Scores and superior performance on T2I-CompBench++ and GenEval metrics, leading to improved general preference scores. The user study results indicate similar findings, as discussed in \cref{sec:add_exp}. Notably, the LPO-optimized SD1.5 even exhibits performance comparable to the original SDXL model across various metrics.  We further validate the effectiveness of \textit{heterogeneous optimization} in Tab.\;\ref{tab:sd15_for_sd21}. SD1.5 and SD2.1 \cite{sd1} share the same VAE encoder, but SD1.5 has a smaller text encoder. Remarkably, fine-tuning SD2.1 using LRM-1.5 still yields significant improvements across various aspects, demonstrating that a smaller and inferior diffusion model can effectively fine-tune a larger and more advanced model as long as they share the same VAE encoder. In contrast, applying LRM-1.5 for the LPO of SDXL is ineffective due to the distribution mismatch in their latent spaces, which arises from differences in their VAE encoders.

\textbf{Qualitative Comparison.} The qualitative comparisons of various methods are illustrated in Fig.\;\ref{fig:main_comparison} and Fig.\;\ref{fig:vis_15_1}-Fig.\;\ref{fig:vis_xl_4}. The images generated by Diffusion-DPO exhibit deficiencies in color and detail, whereas those produced by SPO demonstrate lower semantic relevance. Additionally, SPO's excessive focus on aesthetics may lead to an overabundance of details in some images, making them appear cluttered. In contrast, the images produced by LPO achieve a strong balance between text-image alignment and aesthetic quality, delivering a higher overall image quality.


\textbf{Training Efficiency Comparison.} LPO achieves significantly faster training speed. As shown in Tab.\;\ref{tab:speed}, considering the time required for both reward modeling and preference optimization, LPO requires only 23 A100 hours for SD1.5---just 1/10 of the training time needed for Diffusion-DPO and 1/3.5 of that for SPO. For SDXL, LPO's training time is reduced to 1/28 and 1/2.5 of that for Diffusion-DPO and SPO, respectively. This efficiency is primarily due to LPO performing reward modeling and preference optimization directly in the latent space, avoiding the additional computational overhead of converting to pixel space.

\subsection{Ablation Studies}
\label{sec:ablation_study}
If not specified, ablation experiments are conducted on SD1.5. Due to space limitations, we only use PickScore to reflect general preference in Tab.\;\ref{tab:ablation_data} and Tab.\;\ref{tab:ablation_lrm}.


\textbf{MPCF.} As shown in Tab.\;\ref{tab:ablation_data}, MPCF plays a critical role in LRM training. As discussed in Sec.\;\ref{sec:lrm_train}, the inconsistent preference issue makes training on the full dataset (wo MPCF) ineffective, since it hinders the LRM from adequately focusing on aesthetics or text-image alignment, resulting in inferior LPO performance. On the other hand, different filtering strategies can profoundly impact the preference patterns of both the LRM and LPO-optimized models. The first filtering strategy strictly requires that winning images score higher than losing images across all aspects. However, since the diffusion model lacks explicit text-image alignment pre-training like CLIP, it is prone to overfitting to the visual features of the images, as indicated by a higher Aes-Corr. This overfitting results in reduced attention to alignment, as reflected by lower CLIP-Corr and GenEval scores. The second and third strategies relax the aesthetic constraints to varying degrees. However, excessively lenient constraints (the 3rd strategy) may cause LRM to focus solely on text-image alignment while neglecting image quality, resulting in a negative Aes-Corr. In contrast, the second strategy balances these two aspects better, leading to the highest general preference scores.


\begin{table}[t]
    \centering
    \vspace{-2.5mm}
    \caption{Ablation results on MPCF of LRM's training data. The second strategy balances aesthetics and alignment better.}
    \vskip 0.05in
    \label{tab:ablation_data}
    \scriptsize
    \setlength{\tabcolsep}{1.0mm}{
    \scalebox{1.1}{
    \begin{tabular}{c c c c c c}
         \toprule
         \multirow{2}{*}{Strategy} & \multicolumn{2}{c}{LRM} & \multicolumn{3}{c}{LPO} \\
         \cmidrule(lr){2-3} \cmidrule(lr){4-6}
          & Aes-Corr & CLIP-Corr & Aesthetic & GenEval & PickScore \\
         \midrule
         wo MPCF & 0.1342 & 0.2274 & 5.772 & 45.66 & 21.49 \\
         1 & \textbf{0.4860} & 0.1011 & \textbf{6.390} & 45.77 & \underline{21.61} \\
         \rowcolor{cyan!15}2 & 0.1136 & 0.3588 & \underline{5.945} & \underline{48.39} & \textbf{21.69} \\
         3 & -0.1152 & \textbf{0.4480} & 5.750 & \textbf{48.62} & 21.47 \\
         \bottomrule
    \end{tabular}}}
    % \vspace{-2mm}
    \vskip -0.1in
\end{table}


\begin{table}[t]
    \centering
    \vspace{-2mm}
    \caption{Ablation results on the VFE module of LRM. Introducing VFE leads to better alignment and general preferences.}
    \vskip 0.05in
    \label{tab:ablation_lrm}
    \scriptsize
    \setlength{\tabcolsep}{1.0mm}{
    \scalebox{1.1}{
    \begin{tabular}{c c c c c c c }
         \toprule
         \multirow{2}{*}{VFE} & \multirow{2}{*}{$gs$} & \multicolumn{2}{c}{LRM} & \multicolumn{3}{c}{LPO} \\
         \cmidrule(lr){3-4} \cmidrule(lr){5-7}
          &  & Aes-Corr & CLIP-Corr & Aesthetic & GenEval & PickScore\\
         \midrule
         \xmark & 1.0 & \textbf{0.1712} & 0.3211 & \textbf{6.053} & 46.60 & 21.51  \\
         \cmark & 3.0 & 0.1233 & 0.3441 & 5.923 & 47.35 & 21.53 \\
         \rowcolor{cyan!15}\cmark & 7.5 & 0.1136 & 0.3588 & \underline{5.945} & \textbf{48.39} & \textbf{21.69}\\
         \cmark & 10.0 & 0.1063 & \textbf{0.3592} & 5.937 & \underline{48.13} & \underline{21.56}\\
         \bottomrule
    \end{tabular}}}
    % \vspace{-2mm}
    \vskip -0.1in
\end{table}


\begin{table}[t]
    \centering
    \vspace{-2.5mm}
    \caption{Ablation results on optimization timestep ranges in LPO.}
    \vskip 0.05in
    \label{tab:ablation_timestep}
    \scriptsize
    \setlength{\tabcolsep}{1.0mm}{
    \scalebox{1.1}{
    \begin{tabular}{c c c c c c c}
         \toprule
         Range of $t$ & Aesthetic & GenEval & P-S & I-R & HPSv2 & HPSv2.1 \\
         \midrule
         \texttt{[}0, 200\texttt{]} & 5.434 & 40.11 & 20.46 & -0.0987 & 26.25 & 23.61 \\
         \texttt{[}250, 450\texttt{]} & 5.527 & 43.00 & 20.76 & 0.1430 & 26.90 & 25.37 \\
         \texttt{[}500, 700\texttt{]} & 5.742 & 44.44 & 20.95 & 0.1591 & 26.71 & 25.16\\
         \texttt{[}750, 950\texttt{]} & \underline{5.853} & \underline{48.28} & 
         \underline{21.54} & \underline{0.6337} & \underline{27.47} & \underline{27.64} \\
         \midrule
         \texttt{[}0, 450\texttt{]} & 5.573 & 42.71 & 20.63 & 0.0204 & 26.69 & 24.88 \\
         \texttt{[}0, 700\texttt{]} & 5.765 & 44.93 & 21.02 & 0.3087 &  27.10 & 26.25\\
         \rowcolor{cyan!15}\texttt{[}0, 950\texttt{]} & \textbf{5.945} & \textbf{48.39} & \textbf{21.69} & \textbf{0.6588} & \textbf{27.64} & \textbf{27.86} \\
         \bottomrule
    \end{tabular}}}
    % \vspace{-2mm}
    \vskip -0.1in
\end{table}


\begin{table}[t]
    \centering
    \vspace{-2mm}
    \caption{Ablation results on different threshold strategies.}
    \vskip 0.05in
    \label{tab:ablation_threshold}
    \scriptsize
    \setlength{\tabcolsep}{1.0mm}{
    \scalebox{1.1}{
    \begin{tabular}{c c c c c c c }
         \toprule
          Threshold & Aesthetic & GenEval & P-S & I-R & HPSv2 & HPSv2.1\\
         \midrule
         0.3 & 5.853 & 46.75 & 21.22 & 0.5112  & 27.30 & 27.12 \\ 
         0.4 & 5.832 & 48.32 & 21.32 & 0.4789 & 27.08 & 26.37 \\
         0.5 & 5.900 & 48.39 & 21.57 & 0.6088 & 27.54 & \underline{27.42} \\
         0.6 & 5.877 & 47.97 & 21.35 & 0.5510 & 27.25 & 26.73 \\
         \midrule
         \texttt{[}0.3, 0.45\texttt{]} & \underline{5.916} & \textbf{49.43} & \underline{21.58} & \underline{0.6405} & \underline{27.55} & 27.33\\
         \rowcolor{cyan!15}\texttt{[}0.35, 0.5\texttt{]} & \textbf{5.945} & 48.39 & \textbf{21.69} & \textbf{0.6588} & \textbf{27.64} & \textbf{27.86} \\
         \texttt{[}0.4, 0.55\texttt{]} & 5.882 & \underline{48.77} & 21.48 & 0.4791 & 27.30 & 27.13\\
         \bottomrule
    \end{tabular}}}
    % \vspace{-2mm}
    \vskip -0.1in
\end{table}


\textbf{Structure of LRM.} As illustrated in Tab.\;\ref{tab:ablation_lrm}, the introduction of VFE ($gs>1$) leads to lower Aes-Corr values but higher CLIP-Corr values, indicating an enhanced emphasis on text-image alignment. This results in improvements in both the GenEval score and PickScore, with only a minor decline in the Aesthetic Score. As $gs$ increases, the LRM's correlation with alignment steadily improves, while its correlation with aesthetics decreases. When $gs$ is set to 7.5, the model achieves the best overall performance.

\textbf{Optimization Timesteps.} Tab.\;\ref{tab:ablation_timestep} ablates different optimization timestep ranges, indicating that larger timesteps lead to better performance. The results achieved within the range of $[750, 950]$ are nearly comparable to those achieved through optimization across the entire denoising process, \ie $[0,950]$. We suggest this is because diffusion models focus on low-frequency information, such as image layout and style, during larger timesteps, while emphasizing high-frequency texture details during smaller timesteps. The low-frequency components formed in higher timesteps play a decisive role in determining the overall quality of the generated images. This observation also demonstrates the effectiveness of LRM, even in very large timesteps. The qualitative comparison of different ranges is shown in Fig.\;\ref{fig:vis_timestep}.

\textbf{Dynamic Sampling Threshold.} The standard deviation $\sigma_t$ of samples at smaller timesteps is relatively small according to the DDPM scheduling \cite{ddpm}, making the constant threshold insufficient to accommodate all timesteps. As indicated in Tab.\;\ref{tab:ablation_threshold}, the dynamic threshold strategy generally outperforms the constant threshold across different intervals, effectively alleviating this problem. We further explore other dynamic strategies in \cref{sec:add_exp}.


\section{Discussion and Conclusion}
\label{sec:discuss}

We presented \bench, the first framework  and experimental platform to benchmark AI Agents for IT automation tasks. \bench strives to capture the complexity of modern IT systems and the diversity of IT tasks. The reproducibility of \bench ensures the community-driven effort despite inherent nondeterminism of large-scale IT systems. 

One of the key design principles of \bench is ensuring its flexibility to support diverse areas of different IT systems
and its extensibility to new scenarios. While current scope of \bench is comprehensive and representative, we plan to further enrich the benchmark suites by adding other important processes essential to modern IT automation. Furthermore, we plan to expand our benchmarking beyond event-triggered scenarios. 
We are actively working to expand scenario coverage for the supported processes and promote growth through open-community contributions.
 We invite the community to reproduce their real-world-inspired incidents in a synthetic sandboxed environment leveraging the \bench. We expect that everyone contributing can bring their expertise to the table.

We expect \bench to drive the innovations of AI agent-based techniques with a direct impact on the safety, efficiency, and intelligence of today’s IT infrastructures. 
With \bench, we are starting to explore many deep, exciting open problems: How to develop domain-specific AI agents that specialize in certain types of IT tasks? How to orchestrate multiple agents with various expertise to collaborate on bigger projects? How can we ensure safety of agent-driven solutions? How can we effectively use human-in-the-loop while developing diverse adaptive agents? We invite everyone to participate in answering these questions and realizing the vision of using AI agents to automate critical IT tasks.



\newpage
\bibliographystyle{tencent_ailab_tech_report}
\bibliography{tencent_ailab_tech_report}

\newpage
% \section{Appendix}
\subsection{Data - Survey on current passenger behaviour regarding POIs}

\begin{center}
    \begin{minipage}{\textwidth}

        \centering
        \captionof{table}{Usage of various navigation methods by our survey participants.}
        \begin{tabular}{r|cc|cc|cc|cc|cc}
            \toprule
            & \multicolumn{2}{c|}{AppleCar/AndroidAuto} & \multicolumn{2}{c|}{InVehicleSystems} & \multicolumn{2}{c|}{SmartphoneApps} & \multicolumn{2}{c|}{Compass} & \multicolumn{2}{c}{PaperMaps} \\
            & N & \% & N & \% & N & \% & N & \% & N & \% \\
            \midrule
            Never & 39 & 36.4\% & 13 & 12.1\% & 2 & 1.9\% & 101 & 94.4\% & 79 & 73.8\% \\
            Rarely & 19 & 17.8\% & 19 & 17.8\% & 15 & 14.0\% & 2 & 1.9\% & 26 & 24.3\% \\
            Sometimes & 18 & 16.8\% & 22 & 20.6\% & 33 & 30.8\% & 2 & 1.9\% & 2 & 1.9\% \\
            Often & 22 & 20.6\% & 34 & 31.8\% & 32 & 29.9\% & 0 & 0\% & 0 & 0\% \\
            Always & 9 & 8.4\% & 19 & 17.8\% & 25 & 23.4\% & 2 & 1.9\% & 0 & 0\% \\
            \bottomrule
        \end{tabular}

        \vspace{\baselineskip}
        
        \captionof{table}{The types of information drivers and passengers need to find a missed point of interest. Multiple choice was possible.}
        \begin{tabular}{r|cc|cc}
            \toprule
            \textbf{Type of information} & \multicolumn{2}{c|}{Drivers} & \multicolumn{2}{c}{Passengers} \\
            & N & \% & N & \% \\
            \midrule
            Name & 71 & 89\% & 65 & 90\% \\
            Perspective Picture & 39 & 49\% & 38 & 53\% \\
            Web Picture & 46 & 58\% & 44 & 61\% \\
            Description Text & 45 & 56\% & 49 & 68\% \\
            Category & 45 & 56\% & 49 & 68\% \\
            \bottomrule
        \end{tabular}

        \vspace{\baselineskip}

        \captionof{table}{Percentages on how and when passengers and drivers look for a missed point of interest.}
        \begin{tabular}{r|c|c}
            \toprule
            \textbf{Action}             & \textbf{Drivers}   & \textbf{Passengers} \\
            \midrule
            Nothing                     & 5\%                 & 4\% \\
            System saves Automatically  & 0\%                 & 3\% \\
            Stop the Car                & 3\%                 & 0\% \\
            Search later on Smartphone  & 46\%                & 8\% \\
            Search immediately on Smartphone & 11\%           & 49\% \\
            Search later on Navigation system & 24\%          & 5\% \\
            Search immediately on Navigation system & 11\%    & 31\% \\
            \bottomrule
        \end{tabular}

    \end{minipage}
\end{center}



\clearpage
\subsection{Data - Pre-Study on Eye-Gaze Interaction}

\begin{center}
    \begin{minipage}{\textwidth}
      
    \centering
    \captionof{table}{Descriptive Statistics for the pre-study Raw NASA Task Load Index.}
    \begin{tabular}{l|c|c|c|c|c|c|c}
        \toprule
        & \textbf{Mental} & \textbf{Physical} & \textbf{Temporal} & \textbf{Performance} & \textbf{Effort} & \textbf{Frustration} & \textbf{Score} \\
        \midrule
        Mean & 15.0 & 19.5 & 40.0 & 33.5 & 28.0 & 13.5 & 24.8 \\
        Median & 12.5 & 15.0 & 42.5 & 25.0 & 32.5 & 15.0 & 25.0 \\
        Standard Deviation & 10.5 & 19.9 & 23.1 & 21.4 & 18.1 & 11.6 & 9.47 \\
        Shapiro-Wilk W & 0.942 & 0.782 & 0.961 & 0.848 & 0.898 & 0.882 & 0.959 \\
        Shapiro-Wilk p & 0.573 & 0.009 & 0.794 & 0.055 & 0.206 & 0.139 & 0.769 \\
        \bottomrule
    \end{tabular}  

    \vspace{\baselineskip}

    \captionof{table}{Descriptive Statistics for the pre-study System Usability Scale.}
    \begin{tabular}{l|c|c|c|c|c}
        \toprule
        \textbf{Question} & \textbf{Mean} & \textbf{Median} & \textbf{Std. Deviation} & \textbf{Shapiro-Wilk W} & \textbf{Shapiro-Wilk p} \\
        \midrule
        Frequent Use & 4.00 & 4.00 & 0.943 & 0.841 & 0.045 \\
        Unnecessary Complex & 1.40 & 1.00 & 0.516 & 0.640 & < .001 \\
        Easy to Use & 4.60 & 5.00 & 0.516 & 0.640 & < .001 \\
        Support of Technical & 1.50 & 1.00 & 0.972 & 0.603 & < .001 \\
        Well Integrated & 4.40 & 4.00 & 0.516 & 0.640 & < .001 \\
        Inconsistency & 1.80 & 2.00 & 0.789 & 0.820 & 0.025 \\
        Learn Quickly & 4.60 & 5.00 & 0.699 & 0.650 & < .001 \\
        Cumbersome & 1.60 & 1.00 & 0.966 & 0.678 & < .001 \\
        Confident & 4.20 & 4.00 & 0.632 & 0.794 & 0.012 \\
        Learn a Lot Before & 1.10 & 1.00 & 0.316 & 0.366 & < .001 \\
        Score & 86.0 & 87.5 & 8.01 & 0.925 & 0.398 \\
        \bottomrule
    \end{tabular}

    \end{minipage}
\end{center}



\clearpage
\subsection{Data - Study on Visualizing Passed and Upcoming POIs}

\begin{center}
    \begin{minipage}{\textwidth}

        \centering
        \captionof{table}{Descriptive Statistics for the visualization-study Raw NASA Task Load Index, grouped by condition.}
        \begin{tabular}{llccccccc}
            \toprule
            \textbf{Statistic} & \textbf{Condition} & \textbf{Mental} & \textbf{Physical} & \textbf{Temporal} & \textbf{Performance} & \textbf{Effort} & \textbf{Frustration} & \textbf{Score} \\
            \midrule
            \multirow{3}{*}{Mean} & List & 28.6 & 25.5 & 20.7 & 14.8 & 29.5 & 19.5 & 23.1 \\
            & Timeline & 29.8 & 27.1 & 23.3 & 20.0 & 27.4 & 28.6 & 26.0 \\
            & Minimap & 46.7 & 36.4 & 42.4 & 42.6 & 56.9 & 50.2 & 45.9 \\
            \midrule
            \multirow{3}{*}{Median} & List & 25 & 20 & 15 & 5 & 25 & 15 & 25.0 \\
            & Timeline & 25 & 20 & 15 & 5 & 20 & 20 & 22.5 \\
            & Minimap & 40 & 30 & 45 & 40 & 60 & 55 & 49.2 \\
            \midrule
            \multirow{3}{1.5cm}{Std. Deviation} & List & 19.8 & 21.3 & 17.0 & 19.5 & 25.3 & 18.5 & 14.6 \\
            & Timeline & 22.8 & 21.3 & 19.6 & 29.0 & 26.6 & 26.0 & 19.3 \\
            & Minimap & 21.5 & 24.7 & 22.3 & 22.8 & 23.5 & 22.6 & 15.8 \\
            \midrule
            \multirow{3}{1.5cm}{Shapiro-Wilk W} & List & 0.917 & 0.879 & 0.909 & 0.767 & 0.835 & 0.851 & 0.941 \\
            & Timeline & 0.868 & 0.904 & 0.911 & 0.683 & 0.821 & 0.874 & 0.927 \\
            & Minimap & 0.947 & 0.886 & 0.920 & 0.938 & 0.954 & 0.946 & 0.927 \\
            \midrule
            \multirow{3}{1.5cm}{Shapiro-Wilk p} & List & 0.076 & 0.014 & 0.052 & < .001 & 0.002 & 0.004 & 0.229 \\
            & Timeline & 0.009 & 0.043 & 0.058 & < .001 & 0.001 & 0.011 & 0.118 \\
            & Minimap & 0.297 & 0.019 & 0.088 & 0.201 & 0.399 & 0.290 & 0.118 \\
            \bottomrule
        \end{tabular}


        \vspace{\baselineskip}

    
        \captionof{table}{Descriptive Statistics for the visualization-study System Usability Scale scores, grouped by condition.}
        \begin{tabular}{lccccc}
            \toprule
            \textbf{Condition} & \textbf{Mean} & \textbf{Median} & \textbf{Std. Deviation} & \textbf{Shapiro-Wilk W} & \textbf{Shapiro-Wilk p} \\
            \midrule
            List & 78.5 & 77.5 & 10.2 & 0.885 & 0.018 \\
            Minimap & 61.1 & 60.0 & 14.3 & 0.962 & 0.563 \\
            Timeline & 71.5 & 75.0 & 15.9 & 0.921 & 0.091 \\
            \bottomrule
        \end{tabular}


        \vspace{\baselineskip}

        
        \caption{Descriptive Statistics for the visualization-study Motion Sickness Questionnaire, grouped by time of completion.}
        \begin{tabular}{lccccc}
            \toprule
            \textbf{Study Order} & \textbf{Mean} & \textbf{Median} & \textbf{Std. Deviation} & \textbf{Shapiro-Wilk W} & \textbf{Shapiro-Wilk p} \\
            \midrule
            Pre-study & 0.381 & 0 & 0.669 & 0.617 & < .001 \\
            After First Condition & 0.857 & 0 & 1.15 & 0.732   & < .001 \\
            After Second Condition & 1.19 & 1 & 1.36 & 0.822   & 0.001  \\
            After Third Condition & 1.00 & 1 & 1.05 & 0.826    & 0.002  \\
            \bottomrule
        \end{tabular}


    \end{minipage}
\end{center}


\begin{table*}[ht]
    \centering
    \caption{Descriptive Statistics for the visualization-study User Experience Questionnaire, grouped by condition.}
    \begin{tabular}{llcccccc}
    \toprule
    \textbf{} & \textbf{Condition} & \textbf{Attractiveness} & \textbf{Perspicuity} & \textbf{Efficiency} & \textbf{Dependability} & \textbf{Stimulation} & \textbf{Novelty} \\
    \midrule
    \multirow{3}{*}{\textbf{M}} & Timeline & 1.26 & 1.42 & 1.10 & 1.12 & 1.11 & 1.37 \\
    & Minimap & 0.952 & 0.964 & 0.619 & 0.821 & 1.00 & 1.57 \\
    & List & 1.78 & 2.05 & 1.79 & 1.68 & 1.54 & 1.31 \\
    \midrule
    \multirow{3}{*}{\textbf{Mdn}} & Timeline & 1.50 & 1.75 & 1.25 & 1.25 & 1.25 & 1.50 \\
    & Minimap & 1.00 & 0.750 & 0.500 & 0.500 & 1.00 & 1.25 \\
    & List & 1.83 & 2.25 & 1.50 & 1.50 & 1.50 & 1.25 \\
    \midrule
    \multirow{3}{*}{\textbf{SD}} & Timeline & 0.921 & 1.03 & 0.937 & 0.883 & 0.986 & 1.11 \\
    & Minimap & 0.972 & 1.01 & 0.993 & 0.946 & 0.939 & 1.13 \\
    & List & 0.642 & 0.692 & 0.755 & 0.717 & 0.704 & 0.798 \\
    \midrule
    \multirow{3}{*}{\textbf{W}} & Timeline & 0.938 & 0.833 & 0.951 & 0.912 & 0.970 & 0.933 \\
    & Minimap & 0.972 & 0.975 & 0.980 & 0.948 & 0.967 & 0.877 \\
    & List & 0.976 & 0.893 & 0.847 & 0.926 & 0.942 & 0.982 \\
    \midrule
    \multirow{3}{*}{\textbf{p}} & Timeline & 0.200 & 0.002 & 0.359 & 0.059 & 0.737 & 0.160 \\
    & Minimap & 0.783 & 0.844 & 0.918 & 0.313 & 0.661 & 0.013 \\
    & List & 0.866 & 0.026 & 0.004 & 0.112 & 0.235 & 0.953 \\
    \bottomrule
    \end{tabular}
\end{table*}

\begin{table*}[ht]
    \centering
    \caption{User ranking values for the visualization-study conditions.}
    \begin{tabular}{l|cc|cc|cc}
    \toprule
    \multirow{2}{*}{\textbf{Condition}}& \multicolumn{2}{c|}{\textbf{Least Favorite}} & \multicolumn{2}{c|}{\textbf{Middle}} & \multicolumn{2}{c}{\textbf{Favorite}} \\
                & N     & \%             & N    & \%                  & N       & \%                 \\
    \midrule
    List        & 2     & 9.5\%          & 4    & 19.0\%              & 15      & 71.4\%              \\
    Minimap     & 11    & 52.4\%         & 7    & 33.3\%              & 3       & 14.3\%             \\
    Timeline    & 8     & 38.1\%         & 10   & 47.6\%              & 3       & 14.3\%             \\
    \bottomrule
    \end{tabular}
\end{table*}


\end{document}