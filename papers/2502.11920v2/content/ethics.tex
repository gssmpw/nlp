
\section{Ethical Considerations}
\label{sec:ethics}

Several important ethical considerations are relevant to this research. We now outline how we considered them during the study design and execution. The Science Ethics Review Board at Leiden University reviewed and approved our study. 

Our study involved human participants, and, moreover, these participants were students taking a course taught by the authors of this paper. This introduces ethical concerns due to the dual role of the authors being both in the research team and responsible for the education of the students in the course. We have done our utmost to ensure that the students were not pressured to participate in this study and that they did not perceive being pressured or nudged to participate. Below we discuss the multiple safeguards in this regard that we introduced. 

Students were informed of the study objectives and design and then were asked to fill in and sign an informed consent form. In this form, they could choose to provide consent for their responses to the assignment to be included in the study and for the data they submit to be used for research purposes in an anonymized format. The consent forms were collected blindly for the teachers. 

We made it clear to the students that the assignment was a mandatory, graded course component, but participation in the study was entirely optional and would not affect their grades. Students were informed that teaching assistants would grade their submissions according to defined grading rubrics, and teaching assistants had no knowledge of who had elected to participate in the study. The grading rubric did not account for study participation in any way; thus, the grade was not influenced by (non-)participation. Finally, students were told that they could withdraw their consent at any time. We informed students that we would not collect responses until one month after final grades were submitted (and were no longer able to be modified). For any students who were still concerned, we offered the protocol of initially providing consent to participate, and withdrawing said consent after final grades were submitted. Withdrawing consent required filling out an online form, which we provided with the intention of making it as easy and straightforward as possible for students who no longer wished for their responses to be included.

We further provided resources when presenting the research to students, in the participation consent form, and in the introduction to the assignment that students could reach out to if they were concerned about any negative effects resulting from the study. These resources included the contact information for the relevant Ethical Review Board, the university ombudsman's office, and a student counselor. To our knowledge, no students reached out to these resources with questions or concerns about the study.

The assignments were submitted via the university's learning management software (LMS), which is a standard and accepted practice for course assignments. For the students who opted to participate in the study, once the data processing started, students were assigned a ``participant number'' which was stored in a password-protected reference list on the first author's university-issue computer. The participant number was used to anonymize the data for analysis. All other study data was pulled directly off of the LMS into a spreadsheet for further processing. The data collected and analyzed for the study did not contain any personal information.


Participants were not provided compensation for their participation in the study. As the assignment was a mandatory course component, it would have been inappropriate to compensate students for completing it. We designed our study following the Menlo Report's guidelines for ethical research~\cite{kenneallyMenloReportEthical2012} and we strived to carefully balance the benefits of the study against potential harms. The assignment itself is useful for students as it helps them learn about important concepts within cybersecurity and develops their analysis skills. We also believe that our students benefited from the study because they experienced the scientific process in the computer science domain. Moreover, the findings from this study allow us to further improve our research-based teaching, which will benefit future generations of students. It is important for the community that teachers can confidently teach attack trees to students without a substantial computer science background. Our personal experience told us that attack trees are accessible to such audiences, but only via doing a properly designed study can we be confident about this.  

We believe that the potential harm to our students, on the other hand, is limited, because we actively emphasized that non-participation does not entail any consequences for the course and we placed multiple safeguards to protect the students. Participation in the study did not entail any extra effort for the students (because they would still be doing the work as a course assignment). 

Our Ethics Review Board agreed with this risk-benefit analysis and approved our study (ref. 2022-016). 



\section{Open Science}
\label{sec:open-science}

The full set of anonymized, qualitative, perception data is shared alongside this work in our supplementary data material~\cite{zenodo-dataset}. This includes all of the values used to calculate the results presented in this paper, as well as additional elements of data that were ultimately excluded.  The data are shared in a \texttt{.csv} format. To enable verification, we provide the code we used to analyze the data and generate the results presented in this paper. This code is provided as a Jupyter notebook.

Further, we provide the dataset of ADTs generated by participants. All ADTs are provided as \texttt{.png} files, with trees without structural errors provided as \texttt{.xml} files in the ADTool schema. 
Finally, we also share the slides used in the training of the study alongside a summary of the training and indicative time amounts spent on each part of the training. All these materials are available in~\cite{zenodo-dataset}.