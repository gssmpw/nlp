\section{Study Results}\label{sec:results}
In this section, we present the study results per our main research questions \textbf{RQ1--RQ4}.


\begin{table*}[t!]
    \centering
    \caption{Check for understanding.}
    \label{tab:cfu-results}
    % \resizebox{\columnwidth}{!}{
    % \begin{tabular}{@{}lcccccc@{}}
    %     \toprule
    %     \textbf{Description}
    %      & \textbf{\# \ICS} & \textbf{\% Correct \ICS} & \textbf{\# \SEC} & \textbf{\% Correct \SEC} & \textbf{U} & \textbf{p} \\ \midrule
    %     Root Nodes
    %      & 35          & 91.43               & 28          & 89.28               & 500.5      & 0.7856     \\
    %     Leaf Nodes
    %      & 49          & 40.13               & 53          & 70.13               & 806.0      & 0.0005     \\
    %     Defense Nodes\footnote{Many of these questions were asked alongside attack vectors, necessitating understanding of both concepts}
    %      & 49          & 51.19               & 53          & 61.13               & 1117.5     & 0.2123     \\
    %     %Leaf Nodes small study only          
    %     % & 35          & 50.0               & 28          & 91.07               & 248.5         & 0.00016     \\
    %     Attack Vectors
    %      & 35          & 34.29               & 28          & 44.64               & 433.0      & 0.3960     \\
    %     Levels of Abstraction
    %      & 34          & 26.47               & 27          & 37.04               & 378.0      & 0.1795     \\            \bottomrule
    % \end{tabular}
    \resizebox{.68\textwidth}{!}{
    \begin{tabular}{@{}lcclclcccc@{}}

        \toprule
\textbf{Description}  & \textbf{Questions} & \multicolumn{2}{c}{\textbf{\ICS}} & \multicolumn{2}{c}{\textbf{\SEC\text{  }}} & \multicolumn{2}{c}{\textbf{BM test}} & \textbf{TOST} &     \textbf{Effect Size}             \\
&                    & $n$      & \% Correct        & $n$         & \% Correct    & statistic         & $p*m$   & $p*m$& Cohen's $d$                    \\ \midrule
Root nodes            & 1                  & 35       & 91.43             & 28          & 89.28         & -0.28                & 1.0          & \revised{\textbf{1.98e-17}} & 0.07 \\
Leaf nodes            & 3                  & 49       & 40.13             & 53          & 70.13         & \revised{3.73}                & \revised{\textbf{0.025}} &                 & 0.75 \\
Defense nodes         & 4                  & 49       & 51.19             & 53          & 61.13         & 1.25               & 1.0           & \revised{\textbf{2.65e-19}} & \revised{0.27} \\
        %Leaf Nodes small study only          
        % & 35          & 50.0               & 28          & 91.07               & 248.5         & 0.00016     \\
Attack vectors        & 2                  & 35       & 34.29             & 28          & 44.64         & 0.84               & 1.0           & \revised{\textbf{6.19e-10}} & 0.24 \\
Levels of abstraction & 2                  & 34       & 26.47             & 27          & 37.04         &  \revised{1.36}               & 1.0          & \revised{\textbf{8.81e-17}} & 0.38 \\            \bottomrule
    \end{tabular}
    }
\end{table*}
% Root nodes           0.000000    1.000000  8.602000e-18
%       Leaf nodes           1.000000       
%       Defense nodes        2.000000     1.000000  1.156214e-19
%       Attack vectors       3.000000     1.000000  2.698093e-10
%       LoA                  1.362944     1.000000  3.775514e-17


\subsection{\RQ{1}: Effect of the background on \textbf{AE}}\label{sec:rq1}


\subsubsection{\hypothesis{\hypoCheckUnderstand}: Understanding ADT concepts}
\label{ssec:results-concept-understanding}

As mentioned in Section~\ref{ssec:training}, both \ICS\ and \SEC\ students received the same training in the form of a lecture. The lecture covered ADTs as a whole and delved into specific important concepts such as the types of nodes and refinements, levels of abstraction (LoA), and attack vectors. These concepts were addressed in detail during the lecture and practiced by students in small groups. We then tested the understanding of these concepts in the small study. 

In Table~\ref{tab:cfu-results} we see the aggregated responses to questions covering five chosen concepts related to ADTs. For each concept, Table~\ref{tab:cfu-results} presents the number of questions asked about each concept. We see the number of respondents from both groups as well as the average percentage of correct answers for each population and concept. Finally, we can see the statistics and $p * m$ values (with HB correction)  from the BM test. We can see that on four out of five concepts \ICS\ students scored, on average, somewhat worse than the \SEC\ students. However, only one of five topics (leaf nodes) has a statistically significant difference ($p * m <0.05$) between the two populations according to the BM test, and all topics show statistically significant results for equivalence according to TOST. We provide a visualization of this comparison in Figure~\ref{img:cfu}.

\highlight{\textbf{$H_1$}: We find evidence of equivalence between \ICS\ and \SEC\ students on understanding ADT concepts.}



\begin{figure}[h]
    \imgResize{
        \includegraphics[width=\linewidth]{img/ConceptUnderstanding}
    }
    \caption{Comparison of the average scores across check-for-understanding questions.}
    \label{img:cfu}
\end{figure}




\begin{figure}[t]
    \imgResize{
        \includegraphics[width=\linewidth]{img/ADT2Comparison.png}
    }
    \caption{Comparison on creating ADTs from a written description.}
    \label{fig:adt2-comparison}
\end{figure}




\subsubsection{\hypothesis{\hypoSecondADT}: Successfully creating ADTs}\label{ssec:results-h2}


\textbf{\hypothesis{\hypoSecondADTsub}: Creating ADTs from a written description.}%\label{ssec:results-adt2-comparison}
The second task of the large study was to create an ADT from a written description of an attack scenario. The written scenario was the result of reading out an existing ADT chosen by the research team into text. The students were tasked with reconstructing the original ADT from the text alone and were not told of the existence of the original ADT. They were specifically instructed to only include information from the scenario and to not introduce new information. From this task, we have 89 submitted ADTs \revised{(one participant did not submit an ADT for this task)}\ that are all nearly identical, as they are drawn from the same source material. Because of how the task was designed, we consider that this task has a correct answer. As such, we can compare the ADTs created by students to the original ADT to find where the participants deviated. 

Fifty-seven (57) ADTs (64\%) were identical to the original ADT according to the seven metrics we chose to measure the similarity of ADTs\footnote{To the best of our knowledge there is no established metric to measure distance or similarity between attack trees.}: the numbers of attack and defense nodes (we also separately count the number of attack and defense leaf nodes), the number of \OR\ and \AND\ refinements, and the number of levels of abstraction in a tree.  Of those identical ADTs, 26 (61.9\%) were provided by \ICS\ students, and 31 (64.6\%) were provided by \SEC\ students. \revised{The complete set of results is found in Table~\ref{h3h9-table}.}


Figure~\ref{fig:adt2-comparison} summarizes the 32 answers that deviated from the correct  ADT on at least one of the seven metrics. For example, if a student had one extra attack node, the figure would represent this answer as +1 in the ``\# atk nodes'' category. This figure shows that most students tended to make errors on only a few metrics, and produced results similar enough to the correct ADT. 
We see that only one \SEC\ student and no \ICS\ students made any errors regarding levels of abstraction (LoA); this could indicate that it is relatively easy for human participants to infer the different LoA from a textual description, and this holds for both participants with and without technical background. 


\begin{table}[t!]s
    \centering
    \caption{Qualitative analysis of self-drawn ADTs.}
    \label{tab:results-qualitative}
    \revised{
        \resizebox{\linewidth}{!}{
    \begin{tabular}{lrlrlrlcccc}
        \toprule
    \textbf{Quality} & \multicolumn{2}{c}{\textbf{\shortstack{Largely\\Correct}}} & \multicolumn{2}{c}{\textbf{\shortstack{Neither}}} & \multicolumn{2}{c}{\textbf{\shortstack{Largely\\Incorrect}}}  & \multicolumn{2}{c}{\textbf{BM Test}} & \textbf{TOST} & \textbf{Effect Size} \\
     & \ICS & \SEC & \ICS & \SEC & \ICS & \SEC & statistic & $p*m$ & $p*m$ & Cohen's $d$ \\
     \midrule
Cohesive&21 & 22 & 15 & 19 & 4 & 6 & 0.46 & 1 & \textbf{8.21e-07} & 0.1\\
Clear&26 & 35 & 11 & 8 & 3 & 4 & -0.91 & 1 & \textbf{2.00e-07} & 0.15\\
Concise&24 & 24 & 16 & 20 & 0 & 3 & 1.20 & 1 & \textbf{6.96e-08} & 0.3\\
Complete&30 & 28 & 9 & 15 & 1 & 4 & 1.67 & 1 & \textbf{1.07e-06} & 0.35\\
    \bottomrule
    \end{tabular}
    }
    }
    \end{table}



\begin{table}
\centering
\caption{Results for hypotheses \revised{\hypothesis{\hypoSecondADTsub}}, \hypothesis{\hypoErrorAmount}, and \hypothesis{\hypoThirdADT}.}
\label{h3h9-table}
\resizebox{\linewidth}{!}{
\begin{tabular}{llllllc}
\toprule
\multicolumn{2}{l}{\textbf{Hypothesis}}         & \textbf{Component} & \multicolumn{2}{c}{\textbf{BM Test}} & \textbf{TOST} & \textbf{Effect Size}                             \\
            
&          && statistic             & \revised{$p*m$}     & \revised{$p*m$}              & Cohen's $d$      \\

\midrule
\revised{\multirow{7}{*}{\hypothesis{\hypoSecondADTsub}}} & & \revised{ADT2 defense leaf nodes} & \revised{0.656} & \revised{1.0} & \revised{\textbf{1.44e-11}}& \revised{0.07}\\
& & \revised{ADT2 defense nodes} & \revised{1.276} & \revised{1.0} & \revised{\textbf{4.88e-07}}& \revised{0.23}\\
& & \revised{ADT2 attack leaf nodes} & \revised{-1.435} & \revised{1.0} & \revised{\textbf{1.73e-16}}& \revised{0.33}\\
& & \revised{ADT2 attack nodes} & \revised{-1.727} & \revised{1.0} & \revised{\textbf{3.53e-04}}& \revised{0.31}\\
& & \revised{ADT2 AND (attack)} & \revised{-0.111} & \revised{1.0} & \revised{\textbf{3.77e-25}}& \revised{0.02}\\
& & \revised{ADT2 OR (attack)} & \revised{-2.496} & \revised{\textbf{.969}}& \revised{\textbf{5.15e-13}}& \revised{0.52}\\
& & \revised{ADT2 levels of abstraction} & \revised{-1.000} & \revised{1.0} & \revised{\textbf{5.57e-59}}& \revised{0.2}\\
\midrule\multirow{10}{*}{\hypothesis{\hypoErrorAmount}} &\multirow{2}{*}{\hypothesis{\hypoMultipleParent}}                      & ADT1 multi-parent nodes              & 0.32          & 1.0                   & \revised{\textbf{3.70e-13}} & 0.17 \\
&                  & ADT3 multi-parent nodes              & -0.64         & 1.0                   & \revised{\textbf{9.87e-17}} & 0.21  \\
\cmidrule{2-7}&\multirow{2}{*}{\hypothesis{\hypoMultipleRefinement}}                    & ADT1 multi refinement                & -0.53         & 1.0                   & \revised{\textbf{8.23e-15}} & 0.03  \\
&                    & ADT3 multi refinement                & -0.13         & 1.0                   & \revised{\textbf{3.54e-26}} & 0.04 \\
\cmidrule{2-7}&    \multirow{3}{*}{\hypothesis{\hypoMulipleCountermeasure}}                & ADT1 multi countermeasure            & 1.76          & 1.0                   & \revised{\textbf{4.59e-12}} & 0.38 \\
&                    & ADT2 multi countermeasure            & 0.46          & 1.0                   & \revised{\textbf{9.86e-22}} & \revised{0.02} \\
&                    & ADT3 multi countermeasure            & -0.96         & 1.0                   & \revised{\textbf{0.035} }    & \revised{0.1}  \\
\cmidrule{2-7}& \multirow{3}{*}{\hypothesis{\hypoSingleChildNodes}}                   & ADT1 single child  (attack)              & -4.68         & \revised{\textbf{8.66e-04}}      &               & 0.79  \\
&                    & ADT2 single child  (attack)              & 1.50          & 1.0                   & \revised{\textbf{1.01e-10}} & 0.19 \\
&                    & ADT3 single child  (attack)              & -2.66         &  \revised{0.641}                & \revised{1.0}              & 0.47  \\

\midrule
\multirow{7}{*}{\hypothesis{\hypoThirdADT}}     &                    & ADT3 defense leaf nodes               & -0.35         & 1.0                   & 1.0               & 0.10 \\
&                    & ADT3 defense nodes                    & -0.31         & 1.0                   & 1.0               & 0.13 \\
&                    & ADT3 attack leaf nodes               & 0.83          & 1.0                   & 1.0               & 0.36 \\
&                    & ADT3 attack nodes                    & 0.19          & 1.0                   & 1.0               & 0.24 \\
&                    & ADT3 AND  (attack)                    & 1.24          & 1.0                   & 1.0               & 0.34 \\
&                    & ADT3 OR  (attack)                      & 0.40          & 1.0                   & 1.0               & 0.23 \\
&                    & ADT3 levels of abstraction                             & -0.53         & 1.0                   & \revised{0.105}     & 0.10  \\
&                    & \revised{ADT3 and:or ratio}                          & \revised{0.19}         & \revised{1.0}                   & \revised{\textbf{1.95e-03}}     & \revised{0.11}  \\
\bottomrule
\end{tabular}
}

\end{table}

% 0.00001


\textbf{\hypothesis{\hypoThirdADTsub}: Creating ADTs for a self-selected scenario.}
%\label{sssec:results-adt3-effectiveness}
It is important to assess whether the participants are able to produce high-quality ADTs to represent a diverse set of attack scenarios. In total, there were 88 ADTs (two participants did not submit an ADT for this task) drawn for the task where students had to model their own scenarios. 

We qualitatively evaluated the ADTs designed for self-selected scenarios (we call them \emph{self-drawn ADTs}) based on four criteria: how meaningful are the refinements (\emph{cohesiveness}), how clear are the labels (\emph{clarity}), how relevant are the suggested attack components and whether there are any excessive steps (\emph{conciseness}), and how complete are the scenarios (\emph{completeness}). These qualities were selected to represent together a quality evaluation of the designed models. 

 The evaluation was done by two researchers experienced in attack trees and cybersecurity. First, the researchers designed together a rubric to evaluate ADTs based on these four criteria. The rubric was adjusted and calibrated in two iterations, when the researchers would first independently evaluate a set of randomly selected ADTs from both \ICS\ and \SEC\ participants and then jointly discuss the results. In the second iteration, the two researchers independently assessed all considered trees in the same way (reaching an agreement).  This final rubric used to evaluate the ADTs according to these criteria is available in the provided data artifact~\cite{zenodo-dataset}. The principal researcher then evaluated the whole set of ADTs based on the final rubric. The results of the evaluation according to this rubric can be found in Table~\ref{tab:results-qualitative}, which shows that there is statistically significant equivalence between the groups on all four criteria.
 

\highlight{\textbf{$H_2$}: We find no significant evidence of a difference between \ICS\ and \SEC\ students on effectively creating ADTs.}


\subsubsection{\hypothesis{\hypoErrorAmount}: Common errors when designing ADTs}
\label{ssec:results-common-errors}

Another metric we used to compare the two populations of students is the common mistakes they made while creating ADTs. After manually checking all 180 received ADT images, we identified four common types of mistakes described below.


\textbf{\hypothesis{\hypoMultipleParent}: Multi-parent nodes.} These describe nodes that have more than one parent. ADT construction rules (syntax) allow only a single parent for every node~\cite{kordyFoundationsAttackDefense2011}. For each node that had more than one parent, we counted that node as an error. If a node had more than two parents, the node was still counted only once.

\textbf{\hypothesis{\hypoMultipleRefinement}: Multi-refinement nodes} These are nodes that have children with multiple refinement relationships. ADT construction rules allow for one refinement per node, in our case either \AND\ or \OR~\cite{kordyFoundationsAttackDefense2011}. Some students would have two child nodes in an \AND\ relationship, and then a third or fourth child node that was not included in the \AND. This was expressed by the \AND\ arc not extending to the connecting edge of these other children. It was clear to us, also based on the node labels, that some children were in an \AND\ relationship, while the remainder was in an \OR\ relationship. We counted each node with multiple refinements regardless of the number of children that node had.

\textbf{\hypothesis{\hypoMulipleCountermeasure}: Multi-countermeasure attack nodes.} These are attack nodes that have multiple countermeasures. ADT construction rules only allow for one countermeasure child per node~\cite{kordyFoundationsAttackDefense2011}. If multiple countermeasures are possible, there should first be an intermediate defense node with the single countermeasure edge, and then the multiple countermeasures can be added to the intermediate node in either \AND\ or \OR\ relationship. We counted each time an attack node had more than one countermeasure, regardless of the number of countermeasures attached to that node. %Note that some students had relationships between multiple countermeasure edges; for example, a node with two countermeasures would show an \AND\ arc between the two countermeasure edges. 

\textbf{\hypothesis{\hypoSingleChildNodes}: Single-child nodes.} These are nodes that had only one child node. This type of error is unlike the previous three in that it is not a semantic error. Semantically, there is no issue with having a single child, with multiple semantic representations of ADTs allowing a single child node~\cite{mauwFoundationsAttackTrees2006,kordyFoundationsAttackDefense2011}. A single child node can be shown to be equivalent in both \AND\ and \OR\ refinements, thus technically we can admit attack trees with such refinements as valid.
The primary reason for single-child nodes to be included in this section is students were explicitly instructed to avoid using single-child nodes, as the syntactic ADT definition requires that each refined node has at least two children of the same type in either \AND\ or \OR\ relationship, and if only one child is needed, it can be absorbed in the parent node itself. We acknowledge that this argument is flawed for practical reasons, as single child nodes may be necessary to cognitively help the analysts to consider different sub-scenarios and keep the levels of abstraction of a tree consistent across different branches. However, levels of abstraction and the cognitive needs of the analysts were not a focus of our research, while the use of ADTs in a syntactically correct manner was a focus; thus, we have elected to consider single child nodes as an error.


% \begin{comment}
% There was one other semantic error; multiple root nodes where multiple nodes were present that did not have a parent node. ADT semantic representations only allow for a single root node without parents.  there were only 2 of such instances across approximately 180 ADTs (3 ADTs per participant in the large study), making this error so infrequent that any analysis of it would not be valuable. While there is no way to represent this any semantic representation of ADTs, classifying this error similarly to the ones above, it occurred so infrequently that we have decided not to conduct further analysis.
% \end{comment}



\begin{figure}[t]
    \imgResize{
        \includegraphics[width=0.8\linewidth]{img/ErrorAmounts.png}
    }
    \caption{Comparison of of the amount of semantic errors made by \ICS\ and \SEC\ students.}
    \label{fig:error-amounts}
\end{figure}

\begin{figure}[t]
    \imgResize{
        \includegraphics[width=0.8\linewidth]{img/ADT3Comparison.png}
    }
    \caption{Comparison of \ICS\ and \SEC\ students' self-drawn ADTs on \revised{quantitative}\ metrics. }.
    \label{fig:adt3-comparison}
\end{figure}



\textbf{Analysis of common errors.}
Figure~\ref{fig:error-amounts} shows the total number of errors present in the ADTs of both \ICS\ and \SEC\ students. The colored bars show the total error count; if a student made an error three times on the same ADT, then this would be counted three times in the total error count. By contrast, the small black bar inside each colored bar shows the total number of ADTs that have errors in them (the large study consisted of three separate ADTs). The small white bar within the black bar shows the total number of students who made these errors. If the height of the colored and black bars is similar, it indicates that the number of errors present per ADT is closer to 1. If the height of the white and black bar is similar, this indicates that students only made this mistake on one of their three ADTs; a significant height difference here indicates that some students made this mistake on more than one ADT.

In Figure~\ref{fig:error-amounts}, we see that multi-refinement and multi-countermeasure errors are made very infrequently at very similar rates between \ICS\ and \SEC\ students. For the single-child error (\hypothesis{\hypoSingleChildNodes}), we see that a similar number of students made these errors across similar numbers of ADTs; however, \ICS\ students made this error nearly twice as many times as \SEC\ students (this difference is statistically significant in ADT1 according to the BM test). The results of our testing can be found in Table~\ref{h3h9-table}.  Across the other errors \hypothesis{\hypoMultipleParent}, \hypothesis{\hypoMultipleRefinement}, and \hypothesis{\hypoMulipleCountermeasure}, there is no statistically significant difference between \ICS\ and \SEC\ students, but there is statistically significant equivalence according to TOST.

\highlight{\textbf{$H_3$}: We find a significant difference between the groups with respect to single-child nodes. We see evidence of groups' equivalence for all other types of errors. Overall, we find little evidence of a difference and significant evidence of equivalence between \ICS\ and \SEC\ on common errors.}

\textbf{Conclusions on the actual effectiveness of ADTs.}
We can conclude that, while we observed a statistically significant difference between the treatment groups for the two types of errors we considered, the majority of the other tested components of the actual effectiveness show the absence of a statistically significant difference between groups' performances. On some measured components, like the quality of self-drawn ADTs, the two treatment groups show statistically significant equivalent behavior. Overall, while both groups show the same lack of understanding of some aspects of ADTs, both groups have demonstrated sufficient mastery of the topic at a similar rate, allowing us to conclude that the actual effectiveness of ADTs is high for both groups.



\highlight{\RQ{1}: Actual effectiveness of ADTs is high for both groups and does not appear to be affected by technical background.}




%%%%%%%%%%%%%%%%%%%%%%%%%%%%%%%%%%%%%%%%%%%%%%%%%%%%%%%%%%%%%%%%%%%%%%%%%%%%%%%%%%%%%%%%%%%%%%%%%%%%%%%%%%%%%%%


\subsection{\RQ{2}: Effect of the background on \textbf{PU} and \textbf{PEOU}}\label{sec:rq2}


\subsubsection{Perceived Ease of Use (\textbf{PEOU})}


\textbf{\hypothesis{\hypoSelfUnderstand}: Self-assessment of understanding.}
%\label{ssec:results-understanding-self-assessment}
Alongside the check-for-understanding questions we discussed in Section~\ref{ssec:results-concept-understanding}, we asked students if they found a given ADT easy to understand. For the small study, we asked if the provided ADT was easy to understand, and for the large study, we asked if the structure of ADTs was easy to understand. These questions were all in service of the same goal: assessing how students perceived their own understanding of ADTs.

In general, \ICS\ and \SEC\ students both assessed their understanding similarly (see Figure~\ref{img:likert-understanding}). With the small study questions (labeled \texttt{SS-Q\#}), the students reported a steady decrease in their confidence in understanding. This is to be expected since, as we describe in Sec.~\ref{sec:studycomponents}, there were four ADTs with increasing complexity. The same question was asked about each ADT, and students were less confident with more complex trees.

In Table~\ref{tab:likert}, we can see that none of the understanding Likert questions shows any statistically significant difference between the groups according to the BM test (and some of the questions demonstrate significant equivalence of the groups). 

\highlight{\textbf{$H_4$}: We find evidence of equivalence between \ICS\ and \SEC\ students on self-assessment of understanding.}


\begin{table*}[t!]
\centering
\caption{Table showing the statistics and analysis of answers to Likert questions per hypothesis and treatment group.}
\label{tab:likert}
\resizebox{0.88\textwidth}{!}{
\begin{tabular}{@{}llrlrlrlrlrlrllllc@{}} 
\toprule
\textbf{Hypothesis}           & \textbf{Question}\footnotemark   & \multicolumn{2}{c}{\textbf{Str. Agree}} & \multicolumn{2}{c}{\textbf{Agree}} & \multicolumn{2}{c}{\textbf{Neither}} & \multicolumn{2}{c}{\textbf{Disagree}} & \multicolumn{2}{c}{\textbf{Str. Disagree}} & \multicolumn{2}{c}{\textbf{Average}} & \multicolumn{2}{c}{\textbf{BM test}}      & \textbf{TOST}& \textbf{Effect Size} \\&                 &     \ICS            &      \SEC           &     \ICS            &     \SEC            &     \ICS            &    \SEC   &    \ICS   &   \SEC    &   \ICS    &   \SEC    & \ICS   &     \SEC   &statistic&$p*m$         &$p*m$  &Cohen's $d$        \\ \midrule
\multirow{5}{*}{\shortstack[l]{Understanding\\(\hypothesis{\hypoSelfUnderstand})}}       

& \texttt{LS-ADT1-L1} & 21           & 26  & 17    & 19     & 2               & 0     & 2  & 2  & 0 & 1 & 1.64 & 1.6 &  -0.46& 1.0  &\revised{\textbf{1.92e-05}}&0.05\\
& \texttt{SS-Q5}      & 17           & 17  & 13    & 10     & 4               & 1     & 1  & 0  & 0 & 0 & 1.69 & 1.43 & -1.25& 1.0   &\revised{\textbf{4.30e-03}}&0.37\\
& \texttt{SS-Q10}     & 4            & 6   & 24    & 14     & 2               & 7     & 5  & 1  & 0 & 0 & 2.23 & 2.11 & -0.28& 1.0    &\revised{\textbf{3.26e-03}}&0.15\\
& \texttt{SS-Q15}     & 4            & 4   & 17    & 13     & 4               & 6     & 7  & 3  & 2 & 1 & 2.59 & 2.41 & -0.50& 1.0   &\revised{0.162}&0.17\\
& \texttt{SS-Q19}     & 2            & 5   & 14    & 6      & 5               & 8     & 8  & 7  & 4 & 1 & 2.94 & 2.74 & -0.49& 1.0    & \revised{0.395}&0.17\\\midrule

Communication (\hypothesis{\hypoCommunicationTool}) 

& \texttt{LS-ADT3-L3} & 24           & 28  & 11    & 14     & 1               & 3     & 2  & 3  & 2 & 0 & 1.68 & 1.6  & 0.06& 1.0   &\revised{\textbf{1.14e-03}}&0.08\\\midrule

\multirow{3}{*}{\shortstack[l]{Analysis\\(\hypothesis{\hypoAnalysisTool})}}            

& \texttt{LS-ADT1-L5} & 19           & 18  & 12    & 17     & 5               & 5     & 5  & 5  & 1 & 2 & 1.98 & 2.06 & 0.45& 1.0   & \revised{\textbf{0.013}}&0.07\\
& \texttt{LS-ADT2-L2} & 28           & 13  & 8     & 22     & 3               & 3     & 0  & 5  & 3 & 5 & 1.62 & 2.31 & 3.60 & \revised{\textbf{0.041}}  & & 0.57\\
& \texttt{LS-ADT3-L1} & 21           & 16  & 16    & 18     & 2               & 5     & 2  & 5  & 1 & 4 & 1.71 & 2.23 & 2.14 & \revised{1.0} &   \revised{1.0} &0.46\\ \midrule

\multirow{6}{*}{\shortstack[l]{Written\\description\\(\hypothesis{\hypoWrittenComparison})}} 

& \texttt{LS-ADT2-L1} & 18           & 24  & 12    & 12     & 5               & 6     & 3  & 3  & 4 & 3 & 2.12 & 1.94 & -0.68 & 1.0  &\revised{0.101}&0.14\\
& \texttt{LS-ADT3-L2} & 6            & 7   & 11    & 14     & 7               & 3     & 16 & 21 & 2 & 3 & 2.93 & 2.98  & 0.26& 1.0  &\revised{\textbf{0.018}}&0.04\\
& \texttt{SS-Q6}      & 11           & 13  & 13    & 13     & 7               & 0     & 4  & 1  & 0 & 1 & 2.11 & 1.71 & -1.89 & 1.0   &\revised{0.595}&0.41\\
& \texttt{SS-Q11}     & 10           & 7   & 14    & 16     & 6               & 3     & 5  & 1  & 0 & 1 & 2.17 & 2.04 & -0.51& 1.0   &\revised{\textbf{0.034}}&0.14\\
& \texttt{SS-Q16 }    & 12           & 7   & 10    & 14     & 4               & 2     & 8  & 2  & 0 & 2 & 2.24 & 2.19 & -0.05& 1.0   &\revised{0.092}&0.04\\
& \texttt{SS-Q20}     & 11           & 10  & 12    & 7      & 3               & 5     & 5  & 3  & 2 & 2 & 2.24 & 2.26 & 0.01 & \revised{0.994}     &\revised{0.146}&0.01\\
  %&0.46
\bottomrule
\end{tabular}
}
\end{table*}

%-0.576912
%-0.462689

% LS-ADT1-L1           5.000000   1.000000  8.512490e-06
%       SS-Q5                5.000000   1.000000  1.832097e-03
%       SS-Q10               5.000000    1.000000  1.413581e-03
%       SS-Q15               5.000000    1.000000  6.251503e-02
%       SS-Q19               5.000000    1.000000  1.487202e-01






\begin{figure}[t]
    \imgResize{
        \includegraphics[width=\linewidth]{img/UnderstandLikertComparison.png}
    }
    \caption{Comparison of \ICS\ and \SEC\ students on responses to questions self-assessing their understanding of ADTs. The ADTs used in the questions are referenced in Appendices~\ref{sec:small-study-q} and \ref{sec:large-study-q}.}
    \label{img:likert-understanding}
\end{figure}



\begin{figure}[t]
    \imgResize{
        \includegraphics[width=\linewidth]{img/WrittenDescriptionComparison.png}
    }
    \caption{Responses to questions concerning the preference of ADTs to a written description.}
    \label{fig:likert-written}
\end{figure}


\textbf{\hypothesis{\hypoWrittenComparison}: Written description preference.}
%\label{ssec:results-written-description-preference}
We asked students across every ADT model in the small study and across the final two ADTs in the large study if they prefer ADTs to a written description of an attack scenario. In all questions save one, there was no written description provided; students were asked if their preference was for an ADT that was either presented or to an ADT they had drawn, without an alternative written text about the scenario present (there is one exception to this: the task on building an ADT in the large study where students converted a textual attack scenario description to an ADT). The responses for both \ICS\ and \SEC\ students were similar: Table~\ref{tab:likert} shows that there is a statistically significant equivalence between \ICS\ and \SEC\ for questions in the written description category. This is also demonstrated by Figure~\ref{fig:likert-written}.


\highlight{\textbf{$H_7$}: We find evidence of equivalence between \ICS\ and \SEC\ students on preference of ADTs to a written description.}


\subsubsection{Perceived Usefulness (\textbf{PU})}


\begin{figure}[t]
    \imgResize{
        \includegraphics[width=\linewidth]{img/MeansOfCommunicationAndAnalysis.png}
    }
    \caption{Replies concerning ADTs as a means of analysis and communication.}
    \label{fig:means-of-commanalysis}
\end{figure}


\textbf{\hypothesis{\hypoCommunicationTool}\&\hypothesis{\hypoAnalysisTool}: ADTs as a means of analysis and communication.}
We asked three questions about how students perceived ADTs as a means of analysis and one question about how they perceived ADTs as a means of communication. The data shape of responses can be seen in Figure~\ref{fig:means-of-commanalysis}.



We have more detailed information in Table~\ref{tab:likert}, where we see strong equivalence between \ICS\ and \SEC\ students when considering ADTs as a means of communication. Both groups overwhelmingly agree that ADTs are useful as a tool for communicating attack scenarios. We see more agreement than disagreement about ADTs as a means of analysis, however, it is not as strong as the agreement we see for ADTs as a means of communication. Additionally, we see a statistically significant difference on two of the three questions concerning ADTs as a means of analysis. On these two questions, the \ICS\ students agreed more than the \SEC\ students that ADTs are a useful tool for analysis, with moderate effect sizes (see the Cohen's $d$ values in Table~\ref{tab:likert}).

\highlight{\textbf{$H_5\&H_6$}: \ICS\ and \SEC\ students equally perceive ADTs to be useful as a means of communication, but we find some evidence of a difference in their perceptions of ADTs as a means of analysis.}



\textbf{Conclusions on perceptions of ADTs.}
Overall, we find that the treatment groups largely perceived ADTs to be useful and easy to use (thus, the perceived efficacy is high). \textbf{PEOU} is statistically significantly equivalent in both groups, while \textbf{PU}, while similar, is not equivalent, and is significantly diverging on one measured aspect (ADTs perceived as a useful means of analysis when designing a model from a textual description). 

The only aspect for which we have found a statistically significant difference between the populations revolved around the Likert question concerning ADTs as a means of analysis. One interpretation of this result could be that \ICS\ students were introduced to a novel means of organizing information (in the tree structure), which would aid in analysis. In contrast, \SEC\ students should have seen tree structures in their previous coursework, which would lead to ADTs not introducing a new means of organizing information. 
This hypothesis would need further study in order to be tested.


\highlight{\RQ{2}: We find little evidence that the perceived efficacy of ADTs is affected by technical background. The only hypothesis \hypothesis{\hypoAnalysisTool} for which we have observed a statistically significant difference affects the perception of ADTs in a specific context only, as a means of analysis. The perceived efficacy of ADTs is high for both groups.}



%%%%%%%%%%%%%%%%%%%%%%%%%%%%%%%%%%%%%%%%%%%%%%%%%%%%%%%%%%%%%%%%%%%%%%%%%%%%%%%%%%%%%%%%%%%%%%%%%%%%%%%%%%%%%%%


\subsection{\RQ{3}: Effect of the background on \textbf{ITU}}\label{sec:rq3}



\textbf{\hypothesis{\hypoIntentionToUse}: Intention to use.}
%\label{ssec:results-intention-to-use}
We asked two open questions relevant to this hypothesis: \texttt{LS-ADT3-W3} asked the participants if they believe ADTs have a place in the cybersecurity field, and if so, where, while \texttt{LS-ADT3-W5} asked the students if they would like to see ADTs again. To analyze these questions, we applied a simple coding. If students responded in the affirmative, we applied a value of $1$ to the code ``Yes''. If the student replied in the negative, we applied a value of $0$, and if the student replied in a manner that was open to interpretation, we applied a value of $0.5$. We followed this structure for the other codes. The ``Communication'' code refers to a response describing the utility of ADTs as a means of communication and the ``Analysis'' code refers to a response describing the utility of ADTs as a means of analysis. These codes are not mutually exclusive, as many responses were coded as neither or both. In this way, we obtain a quantitative evaluation of a qualitative question. \revised{The coding guidelines were developed by the two researchers together, and several randomly selected answers from each category were evaluated independently to verify that the assessment aligns. After the establishment of the guidelines, the coding was done by a single coder (the first author of this work).  }


Table~\ref{tab:coded-future-use} contains the \ICS\ and \SEC\ averages of these codes.
We can see that there is a statistically significant equivalence between the responses. Additionally, we see that both \ICS\ and \SEC\ students strongly agreed that ADTs have a place in the cybersecurity industry, and fairly strongly agreed that they would like to see ADTs again in the future.

\highlight{\textbf{$H_8$}: We find evidence of equivalence between the treatment groups on intention to use ADTs.}



\textbf{Conclusions on intention to use ADTs.}

\highlight{\RQ{3}: The intention to use ADTs is high for both groups and is not affected by technical background.}


%%%%%%%%%%%%%%%%%%%%%%%%%%%%%%%%%%%%%%%%%%%%%%%%%%%%%%%%%%%%%%%%%%%%%%%%%%%%%%%%%%%%%%%%%%%%%%%


\subsection{\RQ{4}: Effect of the background on creative aspects of ADT design}\label{sec:rq4}

While the equivalence of two ADTs can be assessed based on a chosen semantics~\cite{mauwFoundationsAttackTrees2006}, to the best of our knowledge, ADT comparison and metrics of distance between two ADTs have not yet been investigated in the literature. Thus, we opted to compare the self-drawn ADTs based on several \revised{quantitative and qualitative}\ metrics.

\textbf{\hypothesis{\hypoThirdADT}: Self-drawn ADT comparison.}
The third task in the large study required the participants to design an ADT for their scenario of choice. As we mentioned in Sec.~\ref{sec:studycomponents}, we intentionally did not give any indication of the acceptable size for the tree, as we wanted to assess what differences, if any, would appear between ADTs drawn by \ICS\ and \SEC\ students when there are no priming restrictions, thereby evaluating the creative component.


We \revised{quantitatively}\ assessed the ADTs on 8 metrics: the total number of attack and defense nodes, the number of attack and defense leaf nodes, the number of \OR\ and \AND\ refinements, the ratio of \OR\ to \AND\ refinements, and the levels of abstraction. For these criteria, we define a leaf node as any node that does not have children of the same type. Thus, a node that only has a countermeasure edge would also be defined as a leaf node. We define levels of abstraction to be the greatest depth in the tree, not including countermeasures.

We compared \ICS\ and \SEC\ students' answers on these eight metrics using the BM test and found that there is no statistically significant difference between the ADTs drawn by \ICS\ and \SEC\ students on any metric. The results of our testing can be found in Table~\ref{h3h9-table}. Overall, we find the ADTs drawn by these two groups of students to be remarkably similar (though not equivalent in a statistically significant way).


\revised{We qualitatively evaluated the trees using two methods. Besides the quality evaluation results reported in Sec.~\ref{ssec:results-h2} that show that both groups designed ADTs with equivalent quality, we processed the labels of the root nodes, taking the main verb from each label (when present) and standardizing these (for example, ``steal'' and ``rob'' were considered equivalent in meaning). In Figure~\ref{fig:root-verbs}, we can see the prevalence of verbs across the two groups for all verbs that were present in at least two ADTs. While there are some differences in the verbs, as with the quality analysis, overall, the verbs used in the root nodes are similar between the groups.}


    \begin{figure}[h]
        \imgResize{
            \includegraphics[width=0.68\linewidth]{img/RootLabels.png}
        }
        \caption{\revised{Comparison of main verbs in the root nodes of the self-drawn ADTs.}}
        \label{fig:root-verbs}
    \end{figure}

\highlight{\textbf{$H_9$}: We find no evidence of difference between the treatment groups on self-drawn ADTs.}





\begin{table}[t!]
    \caption{Coded responses to written questions concerning the future use of ADTs.}
    \label{tab:coded-future-use}
    \resizebox{.48\textwidth}{!}{
        \begin{tabular}{@{}llllllll@{}}
            \toprule
            \textbf{Question} & \textbf{Code} & \multicolumn{2}{c}{\textbf{Average}} & \multicolumn{2}{c}{\textbf{BM Test}} & \textbf{TOST} &\textbf{Effect Size}                 \\
     &               & \ICS        & \SEC               & statistic           & $p*m$   & $p*m$  &Cohen's $d$  \\\midrule
     \texttt{LS-ADT3-W3}        & Yes           & 0.92        & 0.96               & \revised{0.90}       & 1.0   & \revised{\textbf{3.99e-40}} &\revised{0.23}\\
     \texttt{LS-ADT3-W3}        & Communication & 0.61        & 0.52               & \revised{-0.91}       & 1.0 & \revised{\textbf{3.57e-13}} &\revised{0.19}\\
     \texttt{LS-ADT3-W3}        & Analysis      & 0.44        & 0.45               & \revised{0.12}       & 1.0   & \revised{\textbf{3.98e-15}} &\revised{0.02}\\
     \texttt{LS-ADT3-W5}        & Yes           & 0.85        & 0.73               & \revised{-1.91}       & 1.0 & \revised{\textbf{1.37e-17}} &0.31\\ \bottomrule
        \end{tabular}
    }
\end{table}



\textbf{Conclusions on creative expression with ADTs.}
If ADTs were understood and used differently, we would expect to see a statistically significant difference in the ADTs created by the two groups on some qualitative or quantitative metric. As we cannot see a significant and material difference, this supports our conclusion that the technical background does not impact how ADTs are created.

\highlight{\RQ{4}: The creative component in creating ADTs is not affected by technical background.}


