




% \section{Question text for question labels in this work}

% \begin{itemize}
%     \setlength{\itemindent}{\qIndent}
%     \item[\surveyq{SS-Q7}] How many attack leaf nodes are in this ADT?
%     \item[\surveyq{SS-Q8}] How many different attack vectors are represented by this ADT?
%     \item[\surveyq{SS-Q9}] How many attack vectors do not have a defense?
%     \item[\surveyq{SS-Q10}] The attack tree is easy to understand
%     \item[\surveyq{SS-Q11}] I prefer this attack tree to a written description of this attack
%     \item[\surveyq{SS-Q12}] How many attack vectors do not have a defense?
%     \item[\surveyq{SS-Q13}] How many different attack vectors are represented by this ADT?
%     \item[\surveyq{SS-Q14}] How many levels of abstraction are present in this ADT?
%     \item[\surveyq{SS-Q15}] The attack tree is easy to understand
%     \item[\surveyq{SS-Q16}] I prefer this attack tree to a written description of this attack
%     \item[\surveyq{SS-Q17}] Is the overall goal kept? Why or why not?
%     \item[\surveyq{SS-Q18}] How many levels of abstraction are present in this ADT?
%     \item[\surveyq{SS-Q19}] The attack tree is easy to understand
%     \item[\surveyq{SS-Q20}] I prefer this attack tree to a written description of this attack
%     \item[\surveyq{LS-ADT1-L1}] I find the structure of attack tree easy to understand
%     \item[\surveyq{LS-ADT1-L5}] The process of assembling the attack tree helped me better understand the attack scenario.
%     \item[\surveyq{LS-ADT2-L1}] I prefer reading attack trees to text descriptions of attacks.
%     \item[\surveyq{LS-ADT2-L2}] The process of building the attack tree helped me better understand the attack scenario.
%     \item[\surveyq{LS-ADT3-L1}] The process of creating the attack tree helped me better understand the attack scenario I selected
%     \item[\surveyq{LS-ADT3-L2}] I feel I could have achieved the same understanding by writing a text description of the attack.
%     \item[\surveyq{LS-ADT3-L3}] The ADT I created would help me communicate my threat scenario.
%     \item[\surveyq{LS-ADT3-W3}] Do you think ADTs have a place in the cybersecurity industry? If so, where? If not, why not?
%     \item[\surveyq{LS-ADT3-W5}] Do you hope to encounter ADTs in the future?
%   \end{itemize}


% \section{Table of results}

% Table~\ref{tab:hypotheses-response} contains all of the conclusions of our hypotheses as described in Section~\ref{sec:results}.

%\hResponse{REJECT} 

% \begin{table*}[t!]
%     \caption{Conclusions on our hypotheses}
%     \label{tab:hypotheses-response}
% \resizebox{\textwidth}{!}{
%     \begin{tabular}{@{}lll@{}}
    
%         \toprule
%         \textbf{ID}    & \textbf{RQ}      &  \textbf{Conclusions} \\ \midrule


% \textbf{\nullhypothesis{\hypoCheckUnderstand}} / \althypothesis{\hypoCheckUnderstand} & \RQ{1}           
% & There is no evidence of a difference and significant evidence of equivalence in understanding of the core ADT concepts.                    \\
% \revised{\textbf{\nullhypothesis{\hypoSecondADT}} / \althypothesis{\hypoSecondADT}} & \revised{\RQ{1}\& \RQ{4}}           &  \revised{Participants successfully created ADTs at similar rates.} \\

% $\text{   }$\revised{\textbf{\nullhypothesis{\hypoSecondADTsub}} / \althypothesis{\hypoSecondADTsub}} & \RQ{1}           &  Participants successfully created an ADT from a written description at similar rates. \\

% $\text{   }$\revised{\textbf{\nullhypothesis{\hypoThirdADTsub}} / \althypothesis{\hypoThirdADTsub}} & \revised{\RQ{1} \& \RQ{4}}          &  \revised{Participants successfully created an ADT for their own scenario at similar rates.} \\


% \textbf{\nullhypothesis{\hypoErrorAmount}} / \althypothesis{\hypoErrorAmount} & \RQ{1} \& \RQ{4} &  Participants made semantic errors at similar rates.           \\


% $\text{   }$\textbf{\nullhypothesis{\hypoMultipleParent}} / \althypothesis{\hypoMultipleParent} & \RQ{1} \& \RQ{4} &  \ICS\ and \SEC\ students used nodes with multiple parents at similar rates.  \\

% $\text{   }$\textbf{\nullhypothesis{\hypoMultipleRefinement}} / \althypothesis{\hypoMultipleRefinement}  & \RQ{1} \& \RQ{4} &  \ICS\ and \SEC\ students used nodes with multiple refinements at similar rates.  \\

% $\text{   }$\textbf{\nullhypothesis{\hypoMulipleCountermeasure}} / \althypothesis{\hypoMulipleCountermeasure} & \RQ{1} \& \RQ{4} &  \ICS\ and \SEC\ students used nodes with multiple countermeasures at similar rates.  \\


% $\text{   }$\nullhypothesis{\hypoSingleChildNodes} / \textbf{\althypothesis{\hypoSingleChildNodes}} & \RQ{1} \& \RQ{4} & \ICS\ students were much more likely to have single child nodes than \SEC\ students.    \\






% \textbf{\nullhypothesis{\hypoSelfUnderstand}} / \althypothesis{\hypoSelfUnderstand} & \RQ{2} &  Participants self-report understanding at statistically significantly equivalent rates.       \\

% \textbf{\nullhypothesis{\hypoCommunicationTool}} / \althypothesis{\hypoCommunicationTool} & \RQ{2}           &  Results do not support a difference between \ICS\ and \SEC\ for ADTs as a communication tool.    \\




% \nullhypothesis{\hypoAnalysisTool} / \textbf{\althypothesis{\hypoAnalysisTool}} & \RQ{2}           &  \ICS\ students had a significantly stronger preference for ADTs as an analysis tool compared to \SEC\ students.                         \\


% \textbf{\nullhypothesis{\hypoWrittenComparison}} / \althypothesis{\hypoWrittenComparison} & \RQ{2}            &  Participants preferred ADTs over written descriptions at similar rates.           \\

% \nullhypothesis{\hypoIntentionToUse} / \althypothesis{\hypoIntentionToUse} & \RQ{3} &  We find evidence of equivalence in the intention to use between the two groups. \\

%         % \nullhypothesis{7} & \RQ{2}           & The comparison of ADTs as a means of analysis will be different between \ICS\ and \SEC students                             \\
% \textbf{\nullhypothesis{\hypoThirdADT}} / \althypothesis{\hypoThirdADT} & \RQ{4}           &  \revised{There were no significant qualitative or quantitative differences between the two groups for the self-drawn ADT.}                                \\
%         \bottomrule
%     \end{tabular}
%     }
% \end{table*}

\footnotesize

\section{Small Study}\label{sec:small-study-q}

%\textbf{Consent.}
%This section consisted of a single question explaining the study goals and the data collected, and asking if participants consented to have their responses included in this study.

\noindent\textbf{ADT 1.}
The ADT for the following questions was created by Buldas~\etal\. It can be found on page four labeled as Figure~1~\cite{buldasAttributeEvaluationAttack2020}.

\begin{itemize}
    \setlength{\itemindent}{\qsIndent}
\item[\surveyq{SS-Q2}]  How many leaf nodes are in this ADT?
\item[\surveyq{SS-Q3}]  How many root nodes are in this ADT?
\item[\surveyq{SS-Q4}]  How many different attack vectors are represented by this ADT?
\item[\surveyq{SS-Q5}]  The attack tree is easy to understand
\item[\surveyq{SS-Q6}]  I prefer this attack tree to a written description of this attack
\end{itemize}

\noindent\textbf{ADT 2.}
The ADT for the following questions is shown in Figure~\ref{img:ss-adt2}.

\begin{itemize}
    \setlength{\itemindent}{\qsIndent}
    \item[\surveyq{SS-Q7}] How many attack leaf nodes are in this ADT?
    \item[\surveyq{SS-Q8}] How many different attack vectors are represented by this ADT?
    \item[\surveyq{SS-Q9}] How many attack vectors do not have a defense?
    \item[\surveyq{SS-Q10}] The attack tree is easy to understand
    \item[\surveyq{SS-Q11}] I prefer this attack tree to a written description of this attack
\end{itemize}


% \caption{Third ADT in the small study~\cite{mauwRFIDCommunicationBlock}}\caption{Fourth ADT in the small study~\cite{kordyAttackdefenseTrees2014}}

\noindent\textbf{ADT 3.}
The ADT for the following questions was created by Mauw and Oostdijk~\cite{mauwRFIDCommunicationBlock}.
\begin{itemize}
    \setlength{\itemindent}{\qsIndent}
    \item[\surveyq{SS-Q12}] How many attack vectors do not have a defense?
    \item[\surveyq{SS-Q13}] How many different attack vectors are represented by this ADT?
    \item[\surveyq{SS-Q14}] How many levels of abstraction are present in this ADT?
    \item[\surveyq{SS-Q15}] The attack tree is easy to understand
    \item[\surveyq{SS-Q16}] I prefer this attack tree to a written description of this attack
\end{itemize}


\noindent\textbf{ADT 4.}
The ADT for the following questions was created by Kordy~\etal\ can be found on page 58 of that work labeled Figure~1~\cite{kordyAttackdefenseTrees2014}.

\begin{itemize}
    \setlength{\itemindent}{\qsIndent}
    \item[\surveyq{SS-Q17}] Is the overall goal kept? Why or why not?
    \item[\surveyq{SS-Q18}] How many levels of abstraction are present in this ADT?
    \item[\surveyq{SS-Q19}] The attack tree is easy to understand
    \item[\surveyq{SS-Q20}] I prefer this attack tree to a written description of this attack
\end{itemize}






































% \begin{comment}
%     \section{Small Study}


% \subsection*{Consent}

% This section consisted of a single question asking if participants consented to have their responses included in this study.

% \subsection{ADT 1}

% The ADT for the following questions is shown in figure~\ref{img:ss-adt1}.

% \begin{enumerate}
%     \setcounter{enumi}{1}
% \item How many leaf nodes are in this ADT?
% \item How many root nodes are in this ADT?
% \item How many different attack vectors are represented by this ADT?
% \item The attack tree is easy to understand
% \item I prefer this attack tree to a written description of this attack
% \end{enumerate}

% \subsection{ADT 2}


% The ADT for the following questions is shown in figure~\ref{img:ss-adt2}.

% \begin{enumerate}
%     \setcounter{enumi}{6}
%     \item How many attack leaf nodes are in this ADT?
%     \item How many different attack vectors are represented by this ADT?
%     \item How many attack vectors do not have a defense?
%     \item The attack tree is easy to understand
%     \item I prefer this attack tree to a written description of this attack
% \end{enumerate}

% \subsection{ADT 3}
% The ADT for the following questions is shown in figure~\ref{img:ss-adt3}.
% \begin{enumerate}
%     \setcounter{enumi}{11}
%     \item How many attack vectors do not have a defense?
%     \item How many different attack vectors are represented by this ADT?
%     \item How many levels of abstraction are present in this ADT?
%     \item The attack tree is easy to understand
%     \item I prefer this attack tree to a written description of this attack
% \end{enumerate} 



% \subsection{ADT 4}

% The ADT for the following questions is shown in figure~\ref{img:ss-adt4}.

% \begin{enumerate}
%     \setcounter{enumi}{16}
%     \item Is the overall goal kept? Why or why not?
%     \item How many levels of abstraction are present in this ADT?
%     \item The attack tree is easy to understand
%     \item I prefer this attack tree to a written description of this attack
% \end{enumerate}






% \end{comment}
\footnotesize

\section{Large Study}\label{sec:large-study-q}

\subsubsection*{ADT 1: Assembling ADTs}
The following attack \textbf{leaf} nodes are provided. The overall goal of this scenario (and thus the root node of the tree) is \textbf{Rob bank}. Assemble an attack-defense tree using these leaf nodes. Do not add any additional leaf nodes. You may add any intermediary nodes you wish.

\textbf{Attack leaf nodes:} 
Hire Outright; Promise part of the stolen money; Threaten insiders; Buy tools; Steal tools; Gain Access; Walk through front door; Locate start of tunnel; Find direction to tunnel.

\textbf{Defense leaf nodes:} 
Personnel Risk Management; Check employee financial situation.



%\noindent\textbf{Perception Questions.}


\noindent\textbf{Likert Questions.}
\begin{itemize}
  \setlength{\itemindent}{\qIndent}
  \item[\surveyq{LS-ADT1-L1}] I find the structure of attack tree easy to understand
  \item[\surveyq{LS-ADT1-L2}] Given all the nodes of an attack tree, it is easy for me to assemble the tree
  \item[\surveyq{LS-ADT1-L3}] Given only the leaf nodes of an attack tree, it is easy for me to assemble the tree.
  \item[\surveyq{LS-ADT1-L4}] I would rather define my own intermediary nodes
  \item[\surveyq{LS-ADT1-L5}] The process of assembling the attack tree helped me better understand the attack scenario.
\end{itemize}

\noindent\textbf{Short Response Questions.}
\begin{itemize}
  \setlength{\itemindent}{\qIndent}
  \item[\surveyq{LS-ADT1-W1}] What did you find most difficult about this task? Why?
  \item[\surveyq{LS-ADT1-W2}] How did you go about solving this task? What was your methodology?
\end{itemize}



\subsubsection*{ADT 2: Building ADTs}
The following text scenario is provided for you. Please create a complete attack defense tree \textbf{of this scenario}. \textbf{Do not add extra information that is not in the scenario}. Try to encapsulate the entire scenario with an attack-defense tree (don't leave any aspect of the attack scenario out).

\emph{Scenario:} 
The goal is to open a safe. To open the safe, an attacker can pick the lock,
learn the combination, cut open the safe, or install the safe improperly so
that he can easily open it later. Some models of safes are such that they cannot be picked, so if this model is used, then an attacker is unable to pick the lock. There are also auditing services to check if safes and other security technology is installed correctly. To learn the combination, the attacker
either has to find the combination written down or get the combination
from the safe owner. If the password is such that the safe owner can remember it, then the safe owner would not need to write it down.



%\noindent\textbf{Perception Questions.}

\noindent\textbf{Likert Questions.}
\begin{itemize}
  \setlength{\itemindent}{\qIndent}
  \item[\surveyq{LS-ADT\revised{2}-L1}] I prefer reading attack trees to text descriptions of attacks.
  \item[\surveyq{LS-ADT\revised{2}-L2}] The process of building the attack tree helped me better understand the attack scenario.
\end{itemize}

\noindent\textbf{Short Response Questions.}
\begin{itemize}
  \setlength{\itemindent}{\qIndent}
  \item[\surveyq{LS-ADT\revised{2}-W1}] What did you find most difficult about this task? Why?
  \item[\surveyq{LS-ADT\revised{2}-W2}] How did you go about building the ADT?\@ What was your methodology?
  \item[\surveyq{LS-ADT\revised{2}-W3}] What was the first node you added to your tree?
\end{itemize}




\subsubsection*{ADT 3: Creating ADTs}
Construct an attack defense tree of a scenario of your choice. Your tree should be complete (covers all reasonable attack scenarios) and reasonably large.


%\noindent\textbf{Perception Questions.}

\noindent\textbf{Likert Questions.}
\begin{itemize}
  \setlength{\itemindent}{\qIndent}
  \item[\surveyq{LS-ADT\revised{3}-L1}] The process of creating the attack tree helped me better understand the attack scenario I selected
  \item[\surveyq{LS-ADT\revised{3}-L2}] I feel I could have achieved the same understanding by writing a text description of the attack.
  \item[\surveyq{LS-ADT\revised{3}-L3}] The ADT I created would help me communicate my threat scenario.
\end{itemize}

\noindent\textbf{Short Response Questions.}
\begin{itemize}
  \setlength{\itemindent}{\qIndent}
  \item[\surveyq{LS-ADT\revised{3}-W1}] What did you find easy about using ADTs?
  \item[\surveyq{LS-ADT\revised{3}-W2}] What did you find difficult about using ADT?\@
  \item[\surveyq{LS-ADT\revised{3}-W3}] Do you think ADTs have a place in the cybersecurity industry? If so, where? If not, why not?
  \item[\surveyq{LS-ADT\revised{3}-W4}] What aspects, if any, do you think are missing from ADTs?
  \item[\surveyq{LS-ADT\revised{3}-W5}] Do you hope to encounter ADTs in the future?
\end{itemize}










































% \begin{comment}
%   \section{Large Study}

% \subsection{ADT 1: Assembling ADTs}

% The following attack \textbf{leaf} nodes are provided. The overall goal of this scenario (and thus the root node of the tree) is \textbf{Rob bank}. Assemble an attack-defense tree using these leaf nodes. Do not add any additional leaf nodes. You may add any intermediary nodes you wish.

% Attack leaf nodes:
% % Why the fuck is there so much space here?
% % \vspace{-10cm}
% \begin{itemize}
%   \item Hire Outright
%   \item Promise part of the stolen money
%   \item Threaten insiders

%   \item Buy tools
%   \item Steal tools
%   \item Gain Access
%   \item Walk through front door
%   \item Locate start of tunnel
%   \item Find direction to tunnel
% \end{itemize}

% Defense leaf nodes:
% \begin{itemize}
%   \item Personnel Risk Management
%   \item Check employee financial situation
% \end{itemize}


% \subsection*{Perception Questions}

% \normalsize
% \subsubsection{Likert Questions}
% \begin{enumerate}
%   \item I find the structure of attack tree easy to understand
%   \item Given all the nodes of an attack tree, it is easy for me to assemble the tree
%   \item Given only the leaf nodes of an attack tree, it is easy for me to assemble the tree.
%   \item I would rather define my own intermediary nodes
%   \item The process of assembling the attack tree helped me better understand the attack scenario.
% \end{enumerate}

% \subsubsection{Short Response Questions}
% \begin{enumerate}
%   \item What did you find most difficult about this task? Why?
%   \item How did you go about solving this task? What was your methodology?
% \end{enumerate}



% \subsection{ADT 2: Building ADTs}

% The following text scenario is provided for you. Please create a complete attack defense tree \textbf{of this scenario}. \textbf{Do not add extra information that is not in the scenario}. Try to encapsulate the entire scenario with an attack-defense tree (don't leave any aspect of the attack scenario out).
% \\~\\
% Scenario:
% \\~\\
% The goal is to open a safe. To open the safe, an attacker can pick the lock,
% learn the combination, cut open the safe, or install the safe improperly so
% that he can easily open it later. Some models of safes are such that they cannot be picked, so if this model is used, then an attacker is unable to pick the lock. There are also auditing services to check if safes and other security technology is installed correctly. To learn the combination, the attacker
% either has to find the combination written down or get the combination
% from the safe owner. If the password is such that the safe owner can remember it, then the safe owner would not need to write it down.



% \subsection*{Perception Questions}

% \normalsize
% \subsubsection{Likert Questions}
% \begin{enumerate}
%   \item I prefer reading attack trees to text descriptions of attacks.
%   \item The process of building the attack tree helped me better understand the attack scenario.
% \end{enumerate}

% \subsubsection{Short Response Questions}
% \begin{enumerate}
%   \item What did you find most difficult about this task? Why?
%   \item How did you go about building the ADT?\@ What was your methodology?
%   \item What was the first node you added to your tree?
% \end{enumerate}



% \subsection{ADT 3: Creating ADTs}

% Construct an attack defense tree of a scenario of your choice. Your tree should be complete (covers all reasonable attack scenarios) and reasonably large.


% \subsection*{Perception Questions}

% \normalsize
% \subsubsection{Likert Questions}
% \begin{enumerate}
%   \item The process of creating the attack tree helped me better understand the attack scenario I selected
%   \item I feel I could have achieved the same understanding by writing a text description of the attack.
%   \item The ADT I created would help me communicate my threat scenario.
% \end{enumerate}

% \subsubsection{Short Response Questions}
% \begin{enumerate}
%   \item What did you find easy about using ADTs?
%   \item What did you find difficult about using ADT?\@
%   \item Do you think ADTs have a place in the cybersecurity industry? If so, where? If not, why not?
%   \item What aspects, if any, do you think are missing from ADTs?
%   \item Do you hope to encounter ADTs in the future?
% \end{enumerate}

% \end{comment}


% \revised{
% \section{Lecture description}\label{sec:training-description}

% The following is a detailed description of the lecture used as training for the study participants. The provided timing indications are approximate, and the total time of the training (without a break) is about 90 min. The training requires a whiteboard (or another means to quickly draw ADTs in front of an audience). 
% %While the alloted lecture time is 105 minutes, this time also needed to include a break and time for questions, the actual lecture time was closer to 75 minutes.

% \begin{enumerate}
%     \setlength{\itemindent}{\qIndent}
%     \item[6 Min.:] Threat modeling in the context of risk management (why threat modeling is used). The main goals of threat modeling (analysis and communication of threats), as well as why threat modeling is valuable.
%     \item[8 Min.:] Short description of some specific threat modeling methods (STRIDE, Cyber Kill Chain) with examples and clarifications in which contexts they apply. 
%     \item[8 Min.:] Introduction to attack graphs. This introduces the idea of graphs (in graph theory) and how they can be used to model attacks.  
%     \item[15 Min.:] Introduction to attack trees (ATs). This includes how they were developed, the core AT components and design rules (nodes, refinements, levels of abstraction, requirements related to the shape of the tree), a short explanation of how they differ from attack graphs, and how they are read. The lunch example from \cite[Fig. 1]{mauwFoundationsAttackTrees2006} is used to demonstrate attack trees. 
%     \item[20 Min.:] 1st interactive activity. Participants work with the people near them to create an attack tree of a scenario of their choice. Volunteers are asked to draw their attack trees on whiteboards at the front of the lecture hall. After these are drawn, the students present their attack trees and then the lecturer discusses the merits, construction issues, and avenues for improvement as a learning exercise. Trees are NOT erased from the whiteboard as they will be used in the next section. At this moment, the lecturer also encourages questions from the audience about the model.
%     \item[10 Min.:] Introduction of defense nodes and ADTs, along with their construction rules (one child node of the opposite type). Examples from \cite{kordyAttackdefenseTrees2014} are used as support material. 
%     \item[15 Min.:] 2nd interactive activity. Participants take their ATs from the previous activity and attach defense nodes to them. Participants are also encouraged to think of attacks against their countermeasures. Countermeasures for the ATs drawn on the whiteboard (from the previous activity) are discussed as a class and drawn by the lecturer.
%     \item[8 Min.:] Overview of ADT expansions. This includes models such as \SAND\ trees (examples from \cite{jhawarAttackTreesSequential2015}) and Bayesian ATs (examples from \cite{meyurBayesianAttackTree2020}). This is not necessary to understand the core concepts of ADTs, but is included to show that ADTs are a flexible tool.
% \end{enumerate}

% To adapt this to the industry context, the lecture could be shortened to remove the description of other threat modeling methods and the overview of ADT expansions. Further, the interactive activities could be tailored to the needs of the audience (using the target systems as scenarios, changing the duration, etc.).

% }

%\section{Supplementary Figures and Tables}\label{sec:likert_figures}





% \section{Codes for study question}
% \begin{table*}[h!]
% \resizebox{\textwidth}{!}{
%     \begin{tabular}{@{}ll@{}}
%         \toprule
%         \textbf{Code} & \textbf{Question}                                                                                   \\ \midrule
%         LS-ADT1-L1    & I find the structure of (attack)tree easy to understand                                              \\
%         LS-ADT1-L2    & Given all the nodes of an (attack)tree, it is easy for me to assemble the tree                       \\
%         LS-ADT1-L3    & Given only the leaf nodes of an (attack)tree, it is easy for me to assemble the tree.                \\
%         LS-ADT1-L4    & I would rather defenseine my own intermediary nodes                                                     \\
%         LS-ADT1-L5    & The process of assembling the (attack)tree helped me better understand the (attack)scenario.          \\
%         LS-ADT1-W1    & What did you find most difficult about this task? Why?                                              \\
%         LS-ADT1-W2    & How did you go about solving this task? What was your methodology?                                  \\
%         LS-ADT2-L1    & I prefer reading (attack)trees to text descriptions of (attack).                                      \\
%         LS-ADT2-L2    & The process of building the (attack)tree helped me better understand the (attack)scenario.            \\
%         LS-ADT2-W1    & What did you find most difficult about this task? Why?                                              \\
%         LS-ADT2-W2    & How did you go about building the ADT? What was your methodology?                                   \\
%         LS-ADT2-W3    & What was the first node you added to your tree?                                                     \\
%         LS-ADT3-L1    & The process of creating the (attack)tree helped me better understand the (attack)scenario I selected. \\
%         LS-ADT3-L2    & I feel I could have achieved the same understanding by writing a text description of the (attack)    \\
%         LS-ADT3-L3    & The ADT I created would help me communicate my threat scenario.                                     \\
%         LS-ADT3-W1    & What did you find easy about using ADTs?                                                            \\
%         LS-ADT3-W2    & What did you find difficult about using ADT?                                                        \\
%         LS-ADT3-W3    & Do you think ADTs have a place in the cybersecurity industry? If so, where? If not, why not?        \\
%         LS-ADT3-W4    & What aspects, if any, do you think are missing from ADTs?                                           \\
%         LS-ADT3-W5    & Do you hope to encounter ADTs in the future?                                                        \\\midrule
%         SS-Q2         & How many leaf nodes are in this (attack)tree?                                                        \\
%         SS-Q3         & How many root nodes are in this (attack)tree?                                                        \\
%         SS-Q4         & How many different (attack)vectors are in this (attack)tree?                                          \\
%         SS-Q5         & The (attack)tree is easy to understand                                                               \\
%         SS-Q6         & I prefer this (attack)tree to a written description of this (attack)                                  \\
%         SS-Q7         & How many leaf nodes are in this ADT?                                                                \\
%         SS-Q8         & How many different (attack)vectors are represented by this ADT?                                      \\
%         SS-Q9         & How many (attack)vectors do not have a defenseense?                                                      \\
%         SS-Q10        & The (attack)tree is easy to understand                                                               \\
% SS-Q11        & I prefer this (attack)tree to a written description of the (attack)                                   \\
%         SS-Q12        & How many (attack)vectors do not have a defenseense?                                                      \\
%         SS-Q13        & How many different (attack)vectors are represented by this ADT?                                      \\
%         SS-Q14        & How many levels of abstraction are present in this ADT?                                             \\
%         SS-Q15        & The (attack)tree is easy to understand                                                               \\
%         SS-Q16        & I prefer this (attack)tree to a written description of this (attack)                                  \\
%         SS-Q17        & Is the overal kept? Why or why not?                                                                 \\
%         SS-Q18        & How many levels of abstraction are present in this ADT?                                             \\
%         SS-Q19        & This (attack)tree is easy to understand                                                              \\
%         SS-Q20        & I prefer this (attack)tree to a written description of this (attack)                                  \\ \bottomrule
% \end{tabular}
% }
% \caption{Questions in both studies and their corresponding codes}
% \label{tab:appendix-question-codes}
% \end{table*}


 



    % \begin{table*}[]
    %     \caption{Qualitative evaluation rubric for accessing the freely drawn ADTs (ADT3)}
    %     \label{tab:qualitative-evaluation}
    %  \resizebox{\linewidth}{!}{
    %     \revised{
    % \begin{tabular}{p{0.1\linewidth}p{0.07\linewidth}p{0.1\linewidth}p{0.25\linewidth}p{0.25\linewidth}p{0.25\linewidth}}
    %     \toprule
    %     \multirow{3}{*}{\textbf{Component}} & \multirow{3}{*}{\textbf{Checking}} & \multirow{3}{*}{\textbf{Question}}                                                                                     & \multicolumn{3}{c}{\textbf{Evaluations}}                                                                                                                                                                                                                                                                                                                                                                                                                                                                                                                                                                                                                                                                                                                                                                                                                                                                                                                            \\
    %                                         &                                    &                                                                                                                        & \multicolumn{1}{c}{\textbf{1}}                                                                                                                                                                                                                                                                             & \multicolumn{1}{c}{\textbf{2}}                                                                                                                                                                                                                                                                         & \multicolumn{1}{c}{\textbf{3}}                                                                                                                                                                                                                                                                                                                \\
    %                                         &                                    &                                                                                                                        & \multicolumn{1}{c}{\textbf{Largely correct}}                                                                                                                                                                                                                                                              & \multicolumn{1}{c}{\textbf{Neither correct nor incorrect}}                                                                                                                                                                                                                                             & \multicolumn{1}{c}{\textbf{Largely incorrect}}                                                                                                                                                                                                                                                                                               \\ \midrule
    %     Cohesiveness                        & Child nodes                        & Are the child nodes components of their parent?                                                                        & For most/all nodes, children are components of their parents. That is to say that children are on a different level of abstraction from their children. Some children may be poorly related, but are representative of a different level of abstraction. No children that are sequential steps are present & There are some nodes with poor relationships to their parents. An example would be a standard attack tree with a child that is a sequential step as opposed to a component (poor abstraction). Further, several poorly related children (that are still correctly abstracted) could also classify here & Many child nodes have poor relationships to their parents. This can include children which should belong on the same level of abstraction as their parent (i.e. not a child) or children that are unrelated to the parent. Significant examples of sequential step children (i.e. kill chains appended to ADT nodes) would be classified here \\
    %     Clarity                             & Node labels                        & Is it clear what the intended meaning of the nodes is?                                                                 & The meaning of all/almost nodes can be reasonably inferred or guessed. This can include looking at siblings/parents/children to infer the meaning of node labels. A precise meaning is not needed, but a general understanding of the reference should be clear                                            & Some nodes are unclear in what the intended meaning is. This can include nodes that have multiple meanings (an the intended reading of the label is unclear).                                                                                                                                          & The meaning of many nodes cannot be easily interpreted/inferred. For these nodes, it should be possible to describe the different possible meanings. The understanding of the tree as a whole suffers due to the number of nodes which are unclear.                                                                                           \\
    %     Concise                             & Nodes present                      & Are the nodes that are present in the attack tree that are irrelevant, excessive or do not contribute to the scenario? & Most nodes are relevant to the overall tree. There may be some nodes that are tangentially related or excessive, but a reasonable argument can be constructed for the placement of most nodes within the tree                                          & Several nodes are excessive, incorrect or too detailed for the given scenario in the context of their parent.                                   & There are many nodes which are out of place, irrelevant, or excessive. The overall ADT would be improved or be easier to understand if these nodes were removed. The overall ADT is significantly affected by nodes, it is hard to tell which nodes are important because of the number of excessive nodes                                    \\
    %     Completeness                        & Nodes not present                  & Are there any obvious components of the attack scenario that would be expected to be present and aren't?               & In general, there are no obvious additions to the ADT that would be expected to be included. This only applies to the siblings of nodes, we do not evaluate if children should be added to leaf nodes. This should not be applied to nodes deemed excessive under "concise" or unrelated under "cohesiveness".                                                                                                      & There may be some (obvious) addition(s) that would be expected. Overall, the ADT is sufficient on its own but would be improved by the addition of the obvious addition(s). This only applies to the siblings of nodes, we do not evaluate if children should be added to leaf nodes. This should not be applied to nodes deemed excessive under "concise" or unrelated under "cohesiveness".                   & There are several obvious additions that need to be present in the ADT. The ADT without these additions is incomplete and/or insufficient. This only applies to the siblings of nodes, we do not evaluate if children should be added to leaf nodes. This should not be applied to nodes deemed excessive under "concise" or unrelated under "cohesiveness".                                                                                           \\ \bottomrule        
    % \end{tabular}
    % }}
    % \end{table*}
    

    % The ADT has intermediate nodes which ideally should have been leaf nodes. That is, the children of the intermediate nodes should be present, and represent too much extraction.