


\section{Introduction}
\label{sec:introduction}


Threat modeling has taken an increasingly prominent role in risk assessment and security-oriented design~\cite{andersonSecurityEngineeringGuide2020}, especially in the area of secure software engineering~\cite{yskout2020threat,howard2003inside,apvrille2005secure}. \emph{Attack-defense trees} (ADTs), a graphical component-based representation of attack scenarios, are a highly recommended model for analyzing attacks as well as communicating attack-related information to others in a succinct manner~\cite{andersonSecurityEngineeringGuide2020,ncsc2023attacktrees}. ADTs have been long considered to be a suitable, versatile, and easy-to-use threat modeling approach~\cite{schneierSecretsLiesDigital2000,shostack2014threat,tarandach2020threat,saini2008threat,shevchenko2018threat,apvrille2005secure,stringhini2021adversarial,ncsc2023attacktrees,reversinglabs2024attacktrees}. However, as threat modeling process and results \emph{need to be accessible to people with different backgrounds}~\cite{dev2023models,brunner2020risk}, in order to be effective as a threat modeling method, ADTs must be acceptable for all stakeholders in the software development process, including, among others, security analysts, software engineers, product owners, and managers~\cite{verreydt2024threat}. For a model to be \emph{acceptable}, stakeholders need to be able to use the model efficiently and effectively, as well as perceive the model to be useful and usable~\cite{moodyMethodEvaluationModel2003}.

Thus far, there have been relatively few studies focusing on ADT acceptability. A few studies have directly compared attack trees with other threat models. For example, Opdahl and Sindre~\cite{opdahlExperimentalComparisonAttack2009} and Karpati et al.~\cite{karpatiComparingAttackTrees2014} found that attack trees allowed for better analysis than misuse cases. Broccia \etal~\cite{broccia_assessing_2024,broccia2025evaluating} have recently demonstrated high comprehensibility and acceptability of ADTs for users with a technical background. Lallie \etal~\cite{lallieEmpiricalEvaluationEffectiveness2017} compared fault trees (the precursor to attack trees) and attack graphs, a temporal state-based threat model~\cite{schieleNovelApproachAttack2021}, in a study with participants of different backgrounds. They found that those with a technical background strongly outperformed those without on both models~\cite{lallieEmpiricalEvaluationEffectiveness2017}.  


\revised{As threat modeling is an important part of the secure software development lifecycle~\cite{lipner2023inside,verreydt2024threat} with a strong focus on collaboration~\cite{jawad2024m,verreydt2024threat}, it is crucial to understand whether such a popular and recommended method as ADTs is suitable for all involved stakeholders who might have a very limited technical background.}\ To address this gap in the acceptability of ADTs, re-examine the findings from \cite{lallieEmpiricalEvaluationEffectiveness2017}, and guide our research, we formulated the following research questions following the Method Evaluation Model (MEM) as described by Moody~\cite{moodyMethodEvaluationModel2003}:

\begin{itemize}
\setlength{\itemindent}{1.2em}
    \item[\RQ{1}] Is the actual effectiveness of ADTs affected by technical background?
    \item[\RQ{2}] Are the perceived ease of use or perceived usefulness of ADTs affected by technical background?
    \item[\RQ{3}] Is the intention to use ADTs affected by technical background?
    \item[\RQ{4}] Does technical background impact how ADTs are drawn?
\end{itemize}

Our work aims to establish \emph{whether the extent of the technical background affects the ADT acceptability} by conducting a study with student participants from different fields (53 computer science participants; 49 non-computer science participants with a very limited technical background) who complete the same suite of tasks involving using and creating ADTs. Our study is the first to examine ADTs in this context, and, to the best of our knowledge, our study is the first to examine the creative aspect of using any threat model by having participants create an ADT for a scenario of their own choosing and comparing the resulting set of ADTs.


\textbf{Our main findings are}:
\begin{itemize}
    \item \emph{A limited technical (computer science) background is sufficient for the acceptability of ADTs}: Participants of different backgrounds did not show a significant difference in their usage or perceptions of ADTs.

    \item \emph{A creative component in designing ADTs does not appear to be affected by the background}: All self-drawn ADTs fell within the same general limits (in terms of the number of nodes, depth, refinements, etc.), \revised{represent similar types of scenarios, and are of similar quality}, regardless of the background of their authors.
\end{itemize}

Overall, our results strongly support ADTs as a threat modeling tool that is acceptable for threat modeling stakeholders, including those with a very limited technical (computer science) background. We share our study design, training materials, and the anonymized data from the participants in~\cite{zenodo-dataset} to enable further research in this field. 

The remainder of this paper is structured as follows. \revised{We first present the necessary background information on ADTs and threat modeling and summarize the relevant state-of-practice in threat modeling in Sec.~\ref{sec:background}.}\ We then review the related work on empirical studies in Sec.~\ref{sec:empirical_studies}. Sec.~\ref{sec:methodology} presents the methodology of our study. It is followed by Sec.~\ref{sec:results} presenting the study results and answering our four key research questions. We discuss the results in Sec.~\ref{sec:discussion} and acknowledge the study limitations in Sec.~\ref{sec:limitations}. Sec.~\ref{sec:conclusions} concludes this paper. 




