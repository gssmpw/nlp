\section{Analysis}\label{sec:analysis}

\subsection{Concerning understanding}
\label{ssec:analysis-concerning-understanding}

In Section~\ref{ssec:results-concept-understanding} we see that on four out of five metrics assessing student understanding, there is no significant difference between the \ICS\ and \SEC\ students. This would suggest that the level of understanding between the two groups is similar and is not affected by the technical background. There is one concept (leaf nodes) where there is a significant difference. Specifically, \SEC\ students performed better on these tasks compared to \ICS\ students, which would be the result we could expect for all concepts based on the results from the
%This assumption is derived from the idea that tree data structures are a common concept within computer science and the \SEC\ students would have likely seen a similar concept in their prior coursework, which would predispose the \SEC\ students to a stronger understanding of the same concepts within ADTs. In contrast, the \ICS\ students would likely not have seen tree-like data structures, if any complex data structures at all, in their prior coursework. From this and 
study by Lallie \etal~\cite{lallieEmpiricalEvaluationEffectiveness2017}, we expected a significant difference in the measurable conceptual understanding of ADTs between \ICS\ and \SEC\, with specifically \SEC\ students performing better. However, we do not see this on most concepts. While the \SEC\ students seem to perform slightly better than the \ICS\ students, which provides some validation for our assumptions, the \SEC\ students do not significantly outperform the \ICS\ students. When looking at the TOST results, we find that the \SEC\ and \ICS\ students have statistically significant equivalence of understanding.

When looking at these results, As such, we accept the null hypothesis~\nullhypothesis{\hypoCheckUnderstand} and reject the alternative hypothesis~\althypothesis{\hypoCheckUnderstand}. There is evidence to support that there is equivalence between \ICS\ and \SEC\ students with respect to understanding core ADT concepts. This helps establish that the relatively short training we provided, as described in Section~\ref{ssec:training}, was adequate for the understanding and use of ADTs. This would further indicate that any differences we find between \ICS\ and \SEC\ on our other hypotheses are due to the difference in technical background. As for students' self-perception of understanding, as described in Section~\ref{ssec:results-understanding-self-assessment}, there is no statistically significant difference between how the \ICS\ and the \SEC\ students perceive their own understanding of ADTs. As such, we must accept the null-hypothesis \nullhypothesis{\hypoSelfUnderstand} and reject the alternative hypothesis \althypothesis{\hypoSelfUnderstand} as well, as we both find evidence of equivalence and do not find evidence of a difference difference in the self-perception of ADTs.

Overall, it appears that our training was successfully able to both convey the key concepts of ADTs as well as the confidence in understand of those concepts. This implies that in the professional context, stakeholders with and without a technical background can be relatively easily trained to understand attack trees and be confident when interpreting them.



\subsection{Comparison of ADTs to a written description}

We asked students if they preferred ADTs to a written description of attacks six times across the two studies. Across all six questions, we did not see a statistically significant difference and see a statistically significant equivalence. As such we cannot reject the alternative hypothesis \althypothesis{\hypoWrittenComparison} and we reject the null hypothesis~\hypothesis{\hypoWrittenComparison}.

However, we do notice that for self drawn ADT task there is a difference in response to this question, despite the question being identical to the other times this question was asked. As we can see in Figure~\ref{fig:likert-written}, there are more students from both \ICS\ and \SEC\ who disagree with the preference for ADTs over written descriptions specifically when asked about the self-created ADT. Given that this change occurs at the same rate for both \ICS\ and \SEC\ we believe that this change is not due to the background difference. The reasons behind this change in preference would need to be further explored.


\subsection{Errors}
\label{ssec:analysis-errors}

We assessed four types of common errors described in Section~\ref{ssec:results-common-errors}. We provided the total number of errors made, the number of ADTs with errors and the number of students who made errors separated by \ICS\ and \SEC\ in Figure~\ref{fig:error-amounts}.

We can see in this figure that the total number of errors of the Multi-parent and Multi-refinement types to be very low. Each error was generally made once by students across all of their ADTs, with a few students making this error in more than one of their ADTs or more than one time in a single ADT. There is no statistically significant difference in the occurrence of this error between the two groups of students. The occurrence of these errors is so infrequent, that it can likely be attributed to a few students having a misunderstanding of the rules of ADT construction, which is not affected by the background.% rather than an fault in our training.

For the multiple countermeasures error type, we again do not see a statistically significant difference. However, unlike the previous two errors, multiple countermeasures occur significantly more frequently. Additionally, several students included an \AND\ refinement in their ``illegal'' multiple countermeasures. In fact, when we compare most of the multiple countermeasures present in the collected ADTs, we can strongly assume that many students intended for their ADTs to be read as the multiple countermeasures having an \OR\ relationship. Our training did ensure to include that nodes can have at most one countermeasure (node of the opposite type); however, despite this instruction, students frequently included multiple countermeasures anyway, perhaps indicating an inherent preference for multiple countermeasures in ADTs. Subsequently, we accept the null-hypothesis \nullhypothesis{\hypoErrorAmount} and reject the alternative hypothesis \althypothesis{\hypoErrorAmount}. Similarly, find the same results for the three semantic errors (\hypothesis{\hypoMultipleParent}, \hypothesis{\hypoMultipleRefinement}, \hypothesis{\hypoMulipleCountermeasure}) we discussed in Section~\ref{ssec:results-common-errors} .

%While we do not see a significant difference in the amount of errors made by both groups of students, we do see a usage of ADTs that was wholly unexpected. While nodes in ADTs can only have one countermeasure, our research suggests that in practice, users of ADTs would prefer to be able to include multiple countermeasures. However, more research will be required to confirm this finding and further to expand the definition of ADTs to allow for multiple countermeasures. 

Finally, the error of the single-child nodes needs to be discussed separately. As mentioned in Section~\ref{ssec:results-common-errors}, this is a somewhat different type of error than the semantic errors. We do see a statistically significant difference between \ICS\ and \SEC\ students in the number of single child nodes used. Specifically, we see $p$-values of 0.00015 for ADT1, 0.142839 for ADT2, 0.021082 for ADT3 and 0.000450 when these values are combined across all ADTs. The $p$-values for the individual ADT tests are correct for family-wise errors using the Bonferroni correction. Except for ADT2, we see a statistically significant difference, with \ICS\ students using single children on average 2-3 times more than \SEC\ students. On ADT2, there is little room for creativity or interpretation, as the goal is to purely translate the simple written scenario into an ADT. Given the simplicity of the written scenario, there is unlikely to be a significant difference with number of errors, specifically the single child node error, purely due to the lack of opportunity to make such errors. As such, we interpret our findings to say that \ICS\ students were more likely to use single-child nodes, perhaps as means of expressing the different levels of abstraction in their attack scenarios. Based on our data, we can reject the null-hypothesis \nullhypothesis{\hypoSingleChildNodes} and accept the alternative hypothesis \althypothesis{\hypoSingleChildNodes}.



\subsection{ADT usage}
\label{ssec:analysis-adt-usage}

We only asked a single Likert question concerning ADTs as a means of communication. We found there was no statistically significant difference between \ICS\ and \SEC\ with assessing ADTs as a means of communicating. This is shown clearly in Figure~\ref{fig:means-of-commanalysis}. While we did only ask this question a single time, we asked it on the same ADT that yielded the largest statistical difference for the subsequent question about ADTs as a means of analysis. 
%Given that on the same ADT, the \ICS\ and \SEC\ students responded with relative unanimity for ADTs as a means of communication but with significant difference for ADTs as a means of analysis, this single question offers more because of the broader context. 
Additionally, we asked a similar question on the same ADT (question LS-ADT3-W3), which was coded and analyzed in Section~\ref{ssec:results-intention-to-use} and Table~\ref{tab:coded-future-use}. There was no statistically significant difference in the coded ``communication'' responses, which would indicate that both \ICS\ and \SEC\ students responded concerning the use of ADTs as a means of communication at similar rates.
With this, we cannot accept either the null-hypothesis \nullhypothesis{\hypoCommunicationTool} or the alternative hypothesis \althypothesis{\hypoCommunicationTool}, as technical background does not appear to affect the perception of the usability of ADTs for communicating attack scenarios.

We do not see the same when assessing ADTs as a means of analysis. As is commonly cited, ADTs can be used as a tool to help fill in potential attack scenarios and guide analysis of threat intelligence~\cite{andersonSecurityEngineeringGuide2020,schneierSecretsLiesDigital2000}. However, when asking our participants how they perceive the usability of ADTs for analysis, we find there is a statistically significant difference between \ICS\ and \SEC\ students on two of the three questions we asked. Specifically of interest to us is this question when referring to the self-generated ADT, as this would be the ADT in which students would likely use the structure of ADTs to guide the development of their threat scenario. In contrast to these findings, we do see in the responses to question LS-ADT3-W3 that coded student responses do not show a significant difference concerning the use of ADTs as a means of analysis. 
%However, it is noteworthy that the difference is much greater than that of using ADTs as a means of communication.
Overall, we find that \ICS\ students are more likely to agree that ADTs are useful as an analysis tool than \SEC\ students. As such, we can reject the null-hypothesis \nullhypothesis{\hypoAnalysisTool} and accept the alternative hypothesis \althypothesis{\hypoAnalysisTool}.

Finally, we asked students if they saw a place for ADTs in the industry in question LS-ADT3-W3, and if they hoped to see ADTs again in LS-ADT3-W5. We posit that the participants would have a strong intention to use ADTs in the future if they responded in the affirmative to both questions. We could further argue that students intend to use ADTs in the future if they answer positively only to the second question concerning seeing ADTs again. Based on the responses, it appears that this question was interpreted as inquiring if students would use ADTs themselves in the future. While students answered positively to both questions, we do not see a statistically significant difference between \ICS\ and \SEC\ students; so we cannot reject  the null-hypothesis \nullhypothesis{\hypoIntentionToUse}. We do note that there is a high apparent intention to use, which should indicate high potential of actual usage according to the MEM.


\subsection{ADT creation}
\label{ssec:analysis-adt-creation}

While the equivalence of two ADTs can be assessed based on a chosen semantics~\cite{mauwFoundationsAttackTrees2006}, to the best of our knowledge, ADT comparison and metrics of distance between two ADTs have not yet been investigated in the literature. Thus, we opted to compare several objective metrics of each ADT, all of which can be seen in Figures~\ref{fig:adt2-comparison}~and~\ref{fig:adt3-comparison}. 

For the second task of the large study, we compared seven unique metrics. The results of these comparisons can be found in Section~\ref{ssec:results-adt2-comparison}. We find that over 60\% of students were able to successfully convert a written description to ADT without making an error. Of the less than 40\% of both groups who made errors in conversion, we found that the only statistically significant difference was the number of \OR\ refinements present.% (the \ICS\ students added more \OR\ refinements on occasion and 20\% of the \SEC\ students missed an \OR\ refinement).

% We believe this statistically significant difference is created by the confluence of a few factors. Firstly, \SEC\ students similarly missed one or two nodes that should have been present in the final ADT. We are unsure of the cause of these omissions, but we do know that an omission of one or two attack nodes could easily result is a missing \OR\ refinement. Conversely, \ICS\ students tended to add more information than was in the tree, with most students who made errors adding at least one extra attack node.


Similar to the second task in the large study, for the third task we also used metrics as a means of comparison. As these ADTs are all created by the students' themselves with no external influence, we added an additional metric. All eight metrics can be seen in Figure~\ref{fig:adt3-comparison}. The results of these ADTs can be found in Section~\ref{ssec:results-adt3-comparison}. However, across all metrics, we do not see any statistically significant difference. In fact it is quite the opposite, students created a wide range of ADTs but in aggregate, these ADTs tend to fall within the same ranges for both \ICS\ and \SEC. As such, we accept the null-hypothesis~\nullhypothesis{\hypoThirdADT} and reject the alternative hypothesis \althypothesis{\hypoThirdADT}, and we can conclude that the technical background does not affect the creativity in creating ADTs.

%Of note with our analysis of ADT3 are the ranges in the metrics. While this was not one of our research questions, as we were primarily concerned with technical background, we do find that both groups of students produced ADTs within similar ranges. This suggests that there may be ideal metrics for ADT creation, but further research will be needed to explore this.



% \begin{table*}[t!]
%     \caption{Conclusions on our hypotheses}
%     \label{tab:hypotheses-response}
% \resizebox{\textwidth}{!}{
%     \begin{tabular}{@{}lll@{}}
    
%         \toprule
%         \textbf{ID}    & \textbf{RQ}      &  \textbf{Conclusions} \\ \midrule

%         \hypothesis{\hypoCheckUnderstand} & \RQ{1}           
%         & \hResponse{REJECT} We do not see a statistically significant difference on four of five measured metrics                        \\
%         \hypothesis{\hypoSecondADT} & \RQ{1}           & \hResponse{REJECT} Students successfully created an ADT from a written description at similar rates \\


%         \hypothesis{\hypoErrorAmount} & \RQ{1} \& \RQ{4} & \hResponse{REJECT} Students made semantic errors at similar rates           \\


% $\text{   }$\hypothesis{\hypoMultipleParent} & \RQ{1} \& \RQ{4} & \hResponse{REJECT} \ICS\ and \SEC\ students used nodes with multiple parents at similar rates  \\

% $\text{   }$\hypothesis{\hypoMultipleRefinement} & \RQ{1} \& \RQ{4} & \hResponse{REJECT} \ICS\ and \SEC\ students used nodes with multiple refinements at similar rates  \\

% $\text{   }$\hypothesis{\hypoMulipleCountermeasure} & \RQ{1} \& \RQ{4} & \hResponse{REJECT} \ICS\ and \SEC\ students used nodes with multiple countermeasures at similar rates  \\


% $\text{   }$\hypothesis{\hypoSingleChildNodes} & \RQ{1} \& \RQ{4} & \hResponse{ACCEPT} \ICS\ students were much more likely to have single child nodes than \SEC\ students                                                     \\



%         \hypothesis{\hypoSelfUnderstand} & \RQ{2} & \hResponse{REJECT} Students self report understanding at the similarly       \\

%         \hypothesis{\hypoCommunicationTool} & \RQ{2}           & \hResponse{REJECT} Students' perception of ADTs as a communication tool were similar    \\




%         \hypothesis{\hypoAnalysisTool} & \RQ{2}           & \hResponse{ACCEPT} \ICS\ students has a significantly stronger preference for ADTs as an analysis tool compared to \SEC\ students                         \\


% \hypothesis{\hypoWrittenComparison} & \RQ{2} \& \RQ{3}            & \hResponse{REJECT} Students preferred ADT over written descriptions at similar rates           \\

%         \hypothesis{\hypoIntentionToUse} & \RQ{3} & \hResponse{REJECT} Students believe that ADTs have a place in the cybersecurity industry and hope to see them again at similar rates\\

%         % \hypothesis{7} & \RQ{2}           & The comparison of ADTs as a means of analysis will be different between \ICS\ and \SEC students                             \\
%         \hypothesis{\hypoThirdADT} & \RQ{4}           & \hResponse{REJECT} Across the 8 measurements taken of the self-drawn ADT, there were no statistically significant differences                                   \\
%         \bottomrule
%     \end{tabular}
%     }
% \end{table*}


% RQ1 Is the actual efficacy or actual efficiency of ADTs affected by
% technical background?
% RQ2 Is the perceived ease of use or perceived usefulness of ADTs
% affected by technical background?
% RQ3 Is the intention to use ADTs affected by technical back-
% ground?
% RQ4 Does technical background impact how ADTs are drawn?


% Overall, we find little evidence to support the underlying assumption that technical background is the cause of the lack of actual usage of ADTs in industry \NS{include citation I haven't found yet}. 
% \begin{comment}
% \subsection{\RQ{1}: Actual efficacy}
% \label{ssec:analysis-rq1}

% We find little evidence to support the idea of technical background affecting understanding. Our assumption in this research question was that \SEC\ students were more likely to come across tree, and tree like structures, in their coursework, and that this might allow for a greater ease of understanding. This research question can also address the actual efficiency in the MEM. Actual efficiency refers to how much effort it takes a participant to use a given method~\cite{moodyMethodEvaluationModel2003}. We posit that strong understanding is fundamental for efficient use of ADTs. 

% Four of our hypotheses (\hypothesis{\hypoCheckUnderstand},\hypothesis{\hypoSelfUnderstand},\hypothesis{\hypoErrorAmount},\hypothesis{\hypoSingleChildNodes}) were related to our research question concerning the effect of technical background on understand. Of these, three (\hypothesis{\hypoCheckUnderstand},\hypothesis{\hypoSelfUnderstand},\hypothesis{\hypoErrorAmount}) were rejected as there was not a significant difference between \ICS\ and \SEC\ students. There was one hypothesis (\hypothesis{\hypoSingleChildNodes}) concerning the use of single child nodes which was accepted, as there was a statistically significant difference between \ICS\ and \SEC\ students. Typically, when significant numbers of single child nodes are used, they are being used to represent steps in an attack as opposed to components in an attack~\cite{dawkinsSystematicApproachMultiStage2004}. This is akin to attaching a kill-chain to an attack tree, which not the intended use of attack trees. As discussed in Section~\ref{ssec:analysis-errors}, this is the exact behavior we observe on trees with high numbers of single child nodes. We do see this error being made frequently by both \ICS\ and \SEC\ students, though it is made significantly more frequently by \ICS\ students. This could be an indicator of a lack of actual efficiency of ADTs, as students struggled to use the tool as intended, and made errors which largely make ADTs harder to use. However, it is also possible that this was a flaw in our training or a flaw within the model itself, for it could be argued that nodes should have at least two children~\cite{mauwFoundationsAttackTrees2006}.

% With all this considered, we conclude that actual efficacy and efficiency is not affected by technical background. If it were affected, we would have seen more difficulty using ADTs for one group, or that one group was unable to successfully use ADTs. We do not see any such results. 

% \subsection{\RQ{2}: Perceived ease of use and usefulness}
% \label{ssec:analysis-rq2}

% Similarly to the previous section, we find little evidence that technical background affects either perceived ease of use; however, we do find some evidence that perceived usefulness is affected by technical background.  Another four of our hypotheses concern perception (\hypothesis{\hypoSelfUnderstand}, \hypothesis{\hypoCommunicationTool}, \hypothesis{\hypoAnalysisTool}, \hypothesis{\hypoWrittenComparison}). Three of these were rejected as there was no statistically significant difference between \ICS\ and \SEC\ students.

% One hypothesis, \hypothesis{\hypoAnalysisTool}, was accepted as we did see a statistically significant difference between \ICS\ and \SEC\ students with regards to the perception of ADTs as a tool for analyzing a threat scenario. \ICS\ students were significantly more likely than \SEC\ students to see ADT as a useful tool for analysis. This is in stark contrast to the similar hypothesis \hypothesis{\hypoCommunicationTool} concerning ADTs as a communication tool. We show there is evidence to support that the perceived usefulness of ADTs in the analysis domain is affected by technical background. However, given that ADTs are perceived to be useful in the communication domain by both \ICS\ and \SEC\ groups, this does not offer a potential answer to why ADTs are not more widely used. Additionally, this conclusion brings new questions about why \SEC\ students did not find ADTs useful as a tool for analysis. 

% While we did not find a statistically significant difference between \ICS\ and \SEC\ students with respect to ADTs as a communication tool, in LS-ADT3-W3, several students from both \ICS\ and \SEC\ noted that a potential use they see for ADTs is specifically to help people with a technical and non-technical background find common understanding of a threat scenario~\NS{cite gap paper?}. \SEC\ student \#19 wrote, ``I feel [ADTs] can be very helpful in communicating and even `translating' between technical cyber security staff and less technical management staff''. Our research concerning actual efficacy supports this potential.

% \subsection{\RQ{3}: Intention to use}
% \label{ssec:analysis-rq3}

% In the MEM, the biggest factor in the actual use of a method is the intention to use that method. We suppose that if the intention to use ADTs was affected by technical background, that this could be an indicator of why ADTs are not used more broadly within in the cybersecurity industry, as if only technical staff have an intention to use ADTs, then it is understandably that ADTs would not be used in many contexts.

% However, the hypotheses we tests concerning intention to use ADTs (\hypothesis{\hypoWrittenComparison}, \hypothesis{\hypoIntentionToUse}) both were rejected as there is not a statistically significant difference between \ICS\ and \SEC\ students. We do find that both groups of students strongly agree that ADTs have a place within industry and would hope to see them again, indicating a strong intention to use the model. We can thus argue that the lack of widespread use of ADTs within industry is not caused by a difference in use, perception, or intention between technical and non-technical people. The lack of actual usage is likely affected by other factors.


% \subsection{\RQ{4}: Drawn ADTs}
% \label{ssec:analysis-rq4}

% As stated in Section~\ref{ssec:results-adt3-comparison}, we do not see any statistically significant difference in how ADTs are drawn between \ICS\ and \SEC\ students. In contrast and of note, we find almost the exact opposite. In Section~\ref{ssec:analysis-adt-creation}, we find that students of both groups produce ADTs that all fall within the same ranges. While this leads us to conclude that technical background does not affect how ADTs are draw, it does lead us to a new question of whether there are limits beyond which ADTs cease to be effective, as has been found for other graphical visualization~\cite{abdelaalComparativeEvaluationBipartite2023}.

% \end{comment}