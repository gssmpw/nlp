\section{Discussion}\label{sec:discussion}

Our results show that both participants with a highly technical background and a very limited technical background find ADTs acceptable. Moreover, they find it acceptable in an equal way: for most of the concepts we measured, both treatment groups have shown equivalent behavior and perceptions. They also use ADTs creatively in a similar way, designing models of very similar size and quality. These findings confirm the belief in the security community that attack trees are accessible and easy to learn~\cite{reversinglabs2024attacktrees}. 

Our research sought to establish if the technical background is a potential factor in the adoption of ADTs and, specifically, if the participants with a very limited technical background would be disadvantaged in using ADTs. The cyber security industry consists of people with widely varying backgrounds~\cite{andersonSecurityEngineeringGuide2020}. In particular, TM is done by people with diverse skillsets and objectives~\cite{shostack2008experiences,verreydt2024threat}. If a technical background were to impact the acceptability of ADTs, then this could be a reason for not recommending them to be used.


Lallie~\etal\ found that there was a difference between participants with and without a computer science background when using both fault trees and attack graphs in a similar study design to ours~\cite{lallieEmpiricalEvaluationEffectiveness2017}. This result indicated that TM stakeholders who do not possess a highly technical background (e.g., managers) might potentially be disadvantaged if the team uses attack trees for threat modeling. However, it is reasonable to expect that people involved in TM, even managers, might possess at least a limited technical background as they are exposed to software development and/or IT security risk management activities. Our study concludes that ADTs are highly acceptable for such TM stakeholders and do not disadvantage them compared to threat modelers with a highly technical background. 


We believe the difference in the results between our study and \cite{lallieEmpiricalEvaluationEffectiveness2017} to be due to two major differences in the study design and methodology. First, we intentionally used ADT examples that are equally accessible to all participants, attempting to remove any specifically technical jargon from our study questions. For the small study, all of our examples were pulled from papers on ADTs and we specifically looked for ADTs without complex technical labels, i.e., accessible to people with diverse backgrounds. This approach was inspired by Lallie et al.~\cite{lallieEmpiricalEvaluationEffectiveness2017} who used fault trees from previous works. However, two of the fault trees they used are arguably difficult to understand to a layperson, using terms such as ``\texttt{sshd\_bof(1,2)}'', which might be more accessible to someone with a computer science background. As such, their finding that those with a computer science background can use these models more effectively may speak more to the comprehensibility of the language used in their study. While subsequent studies in the attack trees context are required to test this,  previous research has shown that technical language does affect comprehension: e.g., Bravo-Lillo et al.~\cite{bravo2010bridging} have shown that technical terms complicate comprehension of security warnings for non-expert users, compared to security experts. In the TM context, Ingalsbe et al.~\cite{ingalsbe2008threat} explicitly mention that the vocabulary of threat modeling is IT-biased, impeding communication with internal business customers, while Verreydt et al.~\cite{verreydt2024threat} also acknowledge the TM challenges related to communication and the used language. 


One important conclusion that we can draw from our study is that short training is sufficient for making ADTs equally acceptable for users with high and limited technical backgrounds. Threat modeling method training is an established practice in organizations~\cite{verreydt2024threat}, and it can be recommended to improve the outcomes and facilitate the process~\cite{cruzes2018challenges,stevens2018battle}. To help implement training on ADTs in organizations, we share the slides of our training lecture along with a detailed description in our supplementary data material~\cite{zenodo-dataset}.   


Another relevant observation that we can make from the analysis of the related literature (Sec.~\ref{sec:empirical_studies}) is that there are no established protocols for empirical studies of TM methods. While the studies frequently follow reputable frameworks like TAM and MEM, the operationalization of the frameworks' constructs differs a lot. One of the reasons behind this might be the diversity of TM methods themselves. Still, it would be useful to systematize the experiences reported so far and develop guidelines for executing such studies.  

