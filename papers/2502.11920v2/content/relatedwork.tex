
\section{Threat Modeling and Attack Trees}\label{sec:background}
The notion of \emph{threat modeling} (TM) refers to a process to identify relevant attacks or threats; it typically takes place in the context of software development or security risk management~\cite{xiong2019threat,tuma2018threat}. In the context of software development, TM can refer to a requirements elicitation or design analysis technique~\cite{shostack2008experiences}. Given the diversity of secure software development guidelines~\cite{kudriavtseva2022secure} and security risk management methods~\cite{gritzalis2018exiting} -- and the wide variety of organizational contexts and systems where threat modeling is applied -- there are also many established TM approaches~\cite{shevchenko2018threat,xiong2019threat,mitre2018tm,granata2024systematic,tatam2021review,tuma2018threat}. These methods differ substantially in their focus and process to follow: e.g., STRIDE helps with discovering pertinent security issues during software development, LINDDUN is designed for privacy threats, TARA and Persona non Grata focus on identifying relevant attacker profiles, while PASTA and OCTAVE cover the whole security risk assessment process~\cite{shevchenko2018threat,mitre2018tm}. 


\textbf{Attack trees.}
\emph{Attack trees} (sometimes called \emph{threat trees}) were proposed by Bruce Schneier in 1999, inspired by the fault trees model~\cite{schneierAttackTrees1999}. According to Shevchenko et al.~\cite{shevchenko2018threat}, attack trees are one of the oldest and most widely used threat modeling methods that help capture and dissect possible cyber, cyber-physical, or physical attack scenarios. Attack trees are labeled acyclic graphs (trees) in which every node label is either an attacker's goal or an attack component in service of that goal. Each node can have any number of child nodes with a defined relationship, otherwise referred to as a \emph{refinement}, between those nodes. The \OR\ relationship indicates that one child must be completed for the parent to evaluate as complete. The \AND\ relationship indicates that all children must be completed for the parent to evaluate as complete. We note that each parent node can be refined in only one way (either \AND\ or \OR) or have no children at all: such nodes are called \emph{leaf} nodes, and they represent simple attacker's actions that don't need to be further specified. This simple \AND-\OR\ tree model is very versatile and allows representing complex scenarios succinctly~\cite{schneierAttackTrees1999,mauwFoundationsAttackTrees2006,widel2019beyond}. 

Mauw and Oostdijk defined the attack tree theory by proposing several semantics that can be used to represent attack trees formally~\cite{mauwFoundationsAttackTrees2006}. Kordy~\etal\ further expanded on this by introducing attack-defense trees (ADTs)~\cite{kordyFoundationsAttackDefense2011}. ADTs allow each attack node to have a single countermeasure edge to a defense node, representing a defense to the attack goal or component it is attached to. These defense nodes are roots of their own defense subtree, with the same construction rules as attack trees, including being able to have countermeasure edges to attack nodes, representing an attack against the defense. Thus, attack trees are a particular case of ADTs that do not have any defense nodes. The ADT model allows representing complex attack-defense scenarios where defenders can deploy countermeasures against attacks, and attackers can try to circumvent these countermeasures~\cite{kordyFoundationsAttackDefense2011,kordyDAGBasedAttackDefense2013}. Moreover, considering countermeasures explicitly and  collecting a library of best practices for mitigation are recommended in the TM literature~\cite{dhillon2011developer,jawad2024m,trentinaglia2023eliciting}, and ADTs can help with these objectives.
 Fig.~\ref{img:ss-adt2} shows an example ADT from~\cite{sunCyberAttackRisksAnalysis2018}. 



\begin{figure}
\includegraphics[width=0.96\linewidth]{img/SS_ADT2.pdf}
\caption{An example of an ADT (the second ADT in our small study) from~\cite{sunCyberAttackRisksAnalysis2018}.}
    \label{img:ss-adt2}
\end{figure}

Attack trees and ADTs have been further expanded in other ways~\cite{kordyADToolSecurityAnalysis2013,widel2019beyond,hongSurveyUsabilityPractical2017}. Our work focuses on ADTs 
\emph{{\`a} la} Kordy~\etal~\cite{kordyFoundationsAttackDefense2011}, i.e., without any additional attributes.

\textbf{Attack trees usage in practice.}
Attack trees are quite popular, as evidenced by the fact that they are described in many textbooks (for example, Bishop~\cite{bishop2019computer}, Stallings and Brown~\cite{stallings2018computer}, van Oorschot~\cite{van2020computer}, and Anderson~\cite{andersonSecurityEngineeringGuide2020}), authoritative references on threat modeling (Shostack~\cite{shostack2014threat}, Shevchenko et al.~\cite{shevchenko2018threat}, Bodeau et al.~\cite{mitre2018tm}, and Tarandach and Coles~\cite{tarandach2020threat}), adversarial modeling (CyBOK~\cite{stringhini2021adversarial}), and advice from relevant government institutions and industry bodies (e.g., the UK NCSC~\cite{ncsc2023attacktrees,ncsc2020telecom}), OWASP~\cite{owasp2024tm}, or US NIST~\cite{nist202180030}). 

It is frequently recommended to combine STRIDE-based threat modeling with attack trees for more in-depth analysis of critical data flows and threats~\cite{shostack2014threat,tarandach2020threat,reversinglabs2024attacktrees}. This aligns well with the evidence-based recommendation to complement data flow diagram-based analysis of STRIDE with expressive attacker models by Van Landuyt and Joosen~\cite{van2022descriptivestride} and observations from practitioners that having a library of relevant threat scenarios improves the TM outcomes~\cite{dhillon2011developer}. Schneier advises organizations to develop collections of attack trees to share knowledge and alleviate the need for in-depth security expertise~\cite{schneierAttackTrees1999}. LINDDUN implements this advice, featuring a dedicated privacy threat trees catalogue~\cite{deng2011privacy}, which was appreciated as useful by participants in an empirical study evaluating LINDDUN~\cite{wuyts2014empirical}. Jamil et al.~\cite{jamil2021threat} report that attack trees are chosen as a method because they can help covering all possible attack entry points. 

Despite the popularity, to the best of our knowledge, there are few established references that prescribe how to apply attack trees. Sonderen~\cite{sonderenManualAttackTrees2019} designed a manual for producing attack trees. The manual aims to support a single person designing an attack tree for a given scenario (i.e., the context is not a TM exercise done as a team); it was refined and evaluated in both a qualitative study and a case study. Sonderen reports that careful handling of the levels of abstraction is the most important for a structurally solid attack tree. Schneier prescribes to develop an attack tree top-down, revise it over time, and share with one or more colleagues to improve the completeness of the model~\cite{schneierAttackTrees1999}. He also advises having a library of attack trees that could capture relevant attack scenarios and can be reused -- and thus diminish the need to have security experts around. 


\textbf{Threat modeling best practices.}
The TM literature offers substantial insights into the practice of threat modeling. However, it is clear that there is still a gap in understanding how different human factors affect the TM process~\cite{tran2023threat}.
Stevens et al.~\cite{stevens2018battle} reported on their experience with introducing the Center of Gravity TM approach to New York City Cyber Command, highlighting the benefits of threat communication that were reported by the participants. Thompson et al.~\cite{thompson2024there} interviewed and observed 12 medical device security experts to understand their TM practices. They find that the approaches to TM used by different experts vary, and it is important to support a free-flowing, natural approach to ideation (brainstorming).  

Verreydt et al.~\cite{verreydt2024threat} conducted an empirical study of TM methods applied in Dutch organizations within the secure software development process. They found that while the roles involved in the software product development (developers, architects, product owners, and the security team) were central to conducting the TM process itself (this is concurred by other works, e.g.~\cite{shostack2008experiences, bernsmed2022adopting,cruzes2018challenges,dhillon2011developer,trentinaglia2023eliciting}), the outcomes are often communicated to information security officers and managers. Moreover, one of the reasons that management is not involved directly during the TM activities is the belief that such sessions require a strong technical and/or security background~\cite{verreydt2024threat}. Involving a business representative familiar with the key business objectives is also recommended by Ingalsbe et al.~\cite{ingalsbe2008threat}.  Considering security risk management practices, Brunner et al.~\cite{brunner2020risk} also report on the heterogeneity of roles being involved: CxOs, quality and compliance managers, software developers, security-related staff, and others. 

%\NS{This paragraph is hard to follow for me. I'm wondering if we should remove it? Maybe move to appendix?}
To summarize, TM is a team-based activity that involves different stakeholders: developers, security experts, product owners, and managers. Given the multitude of roles involved, communication becomes very prominent. While TM can be an opportunity to raise awareness about security in managers and bring their attention to the importance of security~\cite{cruzes2018challenges,verreydt2024threat}, difficulties in communication and conveying security messages across the teams are known to be a security ``blocker''~\cite{weir2023incorporating,verreydt2024threat}. Thus, it is important to establish whether such a prominent TM method like attack trees is amenable for all stakeholders, especially for people without a substantial technical background. A positive answer would help organizations to recommend that management and other stakeholders with a limited technical background participate in TM more actively, as well as use attack trees for communicating the TM results outside of the product development team.



% \revised{
% Despite this popularity, to the best of our knowledge, there are few established references that prescribe how to apply attack trees. The potential stakeholders for attack trees include security analysists, software develpers, security operations personnel, applying threat modeling broadly (and attack defense trees specifically) in their domains. Stakeholders could also include managers, auditors, and other non-technical personnel who need to understand security risks to best perform their functions. As such, it is vital that threat models are understandable and usable by a wide range of stakeholders, with various backgrounds, as these stakeholders are present within the cybersecurity industry. The acceptability of threat models to the general public is not as critical, and this is outside the scope of our research}.
% \NS{I attempted to write the stakeholder bit}


%%%%%%%%%%%%%%%%%%%%%%%%%%%%%%%%%%%%%%%%%%%%%%%%%%%%%%%%%%%%%%%%%%%%%%

\section{Related Work}\label{sec:empirical_studies}


\textbf{Acceptability of attack trees.}
A common strategy for examining TM notations such as ADTs is a study designed to compare two or more notations against each other. Such studies split participants into several groups, and have them complete tasks designed to measure TM method efficacy, with the same tasks being performed using different methods. Opdahl and Sindre used this design to explore the effectiveness of attack trees compared to misuse cases, finding that attack trees are more effective, but the participants has similar perceptions of the two techniques~\cite{opdahlExperimentalComparisonAttack2009}. This study has been replicated with industry practitioners by Karpati~\etal\ who found similar effects, also showing that in the context of cybersecurity, students make a sufficient proxy for practitioners~\cite{karpatiComparingAttackTrees2014}. 

In Diallo~\etal~\cite{dialloComparativeEvaluationThree2006}, two computer science master students applied Common Criteria, misuse cases, and attack trees to the same scenario, evaluating the methods' learnability, usability, analyzability, and clarity of output and finding advantages and disadvantages for each approach. They concluded that attack trees were easy to learn and use, provided a clear output, but were more difficult to analyze~\cite{dialloComparativeEvaluationThree2006}.  %Sonderen conducted a small study on attack trees, exploring if training enabled the better usage of attack trees, finding that having a simplified manual offers sufficient training for more effective attack tree use~\cite{sonderenManualAttackTrees2019}.


Broccia~\etal\ applied the Method Evaluation Model (MEM) and used 25 human subjects (all with technical background) to examine attack defense tree acceptability~\cite{broccia_assessing_2024}\footnote{This research was performed concurrently with ours, and we had no knowledge of these works when designing and performing our study.}. This study was also recently replicated in another experiment with 49 subjects (computer engineering students)~\cite{broccia2025evaluating}. For their participants with a technical background, Broccia~\etal\ found a good level of understandability and acceptability of ADTs~\cite{broccia_assessing_2024,broccia2025evaluating}. Yet, unlike our study, these studies did not examine participants with a very limited technical background. 


To our knowledge, there has only been one previous study on the effect of technical background on attack tree effectiveness. Lallie~\etal\ compared attack graphs to fault trees (considered as a variant of attack trees), finding attack graphs more effective~\cite{lallieEmpiricalEvaluationEffectiveness2017}. Additionally, in the same study, they compared participants with a computer science background to those without one. Their findings did show that computer science participants were able to significantly outperform those without a computer science background using both models. 

\textbf{Studies of other security methods.}
Moving beyond attack trees, Katta~\etal\ conducted an experiment with student participants to compare understanding, performance, and perception of misuse sequence diagrams and misuse case maps, finding that the models perform similarly~\cite{kattaComparingTwoTechniques2010}. 
Labunets~\etal\ compared visual and textual risk assessment methods with student participants using a similar design, focusing on evaluating perception and effectiveness~\cite{labunetsExperimentalComparisonTwo2013,labunets_no_2018}. They found that each type of method was effective in different tasks. De La Vara~\etal\ conducted a study with students concerning Systems Process Engineering Metamodel-like diagrams, comparing this model to text descriptions; they found that the model was statistically significantly more effective in helping students understand the scenario \cite{de_la_vara_empirical_2020}. Tondel et al.~\cite{tondel2019understanding} examined the acceptability of Protection Poker in a study with computer science students and reported that the participants found it to be acceptable but perceived a limited impact on the security of the project. Wuyts et al.~\cite{wuyts2014empirical} empirically evaluated LINDDUN in a series of studies with students and a case study with experts, finding that the method helps to identify relevant privacy threats (correct), but many threats are also not discovered (incomplete). The participants perceived LINDDUN to be easy to use, but the method's efficiency was lower than expected~\cite{wuyts2014empirical}.  

A group of empirical studies focused on evaluating STRIDE-based threat modeling, as STRIDE is the most commonly used method~\cite{verreydt2024threat,thompson2024there,jamil2021threat}. For example, Bernsmed et al.~\cite{bernsmed2022adopting} conducted a study with students to evaluate user acceptance and usage of two versions of a STRIDE-based threat modeling process. Scandariato et al.~\cite{scandariato2015descriptive} evaluated STRIDE in a study with computer science students, concluding that STRIDE is relatively time-consuming (not very efficient), but it is perceived as easy to learn. The threats identified by the participants were largely correct, but many threats were not discovered (low completeness)~\cite{scandariato2015descriptive}. Tuma and Scandariato~\cite{tuma2018two} followed a similar design and compared time cost and effectiveness in threat elicitation of STRIDE per element and STRIDE per interaction in a controlled experiment with computer science master students, reporting that STRIDE per element provided better results. However, all these studies did not examine the effects of participants' background.   


\textbf{Examining the difference in backgrounds.}
The existing literature demonstrates that technical background can affect comprehension. Hogganvik and Stolen evaluated the background-affected comprehensibility of risk analysis terminology on professionals and students. They found a statistically significant difference in correct responses, concluding that background does affect comprehension~\cite{hogganvik_risk_2005}. Wu~\etal\ conducted a study examining the ability of participants to understand security texts, finding that a significant percentage of security jargon is not comprehensible by those with a limited IT background~\cite{wu_what_2020}. Chen et al.~\cite{chen2023investigating} found that participants with IT background could understand explanations of Alexa skills privacy policies and related terms better than participants without an IT background. 


To summarize, it appears that technical background seems to be an important prerequisite for comprehending many security-related concepts~\cite{lallieEmpiricalEvaluationEffectiveness2017,hogganvik_risk_2005,wu_what_2020}, and, in particular, users without a computer science background might be disadvantaged when using ADTs~\cite{lallieEmpiricalEvaluationEffectiveness2017}. As threat modeling involves participants like managers with possibly a very limited technical background, we set out to examine in our study whether they would be disadvantaged when using ADTs compared to participants with more advanced technical backgrounds.

