\documentclass[11pt,a4paper]{amsart}
\pdfoutput=1
\usepackage[margin=3.25cm]{geometry}

         \documentclass[conference]{IEEEtran}
\IEEEoverridecommandlockouts
% The preceding line is only needed to identify funding in the first footnote. If that is unneeded, please comment it out.
\usepackage[table]{xcolor}
\usepackage{cite}
\usepackage{amsmath,amssymb,amsfonts}
\usepackage{algorithmic}
\usepackage{graphicx}
\usepackage{lipsum}
\usepackage{capt-of}
\usepackage{cuted} 
\usepackage{comment}
\usepackage{svg}
\usepackage{textcomp}
\usepackage{booktabs}
% \usepackage{xcolor}
\usepackage{ifthen}
\usepackage{multirow}
\usepackage{booktabs}
\newboolean{isAnonymous}
\setboolean{isAnonymous}{false}  % Change to false to include names + ACKS

\def\BibTeX{{\rm B\kern-.05em{\sc i\kern-.025em b}\kern-.08em
    T\kern-.1667em\lower.7ex\hbox{E}\kern-.125emX}}
\newcommand\f[1]{\textit{F1}(#1)}

\makeatletter
\newcommand{\newlineauthors}{%
  \end{@IEEEauthorhalign}\hfill\mbox{}\par
  \mbox{}\hfill\begin{@IEEEauthorhalign}
}
\makeatother



\begin{document}
% Should robust decentralized learning systems use non-IID data distributions?
%Inherent Resilience: The Built-In Robustness of Decentralized Averaging to Bad Data
\title{The Built-In Robustness of Decentralized Federated Averaging to Bad Data
\ifthenelse{\boolean{isAnonymous}}{}{\thanks{This work is partially supported by the European Union under the scheme HORIZON-INFRA-2021-DEV-02-01 – Preparatory phase of new ESFRI research infrastructure projects, Grant Agreement n.101079043, “SoBigData RI PPP: SoBigData RI Preparatory Phase Project”. 
This work was partially supported by SoBigData.it. SoBigData.it receives funding from European Union – NextGenerationEU – National Recovery and Resilience Plan (Piano Nazionale di Ripresa e Resilienza, PNRR) – Project: “SoBigData.it – Strengthening the Italian RI for Social Mining and Big Data Analytics” – Prot. IR0000013 – Avviso n. 3264 del 28/12/2021. 
%
S. Sabella's, C. Boldrini's and M. Conti's work was partly funded by the PNRR - M4C2 - Investimento 1.3, Partenariato Esteso PE00000013 - ``FAIR'', A. Passarella's and L. Valerio's work was partially supported by the European Union - Next Generation EU under the Italian National Recovery and Resilience Plan (NRRP), Mission 4, Component 2, Investment 1.3, CUP B53C22003970001, partnership on ``Telecommunications of the Future'' (PE00000001 - program “RESTART”).}
}
}

\ifthenelse{\boolean{isAnonymous}}{\author{\IEEEauthorblockN{Anonymous Authors}}}{
\author{\IEEEauthorblockN{Samuele Sabella}
\IEEEauthorblockA{\textit{Istituto di Informatica e Telematica} \\
\textit{Consiglio Nazionale delle Ricerche}\\
Pisa, Italy \\
samuele.sabella@iit.cnr.it}
\and
\IEEEauthorblockN{Chiara Boldrini\IEEEauthorrefmark{2}}
\IEEEauthorblockA{\textit{Istituto di Informatica e Telematica} \\
\textit{Consiglio Nazionale delle Ricerche}\\
Pisa, Italy \\
chiara.boldrini@iit.cnr.it}
\and
\IEEEauthorblockN{Lorenzo Valerio\IEEEauthorrefmark{2}}
\IEEEauthorblockA{\textit{Istituto di Informatica e Telematica} \\
\textit{Consiglio Nazionale delle Ricerche}\\
Pisa, Italy \\
lorenzo.valerio@iit.cnr.it}
\newlineauthors
\IEEEauthorblockN{Andrea Passarella}
\IEEEauthorblockA{\textit{Istituto di Informatica e Telematica} \\
\textit{Consiglio Nazionale delle Ricerche}\\
Pisa, Italy \\
andrea.passarella@iit.cnr.it}
\and
\IEEEauthorblockN{Marco Conti}
\IEEEauthorblockA{\textit{Istituto di Informatica e Telematica} \\
\textit{Consiglio Nazionale delle Ricerche}\\
Pisa, Italy \\
marco.conti@iit.cnr.it}
\thanks{\IEEEauthorrefmark{2} C. Boldrini and L. Valerio contributed equally to this work.}%
}
} % END if anonymous

\maketitle

\begin{abstract}
Decentralized federated learning (DFL) enables devices to collaboratively train models over complex network topologies without relying on a central controller. In this setting, local data remains private, but its quality and quantity can vary significantly across nodes. The extent to which a fully decentralized system is vulnerable to poor-quality or corrupted data remains unclear, but several factors could contribute to potential risks. Without a central authority, there can be no unified mechanism to detect or correct errors, and each node operates with a localized view of the data distribution, making it difficult for the node to assess whether its perspective aligns with the true distribution. Moreover, models trained on low-quality data can propagate through the network, amplifying errors. 
To explore the impact of low-quality data on DFL, we simulate two scenarios with degraded data quality—one where the corrupted data is evenly distributed in a subset of nodes and one where it is concentrated on a single node—using a decentralized implementation of FedAvg. Our results reveal that averaging-based decentralized learning is remarkably robust to localized bad data, even when the corrupted data resides in the most influential nodes of the network. Counterintuitively, this robustness is further enhanced when the corrupted data is concentrated on a single node, regardless of its centrality in the communication network topology. This phenomenon is explained by the averaging process, which ensures that no single node—however central—can disproportionately influence the overall learning process. 
\end{abstract}

\begin{IEEEkeywords}
fully decentralized learning, federated learning, supervised learning, complex networks, low-quality data, label flip, generative models
\end{IEEEkeywords}

\section{Introduction}
The Federated Learning paradigm leverages the use of multiple entities connected within a network, empowered with enough computational power to solve the task of knowledge extraction and model training. Each device extracts knowledge from locally available data by a training machine learning model and shares it with its neighbors, resulting in local knowledge spread across the network without the raw data leaving the device. Thus, problems that typically require access to all the data produced by the system, as done in centralized solutions, are solved using multiple learning devices trained in parallel on locally available data. Federated learning is studied within environments where it is not possible or convenient to share raw data. Healthcare facilities have strict privacy-related problems preventing them from sharing patients' data outside the infrastructure where they are collected or produced. Sharing locally trained neural networks has been one key solution to combine the knowledge coming from multiple sources and improve the predictions' performance~\cite{tedeschini2022decentralized}\cite{shiranthika2023decentralized}. Internet of Things (\textit{IoT}) networks usually have bandwidth limits that prevent sharing large volumes of data across the network. Sharing a compressed data representation may represent the only solution to learning a shared objective between devices using all the data available to the system. When computing power is not scarce and learning is required, decentralized learning could be a valid approach to avoid sending large quantities of data to a central server where learning is done via a centralized technique. Moreover, end users are increasingly concerned about privacy. Sharing only the knowledge extracted from the end devices without exposing the data used to extract that knowledge offers an effective alternative to centralized learning solutions that require all data to be accessible during training.

Compared to a centralized solution where data is centrally collected and trained in the cloud, federated learning introduces a greater number of degrees of freedom. The devices in the network can be initialized either homogeneously or heterogeneously, and data distribution across devices can be either identically and independently distributed (IID) or non-IID. In non-IID settings, each node possesses a unique perspective, differing from its neighbors. When devices learn exclusively from their local data, generalization across the entire system suffers, as no single node has access to the full data distribution. In star-like topologies, a central entity—typically called a Parameter Server (PS)—can coordinate knowledge sharing, a paradigm known as centrally coordinated federated learning (FL, hereafter). Conversely, when network conditions or topology prevent central coordination, the system operates under a fully decentralized federated learning paradigm (DFL, for short). In this paper, we focus on \emph{fully decentralized} learning systems with non-IID data distributions and homogeneous initialization, where no central coordination is available.

Previous research has demonstrated the robustness and resilience of decentralized learning to structural failures in network topology during training~\cite{palmieri2024robustness}. Specifically, DFL has shown that network disruptions have minimal impact on learning accuracy, provided that the remaining nodes retain sufficiently representative data. Moreover, most of the knowledge acquired before network disruptions is preserved even when some nodes become unreachable. Similarly, DFL can effectively handle highly skewed data distributions, as long as each node has access to a few representative examples~\cite{valerio2023coordination}. However, to the best of our knowledge, the impact of low-quality data on DFL has not been systematically investigated. Building on these insights, this paper examines the robustness of decentralized federated learning from a data quality perspective, focusing on the widely used yet simple model aggregation method based on averaging. We address the following research questions:
%
\begin{itemize}
\item \textbf{RQ1:} \emph{How sensitive is average-based decentralized federated learning to low-quality or corrupted data?}
\item \textbf{RQ2:} \emph{To what extent is this sensitivity influenced by the underlying network topology?}
\end{itemize}
%
To investigate these questions, we simulate decentralized environments where some nodes’ local datasets contain low-quality samples. In real-world scenarios, such data degradation may arise from sensor noise, faulty data augmentation, or intentional adversarial interference aimed at disrupting the learning process. Regardless of the cause, these misleading samples (referred to as \emph{corrupted data}) can affect DFL, and this paper aims to systematically assess their impact.

We generate malformed data using interpolations in the latent space of a pre-trained Generative Adversarial Network (\textit{GAN}), following the method described in~\cite{BKHMSILatentInterpolation}. This interpolation approach enables targeted corruption of specific labels with varying intensities, providing a controlled environment to analyze its effect on model performance.
We compare fully decentralized and federated learning approaches against a centralized baseline to assess their resilience to data corruption. Our experiments consider both balanced and imbalanced data distributions, including scenarios where central nodes (e.g., hubs) concentrate most of the available information. Additionally, we introduce increasing levels of feature corruption, ranging from a small fraction of the dataset up to 90\%, to evaluate the impact on learning outcomes.
 
Our key findings are the following:
%
\begin{itemize}
    \item Corrupted data primarily affects both the target and collateral classes, while overall accuracy remains largely stable, indicating minimal impact on other classes. Interestingly, corrupted samples from the target class that resemble the collateral class degrade the classifier’s performance more on the collateral class than on the target class itself.
    
    \item The distribution of bad data plays a crucial role in its impact: corruption spread across multiple nodes has a far more detrimental effect than when concentrated in a single node, regardless of that node’s influence in the network.
    
    \item When data corruption is unevenly distributed, decentralized learning proves to be more resilient than its centralized counterpart. While this does not definitively establish decentralized learning as the more robust approach, it highlights the need for further comparative analysis.
    
    \item Federated learning demonstrates superior long-term robustness to data corruption compared to a fully decentralized learning setup.
\end{itemize}

\section{Related Work}
We emphasize that this paper focuses on the susceptibility and robustness of decentralized learning to low-quality data distributed across the network topology. While this issue may bear some resemblance to adversarial machine learning, our perspective is fundamentally different, as we do not specifically address adversarial attacks. However, given the conceptual overlap, we include, alongside the relevant literature on decentralized learning, a brief overview of the literature on attacks against decentralized learning.

\subsection{Decentralized learning}
Most of the work studying the robustness of decentralized learning against data poisoning mainly focuses on attacks targeting federated learning and trying to compromise the model integrity by sending ad-hoc updates to the parameter server. Federated learning~\cite{mcmahan2017communication}, which currently has some of its applications in the field of user keyboard input prediction~\cite{hard2018federated} and healthcare~\cite{shiranthika2023decentralized}, is a form of decentralized learning where the underlying topology is star-like and the central server coordinates the learning. 

Decentralized learning, more broadly, offers a solution to learn within networks of devices, each holding a unique dataset, where constraints prevent the sharing of raw data across the network. The constraints are usually due to bandwidth limits or privacy requirements. Briefly, each device in the network starts by training a local model for a few epochs on the locally available data. Then, the distilled knowledge, usually the locally trained model, is shared with the neighbors. Finally, each device collects the models received by its neighbors and merges them into a new, updated model. Three steps are outlined: i) local training, ii) synchronization, iii) update. The three phases are then repeated until all devices reach convergence and models do not change anymore. It must be specified that local learning is analogous to the centralized paradigm, with all the techniques available such as early stopping on validation data extracted from the local dataset. Within these settings, learning is still possible even with non-IID data distributions and heterogeneous initialization~\cite{valerio2023coordination}. 

\subsection{Attacks against decentralized learning}

Unlike adversarial attacks that target models at test time~\cite{tabacof2016exploring}, our work focuses on the local training phase within a decentralized learning paradigm. Most prior research has concentrated on attacks against federated learning, along with corresponding countermeasures and detection mechanisms. In contrast, we aim to analyze overall network performance in a fully decentralized, uncoordinated, and potentially data-imbalanced scenario where nodes hold misleading and incorrectly labeled data.

Our scenario aligns closely with the category of \textit{label flip} attacks. In~\cite{bhagoji2019analyzing}, the authors investigated a federated IID setting with a single malicious user who manipulated gradient updates sent to the parameter server. Their approach leveraged either an additional scaling factor or a custom loss function to maximize the stealth and effectiveness of the attack. In~\cite{sun2024gan}, the authors proposed \textit{VagueGAN}, a modified GAN trained on an altered objective function that degrades the generator’s capabilities. In their experiments, they augmented the local dataset of each malicious node by 10\% using the pre-trained GAN and performed a label flip attack, converting all samples of class 6 to class 0 in the MNIST dataset. Their study examined scenarios with up to 30\% malicious participants.

Similarly, \cite{zhang2020poisongan} introduced \textit{PoisonGAN} to address cases where an attacker lacks access to a pre-trained generator or the necessary data distribution for training it. Their approach leveraged the global model managed by the parameter server as a discriminator to bypass data access limitations. Our work departs from these studies by assuming that nodes have full access to a pre-trained generator, which they use to construct or augment local datasets before training begins . %(further details in Sec.~\ref{subsec:threat_model}).

In~\cite{tolpegin2020data}, the authors conducted an extensive study on label-flip poisoning in federated learning with IID data distributions. They examined the effects of poisoning by deploying multiple attackers at different time steps, specifically targeting selected labels. A comprehensive survey on poisoning attacks in federated learning is provided in~\cite{xia2023poisoning}. Meanwhile,~\cite{pham2024data} explored fully decentralized learning systems by injecting adversarial samples to induce backdoor behavior, forcing the victim model to respond to specific image triggers with predetermined outputs. This type of attack, known as a \textit{backdoor attack}, allows an adversary to embed a hidden objective in the model that can be exploited at test time. In contrast, our approach simply shifts the label of one class to another (specifically, class 4 to class 9) without introducing any additional objectives.

Unlike~\cite{gentz2015detection}, where malicious nodes execute adversarial code, our setting assumes that nodes have control only over their local dataset and do not run any harmful code. Lastly, we extend the ideas presented in~\cite{cao2019understanding}, where the impact of corrupted samples in federated scenarios was analyzed, to the extreme case of a fully decentralized setting with imbalanced data distribution. In our scenario, samples from two easily confusable classes are further corrupted using a generative model and assigned to the wrong class.

To the best of our knowledge, this is the first study to jointly investigate the effects of data corruption on a complex network topology in fully decentralized federated learning.
% Instead of targeting the model at test time as done in adversarial attacks~\cite{tabacof2016exploring}, our work focuses on the local training phase of the decentralized paradigm. Most prior work focuses on attacks against federated learning, countermeasures, and detection mechanisms. Instead, we are interested in analyzing the overall network performance in a decentralized, uncoordinated, and possibly data-unbalanced scenario where nodes hold misleading and wrongly labeled data. Our scenario is close to the \textit{label flip} attacks category. Authors in~\cite{bhagoji2019analyzing} experimented with a unique malicious user in a federated IID scenario. In their work. they leverage either an additional scaling factor or a custom loss for the gradient update sent by the attackers to the parameter server to maximize the stealthiness and effectiveness of the attack. 
% Authors in~\cite{sun2024gan} proposed a modified version of GAN, namely \textit{VagueGAN}, which is trained on a modified objective by lowering the generator capabilities. In their experiments, they augmented the local dataset of each malicious node by a $10\%$ factor using the pre-trained GAN, then executed a label flip over all samples of class $6$ towards class $0$ of MNIST dataset. In their work, they experimented with an increasing number of malicious users, up to $30\%$.~\cite{zhang2020poisongan} proposed \textit{PoisonGAN} to handle scenarios where the attacker has no access to a pre-trained generator or has no access to the data distribution required to train it. Within this scenario, they showed that it is possible to leverage the global model managed by the parameter server to act as a discriminator, thus overcoming the data problem. We move away from that work by assuming that the nodes have full access over a pre-trained generator used to build or augment the local dataset before the learning starts( more details are given in~\ref{subsec:threat_model}). Authors in~\cite{tolpegin2020data} conducted extensive work about label flip poisoning in federated learning with IID data distribution. In their work, they studied the effects of poising by deploying an attacker at different time steps targeting specific labels. A well-detailed survey about federated learning attacks can be found in~\cite{xia2023poisoning}. Authors in~\cite{pham2024data} targeted fully decentralized learning systems by injecting ad-hoc samples to force the victim's model to optimize a hidden objective, namely to respond to specific triggers in the images with constant outputs. This is usually called a \textit{backdoor attack} because an attacker could hide a malicious objective in the model and exploit it at test time with ad-hoc samples. Instead, we shift the output label of a class of samples towards another (namely, class $4$ to class $9$) and do not introduce any additional objective. Within our settings, the nodes do not execute malicious code, as instead assumed in~\cite{gentz2015detection}, and they have control only over their local dataset. Finally, we extended the ideas presented in~\cite{cao2019understanding}, where the authors analyzed the impact of corrupted samples on federated scenarios, to the extreme case of a fully decentralized with unbalanced data distribution across nodes, where samples belonging to two misunderstandable classes are further corrupted via a generative model and assigned to the wrong class. \textcolor{orange}{To the best of our knowledge, this is the first work that jointly investigates the effect of data corruption and the network topology in decentralized, federated learning.}
% This is different from our setup where the attacker can only modify local data and has no control over the learning algorithm. 
% with a poising strategy aiming at achieving a label flip from class $4$ to $9$ and a generative model providing the poisoned samples. 

\begin{figure*}[t]
\includegraphics[width=\linewidth]{res/Dataset_Corruption.png}
\captionof{figure}{The hub's local dataset undergoes corruption at $75\%$ with the corruption strength $\alpha=0.95$. The corrupted images are built using the en }
\label{Fig:dataset_corruption}
\end{figure*}

\section{System Model}
\subsection{Decentralized learning}
\label{subsec:decentralized_learning}

% The following work inherit and extends the system model defined in~\cite{palmieri2024robustness}. 
We start by defining a network $\mathcal{G}(\mathcal{V},\mathcal{E})$ where $\mathcal{V}$ denotes the set of nodes and $\mathcal{E}$ the set of edges. Nodes represent individual learning entities (e.g., devices in a network), and edges represent the connections between them. We will assume that nodes represent devices in a network. Each device follows a three-step algorithm in every communication round: $i$) the device trains a local model using its locally available dataset for a few epochs; $ii$) local model's parameters are sent to the neighboring nodes; $iii$) the models received from the neighbourhood are combined through an \emph{aggregation function} and used to update the local model. A communication round consists of these three steps executed atomically. Decentralized learning proceeds through multiple communication rounds until the training converges across the network, i.e., the local model updates become negligible.

The overall dataset $\mathcal{D}$ can be denoted as the union of all local datasets: 
\[
\mathcal{D} \sim \mathcal{P} = \bigcup\limits_{i=1}^{N} D_i \sim  P_i \quad \text{with} \quad \bigcap\limits_{i=1}^{N} D_i = \emptyset,
\]
where \( D_i = \{(x_j, y_j)\}_{j=1}^{m} \) represents the local dataset of node \( i \), \( N \) is the total number of nodes in the network, and \( \mathcal{P} \) is the global data distribution. Since local data distribution may vary across devices, we explicitly define \( P_i \), the distribution from which \( D_i \) is drawn.

All devices share the same learning objective and underlying neural network architecture. However, local datasets may differ in size and label distribution. For instance, the hub (i.e., the most connected node in the network) may have a small, uniformly distributed dataset (\( \forall i: P_i \sim \mathcal{P} \)) or a large, imbalanced dataset. To isolate the effects of data corruption and avoid confounding factors introduced by model initialization, we assume homogeneous model initialization across devices.

Following~\cite{sun2022decentralized}, let \( h_i \) be the model of device \( i \) at communication round 0, parameterized by \( \mathbf{w}_i \). The device aims to minimize the local loss function \( \ell \) over its dataset \( D_i \):
\begin{equation}
    \tilde{\mathbf{w}}_i = \arg\min_{\mathbf{w}} \sum_{k=1}^{|D_i|} \ell(y_k, h(\mathbf{x}_k; \mathbf{w}_i)).
\end{equation}
At the end of the local training phase, device \( i \) transmits \( \tilde{\mathbf{w}}_i \) to its neighboring nodes. The model for the next communication round is then updated by aggregating the received models in a layer-wise manner, weighted by the respective dataset sizes. Given \( \mathcal{N}(i) \), the set containing node \( i \) and its neighbors, the model update at communication round \( t \) is defined as:
\begin{equation} \label{eq:aggr}
    \mathbf{w}_i^{(t)} \leftarrow
    \frac{\sum_{j \in \mathcal{N}(i)} |D_j| \tilde{\mathbf{w}}_j^{(t-1)}}
    {\sum_{j \in \mathcal{N}(i)} |D_j|}.
\end{equation}
Once \( \mathbf{w}_i^{(t)} \) is updated, the local learning process restarts, and the cycle repeats until convergence.
%
The described approach aligns and generalizes with respect to previous work~\cite{palmieri2024robustness,valerio2023coordination,sun2022decentralized,savazzi2020federated}. The aggregation in Eq.~\eqref{eq:aggr} performs a weighted averaging of the model parameters, akin to the widely used FedAvg\cite{mcmahan2017communication} approach (proposed for centralized FL), but adapted to a decentralized setting.


\subsection{Dataset corruption}
We extend the notation introduced in the previous section by defining $C(\cdot; \mathcal{I}, s)$, a function that maps $\mathcal{D}$ to $\tilde{\mathcal{D}}$, the corrupted dataset. In our case, $\tilde{\mathcal{D}}$ is just $\mathcal{D}$ being subject to a post-processing function $\mathcal{I}$ which targets a specific label class $c_t$ (\emph{target class}), thus:
\begin{equation}
\mathcal{C}(\mathcal{D}; \mathcal{I}, s) = \tilde{\mathcal{D}} =
  \begin{cases}
  (\mathcal{I}(x_i), y_i) & \text{if } \ \ y_i = c_t, \\
  (x_i, y_i) & \text{otherwise}.
  \end{cases}
\end{equation}

To analyze the network's behavior under varying levels of corruption, we conducted experiments where samples for label $c_t$ were drawn either from $\mathcal{D}$ or $\tilde{\mathcal{D}}$ with percentage $p$ gradually increasing from $10\%$ to $90\%$. Given $\mathcal{D}=\{x_i, y_i\}_{1..n}$ and $\tilde{\mathcal{D}}=\{\tilde{x}_i, y_i\}_{1..n}$ we define our final dataset, from which local datasets are drawn, as:
\begin{equation}
\tilde{\mathcal{D}}_p = \{(x_i, y_i)\}_{1..\lfloor p*n \rfloor} \cup \{(\tilde{x}_i, y_i)\}_{\lceil p*n \rceil ..n}
\end{equation}
Our downstream task is supervised image classification. Within this context, we pick two classes, $c_c$ and $c_t$, and we corrupt the features of $c_t$ with an interpolation towards class $c_c$. This corruption makes the learning model more prone to errors when discriminating between the two classes~\cite{sikar2024misclassification}. Interpolated images are built upon the latent representation of a pre-trained GAN~\cite{goodfellow2014generative}, a special case of artificial curiosity~\cite{schmidhuber2020generative}\cite{schmidhuber1990making}. Let   $\mathcal{E}_{net}$ and $\mathcal{D}_{net}$ be the encoder and decoder of our model trained using an adversarial loss, respectively. For each sample $x_i$ subject to corruption, let $x_c, x_t$ be samples randomly drawn from the collateral class $c$ and target class $t$, respectively. We obtain the corrupted sample $\tilde{x}_i$ by combining their latent representation via a simple linear transformation with parameter $\alpha \in [0,1]$ controlling the interpolation strength. We then used the interpolated sample in place of $x_i$.
\begin{equation} \label{eq:corruption}
\tilde{x}_i=\mathcal{I}(x_i) = \mathcal{D}_{net} (\alpha * \mathcal{E}_{net}(x_t) + (1-\alpha) * \mathcal{E}_{net}(x_c))
\end{equation}
% To be noted that the corrupted version of $x_t$ is totally independent from the original sample. 
Figure~\ref{Fig:dataset_corruption} summarizes the process we used to corrupt the datasets in our experiments, applied to the MNIST dataset, where $x_c$ belongs to class 4 and $x_t$ belongs to class 9. In the example, the interpolation transforms the features of class 9 towards those of class 4.   
% Our main focus is the simple supervised image classification task over the MNIST dataset. Within this context, we have chosen to corrupt labels of class nine with an interpolation towards class four. The choice of this corruption exploits the already known underlying cross correlation between the two classes, which makes the learning model more prone to errors when discriminating between the two~\cite{sikar2024misclassification}. Interpolated images are built upon the latent representation of a pre-trained GAN~\cite{goodfellow2014generative}, a special case of artificial curiosity~\cite{schmidhuber2020generative}\cite{schmidhuber1990making}. Let $\mathcal{D}_{net}$ and $\mathcal{E}_{net}$ respectably be the encoder and decoder of our model trained using an adversarial loss. For each $x$ subject to corruption, let $s_4$ and $s_9$ be two samples randomly drawn among class four and class nine. Their latent representation is combined with a simple linear transformation with parameter $\alpha \in [0,1]$ controlling the interpolation strength. We then used the interpolated sample in place of $x$.
% \begin{equation} \label{eq:corruption}
% \mathcal{I}(x) = \mathcal{D}_{net} (\alpha * \mathcal{E}_{net}(s_9) + (1-\alpha) * \mathcal{E}(s_4))
% \end{equation}
% To be noted that the corrupted version of $x$ is totally independent from the original sample. Image~\ref{Fig:dataset_corruption} summarizes the process we used to corrupt the datasets in our experiments.

% \subsection{Data corruption generation model}
% \label{subsec:threat_model}
% In our experiments, we assume that some of the nodes used a generative model to build their local dataset before starting the training process. Such a generative model might have been trained on low or bad-quality data, i.e., this situation might cover several real-world scenarios, such as data collected from malfunctioning sensors or errors in the data labeling process. In this paper, we are not assuming that nodes are maliciously corrupting data. Conversely, users are honest but unaware of the quality of the data they provide to the learning process. Therefore, since we are not assuming a proper adversarial scenario with a proper attacker, we cannot implement any countermeasure or assumption in this aspect. This paper aims to assess the resilience and robustness of vanilla decentralized federated averaging in fully decentralized settings.

% In our experiments, we assumed an external attacker was able to take full control of the device's local dataset. No additional capability over the target devices is granted to malicious users. Attackers have full access to a generative model, pre-trained over the original data distribution of the other clients. We therefore consider the attacker's optimal scenario, impractical under most real non-IID use cases. However, this assumption is acceptable if we think at the current trend in generative artificial intelligence, at the massive adoption of generative foundational models and the general need for classifiers over images and text data. We assume one or more malicious user to be available in the network with the purpose of taking a well established concept, namely that images of the handwritten digit four belongs to class label $4$, and shift the features discriminating class four towards class label $9$. We assume our malicious users compromising the learning process from the first communication round and our attackers' datasets to not change over time.

\section{Experimental Settings}

\subsection{Communication network}

We run experiments for scenarios composed of $50$ nodes both for the federated learning and the fully decentralized setting. In FL the topology is constrained to being star-like (with the coordinating server at the center). In our federated learning setup, we assume that all devices participate in every communication round, whereas some FL algorithms allow only a fraction of nodes to contribute at each round. This choice removes one degree of variability in our experiments, ensuring a consistent participation pattern across all nodes. 

In the decentralised scenarios, we employ a Barabási–Albert (BA) graph topology~\cite{barabasi1999emergence}, which belongs to the family of scale-free networks. These networks are characterized by a few highly connected nodes, known as hubs, which play a crucial role in information spreading due to their extensive connectivity. Given their central position, hubs serve as our primary targets for the investigation, as their models are directly shared with numerous other devices. We generate BA graphs using the NetworkX Python library~\cite{hagberg2008exploring}, setting the parameter  $m$ to $1$, meaning that each newly added node connects to only one existing node. Because the BA model preferentially attaches new nodes to those with higher degrees, this results in a small number of well-connected hubs, while most nodes remain sparsely connected, often with just one link (because $m=1$). 

In both FL and DFL, nodes communicate exclusively with their neighbors in the network. However, in FL, each node has a single neighbor (the central server), whereas in DFL, nodes have varying numbers of neighbors depending on the degree distribution of the communication graph. The BA topology is particularly interesting for studying DFL because its heavy-tailed degree distribution (a common feature in many real-life networks, including social and technological systems~\cite{barabasi2013network}) introduces maximum heterogeneity in node connectivity and communication opportunities.

\subsection{Dataset and data distribution}
\label{sec:data}

\begin{figure}[!t]
    \centering
    \includegraphics[width=0.5\linewidth]{res/Interpolation_strength.png}
    \caption{Examples of interpolated images of nine with images of class four with $\alpha=0.5$ (left) and $\alpha=0.95$ (right)}
    \label{fig:interpolation-strength}
\end{figure}

In our experiments we use the MNIST dataset~\cite{lecun1998mnist}. Given its simplicity and well-defined classes, it provides an ideal testbed for analyzing and understanding the impact of bad data in a controlled manner.
The MNIST dataset consists of $60,000$ training samples of handwritten digits from $0$ to $9$ and a test set containing $10,000$ samples. Each sample consists of a $28 \times 28$ image and a label reporting the digit number. The local datasets of each node are mutually exclusive partitions of the training set, the test set is common to all the nodes. We investigated a scenario where low-quality data affects a specific class. To this aim, we perform label interpolation modifying images labeled as $9$ to resemble those labeled as $4$. We explore interpolation strength (i.e. $\alpha$ in eq.~\ref{eq:corruption}) set to $0.5$ and to $0.95$, the latter basically resulting in a naive label flip. Figure~\ref{fig:interpolation-strength} shows some examples of interpolated samples at different strength levels. 
%
We refer to class $9$ as the \textbf{target class}, as its samples are directly modified, and to class $4$ as the \textbf{collateral class}, since the increasing resemblance of corrupted $9$s to $4$s may indirectly distort the learning process for class $4$. The remaining classes are referred to as \textbf{bystanders or uninvolved classes}. While the parameter $\alpha$ controls the extent to which corrupted $9$s resemble $4$s, the actual number of corrupted samples in class $9$ is governed by parameter $p\in[0,1]$. Consequently, the impact of poor-quality data is maximized when both $\alpha$ and $p$ are close to $1$, i.e., the totality of the target class sample is indistinguishable from those of the collateral class, but the labels remain unchanged, resulting in a sort of label flip.


In our experiments, images from all classes except class $9$ are assigned in an IID manner, ensuring uniform distribution across all nodes. This approach prevents unintended effects where some nodes might receive too little data, which could degrade performance for reasons unrelated to the presence of bad data. By maintaining a balanced data allocation for uninvolved and collateral classes, we ensure that the observed impact is primarily attributable to the presence of corrupted samples.

For class $9$, we consider the following distributions:

\noindent\textbf{Balanced bad data distribution}: Here, the fraction $p$ of corrupted samples is evenly distributed among the most central nodes, which, as previously discussed, play a more influential role in the communication network. Specifically, the $\lceil pN \rceil$ most central nodes receive the corrupted samples\footnote{If the total number of training samples for class \( 9 \) is \( n_9 \), each node is expected to receive approximately \( \frac{n_9}{N} \) samples under a uniform distribution. By concentrating the corrupted samples on the most central nodes, the total number of corrupted samples, \( p n_9 \), is allocated in groups of \( \frac{n_9}{N} \). Thus, the number of nodes affected by corruption is given by \( \frac{p n_9}{n_9/N} = pN \), meaning that \( \lceil pN \rceil\) is the number of nodes receiving corrupted samples.}, meaning that within these nodes, all instances of class $9$ are replaced with corrupted samples.
%If we denote the number of training/validation samples for class $s$ as $n_s$, each node $i$ is expected to receive  $n_s^{(i)} = \frac{n_s}{N}$ samples. 
We allocate bad samples in descending order of centrality, ensuring that only the most central nodes receive them. This setting is termed \textit{balanced} because all affected nodes hold the same number of bad samples, with minor variations due to border effects (e.g., when the number of bad samples is not an integer multiple of the local dataset size for class $9$).
    
\noindent\textbf{Unbalanced bad data distribution}: In this scenario, all corrupted samples are assigned to the most central node, while the remaining nodes receive an equal share of the unaltered samples. Thus, while in the balanced bad data scenario $\lceil pN \rceil$ central nodes receive low-quality data, here only one node gets \textit{all} the corrupted data. Figure~\ref{fig:nonIID_data_scatter_plot} provides a visual representation of this distribution, when the hub holds $90\%$ of class-9 samples.

Table~\ref{tab:datadist} summarizes the data distribution under the two scenarios. Figure~\ref{fig:data_distribution} illustrates the different configurations tested in our simulations. Note that we mimic the balanced and unbalanced corrupted data distributions when testing FL. However, since FL is inherently constrained to a star-like topology, where each device has exactly one connection (to the central server), node centrality does not influence the learning process in this setting.

\begin{figure}[t!]
    \centering
    \includegraphics[width=\linewidth]{res/nonIID_data_distribution.png}
    \caption{Example of unbalanced corrupted data distribution.}
    \label{fig:nonIID_data_scatter_plot}
\end{figure}

\begin{figure}[t!]
    \centering
    \includegraphics[width=\linewidth]{res/Experiment_Data_Distribution.png}
    \caption{We simulated four main scenarios: (a) decentralized with balanced bad data distribution, (b) federated with balanced bad data distribution, (c) decentralized with unbalanced bad data distribution, and (d) federated with unbalanced bad data distribution. Non-target classes (i.e., all classes except class $9$) are distributed uniformly.}
    \label{fig:data_distribution}
\end{figure}

% Please add the following required packages to your document preamble:
% \usepackage{booktabs}
\begin{table}[!t]
\renewcommand{\arraystretch}{2}
\centering
\caption{Data distribution summary. $v_{(i)}$ denotes the $i$-th most central node. $n_i$ is the number of training samples in class $i$. $N$ is the number of nodes. $p$ is the fraction of corrupt samples.}
\label{tab:datadist}
\begin{tabular}{@{}cccccc@{}}
\toprule
\textbf{} & \textbf{Class $i \in \{0,8\}$} & \multicolumn{2}{c}{\textbf{Bal. Class 9}} & \multicolumn{2}{c}{\textbf{Unbal. Class 9}} \\ \cmidrule(lr){3-4} \cmidrule(lr){5-6} 
             &                 & \textbf{Good}   & \textbf{Bad}    & \textbf{Good}           & \textbf{Bad} \\ \midrule
$v_{(1)}$    & $\dfrac{n_i}{N}$ & 0               & $\dfrac{n_9}{N}$ & 0                       & $p n_9$      \\
...          & $\dfrac{n_i}{N}$ & 0               & $\dfrac{n_9}{N}$ & $\dfrac{(1-p) n_9}{N-1}$ & 0            \\
$v_{(pN)}$   & $\dfrac{n_i}{N}$ & 0               & $\dfrac{n_9}{N}$ & $\dfrac{(1-p) n_9}{N-1}$ & 0            \\
$v_{(pN+1)}$ & $\dfrac{n_i}{N}$ & $\dfrac{n_9}{N}$ & 0               & $\dfrac{(1-p) n_9}{N-1}$ & 0            \\
...          & $\dfrac{n_i}{N}$ & $\dfrac{n_9}{N}$ & 0               & $\dfrac{(1-p) n_9}{N-1}$ & 0            \\
$v_{(N)}$    & $\dfrac{n_i}{N}$ & $\dfrac{n_9}{N}$ & 0               & $\dfrac{(1-p) n_9}{N-1}$ & 0            \\ \bottomrule
\end{tabular}
\end{table}

% %%%%%%%%%%%%%%%%% OLD
% In our experiments, we investigated both IID and non-IID data distributions. However, when studying non-IID scenarios, the classes not subject to the attack (i.e. 0-8) were distributed uniformly across the network. This helped us avoid unwanted effect coming from the random distribution of samples from other classes. For this reason we will also refer to IID scenarios as data balanced and non-IID as data unbalanced, to make explicit the difference from a real IID or non-IID scenario where all the classes are distributed uniformly or not. Figure~\ref{fig:nonIID_data_scatter_plot} shows an example of a non-IID data distribution used in one of our experiments. Specifically, the hub holds $90\%$ of the whole dataset for class 9. We started by taking an even number of samples for each class (i.e. $5421$ samples), then we assigned the hub $90\%$ of the total training samples for class $9$ and then reported its training set (i.e. $80\%$).

% Our main purpose is to study the performance of the network under increasing dataset corruption percentages, up to $90\%$ of the globally available samples for the poisoned class $9$. When studying unbalanced data distributions, we were able to assign a different number of samples to each device. All samples belonging to classes other than class $9$ were distribute uniformly across the network. We then corrupted all the samples globally available for class $9$ up to a given percentage $p$ to get $\tilde{\mathcal{D}}_{p}$. Then, we forced the hub to hold all the corrupted samples. The remaining uncorrupted samples for class $9$ were distributed evenly across the network. To summarize, for data unbalanced experiments we:
% \begin{itemize}
%     \item Assigned samples belonging to class zero to eighth uniformly across the network.
%     \item Corrupted up to a percentage $p$ of the globally available samples for class nine.
%     \item While studying unbalanced data distributions we assigned all corrupted samples to the most central node in the network, the hub.
%     \item The remaining non-interpolated samples were distributed uniformly across the network, hub excluded.
% \end{itemize}
% Thus, in unbalanced scenarios with sufficient global corruption percentage (e.g. $p=0.9$) the hub holds most of the data, which is however corrupted. The remaining devices hold the remaining samples available for class $9$ in the original MNIST dataset (i.e. $10\%$). 

% Experiments involving IID data distributions, referred as balanced, required in the most extreme scenario to corrupt and distribute $90\%$ of the globally available samples across the network. In data balanced experiments we were constrained by a uniform sample distribution and were not able assign all corrupted samples to one node only. We built the devices' local datasets one at a time starting by distributing only the corrupted samples. Once all poisoned samples have been distributed we started assigning samples coming from the original dataset and not subject to corruption. That is, when corrupting $90\%$ of the dataset ($\tilde{\mathcal{D}}_{.9}$), given that the dataset is distributed uniformly across the network which is comprised of $50$ devices, we selected $45$ devices and assigned all corrupted samples to them. Then, we assigned the remaining non-interpolated samples to the remaining $5$ devices. In IID experiments, due to the fact that the hub only holds a small percentage of the globally available samples, we poisoned the hub and other random nodes near by. 

\subsection{Learning \& simulation settings}

Each node locally trains a CNN configured according to state-of-the-art practices with a cross-entropy supervised learning objective. The CNN consists of two convolution layers (kernel size is set to $5$), each followed by max pooling and a ReLu activation function. Dropout interleaves the two layers and the output is given by two fully connected layers with a ReLu activation function and dropout applied between the two. We used the stochastic gradient descent with learning rate set to $1e-3$ and momentum set to $0.9$. In all our experiments we set the batch size to $32$. We also set the local validation size to $20\%$ of all the training set local to each node. We run simulations up to $1,000$ communication rounds, with each node training for a maximum of $5$ local epochs. Early stopping is applied to prevent local model degradation. We replicate our simulations varying both the seed governing graph generation and the seed controlling other sources of randomness beyond the network topology. We also present the results for a centralized benchmark, where the same CNN is trained on the overall dataset $\mathcal{D}$.
% We run at least three simulations for each configuration involving either a non-IID data distribution or an IID data distribution but complex topology (i.e. fully decentralized). 


\subsection{Performance metrics}
We evaluated each node’s performance using a common test set shared across all devices. At each communication round, we collected the confusion matrix for each node and extracted two key metrics: \textit{accuracy} and \textit{F1 score}. Depending on the analysis, we present these metrics in different ways: averaged across all nodes, within the neighborhood of target nodes, or for a specific subset of nodes. We will specify the chosen approach as needed throughout the discussion. When not specified, the results we present in this paper report the mean value at each communication round together with the $95\%$ confidence interval computed over the different seeds used.

% ed both the loss and the confusion matrix at each local training and aggregation phase end. From the confusion matrix we extracted: \textit{accuracy}, \textit{recall}, \textit{precision}, \textit{F1 score}. We also measured the mean confusion matrix for neighbors, either locally for a subset of nodes or globally for each node in the network. 

% Moreover, for each experiment we extracted a set of features ranging from metrics computed to describe the graph (e.g. its diameter) to features computed over the datasets of nodes. 
% We then computed a correlation analysis to measure what features are more correlated with the accuracy measured for the class subject to the attack. We measured both the Spearman correlation with $p<0.05$ and the Lasso weights. Finally, most of the results are presented by aggregating each metric recorded for the same experiment configuration run with different seeds both for the data distribution and the graph topology.

\section{Results}
% Given the experimental settings presented in the last section by which we are able to measure the performance when corrupting a percentage $p$ of the globally available samples for class nine (experiments run using $\tilde{\mathcal{D}}_p$) we present the results of our simulations. 

Before proceeding with a detailed analysis, we we first highlight a key observation: our experiments indicate that decentralized federated averaging is largely robust to the presence of corrupted data in the system. Specifically, we found that data corruption effects become clearly pronounced when its strength is set to $\alpha=0.95$. Consequently, unless stated otherwise, all results presented in this work have been obtained using this corruption strength. Refer to Figure~\ref{fig:interpolation-strength} for a qualitative illustration of the impact of corruption for different $\alpha$.

We begin by illustrating the impact of low-quality data over time in Figure~\ref{fig:dec,iid,corruption_strength.95}, where we vary the dataset corruption percentage (\( p \) in \( \tilde{\mathcal{D}}_p \)). The figure presents results for DFL in the most vulnerable scenario identified: balanced corrupted data distribution.
%
The plot tracks performance at each communication round using three key metrics. The first is \textit{accuracy}, which reflects the overall impact of the attack on all classes. The second is the \textit{F1 score for class 9}, denoted as \f{9}, quantifying degradation in the corrupted class. The third is the \textit{F1 score for class 4}, denoted as \f{4}, capturing potential misclassification side effects in the collateral class.  
%
With no corrupted data, the performance of decentralized learning soon approaches that of a centralized approach. However, as $p$ increases, the impact of low-quality data for class $9$ is clearly visible, and as expected, its severity increases with the corruption fraction \( p \). As the proportion of bad samples grows, accuracy and F1 scores progressively decline. However, the overall accuracy experiences only a mild drop, with at most a 10 percentage point decrease compared to an uncorrupted scenario. Notably, the corruption of class 9 strongly affects class 4, leading to increasing misclassification, i.e., low \f{4}. \textit{These results suggest that low-quality data primarily affects both the target class (class 9) directly and the collateral class (class 4) indirectly, while having minimal impact on the remaining classes.}
% We start by showing in Figure~\ref{fig:dec,iid,corruption_strength.95} the effects of a label poisoning attack over time using different dataset corruption percentages (i.e., varying the value of $p$ in $\tilde{\mathcal{D}}_p$). The plot reports the most vulnerable scenario we found, that is fully decentralized learning with an IID data distribution, which we also referred as balanced poisoning. The figure shows the performance computed at each communication round via three different metrics: the accuracy, reported to understand the effects of the attack against all the classes, the $F1$ score computed over class $9$ and class $4$ to understand the impact of the attack. The first thing that can be noticed is that \textit{the effects of poisoning are visible} since the performance for datasets containing progressively more interpolated samples becomes lower and lower. \textit{The attack against class $9$ produces strong effects over class $4$} which starts to be misclassified. However \textit{the overall accuracy present milder effects} of a factor of at most $10\%$ degradation with respect to a corruption free scenario. This strongly suggest that the other classes do not get affected by the attack.

\begin{figure}[t!]
    \centering
    \includegraphics[width=\linewidth]{res/decentralized,iid,corruption_strength.95.png}
    \caption{Comparison of the performance between a centralized paradigm (dashed line) and DFL (continuous line) under \textbf{balanced corrupted data distribution}, with a corruption strength of $\alpha=95\%$ and fraction of bad samples ranging from $p=0.1$ up to $p=0.9$.}
    \label{fig:dec,iid,corruption_strength.95}
\end{figure}

Table~\ref{tab:centralized_baseline} presents the F1 score for the best-performing round under high corruption conditions, i.e., with a corruption strength of $\alpha=95\%$ and $p=90\%$ of class 9 samples globally corrupted. For comparison, the table also includes baseline performance in the absence of corruption. A key observation is that, in a corruption-free scenario, FL, DFL, and centralized learning achieve rather similar performance (highlighted in grey in Table~\ref{tab:centralized_baseline}). However, when corrupted data is introduced, all learning strategies experience performance degradation, as indicated by the green-highlighted cells showing lower F1 scores. Interestingly, the impact of corruption varies depending on the distribution of corrupted data across nodes. In the unbalanced case (highlighted in light green), the performance drop remains relatively small, with a worst-case loss of 8 percentage points (pp) with respect to the corresponding configuration without corruption. In contrast, in the balanced corruption scenario (darker green), the degradation is much more severe, with class 4 (the collateral class) losing approximately 30 pp and class 9 (the target class) losing around 20 pp.
\textit{Surprisingly, the performance drop is more severe for class 4 than for class 9. This suggests that classifying 4s becomes substantially more challenging due to the presence of corrupted samples labeled as 9s that visually resemble 4s. Conversely, corruption on 9s appears to act—counterintuitively—as a form of generalization, making it easier for the model to handle class 9. }As a result, while the model does make mistakes on 9s, it makes significantly more errors on 4s, whose dataset remains unaltered yet is indirectly affected by the misleading samples introduced into class 9.

\definecolor{Balanced}{RGB}{140, 210, 140} % Darker green 
\definecolor{Unbalanced}{RGB}{195, 255, 180} % Light green 

% \begin{table}[!t]
% \renewcommand{\arraystretch}{1.3}
% \caption{Comparison of final F1 scores w/o corruption and w/ corruption ($\alpha=0.95$ and $p=0.9$)}
% \label{tab:centralized_baseline}
% \centering
% % \begin{tabular}{cccccc}
% \begin{tabular}{@{}cccccc@{}}
% \toprule
% \textbf{Method} & \textbf{Data Distr.} & \textbf{Corruption} & \textbf{F1(4)} & \textbf{F1(9)}\\
% \midrule
% \multirow{2}{*}{Centralized} & \multirow{2}{*}{-} & \cellcolor{lightgray}No & \cellcolor{lightgray}0.991 & \cellcolor{lightgray}0.980 &\\
%  & & \cellcolor{Balanced}Yes & \cellcolor{Balanced}0.837 & \cellcolor{Balanced}0.831 \\
% \midrule
% % federated & IID & No & 0.992 & 0.985 \\
% % fully dec. & IID & No & 0.988 & 0.981 \\
% % federated & non-IID & No & 0.985 & 0.977 \\
% % fully dec. & non-IID & No & 0.988 & 0.975 \\
% \multirow{4}{*}{FL} & \multirow{2}{*}{Bal.} & \cellcolor{lightgray}No & \cellcolor{lightgray}0.992 & \cellcolor{lightgray}0.985 \\ 
%  &  & \cellcolor{Balanced}Yes & \cellcolor{Balanced}0.666 & \cellcolor{Balanced}0.790 \\    
%  \cmidrule{2-5}
%  & \multirow{2}{*}{Unbal.} & \cellcolor{lightgray}No & \cellcolor{lightgray}0.985 & \cellcolor{lightgray}0.977 \\
%  &  & \cellcolor{Unbalanced}Yes & \cellcolor{Unbalanced}0.953 & \cellcolor{Unbalanced}0.932 \\
% \midrule
% \multirow{4}{*}{DFL} & \multirow{2}{*}{Bal.} & \cellcolor{lightgray}No & \cellcolor{lightgray}0.988 & \cellcolor{lightgray}0.981 \\
%  &  & \cellcolor{Balanced}Yes & \cellcolor{Balanced}0.632 & \cellcolor{Balanced}0.759 \\  
%  \cmidrule{2-5}
%  & \multirow{2}{*}{Unbal.} & \cellcolor{lightgray}No & \cellcolor{lightgray}0.988 & \cellcolor{lightgray}0.975 \\
%  &  & \cellcolor{Unbalanced}Yes & \cellcolor{Unbalanced}0.935 & \cellcolor{Unbalanced}0.897 \\ 
% \bottomrule
% \end{tabular}
% \vspace{0.5em} 
% \vspace*{-\baselineskip} 
% \end{table}

\begin{table}[!t]
\renewcommand{\arraystretch}{1.3}
\caption{Comparison of final F1 scores w/o corruption and w/ corruption ($\alpha=0.95$ and $p=0.9$)}
\label{tab:centralized_baseline}
\centering
% \begin{tabular}
\begin{tabular}{ccccc}
\toprule
\textbf{Corr.?}   & \textbf{Method}       & \textbf{Bad Data} & \textbf{F1(4)}                & \textbf{F1(9)}                \\ 
 & &\textbf{Dist.}& &  \\\midrule
                     & Centr.           & \cellcolor{lightgray}-          & \cellcolor{lightgray}$0.991\pm0.000$                         & \cellcolor{lightgray}$0.980\pm0.000$                          \\ \cmidrule(l){2-5} 
                     &                       & \cellcolor{lightgray}Balanced   & \cellcolor{lightgray}$0.992\pm0.000$                         & \cellcolor{lightgray}$0.985\pm0.000$                         \\
                     & \multirow{-2}{*}{FL}  & \cellcolor{lightgray}Unbalanced & \cellcolor{lightgray}$0.986\pm0.002$                         & \cellcolor{lightgray}$0.976\pm0.001$                         \\ \cmidrule(l){2-5} 
                     &                       & \cellcolor{lightgray}Balanced   & \cellcolor{lightgray}$0.989\pm0.001$                         & \cellcolor{lightgray}$0.982\pm0.002$                         \\
\multirow{-5}{*}{No} & \multirow{-2}{*}{DFL} & \cellcolor{lightgray}Unbalanced & \cellcolor{lightgray}$0.986\pm0.002$                         & \cellcolor{lightgray}$0.976\pm0.002$                         \\ \midrule
                     & Centr.           & \cellcolor{Unbalanced}-          & \cellcolor{Unbalanced}$0.837 \pm 0.000$ & \cellcolor{Unbalanced}$0.831 \pm 0.000$ \\ \cmidrule(l){2-5} 
                     &                       & \cellcolor{Balanced}Balanced   & \cellcolor{Balanced}$0.666 \pm 0.000$ & \cellcolor{Balanced}$0.790 \pm 0.000$  \\
                      & \multirow{-2}{*}{FL}  & \cellcolor{Unbalanced}Unbalanced              & \cellcolor{Unbalanced}$0.952\pm0.001$ & \cellcolor{Unbalanced}$0.929\pm0.003$ \\ \cmidrule(l){2-5} 
                     &                       & \cellcolor{Balanced}Balanced   & \cellcolor{Balanced}$0.640\pm0.034$ & \cellcolor{Balanced}$0.758\pm0.011$ \\
\multirow{-5}{*}{Yes} & \multirow{-2}{*}{DFL} & \cellcolor{Unbalanced}Unbalanced              & \cellcolor{Unbalanced}$0.927\pm0.007$ & \cellcolor{Unbalanced}$0.886\pm0.014$ \\ \bottomrule
\end{tabular}
\vspace{0.5em} 
\vspace*{-\baselineskip} 
\end{table}

% % Please add the following required packages to your document preamble:
% % \usepackage{booktabs}
% % \usepackage{multirow}
% % \usepackage[table,xcdraw]{xcolor}
% % Beamer presentation requires \usepackage{colortbl} instead of \usepackage[table,xcdraw]{xcolor}
% \begin{table}[]
% \renewcommand{\arraystretch}{1.3}
% \centering
% \begin{tabular}{@{}cccccc@{}}
% % \centering
% \toprule
%                               &     & \multicolumn{2}{c}{\textbf{Balanced}}                         & \multicolumn{2}{c}{\textbf{Unbalanced}}                       \\ 
% \multirow{-2}{*}{\textbf{Method}} &
%   \multirow{-2}{*}{\textbf{Corruption}} &
%   \multicolumn{1}{c}{\textbf{F1(4)}} &
%   \multicolumn{1}{c}{\textbf{F1(9)}} &
%   \multicolumn{1}{c}{\textbf{F1(4)}} &
%   \multicolumn{1}{c}{\textbf{F1(9)}} \\ \midrule
%                               & No  & 0.991                         & 0.98                          &                               &                               \\
%                             \multirow{-2}{*}{Centralized} & Yes & \cellcolor[HTML]{EBF1DE}0.837 & \cellcolor[HTML]{EBF1DE}0.831 &                               &                               \\ \midrule
%                               & No  & 0.992                         & 0.985                         & 0.985                         & 0.977                         \\
% \multirow{-2}{*}{FL}          & Yes & \cellcolor[HTML]{D8E4BC}0.666 & \cellcolor[HTML]{D8E4BC}0.79  & \cellcolor[HTML]{76933C}0.953 & \cellcolor[HTML]{76933C}0.932 \\ \midrule
%                               & No  & 0.988                         & 0.981                         & 0.988                         & 0.975                         \\
% \multirow{-2}{*}{DFL}         & Yes & \cellcolor[HTML]{D8E4BC}0.632 & \cellcolor[HTML]{D8E4BC}0.759 & \cellcolor[HTML]{76933C}0.935 & \cellcolor[HTML]{76933C}0.897\\
% \bottomrule
% \end{tabular}
% \end{table}


Figures~\ref{fig:iid,corruption_strength.95} and~\ref{fig:noniid,corruption_strength.95} report accuracy, \f{4}, \f{9} for both FL and DFL. In the \emph{balanced bad data} configuration (Fig.~\ref{fig:iid,corruption_strength.95}), we observe curves that are clearly separated for different values of $p$. From the temporal evolution standpoint, the figure highlights that FL also converges faster than DFL over time, especially for small $p$. In the unbalanced bad data scenario (Fig.~\ref{fig:noniid,corruption_strength.95}), where corrupted samples are concentrated on a single node, performance differences are minimal for low to intermediate values of  $p$, with FL and DFL exhibiting nearly identical behavior. For high  $p$, DFL initially learns faster but is eventually overtaken by FL, which surpasses its performance at later stages. As previously noted, the unbalanced bad data case has a significantly smaller impact on overall performance, as the corruption remains localized and does not propagate widely across the network.

% The performance gap between balanced and unbalanced cases becomes more visible when comparing the performance over time of an IID data distribution to a non-IID scenario. Figure~\ref{fig:iid,corruption_strength.95} and~\ref{fig:noniid,corruption_strength.95} report the accuracy and $F1$ score computed at each communication round. In figure~\ref{fig:iid,corruption_strength.95}, both the overall accuracy and the ability of the networks to correctly classify images of class $4$ and $9$ gets severely affected as the number of poisoned samples grows. Conversely, in the non-IID unbalanced scenario shown in figure~\ref{fig:noniid,corruption_strength.95} the impact of the attack is significantly smaller, with the F1 score for class $9$ achieving a value over $0.75$.

\begin{figure}[!t]
    \centering
    \includegraphics[width=\linewidth]{res/iid,corruption_strength.95.png}
    \caption{Balanced corrupted data: DFL vs FL, with corruption strength $95\%$.}
    \label{fig:iid,corruption_strength.95}
\end{figure}
\begin{figure}[!t]
    \centering
    \includegraphics[width=\linewidth]{res/noniid,corruption_strength.95.png}
    \caption{Unbalanced corrupted data: DFL vs FL, with corruption strength $95\%$.}
    \label{fig:noniid,corruption_strength.95}
\end{figure}

Taken together, these results indicate that, for the same amount of bad-quality data,\textit{ the impact is significantly greater when the corrupted samples are distributed across multiple nodes rather than concentrated within a single node, regardless of the node’s centrality.}
% We argue that the difference in performance between the balanced and unbalanced low-quality sample distributions arises from the fact that having a neighbor with uncorrupted data helps mitigate the effects of corruption. This is supported by analyzing both the performance of individual nodes after the aggregation phase and the underlying network topology. 
% 
Building on this finding, we further argue that having a neighbor with uncorrupted data helps mitigate the effects of corruption. Figure~\ref{fig:topology-F1-4} presents an analysis of a DFL simulation with $p = 0.9$. The leftmost figure visualizes the network topology, where each node is colored based on its mean \f{4}. Nodes circled in red represent devices that do not contain corrupted samples. The rightmost figure plots the mean \f{4} score across the entire network against the \f{4} scores of individual nodes, particularly those connected to nodes with uncorrupted data. 
The results clearly show that having at least one ``clean" neighbor significantly mitigates the negative effects of corrupted data, leading to a local \f{4} score that exceeds the network-wide average. From this, we conclude that denser networks may exhibit greater resilience to low-quality data when corruption is confined to a subset of nodes. In such networks, uncorrupted nodes have more connections, potentially acting as stabilizing points in the aggregation process and mitigating the impact of corrupted data.

%The presence of well-connected, uncorrupted nodes allows for more effective knowledge aggregation, thereby preserving overall model performance.

% In line with this result, we also noticed another trend. When the effects of low-quality data start to become visible and the performance gets significantly affected, \textit{the best-performing paradigm in the long term result to be federated learning}. Figure~\ref{fig:noniid,corruption_strength.95} shows the fully decentralized solution converging faster but failing to further improve over the communication rounds with respect to a federated solution, able to achieve greater performance from communication round $200$ on. 

The increased robustness of federated learning compared to a fully decentralized approach becomes less noticeable in milder scenarios. By using a corruption strength of $\alpha=0.5$ both presented similar convergence for the balanced and unbalanced scenario as shown in Figures~\ref{fig:iid,corruption_strength.5} and~\ref{fig:noniid,corruption_strength.5}.


\begin{figure*}[!t]
    \centering
    \includegraphics[width=\linewidth]{res/topology-F1-4}
    \caption{Example of a fully decentralized learning system with $\alpha=.95$ and $p=0.9$. Left: the network topology for one of our experiments. The color of the nodes represents the mean \f{4} score computed after the aggregation phase for class $4$ after $500$ communication rounds. The nodes with a red circle are the ones not containing corrupted samples. Right: Mean F1 score computed after the aggregation phase for class $4$ over communication rounds. The red line reports the mean \f{4} score computed across all the nodes at each communication round.}
    \label{fig:topology-F1-4}
\end{figure*}

\begin{figure}[!t]
    \centering
    \includegraphics[width=\linewidth]{res/iid,corruption_strength.5.png}
    \caption{Corruption strength $\alpha=0.5$ with balanced corrupted data: DFL vs FL for varying $p$.}
    \label{fig:iid,corruption_strength.5}
\end{figure}
\begin{figure}[!t]
    \centering
    \includegraphics[width=\linewidth]{res/noniid,corruption_strength.5.png}
    \caption{Corruption strength $\alpha=0.5$ with unbalanced corrupted data: DFL vs FL for varying $p$.}
    \label{fig:noniid,corruption_strength.5}
\end{figure}

To summarize, the following findings emerged from our observations:
\begin{itemize}
    \item The impact of corrupted data is most pronounced in the F1 scores for class 4 and class 9, while overall accuracy is only mildly affected, suggesting that other classes remain largely unaffected. Surprisingly, corrupted 9s that resemble 4s degrade the performance of the decentralized classifier more on class 4 than on class 9.
    \item The same amount of bad data has a significantly greater impact when distributed across multiple nodes rather than concentrated in a single, even highly influential, node.
    \item Decentralized learning under an unbalanced data distribution outperforms its centralized counterpart. While this does not necessarily imply that decentralized learning is inherently more robust, it warrants further investigation.
    % \item We proposed that the performance should increase by increasing the number of edges due to the increased number of nodes having corruption-free neighbors.
    \item Federated learning exhibits greater long-term resilience to corrupted data compared to a fully decentralized approach.
\end{itemize} 
% Finally, we also computed a rich set of features for each of our experiments ranging from topological ones, like the diameter, to data oriented attributes like the total percentage of samples belonging to a specific class in the neighborhood of each node. Using this features we started to explore the relationship between the performance (both class $4$ and $9$ F1 score), network topology and data distribution. We computed the significant correlations at each communication round (p-value less than $0.05$) and tried to fit a LASSO regression model to predict the performance for all the nodes at each round. Our results are still in an early stage and only suggest very simple findings like that the maximum degree of the network and the distance from the source of corruption may influence the robustness of the system. These observations, while promising, warrant further investigation and are left for future work.

\section{Discussion and Conclusion}
n this work, we investigated the impact of corrupted data on decentralized federated learning (DFL) and compared its resilience to federated learning (FL) and centralized learning. Our evaluation was conducted on a realistic communication network modeled as a heavy-tailed Barabási–Albert (BA) graph, with a focus on a standard image classification task. We simulated decentralized environments where some nodes’ local datasets contained low-quality samples, mimicking real-world scenarios in which data degradation arises from sensor noise, faulty data augmentation, or intentional adversarial interference.  
To introduce controlled corruption, we leveraged interpolations in the latent space of a pre-trained GAN, enabling targeted corruption of specific labels with varying intensities. Our experiments examined both balanced and unbalanced corruption distributions, considering scenarios where corrupted data was either spread across multiple nodes or concentrated on a single node, including highly connected hubs. We also explored different corruption levels, ranging from minor perturbations to extreme cases where up to 90\% of class 9 samples were corrupted.  

Our key findings indicate that corruption primarily affects the target and collateral classes while leaving other classes largely unaffected. Interestingly, misclassified target samples that resemble the collateral class degrade performance more on the collateral class than on the target class itself. Furthermore, we observed that the distribution of corruption plays a crucial role: when corrupted data is scattered across multiple nodes, the performance drop is significantly larger than when it is concentrated in a single node, even if that node is highly central. This suggests that having uncorrupted neighbors helps mitigate the negative effects of bad data.  
Additionally, we found that under an unbalanced data distribution, DFL can outperform a naive centralized approach in handling corruption. However, this does not necessarily indicate that decentralized learning is universally more robust; rather, its resilience depends on the corruption pattern and network structure, warranting further investigation. Lastly, while fully decentralized learning initially converges faster, FL exhibits greater long-term robustness, ultimately surpassing DFL in later communication rounds.  

Our findings provide new insights into the resilience of decentralized learning systems to low-quality data, emphasizing the strong interplay between network structure and the distribution of corrupted data. This interplay suggests potential design strategies for peer-to-peer overlays that enhance collaboration among nodes when network constraints allow. Future research should explore alternative aggregation strategies beyond simple averaging, extend the analysis to different network topologies, and evaluate additional datasets.  

Finally, given recent studies on the decentralized training of diffusion models to alleviate the computational burden of training infrastructure~\cite{mcallister2025decentralized}, we argue that a deeper understanding of how unbalanced settings impact performance could inform more effective data distribution strategies. This, in turn, may help mitigate the risks associated with training on unverified data sources, such as those scraped from uncontrolled internet environments.  

\bibliographystyle{IEEEtran}
\bibliography{references}

\end{document}


\title[{Exact Sequence Classification with Hardmax Transformers}]{Exact Sequence Classification with \\ Hardmax Transformers}

\author[A. Alcalde]{Albert Alcalde\thankssymb{2}}
\email{albert.alcalde@fau.de}

\author[G. Fantuzzi]{Giovanni Fantuzzi\thankssymb{2}}
\email{giovanni.fantuzzi@fau.de}

\author[E. Zuazua]{Enrique Zuazua\thankssymb{1}\thankssymb{2}\thankssymb{3}}
\email{enrique.zuazua@fau.de}

\thanks{\thankssymb{2}Chair for Dynamics, Control, Machine Learning, and Numerics (Alexander von Humboldt Professorship), Department of Mathematics, Friedrich--Alexander--Universit\"at Erlangen--N\"urnberg, 91058 Erlangen, Germany.
}

\thanks{\thankssymb{1}Departamento de Matem\'{a}ticas,
Universidad Aut\'{o}noma de Madrid, 28049 Madrid, Spain.
}

\thanks{\thankssymb{3}Chair of Computational Mathematics, Fundaci\'{o}n Deusto. Av. de las Universidades, 24, 48007 Bilbao, Basque Country, Spain.
}

\begin{document}
\date{\today}
\begin{abstract}
    We prove that hardmax attention transformers perfectly classify datasets of $N$ labeled sequences in $\R^d$, $d\geq 2$. Specifically, given $N$ sequences with an arbitrary but finite length in $\R^d$, we construct a transformer with $\mathcal{O}(N)$ blocks and $\mathcal{O}(Nd)$ parameters perfectly classifying this dataset. Our construction achieves the best complexity estimate to date, independent of the length of the sequences, by innovatively alternating feed-forward and self-attention layers and by capitalizing on the clustering effect inherent to the latter. Our novel constructive method also uses low-rank parameter matrices within the attention mechanism, a common practice in real-life transformer implementations. Consequently, our analysis holds twofold significance: it substantially advances the mathematical theory of transformers and it rigorously justifies their exceptional real-world performance in sequence classification tasks.
\end{abstract}
\maketitle
%
\section{Introduction}\label{sec:introduction}
Transformers \cite{vaswaniAttentionAllYou2017} have revolutionized machine learning by outperforming traditional residual networks (ResNets) in applications such as natural language processing (NLP) \cite{openai2024gpt4technicalreport} and computer vision \cite{dosovitskiy2021imageworth16x16words}, where the input data is a finite sequence of $d$-dimensional vectors. Their practical success relies on their universal approximation properties \cite{yunAreTransformersUniversal2020, alberti2023sumformer} and, most importantly, on the effectiveness of the self-attention layers, which act between feed-forward layers to inject information of the sequence to each component.

The approximation power of feed-forward layers---the heart of ResNets since their inception \cite{kaimanHe_2016}---has been studied extensively, especially in the context of classification problems. For example, \citet{geshkovski2022turnpike} and \citet{domenec2023NODES} established \emph{simultaneous controllability} results showing that sufficiently deep fully-connected ResNets can exactly classify inputs in Euclidean spaces of dimension $d\geq 2$. They also provide explicit upper bounds on the number of ResNet parameters and layers as a function of the input dimension $d$. 

Since the parameters of self-attention layers can always be chosen such that they act as the identity map, these perfect classification results carry over to transformers. However, the parameter estimates obtained in this way depend strongly on the sequence length $n$, whereas in practice transformers have good performance independently of $n$. It is therefore necessary to develop new theory specific to transformers that can elucidate the role of their self-attention layers and, consequently, justify their outstanding practical performance. 

In this work, we take a step in this direction by proving that transformers equipped with hardmax self-attention layers can perfectly classify sequences with a significant reduction in the number of parameters compared to ResNets (see \Cref{sec:main-thm} for a quantitative statement). Specifically, through careful explicit choices of the self-attention parameters, we show that the self-attention layers serve as a dimensionality reduction mechanism that allows transformers to classify sequences using a number of parameters independent of the sequence length. Our constructive strategy alternates self-attention with feed-forward layers and works with rank-1 parameter matrices in the self-attention mechanism, justifying why practical implementations that maintain a low-rank factorization in terms of `key' and `query' matrices work well. Our results potentially explain why transformers outperform ResNets in sequence-based classification tasks.
%
\subsection{Related Work}
Mathematical insight into the inner workings of transformers can be gained by looking at self-attention layers through the lens of interacting particle systems \cite{lu2019understanding,sanderSinkformers2022}. Specifically, one can view the elements of an input sequence as particles that interact through a kernel determined by the self-attention mechanism. \citet{geshkovski2023emergence, geshkovski_mathematical_2023} adopted this perspective to prove asymptotic clustering results for simplified attention-only transformers with shared weights. \citet{alcalde2024clustering} proved further clustering results for a hardmax formulation of self-attention, which allows for a more geometric understanding of the attention mechanism.

\citet{geshkovski2024measure} recently exploited the clustering effect of self-attention layers to prove approximate interpolation results for a class of transformers acting as measure-to-measure mappings. In these transformers, a softmax self-attention acts as a particular interaction mechanism for $d$-dimensional particles on the unit sphere $\mathbb{S}^{d-1}$. This formulation uses a continuous-time interpretation of transformers similar to the so-called neural ODE interpretation of ResNets \cite{Weinan2017APO,NODES2018}. The resulting transformer model is a nonlocal transport equation for measures, with the transformer parameters acting as control inputs. Approximate interpolation of measures in the Wasserstein-2 distance is achieved using explicit piecewise-constant control inputs with a number of switches (corresponding to the number of transformer blocks needed in practice) proportional to $Nd$, where $N$ is the number of measures \citep[Theorem~1.1]{geshkovski2024measure}. Moreover, if the input and target measures are atomic with the same number of atoms, then one can perfectly solve the classification task stated in \Cref{pb:textClass} below using $\mathcal{O}(Nd)$ transformer blocks (see \cite[\S1.4.2]{geshkovski2024measure}).
%
\subsection{Novelty of this Work}
The theoretical results proved in this paper are similar in spirit to those of \citet{geshkovski2024measure}, but with two key differences. First, we analyze a different transformer architecture with a discrete sequence of transformer blocks and no need for a continuous-time model (see \Cref{ss:transArch} for details). In particular, our transformer is a sequence-to-sequence map and uses a hardmax self-attention mechanism. Second, and more important, we use a different strategy to choose the transformer parameters, reducing the number of blocks required to achieve perfect sequence classification to $\mathcal{O}(N)$ independently of the dimension $d$ of the sequence elements. 
%
\section{Transformer-Based Classification}
Our goal is to prove that a simple transformer architecture can perfectly solve the supervised learning problem of sequence classification. This section gives a mathematical statement of this problem, describes the transformer architecture we use to solve it, and presents our main result.
%
\subsection{Problem Statement}\label{ss:pb_statement}
Fix $d,M,N \in \N$ with $d\geq 2$. Let $(\R^d)^n$ be the set of sequences in $\R^d$ with $n$ elements. The set of finite-length sequences in $\R^d$ is
\begin{equation}
    \mathcal{Z} = \bigcup_{n \in \N} (\R^d)^n.
\end{equation}
Let $Z^1,\ldots,Z^N \in \mathcal{Z}$ be sequences to be classified, let $\smash{S^1,\ldots,S^M}$ be disjoint subsets of $\R^d$ representing classification categories or \textit{labels}, and let $c : [N] \to [M]$ be a map that assigns a category $\smash{S^{c(j)}}$ to every sequence $Z^j$. (Here and throughout the paper, $[k]$ denotes the set $\{1,\ldots,k\}$.) The problem we study in this paper can be stated as follows.
\begin{problem}\label{pb:textClass}
    Construct a function $\phi : \mathcal{Z} \to \mathbb{R}^d$ such that $\phi(Z^j) \in S^{c(j)}$ for all $j \in [N]$.
\end{problem}
Two classical examples fit the framework of \Cref{pb:textClass}. One is text sentiment analysis, where the positive and negative sentiment classes are represented by disjoint open sets $S^1$ and $S^2$, each sequence $Z^j$ is a collection of sentences, and its elements (also called \emph{tokens}) are words or sub-words encoded as points in $\R^d$. Another example is exact next-token prediction, where each sequence $Z^j$ represents an incomplete sentence and the sets $S^{i}=\{y^{i}\}$ are singletons encoding words that can be used to complete them. Both problems can be solved exactly with the transformers described in \Cref{ss:transArch}. In contrast, the continuous-time measure-theoretic transformers of \citet{geshkovski2024measure} can solve the sentiment analysis problem, but not the exact next-token prediction one unless the target tokens $\{y^{c(j)}\}_{j\in[N]}$ are pairwise distinct (see \cite[\S1.4.2]{geshkovski2024measure}).
%
\subsection{Transformer Architecture}\label{ss:transArch}
We solve \Cref{pb:textClass} by taking $\phi = \mathrm{R} \circ \T$ to be the composition of a transformer $\T: \mathcal{Z} \to \mathcal{Z}$ and a so-called `readout map' $\mathrm{R}: \mathcal{Z} \to \mathbb{R}^d$, which we describe below. We denote by $\len(Z)$ the length of a sequence $Z$.

The readout map is defined as
\begin{equation}\label{eq:readout}
    \mathrm{R}(Z) = \frac{1}{\len(Z)} \sum_{i=1}^{\len(Z)} z_i.
\end{equation}
The transformer is instead a composition of \emph{transformer blocks}, in which a feed-forward (FF) layer with a residual connection is followed by a self-attention (SA) layer with a hardmax attention mechanism that may or may not include a residual connection. We define these layers in detail next.
\vspace{3mm}

\noindent\textit{Feed-Forward Layers.} Given $d'\in \N$, feed-forward layers are functions $\FF : \mathcal{Z} \to \mathcal{Z}$ parametrized by $W \in \R^{d \times d'}$, $U \in \R^{d'\times d}$, $b\in \R^{d'}$, and an activation function $\sigma$ acting element-wise on vectors. The $i$-th component of the sequence $\FF(Z)=\{\FF_1(Z),\ldots,\FF_{\len(Z)}(Z)\}$ is given by
\begin{equation}
\FF_i(Z) = z_i + W\sigma(U z_i + b).
\end{equation}
In this work, we fix $\sigma(x) = \max (0,x)$ (the ReLU function).
\begin{remark}
    ResNets can also be used to solve \Cref{pb:textClass} if all input sequences $Z^j$ have the same length $n$. However, the FF layers in these ResNets act on a sequence $Z$ after `flattening' it to a vector in $\R^{nd}$. Therefore, they require a number of parameters linearly proportional to the sequence length $n$. In contrast, the FF layers used in transformers act independently on each token in the sequence $Z$ and require a number of parameters independent of the sequence length. In fact, they can act on sequences of arbitrary (but finite) length. This complexity reduction is a distinguishing feature of transformers and is key to our analysis.
\end{remark}
\begin{figure}
    \centering
    \includegraphics[width=0.25\linewidth]{figures/hardmax-attention.pdf}
    \caption{Geometric interpretation of \Cref{eq:selfatt_c} for $i=1$ with $A=I$. Tokens $z_2$ and $z_3$ have the largest orthogonal projection on the direction of $Az_1 = z_1$, so $\mathcal{C}_i(Z,A) = \{ 2,3 \}$.}
    \label{fig:hm_att}
\end{figure}
\vspace{3mm}

\noindent\textit{Self-Attention Layers.} 
Self-attention layers are functions $\mathrm{SA} :\mathcal{Z} \to \mathcal{Z}$ parametrized by $\rho \in \R$ and matrices $V, A \in \R^{d \times d}$ as follows. For every token index $i \in \len(Z)$, set 
\begin{subequations}\label{eq:selfatt}
\begin{gather}
\label{eq:selfatt_c}
    \mathcal{C}_i(Z,A) = \left\{ j :\;  \langle A z_i, z_j \rangle = \max_{\ell \in \len(Z)} \langle A z_i, z_\ell \rangle \right\}
    \\
\label{eq:selfatt_b}
\text{and}\quad
    \Lambda_{i\ell}(Z, A) = 
    \begin{cases}
        \frac{1}{\abs{ \mathcal{C}_i(Z,A) }} &\text{if } z_\ell \in \mathcal{C}_i(Z,A),\\
        \;\, 0 &\text{otherwise}.
    \end{cases}
\end{gather}
Then, the sequence $\mathrm{SA}(Z) = \{\mathrm{SA}_1(Z),\dots, \mathrm{SA}_{\len(Z)}(Z)\}$ is determined component-wise using the formula
\begin{equation}
\label{eq:selfatt_a}
    \mathrm{SA}_i(Z) = \rho z_i + V \sum_{\ell = 1}^{\len(Z)} \Lambda_{i\ell}(Z,A)z_\ell.
\end{equation}
\end{subequations}
\begin{remark}
    We choose the hardmax attention mechanism because of its simple geometric interpretation: a token $z_i$ is influenced by the tokens with the largest orthogonal projection onto the direction of $A z_i$ (cf. \Cref{fig:hm_att}). 
    A regularized `softmax' attention is usually preferred in practice because it allows for training with standard gradient-based algorithms, but one expects the two models to behave very similarly when the regularization parameter in the softmax attention is small enough. \citet{alcalde2024clustering} verified this expectation for the classification problem of sentiment analysis.
\end{remark}
\vspace{3mm}

\noindent\textit{The Transformer.}
We are now ready to combine FF layers and SA layers into a transformer. Fix a depth $L\in \N$, FF layers $\{ \FF^k \}_{k=1}^L$ with parameters $\{ W^k, U^k, b^k \}_{k=1}^L$, and SA layers $\{\SA^k \}_{k=1}^L$ with parameters $\{ \rho^k, V^k, A^k \}_{k=1}^L$.
A transformer is a map $\T : \mathcal{Z} \rightarrow \mathcal{Z}$ defined by
\begin{equation}\label{eq:trans_model}
\T = \left( \SA^L \circ \FF^L \right) \circ \dots \circ \left(\SA^1 \circ \FF^1 \right).
\end{equation}
For each $k\in [L]$, we call the function $\TB^k = \SA^k \circ \FF^k$ a transformer block and say that $\T$ in \Cref{eq:trans_model} has $L$ blocks.
%
\subsection{Main Result}\label{sec:main-thm}
Our main result states that \Cref{pb:textClass} can be solved exactly with the transformer architecture described in \Cref{ss:transArch}.
\begin{theorem}\label{thm:mainResult}
    Fix $d,M,N \in \N$ with $d\geq 2$, disjoint sets $\smash{S^1,\ldots,S^M} \subset \R^d$, sequences $Z^1,\ldots,Z^N \in \mathcal{Z}$,  and a map $c : [N] \to [M]$. Assume that:
    \begin{enumerate}[noitemsep, topsep=0pt]
    \item[i)] The sequences $Z^1,\ldots, Z^N$ are pairwise distinct;
    \item[ii)] Tokens within each sequence are pairwise distinct.
    \end{enumerate}
Let $\mathrm{R}$ be the readout map in \Cref{eq:readout}. There exists a transformer $\T$ of the form \Cref{eq:trans_model} with $L \leq 8N + 4$ blocks and $P = \mathcal{O}(Nd)$ parameters such that
\begin{equation}
    \left( \mathrm{R} \circ \mathrm{T} \right)(Z^j) \in S^{c(j)}\quad \text{for all}\quad j \in [N].
\end{equation}
\end{theorem}
\begin{remark}
    Assumption \textit{i)} is very mild. Indeed, if $Z^i=Z^j$ and $c(i) = c(j)$, then one of the sequences can be dropped without changing \Cref{pb:textClass}. If instead, $Z^i=Z^j$ and $c(i) \neq c(j)$, then \Cref{pb:textClass} trivially has no solution.
\end{remark}
\begin{remark}
    Assumption \textit{ii)} is mild since positional encoding \cite{press2022trainshorttestlong} usually ensures that tokens in a sequence are distinct. However, it can also be removed with relatively straightforward modifications to the proof, but more cumbersome notation. Indeed, repeated tokens in an input sequence $Z^j$ can be replaced by a single token with more `mass'. This can be handled by replacing the constant $1/|\mathcal{C}_i(Z,A)|$ in \Cref{eq:selfatt_b} with 
    \begin{equation}
    \frac{w_\ell}{\sum_{k \in \mathcal{C}_i(Z,A)} w_k},
    \end{equation}
    where $w_\ell$ counts the number of times token $z_\ell$ is repeated.
\end{remark}
\begin{remark}
    Our transformer has $\mathcal{O}(Nd)$ parameters, rather than the $\mathcal{O}(Nd^2)$ one might expect at first, because in the self-attention layers we always choose $V$ as a multiple of the identity and $A = vv^\top$ for some $v\in \R^d$. Moreover, the feed-forward layers use $d'= 1$ as hidden dimension, so $W$ and $U$ are always described by vectors in $\R^d$.
\end{remark}
\Cref{thm:mainResult} highlights the efficiency of transformers for sequence classification since this is achieved perfectly with a number of parameters independent of sequence length, which is arbitrary in our setting. This is a key advantage compared to ResNets, which require sequences of fixed length $n$ and simply view them as vectors in $\mathbb{R}^{nd}$. Indeed, analysis by \citet{domenec2023NODES} shows that ResNets can perfectly classify sequences, but require $\mathcal{O}(nNd)$ parameters. 
Our analysis, therefore, justifies why transformers outperform ResNets in sequence classification tasks.
%
\section{Proof of \Cref{thm:mainResult}}
We now prove \Cref{thm:mainResult} by prescribing explicit transformer parameters. The construction is technical but follows an intuitive strategy that illuminates how transformers work. To better highlight it, we outline only the main steps of the proof in \Cref{sec:proof-outline} and postpone all technicalities to~\Cref{sec:lemmas}.
\begin{figure}
    \centering
    \includegraphics[width=1\linewidth]{figures/sketch-thm.pdf}
    \caption{Proof strategy of \Cref{thm:mainResult} applied to $N=3$ sequences (denoted with circles, squares and stars) in $\R^2$, with $M=2$ different categories (blue and orange). The initial sequences' tokens are shown in panel $(a)$. Panels $(b)$--$(f)$ are obtained by applying a transformer block to the previous panel. Red, pink, blue and green arrows correspond to the separation, clustering, matching labels and classification steps, respectively. In panels $(e)$ and $(f)$, the blue circle and blue square overlap.}
    \label{fig:thm_sketch}
\end{figure}

%
\subsection{Proof Strategy}\label{sec:proof-outline}
Following \citet{geshkovski2024measure}, our proof proceeds in steps, illustrated in \Cref{fig:thm_sketch} for a simple sequence classification problem with $N=3$ input sequences in $\R^2$ and $M=2$ categories represented by the upper and lower half-planes. The first step uses $\mathcal{O}(N)$ transformer blocks to separate overlapping tokens from different input sequences (see panels \textit{(b)} and \textit{(c)} in \Cref{fig:thm_sketch}). This is the most delicate step and requires alternation of feed-forward layers and self-attention layers. In the second step (\Cref{fig:thm_sketch}\textit{(d)}) a single transformer block collapses the sequences into a single token using the clustering properties of self-attention layers. The third step (\Cref{fig:thm_sketch}\textit{(e)}) drives sequences with the same labels to the same point using $\mathcal{O}(N)$ blocks. In the final step, the collapsed sequences are driven to their target set (\Cref{fig:thm_sketch}\textit{(f)}). This requires at most $\mathcal{O}(N)$ transformer blocks thanks to existing classification results for ResNets~\cite{domenec2023NODES}.
We now make these steps rigorous to prove \Cref{thm:mainResult}, but still relegate all technical constructions to~\Cref{sec:lemmas} for clarity.
%
\begin{proof}[Proof of \Cref{thm:mainResult}]
We combine transformers $\T_{\text{cla}}$, $\T_{\text{lab}}$, $\T_{\text{clu}}$, $\T_{\text{sep}}$, $\T_{\text{pre}}$ of the form \Cref{eq:trans_model} to preprocess, separate, cluster, match up, and classify the input sequences. The final transformer is 
\begin{equation}
\T = \T_{\text{cla}} \circ \T_{\text{lab}} \circ \T_{\text{clu}} \circ \T_{\text{sep}} \circ \T_{\text{pre}}.
\end{equation}

\noindent\textit{Preprocessing.} The transformer $\T_{\text{pre}}$ has a single block ensuring that $\T_{\text{pre}}(z_\ell^j) \neq 0$ for all $\ell \in [\len (Z^j)]$ and all $j\in [N]$. This is achieved by setting $U = 0$, $b=1$ and $W = w \in \R^d$ any non-zero vector in the FF layer, and $\rho = 1$, $V = 0$, and $A=0$ in the self-attention layer. With these choices, the FF layer shifts tokens by $w$ and the self-attention layer acts as the identity. No tokens in $\T_{\text{pre}}(Z^1),\ldots,\T_{\text{pre}}(Z^N)$ are zero if $\| w\|$ is sufficiently small, which is easily ensured.
\vspace{3mm}

\noindent\textit{Separation.} The transformer $\T_{\text{sep}}$ has $2(N-1)$ blocks and is built using technical steps in \Cref{lem:splitOverlappingSequences} to ensure that 
\begin{equation}
\T_{\text{sep}}(\T_{\text{pre}}(Z^j)) \cap \T_{\text{sep}}(\T_{\text{pre}}(Z^{j'})) = \emptyset \quad \text{for all} \quad j\neq j'.
\end{equation}

\noindent\textit{Clustering.} The transformer $\T_{\text{clu}}$ has a single block that collapses each of the sequences $\T_{\text{sep}}(\T_{\text{pre}}(Z^j))$. This block has a FF layer constructed as in \Cref{lem:chooseLeader} and a SA layer with $\rho = 0$, $V = I$ and $A=v v^\top$, where $v \in \R^d$ is also given by \Cref{lem:chooseLeader}. Then, for every $j\in [N]$, the sequence $\T_{\text{clu}}(\T_{\text{sep}}(\T_{\text{pre}}(Z^j)))$ has a single token $\T_{\text{sep}}(\T_{\text{pre}}(z_{i_j}^j))$ repeated $\len(Z^j)$ times. The index $i_j$ is given by \Cref{lem:chooseLeader}. In what follows, we identify the sequence $\T_{\textrm{clu}}(\T_{\text{sep}}(\T_{\text{pre}}(Z^j)))$ with the token $\Tilde{z}^j \coloneqq \T_{\text{sep}}(\T_{\text{pre}}(z_{i_j}^j))$.
\vspace{3mm}

\noindent\textit{Matching Labels.} We now build a transformer $\T_{\text{lab}}$ with $N-1$ blocks such that $\T_{\text{lab}}(\Tilde{z}^i) = \T_{\text{lab}}(\Tilde{z}^j)$ if $S^{c(i)}=S^{c(j)}$. The blocks have self-attention layers acting as the identity ($\rho = 1$, $V = 0$, $A = 0$) and $\FF$ layers built as follows. Fix distinct $i_0, i_1 \in [N]$ such that $\Tilde{z}^{i_0}, \Tilde{z}^{i_1}\in \ext \co (\Tilde{z}^1,\dots,\Tilde{z}^N)$. By the hyperplane separation theorem, for each $\ell\in\{0,1\}$ there exist $u_\ell\in\R^d$ and $b_\ell\in \R$ such that
\begin{equation}
\begin{cases} 
   \ip{u_\ell}{\Tilde{z}^{i_\ell}} + b_\ell >0, \\ 
   \ip{u_\ell}{\Tilde{z}^j} + b_\ell <0  &\text{if} \quad j \neq i_\ell.  
\end{cases}
\end{equation}
To choose the parameters $U$, $b$ and $W$ of the first FF layer of $\T_{\textrm{lab}}$ we distinguish two cases depending on whether there are other input sequences $Z^j$ with the same label $S^{c(i_1)}$.

\textit{Case 1:} $S^{c(j)} \neq S^{c(i_1)}$  $\forall j \in [N] \setminus\{i_1\}$. Then, we set 
\begin{equation}
    U = u_1,\quad b = b_1, \quad \text{and} \quad W = (\langle u_1, \Tilde{z}^{i_1} \rangle + b_1)^{-1}(y - \Tilde{z}^{i_1}),
\end{equation} where $y\neq \Tilde{z}^{i_0}$ is any point such that $\ip{u_0}{y} + b_0 > 0$. This gives $\ip{u_0}{\FF (\Tilde{z}^{i_1})} + b_0 > 0$ and $\FF (\Tilde{z}^{i_1}) \neq \Tilde{z}^{i_0}$.

\textit{Case 2:} $\exists j\in[N] \setminus\{i_1\}$ such that $S^{c(j)} = S^{c(i_1)}$. Then, we set 
\begin{equation}
    U = u_1, \quad b = b_1, \quad \text{and} \quad W = (\ip{u_1}{\Tilde{z}^{i_1}} + b_1)^{-1}(\Tilde{z}^j - \Tilde{z}^{i_1}).
\end{equation}
This gives $\FF (\Tilde{z}^{i_1}) = \Tilde{z}^j$.

The FF layers for blocks $k\in\{2,\ldots,N-1\}$ are constructed similarly. Precisely, with a small abuse of notation, let $\Tilde{z}^{1},\ldots,\Tilde{z}^{N}$ now be the tokens returned by block $k-1$. One picks $i_k \in [N]\setminus\{i_0,\ldots,i_{k-1}\}$ such that $\Tilde{z}^{i_k} \in \ext\co(\Tilde{z}^1,\dots,\Tilde{z}^N)$ satisfies $\ip{u_0}{\Tilde{z}^{i_k}} + b_0 < 0$, and sets the FF parameters according to cases~1 and~2 above using $u_k \in \R^d$ and $b_k \in \R$ such that the hyperplane $\ip{u_k}{z}+b_k=0$ separates $\Tilde{z}^{i_k}$ from the other tokens.
\vspace{3mm}

\noindent\textit{Classification.} 
For every $j \in [N]$, the sequence $(\T_{\text{lab}}\circ\T_{\text{clu}}\circ \T_{\text{sep}}\circ \T_{\text{pre}})(Z^j)$ has collapsed to one of $M$ distinct points $x^1,\ldots,x^m \in \R^d$. We now only need to map each point $x^{c(j)}$ to its corresponding label $\{ S^{c(j)}\}$. We do this with a transformer $\T_{\text{cla}}$ with at most $5(N+1)$ in which the self-attention layers act as the identity ($\rho = 1$, $V = 0$, $A = 0$) and the FF layers are tuned as described by \cite[Theorem~4.1]{domenec2023NODES}. These layers have inputs in $\R^d$ and hidden dimension $d' = 1$, giving $\mathcal{O}(d)$ parameters per transformer block.
The resulting transformer satisfies 
\begin{equation}
    (\T_{\text{cla}} \circ \T_{\text{lab}}\circ \T_{\text{clu}} \circ \T_{\text{sep}} \circ \T_{\text{pre}})(Z^j) \in S^{c(j)} \quad \text{for all} \quad j\in [N].
\end{equation}

\noindent\textit{Complexity of the Construction.}
Finally, we count the number of blocks and parameters in our transformer. It is clear that at most $1 + 2(N-1) + 1 + N-1 + 5(N+1) = 8N + 4$ blocks are needed. Moreover, the parameters in each block have been chosen as vectors in $\R^d$ or constants, giving $\mathcal{O}(d)$ parameters per block. The total number of non-zero parameters is therefore $\mathcal{O}(Nd)$.
\end{proof}
%
\subsection{Technical Lemmas}\label{sec:lemmas}
The proof of \Cref{thm:mainResult} requires understanding how different parameter choices in a transformer block affect tokens. We start by asking if a self-attention layer can leave a specific token $z_p^j$ from sequence $Z^j$ unchanged. This helps us determine which tokens can become clusters for their sequence in the clustering step of our proof. 

It is not hard to see that $z_p^j$ will not be changed by a self-attention layer if $\mathcal{C}_p(Z,A) = \{ p \}$. In such a case, we say that token $z_p\in Z$ is a \textit{leader}. The following lemma asserts that a token can be a leader if and only if it is extreme for the convex hull of the tokens in the sequence $Z$.
\begin{lemma}\label{lem:howToLeader}
   Let $Z$ be a sequence of non-zero tokens. There exists $A\in \R^{d\times d}$ such that $z_p \in Z$ is a leader if and only if $z_p \in \ext \co (Z)$.
\end{lemma}
\begin{proof}
Suppose $A\in \R^{d\times d}$ is such that $z_p$ is a leader. Then,
\begin{equation}
    \langle A z_p, z_r \rangle < \langle A z_p, z_p\rangle \quad \forall r\neq p.
\end{equation}
It is easy to verify that this implies that $z_p$ is the unique maximizer of the linear function $z \mapsto \langle Az_p, z\rangle$ over $\co(Z)$. Since linear functions on convex sets are always maximized by at least one extreme point of the set, we conclude that $z_p \in \ext \co (Z)$. This proves the `only if' part of the lemma.

For the reverse implication, we fix $z_p\in\ext\co (Z)$ and explicitly construct $A$ for which $z_p$ is a leader. Since $z_p\in\ext\co (Z)$, then $z_p\notin \co (Z \setminus \{z_p \})$, which is a closed and convex set. By the hyperplane separation theorem, there exists a non-zero vector $v\in \R^d$ and a constant $\alpha\in \R$ such that
\begin{equation}
\begin{cases} 
   \ip{v}{z_p} > \alpha, \\ 
   \ip{v}{z_\ell} < \alpha  &\forall \ell\neq p. 
\end{cases}
\end{equation}
Since the maps $v \mapsto \ip{v}{z_\ell}$ are continuous for all $\ell \in [\len (Z)]$, we can assume without loss of generality that $\ip{v}{z_p} \neq 0$. Then, if $\ip{v}{z_p} > 0$ we set $A = vv^\top$ and verify that
\begin{equation}
    \begin{aligned}
    \langle A z_p, z_p\rangle &= \ip{v}{z_p} \ip{v}{z_p} \\
    &> \ip{v}{z_p}\ip{v}{z_\ell} \\
    &= \langle Az_p, z_\ell\rangle \quad \forall \ell \neq p,
    \end{aligned}
\end{equation}
because $\ip{v}{z_p} > \alpha > \ip{v}{z_\ell}$. This yields $\mathcal{C}_p(Z,A) = \{ p \}$, as desired. If, instead, $\ip{v}{z_p} < 0$, then we set $A = -vv^\top$ and obtain
\begin{equation}\label{eq:lemleader}
\begin{aligned}
    \langle A z_p, z_p\rangle &= -\ip{v}{z_p} \ip{v}{z_p} \\
    &> -\ip{v}{z_p}\ip{v}{z_\ell} \\
    &= \langle Az_p, z_\ell\rangle \quad \forall \ell \neq p. 
\end{aligned}
\end{equation}
This yields $\mathcal{C}_p(Z,A) = \{ p \}$ and finishes the proof.
\end{proof}
The next two lemmas ensure that all tokens in a sequence $Z^j$ can be attracted by a single token in that sequence. This is a crucial result for the clustering step in the proof of \Cref{thm:mainResult}, since it ensures that the self-attention layer can collapse all sequences into a single token. Moreover, the lemmas allow prescribing the attracting token $z_p^q$ in a sequence $Z^q$. This is used in the separation step, for which the result is iterated to separate all sequences one by one.
\begin{lemma}\label{lem:auxLemmaSingleLeader}
Given sequences $Z^1, \dots, Z^N$ of non-zero tokens, fix $q\in [N]$ and $p\in [\len (Z^q)]$ such that $z_p^q \in \ext\co(Z^q)$. There exists $v\in \R^d$ such that:
\begin{enumerate}[topsep=0pt,itemsep=0pt]
    \item[i)] $\ip{v}{z_p^q} \neq 0$.
    \item[ii)] For every $j \in [N]$, there exists $i_j\in [\len (Z^j)]$ such that $\ip{v}{z_\ell^j} < \ip{v}{z_{i_j}^j}$ $\forall \ell \in [\len (Z^j)].$
    \item[iii)] $i_q = p$.
\end{enumerate}
\end{lemma}
\begin{proof}
We assume without loss of generality that $q = 1$, and proceed by induction on $N$. For $N=1$, a vector $v_1\in \R^d$ such that conditions \textit{i)--iii)} hold can be chosen using the same arguments in \Cref{lem:howToLeader}. 

Next, we assume that $v_{N-1}\in \R^d$ is such that \textit{i)--iii)} hold for the first $N-1$ sequences, and construct $v_{N}\in \R^d$ such that they also hold when the last sequence $Z^N$ is considered. Since only condition \textit{ii)} can fail when $Z^N$ is added to the problem, we need only consider two cases.

\textit{Case 1: Condition \textit{ii)} holds also for $j=N$.} In this case, we can simply set $v_{N} = v_{N-1}$.

\textit{Case 2: Condition \textit{ii)} does not hold for $j=N$.} In this case, the quantity $\ip{v}{z_\ell^{N}}$ is maximized at $s>1$ tokens, which we may take to be $z_1^{N}, \dots, z_s^{N}$ without loss of generality after reordering the sequence if necessary. We may also assume without loss of generality that $z_1^{N}$ is extreme for $\co(z_1^{N}, \dots, z_s^{N})$. Then, by the hyperplane separation theorem, there exists $u\in\R^d$ such that 
\begin{equation}
    \ip{u}{z_r^{N}} < \ip{u}{z_1^{N}} \quad \text{for all} \quad r \in \{2,\dots,s\}.
\end{equation}
Set $v_{N} = v_{N-1} + \varepsilon u$ for some $\varepsilon>0$ to be specified below. Then, we obtain
\begin{equation}
\begin{aligned}
    \ip{v_{N}}{z_r^{N}} &= \ip{v_{N-1}}{z_r^{N}} + \varepsilon \ip{u}{z_r^{N}}  \\
    &< \ip{v_{N-1}}{z_1^{N}} + \varepsilon \ip{u}{z_1^{N}}  \\
    &= \ip{v_{N}}{z_1^{N}}
\end{aligned}
\end{equation}
for every $r \in \{2,\dots, s\}$. We now fix $\varepsilon$ small enough that $\ip{v_{N}}{z_r^{N}} < \ip{v_{N}}{z_1^{N}}$ also for $r\in\{s+1,\ldots,\len(Z^N)\}$ and that conditions \textit{i)--iii)} remain true for sequence indices $j\in[N-1]$. This is possible because the maps $v\mapsto \ip{v}{z}$ are continuous. \qedhere
\end{proof}
\begin{lemma}\label{lem:chooseLeader}
Given sequences $Z^1, \dots, Z^N$ of non-zero tokens, fix $q\in [N]$ and $p\in [\len (Z^q)]$ such that $z_p^q \in \ext\co(Z^q)$. There exist parameters $\{ U, W, b\}$ of a feed-forward layer $\FF$, a vector $v\in \R^d$, and indices $\{i_j\}_{j\in [N]}$ with $i_q = p$ such that, for all $\{\ell, j\}\in [\len (Z^j)]\times [N]$,
\begin{equation}\label{eq:chooseLeaderLemma}
   \FF(z_\ell^j) \neq 0 
   \quad\text{and}\quad
    \C_\ell(\FF(Z^j), vv^\top) = \{ i_j \}.
\end{equation}
\end{lemma}
\begin{proof}
Choose $v\in\R^d$ as in \Cref{lem:auxLemmaSingleLeader}. We will choose the parameters of the FF layer such that \Cref{eq:chooseLeaderLemma} holds. For this, we first observe that adding a constant vector to every token of every sequence does not change properties \textit{ii)--iii)} of \Cref{lem:auxLemmaSingleLeader}. We then set the parameters of $\FF$ as $W = v / \| v \|^2$, $U = 0$ and 
\begin{equation}
    b = \max_{k\in [N]} \max_{r\in [\len (Z^k)]}| \ip{v}{z_r^k}| + \varepsilon,
\end{equation} 
where $\varepsilon > 0$. For this choice, $\sigma(Uz + b) = \sigma (b) = b$, so that $\FF(z_\ell^j) = z_\ell^j + b/ {\| v \|^2} v$. It is clear that, for all $\varepsilon$ sufficiently small, $\FF(z_\ell^j) \neq 0$ for all $\{\ell, j\}\in [\len (Z^j)]\times [N]$, giving the first condition in \Cref{eq:chooseLeaderLemma}. For the second one, note that
\begin{align}\label{eq:lemPosTerms}
    \ip{v}{\FF({z}_\ell^j)} &= \ip{v}{z_\ell^j} +  \frac{\ip{v}{v}}{\|v\|^2} b \geq \varepsilon > 0
\end{align}
for all $\{ \ell, j\} \in [\len (Z^j)] \times [N]$. With this inequality we find that $i\in \C_\ell (\FF(Z^j), vv^\top)$ if and only if
\begin{equation}
\begin{aligned}
    i &\overset{\phantom{\text{by \Cref{eq:lemPosTerms}}}}{\in}\argmax_{r \in [\len (Z^j)]} \ip{v}{\FF(z^j_\ell)} \ip{v}{\FF(z_r^j)} \\
    &\overset{\text{by \Cref{eq:lemPosTerms}}}{=} \argmax_{r \in [\len (Z^j)]}  \ip{v}{\FF(z_r^j)}.
\end{aligned}
\end{equation}
This fact, combined with property \textit{ii)} of \Cref{lem:auxLemmaSingleLeader} implies that $\C_\ell(\FF(Z^j), vv^\top) = \{ i_j \}$ for all $\ell \in [\len (Z^j)]$ and all $j\in [N]$. Finally, $i_q = p$ by property \textit{iii)} of \Cref{lem:auxLemmaSingleLeader}. 
\end{proof}
We now give results related to the separation step in the proof of \Cref{thm:mainResult}. First, we show that a FF layer can split the centers of mass of two sequences. We denote the center of mass of a sequence $Z^j$ and of $\FF(Z^j)$ by
\begin{subequations}
    \begin{align}
    m^{j} &:= \frac{1}{\len (Z^j)}\sum_{\ell=1}^{\len (Z^j)} z_\ell^j,\\
    \Tilde{m}^j &:= \frac{1}{\len (Z^j)}\sum_{\ell=1}^{\len (Z^j)} \FF(z_\ell^j).
\end{align}
\end{subequations}
\begin{lemma}\label{lem:splitCentersOfMass}
Fix $j, j'\in [N]$, $j\neq j'$. There exist parameters $\{ U, W, b\}$ of a feed-forward layer $\FF$ such that $\Tilde{m}^j \neq \Tilde{m}^{j'}$. Moreover, for any token $z_i^j\in Z^j$, $\FF(z_i^j)\in \ext \co (\FF(Z^j))$ if and only if $z_i^j\in \ext \co (Z^j)$. 
\end{lemma}
\begin{proof}
We first show the result for the case in which the centers of mass are different, that is, $m^j \neq m^{j'}$. Then, we set $W = 0$, $U=0$, $b=0$ to obtain $\FF(Z) = Z$ and thus $\Tilde{m}^j \neq \Tilde{m}^{j'}$.

We now deal with the more delicate case $m^j = m^{j'}$. Possibly after relabeling, denote by $Z^j \cap Z^{j'} = \{ z_1^j, \dots, z_p^j\} = \{ z_1^{j'}, \dots, z_p^{j'}\}$ for some $p \leq \min (\len (Z^j), \len (Z^{j'}))$ the set of tokens appearing in both sequences. Then, possibly after relabeling, there exists $J \in \{ j, j'\}$ such that $z_{p+1}^J \in \ext \co (\mathcal{S})$ where $\mathcal{S} = (Z^j \cup Z^{j'}) \setminus (Z^j \cap Z^{j'})$. Assume without loss of generality that $J = j$ and note that the set $\mathcal{S} \setminus \{ z_{p+1}^j\}$ cannot be empty since we assumed that $m^j = m^{j'}$. Note also that $\{ z_{p+1}^j \}$ is disjoint from the closed convex set $\co(\mathcal{S} \setminus \{ z_{p+1}^j\})$, so there exists $u\in \R^d$ and $\beta \in \R$ such that
\begin{equation}\label{e:hyperplane}
\begin{cases} 
    \ip{u}{z_{p+1}^j} + \beta > 0, \\ 
    \ip{u}{z} + \beta < 0  &\forall z\in \co (\mathcal{S}\setminus \{ z_{p+1}^j \}).  
\end{cases}
\end{equation}
Now, let the parameters in $\FF$ be $W = w \in \R^{d\times 1}$, for $w \neq 0$, $U = u^\top \in \R^{1 \times d}$ and $b = \beta \in \R$. By \Cref{e:hyperplane}, this means that $\sigma (\ip{u}{z_\ell^{j}} + b) = \sigma (\ip{u}{z_\ell^{j'}} + b) = 0$ for all $\ell > p+1$. Then, recalling that $m^j = m^{j'}$ by assumption and that $\sigma(\ip{u}{z_\ell^j} + b) = \sigma(\ip{u}{z_\ell^{j'}} + b)$ for all $\ell\in [p]$ by construction, we can calculate
\begin{equation}
\begin{aligned}
    \Tilde{m}^j - \Tilde{m}^{j'} &= w\frac{1}{n} \Big[ \sum_{\ell = 1}^p \sigma(\ip{u}{z_\ell^j} + b) - \sum_{\ell = 1}^p \sigma(\ip{u}{z_\ell^{j'}} + b)
    + \ip{u}{z_{p+1}^j} + b \Big] \\
    &= w\frac{1}{n} \Big[\ip{u}{z_{p+1}^j} + b\Big].
\end{aligned}
\end{equation}
The last term is positive by \Cref{e:hyperplane}, giving $\Tilde{m}^j \neq \Tilde{m}^{j'}$. Moreover, $\|w \|$ can be chosen small enough that $\FF(z_i^j) \in \ext\co(\FF(Z^j))$ if and only if $z_i^j \in \ext \co (Z^j)$.
\end{proof}
Next, we ensure that a family of $N$ sequences can be made pairwise disjoint through the action of suitably chosen transformer blocks. This is the key construction that enables the separation step in the proof of \Cref{thm:mainResult}.

\begin{lemma}\label{lem:splitOverlappingSequences}
Given sequences $Z^1,\ldots,Z^N$ with nonzero tokens, there exists a transformer $\T$ with $2 (N-1)$ blocks such that $\T({Z}^j) \cap \T({Z}^{j'}) = \emptyset$ for all $j\neq j'\in [N]$.
\end{lemma}
\begin{proof}
We prove the result constructively by induction on~$N$. Throughout the proof, $B_\delta(x)$ denotes the open ball of radius $\delta > 0$ centered at $x\in \R^d$.
\vspace{3mm}

\noindent{$\bm{N = 2}$.} By relabeling tokens if necessary, we may assume without loss of generality that $z_1^1 \in \ext\co(Z^1)$. We then consider three cases, which can be handled using at most two transformer blocks.

\textit{Case 1: $z_1^1 \cap Z^2 = \emptyset$ or $z_1^1 = z_1^2$ for $z_1^2 \notin \ext\co (Z^2)$.} In this case, the extreme token $z_1^1 \in Z^1$ does not coincide with any extreme token for $Z^2$, which allows for sequence separation using a single transformer block. For the FF layer of this block, we apply \Cref{lem:chooseLeader} with $p = q = 1$ to construct a feed-forward layer $\FF$ and a vector $v_1\in \R^d$ such that
\begin{subequations}
    \begin{align}
    \C_{\ell} (\FF (Z^1), v_1 v_1^\top) = \{ 1 \} \quad \forall \ell \in [\len (Z^1)],
    \\ 
    \C_r (\FF (Z^2), v_1 v_1^\top) = \{ i_2 \} \quad \forall r \in [\len (Z^2)].
\end{align}
\end{subequations}
Note that $z_{i_2}^2\in \ext \co (Z^2)$ by  \Cref{lem:howToLeader}, so $z_{i_2}^2\neq z_1^1$ by the assumption defining our Case 1. For the self-attention layer, instead, we take $\alpha > 0$ to be specified precisely below and set $\rho = 1 -\alpha$, $V = \alpha I$ and $A=vv^\top$ to obtain
\begin{subequations}
    \begin{align}
    \SA_{\ell}(\FF (Z^1)) &=  {z}_{\ell}^1 + \alpha ({z}_1^1 -{z}_{\ell}^1)  + b\frac{v_1}{\| v_1 \|^2}, \\ 
    \SA_{r}(\FF (Z^2)) &= {z}_r^2 + \alpha ({z}_{i_2}^2 - {z}_r^2) + b\frac{v_1}{\| v_1 \|^2}.    
\end{align}
\end{subequations}
(The last term in these two equations comes from the shift introduced by the FF layer.) These quantities are different for all $\{\ell, r\} \in [\len (Z^1)] \times [\len (Z^2)]$ if and only if
\begin{equation}\label{eq:lemConditionOnAlpha}
(1 - \alpha) (z_\ell^1 - z_r^2) + \alpha (z_1^1 - z_{i_2}^2) \neq 0.
\end{equation}
This condition fails if and only if there exists $c_{\ell r} \in \R$ such that $z_\ell^1 - z_r^2 = c_{\ell r} (z_1^1 - z_{i_2}^2)$ 
and $\alpha$ is the unique solution to
\begin{equation}
    \frac{\alpha}{\alpha - 1} = c_{\ell r}.
\end{equation}
This means there are at most finitely many choices of $\alpha$ for which \eqref{eq:lemConditionOnAlpha} fails, so it suffices to pick $\alpha$ not from this set. \Cref{fig:case1} illustrates the construction with a simple example.
\begin{figure}
    \centering
    \includegraphics[width=0.85\linewidth]{figures/case1.pdf}
    \caption{Illustration of Case 1 in \Cref{lem:splitOverlappingSequences} for $N=2$ sequences in $\R^2$. The initial tokens in $Z^1$ (stars) and $Z^2$ (circles) are shown in panel $(a)$. In panel $(b)$, a $\FF$ layer has shifted the tokens in the direction of $v\in \R^2$ such that, in the following self-attention layer, the token in each sequence are attracted by a single leader (marked by a red arrow). In panel $(c)$, the sequences have been split by a $\SA$ layer.}
    \label{fig:case1}
\end{figure}

\textit{Case 2: $z_1^1 \in \ext\co (Z^2)$ and $m^1 \neq m^2$.} 
In this case, the extreme token $z_1^1 \in Z^1$ coincides with an extreme token for $Z^2$, which we take to be $z_1^2$ without loss of generality. Nevertheless, we can separate the sequences using their distinct centers of mass through two transformer blocks: one to reduce the problem to Case~1 (cf. \Cref{fig:case2}), and one to handle that case. 

In the first block, we take a FF layer acting as the identity ($W=0$, $U=0$, $b=0$). For the self-attention layer, we fix a constant $\beta > 0$ to be specified later and a nonzero vector $v_2\in \R^d$ such that $\ip{v_2}{z_1^1} = 0 = \ip{v_2}{z_1^2}$. We then set $A = v_2 v_2^\top$, $\rho = 1-\beta$ and $V = \beta I$. This gives
\begin{subequations}
    \begin{align}
    \mathrm{SA}_1(Z^1) &= z_1^1 +  \beta (m^1 - z_1^1), \\ 
    \label{eq:SA_case2}
    \mathrm{SA}_1(Z^2) &= z_1^2 + \beta (m^2 - z_1^2).
\end{align}
\end{subequations}
Since $m^{1} \neq m^{2}$, $\SA_1(Z^1) \neq \SA_1(Z^2)$ for any $\beta>0$. We now pick $\beta$ to ensure that 
\begin{equation}
   \SA_1(Z^1) \in \ext\co (\SA(Z^1)) \quad \text{and} \quad \SA_1(Z^1) \cap \SA (Z^2) = \emptyset, 
\end{equation}
reducing Case~2 to Case~1 as desired.
\begin{figure}
    \centering
    \includegraphics[width=0.5\linewidth]{figures/case2.pdf}
    \caption{Illustration of Case 2 in \Cref{lem:splitOverlappingSequences} for $N=2$ sequences in $\R^2$. Panel $(a)$ shows the sequences' initial tokens and centers of mass. In panel $(b)$, a $\SA$ layer separates $z_1^1\in Z^1$ from $z_1^2\in Z^2$.}
    \label{fig:case2}
\end{figure}
For that, notice that the right-hand side of \Cref{eq:SA_case2} is a perturbation of the identity and is continuous in $\beta$. The same is true for the expressions for $\mathrm{SA}_\ell(Z^2)$, $\ell \in [\len (Z^2)]$. Therefore, for all $\delta > 0$, there exists $\beta > 0$ such that 
\begin{equation}
    \SA_\ell (Z^2) \in B_\delta (z_\ell^2) \quad \text{for all} \quad \ell \in [\len (Z^2)]. 
\end{equation}
We can then take $\delta > 0$ small enough such that the balls $\{B_\delta(z_\ell^2)\}_{\ell\in [\len (Z^2)]}$ are disjoint and $\SA_1(Z^1) \cap \SA (Z^2) = \emptyset$. Similar continuity arguments and the assumption that $z_1^1 \in \ext \co (Z^1)$ show that $\SA_1(Z^1) \in \ext\co (\SA(Z^1))$.
\begin{figure}
    \centering
    \includegraphics[width=0.5\linewidth]{figures/case3.pdf}
    \caption{Illustration of Case 3 in \Cref{lem:splitOverlappingSequences} for $N=2$ sequences in $\R^2$. Panel $(a)$ shows the sequences' initial tokens and centers of mass. In panel $(b)$, a $\FF$ layer separates $m^1$ from $m^2$.}
    \label{fig:case3}
\end{figure}

\textit{Case 3: $z_1^1 \in \ext\co (Z^2)$ and $m^1 = m^2$.} 
In this case, the extreme token $z_1^1 \in Z^1$ coincides with an extreme token for $Z^2$ and the sequences' centers of mass also coincide. We handle this using two transformer blocks, one to reduce the problem to Case 1, and one as described in that case.

To construct the FF layer of the first block, we apply \Cref{lem:splitCentersOfMass} to construct feed-forward layer $\FF$ such that $\Tilde{m}^1 \neq \Tilde{m}^2$ (cf. \Cref{fig:case3}). After this transformation, which ensures that $\FF (z_1^1) \in \ext \co (\FF(Z^1))$, the problem reduces to Case~2 if $\FF (z_1^1) \in \ext \co (\FF (Z^2))$, and to Case~1 if not. In the former case, we take the self-attention layer of the first transformer block as described in Case~2. Otherwise, we use a self-attention layer $\SA$ with $\rho = 1$, $V = 0$ and $A=0$ to have $\SA(Z) = Z$. Either way, we obtain a transformer block that reduces Case~3 to Case~1, as desired. 
\vspace{3mm}

\noindent\textbf{Induction Step.} Assume there exists a transformer $\T_{N-1}$ with at most $2(N-2)$ blocks such that 
\begin{equation}
    \T_{N-1}(Z^j) \cap \T_{N-1}(Z^{j'}) = \emptyset \quad \forall j\neq j'\in [N-1].
\end{equation}
Set $\Hat{Z} \coloneqq \T_{N-1}(Z)$ to ease the notation. Fix $\Hat{z}_i^N \in \ext \co (\Hat{Z}^N)$, assuming without loss of generality that $i=1$.

If $\Hat{z}_1^N \cap \Hat{Z}^j = \emptyset$ for all $j\in [N]$, then we apply the same argument in Case 1 above, but now choose $\alpha > 0$ such that
\begin{equation}\label{eq:lemConditionOnAlphaInduction}
(1 - \alpha) (\Hat{z}_\ell^N - \Hat{z}_r^j) + \alpha (\Hat{z}_1^N - \Hat{z}_{i_j}^j) \neq 0 \quad \forall j\in [N-1]. 
\end{equation}
Again, there are at most finitely many choices of $\alpha$ for which \Cref{eq:lemConditionOnAlphaInduction} fails and we choose $\alpha$ outside of this set.
    
If $\Hat{z}_1^N \cap \Hat{Z}^j \neq \emptyset$ for some $j\in [N]$, we can assume (upon relabeling if necessary) that $\Hat{z}_1^N = \Hat{z}_1^{N-1}$. Then, since since $\hat{Z}^{N-1} \cap \Hat{Z}^j = \emptyset$ for all $j\in [N-2]$ by the induction hypothesis, we also have
\begin{equation}
    \Hat{z}_1^N \cap \Hat{Z}^j = \emptyset \quad \text{for all} \quad j\in [N-2].
\end{equation}
We are therefore back in one of the Cases 1, 2 or 3 above, depending on whether $\Hat{z}_1^{N-1}$ is extreme for the convex hull of its sequence and whether the sequences $\Hat{Z}^{N-1}$ and $\Hat{Z}^N$ have the same center of mass.

Irrespective of which case arises, we construct a transformer $\T'$ with at most $2$ blocks ensuring that $\T' (\Hat{Z}^N) \cap \T' (\Hat{Z}^{N-1}) = \emptyset$ by applying the arguments used for $N=2$. Moreover, by the induction hypothesis, there exists $\delta_1 > 0$ such that the balls $\{ B_{\delta_1} (\hat{z}_\ell^j )\}_{\ell, j}$ are disjoint for all $\ell \in [\len (Z^j)]$ and all $j\in [N-1]$. By taking the parameters $\alpha$ and $\beta$ in the blocks of $\T'$ small enough, we can ensure also that the balls $\{ B_{\delta_1} (\T'(\hat{z}_\ell^j)) \}_{\ell, j}$ remain disjoint for all $\ell \in [\len (Z^j)]$ and all $j\in [N-1]$, whence $\T' (\Hat{Z}^N) \cap \T' (\Hat{Z}^{j}) = \emptyset$ for all $j\in [N - 1]$.

In summary, since $\Hat{Z}^j = \T_{N-1}(Z^j)$ for all $j\in[N]$, we have constructed a transformer $\T = \T' \circ \T_{N-1}$ with at most $2(N-1)$ blocks achieving $\T({Z}^j) \cap \T({Z}^{j'}) = \emptyset$ for all $j\neq j'\in [N]$. The proof is complete.
\end{proof}
%
\section{Conclusion}\label{sec:conclusion}
The results in this paper advance the mathematical understanding of transformers in the context of classification problems. We provide an explicit construction showing that transformers can perform sequence classification and, in doing so, we uncover two key roles played by their self-attention layers. First, they reduce dimensionality thanks to their clustering effect, which allows transformers to classify sequences with a total number of parameters independent of the sequence length. Second, self-attention layers separate tokens shared by different sequences, thanks to the non-locality of the self-attention mechanism as a transformation in the token space. The construction also incorporates two common features of real-life transformers: the alternation of self-attention and feed-forward layers, and the low-rank structure of the matrix inside the inner product of self-attention layers. Our theoretical results, therefore, offer an explanation for the remarkable practical efficiency of transformers when solving sequence classification tasks. In particular, we have shown that self-attention allows for a significant reduction in the number of parameters compared to traditional ResNet architectures used to solve the same tasks, with the number of parameters defining our transformer being independent of the input sequence length. 

Our work leaves room for several future improvements. An interesting question is whether our arguments can be extended to transformers with multi-head attention. In particular, even though empirical work suggests that a single head might suffice~\cite{michel2019sixteen}, we wonder if multiple heads can reduce the number of parameters and blocks required to achieve perfect sequence classification. For this, one should replace \Cref{eq:selfatt_a} with
\begin{equation}
    \mathrm{SA}_i(Z) = \rho z_i + \sum_{h=1}^H V^h \sum_{\ell = 1}^{\len(Z)} \Lambda_{i\ell}(Z,A^h)z_\ell,
\end{equation}
for a given number of heads $H\in \N$, and check, for example, whether the combination of multiple heads can reduce the number of blocks required by the separation step. 

On the other hand, it would be interesting to consider masked self-attention layers. Masking is a common feature of self-attention layers for classification tasks in NLP, where a token $z_i$ should only be influenced by those preceding it. To model this, one should replace \Cref{eq:selfatt_a} with
\begin{equation}
    \mathrm{SA}_i(Z) = \rho z_i + V \sum_{\ell = 1}^i \Lambda_{i\ell}(Z,A)z_\ell.
\end{equation}
Notice that the sum now runs only up to $i\in [\len (Z)]$, forcing tokens ahead of $z_i$ not to interact with it. 
Extending our present analysis to such a model of masked transformers remains an open challenge.
%
\section*{Acknowledgments}
This work was funded by the Humboldt Research
Fellowship for postdoctoral researchers, the Alexander von Humboldt-Professorship program, the European Union (Horizon Europe MSCA project ModConFlex, grant number 101073558), 24IOE027 of AFOSR, the COST Action MAT-DYN-NET, the Transregio 154 Project of the DFG, grants PID2020-112617GB-C22 and TED2021-131390B-I00 of MINECO (Spain), and the Madrid Government--UAM Agreement for the Excellence of the University Research Staff in the context of the V PRICIT (Regional Programme of Research and Technological Innovation). The authors thank Martín Hernández for valuable conversations.
%
\bibliographystyle{abbrvnat}  
\bibliography{refs}
\vfill
\end{document}