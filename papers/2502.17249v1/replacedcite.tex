\section{RELATED WORK}
Here, we review existing research closely related to our work. First, we summarize LiDAR-only odometry and mapping, categorized into direct and feature-based methods. We then provide an overview of loosely-coupled and tightly-coupled LiDAR-Inertial odometry methods.
\subsection{LiDAR-only Odometry and Mapping}
\emph{1) Direct Methods:}
The most prominent methods of scan matching in LiDAR-based SLAM rely on the ICP ____ paradigm. Starting from an initial alignment, ICP alternates between finding the closest point pairs between two point clouds and iteratively minimizing the distances of correspondences to get the optimal transformation parameters. With ICP, multiple LiDAR scans are aligned to generate a local map and output the corresponding poses, forming a direct LOAM module. During this process, the method operates on raw data, with or without using spatial downsample or temporal downsample. To reduce the mismatch rate, in addition to using point-to-point distance in traditional ICP, various error metrics, such as point-to-plane ____, can be applied in direct LOAM methods. While accounting for all environmental details, direct methods are computationally expensive and also introduce more outliers and noises.

\emph{2) Feature-based Methods:}
To improve accuracy and reduce computation load, many feature-based methods match only edge and planar features. The introduction of LOAM ____ establishes a standard for numerous subsequent works. For each incoming scan, edge and planar features are pre-extracted through local smoothness analysis and then aligned with the preceding scan to obtain odometry. With the odometry, multiple scans form a sweep, which is then registered to a global map incrementally. The cost in optimization is calculated as the Euclidean distance between points and their corresponding edges and surfaces.
To enable the algorithm to run in real-time on computationally constrained platforms, LeGO-LOAM ____ separates ground plane optimization before feature extraction. To further enhance running performance and accuracy, F-LOAM ____ adopts a non-iterative two-stage distortion compensation method and solves a weighted nonlinear least squares problem to achieve scan-to-map matching. During the pose optimization, a weight function based on local smoothness is introduced to increase the penalty on points with lower curvature in edge features and points with higher curvature in planar features to balance the matching process. 
The above works ____ mainly focus on multi-line spinning LiDARs. To solve the problem of algorithm adaptation for solid-state LiDARs, LOAM-Livox ____ considers the scanning mechanisms and low-level physical properties of solid-state LiDARs and performs point-level selection to extract reliable features. However, existing algorithms generally lack mechanisms for outlier suppression, which limits their accuracy. 
\subsection{LiDAR-Inertial Odometry}
Since pure LiDAR odometry often incurs drift over long-term operation and degenerates in featureless environments, fusion with the Inertial Measurement Unit (IMU) is an elegant solution. Current works on LiDAR-Inertial fusion can be categorized into two classes: loosely coupled and tightly coupled.

\emph{1) Loosely-coupled Methods:}
Loosely-coupled methods commonly handle LiDAR and IMU data independently and fuse their results in a modular fashion. In addition to LiDAR-only methods, ____ also provide typical examples of loosely-coupled IMU-aided LiDAR odometry, where poses integrated from IMU data serve as the initial guesses for LiDAR scan registration.

\emph{2) Tightly-coupled Methods:}
Tightly-coupled LiDAR-inertial odometry can be divided into optimization-based and filter-based approaches. In various optimization processes (e.g., sliding window joint optimization ____, and factor graph smoothing ____), optimization-based approaches account for the intrinsic characteristics of the two sensors, such as IMU observation noise, biases, and residuals from LiDAR scan registration, rather than simply fusing the final results.
For filter-based approaches, LINS ____ is the first tightly-coupled method that achieves the 6 DOF ego-motion estimation using the iterated Kalman filter. To further enhance computational efficiency and lower the odometry drift, FAST-LIO ____ proposes a new formula to compute the Kalman gain and a back-propagation to resolve motion distortion. In its subsequent work FAST-LIO2 ____, an incremental iKD-Tree map is maintained, significantly accelerating the mapping process.
Furthermore, with regard to map representation, ImMesh ____, building on the work of ____, becomes the first tightly-coupled LiDAR-inertial framework capable of reconstructing triangle meshes of large-scale environments online without relying on GPU acceleration.
Although fusion with IMU can achieve higher accuracy in long-range and degenerate scenarios, it introduces complex calibration and substantial computation.

Our method falls into the feature-based LOAM. We adopt scan-to-map matching similar to ____, but with a significant suppression of color and positional correspondence outliers, requiring only the addition of color image input. 
\begin{figure*}[!ht] 
    \vspace{0.3cm}
    \centering
    \includegraphics[width=7in]{Figures/workflow2.pdf} 
    \caption{The overview of our workflows. Each new LiDAR frame, colored by the corresponding RGB color image from the camera, is matched with the global map to provide the odometry output. The matching result is in turn used to register the frame to the global map. During the matching process, each point on the new colored edge/planar feature is used to calculate the distance and perceptually uniform color difference to its corresponding edge/planar feature retrieved from the global map. The color of the edge/planar feature is represented by the color of the nearest neighbor point of the current query point. The distances serve as inputs to the Welsch's function to get robust normalized residuals. Similarly, perceptually uniform color differences are processed using the Gaussian robust function to generate normalized color weights. The residuals are then multiplied by their corresponding weights and utilized in iterative pose optimization.}
    \label{fig:2}
\end{figure*}
\begin{figure}[t]
    \centering
    \includegraphics[width=3.4in]{Figures/compr_coloring.pdf}
    \caption{Steps of the coloring of LiDAR clouds and features.}
    \label{fig:3}
\end{figure}