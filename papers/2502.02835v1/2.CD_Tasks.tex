\section{CD Tasks}\label{sec_outline}

According to the granularity of results and the type of input images, CD in RSIs can be further divided into various sub-categories, including Binary CD (BCD), Multi-class CD/Semantic CD (MCD/SCD), and Time-series CD (TSCD). Fig.\ref{Fig.tasks} presents an overview of these different tasks. In the following, we summarize the benchmarks, applicational scope, and representative works related to each CD task.

\begin{figure*}[t]
\centering
    \includegraphics[width=1\linewidth]{Figs/CD_Tasks.png}
    \caption{A comparison between (a) BCD, (b) MCD/SCD, and (c) TSCD. The color regions in $Y_1, Y_2, Y_3$ and $Y_c^{1 \rightarrow t}$ indicate the pre-defined LCLU/change categories.}
    \label{Fig.tasks}
\end{figure*}

\subsection{Binary CD}

\textbf{Background:} BCD has been the most extensively studied CD task in the past few decades. Unless otherwise specified, BCD is also commonly abbreviated as CD in literature. As BCD has been comprehensively reviewed in existing literature, here we only provide a brief summary of the typical paradigms and representative work.

In the initial stages, DL-based BCD was seen as a segmentation task, where UNet-like Convolutional Neural Networks (CNNs) are employed to directly segment changes \cite{peng2019end}. Let $I_1$ and $I_2$ denote a pair of RSIs obtained on the dates $t_1$ and $t_2$, respectively. The general function of CD can be represented as:

\begin{equation}
    \mathcal{F}_{\theta}(I_1, I_2) = Y_c,
\end{equation}
where $Y_c$ is the predicted change map, $\mathcal{F}$ is the mapping function of a DNN with the trained parameters $\theta$. Differently, Daudt et al. \cite{daudt2018fully} proposed to first extract the temporal features, then model the change representations:
\begin{equation} \label{eq.BCD}
    \nu [\mu_1(I_1), \mu_2(I_2)] = Y_c,
\end{equation}
where $\mu_1$ and $\mu_2$ are two DNN encoders, $\nu$ is a DNN decoder. 
Under the circumstance that $I_1$ and $I_2$ exhibit homogeneity (e.g., collected by the same sensor or have similar spatial and spectral characteristics), $\mu_1$ and $\mu_2$ can be configured as siamese networks \cite{daudt2018fully}, i.e., share the same weight. This approach has been widely accepted as a paradigm for DL-based CD, as it allows effective exploitation of the temporal features.

\textbf{Techniques:} The major challenges in BCD are distinguishing semantic changes between seasonal changes and mitigating spatial misalignment as well as illumination differences. In CNN-based methods, channel-wise feature difference operations are commonly used to extract change features \cite{daudt2018fully, zhang2020feature}. Another common strategy is to leverage multiscale features to reduce the impact of redundant spatial details \cite{hou2021high}. Multiscale binary supervisions are also introduced in \cite{peng2019end} to align the embedding of change features. As an effective technique to aggregate global context, the attention mechanism is also widely used in CD of RSIs. Channel-wise attention is often used to improve the change representations \cite{li2022remote, peng2021scdnet}, while spatial attention is often used to exploit the long-range context dependencies \cite{chen2020dasnet, shi2021deeply}.

Another research focus in BCD is to model the temporal dependencies in pairs of RSIs. In \cite{chen2019change} a multilayer RNN module is adopted to learn change probabilities. Graph convolutional networks are also an efficient technique to propagate Land Cover Land Use (LCLU) information to identify changes\cite{wu2021multiscale}. Recently, Vision Transformers (ViTs) \cite{dosovitskiy2020image,li2024casformer} have emerged and gained great research interests in the RS field \cite{ding2022looking,hong2023cross}. There are two strategies to utilize ViTs for CD in RSIs. The first is to replace CNN backbones with ViTs to extract temporal features, such as ChangeFormer\cite{yuan2022transformer} and ScratchFormer\cite{noman2024remote}. Meanwhile, ViTs can also be used to model the temporal dependencies. In BiT\cite{chen2021remote}, a transformer encoder is employed to extract changes of interest, while two siamese transformer decoders are placed to refine the change maps. In CTD-Former\cite{zhang2023relation}, a cross-temporal transformer is proposed to interact between the different temporal branches.


\subsection{Multi-class CD/Semantic CD}

\textbf{Background:} In BCD, the results only indicate location of the change, leaving out the detailed change type. This is often not informative enough to support RS applications. In contrast, multi-class CD (MCD) refers to the task of classifying changes into multiple predefined classes or categories \cite{bovolo2015time}. On the other hand, semantic change detection (SCD) is introduced in recent DL-based CD literature to classify and represent the pre-event and after-event change classes \cite{daudt2019multitask, yang2021asymmetric}. Although there are slight differences in the representation of results, both MCD and SCD enable a detailed analysis of the changed regions, e.g., identifying the major changes and calculating the proportion of each type of change. The results can further be represented in an occurrence matrix indicating pre-event and after-event LCLU classes, such as presented in Fig.\ref{Fig.tasks}(b).

\textbf{Architectures:} MCD/SCD, with its provision of more detailed information, is indeed a more challenging task compared to BCD due to the need for modeling semantic information in particularly changed areas. According to the order of semantic modeling and CD, conventional methods for MCD can be roughly divided into two types, i.e. the post-classification comparison \cite{singh1990digital} and compound classification \cite{bruzzone1997iterative, wu2017post}. In DL, it is feasible to perform multi-task learning by jointly using different training objectives. There are two types of deep architectures for MCD/SCD in RSIs. The first architecture applies the common CD architecture in Equation. (\ref{eq.BCD}), and fuses bi-temporal information to classify multiple change types \cite{mou2018learning, zhu2022landuse}. The second approach employs a joint learning paradigm to learn semantic features and change representations simultaneously through different network branches \cite{daudt2019multitask, yang2021asymmetric}. This can be formulated as follows:

\begin{equation}
    \begin{aligned}
    & \phi_1 [\mu_1(I_1)] = Y_1, \phi_2 [\mu_1(I_1)] = Y_2,\\
    & \nu [\mu_1(I_1), \mu_2(I_2)] = Y_c,
\end{aligned}
\end{equation}
where $\phi_1$, $\phi_2$, and $\nu$ are three DNN modules that project the temporal features into semantic maps $Y_1, Y_2$ and change map $Y_c$, respectively.

\textbf{Techniques:} The techniques used in MCD/SCD can be categorized into two types: i) spatio-temporal fusion \cite{zhu2022landuse, zheng2022changemask} and ii) semantic dependency modeling \cite{ding2024scannet}.
In \cite{mou2018learning}, Mou et al. made an early attempt to employ DNNs for MCD. It is a joint CNN-RNN network where the CNN extracts semantic features, while the RNN models temporal dependencies to classify multi-class changes.


\subsection{Time-series CD}

\textbf{Background:} Differently from common CD studies that analyze bi-temporal RSIs, Time-Series CD (TSCD) aims to capture changes that have occurred over multiple periods or across a series of temporal images. This can better characterize the dynamics of change \cite{li2023sartscc} and discriminate between transient fluctuations and persistent changes, leading to more reliable and informative CD results. Conventional algorithms analyze the temporal curves to model the change patterns. Among these algorithms, \textit{trajectory classification} models the trajectory in change regions, \textit{statistical boundary} detects departure from common variations to detect changes, and \textit{regression} models the long-term momentum in the observed regions\cite{stahl2023automated}. Since these methods commonly do not consider spatial contexts, they are sensitive to noise and seasonal variations. Moreover, they have difficulty modeling complex or long-term change patterns. 

\textbf{Architectures:} Due to the scarcity of training data, DL-based TSCD did not emerge until very recent years. An intuitive approach is to employ RNNs to model temporal variations in time-series observations, as RNNs were originally designed for sequence processing. In \cite{yuan2020using} Long-Short-Term Memory (LSTM) network, a more delicate type of RNN is first introduced to detect and predict the burned areas in forests. Experimental results reveal that LSTM can better model the nonlinear characteristics in temporal data. In \cite{he2024timeseries} a temporal semantic segmentation method for time-series images is proposed. LSTM is employed to classify the spectral vectors into different LCLU types at different timestamps.

In these LSTM-based methods, the analysis is limited to the temporal dimension. Although the method in \cite{he2024timeseries} involves analysis of the spatial consistency, this is conducted as post-processing to reduce noise and is not end-to-end trainable. To overcome this limitation and to consider the spatial context in time-series RSIs, in \cite{sefrin2020deep} LSTM is combined with a CNN for joint spatiotemporal analysis. A CNN is employed to project time-series RSIs into spatial features, followed by an LSTM to model the temporal dependencies. This can be formulated as:

\begin{equation}
    \begin{aligned}
    & x_i = \psi (I_i),\\
    & \omega [x_1, x_2, ..., x_t] = [h_1, h_2, ..., h_t],\\
    & \nu [h_1, h_2, ..., h_t] = Y_c,
\end{aligned}
\end{equation}
where $i=1,2,...t$ is the time step, $x_i$ and $h_i$ are the extracted spatial and temporal features, $\psi$ and $\omega$ are the CNN and RNN units, respectively. $\nu$ can be a $softmax$ operation in multi-date LCLU CD applications\cite{sefrin2020deep}, or an anomaly detection function in disaster monitoring applications \cite{saha2022change}.
%However, in these calculations, the spatial and temporal features are independently considered. To jointly consider the
%In TSCD, a major challenge is the analysis of gradual or cumulative changes over time.  Although it also involves analysis of the spatial consistency, this is conducted as a post-processing to reduce the impact of noise.
%In \cite{meshkini2024multiannual} 3D convolutions are introduced to jointly consider the spatio-temporal contexts. To reduce parameters and to alleviate over-fitting, a 2D to 3D transition technique is proposed to adapt pre-trained 2D convolutional layers to process 3D features.


%\subsection{Map CD \& updating}

%\subsection{Heterogeneous CD}
%Heterogeneous CD aims to identify and analyze differences or changes in remote sensing data observed from data sources. These data sources may include different types of sensors, such as optical sensors, hyperspectral sensors, and radar sensors, as well as different data acquisition methods and periods.


