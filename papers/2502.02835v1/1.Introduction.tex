
\section{Introduction}\label{sec_intro}

Over the last 10 years, the emergence and success of Deep Learning (DL) techniques \cite{hong2024spectralgpt} have significantly advanced the field of Change Detection (CD) in Remote Sensing Images (RSIs). DL-based CD enables data-driven learning of specific changes of interest and, as a result, facilitates accurate and fully automatic processing of vast amounts of data. State-Of-The-Art (SOTA) methods~\cite{chen2021remote, Liky2023Changer, ding2024samcd} have reached an accuracy exceeding 90\% in the $F_1$ metric across multiple benchmark datasets for CD, highlighting the remarkable identification capability of DL-based CD approaches.

Despite these advances, the translation of CD methods into practical real-world applications remains a significant challenge. This arises from the inherent diversity present in the input RSIs, as well as the wide variety of scenarios to conduct CD algorithms. For instance, the multi-temporal RSIs that CD methods process can exhibit significant heterogeneity or spatial misalignment \cite{shi2020change}, and more fine-grained information is required to indicate the time of change or the specific change category. This necessitates the development of CD methodologies that can operate effectively within such varied and intricate environments.

Moreover, training a robust Deep Neural Network (DNN) for CD requires extensive and accurately labeled datasets. In many real-world scenarios, the presence of such data is scarce. The construction of a CD training set requires the collection of RSIs with expansive region coverage and adequate temporal intervals to capture changes of interest \cite{shen2021s2looking}. For some small or rare types of change, it is often difficult to collect a sufficient number of training samples. This poses negative impacts on the efficacy and generalization of DL-based CD approaches.

\begin{figure}[!t]
	\begin{center}
    \includegraphics[width = 0.5\textwidth]{Figs/Literature_num.png}
	\end{center}
	\caption{The number of literature publications associated with different CD topics over the past 10 years. Solid lines present different CD tasks, while the dashed lines indicate different supervision strategies.}
	\label{fig.paper_num}
\end{figure}

In response to these challenges, researchers have developed a variety of specialized CD methodologies that are customized to specific application contexts and training limitations. These methodologies encompass various subdivided CD tasks, each designed to meet the unique demands of a particular scenario. Concurrently, innovative training techniques and strategies have been introduced to mitigate the issue of 'data-hungry' in training DNNs for CD. By exploring the underlying semantic context and multi-temporal correlations that are inherent to RSIs, the demand for extensive training labels can be reduced. Based on the different levels of supervision signals introduced, DL-based CD methods can be divided into several categories, such as fully supervised, semi-supervised, self-supervised, weakly supervised, and unsupervised. To display the dynamics in recent CD-related studies, in Fig.\ref{fig.paper_num} we present the number of publications associated with different CD tasks and supervision strategies. The statistics are obtained through a search at \textit{Web of Science} \footnote{https://www.webofscience.com/wos/} using related keywords while filtering the metadata to exclude those irrelevant to \textit{remote sensing}. One can observe that there has been a rapid growth of interest in several CD topics, including multi-class CD, self-supervised CD, and semi-supervised CD. Additionally, some incomplete supervision settings have been rarely studied until very recent years (e.g., weakly supervised CD). These statistics indicate a trend of research focus in recent studies: as fully-supervised CD has already reached a high level of accuracy, an increasing number of investigations are being conducted on more challenging CD topics with incomplete supervision setups \cite{cheng2023change}.

In light of these developments, there is a pressing need to comprehensively review and analyze the recent research on DL-based CD methods, particularly those tailored to diverse applicational contexts and incomplete supervision circumstances. This review aims to fill this gap by providing a detailed examination of the literature on CD tasks, which have been partitioned into specialized domains to address the unique challenges of each setting. In doing so, we expect to provide an in-depth understanding of the techniques and strategies employed to train and deploy DNN-based CD methods in real-world scenarios. Furthermore, we seek to identify gaps in the existing literature and highlight areas for future research, thus contributing to the multifaceted advancement and broader application of CD methodologies. 


\begin{comment}
    
\section{Introduction}\label{sc1}
    In the last decade, the rapid development of deep learning (DL) has made it possible to perform automatic, accurate, and robust Change Detection (CD) on large volumes of Remote Sensing Images (RSIs). However, despite the advancements in CD methods, their practical application in real-world contexts remains limited due to the diverse input data and results encountered in many complex applications. For example, the input multi-temporal RSIs can be heterogeneous or unregistered, and more fine-grained information is required to indicate the time of change or the specific change category. Moreover, training a Deep Neural Network (DNN) requires a massive amount of training samples,  but in many cases, the available change samples are rare. To address these challenges, various specific CD methods have been developed considering different applicational scenarios and training resources. These methods span different sub-divided CD tasks that are tailored to specific scenarios. Additionally, recent advancements in self-supervision techniques and Visual Foundation Models (VFMs) have opened up new possibilities to address the 'data-hungry' issue of DL-based CD. Furthermore, the development of these methods in broader application scenarios requires further investigation and discussion. Therefore, this article summarizes the literature methods on different sub-divided CD tasks, as well as the available strategies and techniques to train and deploy DNN-based CD methods in a sample-limited scenario. We expect that this survey can provide new insights and inspiration for researchers in this field to develop more effective CD methods that can be applied in a wider range of contexts.
    
\section{Background}
    Change detection is crucial in remote sensing and has been a research focus in the past few decades. It provides insights into the dynamics of the Earth's surface and enables continuous tracking in the observed regions. It is important for a wide range of real-world applications \cite{hong2024multimodal}, including environmental monitoring, urban management, disaster alerting, damage assessment, land cover/land use (LCLU) monitoring, and agriculture monitoring.
    
    Before deep learning (DL) gained prominence, CD methods extracted and analyzed the change features to segment changes. They can be categorized into three categories based on the types of analyzed features, including texture features, object-based features, and angular features \cite{wen2021change}. Since these methods typically include many stages and require the selection of hand-crafted rules and hyper-parameters, they mostly suffer from error accumulation problems and have limited generalization. In recent years, DNNs have been extensively utilized in CD due to their ability to automatically learn complex spatiotemporal patterns and the generalization to large volumes of data. In particular, deep neural networks, including Convolutional Neural Networks (CNNs), Recurrent Neural Networks (RNNs), and Vision Transformers (ViTs) \cite{dosovitskiy2020image}, are widely used for CD in VHR RSIs. A common practice is to use weight-sharing CNNs, i.e., siamese CNNs, to aggregate and segment the multi-temporal semantic changes. Since RNNs are designed for modeling the temporal dependency, they are often employed to learn the change information from the CNN features \cite{mou2018learning}. ViTs are also leveraged as encoders to aggregate the features or to model the semantic-change correlations \cite{hong2024spectralgpt}. State-of-the-art (SOTA) methods can detect changes with high accuracy.

    Although much progress has been made, the use of CD techniques in real-world applications is still limited. There are two major barriers. First, the scarcity of training samples in practical applications poses a great challenge to the generalization of CD methods. To train a CD DNN, large volumes of high-quality and well-annotated change instances are required. The construction of a CD training set requires that there is a certain time gap between the acquisition dates, and the observation platform covers large areas. In cases where the change type is rare or the observation platforms are heterogeneous, it is often difficult to collect enough training samples. Second, the spatial and temporal granularity of CD results is not fine enough to support real-world applications. Most of the CD methods only perform binary change detection (BCD), i.e., producing a single binary change map, whereas in real-world applications detailed information w.r.t. the specific date and type of the change event is often required.
    
    To conquer these limitations and to promote the application of CD, many label-efficient CD methods are developed, and some sub-divided CD tasks considering the different application scenarios are studied. However, although there are literature reviews that provide an overview of CD and the state-of-the-art (SOTA) DL-based methods, limited attention has been paid to the granularity and supervision mode of CD methods. Given this, this study conducts a systematic review of DL-based CD from task and sample perspectives, respectively. By doing so, we contribute to the literature by summarizing the available techniques, strategies, and learning frameworks for applying CD in different scenarios and providing an outlook of future trends.

\end{comment}


%Change Detection (CD) in Remote Sensing (RS) is to discriminate and locate the Earth's surface changes in multi-temporal images captured at different times. %It is crucial for a wide variety of Remote Sensing applications, including Land Cover Land Use (LCLU) monitoring\cite{lv2021land}, urban management, agriculture monitoring\cite{Liu2019reviewHSI}, disaster alerting, damage assessment\cite{zheng2021building}, etc.
