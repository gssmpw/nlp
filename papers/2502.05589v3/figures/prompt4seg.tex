\begin{figure}[htb]
\small
\begin{tcolorbox}[left=3pt,right=3pt,top=3pt,bottom=3pt,title=Instruction Part of the Segmentation Prompt (W/ Reflection).]
\begin{verbatim}
# Instruction
## Context
- **Goal**: Your task is to segment a multi-turn conversation between a 
user and a chatbot into topically coherent units based on semantics. 
Successive user-bot exchanges with the same topic should be grouped 
into the same segmentation unit, and new segmentation units should 
be created when topic shifts.
- **Data**: The input data is a series of user-bot exchanges separated 
by "\n\n". Each exchange consists of a single-turn conversation between 
the user and the chatbot, started with "[Exchange (Exchange Number)]: ".
- **Tips**: Refer fully to the provided rubric 
and examples for guidance on segmentation.
## Requirements
### Output Format
- Output the segmentation results in **JSONL (JSON Lines)** format. 
Each dictionary represents a segment, consisting of one or more 
user-bot exchanges on the same topic. 
Each dictionary should include the following keys:
  - **segment_id**: The index of this segment, starting from 0.
  - **start_exchange_number**: The number of the **first** user-bot 
  exchange in this segment.
  - **end_exchange_number**: The number of the **last** 
  user-bot exchange in this segment.
  - **num_exchanges**: An integer indicating the number of 
  user-bot exchanges in this segment, calculated as:
  **end_exchange_number** - **start_exchange_number** + 1.
Here is an example of the expected output:
```
<segmentation>
{"segment_id": 0, "start_exchange_number": 0, 
"end_exchange_number": 5, "num_exchanges": 6}
{"segment_id": 1, "start_exchange_number": 6, 
"end_exchange_number": 8, "num_exchanges": 3}
...
</segmentation>
```
## Segment Rubric
{{segment_rubric}}
## Segment Examples
{{segment_examples}}
# Data
{{text_to_be_segmented}}
# Question
## Please generate the segmentation result from the input data that 
meets the following requirements:
- **No Missing Exchanges**:  Ensure that the exchange numbers cover 
all exchanges in the given conversation without omission. 
- **No Overlapping Exchanges**: Ensure that successive segments have 
no overlap in exchanges.
- **Accurate Counting**:  The sum of **num_exchanges**
across all segments should equal the total number of user-bot exchanges.
- **Utilize Segment Rubric**: Use the given segment rubric 
and examples to better segment.
- Provide your segmentation result between the tags:
<segmentation></segmentation>.
# Output
Now, provide the segmentation result based on the instructions above.
\end{verbatim}
\end{tcolorbox}
\caption{Prompt for GPT-4 segmentation (w/ reflection).}
\label{fig: prompt4seg}
\end{figure}