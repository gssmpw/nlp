\begin{figure}[htbp]
    \small
    \vspace{-5mm}
    \begin{tcolorbox}[left=3pt,right=3pt,top=3pt,bottom=3pt,title=\textbf{Segmentation rubric learned from \textit{TIAGE}}]
\begin{itemize}
    \item Ensure segments encapsulate a complete thematic or topical exchange before initiating a new segment. This includes recognizing when a topic shift is part of the same thematic exchange and should not trigger a new segment.

    \item Segments should not only capture the flow of conversation by recognizing subtle topic shifts but also ensure that related questions and answers, or setup and response exchanges, are included within the same segment to preserve the natural flow and context of the dialogue.

    \item Maintain the integrity of conversational dynamics, ensuring that exchanges which include setup and response (or question and answer) are not divided across segments. This preserves the context and flow of the dialogue, recognizing that some topic shifts, while apparent, are part of a larger thematic discussion.

    \item Segments must accurately reflect the thematic depth of the conversation, ensuring that all parts of a thematic exchange, including indirect responses or tangentially related comments, are grouped within the same segment to maintain conversational coherence.

    \item Evaluate the conversational cues and context to determine the thematic linkage between exchanges. Avoid creating new segments for responses that, while seemingly off-topic, are contextually related to the preceding messages, ensuring a coherent and unified thematic narrative.

    \item Prioritize the preservation of conversational momentum when determining segment boundaries, ensuring that the segmentation does not interrupt the natural progression of dialogue or the development of thematic elements, even when the conversation takes unexpected turns.

    \item Assess the thematic relevance of each conversational turn, ensuring segments are not prematurely divided by superficial topic changes that are part of a broader thematic dialogue. This includes recognizing when a seemingly new topic is a direct continuation or an elaboration of the previous exchange, thereby maintaining thematic coherence and conversational flow.

    \item Consider the conversational and thematic continuity over superficial changes in topic or structure when segmenting conversations. This ensures that segments reflect the natural flow and thematic integrity of the dialogue, even when the conversation takes subtle turns.

    \item Incorporate flexibility in segment boundaries to accommodate for the natural ebb and flow of conversational topics, ensuring that segments are not overly fragmented by minor topic shifts that remain within the scope of the overarching thematic dialogue.

    \item Avoid over-segmentation by recognizing the thematic bridges between conversational turns. Even when a conversation appears to shift topics, if the underlying theme or narrative purpose connects the exchanges, they should be considered part of the same segment to preserve the dialogue's natural progression and thematic integrity.
\end{itemize}
\end{tcolorbox}
\caption{Segmentation rubric learned on \textit{TIAGE}~\citep{xie2021tiage}.}
\label{fig: segmentation_rubric_tiage}
\end{figure}

\begin{figure}[htbp]
    \small
    \vspace{-5mm}
    \begin{tcolorbox}[left=3pt,right=3pt,top=3pt,bottom=3pt,title=\textbf{Segmentation rubric learned from \textit{SuperDialSeg}}]
\begin{itemize}
    \item Segmentation should reflect natural pauses or shifts in the conversation, indicating a change in topic or focus.
    \item Each segment should aim to be self-contained, providing enough context for the reader to understand the topic or question being addressed without needing to refer to other segments.
    \item Ensure segmentation captures the full scope of a thematic exchange, using linguistic cues and conversational context to guide the identification of natural breaks or transitions in dialogue.
    \item Segmentation should prioritize thematic continuity over structural cues alone, ensuring that all parts of a thematic exchange, including follow-up questions or clarifications, are contained within the same segment.
    \item Segments must ensure logical and thematic coherence, grouping together all elements of an exchange that contribute to a single topic or question, even if the conversation appears structurally disjointed.
    \item Ensure segments maintain thematic progression, especially in conversations where multiple inquiries and responses explore different facets of the same overarching topic.
    \item Segmentation should avoid over-segmentation by ensuring that a series of inquiries and responses that explore different aspects of a single overarching topic are grouped within the same segment, even if they contain multiple question-answer pairs.
    \item Ensure that segments are not prematurely divided based on superficial structural cues like greetings or sign-offs, but rather on the substantive thematic content of the exchange.
    \item Ensure segmentation recognizes and preserves the thematic progression within a conversation, even when minor topic shifts occur, by evaluating the overall context and goal of the exchange rather than segmenting based on immediate linguistic cues alone.
    \item Ensure that segments accurately reflect the inquiry-response cycle, grouping all related questions and their corresponding answers into a single segment to preserve the flow and coherence of the conversation.
\end{itemize}
\end{tcolorbox}
\caption{Segmentation rubric learned on \textit{SuperDialSeg}~\citep{jiang2023superdialseg}.}
\label{fig: segmentation_rubric_superseg}
\end{figure}