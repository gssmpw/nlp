\begin{figure}[!h]
    \centering
    \includegraphics[width=\linewidth]{figures/intro_example.pdf}
    \caption{Illustration of retrieval augmented response generation with different memory granularities. \textcolor{dark_blue}{\textit{Turn-level memory}} is too fine-grained, leading to fragmentary and incomplete context. \textcolor{brown}{\textit{Session-level memory}} is too coarse-grained, containing too much irrelevant information. \textcolor{plum}{\textit{Summary based methods}} suffer from information loss that occurs during summarization. \textcolor{dark_green}{\textit{Ours (segment-level memory)}} can better capture topically coherent units in long conversations, striking a balance between including more relevant, coherent information while excluding irrelevant content. Bullseye $\odot$ indicates the retrieved memory units at \textcolor{dark_blue}{turn level} or \textcolor{dark_green}{segment level} under the same context budget. [0.xx]: similarity between target query and history content. \textcolor{dark_blue}{Turn-level} retrieval errors: \colorbox{light_green}{\textcolor{dark_green}{false negative}}, \colorbox{light_blue}{\textcolor{red}{false positive}}.}
    \label{fig: intro_example}
\end{figure}