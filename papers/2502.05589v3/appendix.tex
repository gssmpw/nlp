\section{Appendix}

\subsection{Details of Conversation Segmentation Model}
\label{sec: segmentation_details}

We use GPT-4-0125 as the backbone LLM for segmentation. The zero-shot segmentation prompt is provided in Figure~\ref{fig: prompt4seg-zero-shot}. It instructs the segmentation model to generate all segmentation indices at once, avoiding the iterative segmentation process used in LumberChunker~\citep{duarte2024lumberchunker}, which can lead to unacceptable latency. We specify that the output should be in \textbf{JSONL} format to facilitate subsequent processing.
To generate segmentation guidance, we select the top 100 poorly segmented samples with the largest Window Diff metric from the training set. The segmentation guidance consists of two parts: (1) \textit{\textbf{Segmentation Rubric}}: Criteria items on how to make better segmentation. (2) \textit{\textbf{Representative Examples}}: The most representative examples that include the ground-truth segmentation, the model's prediction, and the reflection on the model's errors.
The number of rubric items is set to 10. To meet this requirement, we divide the top 100 poorly segmented samples into 10 mini-batches and prompt the LLM-based segmentation model to reflect on each batch individually. The segmentation model is also asked to select the most representative example in each batch, which is done concurrently with rubric generation. Figure~\ref{fig: prompt4rubric} presents the prompt used to generate rubric. The generated rubric is shown at Fig.~\ref{fig: segmentation_rubric_tiage} and Fig.~\ref{fig: segmentation_rubric_superseg} on \textit{TIAGE} and \textit{SuperDialSeg}, respectively. After the segmentation guidance is learned, we utilize the prompt shown in Figure~\ref{fig: prompt4seg} as a few-shot segmentation prompt. For simplicity and fair comparison, we do not use any rubric for conversation segmentation in \textit{LOCOMO} and \textit{Long-MT-Bench+}.

\begin{figure}[htb]
\small
\begin{tcolorbox}[left=3pt,right=3pt,top=3pt,bottom=3pt,title=Instruction Part of the Segmentation Prompt (Zero-Shot).]
\begin{verbatim}
# Instruction
## Context
- **Goal**: Your task is to segment a multi-turn conversation between a 
user and a chatbot into topically coherent units based on semantics. 
Successive user-bot exchanges with the same topic should be grouped 
into the same segmentation unit, and new segmentation units should 
be created when topic shifts.
- **Data**: The input data is a series of user-bot exchanges separated 
by "\n\n". Each exchange consists of a single-turn conversation between 
the user and the chatbot, started with "[Exchange (Exchange Number)]: ".
### Output Format
- Output the segmentation results in **JSONL (JSON Lines)** format. 
Each dictionary represents a segment, consisting of one or more 
user-bot exchanges on the same topic. 
Each dictionary should include the following keys:
  - **segment_id**: The index of this segment, starting from 0.
  - **start_exchange_number**: The number of the **first** user-bot 
  exchange in this segment.
  - **end_exchange_number**: The number of the **last** 
  user-bot exchange in this segment.
  - **num_exchanges**: An integer indicating the number of 
  user-bot exchanges in this segment, calculated as:
  **end_exchange_number** - **start_exchange_number** + 1.
Here is an example of the expected output:
```
<segmentation>
{"segment_id": 0, "start_exchange_number": 0, 
"end_exchange_number": 5, "num_exchanges": 6}
{"segment_id": 1, "start_exchange_number": 6, 
"end_exchange_number": 8, "num_exchanges": 3}
...
</segmentation>
```
# Data
{{text_to_be_segmented}}
# Question
## Please generate the segmentation result from the input data that 
meets the following requirements:
- **No Missing Exchanges**:  Ensure that the exchange numbers cover 
all exchanges in the given conversation without omission. 
- **No Overlapping Exchanges**: Ensure that successive segments have 
no overlap in exchanges.
- **Accurate Counting**:  The sum of **num_exchanges**
across all segments should equal the total number of user-bot exchanges.
- Provide your segmentation result between the tags:
<segmentation></segmentation>.
# Output
Now, provide the segmentation result based on the instructions above.
\end{verbatim}
\end{tcolorbox}
\caption{Prompt for GPT-4 segmentation (zero-shot).}
\label{fig: prompt4seg-zero-shot}
\end{figure}
\begin{figure}[htb]
\small
\begin{tcolorbox}[left=3pt,right=3pt,top=3pt,bottom=3pt,title=Prompt for Generating the Segmentation Guidance]
\begin{verbatim}
# Instruction
## Context
**Goal**: Your task is to evaluate the differences between a language 
model's predicted segmentation and the ground-truth segmentation made 
by expert annotators for multiple human-bot conversations. 
Analyze these differences, reflect on the prediction errors, and 
generate one concise rubric item for future conversation segmentation. 
You will be provided with some existing rubric items derived 
from previous examples. 
1. Begin by reviewing and copying the existing rubric items.
2. Modify, update, or replace the existing items if they do not 
adequately address the current segmentation errors.
3. Generate only one new rubric item to minimize segmentation errors 
in the given examples.
4. Select and reflect on the most representative example 
from the provided data.
**Data**: You will receive a segmented conversation example, 
including both the prediction and the ground-truth segmentation.
Each segment begins with "Segment segment_id:". 
Additionally, you will be provided with some existing rubric items 
derived from previous examples. Modify, update, or even replace them 
if they do not adequately explain the current segmentation mistakes.
## Requirements
- Add at most one new rubric item at a time even 
though multiple examples are provided.
- Ensure the rubric is user-centric, concise, and each item 
is mutually exclusive.
- You can modify, update, or replace the existing items 
if they do not adequately 
address the current segmentation errors.
- Present your new rubric item within `<rubric></rubric>`. 
- Provide the most representative example with your reflection 
within `<example></example>`. Here is an example:
```
<reflection>
Your reflection on the prediction errors, 
example by example.
</reflection>
<rubric>
- [one and only one new rubric item]
</rubric>
<example>
Present the most representative example, 
along with your reflection on this example. 
</example>
```
# Existing Rubric: {{past_rubric}}
# Examples: {{examples}}

# Output
\end{verbatim}
\end{tcolorbox}
\caption{Prompt for generating segmentation guidance.}
\label{fig: prompt4rubric}
\end{figure}
\input{figures/prompt4seg}
\begin{figure}[htbp]
    \small
    \vspace{-5mm}
    \begin{tcolorbox}[left=3pt,right=3pt,top=3pt,bottom=3pt,title=\textbf{Segmentation rubric learned from \textit{TIAGE}}]
\begin{itemize}
    \item Ensure segments encapsulate a complete thematic or topical exchange before initiating a new segment. This includes recognizing when a topic shift is part of the same thematic exchange and should not trigger a new segment.

    \item Segments should not only capture the flow of conversation by recognizing subtle topic shifts but also ensure that related questions and answers, or setup and response exchanges, are included within the same segment to preserve the natural flow and context of the dialogue.

    \item Maintain the integrity of conversational dynamics, ensuring that exchanges which include setup and response (or question and answer) are not divided across segments. This preserves the context and flow of the dialogue, recognizing that some topic shifts, while apparent, are part of a larger thematic discussion.

    \item Segments must accurately reflect the thematic depth of the conversation, ensuring that all parts of a thematic exchange, including indirect responses or tangentially related comments, are grouped within the same segment to maintain conversational coherence.

    \item Evaluate the conversational cues and context to determine the thematic linkage between exchanges. Avoid creating new segments for responses that, while seemingly off-topic, are contextually related to the preceding messages, ensuring a coherent and unified thematic narrative.

    \item Prioritize the preservation of conversational momentum when determining segment boundaries, ensuring that the segmentation does not interrupt the natural progression of dialogue or the development of thematic elements, even when the conversation takes unexpected turns.

    \item Assess the thematic relevance of each conversational turn, ensuring segments are not prematurely divided by superficial topic changes that are part of a broader thematic dialogue. This includes recognizing when a seemingly new topic is a direct continuation or an elaboration of the previous exchange, thereby maintaining thematic coherence and conversational flow.

    \item Consider the conversational and thematic continuity over superficial changes in topic or structure when segmenting conversations. This ensures that segments reflect the natural flow and thematic integrity of the dialogue, even when the conversation takes subtle turns.

    \item Incorporate flexibility in segment boundaries to accommodate for the natural ebb and flow of conversational topics, ensuring that segments are not overly fragmented by minor topic shifts that remain within the scope of the overarching thematic dialogue.

    \item Avoid over-segmentation by recognizing the thematic bridges between conversational turns. Even when a conversation appears to shift topics, if the underlying theme or narrative purpose connects the exchanges, they should be considered part of the same segment to preserve the dialogue's natural progression and thematic integrity.
\end{itemize}
\end{tcolorbox}
\caption{Segmentation rubric learned on \textit{TIAGE}~\citep{xie2021tiage}.}
\label{fig: segmentation_rubric_tiage}
\end{figure}

\begin{figure}[htbp]
    \small
    \vspace{-5mm}
    \begin{tcolorbox}[left=3pt,right=3pt,top=3pt,bottom=3pt,title=\textbf{Segmentation rubric learned from \textit{SuperDialSeg}}]
\begin{itemize}
    \item Segmentation should reflect natural pauses or shifts in the conversation, indicating a change in topic or focus.
    \item Each segment should aim to be self-contained, providing enough context for the reader to understand the topic or question being addressed without needing to refer to other segments.
    \item Ensure segmentation captures the full scope of a thematic exchange, using linguistic cues and conversational context to guide the identification of natural breaks or transitions in dialogue.
    \item Segmentation should prioritize thematic continuity over structural cues alone, ensuring that all parts of a thematic exchange, including follow-up questions or clarifications, are contained within the same segment.
    \item Segments must ensure logical and thematic coherence, grouping together all elements of an exchange that contribute to a single topic or question, even if the conversation appears structurally disjointed.
    \item Ensure segments maintain thematic progression, especially in conversations where multiple inquiries and responses explore different facets of the same overarching topic.
    \item Segmentation should avoid over-segmentation by ensuring that a series of inquiries and responses that explore different aspects of a single overarching topic are grouped within the same segment, even if they contain multiple question-answer pairs.
    \item Ensure that segments are not prematurely divided based on superficial structural cues like greetings or sign-offs, but rather on the substantive thematic content of the exchange.
    \item Ensure segmentation recognizes and preserves the thematic progression within a conversation, even when minor topic shifts occur, by evaluating the overall context and goal of the exchange rather than segmenting based on immediate linguistic cues alone.
    \item Ensure that segments accurately reflect the inquiry-response cycle, grouping all related questions and their corresponding answers into a single segment to preserve the flow and coherence of the conversation.
\end{itemize}
\end{tcolorbox}
\caption{Segmentation rubric learned on \textit{SuperDialSeg}~\citep{jiang2023superdialseg}.}
\label{fig: segmentation_rubric_superseg}
\end{figure}
\begin{figure}[htbp]
    \small
\begin{tcolorbox}[left=3pt,right=3pt,top=3pt,bottom=3pt,title=\textbf{Ground-truth Segment:}]
\begin{itemize}
\item \textbf{Segment 0: }
hello, how are you doing?
hello. pretty good, thanks. and yourself?
awesome, i just got back from a bike ride.
cool! do you spend a lot of time biking?
yup. its my favorite thing to do. do you?
i love playing folk music. i actually hope to be a professional musician someday
that is interesting. what instruments do you play?
i can play the guitar and the piano and i also like to sing.
i can only sing when i drink, but i do not like to do that anymore.
\item \textbf{Segment 1: }
i m not a big drinker either. do you have a job?
construction, like my dad. what do you do when you are not being a rock star
nice! i work as a custodian. not too glamorous but it pays the bills haha
i feel ya. you gotta do what you gotta do.
exactly. do you have other hobbies besides biking?
\end{itemize}
\end{tcolorbox}
\begin{tcolorbox}[left=3pt,right=3pt,top=3pt,bottom=3pt,title=\textbf{Predicted Segment:}]
\begin{itemize}
\item \textbf{Segment 0: }
hello, how are you doing?
hello. pretty good, thanks. and yourself?
awesome, i just got back from a bike ride.
\item \textbf{Segment 1: }
cool! do you spend a lot of time biking?
yup. its my favorite thing to do. do you?
i love playing folk music. i actually hope to be a professional musician someday

\item \textbf{Segment 2: }
that is interesting . what instruments do you play?
i can play the guitar and the piano and i also like to sing.
i can only sing when i drink, but i do not like to do that anymore.

\item \textbf{Segment 3: }
i m not a big drinker either. do you have a job?
construction, like my dad. what do you do when you are not being a rock star
nice! i work as a custodian. not too glamorous but it pays the bills haha

    \item \textbf{Segment 4: }
i feel ya. you gotta do what you gotta do.
exactly. do you have other hobbies besides biking?
\end{itemize}
\end{tcolorbox}
\caption{An example of poor segmentation from GPT-4 zero-shot segmentation illustrates that the GPT-4 powered segmentation model favors a more fine-grained segmentation. The Window Diff metric between the ground truth and the prediction is 0.80.}
\label{fig: poorly_segmented_example}
\end{figure}

\subsection{Additional Cost Analysis}
\label{sec: cost}

\section{Supplementary Materials for Computational Cost Analysis}\label{sec:detail_cost}

\begin{table}[!h]
    \centering
    \caption{\textbf{Total training times of different methods in semi-supervised settings.} All recorded experiment times are based on a single NVIDIA H100-80G GPU.}
    \resizebox{\linewidth}{!}{

      \begin{tabular}{cc|ccccccccc}
       \toprule
       \rowcolor{COLOR_MEAN}  \textbf{Type} & \textbf{Method} & \textbf{Cora} & \textbf{Citeseer} & \textbf{Pubmed} & \textbf{WikiCS} & \textbf{Instagram} & \textbf{Reddit} & \textbf{Books} & \textbf{Photo} & \textbf{Computer} \\ \midrule
       \multicolumn{2}{c|}{\# Training Samples} & 140 & 120 & 60 & 580 & 1,160 & 3,344 & 4,155 & 4,836 & 8,722 \\ \midrule
       \multirow{5}{*}{\textbf{Classic}} & GCN$_{\text{ShallowEmb}}$ & 2.8s & 2.7s & 2.7s & 3.2s & 1.8s & 7.7s & 12.7s & 13.8s & 33.5s \\ 
       & {GAT$_{\text{ShallowEmb}}$} & 1.9s & 2.4s & 3.8s & 3.3s & 2.0s & 6.0s & 10.5s & 12.3s & 38.0s \\ 
       & {SAGE$_{\text{ShallowEmb}}$} & 1.9s & 3.9s & 5.0s & 2.9s & 1.8s & 6.4s & 16.9s & 21.1s & 33.3s \\ 
       & {SenBERT-66M} & 8.5s & 7.9s & 5.9s & 27.9s & 14.7s & 1.2m & 1.5m & 1.8m & 3.3m \\ 
       & {RoBERTa-355M} & 21.2s & 18.9s & 12.7s & 1.2m & 2.3m & 6.5m & 8.1m & 3.8m & 6.9m \\ \midrule 

      \multirow{2}{*}{\textbf{Encoder}} & GCN$_{\text{LLMEmb}}$ & 1.2m & 1.4m & 13.4m & 7.4m & 4.5m & 16.0m & 23.5m & 26.8m & 44.7m \\ 
      & ENGINE  & 2.2m & 2.4m & 16.1m & 15.2m & 9.3m & 22.9m & 31.1m & 38.8m & 1.1h \\ \midrule

      \textbf{Reasoner} & TAPE & 25.5m & 27.8m & 5.6h & 2.7h & 2.0h & 8.0h & 9.9h & 11.7h & 14.5h \\ \midrule

      \multirow{3}{*}{\textbf{Predictor}} & {LLM$_{\text{IT}}$} & 25.6m & 22.0m & 3.9m & 1.1h & 49.1m & 1.1m & 2.0h & 2.4h & 2.7h \\ 
      & GraphGPT & 16.4m & 15.5m & 1.2h & 48.5m & 30.1m & 1.8h & 2.4h & 2.2h & 5.8h \\ 
       & LLaGA & 1.7m & 2.2m & 5.2m & 5.8m & 3.0m & 20.9m & 19.5m & 23.5m & 43.1m  \\    
        \bottomrule
      \end{tabular}
    }
    \label{tab:timecost}
\end{table}


\begin{table}[!h]
    \centering
    \caption{\textbf{Total training times of different methods in supervised settings.} All recorded experiment times are based on a single NVIDIA H100-80G GPU.}
    \resizebox{\linewidth}{!}{

      \begin{tabular}{cc|cccccccccc}
       \toprule
       \rowcolor{COLOR_MEAN}  \textbf{Type} & \textbf{Method} & \textbf{Cora} & \textbf{Citeseer} & \textbf{Pubmed} & \textbf{WikiCS} & \textbf{arXiv} & \textbf{Instagram} & \textbf{Reddit} & \textbf{Books} & \textbf{Photo} & \textbf{Computer} \\ \midrule
       \multicolumn{2}{c|}{\# Training Samples} & 1,624 & 1,911 & 11,830 & 7,020 &  90,941  & 6,803 & 20,060 & 24,930 & 29,017 & 52,337  \\ \midrule
       \multirow{5}{*}{\textbf{Classic}} & {GCN$_{\text{ShallowEmb}}$} & 1.8s & 1.7s & 5.2s & 5.1s & 51.2s & 19.5s & 8.5s & 14.9s & 19.7s & 25.8s \\ 
       & {GAT$_{\text{ShallowEmb}}$} & 2.1s & 1.9s & 7.9s & 5.7s & 1.5m & 2.7s & 6.9s & 16.6s & 28.0s & 44.6s \\ 
       & {SAGE$_{\text{ShallowEmb}}$} & 1.7s & 3.0s & 7.6s & 4.0s & 1.3m & 2.0s & 7.2s & 19.6s & 20.1s & 43.2s \\ 
       & {SenBERT-66M} &  35s & 41s & 2.6m & 2.5m & 7.4m & 1.2m & 4.4m & 1.8m & 2.2m & 4.1m \\ 
       & {RoBERTa-355M} & 1.3m & 1.6m & 9.2m & 5.5m & 40.8m & 5.3m & 15.9m & 9.7m & 11.9m & 22.4m \\ \midrule 

      \multirow{2}{*}{\textbf{Encoder}} & GCN$_{\text{LLMEmb}}$ & 1.2m & 1.4m & 13.4m & 7.5m & 1.4h & 4.5m & 16.1m & 23.6m & 26.8m & 44.8m  \\ 
      & ENGINE & 2.2m & 2.4m & 16.1m & 19.4m & 2.6h & 8.9m & 24.2m & 35.2m & 44.2m & 1.2h \\ \midrule

      \textbf{Reasoner} & TAPE & 27.4m & 30.3m & 5.9h & 2.8h & 37.4h &  2.1h & 8.3h & 10.0h & 12.0h & 15.0h \\ \midrule

      \multirow{3}{*}{\textbf{Predictor}} & {LLM$_{\text{IT}}$} & 1.0h & 1.3h & 9.9h & 4.2h & 36.3h & 2.7h & 3.4h & 5.7h & 7.4h & 12.4h \\ 
      & GraphGPT & 26.4m & 29.5m & 2.7h & 1.7h & 7.8h & 49.1m & 3.4h & 3.8h & 3.6h & 7.8h \\ 
       & LLaGA &  5.6m & 7.7m & 25.6m & 18.8m & 7.7h & 10.6m & 32.2m & 1.0h & 1.4h & 2.5h \\    
        \bottomrule
      \end{tabular}
    }
    \label{tab:timecost_supervised}
\end{table}


\begin{table}[!h]
    \centering
    \caption{\textbf{Inference times of different methods.} Values in brackets denote the average inference time per case in milliseconds (ms). All recorded experiment times are based on a single NVIDIA H100-80G GPU.}
    \resizebox{0.8\linewidth}{!}{
     \begin{tabular}{cc|ccccc}
     \toprule
        \rowcolor{COLOR_MEAN} \multicolumn{2}{c|}{\textbf{Method}} & \textbf{Cora} & \textbf{arXiv}  & \textbf{Instagram} & \textbf{Photo} & \textbf{WikiCS} \\ \midrule
        \multicolumn{2}{c|}{\# Test Samples} &  542 & 48,603 & 5,847 & 2,268 & 9,673 \\ \midrule
        \textbf{Classic} & GCN & 0.9ms & 21.8ms & 2.0ms & 7.5ms & 4.4ms \\ \midrule 
        \textbf{Encoder} & GCN$_{\text{LLMEmb}}$ & 14.0s (26ms) & 23.8m (29ms) & 53.6s (24ms) & 5.3m (33ms) & 3.7m (38ms) \\  \midrule
        \textbf{Reasoner} & TAPE & 5.0m (551ms) & 10.4h (767ms) & 23.7m (627ms) & 2.3h (863ms) & 1.3h (813ms) \\ \midrule
        \multirow{3}{*}{\textbf{Predictor}} & LLM$_{\text{IT}}$ & 1.2m (129ms) & 3.3h (243ms) & 2.7m (71ms) & 24.1m (149ms) & 5.8m (60ms) \\ 
        & GraphGPT & 1m (104ms) & 1.2h (87ms) & 2.0m (52ms) & 10.4m (64ms) & 11.0m (112ms) \\ 
        & LLaGA & 11.2s (21ms) & 57.1m (70ms) & 1.3m (35ms) & 4.4m (27ms) & 2.4m (25ms) \\
        \bottomrule
    \end{tabular}
    }
    \label{tab:inference_cost}
\end{table}



Table~\ref{tab: cost_analysis} compares the overall costs involved in memory construction, memory retrieval, and response generation across different methods. The results demonstrate that our method significantly enhances performance compared to the baseline while only slightly increasing computational overhead, and it outperforms the MemoChat method in both efficiency and effectiveness.


\subsection{The Analogy between the Reflection Augmentation and Prefix-Tuning}
When a small amount of conversation data with segment annotations is available, we explore how to leverage this data to transfer segmentation knowledge and better align the LLM-based segmentation model with human preferences. Inspired by the prefix-tuning technique~\citep{li2021prefix} and reflection mechanism~\citep{shinn2023reflexion,renze2024self}, we treat the segmentation prompt as the ``prefix'' and iteratively optimize it through LLM self-reflection, ultimately obtaining a segmentation guidance $\bm{G}$.

Prefix-tuning seeks to learn a prefix matrix $\bm{P}$ to boost the performance of the language model $\operatorname{LM}_{\phi}$ without fine-tuning its parameter $\phi$. The prefix matrix $\bm{P}$ is prepended to the activation $h$ of the Transformer layer: 
\begin{equation}
h_i= \begin{cases}\bm{P}[i,:], & \text { if } i \in \mathcal{P}_{idx} \\ \operatorname{LM}_\phi\left(z_i, h_{<i}\right), & \text { otherwise }\end{cases}
\end{equation}
where $\mathcal{P}_{idx}$ is the prefix indices.

In the context of our segmentation scenario, our goal is to ``learn'' a textual guidance $\boldsymbol{G}$ that directs the segmentation model toward improved segmentation outcomes. The process of updating the segmentation guidance $\boldsymbol{G}$ parallels the optimization of the prefix parameter $\bm{P}$ in prefix-tuning. Initially, the segmentation guidance $\boldsymbol{G}_{0}$ is set to empty, analogous to the initial prefix parameter $\bm{P}_{0}$. During each iteration of guidance updating, we first apply our conversation segmentation model in a zero-shot manner to a batch of conversation data. Building upon the insights that LLMs possess the ability for self-reflection and improvement~\citep{shinn2023reflexion, renze2024self}, we then instruct the segmentation model to reflect on its mistakes given the ground-truth segmentation and update the segmentation guidance $\boldsymbol{G}$. This process mirrors Stochastic Gradient Descent (SGD) optimization:
\begin{equation}
    \boldsymbol{G}_{m+1}=\boldsymbol{G}_m-\eta \nabla \mathcal{L}\left(\boldsymbol{G}_m\right), 
\end{equation}
where $\nabla \mathcal{L}\left(\boldsymbol{G}_m\right)$ denotes the gradient of segmentation loss, which we assume is estimated implicitly by the LLM itself and used to adjust the next segmentation guidance $\boldsymbol{G}_{m+1}$.


\subsection{Prompt for GPT-4 Evaluation}
\label{sec: prompt4eval}

We use the same evaluation prompts as  MemoChat~\citep{lu2023memochat}. The LLM-powered evaluation consists of single-sample scoring (GPT4Score) and pair-wise comparison. The evaluation prompts are displayed in Figure~\ref{fig: prompt4eval}. For pair-wise comparison, we alternate the order of the responses and conduct a second comparison for each pair to minimize position bias. 

\begin{figure*}[htbp]
\small
\begin{tcolorbox}[left=3pt,right=3pt,top=3pt,bottom=3pt,title=\textbf{Single-Sample Score}]
\begin{verbatim}
You are an impartial judge. You will be shown Related
Conversation History, User Question and Bot Response.
```\nRelated Conversation History\nRCH\_0\n```
```\nUser Question\nUQ\_1\n```
```\nBot Response\nBR\_2\n```
Please evaluate whether Bot Response is faithful to the content of 
Related Conversation History to answer User Question. 
Begin your evaluation by providing a short explanation, 
then you must rate Bot Response on an integer rating of 1 to 
100 
by strictly following this format: 
<rating>an integer rating of 1 to 100</rating>.
\end{verbatim}
\end{tcolorbox}
\begin{tcolorbox}[left=3pt,right=3pt,top=3pt,bottom=3pt,title=\textbf{Pair-Wise Comparison}]
\begin{verbatim}
You are an impartial judge. You will be shown 
Related Conversation History, User Question and Bot Response.
```\nRelated Conversation History\nRCH_0\n```
```\nUser Question\nUQ_1\n```
```\nBot Response A\nBR_2\n```
```\nBot Response B\nBR_3\n```
Please evaluate whether Bot Response is faithful to the content of 
Related Conversation History to answer User Question. 
Begin your evaluation by 
providing a short explanation, 
then you must choose the better bot response by giving 
either A or B. 
If the two responses are the same, you can choose NONE:
<chosen>A (or B or NONE)</chosen>.
\end{verbatim}
\end{tcolorbox}
\caption{Prompt used in GPT-4 evaluation, following \citet{lu2023memochat}.}
\label{fig: prompt4eval}
\end{figure*}


\subsection{Evaluation Results on the Official QA Pairs of LOCOMO}
\label{sec: main_locomo2}
As \textit{LOCOMO}~\citep{maharana2024evaluating} released a subset containing QA pairs recently. To ensure reproducibility, we evaluate our method on these official QA pairs. Table~\ref{tab: main_locomo2} presents the evaluation results. The superiority of our \sysname\ is also evident on these QA pairs, demonstrating its superior effectiveness and robustness.

\begin{table*}[!t]
    \small
    \centering
    \setlength{\tabcolsep}{1mm}
    \caption{Performance comparison on the official question-answer pairs of \textit{LOCOMO} using MPNet retriever. All other settings remain the same as in Table~\ref{tab: main_results}. MemoChat~\citep{lu2023memochat} is not applicable in \textit{Mistral-7B-Instruct-v0.3} due to Mistral's inability to execute the ``Memo Writing'' step, as it often fails to generate a valid JSON response needed to construct the memory bank in \citet{lu2023memochat}.
    }
    \label{tab: main_locomo2}
    \resizebox{1\columnwidth}{!}{
    \begin{tabular}{l|cccccc|cc}
    \toprule
    
    \multirow{2}{*}{\textbf{Methods}} &  \multicolumn{6}{@{}c|}{{\bf QA Performance}} & \multicolumn{2}{@{}c}{{\ \bf Context Length}} \\
    \cmidrule (lr){2-7} \cmidrule (lr){8-9}
    & GPT4Score & BLEU & Rouge1 & Rouge2 & RougeL & BERTScore & \# Turns & \# Tokens \\
    \midrule
    \multicolumn{9}{@{}c}{ \textit{ GPT-35-Turbo } } \\
    \midrule
    Full History & 66.28 & 7.51 & 28.73 & 14.07 & 27.90 & 87.82 & 293 & 18,655 \\
    MemoChat & 75.77 & 11.28 & 32.91 & 18.82 & 29.78 & 87.98 & - & 1,159 \\
    \midrule
    Turn-Level & 81.52 & 11.91 & 36.00 & 19.59 & 34.99 & \textbf{88.64} & 55.00 & 3,026 \\
    Session-Level & 74.20 & 10.95 & 29.92 & 14.64 & 29.27 & 87.96 & 54.48 & 3,442 \\
    \midrule
    \sysname\ & \textbf{84.21} & \textbf{12.80} & \textbf{36.70} & \textbf{19.90} & \textbf{35.61} & 88.59 & 56.49 & 3,565 \\
    \midrule
    \multicolumn{9}{@{}c}{ \textit{ Mistral-7B-v0.3 } } \\
    \midrule
    Full History & 69.13 & 6.77 & 30.40 & 15.02 & 29.20& 87.29 & 293 & 18,655 \\
    \midrule
    Turn-Level & 78.82 & 10.09 & 32.75 & 16.25 & 31.75 & \textbf{87.97} & 55.00 & 3,026 \\
    Session-Level & 62.68 & 7.37 & 26.68 & 12.38 & 25.86 & 86.98 & 54.48 & 3,442 \\
    \midrule
    \sysname\ & \textbf{80.07} & \textbf{10.67} & \textbf{32.82} & \textbf{16.65} & \textbf{31.81} & 87.87 & 56.49 & 3,565 \\
    \bottomrule
    \end{tabular}
    }
\end{table*}

\subsection{Case Study}
\label{sec: case_study}

To further demonstrate the advantages of our method, we conduct a qualitative evaluation. Figure~\ref{fig: case_study_segment_vs_turn} presents a specific case comparing the segment-level memory with the turn-level memory. It demonstrates that using turn-level memory units fails to address the user's request. We attribute this to the fragmentation of user-agent turns, and the critical turns may not explicitly contain or relate to the keywords in the user's request.

Similarly, using session-level memory units is also sub-optimal, as illustrated in Figure~\ref{fig: case_study_segment_vs_session}. This issue arises because a session often includes multiple topics, introducing a significant amount of irrelevant information that hampers effective retrieval. The irrelevant information also distracts the LLM, as noted in previous studies~\citep{shi2023large, liu2024lost}. 

We also conduct a case study to compare our method with two recent, powerful memory management techniques: \textit{RecurSum}~\citep{wang2023recursively} and \textit{ConditionMem}~\citep{yuan2023evolving}, as shown in Figure~\ref{fig: case_study_segment_vs_rsum} and Figure~\ref{fig: case_study_segment_vs_condmem}. The results indicate that the summarization process in these methods often omits detailed information that is essential for accurately answering the user's request.


\begin{table*}[htbp]
    \centering
    \small
    \begin{tabular}{p{14cm}}
     \toprule
\#\#\#  Objective: \\
Generate a 5-day family travel itinerantry that satisfies all specified requirements while adhering to highly fine-grained constraints. The generated itinerary should balance real-time adaptability, strict hard attributes, and semantic soft attributes. \\

\#\#\# User Profile: \\
 - Travelers: 2 adults + 1 child (age 8) \\
 - Budget: $<=$ \$300/day (total \$1,500 for the trip) \\
 - Activity Balance: 70\% educational/cultural experiences, 20\% relaxation, 10\% family-friendly shopping. \\

\#\#\# Hard Attributes: \\
- Activity Scheduling: \\
\quad- Each activity must have a defined start and end time, ensuring there is no overlap between activities. \\
\quad- A break period from 13:00-14:30 is mandatory daily. \\
\quad- Each activity must fit within a 2-hour window unless otherwise specified. \\

- Budget Requirements: \\
\quad- Each day’s total cost (including transportation, food, and activities) must not exceed \$300. \\
\quad- Transportation is limited to metro and walking only, with a maximum of 3 metro rides per day. \\

- Location Constraints: \\
\quad- Must-visit locations: City Zoo (Day 1) and Science Museum (Day 3). \\
\quad- Activities must occur in geographically adjacent areas to minimize walking distance. \\

- Keyword Requirements: \\
\quad- Each day’s description must include specific keywords. For example: \\
\quad- Day 1: “wildlife,” “exploration,” and “interactive learning.” \\
\quad- Day 3: “science,” “innovation,” and “hands-on exhibits.” \\

- Structure Constraints: \\
\quad- Each day’s itinerary must consist of 4 sections: \\
\quad\quad- Morning activity \\
\quad\quad- Break/lunch period \\ 
\quad\quad- Afternoon activity \\
\quad\quad- Evening summary (limited to 50 words) \\

\#\#\# Soft Attributes \\
- Tone and Emotion: \\ 
\quad- Day 1: Use a tone that conveys “excitement and discovery.” \\ 
\quad- Day 3: Use a tone that conveys “curiosity and wonder.” \\
- Language Style: \\ 
\quad- Use descriptive, vivid, and family-friendly language throughout. \\
\quad- Include at least one metaphor or simile per day (e.g., "The Science Museum felt like stepping into the future!"). \\
- Visual Details: \\
\quad- Each activity must include specific sensory details (e.g., "the bright colors of the parrots at the zoo" or "the tinkling sound of water fountains at the park").

- Adaptive Adjustments (Real-time Constraints): \\
\quad- Weather Sensitivity: \\
\quad\quad- If the rain forecast exceeds 60\%, replace outdoor activities with indoor alternatives while keeping the overall tone and keywords intact. \\ 
\quad- Physical Endurance: \\
\quad\quad- If a day’s total walking distance exceeds 10 kilometers, the next day’s activities must reduce walking by 30\%. \\
\quad- Health Responsiveness: \\
\quad\quad- If a health-related issue arises (e.g., fatigue or illness), adjust the itinerary dynamically to: \\
\quad\quad- Reduce activity duration to half. \\ 
\quad\quad- Substitute the activity with a more relaxing or passive option. \\
\bottomrule
    \end{tabular}
    \caption{The complete travel planner case study.}
    \label{tab:travel_planner_case}
\end{table*}

\subsection{Details of Dataset Construction}
\label{sec: dataset_details}

(i) \textit{LOCOMO}~\citep{maharana2024evaluating}: this dataset contains the longest conversations to date, with an average of  more than 9K tokens per sample. Since \textit{LOCOMO} does not release the corresponding question-answer pairs when we conduct our experiment, we prompt GPT-4 to generate QA pairs for each session as in \citet{alonso2024toward}. We also conduct evaluation on the recently released official QA pairs in Appendix~\ref{sec: main_locomo2}.

(ii) \textit{Long-MT-Bench+}: \textit{Long-MT-Bench+} is reconstructed from the 
\textit{MT-Bench+}~\citep{lu2023memochat} dataset. In \textit{MT-Bench+}, human experts are invited to expand the original questions and create long-range questions as test samples. However, there are two drawbacks when using this dataset to evaluate the memory mechanism of conversational agents: (1) the number of QA pairs is relatively small, with only 54 human-written long-range questions; and (2) the conversation length is not sufficiently long, with each conversation containing an average of 13.3 dialogue turns and a maximum of 16 turns. In contrast, the conversation in \textit{LOCOMO} has an average of 300 turns and 9K tokens. To address (1), we use these human-written questions as few-shot examples and ask GPT-4 to generate a long-range test question for each dialogue topic. For (2), following~\citep{yuan2023evolving}, we merge five consecutive sessions into one, forming longer dialogues that are more suitable for evaluating memory in long-term conversation. We refer to the reconstructed dataset as \textit{Long-MT-Bench+} and present its statistics in Table~\ref{tab: datasets_statistics}.

\begin{table}[ht]
\fontsize{6}{7}\selectfont
\centering
% \renewcommand{\arraystretch}{0.8} % Reduce vertical space
\resizebox{0.75\columnwidth}{!}{%
\begin{tabular}{@{}p{2.1cm}p{0.5cm}p{0.5cm}@{}}
\midrule
               & \ZhEn & \EnDe \\ \cmidrule{1-3}
Documents      & 38            & 30 \\
Segments      & 377           & 104 \\
Avg. English tokens/seg   &     32.02     & 71.91\\\midrule
\end{tabular}%
}
\caption{
Basic dataset statistics. For \ZhEn, average tokens per segment are based on the English reference translation, and for \EnDe, on the English source. Tokens are counted using whitespace in both cases.
}
\label{tab:basic_stats}
\vspace{-10pt}
\end{table}

\subsection{Details of Retrieval Performance Measurement}
\label{sec: retrieval_measurement}

We measure the retrieval performance in terms of the discounted cumulative gain (DCG) metric~\citep{jarvelin2002cumulated}: 
\begin{equation} 
\textit{DCG}=\sum_{i=1}^{p}\frac{rel_{i}}{\log_{2}(i+1)}, 
\label{eq: dcg} 
\end{equation} 
where $rel_{i}$ denotes the relevance score of the retrieved user-agent turn ranked at position $i$, and $p$ represents the total number of retrieved turns. Note that in the \textit{Long-MT-Bench+} dataset, answering a single question often requires referring to several consecutive turns. Therefore, we distribute the relevance score evenly across these relevant turns and set the relevance score of irrelevant turns to zero. For instance, assume that the ground truth reference turn set for question $q$ is $\mathcal{R}(q) = \{r_{k+j}\}_{j=1}^{N}$, which is provided by the dataset. In this case, the relevance score for each turn is set as follows:
$$
\textit{rel}_{i} =
\begin{cases}
0 & i < k+1 \\ 
\frac{1}{N} & k+1 \leq i \leq k+N \\ 
0 & i > k+N
\end{cases}.
$$
This approach allows us to evaluate retrieval performance at different granularity.

\subsection{Additional Experiments on CoQA and Persona-Chat}

To further validate SeCom's robustness and versatility across a broader range of dialogue types, we conduct additional experiments on other benchmarks, \textbf{Persona-Chat}~\citep{zhang-etal-2018-personalizing} and \textbf{CoQA}~\citep{reddy2019coqa}.

Given the relatively short context length of individual samples in these datasets, we adopt an approach similar to Long-MT-Bench+ by aggregating multiple adjacent samples into a single instance. For CoQA, each sample is supplemented with the text passages of its 10 surrounding samples. Since CoQA answers are derived from text passages rather than dialogue turns, we replace the turn-level baseline with a sentence-level baseline.
For Persona-Chat, we utilize the expanded version provided by~\citet{jandaghi2023faithful}. Conversations are aggregated by combining each sample with its 5 surrounding samples. Following the next utterance prediction protocol, we include the personas of both conversational roles in the prompt. Due to the large scale of these datasets, we select subsets for experimentation. From CoQA, we randomly sample 50 instances from an initial pool of 500, resulting in a subset containing over 700 QA pairs. Similarly, for Persona-Chat, we randomly select 100 instances, encompassing over 1,000 utterances in total. 

As shown in Table~\ref{tab: results_on_coqa} and Table~\ref{tab: results_on_spc}, \sysname\ consistently outperforms baseline methods across these datasets, highlighting its effectiveness in handling diverse dialogue scenarios, including open-ended and multi-turn interactions.

\begin{table*}[!t]
    \small
    \centering
    % \setlength{\tabcolsep}{1mm}
    \caption{QA performance comparison on  \textit{CoQA} using MPNet-based retrieval model. The response generation model is \texttt{GPT-3.5-Turbo}. 
    }
    \label{tab: results_on_coqa}
    \resizebox{1\columnwidth}{!}{
    \begin{tabular}{l|cccccc|c}
    \toprule
    
    \textbf{Methods}        & \textbf{GPT4Score} & \textbf{BLEU} & \textbf{Rouge1} & \textbf{Rouge2} & \textbf{RougeL} & \textbf{BERTScore} & \textbf{\#Tokens} \\ 
    \midrule
    Sentence-Level & 95.55 & 36.02 & 48.58 & 37.96 & 47.03 & 90.01 & 993 \\ 
    Session-Level  & 91.58 & 31.22 & 47.18 & 37.32 & 45.92 & 89.65 & 3,305 \\ 
    \midrule
    ConditionMem   & 94.32 & 34.35 & 47.91 & 37.55 & 46.38 & 89.77 & 1,352 \\ 
    MemoChat       & 97.16 & 38.17 & 49.54 & 38.23 & 47.77 & 90.14 & 1,041 \\ 
    COMEDY         & 97.48 & 38.02 & 49.41 & 38.19 & 47.63 & 90.06 & 3,783 \\ 
    \midrule
    \sysname\ (Ours)   & \bf98.31 & \bf39.57 & \bf50.44 & \bf39.51 & \bf48.98 & \bf90.37 & 1,016 \\ 

    \bottomrule
    \end{tabular}
    }
\end{table*}
\begin{table*}[!t]
    \small
    \centering
    % \setlength{\tabcolsep}{1mm}
    \caption{Next utterance prediction performance comparison on  \textit{Persona-Chat} using MPNet-based retrieval model. The response generation model is \texttt{GPT-3.5-Turbo}. 
    }
    \label{tab: results_on_spc}
    \resizebox{1\columnwidth}{!}{
    \begin{tabular}{l|cccccc|cc}
    \toprule

    \multirow{2}{*}{\textbf{Methods}} &  \multicolumn{6}{@{}c|}{{\bf Performance}} & \multicolumn{2}{@{}c}{{\bf Context Length}} \\
    \cmidrule (lr){2-7} \cmidrule (lr){8-9}
    & GPT4Score & BLEU & Rouge1 & Rouge2 & RougeL & BERTScore & \# Turns & \# Tokens \\
    \midrule
    Turn-Level & 69.23 & 5.73 & 21.38 & 9.06 & 19.87 & 87.28 & 24.00 & 682 \\
    Session-Level & 67.35 & 5.45 & 21.80 & 8.86 & 20.04 & 87.34 & 116.91 & 3,593 \\
    \midrule
    ConditionMem & 73.21 & 6.16 & 22.52 & 9.88 & 20.95 & 87.44 & - & 1,388 \\ 
    MemoChat & 76.83 & 7.21 & 25.13 & 10.81 & 22.31 & 87.68 & - & 1,296 \\ 
    COMEDY & 76.52 & 7.05 & 24.97 & 10.54 & 22.18 & 87.60 & - & 3,931 \\ 
    \midrule
    \sysname\ (Ours) & \textbf{78.34} & \textbf{7.75} & \textbf{26.01} & \textbf{11.57} & \textbf{23.98} & \textbf{87.82} & 23.48 & 702 \\

    \bottomrule
    \end{tabular}
    }
\end{table*}

\subsection{Human Evaluation Results}
\label{sec: human_evaluation}

To ensure a holistic assessment, we conduct human evaluation to gauge the quality of the LLM's response in conversation. We adopt the human evaluation scheme of COMEDY~\citep{chen2024compress}, which encompasses five perspectives: Coherence, Consistency, Engagingness, Humanness and Memorability. Ten human annotators are asked to score the responses following a detailed rubric for each perspective. Results in Table~\ref{tab: human_evaluation} show that the rank of different methods from human evaluation is generally consistent with those obtained from automated metrics, confirming the practical effectiveness of our proposed approach. 

\begin{figure}[!t]

    \centering
    \includegraphics[width=0.9\linewidth]{sec/figs/human_evaluation_fig.pdf}
    \vspace{-3mm}
    \caption{Human evaluation of IPO and CogVideoX-2B.}
    \label{fig:human_evaltion}
    \vspace{-5mm}
\end{figure}

\subsection{Performance Using Smaller Segmentation Model}

To make our method applicable in resource-constrained environments, we conduct additional experiments by replacing the \texttt{GPT-4-Turbo} used for the segmentation model with the \texttt{Mistral-7B-Instruct-v0.3} and a \texttt{RoBERTa} based model fine-tuned on SuperDialseg~\citep{jiang2023superdialseg}. Table~\ref{tab: main_results_slm_seg} shows that \sysname\ maintains the advantage over baseline methods when switching from GPT-4 to Mistral-7B. Notably, even with a RoBERTa based segmentation model, \sysname\ retains a substantial performance gap over other granularity-based baselines.

\input{tables/main_results_SLM_seg}

% \begin{table*}[t]
    \small
    \centering
    \setlength{\tabcolsep}{1mm}
    \caption{Segmentation performances on three datasets.
    $^{\dag}$: numbers reported in \citet{gao2023unsupervised}. Other baselines are reported in \citet{jiang2023superdialseg}. The best performance is highlighted in \textbf{bold}, and the second best is highlighted by \underline{underline}. \colorbox{light_gray}{Numbers in gray} correspond to \textbf{supervised} setting.}
    \label{tab: segment_main}
    \resizebox{\columnwidth}{!}{
    
    \begin{tabular}{l|cccc|cccc|cccc}
    \toprule
    \multirow{2}{*}{\textbf{Methods}} & \multicolumn{4}{@{}c}{{\bf Dialseg711}} & \multicolumn{4}{@{}c}{{\bf SuperDialSeg}} & \multicolumn{4}{@{}c}{{\bf TIAGE}}  \\
    \cmidrule (lr){2-5} \cmidrule (lr){6-9} \cmidrule (lr){10-13} 
    & Pk↓ & WD↓ & F1↑ & Score↑ & Pk↓ & WD↓ & F1↑ & Score↑ & Pk↓ & WD↓ & F1↑ & Score↑ \\

    \midrule
    \multicolumn{13}{@{}c}{\textbf{Unsupervised}} \\
    \midrule
    
    BayesSeg & 0.306 & 0.350 & 0.556 & 0.614 & \underline{0.433} & 0.593 & \underline{0.438} & 0.463 & 0.486 & 0.571 & 0.366 & 0.419\\
    TextTiling & 0.470 & 0.493 & 0.245 & 0.382 & 0.441 & \underline{0.453} & 0.388 & \underline{0.471} & 0.469 & 0.488 & 0.204 & 0.363 \\
    GraphSeg & 0.412 & 0.442 & 0.392 & 0.483 & 0.450 & 0.454 & 0.249 & 0.398 & 0.496 & 0.515 & 0.238 & 0.366\\
    \midrule
    TextTiling+Glove & 0.399 & 0.438 & 0.436 & 0.509 & 0.519 & 0.524 & 0.353 & 0.416 & 0.486 & 0.511 & 0.236 & 0.369\\
    TextTiling+[CLS] & 0.419 & 0.473 & 0.351 & 0.453 & 0.493 & 0.523 & 0.277 & 0.385 & 0.521 & 0.556 & 0.218 & 0.340 \\
    TextTiling+NSP & 0.347 & 0.360 & 0.347 & 0.497 & 0.512 & 0.521 & 0.208 & 0.346 & 0.425 & 0.439 & 0.285 & 0.426\\
    GreedySeg & 0.381 & 0.410 & 0.445 & 0.525 & 0.490 & 0.494 & 0.365 & 0.437 & 0.490 & 0.506 & 0.181 & 0.341\\
    CSM & 0.278 & 0.302 & \underline{0.610} & \underline{0.660} & 0.462 & 0.467 & 0.381 & 0.458 & \underline{0.400} & \underline{0.420} & \underline{0.427} & \underline{0.509} \\
    DialSTART $^{\dag}$ & \underline{0.178} & \underline{0.198} & - & - & - & - & - & - & - & - & - & - \\
    \midrule
    \textbf{Ours} & \textbf{0.093} & \textbf{0.103} & \textbf{0.888} & \textbf{0.895} & \textbf{0.277} & \textbf{0.289} & \textbf{0.758} & \textbf{0.738} & \textbf{0.363} & \textbf{0.401} & \textbf{0.596} & \textbf{0.607} \\
    
    \bottomrule
    \end{tabular}
    }
\end{table*}
