\section{Appendix}

\subsection{Details of Conversation Segmentation Model}
\label{sec: segmentation_details}

We use GPT-4-0125 as the backbone LLM for segmentation. The zero-shot segmentation prompt is provided in Figure~\ref{fig: prompt4seg-zero-shot}. It instructs the segmentation model to generate all segmentation indices at once, avoiding the iterative segmentation process used in LumberChunker~\citep{duarte2024lumberchunker}, which can lead to unacceptable latency. We specify that the output should be in \textbf{JSONL} format to facilitate subsequent processing.
To generate segmentation guidance, we select the top 100 poorly segmented samples with the largest Window Diff metric from the training set. The segmentation guidance consists of two parts: (1) \textit{\textbf{Segmentation Rubric}}: Criteria items on how to make better segmentation. (2) \textit{\textbf{Representative Examples}}: The most representative examples that include the ground-truth segmentation, the model's prediction, and the reflection on the model's errors.
The number of rubric items is set to 10. To meet this requirement, we divide the top 100 poorly segmented samples into 10 mini-batches and prompt the LLM-based segmentation model to reflect on each batch individually. The segmentation model is also asked to select the most representative example in each batch, which is done concurrently with rubric generation. Figure~\ref{fig: prompt4rubric} presents the prompt used to generate rubric. The generated rubric is shown at Fig.~\ref{fig: segmentation_rubric_tiage} and Fig.~\ref{fig: segmentation_rubric_superseg} on \textit{TIAGE} and \textit{SuperDialSeg}, respectively. After the segmentation guidance is learned, we utilize the prompt shown in Figure~\ref{fig: prompt4seg} as a few-shot segmentation prompt. For simplicity and fair comparison, we do not use any rubric for conversation segmentation in \textit{LOCOMO} and \textit{Long-MT-Bench+}.

\begin{figure}[htb]
\small
\begin{tcolorbox}[left=3pt,right=3pt,top=3pt,bottom=3pt,title=Instruction Part of the Segmentation Prompt (Zero-Shot).]
\begin{verbatim}
# Instruction
## Context
- **Goal**: Your task is to segment a multi-turn conversation between a 
user and a chatbot into topically coherent units based on semantics. 
Successive user-bot exchanges with the same topic should be grouped 
into the same segmentation unit, and new segmentation units should 
be created when topic shifts.
- **Data**: The input data is a series of user-bot exchanges separated 
by "\n\n". Each exchange consists of a single-turn conversation between 
the user and the chatbot, started with "[Exchange (Exchange Number)]: ".
### Output Format
- Output the segmentation results in **JSONL (JSON Lines)** format. 
Each dictionary represents a segment, consisting of one or more 
user-bot exchanges on the same topic. 
Each dictionary should include the following keys:
  - **segment_id**: The index of this segment, starting from 0.
  - **start_exchange_number**: The number of the **first** user-bot 
  exchange in this segment.
  - **end_exchange_number**: The number of the **last** 
  user-bot exchange in this segment.
  - **num_exchanges**: An integer indicating the number of 
  user-bot exchanges in this segment, calculated as:
  **end_exchange_number** - **start_exchange_number** + 1.
Here is an example of the expected output:
```
<segmentation>
{"segment_id": 0, "start_exchange_number": 0, 
"end_exchange_number": 5, "num_exchanges": 6}
{"segment_id": 1, "start_exchange_number": 6, 
"end_exchange_number": 8, "num_exchanges": 3}
...
</segmentation>
```
# Data
{{text_to_be_segmented}}
# Question
## Please generate the segmentation result from the input data that 
meets the following requirements:
- **No Missing Exchanges**:  Ensure that the exchange numbers cover 
all exchanges in the given conversation without omission. 
- **No Overlapping Exchanges**: Ensure that successive segments have 
no overlap in exchanges.
- **Accurate Counting**:  The sum of **num_exchanges**
across all segments should equal the total number of user-bot exchanges.
- Provide your segmentation result between the tags:
<segmentation></segmentation>.
# Output
Now, provide the segmentation result based on the instructions above.
\end{verbatim}
\end{tcolorbox}
\caption{Prompt for GPT-4 segmentation (zero-shot).}
\label{fig: prompt4seg-zero-shot}
\end{figure}
\begin{figure}[htb]
\small
\begin{tcolorbox}[left=3pt,right=3pt,top=3pt,bottom=3pt,title=Prompt for Generating the Segmentation Guidance]
\begin{verbatim}
# Instruction
## Context
**Goal**: Your task is to evaluate the differences between a language 
model's predicted segmentation and the ground-truth segmentation made 
by expert annotators for multiple human-bot conversations. 
Analyze these differences, reflect on the prediction errors, and 
generate one concise rubric item for future conversation segmentation. 
You will be provided with some existing rubric items derived 
from previous examples. 
1. Begin by reviewing and copying the existing rubric items.
2. Modify, update, or replace the existing items if they do not 
adequately address the current segmentation errors.
3. Generate only one new rubric item to minimize segmentation errors 
in the given examples.
4. Select and reflect on the most representative example 
from the provided data.
**Data**: You will receive a segmented conversation example, 
including both the prediction and the ground-truth segmentation.
Each segment begins with "Segment segment_id:". 
Additionally, you will be provided with some existing rubric items 
derived from previous examples. Modify, update, or even replace them 
if they do not adequately explain the current segmentation mistakes.
## Requirements
- Add at most one new rubric item at a time even 
though multiple examples are provided.
- Ensure the rubric is user-centric, concise, and each item 
is mutually exclusive.
- You can modify, update, or replace the existing items 
if they do not adequately 
address the current segmentation errors.
- Present your new rubric item within `<rubric></rubric>`. 
- Provide the most representative example with your reflection 
within `<example></example>`. Here is an example:
```
<reflection>
Your reflection on the prediction errors, 
example by example.
</reflection>
<rubric>
- [one and only one new rubric item]
</rubric>
<example>
Present the most representative example, 
along with your reflection on this example. 
</example>
```
# Existing Rubric: {{past_rubric}}
# Examples: {{examples}}

# Output
\end{verbatim}
\end{tcolorbox}
\caption{Prompt for generating segmentation guidance.}
\label{fig: prompt4rubric}
\end{figure}
\begin{figure}[htb]
\small
\begin{tcolorbox}[left=3pt,right=3pt,top=3pt,bottom=3pt,title=Instruction Part of the Segmentation Prompt (W/ Reflection).]
\begin{verbatim}
# Instruction
## Context
- **Goal**: Your task is to segment a multi-turn conversation between a 
user and a chatbot into topically coherent units based on semantics. 
Successive user-bot exchanges with the same topic should be grouped 
into the same segmentation unit, and new segmentation units should 
be created when topic shifts.
- **Data**: The input data is a series of user-bot exchanges separated 
by "\n\n". Each exchange consists of a single-turn conversation between 
the user and the chatbot, started with "[Exchange (Exchange Number)]: ".
- **Tips**: Refer fully to the provided rubric 
and examples for guidance on segmentation.
## Requirements
### Output Format
- Output the segmentation results in **JSONL (JSON Lines)** format. 
Each dictionary represents a segment, consisting of one or more 
user-bot exchanges on the same topic. 
Each dictionary should include the following keys:
  - **segment_id**: The index of this segment, starting from 0.
  - **start_exchange_number**: The number of the **first** user-bot 
  exchange in this segment.
  - **end_exchange_number**: The number of the **last** 
  user-bot exchange in this segment.
  - **num_exchanges**: An integer indicating the number of 
  user-bot exchanges in this segment, calculated as:
  **end_exchange_number** - **start_exchange_number** + 1.
Here is an example of the expected output:
```
<segmentation>
{"segment_id": 0, "start_exchange_number": 0, 
"end_exchange_number": 5, "num_exchanges": 6}
{"segment_id": 1, "start_exchange_number": 6, 
"end_exchange_number": 8, "num_exchanges": 3}
...
</segmentation>
```
## Segment Rubric
{{segment_rubric}}
## Segment Examples
{{segment_examples}}
# Data
{{text_to_be_segmented}}
# Question
## Please generate the segmentation result from the input data that 
meets the following requirements:
- **No Missing Exchanges**:  Ensure that the exchange numbers cover 
all exchanges in the given conversation without omission. 
- **No Overlapping Exchanges**: Ensure that successive segments have 
no overlap in exchanges.
- **Accurate Counting**:  The sum of **num_exchanges**
across all segments should equal the total number of user-bot exchanges.
- **Utilize Segment Rubric**: Use the given segment rubric 
and examples to better segment.
- Provide your segmentation result between the tags:
<segmentation></segmentation>.
# Output
Now, provide the segmentation result based on the instructions above.
\end{verbatim}
\end{tcolorbox}
\caption{Prompt for GPT-4 segmentation (w/ reflection).}
\label{fig: prompt4seg}
\end{figure}
\begin{figure}[htbp]
    \small
    \vspace{-5mm}
    \begin{tcolorbox}[left=3pt,right=3pt,top=3pt,bottom=3pt,title=\textbf{Segmentation rubric learned from \textit{TIAGE}}]
\begin{itemize}
    \item Ensure segments encapsulate a complete thematic or topical exchange before initiating a new segment. This includes recognizing when a topic shift is part of the same thematic exchange and should not trigger a new segment.

    \item Segments should not only capture the flow of conversation by recognizing subtle topic shifts but also ensure that related questions and answers, or setup and response exchanges, are included within the same segment to preserve the natural flow and context of the dialogue.

    \item Maintain the integrity of conversational dynamics, ensuring that exchanges which include setup and response (or question and answer) are not divided across segments. This preserves the context and flow of the dialogue, recognizing that some topic shifts, while apparent, are part of a larger thematic discussion.

    \item Segments must accurately reflect the thematic depth of the conversation, ensuring that all parts of a thematic exchange, including indirect responses or tangentially related comments, are grouped within the same segment to maintain conversational coherence.

    \item Evaluate the conversational cues and context to determine the thematic linkage between exchanges. Avoid creating new segments for responses that, while seemingly off-topic, are contextually related to the preceding messages, ensuring a coherent and unified thematic narrative.

    \item Prioritize the preservation of conversational momentum when determining segment boundaries, ensuring that the segmentation does not interrupt the natural progression of dialogue or the development of thematic elements, even when the conversation takes unexpected turns.

    \item Assess the thematic relevance of each conversational turn, ensuring segments are not prematurely divided by superficial topic changes that are part of a broader thematic dialogue. This includes recognizing when a seemingly new topic is a direct continuation or an elaboration of the previous exchange, thereby maintaining thematic coherence and conversational flow.

    \item Consider the conversational and thematic continuity over superficial changes in topic or structure when segmenting conversations. This ensures that segments reflect the natural flow and thematic integrity of the dialogue, even when the conversation takes subtle turns.

    \item Incorporate flexibility in segment boundaries to accommodate for the natural ebb and flow of conversational topics, ensuring that segments are not overly fragmented by minor topic shifts that remain within the scope of the overarching thematic dialogue.

    \item Avoid over-segmentation by recognizing the thematic bridges between conversational turns. Even when a conversation appears to shift topics, if the underlying theme or narrative purpose connects the exchanges, they should be considered part of the same segment to preserve the dialogue's natural progression and thematic integrity.
\end{itemize}
\end{tcolorbox}
\caption{Segmentation rubric learned on \textit{TIAGE}~\citep{xie2021tiage}.}
\label{fig: segmentation_rubric_tiage}
\end{figure}

\begin{figure}[htbp]
    \small
    \vspace{-5mm}
    \begin{tcolorbox}[left=3pt,right=3pt,top=3pt,bottom=3pt,title=\textbf{Segmentation rubric learned from \textit{SuperDialSeg}}]
\begin{itemize}
    \item Segmentation should reflect natural pauses or shifts in the conversation, indicating a change in topic or focus.
    \item Each segment should aim to be self-contained, providing enough context for the reader to understand the topic or question being addressed without needing to refer to other segments.
    \item Ensure segmentation captures the full scope of a thematic exchange, using linguistic cues and conversational context to guide the identification of natural breaks or transitions in dialogue.
    \item Segmentation should prioritize thematic continuity over structural cues alone, ensuring that all parts of a thematic exchange, including follow-up questions or clarifications, are contained within the same segment.
    \item Segments must ensure logical and thematic coherence, grouping together all elements of an exchange that contribute to a single topic or question, even if the conversation appears structurally disjointed.
    \item Ensure segments maintain thematic progression, especially in conversations where multiple inquiries and responses explore different facets of the same overarching topic.
    \item Segmentation should avoid over-segmentation by ensuring that a series of inquiries and responses that explore different aspects of a single overarching topic are grouped within the same segment, even if they contain multiple question-answer pairs.
    \item Ensure that segments are not prematurely divided based on superficial structural cues like greetings or sign-offs, but rather on the substantive thematic content of the exchange.
    \item Ensure segmentation recognizes and preserves the thematic progression within a conversation, even when minor topic shifts occur, by evaluating the overall context and goal of the exchange rather than segmenting based on immediate linguistic cues alone.
    \item Ensure that segments accurately reflect the inquiry-response cycle, grouping all related questions and their corresponding answers into a single segment to preserve the flow and coherence of the conversation.
\end{itemize}
\end{tcolorbox}
\caption{Segmentation rubric learned on \textit{SuperDialSeg}~\citep{jiang2023superdialseg}.}
\label{fig: segmentation_rubric_superseg}
\end{figure}
\begin{figure}[htbp]
    \small
\begin{tcolorbox}[left=3pt,right=3pt,top=3pt,bottom=3pt,title=\textbf{Ground-truth Segment:}]
\begin{itemize}
\item \textbf{Segment 0: }
hello, how are you doing?
hello. pretty good, thanks. and yourself?
awesome, i just got back from a bike ride.
cool! do you spend a lot of time biking?
yup. its my favorite thing to do. do you?
i love playing folk music. i actually hope to be a professional musician someday
that is interesting. what instruments do you play?
i can play the guitar and the piano and i also like to sing.
i can only sing when i drink, but i do not like to do that anymore.
\item \textbf{Segment 1: }
i m not a big drinker either. do you have a job?
construction, like my dad. what do you do when you are not being a rock star
nice! i work as a custodian. not too glamorous but it pays the bills haha
i feel ya. you gotta do what you gotta do.
exactly. do you have other hobbies besides biking?
\end{itemize}
\end{tcolorbox}
\begin{tcolorbox}[left=3pt,right=3pt,top=3pt,bottom=3pt,title=\textbf{Predicted Segment:}]
\begin{itemize}
\item \textbf{Segment 0: }
hello, how are you doing?
hello. pretty good, thanks. and yourself?
awesome, i just got back from a bike ride.
\item \textbf{Segment 1: }
cool! do you spend a lot of time biking?
yup. its my favorite thing to do. do you?
i love playing folk music. i actually hope to be a professional musician someday

\item \textbf{Segment 2: }
that is interesting . what instruments do you play?
i can play the guitar and the piano and i also like to sing.
i can only sing when i drink, but i do not like to do that anymore.

\item \textbf{Segment 3: }
i m not a big drinker either. do you have a job?
construction, like my dad. what do you do when you are not being a rock star
nice! i work as a custodian. not too glamorous but it pays the bills haha

    \item \textbf{Segment 4: }
i feel ya. you gotta do what you gotta do.
exactly. do you have other hobbies besides biking?
\end{itemize}
\end{tcolorbox}
\caption{An example of poor segmentation from GPT-4 zero-shot segmentation illustrates that the GPT-4 powered segmentation model favors a more fine-grained segmentation. The Window Diff metric between the ground truth and the prediction is 0.80.}
\label{fig: poorly_segmented_example}
\end{figure}

\subsection{Additional Cost Analysis}
\label{sec: cost}

\section{Supplementary Materials for Computational Cost Analysis}\label{sec:detail_cost}

\begin{table}[!h]
    \centering
    \caption{\textbf{Total training times of different methods in semi-supervised settings.} All recorded experiment times are based on a single NVIDIA H100-80G GPU.}
    \resizebox{\linewidth}{!}{

      \begin{tabular}{cc|ccccccccc}
       \toprule
       \rowcolor{COLOR_MEAN}  \textbf{Type} & \textbf{Method} & \textbf{Cora} & \textbf{Citeseer} & \textbf{Pubmed} & \textbf{WikiCS} & \textbf{Instagram} & \textbf{Reddit} & \textbf{Books} & \textbf{Photo} & \textbf{Computer} \\ \midrule
       \multicolumn{2}{c|}{\# Training Samples} & 140 & 120 & 60 & 580 & 1,160 & 3,344 & 4,155 & 4,836 & 8,722 \\ \midrule
       \multirow{5}{*}{\textbf{Classic}} & GCN$_{\text{ShallowEmb}}$ & 2.8s & 2.7s & 2.7s & 3.2s & 1.8s & 7.7s & 12.7s & 13.8s & 33.5s \\ 
       & {GAT$_{\text{ShallowEmb}}$} & 1.9s & 2.4s & 3.8s & 3.3s & 2.0s & 6.0s & 10.5s & 12.3s & 38.0s \\ 
       & {SAGE$_{\text{ShallowEmb}}$} & 1.9s & 3.9s & 5.0s & 2.9s & 1.8s & 6.4s & 16.9s & 21.1s & 33.3s \\ 
       & {SenBERT-66M} & 8.5s & 7.9s & 5.9s & 27.9s & 14.7s & 1.2m & 1.5m & 1.8m & 3.3m \\ 
       & {RoBERTa-355M} & 21.2s & 18.9s & 12.7s & 1.2m & 2.3m & 6.5m & 8.1m & 3.8m & 6.9m \\ \midrule 

      \multirow{2}{*}{\textbf{Encoder}} & GCN$_{\text{LLMEmb}}$ & 1.2m & 1.4m & 13.4m & 7.4m & 4.5m & 16.0m & 23.5m & 26.8m & 44.7m \\ 
      & ENGINE  & 2.2m & 2.4m & 16.1m & 15.2m & 9.3m & 22.9m & 31.1m & 38.8m & 1.1h \\ \midrule

      \textbf{Reasoner} & TAPE & 25.5m & 27.8m & 5.6h & 2.7h & 2.0h & 8.0h & 9.9h & 11.7h & 14.5h \\ \midrule

      \multirow{3}{*}{\textbf{Predictor}} & {LLM$_{\text{IT}}$} & 25.6m & 22.0m & 3.9m & 1.1h & 49.1m & 1.1m & 2.0h & 2.4h & 2.7h \\ 
      & GraphGPT & 16.4m & 15.5m & 1.2h & 48.5m & 30.1m & 1.8h & 2.4h & 2.2h & 5.8h \\ 
       & LLaGA & 1.7m & 2.2m & 5.2m & 5.8m & 3.0m & 20.9m & 19.5m & 23.5m & 43.1m  \\    
        \bottomrule
      \end{tabular}
    }
    \label{tab:timecost}
\end{table}


\begin{table}[!h]
    \centering
    \caption{\textbf{Total training times of different methods in supervised settings.} All recorded experiment times are based on a single NVIDIA H100-80G GPU.}
    \resizebox{\linewidth}{!}{

      \begin{tabular}{cc|cccccccccc}
       \toprule
       \rowcolor{COLOR_MEAN}  \textbf{Type} & \textbf{Method} & \textbf{Cora} & \textbf{Citeseer} & \textbf{Pubmed} & \textbf{WikiCS} & \textbf{arXiv} & \textbf{Instagram} & \textbf{Reddit} & \textbf{Books} & \textbf{Photo} & \textbf{Computer} \\ \midrule
       \multicolumn{2}{c|}{\# Training Samples} & 1,624 & 1,911 & 11,830 & 7,020 &  90,941  & 6,803 & 20,060 & 24,930 & 29,017 & 52,337  \\ \midrule
       \multirow{5}{*}{\textbf{Classic}} & {GCN$_{\text{ShallowEmb}}$} & 1.8s & 1.7s & 5.2s & 5.1s & 51.2s & 19.5s & 8.5s & 14.9s & 19.7s & 25.8s \\ 
       & {GAT$_{\text{ShallowEmb}}$} & 2.1s & 1.9s & 7.9s & 5.7s & 1.5m & 2.7s & 6.9s & 16.6s & 28.0s & 44.6s \\ 
       & {SAGE$_{\text{ShallowEmb}}$} & 1.7s & 3.0s & 7.6s & 4.0s & 1.3m & 2.0s & 7.2s & 19.6s & 20.1s & 43.2s \\ 
       & {SenBERT-66M} &  35s & 41s & 2.6m & 2.5m & 7.4m & 1.2m & 4.4m & 1.8m & 2.2m & 4.1m \\ 
       & {RoBERTa-355M} & 1.3m & 1.6m & 9.2m & 5.5m & 40.8m & 5.3m & 15.9m & 9.7m & 11.9m & 22.4m \\ \midrule 

      \multirow{2}{*}{\textbf{Encoder}} & GCN$_{\text{LLMEmb}}$ & 1.2m & 1.4m & 13.4m & 7.5m & 1.4h & 4.5m & 16.1m & 23.6m & 26.8m & 44.8m  \\ 
      & ENGINE & 2.2m & 2.4m & 16.1m & 19.4m & 2.6h & 8.9m & 24.2m & 35.2m & 44.2m & 1.2h \\ \midrule

      \textbf{Reasoner} & TAPE & 27.4m & 30.3m & 5.9h & 2.8h & 37.4h &  2.1h & 8.3h & 10.0h & 12.0h & 15.0h \\ \midrule

      \multirow{3}{*}{\textbf{Predictor}} & {LLM$_{\text{IT}}$} & 1.0h & 1.3h & 9.9h & 4.2h & 36.3h & 2.7h & 3.4h & 5.7h & 7.4h & 12.4h \\ 
      & GraphGPT & 26.4m & 29.5m & 2.7h & 1.7h & 7.8h & 49.1m & 3.4h & 3.8h & 3.6h & 7.8h \\ 
       & LLaGA &  5.6m & 7.7m & 25.6m & 18.8m & 7.7h & 10.6m & 32.2m & 1.0h & 1.4h & 2.5h \\    
        \bottomrule
      \end{tabular}
    }
    \label{tab:timecost_supervised}
\end{table}


\begin{table}[!h]
    \centering
    \caption{\textbf{Inference times of different methods.} Values in brackets denote the average inference time per case in milliseconds (ms). All recorded experiment times are based on a single NVIDIA H100-80G GPU.}
    \resizebox{0.8\linewidth}{!}{
     \begin{tabular}{cc|ccccc}
     \toprule
        \rowcolor{COLOR_MEAN} \multicolumn{2}{c|}{\textbf{Method}} & \textbf{Cora} & \textbf{arXiv}  & \textbf{Instagram} & \textbf{Photo} & \textbf{WikiCS} \\ \midrule
        \multicolumn{2}{c|}{\# Test Samples} &  542 & 48,603 & 5,847 & 2,268 & 9,673 \\ \midrule
        \textbf{Classic} & GCN & 0.9ms & 21.8ms & 2.0ms & 7.5ms & 4.4ms \\ \midrule 
        \textbf{Encoder} & GCN$_{\text{LLMEmb}}$ & 14.0s (26ms) & 23.8m (29ms) & 53.6s (24ms) & 5.3m (33ms) & 3.7m (38ms) \\  \midrule
        \textbf{Reasoner} & TAPE & 5.0m (551ms) & 10.4h (767ms) & 23.7m (627ms) & 2.3h (863ms) & 1.3h (813ms) \\ \midrule
        \multirow{3}{*}{\textbf{Predictor}} & LLM$_{\text{IT}}$ & 1.2m (129ms) & 3.3h (243ms) & 2.7m (71ms) & 24.1m (149ms) & 5.8m (60ms) \\ 
        & GraphGPT & 1m (104ms) & 1.2h (87ms) & 2.0m (52ms) & 10.4m (64ms) & 11.0m (112ms) \\ 
        & LLaGA & 11.2s (21ms) & 57.1m (70ms) & 1.3m (35ms) & 4.4m (27ms) & 2.4m (25ms) \\
        \bottomrule
    \end{tabular}
    }
    \label{tab:inference_cost}
\end{table}



Table~\ref{tab: cost_analysis} compares the overall costs involved in memory construction, memory retrieval, and response generation across different methods. The results demonstrate that our method significantly enhances performance compared to the baseline while only slightly increasing computational overhead, and it outperforms the MemoChat method in both efficiency and effectiveness.


\subsection{The Analogy between the Reflection Augmentation and Prefix-Tuning}
When a small amount of conversation data with segment annotations is available, we explore how to leverage this data to transfer segmentation knowledge and better align the LLM-based segmentation model with human preferences. Inspired by the prefix-tuning technique~\citep{li2021prefix} and reflection mechanism~\citep{shinn2023reflexion,renze2024self}, we treat the segmentation prompt as the ``prefix'' and iteratively optimize it through LLM self-reflection, ultimately obtaining a segmentation guidance $\bm{G}$.

Prefix-tuning seeks to learn a prefix matrix $\bm{P}$ to boost the performance of the language model $\operatorname{LM}_{\phi}$ without fine-tuning its parameter $\phi$. The prefix matrix $\bm{P}$ is prepended to the activation $h$ of the Transformer layer: 
\begin{equation}
h_i= \begin{cases}\bm{P}[i,:], & \text { if } i \in \mathcal{P}_{idx} \\ \operatorname{LM}_\phi\left(z_i, h_{<i}\right), & \text { otherwise }\end{cases}
\end{equation}
where $\mathcal{P}_{idx}$ is the prefix indices.

In the context of our segmentation scenario, our goal is to ``learn'' a textual guidance $\boldsymbol{G}$ that directs the segmentation model toward improved segmentation outcomes. The process of updating the segmentation guidance $\boldsymbol{G}$ parallels the optimization of the prefix parameter $\bm{P}$ in prefix-tuning. Initially, the segmentation guidance $\boldsymbol{G}_{0}$ is set to empty, analogous to the initial prefix parameter $\bm{P}_{0}$. During each iteration of guidance updating, we first apply our conversation segmentation model in a zero-shot manner to a batch of conversation data. Building upon the insights that LLMs possess the ability for self-reflection and improvement~\citep{shinn2023reflexion, renze2024self}, we then instruct the segmentation model to reflect on its mistakes given the ground-truth segmentation and update the segmentation guidance $\boldsymbol{G}$. This process mirrors Stochastic Gradient Descent (SGD) optimization:
\begin{equation}
    \boldsymbol{G}_{m+1}=\boldsymbol{G}_m-\eta \nabla \mathcal{L}\left(\boldsymbol{G}_m\right), 
\end{equation}
where $\nabla \mathcal{L}\left(\boldsymbol{G}_m\right)$ denotes the gradient of segmentation loss, which we assume is estimated implicitly by the LLM itself and used to adjust the next segmentation guidance $\boldsymbol{G}_{m+1}$.


\subsection{Prompt for GPT-4 Evaluation}
\label{sec: prompt4eval}

We use the same evaluation prompts as  MemoChat~\citep{lu2023memochat}. The LLM-powered evaluation consists of single-sample scoring (GPT4Score) and pair-wise comparison. The evaluation prompts are displayed in Figure~\ref{fig: prompt4eval}. For pair-wise comparison, we alternate the order of the responses and conduct a second comparison for each pair to minimize position bias. 

\begin{figure*}[htbp]
\small
\begin{tcolorbox}[left=3pt,right=3pt,top=3pt,bottom=3pt,title=\textbf{Single-Sample Score}]
\begin{verbatim}
You are an impartial judge. You will be shown Related
Conversation History, User Question and Bot Response.
```\nRelated Conversation History\nRCH\_0\n```
```\nUser Question\nUQ\_1\n```
```\nBot Response\nBR\_2\n```
Please evaluate whether Bot Response is faithful to the content of 
Related Conversation History to answer User Question. 
Begin your evaluation by providing a short explanation, 
then you must rate Bot Response on an integer rating of 1 to 
100 
by strictly following this format: 
<rating>an integer rating of 1 to 100</rating>.
\end{verbatim}
\end{tcolorbox}
\begin{tcolorbox}[left=3pt,right=3pt,top=3pt,bottom=3pt,title=\textbf{Pair-Wise Comparison}]
\begin{verbatim}
You are an impartial judge. You will be shown 
Related Conversation History, User Question and Bot Response.
```\nRelated Conversation History\nRCH_0\n```
```\nUser Question\nUQ_1\n```
```\nBot Response A\nBR_2\n```
```\nBot Response B\nBR_3\n```
Please evaluate whether Bot Response is faithful to the content of 
Related Conversation History to answer User Question. 
Begin your evaluation by 
providing a short explanation, 
then you must choose the better bot response by giving 
either A or B. 
If the two responses are the same, you can choose NONE:
<chosen>A (or B or NONE)</chosen>.
\end{verbatim}
\end{tcolorbox}
\caption{Prompt used in GPT-4 evaluation, following \citet{lu2023memochat}.}
\label{fig: prompt4eval}
\end{figure*}


\subsection{Evaluation Results on the Official QA Pairs of LOCOMO}
\label{sec: main_locomo2}
As \textit{LOCOMO}~\citep{maharana2024evaluating} released a subset containing QA pairs recently. To ensure reproducibility, we evaluate our method on these official QA pairs. Table~\ref{tab: main_locomo2} presents the evaluation results. The superiority of our \sysname\ is also evident on these QA pairs, demonstrating its superior effectiveness and robustness.

\begin{table*}[!t]
    \small
    \centering
    \setlength{\tabcolsep}{1mm}
    \caption{Performance comparison on the official question-answer pairs of \textit{LOCOMO} using MPNet retriever. All other settings remain the same as in Table~\ref{tab: main_results}. MemoChat~\citep{lu2023memochat} is not applicable in \textit{Mistral-7B-Instruct-v0.3} due to Mistral's inability to execute the ``Memo Writing'' step, as it often fails to generate a valid JSON response needed to construct the memory bank in \citet{lu2023memochat}.
    }
    \label{tab: main_locomo2}
    \resizebox{1\columnwidth}{!}{
    \begin{tabular}{l|cccccc|cc}
    \toprule
    
    \multirow{2}{*}{\textbf{Methods}} &  \multicolumn{6}{@{}c|}{{\bf QA Performance}} & \multicolumn{2}{@{}c}{{\ \bf Context Length}} \\
    \cmidrule (lr){2-7} \cmidrule (lr){8-9}
    & GPT4Score & BLEU & Rouge1 & Rouge2 & RougeL & BERTScore & \# Turns & \# Tokens \\
    \midrule
    \multicolumn{9}{@{}c}{ \textit{ GPT-35-Turbo } } \\
    \midrule
    Full History & 66.28 & 7.51 & 28.73 & 14.07 & 27.90 & 87.82 & 293 & 18,655 \\
    MemoChat & 75.77 & 11.28 & 32.91 & 18.82 & 29.78 & 87.98 & - & 1,159 \\
    \midrule
    Turn-Level & 81.52 & 11.91 & 36.00 & 19.59 & 34.99 & \textbf{88.64} & 55.00 & 3,026 \\
    Session-Level & 74.20 & 10.95 & 29.92 & 14.64 & 29.27 & 87.96 & 54.48 & 3,442 \\
    \midrule
    \sysname\ & \textbf{84.21} & \textbf{12.80} & \textbf{36.70} & \textbf{19.90} & \textbf{35.61} & 88.59 & 56.49 & 3,565 \\
    \midrule
    \multicolumn{9}{@{}c}{ \textit{ Mistral-7B-v0.3 } } \\
    \midrule
    Full History & 69.13 & 6.77 & 30.40 & 15.02 & 29.20& 87.29 & 293 & 18,655 \\
    \midrule
    Turn-Level & 78.82 & 10.09 & 32.75 & 16.25 & 31.75 & \textbf{87.97} & 55.00 & 3,026 \\
    Session-Level & 62.68 & 7.37 & 26.68 & 12.38 & 25.86 & 86.98 & 54.48 & 3,442 \\
    \midrule
    \sysname\ & \textbf{80.07} & \textbf{10.67} & \textbf{32.82} & \textbf{16.65} & \textbf{31.81} & 87.87 & 56.49 & 3,565 \\
    \bottomrule
    \end{tabular}
    }
\end{table*}

\subsection{Case Study}
\label{sec: case_study}

To further demonstrate the advantages of our method, we conduct a qualitative evaluation. Figure~\ref{fig: case_study_segment_vs_turn} presents a specific case comparing the segment-level memory with the turn-level memory. It demonstrates that using turn-level memory units fails to address the user's request. We attribute this to the fragmentation of user-agent turns, and the critical turns may not explicitly contain or relate to the keywords in the user's request.

Similarly, using session-level memory units is also sub-optimal, as illustrated in Figure~\ref{fig: case_study_segment_vs_session}. This issue arises because a session often includes multiple topics, introducing a significant amount of irrelevant information that hampers effective retrieval. The irrelevant information also distracts the LLM, as noted in previous studies~\citep{shi2023large, liu2024lost}. 

We also conduct a case study to compare our method with two recent, powerful memory management techniques: \textit{RecurSum}~\citep{wang2023recursively} and \textit{ConditionMem}~\citep{yuan2023evolving}, as shown in Figure~\ref{fig: case_study_segment_vs_rsum} and Figure~\ref{fig: case_study_segment_vs_condmem}. The results indicate that the summarization process in these methods often omits detailed information that is essential for accurately answering the user's request.

\begin{figure}[htb]
\small
\begin{tcolorbox}[left=3pt,right=3pt,top=3pt,bottom=3pt,title=\textbf{Conversation History:}]
[human]: Craft an intriguing opening paragraph for a fictional short story. The story should involve a character who wakes up one morning to find that they can time travel.

...(Human-Bot Dialogue Turns)... \textcolor{blue}{(Topic: Time-Travel Fiction)}

[human]: Please describe the concept of machine learning. Could you elaborate on the differences between supervised, unsupervised, and reinforcement learning? Provide real-world examples of each.

...(Human-Bot Dialogue Turns)... \textcolor{blue}{(Topic: Machine learning Concepts and Types)}


[human]: Discuss antitrust laws and their impact on market competition. Compare the antitrust laws in US and China along with some case studies

...(Human-Bot Dialogue Turns)... \textcolor{blue}{(Topic: Antitrust Laws and Market Competition)}

[human]: The vertices of a triangle are at points (0, 0), (-1, 1), and (3, 3). What is the area of the triangle?

...(Human-Bot Dialogue Turns)... \textcolor{blue}{(Topic: Area and Circumcircle of Triangle)}

...(Human-Bot Dialogue Interaction on Other Topics)... 

\end{tcolorbox}
\begin{tcolorbox}[left=3pt,right=3pt,top=3pt,bottom=3pt,title=\textbf{Question:}]
What title reflects the thematic essence of the Doctor's experience with time travel as described in the story?
\end{tcolorbox}
\begin{tcolorbox}[left=3pt,right=3pt,top=3pt,bottom=3pt,title=\textbf{Ground-Truth Answer:}]
Awakening to Eternity: The Doctor's New Dawn
\end{tcolorbox}
\begin{tcolorbox}[left=3pt,right=3pt,top=3pt,bottom=3pt,title=\textbf{Retrieved Memory (Turn-Level):}]
[human]: Describe the process and write out the balanced chemical equation for the reaction that occurs when solid calcium carbonate reacts with hydrochloric acid to form aqueous calcium chloride, carbon dioxide, and water. What type of reaction is this, and what observations might indicate that the reaction is taking place?

[bot]: When solid calcium carbonate ($\text{CaCO}_{3}$) reacts with hydrochloric acid (HCl), the products formed are aqueous calcium chloride ($\text{CaCl}_{2}$), carbon dioxide ($\text{CO}_{2}$), and water ($\text{H}_{2}\text{O}$). The balanced chemical equation for this reaction is: ... \textcolor{blue}{(Topic: Calcium Carbonate and Hydrochloric Acid Reaction)}

\textcolor{brown}{
[human]: Write a title for your story.}

\textcolor{brown}{
[bot]: "Awakening to Eternity: The Doctor's New Dawn".
} \textcolor{blue}{(Topic: Time-Travel Fiction)}

[human]: Analyze the following customer reviews from different sources for three different smartphones - the latest iPhone, Samsung Galaxy, and Google Pixel - and provide an overall rating for each phone on a scale of 1 to 10. Consider the following complex and contradictory reviews: - TechRadar's review of the latest iPhone: The new iPhone is a stunning triumph of engineering that sets a new bar for smartphone performance and camera quality. However, the incremental design and high price mean it lacks the 'wow' factor of previous iPhones. Still, its power and intelligence are unrivaled. - CNET's review of the latest Samsung Galaxy: ... Return the answer as a JSON object with the overall ratings for each phone out of 10, to one decimal place.

[bot]: ```json { "iPhone": 8.5, "Samsung Galaxy": 8.0, "Google Pixel": 7.5 } ```

\textcolor{blue}{(Topic: Smartphone Review Ratings Analysis)}
\end{tcolorbox}
\begin{tcolorbox}[left=3pt,right=3pt,top=3pt,bottom=3pt,title=\textbf{Answer (Turn-Level Retrieval):}]
As an AI language model, I do not have access to the story you mentioned earlier. Please provide me with the story or the necessary details so that I can assist you better.

\textbf{\textcolor{red}{(The next part of this figure~\ref{fig: case_study_segment_vs_turn} will be continued on the next page.)}}
\end{tcolorbox}
\end{figure}

\begin{figure}[htb]
% \ContinuedFloat
\small
\begin{tcolorbox}[left=3pt,right=3pt,top=3pt,bottom=3pt,title=\textbf{Retrieved Memory (Ours):}]
[human]: Craft an intriguing opening paragraph for a fictional short story. The story should involve a character who wakes up one morning to find that they can time travel.

[bot]: In the hushed pre-dawn hours, when the world still slumbers under the cloak of night, Alex found themselves caught in the fragile boundary between sleep and wakefulness. ...

[human]: Summarize the story with three bullet points using only nouns and adjectives, without verbs.

[bot]: - Alex, pre-dawn awakening, time-travel discovery - Disorientation, vertigo, reality's malleability - Sunlight, new day, astonishing truth

[human]: Is there any famous fictional character that can time travel? If so, can you rewrite your story with him or her?

[bot]: - \textcolor{brown}{Doctor Who, a renowned time-traveling character from British television, is a perfect fit for this scenario.} In the silent embrace of early morning, where shadows merge with the fading night, the Doctor found themselves in the thin veil separating dreams from consciousness. It was here, amidst the soft whispers of time, that an unusual sensation enveloped them—a sense of time's fabric slipping through their fingers like grains of sand. As dawn's light broke through the darkness, painting the room in hues of gold and amber, the Doctor's eyes opened to what appeared to be an ordinary day. \textcolor{brown}{However, upon standing, a wave of dizziness overwhelmed them, the surroundings blurred, and in that moment of confusion, the Doctor realized a profound truth: they had not only awakened to a new dawn but had also stumbled upon the incredible ability to navigate the currents of time at will.}

\textcolor{brown}{[human]: Write a title for your story.}

\textcolor{brown}{[bot]: "Awakening to Eternity: The Doctor's New Dawn"} \textcolor{blue}{(Topic: Time-Travel Fiction)}
\end{tcolorbox}
\begin{tcolorbox}[left=3pt,right=3pt,top=3pt,bottom=3pt,title=\textbf{Answer (Ours):}]
The title "Awakening to Eternity: The Doctor's New Dawn" reflects the thematic essence of the Doctor's experience with time travel as described in the story. It captures the idea of a new beginning, a sense of eternity, and the awakening of a new power that will change the course of the Doctor's life forever.
\end{tcolorbox}
\caption{Case study comparing our \sysname\ with the \textit{turn-level} memory baseline. When constructing the memory bank at the \textit{turn-level}, the agent fails to provide a satisfactory response due to fragmentation: the retrieved dialogue turns can not capture a complete human-agent interaction related to the query, missing critical information highlighted in \textcolor{brown}{brown}. The dialogue topic is emphasized in \textcolor{blue}{blue}. Some context is omitted (replaced by ...) due to length constraints.}
\label{fig: case_study_segment_vs_turn}
\end{figure}

\newpage

\begin{figure}[htbp]
\small
\begin{tcolorbox}[left=3pt,right=3pt,top=3pt,bottom=3pt,title=\textbf{Conversation History:}]
[human]: Explain what's base rate fallacy and list five specific examples of ...

... (Human-Bot Dialogue Turns) ... \textcolor{blue}{(Topic: Base Rate Fallacy In Politics)}


[human]: How many integers are in the solution of the inequality $|x + 5| < 10$?

... (Human-Bot Dialogue Turns) ... \textcolor{blue}{(Topic: Number of Integers in Inequalities)}

[human]: How have the Alps and Rhine River influenced settlement and agriculture in Western Europe?

... (Human-Bot Dialogue Turns) ... \textcolor{blue}{(Topic: Alps and Rhine River Impact)}

[human]: In a survey conducted at a local high school, preferences for a new school color were measured: 58\% of students liked the color blue, 45\% preferred green, and 22\% liked both colors. If we randomly pick a student from the school, what's the probability that they would like neither blue nor green?

... (Human-Bot Dialogue Turns) ... \textcolor{blue}{(Topic: Probability of Liking Colors in Survey)}

[human]: Imagine you are participating in a race with a group of people. If you have just overtaken the second person, what's your current position? Where is the person you just overtook?

... (Human-Bot Dialogue Turns) ... \textcolor{blue}{(Topic: Overtaking Positions in Race)}
\end{tcolorbox}
\begin{tcolorbox}[left=3pt,right=3pt,top=3pt,bottom=3pt,title=\textbf{Question:}]
[human]: In the inequality question, how many positive integers are in the solution of the second inequality?
\end{tcolorbox}
\begin{tcolorbox}[left=3pt,right=3pt,top=3pt,bottom=3pt,title=\textbf{Ground-Truth Answer:}]
In the second inequality, $(|x + 10| < 5)$, the solution range is $(-15 < x < -5)$. Within this range, there are no positive integers, as all values of $(x)$ that satisfy the inequality are negative.
\end{tcolorbox}
\begin{tcolorbox}[left=3pt,right=3pt,top=3pt,bottom=3pt,title=\textbf{Retrieved Memory (Session-Level):}]
[human]: Explain what's base rate fallacy and list five specific examples of ...

... (Human-Bot Dialogue Turns) ... \textcolor{blue}{(Topic: Base Rate Fallacy In Politics)}

[human]: How many integers are in the solution of the inequality $|x + 5| < 10$?

[bot]: To solve the inequality $|x + 5| < 10$, ... 

\textcolor{brown}{
[human]: What about $|x + 10| < 5$?
}

\textcolor{brown}{
[bot]: To solve the inequality $|x + 10| < 5$, we ...} \textcolor{gray}{(``overwhelmed'' by other irrelevant, redundant content)}

... (Human-Bot Dialogue Turns) ...  \textcolor{blue}{(Topic: Number of Integers in Inequalities)}

% \textcolor{brown}{
% [human]: What about $|x + 10| < 5$
% }

% \textcolor{brown}{
% [bot]: To solve the inequality $|x + 10| < 5$, ...
% } \textcolor{blue}{(Topic: Number of Integers in Inequalities)}

[human]: How have the Alps and Rhine River influenced settlement and agriculture in Western Europe?

... (Human-Bot Dialogue Turns) ... \textcolor{blue}{(Topic: Alps and Rhine River Impact)}

[human]: In a survey conducted at a local high school, ...

... (Human-Bot Dialogue Turns) ... \textcolor{blue}{(Topic: Probability of Liking Colors in Survey)}
\end{tcolorbox}
\begin{tcolorbox}[left=3pt,right=3pt,top=3pt,bottom=3pt,title=\textbf{Answer (Session-Level Retrieval):}]
I apologize for the confusion earlier. To answer your question, the second inequality was not provided in our previous conversation. If you could provide the inequality, I would be happy to help you find the solution and the number of positive integers in the solution set.

\textbf{\textcolor{red}{(The next part of this figure~\ref{fig: case_study_segment_vs_session} will be continued on the next page.)}}
\end{tcolorbox}
\end{figure}
\begin{figure}[htbp]
% \ContinuedFloat
\small
\begin{tcolorbox}[left=3pt,right=3pt,top=3pt,bottom=3pt,title=\textbf{Retrieved Memory (Ours):}]
[human]: How many integers are in the solution of the inequality $|x + 5| < 10$?

[bot]: To solve the inequality $|x + 5| < 10$, we break it into two cases based on the definition of absolute value: 
1. $x + 5 < 10$ 
2. $-(x + 5) < 10$, which simplifies to $x + 5 > -10$ 
Solving each case: 
1. $x < 5$ 
2. $x > -15$ 
Combining these, we get $-15 < x < 5$. This inequality means that $x$ can be any number greater than $-15$ and less than $5$. To find how many integers satisfy this condition, we list them: $-14, -13, -12, -11, -10, -9, -8, -7, -6, -5, -4, -3, -2, -1, 0, 1, 2, 3, 4$. There are 19 integers that satisfy the inequality $|x + 5| < 10$.

\textcolor{brown}{
[human]: What about $|x + 10| < 5$?
}

\textcolor{brown}{
[bot]: To solve the inequality $|x + 10| < 5$, we again consider two cases: 
1. $x + 10 < 5$ 
2. $-(x + 10) < 5$, which simplifies to $x + 10 > -5$ 
Solving each case: 
1. $x < -5$ 
2. $x > -15$ 
Combining these, we get $-15 < x < -5$. This inequality means that $x$ can be any number greater than $-15$ and less than $-5$. To find how many integers satisfy this condition, we list them: $-14, -13, -12, -11, -10, -9, -8, -7, -6$. There are 9 integers that satisfy the inequality $|x + 10| < 5$.
}

[human]: What about $|x + 10| < 10$?

[bot]: To solve the inequality $|x + 10| < 10$, we consider two cases: 
1. $x + 10 < 10$ 
2. $-(x + 10) < 10$, which simplifies to $x + 10 > -10$ 
Solving each case: 
1. $x < 0$ 
2. $x > -20$ 
Combining these, we get $-20 < x < 0$. This inequality means that $x$ can be any number greater than $-20$ and less than $0$. To find how many integers satisfy this condition, we list them: $-19, -18, -17, -16, -15, -14, -13, -12, -11, -10, -9, -8, -7, -6, -5, -4, -3, -2, -1$. There are 19 integers that satisfy the inequality $|x + 10| < 10$. \textcolor{blue}{(Topic: Number of Integers in Inequalities)}
\end{tcolorbox}
\begin{tcolorbox}[left=3pt,right=3pt,top=3pt,bottom=3pt,title=\textbf{Answer (Ours):}]
To find the number of positive integers that satisfy the inequality $(|x + 10| < 5)$, we need to look for positive integers in the solution set $(-15 < x < -5)$. Since there are no positive integers in this interval, the answer is 0.

\end{tcolorbox}
\caption{Case study comparing our \sysname\ with the \textit{session-level} memory baseline. When constructing the memory bank at the \textit{session-level}, the agent fails to provide a satisfactory response due to the inclusion of excessive irrelevant information within a session. This distracts the generator from focusing on the truly useful information highlighted in \textcolor{brown}{brown}. The dialogue topic is emphasized in \textcolor{blue}{blue}. Some context is omitted (replaced by ...) due to length constraints.}
\label{fig: case_study_segment_vs_session}
\end{figure}



\begin{figure}[htb]
\small
\begin{tcolorbox}[left=3pt,right=3pt,top=3pt,bottom=3pt,title=\textbf{Conversation History:}]
[human]: Photosynthesis is a vital process for life on Earth. Could you outline the two main stages of photosynthesis, including where they take place within the chloroplast, and the primary inputs and outputs for each stage? ... (Human-Bot Dialogue Turns)... \textcolor{blue}{(Topic: Photosynthetic Energy Production)}

[human]: Please assume the role of an English translator, tasked with correcting and enhancing spelling and language. Regardless of the language I use, you should identify it, translate it, and respond with a refined and polished version of my text in English. 

... (Human-Bot Dialogue Turns)...  \textcolor{blue}{(Topic: Language Translation and Enhancement)}

[human]: Suggest five award-winning documentary films with brief background descriptions for aspiring filmmakers to study.

\textcolor{brown}{[bot]: ...
5. \"An Inconvenient Truth\" (2006) - Directed by Davis Guggenheim and featuring former United States Vice President Al Gore, this documentary aims to educate the public about global warming. It won two Academy Awards, including Best Documentary Feature. The film is notable for its straightforward yet impactful presentation of scientific data, making complex information accessible and engaging, a valuable lesson for filmmakers looking to tackle environmental or scientific subjects.}

... (Human-Bot Dialogue Turns)... 
\textcolor{blue}{(Topic: Documentary Films Recommendation)}

[human]: Given the following records of stock prices, extract the highest and lowest closing prices for each month in the year 2022. Return the results as a CSV string, with one line allocated for each month. Date,Open,High,Low,Close,Volume ... ... (Human-Bot Dialogue Turns)...  \textcolor{blue}{(Topic: Stock Prices Analysis)}

[human]: The city of Vega intends to build a bridge that will span the Vegona River, covering a distance of 1.8 kilometers. The proposed location falls within a seismically active area that has experienced several high-magnitude earthquakes. Given these circumstances, what would be the best approach to constructing the bridge? ... (Human-Bot Dialogue Turns)...  \textcolor{blue}{(Topic: Seismic Bridge Construction)}
\end{tcolorbox}
\begin{tcolorbox}[left=3pt,right=3pt,top=3pt,bottom=3pt,title=\textbf{Question:}]
Which documentary won two Academy Awards and what lesson does it offer to filmmakers?
\end{tcolorbox}
\begin{tcolorbox}[left=3pt,right=3pt,top=3pt,bottom=3pt,title=\textbf{Ground-Truth Answer:}]
``An Inconvenient Truth" (2006) won two Academy Awards. It offers the lesson of making complex information accessible and engaging, a valuable lesson for filmmakers looking to tackle environmental or scientific subjects.
\end{tcolorbox}
\begin{tcolorbox}[left=3pt,right=3pt,top=3pt,bottom=3pt,title=\textbf{Memory generated by RecurSum~\citep{wang2023recursively}:}]
The dialogue provided detailed information on various topics, including the stages of photosynthesis, energy production estimates through photosynthesis for different organisms, translations and enhancements of phrases in different languages, recommendations for documentary films, a pitch for a film about overcoming adversity, analysis of stock prices, and considerations for constructing a bridge in a seismically active area. Here's a summary of the key points: ...

\textbf{Documentary Film Recommendations}: Suggested five award-winning documentary films for aspiring filmmakers to study, with brief background descriptions for each. The response included:

-  "March of the Penguins" (2005) - Emperor penguins' journey in Antarctica; noted for its cinematography and portrayal of the penguins' life cycle.

-  "Citizenfour" (2014) - Edward Snowden and NSA spying scandal; known for its real-time exploration of Snowden's revelations.

-  "20 Feet from Stardom" (2013) - Stories of backup singers behind musical legends; celebrates unsung heroes in the music industry.

-  "Man on Wire" (2008) - Philippe Petit's 1974 high-wire walk between the Twin Towers; combines personal footage, interviews, and re-enactments.

-  \textcolor{brown}{"An Inconvenient Truth" (2006) - Al Gore's documentary on global warming; known for its presentation of scientific data.}
\end{tcolorbox}
\begin{tcolorbox}[left=3pt,right=3pt,top=3pt,bottom=3pt,title=\textbf{Answer (RSum):}]
I'm sorry, but based on the updated context you provided, I don't see any previous discussion related to the question you asked. However, to answer your question, the documentary that won two Academy Awards is "Free Solo," which won for Best Documentary Feature and Best Achievement in Film Editing in 2019. 
% The lesson it offers to filmmakers is the importance of taking risks and pushing boundaries in order to create something truly unique and impactful.
....
The film follows climber Alex Honnold as he attempts to free solo climb El Capitan in Yosemite National Park, and the filmmakers had to take risks themselves in order to capture the footage and tell the story.

\textbf{\textcolor{red}{(The next part of this figure~\ref{fig: case_study_segment_vs_rsum} will be continued on the next page.)}}
\end{tcolorbox}
\end{figure}
\begin{figure}[htbp]
% \ContinuedFloat
\small
\begin{tcolorbox}[left=3pt,right=3pt,top=3pt,bottom=3pt,title=\textbf{Retrieved Memory (Ours):}]
[human]: Suggest five award-winning documentary films with brief background descriptions for aspiring filmmakers to study.

[bot]: 1. **"March of the Penguins" (2005)** - Directed by Luc Jacquet, this French feature-length nature documentary delves into the annual journey of Emperor penguins across the Antarctic. It won the Academy Award for Best Documentary Feature. The film is celebrated for its stunning cinematography and its detailed, heartfelt portrayal of the penguins' life cycle, making it a must-study for filmmakers interested in wildlife documentaries. 2. **"Citizenfour" (2014)** ... 5. \textcolor{brown}{**"An Inconvenient Truth" (2006)** - Directed by Davis Guggenheim and featuring former United States Vice President Al Gore, this documentary aims to educate the public about global warming. It won two Academy Awards, including Best Documentary Feature.} The film is notable for its straightforward yet impactful presentation of scientific data, making complex information accessible and engaging, \textcolor{brown}{a valuable lesson for filmmakers looking to tackle environmental or scientific subjects.}

[human]: With the spirit in the first film, craft a succinct and persuasive pitch for a film about overcoming adversity.

[bot]: Drawing inspiration from the resilience and majestic journey of Emperor penguins in "March of the Penguins," our documentary, "Unyielding Spirits: The Human Odyssey," will explore the profound stories of individuals around the globe who have faced and overcome staggering adversities. Through breathtaking cinematography and intimate storytelling, we ...
\textcolor{blue}{(Topic: Documentary Films Recommendation)}
\end{tcolorbox}
\begin{tcolorbox}[left=3pt,right=3pt,top=3pt,bottom=3pt,title=\textbf{Answer (Ours):}]
``An Inconvenient Truth" won two Academy Awards, including Best Documentary Feature. The film offers a valuable lesson to filmmakers looking to tackle environmental or scientific subjects, as it presents complex information in a straightforward yet impactful way, making it accessible and engaging to a wider audience.


\end{tcolorbox}
\caption{Case study comparing our \sysname\ with RecurSum~\citep{wang2023recursively}. When using RecurSum, the agent fails to provide a satisfactory answer due to information loss during the summarization process. Summarization omits essential details needed to answer the user's new request, as highlighted in \textcolor{brown}{brown}. The dialogue topic is emphasized in \textcolor{blue}{blue}. Some context is omitted (replaced by ...) due to length constraints.}
\label{fig: case_study_segment_vs_rsum}
\end{figure}




\begin{figure}[htb]
\small
\begin{tcolorbox}[left=3pt,right=3pt,top=3pt,bottom=3pt,title=\textbf{Conversation History:}]
[human]: Photosynthesis is a vital process for life on Earth. Could you outline the two main stages of photosynthesis, including where they take place within the chloroplast, and the primary inputs and outputs for each stage? ... (Human-Bot Dialogue Turns)... \textcolor{blue}{(Topic: Photosynthetic Energy Production)}

[human]: Please assume the role of an English translator, tasked with correcting and enhancing spelling and language. Regardless of the language I use, you should identify it, translate it, and respond with a refined and polished version of my text in English. 

... (Human-Bot Dialogue Turns)...  \textcolor{blue}{(Topic: Language Translation and Enhancement)}

[human]: Suggest five award-winning documentary films with brief background descriptions for aspiring filmmakers to study.

\textcolor{brown}{[bot]: ...
5. \"An Inconvenient Truth\" (2006) - Directed by Davis Guggenheim and featuring former United States Vice President Al Gore, this documentary aims to educate the public about global warming. It won two Academy Awards, including Best Documentary Feature. The film is notable for its straightforward yet impactful presentation of scientific data, making complex information accessible and engaging, a valuable lesson for filmmakers looking to tackle environmental or scientific subjects.}

... (Human-Bot Dialogue Turns)... 
\textcolor{blue}{(Topic: Documentary Films Recommendation)}

[human]: Given the following records of stock prices, extract the highest and lowest closing prices for each month in the year 2022. Return the results as a CSV string, with one line allocated for each month. Date,Open,High,Low,Close,Volume ... ... (Human-Bot Dialogue Turns)...  \textcolor{blue}{(Topic: Stock Prices Analysis)}

[human]: The city of Vega intends to build a bridge that will span the Vegona River, covering a distance of 1.8 kilometers. The proposed location falls within a seismically active area that has experienced several high-magnitude earthquakes. Given these circumstances, what would be the best approach to constructing the bridge? ... (Human-Bot Dialogue Turns)...  \textcolor{blue}{(Topic: Seismic Bridge Construction)}
\end{tcolorbox}
\begin{tcolorbox}[left=3pt,right=3pt,top=3pt,bottom=3pt,title=\textbf{Question:}]
Which documentary won two Academy Awards and what lesson does it offer to filmmakers?
\end{tcolorbox}
\begin{tcolorbox}[left=3pt,right=3pt,top=3pt,bottom=3pt,title=\textbf{Ground-Truth Answer:}]
"An Inconvenient Truth" (2006) won two Academy Awards. It offers the lesson of making complex information accessible and engaging, a valuable lesson for filmmakers looking to tackle environmental or scientific subjects.
\end{tcolorbox}
\begin{tcolorbox}[left=3pt,right=3pt,top=3pt,bottom=3pt,title=\textbf{Memory generated by ConditionMem~\citep{yuan2023evolving}:}]
- \textcolor{brown}{The user is seeking recommendations for award-winning documentary films, along with brief background descriptions, for educational purposes.} This request likely stems from a desire to learn from successful examples in the documentary filmmaking field, possibly to improve their own filmmaking skills or to gain inspiration for their projects. 

- \textcolor{brown}{User is interested in award-winning documentary films for educational purposes, aiming to study their backgrounds and filmmaking techniques.}

- The user's request for crafting a succinct and persuasive pitch for a film about overcoming adversity is inspired by the spirit of "March of the Penguins," a documentary that beautifully captures the Emperor penguins' annual journey across the Antarctic, highlighting themes of endurance and survival against the odds. The user seeks to emulate the emotional depth and cinematic quality that made "March of the Penguins" compelling, aiming to create a pitch that resonates with similar themes of resilience and triumph over challenges. 

- User is interested in creating content that focuses on themes of overcoming adversity, inspired by the emotional depth and cinematic excellence of documentaries like "March of the Penguins." They value storytelling that captures the essence of endurance and survival against the odds.
% : The user's request to write a budget plan for the video production directly follows the task of creating a video script aimed at training new employees of a car wash business in Japan, focusing on business etiquette norms.
% The user is likely seeking to understand the financial requirements to bring the previously crafted script to life, indicating a move from the planning and content development phase to the practical aspects of production. 

... (Other Irrelevant Memory Records) ...
\end{tcolorbox}
\begin{tcolorbox}[left=3pt,right=3pt,top=3pt,bottom=3pt,title=\textbf{Answer (Conditional Memory):}]
The documentary that won two Academy Awards is "Free Solo," which follows climber Alex Honnold as he attempts to free solo climb El Capitan in Yosemite National Park. The film offers a lesson to filmmakers on the importance of capturing authentic and compelling human stories, as well as the technical challenges of filming in extreme environments.

\textbf{\textcolor{red}{(The next part of this figure~\ref{fig: case_study_segment_vs_condmem} will be continued on the next page.)}}
\end{tcolorbox}
\end{figure}
\begin{figure}[htbp]
% \ContinuedFloat
\small
\begin{tcolorbox}[left=3pt,right=3pt,top=3pt,bottom=3pt,title=\textbf{Retrieved Memory (Ours):}]
[human]: Suggest five award-winning documentary films with brief background descriptions for aspiring filmmakers to study.

[bot]: 1. **"March of the Penguins" (2005)** - Directed by Luc Jacquet, this French feature-length nature documentary delves into the annual journey of Emperor penguins across the Antarctic. It won the Academy Award for Best Documentary Feature. The film is celebrated for its stunning cinematography and its detailed, heartfelt portrayal of the penguins' life cycle, making it a must-study for filmmakers interested in wildlife documentaries. 2. **"Citizenfour" (2014)** ... 5. \textcolor{brown}{**"An Inconvenient Truth" (2006)** - Directed by Davis Guggenheim and featuring former United States Vice President Al Gore, this documentary aims to educate the public about global warming. It won two Academy Awards, including Best Documentary Feature.} The film is notable for its straightforward yet impactful presentation of scientific data, making complex information accessible and engaging, a valuable lesson for filmmakers looking to tackle environmental or scientific subjects.

[human]: With the spirit in the first film, craft a succinct and persuasive pitch for a film about overcoming adversity.

[bot]: Drawing inspiration from the resilience and majestic journey of Emperor penguins in "March of the Penguins," our documentary, "Unyielding Spirits: The Human Odyssey," will explore the profound stories of individuals around the globe who have faced and overcome staggering adversities. Through breathtaking cinematography and intimate storytelling, we ...
\textcolor{blue}{(Topic: Documentary Films Recommendation)}
\end{tcolorbox}
\begin{tcolorbox}[left=3pt,right=3pt,top=3pt,bottom=3pt,title=\textbf{Answer (Ours):}]
``An Inconvenient Truth" won two Academy Awards, including Best Documentary Feature. The film offers a valuable lesson to filmmakers looking to tackle environmental or scientific subjects, as it presents complex information in a straightforward yet impactful way, making it accessible and engaging to a wider audience.
\end{tcolorbox}
\caption{Case study comparing our \sysname\ with ConditionMem~\citep{yuan2023evolving}. When using ConditionMem, the agent fails to provide a satisfactory answer due to (1) information loss during the summarization process and (2) the incorrect discarding of turns that are actually useful, as highlighted in \textcolor{brown}{brown}. The dialogue topic is emphasized in \textcolor{blue}{blue}. Some context is omitted (replaced by ...) due to length constraints.}
\label{fig: case_study_segment_vs_condmem}
\end{figure}



\subsection{Details of Dataset Construction}
\label{sec: dataset_details}

(i) \textit{LOCOMO}~\citep{maharana2024evaluating}: this dataset contains the longest conversations to date, with an average of  more than 9K tokens per sample. Since \textit{LOCOMO} does not release the corresponding question-answer pairs when we conduct our experiment, we prompt GPT-4 to generate QA pairs for each session as in \citet{alonso2024toward}. We also conduct evaluation on the recently released official QA pairs in Appendix~\ref{sec: main_locomo2}.

(ii) \textit{Long-MT-Bench+}: \textit{Long-MT-Bench+} is reconstructed from the 
\textit{MT-Bench+}~\citep{lu2023memochat} dataset. In \textit{MT-Bench+}, human experts are invited to expand the original questions and create long-range questions as test samples. However, there are two drawbacks when using this dataset to evaluate the memory mechanism of conversational agents: (1) the number of QA pairs is relatively small, with only 54 human-written long-range questions; and (2) the conversation length is not sufficiently long, with each conversation containing an average of 13.3 dialogue turns and a maximum of 16 turns. In contrast, the conversation in \textit{LOCOMO} has an average of 300 turns and 9K tokens. To address (1), we use these human-written questions as few-shot examples and ask GPT-4 to generate a long-range test question for each dialogue topic. For (2), following~\citep{yuan2023evolving}, we merge five consecutive sessions into one, forming longer dialogues that are more suitable for evaluating memory in long-term conversation. We refer to the reconstructed dataset as \textit{Long-MT-Bench+} and present its statistics in Table~\ref{tab: datasets_statistics}.

\sisetup{table-text-alignment=center,table-format=6.0}
\begin{tabular}{lSSSSrrS}
\toprule
{\textbf{Attribute}} & {\textbf{Declarations}} & {\textbf{Users}} & {\textbf{Inactive Users}} & {\textbf{Ambiguous Users}} & \multicolumn{2}{c}{\textbf{Class Proportion}} & {\textbf{Subreddits}} \\
\midrule
Year of Birth & 420803 & 401390 & 1630 & 17341 & Old: 56.19\% & Young: 43.81\% & 9806 \\
Gender & 424330 & 403428 & 1634 & 18337 & Male: 50.89\% & Female: 49.11\% & 9809 \\
Partisan Affiliation & 6369 & 6118 & 4 & 251 & Dem.: 54.55\% & Rep.: 45.45\% & 9137 \\
\bottomrule
\end{tabular}


\subsection{Details of Retrieval Performance Measurement}
\label{sec: retrieval_measurement}

We measure the retrieval performance in terms of the discounted cumulative gain (DCG) metric~\citep{jarvelin2002cumulated}: 
\begin{equation} 
\textit{DCG}=\sum_{i=1}^{p}\frac{rel_{i}}{\log_{2}(i+1)}, 
\label{eq: dcg} 
\end{equation} 
where $rel_{i}$ denotes the relevance score of the retrieved user-agent turn ranked at position $i$, and $p$ represents the total number of retrieved turns. Note that in the \textit{Long-MT-Bench+} dataset, answering a single question often requires referring to several consecutive turns. Therefore, we distribute the relevance score evenly across these relevant turns and set the relevance score of irrelevant turns to zero. For instance, assume that the ground truth reference turn set for question $q$ is $\mathcal{R}(q) = \{r_{k+j}\}_{j=1}^{N}$, which is provided by the dataset. In this case, the relevance score for each turn is set as follows:
$$
\textit{rel}_{i} =
\begin{cases}
0 & i < k+1 \\ 
\frac{1}{N} & k+1 \leq i \leq k+N \\ 
0 & i > k+N
\end{cases}.
$$
This approach allows us to evaluate retrieval performance at different granularity.

\subsection{Additional Experiments on CoQA and Persona-Chat}

To further validate SeCom's robustness and versatility across a broader range of dialogue types, we conduct additional experiments on other benchmarks, \textbf{Persona-Chat}~\citep{zhang-etal-2018-personalizing} and \textbf{CoQA}~\citep{reddy2019coqa}.

Given the relatively short context length of individual samples in these datasets, we adopt an approach similar to Long-MT-Bench+ by aggregating multiple adjacent samples into a single instance. For CoQA, each sample is supplemented with the text passages of its 10 surrounding samples. Since CoQA answers are derived from text passages rather than dialogue turns, we replace the turn-level baseline with a sentence-level baseline.
For Persona-Chat, we utilize the expanded version provided by~\citet{jandaghi2023faithful}. Conversations are aggregated by combining each sample with its 5 surrounding samples. Following the next utterance prediction protocol, we include the personas of both conversational roles in the prompt. Due to the large scale of these datasets, we select subsets for experimentation. From CoQA, we randomly sample 50 instances from an initial pool of 500, resulting in a subset containing over 700 QA pairs. Similarly, for Persona-Chat, we randomly select 100 instances, encompassing over 1,000 utterances in total. 

As shown in Table~\ref{tab: results_on_coqa} and Table~\ref{tab: results_on_spc}, \sysname\ consistently outperforms baseline methods across these datasets, highlighting its effectiveness in handling diverse dialogue scenarios, including open-ended and multi-turn interactions.

\begin{table*}[!t]
    \small
    \centering
    % \setlength{\tabcolsep}{1mm}
    \caption{QA performance comparison on  \textit{CoQA} using MPNet-based retrieval model. The response generation model is \texttt{GPT-3.5-Turbo}. 
    }
    \label{tab: results_on_coqa}
    \resizebox{1\columnwidth}{!}{
    \begin{tabular}{l|cccccc|c}
    \toprule
    
    \textbf{Methods}        & \textbf{GPT4Score} & \textbf{BLEU} & \textbf{Rouge1} & \textbf{Rouge2} & \textbf{RougeL} & \textbf{BERTScore} & \textbf{\#Tokens} \\ 
    \midrule
    Sentence-Level & 95.55 & 36.02 & 48.58 & 37.96 & 47.03 & 90.01 & 993 \\ 
    Session-Level  & 91.58 & 31.22 & 47.18 & 37.32 & 45.92 & 89.65 & 3,305 \\ 
    \midrule
    ConditionMem   & 94.32 & 34.35 & 47.91 & 37.55 & 46.38 & 89.77 & 1,352 \\ 
    MemoChat       & 97.16 & 38.17 & 49.54 & 38.23 & 47.77 & 90.14 & 1,041 \\ 
    COMEDY         & 97.48 & 38.02 & 49.41 & 38.19 & 47.63 & 90.06 & 3,783 \\ 
    \midrule
    \sysname\ (Ours)   & \bf98.31 & \bf39.57 & \bf50.44 & \bf39.51 & \bf48.98 & \bf90.37 & 1,016 \\ 

    \bottomrule
    \end{tabular}
    }
\end{table*}
\begin{table*}[!t]
    \small
    \centering
    % \setlength{\tabcolsep}{1mm}
    \caption{Next utterance prediction performance comparison on  \textit{Persona-Chat} using MPNet-based retrieval model. The response generation model is \texttt{GPT-3.5-Turbo}. 
    }
    \label{tab: results_on_spc}
    \resizebox{1\columnwidth}{!}{
    \begin{tabular}{l|cccccc|cc}
    \toprule

    \multirow{2}{*}{\textbf{Methods}} &  \multicolumn{6}{@{}c|}{{\bf Performance}} & \multicolumn{2}{@{}c}{{\bf Context Length}} \\
    \cmidrule (lr){2-7} \cmidrule (lr){8-9}
    & GPT4Score & BLEU & Rouge1 & Rouge2 & RougeL & BERTScore & \# Turns & \# Tokens \\
    \midrule
    Turn-Level & 69.23 & 5.73 & 21.38 & 9.06 & 19.87 & 87.28 & 24.00 & 682 \\
    Session-Level & 67.35 & 5.45 & 21.80 & 8.86 & 20.04 & 87.34 & 116.91 & 3,593 \\
    \midrule
    ConditionMem & 73.21 & 6.16 & 22.52 & 9.88 & 20.95 & 87.44 & - & 1,388 \\ 
    MemoChat & 76.83 & 7.21 & 25.13 & 10.81 & 22.31 & 87.68 & - & 1,296 \\ 
    COMEDY & 76.52 & 7.05 & 24.97 & 10.54 & 22.18 & 87.60 & - & 3,931 \\ 
    \midrule
    \sysname\ (Ours) & \textbf{78.34} & \textbf{7.75} & \textbf{26.01} & \textbf{11.57} & \textbf{23.98} & \textbf{87.82} & 23.48 & 702 \\

    \bottomrule
    \end{tabular}
    }
\end{table*}

\subsection{Human Evaluation Results}
\label{sec: human_evaluation}

To ensure a holistic assessment, we conduct human evaluation to gauge the quality of the LLM's response in conversation. We adopt the human evaluation scheme of COMEDY~\citep{chen2024compress}, which encompasses five perspectives: Coherence, Consistency, Engagingness, Humanness and Memorability. Ten human annotators are asked to score the responses following a detailed rubric for each perspective. Results in Table~\ref{tab: human_evaluation} show that the rank of different methods from human evaluation is generally consistent with those obtained from automated metrics, confirming the practical effectiveness of our proposed approach. 

\begin{table*}[!t]
    \small
    \centering
    \caption{Human evaluation results on  \textit{Long-MT-Bench+} using MPNet-based retrieval model. The response generation model is \texttt{GPT-3.5-Turbo}. 
    }
    \label{tab: human_evaluation}
    \resizebox{1\columnwidth}{!}{
    \begin{tabular}{l|cccccc}
    \toprule
    
    \textbf{Methods} & \textbf{Coherence} & \textbf{Consistency} & \textbf{Memorability} & \textbf{Engagingness} & \textbf{Humanness} & \textbf{Average} \\ 
    \midrule
    Full-History & 1.55 & 1.11 & 0.43 & 0.33 & 1.85 & 1.05 \\
    Sentence-Level & 1.89 & 1.20 & 1.06 & 0.78 & 2.00 & 1.39 \\
    Session-Level & 1.75 & 1.25 & 0.98 & 0.80 & 1.92 & 1.34 \\
    \midrule
    ConditionMem & 1.58 & 1.08 & 0.57 & 0.49 & 1.77 & 1.10 \\
    MemoChat & 2.05 & 1.25 & 1.12 & 0.86 & \textbf{2.10} & 1.48 \\
    COMEDY & \textbf{2.20} & 1.28 & 1.20 & 0.90 & 1.97 & 1.51 \\
    \midrule
    \sysname\ (Ours) & 2.13 & \textbf{1.34} & \textbf{1.28} & \textbf{0.94} & 2.06 & 1.55 \\

    \bottomrule
    \end{tabular}
    }
\end{table*}

\subsection{Performance Using Smaller Segmentation Model}

To make our method applicable in resource-constrained environments, we conduct additional experiments by replacing the \texttt{GPT-4-Turbo} used for the segmentation model with the \texttt{Mistral-7B-Instruct-v0.3} and a \texttt{RoBERTa} based model fine-tuned on SuperDialseg~\citep{jiang2023superdialseg}. Table~\ref{tab: main_results_slm_seg} shows that \sysname\ maintains the advantage over baseline methods when switching from GPT-4 to Mistral-7B. Notably, even with a RoBERTa based segmentation model, \sysname\ retains a substantial performance gap over other granularity-based baselines.

\begin{table}[!t]
    \small
    \centering
    \caption{Performance comparison on \textit{LOCOMO} and \textit{Long-MT-Bench+} using different segmentation model. The retriever is MPNet-based and other settings follow Table~\ref{tab: main_results}.
    }
    \label{tab: main_results_slm_seg}
    
    \resizebox{1\columnwidth}{!}{
    \begin{tabular}{l|cccccc|cc}
    \toprule
    
    \multirow{2}{*}{\textbf{Methods}} &  \multicolumn{6}{@{}c|}{{\bf QA Performance}} & \multicolumn{2}{@{}c}{{\bf Context Length}} \\
    \cmidrule (lr){2-7} \cmidrule (lr){8-9}
    & GPT4Score & BLEU & Rouge1 & Rouge2 & RougeL & BERTScore & \# Turns & \# Tokens \\

    \midrule
    \multicolumn{9}{@{}c}{{ \textit{ LOCOMO } }} \\
    \midrule
    Zero History & 24.86 & 1.94 & 17.36 & 3.72 & 13.24 & 85.83 & 0.00 & 0 \\
    Full History & 54.15 & 6.26 & 27.20 & 12.07 & 22.39 & 88.06 & 210.34 & 13,330 \\
    \midrule
    Turn-Level (MPNet) & 57.99 & 6.07 & 26.61 & 11.38 & 21.60 & 88.01 & 54.77 & 3,288 \\
    \midrule
    Session-Level (MPNet) & 51.18 & 5.22 & 24.23 & 9.33 & 19.51 & 87.45 & 53.88 & 3,471 \\
    \midrule
    SumMem & 53.87 & 2.87 & 20.71 & 6.66 & 16.25 & 86.88 & - & 4,108 \\
    RecurSum & 56.25 & 2.22 & 20.04 & 8.36 & 16.25 & 86.47 & - & 400 \\
    ConditionMem & 65.92 & 3.41 & 22.28 & 7.86 & 17.54 & 87.23 & - & 3,563 \\
    MemoChat & 65.10 & 6.76 & 28.54 & 12.93 & 23.65 & 88.13 & - & 1,159 \\
    \midrule
    \textbf{\sysname} (RoBERTa-Seg) & 61.84 & 6.41 & 27.51 & 12.27 & 23.06 & 88.08 & 56.32 & 3,767 \\
    \textbf{\sysname} (Mistral-7B-Seg) & 66.37 & 6.95 & 28.86 & 13.21 & 23.96 & 88.27 & 55.80 & 3,720 \\
    \textbf{\sysname} (GPT-4-Seg) & \bf{69.33} & \bf{7.19} & \bf{29.58} & \bf{13.74} & \bf{24.38} & \bf{88.60} & 55.51 & 3,716 \\

    \midrule
    \multicolumn{9}{@{}c}{{ \textit{ Long-MT-Bench+ } }} 
    \\
    \midrule
    Zero History & 49.73 & 4.38 & 18.69 & 6.98 & 13.94 & 84.22 & 0.00 & 0 \\
    Full History & 63.85 & 7.51 & 26.54 & 12.87 & 20.76 & 85.90 & 65.45 & 19,287 \\
    \midrule
    Turn-Level (MPNet) & 84.91 & 12.09 & 34.31 & 19.08 & \textbf{27.82} & 86.49 & 3.00 & 909 \\
    \midrule
    Session-Level (MPNet) & 73.38 & 8.89 & 29.34 & 14.30 & 22.79 & 86.61 & 13.43 & 3,680 \\
    \midrule
    SumMem & 63.42 & 7.84 & 25.48 & 10.61 & 18.66 & 85.70 & - & 1,651 \\
    RecurSum & 62.96 & 7.17 & 22.53 & 9.42 & 16.97 & 84.90 & - & 567 \\
    ConditionMem & 63.55 & 7.82 & 26.18 & 11.40 & 19.56 & 86.10 & - & 1,085 \\
    MemoChat & 85.14 & 12.66 & 33.84 & 19.01 & 26.87 & 87.21 & - & 1,615 \\
    \midrule
    \textbf{\sysname} (RoBERTa-Seg) & 81.52 & 11.27 & 32.66 & 16.23 & 25.51 & 86.63 & 2.96 & 841 \\
    \textbf{\sysname} (Mistral-7B-Seg) & 86.32 & 12.41 & 34.37 & 19.01 & 26.94 & 87.43 & 2.85 & 834 \\
    \textbf{\sysname} (GPT-4-Seg) & \textbf{88.81} & \textbf{13.80} & \textbf{34.63} & \textbf{19.21} & 27.64 & \textbf{87.72} & 2.77 & 820 \\
    \bottomrule
    \end{tabular}
    }
\end{table}

% \begin{table*}[t]
    \small
    \centering
    \setlength{\tabcolsep}{1mm}
    \caption{Segmentation performances on three datasets.
    $^{\dag}$: numbers reported in \citet{gao2023unsupervised}. Other baselines are reported in \citet{jiang2023superdialseg}. The best performance is highlighted in \textbf{bold}, and the second best is highlighted by \underline{underline}. \colorbox{light_gray}{Numbers in gray} correspond to \textbf{supervised} setting.}
    \label{tab: segment_main}
    \resizebox{\columnwidth}{!}{
    
    \begin{tabular}{l|cccc|cccc|cccc}
    \toprule
    \multirow{2}{*}{\textbf{Methods}} & \multicolumn{4}{@{}c}{{\bf Dialseg711}} & \multicolumn{4}{@{}c}{{\bf SuperDialSeg}} & \multicolumn{4}{@{}c}{{\bf TIAGE}}  \\
    \cmidrule (lr){2-5} \cmidrule (lr){6-9} \cmidrule (lr){10-13} 
    & Pk↓ & WD↓ & F1↑ & Score↑ & Pk↓ & WD↓ & F1↑ & Score↑ & Pk↓ & WD↓ & F1↑ & Score↑ \\

    \midrule
    \multicolumn{13}{@{}c}{\textbf{Unsupervised}} \\
    \midrule
    
    BayesSeg & 0.306 & 0.350 & 0.556 & 0.614 & \underline{0.433} & 0.593 & \underline{0.438} & 0.463 & 0.486 & 0.571 & 0.366 & 0.419\\
    TextTiling & 0.470 & 0.493 & 0.245 & 0.382 & 0.441 & \underline{0.453} & 0.388 & \underline{0.471} & 0.469 & 0.488 & 0.204 & 0.363 \\
    GraphSeg & 0.412 & 0.442 & 0.392 & 0.483 & 0.450 & 0.454 & 0.249 & 0.398 & 0.496 & 0.515 & 0.238 & 0.366\\
    \midrule
    TextTiling+Glove & 0.399 & 0.438 & 0.436 & 0.509 & 0.519 & 0.524 & 0.353 & 0.416 & 0.486 & 0.511 & 0.236 & 0.369\\
    TextTiling+[CLS] & 0.419 & 0.473 & 0.351 & 0.453 & 0.493 & 0.523 & 0.277 & 0.385 & 0.521 & 0.556 & 0.218 & 0.340 \\
    TextTiling+NSP & 0.347 & 0.360 & 0.347 & 0.497 & 0.512 & 0.521 & 0.208 & 0.346 & 0.425 & 0.439 & 0.285 & 0.426\\
    GreedySeg & 0.381 & 0.410 & 0.445 & 0.525 & 0.490 & 0.494 & 0.365 & 0.437 & 0.490 & 0.506 & 0.181 & 0.341\\
    CSM & 0.278 & 0.302 & \underline{0.610} & \underline{0.660} & 0.462 & 0.467 & 0.381 & 0.458 & \underline{0.400} & \underline{0.420} & \underline{0.427} & \underline{0.509} \\
    DialSTART $^{\dag}$ & \underline{0.178} & \underline{0.198} & - & - & - & - & - & - & - & - & - & - \\
    \midrule
    \textbf{Ours} & \textbf{0.093} & \textbf{0.103} & \textbf{0.888} & \textbf{0.895} & \textbf{0.277} & \textbf{0.289} & \textbf{0.758} & \textbf{0.738} & \textbf{0.363} & \textbf{0.401} & \textbf{0.596} & \textbf{0.607} \\
    
    \bottomrule
    \end{tabular}
    }
\end{table*}
