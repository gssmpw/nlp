\documentclass{article}
\usepackage{iclr2025_conference,times}

% Optional math commands from https://github.com/goodfeli/dlbook_notation.
%%%%% NEW MATH DEFINITIONS %%%%%

\usepackage{amsmath,amsfonts,bm}
\usepackage{derivative}
% Mark sections of captions for referring to divisions of figures
\newcommand{\figleft}{{\em (Left)}}
\newcommand{\figcenter}{{\em (Center)}}
\newcommand{\figright}{{\em (Right)}}
\newcommand{\figtop}{{\em (Top)}}
\newcommand{\figbottom}{{\em (Bottom)}}
\newcommand{\captiona}{{\em (a)}}
\newcommand{\captionb}{{\em (b)}}
\newcommand{\captionc}{{\em (c)}}
\newcommand{\captiond}{{\em (d)}}

% Highlight a newly defined term
\newcommand{\newterm}[1]{{\bf #1}}

% Derivative d 
\newcommand{\deriv}{{\mathrm{d}}}

% Figure reference, lower-case.
\def\figref#1{figure~\ref{#1}}
% Figure reference, capital. For start of sentence
\def\Figref#1{Figure~\ref{#1}}
\def\twofigref#1#2{figures \ref{#1} and \ref{#2}}
\def\quadfigref#1#2#3#4{figures \ref{#1}, \ref{#2}, \ref{#3} and \ref{#4}}
% Section reference, lower-case.
\def\secref#1{section~\ref{#1}}
% Section reference, capital.
\def\Secref#1{Section~\ref{#1}}
% Reference to two sections.
\def\twosecrefs#1#2{sections \ref{#1} and \ref{#2}}
% Reference to three sections.
\def\secrefs#1#2#3{sections \ref{#1}, \ref{#2} and \ref{#3}}
% Reference to an equation, lower-case.
\def\eqref#1{equation~\ref{#1}}
% Reference to an equation, upper case
\def\Eqref#1{Equation~\ref{#1}}
% A raw reference to an equation---avoid using if possible
\def\plaineqref#1{\ref{#1}}
% Reference to a chapter, lower-case.
\def\chapref#1{chapter~\ref{#1}}
% Reference to an equation, upper case.
\def\Chapref#1{Chapter~\ref{#1}}
% Reference to a range of chapters
\def\rangechapref#1#2{chapters\ref{#1}--\ref{#2}}
% Reference to an algorithm, lower-case.
\def\algref#1{algorithm~\ref{#1}}
% Reference to an algorithm, upper case.
\def\Algref#1{Algorithm~\ref{#1}}
\def\twoalgref#1#2{algorithms \ref{#1} and \ref{#2}}
\def\Twoalgref#1#2{Algorithms \ref{#1} and \ref{#2}}
% Reference to a part, lower case
\def\partref#1{part~\ref{#1}}
% Reference to a part, upper case
\def\Partref#1{Part~\ref{#1}}
\def\twopartref#1#2{parts \ref{#1} and \ref{#2}}

\def\ceil#1{\lceil #1 \rceil}
\def\floor#1{\lfloor #1 \rfloor}
\def\1{\bm{1}}
\newcommand{\train}{\mathcal{D}}
\newcommand{\valid}{\mathcal{D_{\mathrm{valid}}}}
\newcommand{\test}{\mathcal{D_{\mathrm{test}}}}

\def\eps{{\epsilon}}


% Random variables
\def\reta{{\textnormal{$\eta$}}}
\def\ra{{\textnormal{a}}}
\def\rb{{\textnormal{b}}}
\def\rc{{\textnormal{c}}}
\def\rd{{\textnormal{d}}}
\def\re{{\textnormal{e}}}
\def\rf{{\textnormal{f}}}
\def\rg{{\textnormal{g}}}
\def\rh{{\textnormal{h}}}
\def\ri{{\textnormal{i}}}
\def\rj{{\textnormal{j}}}
\def\rk{{\textnormal{k}}}
\def\rl{{\textnormal{l}}}
% rm is already a command, just don't name any random variables m
\def\rn{{\textnormal{n}}}
\def\ro{{\textnormal{o}}}
\def\rp{{\textnormal{p}}}
\def\rq{{\textnormal{q}}}
\def\rr{{\textnormal{r}}}
\def\rs{{\textnormal{s}}}
\def\rt{{\textnormal{t}}}
\def\ru{{\textnormal{u}}}
\def\rv{{\textnormal{v}}}
\def\rw{{\textnormal{w}}}
\def\rx{{\textnormal{x}}}
\def\ry{{\textnormal{y}}}
\def\rz{{\textnormal{z}}}

% Random vectors
\def\rvepsilon{{\mathbf{\epsilon}}}
\def\rvphi{{\mathbf{\phi}}}
\def\rvtheta{{\mathbf{\theta}}}
\def\rva{{\mathbf{a}}}
\def\rvb{{\mathbf{b}}}
\def\rvc{{\mathbf{c}}}
\def\rvd{{\mathbf{d}}}
\def\rve{{\mathbf{e}}}
\def\rvf{{\mathbf{f}}}
\def\rvg{{\mathbf{g}}}
\def\rvh{{\mathbf{h}}}
\def\rvu{{\mathbf{i}}}
\def\rvj{{\mathbf{j}}}
\def\rvk{{\mathbf{k}}}
\def\rvl{{\mathbf{l}}}
\def\rvm{{\mathbf{m}}}
\def\rvn{{\mathbf{n}}}
\def\rvo{{\mathbf{o}}}
\def\rvp{{\mathbf{p}}}
\def\rvq{{\mathbf{q}}}
\def\rvr{{\mathbf{r}}}
\def\rvs{{\mathbf{s}}}
\def\rvt{{\mathbf{t}}}
\def\rvu{{\mathbf{u}}}
\def\rvv{{\mathbf{v}}}
\def\rvw{{\mathbf{w}}}
\def\rvx{{\mathbf{x}}}
\def\rvy{{\mathbf{y}}}
\def\rvz{{\mathbf{z}}}

% Elements of random vectors
\def\erva{{\textnormal{a}}}
\def\ervb{{\textnormal{b}}}
\def\ervc{{\textnormal{c}}}
\def\ervd{{\textnormal{d}}}
\def\erve{{\textnormal{e}}}
\def\ervf{{\textnormal{f}}}
\def\ervg{{\textnormal{g}}}
\def\ervh{{\textnormal{h}}}
\def\ervi{{\textnormal{i}}}
\def\ervj{{\textnormal{j}}}
\def\ervk{{\textnormal{k}}}
\def\ervl{{\textnormal{l}}}
\def\ervm{{\textnormal{m}}}
\def\ervn{{\textnormal{n}}}
\def\ervo{{\textnormal{o}}}
\def\ervp{{\textnormal{p}}}
\def\ervq{{\textnormal{q}}}
\def\ervr{{\textnormal{r}}}
\def\ervs{{\textnormal{s}}}
\def\ervt{{\textnormal{t}}}
\def\ervu{{\textnormal{u}}}
\def\ervv{{\textnormal{v}}}
\def\ervw{{\textnormal{w}}}
\def\ervx{{\textnormal{x}}}
\def\ervy{{\textnormal{y}}}
\def\ervz{{\textnormal{z}}}

% Random matrices
\def\rmA{{\mathbf{A}}}
\def\rmB{{\mathbf{B}}}
\def\rmC{{\mathbf{C}}}
\def\rmD{{\mathbf{D}}}
\def\rmE{{\mathbf{E}}}
\def\rmF{{\mathbf{F}}}
\def\rmG{{\mathbf{G}}}
\def\rmH{{\mathbf{H}}}
\def\rmI{{\mathbf{I}}}
\def\rmJ{{\mathbf{J}}}
\def\rmK{{\mathbf{K}}}
\def\rmL{{\mathbf{L}}}
\def\rmM{{\mathbf{M}}}
\def\rmN{{\mathbf{N}}}
\def\rmO{{\mathbf{O}}}
\def\rmP{{\mathbf{P}}}
\def\rmQ{{\mathbf{Q}}}
\def\rmR{{\mathbf{R}}}
\def\rmS{{\mathbf{S}}}
\def\rmT{{\mathbf{T}}}
\def\rmU{{\mathbf{U}}}
\def\rmV{{\mathbf{V}}}
\def\rmW{{\mathbf{W}}}
\def\rmX{{\mathbf{X}}}
\def\rmY{{\mathbf{Y}}}
\def\rmZ{{\mathbf{Z}}}

% Elements of random matrices
\def\ermA{{\textnormal{A}}}
\def\ermB{{\textnormal{B}}}
\def\ermC{{\textnormal{C}}}
\def\ermD{{\textnormal{D}}}
\def\ermE{{\textnormal{E}}}
\def\ermF{{\textnormal{F}}}
\def\ermG{{\textnormal{G}}}
\def\ermH{{\textnormal{H}}}
\def\ermI{{\textnormal{I}}}
\def\ermJ{{\textnormal{J}}}
\def\ermK{{\textnormal{K}}}
\def\ermL{{\textnormal{L}}}
\def\ermM{{\textnormal{M}}}
\def\ermN{{\textnormal{N}}}
\def\ermO{{\textnormal{O}}}
\def\ermP{{\textnormal{P}}}
\def\ermQ{{\textnormal{Q}}}
\def\ermR{{\textnormal{R}}}
\def\ermS{{\textnormal{S}}}
\def\ermT{{\textnormal{T}}}
\def\ermU{{\textnormal{U}}}
\def\ermV{{\textnormal{V}}}
\def\ermW{{\textnormal{W}}}
\def\ermX{{\textnormal{X}}}
\def\ermY{{\textnormal{Y}}}
\def\ermZ{{\textnormal{Z}}}

% Vectors
\def\vzero{{\bm{0}}}
\def\vone{{\bm{1}}}
\def\vmu{{\bm{\mu}}}
\def\vtheta{{\bm{\theta}}}
\def\vphi{{\bm{\phi}}}
\def\va{{\bm{a}}}
\def\vb{{\bm{b}}}
\def\vc{{\bm{c}}}
\def\vd{{\bm{d}}}
\def\ve{{\bm{e}}}
\def\vf{{\bm{f}}}
\def\vg{{\bm{g}}}
\def\vh{{\bm{h}}}
\def\vi{{\bm{i}}}
\def\vj{{\bm{j}}}
\def\vk{{\bm{k}}}
\def\vl{{\bm{l}}}
\def\vm{{\bm{m}}}
\def\vn{{\bm{n}}}
\def\vo{{\bm{o}}}
\def\vp{{\bm{p}}}
\def\vq{{\bm{q}}}
\def\vr{{\bm{r}}}
\def\vs{{\bm{s}}}
\def\vt{{\bm{t}}}
\def\vu{{\bm{u}}}
\def\vv{{\bm{v}}}
\def\vw{{\bm{w}}}
\def\vx{{\bm{x}}}
\def\vy{{\bm{y}}}
\def\vz{{\bm{z}}}

% Elements of vectors
\def\evalpha{{\alpha}}
\def\evbeta{{\beta}}
\def\evepsilon{{\epsilon}}
\def\evlambda{{\lambda}}
\def\evomega{{\omega}}
\def\evmu{{\mu}}
\def\evpsi{{\psi}}
\def\evsigma{{\sigma}}
\def\evtheta{{\theta}}
\def\eva{{a}}
\def\evb{{b}}
\def\evc{{c}}
\def\evd{{d}}
\def\eve{{e}}
\def\evf{{f}}
\def\evg{{g}}
\def\evh{{h}}
\def\evi{{i}}
\def\evj{{j}}
\def\evk{{k}}
\def\evl{{l}}
\def\evm{{m}}
\def\evn{{n}}
\def\evo{{o}}
\def\evp{{p}}
\def\evq{{q}}
\def\evr{{r}}
\def\evs{{s}}
\def\evt{{t}}
\def\evu{{u}}
\def\evv{{v}}
\def\evw{{w}}
\def\evx{{x}}
\def\evy{{y}}
\def\evz{{z}}

% Matrix
\def\mA{{\bm{A}}}
\def\mB{{\bm{B}}}
\def\mC{{\bm{C}}}
\def\mD{{\bm{D}}}
\def\mE{{\bm{E}}}
\def\mF{{\bm{F}}}
\def\mG{{\bm{G}}}
\def\mH{{\bm{H}}}
\def\mI{{\bm{I}}}
\def\mJ{{\bm{J}}}
\def\mK{{\bm{K}}}
\def\mL{{\bm{L}}}
\def\mM{{\bm{M}}}
\def\mN{{\bm{N}}}
\def\mO{{\bm{O}}}
\def\mP{{\bm{P}}}
\def\mQ{{\bm{Q}}}
\def\mR{{\bm{R}}}
\def\mS{{\bm{S}}}
\def\mT{{\bm{T}}}
\def\mU{{\bm{U}}}
\def\mV{{\bm{V}}}
\def\mW{{\bm{W}}}
\def\mX{{\bm{X}}}
\def\mY{{\bm{Y}}}
\def\mZ{{\bm{Z}}}
\def\mBeta{{\bm{\beta}}}
\def\mPhi{{\bm{\Phi}}}
\def\mLambda{{\bm{\Lambda}}}
\def\mSigma{{\bm{\Sigma}}}

% Tensor
\DeclareMathAlphabet{\mathsfit}{\encodingdefault}{\sfdefault}{m}{sl}
\SetMathAlphabet{\mathsfit}{bold}{\encodingdefault}{\sfdefault}{bx}{n}
\newcommand{\tens}[1]{\bm{\mathsfit{#1}}}
\def\tA{{\tens{A}}}
\def\tB{{\tens{B}}}
\def\tC{{\tens{C}}}
\def\tD{{\tens{D}}}
\def\tE{{\tens{E}}}
\def\tF{{\tens{F}}}
\def\tG{{\tens{G}}}
\def\tH{{\tens{H}}}
\def\tI{{\tens{I}}}
\def\tJ{{\tens{J}}}
\def\tK{{\tens{K}}}
\def\tL{{\tens{L}}}
\def\tM{{\tens{M}}}
\def\tN{{\tens{N}}}
\def\tO{{\tens{O}}}
\def\tP{{\tens{P}}}
\def\tQ{{\tens{Q}}}
\def\tR{{\tens{R}}}
\def\tS{{\tens{S}}}
\def\tT{{\tens{T}}}
\def\tU{{\tens{U}}}
\def\tV{{\tens{V}}}
\def\tW{{\tens{W}}}
\def\tX{{\tens{X}}}
\def\tY{{\tens{Y}}}
\def\tZ{{\tens{Z}}}


% Graph
\def\gA{{\mathcal{A}}}
\def\gB{{\mathcal{B}}}
\def\gC{{\mathcal{C}}}
\def\gD{{\mathcal{D}}}
\def\gE{{\mathcal{E}}}
\def\gF{{\mathcal{F}}}
\def\gG{{\mathcal{G}}}
\def\gH{{\mathcal{H}}}
\def\gI{{\mathcal{I}}}
\def\gJ{{\mathcal{J}}}
\def\gK{{\mathcal{K}}}
\def\gL{{\mathcal{L}}}
\def\gM{{\mathcal{M}}}
\def\gN{{\mathcal{N}}}
\def\gO{{\mathcal{O}}}
\def\gP{{\mathcal{P}}}
\def\gQ{{\mathcal{Q}}}
\def\gR{{\mathcal{R}}}
\def\gS{{\mathcal{S}}}
\def\gT{{\mathcal{T}}}
\def\gU{{\mathcal{U}}}
\def\gV{{\mathcal{V}}}
\def\gW{{\mathcal{W}}}
\def\gX{{\mathcal{X}}}
\def\gY{{\mathcal{Y}}}
\def\gZ{{\mathcal{Z}}}

% Sets
\def\sA{{\mathbb{A}}}
\def\sB{{\mathbb{B}}}
\def\sC{{\mathbb{C}}}
\def\sD{{\mathbb{D}}}
% Don't use a set called E, because this would be the same as our symbol
% for expectation.
\def\sF{{\mathbb{F}}}
\def\sG{{\mathbb{G}}}
\def\sH{{\mathbb{H}}}
\def\sI{{\mathbb{I}}}
\def\sJ{{\mathbb{J}}}
\def\sK{{\mathbb{K}}}
\def\sL{{\mathbb{L}}}
\def\sM{{\mathbb{M}}}
\def\sN{{\mathbb{N}}}
\def\sO{{\mathbb{O}}}
\def\sP{{\mathbb{P}}}
\def\sQ{{\mathbb{Q}}}
\def\sR{{\mathbb{R}}}
\def\sS{{\mathbb{S}}}
\def\sT{{\mathbb{T}}}
\def\sU{{\mathbb{U}}}
\def\sV{{\mathbb{V}}}
\def\sW{{\mathbb{W}}}
\def\sX{{\mathbb{X}}}
\def\sY{{\mathbb{Y}}}
\def\sZ{{\mathbb{Z}}}

% Entries of a matrix
\def\emLambda{{\Lambda}}
\def\emA{{A}}
\def\emB{{B}}
\def\emC{{C}}
\def\emD{{D}}
\def\emE{{E}}
\def\emF{{F}}
\def\emG{{G}}
\def\emH{{H}}
\def\emI{{I}}
\def\emJ{{J}}
\def\emK{{K}}
\def\emL{{L}}
\def\emM{{M}}
\def\emN{{N}}
\def\emO{{O}}
\def\emP{{P}}
\def\emQ{{Q}}
\def\emR{{R}}
\def\emS{{S}}
\def\emT{{T}}
\def\emU{{U}}
\def\emV{{V}}
\def\emW{{W}}
\def\emX{{X}}
\def\emY{{Y}}
\def\emZ{{Z}}
\def\emSigma{{\Sigma}}

% entries of a tensor
% Same font as tensor, without \bm wrapper
\newcommand{\etens}[1]{\mathsfit{#1}}
\def\etLambda{{\etens{\Lambda}}}
\def\etA{{\etens{A}}}
\def\etB{{\etens{B}}}
\def\etC{{\etens{C}}}
\def\etD{{\etens{D}}}
\def\etE{{\etens{E}}}
\def\etF{{\etens{F}}}
\def\etG{{\etens{G}}}
\def\etH{{\etens{H}}}
\def\etI{{\etens{I}}}
\def\etJ{{\etens{J}}}
\def\etK{{\etens{K}}}
\def\etL{{\etens{L}}}
\def\etM{{\etens{M}}}
\def\etN{{\etens{N}}}
\def\etO{{\etens{O}}}
\def\etP{{\etens{P}}}
\def\etQ{{\etens{Q}}}
\def\etR{{\etens{R}}}
\def\etS{{\etens{S}}}
\def\etT{{\etens{T}}}
\def\etU{{\etens{U}}}
\def\etV{{\etens{V}}}
\def\etW{{\etens{W}}}
\def\etX{{\etens{X}}}
\def\etY{{\etens{Y}}}
\def\etZ{{\etens{Z}}}

% The true underlying data generating distribution
\newcommand{\pdata}{p_{\rm{data}}}
\newcommand{\ptarget}{p_{\rm{target}}}
\newcommand{\pprior}{p_{\rm{prior}}}
\newcommand{\pbase}{p_{\rm{base}}}
\newcommand{\pref}{p_{\rm{ref}}}

% The empirical distribution defined by the training set
\newcommand{\ptrain}{\hat{p}_{\rm{data}}}
\newcommand{\Ptrain}{\hat{P}_{\rm{data}}}
% The model distribution
\newcommand{\pmodel}{p_{\rm{model}}}
\newcommand{\Pmodel}{P_{\rm{model}}}
\newcommand{\ptildemodel}{\tilde{p}_{\rm{model}}}
% Stochastic autoencoder distributions
\newcommand{\pencode}{p_{\rm{encoder}}}
\newcommand{\pdecode}{p_{\rm{decoder}}}
\newcommand{\precons}{p_{\rm{reconstruct}}}

\newcommand{\laplace}{\mathrm{Laplace}} % Laplace distribution

\newcommand{\E}{\mathbb{E}}
\newcommand{\Ls}{\mathcal{L}}
\newcommand{\R}{\mathbb{R}}
\newcommand{\emp}{\tilde{p}}
\newcommand{\lr}{\alpha}
\newcommand{\reg}{\lambda}
\newcommand{\rect}{\mathrm{rectifier}}
\newcommand{\softmax}{\mathrm{softmax}}
\newcommand{\sigmoid}{\sigma}
\newcommand{\softplus}{\zeta}
\newcommand{\KL}{D_{\mathrm{KL}}}
\newcommand{\Var}{\mathrm{Var}}
\newcommand{\standarderror}{\mathrm{SE}}
\newcommand{\Cov}{\mathrm{Cov}}
% Wolfram Mathworld says $L^2$ is for function spaces and $\ell^2$ is for vectors
% But then they seem to use $L^2$ for vectors throughout the site, and so does
% wikipedia.
\newcommand{\normlzero}{L^0}
\newcommand{\normlone}{L^1}
\newcommand{\normltwo}{L^2}
\newcommand{\normlp}{L^p}
\newcommand{\normmax}{L^\infty}

\newcommand{\parents}{Pa} % See usage in notation.tex. Chosen to match Daphne's book.

\DeclareMathOperator*{\argmax}{arg\,max}
\DeclareMathOperator*{\argmin}{arg\,min}

\DeclareMathOperator{\sign}{sign}
\DeclareMathOperator{\Tr}{Tr}
\let\ab\allowbreak


\usepackage{hyperref}
\usepackage{url}
\usepackage{xurl}
\usepackage{graphicx}
\usepackage{multirow}
\usepackage[most]{tcolorbox}
\usepackage{graphicx}
\usepackage{amsfonts}
\usepackage{soul}
\usepackage{amsmath}
\usepackage{makecell}
\usepackage{bm}
\usepackage{pifont}
\usepackage[export]{adjustbox}
\usepackage{booktabs}
\usepackage{ulem}

% This assumes your files are encoded as UTF8
\usepackage[utf8]{inputenc}
\usepackage{algorithm2e}
\usepackage{booktabs}
% This is not strictly necessary, and may be commented out,
% but it will improve the layout of the manuscript,
% and will typically save some space.
\usepackage{microtype}
% If the title and author information does not fit in the area allocated, uncomment the following
%
%\setlength\titlebox{<dim>}
%
% and set <dim> to something 5cm or larger.

\usepackage{color}
\usepackage{colortbl}
\usepackage{xcolor}
% \definecolor{red_dark}{rgb}{1,0,0}
% \definecolor{red_medium}{rgb}{1,0,0,0.5}
% \definecolor{red_light}{rgb}{1,0,0,0.25}
% \definecolor{}{rgb}{1,0,0,0.5} % 50% alpha 红色
\usepackage{wrapfig}
\usepackage{subfig}
% \usepackage{subfigure}
% \usepackage{subcaption}

\definecolor{light_gray}{RGB}{211,211,211}
\definecolor{plum}{RGB}{160,43,147}
\definecolor{brown}{RGB}{192,79,21}
\definecolor{light_green}{RGB}{217,242,208}
\definecolor{dark_green}{RGB}{59,125,35}
\definecolor{light_blue}{RGB}{220,234,247}
\definecolor{dark_blue}{RGB}{33,95,154}

\newcommand{\sysname}{{\scshape SeCom}}
\newcommand{\ie}{\textit{i.e.}}
\newcommand{\eg}{\textit{e.g.}}
\newcommand{\wrt}{\textit{w.r.t}}
\newcommand{\etc}{\textit{etc}}
\newcommand{\citehere}{\textcolor{red}{[citation]}}
\newcommand{\red}[1]{\textcolor{red}{#1}}

\definecolor{darkred}{rgb}{0.5, 0.0, 0.0}
\definecolor{darkblue}{rgb}{0.0, 0.0, 0.5}
\definecolor{darkgreen}{rgb}{0.0, 0.5, 0.0}
\definecolor{darkcyan}{rgb}{0.0, 0.5, 0.5}
\definecolor{darkmagenta}{rgb}{0.5, 0.0, 0.5}
\definecolor{darkgray}{rgb}{0.3, 0.3, 0.3}
\definecolor{navy}{rgb}{0.0, 0.0, 0.5}
\definecolor{olive}{rgb}{0.5, 0.5, 0.0}
\definecolor{brown}{rgb}{0.5, 0.25, 0.0}

\title{On Memory Construction and Retrieval for Personalized Conversational Agents}

% Authors must not appear in the submitted version. They should be hidden
% as long as the \iclrfinalcopy macro remains commented out below.
% Non-anonymous submissions will be rejected without review.

\author{Zhuoshi Pan\textsuperscript{1}\footnotemark[2], Qianhui Wu\textsuperscript{2}\footnotemark[3], Huiqiang Jiang\textsuperscript{2}, Xufang Luo\textsuperscript{2}, Hao Cheng\textsuperscript{2}, \\ \textbf{Dongsheng Li\textsuperscript{2}, Yuqing Yang\textsuperscript{2}, Chin-Yew Lin\textsuperscript{2}, H. Vicky Zhao\textsuperscript{1}\footnotemark[3], Lili Qiu\textsuperscript{2}, Jianfeng Gao\textsuperscript{2}} \\
\\
\texttt{\textsuperscript{1} Tsinghua University, \textsuperscript{2} Microsoft Corporation} \\
% \texttt{\{qianhuiwu, hjiang, xufang.luo\}@microsoft.com}
}

\renewcommand{\thefootnote}{\fnsymbol{footnote}}


% The \author macro works with any number of authors. There are two commands
% used to separate the names and addresses of multiple authors: \And and \AND.
%
% Using \And between authors leaves it to \LaTeX{} to determine where to break
% the lines. Using \AND forces a linebreak at that point. So, if \LaTeX{}
% puts 3 of 4 authors names on the first line, and the last on the second
% line, try using \AND instead of \And before the third author name.

\newcommand{\fix}{\marginpar{FIX}}
\newcommand{\new}{\marginpar{NEW}}

\iclrfinalcopy % Uncomment for camera-ready version, but NOT for submission.
\begin{document}

\maketitle
\footnotetext[2]{Work during internship at Microsoft.}
\footnotetext[3]{Corresponding author.}
\footnotetext[4]{Project page:~\href{https://aka.ms/secom}{https://aka.ms/secom}}

% \vspace{-9mm}
% \hspace{0.5em} {Project page:~\href{https://aka.ms/secom}{https://aka.ms/secom}}
% \vspace{3mm}

% \begin{center}
%  \vspace{-10mm}
%   % \fontsize{9pt}{\baselineskip}\selectfont
%   {Project page:}~\href{https://www.microsoft.com/en-us/research/project/secom/}{\textbf{https://www.microsoft.com/en-us/research/project/secom}}
%  \vspace{3mm}
% \end{center}

\begin{abstract}
To deliver coherent and personalized experiences in long-term conversations, existing approaches typically perform retrieval augmented response generation by constructing memory banks from conversation history at either the turn-level, session-level, or through summarization techniques.
In this paper, we present two key findings: (1) The granularity of memory unit matters: Turn-level, session-level, and summarization-based methods each exhibit limitations in both memory retrieval accuracy and the semantic quality of the retrieved content. (2) Prompt compression methods, such as \textit{LLMLingua-2}, can effectively serve as a denoising mechanism, enhancing memory retrieval accuracy across different granularities.

Building on these insights, we propose \textbf{\sysname}, a method that constructs the memory bank at segment level by introducing a conversation \textbf{{\scshape Se}}gmentation model that partitions long-term conversations into topically coherent segments, while applying \textbf{{\scshape Com}}pression based denoising on memory units to enhance memory retrieval.
Experimental results show that {\sysname} exhibits a significant performance advantage over baselines on long-term conversation benchmarks \textit{LOCOMO} and \textit{Long-MT-Bench+}. Additionally, the proposed conversation segmentation method demonstrates superior performance on dialogue segmentation datasets such as \textit{DialSeg711}, \textit{TIAGE}, and \textit{SuperDialSeg}.

\end{abstract}

\section{Introduction}
\label{sec:intro}
Large language models (LLMs) have developed rapidly in recent years and have been widely used in conversational agents.
In contrast to traditional dialogue systems, which typically focus on short conversations within specific domains \citep{dinanwizard}, 
LLM-powered conversational agents engage in significantly more interaction turns across a broader range of topics in open-domain conversations~\citep{kim2023aligning, zhou-etal-2023-facilitating}.
Such long-term, open-domain conversations over multiple sessions present significant challenges, as they require the system to retain past events and user preferences to deliver coherent and personalized responses~\citep{chen2024compress}. 

Some methods maintain context by concatenating all historical utterances or their summarized versions~\citep{langchain2023buffer, wang2023recursively}.
However, these strategies can result in excessively long contexts that include irrelevant information, which may not be relevant to the user's current request.
As noted by~\citet{maharana2024evaluating}, LLMs struggle with understanding lengthy conversations and grasping long-range temporal and causal dynamics, particularly when the dialogues contain irrelevant information~\citep{jiang2023longllmlingua}.
Some other works focus on 
retrieving query-related conversation history to 
enhance response generation~\citep{yuan2023evolving, alonso2024toward, kim2024theanine, maharana2024evaluating}.
These approaches typically construct a memory bank from the conversation history at either the \textit{turn-level}~\citep{yuan2023evolving} or \textit{session-level}~\citep{wang2023recursively}. 
% \citet{xu2022beyond}, 
\citet{chen2024compress}, \citet{li2024hello} and \citet{zhong2024memorybank} further leverage \textit{summarization} techniques to build memory units, which are then retrieved as context for response generation.

Building on these works, a key question arises: Which level of memory granularity—turn-level, session-level, or their summarized forms—yields the highest effectiveness? Moreover, is there a novel memory structure that could outperform these three formats?

In this paper, we first systematically investigate the impact of different memory granularities on conversational agents within the paradigm of retrieval augmented response generation~\citep{lewis2020retrieval,ye2024boosting}.
Our findings indicate that turn-level, session-level, and summarization-based methods all exhibit limitations in terms of the accuracy of the retrieval module as well as the semantics of the retrieved content, which ultimately lead to sub-optimal responses, as depicted in Figure~\ref{fig: intro_example}, Figure~\ref{fig: intro_statistics}, and Table~\ref{tab: main_results}.

\begin{figure}[!h]
    \centering
    \includegraphics[width=\linewidth]{figures/intro_example.pdf}
    \caption{Illustration of retrieval augmented response generation with different memory granularities. \textcolor{dark_blue}{\textit{Turn-level memory}} is too fine-grained, leading to fragmentary and incomplete context. \textcolor{brown}{\textit{Session-level memory}} is too coarse-grained, containing too much irrelevant information. \textcolor{plum}{\textit{Summary based methods}} suffer from information loss that occurs during summarization. \textcolor{dark_green}{\textit{Ours (segment-level memory)}} can better capture topically coherent units in long conversations, striking a balance between including more relevant, coherent information while excluding irrelevant content. Bullseye $\odot$ indicates the retrieved memory units at \textcolor{dark_blue}{turn level} or \textcolor{dark_green}{segment level} under the same context budget. [0.xx]: similarity between target query and history content. \textcolor{dark_blue}{Turn-level} retrieval errors: \colorbox{light_green}{\textcolor{dark_green}{false negative}}, \colorbox{light_blue}{\textcolor{red}{false positive}}.}
    \label{fig: intro_example}
\end{figure}
\begin{figure*}[!h]
  \centering
  \subfloat[Response quality as a function of chunk size, given a total budget of 50 turns to retrieve as context.]{
    \label{fig: score_wrt_chunk_size}
    \includegraphics[width=0.31\columnwidth]{figures/score_wrt_chunk_size.pdf}}
    \hspace{0.2em}
  \subfloat[Retrieval DCG obtained with different memory granularities using BM25 based retriever.]{
    \label{fig: bm25_granularity}
    \includegraphics[width=0.31\columnwidth]{figures/dcg_bm25_mtbp.pdf}}
    \hspace{0.2em}
  \subfloat[Retrieval DCG obtained with different memory granularities using MPNet based retriever.]{
    \label{fig: mpnet_granularity}
    \includegraphics[width=0.31\columnwidth]{figures/dcg_mpnet_mtbp.pdf}}
  \caption{The impact of memory granularity on the response quality (a) and retrieval accuracy (b, c).}
  \label{fig: intro_statistics}
\end{figure*}


% \begin{figure*}[htbp]
% \centering
% \caption{The DCG metric of retrieved result at varying number of retrieved budget on \textit{Long-MT-Bench+}. \Zhuoshi{Since the length of the retrieval unit varies a lot at different granularity. For a fair comparison, we set the x-coordinate to the total number of retrieved rounds .}}
% \label{fig: DCG_mtbp}
% \end{figure*}

Specifically, users often interact with agents over multiple turns to achieve their goals, causing relevant information to be dispersed across multiple interactions. This dispersion can pose a great challenge to the retrieval of turn-level memory units as some of the history conversation turns may not explicitly contain or relate to keywords mentioned in the current request (\eg, Turn-5 in Figure~\ref{fig: intro_example}). As a result, the retrieved contexts (\eg, Turn-3 and Turn-6 in Figure~\ref{fig: intro_example}) can be fragmentary and fail to encompass the complete request-related information flow, leading to responses that may lack coherence or omit essential information.
On the other hand, a single conversation session may cover multiple topics, especially when users do not initiate a new chat session upon switching topics. Therefore, constructing memory units at the session level risks including irrelevant content (\eg, definition of the prosecutor's fallacy and reasons of World War II in Figure~\ref{fig: intro_example}). Such extraneous content in the session-level memory unit may not only distract the retrieval module but also disrupt the language model's comprehension of the context, causing the agent to produce responses that are off-topic or include unnecessary details.

Long conversations are naturally composed of coherent discourse units. To capture this structure, we introduce a conversation segmentation model that partitions long-term conversations into topically coherent segments, constructing the memory bank at the segment level. During response generation, we directly concatenate the retrieved segment-level memory units as the context as in \citet{yuan2023evolving, kim2024theanine},
bypassing summarization to avoid the information loss that often occurs when converting dialogues into summaries~\citep{maharana2024evaluating}.

Furthermore, inspired by the notion that natural language tends to be inherently redundant~\citep{shannon1951prediction, jiang2023llmlingua, pan2024llmlingua}, we hypothesize that such redundancy can act as noise for retrieval systems, complicating the extraction of key information~\citep{grangier2003information, ma2021simple}.
Therefore, we propose removing such redundancy from memory units prior to retrieval by leveraging prompt compression methods such as LLMLingua-2~\citep{pan2024llmlingua}.
Figure~\ref{fig: similarity_mpnet} shows the results obtained with a BM25 based retriever and an MPNet based retriever~\citep{song2020mpnet} on \textit{Long-MT-Bench+}.
As demonstrated in Figure~\ref{fig: recall_wrt_comp_rate_bm25} and Figure~\ref{fig: recall_wrt_comp_rate_mpnet}, LLMLingua-2 consistently improves retrieval recall given different retrieval budgets $K$ (\ie, the number of retrieved segments) when the compression rate exceeds 50\%.
Figure~\ref{fig: similarity_change} further illustrates that, after denoising, similarity between the query and relevant segments increases, while the similarity with irrelevant segments decreases.

\begin{figure*}[!h]
\centering
\subfloat[Retrieval recall v.s. compression rate: $\frac{\text{\# tokens after compression}}{{\text{\# tokens before compression}}}$.\\K: number of retrieved segments. \\Retriever: BM25]{
    \label{fig: recall_wrt_comp_rate_bm25} \includegraphics[width=0.3\columnwidth]{figures/recall_mtbp_bm25.pdf}
}
\hspace{0.5em}
\subfloat[Retrieval recall v.s. compression rate: $\frac{\text{\# tokens after compression}}{{\text{\# tokens before compression}}}$.\\K: number of retrieved segments. Retriever: MPNet]{
    \label{fig: recall_wrt_comp_rate_mpnet} \includegraphics[width=0.29\columnwidth]{figures/recall_mtbp_mpnet.pdf}
}
\hspace{0.5em}
\subfloat[Similarity between the query and different dialogue segments. Blue: \textcolor{blue}{relevant} segments. Orange: \textcolor{orange}{irrelevant} segments. Retriever: MPNet]
{
    \label{fig: similarity_change}
    \includegraphics[width=0.33\columnwidth]{figures/similarity_mpnet.pdf}
}
  \caption{Prompt compression method (LLMLingua-2) can serve as an effective denoising technique to enhance the memory retrieval system by: (a) improving the retrieval recall with varying context budget $K$; (b) benefiting the retrieval system by increasing the similarity between the query and relevant segments while decreasing the similarity with irrelevant ones.}
  \label{fig: similarity_mpnet}
\end{figure*}

Our contributions can be summarized as follows:
\begin{itemize}
\item We systematically investigate the effects of memory granularity on retrieval augmented response generation in conversational agents. Our findings reveal that turn-level, session-level, and summarization-based approaches each face challenges in ensuring precise retrieval and providing a complete, relevant, and coherent context for generating accurate responses.

\item We contend that the inherent redundancy in natural language can act as noise for retrieval systems. We demonstrate that prompt compression technique, LLMLingua-2, can serve as an effective denoising method to enhance memory retrieval performance.

\item We present \sysname, a system that constructs memory bank at segment level by introducing a conversation {\scshape Se}gmentation model, while applying {\scshape Com}pression based denoising on memory units to enhance memory retrieval. The experimental results show that \sysname\ outperforms baselines on two long-term conversation benchmark LOCOMO and Long-MT-Bench+. Further analysis and ablation studies confirm the contributions of the segment-level memory units and the compression-based denoising technique within our framework. 

\end{itemize}



\section{SeCom}
\subsection{Preliminary}
Let $\mathcal{H} = \{\bm{c}_i\}_{i=1}^C$ represent the available conversation history between a user and an agent, which consists of $C$ sessions.
$\bm{c}_i = \{\bm{t}_j\}_{j=1}^{T_{i}}$ denotes the $i$-th session that is composed of $T_{i}$ sequential user-agent interaction turns, with each turn $\bm{t}_j=(u_j,r_j)$ consisting of a user request $u_j$ and the corresponding response from the agent $r_j$.
Denote the base retrieval system as $f_R$ and the response generation model as $f_{\text{LLM}}$.
The research framework here can be defined as:
(1) \textit{Memory construction}: construct a memory bank $\mathcal{M}$ using conversation history $\mathcal{H}$; For a turn-level memory bank, each memory unit $\bm{m}\in\mathcal{M}$ corresponds to an interaction turn $\bm{t}$, with $|\mathcal{M}|=\sum_{i=1}^C T_i$. For a session-level memory bank, each memory unit $\bm{m}$ corresponds to a session $\bm{c}$, with $|\mathcal{M}|=C$.
(2) \textit{Memory retrieval}: given a target user request $u^*$ and context budget $N$, retrieve $N$ memory units $\{\bm{m}_n\in\mathcal{M}\}_{n=1}^N \leftarrow f_R(u^*, \mathcal{M}, N)$ that are relevant to user request $u^*$;
(3) \textit{Response generation}: take the retrieved $N$ memory units in time order as the context and query the response generation model for response $r^*=f_{\text{LLM}}(u^*, \{\bm{m}_n\}_{n=1}^N)$.

In the remainder of this section, we first elaborate on the proposed conversation segmentation model that splits each session $\bm{c}_i$ into $K_i$ topical segments $\{\bm{s}_{k}\}_{k=1}^{K_i}$ in Section~\ref{sec: method_segment}, with which we construct a segment-level memory bank with each memory unit $\bm{m}$ corresponding to a segment $\bm{s}$ and $|\mathcal{M}|=\sum_{i=1}^{C}K_i$.
In Section~\ref{sec: method_retrieval}, we describe how to denoise memory units to enhance the accuracy of memory retrieval.

\subsection{Conversation Segmentation}
\label{sec: method_segment}

\paragraph{Zero-shot Segmentation} Given a conversation session $\bm{c}$, the conversation segmentation model $f_{\mathcal{I}}$ aims to identify \textit{a set of segment indices} $\mathcal{I}=\{(p_{k}, q_{k})\}_{k=1}^{K}$, where $K$ denotes the total number of segments within the session $\bm{c}$, $p_{k}$ and $q_{k}$ represent the indexes of the first and last interaction turns for the $k$-th segment $\bm{s}_{k}$, with $p_{k} \leq q_{k}$, $p_{k+1} = q_k + 1$.
This can be formulated as:
\begin{equation}
    f_{\mathcal{I}}(\bm{c}) = \{\bm{s}_k\}_{k=1}^K, \\
    \text{where}~\bm{s}_k =\{\bm{t}_{p_k}, \bm{t}_{p_k+1}, ..., \bm{t}_{q_k}\}
\end{equation}
However, building a segmentation model for open-domain conversation is challenging, primarily due to the difficulty of acquiring large amounts of annotated data.
As noted by \citet{jiang2023superdialseg}, the ambiguous nature of segmentation points complicates data collection, making the task difficult even for human annotators.
Consequently, we employ GPT-4 as the conversation segmentation model $f_{\mathcal{I}}$ to leverage its powerful text understanding ability across various domains. To provide clearer context and facilitate reasoning, we enhance session data $\bm{c}$ by adding turn indices and role identifiers to each interaction $\bm{t}_j$ as: ``\text{Turn $j$: \textbackslash n[user]: $u_j$\textbackslash n[agent]: $r_j$}''. We empirically demonstrate that segmentation can also be accomplished  with more lightweight models, such 
as \textit{Mistral-7B} and even \textit{RoBERTa} scale models, making our approach applicable in resource-constrained environments. Figure~\ref{fig: prompt4seg-zero-shot} in Appendix~\ref{sec: segmentation_details} presents the detailed instruction used for zero-shot conversation segmentation here. 

\paragraph{Segmentation with Reflection on Limited Annotated Data}
When a small amount of conversation data with segment annotations is available, we leverage this annotated data to inject segmentation knowledge into LLMs and better align the LLM-based segmentation model with human preferences. Inspired by the prefix-tuning technique~\citep{li2021prefix} and reflection mechanism~\citep{shinn2023reflexion,renze2024self}, we treat the segmentation prompt as the ``prefix'' and iteratively optimize it through LLM self-reflection, ultimately obtaining a segmentation guidance $\bm{G}$.

Specifically, in each iteration, we first apply our segmentation model in a zero-shot manner to a batch of conversation data and select the ``hard examples'', \textit{i.e.,} the top $K$ sessions with the most significant segmentation errors based on the WindowDiff metric~\citep{pevzner2002critique}. The LLM-based segmentation model is then instructed to reflect on its mistakes given the ground-truth segmentation annotations and update the segmentation guidance $\bm{G}$. This process mirrors Stochastic Gradient Descent (SGD) optimization, \textit{i.e.,} $\boldsymbol{G}_{m+1}=\boldsymbol{G}_m-\eta \nabla \mathcal{L}\left(\boldsymbol{G}_m\right)$, where $\nabla \mathcal{L}\left(\boldsymbol{G}_m\right)$ denotes the gradient of segmentation loss, which we assume is estimated implicitly by the LLM itself and is used to adjust the next segmentation guidance $\boldsymbol{G}_{m+1}$.
Figure~\ref{fig: prompt4rubric} shows the self-reflection prompt and Figure~\ref{fig: prompt4seg} illustrates the final prompt with the learned rubric for segmentation.

\subsection{Compression based Memory Denoising}
\label{sec: method_retrieval}
Given a target user request $u^*$ and context budget $N$, the memory retrieval system $f_R$ retrieves $N$ memory units $\{\bm{m}_n\in\mathcal{M}\}_{n=1}^N$ from the memory bank $\mathcal{M}$ as the context in response to the user request $u^*$ .
With the consideration that the inherent redundancy in natural language can act as noise for the retrieval system~\citep{grangier2003information, ma2021simple}, we denoise memory units by removing such redundancy via a prompt compression model $f_{Comp}$ before retrieval:
\begin{equation}
\{\bm{m}_n\in\mathcal{M}\}_{n=1}^N \leftarrow f_R(u^*, f_{Comp}(\mathcal{M}), N).
\end{equation}
Specifically, we use LLMLingua-2~\citep{pan2024llmlingua} as the denoising function $f_{Comp}$ here. 

\section{Experiments}
\label{sec: experiments}

\paragraph{Implementation Details}
We use \texttt{GPT-35-Turbo} for response generation in our main experiment. We also adopt \texttt{Mistral-7B-Instruct-v0.3}\footnote{\url{https://huggingface.co/mistralai/Mistral-7B-Instruct-v0.3}}~\citep{jiang2023mistral7b} for robustness evaluation across different LLMs. 
We employ zero-shot segmentation for QA benchmarks and further incorporate the reflection mechanism for segmentation benchmarks to leverage the available annotated data. To make our method applicable in resource-constrained environments, we conduct additional experiments by using  \texttt{Mistral-7B-Instruct-v0.3} and a \texttt{RoBERTa} based model fine-tuned on SuperDialseg~\citep{jiang2023superdialseg}.
Details for the conversation segmentation such as the prompt and hyper-parameters are described in Appendix~\ref{sec: segmentation_details}.
We use \texttt{LLMLingua-2}~\citep{pan2024llmlingua} with a compression rate of 75\% and \texttt{xlm-roberta-large}~\citep{conneau2019unsupervised} as the base model to denoise memory units.
Following \citet{alonso2024toward}, we apply MPNet (\texttt{multi-qa-mpnet-base-dot-v1})~\citep{song2020mpnet} with FAISS~\citep{johnson2019billion} and BM25~\citep{Amati2009} for memory retrieval.

\paragraph{Datasets \& Evaluation Metrics}
We evaluate {\sysname} and other baseline methods for long-term conversations on the following benchmarks:

(i) \textit{LOCOMO}~\citep{maharana2024evaluating}, which is the longest conversation dataset to date, with an average of 300 turns with 9K tokens per sample. 
For the test set, we prompt GPT-4 to generate QA pairs for each session as in \citet{alonso2024toward}.
We also conduct evaluation on the recently released official QA pairs in Appendix~\ref{sec: main_locomo2}.

(ii) \textit{Long-MT-Bench+}, which is reconstructed from \textit{MT-Bench+}~\citep{lu2023memochat}, where human experts are invited to expand the original questions and create long-range questions as test user requests. Since each conversation only contains an average of 13.3 dialogue turns, following \citet{yuan2023evolving}, we merge five consecutive sessions into one long-term conversation. We also use these human-written questions as few-shot examples to prompt GPT-4 to generate a long-range test question for each dialogue topic as the test set. More details such as the statistics of the constructed \textit{Long-MT-Bench+} are listed in Appendix~\ref{sec: dataset_details}.

For evaluation metrics, we use the conventional \textit{BLEU}~\citep{papineni2002bleu}, \textit{ROUGE}~\citep{lin2004rouge}, and \textit{BERTScore}~\citep{zhangbertscore} for basic evaluation. Inspired by~\citep{pan2023rewards}, we employ \textit{GPT4Score} 
for more accurate evaluation, where \texttt{GPT-4-0125} is prompted to assign an integer rating from 0 (poor) to 100 (excellent).
We also perform \textit{pairwise comparisons} by instructing GPT-4 to determine the superior response. The evaluation prompts are detailed in Figure~\ref{fig: prompt4eval} of Appendix~\ref{sec: prompt4eval}. Human evaluation is also conducted, with results summarized in Table~\ref{tab: human_evaluation} in Appendix~\ref{sec: human_evaluation}.

\paragraph{Baselines} We evaluate our method against four intuitive approaches and four state-of-the-art models. As Figure~\ref{fig: similarity_mpnet} indicates, the compression-based memory denoising mechanism can benefit % also benefits the turn-level and session-level 
memory retrieval, in the main results, we directly compare our method to the denoising-enhanced turn-level and session-level baselines.
(1) \textit{Turn-Level}, which constructs the memory bank by treating each user-agent interaction as a distinct memory unit.
(2) \textit{Session-Level}, which uses each entire conversation session as a memory unit.
(3) \textit{Zero History}, which generates responses without incorporating any conversation history, operating in a zero-shot manner.
(4) \textit{Full History}, which concatenates all prior conversation history as the context for response generation.
(5) \textit{SumMem}~\citep{langchain2023summary}, which dynamically generates summaries of past dialogues relevant to the target user request, and uses these summaries as context for response generation. % No retriever here.
(6) \textit{RecurSum}~\citep{wang2023recursively}, which recursively updates summary using current session and previous summaries, and takes the updated summary of current session as the context. % No retriever here.
(7) \textit{ConditionMem}~\citep{yuan2023evolving}, which generates summaries and knowledge for each dialogue turn, then retrieves the most relevant summary, knowledge, and raw conversation turn as the context in response to a new user request. 
(8) \textit{MemoChat}~\citep{lu2023memochat}, which operates memories at segment level, but focuses on tuning LLMs for both memory construction and retrieval.

% Please add the following required packages to your document preamble:

% Beamer presentation requires \usepackage{colortbl} instead of \usepackage[table,xcdraw]{xcolor}
\begin{table*}[t]
\centering
\caption{Main Results. Eurus-2-7B-PRIME demonstrates the best reasoning ability.}
\label{tab:main_results}
\resizebox{\textwidth}{!}{
\begin{tabular}{lcccccc}
\toprule
\textbf{Model}                     & \textbf{AIME 2024}                           & \textbf{MATH-500} & \textbf{AMC}          & \textbf{Minerva Math} & \textbf{OlympiadBench} & \textbf{Avg.}          \\ \midrule
\textbf{GPT-4o}                    & 9.3                                          & 76.4              & 45.8                  & 36.8                  & \textbf{43.3}          & 43.3                   \\
\textbf{Llama-3.1-70B-Instruct}    & 16.7                                         & 64.6              & 30.1                  & 35.3                  & 31.9                   & 35.7                   \\
\textbf{Qwen-2.5-Math-7B-Instruct} & 13.3                                         & \textbf{79.8}     & 50.6                  & 34.6                  & 40.7                   & 43.8                   \\
\textbf{Eurus-2-7B-SFT}            & 3.3                                          & 65.1              & 30.1                  & 32.7                  & 29.8                   & 32.2                   \\
\textbf{Eurus-2-7B-PRIME}          & \textbf{26.7 {\color[HTML]{009901} (+23.3)}} & 79.2 {\color[HTML]{009901}(+14.1)}      & \textbf{57.8 {\color[HTML]{009901}(+27.7)}} & \textbf{38.6 {\color[HTML]{009901}(+5.9)}}  & 42.1 {\color[HTML]{009901}(+12.3) }          & \textbf{48.9 {\color[HTML]{009901}(+ 16.7)}} \\ \bottomrule
\end{tabular}
}
\end{table*}
\begin{figure}[t]
    \centering
    \subfloat[{\sysname} \textit{v.s.}  state-of-the-art methods]{
    \includegraphics[width=.47\textwidth]{figures/win_rate_locomo_baselines.pdf}
    \label{fig: compare_baselines}
    }
    \subfloat[{\sysname} (segment-level) \textit{v.s.} other granularities]{
    \includegraphics[width=.47\textwidth]{figures/win_rate_locomo_granularity.pdf}
    \label{fig: compare_granularity}
    }
    \caption{GPT-4 based pairwise performance comparison on LOCOMO with BM25 based retriever.}
    \label{fig: main_compare}
\end{figure}

\paragraph{Main Results}
As shown in Table~\ref{tab: main_results} and Figure~\ref{fig: main_compare}, \textit{{\sysname} outperforms all baseline approaches}, exhibiting a significant performance advantage, particularly on the long-conversation benchmark LOCOMO.
Interestingly, there is a significant performance disparity in Turn-Level and Session-Level methods when using different retrieval models. For instance, switching from the MPNet-based retriever to the BM25-based retriever results in performance improvements up to 11.98 and 7.89 points in terms of GPT4Score on LOCOMO and Long-MT-Bench+, respectively. 
In contrast, \textit{{\sysname} demonstrates greater robustness in terms of the deployed retrieval system}. We attribute this to the following reason:
As discussed in Section \ref{sec:intro}, turn-level memory units are often fragmented and may not explicitly include or relate to keywords mentioned in the target user request. On the other hand, session-level memory units contain a large amount of irrelevant information. Both of these scenarios
make the retrieval performance sensitive to the capability of the deployed retrieval system.
However, topical segments in {\sysname} can strike a balance between including more relevant, coherent information while excluding irrelevant content, thus leading to more robust and superior retrieval performance.
Table~\ref{tab: main_results} and Figure~\ref{fig: main_compare} also reveal that \textit{summary based methods, such as SumMem and RecurSum fall behind turn-level or session-level baselines}. Our case study, Figure~\ref{fig: case_study_segment_vs_rsum} and~\ref{fig: case_study_segment_vs_condmem} in Appendix~\ref{sec: case_study}, suggests that this is likely due to the loss of crucial details during the process of converting dialogues into summaries~\citep{maharana2024evaluating}, which are essential for accurate question answering. Furthermore, Table~\ref{tab: main_results} shows that \sysname\ maintains the advantage over baseline methods when switching the segmentation model from GPT-4 to Mistral-7B. Notably, even with a RoBERTa-based segmentation model, \sysname\ retains competitive performance compared to other granularity-based baselines. 
% We also conduct additional experiments on a subset of \textit{Persona-Chat} and \textit{CoQA}. The results in Table~\ref{tab: results_on_coqa} and~\ref{tab: results_on_spc} further validate SeCom's robustness in more conversation scenarios.

\paragraph{Ablation Study on Granularity of Memory Units}
Figure~\ref{fig: bm25_granularity}, Figure~\ref{fig: mpnet_granularity}, and Table~\ref{tab: main_mistral} have clearly demonstrated the superiority of segment-level memory over turn-level and session-level memory in terms of both retrieval accuracy and end-to-end QA performance.
Figure~\ref{fig: gpt4score_bm25} and Figure~\ref{fig: gpt4score_mpnet} further compare QA performance across different memory granularities under varying context budgets. Compression-based memory unit denoising was applied in all experiments here to isolate the end-to-end impact of memory granularity on performance. The results show that segment-level memory consistently outperforms both turn-level and session-level memory across a range of context budgets, reaffirming its superiority. Figures~\ref{fig: case_study_segment_vs_turn} and \ref{fig: case_study_segment_vs_session} in Appendix~\ref{sec: case_study} provide detailed case studies.

\begin{figure}[htbp]
\centering
\hspace{1.65em} \includegraphics[width=.71\linewidth]{figures/gpt4score_mpnet_mtbp_legend.pdf}
\vspace{-4mm}
\\
\subfloat[BM25 based Retriever] {\includegraphics[width=.35\textwidth]{figures/gpt4score_bm25_mtbp.pdf}
\label{fig: gpt4score_bm25}
}
\hspace{2em}
\subfloat[MPNet based Retriever] {\includegraphics[width=.35\textwidth]{figures/gpt4score_mpnet_mtbp.pdf}
\label{fig: gpt4score_mpnet}
}
\caption{Performance comparison of different memory granularities with various context budget on \textit{Long-MT-Bench+}.}
\label{fig: score_wrt_token}
\end{figure}

\begin{table*}[!h]
\centering
\small
% \setlength{\tabcolsep}{1.5mm}
% \caption{Ablation study on compression based memory denoising. Compression rate: 75\%. Retriever: MPNet.}
\caption{Ablation study on compression-based memory denoising with a compression rate of 75\% using the MPNet based retriever.}
\label{tab: ablation_compression}
    % \newcommand{\redcross}{\textcolor{red}{$\times$}}
    % \begin{adjustbox}{width=\linewidth, height=6cm, keepaspectratio}
    \resizebox{1.\columnwidth}{!}{

    \begin{tabular}{l|cccc|cccc}
    \toprule
    \multirow{2}{*}{\textbf{Methods}} &  \multicolumn{4}{@{}c|}{{\bf LOCOMO}} &  \multicolumn{4}{@{}c}{{\bf Long-MT-Bench+}}\\
    \cmidrule (lr){2-5} \cmidrule (lr){6-9}
    & GPT4Score & BLEU & Rouge2 &  BERTScore & GPT4Score & BLEU & Rouge2 &  BERTScore\\
    \midrule
    {\sysname} & \textbf{69.33} & \textbf{7.19} & \textbf{13.74} & \textbf{88.60} & \textbf{88.81} & \textbf{13.80} & \textbf{19.21} & \textbf{87.72}\\
    {~~$-$ Denoise} & 59.87 & 6.49 & 12.11 & 88.16 & 87.51 & 12.94 & 18.73 & 87.44 \\ 
    \bottomrule
    \end{tabular}
    }
\end{table*}
\paragraph{Ablation Study on Compression based Memory Denoising}
As shown in Table~\ref{tab: ablation_compression}, removing the proposed compression based memory denoising mechanism will result in a performance drop up to 9.46 points of GPT4Score on LOCOMO, highlighting the critical role of this denoising mechanism: by effectively improving the retrieval system (Figure~\ref{fig: recall_wrt_comp_rate_mpnet}), it significantly enhances the overall effectiveness of the system.

\paragraph{Mistral-7B Powered Response Generation}
Table~\ref{tab: main_mistral} presents the results of {\sysname} and baselines using \texttt{Mistral-7B-Instruct-v0.3}\footnote{\url{https://huggingface.co/mistralai/Mistral-7B-Instruct-v0.3}}~\citep{jiang2023mistral7b} as the response generator.
Our method demonstrates a significant performance gain over other baselines, showcasing its good generalization ability across different LLM-powered conversation agents.
Interestingly, although the Mistral-7B here features a 32K context window capable of accommodating the entire conversation history, in other words, it is able to include and comprehend the entire conversation history without truncation, the performance of the ``Full History'' approach still falls short compared to \sysname. This highlights the effectiveness of our memory construction and retrieval mechanisms, which prioritize relevant context and reduce noise, leading to more accurate and contextually appropriate responses.

\begin{table*}[h]
    \small
    \centering
    \caption{Performance comparison on \textit{Long-MT-Bench+} using \texttt{Mistral-7B-Instruct-v0.3}. Other settings are the same as Table~\ref{tab: main_results}.
    }
    \label{tab: main_mistral}
    \resizebox{\linewidth}{!}{
    \begin{tabular}{l|cccccc|cc}
    \toprule
    
    \multirow{2}{*}{\textbf{Methods}} &  \multicolumn{6}{@{}c|}{{\bf QA Performance}} & \multicolumn{2}{@{}c}{{\bf Context Length}} \\
    \cmidrule (lr){2-7} \cmidrule (lr){8-9}
    & GPT4Score & BLEU & Rouge1 & Rouge2 & RougeL & BERTScore & \# Turns & \# Tokens \\
    
    \midrule
    Full History & 78.73 & 10.25 & 29.43 & 14.32 & 23.37 & 86.77 & 65.45 & 19,287 \\
    \midrule
    \multicolumn{9}{@{}c}{{ \textit{ BM25 Based Retriever } }} 
    \\
    \midrule
    Turn-Level & 83.14 & 13.60 & 33.28 & 19.11 & 27.32 & 87.52 & 3.00 & 1,047 \\
    Session-Level & 81.03 & 12.49 & 32.39 & 17.11 & 25.66 & 87.21 & 13.35 & 4,118 \\
    \sysname\ & \textbf{89.43} & \textbf{15.06} & \textbf{35.77} & \textbf{21.35} & \textbf{29.50} & \textbf{87.89} & 2.87 & 906 \\
    \midrule
    \multicolumn{9}{@{}c}{{ \textit{ MPNet Based Retriever } }} 
    \\
    \midrule
    Turn-Level & 85.61 & 12.78 & 35.06 & 19.61 & 28.51 & 87.77 & 3.00 & 909 \\
    Session-Level & 75.29 & 9.14 & 28.65 & 13.91 & 22.52 & 86.51 & 13.43 & 3,680 \\
    \sysname\ & \textbf{90.58} & \textbf{15.80} & \textbf{36.14} & \textbf{21.49} & \textbf{29.94} & \textbf{88.07} & 2.77 & 820 \\
    
    \bottomrule
    \end{tabular}
    }
\end{table*}

\paragraph{Evaluation of Conversation Segmentation Model}
To evaluate the conversation segmentation module described in Section \ref{sec: method_segment} independently, we use three widely used dialogue segmentation datasets: DialSeg711~\citep{xu2021topic}, TIAGE~\citep{xie2021tiage}, and SuperDialSeg~\citep{jiang2023superdialseg}. In addition to the unsupervised (zero-shot) setting, we also assess performance in a transfer learning setting, where baseline models are trained on the full training set of the source dataset, while our model learns the segmentation rubric through LLM reflection on the top 100 most challenging examples.
We evaluate transfer learning only using SuperDialSeg and TIAGE as the source datasets since DialSeg711 lacks a training set.
For evaluation metrics, following \citet{jiang2023superdialseg}, we use the F1 score, $P_k$~\citep{beeferman1999statistical}, Window Diff (WD)~\citep{pevzner2002critique} and the segment score\footnote{Recommended by ICASSP2023 General Meeting Understanding and Generation Challenge \url{https://2023.ieeeicassp.org/signal-processing-grand-challenges}.}:
\begin{equation}
\textit{Score}=\frac{2 * F 1+\left(1-\textit{P}_k\right)+(1-\textit{WD})}{4}.
\label{eq:segment_score}
\end{equation}
Table \ref{tab: segment_main} presents the results, showing that our segmentation model consistently outperforms baselines in the unsupervised setting.
In the transfer learning setting, despite the segmentation rubric being learned from LLM reflection on only 100 examples from the source dataset, it generalizes well to the target dataset, surpassing the baseline model trained on the full source training set and even outperforming some supervised baselines. 
% \textcolor{red}{We also experiment with more light-weight segmentation models, \textit{Mistral-7B-Instruct-v0.3} and a model fine-tuned from \textit{RoBERTa}. The experiment results in Table~\ref{tab: main_results_slm_seg} demonstrate that smaller segmentation models can perform segmentation well, making our approach applicable to low-resource environments.}

\begin{table*}[t]
    \small
    \centering
    \setlength{\tabcolsep}{1mm}
    \caption{Segmentation performances on three datasets.
    $^{\dag}$: numbers reported in \citet{gao2023unsupervised}. Other baselines are reported in \citet{jiang2023superdialseg}. The best performance is highlighted in \textbf{bold}, and the second best is highlighted by \underline{underline}. \colorbox{light_gray}{Numbers in gray} correspond to \textbf{supervised} setting.}
    \label{tab: segment_main}
    \resizebox{\columnwidth}{!}{
    
    \begin{tabular}{l|cccc|cccc|cccc}
    \toprule
    \multirow{2}{*}{\textbf{Methods}} & \multicolumn{4}{@{}c}{{\bf Dialseg711}} & \multicolumn{4}{@{}c}{{\bf SuperDialSeg}} & \multicolumn{4}{@{}c}{{\bf TIAGE}}  \\
    \cmidrule (lr){2-5} \cmidrule (lr){6-9} \cmidrule (lr){10-13} 
    & Pk↓ & WD↓ & F1↑ & Score↑ & Pk↓ & WD↓ & F1↑ & Score↑ & Pk↓ & WD↓ & F1↑ & Score↑ \\

    \midrule
    \multicolumn{13}{@{}c}{\textbf{Unsupervised}} \\
    \midrule
    
    BayesSeg & 0.306 & 0.350 & 0.556 & 0.614 & \underline{0.433} & 0.593 & \underline{0.438} & 0.463 & 0.486 & 0.571 & 0.366 & 0.419\\
    TextTiling & 0.470 & 0.493 & 0.245 & 0.382 & 0.441 & \underline{0.453} & 0.388 & \underline{0.471} & 0.469 & 0.488 & 0.204 & 0.363 \\
    GraphSeg & 0.412 & 0.442 & 0.392 & 0.483 & 0.450 & 0.454 & 0.249 & 0.398 & 0.496 & 0.515 & 0.238 & 0.366\\
    \midrule
    TextTiling+Glove & 0.399 & 0.438 & 0.436 & 0.509 & 0.519 & 0.524 & 0.353 & 0.416 & 0.486 & 0.511 & 0.236 & 0.369\\
    TextTiling+[CLS] & 0.419 & 0.473 & 0.351 & 0.453 & 0.493 & 0.523 & 0.277 & 0.385 & 0.521 & 0.556 & 0.218 & 0.340 \\
    TextTiling+NSP & 0.347 & 0.360 & 0.347 & 0.497 & 0.512 & 0.521 & 0.208 & 0.346 & 0.425 & 0.439 & 0.285 & 0.426\\
    GreedySeg & 0.381 & 0.410 & 0.445 & 0.525 & 0.490 & 0.494 & 0.365 & 0.437 & 0.490 & 0.506 & 0.181 & 0.341\\
    CSM & 0.278 & 0.302 & \underline{0.610} & \underline{0.660} & 0.462 & 0.467 & 0.381 & 0.458 & \underline{0.400} & \underline{0.420} & \underline{0.427} & \underline{0.509} \\
    DialSTART $^{\dag}$ & \underline{0.178} & \underline{0.198} & - & - & - & - & - & - & - & - & - & - \\
    \midrule
    \textbf{Ours (zero-shot)} & \textbf{0.093} & \textbf{0.103} & \textbf{0.888} & \textbf{0.895} & \textbf{0.277} & \textbf{0.289} & \textbf{0.758} & \textbf{0.738} & \textbf{0.363} & \textbf{0.401} & \textbf{0.596} & \textbf{0.607} \\

    \midrule
    \multicolumn{13}{@{}c}{\textbf{Transfer from TIAGE to Target}}\\
    \midrule
    $\text{TextSeg}_{dial}$ & 0.476 & 0.491 & 0.182 & 0.349 & 0.552 & 0.570 & 0.199 & 0.319 & \colorbox{light_gray}{0.357} & \colorbox{light_gray}{0.386} & \colorbox{light_gray}{0.450} & \colorbox{light_gray}{0.539} \\
    BERT & 0.441 & 0.411 & 0.005 & 0.297 & 0.511 & 0.513 & 0.043 & 0.266 & \colorbox{light_gray}{0.418} & \colorbox{light_gray}{0.435} & \colorbox{light_gray}{0.124} & \colorbox{light_gray}{0.349} \\
    RoBERTa & \underline{0.197} & \underline{0.210} & \underline{0.650} & \underline{0.723} & \underline{0.434} & \underline{0.436} & \underline{0.276} & \underline{0.420} & \colorbox{light_gray}{\textbf{0.265}} & \colorbox{light_gray}{\textbf{0.287}} & \colorbox{light_gray}{\underline{0.572}} & \colorbox{light_gray}{\textbf{0.648}} \\
    \midrule
    \textbf{Ours (w/ reflection)} & \textbf{0.050} & \textbf{0.056} & \textbf{0.921} & \textbf{0.934} & \textbf{0.265} & \textbf{0.273} & \textbf{0.765} & \textbf{0.748} & \underline{0.333} & \underline{0.362} & \textbf{0.632} & \underline{0.642} \\
    
    \midrule
    \multicolumn{13}{@{}c}{\textbf{Transfer from SuperDialseg to Target}}\\
    \midrule

    $\text{TextSeg}_{dial}$ & 0.453 & 0.461 & 0.367 & 0.455 & \colorbox{light_gray}{\underline{0.199}} & \colorbox{light_gray}{\underline{0.204}} & \colorbox{light_gray}{0.760} & \colorbox{light_gray}{\underline{0.779}} & 0.489 & 0.508 & 0.266 & 0.384 \\
    BERT & 0.401 & 0.473 & 0.381 & 0.472 & \colorbox{light_gray}{0.214} & \colorbox{light_gray}{0.225} & \colorbox{light_gray}{0.725} & \colorbox{light_gray}{0.753} & 0.492 & 0.526 & 0.226 & 0.359 \\
    RoBERTa & \underline{0.241} & \underline{0.272} & \underline{0.660} & \underline{0.702} & \colorbox{light_gray}{\textbf{0.185}} & \colorbox{light_gray}{\textbf{0.192}} & \colorbox{light_gray}{\textbf{0.784}} & \colorbox{light_gray}{\textbf{0.798}} & \underline{0.401} & \underline{0.418} & \underline{0.373} & \underline{0.482} \\
    \midrule
    \textbf{Ours (w/ refletion)} & \textbf{0.049} & \textbf{0.054} & \textbf{0.924} & \textbf{0.936} & 0.256 & 0.264 & \underline{0.776} & 0.758 & \textbf{0.318} & \textbf{0.345} & \textbf{0.634} & \textbf{0.651} \\
    
    \bottomrule
    \end{tabular}
    }
\end{table*}



\section{Related Works}
\subsection{Memory Management in Conversation}
Long-term open-domain conversation~\citep{feng2020doc2dial, xu2022beyond, maharana2024evaluating} poses significant challenges for LLM-powered conversational agents. To address this, memory management~\citep{lu2023memochat, wang2023recursively, zhong2024memorybank, wu2024tokenselect, li2024hello, zhang2024survey} is widely adopted. The core of memory management involves leveraging dialogue history to provide background information, extract persona, understand the user's intent, and generate history-aware responses.
For instance, MPC~\citep{lee2023prompted}, MemoryBank~\citep{zhong2024memorybank} and COMEDY~\citep{chen2024compress} further summarize past events in the conversation history as memory records. Methods such as RecurSum~\citep{wang2023recursively} and ConditionMem~\citep{yuan2023evolving} consider the memory updating process through recursive summarization.

Inspired by the success of retrieval-augmented generation (RAG), many recent works introduce retrieval modules into memory management. For example, MSC~\citep{xu2022beyond} utilizes a pre-trained Dense Passage Retriever (DPR)~\citep{karpukhin2020dense} model to select the top \textit{N} relevant summaries. 
Instead of using a retrieval model, MemoChat~\citep{lu2023memochat} employs an LLM to retrieve relevant memory records.
Recently, \citet{maharana2024evaluating} release a dataset, \textit{LOCOMO}, which is specifically designed to assess long-term conversational memory, highlighting the effectiveness of RAG in maintaining long-term memory. Their experiment results indicate that long-context LLMs are prone to generating hallucinations, and summary-only memory  results in sub-optimal performance due to information loss.

\subsection{Chunking Granularity in RAG System}

Chunking granularity~\citep{duarte2024lumberchunker} (i.e., how the entire context is segmented into retrieval units) is a crucial aspect of RAG systems. Ineffective segmentation can result in incomplete or noisy retrieval units, which can impair the retrieval module~\citep{yu2023chain} and negatively impact the subsequent response generation~\citep{shi2023large}.

Semantic-based chunking strategies~\citep{anurag2023chunkingstrategies, antematter2024optimizing, kamradt2024semantic} use representation similarity to identify topic shifts and decide chunk boundaries. 
With the advancement of LLMs, some studies leverage their capabilities to segment context into retrieval units. For instance, 
LumberChunker~\citep{duarte2024lumberchunker} segments narrative documents into semantically coherent chunks using Gemini~\citep{team2023gemini}. However, existing research mainly focuses on document chunking, overlooking conversation chunking. Common chunking practices~\citep{langchain2023conversational, llamaindex2023buffer} in conversations directly rely on the natural structure (\textit{i.e.,} utterances or dialogue turns) of dialogue to divide conversation into retrieval units.

\subsection{Denoising in RAG system}
Recent studies have observed that noise in conversations can negatively impact the retrieval module in RAG systems. For example, COTED~\citep{mao2022curriculum} found that redundant noise in dialogue rounds significantly impairs conversational search. 
Earlier research~\citep{strzalkowski1998summarization, wasson2002using} investigates the use of summaries in retrieval systems. With the advent of LLM, recent approaches~\citep{ravfogel2023retrieving, lee2024effective} denoise raw dialogues by prompting LLMs to summarize. Subsequently, they fine-tune the retriever's embedding model to align vector representations of original text with those of generated summaries. However, these methods have several drawbacks: (1) summarization introduces latency and computational costs, whereas dialogue state methods require high-quality annotated data. (2) Fine-tuning the retriever's embedding model limits flexibility and scalability, restricting it from being used as a plug-and-play method. (3) Fine-tuning risks overfitting and catastrophic forgetting \citep{mccloskey1989catastrophic, lee2022sequential}, potentially impeding domain adaptation and generalization ability of pre-trained retrievers.

\section{Conclusion}

In this paper, we systematically investigate the impact of memory granularity on retrieval-augmented response generation for long-term conversational agents. Our findings reveal the limitations of turn-level and session-level memory granularities, as well as summarization-based methods. To overcome these challenges, we introduce \sysname, a novel memory management system that constructs a memory bank at the segment-level and employs compression-based denoising techniques to enhance retrieval performance. The experimental results underscore the effectiveness of \sysname\ in handling long-term conversations. Further analysis and ablation studies confirm the contributions of the segment-level memory units and the compression-based denoising technique within our framework. 


\documentclass{article} % For LaTeX2e
\usepackage{iclr2025_conference,times}

% Optional math commands from https://github.com/goodfeli/dlbook_notation.
%%%%% NEW MATH DEFINITIONS %%%%%

\usepackage{amsmath,amsfonts,bm}
\usepackage{derivative}
% Mark sections of captions for referring to divisions of figures
\newcommand{\figleft}{{\em (Left)}}
\newcommand{\figcenter}{{\em (Center)}}
\newcommand{\figright}{{\em (Right)}}
\newcommand{\figtop}{{\em (Top)}}
\newcommand{\figbottom}{{\em (Bottom)}}
\newcommand{\captiona}{{\em (a)}}
\newcommand{\captionb}{{\em (b)}}
\newcommand{\captionc}{{\em (c)}}
\newcommand{\captiond}{{\em (d)}}

% Highlight a newly defined term
\newcommand{\newterm}[1]{{\bf #1}}

% Derivative d 
\newcommand{\deriv}{{\mathrm{d}}}

% Figure reference, lower-case.
\def\figref#1{figure~\ref{#1}}
% Figure reference, capital. For start of sentence
\def\Figref#1{Figure~\ref{#1}}
\def\twofigref#1#2{figures \ref{#1} and \ref{#2}}
\def\quadfigref#1#2#3#4{figures \ref{#1}, \ref{#2}, \ref{#3} and \ref{#4}}
% Section reference, lower-case.
\def\secref#1{section~\ref{#1}}
% Section reference, capital.
\def\Secref#1{Section~\ref{#1}}
% Reference to two sections.
\def\twosecrefs#1#2{sections \ref{#1} and \ref{#2}}
% Reference to three sections.
\def\secrefs#1#2#3{sections \ref{#1}, \ref{#2} and \ref{#3}}
% Reference to an equation, lower-case.
\def\eqref#1{equation~\ref{#1}}
% Reference to an equation, upper case
\def\Eqref#1{Equation~\ref{#1}}
% A raw reference to an equation---avoid using if possible
\def\plaineqref#1{\ref{#1}}
% Reference to a chapter, lower-case.
\def\chapref#1{chapter~\ref{#1}}
% Reference to an equation, upper case.
\def\Chapref#1{Chapter~\ref{#1}}
% Reference to a range of chapters
\def\rangechapref#1#2{chapters\ref{#1}--\ref{#2}}
% Reference to an algorithm, lower-case.
\def\algref#1{algorithm~\ref{#1}}
% Reference to an algorithm, upper case.
\def\Algref#1{Algorithm~\ref{#1}}
\def\twoalgref#1#2{algorithms \ref{#1} and \ref{#2}}
\def\Twoalgref#1#2{Algorithms \ref{#1} and \ref{#2}}
% Reference to a part, lower case
\def\partref#1{part~\ref{#1}}
% Reference to a part, upper case
\def\Partref#1{Part~\ref{#1}}
\def\twopartref#1#2{parts \ref{#1} and \ref{#2}}

\def\ceil#1{\lceil #1 \rceil}
\def\floor#1{\lfloor #1 \rfloor}
\def\1{\bm{1}}
\newcommand{\train}{\mathcal{D}}
\newcommand{\valid}{\mathcal{D_{\mathrm{valid}}}}
\newcommand{\test}{\mathcal{D_{\mathrm{test}}}}

\def\eps{{\epsilon}}


% Random variables
\def\reta{{\textnormal{$\eta$}}}
\def\ra{{\textnormal{a}}}
\def\rb{{\textnormal{b}}}
\def\rc{{\textnormal{c}}}
\def\rd{{\textnormal{d}}}
\def\re{{\textnormal{e}}}
\def\rf{{\textnormal{f}}}
\def\rg{{\textnormal{g}}}
\def\rh{{\textnormal{h}}}
\def\ri{{\textnormal{i}}}
\def\rj{{\textnormal{j}}}
\def\rk{{\textnormal{k}}}
\def\rl{{\textnormal{l}}}
% rm is already a command, just don't name any random variables m
\def\rn{{\textnormal{n}}}
\def\ro{{\textnormal{o}}}
\def\rp{{\textnormal{p}}}
\def\rq{{\textnormal{q}}}
\def\rr{{\textnormal{r}}}
\def\rs{{\textnormal{s}}}
\def\rt{{\textnormal{t}}}
\def\ru{{\textnormal{u}}}
\def\rv{{\textnormal{v}}}
\def\rw{{\textnormal{w}}}
\def\rx{{\textnormal{x}}}
\def\ry{{\textnormal{y}}}
\def\rz{{\textnormal{z}}}

% Random vectors
\def\rvepsilon{{\mathbf{\epsilon}}}
\def\rvphi{{\mathbf{\phi}}}
\def\rvtheta{{\mathbf{\theta}}}
\def\rva{{\mathbf{a}}}
\def\rvb{{\mathbf{b}}}
\def\rvc{{\mathbf{c}}}
\def\rvd{{\mathbf{d}}}
\def\rve{{\mathbf{e}}}
\def\rvf{{\mathbf{f}}}
\def\rvg{{\mathbf{g}}}
\def\rvh{{\mathbf{h}}}
\def\rvu{{\mathbf{i}}}
\def\rvj{{\mathbf{j}}}
\def\rvk{{\mathbf{k}}}
\def\rvl{{\mathbf{l}}}
\def\rvm{{\mathbf{m}}}
\def\rvn{{\mathbf{n}}}
\def\rvo{{\mathbf{o}}}
\def\rvp{{\mathbf{p}}}
\def\rvq{{\mathbf{q}}}
\def\rvr{{\mathbf{r}}}
\def\rvs{{\mathbf{s}}}
\def\rvt{{\mathbf{t}}}
\def\rvu{{\mathbf{u}}}
\def\rvv{{\mathbf{v}}}
\def\rvw{{\mathbf{w}}}
\def\rvx{{\mathbf{x}}}
\def\rvy{{\mathbf{y}}}
\def\rvz{{\mathbf{z}}}

% Elements of random vectors
\def\erva{{\textnormal{a}}}
\def\ervb{{\textnormal{b}}}
\def\ervc{{\textnormal{c}}}
\def\ervd{{\textnormal{d}}}
\def\erve{{\textnormal{e}}}
\def\ervf{{\textnormal{f}}}
\def\ervg{{\textnormal{g}}}
\def\ervh{{\textnormal{h}}}
\def\ervi{{\textnormal{i}}}
\def\ervj{{\textnormal{j}}}
\def\ervk{{\textnormal{k}}}
\def\ervl{{\textnormal{l}}}
\def\ervm{{\textnormal{m}}}
\def\ervn{{\textnormal{n}}}
\def\ervo{{\textnormal{o}}}
\def\ervp{{\textnormal{p}}}
\def\ervq{{\textnormal{q}}}
\def\ervr{{\textnormal{r}}}
\def\ervs{{\textnormal{s}}}
\def\ervt{{\textnormal{t}}}
\def\ervu{{\textnormal{u}}}
\def\ervv{{\textnormal{v}}}
\def\ervw{{\textnormal{w}}}
\def\ervx{{\textnormal{x}}}
\def\ervy{{\textnormal{y}}}
\def\ervz{{\textnormal{z}}}

% Random matrices
\def\rmA{{\mathbf{A}}}
\def\rmB{{\mathbf{B}}}
\def\rmC{{\mathbf{C}}}
\def\rmD{{\mathbf{D}}}
\def\rmE{{\mathbf{E}}}
\def\rmF{{\mathbf{F}}}
\def\rmG{{\mathbf{G}}}
\def\rmH{{\mathbf{H}}}
\def\rmI{{\mathbf{I}}}
\def\rmJ{{\mathbf{J}}}
\def\rmK{{\mathbf{K}}}
\def\rmL{{\mathbf{L}}}
\def\rmM{{\mathbf{M}}}
\def\rmN{{\mathbf{N}}}
\def\rmO{{\mathbf{O}}}
\def\rmP{{\mathbf{P}}}
\def\rmQ{{\mathbf{Q}}}
\def\rmR{{\mathbf{R}}}
\def\rmS{{\mathbf{S}}}
\def\rmT{{\mathbf{T}}}
\def\rmU{{\mathbf{U}}}
\def\rmV{{\mathbf{V}}}
\def\rmW{{\mathbf{W}}}
\def\rmX{{\mathbf{X}}}
\def\rmY{{\mathbf{Y}}}
\def\rmZ{{\mathbf{Z}}}

% Elements of random matrices
\def\ermA{{\textnormal{A}}}
\def\ermB{{\textnormal{B}}}
\def\ermC{{\textnormal{C}}}
\def\ermD{{\textnormal{D}}}
\def\ermE{{\textnormal{E}}}
\def\ermF{{\textnormal{F}}}
\def\ermG{{\textnormal{G}}}
\def\ermH{{\textnormal{H}}}
\def\ermI{{\textnormal{I}}}
\def\ermJ{{\textnormal{J}}}
\def\ermK{{\textnormal{K}}}
\def\ermL{{\textnormal{L}}}
\def\ermM{{\textnormal{M}}}
\def\ermN{{\textnormal{N}}}
\def\ermO{{\textnormal{O}}}
\def\ermP{{\textnormal{P}}}
\def\ermQ{{\textnormal{Q}}}
\def\ermR{{\textnormal{R}}}
\def\ermS{{\textnormal{S}}}
\def\ermT{{\textnormal{T}}}
\def\ermU{{\textnormal{U}}}
\def\ermV{{\textnormal{V}}}
\def\ermW{{\textnormal{W}}}
\def\ermX{{\textnormal{X}}}
\def\ermY{{\textnormal{Y}}}
\def\ermZ{{\textnormal{Z}}}

% Vectors
\def\vzero{{\bm{0}}}
\def\vone{{\bm{1}}}
\def\vmu{{\bm{\mu}}}
\def\vtheta{{\bm{\theta}}}
\def\vphi{{\bm{\phi}}}
\def\va{{\bm{a}}}
\def\vb{{\bm{b}}}
\def\vc{{\bm{c}}}
\def\vd{{\bm{d}}}
\def\ve{{\bm{e}}}
\def\vf{{\bm{f}}}
\def\vg{{\bm{g}}}
\def\vh{{\bm{h}}}
\def\vi{{\bm{i}}}
\def\vj{{\bm{j}}}
\def\vk{{\bm{k}}}
\def\vl{{\bm{l}}}
\def\vm{{\bm{m}}}
\def\vn{{\bm{n}}}
\def\vo{{\bm{o}}}
\def\vp{{\bm{p}}}
\def\vq{{\bm{q}}}
\def\vr{{\bm{r}}}
\def\vs{{\bm{s}}}
\def\vt{{\bm{t}}}
\def\vu{{\bm{u}}}
\def\vv{{\bm{v}}}
\def\vw{{\bm{w}}}
\def\vx{{\bm{x}}}
\def\vy{{\bm{y}}}
\def\vz{{\bm{z}}}

% Elements of vectors
\def\evalpha{{\alpha}}
\def\evbeta{{\beta}}
\def\evepsilon{{\epsilon}}
\def\evlambda{{\lambda}}
\def\evomega{{\omega}}
\def\evmu{{\mu}}
\def\evpsi{{\psi}}
\def\evsigma{{\sigma}}
\def\evtheta{{\theta}}
\def\eva{{a}}
\def\evb{{b}}
\def\evc{{c}}
\def\evd{{d}}
\def\eve{{e}}
\def\evf{{f}}
\def\evg{{g}}
\def\evh{{h}}
\def\evi{{i}}
\def\evj{{j}}
\def\evk{{k}}
\def\evl{{l}}
\def\evm{{m}}
\def\evn{{n}}
\def\evo{{o}}
\def\evp{{p}}
\def\evq{{q}}
\def\evr{{r}}
\def\evs{{s}}
\def\evt{{t}}
\def\evu{{u}}
\def\evv{{v}}
\def\evw{{w}}
\def\evx{{x}}
\def\evy{{y}}
\def\evz{{z}}

% Matrix
\def\mA{{\bm{A}}}
\def\mB{{\bm{B}}}
\def\mC{{\bm{C}}}
\def\mD{{\bm{D}}}
\def\mE{{\bm{E}}}
\def\mF{{\bm{F}}}
\def\mG{{\bm{G}}}
\def\mH{{\bm{H}}}
\def\mI{{\bm{I}}}
\def\mJ{{\bm{J}}}
\def\mK{{\bm{K}}}
\def\mL{{\bm{L}}}
\def\mM{{\bm{M}}}
\def\mN{{\bm{N}}}
\def\mO{{\bm{O}}}
\def\mP{{\bm{P}}}
\def\mQ{{\bm{Q}}}
\def\mR{{\bm{R}}}
\def\mS{{\bm{S}}}
\def\mT{{\bm{T}}}
\def\mU{{\bm{U}}}
\def\mV{{\bm{V}}}
\def\mW{{\bm{W}}}
\def\mX{{\bm{X}}}
\def\mY{{\bm{Y}}}
\def\mZ{{\bm{Z}}}
\def\mBeta{{\bm{\beta}}}
\def\mPhi{{\bm{\Phi}}}
\def\mLambda{{\bm{\Lambda}}}
\def\mSigma{{\bm{\Sigma}}}

% Tensor
\DeclareMathAlphabet{\mathsfit}{\encodingdefault}{\sfdefault}{m}{sl}
\SetMathAlphabet{\mathsfit}{bold}{\encodingdefault}{\sfdefault}{bx}{n}
\newcommand{\tens}[1]{\bm{\mathsfit{#1}}}
\def\tA{{\tens{A}}}
\def\tB{{\tens{B}}}
\def\tC{{\tens{C}}}
\def\tD{{\tens{D}}}
\def\tE{{\tens{E}}}
\def\tF{{\tens{F}}}
\def\tG{{\tens{G}}}
\def\tH{{\tens{H}}}
\def\tI{{\tens{I}}}
\def\tJ{{\tens{J}}}
\def\tK{{\tens{K}}}
\def\tL{{\tens{L}}}
\def\tM{{\tens{M}}}
\def\tN{{\tens{N}}}
\def\tO{{\tens{O}}}
\def\tP{{\tens{P}}}
\def\tQ{{\tens{Q}}}
\def\tR{{\tens{R}}}
\def\tS{{\tens{S}}}
\def\tT{{\tens{T}}}
\def\tU{{\tens{U}}}
\def\tV{{\tens{V}}}
\def\tW{{\tens{W}}}
\def\tX{{\tens{X}}}
\def\tY{{\tens{Y}}}
\def\tZ{{\tens{Z}}}


% Graph
\def\gA{{\mathcal{A}}}
\def\gB{{\mathcal{B}}}
\def\gC{{\mathcal{C}}}
\def\gD{{\mathcal{D}}}
\def\gE{{\mathcal{E}}}
\def\gF{{\mathcal{F}}}
\def\gG{{\mathcal{G}}}
\def\gH{{\mathcal{H}}}
\def\gI{{\mathcal{I}}}
\def\gJ{{\mathcal{J}}}
\def\gK{{\mathcal{K}}}
\def\gL{{\mathcal{L}}}
\def\gM{{\mathcal{M}}}
\def\gN{{\mathcal{N}}}
\def\gO{{\mathcal{O}}}
\def\gP{{\mathcal{P}}}
\def\gQ{{\mathcal{Q}}}
\def\gR{{\mathcal{R}}}
\def\gS{{\mathcal{S}}}
\def\gT{{\mathcal{T}}}
\def\gU{{\mathcal{U}}}
\def\gV{{\mathcal{V}}}
\def\gW{{\mathcal{W}}}
\def\gX{{\mathcal{X}}}
\def\gY{{\mathcal{Y}}}
\def\gZ{{\mathcal{Z}}}

% Sets
\def\sA{{\mathbb{A}}}
\def\sB{{\mathbb{B}}}
\def\sC{{\mathbb{C}}}
\def\sD{{\mathbb{D}}}
% Don't use a set called E, because this would be the same as our symbol
% for expectation.
\def\sF{{\mathbb{F}}}
\def\sG{{\mathbb{G}}}
\def\sH{{\mathbb{H}}}
\def\sI{{\mathbb{I}}}
\def\sJ{{\mathbb{J}}}
\def\sK{{\mathbb{K}}}
\def\sL{{\mathbb{L}}}
\def\sM{{\mathbb{M}}}
\def\sN{{\mathbb{N}}}
\def\sO{{\mathbb{O}}}
\def\sP{{\mathbb{P}}}
\def\sQ{{\mathbb{Q}}}
\def\sR{{\mathbb{R}}}
\def\sS{{\mathbb{S}}}
\def\sT{{\mathbb{T}}}
\def\sU{{\mathbb{U}}}
\def\sV{{\mathbb{V}}}
\def\sW{{\mathbb{W}}}
\def\sX{{\mathbb{X}}}
\def\sY{{\mathbb{Y}}}
\def\sZ{{\mathbb{Z}}}

% Entries of a matrix
\def\emLambda{{\Lambda}}
\def\emA{{A}}
\def\emB{{B}}
\def\emC{{C}}
\def\emD{{D}}
\def\emE{{E}}
\def\emF{{F}}
\def\emG{{G}}
\def\emH{{H}}
\def\emI{{I}}
\def\emJ{{J}}
\def\emK{{K}}
\def\emL{{L}}
\def\emM{{M}}
\def\emN{{N}}
\def\emO{{O}}
\def\emP{{P}}
\def\emQ{{Q}}
\def\emR{{R}}
\def\emS{{S}}
\def\emT{{T}}
\def\emU{{U}}
\def\emV{{V}}
\def\emW{{W}}
\def\emX{{X}}
\def\emY{{Y}}
\def\emZ{{Z}}
\def\emSigma{{\Sigma}}

% entries of a tensor
% Same font as tensor, without \bm wrapper
\newcommand{\etens}[1]{\mathsfit{#1}}
\def\etLambda{{\etens{\Lambda}}}
\def\etA{{\etens{A}}}
\def\etB{{\etens{B}}}
\def\etC{{\etens{C}}}
\def\etD{{\etens{D}}}
\def\etE{{\etens{E}}}
\def\etF{{\etens{F}}}
\def\etG{{\etens{G}}}
\def\etH{{\etens{H}}}
\def\etI{{\etens{I}}}
\def\etJ{{\etens{J}}}
\def\etK{{\etens{K}}}
\def\etL{{\etens{L}}}
\def\etM{{\etens{M}}}
\def\etN{{\etens{N}}}
\def\etO{{\etens{O}}}
\def\etP{{\etens{P}}}
\def\etQ{{\etens{Q}}}
\def\etR{{\etens{R}}}
\def\etS{{\etens{S}}}
\def\etT{{\etens{T}}}
\def\etU{{\etens{U}}}
\def\etV{{\etens{V}}}
\def\etW{{\etens{W}}}
\def\etX{{\etens{X}}}
\def\etY{{\etens{Y}}}
\def\etZ{{\etens{Z}}}

% The true underlying data generating distribution
\newcommand{\pdata}{p_{\rm{data}}}
\newcommand{\ptarget}{p_{\rm{target}}}
\newcommand{\pprior}{p_{\rm{prior}}}
\newcommand{\pbase}{p_{\rm{base}}}
\newcommand{\pref}{p_{\rm{ref}}}

% The empirical distribution defined by the training set
\newcommand{\ptrain}{\hat{p}_{\rm{data}}}
\newcommand{\Ptrain}{\hat{P}_{\rm{data}}}
% The model distribution
\newcommand{\pmodel}{p_{\rm{model}}}
\newcommand{\Pmodel}{P_{\rm{model}}}
\newcommand{\ptildemodel}{\tilde{p}_{\rm{model}}}
% Stochastic autoencoder distributions
\newcommand{\pencode}{p_{\rm{encoder}}}
\newcommand{\pdecode}{p_{\rm{decoder}}}
\newcommand{\precons}{p_{\rm{reconstruct}}}

\newcommand{\laplace}{\mathrm{Laplace}} % Laplace distribution

\newcommand{\E}{\mathbb{E}}
\newcommand{\Ls}{\mathcal{L}}
\newcommand{\R}{\mathbb{R}}
\newcommand{\emp}{\tilde{p}}
\newcommand{\lr}{\alpha}
\newcommand{\reg}{\lambda}
\newcommand{\rect}{\mathrm{rectifier}}
\newcommand{\softmax}{\mathrm{softmax}}
\newcommand{\sigmoid}{\sigma}
\newcommand{\softplus}{\zeta}
\newcommand{\KL}{D_{\mathrm{KL}}}
\newcommand{\Var}{\mathrm{Var}}
\newcommand{\standarderror}{\mathrm{SE}}
\newcommand{\Cov}{\mathrm{Cov}}
% Wolfram Mathworld says $L^2$ is for function spaces and $\ell^2$ is for vectors
% But then they seem to use $L^2$ for vectors throughout the site, and so does
% wikipedia.
\newcommand{\normlzero}{L^0}
\newcommand{\normlone}{L^1}
\newcommand{\normltwo}{L^2}
\newcommand{\normlp}{L^p}
\newcommand{\normmax}{L^\infty}

\newcommand{\parents}{Pa} % See usage in notation.tex. Chosen to match Daphne's book.

\DeclareMathOperator*{\argmax}{arg\,max}
\DeclareMathOperator*{\argmin}{arg\,min}

\DeclareMathOperator{\sign}{sign}
\DeclareMathOperator{\Tr}{Tr}
\let\ab\allowbreak


\usepackage{hyperref}
\usepackage{url}
\usepackage{cleveref}
\usepackage{booktabs}
\usepackage{multirow}
\usepackage{subcaption}
\usepackage{adjustbox} % To adjust table sizes
\usepackage{float}

% \iclrfinalcopy

% For table
\usepackage{multirow}

% For figures
\usepackage{graphicx}
% \usepackage[table]{xcolor}
\title{Improved Training Technique for Latent Consistency Models}

% Authors must not appear in the submitted version. They should be hidden
% as long as the \iclrfinalcopy macro remains commented out below.
% Non-anonymous submissions will be rejected without review.
\iclrfinalcopy

\author{Quan Dao$^{*\dagger}$\\
Rutgers University \\
\texttt{quan.dao@rutgers.edu} \\ \And 
Khanh Doan$^{*}$\\
Movian AI, Vietnam \\
\texttt{dnkhanh.k63.bk@gmail.com} \\ \And
Di Liu\\
Rutgers University \\
\texttt{di.liu@rutgers.edu} \\   \And
Trung Le\\
Monash University \\
\texttt{trunglm@monash.edu} \\   \And
Dimitris Metaxas\\
Rutgers University \\
\texttt{dnm@cs.rutgers.edu} \\
}


% The \author macro works with any number of authors. There are two commands
% used to separate the names and addresses of multiple authors: \And and \AND.
%
% Using \And between authors leaves it to \LaTeX{} to determine where to break
% the lines. Using \AND forces a linebreak at that point. So, if \LaTeX{}
% puts 3 of 4 authors names on the first line, and the last on the second
% line, try using \AND instead of \And before the third author name.

\newcommand{\fix}{\marginpar{FIX}}
\newcommand{\new}{\marginpar{NEW}}
\newcommand{\khanh}[1]{\textcolor{orange}{[Khanh: #1]}}
\newcommand{\quan}[1]{\textcolor{red}{[Quan: #1]}}
\newcommand{\diliu}[1]{\textcolor{purple}{[Di Liu: #1]}}
\newcommand{\trung}[1]{\textcolor{cyan}{[Trung: #1]}}
\newcommand{\metaxas}[1]{\textcolor{blue}{[Metaxas: #1]}}
\newcommand{\minisection}[1]{\noindent{\textbf{#1}}}


%\iclrfinalcopy % Uncomment for camera-ready version, but NOT for submission.
\begin{document}


\maketitle
\def\thefootnote{\textsuperscript{$*$}}\footnotetext{Equal contributions.}
\def\thefootnote{\textsuperscript{$\dagger$}}\footnotetext{Project Lead \& Corresponding Author.}

\begin{abstract}
Consistency models are a new family of generative models capable of producing high-quality samples in either a single step or multiple steps. Recently, consistency models have demonstrated impressive performance, achieving results on par with diffusion models in the pixel space. However, the success of scaling consistency training to large-scale datasets, particularly for text-to-image and video generation tasks, is determined by performance in the latent space. In this work, we analyze the statistical differences between pixel and latent spaces, discovering that latent data often contains highly impulsive outliers, which significantly degrade the performance of iCT in the latent space. To address this, we replace Pseudo-Huber losses with Cauchy losses, effectively mitigating the impact of outliers. Additionally, we introduce a diffusion loss at early timesteps and employ optimal transport (OT) coupling to further enhance performance. Lastly, we introduce the adaptive scaling-$c$ scheduler to manage the robust training process and adopt Non-scaling LayerNorm in the architecture to better capture the statistics of the features and reduce outlier impact. With these strategies, we successfully train latent consistency models capable of high-quality sampling with one or two steps, significantly narrowing the performance gap between latent consistency and diffusion models. The implementation is released here: \url{https://github.com/quandao10/sLCT/}
\end{abstract}

\section{Introduction}
In recent years, generative models have gained significant prominence, with models like ChatGPT excelling in language generation and Stable Diffusion \citep{rombach2021highresolution}. In computer vision, the diffusion model \citep{song2020score, song2019generative, ho2020denoising, sohl2015deep} has quickly popularized and dominated the Adversarial Generative Model (GAN) \citep{goodfellow2014generative}. It is capable of generating high-quality diverse images that beat SoTA GAN models \citep{dhariwal2021diffusion}. Additionally, diffusion models are easier to train, as they avoid the common pitfalls of training instability and the need for meticulous hyperparameter tuning associated with GANs. The application of diffusion spans the entire computer vision field, including text-to-image generation \citep{rombach2021highresolution, gu2022vector}, image editing \citep{meng2021sdedit, cyclediffusion, huberman2024edit, han2024proxedit, he2024dice}, text-to-3D generation \citep{poole2022dreamfusion, wang2024prolificdreamer}, personalization \citep{ruiz2022dreambooth, van2023anti, kumari2023multi} and control generation \citep{zhang2023adding, brooks2022instructpix2pix, zhangli2024layout}. Despite their powerful capabilities, they require thousands of function evaluations for sampling, which is computationally expensive and hinders their application in the real world. Numerous efforts have been made to address this sampling challenge, either by proposing new training frameworks \citep{xiao2021tackling, rombach2021highresolution} or through distillation techniques \citep{meng2023distillation, yin2024one, sauer2023adversarial, dao2024self}. However, methods like \citep{xiao2021tackling} suffer from low recall due to the inherent challenges of GAN training, while \citep{rombach2021highresolution} still requires multi-step sampling. Distillation-based approaches, on the other hand, rely heavily on pretrained diffusion models and demand additional training.

Recently, \citep{song2023consistency} introduced a new family of generative models called the consistency model. Compared to the diffusion model \citep{song2019generative, song2020score, ho2020denoising}, the consistency model could both generate high-quality samples in a single step and multi-steps. The consistency model could be obtained by either consistency distillation (CD) or consistency training (CT). In previous work \citep{song2023consistency}, CD significantly outperforms CT. However, the CD requires additional training budget for using pretrained diffusion, and its generation quality is inherently limited by the pretrained diffusion. Subsequent research \citep{song2023improved} improves the consistency training procedure, resulting in performance that not only surpasses consistency distillation but also approaches SoTA performance of diffusion models. Additionally, several works \citep{kim2023consistency, geng2024consistency} have further enhanced the efficiency and performance of CT, achieving significant results. However, all of these efforts have focused exclusively on pixel space, where data is perfectly bounded. In contrast, most large-scale applications of diffusion models, such as text-to-image or video generation, operate in latent space \citep{rombach2021highresolution, gu2022vector}, as training on pixel space for large-scale datasets is impractical. Therefore, to scale consistency models for large datasets, the consistency must perform effectively in latent space. This work addresses the key question: How well can consistency models perform in latent space? To explore this, we first directly applied the SoTA pixel consistency training method, iCT \citep{song2023improved}, to latent space. The preliminary results were extremely poor, as illustrated in \cref{fig:qualitative_ict}, motivating a deeper investigation into the underlying causes of this suboptimal performance. We aim to improve CT in latent space, narrowing the gap between the performance of latent consistency and diffusion.

We first conducted a statistical analysis of both latent and pixel spaces. Our analysis revealed that the latent space contains impulsive outliers, which, while accounting for a very small proportion, exhibit extremely high values akin to salt-and-pepper noise. We also drew a parallel between Deep Q-Networks (DQN) and the Consistency Model, as both employ temporal difference (TD) loss. This could lead to training instability compared to the Kullback-Leibler (KL) loss used in diffusion models. Even in bounded pixel space, the TD loss still contains impulsive outliers, which \citep{song2023improved} addressed by proposing the use of Pseudo-Huber loss to reduce training instability. As shown in \cref{fig:impulsive_noise}, the latent input contains extremely high impulsive outliers, leading to very large TD values. Consequently, the Pseudo-Huber loss fails to sufficiently mitigate these outliers, resulting in poor performance as demonstrated in \cref{fig:qualitative_ict}. To overcome this challenge, we adopt Cauchy loss, which heavily penalizes extremely impulsive outliers. Additionally, we introduce diffusion loss at early timesteps along with optimal transport (OT) matching, both of which significantly enhance the model's performance. Finally, we propose an adaptive scaling $c$ schedule to effectively control the robustness of the model, and we incorporate Non-scaling LayerNorm into the architecture. With these techniques, we significantly boost the performance of latent consistency model compared to the baseline iCT framework and bridge the gap between the latent diffusion and consistency training.

\section{Related Works} \label{related}
% \subsection{Diffusion Model and Fast Sampling Technique}
% Diffusion models \citep{song2020score, song2019generative, ho2020denoising} have recently been raised as the most powerful generative model and outperform GAN \citep{goodfellow2014generative} in many applications. Diffusion models can generate high-fidelity images and possess good mode coverage, allowing diverse samples compared to GAN. However, diffusion models require many function evaluations (NFEs) during inference time. This drawback hinders its application in the real world. Many works are trying to tackle this drawback and achieve promising results. They can be divided into two main research categories: training from scratch and building upon pretrained diffusion models. Following the first category, there are several works, such as DDGAN \citep{xiao2021tackling}, LDM \citep{rombach2021highresolution}, and VQDiff \citep{gu2022vector}. DDGAN \citep{xiao2021tackling} proposes to use a larger step size in the forward process to reduce the NFEs; they use GAN models to implicitly learn the backward transition. Even though DDGAN \citep{xiao2021tackling} requires only a few sampling timesteps, it still suffers from low recall due to mode collapse from GAN. LDM \citep{rombach2021highresolution} does not directly reduce the number of sampling time steps; instead, it compresses the image to latent with a much smaller resolution. By training on latent space, LDM \citep{rombach2021highresolution} can significantly reduce both time and memory budget, and the inference is much faster than other pixel diffusion models. LDM \citep{rombach2021highresolution} has become the core technique in many large-scale diffusion models. Most real-world applications rely on LDM since it allows to scale diffusion models up on enormous high-resolution training datasets, which is impractical if training pixel diffusion models. In the second category, several works such as \citep{lu2022dpm, zhang2022fast} propose the high-order solver during inference. These works could successfully reduce sample NFEs to 10 without any training. However, they cannot sample with little NFEs such as 1 or 2. The other line of work is a distillation-based method. Progressive distillation \citep{salimans2022progressive} proposes to progressively distill diffusion model; each stage distills to reduce the sampling NFEs by half. This technique is costly since it requires to train many stages. Later works such as Guided-Distill \citep{meng2023distillation}, Swiftbrush \citep{nguyen2024swiftbrush}, DMD \citep{yin2024one}, and UFOGEN \citep{xu2024ufogen} manage to distill diffusion into few-step generation without compromising generative quality. The major drawback of these techniques is that additional training is required.
% Furthermore, some techniques, such as Swiftbrush \citep{nguyen2024swiftbrush} and DMD \citep{yin2024one}, do not have few-step sampling. Other methods, such as UFOGEN \citep{xu2024ufogen} and ADD \citep{sauer2023adversarial}, require training GAN, which could lead to training instability and low mode coverage. A standout among these techniques is the consistency model. The consistency model \citep{song2023consistency} is defined based on probability flow ODE (PF-ODE), allowing single- and multi-step sampling. The consistency model could be achieved via training from scratch and distillation from the diffusion model.

% \subsection{Consistency Model}

Consistency model \citep{song2023consistency, song2023improved} proposes a new type of generative model based on PF-ODE, which allows 1, 2 or multi-step sampling. The consistency model could be obtained by either training from scratch using an unbiased score estimator or distilling from a pretrained diffusion model. Several works improve the training of the consistency model. ACT \citep{kong2023act}, CTM \citep{kim2023consistency} propose to use additional GAN along with consistency objective. While these methods could improve the performance of consistency training, they require an additional discriminator, which could need to tune the hyperparameters carefully. MCM \citep{heek2024multistep} introduces multistep consistency training, which is a combination of TRACT \citep{berthelot2023tract} and CM \citep{song2023consistency}. MCM increases the sampling budget to 2-8 steps to tradeoff with efficient training and high-quality image generation. ECM \citep{geng2024consistency} initializes the consistency model by pretrained diffusion model and fine-tuning it using the consistency training objective. ECM vastly achieves improved training times while maintaining good generation performance. However, ECM requires pretrained diffusion model, which must use the same architecture as the pretrained diffusion architecture. Although these works successfully improve the performance and efficiency of consistency training, they only investigate consistency training on pixel space. As in the diffusion model, where most applications are now based on latent space, scaling the consistency training \citep{song2023consistency, song2023improved} to text-to-image or higher resolution generation requires latent space training. Otherwise, with pretrained diffusion model, we could either finetune consistency training \citep{geng2024consistency} or distill from diffusion model \citep{song2023consistency, luo2023latent}. CM \citep{song2023consistency} is the first work proposing consistency distillation (CD) on pixel space. LCM \citep{luo2023latent} later applies consistency technique on latent space and can generate high-quality images within a few steps. However, LCM's generated images using 1-2 steps are still blurry \citep{luo2023latent}. Recent works, such as Hyper-SD \cite{ren2024hyper} and TCD \cite{zheng2024trajectory}, have introduced notable improvements to latent consistency distillation. TCD \cite{zheng2024trajectory} employed CTM \cite{kim2023consistency} instead of CD \cite{song2023consistency}, significantly enhancing the performance of the distilled student model. Building on this, Hyper-SD \cite{ren2024hyper} divided the Probability Flow ODE (PF-ODE) into multiple components inspired by Multistep Consistency Models (MCM) \cite{heek2024multistep}, and applied TCD \cite{zheng2024trajectory} to each segment. Subsequently, Hyper-SD \cite{ren2024hyper} merged these segments progressively into a final model, integrating human feedback learning and score distillation \cite{yin2024one} to optimize one-step generation performance.

\section{Preliminaries} \label{sec:bg}
Denote $\pdata(\rvx_0)$ as the data distribution, the forward diffusion process gradually adds Gaussian noise with monotonically increasing standard deviation $\sigma(t)$ for $t \in \{0,1,\dots,T\}$ such that $p_t(\rvx_t|\rvx_0) = \gN(\rvx_0, \sigma^2(t)\mI)$ and $\sigma(t)$ is handcrafted such that $\sigma(0) = \sigma_{\min}$ and $\sigma(T)=\sigma_{\max}$. By setting $\sigma(t) = t$, the probability flow ODE (PF-ODE) from \citep{Karras2022edm} is defined as:
\begin{equation}
    \frac{\rd\rvx_t}{\rd t} = -t\nabla_{\rvx_t} \log p_t(\rvx_t) = \frac{\left( \rvx_t - \vf(\rvx_t, t) \right)}{t},  \label{eq:pf_ode}
\end{equation}
where $\vf:(\rvx_t, t) \rightarrow \rvx_0$ is the denoising function which directly predicts clean data $\rvx_0$ from given perturbed data $\rvx_t$. 
\citep{song2023consistency} defines consistency model based on PF-ODE in \cref{eq:pf_ode}, which builds a bijective mapping $\vf$ between noisy distribution $p(\rvx_t)$ and data distribution $\pdata(\rvx_0)$. The bijective mapping $\vf:(\rvx_t, t) \rightarrow \rvx_0$ is termed the consistency function. A consistency model $\vf_\theta(\rvx_t, t)$ is trained to approximate this consistency function $\vf(\rvx_t, t)$. The previous works \citep{song2023consistency, song2023improved, Karras2022edm} impose the boundary condition by parameterizing the consistency model as:
\begin{equation}
    \vf_\theta(\rvx_t, t) = c_{skip}(t)\rvx_t + c_{out}(t)\mF_\theta(\rvx_t, t), \label{eq:cm_param}
\end{equation}
where $\mF_\theta(\rvx_t, t)$ is a neural network to train. Note that, since $\sigma(t) = t$, we hereafter use $t$ and $\sigma$ interchangeably. $c_{skip}(t)$ and $c_{out}(t)$ are time-dependent functions such that $c_{skip}(\sigma_{\min}) = 1$ and $c_{out}(\sigma_{\max}) = 0$.

To train or distill consistency model, \citep{song2023consistency, song2023improved, Karras2022edm} firstly discretize the PF-ODE using a sequence of noise levels $\sigma_{\min} = t_{\min} = t_1 < t_2 < \dots < t_{N} = t_{\max} = \sigma_{\max}$, where $t_i = \left( t_{\min}^{1/\rho} + \frac{i-1}{N-1}(t_{\max}^{1/\rho
} - t_{\min}^{1/\rho})\right)^\rho$ and $\rho = 7$. 

\textbf{Consistency Distillation} Given the pretrained diffusion model $\vs_\phi(\rvx_t, t) \approx \nabla_{\rvx_t} \log p_t(\rvx_t)$, the consistency model could be distilled from the pretrained diffusion model using the following CD loss:
\begin{equation}
    \gL_{\text{CD}}(\theta, \theta^-) = \E\left[ \lambda(t_i)d(\vf_\theta(\rvx_{t_{i+1}}, t_{i+1}), \vf_{\theta^{-}}(\Tilde{\rvx}_{t_i}, t_{i})) \right], \label{loss:cd}
\end{equation}
where $\rvx_{t_{i+1}} = \rvx_0 + t_{i+1} \rvz$ with the $\rvx_0 \sim \pdata(\rvx_0)$ and $\rvz \sim \gN(0, \mI)$ and $\rvx_{t_i} = \rvx_{t_{i+1}} - (t_{i}-t_{i+1})t_{i+1} \nabla_{\rvx_{t_{i+1}}} \log p_{t_{i+1}}(\rvx_{t_{i+1}}) = \rvx_{t_{i+1}} - (t_{i}-t_{i+1})t_{i+1}\vs_\phi(\rvx_{t_{i+1}}, t_{i+1})$. 

\textbf{Consistency Training}
The consistency model is trained by minimizing the following CT loss:
\begin{equation}
    \gL_{\text{CT}}(\theta, \theta^-) = \E\left[ \lambda(t_i)d(\vf_\theta(\rvx_{t_{i+1}}, t_{i+1}), \vf_{\theta^{-}}(\rvx_{t_i}, t_{i})) \right], \label{loss:ct}
\end{equation}
where $\rvx_{t_i} = \rvx_0 + t_{i} \rvz$ and $\rvx_{t_{i+1}} = \rvx_0 + t_{i+1} \rvz$ with the same $\rvx_0 \sim \pdata(\rvx_0)$ and $\rvz \sim \gN(0, \mI)$

In \cref{loss:cd} and \cref{loss:ct}, $\vf_\theta$ and $\vf_{\theta^-}$ are referred to as the online network and the target network, respectively. The target's parameter $\theta^-$ is obtained by applying the Exponential Moving Average (EMA) to the student's parameter $\theta$ during the training and distillation as follows:
\begin{equation}
    \theta^- \leftarrow \text{stopgrad}(\mu\theta^- + (1-\mu)\theta), \label{ema}
\end{equation}
with $0\leq\mu<1$ as the EMA decay rate,  weighting function $\lambda(t_i)$ for each timestep $t_i$, and $d(\cdot, \cdot)$ is a predefined metric function. 

In CM \citep{song2023consistency}, the consistency training still lags behind the consistency distillation and diffusion models. iCT \citep{song2023improved} later propose several improvements that significantly boost the training performance and efficiency. First, the EMA decay rate $\mu$ is set to $0$ for better training convergence. Second, the Fourier scaling factor of noise embedding and the dropout rate are carefully examined. Third, iCT introduces Pseudo-Huber losses to replace $L_2$ and LPIPS since LPIPS introduces the undesirable bias in generative modeling \citep{song2023improved}. Furthermore, the Pseudo-Huber is more robust to outliers since it imposes a smaller penalty for larger errors than the $L_2$ metric. Fourth, iCT proposes an exp curriculum for total discretization steps N, which doubles N after a predefined number of training iterations. Moreover, uniform weighting $\lambda(t_i) = 1$ is replaced by $\lambda(t_i)=1/(t_{i+1}-t_i)$. Finally, iCT adopts a discrete Lognormal distribution for timestep sampling as EDM \citep{Karras2022edm}. With all these improvements, CT is now better than CD and performs on par with the diffusion models in pixel space.

\section{Method}
\label{method}
In this paper, we first investigate the underlying reason behind the performance discrepancy between latent and pixel space using the same training framework in \cref{sec:analysis}. Based on the analysis, we find out the root of unsatisfied performance on latent space could be attributed to two factors: the impulsive outlier and the unstable temporal difference (TD) for computing consistency loss. To deal with impulsive outliers of TD on pixel space, \citep{song2023improved} proposes the Pseudo-Huber function as training loss. For the latent space, the impulsive outlier is even more severe, making Pseudo-Huber loss not enough to resist the outlier. Therefore,  \cref{sec:cauchy} introduces Cauchy loss, which is more effective with extreme outliers. In the next \cref{sec:diff_loss} and \cref{sec:ot}, we propose to use diffusion loss at early timesteps and OT matching for regularizing the overkill effect of consistency at the early step and training variance reduction, respectively. Section \ref{sec:c} designs an adaptive scheduler of scaling $c$ to control the robustness of the proposed loss function more carefully, leading to better performance. Finally, in \cref{sec:norm}, we investigate the normalization layers of architecture and introduce Non-scaling LayerNorm to both capture feature statistic better and reduce the sensitivity to outliers.

\subsection{Analysis of latent space} \label{sec:analysis}

We first reimplement the iCT model \citep{song2023improved} on the latent dataset CelebA-HQ $32 \times 32 \times 4$ and pixel dataset Cifar-10 $32 \times 32 \times 3$. Hereafter, we refer to the latent iCT model as iLCT. We find that iCT framework works well on pixel datasets as claim \citep{song2023improved}. However, it produces worse results on latent datasets as in \cref{fig:qualitative_ict} and \cref{tab:main_exp}. The iLCT gets a very high FID above 30 for both datasets, and the generative images are not usable in the real world. This observation raises concern about the sensitivity of CT algorithm with training data, and we should carefully examine the training dataset. In addition, we notice that the DQN and CM use the same TD loss, which update the current state using the future state. Furthermore, they also possess the training instability. This motivates to carefully examine the behavior of TD loss with different training data.


While the pixel data lies within the range $[-1, 1]$ after being normalized, the range of latent data varies depending on the encoder model, which is blackbox and unbound. After normalizing latent data using mean and variance, we observe that the latent data contains high-magnitude values. We call them the impulsive outliers since they account for small probability but are usually very large values. In the bottom left of \cref{fig:impulsive_noise}, the impulsive outlier of latent data is red, spanning from $-9$ to $7$, while the first and third quartiles are just around $-1.4$ and $1.4$, respectively. We evaluate how the iCT will be affected by data outliers by analyzing the temporal difference $\text{TD} = f_\theta(\rvx_{t_{i+1}}, t_{i+1})-f_{\theta^-}(\rvx_{t_i}, t_{i})$. In the top right of \cref{fig:impulsive_noise}, the impulsive outliers of pixel TD range from -1.5 to 1.7, which are not too far from the interquartile range compared to latent TD. The impulsive outliers of latent TD range is much wider from -3.2 to 5. iCT uses Pseudo-Huber loss instead of $L_2$ loss since the Huber is less sensitive to outliers, see \cref{fig:loss}. However, for latent data, the Huber's reduction in sensitivity to outliers is not enough. This indicates that even using Pseudo-Huber loss, the iLCT training on latent space could still be unstable and lead to worse performance, which matches our experiment results on iLCT. Based on the above analysis, we hypothesize that the TD value statistic highly depends on the training data statistic.

\begin{figure}[!t]
    \centering
    \includegraphics[width=0.85\linewidth]{figures/impulsive_noise.pdf}
    \caption{\textbf{Box and Whisker Plot:} Impulsive noise comparison between pixel and latent spaces. The right column shows the statistics of TD values at 21 discretization steps. Other discretization steps exhibit same behavior, where impulsive outliers are consistently present regardless of the total discretization steps. The blue boxes represent interquartile ranges of the data, while the green and orange dashed lines indicate inner and outer fences, respectively. Outliers are marked with red dots.}
    \label{fig:impulsive_noise}
\end{figure}

%To understand the root of TD's impulsive outlier, we look into Deep Q Learning (DQN) from Reinforcement Learning (RL). There is a strong correlation between DQN and CM. While the DQN uses Q-value of future state as the ground truth for Q-value of the current state, the CM updates the current timesteps $f(\rvx_{t_{i+1}}, t_{i+1})$ using the smaller timesteps $f(\rvx_{t_{i}}, t_{i})$. This loss type is called the temporal difference in RL. For stable training, DQN uses target network $\theta^-$ to estimate Q-value of future state, and CM similarly adopts the same technique for consistency loss. The target network $\theta^-$ could be updated differently, such as Polyak-averaging, periodic, and standard TD updates \citep{lee2019target}. The Polyak-averaging update is simply EMA update using in \citep{song2023consistency}, and standard update corresponds to \citep{song2023improved} which $\theta^- \leftarrow \theta$ every iteration. The periodic update is a standard update but after every fixed number of iterations. In RL, Polyak-averaging and periodic updates are more stable but slowly convergent \citep{lee2019target}. Even using these stable target updates, there is still instability in TD training. Since the target network needs to change along with the online model, the target value of TD can never be fixed, which makes the loss highly oscillate. Therefore, even though the pixel data is very well-bounded within [-1, 1], the CM training is still affected by impulsive outliers due to the nature of TD loss.

To mitigate the impact of impulsive outliers, we could use more stable target updates like Polyak or periodic in TD loss \cite{lee2019target}, but they lead to very slow convergence, as shown in \citep{song2023consistency}. Even though CM is initialized by a pretrained diffusion model, the Polyak update still takes a long time to converge. Therefore, using Polyak or periodic updates is computationally expensive, and we keep the standard target update as in \citep{song2023improved}. Another direction is using a special metric for latent like LPIPS on pixel space \citep{song2023consistency}. \citep{kang2024diffusion2gan} proposes the E-LatentLPIPS as a metric for distillation and performs well on distillation tasks. However, this requires training a network as a metric and using this metric during the training process will also increase the training budget. To avoid the overhead of the training, we seek a simple loss function like Pseudo-Huber but be more effective with outliers. We find that the Cauchy loss function \citep{black1996robust, barron2019general} could be a promising candidate in place of Pseudo-Huber for latent space.
\subsection{Cauchy Loss against Impulsive Outlier} \label{sec:cauchy}
In this section, we introduce the Cauchy loss \citep{black1996robust, barron2019general} function to deal with extreme impulsive outliers. The Cauchy loss function has the following form:
\begin{equation}
    d_{\text{Cauchy}}(\rvx, \rvy)=  \log \left(1+\frac{||\rvx-\rvy||_2^2}{2c^2}\right), \label{loss:cauchy}
\end{equation}
and we also consider two additional robust losses, which are Pseudo-Huber \citep{song2023improved, barron2019general} and Geman-McClure \citep{geman1986bayesian, barron2019general}
\begin{equation}
    d_{\text{Pseudo-Huber}}(\rvx, \rvy)= \sqrt{||\rvx-\rvy||_2^2 + c^2} - c, \label{loss:huber}
\end{equation}
\begin{equation}
    d_{\text{Geman-McClure}}(\rvx, \rvy)= \frac{2||\rvx-\rvy||_2^2}{||\rvx-\rvy||_2^2 + 4c^2}, \label{loss:gm}
\end{equation}
where $c$ is the scaling parameter to control how robust the loss is to the outlier. We analyze their robustness behavior against outliers. As shown in \cref{fig:loss_val}, the Pseudo-Huber loss linearly increases like $L_1$ loss for the large residuals $\rvx-\rvy$. In contrast, the Cauchy loss only grows logarithmically, and the Geman-McClure suppresses the loss value to $1$ for the outliers. 

The Pseudo-Huber loss works well if the residual value does not grow too high and, therefore, has a good performance on the pixel space. However, for the latent space, as shown in the bottom right of \cref{fig:impulsive_noise}, the TD suffers from extremely high values coming from the impulsive outlier in the latent dataset, the Cauchy loss could be more suitable since it significantly dampens the influence of extreme outliers. Otherwise, even Geman-McClure is very highly effective for removing outlier effects than two previous losses; it gives a gradient $0$ for high TD value and completely ignores the impulsive outliers as \cref{fig:loss_derivative}. This is unexpected behavior because even though we call the high-value latent impulsive outlier, they actually could encode important information from original data. Completely ignoring them could significantly hurt the performance of training model. Based on this analysis, we choose Cauchy loss as the default loss for latent CM for the rest of the paper. The loss ablation is provided in \cref{tab:ablate_robust}.


\begin{figure}[!ht]
    \centering
    \begin{subfigure}[t]{0.40\textwidth}
        \centering
        \includegraphics[width=1.0\textwidth]{figures/func.png}
        \caption{Robust Loss}
        \label{fig:loss_val}
    \end{subfigure}%
    ~ 
    \begin{subfigure}[t]{0.40 \textwidth}
        \centering
        \includegraphics[width=1.0\textwidth]{figures/derivative.png}
        \caption{Derivative of Robust Loss}
        \label{fig:loss_derivative}
    \end{subfigure}
    \caption{Analysis of robust loss: Pseudo-Huber, Cauchy, and Geman-McClure}
    \label{fig:loss}
\end{figure}
% \vspace{-3mm}


\subsection{Diffusion Loss at small timestep} \label{sec:diff_loss}
For small noise level $\sigma$, the ground truth of $f(\rvx_\sigma, \sigma)$ can be well approximated by $\rvx_0$, but this does not hold for large noise levels. Therefore, for low-level noise, the consistency objective seems to be overkill and harms the model's performance since instead of optimizing $f_\theta(\rvx_\sigma, \sigma)$ to approximated ground truth $\rvx_0$, the consistency objective optimizes through a proxy estimator $f_{\theta^-}(\rvx_{<\sigma}, <\sigma)$ leading to error accumulation over timestep. To regularize this overkill, we propose to apply an additional diffusion loss on small noise level as follows:

\begin{equation}
    L_{diff} = ||f_\theta(\rvx_{t_i}, t_i) - \rvx_0||^2_2 \quad \forall i \leq \text{int(N $\cdot$ r)}, \label{loss:diff}
\end{equation}

where N is the number of training discretization steps and $r\in[0;1]$ is the diffusion threshold, and we heuristicly choose $r=0.25$. We do not apply diffusion loss for large noise levels since $f(\rvx_\sigma, \sigma)$ will differ greatly from the target $\rvx_0$, leading to very high $L_2$ diffusion loss. This could harm the training consistency process, misleading to the wrong solution. We provide the ablation study in \cref{tab:diff_loss}. Furthermore, CTM \citep{kim2023consistency} also proposes to use diffusion loss, but they use them on both high and low-level noise, which is different from us. 

\subsection{OT matching reduces the variance} \label{sec:ot}
% \vspace{-5mm}
In this section, we adopt the OT matching technique from previous works \citep{pooladian2023multisample, lee2023minimizing}. \citep{pooladian2023multisample} proposes to use OT to match noise and data in the training batch, such as the moving $L_2$ cost is optimal. On the other hand, \citep{lee2023minimizing} introduces $\beta\text{VAE}$ for creating noise corresponding to data and train flow matching on the defined data-noise pairs. By reassigning noise-data pairs, these works significantly reduce the variance during the diffusion/flow matching training process, leading to a faster and more stable training process. According to \citep{zhang2023emergence}, the consistency training and diffusion models produce highly similar images given the same noise input. Therefore, the final output solution of the consistency and diffusion models should be close to each other. Since OT matching helps reduce the variance during training diffusion, it could be useful to reduce the variance of consistency training. In our implementation, we follow \citep{pooladian2023multisample, tong2023improving} using the POT library to map from noise to data in the training batch. The overhead caused by minibatch OT is relatively small, only around $0.93\%$ training time, but gains significant performance improvement as shown in \cref{tab:strategy}.

\subsection{Adaptive $c$ scheduler} \label{sec:c}

% \begin{figure}[h!]
%     \centering
%     \includegraphics[width=0.5\linewidth]{figures/C_by_NFE.pdf}
%     \caption{Our robust adaptive $c$ scheduler}
%     \label{fig:proposed_c}
% \end{figure}

\begin{figure}[h!]
    \centering
    \includegraphics[width=0.8\linewidth]{figures/C_merge.pdf}
    \caption{Model convergence plot on different $c$ schedule. (Left) Our proposed $c$ values. Performance on FID (Middle) and Recall (Right) of our proposed $c$ in comparison with different choices.}
    \label{fig:fid_vary_c}
\end{figure}
% \vspace{-5mm}

In this section, we examine the choice of scaling parameter $c$ in robust loss functions. The scaling parameter controls the robustness level, which is very important for model performance. The previous work \citep{song2023improved} proposes to use fixed constant $c_0 = 0.00054\sqrt{d}$, where $d$ is the dimension of data. We find that using this simple fixed $c$ is not yet optimal for the training consistency model. Especially in this paper, we follow the Exp curriculum specified by \cref{exp_cur} in \citep{song2023improved}, which doubles the total discretization step after a defined number of training iterations. 
\begin{equation}
    \text{NFE}(k)=\min \left(s_0 2^{\left\lfloor\frac{k}{K^{\prime}}\right\rfloor}, s_1\right)+1, \quad K^{\prime}=\left\lfloor\frac{K}{\log _2\left\lfloor s_1 / s_0\right\rfloor+1}\right\rfloor, \label{exp_cur}
\end{equation}
where $k$ is current training iteration, $K$ is total training iteration and $s_0 = 10, s_1=640$. During training, we notice that the variance of TD is significantly reduced as doubling total discretization steps using \cref{exp_cur}. Since the more discretization steps, the closer distance of $\rvx_{t_i}$ and $\rvx_{t_{i+1}}$, the TD value's range between them should be smaller. However, the impulsive outlier still exists regardless of the number of discretization steps. Intuitively, we propose a heuristic adaptive $c$ scheduler where the $c$ is scaled down proportional to the reduction rate of TD variance as the number of discretization steps increases. We plot our $c$ scheduler versus discretization steps in \cref{fig:fid_vary_c} and we fit the $c$ scheduler to get the scheduler equation as following:

\begin{equation}
    c = \exp(-1.18 * \log(\text{NFE}(k) - 1) - 0.72) \label{eq:c_scheduler}
\end{equation}

\subsection{Non-scaling Layernorm} \label{sec:norm}
As mentioned in \cref{sec:analysis}, the statistic of training data could play an important role in the success of consistency training. Furthermore, in architecture design, the normalization layer specifically handles the statistics of input, output, and hidden features. In this section, we investigate the normalization layer choice for consistency training, which is sensitive to training data statistics. 

Currently, both \citep{song2023improved, song2023consistency} use the UNet architecture from \citep{dhariwal2021diffusion}. In UNet \citep{dhariwal2021diffusion}, GroupNorm is used in every layer by default. The GroupNorm only captures the statistics over groups of local channels, while the LayerNorm further captures the statistics' overall features. Therefore, LayerNorm is better at capturing fine-grained statistics over the entire feature. We further carry out the experiments for other types of normalization, such as LayerNorm, InstanceNorm, RMSNorm in \cref{tab:norm_layer} and observe that the GroupNorm and InstanceNorm perform relatively well compared to others, especially LayerNorm. This could be due to that they are less sensitive to the outliers since they only capture the statistic over groups of channels. Therefore, the impulsive features only affect the normalization of a group containing them. For the LayerNorm, the impulsive features could negatively impact the overall features's normalization. We further look into the LayerNorm implementation and suspect that the scaling term could significantly amplify the outliers across features by serving as a shared parameter. This observation is also mentioned in \citep{wei2022outlier} for LLM quantization. In implementation, we set the \textbf{scaling term of LayerNorm to $1$} and \textbf{disabled the gradient update} for it \eqref{operation:layernorm}. We refer to it as Non-scaling LayerNorm (NsLN) as \citep{wei2022outlier}.

\begin{equation}
    \text{LN}_{\gamma, \beta}(\rvx) = \frac{\rvx - u(\rvx)}{\sqrt{\sigma^{2}(\rvx) + \epsilon}} \cdot \gamma + \beta, \quad
    \text{NsLN}_{\beta}(\rvx) = \frac{\rvx - u(\rvx)}{\sqrt{\sigma^{2}(\rvx) + \epsilon}} + \beta, \label{operation:layernorm}
\end{equation}

where $u(\rvx)$ and $\sigma^{2}(\rvx)$ are mean and variance of $\rvx$.

% \subsection{Improve Consistency Distillation}

% \vspace{-15mm}
\section{Experiment} \label{exp}

\subsection{Performance of our training technique} \label{exp:main}
% \vspace{-3mm}
\begin{table}[t]
    \centering
    \begin{tabular}{cc}
        \begin{minipage}[c]{0.58\textwidth}
            \centering
            \begin{subtable}[t]{\textwidth}
                \resizebox{\textwidth}{!}{%
                \begin{tabular}{l c c c c c}
                    \toprule
                    Model & NFE$\downarrow$ & FID$\downarrow$ & Recall$\uparrow$ & Epochs & Total Bs\\
                    \midrule 
                    \multicolumn{5}{c}{\textbf{Pixel Diffusion Model}}\\
                    \midrule
                    WaveDiff \citep{phung2023wavediff} & 2 & 5.94 & 0.37 & 500 & 64\\
                    Score SDE \citep{song2020score} & 4000 & 7.23 & - & ~6.2K & - \\
                    DDGAN \citep{xiao2021tackling} & 2 & 7.64 & 0.36 & 800 & 32 \\
                    RDUOT \citep{dao2024high} & 2 & 5.60 & 0.38 & 600 & 24 \\
                    RDM \citep{teng2023relay} & 270 & 3.15 & 0.55 & 4K & - \\
                    UNCSN++ \citep{kim2021soft} & 2000 & 7.16 & - & - & -\\
                    \midrule 
                    \multicolumn{5}{c}{\textbf{Latent Diffusion Model}}\\
                    \midrule
                    LFM-8 \citep{dao2023flow} & 85 & 5.82 & 0.41 & 500 & 112\\ 
                    LDM-4 \citep{rombach2021highresolution} & 200 & 5.11 & 0.49 & 600 &48 \\
                    LSGM \citep{vahdat2021score} & 23 & 7.22 & - & 1K &-\\
                    DDMI \citep{park2024ddmi} & 1000 & 7.25 & - & - &-\\
                    
                    DIMSUM \citep{phung2024dimsum} & 73 & 3.76  & 0.56 & 395 &32\\
                    $\text{LDM-8}^\dagger$ & 250 & {8.85}  & - & 1.4K &128\\
                    
                    \midrule
                    \multicolumn{5}{c}{\textbf{Latent Consistency Model}}\\
                    \midrule
                    iLCT \citep{song2023improved} & 1 & 37.15 & 0.12 & 1.4K &128\\
                    iLCT \citep{song2023improved} & 2 & 16.84 & 0.24 & 1.4K &128\\
                    Ours  & 1 & 7.27 & 0.50 & 1.4K &128\\
                    Ours  & 2 & 6.93 & 0.52 & 1.4K &128\\
                    \bottomrule
                \end{tabular}%
                }
            \caption{CelebA-HQ}
            \label{tab:celeb}
            \end{subtable}
        \end{minipage}
        \hfill
        \begin{minipage}[c]{0.42\textwidth}
            \centering
            \begin{subtable}[t]{\textwidth}
                \resizebox{\textwidth}{!}{%
                \begin{tabular}{l c c c c c}
                    \toprule
                    Model & NFE$\downarrow$ & FID$\downarrow$ & Recall$\uparrow$ & Epochs & Total Bs\\
                    \midrule 
                    \multicolumn{5}{c}{\textbf{Pixel Diffusion Model}}\\
                    \midrule
                    WaveDiff \citep{phung2023wavediff} & 2 & 5.94 & 0.37 & 500 & 64\\
                    Score SDE \citep{song2020score} & 4000 & 7.23 & - &6.2K & -\\
                    DDGAN \citep{xiao2021tackling} & 2 & 5.25 & 0.36 & 500 & 32\\
                    \midrule 
                    \multicolumn{5}{c}{\textbf{Latent Diffusion Model}}\\
                    \midrule
                    LFM-8 \citep{dao2023flow} & 90 & 7.70 & 0.39 & 90 &112\\
                    LDM-8 \citep{rombach2021highresolution} & 400 & 4.02 & 0.52 & 400 &96\\
                    $\text{LDM-8}^\dagger$ & 250 & {10.81} & - & 1.8K &256\\
                    \midrule
                    \multicolumn{5}{c}{\textbf{Latent Consistency Model}}\\
                    \midrule
                    iLCT \citep{song2023improved} & 1 &52.45  &0.11  & 1.8K &256\\
                    iLCT \citep{song2023improved} & 2 &24.67  &0.17  & 1.8K &256\\
                    Ours  & 1 &8.87  &0.47  & 1.8K &256\\
                    Ours  & 2 &7.71  &0.48  & 1.8K &256\\
                    \bottomrule
                \end{tabular}%
                }
            \caption{LSUN Church}
            \label{tab:lsun}
            \end{subtable}
            \hfill
            \begin{subtable}[t]{\textwidth}
                \resizebox{\textwidth}{!}{%
                \begin{tabular}{l c c c c c}
                    \toprule
                    Model & NFE$\downarrow$ & FID$\downarrow$ & Recall$\uparrow$ & Epochs &Total Bs \\
                    \midrule 
                    \multicolumn{5}{c}{\textbf{Latent Diffusion Model}}\\
                    \midrule
                    LFM-8 \citep{dao2023flow} & 84 & 8.07 & 0.40 & 700 &128\\
                    LDM-4 \citep{rombach2021highresolution} & 200 & 4.98 & 0.50 & 400 &42\\
                    $\text{LDM-8}^\dagger$ & 250 &{10.23} & - & 1.4K &128\\
                    \midrule
                    \multicolumn{5}{c}{\textbf{Latent Consistency Model}}\\
                    \midrule
                    iLCT \citep{song2023improved} & 1 & 48.82  & 0.15 & 1.4K &128 \\
                    iLCT \citep{song2023improved} & 2 & 21.15 & 0.19 & 1.4K &128\\
                    Ours  & 1 & 8.72  &0.42 & 1.4K &128\\
                    Ours  & 2 & 8.29  &0.43  & 1.4K &128\\
                    \bottomrule
                \end{tabular}%
                }
            \caption{FFHQ}
            \label{tab:ffhq}
            \end{subtable}
        \end{minipage}
    \end{tabular}
    \caption{Our performance on CelebA-HQ, LSUN Church, FFHQ datasets at resolution $256 \times 256$. ($\dagger$) means training on our machine with the same diffusion forward and equivalent architecture.}
    \label{tab:main_exp}
\end{table}


\minisection{Experiment Setting:}
We measure the performance of our proposed technique on three datasets: CelebA-HQ \citep{celeba}, FFHQ \citep{karras2019style}, and LSUN Church \citep{lsun}, at the same resolution of $256 \times 256$. Following LDM \citep{rombach2021highresolution}, we use pretrained VAE KL-8 \footnote{https://huggingface.co/stabilityai/sd-vae-ft-ema} to obtain latent data with the dimensionality of $32 \times 32 \times 4$. We adopt the OpenAI UNet architecture \citep{dhariwal2021diffusion} as the default architecture throughout the paper. Furthermore, we use the variance exploding (VE) forward process for all the consistency and diffusion experiments following \citep{song2023consistency, song2023improved}. The baseline iCT is self-implemented based on official implementation CM \citep{song2023consistency} and iCT \citep{song2023improved}. We refer to this baseline as iLCT. Furthermore, we also train the latent diffusion model for each dataset using the same VE forward noise process for fair comparisons with our technique. This LDM model is referred to as $\text{LDM-8}^{\dagger}$ in \cref{tab:main_exp}. All three frameworks, including ours, iLCT, and $\text{LDM-8}^{\dagger}$, use the same architecture.

\minisection{Evaluation:} During the evaluation, we first generate 50K latent samples and then pass them through VAE's decoder to obtain the pixel images. We use two well-known metrics, Fréchet Inception Distance (FID) \citep{fid} and Recall \citep{kynkaanniemi2019improved}, for measuring the performance of the model given the training data and 50K generated images. 

\minisection{Model Performance:} We report the performance of our model across all three datasets in \cref{tab:main_exp}, primarily to compare it with the baseline iLCT \citep{song2023improved} and LDM \citep{rombach2021highresolution}. For both 1 and 2 NFE sampling, we observe that the FIDs of iLCT for all datasets are notably high (over 30 for 1-NFE sampling and over 16 for 2-NFE sampling), consistent with the qualitative results shown in \cref{fig:qualitative_ict}, where the generated image is unrealistic and contain many artifacts. This poor performance of iLCT in latent space is expected, as the Pseudo-Huber training losses are insufficient in mitigating extreme impulsive outliers, as discussed in \cref{sec:analysis} and \cref{sec:cauchy}. In contrast, our proposed framework demonstrates significantly better FID and Recall than iLCT. Specifically, we achieve 1-NFE sampling FIDs of 7.27, 8.87, and 8.29 for CelebA-HQ, LSUN Church, and FFHQ, respectively. For 2-NFE sampling, our FID scores improve across all three datasets. Notably, our 1-NFE sampling outperforms $\text{LDM-8}^{\dagger}$, using the same noise scheduler and architecture. However, our models still exhibit higher FIDs compared to LDM \citep{rombach2021highresolution} and LFM \citep{dao2023flow}. In contrast, we only need 1 or 2 timestep sampling, whereas they require multiple timesteps for high-fidelity generation.
 It's important to note that we employ the VE forward process, whereas these other methods use VP and flow-matching forward processes. Furthermore, the qualitative results of our framework, as shown in \cref{fig:qualitative_1nfe}, highlight our ability to generate high-quality images.

\begin{figure}[ht]
\centering
    \begin{subfigure}[b]{0.3\textwidth}
    \centering
    \includegraphics[width=\textwidth]{figures/lct_celeba.pdf}
    \caption{CelebA-HQ}
    \label{fig:qualitative_celeba}
    \end{subfigure}
    \hfill
    \begin{subfigure}[b]{0.3\textwidth}
    \centering
    \includegraphics[width=\textwidth]{figures/lct_church.pdf}
    \caption{LSUN Church}
    \label{fig:qualitative_lsun_church}
    \end{subfigure}
    \hfill
    \begin{subfigure}[b]{0.3\textwidth}
    \centering
    \includegraphics[width=\textwidth]{figures/ffhq.pdf}
    \caption{FFHQ}
    \label{fig:qualitative_ffhq}
    \end{subfigure}
    \caption{Our qualitative results using 1-NFE at resolution $256 \times 256$}
    \label{fig:qualitative_1nfe}
\end{figure}

\begin{figure}[ht]
\centering
    \begin{subfigure}[b]{0.3\textwidth}
    \centering
    \includegraphics[width=\textwidth]{figures/lct_celeba_baseline.pdf}
    \caption{CelebA-HQ}
    \label{fig:qualitative_ict_celeba}
    \end{subfigure}
    \hfill
    \begin{subfigure}[b]{0.3\textwidth}
    \centering
    \includegraphics[width=\textwidth]{figures/lsun_church.pdf}
    \caption{LSUN Church}
    \label{fig:qualitative_ict_lsun_church}
    \end{subfigure}
    \hfill
    \begin{subfigure}[b]{0.3\textwidth}
    \centering
    \includegraphics[width=\textwidth]{figures/lct_ffhq_baseline.pdf}
    \caption{FFHQ}
    \label{fig:qualitative_ict_ffhq}
    \end{subfigure}
    \caption{iLCT qualitative results using 1-NFE at resolution  $256 \times 256$}
    \label{fig:qualitative_ict}
\end{figure}


\subsection{Ablation of proposed framework} \label{exp:ablation}

We ablate our proposed techniques on the CelebA-HQ $256\times256$ dataset, with all FID and Recall metrics measured using 1-NFE sampling. All models are trained for 1,400 epochs with the same hyperparameters. As shown in \cref{tab:strategy}, replacing Pseudo-Huber losses with Cauchy losses makes our model's training less sensitive to impulsive outliers, resulting in a significant FID reduction from $37.15$ to $13.02$. This demonstrates the effectiveness of Cauchy losses in handling extremely high-value outliers, as discussed in \cref{sec:cauchy}. Additionally, applying diffusion loss at small timesteps further reduces FID by approximately 4 points to $9.11$, as this loss term stabilizes the training process at small timesteps, as described in \cref{sec:diff_loss}. Introducing OT coupling during minibatch training reduces training variance, improving the FID to $8.89$. Notably, by replacing the fixed scaling term $c=c_0$, \citep{song2023improved} with an adaptive scaling schedule, our model achieves an additional FID reduction of more than 1 point, reaching $7.76$, highlighting the importance of the scaling term $c$ in robustness control. Finally, we propose using NsLN, which removes the scaling term from LayerNorm to handle outliers more effectively. NsLN captures feature statistics while mitigating the negative impact of outliers, resulting in our best FID of $7.27$.

\minisection{Robustness Loss} \label{exp:ablation:robust_loss}
To analyze the impact of different robust loss functions, we conduct an ablation study using our best settings but replace the Cauchy loss with alternatives such as L2, E-LatentLPIPS \cite{kang2024diffusion2gan}, the Huber and the Geman-McClure loss. The results, shown in \cref{tab:ablate_robust}, indicate that both Huber and Geman-McClure underperform compared to the Cauchy loss when applied in the latent space. This is because the Huber loss remains too sensitive to extremely impulsive outliers, while the Geman-McClure loss tends to ignore such outliers entirely, leading to a loss of important information. This behavior is also discussed in \cref{sec:cauchy}.

% \vspace{-2mm}
\begin{table}[h!]
    \centering
    \begin{tabular}{cc}
        \begin{minipage}[c]{0.40\textwidth}
            \centering
            
            \begin{subtable}[t]{\textwidth}
                \centering
                \begin{tabular}{lcc}
                    \toprule
                    Framework                      & FID $\downarrow$   & Recall $\uparrow$   \\
                    \midrule
                    iLCT                           & 37.15              & 0.12                \\
                    \midrule
                    Cauchy                         & 13.02              & 0.36                \\
                    + Diff                         & 9.11               & 0.41                \\
                    + OT                           & 8.89               & 0.42                \\
                    + Scaled $c$                   & 7.76               & 0.47                \\
                    + NsLN       & \textbf{7.27}               &\textbf{0.50}                \\
                    % \rowcolor{pink!60}+ NsLN       & 7.27               & 0.50                \\
                    \bottomrule
                \end{tabular}
                \caption{Components of proposed framework}
                \label{tab:strategy}
            \end{subtable}
            \hfill
            % \vspace{2mm}
            \begin{subtable}[t]{\textwidth}
                \centering
                \begin{tabular}{lcc}
                    \toprule
                    $r$           & FID $\downarrow$   & Recall $\uparrow$   \\
                    \midrule
                    1.0                    & 7.47               & 0.49                \\
                    0.6                      & 7.33               & 0.49                \\
                    % \rowcolor{pink!60}0.25   & 7.27               & 0.50                \\
                    0.25   & \textbf{7.27}               & \textbf{0.50}                \\
                    \bottomrule
                \end{tabular}
                \caption{Threshold using Diffusion loss}
                \label{tab:diff_loss}
            \end{subtable}
            
        \end{minipage}
        \hfill
        \begin{minipage}[c]{0.40\textwidth}
            \centering
            \begin{subtable}[t]{\textwidth}
                \centering
                \begin{tabular}{lcc}
                    \toprule
                    Loss                        & FID $\downarrow$   & Recall $\uparrow$   \\
                    \midrule
                     L2                          & 50.40              & 0.04                \\
                    E-LatentLPIPS               & 11.49              & 0.47                \\
                    \midrule
                    Huber                       & 9.97               & 0.44                \\
                    Geman McClure               & 11.28              & 0.44                \\
                    % \rowcolor{pink!60} Cauchy   & 7.27               & 0.50                \\
                    Cauchy   & \textbf{7.27}               & \textbf{0.50}                \\
                    \bottomrule
                \end{tabular}
                \caption{Robust losses.}
                \label{tab:ablate_robust}
            \end{subtable}
            \hfill
            % \vspace{2mm}
            \begin{subtable}[t]{\textwidth}
                \centering
                \begin{tabular}{lcc}
                    \toprule
                    Norm layer                             & FID $\downarrow$   & Recall $\uparrow$   \\
                    \midrule
                    $\text{GN}$                           & 7.76               & 0.47                \\
                    \midrule
                    % \midrule
                    IN                                      & 8.47               & 0.43                \\
                    % $\text{IN}^\dagger$                     & 8.03               & 0.46                \\
                    % \midrule
                    LN                                      & 9.05               & 0.46                \\
                    % $\text{LN}^\dagger$                     & 7.92               & 0.47                \\
                    % \midrule
                    RMS                                     & 8.96               & 0.46                \\
                    % $\text{RMS}^\dagger$                    & 7.62               & 0.47                \\
                    % \midrule
                    NsLN                 &\textbf{7.27}               &\textbf{0.50}                \\
                    % \rowcolor{pink!60} NsLN                 & 7.27               & 0.50                \\
                    % \rowcolor{white}$\text{NsLN}^\dagger$   & 7.64               & 0.47                \\
                    \bottomrule
                \end{tabular}
                \caption{Norm Layer}
                \label{tab:norm_layer}
            \end{subtable}
        \end{minipage}
    \end{tabular}
    \caption{Ablation Studies on CelebA-HQ $256\times256$ dataset at epoch 1400}
    \label{tab:ablation}
\end{table}

\minisection{Diffusion Threshold} \label{exp:ablation:diff_loss}
In this section, we explore the impact of varying the threshold for applying the diffusion loss function in combination with the consistency loss. We observe that using the diffusion loss at every timestep improves consistency training; however, it underperforms compared to applying the diffusion loss selectively at smaller timesteps such as $r=0.25$ as shown in \cref{tab:diff_loss}. This suggests that applying diffusion losses primarily at small noise levels improves performance as discussed \cref{sec:diff_loss}. At larger timesteps, the diffusion loss may conflict with the consistency loss, potentially guiding the model toward incorrect solutions, thereby reducing overall performance.



\minisection{Scaling term $c$ scheduler} \label{exp:ablation:vary_c}
In this section, we compare the performance of our adaptive scaling $c$ scheduler with the fixed scaling $c$ scheduler proposed in \citep{song2023improved}. Our model demonstrates better convergence with the proposed adaptive $c$ scheduler. The rationale behind this improvement lies in the fact that, as the discretization steps increases using the exponential curriculum, the value of the TD scales down. Despite the reduced TD value, impulsive outliers still persist. A fixed large scaling $c$ is not effective in handling these outliers. To address this, we scale $c$ down as discretization steps increases, which leads to better performance, as shown in \cref{fig:fid_vary_c}.


\minisection{Normalizing Layer} \label{exp:ablation:norm_layer}
We denote GN, IN, LN, RMS, and NsLN as GroupNorm, InstanceNorm, LayerNorm, RMSNorm, and Non-scaling LayerNorm, respectively. The baseline UNet architecture from \citep{dhariwal2021diffusion} uses GroupNorm by default. We replace the normalization layers in the baseline with each of these types and train the model on CelebA-HQ using the best settings. The results are reported in \cref{tab:norm_layer}. GN and IN only capture local statistics, making them more robust to outliers, as outliers in one region do not affect others. In contrast, LN captures statistics from all features, making it more vulnerable to outliers because an outlier affects all features through a shared scaling term. By removing the scaling term in LN, we obtain NsLN, which is both effective in capturing feature statistics and resistant to outliers. As shown in \cref{tab:norm_layer}, NsLN outperforms the second-best GN by 0.5 FID and significantly outperforms LN.
\section{Conclusion}
CT is highly sensitive to the statistical properties of the training data. In particular, when the data contains impulsive noise, such as latent data, CT becomes unstable, leading to poor performance. In this work, we propose using the Cauchy loss, which is more robust to outliers, along with several improved training strategies to enhance model performance. As a result, we can generate high-fidelity images from latent CT, effectively bridging the gap between latent diffusion models and consistency models. Future work could explore further improvements to the architecture, specifically by investigating normalization methods that reduce the impact of outliers. For example, removing the scaling term from group normalization or instance normalization may help mitigate outlier effects. Another promising future direction is the integration of this technique with Consistency Trajectory Models (CTM) \cite{kim2023consistency}, as CTM has demonstrated improved performance compared to traditional Consistency Models (CM) \cite{song2023consistency}.
\section*{Acknowledgements}
Research funded by research grants to Prof. Dimitris Metaxas from NSF: 2310966, 2235405, 2212301, 2003874, 1951890, AFOSR 23RT0630, and NIH 2R01HL127661.


\bibliography{iclr2025_conference}
\bibliographystyle{iclr2025_conference}

\newpage
\appendix
\section{Appendix}
We provide additional uncurated samples of our models for three datasets: CelebaA-HQ (\ref{fig:appendix:celeba_onestep}, \ref{fig:appendix:celeba_twostep}), LSUN Church (\ref{fig:appendix:lsun_onestep}, \ref{fig:appendix:lsun_twostep}), and FFHQ (\ref{fig:appendix:ffhq_onestep}, \ref{fig:appendix:ffhq_twostep}). We also provide additional uncurated samples of our models on CelebaA-HQ trained with L2 loss (\ref{fig:appendix:celeba_onestep_ilct_l2}) and E-LatentLPIPS loss (\ref{fig:appendix:celeba_onestep_ilct_elatentlpips}).

\begin{figure}[h]
\centering
    \includegraphics[width=0.6\textwidth]{figures/lct_celeba_more.pdf}
    \caption{One-step samples on CelebA-HQ $256 \times 256$}
    \label{fig:appendix:celeba_onestep}
\end{figure}

\begin{figure}[h]
\centering
    \includegraphics[width=0.6\textwidth]{figures/lct_celeba_more_2step.pdf}
    \caption{Two-step samples on CelebA-HQ $256 \times 256$}
    \label{fig:appendix:celeba_twostep}
\end{figure}

\begin{figure}[h]
\centering
    \includegraphics[width=0.6\textwidth]{figures/lct_lsun_more.pdf}
    \caption{One-step samples on LSUN Church $256 \times 256$}
    \label{fig:appendix:lsun_onestep}
\end{figure}

\begin{figure}[h]
\centering
    \includegraphics[width=0.6\textwidth]{figures/lct_lsun_more_2step.pdf}
    \caption{Two-step samples on LSUN Church $256 \times 256$}
    \label{fig:appendix:lsun_twostep}
\end{figure}

\begin{figure}[h]
\centering
    \includegraphics[width=0.6\textwidth]{figures/lct_ffhq_more.pdf}
    \caption{One-step samples on FFHQ $256 \times 256$}
    \label{fig:appendix:ffhq_onestep}
\end{figure}

\begin{figure}[h]
\centering
    \includegraphics[width=0.6\textwidth]{figures/lct_ffhq_more_2step.pdf}
    \caption{Two-step samples on FFHQ $256 \times 256$}
    \label{fig:appendix:ffhq_twostep}
\end{figure}


\begin{figure}[h]
\centering
    \includegraphics[width=0.6\textwidth]{figures/ilct_l2_celeba.pdf}
    \caption{One-step samples on CelebA-HQ $256 \times 256$ (L2 loss)}
    \label{fig:appendix:celeba_onestep_ilct_l2}
\end{figure}

\begin{figure}[h]
\centering
    \includegraphics[width=0.6\textwidth]{figures/ilct_latentlpips_celeba.pdf}
    \caption{One-step samples on CelebA-HQ $256 \times 256$ (E-LatentLPIPS loss)}
    \label{fig:appendix:celeba_onestep_ilct_elatentlpips}
\end{figure}

\end{document}

% \bibliography{iclr2025_conference}
% \bibliographystyle{iclr2025_conference}

\appendix

\subsection{Lloyd-Max Algorithm}
\label{subsec:Lloyd-Max}
For a given quantization bitwidth $B$ and an operand $\bm{X}$, the Lloyd-Max algorithm finds $2^B$ quantization levels $\{\hat{x}_i\}_{i=1}^{2^B}$ such that quantizing $\bm{X}$ by rounding each scalar in $\bm{X}$ to the nearest quantization level minimizes the quantization MSE. 

The algorithm starts with an initial guess of quantization levels and then iteratively computes quantization thresholds $\{\tau_i\}_{i=1}^{2^B-1}$ and updates quantization levels $\{\hat{x}_i\}_{i=1}^{2^B}$. Specifically, at iteration $n$, thresholds are set to the midpoints of the previous iteration's levels:
\begin{align*}
    \tau_i^{(n)}=\frac{\hat{x}_i^{(n-1)}+\hat{x}_{i+1}^{(n-1)}}2 \text{ for } i=1\ldots 2^B-1
\end{align*}
Subsequently, the quantization levels are re-computed as conditional means of the data regions defined by the new thresholds:
\begin{align*}
    \hat{x}_i^{(n)}=\mathbb{E}\left[ \bm{X} \big| \bm{X}\in [\tau_{i-1}^{(n)},\tau_i^{(n)}] \right] \text{ for } i=1\ldots 2^B
\end{align*}
where to satisfy boundary conditions we have $\tau_0=-\infty$ and $\tau_{2^B}=\infty$. The algorithm iterates the above steps until convergence.

Figure \ref{fig:lm_quant} compares the quantization levels of a $7$-bit floating point (E3M3) quantizer (left) to a $7$-bit Lloyd-Max quantizer (right) when quantizing a layer of weights from the GPT3-126M model at a per-tensor granularity. As shown, the Lloyd-Max quantizer achieves substantially lower quantization MSE. Further, Table \ref{tab:FP7_vs_LM7} shows the superior perplexity achieved by Lloyd-Max quantizers for bitwidths of $7$, $6$ and $5$. The difference between the quantizers is clear at 5 bits, where per-tensor FP quantization incurs a drastic and unacceptable increase in perplexity, while Lloyd-Max quantization incurs a much smaller increase. Nevertheless, we note that even the optimal Lloyd-Max quantizer incurs a notable ($\sim 1.5$) increase in perplexity due to the coarse granularity of quantization. 

\begin{figure}[h]
  \centering
  \includegraphics[width=0.7\linewidth]{sections/figures/LM7_FP7.pdf}
  \caption{\small Quantization levels and the corresponding quantization MSE of Floating Point (left) vs Lloyd-Max (right) Quantizers for a layer of weights in the GPT3-126M model.}
  \label{fig:lm_quant}
\end{figure}

\begin{table}[h]\scriptsize
\begin{center}
\caption{\label{tab:FP7_vs_LM7} \small Comparing perplexity (lower is better) achieved by floating point quantizers and Lloyd-Max quantizers on a GPT3-126M model for the Wikitext-103 dataset.}
\begin{tabular}{c|cc|c}
\hline
 \multirow{2}{*}{\textbf{Bitwidth}} & \multicolumn{2}{|c|}{\textbf{Floating-Point Quantizer}} & \textbf{Lloyd-Max Quantizer} \\
 & Best Format & Wikitext-103 Perplexity & Wikitext-103 Perplexity \\
\hline
7 & E3M3 & 18.32 & 18.27 \\
6 & E3M2 & 19.07 & 18.51 \\
5 & E4M0 & 43.89 & 19.71 \\
\hline
\end{tabular}
\end{center}
\end{table}

\subsection{Proof of Local Optimality of LO-BCQ}
\label{subsec:lobcq_opt_proof}
For a given block $\bm{b}_j$, the quantization MSE during LO-BCQ can be empirically evaluated as $\frac{1}{L_b}\lVert \bm{b}_j- \bm{\hat{b}}_j\rVert^2_2$ where $\bm{\hat{b}}_j$ is computed from equation (\ref{eq:clustered_quantization_definition}) as $C_{f(\bm{b}_j)}(\bm{b}_j)$. Further, for a given block cluster $\mathcal{B}_i$, we compute the quantization MSE as $\frac{1}{|\mathcal{B}_{i}|}\sum_{\bm{b} \in \mathcal{B}_{i}} \frac{1}{L_b}\lVert \bm{b}- C_i^{(n)}(\bm{b})\rVert^2_2$. Therefore, at the end of iteration $n$, we evaluate the overall quantization MSE $J^{(n)}$ for a given operand $\bm{X}$ composed of $N_c$ block clusters as:
\begin{align*}
    \label{eq:mse_iter_n}
    J^{(n)} = \frac{1}{N_c} \sum_{i=1}^{N_c} \frac{1}{|\mathcal{B}_{i}^{(n)}|}\sum_{\bm{v} \in \mathcal{B}_{i}^{(n)}} \frac{1}{L_b}\lVert \bm{b}- B_i^{(n)}(\bm{b})\rVert^2_2
\end{align*}

At the end of iteration $n$, the codebooks are updated from $\mathcal{C}^{(n-1)}$ to $\mathcal{C}^{(n)}$. However, the mapping of a given vector $\bm{b}_j$ to quantizers $\mathcal{C}^{(n)}$ remains as  $f^{(n)}(\bm{b}_j)$. At the next iteration, during the vector clustering step, $f^{(n+1)}(\bm{b}_j)$ finds new mapping of $\bm{b}_j$ to updated codebooks $\mathcal{C}^{(n)}$ such that the quantization MSE over the candidate codebooks is minimized. Therefore, we obtain the following result for $\bm{b}_j$:
\begin{align*}
\frac{1}{L_b}\lVert \bm{b}_j - C_{f^{(n+1)}(\bm{b}_j)}^{(n)}(\bm{b}_j)\rVert^2_2 \le \frac{1}{L_b}\lVert \bm{b}_j - C_{f^{(n)}(\bm{b}_j)}^{(n)}(\bm{b}_j)\rVert^2_2
\end{align*}

That is, quantizing $\bm{b}_j$ at the end of the block clustering step of iteration $n+1$ results in lower quantization MSE compared to quantizing at the end of iteration $n$. Since this is true for all $\bm{b} \in \bm{X}$, we assert the following:
\begin{equation}
\begin{split}
\label{eq:mse_ineq_1}
    \tilde{J}^{(n+1)} &= \frac{1}{N_c} \sum_{i=1}^{N_c} \frac{1}{|\mathcal{B}_{i}^{(n+1)}|}\sum_{\bm{b} \in \mathcal{B}_{i}^{(n+1)}} \frac{1}{L_b}\lVert \bm{b} - C_i^{(n)}(b)\rVert^2_2 \le J^{(n)}
\end{split}
\end{equation}
where $\tilde{J}^{(n+1)}$ is the the quantization MSE after the vector clustering step at iteration $n+1$.

Next, during the codebook update step (\ref{eq:quantizers_update}) at iteration $n+1$, the per-cluster codebooks $\mathcal{C}^{(n)}$ are updated to $\mathcal{C}^{(n+1)}$ by invoking the Lloyd-Max algorithm \citep{Lloyd}. We know that for any given value distribution, the Lloyd-Max algorithm minimizes the quantization MSE. Therefore, for a given vector cluster $\mathcal{B}_i$ we obtain the following result:

\begin{equation}
    \frac{1}{|\mathcal{B}_{i}^{(n+1)}|}\sum_{\bm{b} \in \mathcal{B}_{i}^{(n+1)}} \frac{1}{L_b}\lVert \bm{b}- C_i^{(n+1)}(\bm{b})\rVert^2_2 \le \frac{1}{|\mathcal{B}_{i}^{(n+1)}|}\sum_{\bm{b} \in \mathcal{B}_{i}^{(n+1)}} \frac{1}{L_b}\lVert \bm{b}- C_i^{(n)}(\bm{b})\rVert^2_2
\end{equation}

The above equation states that quantizing the given block cluster $\mathcal{B}_i$ after updating the associated codebook from $C_i^{(n)}$ to $C_i^{(n+1)}$ results in lower quantization MSE. Since this is true for all the block clusters, we derive the following result: 
\begin{equation}
\begin{split}
\label{eq:mse_ineq_2}
     J^{(n+1)} &= \frac{1}{N_c} \sum_{i=1}^{N_c} \frac{1}{|\mathcal{B}_{i}^{(n+1)}|}\sum_{\bm{b} \in \mathcal{B}_{i}^{(n+1)}} \frac{1}{L_b}\lVert \bm{b}- C_i^{(n+1)}(\bm{b})\rVert^2_2  \le \tilde{J}^{(n+1)}   
\end{split}
\end{equation}

Following (\ref{eq:mse_ineq_1}) and (\ref{eq:mse_ineq_2}), we find that the quantization MSE is non-increasing for each iteration, that is, $J^{(1)} \ge J^{(2)} \ge J^{(3)} \ge \ldots \ge J^{(M)}$ where $M$ is the maximum number of iterations. 
%Therefore, we can say that if the algorithm converges, then it must be that it has converged to a local minimum. 
\hfill $\blacksquare$


\begin{figure}
    \begin{center}
    \includegraphics[width=0.5\textwidth]{sections//figures/mse_vs_iter.pdf}
    \end{center}
    \caption{\small NMSE vs iterations during LO-BCQ compared to other block quantization proposals}
    \label{fig:nmse_vs_iter}
\end{figure}

Figure \ref{fig:nmse_vs_iter} shows the empirical convergence of LO-BCQ across several block lengths and number of codebooks. Also, the MSE achieved by LO-BCQ is compared to baselines such as MXFP and VSQ. As shown, LO-BCQ converges to a lower MSE than the baselines. Further, we achieve better convergence for larger number of codebooks ($N_c$) and for a smaller block length ($L_b$), both of which increase the bitwidth of BCQ (see Eq \ref{eq:bitwidth_bcq}).


\subsection{Additional Accuracy Results}
%Table \ref{tab:lobcq_config} lists the various LOBCQ configurations and their corresponding bitwidths.
\begin{table}
\setlength{\tabcolsep}{4.75pt}
\begin{center}
\caption{\label{tab:lobcq_config} Various LO-BCQ configurations and their bitwidths.}
\begin{tabular}{|c||c|c|c|c||c|c||c|} 
\hline
 & \multicolumn{4}{|c||}{$L_b=8$} & \multicolumn{2}{|c||}{$L_b=4$} & $L_b=2$ \\
 \hline
 \backslashbox{$L_A$\kern-1em}{\kern-1em$N_c$} & 2 & 4 & 8 & 16 & 2 & 4 & 2 \\
 \hline
 64 & 4.25 & 4.375 & 4.5 & 4.625 & 4.375 & 4.625 & 4.625\\
 \hline
 32 & 4.375 & 4.5 & 4.625& 4.75 & 4.5 & 4.75 & 4.75 \\
 \hline
 16 & 4.625 & 4.75& 4.875 & 5 & 4.75 & 5 & 5 \\
 \hline
\end{tabular}
\end{center}
\end{table}

%\subsection{Perplexity achieved by various LO-BCQ configurations on Wikitext-103 dataset}

\begin{table} \centering
\begin{tabular}{|c||c|c|c|c||c|c||c|} 
\hline
 $L_b \rightarrow$& \multicolumn{4}{c||}{8} & \multicolumn{2}{c||}{4} & 2\\
 \hline
 \backslashbox{$L_A$\kern-1em}{\kern-1em$N_c$} & 2 & 4 & 8 & 16 & 2 & 4 & 2  \\
 %$N_c \rightarrow$ & 2 & 4 & 8 & 16 & 2 & 4 & 2 \\
 \hline
 \hline
 \multicolumn{8}{c}{GPT3-1.3B (FP32 PPL = 9.98)} \\ 
 \hline
 \hline
 64 & 10.40 & 10.23 & 10.17 & 10.15 &  10.28 & 10.18 & 10.19 \\
 \hline
 32 & 10.25 & 10.20 & 10.15 & 10.12 &  10.23 & 10.17 & 10.17 \\
 \hline
 16 & 10.22 & 10.16 & 10.10 & 10.09 &  10.21 & 10.14 & 10.16 \\
 \hline
  \hline
 \multicolumn{8}{c}{GPT3-8B (FP32 PPL = 7.38)} \\ 
 \hline
 \hline
 64 & 7.61 & 7.52 & 7.48 &  7.47 &  7.55 &  7.49 & 7.50 \\
 \hline
 32 & 7.52 & 7.50 & 7.46 &  7.45 &  7.52 &  7.48 & 7.48  \\
 \hline
 16 & 7.51 & 7.48 & 7.44 &  7.44 &  7.51 &  7.49 & 7.47  \\
 \hline
\end{tabular}
\caption{\label{tab:ppl_gpt3_abalation} Wikitext-103 perplexity across GPT3-1.3B and 8B models.}
\end{table}

\begin{table} \centering
\begin{tabular}{|c||c|c|c|c||} 
\hline
 $L_b \rightarrow$& \multicolumn{4}{c||}{8}\\
 \hline
 \backslashbox{$L_A$\kern-1em}{\kern-1em$N_c$} & 2 & 4 & 8 & 16 \\
 %$N_c \rightarrow$ & 2 & 4 & 8 & 16 & 2 & 4 & 2 \\
 \hline
 \hline
 \multicolumn{5}{|c|}{Llama2-7B (FP32 PPL = 5.06)} \\ 
 \hline
 \hline
 64 & 5.31 & 5.26 & 5.19 & 5.18  \\
 \hline
 32 & 5.23 & 5.25 & 5.18 & 5.15  \\
 \hline
 16 & 5.23 & 5.19 & 5.16 & 5.14  \\
 \hline
 \multicolumn{5}{|c|}{Nemotron4-15B (FP32 PPL = 5.87)} \\ 
 \hline
 \hline
 64  & 6.3 & 6.20 & 6.13 & 6.08  \\
 \hline
 32  & 6.24 & 6.12 & 6.07 & 6.03  \\
 \hline
 16  & 6.12 & 6.14 & 6.04 & 6.02  \\
 \hline
 \multicolumn{5}{|c|}{Nemotron4-340B (FP32 PPL = 3.48)} \\ 
 \hline
 \hline
 64 & 3.67 & 3.62 & 3.60 & 3.59 \\
 \hline
 32 & 3.63 & 3.61 & 3.59 & 3.56 \\
 \hline
 16 & 3.61 & 3.58 & 3.57 & 3.55 \\
 \hline
\end{tabular}
\caption{\label{tab:ppl_llama7B_nemo15B} Wikitext-103 perplexity compared to FP32 baseline in Llama2-7B and Nemotron4-15B, 340B models}
\end{table}

%\subsection{Perplexity achieved by various LO-BCQ configurations on MMLU dataset}


\begin{table} \centering
\begin{tabular}{|c||c|c|c|c||c|c|c|c|} 
\hline
 $L_b \rightarrow$& \multicolumn{4}{c||}{8} & \multicolumn{4}{c||}{8}\\
 \hline
 \backslashbox{$L_A$\kern-1em}{\kern-1em$N_c$} & 2 & 4 & 8 & 16 & 2 & 4 & 8 & 16  \\
 %$N_c \rightarrow$ & 2 & 4 & 8 & 16 & 2 & 4 & 2 \\
 \hline
 \hline
 \multicolumn{5}{|c|}{Llama2-7B (FP32 Accuracy = 45.8\%)} & \multicolumn{4}{|c|}{Llama2-70B (FP32 Accuracy = 69.12\%)} \\ 
 \hline
 \hline
 64 & 43.9 & 43.4 & 43.9 & 44.9 & 68.07 & 68.27 & 68.17 & 68.75 \\
 \hline
 32 & 44.5 & 43.8 & 44.9 & 44.5 & 68.37 & 68.51 & 68.35 & 68.27  \\
 \hline
 16 & 43.9 & 42.7 & 44.9 & 45 & 68.12 & 68.77 & 68.31 & 68.59  \\
 \hline
 \hline
 \multicolumn{5}{|c|}{GPT3-22B (FP32 Accuracy = 38.75\%)} & \multicolumn{4}{|c|}{Nemotron4-15B (FP32 Accuracy = 64.3\%)} \\ 
 \hline
 \hline
 64 & 36.71 & 38.85 & 38.13 & 38.92 & 63.17 & 62.36 & 63.72 & 64.09 \\
 \hline
 32 & 37.95 & 38.69 & 39.45 & 38.34 & 64.05 & 62.30 & 63.8 & 64.33  \\
 \hline
 16 & 38.88 & 38.80 & 38.31 & 38.92 & 63.22 & 63.51 & 63.93 & 64.43  \\
 \hline
\end{tabular}
\caption{\label{tab:mmlu_abalation} Accuracy on MMLU dataset across GPT3-22B, Llama2-7B, 70B and Nemotron4-15B models.}
\end{table}


%\subsection{Perplexity achieved by various LO-BCQ configurations on LM evaluation harness}

\begin{table} \centering
\begin{tabular}{|c||c|c|c|c||c|c|c|c|} 
\hline
 $L_b \rightarrow$& \multicolumn{4}{c||}{8} & \multicolumn{4}{c||}{8}\\
 \hline
 \backslashbox{$L_A$\kern-1em}{\kern-1em$N_c$} & 2 & 4 & 8 & 16 & 2 & 4 & 8 & 16  \\
 %$N_c \rightarrow$ & 2 & 4 & 8 & 16 & 2 & 4 & 2 \\
 \hline
 \hline
 \multicolumn{5}{|c|}{Race (FP32 Accuracy = 37.51\%)} & \multicolumn{4}{|c|}{Boolq (FP32 Accuracy = 64.62\%)} \\ 
 \hline
 \hline
 64 & 36.94 & 37.13 & 36.27 & 37.13 & 63.73 & 62.26 & 63.49 & 63.36 \\
 \hline
 32 & 37.03 & 36.36 & 36.08 & 37.03 & 62.54 & 63.51 & 63.49 & 63.55  \\
 \hline
 16 & 37.03 & 37.03 & 36.46 & 37.03 & 61.1 & 63.79 & 63.58 & 63.33  \\
 \hline
 \hline
 \multicolumn{5}{|c|}{Winogrande (FP32 Accuracy = 58.01\%)} & \multicolumn{4}{|c|}{Piqa (FP32 Accuracy = 74.21\%)} \\ 
 \hline
 \hline
 64 & 58.17 & 57.22 & 57.85 & 58.33 & 73.01 & 73.07 & 73.07 & 72.80 \\
 \hline
 32 & 59.12 & 58.09 & 57.85 & 58.41 & 73.01 & 73.94 & 72.74 & 73.18  \\
 \hline
 16 & 57.93 & 58.88 & 57.93 & 58.56 & 73.94 & 72.80 & 73.01 & 73.94  \\
 \hline
\end{tabular}
\caption{\label{tab:mmlu_abalation} Accuracy on LM evaluation harness tasks on GPT3-1.3B model.}
\end{table}

\begin{table} \centering
\begin{tabular}{|c||c|c|c|c||c|c|c|c|} 
\hline
 $L_b \rightarrow$& \multicolumn{4}{c||}{8} & \multicolumn{4}{c||}{8}\\
 \hline
 \backslashbox{$L_A$\kern-1em}{\kern-1em$N_c$} & 2 & 4 & 8 & 16 & 2 & 4 & 8 & 16  \\
 %$N_c \rightarrow$ & 2 & 4 & 8 & 16 & 2 & 4 & 2 \\
 \hline
 \hline
 \multicolumn{5}{|c|}{Race (FP32 Accuracy = 41.34\%)} & \multicolumn{4}{|c|}{Boolq (FP32 Accuracy = 68.32\%)} \\ 
 \hline
 \hline
 64 & 40.48 & 40.10 & 39.43 & 39.90 & 69.20 & 68.41 & 69.45 & 68.56 \\
 \hline
 32 & 39.52 & 39.52 & 40.77 & 39.62 & 68.32 & 67.43 & 68.17 & 69.30  \\
 \hline
 16 & 39.81 & 39.71 & 39.90 & 40.38 & 68.10 & 66.33 & 69.51 & 69.42  \\
 \hline
 \hline
 \multicolumn{5}{|c|}{Winogrande (FP32 Accuracy = 67.88\%)} & \multicolumn{4}{|c|}{Piqa (FP32 Accuracy = 78.78\%)} \\ 
 \hline
 \hline
 64 & 66.85 & 66.61 & 67.72 & 67.88 & 77.31 & 77.42 & 77.75 & 77.64 \\
 \hline
 32 & 67.25 & 67.72 & 67.72 & 67.00 & 77.31 & 77.04 & 77.80 & 77.37  \\
 \hline
 16 & 68.11 & 68.90 & 67.88 & 67.48 & 77.37 & 78.13 & 78.13 & 77.69  \\
 \hline
\end{tabular}
\caption{\label{tab:mmlu_abalation} Accuracy on LM evaluation harness tasks on GPT3-8B model.}
\end{table}

\begin{table} \centering
\begin{tabular}{|c||c|c|c|c||c|c|c|c|} 
\hline
 $L_b \rightarrow$& \multicolumn{4}{c||}{8} & \multicolumn{4}{c||}{8}\\
 \hline
 \backslashbox{$L_A$\kern-1em}{\kern-1em$N_c$} & 2 & 4 & 8 & 16 & 2 & 4 & 8 & 16  \\
 %$N_c \rightarrow$ & 2 & 4 & 8 & 16 & 2 & 4 & 2 \\
 \hline
 \hline
 \multicolumn{5}{|c|}{Race (FP32 Accuracy = 40.67\%)} & \multicolumn{4}{|c|}{Boolq (FP32 Accuracy = 76.54\%)} \\ 
 \hline
 \hline
 64 & 40.48 & 40.10 & 39.43 & 39.90 & 75.41 & 75.11 & 77.09 & 75.66 \\
 \hline
 32 & 39.52 & 39.52 & 40.77 & 39.62 & 76.02 & 76.02 & 75.96 & 75.35  \\
 \hline
 16 & 39.81 & 39.71 & 39.90 & 40.38 & 75.05 & 73.82 & 75.72 & 76.09  \\
 \hline
 \hline
 \multicolumn{5}{|c|}{Winogrande (FP32 Accuracy = 70.64\%)} & \multicolumn{4}{|c|}{Piqa (FP32 Accuracy = 79.16\%)} \\ 
 \hline
 \hline
 64 & 69.14 & 70.17 & 70.17 & 70.56 & 78.24 & 79.00 & 78.62 & 78.73 \\
 \hline
 32 & 70.96 & 69.69 & 71.27 & 69.30 & 78.56 & 79.49 & 79.16 & 78.89  \\
 \hline
 16 & 71.03 & 69.53 & 69.69 & 70.40 & 78.13 & 79.16 & 79.00 & 79.00  \\
 \hline
\end{tabular}
\caption{\label{tab:mmlu_abalation} Accuracy on LM evaluation harness tasks on GPT3-22B model.}
\end{table}

\begin{table} \centering
\begin{tabular}{|c||c|c|c|c||c|c|c|c|} 
\hline
 $L_b \rightarrow$& \multicolumn{4}{c||}{8} & \multicolumn{4}{c||}{8}\\
 \hline
 \backslashbox{$L_A$\kern-1em}{\kern-1em$N_c$} & 2 & 4 & 8 & 16 & 2 & 4 & 8 & 16  \\
 %$N_c \rightarrow$ & 2 & 4 & 8 & 16 & 2 & 4 & 2 \\
 \hline
 \hline
 \multicolumn{5}{|c|}{Race (FP32 Accuracy = 44.4\%)} & \multicolumn{4}{|c|}{Boolq (FP32 Accuracy = 79.29\%)} \\ 
 \hline
 \hline
 64 & 42.49 & 42.51 & 42.58 & 43.45 & 77.58 & 77.37 & 77.43 & 78.1 \\
 \hline
 32 & 43.35 & 42.49 & 43.64 & 43.73 & 77.86 & 75.32 & 77.28 & 77.86  \\
 \hline
 16 & 44.21 & 44.21 & 43.64 & 42.97 & 78.65 & 77 & 76.94 & 77.98  \\
 \hline
 \hline
 \multicolumn{5}{|c|}{Winogrande (FP32 Accuracy = 69.38\%)} & \multicolumn{4}{|c|}{Piqa (FP32 Accuracy = 78.07\%)} \\ 
 \hline
 \hline
 64 & 68.9 & 68.43 & 69.77 & 68.19 & 77.09 & 76.82 & 77.09 & 77.86 \\
 \hline
 32 & 69.38 & 68.51 & 68.82 & 68.90 & 78.07 & 76.71 & 78.07 & 77.86  \\
 \hline
 16 & 69.53 & 67.09 & 69.38 & 68.90 & 77.37 & 77.8 & 77.91 & 77.69  \\
 \hline
\end{tabular}
\caption{\label{tab:mmlu_abalation} Accuracy on LM evaluation harness tasks on Llama2-7B model.}
\end{table}

\begin{table} \centering
\begin{tabular}{|c||c|c|c|c||c|c|c|c|} 
\hline
 $L_b \rightarrow$& \multicolumn{4}{c||}{8} & \multicolumn{4}{c||}{8}\\
 \hline
 \backslashbox{$L_A$\kern-1em}{\kern-1em$N_c$} & 2 & 4 & 8 & 16 & 2 & 4 & 8 & 16  \\
 %$N_c \rightarrow$ & 2 & 4 & 8 & 16 & 2 & 4 & 2 \\
 \hline
 \hline
 \multicolumn{5}{|c|}{Race (FP32 Accuracy = 48.8\%)} & \multicolumn{4}{|c|}{Boolq (FP32 Accuracy = 85.23\%)} \\ 
 \hline
 \hline
 64 & 49.00 & 49.00 & 49.28 & 48.71 & 82.82 & 84.28 & 84.03 & 84.25 \\
 \hline
 32 & 49.57 & 48.52 & 48.33 & 49.28 & 83.85 & 84.46 & 84.31 & 84.93  \\
 \hline
 16 & 49.85 & 49.09 & 49.28 & 48.99 & 85.11 & 84.46 & 84.61 & 83.94  \\
 \hline
 \hline
 \multicolumn{5}{|c|}{Winogrande (FP32 Accuracy = 79.95\%)} & \multicolumn{4}{|c|}{Piqa (FP32 Accuracy = 81.56\%)} \\ 
 \hline
 \hline
 64 & 78.77 & 78.45 & 78.37 & 79.16 & 81.45 & 80.69 & 81.45 & 81.5 \\
 \hline
 32 & 78.45 & 79.01 & 78.69 & 80.66 & 81.56 & 80.58 & 81.18 & 81.34  \\
 \hline
 16 & 79.95 & 79.56 & 79.79 & 79.72 & 81.28 & 81.66 & 81.28 & 80.96  \\
 \hline
\end{tabular}
\caption{\label{tab:mmlu_abalation} Accuracy on LM evaluation harness tasks on Llama2-70B model.}
\end{table}

%\section{MSE Studies}
%\textcolor{red}{TODO}


\subsection{Number Formats and Quantization Method}
\label{subsec:numFormats_quantMethod}
\subsubsection{Integer Format}
An $n$-bit signed integer (INT) is typically represented with a 2s-complement format \citep{yao2022zeroquant,xiao2023smoothquant,dai2021vsq}, where the most significant bit denotes the sign.

\subsubsection{Floating Point Format}
An $n$-bit signed floating point (FP) number $x$ comprises of a 1-bit sign ($x_{\mathrm{sign}}$), $B_m$-bit mantissa ($x_{\mathrm{mant}}$) and $B_e$-bit exponent ($x_{\mathrm{exp}}$) such that $B_m+B_e=n-1$. The associated constant exponent bias ($E_{\mathrm{bias}}$) is computed as $(2^{{B_e}-1}-1)$. We denote this format as $E_{B_e}M_{B_m}$.  

\subsubsection{Quantization Scheme}
\label{subsec:quant_method}
A quantization scheme dictates how a given unquantized tensor is converted to its quantized representation. We consider FP formats for the purpose of illustration. Given an unquantized tensor $\bm{X}$ and an FP format $E_{B_e}M_{B_m}$, we first, we compute the quantization scale factor $s_X$ that maps the maximum absolute value of $\bm{X}$ to the maximum quantization level of the $E_{B_e}M_{B_m}$ format as follows:
\begin{align}
\label{eq:sf}
    s_X = \frac{\mathrm{max}(|\bm{X}|)}{\mathrm{max}(E_{B_e}M_{B_m})}
\end{align}
In the above equation, $|\cdot|$ denotes the absolute value function.

Next, we scale $\bm{X}$ by $s_X$ and quantize it to $\hat{\bm{X}}$ by rounding it to the nearest quantization level of $E_{B_e}M_{B_m}$ as:

\begin{align}
\label{eq:tensor_quant}
    \hat{\bm{X}} = \text{round-to-nearest}\left(\frac{\bm{X}}{s_X}, E_{B_e}M_{B_m}\right)
\end{align}

We perform dynamic max-scaled quantization \citep{wu2020integer}, where the scale factor $s$ for activations is dynamically computed during runtime.

\subsection{Vector Scaled Quantization}
\begin{wrapfigure}{r}{0.35\linewidth}
  \centering
  \includegraphics[width=\linewidth]{sections/figures/vsquant.jpg}
  \caption{\small Vectorwise decomposition for per-vector scaled quantization (VSQ \citep{dai2021vsq}).}
  \label{fig:vsquant}
\end{wrapfigure}
During VSQ \citep{dai2021vsq}, the operand tensors are decomposed into 1D vectors in a hardware friendly manner as shown in Figure \ref{fig:vsquant}. Since the decomposed tensors are used as operands in matrix multiplications during inference, it is beneficial to perform this decomposition along the reduction dimension of the multiplication. The vectorwise quantization is performed similar to tensorwise quantization described in Equations \ref{eq:sf} and \ref{eq:tensor_quant}, where a scale factor $s_v$ is required for each vector $\bm{v}$ that maps the maximum absolute value of that vector to the maximum quantization level. While smaller vector lengths can lead to larger accuracy gains, the associated memory and computational overheads due to the per-vector scale factors increases. To alleviate these overheads, VSQ \citep{dai2021vsq} proposed a second level quantization of the per-vector scale factors to unsigned integers, while MX \citep{rouhani2023shared} quantizes them to integer powers of 2 (denoted as $2^{INT}$).

\subsubsection{MX Format}
The MX format proposed in \citep{rouhani2023microscaling} introduces the concept of sub-block shifting. For every two scalar elements of $b$-bits each, there is a shared exponent bit. The value of this exponent bit is determined through an empirical analysis that targets minimizing quantization MSE. We note that the FP format $E_{1}M_{b}$ is strictly better than MX from an accuracy perspective since it allocates a dedicated exponent bit to each scalar as opposed to sharing it across two scalars. Therefore, we conservatively bound the accuracy of a $b+2$-bit signed MX format with that of a $E_{1}M_{b}$ format in our comparisons. For instance, we use E1M2 format as a proxy for MX4.

\begin{figure}
    \centering
    \includegraphics[width=1\linewidth]{sections//figures/BlockFormats.pdf}
    \caption{\small Comparing LO-BCQ to MX format.}
    \label{fig:block_formats}
\end{figure}

Figure \ref{fig:block_formats} compares our $4$-bit LO-BCQ block format to MX \citep{rouhani2023microscaling}. As shown, both LO-BCQ and MX decompose a given operand tensor into block arrays and each block array into blocks. Similar to MX, we find that per-block quantization ($L_b < L_A$) leads to better accuracy due to increased flexibility. While MX achieves this through per-block $1$-bit micro-scales, we associate a dedicated codebook to each block through a per-block codebook selector. Further, MX quantizes the per-block array scale-factor to E8M0 format without per-tensor scaling. In contrast during LO-BCQ, we find that per-tensor scaling combined with quantization of per-block array scale-factor to E4M3 format results in superior inference accuracy across models. 


\end{document}