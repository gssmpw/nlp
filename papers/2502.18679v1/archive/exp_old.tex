\documentclass{article}


% if you need to pass options to natbib, use, e.g.:
%     \PassOptionsToPackage{numbers, compress}{natbib}
% before loading neurips_2023


% ready for submission
% \usepackage{neurips_2023}


% to compile a preprint version, e.g., for submission to arXiv, add add the
% [preprint] option:
    % \usepackage[preprint]{neurips_2023}


% to compile a camera-ready version, add the [final] option, e.g.:
    \usepackage[final]{neurips_2023}


% to avoid loading the natbib package, add option nonatbib:
%    \usepackage[nonatbib]{neurips_2023}


\usepackage[utf8]{inputenc} % allow utf-8 input
\usepackage[T1]{fontenc}    % use 8-bit T1 fonts
\usepackage{hyperref}       % hyperlinks
\usepackage{url}            % simple URL typesetting
\usepackage{booktabs}       % professional-quality tables
\usepackage{amsfonts}       % blackboard math symbols
\usepackage{nicefrac}       % compact symbols for 1/2, etc.
\usepackage{microtype}      % microtypography
\usepackage{xcolor}         % colors

%%%%%%%%%%%%%%%%%%%% Added package%%%%%%%%%%%%%%%%%%%%
\usepackage{graphicx}
\usepackage{subfigure}
\usepackage{makecell}
\usepackage{algorithm,algorithmic}
\usepackage{paralist,amsmath, amssymb,bm}
\usepackage{multirow}
\usepackage{ntheorem}
\newtheorem{thm}{Theorem}
\newtheorem{prop}{Proposition}
\newtheorem{lemma}{Lemma}
\newtheorem*{proof}{Proof}
\newtheorem{cor}{Corollary}
\newtheorem{definition}{Definition}
\newtheorem{asm}{Assumption}
\usepackage{enumitem}
\usepackage{textcase}
\usepackage{booktabs}
\usepackage{fancybox}
\usepackage{mathtools}

%%%% mathcal
\def \S {\mathbf{S}}
\def \L {\mathcal{L}}
\def \A {\mathcal{A}}
\def \X {\mathcal{X}}
\def \O {\mathcal{O}}
\def \Y {\mathcal{Y}}
\def \Ab {\bar{\A}}
\def \R {\mathbb{R}}
\def \Kt {\widetilde{K}}
\def \k {\mathbf{k}}
\def \w {\mathbf{w}}
\def \v {\mathbf{v}}
\def \t {\mathbf{t}}
\def \x {\mathbf{x}}
\def \Se {\mathcal{S}}
\def \E {\mathrm{E}}
\def \Rh {\widehat{R}}
\def \x {\mathbf{x}}
\def \p {\mathbf{p}}
\def \a {\mathbf{a}}
\def \diag {\mbox{diag}}
\def \b {\mathbf{b}}
\def \e {\mathbf{e}}
\def \ba {\boldsymbol{\alpha}}
\def \c {\mathbf{c}}
\def \tr {\mbox{tr}}
\def \d {\mathbf{d}}
\def \db {\bar\mathbf{d}}
\def \1 {\mathbf{1}}

\def \z {\mathbf{z}}
\def \s {\mathbf{s}}
\def \bh {\widehat{\b}}
\def \y {\mathbf{y}}
\def \u {\mathbf{u}}
\def \uh {\widehat{\u}}
\def \H {\mathcal{H}}
\def \g {\mathbf{g}}
\def \F {\mathcal{F}}
\def \I {\mathbb{I}}
\def \P {\mathcal{P}}
\def \Q {\mathcal{Q}}
\def \xh {\widehat{\x}}
\def \wh {\widehat{\w}}
\def \lambdah {\widehat{\lambda}}

\def \ah {\widehat{\a}}
\def \Rc {\mathcal R}
\def \Sigmah {\widehat\Sigma}

\def \Bh {\widehat B}
\def \Ah {\widehat A}
\def \Uh {\widehat U}
\def \Ut {\widetilde U}
\def \B {\mathcalB}
\def \C {\mathbf C}
\def \U {\mathbf U}
\def \Kh {\widehat K}
\def \fh {\widehat f}
\def \yh {\widehat\y}
\def \Xh {\widehat{X}}
\def \Fh {\widehat{F}}

\def \m {\mathbf{m}}
\def \y {\mathbf{y}}
\def \E {\mathrm{E}}
\def \x {\mathbf{x}}
\def \g {\nabla{g}}
\def \D {\mathcal{D}}
\def \z {\mathbf{z}}
\def \u {\mathbf{u}}
\def \H {\mathcal{H}}
\def \Z {\mathcal{Z}}
\def \Pc {\mathcal{P}}
\def \w {\mathbf{w}}
\def \s {\mathbf{s}}
\def \r {\mathbf{r}}
\def \R {\mathbb{R}}
\def \S {\mathcal{S}}
\def \regret {\mbox{regret}}
\def \Uh {\widehat{U}}
\def \Q {\mathcal{Q}}
\def \W {\mathcal{W}}
\def \N {\mathcal{N}}
\def \A {\mathcal{A}}
\def \q {\mathbf{q}}
\def \v {\mathbf{v}}
\def \M {\mathcal{M}}
\def \c {\mathbf{c}}
\def \ph {\widehat{p}}
\def \d {\mathbf{d}}
\def \p {\mathbf{p}}
\def \q {\mathbf{q}}
\def \db {\bar{\d}}
\def \dbb {\bar{d}}

\def \I {\mathbb{I}}
\def \xt {\widetilde{\x}}
\def \yt {\widetilde{\y}}
\def \hrho {\hat{\rho}}

\def \f {\mathbf{f}}
\def \a {\mathbf{a}}
\def \b {\mathbf{b}}
\def \ft {\widetilde{\f}}
\def \bt {\widetilde{\b}}
\def \h {\mathbf{h}}
\def \B {\mathcal{B}}
\def \bts {\widetilde{b}}
\def \fts {\widetilde{f}}
\def \Gh {\widehat{G}}
\def \G {\mathcal {G}}
\def \bh {\widehat{b}}
\def \wh {\widehat{\w}}
\def \Dth {\widehat{\Delta}}
\def \vb {\bar{\mathbf v}}
\def \zt {\widetilde{\z}}
\def \zh {\widehat{\z}}
\def \zts {\widetilde{z}}
\def \s {\mathbf{s}}
\def \gh {\widehat{\g}}
\def \vh {\widehat{\v}}
\def \Sh {\widehat{S}}
\def \rhoh {\widehat{\rho}}
\def \hh {\widehat{\h}}
\def \C {\mathcal{C}}
\def \V {\mathcal{L}}
\def \t {\mathbf{t}}
\def \xh {\widehat{\x}}
\def \Ut {\widetilde{U}}
\def \wt {\m}
\def \Th {\widehat{T}}
\def \Ot {\tilde{\mathcal{O}}}
\def \X {\mathcal{X}}
\def \nb {\widehat{\nabla}}
\def \K {\mathcal{K}}
\def \P {\mathbb{P}}
\def \T {\mathcal{T}}
\def \F {\mathcal{F}}
\def \ft{\widetilde{f}}
\def \Rt {\mathcal{R}}
\def \Rb {\bar{\Rt}}
\def \wb {\bar{\w}}
\def \zu {\underline{\z}}
\def \vect {\text{vec}}
\def \E {\mathbb{E}}
\def \bftau {\boldsymbol{\tau}}
\def\Tau{{\rm T}}
\def\bw{\mathbf{w}}
\def \start {\textsc{Start}}

\usepackage{comment}

\usepackage{tcolorbox}
\tcbuselibrary{minted,breakable,xparse,skins}

\definecolor{bg}{gray}{0.95}
\DeclareTCBListing{mintedbox}{O{}m!O{}}{%
  breakable=true,
  listing engine=minted,
  listing only,
  minted language=#2,
  minted style=default,
  minted options={%
    linenos,
    gobble=0,
    breaklines=true,
    breakafter=,,
    fontsize=\small,
    numbersep=8pt,
    #1},
  boxsep=0pt,
  left skip=0pt,
  right skip=0pt,
  left=25pt,
  right=0pt,
  top=3pt,
  bottom=3pt,
  arc=5pt,
  leftrule=0pt,
  rightrule=0pt,
  bottomrule=2pt,
  toprule=2pt,
  colback=bg,
  colframe=orange!70,
  enhanced,
  overlay={%
    \begin{tcbclipinterior}
    \fill[orange!20!white] (frame.south west) rectangle ([xshift=20pt]frame.north west);
    \end{tcbclipinterior}},
  #3}

\title{ \textsc{Fox}: Finetuning Generative Language Models with A Discriminative Approach}


% The \author macro works with any number of authors. There are two commands
% used to separate the names and addresses of multiple authors: \And and \AND.
%
% Using \And between authors leaves it to LaTeX to determine where to break the
% lines. Using \AND forces a line break at that point. So, if LaTeX puts 3 of 4
% authors names on the first line, and the last on the second line, try using
% \AND instead of \And before the third author name.


\author{%
  Tianbao Yang \\
  % Department of Computer Science and Engineering \\
  Texas A\&M University \\
   College Station, USA \\
  \texttt{tianbao-yang@tamu.edu} \\
  % \And
  % Coauthor \\
  % Affiliation \\
  % Address \\
  % \texttt{email} \\
  % \And
  % Coauthor \\
  % Affiliation \\
  % Address \\
  % \texttt{email} \\
}

\usepackage{enumitem,amssymb}
\newlist{todolist}{itemize}{3}
\setlist[todolist]{label=$\square$}
\usepackage{pifont}
\newcommand{\cmark}{\ding{51}}%
\newcommand{\xmark}{\ding{55}}%
\newcommand{\done}{\rlap{$\square$}{\raisebox{2pt}{\large\hspace{1pt}\cmark}}%
\hspace{-2.5pt}}
\newcommand{\wontfix}{\rlap{$\square$}{\large\hspace{1pt}\xmark}}


\begin{document}


\maketitle


\section{Baselines}

\begin{todolist}
  \item zephyr-sft-full + fox, compare with SPIN. Dataset: ultrachat 50k.
  \begin{todolist}
  \item[\done] equivalence
        \begin{todolist}
            \item[\done] zephyr-fox-s1-e2-tau1e9, finetune zephyr with fox-dro, 1 negative sample and 2 epochs (practically, we set epoch to 6 and early stop at epoch 2) 
            \item[\done] reproduce SPIN
        \end{todolist}
  \item For each $s$, tune the $\tau$ and compare with the baseline $\tau=10^9$.
        \begin{todolist}
            \item zephyr-fox-s1-e2-tau1.0-sampling
            \item zephyr-fox-s1-e2-tau1e3-sampling
            \item zephyr-fox-s1-e2-tau1e6-sampling
            \item zephyr-fox-s1-e2-tau1e9-sampling
            \item zephyr-fox-s2-e2-tau1.0-sampling
            \item zephyr-fox-s2-e2-tau1e3-sampling
            \item zephyr-fox-s2-e2-tau1e6-sampling
            \item zephyr-fox-s2-e2-tau1e9-sampling
            \item zephyr-fox-s4-e2-tau1.0-sampling
            \item zephyr-fox-s4-e2-tau1e3-sampling
            \item zephyr-fox-s4-e2-tau1e6-sampling
            \item zephyr-fox-s4-e2-tau1e9-sampling
            \item possibly tuning $\tau$ in shorter range 
        \end{todolist}
  \end{todolist}
  \item mistral-7B-v0.1 + fox vs ORPO. Dataset: HuggingFaceH4/ultrafeedback\_binarized 61k.
  \begin{todolist}
        \item mistral-fox-s2-e5-tau1.0-sampling
        \item mistral-fox-s2-e5-tau1e3-sampling
        \item mistral-fox-s2-e5-tau1e6-sampling
        \item mistral-fox-s2-e5-tau1e9-sampling
        \item possibly tuning $\tau$ in shorter range 
  \end{todolist}
  \item Fox-Llama-3-8B vs Llama-3-8B-Instruct.
        
\end{todolist}

\section{Ablation Study}
\begin{todolist}
    \item Whether to use the same negative or not in each epoch.
        \begin{todolist}
            \item[\done] zephyr-fox-s1-e2-tau1e9
            \item zephyr-fox-s1-e2-tau1e9-sampling
        \end{todolist}
   \item Whether the larger size of the synthetic data increases the performance? (use a large tau to eliminate the influence of tau on training)
        \begin{todolist}
            \item zephyr-fox-s1-e2-tau1e9-sampling
            \item zephyr-fox-s2-e2-tau1e9-sampling
            \item zephyr-fox-s3-e2-tau1e9-sampling
            \item zephyr-fox-s4-e2-tau1e9-sampling
        \end{todolist}
    \item Increasing the number of negatives vs. increasing the size of the dataset. (How to design the experiment?) 
    \item Change the model used to generate negative data.
        \begin{todolist}
            \item foxllama-s1-e3-tau1e9-lr1e-5(alpaca-zephyr-gen)
            \item foxllama-s1-e3-tau1e9-lr1e-5(alpaca-llama-gen)
            \item generate the negatives using Llama 3.1 70B Instruct offline
            \item zephyr-sft-full+fox (ultrachat, Llama 3.1 70B Instruct generation)
        \end{todolist}
    \item Finetune and evaluate LLMs on specific tasks. (And use a larger number of epochs)
\end{todolist}



\section{Evaluation Method}

\begin{itemize}
    \item Huggingface Leaderboard
    \item perplexity
    \item alpaca eval
\end{itemize}

\section{Results}
\begin{table}[ht]
\begin{tabular}{l|lllllll}
\hline
                 & GSM8k & arc\_c & truthfulqa & winogrande & mmlu  & hellaswag & avg.  \\ \hline
fox-s1-e6-tau1.0 & 39.88 & 64.25  & 54.19      & 77.51      & 59.64 & 83.82     & 63.21 \\
fox-s1-e6-tau5.0 & 30.40 & 64.25  & 52.09      & 76.95      & 59.10 & 83.99     & 61.13 \\
fox-s1-e6-tau1e2 & 29.57 & 64.42  & 51.73      & 77.51      & 59.06 & 83.74     & 61.00 \\
fox-s1-e6-tau1e6 & 36.77 & 65.02  & 54.14      & 77.19      & 59.47 & 83.61     & 62.70 \\
fox-s1-e6-tau1e9 & 34.27 & 66.13  & 50.99      & 77.19      & 58.43 & 83.98     & 61.83 \\
SPIN-iter0       & 38.44 & 64.33  & 49.15      & 77.03      & 60.25 & 84.40     & 62.27 \\
SPIN-iter3       & 37.45 & 65.27  & 54.88      & 77.74      & 60.08 & 85.49     & 63.49 \\ \hline
\end{tabular}
\end{table}

\bibliographystyle{unsrt}
\bibliography{refs}
\end{document}
