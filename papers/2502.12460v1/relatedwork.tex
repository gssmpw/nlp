\section{Related Work}
\label{sec:related}

% \textbf{Access Control Rule Extraction using Type Dependency Graph.}\\
There are broadly two categories of work on generating ABAC rules from existing systems. One of these assumes that the list of authorizations is available in the form of an Access Control Matrix or a set of Role-based Access Control roles and permissions \cite{xu_stoller}\cite{das_et_al}\cite{nakul}. 
% policies written in natural language is a fairly new area of research. 
The other category, which is relatively a more recent effort, involves deriving ABAC rules from security policies specified in a natural language. 
% We focus on this second type here.    
One of the early attempts towards this was made by Alohaly et al. \cite{AlohalyT019}, which extracts only the attributes from natural language ABAC policies. A Convolutional Neural Network (CNN) is used for this purpose. However, this approach does not produce the complete ABAC policies. In contrast, Xia et al. \cite{9995559} generate ABAC policies from natural language text. But their work is restricted only to the healthcare domain. 
Slankas et al. \cite{Slankas} propose a method for extracting relations among different items from natural language artifacts in text, often penned down by domain experts. It uses type dependency graph for rule extraction. 
% These relations are useful in constructing access control rules. 
In another related work, Masoud et al. \cite{masouddbsec2017} start with tokenization and sentence segmentation, followed by lexical parsing with the help of the CoreNLP toolkit. 


Very recently, Abdelgawad et al. \cite{Abdelgawad} have used the spaCy NLP library to obtain ABAC models. A drawback of this approach is the difficulty in obtaining a large set of training data with proper annotation. The associated issues like class imbalance, noisy and incomplete data, etc., further exacerbate the problem. Heaps et al. \cite{userstories} 
use BERT  for access control classification and named entity recognition.  Whether a sentence contains access control information or not is identified by BERT Large transformer. The named entity recognition stage identifies the subjects, objects and the operations that are to be performed in each policy. This is also done by BERT Large transformer. For ensuring correct predictions, attention masks are used to highlight relevant tokens in the input sequence.

% To the best of our knowledge, 
None of the existing work presents a web based tool that lets any user upload their NLACP specification and optionally a list of attributes to generate the corresponding machine enforceable ABAC policies harnessing the power of LLMs as done in LMN.