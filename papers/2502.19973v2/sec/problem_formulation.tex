\section{Problem Formulation}
\label{sec:formatting}
In this section, we introduce the problem formulation for the two benchmarks \textbf{CVQA} and \textbf{CPVQA}.
\subsection{Formulation of CVQA}
\label{subsec:CVQA}
Given a question text $Q_c$ and an image set \(I_c = \{i_1,i_2,\dots,i_n\}\), \textbf{CVQA} can be categorized into three types: props search, props usage, and password clues.
\begin{itemize}
    \item \textbf{Props search.} The question text $Q_{c}$ serves as a guide, directing the model to combine information in complex scenarios $I_{c}$ and identify objects that can be utilized. As illustrated in Figure~\ref{fig:problem_formulation}, the $Q_{c}$ is `\textit{Please find the props or related clues that can be used (interactive) in these scenes.}', guiding the model to infer that the half-pipe in scene \textit{4} can be applied to the missing pipe in scene \textit{1}.
    \item \textbf{Props usage.} The question text $Q_{c}$ explicitly specifies the name of the prop $t_{c}$, and infers where $t_{c}$ can be applied from complex scenarios $I_{c}$. As shown in Figure~\ref{fig:figappendix1}, the $Q_{c}$ is `\textit{And you have a prop: USB, what can this prop interact with?}', guiding the model to infer the appropriate application of $t_{c}$ based on $Q_{c}$ and the $t_{c}$ combination, specifically the computer.
    \item \textbf{Password clues.} The clues for the password, such as numeric codes or letter sequences, are explicitly provided in the question text $Q_{c}$, while clues to decode the password in complex scenarios $I_{c}$ are inferred, or the final sequence is directly given. As shown in Figure~\ref{fig:figappendix2}, $Q_{c}$ is `\textit{Reason within this scenario and derive clues about the numeric code.}', where `\textit{numeric code}' represents password clues. The model is instructed to either infer password-related clues by integrating all scenarios in $I_{c}$ (i.e., computer) or directly provide the password, in this case, `8462'.
\end{itemize}

\begin{figure}[ht]
    \centering
    \includegraphics[width=\linewidth]{figure/figure3_38.pdf}
    \caption{\textbf{Solution Formats.} The solution formats for different benchmarks in \textbf{CVQA} and \textbf{CPVQA}.}
    \label{fig:problem_formulation}
\end{figure}


\subsection{Formulation of CPVQA}
\label{subsec:CPVQA}
Given a question text $Q_p$ and an image set \(I_p = \{i_a,i_b,\dots,i_k\}\), \textbf{CPVQA} is divided into two categories: passwords and sequence rearrangement.
\begin{itemize}
    \item \textbf{Passwords.} The clues for the password, such as a numeric code or a sequence of letters, are explicitly provided in the problem text $Q_{p}$, while the sequence is inferred within complex scenarios $I_{p}$. As illustrated in Figure~\ref{fig:figappendix3}, $Q_{p}$ is `\textit{Reasoning in this scenario and deriving a numerical code.}', where `\textit{numeric code}' serves as the password clue. The model is then directed to infer the numeric password sequence (e.g., 8462) by combining the password sequence found on the computer with the clues in the $Q_{p}$.
    \item \textbf{Sequence rearrangement.} The problem text $Q_{p}$ utilizes codes to represent entities that need to be rearranged within a scene $i_{k}$ in $I_{p}$ , requiring the model to integrate information from all complex scenarios in $I_{p}$ to infer the final sequence code. As illustrated in Figure~\ref{fig:problem_formulation}, $Q_{p}$ encodes the trophies in $i_{k}$, directing the model to combine the trophy sequence in scene x with the encoding defined in $Q_{p}$ to infer the final sequence,  (e.g., BCDA).
\end{itemize}