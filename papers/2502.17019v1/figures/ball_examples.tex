\begin{figure*}
\centering
\begin{tikzpicture}
    \node[anchor=south west,inner sep=0] (image) at (0,0) 
        {\includegraphics[width=0.9\linewidth]{figures/raw/trees.pdf}};

    \begin{scope}[x={(image.south east)},y={(image.north west)}]
        \node[color=black, font=\fontsize{9}{9}\selectfont] at (0.125,1.1) {ball size $512$};
    \end{scope}

    \begin{scope}[x={(image.south east)},y={(image.north west)}]
        \node[color=black, font=\fontsize{9}{9}\selectfont] at (0.475,1.1) {ball size $256$};
    \end{scope}

    \begin{scope}[x={(image.south east)},y={(image.north west)}]
        \node[color=black, font=\fontsize{9}{9}\selectfont] at (0.865,1.1) {ball size $128$};
    \end{scope}


\end{tikzpicture}
\vspace{-1pt}
\caption{Examples of ball trees built on top of data. Partitions at different levels of ball trees are shown. \textbf{Top:} a polypeptide from the molecular dynamics task. \textbf{Center:} a domain from the EAGLE dataset. \textbf{Bottom:} a car surface from the ShapeNet-Car dataset.
}
\label{fig:ball_examples}
\vspace{-3pt}
\end{figure*}
