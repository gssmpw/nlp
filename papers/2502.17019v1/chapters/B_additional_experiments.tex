\section{Extended experiments}
\label{appendix:additional_experiments}
\subsection{Turbulent fluid dynamics} 
We provide additional exemplary rollouts of Erwin for both velocity  (Fig.~\ref{appendix:velocity}) and pressure (Fig.~\ref{appendix:pressure}) fields.

\subsection{Airflow pressure modeling}

\paragraph{Dataset}
We use the ShapeNet-Car dataset generated by \citet{Umetani2018LearningTF} and preprocessed by \citet{alkin2024upt}. It consists of $889$ car models, each car being represented by $3586$ surface points in $3$D space. Airflow was simulated around each car for $10$s (Reynolds number $Re = 5 \times 10^6$) and averaged over the last $4$s to obtain pressure values at each point. The dataset is randomly split into $700$ training and $189$ test samples.

\vspace{-7pt}
\paragraph{Task}
Given surface points, the task is to predict the value of pressure $P \in \mathbb{R}^{N \times 1}$ at each point in $X^{N \times 3}$. The training is done by optimizing the mean squared error loss between predicted and ground truth pressures.

\begin{table}
\vspace{-5pt}
\caption{Test MSE for ShapeNet-Car pressure prediction. [U] and [R] indicate U-Net- and ResNet-like structures respectively.}
\label{table:shapenet}
\begin{center}
\begin{small}
\begin{sc}
\begin{tabular}{lc}
\toprule
Model & MSE, $\times 10^{-2}$ \\
\midrule
U-Net & $6.13$ \\
FNO & $4.04$ \\
GINO & $2.34$ \\
UPT & $2.31$ \\
PointTransformer v3 [U] & $1.78$ \\
Erwin (Ours) [U] & $1.42$ \\
PointTransformer v3 [R] & $\mathbf{0.92}$ \\
\textbf{Erwin (Ours)} [R] & $\mathbf{0.92}$ \\
\bottomrule
\end{tabular}
\vspace{-10pt}
\end{sc}
\end{small}
\end{center}
\end{table}

\vspace{-7pt}
\paragraph{Results}
The results are given in Table~\ref{table:shapenet}. We evaluate PointTransformer v3 and use the baseline results obtained by \citet{alkin2024upt} for U-Net \cite{Ronneberger2015UNetCN}, FNO \cite{Li2020FourierNO}, GINO \cite{Li2023GeometryInformedNO}, and UPT \cite{alkin2024upt}. Both Erwin and PointTransformer v3 achieve significantly lower test MSE compared to other models, which can be attributed to their ability to capture fine geometric details by operating directly on the original point cloud. In comparison, other approaches introduce information loss through compression - UPT encodes the mesh into a latent space representation, while the remaining baselines interpolate the geometry onto regular grids and back. Moreover, in our experiments, ResNet-like configurations that did not include any coarsening performed dramatically better than the ones following the U-Net structure. Overall, this result highlights the potential for Erwin to be used as a scalable neural operator \cite{Wang2024CViTCV}.

\begin{figure*}
\centering
\begin{tikzpicture}
    \node[anchor=south west,inner sep=0] (image) at (0,0) 
        {\includegraphics[width=0.9\linewidth]{figures/raw/trees.pdf}};

    \begin{scope}[x={(image.south east)},y={(image.north west)}]
        \node[color=black, font=\fontsize{9}{9}\selectfont] at (0.125,1.1) {ball size $512$};
    \end{scope}

    \begin{scope}[x={(image.south east)},y={(image.north west)}]
        \node[color=black, font=\fontsize{9}{9}\selectfont] at (0.475,1.1) {ball size $256$};
    \end{scope}

    \begin{scope}[x={(image.south east)},y={(image.north west)}]
        \node[color=black, font=\fontsize{9}{9}\selectfont] at (0.865,1.1) {ball size $128$};
    \end{scope}


\end{tikzpicture}
\vspace{-1pt}
\caption{Examples of ball trees built on top of data. Partitions at different levels of ball trees are shown. \textbf{Top:} a polypeptide from the molecular dynamics task. \textbf{Center:} a domain from the EAGLE dataset. \textbf{Bottom:} a car surface from the ShapeNet-Car dataset.
}
\label{fig:ball_examples}
\vspace{-3pt}
\end{figure*}

