\section{Related Work}
\label{Related Work}
%===================================

Academics and industries across multiple domains are actively exploring post-quantum cryptographic solutions in response to the emerging threats posed by quantum computing advancements.
In the realm of end-to-end secure messaging applications, Signal Messenger recently published a new version of their Extended Triple Diffie-Hellman (X3DH) protocol, called PQXDH, as a quantum-secure protocol \citep{Signal_pqxdh}. 
Although PQXDH uses the NIST standard Key Encapsulation Mechanism (KEM), Kyber \citep{kyber}, which provides post-quantum forward secrecy and a form of cryptographic deniability, it still relies on the hardness of the discrete log problem for authentication. 
Moreover, Apple proposed a post-quantum security protocol called PQ3 for conversations in the iMessage application using Kyber, which was made available to the public with iOS 17.4 and macOS 14.4 \citep{Apple_pq3}.  
 Similarly, the Transport Layer Security (TLS) protocol \citep{TLS}, which secures communication between web browsers and servers, is undergoing a transformation to mitigate quantum threats.  
 The Open Quantum Safe (OQS) \citep{OQS_project} is an open-source project that aims to support the transition to quantum-resistant cryptography.  
 They integrated a library called liboqs into the forks of BoringSSL and OpenSSL1.1.1, and a standalone OQS provider for OpenSSL3 to provide a prototype post-quantum key exchange, authentication, and ciphersuites in a hybrid key exchange in TLS 1.3 \citep{hybrid_key_exchange_TLS_1.3}.  
 
 In the field of federated learning, researchers seek to provide a security mechanism that guarantees the privacy of participants in collaboration with FL projects using various approaches like Homomorphic Encryption (HE) and Differential Privacy (DP) \citep{A_survey_on_security}. 
 As Table \ref{tab: Comparison } shows, only two studies, LaF \citep{Performance_Analysis} and BFL\citep{LaF}, considered post-quantum security concerns in FL environments.  
 This table compares recent studies and identifies their contributions. 
 For instance, Gurung et al. \citep{Performance_Analysis} combined two post-quantum signature schemes, Dilithium and XMSS, to sign transactions in blockchain-based FL.  
 In this study, participants transfer models through signed transactions, which can incur blockchain costs for information transfers.  
 This scheme can provide post-quantum authentication for transmission models; however, the confidentiality of the update models has not been addressed.   Other quantum security studies in the field of federated learning include those of \citep{LaF} and \citep{A_post-quantum_secure}.   
 These two schemes are improved versions of the Google Group scheme \citep{Practical_secure_aggregation} that uses secret sharing to increase the prevention of privacy models against honest-but-curious servers. 
 In these studies, the authors employed two lattice-based cryptosystems, NewHope \citep{NewHope} and Kyber, to encrypt shares between the server and participants. 
 Given that FL systems typically involve several rounds of training, these schemes require key exchange for each round.   
 Although this can provide forward and post-compromise secrecy, it creates heavy data, communication overhead, and time consumption because of the post-quantum key size, which is usually much larger than traditional ones.

\begin{table}[ht]
\large
\centering
\caption{Comparison functionality of related schemes}
\label{tab: Comparison }
\begin{adjustbox}{width=3.2in}
\begin{tblr}{
  column{even} = {c},
  column{3} = {c},
  column{5} = {c},
  column{7} = {c},
  vline{2} = {-}{},
  hline{1-2,11} = {-}{},
}
Capability            &  DAFL  &   BSAFL &  BESIFL   & LaF & BFL  & PQBFL \\
Decentralization      &  \checkmark   &  \checkmark   &   \checkmark   &  \ding{53}   &  \checkmark  & \checkmark   \\
Authentication        &  \checkmark   &  \checkmark   &   \checkmark   &  \ding{53}   &  \checkmark  & \checkmark   \\
Traceability          &  \checkmark   &  \checkmark   &   \checkmark   &  \ding{53}   &  \checkmark  & \checkmark   \\
User privacy          &  \checkmark   &  \ding{53}    &   \checkmark   &  \ding{53}   &  \ding{53}   & \checkmark   \\
Confidentiality       &  \ding{53}    &  \checkmark   &   \ding{53}    &  \checkmark  &  \ding{53}   & \checkmark   \\
Quantum-security      &  \ding{53}    &  \ding{53}    &   \ding{53}    &  \checkmark  &  \checkmark  & \checkmark   \\
Lightweight           &  \checkmark   &  \ding{53}    &   \ding{53}    &  \ding{53}   &  \ding{53}   & \checkmark   \\
Forward secrecy       &  \ding{53}    &  \ding{53}    &   \ding{53}    &  \checkmark  &  \ding{53}   & \checkmark   \\
Post-compromise       &  \ding{53}    &  \ding{53}    &   \ding{53}    &  \checkmark  &  \ding{53}   & \checkmark  
\end{tblr}
\end{adjustbox}
\end{table} 



In addition, \cite{Blockchain-based_decentralized} proposed DAFL as a lightweight digital signature method that facilitates batch verification for authentication to provide a decentralized and simpler framework for FL authentication. 
\cite{Blockchain_and_signcryption} considered the problem of centralized single-layer aggregation in FL and proposed a distributed aggregation architecture called BSAFL by integrating blockchain. 
They introduced signcryption schemes to guarantee the authenticity and confidentiality  of messages in the FL. 
These schemes use quantum-vulnerable cryptography methods to verify the identities of parties. 
They also share public key information and models through blockchain transactions, which are extremely expensive and not cost-effective in the real world. 
Moreover, \cite{BESIFL} proposed a BESIFL paradigm for distributed environments, such as IoT, which leverages blockchain to achieve security using a fully decentralized FL system, integrating mechanisms for the detection of malicious nodes and incentive management in a unified framework.

The PQBFL does not utilize blockchain for key exchanges or model transfers; instead, it uses it to improve security, decentralization, and tracking keys and models.  
In addition to quantum security, the proposed scheme employs a key ratchet mechanism that eliminates the need to exchange keys during each training round.  
It can reduce network overhead and provide forward and post-compromise security, which is ideal for FL systems to improve their performance and security.  
These claimed advantages of the PQBFL are demonstrated and discussed in detail throughout the paper.


%============================