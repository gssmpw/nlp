\section{Method}\label{sec-method}

\subsection{Problem Formulation}

In crowd navigation, a mobile robot navigates to a goal position $g$ without colliding with any of $N$ humans moving in a shared 2D space. The position of the robot is denoted as $x_r^t$, and the positions of humans are denoted as $x_{h,i}^t$, $i\!\in\!\{1, \ldots, N\}$. The robot has a speed limit $v_{max}$. We need to plan velocity action $v_r^t$ given $\left(g, x_r^{1:t}, x_{h,1:N}^{1:t}\right)$, which the robot takes to reach the position at the next time step $x_r^{t+1}$.

\subsection{Preliminaries}

\textbf{Trajectory Prediction.} A trajectory prediction model $TP$ offers explicit modeling of human motion in a near future for planning socially aware robot motion. $TP$ takes robot and human positions in an observation window of length $T_{obs}$ as input, and predict human positions in a future time window of length $T_{pred}$. Our algorithm can take arbitrary models as $TP$. In this work, we use a learning-based trajectory prediction model Gumbel Social Transformer (GST)~\cite{huang2022learning}, which captures social interaction among multiple agents. Note we use a pre-trained GST with frozen weights which is only for inference.

\begin{equation}
    \hat{x}_{h,1:N}^{t+1:t+T_{pred}} = TP\left(x_r^{t-T_{obs}+1:t}, x_{h,1:N}^{t-T_{obs}+1:t}\right)
\end{equation}

\textbf{Human Simulator.} We use a human simulator to generate synthetic human trajectories for conformal prediction. We initialize the robot and the human positions in simulation as $x_r^{t}$ and $x_{h,1:N}^{t}$. Assuming that we have generated a robot plan, we enforce the robot to move along the planned robot trajectory $\bar{x}_{r}^{t+1:t+T_{plan}}$ and run ORCA~\cite{van2011reciprocal} to generate actions for each human to interact with the robot and human neighbors, so we can collect simulated human trajectories $x_{h, 1:N, sim}^{t+1:t+T_{plan}}$ at the end of one simulation episode. ORCA is a multi-agent motion planning algorithm which applies low-dimensional linear programming to compute velocities for all agents to reach respective goals without collision. ORCA has been extensively applied as a crowd simulator for training reinforcement learning crowd navigation policies that are successfully transferred to real world application without fine-tuning~\cite{chen2019crowd}, ~\cite{liu2021decentralized}, ~\cite{liu2023intention}. We randomize human goals and run multiple simulation episodes, and use a sliding window of length $T_{obs}+T_{pred}$ to split the collected trajectories into batches to create a trajectory prediction calibration dataset $\mathcal{D}$.

\textbf{Conformal Prediction.} The trajectory prediction calibration dataset $\mathcal{D}$ includes $M$ samples of past robot trajectories, past human trajectories and future human trajectories. We use $TP$ to make trajectory prediction for each sample, and compute prediction errors of each human's position at each prediction time step in each sample.

\begin{equation}
    e_{i}^{\tau, j} = ||\hat{x}_{h,i}^{T_{obs}+\tau, j} - x_{h,i}^{T_{obs}+\tau, j}||_2, i\!\in\!\{1, \ldots, N\}, j\!\in\!\{1, \ldots, M\}, \tau\!\in\!\{1, \ldots, T_{pred}\}
\end{equation}

We aggregate the prediction errors in terms of the prediction time step $\mathcal{E}^\tau = \{e_{i}^{\tau, j}\}_{i=1:N,j=1:M}$. We assume the errors in $\mathcal{E}^\tau$ are from an exchangeable probability distribution, and sort the errors in a non-decreasing order $\{{e}_{(l)}^{\tau}\}_{l=1:N\times M}$. For a given failure probability $\alpha$, we define confidence interval radius at each prediction time step

\begin{equation}
r_{cp}^{\tau} := {e}_{(\lceil(1-\alpha)(N\times M)\rceil)}, \quad \tau\!\in\!\{1, \ldots, T_{pred}\}
\end{equation}
By treating the trajectory prediction error for any human $i$ at the current time step $t$ as the $N\times M+1$th sample from the exchangeable error distribution, we achieve probabilistic guarantees for prediction at each time step
\begin{equation}
Pr\left(||\hat{x}_{h,i}^{t+\tau}-x_{h,i}^{t+\tau}||_2 \leq
r_{cp}^{\tau}\right) = Pr\left({e}_{(N\times M+1)} \leq
{e}_{(\lceil(1-\alpha)(N\times M)\rceil)}\right) \geq 1 - \alpha
\end{equation}

Note that in ICP, the calibration dataset is composed of the robot plan generated by model predictive control, and the human trajectories simulated based on the assumption that the robot will execute the generated plan.

\textbf{Model Predictive Control.} The model predictive control module (MPC) plans robot motion to reach the goal while satisfying dynamics constraints, control limit constraints, and collision avoidance constraints. In the collision avoidance constraints, $r_r$ is robot radius and $r_h$ is human radius. The conformal interval radii $r_{cp}^{\tau}$'s are incorporated in the collision avoidance constraints to inform MPC about the uncertainty on the predicted human positions. In the optimization problem presented in Equation~\ref{mpc-optimization-v2}, the step-wise cost function includes the goal-reaching cost, the velocity jerk cost, and a regularization cost from the last round of MPC solution, since our algorithm iteratively runs MPC. The regularization cost helps constrain the change of robot plan through iterations, which prevents the drastic oscillation of conformal interval radii and facilitates convergence. The regularization cost is ignored when it is the first round of MPC. Note the collision avoidance constraints make the optimization problem non-convex, but there are usually feasible solutions in practice. 

\vspace{-10pt}
\begin{equation}
\label{mpc-optimization-v2}
\begin{aligned}
\operatorname*{minimize}_{\bar{\mathbf{x}}_r, \bar{\mathbf{v}}_r} \,
& \sum_{\tau=t}^{t+T_{mpc}} \omega_\text{g} ||\bar{x}_r^{\tau} - g||_2^2
+
\sum_{\tau=t}^{t+T_{mpc}-2} \omega_\text{v} ||\bar{v}_r^{\tau + 1} - \bar{v}_r^{\tau}||_2^2 \\
& \qquad \qquad \qquad \qquad \qquad \qquad \qquad \qquad +
\sum_{\tau=t}^{t+T_{mpc}} \omega_\text{reg} ||\bar{x}_r^{\tau} - \bar{x}_{r, k-1}^{\tau}||_2^2 \\
\text{subject to} \quad &\bar{x}_r^{\tau+1} = \bar{x}_r^\tau + \bar{v}_r^\tau \, \Delta T, \quad \tau = t, \ldots, t+T_{mpc}-1, \\
&||\bar{v}_r^\tau||_2 \leq v_{max}, \quad \tau = t, \ldots, t+T_{mpc}-1, \\
&||\bar{x}_r^{t + \tau} - \hat{x}_{h, i}^{t + \tau}||_2 \geq r_r + r_h + r_{cp}^\tau, \quad i = 1, \dots, N, \; \; \tau = 1, \dots, T_{pred}, \\
& \bar{x}_r^t = x_r^t.
\end{aligned}
\end{equation}
where the optimization variable $\bar{\mathbf{x}}_r = (\bar{x}_r^t, \bar{x}_r^{t + 1}, \dots \bar{x}_r^{t + T_{mpc}})$ is the planned robot trajectory, $\bar{\mathbf{v}}_r = (\bar{v}_r^t, \bar{v}_r^{t + 1}, \dots, \bar{v}_r^{t + T_{mpc} - 1})$ is the planned robot velocity, and $\Delta T$ is the time interval between two points on the planned robot trajectory.

\subsection{Interaction-aware Conformal Prediction}
To explicitly address the mutual influence between the robot and the humans during the interaction, Interaction-aware Conformal Prediction (ICP) alternates model predictive control for robot motion planning and conformal prediction for human trajectory prediction, which is presented in Algorithm~\ref{algo:ICP}.

At time $t$, we first feed observed robot and human trajectories into $TP$ to generate predictions of human trajectories $\hat{x}_{h, 1:N}^{t+1:t+T_{pred}}$. With the predictions, we run MPC by assuming confidence interval radii as zero and obtain a nominal robot trajectory $\bar{x}_{r, 0}^{t+1:t+T_{mpc}}$. This nominal robot plan does not have any safety guarantees yet, because no uncertainty quantification has been done for the predicted human trajectories used in MPC.

We introduce an inner iteration to iteratively calibrate the uncertainty of the human trajectory prediction and finetune the robot plan. We simulate crowd motions reacting to the most recent robot plan $\bar{x}_{r, {k-1}}^{t+1:t+T_{mpc}}$ and to collect a trajectory prediction calibration dataset $\mathcal{D}_k$. We perform conformal prediction for the prediction model $TP$ on the calibration dataset $\mathcal{D}_k$ to calculate the conformal interval radii $r_{cp, k}^{1:T_{pred}}$. We then run MPC with the collision avoidance constraints incorporating the updated conformal interval radii, and the regularization cost with respect to the latest MPC solution to generate a new robot plan $\bar{x}_{r, {k}}^{t+1:t+T_{mpc}}$. The new robot plan is used to initiate the next iteration until an iteration limit is reached, or the robot plan and the conformal prediction radii converge. The robot will execute the actions of next $T_{exec}$ steps $\bar{v}_{r, K}^{t:t+T_{exec}-1}$ of the final robot plan generated from the iterations.
\vspace{-15pt}

\begin{algorithm}[hbt!]
    \caption{Interaction-aware Conformal Prediction}\label{algo:ICP}
    \begin{algorithmic}
    \State Load a pre-trained \textbf{trajectory prediction} model $\textit{\textbf{TP}}$.
    \State Set terminal constraints $g$, and control limit constraints $v_{max}$ for model predictive control.
    \For {$t = 1$ to $T$}
        \State Predict future human trajectories $\hat{x}_{h, 1:N}^{t+1:t+T_{pred}}$ by taking past robot and human trajectories as input to $\textit{\textbf{TP}}$.
        \State Initialize conformal interval radii $r_{cp, 0}^{1:T_{pred}}$ as zero.
        \State Set the initial constraints of model predictive control with robot position $x_r^t$.
        \State Set the collision avoidance constraints of model predictive control with predicted human trajectories $\hat{x}_{h, 1:N}^{t+1:t+T_{pred}}$, human radius, robot radius, and the initialized conformal interval radii $r_{cp, 0}^{1:T_{pred}}$.
        \State Run \textbf{model predictive control} to generate action sequence $\bar{v}_{r, 0}^{t:t+T_{mpc}-1}$ and corresponding robot trajectory $\bar{x}_{r, 0}^{t+1:t+T_{mpc}}$.
        \For {$k = 1$ to $K$}
            \State \textbf{Simulate human motion} by assuming robot executes the plan $\bar{x}_{r, k-1}^{t+1:t+T_{mpc}}$ with multiple runs, and collect a trajectory prediction calibration dataset $\mathcal{D}_k$.
            \State Run \textbf{conformal prediction} by evaluating $TP$ on $\mathcal{D}_k$, collecting trajectory prediction errors, and computing the $k$th conformal interval radii $r_{cp, k}^{1:T_{pred}}$ with safe probability $1-\alpha$.
            \State Update the collision avoidance constraints of model predictive control with the $k$th conformal interval radii $r_{cp, k}^{1:T_{pred}}$.
            \State Run \textbf{model predictive control} to generate the $k$th action sequence $\bar{v}_{r, k}^{t:t+T_{mpc}-1}$ and the $k$th robot trajectory $\bar{x}_{r, k}^{t+1:t+T_{mpc}}$.
        \EndFor
        \State The robot executes the actions $\bar{v}_{r, K}^{t:t+T_{exec}-1}$.
    \EndFor
 \end{algorithmic}
 \end{algorithm}
\vspace{-15pt}

When Algorithm~\ref{algo:ICP} converges in the sense that the planned trajectory
from the last iteration induces the same human behavior as 
the previous iteration, then we have the following safety guarantee
\begin{theorem}
    Assume that Algorithm~\ref{algo:ICP} converges at time $t$,
    and the optimization problem in Equation \ref{mpc-optimization-v2} is feasible
    at $t$ with prediction horizon $T_{pred}$. Then the planned trajectory $\bar{x}_r^{t+1:t+T_{mpc}}$ satisfies
    \[
        Pr \left(
            ||\bar{x}_r^{t+\tau} - x_{h, i}^{t+\tau}||_2
            \geq r_r + r_h,
            \forall
            \tau \in
            [T_{pred}],
            \forall
            h \in
            [N]
        \right)
        \ge
        1 - \alpha N T_{pred}.
    \]
\end{theorem}
\begin{proof}
    By the convergence assumption and Theorem~\ref{thm:cp}, 
    for each $h \in \{1, \dots, N\}$ and each 
    $\tau \in \{1, \dots, T_{pred}\}$,
    we have
    \[
        Pr \left(
            ||\hat{x}_{h, i}^{t+\tau} - x_{h, i}^{t+\tau}||_2
            \le r^{\tau}_{cp}
        \right)
        \ge
        1 - \alpha.
    \]
    Further, from the optimization constraints in Equation~\ref{mpc-optimization-v2}, we have
    \[
        ||\bar{x}_r^{t+\tau} - \hat{x}_{h, i}^{t+\tau}||_2
        \geq r_r + r_h + r^{\tau}_{cp},
    \]
    and thus
    \[
        Pr \left(
            ||\bar{x}_r^{t+\tau} - x_{h, i}^{t+\tau}||_2
            \geq r_r + r_h
        \right)
        \ge
        1 - \alpha.
    \]
    Then taking the union bound over $\tau$ and $h$, we get our result.
    \qed
\end{proof}