\section{Experiments}\label{sec-experiments}

\subsection{Experiment Setup}

\textbf{Simulation Environment.} We conduct crowd navigation simulation experiments for evaluation. In each test case, the initial and goal positions of the robot and the humans are randomized. The humans are controlled by ORCA~\cite{van2011reciprocal} to react to each other and the robot. Both robot radius $r_r$ and human radius $r_h$ are set as $0.4\,m$. Both robot max speed and human max speed are set as $v_{max} = 1 \,m/s$. We apply holonomic kinematics to both robot and humans. One time step $\Delta t$ is set as $0.25\,s$. We run 100 test cases for each configuration of ICP and each baseline to report the performance.

\textbf{Baselines.} We show effectiveness of alternating conformal prediction and planning for interaction awareness by comparing ICP to the following methods: 

\begin{itemize}
    \item Offline Conformal Prediction (OffCP) \cite{lindemann2023safe}: an offline method to pre-compute conformal interval radius, and use the fixed conformal interval radius throughout planning and execution. Note OffCP performs simulation of crowd motion among human agents without the robot agent offline, of which the data distribution ignores interaction between robot and humans.  
    \item Adaptive Conformal Prediction (ACP) \cite{dixit2023adaptive}: an online method which adaptively modifies the failure probability $\alpha$ to adjust the conformal interval radius based on conformal prediction from the dataset formed by most recent human and robot trajectories. The original work did not discuss how to compute gradient on the failure probability for multiple humans scenario. We propose two versions: ACP-A averages the gradients computed for each human whether their trajectory prediction error is within the conformal interval radius; ACP-W takes the worst possible gradient, by treating all humans have prediction error beyond the conformal interval radius whenever any of them has prediction error beyond the conformal interval radius. Note that ACP-W retains the asymptotically probabilistic safety guarantees, while ACP-A does not.
    \item Optimal Reciprocal Collision Avoidance (ORCA) \cite{van2011reciprocal}: a reactive navigation method based on the assumption that all agents are velocity obstacles which make similar reasoning on collision avoidance. The robot ORCA configuration is set the same as the human ORCA configuration in simulation experiments. 
\end{itemize}

\textbf{Metrics.} We use metrics in terms of navigation efficiency, social awareness and uncertainty quantification to comprehensively evaluate our method compared to the baselines. We use success rate (SR), robot navigation time (NT) and navigation path length (PL) as performance metrics. SR is the ratio of test cases where the robot successfully reaches the goal without collision with humans. We use intrusion time ratio (ITR) and social distance during intrusion (SD) adopted in \cite{liu2023intention} as social awareness metrics. ITR is the ratio between the number of time steps when the robot collide with any human's ground truth future positions in the prediction horizon window of $T_{pred}$, and the number of time steps for robot to reach the goal. SD is the average distance between the robot and the closest human during intrusion. We use coverage rate (CR) to check the performance on uncertainty quantification. For each human $i$ at time $t$, we check whether the trajectory prediction within the prediction horizon are within the computed conformal interval radii of the ground truth future positions. We average across all $N$ humans through the whole time period $T$ to obtain the coverage rate of one test case. Note the unit of NT is second, and the unit of PL and SD is meter.
\begin{equation}
    CR = \frac{1}{T \times N} \sum_{t=1}^{T} \sum_{i=1}^{N} \prod_{\tau=1}^{T_{pred}} \mathbf{1}[||\hat{x}_{h, i}^{t+\tau} - x_{h, i}^{t+\tau}||_2 < r_{cp}^{\tau}]
\end{equation}

\textbf{Implementation Details.} We set the number of humans as 5, 10, 15, and 20, and run 100 test cases for each crowd setup. We set the number of iterations as 1, 3, and 10 for ICP. Note 1 iteration is also interaction-aware, because it includes two rounds of MPC and one round of simulation. The size of the calibration dataset is adjusted by controlling the number of episodes to run in the simulator, which we define as calibration size (CS). The calibration size is tested across 2, 4, 8, 16, 32, and 64 with 100 test cases run for each configuration. We evaluate two types of execution scheme (ES). The first type is we execute a sequence of actions $\bar{v}_{r, K}^{t:t+T_{pred}-1}$ of length $T_{pred}$ from the robot plan, which is named as Pred-Step Execution (PSE). The second type is we execute only one step of action $\bar{v}_{r, K}^{t}$, which is named as Single-Step Execution (SSE). We set prediction horizon $T_{pred}$ as 5 time steps (1.25 second). The failure probability $\alpha$ is set as 0.05 for all conformal prediction related methods. Based on union bound argument, we can bound the probability that a human trajectory stays within the conformal radius of the entire prediction horizon as follows
\begin{equation}
    \label{eq:coverage}
    Pr\left(
        ||\hat{x}_{h,i}^{t+\tau}-x_{h,i}^{t+\tau}||_2 
        \leq
        r_{cp}^{\tau},
        \forall
        \tau \in \{1, \dots, T_{pred}\}
    \right) 
    \geq 1 - \alpha T_{pred}.
\end{equation}
Thus, the lower bound of coverage rate is $1-0.05\times 5 = 0.75$.

The human simulator used in ICP is run separately from the experiment simulator. In the ICP simulator, we add noises to human goals at random steps during each episode to add randomness to the human behavior and diversify the collected data for calibration. We run the ICP simulator in multiple threads to parallelize the calibration data collection process. When the calibration size is less than 8, the number of threads is equal to the calibration size. Otherwise, we set the number of threads as 8. 

We set the weight parameters in the cost function of MPC $\omega_{g}$ as 1, $\omega_{v}$ as 5, and $\omega_{reg}$ as 0.5 across ICP, ACP and OffCP. To handle the cases when the constraints are too extreme and there are no feasible MPC solutions, we cache the most recent feasible plans for execution. We tune parameters of ACP-A and ACP-W, where the learning rate of ACP-A is set as 0.05, and the learning rate of ACP-W is set as 0.01. The time window used for online calibration dataset collection is set as 30 time steps (7.5 second) for both ACP-A and ACP-W. 

\subsection{Experiment Results}
We report quantitative performance of ICP with different configurations and baselines in 10-human crowd test cases for both PSE and SSE scheme in Table~\ref{table-quantitative}. 
% \vspace{-15pt}
\begin{table}[hbt!]
\caption{Performance of ICP with different configurations and baseline algorithms in 100 crowd navigation test cases of 10 humans. The subscript of ICP denotes the index of configuration. NI denotes number of iterations, CS denotes calibration size, ES denotes execution scheme, SR denotes success rate, ITR denotes intrusion time ratio, SD denotes social distance, PL denotes robot path length, NT denotes robot navigation time. CR denotes coverage rate, where 0.75 is the lower bound corresponding to the failure probability $\alpha$ as 0.05. The best performance for PSE and SSE configurations are independently highlighted.}\label{table-quantitative}
\begin{center}
\fontsize{7}{10}\selectfont
\begin{tabular}{c|ccc|cccccc}
    \hline
    Method & NI & CS & ES & SR$\uparrow$ & ITR$\downarrow$ & SD$\uparrow$ & PL$\downarrow$ & NT$\downarrow$ & CR$\uparrow$ (0.75) \\
    \hline
    \, ORCA\, & \,-\, & \,-\, & \,-\, & \,\textbf{0.99}\, & \,0.26$\pm$0.17\, & \,1.23$\pm$0.10\, & \,12.66$\pm$1.24\, & \,17.48$\pm$4.02\, & \,- \\
    \, OffCP\, & \,-\, & \,8\, & \,PSE\, & \,\textbf{0.99}\, & \,0.17$\pm$0.14\, & \,1.28$\pm$0.14\, & \,\textbf{12.57$\pm$0.62}\, & \,11.35$\pm$1.65\, & \,0.85$\pm$0.08 \\
    \, ACP-A\, & \,-\, & \,-\, & \,PSE\, & \,\textbf{0.99}\, & \,0.16$\pm$0.13\, & \,1.29$\pm$0.14\, & \,12.89$\pm$1.38\, & \,11.60$\pm$2.29\, & \,0.89$\pm$0.04 \\
    \, ACP-W\, & \,-\, & \,-\, & \,PSE\, & \,0.98\, & \,0.16$\pm$0.14\, & \,1.30$\pm$0.14\, & \,12.96$\pm$1.43\, & \,11.66$\pm$2.32\, & \,0.91$\pm$0.04 \\
    \, ICP$_1$\, & \,3\, & \,8\, & \,PSE\, & \,0.98\, & \,0.15$\pm$0.12\, & \,\textbf{1.32$\pm$0.16}\, & \,12.59$\pm$0.67\, & \,11.12$\pm$1.25\, & \,\textbf{0.93$\pm$0.05} \\
    \hline
    \, ICP$_2$\, & \,3\, & \,2\, & \,PSE\, & \,0.97\, & \,0.15$\pm$0.12\, & \,\textbf{1.32$\pm$0.15}\, & \,12.68$\pm$0.98\, & \,11.21$\pm$1.64\, & \,\textbf{0.93$\pm$0.05} \\
    \, ICP$_3$\, & \,3\, & \,4\, & \,PSE\, & \,0.96\, & \,\textbf{0.14$\pm$0.12}\, & \,\textbf{1.32$\pm$0.16}\, & \,12.61$\pm$0.61\, & \,11.13$\pm$1.32\, & \,\textbf{0.93$\pm$0.05} \\
    \, ICP$_4$\, & \,3\, & \,16\, & \,PSE\, & \,0.97\, & \,0.15$\pm$0.12\, & \,\textbf{1.32$\pm$0.16}\, & \,12.59$\pm$0.62\, & \,\textbf{11.09$\pm$1.28}\, & \,\textbf{0.93$\pm$0.05} \\
    \, ICP$_5$\, & \,3\, & \,32\, & \,PSE\, & \,0.96\, & \,0.15$\pm$0.12\, & \,\textbf{1.32$\pm$0.16}\, & \,12.61$\pm$0.78\, & \,11.11$\pm$1.35\, & \,\textbf{0.93$\pm$0.05} \\
    \, ICP$_6$\, & \,3\, & \,64\, & \,PSE\, & \,0.96\, & \,0.15$\pm$0.12\, & \,\textbf{1.32$\pm$0.16}\, & \,12.62$\pm$0.78\, & \,11.12$\pm$1.36\, & \,\textbf{0.93$\pm$0.05} \\
    \hline
    \, ICP$_7$\, & \,1\, & \,8\, & \,PSE\, & \,0.98\, & \,0.16$\pm$0.12\, & \,\textbf{1.32$\pm$0.16}\, & \,12.61$\pm$0.61\, & \,11.16$\pm$1.28\, & \,\textbf{0.93$\pm$0.05} \\
    \, ICP$_8$\, & \,10\, & \,8\, & \,PSE\, & \,\textbf{0.99}\, & \,0.16$\pm$0.12\, & \,1.30$\pm$0.15\, & \,12.58$\pm$0.55\, & \,11.10$\pm$1.20\, & \,\textbf{0.93$\pm$0.05} \\
    \hline
    \hline
    \, OffCP\, & \,-\, & \,8\, & \,SSE\, & \,\textbf{1.00}\, & \,0.16$\pm$0.13\, & \,1.29$\pm$0.14\, & \,\textbf{12.39$\pm$0.56}\, & \,\textbf{11.25$\pm$1.22}\, & \,0.82$\pm$0.08 \\
    \, ACP-A\, & \,-\, & \,-\, & \,SSE\, & \,0.98\, & \,0.15$\pm$0.12\, & \,1.27$\pm$0.14\, & \,12.58$\pm$0.79\, & \,11.58$\pm$1.45\, & \,0.91$\pm$0.03 \\
    \, ACP-W\, & \,-\, & \,-\, & \,SSE\, & \,0.96\, & \,\textbf{0.14$\pm$0.11}\, & \,1.30$\pm$0.13\, & \,13.46$\pm$2.59\, & \,12.95$\pm$4.11\, & \,\textbf{0.96$\pm$0.02} \\
    \, ICP$_9$\, & \,3\, & \,8\, & \,SSE\, & \,0.97\, & \,\textbf{0.14$\pm$0.12}\, & \,\textbf{1.31$\pm$0.15}\, & \,12.58$\pm$0.76\, & \,11.33$\pm$1.40\, & \,0.90$\pm$0.04 \\
    \hline
\end{tabular}
\end{center}
\end{table}
% \vspace{-15pt}

In PSE scheme, we see the coverage rate of ICP is consistently higher than the baselines to provide better safety guarantees, while ICP still achieves state-of-the-art performance in navigation and social-awareness. Fig.~\ref{fig-cr-nt-mean-std} shows that ICP reaches a sweet spot of the lowest navigation time and the highest coverage rate in PSE scheme regardless of the crowd density of the scenes. We find that OffCP has a consistently lower coverage rate than online methods including ICP and ACP. This matches our claim that OffCP calibrates human motion uncertainty of based on samples from a shifted distribution which ignores the interaction between the robot and the humans, and would lead to inaccurate uncertainty quantification.

\begin{figure}[hbt!]
\includegraphics[width=\textwidth]{figs/comparative-num-humans-CR-NT-v3.png}
\caption{Coverage rate (CR) and robot navigation time (NT) of algorithms with Pred-Step Execution scheme in crowd scenes of different number of humans. The error bars denote the standard deviation. The unit of robot navigation time is second. We use ICP$_1$ among all ICPs with PSE configurations for comparison.} \label{fig-cr-nt-mean-std}
\end{figure}

Fig.~\ref{fig-test-case} presents the comparison of performance between ICP$_1$ and ACP-W for each test case in both PSE and SSE schemes. We clearly see the effect of number of humans on the distribution of coverage rate over test cases in ACP-W in both left of Fig.~\ref{fig-cr-nt-mean-std} and the top row of Fig.~\ref{fig-test-case}, where there is a notable number of violations of coverage rate lower bound in 5-human test cases. This is due to the fact that ACP collects the calibration dataset on the fly, of which the size is insufficient and dependent on the number of humans. In contrast, the coverage rate of test cases run with ICP is both high and stable as the simulation provides abundant interaction-aware samples for uncertainty calibration even when the number of humans is low. We argue this is also the reason why ICP$_{2-6}$ whose calibration size spans from 2 to 64 have similar performance across all metrics. Running 2 simulation episodes turns out to be sufficient to calibrate human motion uncertainty when the robot needs to navigate through 10 humans.

\begin{figure}[hbt!]
\includegraphics[width=\textwidth]{figs/icp_vs_acp_per_test_case-v3.png}
\caption{Performance comparison between ICP and ACP-W for both Pred-Step Execution (PSE) Scheme and Single-Step Execution (SSE) Scheme. One black dot is for one test case in the Pred-Step Execution, where ICP and ACP-W share the same configurations on start and goal positions for the robot and the humans. One red triangle is for one test case in the Single-Step Execution. The X value of a black dot or a red triangle shows the performance of ICP, and the Y value shows the performance of ACP-W. Note we use ICP$_1$ for PSE comparison.} \label{fig-test-case}
\end{figure}

It is surprising that the top row of Fig.~\ref{fig-test-case} indicates the coverage rate of ACP-W is in the SSE scheme is better than in the PSE scheme, which is reverse to our expectation as SSE makes the lower bound of the coverage rate not hold anymore. Nevertheless, The coverage rate of ACP-W comes at the price of unstable navigation time performance in contrast to ICP, which is shown in the middle row of Fig.~\ref{fig-test-case}. Regarding the social metrics, the bottom row of Fig.~\ref{fig-test-case} shows that ACP-W and ICP tend to have more comparable intrusion time ratio per test case when the number of humans are lower (e.g., 5). We reason that that lower number of humans indicates simpler interaction patterns, which are less sensitive to different robot plans from ACP and ICP.

Fig.~\ref{fig-snapshot} demonstrates crowd navigation of OffCP, ACP-W, and ICP$_9$ in SSE scheme. We see that the conformal interval radius of ICP during the crowd-robot interaction ($t = 5$) is greater than before ($t = 2.5$) and after ($t= 7.5$), which illustrates that the human motion uncertainty is higher when the crowd-robot interaction is more involved. ACP-W exhibit similar trend by implicitly capturing the mutual influence with online calibration dataset collection. However, the higher coverage rate of ACP-W is at the price of excessive collision constraints caused by the large confidence interval radius, which leads to overly conservative and deviated robot motion. OffCP fails to capture the mutual influence and has fixed small confidence interval radius assuming no presence of the robot through the whole episode, and results in robot motion similar to treating the predictions as ground truth future positions, which exhibits less social awareness.

\begin{figure}[hbt!]
\includegraphics[width=\textwidth]{figs/crowd_nav_visualization-v2.png}
\caption{Snapshots of one crowd navigation test case in SSE scheme. We use ICP$_9$ for ICP visualization. The last column shows the snapshots whe n the robot reaches the goal. The bright yellow disk denotes the robot. The star denotes the robot goal. The orange circles with indices denote the humans with the predicted positions. The bright blue circles denote human radius bloated by the confidence interval radius. The red dots denote the history of the robot positions. The blue dots denote the generated plan to be executed by the robot. The dark yellow dots with indices denote the corresponding human's goal.} \label{fig-snapshot}
\end{figure}

We investigate the practicality of ICP in real world applications by checking the runtime and GPU memory usage. We find that ICP with appropriate configurations can be readily applied in real-time in either PSE (1.25 sec) or SSE (0.25 sec) scheme. The GPU memory usage of the algorithm is also manageable for a standard commercial GPU (e.g., GeForce RTX 2080 with total memory  8192 MiB).


\section{Limitations}\label{sec-limitations}
As demonstrated in Fig.~\ref{fig-snapshot}, higher coverage rates may cause excessively constrained conditions, which leads to infeasible solutions. When this occurs, we use a cached plan which may not remain optimal. If the cached plan is used beyond $T_{pred}$ steps, the probabilistic safety guarantee no longer exists, and the robot and the humans are susceptible to collision. To address this challenge, we are interested in exploring the integration of the adaptive failure probability idea from ACP into ICP, where the adaptation is dependent on the feasibility of optimization problem in MPC.

The performance comparison between ICP$_1$, ICP$_7$, and ICP$_8$ in Table~\ref{table-quantitative} indicates that having 1 iteration of ICP can already capture interaction between robot and humans well. We argue that this is because ORCA is used both in the human simulator of ICP algorithm and for generating human motion in test scenarios. We expect that the sim-to-real gap between the human simulator and the real world human behavior pattern would require more iterations for better performance, which is left for future work.