\section{Introduction}
Despite decades of development in robot motion planning algorithms, it is only recently that mobile robots have started navigating through crowds and serving in our daily lives because of the advancement on data-driven modeling of human motion~\cite{hart1968formal},\cite{karaman2011sampling}, \cite{bemporad2007robust},\cite{alahi2016social}, \cite{huang2022learning}. While these human models are getting more accurate, how to effectively use predicted human motion for robot motion planning remains an open research problem. A classical paradigm performs motion planning by treating the predictions as if they are ground truth future human positions~\cite{huang2023neural}, \cite{liu2023intention}. However, there always exists prediction error, so the planned robot motion does not have any safety guarantees in this paradigm. Recent efforts are focused on calibration of the prediction error with uncertainty quantification techniques like conformal prediction~\cite{vovk2005cp}. As a calibration trajectory dataset is required for conformal prediction, previous works usually suffer from distribution shift on human motion due to (1) offline human-only simulation data collection which overlooks the difference of human reactions to robot from to other humans~\cite{lindemann2023safe}, or (2) impractical amount of online human-robot interaction data required for achieving asymptotic safety guarantee~\cite{dixit2023adaptive}.
% uncertainty quantification of the human models by techniques like conformal prediction. which requires collection of human trajectory dataset for calibration~\cite{lindemann2023safe}, \cite{dixit2023adaptive}. 
% They collect human-only synthetic dataset and calibrate the prediction model to acquire confidence interval radius, which is used to add a data-driven probabilistic safety margin to incorporate human motion uncertainty into robot motion planning. However, these works ignore an important fact that there exists a significant distribution shift between human-only simulation and crowd navigation scenes. The humans behave differently in reaction to different robot motion patterns, so the uncertainty of human motion with respect to the prediction model dynamically changes along with the change of the robot plan. Reversely, the change of the human motion uncertainty should also inform the robot to move differently in order to preserve probabilistic safety.

We introduce Interaction-aware Conformal Prediction (ICP) to address the distribution shift issue by alternation between (1) robot motion planning based on the human motion uncertainty and (2) human motion uncertainty quantification by online human simulation conditioned on the robot motion plan.

In the initial step, ICP assumes predictions are ground truth and generates a robot motion plan (Fig.~\ref{fig-intro} b1, b2). Given the initial robot motion plan, ICP then starts iteration: (1) simulate multiple episodes of crowd motion by assuming the robot will execute the current plan to collect the calibration dataset dependent on the current plan (Fig.~\ref{fig-intro} c1); (2) perform conformal prediction to acquire the decision-dependent confidence interval radii (Fig.~\ref{fig-intro} c2); (3) plan the robot motion by using the current confidence interval radii as the decision-dependent probabilistic safety margin (Fig.~\ref{fig-intro} c3).

By explicitly capturing the mutual influence between the robot plan and the human motion uncertainty, ICP achieves a good tradeoff among navigation efficiency, social awareness, and uncertainty quantification in contrast to previous works in crowd navigation simulation experiments. We demonstrate that ICP generalizes well to crowd scenarios of different number of humans, and its fast runtime and small GPU memory usage show the readiness of real world applications.

% The paper contents are organized as follows. We review the related work on conformal prediction, crowd navigation, and model predictive control used for robot motion planning in Section~\ref{sec-related}. We provide an introduction to conformal prediction in generic context in Section~\ref{sec-intro-cp}. We formulate the crowd navigation problem, and expand the details on the Interaction-aware Conformal Prediction algorithm in Section~\ref{sec-method}. We present the simulation experiments and results in Section~\ref{sec-experiments}, and discuss the limitations in Section~\ref{sec-limitations}. Finally, we conclude this work and discuss future works in Section~\ref{sec-conclusions}.



% While robot motion planning algorithms have been developed for decades, it is not until recently we seeing mobile robots serve in our daily lives.
% To navigate in a crowded space with dynamic human neighbors, robots need to plan safe, socially-aware, and efficient motions. Rule-based methods such as Social Force~\textcolor{red}{[cite]} are collision-free with locally reaction design but can suffer from being stuck in local minimum. Learning-based methods crafted reward functions to balance the motion plan requirements and use reinforcement learning to train neural networks for crowd navigation policy~\textcolor{red}{[cite]}. While demonstrating reactive behavior and impressive efficiency, these end-to-end approaches do not have safety guarantees. 

% More recent efforts have focused on harnessing neural networks for human motion prediction and reasoning prediction uncertainty for safe robot motion planning~\textcolor{red}{[cite]}. Due to distribution-free property and weak assumption on data exchangeability, conformal prediction is applied in these works to quantify uncertainty of trajectory prediction on a collected trajectory calibration dataset~\textcolor{red}{[cite]}. The calibration dataset can be collected offline or online, but we observe limitations with both methods. The offline method simulates human-only scenarios to collect trajectories for calibration, which ignores the distribution shift on human behavior caused by the robot motion. The online method uses the history of human and robot trajectories to build the calibration dataset. ~\textcolor{red}{[Tradeoff: if only use most recent, then the dataset size is small. If not the most recent, the distribution shift exists.]}
\begin{figure*}[hbt!]
\centering
\includegraphics[width=0.8\textwidth]{figs/introduction_fig-v2.png}
\caption{Interaction-aware Conformal Prediction (ICP) iteratively quantifies uncertainty of human trajectory prediction by human motion simulation under the assumption that the robot would execute the latest plan, and plans robot motion with the conformal interval radii calibrated from the latest simulation dataset.} \label{fig-intro}
\end{figure*}
% \vspace{-50pt}
