\section{Related Work}\label{sec-related}
\subsection{Conformal Prediction}
Conformal prediction is a statistical tool designed to produce reliable and 
valid prediction intervals or sets in machine learning. 
First introduced in \cite{vovk2005cp}, it offers 
a rigorous framework to quantify the uncertainty 
of predictions without making assumptions about 
the underlying distribution or predictive model \cite{angelopoulos2022cp}.
Due to its model-agnostic nature, conformal 
prediction has gained increasing popularity in various communities ranging from healthcare \cite{vazquez2022conformal}, \cite{olsson2022estimating},
to finance \cite{pmlr-v128-wisniewski20a}, \cite{KATH2021777}.

The application of conformal prediction has
also found great success in robotics, including combination with reachability analysis~\cite{dietterich2022conformal},~\cite{Muthali2023Multi-agent}, adding safety guarantees to trajectory prediction~\cite{lindemann2023safe},~\cite{dixit2023adaptive},~\cite{Sun2023ConformalPF}, and integration with robust control to find control laws with probabilistic safety guarantees~\cite{zhang2024distributionfree}. Most relevant to our work is the study by 
\cite{lindemann2023safe},
which applies conformal prediction and model predictive control (MPC) to plan robot motion with safety guarantees. Note that they run simulation of only human agents to collect a synthetic trajectory calibration dataset and perform conformal prediction offline. Thus, they compute fixed conformal interval radius as safety clearance for MPC. However, human agents adjust their behavior according to the robot action during human-robot interaction. A new robot plan will alter the distribution of human motion and break the guarantees offered by these fixed conformal sets.

Adaptive Conformal Prediction (ACP) attempts to address this issue by collecting human and robot trajectories, updating calibration datasets and adjusting failure probability on the fly~\cite{dixit2023adaptive}. A practical limitation of ACP is its asymptotic safety guarantee, where the average 
safety rate over all time steps approaches 
the designated safety rate as time goes to infinity. This indicates that a long warm-up period of online human-robot interaction data collection is necessary for achieving the asymptotic safety guarantee, which does not meet the efficiency requirement of crowd navigation applications. In contrast, our ICP algorithm offers distribution-free safety guarantees with robot plan refinement and human motion conformal set re-computation by leveraging online human simulation conditioned on robot plans.


\subsection{Crowd Navigation}
Various methods have been developed to enhance robot navigation in crowded environments. Reaction-based methods such as Optimal Reciprocal Collision Avoidance (ORCA) \cite{van2011reciprocal} treat agents as velocity obstacles, whereas methods like Social Force \cite{helbing1995social}, DS-RNN \cite{liu2021decentralized}, and \cite{liu2023intention} leverage attractive and repulsive forces or interaction-based graphs to model interactions between agents.

While these works have made notable contributions, their frameworks suffer from undetermined uncertainty quantification and are prone to safety problems. Hence, \cite{lindemann2023safe} and \cite{Muthali2023Multi-agent} have made use of conformal prediction to endow crowd navigation with probabilistic safety guarantees, where conformal prediction empowers their frameworks to deal with unknown data distribution.

\cite{Muthali2023Multi-agent} uses a specific prediction model and takes advantage of quantile regression models to generate approximate confidence intervals on predicted actions. Their approach is followed by implementing RollingRC, a conformal prediction method, to adjust composed intervals. Owing to the desirability of having confidence sets in the spatial domain, \cite{Muthali2023Multi-agent} and \cite{lin2024verification} use HJ reachability method to form reachable tubes for each agent. \cite{Muthali2023Multi-agent} obtains optimal trajectory for the ego agent by treating each agent's final forward reachable tube as a time-growing obstacle and maximizing the Hamiltonian.
Unlike previous works, ICP captures the interactions by iteratively computing conformal prediction sets and considering the effect of planner outputs on agents trajectories. 

\subsection{Model Predictive Control}
Model predictive control (MPC) is a control technique based on the iterative solution of an optimization problem~\cite{bemporad2007robust}. By using the system model and the current state, MPC plans the optimal control sequence based on a cost function. Due to its ability to handle multi-variable systems and state/input constraints, MPC has received considerable attention and has been studied within diverse research areas and application domains~\cite{garcia1989model,qin2003survey}. In robotics, MPC has been used to plan motions for mobile robots~\cite{chen2021interactive,sivakumar2021learned}, manipulators~\cite{incremona2017mpc}, and drones~\cite{ji2022robust,kamel2017robust}.

Existing MPC-based methods for robot navigation in social environments are often composed of two steps, prediction and planning, where the future trajectories of the surrounding agents are first predicted and then the robot action is planned by solving an optimization problem~\cite{chen2021interactive}. Park \textit{et al.} propose the model predictive equilibrium point control (MPEPC) for wheelchair robot navigation, where the uncertainty of obstacle motions is predefined and fixed~\cite{park2012robot}. Kamel \textit{et al.} employ a model-based controller to navigate a micro aerial vehicle (MAV) while avoiding collisions with other MAVs, where a constant velocity model is used for obstacle trajectory prediction and the obstacles are inflated for safety based on the uncertainty estimated by an extended Kalman filter~\cite{kamel2017robust}. A similar idea of enlarging the safety distance between the robot and a human based on the covariance of the estimated state is adopted by Toit \textit{et al.}, where several predefined dynamics are also explored for human trajectory prediction~\cite{du2011robot}. Chen \textit{et al.} propose an intention-enhanced ORCA (iORCA) as an advanced pedestrian motion model, which can dynamically adjust the preferred velocity of a pedestrian~\cite{chen2021interactive}. The predicted human trajectories from iORCA are then incorporated into an MPC framework to realize safe navigation in dense crowds. However, these approaches fail to take into account the effects of robot actions on future human trajectories, and thus the distribution shift on human behaviors exists, which can potentially lead to safety violations during execution.