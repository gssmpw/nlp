

% \usepackage{times}
\usepackage[utf8]{inputenc} % allow utf-8 input
\usepackage[T1]{fontenc}    % use 8-bit T1 fonts
\usepackage{url}            % simple URL typesetting
\usepackage{booktabs}       % professional-quality tables
\usepackage{nicefrac}       % compact symbols for 1/2, etc.
\usepackage{microtype}      % microtypography
\usepackage{graphics,graphicx,color}

\usepackage{algorithm}
% \usepackage{algorithmic}
\usepackage{enumitem}

\usepackage{amsfonts}       % blackboard math symbols
\usepackage{amsmath}       % blackboard math symbols
\usepackage{amssymb}

% pseudo-code
\usepackage{algpseudocode}
\renewcommand{\algorithmicrequire}{\textbf{Input:}}
\renewcommand{\algorithmicensure}{\textbf{Output:}}

%%% useful macros
\newcommand{\Figref}[1]{Figure~\ref{#1}}  % beginning of sentence
\newcommand{\figref}[1]{Fig.~\ref{#1}}    % somewhere
\newcommand{\Tabref}[1]{Table~\ref{#1}}
\newcommand{\tabref}[1]{Table~\ref{#1}}
\newcommand{\Eqnref}[1]{Equation~\ref{#1}}
\newcommand{\eqnref}[1]{Eq.~\ref{#1}} % Eq. (1)
\newcommand{\eqnpref}[1]{(Eq.~\ref{#1})} % (Eq. 1)
\newcommand{\Secref}[1]{Sec.~\ref{#1}} % Section 1
\newcommand{\secref}[1]{Sec.~\ref{#1}} % Sec. 1
\newcommand{\suppref}[1]{Suppl.~\ref{#1}}
% Command for first sentence of each paragraph (it makes it bold)
\newtheorem{assumption}{Assumption}
\newcommand{\Assumpref}[1]{Assumption~\ref{#1}}  % beginning of sentence
\newcommand{\assumpref}[1]{Assump.~\ref{#1}}
\newcommand{\Algref}[1]{Algorithm~\ref{#1}}
\newcommand{\fs}[1]{{\bf #1}}

\usepackage{xspace}
% Add a period to the end of an abbreviation unless there's one
% already, then \xspace.
\makeatletter
\DeclareRobustCommand\onedot{\futurelet\@let@token\@onedot}
\def\@onedot{\ifx\@let@token.\else.\null\fi\xspace}
\makeatother

\newcommand{\eg}{e.g\onedot}
\newcommand{\ie}{i.e\onedot}
\newcommand{\cf}{cf\onedot}
\newcommand{\etc}{etc\onedot}
\newcommand{\wrt}{w.r.t\onedot}

% hline for normal tables (I would recommend to use booktab anyway)
\newcommand\Tstrut{\rule{0pt}{2.4ex}}
\newcommand\hlinespace{\hline\Tstrut}    % use for tables to get enough space after \hline

% cross referencing between files (main and appendix)
\usepackage{xr-hyper}
\makeatletter
\newcommand*{\addFileDependency}[1]{% argument=file name and extension
  \typeout{(#1)}
  \@addtofilelist{#1}
  \IfFileExists{#1}{}{\typeout{No file #1.}}
}
\makeatother
\newcommand*{\myexternaldocument}[1]{%
    \externaldocument{#1}%
    \addFileDependency{#1.tex}%
    \addFileDependency{#1.aux}%
  }

% color-code
\usepackage{xcolor}
% \definecolor{ourblue}{rgb}{0.368,0.507,0.71}
% \definecolor{ourorange}{rgb}{0.881,0.611,0.142}
% \definecolor{ourgreen}{rgb}{0.56,0.692,0.195}
% \definecolor{ourred}{rgb}{0.923,0.386,0.209}
% \definecolor{ourviolet}{rgb}{0.528,0.471,0.701}
% \definecolor{ourbrown}{rgb}{0.772,0.432,0.102}
% \definecolor{ourlightblue}{rgb}{0.364,0.619,0.782}
% \definecolor{ourdarkgreen}{rgb}{0.572,0.586,0.}

% % color-code 2
% \definecolor{ourcyan2}{rgb}{0.125,0.722,0.804}
% \definecolor{ourred2}{rgb}{0.863,0.184,0.047}
% \definecolor{ouryellow2}{cmyk}{0,0.16,1.0,0.07}
% \definecolor{ourviolet2}{cmyk}{0.55,0.56,0,0.47}
% \definecolor{ourorange2}{cmyk}{0,0.46,0.89,0.11}

\definecolor{ourorange}{HTML}{e19c24}
\definecolor{ourgreen}{HTML}{97b567}
\definecolor{ourred}{HTML}{ec6235}
\definecolor{ourblue}{HTML}{5e81b5}
\definecolor{ourgrey}{HTML}{919191}

\usepackage{subcaption}
