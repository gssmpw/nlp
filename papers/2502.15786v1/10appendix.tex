\section{Dataset Details}
\label{app:dataset}

\subsection{Details of each dataset in brain instruction tuning}
In this section, we give a brief description of each source of our brain instruction datasets as well as examples from them.

\setlength{\LTcapwidth}{\textwidth}

{\scriptsize 
\begin{longtable}{>{\raggedright\arraybackslash}m{0.2\textwidth}m{0.2\textwidth}m{0.5\textwidth}}
\caption{Dataset details and examples.}
\label{tab:dataset_details}\\

\toprule
\textbf{Dataset} & \textbf{Description} & \textbf{Example}\\
\midrule
\endfirsthead

% \multicolumn{3}{c}{\tablename\ \thetable{} -- Continued from previous page}\\
\toprule
\textbf{Dataset} & \textbf{Description} & \textbf{Example}\\
\midrule
\endhead

\midrule
\multicolumn{3}{r}{Continued on next page}\\
\endfoot

\bottomrule
\endlastfoot

\begin{minipage}[c]{\linewidth}
    \vspace{0.5em}
    Previous Caption
    \vspace{0.5em}
\end{minipage} & 
\begin{minipage}[c]{\linewidth}
    \vspace{0.5em}
    Generating a one-sentence caption of the image that the subject previously saw.
    \vspace{0.5em}
\end{minipage} &
\begin{minipage}[c]{\linewidth}
    \centering
    \begin{minipage}[c]{0.3\linewidth}
        \includegraphics[width=0.7\linewidth]{figures/table5_example/prev-coco.jpg}
    \end{minipage}%
    \hfill
    \begin{minipage}[c]{0.65\linewidth}
        A neat bedroom pairs modern chairs with a glass table.
    \end{minipage}
\end{minipage}\\
\midrule

\begin{minipage}[c]{\linewidth}
    \vspace{0.5em}
    \makecell[l]{COCO Caption\\\cite{chen2015microsoft}}
    \vspace{0.5em}
\end{minipage} & 
\begin{minipage}[c]{\linewidth}
    \vspace{0.5em}
    Generate a one-sentence caption of the image the subject currently sees.
    \vspace{0.5em}
\end{minipage} &
\begin{minipage}[c]{\linewidth}
    \centering
    \begin{minipage}[c]{0.3\linewidth}
        \includegraphics[width=0.7\linewidth]{figures/table5_example/coco-caption.jpg}
    \end{minipage}%
    \hfill
    \begin{minipage}[c]{0.65\linewidth}
        The pedestrian is walking down the side of the highway by the bus.
    \end{minipage}
\end{minipage}\\
\midrule

\begin{minipage}[c]{\linewidth}
    \vspace{0.5em}
Image Paragraph Captioning\\ \cite{krause2017hierarchical}
\vspace{0.5em}
\end{minipage} & 
\begin{minipage}[c]{\linewidth}
    \vspace{0.5em}
 Generate a one-paragraph caption of the image the subject currently sees    \vspace{0.5em}
\end{minipage} &
\begin{minipage}[c]{\linewidth}
    \centering
    \begin{minipage}[c]{0.3\linewidth}
        \includegraphics[width=0.7\linewidth]{figures/table5_example/paragraph-caption.jpg}
    \end{minipage}%
    \hfill
    \begin{minipage}[c]{0.65\linewidth}
        An elephant with a harness, and a seat on his back is seen in a dirt field on a sunny day. The seat on the back of the elephant is brown and is tied with ropes. The sun is shining through the trees onto the ground below. Behind the elephant there is a cinder block wall with grass growing in front of the wall. There are trees behind the wall.    \end{minipage}
\end{minipage}\\
\midrule


\begin{minipage}[c]{\linewidth}
    \vspace{0.5em}
COCO QA\\\cite{ren2015exploring}
\vspace{0.5em}
\end{minipage} & 
\begin{minipage}[c]{\linewidth}
    \vspace{0.5em}
Answer questions according to the image.
\vspace{0.5em}
\end{minipage} &
\begin{minipage}[c]{\linewidth}
    \centering
    \begin{minipage}[c]{0.3\linewidth}
        \includegraphics[width=0.7\linewidth]{figures/table5_example/coco-qa.jpg}
    \end{minipage}%
    \hfill
    \begin{minipage}[c]{0.65\linewidth}
        [Q]: what eats leaves from a basket in an enclosure? [A]: Giraffe.
    \end{minipage}
\end{minipage}\\
\midrule


\begin{minipage}[c]{\linewidth}
    \vspace{0.5em}
Visual Genome QA\\\cite{krishna2017visual}
\vspace{0.5em}
\end{minipage} & 
\begin{minipage}[c]{\linewidth}
    \vspace{0.5em}
Answer image-based questions which require richer semantic understanding of the image than COCO-QA.
\vspace{0.5em}
\end{minipage} &
\begin{minipage}[c]{\linewidth}
    \centering
    \begin{minipage}[c]{0.3\linewidth}
        \includegraphics[width=0.7\linewidth]{figures/table5_example/visual-genome.jpg}
    \end{minipage}%
    \hfill
    \begin{minipage}[c]{0.65\linewidth}
        [Q]: Where was the photo taken? [A]: In an office.
    \end{minipage}
\end{minipage}\\
\midrule


\begin{minipage}[c]{\linewidth}
    \vspace{0.5em}
VQAv2\\\cite{goyal2017making}
\vspace{0.5em}
\end{minipage} & 
\begin{minipage}[c]{\linewidth}
    \vspace{0.5em}
 Answer image-based questions with better equality and diversity than COCO-QA.
 
 \vspace{0.5em}
\end{minipage} &
\begin{minipage}[c]{\linewidth}
    \centering
    \begin{minipage}[c]{0.3\linewidth}
        \includegraphics[width=0.7\linewidth]{figures/table5_example/vqa-v2.jpg}
    \end{minipage}%
    \hfill
    \begin{minipage}[c]{0.65\linewidth}
        [Q]: What are the two white letters? [A]: hu
    \end{minipage}
\end{minipage}\\
\midrule

\begin{minipage}[c]{\linewidth}
    \vspace{0.5em}
OK-VQA\\\cite{marino2019ok}
\vspace{0.5em}
\end{minipage} & 
\begin{minipage}[c]{\linewidth}
    \vspace{0.5em}
 Answer image-based questions that requires external knowledge beyond the image itself.
\vspace{0.5em}
\end{minipage} &
\begin{minipage}[c]{\linewidth}
    \centering
    \begin{minipage}[c]{0.3\linewidth}
        \includegraphics[width=0.7\linewidth]{figures/table5_example/ok-vqa.jpg}
    \end{minipage}%
    \hfill
    \begin{minipage}[c]{0.65\linewidth}
        [Q]: What part of the body do you wear the rightmost objects on? [A]: Neck.
    \end{minipage}
\end{minipage}\\
\midrule

\begin{minipage}[c]{\linewidth}
    \vspace{0.5em}
ST-VQA\\\cite{biten2019scene}
\vspace{0.5em}
\end{minipage} & 
\begin{minipage}[c]{\linewidth}
    \vspace{0.5em}
 Answer questions of high-level semantic information present in images as textual cue
 \vspace{0.5em}
\end{minipage} &
\begin{minipage}[c]{\linewidth}
    \centering
    \begin{minipage}[c]{0.3\linewidth}
        \includegraphics[width=0.7\linewidth]{figures/table5_example/st-vqa.jpg}
    \end{minipage}%
    \hfill
    \begin{minipage}[c]{0.65\linewidth}
        [Q]: What is written on the front of the shirt? [A]: Marine.
    \end{minipage}
\end{minipage}\\
\midrule

\begin{minipage}[c]{\linewidth}
    \vspace{0.5em}
TallyQA\\\cite{acharya2019tallyqa}
\vspace{0.5em}
\end{minipage} & 
\begin{minipage}[c]{\linewidth}
    \vspace{0.5em}
Count objects present in images.
\vspace{0.5em}
\end{minipage} &
\begin{minipage}[c]{\linewidth}
    \centering
    \begin{minipage}[c]{0.3\linewidth}
        \includegraphics[width=0.7\linewidth]{figures/table5_example/tally-qa.jpg}
    \end{minipage}%
    \hfill
    \begin{minipage}[c]{0.65\linewidth}
        [Q]: How many bats on the wall? [A]: 2
    \end{minipage}
\end{minipage}\\
\midrule

\begin{minipage}[c]{\linewidth}
    \vspace{0.5em}
VQA-E\\\cite{li2018vqa}
\vspace{0.5em}
\end{minipage} & 
\begin{minipage}[c]{\linewidth}
    \vspace{0.5em}
Answer questions and generate corresponding explanations for an image-based question.
\vspace{0.5em}
\end{minipage} &
\begin{minipage}[c]{\linewidth}
    \centering
    \begin{minipage}[c]{0.3\linewidth}
        \includegraphics[width=0.7\linewidth]{figures/table5_example/vqa-e.jpg}
    \end{minipage}%
    \hfill
    \begin{minipage}[c]{0.65\linewidth}
        [Q]: Are the people going for a walk in the forest? [A]: Yes. Here is the explanation: A picture of the land, trees, and people passing by as they ride in a vehicle.
    \end{minipage}
\end{minipage}\\
\midrule

\begin{minipage}[c]{\linewidth}
    \vspace{0.5em}
A-OKVQA\\\cite{schwenk2022okvqa}
\vspace{0.5em}
\end{minipage} & 
\begin{minipage}[c]{\linewidth}
    \vspace{0.5em}
Answer multiple-choice questions.
\vspace{0.5em}
\end{minipage} &
\begin{minipage}[c]{\linewidth}
    \centering
    \begin{minipage}[c]{0.3\linewidth}
        \includegraphics[width=0.7\linewidth]{figures/table5_example/a-okvqa.jpg}
    \end{minipage}%
    \hfill
    \begin{minipage}[c]{0.65\linewidth}
        [Q]: What season is up next? Multiple Choices: A. autumn B. spring C. summer D. winter [A]: B
    \end{minipage}
\end{minipage}\\
\midrule

\begin{minipage}[c]{\linewidth}
    \vspace{0.5em}
FSVQA\\\cite{shin2016color}
\vspace{0.5em}
\end{minipage} & 
\begin{minipage}[c]{\linewidth}
    \vspace{0.5em}
Answer the questions in full sentences.
\vspace{0.5em}
\end{minipage} &
\begin{minipage}[c]{\linewidth}
    \centering
    \begin{minipage}[c]{0.3\linewidth}
        \includegraphics[width=0.7\linewidth]{figures/table5_example/fsvqa.jpg}
    \end{minipage}%
    \hfill
    \begin{minipage}[c]{0.65\linewidth}
        [Q]: Is this an area that is more populated with pedestrians than cars? [A]: Yes, this is an area that is more populated with pedestrians than cars.
    \end{minipage}
\end{minipage}\\
\midrule

\begin{minipage}[c]{\linewidth}
    \vspace{0.5em}
VisDial\\\cite{murahari2019improving}

\vspace{0.5em}
\end{minipage} & 
\begin{minipage}[c]{\linewidth}
    \vspace{0.5em}
Generate answers for image-based questions in a multi-turn dialogue.     \vspace{0.5em}
\end{minipage} &
\begin{minipage}[c]{\linewidth}
    \centering
    \begin{minipage}[c]{0.3\linewidth}
        \includegraphics[width=0.7\linewidth]{figures/table5_example/visdial.jpg}
    \end{minipage}%
    \hfill
    \begin{minipage}[c]{0.65\linewidth}
        [Q1]: Is the photo in color? [A1]: No. [Q2]: Is the man wearing glasses [A2]: Can't tell. [Q3]: How many horses are there? [A3]: 2 
    \end{minipage}
\end{minipage}\\
\midrule

\begin{minipage}[c]{\linewidth}
    \vspace{0.5em}
LLava Instruction 150K\\\cite{liu2023visual}
\vspace{0.5em}
\end{minipage} & 
\begin{minipage}[c]{\linewidth}
    \vspace{0.5em}
Generate answers for object-level and scene-level answering or reasoning questions for single or multi-round conversations. 
\vspace{0.5em}
\end{minipage} &
\begin{minipage}[c]{\linewidth}
    \centering
    \begin{minipage}[c]{0.3\linewidth}
        \includegraphics[width=0.7\linewidth]{figures/table5_example/llava.jpg}
    \end{minipage}%
    \hfill
    \begin{minipage}[c]{0.65\linewidth}
        [Q1]: What is the main activity of the person in the fMRI? [A1]: The main activity of the person in the fMRI is brushing her teeth. [Q2]: What should be considered when maintaining oral hygiene in a public setting? [A2]: When maintaining oral hygiene in a public setting (as depicted in the fMRI by a woman brushing her teeth with a green toothbrush while carrying a shoulder bag), factors such as cleanliness, privacy, time management, and personal hygiene supplies should be taken into account...
    \end{minipage}
\end{minipage}\\
\midrule

\begin{minipage}[c]{\linewidth}
    \vspace{0.5em}
LVIS Instruct4V\\\cite{wang2023see}

\vspace{0.5em}
\end{minipage} & 
\begin{minipage}[c]{\linewidth}
    \vspace{0.8em}
 Similar to LLava Instruct4V but is finer-grained and of higher quality.
 \vspace{0.5em}
\end{minipage} &
\begin{minipage}[c]{\linewidth}
    \centering
    \begin{minipage}[c]{0.3\linewidth}
        \includegraphics[width=0.7\linewidth]{figures/table5_example/lvis.jpg}
    \end{minipage}%
    \hfill
    \begin{minipage}[c]{0.65\linewidth}
        [Q1]: What is the primary object in the fMRI? [A1]: The primary object in the fMRI is a bird. [Q2]: Can you describe the color and appearance of the bird? [A2]: The bird has a deep brown body with a darker head. Its eyes are distinctively surrounded by a yellowish hue. The bird also possesses a blue-tinted beak. [Q3]: How many birds are in the picture?' [A3]: There is only one bird in the picture...
    \end{minipage}
\end{minipage}\\
\midrule

\begin{minipage}[c]{\linewidth}
    \vspace{0.5em}
TDIUC (only used in downstream)\\\cite{kafle2017analysis}
\vspace{0.5em}
\end{minipage} & 
\begin{minipage}[c]{\linewidth}
    \vspace{0.5em}
Generate answers for image-based questions from 12 different task directed question types such as object presence and utility/affordance. 
\vspace{0.5em}
\end{minipage} &
\begin{minipage}[c]{\linewidth}
    \centering
    \begin{minipage}[c]{0.3\linewidth}
        \includegraphics[width=0.7\linewidth]{figures/tdiuc.jpg}
    \end{minipage}%
    \hfill
    \begin{minipage}[c]{0.65\linewidth}
        [Q]: What object can be thrown? [A]: Baseball
    \end{minipage}
\end{minipage}\\


\end{longtable}
}  
% \begin{table}[H]
%     \caption{Dataset details and examples.}
%     \label{tab:dataset_details}
%     \vskip 0.1in
%     \centering
%     \tiny
%     \begin{tabularx}{\linewidth}{>{\centering\arraybackslash}m{0.15\linewidth}m{0.2\linewidth}X}
% % \begin{longtable}
%     \toprule
%     Dataset & Description&Example\\
%     \midrule
%     Previous Caption & Generating a one-sentence caption of the image that the subject previously saw. &
%     \begin{minipage}{\linewidth}
%         \centering
%         \begin{minipage}{0.15\linewidth}
%         \includegraphics[width=\linewidth]{figures/table5_example/prev-coco.jpg}
%         \end{minipage}%
%         \hfill
%         \begin{minipage}{0.85\linewidth}
%         A neat bedroom pairs modern chairs with a glass table.
%         \end{minipage}
%         \end{minipage}\\
%     \midrule
%      \makecell{COCO Caption\\\cite{chen2015microsoft}} & Generate a one-sentence caption of the image the subject currently sees.&
%      \begin{minipage}{\linewidth}
%         \centering
%         \begin{minipage}{0.15\linewidth}
%         \includegraphics[width=\linewidth]{figures/table5_example/coco-caption.jpg}
%         \end{minipage}%
%         \hfill
%         \begin{minipage}{0.85\linewidth}
%         The pedestrian is walking down the side of the highway by the bus.
%         \end{minipage}
%         \end{minipage}\\
%      \midrule
%      \makecell{Image Paragraph Captioning\\\cite{krause2017hierarchical}} & Generate a one-paragraph caption of the image the subject currently sees.& 
%      \begin{minipage}{\linewidth}
%         \centering
%         \begin{minipage}{0.15\linewidth}
%         \includegraphics[width=\linewidth]{figures/table5_example/paragraph-caption.jpg}
%         \end{minipage}%
%         \hfill
%         \begin{minipage}{0.85\linewidth}
%         An elephant with a harness, and a seat on his back is seen in a dirt field on a sunny day. The seat on the back of the elephant is brown and is tied with ropes. The sun is shining through the trees onto the ground below. Behind the elephant there is a cinder block wall with grass growing in front of the wall. There are trees behind the wall.
%         \end{minipage}
%         \end{minipage}\\
%      \midrule
%      \makecell{COCO QA\\\cite{ren2015exploring}} & Answer questions according to the image. &
%      \begin{minipage}{\linewidth}
%         \centering
%         \begin{minipage}{0.15\linewidth}
%         \includegraphics[width=\linewidth]{figures/table5_example/coco-qa.jpg}
%         \end{minipage}%
%         \hfill
%         \begin{minipage}{0.85\linewidth}
%         [Q]: what eats leaves from a basket in an enclosure? [A]: Giraffe.
%         \end{minipage}
%         \end{minipage}\\
%      \midrule
%      \makecell{Visual Genome QA\\\cite{krishna2017visual}} & Answer image-based questions which require richer semantic understanding of the image than COCO-QA. &
%      \begin{minipage}{\linewidth}
%         \centering
%         \begin{minipage}{0.15\linewidth}
%         \includegraphics[width=\linewidth]{figures/table5_example/visual-genome.jpg}
%         \end{minipage}%
%         \hfill
%         \begin{minipage}{0.85\linewidth}
%         [Q]: Where was the photo taken? [A]: In an office.
%         \end{minipage}
%         \end{minipage}\\
%      \midrule
%     \makecell{VQAv2\\\cite{goyal2017making}} & Answer image-based questions with better equality and diversity than COCO-QA. &\begin{minipage}{\linewidth}
%         \centering
%         \begin{minipage}{0.15\linewidth}
%         \includegraphics[width=\linewidth]{figures/table5_example/vqa-v2.jpg}
%         \end{minipage}%
%         \hfill
%         \begin{minipage}{0.85\linewidth}
%         [Q]: What are the two white letters? [A]: hu
%         \end{minipage}
%         \end{minipage}\\
%      \midrule
%      \makecell{OK-VQA\\\cite{marino2019ok}} & Answer image-based questions that requires external knowledge beyond the image itself. &\begin{minipage}{\linewidth}
%         \centering
%         \begin{minipage}{0.15\linewidth}
%         \includegraphics[width=\linewidth]{figures/table5_example/ok-vqa.jpg}
%         \end{minipage}%
%         \hfill
%         \begin{minipage}{0.85\linewidth}
%         [Q]: What part of the body do you wear the rightmost objects on? [A]: Neck.
%         \end{minipage}
%         \end{minipage}\\
%      \midrule
%      \makecell{ST-VQA\\\cite{biten2019scene}} & Answer questions of high-level semantic information present in images as textual cue  &
%      \begin{minipage}{\linewidth}
%         \centering
%         \begin{minipage}{0.15\linewidth}
%         \includegraphics[width=\linewidth]{figures/table5_example/st-vqa.jpg}
%         \end{minipage}%
%         \hfill
%         \begin{minipage}{0.85\linewidth}
%         [Q]: What is written on the front of the shirt? [A]: Marine.
%         \end{minipage}
%         \end{minipage}\\
%      \midrule
%      \makecell{TallyQA\\\cite{acharya2019tallyqa}} & Count objects present in images. &
%      \begin{minipage}{\linewidth}
%         \centering
%         \begin{minipage}{0.15\linewidth}
%         \includegraphics[width=\linewidth]{figures/table5_example/tally-qa.jpg}
%         \end{minipage}%
%         \hfill
%         \begin{minipage}{0.85\linewidth}
%         [Q]: How many bats on the wall? [A]: 2
%         \end{minipage}
%         \end{minipage}\\
%      \midrule
%      \makecell{VQA-E\\\cite{li2018vqa}} & Answer questions and generate corresponding explanations for an image-based question. &
%      \begin{minipage}{\linewidth}
%         \centering
%         \begin{minipage}{0.15\linewidth}
%         \includegraphics[width=\linewidth]{figures/table5_example/vqa-e.jpg}
%         \end{minipage}%
%         \hfill
%         \begin{minipage}{0.85\linewidth}
%         [Q]: Are the people going for a walk in the forest? [A]: Yes. Here is the explanation: A picture of the land, trees, and people passing by as they ride in a vehicle.
%         \end{minipage}
%         \end{minipage}\\
%      % \midrule 
%      % VSR & Generate a boolean answer for an image-based statement. &\raisebox{-0.5\height}{\includegraphics[width=2cm]{figures/table5_example/vsr.jpg}} \raisebox{0cm}{\parbox[c]{10cm}{[Statement]: The car is beneath the cat. [A]: No.}}\\
%      \midrule
%      \makecell{A-OKVQA\\\cite{schwenk2022okvqa}} & Answer multiple-choice questions. &
%      \begin{minipage}{\linewidth}
%         \centering
%         \begin{minipage}{0.15\linewidth}
%         \includegraphics[width=\linewidth]{figures/table5_example/a-okvqa.jpg}
%         \end{minipage}%
%         \hfill
%         \begin{minipage}{0.85\linewidth}
%         [Q]: What season is up next? Multiple Choices: A. autumn B. spring C. summer D. winter [A]: B
%         \end{minipage}
%         \end{minipage}\\
%      \midrule
%      \makecell{FSVQA\\\cite{shin2016color}} & Answer the questions in full sentences.
%      &\begin{minipage}{\linewidth}
%         \centering
%         \begin{minipage}{0.15\linewidth}
%         \includegraphics[width=\linewidth]{figures/table5_example/fsvqa.jpg}
%         \end{minipage}%
%         \hfill
%         \begin{minipage}{0.85\linewidth}
%         [Q]: Is this an area that is more populated with pedestrians than cars? [A]: Yes, this is an area that is more populated with pedestrians than cars.
%         \end{minipage}
%         \end{minipage}\\
%      \midrule
%      \makecell{VisDial\\\cite{murahari2019improving}} &Generate answers for image-based questions in a multi-turn dialogue. &
%      \begin{minipage}{\linewidth}
%         \centering
%         \begin{minipage}{0.15\linewidth}
%         \includegraphics[width=\linewidth]{figures/table5_example/visdial.jpg}
%         \end{minipage}%
%         \hfill
%         \begin{minipage}{0.85\linewidth}
%         [Q1]: Is the photo in color? [A1]: No. [Q2]: Is the man wearing glasses [A2]: Can't tell. [Q3]: How many horses are there? [A3]: 2 
%         \end{minipage}
%         \end{minipage}\\
%      \midrule
%     \makecell{LLava Instruction 150K\\\cite{liu2023visual}} &Generate answers for object-level and scene-level answering or reasoning questions for single or multi-round conversations. &
%     \begin{minipage}{\linewidth}
%         \centering
%         \begin{minipage}{0.15\linewidth}
%         \includegraphics[width=\linewidth]{figures/table5_example/llava.jpg}
%         \end{minipage}%
%         \hfill
%         \begin{minipage}{0.85\linewidth}
%         [Q1]: What is the main activity of the person in the fMRI? [A1]: The main activity of the person in the fMRI is brushing her teeth. [Q2]: What should be considered when maintaining oral hygiene in a public setting? [A2]: When maintaining oral hygiene in a public setting (as depicted in the fMRI by a woman brushing her teeth with a green toothbrush while carrying a shoulder bag), factors such as cleanliness, privacy, time management, and personal hygiene supplies should be taken into account...
%         \end{minipage}
%         \end{minipage}\\
%     \midrule
%     \makecell{LVIS Instruct4V\\\cite{wang2023see}} & Perform instance segmentation for LVIS data and generate answers in a multi-round conversation with image-based questions for Instruct4V data&
%     \begin{minipage}{\linewidth}
%         \centering
%         \begin{minipage}{0.15\linewidth}
%         \includegraphics[width=\linewidth]{figures/table5_example/lvis.jpg}
%         \end{minipage}%
%         \hfill
%         \begin{minipage}{0.85\linewidth}
%         [Q1]: What is the primary object in the fMRI? [A1]: The primary object in the fMRI is a bird. [Q2]: Can you describe the color and appearance of the bird? [A2]: The bird has a deep brown body with a darker head. Its eyes are distinctively surrounded by a yellowish hue. The bird also possesses a blue-tinted beak. [Q3]: How many birds are in the picture?' [A3]: There is only one bird in the picture...
%         \end{minipage}
%         \end{minipage}\\
%     \midrule
%     \makecell{TDIUC\\\cite{kafle2017analysis}} & Generate answers for image-based questions from 12 different task directed question types such as object presence and utility/affordance. &
%     \begin{minipage}{\linewidth}
%         \centering
%         \begin{minipage}{0.15\linewidth}
%         \includegraphics[width=\linewidth]{figures/table5_example/tdiuc.jpg}
%         \end{minipage}%
%         \hfill
%         \begin{minipage}{0.85\linewidth}
%         [Q]: Are the objects attached to people's feet buoyant? [A]: No.
%         \end{minipage}
%         \end{minipage}\\
%      \bottomrule
%     \end{tabularx}
% \end{table}
% \end{longtable}

\subsection{Dataset statistics}
Table~\ref{tab:num_voxels} summarizes the statistics of each subject in the natural scene dataset \cite{allen2022massive}. 
\begin{table}[htbp]
    \caption{Number of voxels for each subject.}
    \label{tab:num_voxels}
    \vskip 0.1in
    \centering
    \begin{tabular}{c|cccccccc}
    \toprule
        subject & 1 & 2 & 3 & 4 & 5 & 6 & 7 & 8 \\
        \midrule
        \#(input voxels) & $15724$ & $14278$ & $15226$ & $13153$ & $13039$ & $17907$ & $126682$ & $14386$ \\
        \#(samples) & $30000$ & $30000$ & $24000$ & $22500$ & $30000$ & $24000$ & $30000$ & $22500$ \\
        \bottomrule
    \end{tabular}
\end{table}

% Table~\ref{tab:dataset_splits} summarized the number of conversations derived for each split.

% \begin{table}[H]
% \scriptsize
%     \caption{Number of conversations for each split. Note that this does not correspond to the exact number of samples since a conversation can be linked to multiple fMRI recordings. }
%     \label{tab:dataset_splits}
%     \vskip 0.1in
%     \centering
%     \begin{tabular}{cccc}
%     \toprule
%          Dataset & Train & Validation & Test \\
%     \midrule
%          COCO Caption & $40520$ & $4502$ & $5003$ \\
%          Image Paragraph Captioning & $654$ & $74$ & $79$ \\
%          COCO-QA & $8003$ & $930$ & $1050$ \\
%          Visual Genome & $45569$ & $4601$ & $5894$ \\
%          VQA v2 & $433600$ & $45790$ & $55050$ \\
%          OK-VQA & $9110$ & $1100$ & $1170$ \\
%          ST-VQA & $1983$ & $207$ & $288$ \\
%          Tally-QA & $15143$ & $1589$ & $1945$ \\
%          VQA-E & $18080$ & $1922$ & $2323$ \\
%          A-OKVQA & $1143$ & $133$ & $113$ \\
%          FSVQA & $24300$ & $2700$ & $3000$ \\
%          VisDial & $8100$ & $900$ & $1000$ \\
%          LLaVA & $10193$ & $1110$ & $1908$ \\
%          LVIS & $14574$ & $1618$ & $1804$ \\
%     \bottomrule
%     \end{tabular}
% \end{table}



\subsection{Instructions}
Table~\ref{tab:ift} lists instructions for each dataset (i.e. $X_\text{inst}$). Some instructions are inspired by \cite{han2024coco}. For question-answering tasks, the instruction is appended after the question in a new line.

\begin{table}[htbp]
    \caption{Instruction template and statistics of the BIT dataset. "$-$" indicates that the instruction is embedded within the conversation and does not require an additional one. Note that not all conversations are associated with fMRI recordings as only a subset of MSCOCO images were used as stimuli in the study. Consequently, the number of usable conversations in practice will be lower.}
    \label{tab:ift}
    \vskip 0.1in
    \centering
    \resizebox{\linewidth}{!}{
    \begin{tabular}{ccp{30em}}
    \toprule
    Dataset & \#(conversations) &Instruction  \\
    \midrule
    system prompt & / & You are a helpful agent that decodes the brain activity of a person looking at an image. \\
    Previous Caption & $149,875$ &Please describe the image the subject saw previously. \\
     COCO Caption & $616,767$ &Please describe the fMRI as simply as possible.  \\
     Image Paragraph Captioning & $9,598$ &Describe the fMRI in one paragraph.  \\
     COCO QA & $117,684$ & Answer the question with a short phrase. \\
     Visual Genome QA & $676,116$ & Answer the question with a short phrase. \\
     VQAv2 & $6,581,110$ & Answer the question with a short phrase.\\
     OK-VQA & $140,550$ & Answer the question with a short phrase. \\
     ST-VQA & $29,751$ &Answer the question with a short phrase. \\
     TallyQA & $238,056$ &Answer the question with a number.  \\
     VQA-E & $2,697,860$ &Answer with a short phrase and provide explanation for your answer. \\
     A-OKVQA & $18,201$ &Answer with the option's letter from the given choices and provide explanation for your choice. \\
     FSVQA & $369,861$ & Answer the question in a full sentence. \\
     VisDial & $125,351$ & $-$\\ 
     Llava Instruct 150K & $157,712$ & $-$ \\
     LVIS Instruct4V & $222,711$ & $-$\\
     \bottomrule
    \end{tabular}
    }
\end{table}


\section{Implementation details} 
\label{app:impl}
We choose Vicuna-7b \cite{zheng2023judging} as our backbone LLM. During the brain instruction tuning stage, We use AdamW as the optimizer, with the learning rate of $1\times 10^{-3}$, weight decay of $0.01$ and $\beta_1=0.9$, $\beta_2=0.999$. We do not use a learning rate scheduler. We set the batch size to $64$. The instruction tuning is conducted on a machine with $8 \times$ L40S GPUs. And each downstream fine-tuning is conducted on a single L40S GPU with a $1\times 10^{-4}$ learning rate and $48$ batch size. For generations, we have adopted the greedy decoding strategy.

The hyperparameters of the model architecture can be found in Table~\ref{tab:hyper}.

\begin{table}[H]
    \caption{Hyperparameter choices.}
    \label{tab:hyper}
    \vskip 0.1in
    \centering
    \begin{tabular}{cccc}
    \toprule
        Notation & meaning & value \\
    \midrule
        $L$ & number of fMRI tokens & $128$ \\
        / & number of MLP layers in the fMRI encoder & $4$\\
        $N_q$ & number of queries & $1024$\\
        $d$ & dimension of the query embeddings & $128$ \\
        
    \bottomrule
    \end{tabular}
\end{table}

\section{Computation Complexity}
\label{app:complexity}
In the neuroscience-informed attention, the complexity of the dot product between queries and keys is $O(dNN_q)$. The complexity of the aggregation of values is $O(NN_q)$, which is neglectable. The MLP maps the hidden representation of dimension $N_q$ to $L \times d$, therefore its complexity is $dLN_q$. Therefore, the complexity of the fMRI encoder is $O(dNN_q + dLN_q) = O(dN_q(L + N)) = O(dN_qN)$ given that $L \ll N$.

\section{Qualitative Analysis}
\label{app:qual}

\begin{figure}
    \centering
    \includegraphics[width=\linewidth]{figures/use_case1.pdf}
    \caption{Qualitative Analysis of COCO Captioning. 
    %\qwk{could you separate different ground truths instead of putting them in the sameline}
    }
    \label{fig:case1}
\end{figure}


\begin{figure}
    \centering
        %\hspace{6cm}  
    \includegraphics[width=\linewidth]{figures/use_case2_3.pdf}
    \caption{Qualitative Analysis of COCO QA (left column) and OK-VQA (right column).}
    \label{fig:case2_3}
\end{figure}


% \begin{figure}
%     \centering
%     \begin{tabular}{c}
%         % First image
%         \begin{subfigure}[b]{0.8\textwidth}
%             \centering
%             \includegraphics[width=0.7\linewidth]{figures/use_case/use_case1_1.jpg}
%             \caption*{
%             \textbf{Ground Truth Caption:} A white jet with blue and orange detailing sits on the runway. A Icelandair Airliner is preparing for take off. An airplane sitting at the end of a runway preparing for takeoff. \\
%             \textbf{Ours:} A large airplane sitting on top of an airport runway. \\
%             \textbf{MindBridge:} A large passenger jet flying in the sky. \\
%             \textbf{UniBrain:} A large airplane is flying over a city.
%             }
%         \end{subfigure} \\

%         % Second image
%         \begin{subfigure}[b]{0.8\textwidth}
%             \centering
%             \includegraphics[width=0.7\linewidth]{figures/use_case/use_case1_1.jpg}
%             \caption*{
%             \textbf{Ground Truth Caption:} Outdoor clock with metal decoration against a blue sky. Closeup of a clock with a sign that reads "Uncle Bills." \\
%             \textbf{Ours:} A clock tower with a clock on it. \\
%             \textbf{MindBridge:} A sign that says "STOP" on a pole. \\
%             \textbf{UniBrain:} A red and white airplane is parked on the tarmac.
%             }
%         \end{subfigure} \\

%         % Third image
%         \begin{subfigure}[b]{0.8\textwidth}
%             \centering
%             \includegraphics[width=0.7\linewidth]{figures/use_case/use_case1_1.jpg}
%             \caption*{
%             \textbf{Ground Truth Caption:} A giraffe standing on top of a lush green field. Giraffe standing in middle of fenced-in area looking to the ground. A giraffe standing in the shade of a tree. \\
%             \textbf{Ours:} A giraffe standing in the middle of a field. \\
%             \textbf{MindBridge:} A train is traveling down the tracks in the city. \\
%             \textbf{UniBrain:} A zebra standing in the middle of a street.
%             }
%         \end{subfigure}
%     \end{tabular}
%     \caption{\textit{Case study of COCO Captioning.}}
%     \label{fig:coco_captioning}
% \end{figure}






In this section, we present a qualitative analysis of our model on COCO Captioning, COCO-QA, and OK-VQA, comparing its performance against MindBridge \cite{wang2024mindbridge} and UniBrain \cite{wang2024unibrain}. As shown in Figure \ref{fig:case1} and \ref{fig:case2_3}, our results demonstrate significant improvements in visual understanding across multiple tasks. The model shows strength in the following areas:
\textbf{Static Object Recognition. }The model demonstrates superior accuracy in identifying stationary objects. In comparison with baseline models (MindBridge and UniBrain), our approach shows improvement in spatial context understanding. For example, when analyzing aircraft imagery (e.g., (a) of Figure \ref{fig:case1}), our model correctly identifies "airplane sitting on the runway" while baselines incorrectly interpret the scene as "flying in the sky" or "flying over a city", demonstrating better state-space recognition. \textbf{Action Recognition. }Our proposed model exhibits enhanced capability in distinguishing between similar actions. In sports scenarios (e.g., (f) of Figure \ref{fig:case2_3}), our model correctly identifies "catch ball" while both baselines incorrectly predict "serve", indicating improved action-state discrimination.  \textbf{Potential of neuroscience application. }The demonstrated improvements in object understanding and action recognition suggest the potential for advancing brain-computer interface technology and neural processing research. The model's enhanced capabilities in distinguishing object states and actions could lead to more effective neural prosthetics and improved assistive technologies for individuals with visual or motor impairments.


