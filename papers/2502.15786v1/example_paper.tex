%%%%%%%% ICML 2025 EXAMPLE LATEX SUBMISSION FILE %%%%%%%%%%%%%%%%%

\documentclass{article}

% Recommended, but optional, packages for figures and better typesetting:
\usepackage{microtype}
\usepackage{graphicx}
% \usepackage{subfigure}
\usepackage[dvipsnames]{xcolor}
\usepackage{booktabs} % for professional tables
\usepackage{float}
\usepackage{multirow}
\usepackage{multicol}
\usepackage{enumitem}
\usepackage{makecell}
\usepackage{siunitx}
\usepackage{graphicx}
\usepackage{subcaption}
\usepackage[normalem]{ulem}
\usepackage{caption}
\usepackage{stfloats}
\usepackage{longtable}
\usepackage{array}
\usepackage{makecell}

\sisetup{
  table-format=2.2,  % Specify the format: 3 digits before and 2 after the decimal
  table-align-text-pre=false,  % Center-align numbers in their cells
  table-number-alignment=center  % Center-align numbers horizontally
}
\newcolumntype{M}{>{$}c<{$}}
% hyperref makes hyperlinks in the resulting PDF.
% If your build breaks (sometimes temporarily if a hyperlink spans a page)
% please comment out the following usepackage line and replace
% \usepackage{icml2025} with \usepackage[nohyperref]{icml2025} above.
\usepackage{hyperref}


% Attempt to make hyperref and algorithmic work together better:
\newcommand{\theHalgorithm}{\arabic{algorithm}}

% Use the following line for the initial blind version submitted for review:
% \usepackage{icml2025}

% If accepted, instead use the following line for the camera-ready submission:
\usepackage[accepted]{icml2025}

% For theorems and such
\usepackage{amsmath}
\usepackage{amssymb}
\usepackage{mathtools}
\usepackage{amsthm}
\usepackage{pifont}
\newcommand{\cmark}{\ding{51}}%
\newcommand{\xmark}{\ding{55}}%
\newcommand{\tdag}{\textsuperscript{\textdagger}}
\newcommand{\name}{MindLLM}
\usepackage{subcaption}
% if you use cleveref..
\usepackage[capitalize,noabbrev]{cleveref}
\usepackage{tabularx}
\usepackage{adjustbox}
\usepackage{makecell}
\usepackage[most]{tcolorbox}

%%%%%%%%%%%%%%%%%%%%%%%%%%%%%%%%
% THEOREMS
%%%%%%%%%%%%%%%%%%%%%%%%%%%%%%%%
\theoremstyle{plain}
\newtheorem{theorem}{Theorem}[section]
\newtheorem{proposition}[theorem]{Proposition}
\newtheorem{lemma}[theorem]{Lemma}
\newtheorem{corollary}[theorem]{Corollary}
\theoremstyle{definition}
\newtheorem{definition}[theorem]{Definition}
\newtheorem{assumption}[theorem]{Assumption}
\theoremstyle{remark}
\newtheorem{remark}[theorem]{Remark}


\newcommand{\rex}[1]{\textcolor{magenta}{(Rex: #1)}}
\newcommand{\yy}[1]{\textcolor{orange}{(Yujun: #1)}}
\newcommand{\qwk}[1]{\textcolor{blue}{(Weikang: #1)}}

% Todonotes is useful during development; simply uncomment the next line
%    and comment out the line below the next line to turn off comments
%\usepackage[disable,textsize=tiny]{todonotes}
\usepackage[textsize=tiny]{todonotes}

\newcommand{\papertitle}{\name{}: A Subject-Agnostic and Versatile Model for fMRI-to-Text Decoding}

% The \icmltitle you define below is probably too long as a header.
% Therefore, a short form for the running title is supplied here:
\icmltitlerunning{\papertitle{}}

\begin{document}

\twocolumn[
\icmltitle{\raisebox{-15pt}{\includegraphics[width=0.08\textwidth]{figures/LOGO.png}}\name{}:\\A Subject-Agnostic and Versatile Model for fMRI-to-Text Decoding}

% It is OKAY to include author information, even for blind
% submissions: the style file will automatically remove it for you
% unless you've provided the [accepted] option to the icml2025
% package.

% List of affiliations: The first argument should be a (short)
% identifier you will use later to specify author affiliations
% Academic affiliations should list Department, University, City, Region, Country
% Industry affiliations should list Company, City, Region, Country

% You can specify symbols, otherwise they are numbered in order.
% Ideally, you should not use this facility. Affiliations will be numbered
% in order of appearance and this is the preferred way.
\icmlsetsymbol{equal}{*}

\begin{icmlauthorlist}
\icmlauthor{Weikang Qiu}{yale}
\icmlauthor{Zheng Huang}{equal,dartmouth}
\icmlauthor{Haoyu Hu}{equal,cambridge}
\icmlauthor{Aosong Feng}{yale}
\icmlauthor{Yujun Yan}{dartmouth}
\icmlauthor{Rex Ying}{yale}
%\icmlauthor{}{sch}
%\icmlauthor{}{sch}
%\icmlauthor{}{sch}
\end{icmlauthorlist}

\icmlaffiliation{yale}{Yale University}
\icmlaffiliation{dartmouth}{Dartmouth College}
\icmlaffiliation{cambridge}{University of Cambridge}

\icmlcorrespondingauthor{Weikang Qiu}{weikang.qiu@yale.edu}
% \icmlcorrespondingauthor{Firstname2 Lastname2}{first2.last2@www.uk}

% You may provide any keywords that you
% find helpful for describing your paper; these are used to populate
% the "keywords" metadata in the PDF but will not be shown in the document
\icmlkeywords{Machine Learning, ICML}

\vskip 0.3in
]

% this must go after the closing bracket ] following \twocolumn[ ...

% This command actually creates the footnote in the first column
% listing the affiliations and the copyright notice.
% The command takes one argument, which is text to display at the start of the footnote.
% The \icmlEqualContribution command is standard text for equal contribution.
% Remove it (just {}) if you do not need this facility.

%\printAffiliationsAndNotice{}  % leave blank if no need to mention equal contribution
\printAffiliationsAndNotice{\icmlEqualContribution} % otherwise use the standard text.

\begin{abstract}
Decoding functional magnetic resonance imaging (fMRI) signals into text has been a key challenge in the neuroscience community, with the potential to advance brain-computer interfaces and uncover deeper insights into brain mechanisms. However, existing approaches often struggle with suboptimal predictive performance, limited task variety, and poor generalization across subjects. In response to this, we propose \name{}, a model designed for subject-agnostic and versatile fMRI-to-text decoding. \name{} consists of an fMRI encoder and an off-the-shelf LLM. The fMRI encoder employs a neuroscience-informed attention mechanism, which is capable of accommodating subjects with varying input shapes and thus achieves high-performance subject-agnostic decoding. Moreover, we introduce Brain Instruction Tuning (BIT), a novel approach that enhances the model’s ability to capture diverse semantic representations from fMRI signals, facilitating more versatile decoding. We evaluate \name{} on comprehensive fMRI-to-text benchmarks. Results demonstrate that our model outperforms the baselines, improving downstream tasks by $12.0\%$, unseen subject generalization by $16.4\%$, and novel task adaptation by $25.0\%$. Furthermore, the attention patterns in \name{} provide interpretable insights into its decision-making process.

\end{abstract}

Large language models (LLMs) show significant performance in various downstream
tasks~\citep{brown_language_2020,openai_gpt-4_2024,dubey_llama_2024}. Studies
have found that training on high quality corpus improves the ability of LLMs
to solve different problems such as writing code, doing math exercises, and
answering logic questions~\citep{cai_internlm2_2024,deepseek-ai_deepseek-v3_2024,qwen_qwen25_2024}.
Therefore, effectively selecting high-quality text data is an important subject for
training LLM.

\begin{figure}[t]
    \centering
    \includegraphics[width=\linewidth]{figures/head.pdf}
    \caption{The overview of CritiQ. We (1) employ human annotators to annotate $\sim$30
    pairwise quality comparisons, (2) use CritiQ Flow to mine quality criteria, (3)
    use the derived criteria to annotate 25k pairs, and (4) train the CritiQ Scorer to
    perform efficient data selection.}
    \label{fig:overview}
\end{figure}

To select high-quality data from a large corpus, researchers manually design heuristics~\citep{dubey_llama_2024,rae_scaling_2022},
calculate perplexity using existing LLMs~\citep{marion2023moreinvestigatingdatapruning,wenzek2019ccnetextractinghighquality},
train classifiers~\citep{brown_language_2020,dubey_llama_2024,xie_data_2023} and
query LLMs for text quality through careful prompt engineering~\citep{gunasekar_textbooks_2023,wettig_qurating_2024,sachdeva_how_2024}.
Large-scale human annotation and prompt engineering require a lot of human
effort. Giving a comprehensive description of what high-quality data is like is also
challenging. As a result, manually designing heuristics lacks robustness and introduces
biases to the data processing pipeline, potentially harming model performance
and generalization. In addition, quality standards vary across different
domains. These methods can not be directly applied to other domains without significant
modifications.

To address these problems, we introduce CritiQ, a novel method to automatically
and effectively capture human preferences for data quality and perform efficient data
selection. Figure~\ref{fig:overview} gives an overview of CritiQ, comprising an agent
workflow, CritiQ Flow, and a scoring model, CritiQ Scorer. Instead of manually describing
how high quality is defined, we employ LLM-based agents to summarize quality
criteria from only $\sim$30 human-annotated pairs.

CritiQ Flow starts from a knowledge base of data quality criteria. The worker
agents are responsible to perform pairwise judgment under a given
criterion. The manager agent generates new criteria and refines them through reflection
on worker agents' performance. The final judgment is made by majority voting among
all worker agents, which gives a multi-perspective view of data quality.

To perform efficient data selection, we employ the worker agents to annotate a randomly
selected pairwise subset, which is ~1000x larger than the human-annotated one.
Following \citet{korbak_pretraining_2023,wettig_qurating_2024}, we train CritiQ
Scorer, a lightweight Bradley-Terry model~\citep{bradley_rank_1952} to convert
pairwise preferences into numerical scores for each text. We use CritiQ Scorer to
score the entire corpus and sample the high-quality subset.

For our experiments, we established human-annotated test sets to quantitatively
evaluate the agreement rate with human annotators on data quality preferences. We implemented the manager agent by \texttt{GPT-4o} and the worker
agent by \texttt{Qwen2.5-72B-Insruct}. We conducted experiments on different
domains including code, math, and logic, in which CritiQ Flow shows a consistent
improvement in the accuracies on the test sets, demonstrating the effectiveness
of our method in capturing human preferences for data quality. To validate the quality
of the selected dataset, we continually train \texttt{Llama 3.1}~\citep{dubey_llama_2024}
models and find that the models achieve better performance on downstream tasks
compared to models trained on the uniformly sampled subsets.

We highlight our contributions as follows. We will release the code to facilitate
future research.

\begin{itemize}
    \item We introduce CritiQ, a method that captures human preferences for data
        quality and performs efficient data selection at little cost of human
        annotation effort.

    \item Continual pretraining experiments show improved model performance in code,
        math, and logic tasks trained on our selected high-quality subset compared to the raw dataset.

    \item Ablation studies demonstrate the effectiveness of the knowledge base and
        the the reflection process.
\end{itemize}

\begin{figure*}[t]
    \centering
    \includegraphics[width=\linewidth]{figures/method.pdf}
    \caption{CritiQ Flow comprises two major components: multi-criteria pairwise
    judgment and the criteria evolution process. The multi-criteria pairwise
    judgment process employs a series of worker agents to make quality
    comparisons under a certain criterion. The criteria evolution process aims to
    obtain data quality criteria that highly align with human judgment through
    an iterative evolution. The initial criteria are retrieved from the
    knowledge base. After evolution, we select the final criteria to annotate
    the dataset for training CritiQ Scorer.}
    \label{fig:method}
\end{figure*}

\section{Related Works}
\noindent\textbf{Brain-Conditioned Text Generation}
\begin{figure}
    \centering
    \includegraphics[width=\linewidth]{figures/uni_compare.pdf}
    \vspace{-2em}
    \caption{Comparison between our model and previous unified models. MindBridge \cite{wang2024mindbridge} flattens the voxels and adaptively pools them to a fixed dimension, which overlooks the rich information in positions. UniBrain~ \cite{wang2024unibrain} uniformly samples a subset of voxels and aggregates their neighbors. Different from these methods, we propose neuroscience-informed attention, where each query token attends to all voxels, which minimizes potential information loss in pooling or sampling.}
    \label{fig:uni_compare}
    \vspace{-1em}
\end{figure}
%There have been works that decode text from brain signals. 
This line of research mostly focuses on decoding perceived visual stimuli into natural language from fMRI signals. MindGPT~\cite{chen2023mindgpt}, UniBrain\cite{mai2023unibrain} and BrainCap \cite{ferrante2023brain} employ an fMRI encoder guided by CLIP~\cite{radford2021learning} and use a language model \cite{radford2019language,wang2022git} to decode natural language from the encoded representations. BrainChat~\cite{huang2024brainchat} utilizes multiple pretraining strategies~\cite{devlin2018bert, he2022masked,yu2022coca} to align fMRI with image and text embeddings. These methods fall short in performance and versatility. UMBRAE~\cite{xia2024umbrae} proposes to learn a mapping from fMRI to stimulus images, which later serves as a proxy input for an off-the-shelf visual language model (VLM). Although they achieve performance improvements, the strategy prevents the model from performing tasks that are not directly related to the stimulus images (e.g. answering memory-related questions). In contrast, our end-to-end Brain Instruction Tuning (BIT) ensures seamless and versatile fMRI-to-text decoding, offering the potential to tackle tasks beyond vision-related ones.

\noindent\textbf{Cross-subjects Decoding}
In voxel-level machine learning for brain decoding, the number of voxels varies between subjects \cite{allen2022massive}. Most prior works \cite{scotti2024reconstructing,scotti2024mindeye2} use an MLP for each subject individually. However, due to the fixed input size required by MLP architectures, these models cannot handle varying input shapes. As illustrated in Figure~\ref{fig:uni_compare}, MindBridge \cite{wang2024mindbridge} proposed to use an adaptive max pooling layer to standardize the input shapes. However, unlike images, which are considered translation invariance, positions in fMRI carry specific bio-informative significance that pooling operations may overlook. UniBrain \cite{wang2024unibrain} proposed to sample groups of voxels. Such a sampling strategy, on the one hand, may lead to information loss if some voxels are not included in any group. On the other hand, the irregular spatial distribution of 3D voxels with varying density and curvature may result in underrepresentation or overrepresentation of certain areas. Different from these methods, our model employs a neuroscience-informed attention mechanism that accounts for every single voxel while preserving their bio-informative positional information, ensuring a more comprehensive and precise representation. 
% BrainChat trains the model based on the caption and VQA dataset. UMBRAE introduces a universal encoder that aligns fMRI signals with image stimuli.


\noindent\textbf{Multi-Modal Large Language Model}
Aiming to augment the perceptual capacities of Large Language Models (LLMs), there has been a growing interest in extending them to handle multiple modalities within a unified model. Numerous studies have attempted to incorporate modalities such as images \cite{alayrac2022flamingo,zhang2023internlm,wang2023cogvlm}, videos \cite{cheng2024videollama, kondratyuk2023videopoet,zhang2023video}, and point clouds \cite{xu2023pointllm,qi2025shapellm}. %Flamingo~\cite{alayrac2022flamingo} uses a frozen vision encoder and an LLM equipped with gated cross-attention for cross-modality alignment. PaLM-E~\cite{driess2023palm} integrates extracted visual features directly through linear layers into the pre-trained PaLM~\cite{chowdhery2023palm}. BLIP2~\cite{li2023blip} achieves vision-language alignment by introducing a querying transformer to extract visual features from the frozen image encoder and integrating them with the frozen LLM. 
OneLLM~\cite{han2024onellm} stands out by aligning eight different modalities, including fMRI, with language. However, their approach employs an individual convolution network for each subject instead of a unified architecture for fMRI encoding across subjects, which restricts its applicability to new subjects in real-world scenarios. Furthermore, the approach solely relies on captions as textual annotations, which limits the model's capability for versatile fMRI decoding. 

% \paragraph{Instruction Tuning}
% Instruction tuning \cite{zhang2023instruction, aw2023instruction, radford2019language, thoppilan2022lamda} has been widely used to enable LMs to follow specific natural language instructions. 
% A key method is Instruction Dataset Construction, where (instruction, output) pairs are generated from existing annotated datasets by applying templates to convert text-label pairs into instruction-output pairs \cite{longpre2023flan, sanh2021multitask, conover2023free}. For instance, Natural Instructions \cite{mishra2021cross} creates datasets comprising task descriptions with components like title, definition, cautions, prompts, and example pairs, while P3 \cite{sanh2021multitask} maps conventional NLP tasks, such as question answering or text classification, into natural language input-output pairs. In this work, we contribute to instruction tuning from a neuroscience perspective by introducing the Brain Instruction Tuning Dataset, which decodes brain activity into language representations.


\section{Method}
\label{sec:method}
In this section, we propose a neuroscience-informed fMRI encoder designed to achieve high-performance, subject-agnostic decoding. To further enable versatile decoding, we introduce the construction of a brain instruction tuning dataset, which captures diverse semantic representations encoded in fMRI data.

\subsection{Method Overview}
As illustrated in Figure~\ref{fig:arch}, our model consists of an fMRI encoder $f_\theta$ and an off-the-shelf LLM. In practice, we use Vicuna-7b \cite{zheng2023judging} as our LLM to maintain consistency with our baseline \cite{xia2024umbrae}. For each sample, let $\boldsymbol{v} = [v_1, v_2, \cdots, v_N]\in \mathbb{R}^N$ be the fMRI signals of input voxels, where $N$ is the number of voxels. Note that $N$ varies between different subjects, ranging from $12,682$ to $17,907$ in the dataset we use \cite{allen2022massive}.

The fMRI encoder $f_\theta$, featuring a neuroscience-informed attention layer, encodes $\boldsymbol{v}$ to fMRI tokens $X_v = [\boldsymbol{x}_{v,1}, \boldsymbol{x}_{v,2}, \cdots, \boldsymbol{x}_{v,L}] \in \mathbb{R}^{d\times L}$, where $L$ is the number of tokens and $d$ is the dimension of token embeddings. We then prepend these learned fMRI tokens to the language tokens in the BIT dataset we propose.

\subsection{fMRI Encoder}
As mentioned before, currently most models for fMRI decoding can not handle varying input shapes and are not subject-agnostic, with only a few exceptions \cite{mai2023unibrain}. However, these exceptions still suffer from information loss and uneven representations of certain brain areas. To this end, we propose a novel neuroscience-informed attention mechanism to accommodate varying voxel numbers across subjects, enabling a subject-agnostic encoding strategy. Below we talk about the design of \textit{queries} $\{\boldsymbol{q}_i\}$, \textit{keys} $\{\boldsymbol{k}_i\}$ and \textit{values} $\{\boldsymbol{v}_i\}$ in the attention layer. For \textit{values}, we directly use the fMRI signal of each voxel, which means $\boldsymbol{v_i} = v_i \in \mathbb{R}$. Making each voxel a \textit{value} token maximally prevents information loss compared to pooling- \cite{wang2024mindbridge} or sampling-based \cite{mai2023unibrain} methods. The \textit{queries} are randomly initialized and learnable. We expect each \textit{query} to represent a certain pattern of the brain (refer to visualizations in Section \ref{sec:vis}). The design of \textit{keys} will be discussed below.

\noindent\textbf{Exclude fMRI values from \textit{keys}}
The vanilla cross attention \cite{zhu2020deformable,vaswani2017attention} derives both \textit{keys} and \textit{values} from the same input source. However, we found this would lead to poor performance in fMRI. We argue the reason: different from images or text, which are usually considered translation-invariant, the positions of voxels carry specific brain \textit{functional information}, as voxels in different areas are associated with distinct brain functions. Consequently, a voxel's position alone can theoretically serve as effective \textit{keys} for attention weight computation. Including fMRI values into \textit{keys}, however, introduces additional noise instead of valuable information, thus resulting in poorer performance. Moreover, since brain regions tend to serve similar functions across individuals, decoupling voxel positions from fMRI signals can facilitate the sharing of priors across subjects, potentially improving generalization to unseen subjects.

In light of this, instead of the vanilla cross attention, which derives the \textit{keys} and \textit{values} from the same inputs, we exclude the fMRI value of each voxel and use its positional information alone as its \textit{key} embedding. The positional information is encoded from the coordinates of each voxel, i.e. $\boldsymbol{k}_i^{\text{pos}} = \operatorname{PE}(\boldsymbol{c}_i)$ for the $i$-th voxel, where $\boldsymbol{c}_i \in \mathbb{R}^3$ denotes the coordinates of the voxel. In practice, we use the Fourier positional encoding proposed in \cite{tancik2020fourier} due to its superiority in encoding coordinate information.

\noindent\textbf{Incorporation of Brain Parcellations}
% \noindent\textbf{Incorporation of Brain Parcellations}
While positional encoding alone improves performance, it lacks inherent neuroscientific grounding, potentially making it challenging for the model to efficiently learn representations aligned with established principles of brain function. To overcome this, we incorporate existing brain region parcellations \cite{glasser2016multi,rolls2020automated} into the \textit{key} embeddings. Formally, given a parcellation $\mathcal{P}$, with regions indexed by $1, \cdots, N_\mathcal{P}$. Let $\mathcal{P}(i) \in [1, 2, \cdots, N_\mathcal{P}]$ be the region that the $i$-th voxel belongs to, and $E[\mathcal{P}(i)] \in \mathbb{R}^d$ be the corresponding learnable embedding of the region, which will be incorporated in the \textit{key} embeddings as $\boldsymbol{k}_i^{\text{reg}, \mathcal{P}} = E[\mathcal{P}(i)] \in \mathbb{R}^d$.

\noindent\textbf{Combining Multiple Parcellations}
It is crucial to choose an appropriate brain region parcellation. Previous region-based methods \cite{qiu2023learning,li2021braingnn, kan2022brain} can usually only choose one arbitrarily. In contrast, our model design allows us to combine multiple parcellations $\mathcal{P}^1, \mathcal{P}^2, \cdots$ by concatenating their respective region encodings to the \textit{key} embeddings. In conclusion, the final \textit{key} embeddings are the concatenation by the positional encoding and multiple region encodings,
\begin{equation}
    \boldsymbol{k}_i = \boldsymbol{k}_i^\text{pos} \| \boldsymbol{k}_i^{\text{reg}, \mathcal{P}^1} \|  \boldsymbol{k}_i^{\text{reg}, \mathcal{P}^2} \| \cdots
\end{equation}
where $\|$ denotes the concatenation operation. This process is illustrated in Figure~\ref{fig:arch}'s lower right part.

The positional and region encodings complement each other: The region encodings serve as coarse-scale features, providing a neuroscientific-grounded basis, while the fine-scale positional encoding allows our model to learn finer-grained information directly from the data.

This attention design separates a voxel's \textit{functional information}—which is largely consistent across individuals—from its fMRI value, thereby enhancing generalization. Instead of relying on pooling or sampling, the attention mechanism employs learnable aggregation, while the integration of positional encoding and neuroscientifically informed region encodings further ensures high performance.

After the attention layer, we obtain the hidden representations $\boldsymbol{z}_q \in \mathbb{R}^{N_q} $ where $N_q$ is the number of query embeddings. We then employ an MLP and a reshape operation to map the hidden representations to $L$ fMRI tokens, i.e., $   X_v = \operatorname{reshape}\left( \operatorname{MLP}(
    \{\boldsymbol{z}_q\}
    ) \right) \in \mathbb{R}^{L \times d}$.

The process of the fMRI encoder is illustrated in Figure~\ref{fig:arch}. The obtained fMRI tokens are then prepended to the language tokens in conversations.
\begin{figure}
    \centering
    \includegraphics[width=\linewidth]{figures/arch.pdf}
    % \vspace{-2.2em}
    \caption{Model Architecture. The fMRI encoder maps fMRI to a series of fMRI tokens through our proposed neuroscience-informed attention. The large language model, with both fMRI and text tokens, will be trained by brain instruction tuning.}
    \label{fig:arch}
    \vspace{-1em}
\end{figure}

\subsection{Brain Instruction Tuning (BIT)}
To enable versatile fMRI-to-text decoding, an appropriate BIT dataset is required, yet no such dataset currently exists. To bridge this gap, we construct one based on the fact: MSCOCO images \cite{chen2015microsoft} serve as stimuli for fMRI recordings in the fMRI study \cite{allen2022massive}, and an abundance of datasets provide text annotations (e.g., VQA) for MSCOCO images. Using the images as intermediaries, we select those relevant to brain functions and pair the fMRI data with corresponding text annotations. For example, given an image of a billboard with annotated textual content, we can reasonably infer that when a subject perceives textual information (e.g., contents on the billboard), corresponding representations are encoded in the brain. This suggests the possibility of extracting such information from fMRI signals. We select datasets to fulfill various purposes, enabling the model to capture diverse aspects of semantic information embedded in fMRI signals, including visual perception \& scene understanding, language \& symbolic processing, memory \& knowledge retrieval and complex reasoning, which are considered among most fundamental and essential properties of human brains \cite{robertson2002memory,stenning2012human,wade2013visual,friederici2017language}.

\begin{figure}[h]
% \vspace{-0.5em}
    \centering
    \includegraphics[width=\linewidth]{figures/bit.pdf}
\vspace{-1.8em}
    \caption{Dataset Taxonomy in Brain Instruction Tuning.}
    \label{fig:bit}
% \vspace{-1em}
\end{figure}

\noindent\textbf{Perception \& Scene Understanding} As illustrated in Figure~\ref{fig:bit}, we begin by using caption tasks at both coarse and fine-grained levels to train the model’s ability to understand and summarize what the subject perceives visually \cite{chen2015microsoft,krause2017hierarchical}. Additionally, we incorporate QA tasks \cite{ren2015exploring,krishna2017visual,acharya2019tallyqa} to enhance the model's ability to retrieve and reason about visually perceived content.

\noindent\textbf{Memory \& Knowledge Retrieval} To go beyond tasks directly related to present visual perception, we construct the \emph{previous captioning} task, a memory-oriented task that challenges the model to caption images that the subject previously viewed, simulating memory recall processes. Furthermore, we aim to encode knowledge structures in human brains. The OK-VQA \cite{marino2019ok} and A-OKVQA \cite{schwenk2022okvqa} datasets include questions requiring external knowledge that is not present in the image but resides in human brains. For example, A photo of a hydrant may prompt the answer "firetruck," even though the firetruck is absent in the image. This association also reflects the way human cognition operates through a network of interconnected meanings, where one concept unconsciously triggers another. Such a process, which is called "slippage of the signifier" \cite{lacan2001ecrits, lacan1988seminar, miller2018four}, highlights the symbolic processes through which the brain constructs and retrieves meaning. 

\noindent\textbf{Language \& Symbolic Processing} In addition to the aforementioned OK-VQA and A-OKVQA datasets, which are also related to symbolic process, we further combine datasets of text recognition \cite{biten2019scene} and numerical reasoning \cite{acharya2019tallyqa} to facilitate this aspect.

\noindent\textbf{Complex Reasoning} Finally, we try to approximate the reasoning process that happens in human brains with datasets \cite{liu2023visual,wang2023see,li2018vqa} that require intricate logical and inferential processes. We expect these datasets to challenge the model to extract the reasoning process, drawing upon both visual understanding and abstract problem-solving, thus bridging perception, memory, and knowledge into a cohesive cognitive framework.

We ended up with a brain instruction tuning dataset consisting of $980,610$ conversations associated with fMRI recordings from $15$ datasets. Appendix~\ref{app:dataset} lists the instructions and other details for each dataset. The instruction tuning enables versatile fMRI-to-text decoding. In particular, the introduction of tasks like \textit{previous caption} empowers the model to perform a broader range of tasks beyond vision-related ones, which the previous model \cite{xia2024umbrae} fails.

\begingroup
\sisetup{
  table-format=2.2,  % 3 digits before the decimal, 2 after
  table-align-text-pre=false,
  propagate-math-font=true,
  table-number-alignment=center,
  detect-weight=true,detect-inline-weight=math
}
\begin{table*}[bp]
    \centering
    \vspace{-1.7em}
    \caption{Results of brain captioning. The CIDEr metric is scaled by a factor of 100 for consistency with Table~\ref{tab:caption} and baselines.}
    \label{tab:caption}
\vspace{0.1in}
    \resizebox{\linewidth}{!}{
    \begin{tabular}{lcSSSSSSSS}
    \toprule
 % \multirow{2}{*}{Method}&  \multirow{2}{*}{cross-subject}&\multicolumn{5}{c}{fMRI caption} & &  &\\
   {Method} & {\makecell{subject\\agnostic}}  &{{BLEU-1} $\uparrow$} & {BLEU-2 $\uparrow$} & {BLEU-3 $\uparrow$} & {{BLEU-4} $\uparrow$} &{METEOR $\uparrow$}&{ROUGE $\uparrow$}& {CIDEr $\uparrow$}&{SPICE $\uparrow$}\\
    \midrule
    SDRecon \cite{takagi2023high}    & {\xmark} &36.21 & 17.11 & 7.22 & 3.43   &10.03&  25.13&13.83 &5.02 \\
    OneLLM  \cite{han2024onellm}  & {\xmark} &47.04 & 26.97 & 15.49 & 9.51   &13.55&  35.05&22.99 & 6.26\\
    UniBrain \cite{mai2023unibrain}   & {\xmark} & {$-$}   & {$-$}    & {$-$}  & {$-$}     &16.90&  22.20& {$-$} & {$-$}\\
    BrainCap \cite{ferrante2023brain}  & {\xmark} &55.96 & 36.21 & 22.70 & 14.51   &16.68& 40.69&41.30 & 9.06\\
     BrainChat \cite{huang2024brainchat} & {\xmark}   &52.30& 29.20& 17.10& 10.70 &14.30& 45.70&26.10 & {$-$}\\
    UMBRAE \cite{xia2024umbrae}    & {\xmark} &59.44& 40.48& 27.66&19.03&19.45&  43.71&61.06&12.79\\
    \name{} (Ours)  & {\cmark} & \bfseries 61.75 &  \bfseries42.84 & \bfseries29.86&\bfseries21.24  & 
\bfseries 19.54 &\bfseries45.82 & 60.97  & 11.79\\
    \bottomrule
    \end{tabular}}
\end{table*}
\endgroup


To train the model with the BIT dataset, for each sample $\boldsymbol{v}$, we sample a multi-run conversation $X_t = (X_u^1, X_a^1, \cdots, X_u^T, X_a^T)$ from all conversations associated with it, where $T \geq 1$ represents the number of turns. $a$ indicates the message from the assistant and $u$ indicates the message is from the user. The training objective is to maximize the probability of the assistant's response only
$$
\arg\max_\theta p(X_a | X_v, X_{\text{inst}}) = \prod_{t=1}^T p({\color{magenta}X_a^t} | X_u^{\leq t}, X_a^{\le t }, X_\text{inst}, X_v)
$$
Figure~\ref{fig:chat} illustrates the chat template and the training objective. We freeze the weights of the LLM and only train the fMRI encoder since we want to preserve the LLM's language modeling prior and ensure a fair comparison with baselines such as \citet{xia2024umbrae}.

\noindent\textbf{Computational Complexity} According to the analysis in Appendix~\ref{app:complexity}, our model does not introduce additional complexity compared to previous methods \cite{scotti2024mindeye2, wang2024mindbridge}.


\begin{figure}[htbp]
\vspace{-0.8em}
\centering
\begin{minipage}{0.99\columnwidth}\vspace{0mm}    \centering
\begin{tcolorbox}[colback=white,colframe=gray,left=1pt,top=1pt,bottom=1pt]
\sffamily
\footnotesize	
  \texttt{<system message>}\\
  user: $X_v$, $X_\text{inst}$, $X_1^u$ \\
  assistant: {\color{magenta}$X_1^a$}\\
user: $X_2^u$\\
  assistant: {\color{magenta}$X_2^a$}\\
  $\cdots\cdots$
\end{tcolorbox}
\end{minipage}
\caption{The chat template used during instruction tuning, illustrating two turns of conversations. Two turns of conversations are shown. Tokens highlighted in {\color{magenta}magenta} are used for next-token prediction loss computation.}
\label{fig:chat}
\vspace{-1.2em}
\end{figure}


\section{Experiments}
In this section, we first evaluate our model on various downstream tasks, demonstrating its versatile decoding capabilities. Next, we assess its generalizability to novel subjects and its adaptability to real-world applications. Finally, we analyze the functions of queries in our neuroscience-informed attention mechanism.

\subsection{Settings}

\noindent\textbf{fMRI Datasets} 
We use the widely used Natural Scenes Dataset (NSD) \cite{allen2022massive}, a large-scale dataset consisting of fMRI measurements of $8$ healthy adult subjects. During data collection, subjects viewed images from the MS-COCO dataset \cite{lin2014microsoft} and were instructed to press buttons to indicate whether they had previously seen each image.

% \textit{Instruction datasets}
% COCO Caption \cite{chen2015microsoft}, Paragraph Captioning \cite{krause2017hierarchical},
% COCO QA \cite{ren2015exploring}, 
% VQAv2 \cite{goyal2017making}, Visual Genome \cite{krishna2017visual}, A-OKVQA \cite{schwenk2022okvqa}, ST-VQA \cite{biten2019scene}, OK-VQA \cite{marino2019ok}, \cite{acharya2019tallyqa}, VQA-E \cite{li2018vqa}, FSVQA \cite{shin2016color}
% VisDial \cite{murahari2019improving},
% LVIS-instruct4v \cite{wang2023see}, LLaVA Instruct 150K \cite{liu2023visual}

\noindent\textbf{Downstream Datasets} The downstream dataset will be discussed within each experiment section. See examples and a short description for all dataset we will use in Appendex~\ref{app:dataset}.
Implementation details could be found in Appendix~\ref{app:impl}.

\begingroup
\sisetup{
  table-format=3.2,  % 3 digits before the decimal, 2 after
  table-align-text-pre=false,
  propagate-math-font=true,
  table-number-alignment=center,
  detect-weight=true,detect-inline-weight=math
}
\def\Uline#1{#1\llap{\uline{\phantom{#1}}}}
\begin{table*}[t]\centering
\vspace{-1em}
\caption{Versatile decoding. A dash $-$ means the model could not perform this task. The superscript $^\circ$ means the model is trained from scratch in contrast to their BIT version. The CIDEr metric is scaled by a factor of 100 for consistency with Table~\ref{tab:caption} and baselines.}
\vskip 0.1in
\label{tab:versatile}
\resizebox{\linewidth}{!}{
\begin{tabular}{cc|SSS|SSS|SSS}\toprule
& & {OneLLM} & {UMBRAE} &{BrainChat} & {MindBridge$^\circ$} & {UniBrain$^\circ$} & {\makecell{\name{}}$^\circ$} &{MindBridge} & {UniBrain} & {\makecell{\name{}}} \\
\midrule
subj-agnostic &  & {\xmark}& {\xmark} & {\xmark} & {\cmark} & {\cmark} & {\cmark} &{\cmark} & {\cmark} & {\cmark} \\
% subj-agnostic &  & {\xmark}& {\xmark} & {\xmark} & {\xmark} & {\cmark} & {\cmark} &\xmark & {\cmark} & {\cmark} \\
\midrule
COCO-QA &Accuracy$\uparrow$ &11.09\% & 22.23\% & 39.44\% &40.19\% &38.38\% &42.09\% &\Uline{45.33\%} &42.00\% & \bfseries 48.19\% \\
\midrule
VG-QA &Accuracy$\uparrow$ & 8.76\% & 19.67\% & 21.00\% &20.84\% &21.27\% &21.68\% &23.53\% & \Uline{24.02\%} &\bfseries 24.06\% \\
\midrule
VQA-v2 &Accuracy $\uparrow$& 33.68\% & \Uline{51.23}\% & 40.02\% &43.25\% &46.04\% &44.13\% &47.91\% &48.58\% &\bfseries 52.14\% \\
\midrule
A-OKVQA &Accuracy $\uparrow$&25.23\%& 43.24\% &20.52\% &22.12\% &19.47\% &29.20\% & \Uline{50.44\%} &43.36\% &\bfseries 52.21\% \\
\midrule
ST-VQA &ANLS $\uparrow$& 5.74\%& 5.46\% &9.58\% &10.20\% &7.01\% &\Uline{12.76}\% &11.64\% &8.76\% &\bfseries 12.92\% \\
\midrule
OK-VQA &Accuracy $\uparrow$&22.98\% & 10.35\% &17.22\%&27.63\% &18.63\% &27.70\% &32.13\% &\Uline{32.30\%} & \bfseries 33.33\% \\
\midrule
\multirow{2}{*}{TallyQA} &Accuracy $\uparrow$& 8.34\% & 44.10\% & 43.22\%&43.49\% &44.83\% &43.75\% &49.46\% & \Uline{53.77\%} &\bfseries 54.76\% \\
&RMSE $\downarrow$&7.45 & 3.94 & 1.90 &2.03 &1.83 &2.04 &1.86 &\bfseries 1.67 &\Uline{1.76} \\
\midrule
\multirow{6}{*}{Paragraph Caption} &BLEU-1$\uparrow$ &0.26 & \bfseries 29.82 & 22.21&21.82 &25.69 &26.49 &25.69 &28.28 & \Uline{29.43} \\
&BLEU-2 $\uparrow$ &0.08 & 14.26 &10.23 &10.47 &12.62 &12.48 &13.00 &\Uline{15.47} & \bfseries 15.78 \\
&BLEU-3$\uparrow$ &0.03 & 6.52 &6.38 &5.58 &6.70 &6.43 &7.10 &\Uline{8.90} & \bfseries 9.14 \\
&BLEU-4$\uparrow$ &0.01 & 2.95 &2.12 &3.14 &3.81 &3.63 &4.22 &\bfseries5.60 & \Uline{5.51} \\
&METEOR $\uparrow$&2.36 &12.60 &9.10 &10.95 &11.13 &2.44 &3.56 &\bfseries13.50 & \Uline{13.18} \\
&CIDEr $\uparrow$&0.00 & 7.39 &6.02&7.50 &3.92 &\Uline{10.71} &\bfseries11.39 &1.82 & 7.80 \\
\midrule
\multirow{8}{*}{VQA-E} &Accuracy $\uparrow$ &19.60\%& 47.84\% &46.20\% &45.40\% &44.42\% &44.55\% &\Uline{48.48\%} &48.39\% &\bfseries 50.95\% \\
&BLEU-1 $\uparrow$&17.32& 29.83 & 35.99 &35.63 &35.30 &35.08 &36.18 &\Uline{37.26} &\bfseries 37.70 \\
&BLEU-2 $\uparrow$&7.44& 14.76 & 18.33  &18.27 &18.04 &17.82 &19.38 &\Uline{20.41} &\bfseries 20.56 \\
&BLEU-3 $\uparrow$&3.62& 8.17 & 10.01 &10.32 &10.20 &10.05 &11.30 &\Uline{12.25} &\bfseries 12.34 \\
&BLEU-4 $\uparrow$&1.82& 4.87 &6.60 &6.27 &6.14 &6.00 &7.00 &\Uline{7.83} &\bfseries 7.92 \\
&CIDEr $\uparrow$&19.32& 63.26 &78.33 &79.05 &77.31 &76.80 &86.62 &\Uline{92.09} & \bfseries 93.60 \\
&METEOR $\uparrow$&6.69 & 12.25 &13.64 &14.13 &13.89 &13.96 &14.81 &\Uline{15.51} &\bfseries 15.62 \\
&ROUGE $\uparrow$&16.84& 28.38 & 32.82&33.78 &33.25 &33.11 &34.56 &\Uline{35.87} &\bfseries 35.88 \\
\midrule
\multirow{8}{*}{FSVQA} &VQA Acc. $\uparrow$& 31.44\% & 40.67\% &36.30\% &42.00\% &37.05\% &42.53\% &\Uline{45.95\%} &44.58\% &\bfseries 48.03\% \\
&FSVQA Acc. $\uparrow$& 21.02\% & 0.00\% & 30.22\% &37.40\% &32.30\% &38.50\% &\Uline{40.97\%} &37.87\% &\bfseries 43.00\% \\
&BLEU-1 $\uparrow$& 37.42 & 23.11 & 83.99&85.68 &83.84 &85.88 &\Uline{86.52} &85.10 &\bfseries 87.10 \\
&BLEU-2 $\uparrow$ & 31.72 & 5.86&78.50 &81.27 &78.81 &81.62 &\Uline{82.28} &80.01 &\bfseries 83.03 \\
&BLEU-3 $\uparrow$& 26.95 & 2.10&73.00 &77.10 &73.97 &77.62 &\Uline{78.34} &75.49 &\bfseries 79.27 \\
&BLEU-4 $\uparrow$& 22.48 & 1.04 &69.73 &72.89 &68.91 &73.56 &\Uline{74.35} &70.73 &\bfseries 75.50 \\
&METEOR $\uparrow$& 26.35 & 8.93 &44.76 &47.59 &45.94 &47.96 &\Uline{48.63} &46.89 &\bfseries 49.05 \\
&CIDEr $\uparrow$& 312.75 & 4.07 &600.00 &636.40 &609.00 &646.26 &\Uline{657.02} &628.83 &\bfseries 666.26 \\
\midrule
\multirow{8}{*}{Previous Caption} & BLEU-1 $\uparrow$ & 41.86  & {$-$} & 21.19 & 21.17 & 24.84 & \Uline{44.52} & 42.45 & 43.01 & \bfseries 47.20 \\
&BLEU-2 $\uparrow$ & 19.44 & {$-$} & 8.00 & 7.57 & 9.70 & \Uline{22.46} & 20.04 & 20.03 & \bfseries 25.16\\
&BLEU-3 $\uparrow$ & 9.25 & {$-$} & 1.98 & 2.85 & 3.40 & \Uline{10.39} & 9.61 & 9.19 & \bfseries 12.95\\
&BLEU-4 $\uparrow$ & 3.67 & {$-$} & 1.02& 1.28 & 1.46 & \Uline{5.45} &  5.31 & 4.58 & \bfseries 7.49 \\
&METEOR $\uparrow$ & 10.14 & {$-$} & 6.55& 6.46 & 7.20 & \Uline{11.00} & 10.83 & 10.81 & \bfseries 11.96 \\
&ROUGE $\uparrow$ & 30.19 & {$-$} & 21.23& 20.88 & 23.04 & \Uline{33.20} &32.38 & 31.99 & \bfseries 34.58 \\
&CIDEr $\uparrow$ & 6.65 & {$-$} & 9.21& 8.83 & \Uline{11.73} & 9.39 & 7.89 & 7.53 & \bfseries 16.02 \\
&SPICE $\uparrow$ & 2.49 & {$-$} & 2.44& 2.56 & 2.78 & \Uline{3.07} & 2.80 & 2.92 & \bfseries 3.93 \\
\bottomrule
\end{tabular}}
\vspace{-1em}
\end{table*}
\endgroup

\subsection{Brain Captioning}
To evaluate the model's performance on downstream tasks, we start with the widely used brain captioning benchmark \cite{xia2024umbrae}. The task, built upon COCO Caption \cite{chen2015microsoft} requires the model to predict captions of given images as fMRI stimuli.

\noindent\textbf{Baselines}
The following baselines are considered in this experiment: SDRecon \cite{takagi2023high}, UniBrain \cite{mai2023unibrain}, and BrainCap \cite{ferrante2023brain} employs a linear regression, mapping the fMRI to the inputs of an image caption model \cite{li2023blip}. OneLLM \cite{han2024onellm} is a multimodal large language models that align $8$ modalities (including fMRI) with language all in one model. For fair and efficient comparison, we only finetune the encoder, given that we freeze the LLM in our method as well. UMBRAE learns an encoder that maps fMRIs to images through an encoder similar to the MLP mixer \cite{tolstikhin2021mlp}. BrainChat \cite{huang2024brainchat} segments the flattened voxels into 16 patches and employs a transformer to decode text conditioned on the patches.
It is worth noting that all of these baselines require subject-specific layers or parameters. In contrast, our model is subject-agnostic, thus with the potential to generalize on novel subjects.

\noindent\textbf{Metric} Following previous works, we use five standard metrics for text generation: BLEU-$k$ \cite{papineni2002bleu}, ROUGE-L \cite{lin2004rouge}, CIDEr \cite{vedantam2015cider}, SPICE \cite{anderson2016spice}, METEOR \cite{banerjee2005meteor}.

Table \ref{tab:caption} shows that our model outperforms baselines in terms of most metrics, with an average improvement of $3.32\%$, even if our model does not have any subject-specific layers. We argue that this is attributed to both the novel architecture design and the introduction of BIT, which will be evident in the next experiment.

\begingroup
\sisetup{
  table-format=3.2,  % 3 digits before the decimal, 2 after
  table-align-text-pre=false,
  table-number-alignment=center,
detect-weight=true,detect-inline-weight=math
}
\begin{table}[t]
\vspace{-1em}
\centering
\caption{Model generalization, compared with subject-agnostic model. We train the models on subject $1-7$ and evaluate on subject $8$, which is the held-out subject.}
\label{tab:subj8}
\vskip 0.1in
\resizebox{\linewidth}{!}{
\begin{tabular}{ccSSS}\toprule
& & {MindBridge} & {UniBrain} & {\name{}} \\
\midrule
COCO-QA &Accuracy $\uparrow$&35.88 &24.95 &\bfseries 38.75 \\
\midrule
VG-QA &Accuracy $\uparrow$&\bfseries 20.56 &16.23 & 18.81 \\
\midrule
VQA-v2 &Accuracy $\uparrow$&42.80 &40.16 &\bfseries 44.69 \\
\midrule
A-OKVQA &Accuracy $\uparrow$&44.55 &28.71 &\bfseries 45.54 \\
\midrule
ST-VQA &ANLS $\uparrow$&9.33 &9.30 &\bfseries 10.97 \\
\midrule
OK-VQA &Accuracy $\uparrow$&21.94 &17.09 &\bfseries 24.45 \\
\midrule
\multirow{2}{*}{TallyQA} &Accuracy $\uparrow$&38.92 &32.51 &\bfseries 41.28 \\
&RMSE $\downarrow$&2.12 &\bfseries 2.02 &2.16 \\
\midrule
\multirow{8}{*}{COCO-Caption} &BLEU-1 $\uparrow$&39.84 &41.90 &\bfseries 47.3 \\
&BLEU-2 $\uparrow$&19.55 &19.67 &\bfseries 25.35 \\
&BLEU-3 $\uparrow$&9.29 &8.89 &\bfseries 13.61 \\
&BLEU-4 $\uparrow$&5.24 &4.33 &\bfseries 8.15 \\
&METEOR $\uparrow$&10.39 &10.80 &\bfseries 11.4 \\
&ROUGE $\uparrow$&31.10 &31.54 &\bfseries 34.64 \\
&CIDEr $\uparrow$&\bfseries 8.70 &6.40 &6.41 \\
&SPICE $\uparrow$&2.67 &2.39 &\bfseries 3.61 \\
\midrule
\multirow{6}{*}{Paragraph Caption} &BLEU-1 $\uparrow$&23.18 &21.73 &\bfseries 27.21 \\
&BLEU-2 $\uparrow$&10.71 &8.94 &\bfseries 12.48 \\
&BLEU-3 $\uparrow$&4.61 &3.72 &\bfseries 5.81 \\
&BLEU-4 $\uparrow$&2.22 &1.92 &\bfseries 3.01 \\
&METEOR $\uparrow$&9.99 &9.47 &\bfseries 10.24 \\
&CIDEr $\uparrow$&0.71 &1.56 &\bfseries 4.05 \\
\midrule
\multirow{8}{*}{VQA-E} &Accuracy $\uparrow$&41.78 &38.53 &\bfseries 44.81 \\
&BLEU-1 $\uparrow$&32.54 &32.86 &\bfseries 34.46 \\
&BLEU-2 $\uparrow$&16.13 &15.48 &\bfseries 17.86 \\
&BLEU-3 $\uparrow$&8.82 &7.98 &\bfseries 10.23 \\
&BLEU-4 $\uparrow$&5.16 &4.42 &\bfseries 6.19 \\
&CIDEr $\uparrow$&68.13 &58.79 &\bfseries 77.36 \\
&METEOR $\uparrow$&12.74 &12.26 &\bfseries 13.56 \\
&ROUGE $\uparrow$&30.63 &29.38 &\bfseries 32.99 \\
\midrule
\multirow{8}{*}{FSVQA} &VQA Acc. $\uparrow$&42.33 &37.92 &\bfseries 43.65 \\
&FSVQA Acc. $\uparrow$&37.16 &30.83 &\bfseries 38.41 \\
&BLEU-1 $\uparrow$&75.81 &82.94 &\bfseries 85.69 \\
&BLEU-2 $\uparrow$&71.03 &77.24 &\bfseries 81.18 \\
&BLEU-3 $\uparrow$&66.40 &71.94 &\bfseries 77.06 \\
&BLEU-4 $\uparrow$&61.55 &66.54 &\bfseries 72.86 \\
&METEOR $\uparrow$&45.84 &45.24 &\bfseries 47.87 \\
&CIDEr $\uparrow$&428.39 &587.78 &\bfseries 641.11 \\
\bottomrule
\end{tabular}
}
\vspace{-2em}
\end{table}
\endgroup

\subsection{Versatile Decoding}
The purpose of experiments in this section is two-fold: 1) To investigate the impact of our model design and the introduction of BIT on performance improvement. 2) To evaluate the versatility of the model, i.e., its performance on various downstream tasks.

\noindent\textbf{Baselines} Besides baselines that could be adapted to this experiment from the previous one, we further consider the following subject-agnostic models as baselines.
1) MindBridge \cite{wang2024mindbridge} flatten the voxels and adaptively adjust the padding and stride to pool the voxels into a fixed dimension. The original implementation of MindBridge has subject-specific parameters. However, since those parameters are of the same size, we make them shared across subjects and thus make the model subject-agnostic.
2) UniBrain \cite{wang2024unibrain} samples voxels into a fixed number of groups and employs a transformer where groups are treated as tokens. This UniBrain is unrelated to the UniBrain in the previous section; they just share the same name.

\noindent\textbf{Datasets \& Metric}
We use the test split of all QA \& caption datasets in the BIT dataset. We strictly adhere to the official metrics on all datasets. In summary, for sentence generation, we use BLEU-$k$~\cite{papineni2002bleu}, ROUGE-L~\cite{lin2004rouge}, CIDEr~\cite{vedantam2015cider}, SPICE~\cite{anderson2016spice}, METEOR~\cite{banerjee2005meteor}. For QA-related tasks, we use VQA accuracy~\cite{antol2015vqa} as well as special metrics proposed in the original paper (e.g. ANLS for ST-VQA~\cite{biten2019scene}).

The results are shown in Table~\ref{tab:versatile}. Our model outperforms baselines, with an average improvement of $12.0\%$. Further, by comparing instruction tuning and from-scratch models, we find that instruction tuning has a significant positive effect, with an average improvement of $28.0\%$. The results remain stable across different random seeds; for instance, according to our observations, the BLEU-1 score for paragraph captioning exhibits a maximum of $\pm 0.3$ variance.

\begingroup
\sisetup{
  table-format=2.2,  % 3 digits before the decimal, 2 after
  table-align-text-pre=false,
  table-number-alignment=center,
detect-weight=true,detect-inline-weight=math
}
\begin{table}[htbp]
\vspace{-1.5em}
    \caption{Model adaptation to new tasks. \textit{sentiment understanding} and \textit{utility/affordance} are sub-datasets from TDIUC that are particularly relevant to BCI applications.}
    \label{tab:new_task}
\vskip 0.1in
    \centering
    \resizebox{\linewidth}{!}{
    \begin{tabular}{cSSSS}
    \toprule
    \multirow{2}{*}{Method} &  \multicolumn{2}{c}{Overall} & {Sentiment Understanding} & {Utility/Affordance} \\
    \cmidrule{2-3} \cmidrule{4-4} \cmidrule{5-5}
    & {A-MPT} & {H-MPT} & {Accuracy} & {Accuracy} \\
    \midrule
    \name{}$^\circ$ & 41.09\% & 19.38\% & 70.00\% & 0.00\%\\
    \midrule
         MindBridge & 49.77\% & 39.88\% & 80.00\% & 14.29\%\\
         UniBrain & 51.50\% & 36.76\% & 80.00\% & 28.57\%\\
         \name{}& \bfseries 54.08\% &  \bfseries 45.43\% & \bfseries 80.77\% & \bfseries 50.00\% \\
    \bottomrule
    \end{tabular}
    }
    \vspace{-1em}
\end{table}
\endgroup


\subsection{Unseen Subject Generalization}
Our neuroscience-informed, subject-agnostic design enhances generalization to novel subjects, a crucial factor in real-world applications where training a model for each individual is impractical. To evaluate it, we perform instruction tuning on $7$ out of the $8$ subjects in the natural scene dataset \cite{allen2022massive}, and evaluate generalization on the held-out subject. Table~\ref{tab:subj8} shows our model outperforms two other subject-agnostic baselines in most cases, with an average improvement of $16.4\%$ compared to the second-best model.

\subsection{Adapting to New Tasks}
It is common that users want to adapt the \name{} to their own specific use cases. To this end, we aim to assess our model's adaptability to new tasks.

\noindent\textbf{Dataset \& Metrics} We use TDIUC \cite{kafle2017analysis}, a QA dataset consisting of $12$ types of questions, as a benchmark to evaluate the model's various capabilities comprehensively. Additionally, we further select $2$ task types-\textit{sentiment understanding} and \textit{utility/affordance} tasks, that are particularly relevant to BCI applications as sub-datasets. The \textit{utility/affordance} task, for instance, enables the model to identify useful objects in a given scene and autonomously decide whether to utilize them. Following their paper, we compute the accuracy of each type and report the arithmetic mean-per-type (A-MPT) and the harmonic mean-per-type (H-MPT). For the $2$ selected types, we report the accuracy respectively. 

Table~\ref{tab:new_task} shows our model achieves balanced (high harmonic mean) and consistently improved performances with an average of $13.5\%$. We could also observe the performance benefits from BIT, with $25.0\%$ absolute improvement.

\begin{figure*}[t]
    \begin{subfigure}[c]{0.15\textwidth}
        \centering
        \includegraphics[width=\textwidth]{figures/brain_query/brain_query_ppa.pdf}
        % \caption{Query: PPA}
        \vspace{-1.5em}
        \caption{}
        \label{sub:ppa}
    \end{subfigure}
    \hfill
    \begin{subfigure}[c]{0.15\textwidth}
        \centering
        \includegraphics[width=\textwidth]{figures/brain_query/brain_query_ffa_1.pdf}
        % \caption{Query: FFA-1}
        \vspace{-1.5em}
                \caption{}
        \label{sub:ffa-1}
    \end{subfigure}
    \hfill
    \begin{subfigure}[c]{0.15\textwidth}
        \centering
        \includegraphics[width=\textwidth]{figures/brain_query/brain_query_opa_ofa.pdf}
        % \caption{Query: FFA-1}
        \vspace{-1.5em}
        \caption{}
        \label{sub:opa_ofa}
    \end{subfigure}
    \hfill
    \begin{subfigure}[c]{0.15\textwidth}
        \centering
        \includegraphics[width=\textwidth]{figures/brain_query/brain_query_earlyvis_eba.pdf}
        % \caption{Query: FFA-1}
        \vspace{-1.5em}
        \caption{}
        \label{sub:ev_eba}
    \end{subfigure}
    \hfill
    \begin{subfigure}[c]{0.15\textwidth}
        \centering
        \includegraphics[width=\textwidth]{figures/brain_query/brain_query_multi_low_high.pdf}
        % \caption{Query: FFA-1}
        \vspace{-1.5em}
        \caption{}
        \label{sub:low_high}
    \end{subfigure}
    \hfill
    \begin{subfigure}[c]{0.15\textwidth}
        \centering
        \includegraphics[width=\textwidth]{figures/brain_query/brain_query_multi_high_high.pdf}
        % \caption{Query: FFA-1}
        \vspace{-1.5em}
        \caption{}
        \label{sub:high_high}
    \end{subfigure}
    \hfill
    \begin{subfigure}[c]{0.05\textwidth}
        \centering
        \includegraphics[width=\textwidth]{figures/brain_query/colorbar.png}
    \end{subfigure}
    \vspace{-1.4em}
    \caption{Visaulization of attention weights between \textit{queries} and brain voxels. Each subfigure represents a \textit{query} token, and the strength of color indicates its attention weight (after min-max normalization) to each voxel. 
    }
    \label{fig:query}
    \vspace{-1.2em}
\end{figure*}

\subsection{Ablation Study}
We conduct ablation studies on the design of key embeddings in the neuroscience-informed attention module in Figure~\ref{fig:ablation}. The results strongly validate our design. The vanilla cross attention (\textit{Pos Enc.+fMRI}) leads to poor performance while removing fMRI values from the key embeddings (\textit{Pos Enc.}) yields a significant improvement. Replacing positional encoding with region encodings (\textit{Reg. Enc.}) accelerates convergence in the early stages since it is grounded by neuroscientific principles. However, it is eventually outperformed by \textit{Pos Enc.} due to the lack of finer-grained information. Combining the positional encoding and region encodings (\textit{Pos Enc.+Reg Enc.}) achieves the best model design. In addition, replacing positional encoding with an MLP that maps coordinates to embeddings results in poor performance (\textit{(x,y,z)+MLP}), which indicates the amount of high-frequency spatial information in fMRI signals.

\begin{figure}[h]
\vspace{-0.8em}
    \centering
    \includegraphics[width=\linewidth]{figures/ablation.pdf}
    \vspace{-2em}
    \caption{Ablation study of the key embedding design. Pos Enc. stands for positional encoding. Reg Enc. stands for multiple region encodings. \textit{(x, y, z)+MLP} means we employ an MLP to map the coordinates to the embeddings instead of positional encoding.}
    \label{fig:ablation}
    \vspace{-1.3em}
\end{figure}
\subsection{Visualizations and Interpretations}
\label{sec:vis}



% \begin{figure}
%     \centering
%     \includegraphics[width=\linewidth]{figures/fig_brain.pdf}
%     \caption{Voxel-Level Brain Mapping of Model Attention.(a)-(c) present the attention map on a flattened brain with deeper blue indicating higher value. (d) gives an average value rank of different brain regions from (a).}
%     \label{fig:brain}
% \end{figure}

% Interpretability is of high importance in brain-inspired research, providing critical cues for how to leverage brain information more efficiently in the future and how to correctly transfer models into the real-world application \cite{fellous2019explainable,farahani2022explainable}. Other than compared advanced brain-decoding methods in this paper, our neuroscience-informed framework is interpretable on the brain-level. Accordingly, we conducted analysis in this section to show our model’s capability in distilling complex voxel-level brain signals and locating informative regions for high-level semantic tasks. We extracted the average attention feature from $\mathbf{Q}$ and $\mathbf{K}$ before it interacted with voxel values $\mathbf{V}$ and mapped it back to the original 3D brain cortex, as Figure \ref{fig:brain} shows. Notably, compared to early visual areas, brain regions from higher-level information processing stages caught much more attention of the model. This pattern is different from previous brain-imaging decoding methods with interpretations, such as \cite{takagi2023high} and (), in which early visual regions were among the most notable brain areas. Early visual regions are important, initially processing visual information, passing it to next stages, yet not crucial for high-level cognitive conceptualization. MindLLM recognized a map mainly focusing on Parahippocampal Place Area (PPA), Fusiform Face Area (FFA) and Extrastriate Body Area (EBA). PPA locates in parahippocampal gyrus, related to conceptual association, semantic processing and enviornmental memory \cite{epstein1999parahippocampal,kohler2002differential,bar2008scenes,epstein2010reliable}. FFA is known for its critical role in expertise recognition, social cognition and identity memory \cite{schultz2003role,tsantani2021ffa,xu2005revisiting}. And EBA involves body perception and contextual reasoning \cite{urgesi2007representation,carey2019distinct}. Together, they form a pattern of processing and extracting distilled sophisticated objects and spatial information from outside world, producing conceptual semantic cognition, and even inferring hidden messages behind the visual scenes from the information human gets \cite{amoruso2011beyond}. This revealed pattern proves model’s ability in getting high-level accurate information from the brain and provides evidence about potential brain regions for multitask brain decoding as well as the possibility of implementing better brain decoding model design with richer brain signals.

% The design of the learnable \textit{queries} provides a way to dynamically integrate massive brain-anatomical information from the \textit{keys}. 
Unlike previous deep learning models \cite{scotti2024mindeye2,mai2023unibrain}, our model allows interpretations by investigating how \textit{queries} work in the neuroscience-informed attention layer. We inspect the attention weights between queries and voxels in Figure~\ref{fig:query}. 
% Weight values are scaled by min-max normalization and results are presented in Figure \ref{fig:query}. 

We found that some queries primarily focus on processing single brain regions, such as Parahippocampal Place Area (PPA) (Figure~\ref{sub:ppa}) and Fusiform Face Area (FFA) (\ref{sub:ffa-1}). As previous research has shown, PPA is related to conceptual association, semantic processing and environmental memory \cite{epstein1999parahippocampal,kohler2002differential,bar2008scenes,epstein2010reliable} and FFA is known for its critical role in expertise recognition, social cognition and identity memory \cite{schultz2003role,tsantani2021ffa,xu2005revisiting}. Both are important brain regions for the conceptualization of visual information and are responsible for the interaction between real-time stimulus and past memory \cite{brewer1998making,ranganath2004category,golarai2007differential}. 

Moreover, there are some queries that attend to multiple brain regions, revealing the information transmission between low- and high-level brain regions. For instance, interactions between early visual areas and higher-level regions like PPA and IntraParietal Sulcus (IPS) (Figure~\ref{sub:low_high}), revealing a potential pattern for human attention-guided actions \cite{tunik2007beyond,connolly2016coding}. Additionally, queries are also found responsible for communications between high-level brain regions (Figure \ref{sub:opa_ofa},\ref{sub:high_high}). Together, these findings indicate that the learnable queries may reflect the dynamics of human brain activities in the visual task, from seeing and thinking about the image to pressing the button for the visual recall task in NSD \cite{allen2022massive}. 

% This emphasizes the advantage of modeling interactions between brain-informed keys and queries in the context of parallel multitask working\rex{what is this multitask working?}.

% As brain regions defined in neuroscience research likely to be processed individually and interactively in our brain-decoding pipeline\rex{i don't understand. doesn't seem like a useful sentence}, our model is potentially aware of the biological properties of the human brain rather than mechanically mixing all values up in the black box, like \rex{we haven't compared. how do we support such claims?} \cite{scotti2024reconstructing,scotti2024mindeye2,jiang2024mindshot}.

We also provide qualitative analysis of model responses in Appendix~\ref{app:qual}.

% \begin{figure}
%     \centering
%     \includegraphics[width=\linewidth]{figures/brain_query_slice.pdf}
%     \caption{Slices of Model Attention by Query Index.}
%     \label{fig:query}
% \end{figure}
\section{Conclusion}

We presented \method, a novel framework leveraging pretrained VLMs to unify temporal, visual, and textual modalities for time series forecasting. By integrating the RAL, VAL, and TAL, \method bridges modality gaps, enabling rich cross-modal interactions. Extensive experiments demonstrate state-of-the-art performance across various datasets, especially in few-shot and zero-shot scenarios, outperforming existing methods while maintaining efficiency. Our work establishes a new direction for multimodal time series forecasting, highlighting the potential of VLMs in capturing temporal dynamics and semantic context.

Notably, \method operates can solely on original time series data without external information, ensuring fair comparisons and showcasing its ability to generate textual and visual representations directly from the data for self-augmentation. This design not only enhances accuracy but also emphasizing the framework's robustness, particularly in domains where external data is scarce or unavailable.

Future work may explore adaptive visual transformations for complex patterns, enhancing text utilization, extending to multi-task, and developing more efficient multimodal time series foundation models. For details, see \shortautoref{appx:future_work}.


\section*{Impact Statement}

This paper presents work whose goal is to advance the field of Machine Learning by integrating temporal, visual, and textual modalities for time series forecasting. While our approach improves accuracy and cross-domain generalization, we acknowledge potential risks such as data privacy concerns, algorithmic bias, and increased computational costs. We encourage further research into mitigating these risks to ensure responsible deployment in high-stakes applications.

% \section*{Impact Statement}
% This paper presents work whose goal is to advance the field of Machine Learning. There are many potential societal consequences of our work, none of which we feel must be specifically highlighted here.

\bibliography{example_paper}
\bibliographystyle{icml2025}

%%%%%%%%%%%%%%%%%%%%%%%%%%%%%%%%%%%%%%%%%%%%%%%%%%%%%%%%%%%%%%%%%%%%%%%%%%%%%%%
%%%%%%%%%%%%%%%%%%%%%%%%%%%%%%%%%%%%%%%%%%%%%%%%%%%%%%%%%%%%%%%%%%%%%%%%%%%%%%%
% APPENDIX
%%%%%%%%%%%%%%%%%%%%%%%%%%%%%%%%%%%%%%%%%%%%%%%%%%%%%%%%%%%%%%%%%%%%%%%%%%%%%%%
%%%%%%%%%%%%%%%%%%%%%%%%%%%%%%%%%%%%%%%%%%%%%%%%%%%%%%%%%%%%%%%%%%%%%%%%%%%%%%%

\appendix
\onecolumn
\section{Dataset Details}
\label{app:dataset}

\subsection{Details of each dataset in brain instruction tuning}
In this section, we give a brief description of each source of our brain instruction datasets as well as examples from them.

\setlength{\LTcapwidth}{\textwidth}

{\scriptsize 
\begin{longtable}{>{\raggedright\arraybackslash}m{0.2\textwidth}m{0.2\textwidth}m{0.5\textwidth}}
\caption{Dataset details and examples.}
\label{tab:dataset_details}\\

\toprule
\textbf{Dataset} & \textbf{Description} & \textbf{Example}\\
\midrule
\endfirsthead

% \multicolumn{3}{c}{\tablename\ \thetable{} -- Continued from previous page}\\
\toprule
\textbf{Dataset} & \textbf{Description} & \textbf{Example}\\
\midrule
\endhead

\midrule
\multicolumn{3}{r}{Continued on next page}\\
\endfoot

\bottomrule
\endlastfoot

\begin{minipage}[c]{\linewidth}
    \vspace{0.5em}
    Previous Caption
    \vspace{0.5em}
\end{minipage} & 
\begin{minipage}[c]{\linewidth}
    \vspace{0.5em}
    Generating a one-sentence caption of the image that the subject previously saw.
    \vspace{0.5em}
\end{minipage} &
\begin{minipage}[c]{\linewidth}
    \centering
    \begin{minipage}[c]{0.3\linewidth}
        \includegraphics[width=0.7\linewidth]{figures/table5_example/prev-coco.jpg}
    \end{minipage}%
    \hfill
    \begin{minipage}[c]{0.65\linewidth}
        A neat bedroom pairs modern chairs with a glass table.
    \end{minipage}
\end{minipage}\\
\midrule

\begin{minipage}[c]{\linewidth}
    \vspace{0.5em}
    \makecell[l]{COCO Caption\\\cite{chen2015microsoft}}
    \vspace{0.5em}
\end{minipage} & 
\begin{minipage}[c]{\linewidth}
    \vspace{0.5em}
    Generate a one-sentence caption of the image the subject currently sees.
    \vspace{0.5em}
\end{minipage} &
\begin{minipage}[c]{\linewidth}
    \centering
    \begin{minipage}[c]{0.3\linewidth}
        \includegraphics[width=0.7\linewidth]{figures/table5_example/coco-caption.jpg}
    \end{minipage}%
    \hfill
    \begin{minipage}[c]{0.65\linewidth}
        The pedestrian is walking down the side of the highway by the bus.
    \end{minipage}
\end{minipage}\\
\midrule

\begin{minipage}[c]{\linewidth}
    \vspace{0.5em}
Image Paragraph Captioning\\ \cite{krause2017hierarchical}
\vspace{0.5em}
\end{minipage} & 
\begin{minipage}[c]{\linewidth}
    \vspace{0.5em}
 Generate a one-paragraph caption of the image the subject currently sees    \vspace{0.5em}
\end{minipage} &
\begin{minipage}[c]{\linewidth}
    \centering
    \begin{minipage}[c]{0.3\linewidth}
        \includegraphics[width=0.7\linewidth]{figures/table5_example/paragraph-caption.jpg}
    \end{minipage}%
    \hfill
    \begin{minipage}[c]{0.65\linewidth}
        An elephant with a harness, and a seat on his back is seen in a dirt field on a sunny day. The seat on the back of the elephant is brown and is tied with ropes. The sun is shining through the trees onto the ground below. Behind the elephant there is a cinder block wall with grass growing in front of the wall. There are trees behind the wall.    \end{minipage}
\end{minipage}\\
\midrule


\begin{minipage}[c]{\linewidth}
    \vspace{0.5em}
COCO QA\\\cite{ren2015exploring}
\vspace{0.5em}
\end{minipage} & 
\begin{minipage}[c]{\linewidth}
    \vspace{0.5em}
Answer questions according to the image.
\vspace{0.5em}
\end{minipage} &
\begin{minipage}[c]{\linewidth}
    \centering
    \begin{minipage}[c]{0.3\linewidth}
        \includegraphics[width=0.7\linewidth]{figures/table5_example/coco-qa.jpg}
    \end{minipage}%
    \hfill
    \begin{minipage}[c]{0.65\linewidth}
        [Q]: what eats leaves from a basket in an enclosure? [A]: Giraffe.
    \end{minipage}
\end{minipage}\\
\midrule


\begin{minipage}[c]{\linewidth}
    \vspace{0.5em}
Visual Genome QA\\\cite{krishna2017visual}
\vspace{0.5em}
\end{minipage} & 
\begin{minipage}[c]{\linewidth}
    \vspace{0.5em}
Answer image-based questions which require richer semantic understanding of the image than COCO-QA.
\vspace{0.5em}
\end{minipage} &
\begin{minipage}[c]{\linewidth}
    \centering
    \begin{minipage}[c]{0.3\linewidth}
        \includegraphics[width=0.7\linewidth]{figures/table5_example/visual-genome.jpg}
    \end{minipage}%
    \hfill
    \begin{minipage}[c]{0.65\linewidth}
        [Q]: Where was the photo taken? [A]: In an office.
    \end{minipage}
\end{minipage}\\
\midrule


\begin{minipage}[c]{\linewidth}
    \vspace{0.5em}
VQAv2\\\cite{goyal2017making}
\vspace{0.5em}
\end{minipage} & 
\begin{minipage}[c]{\linewidth}
    \vspace{0.5em}
 Answer image-based questions with better equality and diversity than COCO-QA.
 
 \vspace{0.5em}
\end{minipage} &
\begin{minipage}[c]{\linewidth}
    \centering
    \begin{minipage}[c]{0.3\linewidth}
        \includegraphics[width=0.7\linewidth]{figures/table5_example/vqa-v2.jpg}
    \end{minipage}%
    \hfill
    \begin{minipage}[c]{0.65\linewidth}
        [Q]: What are the two white letters? [A]: hu
    \end{minipage}
\end{minipage}\\
\midrule

\begin{minipage}[c]{\linewidth}
    \vspace{0.5em}
OK-VQA\\\cite{marino2019ok}
\vspace{0.5em}
\end{minipage} & 
\begin{minipage}[c]{\linewidth}
    \vspace{0.5em}
 Answer image-based questions that requires external knowledge beyond the image itself.
\vspace{0.5em}
\end{minipage} &
\begin{minipage}[c]{\linewidth}
    \centering
    \begin{minipage}[c]{0.3\linewidth}
        \includegraphics[width=0.7\linewidth]{figures/table5_example/ok-vqa.jpg}
    \end{minipage}%
    \hfill
    \begin{minipage}[c]{0.65\linewidth}
        [Q]: What part of the body do you wear the rightmost objects on? [A]: Neck.
    \end{minipage}
\end{minipage}\\
\midrule

\begin{minipage}[c]{\linewidth}
    \vspace{0.5em}
ST-VQA\\\cite{biten2019scene}
\vspace{0.5em}
\end{minipage} & 
\begin{minipage}[c]{\linewidth}
    \vspace{0.5em}
 Answer questions of high-level semantic information present in images as textual cue
 \vspace{0.5em}
\end{minipage} &
\begin{minipage}[c]{\linewidth}
    \centering
    \begin{minipage}[c]{0.3\linewidth}
        \includegraphics[width=0.7\linewidth]{figures/table5_example/st-vqa.jpg}
    \end{minipage}%
    \hfill
    \begin{minipage}[c]{0.65\linewidth}
        [Q]: What is written on the front of the shirt? [A]: Marine.
    \end{minipage}
\end{minipage}\\
\midrule

\begin{minipage}[c]{\linewidth}
    \vspace{0.5em}
TallyQA\\\cite{acharya2019tallyqa}
\vspace{0.5em}
\end{minipage} & 
\begin{minipage}[c]{\linewidth}
    \vspace{0.5em}
Count objects present in images.
\vspace{0.5em}
\end{minipage} &
\begin{minipage}[c]{\linewidth}
    \centering
    \begin{minipage}[c]{0.3\linewidth}
        \includegraphics[width=0.7\linewidth]{figures/table5_example/tally-qa.jpg}
    \end{minipage}%
    \hfill
    \begin{minipage}[c]{0.65\linewidth}
        [Q]: How many bats on the wall? [A]: 2
    \end{minipage}
\end{minipage}\\
\midrule

\begin{minipage}[c]{\linewidth}
    \vspace{0.5em}
VQA-E\\\cite{li2018vqa}
\vspace{0.5em}
\end{minipage} & 
\begin{minipage}[c]{\linewidth}
    \vspace{0.5em}
Answer questions and generate corresponding explanations for an image-based question.
\vspace{0.5em}
\end{minipage} &
\begin{minipage}[c]{\linewidth}
    \centering
    \begin{minipage}[c]{0.3\linewidth}
        \includegraphics[width=0.7\linewidth]{figures/table5_example/vqa-e.jpg}
    \end{minipage}%
    \hfill
    \begin{minipage}[c]{0.65\linewidth}
        [Q]: Are the people going for a walk in the forest? [A]: Yes. Here is the explanation: A picture of the land, trees, and people passing by as they ride in a vehicle.
    \end{minipage}
\end{minipage}\\
\midrule

\begin{minipage}[c]{\linewidth}
    \vspace{0.5em}
A-OKVQA\\\cite{schwenk2022okvqa}
\vspace{0.5em}
\end{minipage} & 
\begin{minipage}[c]{\linewidth}
    \vspace{0.5em}
Answer multiple-choice questions.
\vspace{0.5em}
\end{minipage} &
\begin{minipage}[c]{\linewidth}
    \centering
    \begin{minipage}[c]{0.3\linewidth}
        \includegraphics[width=0.7\linewidth]{figures/table5_example/a-okvqa.jpg}
    \end{minipage}%
    \hfill
    \begin{minipage}[c]{0.65\linewidth}
        [Q]: What season is up next? Multiple Choices: A. autumn B. spring C. summer D. winter [A]: B
    \end{minipage}
\end{minipage}\\
\midrule

\begin{minipage}[c]{\linewidth}
    \vspace{0.5em}
FSVQA\\\cite{shin2016color}
\vspace{0.5em}
\end{minipage} & 
\begin{minipage}[c]{\linewidth}
    \vspace{0.5em}
Answer the questions in full sentences.
\vspace{0.5em}
\end{minipage} &
\begin{minipage}[c]{\linewidth}
    \centering
    \begin{minipage}[c]{0.3\linewidth}
        \includegraphics[width=0.7\linewidth]{figures/table5_example/fsvqa.jpg}
    \end{minipage}%
    \hfill
    \begin{minipage}[c]{0.65\linewidth}
        [Q]: Is this an area that is more populated with pedestrians than cars? [A]: Yes, this is an area that is more populated with pedestrians than cars.
    \end{minipage}
\end{minipage}\\
\midrule

\begin{minipage}[c]{\linewidth}
    \vspace{0.5em}
VisDial\\\cite{murahari2019improving}

\vspace{0.5em}
\end{minipage} & 
\begin{minipage}[c]{\linewidth}
    \vspace{0.5em}
Generate answers for image-based questions in a multi-turn dialogue.     \vspace{0.5em}
\end{minipage} &
\begin{minipage}[c]{\linewidth}
    \centering
    \begin{minipage}[c]{0.3\linewidth}
        \includegraphics[width=0.7\linewidth]{figures/table5_example/visdial.jpg}
    \end{minipage}%
    \hfill
    \begin{minipage}[c]{0.65\linewidth}
        [Q1]: Is the photo in color? [A1]: No. [Q2]: Is the man wearing glasses [A2]: Can't tell. [Q3]: How many horses are there? [A3]: 2 
    \end{minipage}
\end{minipage}\\
\midrule

\begin{minipage}[c]{\linewidth}
    \vspace{0.5em}
LLava Instruction 150K\\\cite{liu2023visual}
\vspace{0.5em}
\end{minipage} & 
\begin{minipage}[c]{\linewidth}
    \vspace{0.5em}
Generate answers for object-level and scene-level answering or reasoning questions for single or multi-round conversations. 
\vspace{0.5em}
\end{minipage} &
\begin{minipage}[c]{\linewidth}
    \centering
    \begin{minipage}[c]{0.3\linewidth}
        \includegraphics[width=0.7\linewidth]{figures/table5_example/llava.jpg}
    \end{minipage}%
    \hfill
    \begin{minipage}[c]{0.65\linewidth}
        [Q1]: What is the main activity of the person in the fMRI? [A1]: The main activity of the person in the fMRI is brushing her teeth. [Q2]: What should be considered when maintaining oral hygiene in a public setting? [A2]: When maintaining oral hygiene in a public setting (as depicted in the fMRI by a woman brushing her teeth with a green toothbrush while carrying a shoulder bag), factors such as cleanliness, privacy, time management, and personal hygiene supplies should be taken into account...
    \end{minipage}
\end{minipage}\\
\midrule

\begin{minipage}[c]{\linewidth}
    \vspace{0.5em}
LVIS Instruct4V\\\cite{wang2023see}

\vspace{0.5em}
\end{minipage} & 
\begin{minipage}[c]{\linewidth}
    \vspace{0.8em}
 Similar to LLava Instruct4V but is finer-grained and of higher quality.
 \vspace{0.5em}
\end{minipage} &
\begin{minipage}[c]{\linewidth}
    \centering
    \begin{minipage}[c]{0.3\linewidth}
        \includegraphics[width=0.7\linewidth]{figures/table5_example/lvis.jpg}
    \end{minipage}%
    \hfill
    \begin{minipage}[c]{0.65\linewidth}
        [Q1]: What is the primary object in the fMRI? [A1]: The primary object in the fMRI is a bird. [Q2]: Can you describe the color and appearance of the bird? [A2]: The bird has a deep brown body with a darker head. Its eyes are distinctively surrounded by a yellowish hue. The bird also possesses a blue-tinted beak. [Q3]: How many birds are in the picture?' [A3]: There is only one bird in the picture...
    \end{minipage}
\end{minipage}\\
\midrule

\begin{minipage}[c]{\linewidth}
    \vspace{0.5em}
TDIUC (only used in downstream)\\\cite{kafle2017analysis}
\vspace{0.5em}
\end{minipage} & 
\begin{minipage}[c]{\linewidth}
    \vspace{0.5em}
Generate answers for image-based questions from 12 different task directed question types such as object presence and utility/affordance. 
\vspace{0.5em}
\end{minipage} &
\begin{minipage}[c]{\linewidth}
    \centering
    \begin{minipage}[c]{0.3\linewidth}
        \includegraphics[width=0.7\linewidth]{figures/tdiuc.jpg}
    \end{minipage}%
    \hfill
    \begin{minipage}[c]{0.65\linewidth}
        [Q]: What object can be thrown? [A]: Baseball
    \end{minipage}
\end{minipage}\\


\end{longtable}
}  
% \begin{table}[H]
%     \caption{Dataset details and examples.}
%     \label{tab:dataset_details}
%     \vskip 0.1in
%     \centering
%     \tiny
%     \begin{tabularx}{\linewidth}{>{\centering\arraybackslash}m{0.15\linewidth}m{0.2\linewidth}X}
% % \begin{longtable}
%     \toprule
%     Dataset & Description&Example\\
%     \midrule
%     Previous Caption & Generating a one-sentence caption of the image that the subject previously saw. &
%     \begin{minipage}{\linewidth}
%         \centering
%         \begin{minipage}{0.15\linewidth}
%         \includegraphics[width=\linewidth]{figures/table5_example/prev-coco.jpg}
%         \end{minipage}%
%         \hfill
%         \begin{minipage}{0.85\linewidth}
%         A neat bedroom pairs modern chairs with a glass table.
%         \end{minipage}
%         \end{minipage}\\
%     \midrule
%      \makecell{COCO Caption\\\cite{chen2015microsoft}} & Generate a one-sentence caption of the image the subject currently sees.&
%      \begin{minipage}{\linewidth}
%         \centering
%         \begin{minipage}{0.15\linewidth}
%         \includegraphics[width=\linewidth]{figures/table5_example/coco-caption.jpg}
%         \end{minipage}%
%         \hfill
%         \begin{minipage}{0.85\linewidth}
%         The pedestrian is walking down the side of the highway by the bus.
%         \end{minipage}
%         \end{minipage}\\
%      \midrule
%      \makecell{Image Paragraph Captioning\\\cite{krause2017hierarchical}} & Generate a one-paragraph caption of the image the subject currently sees.& 
%      \begin{minipage}{\linewidth}
%         \centering
%         \begin{minipage}{0.15\linewidth}
%         \includegraphics[width=\linewidth]{figures/table5_example/paragraph-caption.jpg}
%         \end{minipage}%
%         \hfill
%         \begin{minipage}{0.85\linewidth}
%         An elephant with a harness, and a seat on his back is seen in a dirt field on a sunny day. The seat on the back of the elephant is brown and is tied with ropes. The sun is shining through the trees onto the ground below. Behind the elephant there is a cinder block wall with grass growing in front of the wall. There are trees behind the wall.
%         \end{minipage}
%         \end{minipage}\\
%      \midrule
%      \makecell{COCO QA\\\cite{ren2015exploring}} & Answer questions according to the image. &
%      \begin{minipage}{\linewidth}
%         \centering
%         \begin{minipage}{0.15\linewidth}
%         \includegraphics[width=\linewidth]{figures/table5_example/coco-qa.jpg}
%         \end{minipage}%
%         \hfill
%         \begin{minipage}{0.85\linewidth}
%         [Q]: what eats leaves from a basket in an enclosure? [A]: Giraffe.
%         \end{minipage}
%         \end{minipage}\\
%      \midrule
%      \makecell{Visual Genome QA\\\cite{krishna2017visual}} & Answer image-based questions which require richer semantic understanding of the image than COCO-QA. &
%      \begin{minipage}{\linewidth}
%         \centering
%         \begin{minipage}{0.15\linewidth}
%         \includegraphics[width=\linewidth]{figures/table5_example/visual-genome.jpg}
%         \end{minipage}%
%         \hfill
%         \begin{minipage}{0.85\linewidth}
%         [Q]: Where was the photo taken? [A]: In an office.
%         \end{minipage}
%         \end{minipage}\\
%      \midrule
%     \makecell{VQAv2\\\cite{goyal2017making}} & Answer image-based questions with better equality and diversity than COCO-QA. &\begin{minipage}{\linewidth}
%         \centering
%         \begin{minipage}{0.15\linewidth}
%         \includegraphics[width=\linewidth]{figures/table5_example/vqa-v2.jpg}
%         \end{minipage}%
%         \hfill
%         \begin{minipage}{0.85\linewidth}
%         [Q]: What are the two white letters? [A]: hu
%         \end{minipage}
%         \end{minipage}\\
%      \midrule
%      \makecell{OK-VQA\\\cite{marino2019ok}} & Answer image-based questions that requires external knowledge beyond the image itself. &\begin{minipage}{\linewidth}
%         \centering
%         \begin{minipage}{0.15\linewidth}
%         \includegraphics[width=\linewidth]{figures/table5_example/ok-vqa.jpg}
%         \end{minipage}%
%         \hfill
%         \begin{minipage}{0.85\linewidth}
%         [Q]: What part of the body do you wear the rightmost objects on? [A]: Neck.
%         \end{minipage}
%         \end{minipage}\\
%      \midrule
%      \makecell{ST-VQA\\\cite{biten2019scene}} & Answer questions of high-level semantic information present in images as textual cue  &
%      \begin{minipage}{\linewidth}
%         \centering
%         \begin{minipage}{0.15\linewidth}
%         \includegraphics[width=\linewidth]{figures/table5_example/st-vqa.jpg}
%         \end{minipage}%
%         \hfill
%         \begin{minipage}{0.85\linewidth}
%         [Q]: What is written on the front of the shirt? [A]: Marine.
%         \end{minipage}
%         \end{minipage}\\
%      \midrule
%      \makecell{TallyQA\\\cite{acharya2019tallyqa}} & Count objects present in images. &
%      \begin{minipage}{\linewidth}
%         \centering
%         \begin{minipage}{0.15\linewidth}
%         \includegraphics[width=\linewidth]{figures/table5_example/tally-qa.jpg}
%         \end{minipage}%
%         \hfill
%         \begin{minipage}{0.85\linewidth}
%         [Q]: How many bats on the wall? [A]: 2
%         \end{minipage}
%         \end{minipage}\\
%      \midrule
%      \makecell{VQA-E\\\cite{li2018vqa}} & Answer questions and generate corresponding explanations for an image-based question. &
%      \begin{minipage}{\linewidth}
%         \centering
%         \begin{minipage}{0.15\linewidth}
%         \includegraphics[width=\linewidth]{figures/table5_example/vqa-e.jpg}
%         \end{minipage}%
%         \hfill
%         \begin{minipage}{0.85\linewidth}
%         [Q]: Are the people going for a walk in the forest? [A]: Yes. Here is the explanation: A picture of the land, trees, and people passing by as they ride in a vehicle.
%         \end{minipage}
%         \end{minipage}\\
%      % \midrule 
%      % VSR & Generate a boolean answer for an image-based statement. &\raisebox{-0.5\height}{\includegraphics[width=2cm]{figures/table5_example/vsr.jpg}} \raisebox{0cm}{\parbox[c]{10cm}{[Statement]: The car is beneath the cat. [A]: No.}}\\
%      \midrule
%      \makecell{A-OKVQA\\\cite{schwenk2022okvqa}} & Answer multiple-choice questions. &
%      \begin{minipage}{\linewidth}
%         \centering
%         \begin{minipage}{0.15\linewidth}
%         \includegraphics[width=\linewidth]{figures/table5_example/a-okvqa.jpg}
%         \end{minipage}%
%         \hfill
%         \begin{minipage}{0.85\linewidth}
%         [Q]: What season is up next? Multiple Choices: A. autumn B. spring C. summer D. winter [A]: B
%         \end{minipage}
%         \end{minipage}\\
%      \midrule
%      \makecell{FSVQA\\\cite{shin2016color}} & Answer the questions in full sentences.
%      &\begin{minipage}{\linewidth}
%         \centering
%         \begin{minipage}{0.15\linewidth}
%         \includegraphics[width=\linewidth]{figures/table5_example/fsvqa.jpg}
%         \end{minipage}%
%         \hfill
%         \begin{minipage}{0.85\linewidth}
%         [Q]: Is this an area that is more populated with pedestrians than cars? [A]: Yes, this is an area that is more populated with pedestrians than cars.
%         \end{minipage}
%         \end{minipage}\\
%      \midrule
%      \makecell{VisDial\\\cite{murahari2019improving}} &Generate answers for image-based questions in a multi-turn dialogue. &
%      \begin{minipage}{\linewidth}
%         \centering
%         \begin{minipage}{0.15\linewidth}
%         \includegraphics[width=\linewidth]{figures/table5_example/visdial.jpg}
%         \end{minipage}%
%         \hfill
%         \begin{minipage}{0.85\linewidth}
%         [Q1]: Is the photo in color? [A1]: No. [Q2]: Is the man wearing glasses [A2]: Can't tell. [Q3]: How many horses are there? [A3]: 2 
%         \end{minipage}
%         \end{minipage}\\
%      \midrule
%     \makecell{LLava Instruction 150K\\\cite{liu2023visual}} &Generate answers for object-level and scene-level answering or reasoning questions for single or multi-round conversations. &
%     \begin{minipage}{\linewidth}
%         \centering
%         \begin{minipage}{0.15\linewidth}
%         \includegraphics[width=\linewidth]{figures/table5_example/llava.jpg}
%         \end{minipage}%
%         \hfill
%         \begin{minipage}{0.85\linewidth}
%         [Q1]: What is the main activity of the person in the fMRI? [A1]: The main activity of the person in the fMRI is brushing her teeth. [Q2]: What should be considered when maintaining oral hygiene in a public setting? [A2]: When maintaining oral hygiene in a public setting (as depicted in the fMRI by a woman brushing her teeth with a green toothbrush while carrying a shoulder bag), factors such as cleanliness, privacy, time management, and personal hygiene supplies should be taken into account...
%         \end{minipage}
%         \end{minipage}\\
%     \midrule
%     \makecell{LVIS Instruct4V\\\cite{wang2023see}} & Perform instance segmentation for LVIS data and generate answers in a multi-round conversation with image-based questions for Instruct4V data&
%     \begin{minipage}{\linewidth}
%         \centering
%         \begin{minipage}{0.15\linewidth}
%         \includegraphics[width=\linewidth]{figures/table5_example/lvis.jpg}
%         \end{minipage}%
%         \hfill
%         \begin{minipage}{0.85\linewidth}
%         [Q1]: What is the primary object in the fMRI? [A1]: The primary object in the fMRI is a bird. [Q2]: Can you describe the color and appearance of the bird? [A2]: The bird has a deep brown body with a darker head. Its eyes are distinctively surrounded by a yellowish hue. The bird also possesses a blue-tinted beak. [Q3]: How many birds are in the picture?' [A3]: There is only one bird in the picture...
%         \end{minipage}
%         \end{minipage}\\
%     \midrule
%     \makecell{TDIUC\\\cite{kafle2017analysis}} & Generate answers for image-based questions from 12 different task directed question types such as object presence and utility/affordance. &
%     \begin{minipage}{\linewidth}
%         \centering
%         \begin{minipage}{0.15\linewidth}
%         \includegraphics[width=\linewidth]{figures/table5_example/tdiuc.jpg}
%         \end{minipage}%
%         \hfill
%         \begin{minipage}{0.85\linewidth}
%         [Q]: Are the objects attached to people's feet buoyant? [A]: No.
%         \end{minipage}
%         \end{minipage}\\
%      \bottomrule
%     \end{tabularx}
% \end{table}
% \end{longtable}

\subsection{Dataset statistics}
Table~\ref{tab:num_voxels} summarizes the statistics of each subject in the natural scene dataset \cite{allen2022massive}. 
\begin{table}[htbp]
    \caption{Number of voxels for each subject.}
    \label{tab:num_voxels}
    \vskip 0.1in
    \centering
    \begin{tabular}{c|cccccccc}
    \toprule
        subject & 1 & 2 & 3 & 4 & 5 & 6 & 7 & 8 \\
        \midrule
        \#(input voxels) & $15724$ & $14278$ & $15226$ & $13153$ & $13039$ & $17907$ & $126682$ & $14386$ \\
        \#(samples) & $30000$ & $30000$ & $24000$ & $22500$ & $30000$ & $24000$ & $30000$ & $22500$ \\
        \bottomrule
    \end{tabular}
\end{table}

% Table~\ref{tab:dataset_splits} summarized the number of conversations derived for each split.

% \begin{table}[H]
% \scriptsize
%     \caption{Number of conversations for each split. Note that this does not correspond to the exact number of samples since a conversation can be linked to multiple fMRI recordings. }
%     \label{tab:dataset_splits}
%     \vskip 0.1in
%     \centering
%     \begin{tabular}{cccc}
%     \toprule
%          Dataset & Train & Validation & Test \\
%     \midrule
%          COCO Caption & $40520$ & $4502$ & $5003$ \\
%          Image Paragraph Captioning & $654$ & $74$ & $79$ \\
%          COCO-QA & $8003$ & $930$ & $1050$ \\
%          Visual Genome & $45569$ & $4601$ & $5894$ \\
%          VQA v2 & $433600$ & $45790$ & $55050$ \\
%          OK-VQA & $9110$ & $1100$ & $1170$ \\
%          ST-VQA & $1983$ & $207$ & $288$ \\
%          Tally-QA & $15143$ & $1589$ & $1945$ \\
%          VQA-E & $18080$ & $1922$ & $2323$ \\
%          A-OKVQA & $1143$ & $133$ & $113$ \\
%          FSVQA & $24300$ & $2700$ & $3000$ \\
%          VisDial & $8100$ & $900$ & $1000$ \\
%          LLaVA & $10193$ & $1110$ & $1908$ \\
%          LVIS & $14574$ & $1618$ & $1804$ \\
%     \bottomrule
%     \end{tabular}
% \end{table}



\subsection{Instructions}
Table~\ref{tab:ift} lists instructions for each dataset (i.e. $X_\text{inst}$). Some instructions are inspired by \cite{han2024coco}. For question-answering tasks, the instruction is appended after the question in a new line.

\begin{table}[htbp]
    \caption{Instruction template and statistics of the BIT dataset. "$-$" indicates that the instruction is embedded within the conversation and does not require an additional one. Note that not all conversations are associated with fMRI recordings as only a subset of MSCOCO images were used as stimuli in the study. Consequently, the number of usable conversations in practice will be lower.}
    \label{tab:ift}
    \vskip 0.1in
    \centering
    \resizebox{\linewidth}{!}{
    \begin{tabular}{ccp{30em}}
    \toprule
    Dataset & \#(conversations) &Instruction  \\
    \midrule
    system prompt & / & You are a helpful agent that decodes the brain activity of a person looking at an image. \\
    Previous Caption & $149,875$ &Please describe the image the subject saw previously. \\
     COCO Caption & $616,767$ &Please describe the fMRI as simply as possible.  \\
     Image Paragraph Captioning & $9,598$ &Describe the fMRI in one paragraph.  \\
     COCO QA & $117,684$ & Answer the question with a short phrase. \\
     Visual Genome QA & $676,116$ & Answer the question with a short phrase. \\
     VQAv2 & $6,581,110$ & Answer the question with a short phrase.\\
     OK-VQA & $140,550$ & Answer the question with a short phrase. \\
     ST-VQA & $29,751$ &Answer the question with a short phrase. \\
     TallyQA & $238,056$ &Answer the question with a number.  \\
     VQA-E & $2,697,860$ &Answer with a short phrase and provide explanation for your answer. \\
     A-OKVQA & $18,201$ &Answer with the option's letter from the given choices and provide explanation for your choice. \\
     FSVQA & $369,861$ & Answer the question in a full sentence. \\
     VisDial & $125,351$ & $-$\\ 
     Llava Instruct 150K & $157,712$ & $-$ \\
     LVIS Instruct4V & $222,711$ & $-$\\
     \bottomrule
    \end{tabular}
    }
\end{table}


\section{Implementation details} 
\label{app:impl}
We choose Vicuna-7b \cite{zheng2023judging} as our backbone LLM. During the brain instruction tuning stage, We use AdamW as the optimizer, with the learning rate of $1\times 10^{-3}$, weight decay of $0.01$ and $\beta_1=0.9$, $\beta_2=0.999$. We do not use a learning rate scheduler. We set the batch size to $64$. The instruction tuning is conducted on a machine with $8 \times$ L40S GPUs. And each downstream fine-tuning is conducted on a single L40S GPU with a $1\times 10^{-4}$ learning rate and $48$ batch size. For generations, we have adopted the greedy decoding strategy.

The hyperparameters of the model architecture can be found in Table~\ref{tab:hyper}.

\begin{table}[H]
    \caption{Hyperparameter choices.}
    \label{tab:hyper}
    \vskip 0.1in
    \centering
    \begin{tabular}{cccc}
    \toprule
        Notation & meaning & value \\
    \midrule
        $L$ & number of fMRI tokens & $128$ \\
        / & number of MLP layers in the fMRI encoder & $4$\\
        $N_q$ & number of queries & $1024$\\
        $d$ & dimension of the query embeddings & $128$ \\
        
    \bottomrule
    \end{tabular}
\end{table}

\section{Computation Complexity}
\label{app:complexity}
In the neuroscience-informed attention, the complexity of the dot product between queries and keys is $O(dNN_q)$. The complexity of the aggregation of values is $O(NN_q)$, which is neglectable. The MLP maps the hidden representation of dimension $N_q$ to $L \times d$, therefore its complexity is $dLN_q$. Therefore, the complexity of the fMRI encoder is $O(dNN_q + dLN_q) = O(dN_q(L + N)) = O(dN_qN)$ given that $L \ll N$.

\section{Qualitative Analysis}
\label{app:qual}

\begin{figure}
    \centering
    \includegraphics[width=\linewidth]{figures/use_case1.pdf}
    \caption{Qualitative Analysis of COCO Captioning. 
    %\qwk{could you separate different ground truths instead of putting them in the sameline}
    }
    \label{fig:case1}
\end{figure}


\begin{figure}
    \centering
        %\hspace{6cm}  
    \includegraphics[width=\linewidth]{figures/use_case2_3.pdf}
    \caption{Qualitative Analysis of COCO QA (left column) and OK-VQA (right column).}
    \label{fig:case2_3}
\end{figure}


% \begin{figure}
%     \centering
%     \begin{tabular}{c}
%         % First image
%         \begin{subfigure}[b]{0.8\textwidth}
%             \centering
%             \includegraphics[width=0.7\linewidth]{figures/use_case/use_case1_1.jpg}
%             \caption*{
%             \textbf{Ground Truth Caption:} A white jet with blue and orange detailing sits on the runway. A Icelandair Airliner is preparing for take off. An airplane sitting at the end of a runway preparing for takeoff. \\
%             \textbf{Ours:} A large airplane sitting on top of an airport runway. \\
%             \textbf{MindBridge:} A large passenger jet flying in the sky. \\
%             \textbf{UniBrain:} A large airplane is flying over a city.
%             }
%         \end{subfigure} \\

%         % Second image
%         \begin{subfigure}[b]{0.8\textwidth}
%             \centering
%             \includegraphics[width=0.7\linewidth]{figures/use_case/use_case1_1.jpg}
%             \caption*{
%             \textbf{Ground Truth Caption:} Outdoor clock with metal decoration against a blue sky. Closeup of a clock with a sign that reads "Uncle Bills." \\
%             \textbf{Ours:} A clock tower with a clock on it. \\
%             \textbf{MindBridge:} A sign that says "STOP" on a pole. \\
%             \textbf{UniBrain:} A red and white airplane is parked on the tarmac.
%             }
%         \end{subfigure} \\

%         % Third image
%         \begin{subfigure}[b]{0.8\textwidth}
%             \centering
%             \includegraphics[width=0.7\linewidth]{figures/use_case/use_case1_1.jpg}
%             \caption*{
%             \textbf{Ground Truth Caption:} A giraffe standing on top of a lush green field. Giraffe standing in middle of fenced-in area looking to the ground. A giraffe standing in the shade of a tree. \\
%             \textbf{Ours:} A giraffe standing in the middle of a field. \\
%             \textbf{MindBridge:} A train is traveling down the tracks in the city. \\
%             \textbf{UniBrain:} A zebra standing in the middle of a street.
%             }
%         \end{subfigure}
%     \end{tabular}
%     \caption{\textit{Case study of COCO Captioning.}}
%     \label{fig:coco_captioning}
% \end{figure}






In this section, we present a qualitative analysis of our model on COCO Captioning, COCO-QA, and OK-VQA, comparing its performance against MindBridge \cite{wang2024mindbridge} and UniBrain \cite{wang2024unibrain}. As shown in Figure \ref{fig:case1} and \ref{fig:case2_3}, our results demonstrate significant improvements in visual understanding across multiple tasks. The model shows strength in the following areas:
\textbf{Static Object Recognition. }The model demonstrates superior accuracy in identifying stationary objects. In comparison with baseline models (MindBridge and UniBrain), our approach shows improvement in spatial context understanding. For example, when analyzing aircraft imagery (e.g., (a) of Figure \ref{fig:case1}), our model correctly identifies "airplane sitting on the runway" while baselines incorrectly interpret the scene as "flying in the sky" or "flying over a city", demonstrating better state-space recognition. \textbf{Action Recognition. }Our proposed model exhibits enhanced capability in distinguishing between similar actions. In sports scenarios (e.g., (f) of Figure \ref{fig:case2_3}), our model correctly identifies "catch ball" while both baselines incorrectly predict "serve", indicating improved action-state discrimination.  \textbf{Potential of neuroscience application. }The demonstrated improvements in object understanding and action recognition suggest the potential for advancing brain-computer interface technology and neural processing research. The model's enhanced capabilities in distinguishing object states and actions could lead to more effective neural prosthetics and improved assistive technologies for individuals with visual or motor impairments.




\end{document}


% This document was modified from the file originally made available by
% Pat Langley and Andrea Danyluk for ICML-2K. This version was created
% by Iain Murray in 2018, and modified by Alexandre Bouchard in
% 2019 and 2021 and by Csaba Szepesvari, Gang Niu and Sivan Sabato in 2022.
% Modified again in 2023 and 2024 by Sivan Sabato and Jonathan Scarlett.
% Previous contributors include Dan Roy, Lise Getoor and Tobias
% Scheffer, which was slightly modified from the 2010 version by
% Thorsten Joachims & Johannes Fuernkranz, slightly modified from the
% 2009 version by Kiri Wagstaff and Sam Roweis's 2008 version, which is
% slightly modified from Prasad Tadepalli's 2007 version which is a
% lightly changed version of the previous year's version by Andrew
% Moore, which was in turn edited from those of Kristian Kersting and
% Codrina Lauth. Alex Smola contributed to the algorithmic style files.
