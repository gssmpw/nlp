%%%%%%%% ICML 2025 EXAMPLE LATEX SUBMISSION FILE %%%%%%%%%%%%%%%%%

\documentclass{article}

% Recommended, but optional, packages for figures and better typesetting:
\usepackage{microtype}
\usepackage{graphicx}
% \usepackage{subfigure}
\usepackage[dvipsnames]{xcolor}
\usepackage{booktabs} % for professional tables
\usepackage{float}
\usepackage{multirow}
\usepackage{multicol}
\usepackage{enumitem}
\usepackage{makecell}
\usepackage{siunitx}
\usepackage{graphicx}
\usepackage{subcaption}
\usepackage[normalem]{ulem}
\usepackage{caption}
\usepackage{stfloats}
\usepackage{longtable}
\usepackage{array}
\usepackage{makecell}

\sisetup{
  table-format=2.2,  % Specify the format: 3 digits before and 2 after the decimal
  table-align-text-pre=false,  % Center-align numbers in their cells
  table-number-alignment=center  % Center-align numbers horizontally
}
\newcolumntype{M}{>{$}c<{$}}
% hyperref makes hyperlinks in the resulting PDF.
% If your build breaks (sometimes temporarily if a hyperlink spans a page)
% please comment out the following usepackage line and replace
% \usepackage{icml2025} with \usepackage[nohyperref]{icml2025} above.
\usepackage{hyperref}


% Attempt to make hyperref and algorithmic work together better:
\newcommand{\theHalgorithm}{\arabic{algorithm}}

% Use the following line for the initial blind version submitted for review:
% \usepackage{icml2025}

% If accepted, instead use the following line for the camera-ready submission:
\usepackage[accepted]{icml2025}

% For theorems and such
\usepackage{amsmath}
\usepackage{amssymb}
\usepackage{mathtools}
\usepackage{amsthm}
\usepackage{pifont}
\newcommand{\cmark}{\ding{51}}%
\newcommand{\xmark}{\ding{55}}%
\newcommand{\tdag}{\textsuperscript{\textdagger}}
\newcommand{\name}{MindLLM}
\usepackage{subcaption}
% if you use cleveref..
\usepackage[capitalize,noabbrev]{cleveref}
\usepackage{tabularx}
\usepackage{adjustbox}
\usepackage{makecell}
\usepackage[most]{tcolorbox}

%%%%%%%%%%%%%%%%%%%%%%%%%%%%%%%%
% THEOREMS
%%%%%%%%%%%%%%%%%%%%%%%%%%%%%%%%
\theoremstyle{plain}
\newtheorem{theorem}{Theorem}[section]
\newtheorem{proposition}[theorem]{Proposition}
\newtheorem{lemma}[theorem]{Lemma}
\newtheorem{corollary}[theorem]{Corollary}
\theoremstyle{definition}
\newtheorem{definition}[theorem]{Definition}
\newtheorem{assumption}[theorem]{Assumption}
\theoremstyle{remark}
\newtheorem{remark}[theorem]{Remark}


\newcommand{\rex}[1]{\textcolor{magenta}{(Rex: #1)}}
\newcommand{\yy}[1]{\textcolor{orange}{(Yujun: #1)}}
\newcommand{\qwk}[1]{\textcolor{blue}{(Weikang: #1)}}

% Todonotes is useful during development; simply uncomment the next line
%    and comment out the line below the next line to turn off comments
%\usepackage[disable,textsize=tiny]{todonotes}
\usepackage[textsize=tiny]{todonotes}

\newcommand{\papertitle}{\name{}: A Subject-Agnostic and Versatile Model for fMRI-to-Text Decoding}

% The \icmltitle you define below is probably too long as a header.
% Therefore, a short form for the running title is supplied here:
\icmltitlerunning{\papertitle{}}

\begin{document}

\twocolumn[
\icmltitle{\raisebox{-15pt}{\includegraphics[width=0.08\textwidth]{figures/LOGO.png}}\name{}:\\A Subject-Agnostic and Versatile Model for fMRI-to-Text Decoding}

% It is OKAY to include author information, even for blind
% submissions: the style file will automatically remove it for you
% unless you've provided the [accepted] option to the icml2025
% package.

% List of affiliations: The first argument should be a (short)
% identifier you will use later to specify author affiliations
% Academic affiliations should list Department, University, City, Region, Country
% Industry affiliations should list Company, City, Region, Country

% You can specify symbols, otherwise they are numbered in order.
% Ideally, you should not use this facility. Affiliations will be numbered
% in order of appearance and this is the preferred way.
\icmlsetsymbol{equal}{*}

\begin{icmlauthorlist}
\icmlauthor{Weikang Qiu}{yale}
\icmlauthor{Zheng Huang}{equal,dartmouth}
\icmlauthor{Haoyu Hu}{equal,cambridge}
\icmlauthor{Aosong Feng}{yale}
\icmlauthor{Yujun Yan}{dartmouth}
\icmlauthor{Rex Ying}{yale}
%\icmlauthor{}{sch}
%\icmlauthor{}{sch}
%\icmlauthor{}{sch}
\end{icmlauthorlist}

\icmlaffiliation{yale}{Yale University}
\icmlaffiliation{dartmouth}{Dartmouth College}
\icmlaffiliation{cambridge}{University of Cambridge}

\icmlcorrespondingauthor{Weikang Qiu}{weikang.qiu@yale.edu}
% \icmlcorrespondingauthor{Firstname2 Lastname2}{first2.last2@www.uk}

% You may provide any keywords that you
% find helpful for describing your paper; these are used to populate
% the "keywords" metadata in the PDF but will not be shown in the document
\icmlkeywords{Machine Learning, ICML}

\vskip 0.3in
]

% this must go after the closing bracket ] following \twocolumn[ ...

% This command actually creates the footnote in the first column
% listing the affiliations and the copyright notice.
% The command takes one argument, which is text to display at the start of the footnote.
% The \icmlEqualContribution command is standard text for equal contribution.
% Remove it (just {}) if you do not need this facility.

%\printAffiliationsAndNotice{}  % leave blank if no need to mention equal contribution
\printAffiliationsAndNotice{\icmlEqualContribution} % otherwise use the standard text.

\begin{abstract}
Decoding functional magnetic resonance imaging (fMRI) signals into text has been a key challenge in the neuroscience community, with the potential to advance brain-computer interfaces and uncover deeper insights into brain mechanisms. However, existing approaches often struggle with suboptimal predictive performance, limited task variety, and poor generalization across subjects. In response to this, we propose \name{}, a model designed for subject-agnostic and versatile fMRI-to-text decoding. \name{} consists of an fMRI encoder and an off-the-shelf LLM. The fMRI encoder employs a neuroscience-informed attention mechanism, which is capable of accommodating subjects with varying input shapes and thus achieves high-performance subject-agnostic decoding. Moreover, we introduce Brain Instruction Tuning (BIT), a novel approach that enhances the model’s ability to capture diverse semantic representations from fMRI signals, facilitating more versatile decoding. We evaluate \name{} on comprehensive fMRI-to-text benchmarks. Results demonstrate that our model outperforms the baselines, improving downstream tasks by $12.0\%$, unseen subject generalization by $16.4\%$, and novel task adaptation by $25.0\%$. Furthermore, the attention patterns in \name{} provide interpretable insights into its decision-making process.

\end{abstract}

Large language models (LLMs) show significant performance in various downstream
tasks~\citep{brown_language_2020,openai_gpt-4_2024,dubey_llama_2024}. Studies
have found that training on high quality corpus improves the ability of LLMs
to solve different problems such as writing code, doing math exercises, and
answering logic questions~\citep{cai_internlm2_2024,deepseek-ai_deepseek-v3_2024,qwen_qwen25_2024}.
Therefore, effectively selecting high-quality text data is an important subject for
training LLM.

\begin{figure}[t]
    \centering
    \includegraphics[width=\linewidth]{figures/head.pdf}
    \caption{The overview of CritiQ. We (1) employ human annotators to annotate $\sim$30
    pairwise quality comparisons, (2) use CritiQ Flow to mine quality criteria, (3)
    use the derived criteria to annotate 25k pairs, and (4) train the CritiQ Scorer to
    perform efficient data selection.}
    \label{fig:overview}
\end{figure}

To select high-quality data from a large corpus, researchers manually design heuristics~\citep{dubey_llama_2024,rae_scaling_2022},
calculate perplexity using existing LLMs~\citep{marion2023moreinvestigatingdatapruning,wenzek2019ccnetextractinghighquality},
train classifiers~\citep{brown_language_2020,dubey_llama_2024,xie_data_2023} and
query LLMs for text quality through careful prompt engineering~\citep{gunasekar_textbooks_2023,wettig_qurating_2024,sachdeva_how_2024}.
Large-scale human annotation and prompt engineering require a lot of human
effort. Giving a comprehensive description of what high-quality data is like is also
challenging. As a result, manually designing heuristics lacks robustness and introduces
biases to the data processing pipeline, potentially harming model performance
and generalization. In addition, quality standards vary across different
domains. These methods can not be directly applied to other domains without significant
modifications.

To address these problems, we introduce CritiQ, a novel method to automatically
and effectively capture human preferences for data quality and perform efficient data
selection. Figure~\ref{fig:overview} gives an overview of CritiQ, comprising an agent
workflow, CritiQ Flow, and a scoring model, CritiQ Scorer. Instead of manually describing
how high quality is defined, we employ LLM-based agents to summarize quality
criteria from only $\sim$30 human-annotated pairs.

CritiQ Flow starts from a knowledge base of data quality criteria. The worker
agents are responsible to perform pairwise judgment under a given
criterion. The manager agent generates new criteria and refines them through reflection
on worker agents' performance. The final judgment is made by majority voting among
all worker agents, which gives a multi-perspective view of data quality.

To perform efficient data selection, we employ the worker agents to annotate a randomly
selected pairwise subset, which is ~1000x larger than the human-annotated one.
Following \citet{korbak_pretraining_2023,wettig_qurating_2024}, we train CritiQ
Scorer, a lightweight Bradley-Terry model~\citep{bradley_rank_1952} to convert
pairwise preferences into numerical scores for each text. We use CritiQ Scorer to
score the entire corpus and sample the high-quality subset.

For our experiments, we established human-annotated test sets to quantitatively
evaluate the agreement rate with human annotators on data quality preferences. We implemented the manager agent by \texttt{GPT-4o} and the worker
agent by \texttt{Qwen2.5-72B-Insruct}. We conducted experiments on different
domains including code, math, and logic, in which CritiQ Flow shows a consistent
improvement in the accuracies on the test sets, demonstrating the effectiveness
of our method in capturing human preferences for data quality. To validate the quality
of the selected dataset, we continually train \texttt{Llama 3.1}~\citep{dubey_llama_2024}
models and find that the models achieve better performance on downstream tasks
compared to models trained on the uniformly sampled subsets.

We highlight our contributions as follows. We will release the code to facilitate
future research.

\begin{itemize}
    \item We introduce CritiQ, a method that captures human preferences for data
        quality and performs efficient data selection at little cost of human
        annotation effort.

    \item Continual pretraining experiments show improved model performance in code,
        math, and logic tasks trained on our selected high-quality subset compared to the raw dataset.

    \item Ablation studies demonstrate the effectiveness of the knowledge base and
        the the reflection process.
\end{itemize}

\begin{figure*}[t]
    \centering
    \includegraphics[width=\linewidth]{figures/method.pdf}
    \caption{CritiQ Flow comprises two major components: multi-criteria pairwise
    judgment and the criteria evolution process. The multi-criteria pairwise
    judgment process employs a series of worker agents to make quality
    comparisons under a certain criterion. The criteria evolution process aims to
    obtain data quality criteria that highly align with human judgment through
    an iterative evolution. The initial criteria are retrieved from the
    knowledge base. After evolution, we select the final criteria to annotate
    the dataset for training CritiQ Scorer.}
    \label{fig:method}
\end{figure*}

\section{Related Works}
\noindent\textbf{Brain-Conditioned Text Generation}
\begin{figure}
    \centering
    \includegraphics[width=\linewidth]{figures/uni_compare.pdf}
    \vspace{-2em}
    \caption{Comparison between our model and previous unified models. MindBridge \cite{wang2024mindbridge} flattens the voxels and adaptively pools them to a fixed dimension, which overlooks the rich information in positions. UniBrain~ \cite{wang2024unibrain} uniformly samples a subset of voxels and aggregates their neighbors. Different from these methods, we propose neuroscience-informed attention, where each query token attends to all voxels, which minimizes potential information loss in pooling or sampling.}
    \label{fig:uni_compare}
    \vspace{-1em}
\end{figure}
%There have been works that decode text from brain signals. 
This line of research mostly focuses on decoding perceived visual stimuli into natural language from fMRI signals. MindGPT~\cite{chen2023mindgpt}, UniBrain\cite{mai2023unibrain} and BrainCap \cite{ferrante2023brain} employ an fMRI encoder guided by CLIP~\cite{radford2021learning} and use a language model \cite{radford2019language,wang2022git} to decode natural language from the encoded representations. BrainChat~\cite{huang2024brainchat} utilizes multiple pretraining strategies~\cite{devlin2018bert, he2022masked,yu2022coca} to align fMRI with image and text embeddings. These methods fall short in performance and versatility. UMBRAE~\cite{xia2024umbrae} proposes to learn a mapping from fMRI to stimulus images, which later serves as a proxy input for an off-the-shelf visual language model (VLM). Although they achieve performance improvements, the strategy prevents the model from performing tasks that are not directly related to the stimulus images (e.g. answering memory-related questions). In contrast, our end-to-end Brain Instruction Tuning (BIT) ensures seamless and versatile fMRI-to-text decoding, offering the potential to tackle tasks beyond vision-related ones.

\noindent\textbf{Cross-subjects Decoding}
In voxel-level machine learning for brain decoding, the number of voxels varies between subjects \cite{allen2022massive}. Most prior works \cite{scotti2024reconstructing,scotti2024mindeye2} use an MLP for each subject individually. However, due to the fixed input size required by MLP architectures, these models cannot handle varying input shapes. As illustrated in Figure~\ref{fig:uni_compare}, MindBridge \cite{wang2024mindbridge} proposed to use an adaptive max pooling layer to standardize the input shapes. However, unlike images, which are considered translation invariance, positions in fMRI carry specific bio-informative significance that pooling operations may overlook. UniBrain \cite{wang2024unibrain} proposed to sample groups of voxels. Such a sampling strategy, on the one hand, may lead to information loss if some voxels are not included in any group. On the other hand, the irregular spatial distribution of 3D voxels with varying density and curvature may result in underrepresentation or overrepresentation of certain areas. Different from these methods, our model employs a neuroscience-informed attention mechanism that accounts for every single voxel while preserving their bio-informative positional information, ensuring a more comprehensive and precise representation. 
% BrainChat trains the model based on the caption and VQA dataset. UMBRAE introduces a universal encoder that aligns fMRI signals with image stimuli.


\noindent\textbf{Multi-Modal Large Language Model}
Aiming to augment the perceptual capacities of Large Language Models (LLMs), there has been a growing interest in extending them to handle multiple modalities within a unified model. Numerous studies have attempted to incorporate modalities such as images \cite{alayrac2022flamingo,zhang2023internlm,wang2023cogvlm}, videos \cite{cheng2024videollama, kondratyuk2023videopoet,zhang2023video}, and point clouds \cite{xu2023pointllm,qi2025shapellm}. %Flamingo~\cite{alayrac2022flamingo} uses a frozen vision encoder and an LLM equipped with gated cross-attention for cross-modality alignment. PaLM-E~\cite{driess2023palm} integrates extracted visual features directly through linear layers into the pre-trained PaLM~\cite{chowdhery2023palm}. BLIP2~\cite{li2023blip} achieves vision-language alignment by introducing a querying transformer to extract visual features from the frozen image encoder and integrating them with the frozen LLM. 
OneLLM~\cite{han2024onellm} stands out by aligning eight different modalities, including fMRI, with language. However, their approach employs an individual convolution network for each subject instead of a unified architecture for fMRI encoding across subjects, which restricts its applicability to new subjects in real-world scenarios. Furthermore, the approach solely relies on captions as textual annotations, which limits the model's capability for versatile fMRI decoding. 

% \paragraph{Instruction Tuning}
% Instruction tuning \cite{zhang2023instruction, aw2023instruction, radford2019language, thoppilan2022lamda} has been widely used to enable LMs to follow specific natural language instructions. 
% A key method is Instruction Dataset Construction, where (instruction, output) pairs are generated from existing annotated datasets by applying templates to convert text-label pairs into instruction-output pairs \cite{longpre2023flan, sanh2021multitask, conover2023free}. For instance, Natural Instructions \cite{mishra2021cross} creates datasets comprising task descriptions with components like title, definition, cautions, prompts, and example pairs, while P3 \cite{sanh2021multitask} maps conventional NLP tasks, such as question answering or text classification, into natural language input-output pairs. In this work, we contribute to instruction tuning from a neuroscience perspective by introducing the Brain Instruction Tuning Dataset, which decodes brain activity into language representations.


\begin{figure}[t!]
    \centering
    \includegraphics[width=0.45\textwidth]{images/method_visual-crop.pdf}
    \caption{Process of Visual Attention Evaluation.}
    \label{img1}
\end{figure}

\subsection{API Prompting}
API Prompting~(sketched in Figure~\ref{api}) is a Visual Prompting method that highlights important parts in an image using a Visual Attention Heatmap derived from a Vision-Language Model~\citep{api}. The Attribution Map, representing the contribution of image tokens to model outputs, is extracted from a VLM (referred to as Heatmap VLM or H-VLM), convolved, resized to match the image size, and then overlaid on the original image.

Following the study by \citet{api}, Vision-Transformer-based CLIP and LLaVA are employed as Heatmap VLMs. The methods for extracting Visual Attention Attribution Maps from each model are described below.

\paragraph{CLIP Attribution Map}
CLIP computes similarity between text and image representations, and the Attribution Map \( \Psi \) is obtained by decomposing the similarity function \( \text{sim}(\hat{I}, \hat{T}) \). %The image representation \( \hat{I} \) is expressed as:
% \begin{equation}
%    \begin{aligned}
%        \hat{I} &= \mathcal{L}([Z^{0}_{\text{cls}}])
%         + \sum_{\ell=1}^{L} \mathcal{L}([\text{MSA}^{\ell}(Z^{\ell-1})]_{\text{cls}}) \\
%        &\quad + \sum_{\ell=1}^{L} \mathcal{L}([\text{MLP}^{\ell}(\hat{Z}^{\ell})]_{\text{cls}}).
%    \end{aligned}
% \end{equation}

Since later MSA layers greatly impact image representation~\citep{clipdec}, the similarity function is approximated as:
\begin{equation}
   \begin{aligned}
\text{sim}(\hat{I}, \hat{T}) \approx \text{sim}\left(\sum_
{\ell=L'}^{L} \mathcal{L}(\text{MSA}^{\ell}([Z^{\ell-1}]))_{\text{cls}}, \hat{T}\right).
\end{aligned}
\end{equation}
To filter out irrelevant regions, a complementary Attribution Map \( \Psi^{\text{comp}} \) is introduced:
\begin{equation}
   \begin{aligned}
   \Psi_{i,j}^{\text{comp}} &\triangleq 1 - \text{sim}(\mathcal{L}(Z_{t}^{L}), \hat{T}), \\
&\quad \text{where} \quad t = 1 + j + P \cdot (i - 1).
\end{aligned}
\end{equation}
Combining both maps, the final CLIP Attribution Map is defined as:
\begin{equation}
   \begin{aligned}
   \Psi = \Psi^{\text{cls}} + \Psi^{\text{comp}} - \Psi^{\text{comp}} \odot \Psi^{\text{cls}}.
\end{aligned}
\end{equation}

\paragraph{LLaVA Attribution Map}
LLaVA can provide an Attribution Map \( \Psi \) using Multi-Head Self-Attention (MSA) weights between output text tokens and image tokens. The Attribution Map is computed by averaging over all output tokens and attention heads:
\begin{equation}
   \begin{aligned}
\Psi_{i,j} &\triangleq \frac{1}{MH} 
  \sum_{m=1}^{M} 
  \sum_{h=1}^{H} 
  A_{m,t}^{(\bar{L},h)}, \\
&\quad \text{where} \quad 
t = j + P \cdot (i - 1).
\end{aligned}
\end{equation}
Here, \(M\) is the number of output tokens, \(H\) is the number of attention heads, \(P\) is the number of patches per image side, and \(A^{(\bar{L},h)}\) represents cross-attention between output text and image tokens at layer \(\bar{L}\) and attention head \(h\).

\subsection{Background Role Examination}
To assess the necessity of background information for object recognition, ground truth segmentation data is used as a Heatmap during API Prompting and the accuracy of output is evaluated (hereafter referred to as API - Seg.). Binary segmentation masks, overlaid in gray are input into VLMs to evaluate their impact on output accuracy. If POPE response accuracy remains unchanged, background information is deemed unnecessary.

\subsection{Minimum Cutoff}
Minimum cutoff redefines the minimum value in segmentation or Visual Attention Heatmap based on a threshold. Since a threshold of 0.5 showed improvement in Table~\ref{table2}, values below 0.5 in the cutoff are replaced with 0.5, refining segmentation granularity.

\begin{table*}[t]
\centering
\begin{tabular}{lllrrrrr}
\toprule
\textbf{Dataset} & \textbf{Model} & \textbf{Prompting} & \textbf{Acc.} & \textbf{Prec.} & \textbf{Rec.} & \textbf{TNR} & \textbf{F1} %& \textbf{Yes (\%)} 
\\ \cmidrule(lr){1-8}
\multirow{7}{*}{\textbf{MSCOCO}} & \multirow{7}{*}{LLaVA} & w/o prpt. & 86.23 & 84.21  & 89.19 & 83.27 & 86.63 \\ %& 52.96 \\
& & API~(CLIP) & \ensuremath{\blacktriangle}86.52 & \ensuremath{\blacktriangle}84.78 & \ensuremath{\triangledown}89.02 & \ensuremath{\blacktriangle}84.02 & \ensuremath{\blacktriangle}86.85 \\ %& \ensuremath{\blacktriangle}52.50  \\
& & API~(CLIP) w Cutoff & \ensuremath{\blacktriangle}88.59 & \ensuremath{\blacktriangle}85.84 & \ensuremath{\blacktriangle}92.43 & \ensuremath{\blacktriangle}84.75 & \ensuremath{\blacktriangle}89.01 \\
 & & API~(LLaVA) & \ensuremath{\triangledown}86.11 & \ensuremath{\blacktriangle}84.72 & \ensuremath{\triangledown}88.12 & \ensuremath{\blacktriangle}84.10 & \ensuremath{\triangledown}86.39 \\ %&  \ensuremath{\blacktriangle}52.01 \\
 & & API~(LLaVA) w Cutoff & \ensuremath{\blacktriangle}87.98 & \ensuremath{\blacktriangle}85.10 & \ensuremath{\blacktriangle}92.09 & \ensuremath{\blacktriangle}83.88 & \ensuremath{\blacktriangle}88.46 \\
  & & API - Seg. & - & - & \ensuremath{\triangledown}71.78 & - &  - \\ %& - \\
  & & API - Seg. w Cutoff & - & - & \ensuremath{\blacktriangle}89.24 & - &  - \\ %& - \\
 \bottomrule
\end{tabular}
\caption{POPE results on MSCOCO datasets with API Prompting.}
\label{table1}
\end{table*}

\begin{table*}[h!]
\centering
\begin{tabular}{llrrrr}
\toprule
%\toprule
\multirow{2}{*}{\textbf{H-VLM}} & \multirow{2}{*}{\textbf{Output}} & \multicolumn{4}{c}{\textbf{Visual Attention Alignment}} \\ %\cline{2-9} 
& & \textbf{Prec.} & \textbf{Rec.}  & \textbf{IoU}    & \textbf{MSE}   \\ 
%\cmidrule(lr){1-9}
\cmidrule(lr){1-6}
%Overall   & 0.1270   & 0.8318 & 0.1092 & 0.2920  & 0.1010   & 0.6523 & 0.0767 & 0.1337 \\ \hline
%\multirow{3}{*}{CLIP} & - & 12.70 & 83.18 & 10.92 & 29.20 \\
\multirow{2}{*}{CLIP}& Correct~(87\%)  &\ensuremath{\blacktriangle}13.52   & \ensuremath{\blacktriangle}83.94 & \ensuremath{\blacktriangle}11.68 & \ensuremath{\blacktriangle}28.46  \\
& Incorrect~(13\%) & 6.09   & 77.02 & 4.78 & 35.22 \\
%\multirow{3}{*}{LLaVA} & - & 10.10 & 65.23 & 7.67 & 13.37 \\
\multirow{2}{*}{LLaVA}& Correct~(86\%)  & \ensuremath{\blacktriangle}10.62  & 64.62 & \ensuremath{\blacktriangle}8.10 & 13.60 \\ 
& Incorrect~(14\%) & 6.17   & \ensuremath{\blacktriangle}69.78 & 4.44 & \ensuremath{\blacktriangle}11.60 \\
\bottomrule
%\bottomrule
\end{tabular}
\caption{Alignment of CLIP/LLaVA Visual Attention.}
\label{table3}
\end{table*}

\subsection{Evaluation of Visual Attention Alignment}
To assess how well visual attention focuses on target objects, Precision, Recall, Intersection over Union~(IoU), and Mean Squared Error~(MSE) are computed between the object segmentation data and the Visual Attention Heatmap as depicted in Figure~\ref{img1}. 
Visual Attention Heatmaps are converted into binary arrays using thresholds set to the average value of each heatmap.


% \begin{figure}[t]
%      \centering
%     \includegraphics[width=0.47\textwidth]{images/cutoff-crop.pdf}
%      \caption{Example of Cutoff Segmentation Annotation}    \label{clip}
%  \end{figure}
\section{Experiments}

\input{tables/few-shot-forecasting-5}
\input{tables/few-shot-forecasting-10}
\begin{table}[!h]
\renewcommand\arraystretch{1.2}
\begin{center}
\captionsetup{font=small}
\caption{\revision{Zero-shot learning results. Full results see \shortautoref{appx:zero-shot}.}}
\label{tab:zero-shot-forecasting-brief}
\vspace{-1em}
\begin{small}
\scalebox{0.56}{
\setlength\tabcolsep{2.5pt}
\begin{tabular}{c|cc|cc|cc|cc|cc|cc}
\toprule
\multicolumn{1}{c|}{\multirow{2}{*}{Methods}}
&\multicolumn{2}{c|}{\method\textcolor{green!60!black}{\textsubscript{\textbf{143M}}}}&\multicolumn{2}{c|}{Time-LLM\textcolor{orange}{\textsubscript{\textbf{3405M}}}}&\multicolumn{2}{c|}{LLMTime}&\multicolumn{2}{c|}{GPT4TS}&\multicolumn{2}{c|}{DLinear}&\multicolumn{2}{c}{PatchTST}\\

\multicolumn{1}{c|}{} & \multicolumn{2}{c}{\scalebox{0.99}{(\textbf{Ours})}} & 
\multicolumn{2}{|c|}{\scalebox{0.99}{\citeyearpar{jin2023time}}} &
\multicolumn{2}{c|}{\scalebox{0.99}{\citeyearpar{gruver2023large}}} &
\multicolumn{2}{c|}{\scalebox{0.99}{\citeyearpar{zhou2023one}}} & \multicolumn{2}{c|}{\scalebox{0.99}{\citeyearpar{zeng2023transformers}}} & \multicolumn{2}{c}{\scalebox{0.99}{\citeyearpar{nie2022time}}}  \\

\midrule

\multicolumn{1}{c|}{Metric} & MSE & MAE & MSE & MAE & MSE & MAE & MSE & MAE & MSE & MAE& MSE & MAE \\
\midrule
\multirow{1}{*}{\rotatebox{0}{$ETTh1$} $\rightarrow$ \rotatebox{0}{$ETTh2$}}  
& \boldres{0.338} & \boldres{0.385} & \secondres{0.353} & \secondres{0.387} & 0.992 & 0.708 & 0.406 & 0.422 & 0.493 & 0.488 & 0.380 & 0.405 \\
\midrule
\multirow{1}{*}{\rotatebox{0}{$ETTh1 $} $\rightarrow$ \rotatebox{0}{$ETTm2 $}}
& \secondres{0.293} & \secondres{0.350} & \boldres{0.273} & \boldres{0.340} & 1.867 & 0.869 & 0.325 & 0.363 & 0.415 & 0.452 & 0.314 & 0.360 \\
\midrule
\multirow{1}{*}{\rotatebox{0}{$ETTh2 $} $\rightarrow$ \rotatebox{0}{$ETTh1 $}}
& \secondres{0.496} & \secondres{0.480} & \boldres{0.479} & \boldres{0.474} & 1.961 & 0.981 & 0.757 & 0.578 & 0.703 & 0.574 & 0.565 & 0.513 \\
\midrule
\multirow{1}{*}{\rotatebox{0}{$ETTh2 $} $\rightarrow$ \rotatebox{0}{$ETTm2 $}}
& \secondres{0.297} & \secondres{0.353} & \boldres{0.272} & \boldres{0.341} & 1.867 & 0.869 & 0.335 & 0.370 & 0.328 & 0.386 & 0.325 & 0.365 \\
\midrule
\multirow{1}{*}{\rotatebox{0}{$ETTm1 $} $\rightarrow$ \rotatebox{0}{$ETTh2 $}}
& \boldres{0.354} & \boldres{0.397} & \secondres{0.381} & \secondres{0.412} & 0.992 & 0.708 & 0.433 & 0.439 & 0.464 & 0.475 & 0.439 & 0.438 \\
\midrule
\multirow{1}{*}{\rotatebox{0}{$ETTm1 $} $\rightarrow$ \rotatebox{0}{$ETTm2 $}}
& \boldres{0.264} & \boldres{0.319} & \secondres{0.268} & \secondres{0.320} & 1.867 & 0.869 & 0.313 & 0.348 & 0.335 & 0.389 & 0.296 & 0.334 \\
\midrule
\multirow{1}{*}{\rotatebox{0}{$ETTm2 $} $\rightarrow$ \rotatebox{0}{$ETTh2 $}}
& \secondres{0.359} & \boldres{0.399} & \boldres{0.354} & \secondres{0.400} & 0.992 & 0.708 & 0.435 & 0.443 & 0.455 & 0.471 & 0.409 & 0.425 \\
\midrule
\multirow{1}{*}{\rotatebox{0}{$ETTm2 $} $\rightarrow$ \rotatebox{0}{$ETTm1 $}}
& \secondres{0.432} & \boldres{0.426} & \boldres{0.414} & \secondres{0.438} & 1.933 & 0.984 & 0.769 & 0.567 & 0.649 & 0.537 & 0.568 & 0.492 \\

\bottomrule
\end{tabular}
}
\end{small}
\end{center}
\vspace{-2em}
\end{table}

\noindent\textbf{Datasets and Metrics.} We evaluate \method on seven widely-used time series datasets spanning diverse domains, including energy consumption (ETTh1, ETTh2, ETTm1, ETTm2), weather forecasting, electricity load prediction (ECL, 321 variables), and traffic flow estimation (Traffic, 862 variables)~\cite{zhou2021informer, lai2018modeling}. These datasets, extensively adopted for benchmarking long-term forecasting models~\cite{wu2022timesnet}, exhibit varying characteristics in sampling frequency, dimensionality, and temporal patterns. For short-term forecasting, we utilize the M4 benchmark~\citep{makridakis2018m4}, which encompasses marketing data at various sampling frequencies. Forecasting performance is evaluated using Mean Absolute Error (MAE) and Mean Squared Error (MSE), following standard practices in the field. Additional details on datasets and metrics are provided in Appendix~\ref{appx:dataset_details} and~\ref{appx:evaluation_metric}.

% \noindent\textbf{Baselines.} We compare \method with state-of-the-art time series models, including text-augmented methods like TimeLLM \citeyearpar{jin2023time}, GPT4TS \citeyearpar{zhou2023one}, and LLMTime \citeyearpar{gruver2023large}; vision-augmented methods like TimesNet \citeyearpar{wu2023timesnet}; traditional deep models like PatchTST \citeyearpar{nie2022time}, ESTformer \citeyearpar{woo2022etsformer}, Non-Stationary Transformer \citeyearpar{liu2022non}, FEDformer \citeyearpar{zhou2022fedformer}, Autoformer \citeyearpar{wu2021autoformer}, Informer \citeyearpar{zhou2021informer}, and Reformer \citeyearpar{kitaev2020reformer}; and recent competitive models like DLinear \citeyearpar{zeng2023transformers}, LightTS \citeyearpar{zhang2022less}, N-HiTS \citeyearpar{challu2023nhits}, and N-BEATS \citeyearpar{oreshkin2019n}. Performance results for some baselines are cited from \cite{liu2024time} where applicable.


\noindent\textbf{Baselines.} We compare \method with state-of-the-art time series models, including text-augmented methods like TimeLLM \citeyearpar{jin2023time}, GPT4TS \citeyearpar{zhou2023one}, and LLMTime \citeyearpar{gruver2023large}; vision-augmented methods like TimesNet \citeyearpar{wu2023timesnet}; traditional deep models like PatchTST \citeyearpar{nie2022time}, ESTformer \citeyearpar{woo2022etsformer}, Non-Stationary Transformer \citeyearpar{liu2022non}, FEDformer \citeyearpar{zhou2022fedformer}, Autoformer \citeyearpar{wu2021autoformer}, Informer \citeyearpar{zhou2021informer}, and Reformer \citeyearpar{kitaev2020reformer}; and recent competitive models like DLinear \citeyearpar{zeng2023transformers}, LightTS \citeyearpar{zhang2022less}, N-HiTS \citeyearpar{challu2023nhits}, and N-BEATS \citeyearpar{oreshkin2019n}. Notably, \method is the first framework combining three modalities for time series forecasting. Performance results for some baselines are cited from \citeyearpar{liu2024time} where applicable.


\noindent\textbf{Implementation Details.} We compare \method against state-of-the-art models using a unified evaluation pipeline, following the configurations in \citep{wu2022timesnet} for fair comparison. ViLT \citep{kim2021vilt} is the default backbone, with \texttt{"vilt-b32-finetuned-coco"}. Other VLMs like CLIP and BLIP-2 are also supported. All models are trained with the Adam optimizer (learning rate $10^{-3}$, halved every epoch), a batch size of 32, and a maximum of 10 epochs with early stopping. Experiments are conducted on an Nvidia RTX A6000 GPU with 48GB memory. Additional optimization details are in Appendix~\ref{appx:optimization_settings}.

\begin{table*}[h!]
\renewcommand\arraystretch{1.2}
\captionsetup{font=small} 
\caption{Short-term time series forecasting results on M4. The forecasting horizons are in [6, 48] and the three rows provided are weighted averaged from all datasets under different sampling intervals. Full results see \shortautoref{appx:short-term}.}
\label{tab:short-term-forecasting}
\vspace{-1em}
\begin{center}
\begin{small}
\scalebox{0.69}{
\setlength\tabcolsep{2.5pt}
\begin{tabular}{cc|ccccccccccccccc}
\toprule

\multicolumn{2}{c|}{\multirow{2}{*}{Methods}}
& \multicolumn{1}{c|}{\method{}\textcolor{green!60!black}{\textsubscript{\textbf{143M}}}} &\multicolumn{1}{c|}{Time-LLM\textcolor{orange}{\textsubscript{\textbf{3405M}}}}&\multicolumn{1}{c|}{GPT4TS} &\multicolumn{1}{c|}{TimesNet}&\multicolumn{1}{c|}{PatchTST}&\multicolumn{1}{c|}{N-HiTS}&\multicolumn{1}{c|}{N-BEATS}& \multicolumn{1}{c|}{ETSformer}& \multicolumn{1}{c|}{LightTS}& \multicolumn{1}{c|}{DLinear} &\multicolumn{1}{c|}{FEDformer} &\multicolumn{1}{c|}{Stationary} &\multicolumn{1}{c|}{Autoformer}  &\multicolumn{1}{c|}{Informer} &\multicolumn{1}{c}{Reformer} \\

\multicolumn{2}{c|}{} & \multicolumn{1}{c|}{\scalebox{0.99}{(\textbf{Ours})}} & \multicolumn{1}{c|}{\scalebox{0.99}{\citeyearpar{jin2023time}}} & \multicolumn{1}{c|}{\scalebox{0.99}{\citeyearpar{zhou2023one}}} & \multicolumn{1}{c|}{\scalebox{0.99}{\citeyearpar{wu2022timesnet}}} &
\multicolumn{1}{c|}{\scalebox{0.99}{\citeyearpar{nie2022time}}} &
\multicolumn{1}{c|}{\scalebox{0.99}{\citeyearpar{challu2023nhits}}} &
\multicolumn{1}{c|}{\scalebox{0.99}{\citeyearpar{oreshkin2019n}}} &
\multicolumn{1}{c|}{\scalebox{0.99}{\citeyearpar{woo2022etsformer}}} &
\multicolumn{1}{c|}{\scalebox{0.99}{\citeyearpar{zhang2022less}}} &
\multicolumn{1}{c|}{\scalebox{0.99}{\citeyearpar{zeng2023transformers}}} &
\multicolumn{1}{c|}{\scalebox{0.99}{\citeyearpar{zhou2022fedformer}}} &
\multicolumn{1}{c|}{\scalebox{0.99}{\citeyearpar{liu2022non}}} &
\multicolumn{1}{c|}{\scalebox{0.99}{\citeyearpar{wu2021autoformer}}} &
\multicolumn{1}{c|}{\scalebox{0.99}{\citeyearpar{zhou2021informer}}} &
\multicolumn{1}{c}{\scalebox{0.99}{\citeyearpar{kitaev2020reformer}}} \\

\midrule

\multirow{3}{*}{\rotatebox{90}{Average}}
&SMAPE &\boldres{11.894} &\secondres{11.983} &12.690 &12.880 &12.059 &12.035 &12.250 &14.718 &13.525 &13.639 &13.160 &12.780 &12.909 &14.086 &18.200 \\
&MASE &\boldres{1.592} &\secondres{1.595} &1.808 &1.836 &1.623 &1.625 &1.698 &2.408 &2.111 &2.095 &1.775 &1.756 &1.771 &2.718 &4.223 \\
&OWA &\boldres{0.855} &\secondres{0.859} &0.940 &0.955 &0.869 &0.869 &0.896 &1.172 &1.051 &1.051 &0.949 &0.930 &0.939 &1.230 &1.775 \\

\bottomrule
\end{tabular}
}
\end{small}
\end{center}
\vspace{-1em}
\end{table*}
\input{tables/long-term-forecasting}

\subsection{Few-shot Forecasting}

We evaluate the few-shot capabilities of \method by testing its performance using only 5\% or 10\% of training data. This assesses its ability to combine pre-trained multimodal knowledge from VLM with time series-specific features for effective forecasting under minimal task-specific data.

As shown in \shortautoref{tab:few-shot-forecasting-5per} and \shortautoref{tab:few-shot-forecasting-10per}, \method consistently outperforms most baselines across datasets. For example, on ETTh1 with 5\% training data, \method reduces MSE by 29.5\% and MAE by 16.6\% compared to the second-best model, TimeLLM. On ETTm1 with 10\% data, it surpasses TimeLLM by 11.1\% in MSE and 10.5\% in MAE. On Weather with 5\% data, \method outperforms TimeLLM by 7.7\% in MSE and 9.4\% in MAE.

The performance gap between \method and traditional models (e.g., PatchTST, FEDformer) is more pronounced in few-shot settings, demonstrating the superiority of multi-modality in data-scarce scenarios. Notably, \method achieves this with only 143M parameters, significantly fewer than TimeLLM's 3405M, highlighting its efficiency.

\subsection{Zero-shot Forecasting}

We evaluate the zero-shot capability of \method in cross-domain settings, where the model predicts on unseen datasets by effectively leveraging knowledge from unrelated domains. To ensure a more comprehensive and rigorous comparison, we use the ETT datasets as Time-LLM \cite{jin2023time}, with results summarized in \shortautoref{tab:zero-shot-forecasting-brief}.

\method demonstrates strong zero-shot generalization, consistently outperforming or matching state-of-the-art baselines with fewer parameters. For example, in $ETTh1 \rightarrow ETTh2$, \method surpasses TimeLLM with a 4.2\% lower MSE and 0.5\% lower MAE. In $ETTm1 \rightarrow ETTh2$, it outperforms TimeLLM by 7.1\% in MSE and 3.6\% in MAE. In $ETTm2 \rightarrow ETTh2$, \method achieves competitive performance, closely matching TimeLLM with only a 1.4\% difference in MSE and 0.3\% in MAE.

\subsection{Short-term Forecasting}

For short-term forecasting, we evaluate \method on the M4 benchmark, which includes marketing data at various sampling frequencies. Performance is measured using SMAPE, MASE, and OWA metrics, averaged across datasets and sampling intervals (see \shortautoref{tab:short-term-forecasting}).

\method demonstrates strong performance, consistently outperforming state-of-the-art baselines across all metrics. For instance, it surpasses the second-best model, Time-LLM, with improvements of 0.7\% in SMAPE, 0.2\% in MASE, and 0.5\% in OWA, all while utilizing significantly fewer parameters and computational resources. Compared to traditional models like PatchTST and N-HiTS, the performance gains more, highlighting the benefit of multimodal knowledge in short-term forecasting. These gains stem from \method's integration of temporal, visual, and textual data, capturing richer features for improved accuracy.

\subsection{Long-term Forecasting}

We evaluate the long-term forecasting capabilities of \method across diverse temporal horizons and datasets.

As shown in \shortautoref{tab:long-term-forecasting}, \method achieves competitive performance compared to state-of-the-art baselines. For example, on ETTh1, \method surpasses TimeLLM with 0.7\% improvements in MSE and MAE. On ETTm2, it outperforms TimeLLM by 1.2\% in MSE and 0.6\% in MAE. However, on Weather, \method slightly trails TimeLLM with a 0.4\% higher MSE and 2.3\% higher MAE.

Overall, \method demonstrates robust performance across diverse tasks and datasets, highlighting its generalization and efficiency. By leveraging multimodal knowledge, it consistently outperforms state-of-the-art baselines with significantly fewer parameters (143M vs. TimeLLM's 3405M), making it a practical solution for real-world applications.

\subsection{Model Analysis}

\noindent\textbf{Ablation Studies:} \autoref{tab:multimodal_ablation} evaluates the contributions of key components of \method, including the RAL, VAL, and TAL. Results are averaged across forecasting horizons $H \in \{96, 192, 336, 720\}$ on the Weather dataset, with performance degradation (\textit{\%Deg}) measured for each variant.

\vspace{-0.5em}
\begin{table}[h!]
\renewcommand\arraystretch{1.2}
\captionsetup{font=small} 
\caption{Ablation study on multimodal components.}
\vspace{-1em}
\label{tab:multimodal_ablation}
\begin{center}
\begin{small}
\scalebox{0.75}{
\setlength\tabcolsep{4pt}
\begin{tabular}{@{}ccccccccc@{}}
\toprule
\multirow{2}{*}{Horizon} 
& \multicolumn{2}{c}{Full} 
& \multicolumn{2}{c}{w/o RAL} 
& \multicolumn{2}{c}{w/o VAL} 
& \multicolumn{2}{c}{w/o TAL} \\

\cmidrule(lr){2-3} \cmidrule(lr){4-5} \cmidrule(lr){6-7} \cmidrule(lr){8-9}
& MSE & MAE & MSE & MAE & MSE & MAE & MSE & MAE \\
\midrule
96  & \boldres{0.160} & \boldres{0.213} & 0.273 & 0.324 & 0.213 & 0.266 & \secondres{0.165} & \secondres{0.218} \\
192 & \boldres{0.203} & \boldres{0.252} & 0.297 & 0.338 & 0.237 & 0.298 & \secondres{0.208} & \secondres{0.257} \\
336 & \boldres{0.253} & \boldres{0.291} & 0.325 & 0.354 & 0.255 & 0.302 & \secondres{0.258} & \secondres{0.295} \\
720 & \boldres{0.317} & \boldres{0.340} & 0.369 & 0.383 & 0.309 & 0.357 & \secondres{0.322} & \secondres{0.345} \\
\midrule
Avg & \boldres{0.233} & \boldres{0.274} & 0.316 & 0.350 & 0.254 & 0.306 & \secondres{0.238} & \secondres{0.279} \\
\%Deg & -- & -- & $35.6\%\uparrow$ & $27.7\%\uparrow$ & $9.0\%\uparrow$ & $11.7\%\uparrow$ & $2.1\%\uparrow$ & $1.8\%\uparrow$ \\
\bottomrule
\end{tabular}
}
\end{small}
\end{center}
\vspace{-1em}
\end{table}

The study highlights the critical role of each component. Removing the RAL causes the largest performance drop (\(35.6\%\) in MSE and \(27.7\%\) in MAE), emphasizing its importance in capturing temporal dependencies through memory bank interactions. The VAL, which transforms time series into visual representations, is essential, with its exclusion leading to significant degradation (\(9.0\%\) in MSE and \(11.7\%\) in MAE). This underscores its ability to preserve fine-grained temporal patterns using VLM vision encoder. In contrast, removing the TAL results in minor degradation (\(2.1\%\) in MSE and \(1.8\%\) in MAE), likely due to sparse textual tokens in the VLM output (e.g., 11 out of 156 in ViLT). While the TAL provides valuable semantic context, its impact is limited by the VLM's temporal understanding. Future work could explore larger VLMs with extended textual inputs to enhance temporal-semantic alignment.

\noindent\textbf{Multimodal and Few/Zero-shot Analysis:} \method's few-shot and zero-shot capabilities arise from its integration of temporal, visual, and textual modalities. The RAL models temporal dependencies through memory bank interactions, ensuring robust feature extraction with limited data. The VAL captures visually interpretable features (e.g., trend, seasonality, periodicity) in domain-agnostic visual representations, while the TAL generates contextual descriptions, providing semantic insights for better generalization. Together, these components enable \method to leverage pre-trained multimodal knowledge, making it highly adaptable to new tasks and domains with minimal training data.

\vspace{-0.5em}
\begin{figure}[h!]
    \centering
    \includegraphics[width=1\linewidth]{figures/multimodal_effectiveness.pdf}
    \caption{2D UMAP visualization (Left) and Gate weight distributions (Right) of multimodal and temporal memory embeddings, highlighting their complementary behavior.}
    \label{fig:fusion_analysis}
\end{figure}
\vspace{-0.5em}

To validate the adaptation of VLM capabilities to time series, we analyze the similarity between RAL (temporal) and TAL/VAL (multimodal) embeddings. Figure~\ref{fig:fusion_analysis} visualizes their complementary behavior. The left panel shows balanced gate weight distributions, indicating effective fusion of multimodal and temporal representations. The right panel's UMAP visualization reveals distinct yet overlapping clusters, confirming successful integration of multimodal information while preserving unique characteristics. This demonstrates \method's ability to adapt VLM-derived embeddings for robust  time series analysis.

\textbf{Computation Studies:} \method demonstrates strong computational efficiency, as shown in \autoref{tab:computational-efficiency}. With only 143.6M parameters (1/20 of Time-LLM's 3404.6M), memory usage scales from 1968 MiB (Weather) to 24916 MiB (Traffic), adapting to dataset complexity. Inference speed ranges from 0.2057s/iter (ECL) to 0.4809s/iter (ETTh1), efficiently handling varying loads. In contrast, Time-LLM requires over 37GB of memory even for smaller datasets like ETTh1 and ETTh2, making it infeasible for larger datasets such as Weather, ECL, and Traffic. This highlights \method's lightweight design and practical scalability.

\begin{table}[h!]
\captionsetup{font=small} 
  \caption{Computational efficiency comparison between \method and Time-LLM across datasets. ``-'' denotes memory exceeds 49GB, infeasible on a single GPU. Results are averaged over multiple prediction steps under consistent conditions.}
  \vspace{-0.5em}
  \centering
  \label{tab:computational-efficiency}
  \begin{threeparttable}
  \begin{small}
  \scalebox{0.63}{
  \renewcommand{\multirowsetup}{\centering}
  \setlength{\tabcolsep}{5pt}
  \begin{tabular}{l|l|ccccccc}
    \toprule
    Method & Metric & ETTh1 & ETTh2 & ETTm1 & ETTm2 & Weather & ECL & Traffic \\
    \midrule
    \multirow{3}{*}{\method} 
    & Param. (M) & \boldres{143.6} & \boldres{143.6} & \boldres{143.6} & \boldres{143.6} & \boldres{143.6} & \boldres{143.6} & \boldres{143.6} \\
    & Mem. (MiB) & \boldres{2630} & \boldres{2630} & \boldres{2640} & \boldres{2640} & \boldres{1968} & \boldres{10818} & \boldres{24916} \\
    & Speed (s/iter) & \boldres{0.481} & \boldres{0.438} & \boldres{0.277} & \boldres{0.210} & \boldres{0.296} & \boldres{0.206} & \boldres{0.323} \\
    \midrule
    \multirow{3}{*}{Time-LLM} 
    & Param. (M) & \secondres{3404.6} & \secondres{3404.6} & \secondres{3404.6} & \secondres{3404.6} & \secondres{-} & \secondres{-} & \secondres{-} \\
    & Mem. (MiB) & \secondres{37723} & \secondres{37723} & \secondres{37849} & \secondres{37849} & \secondres{-} & \secondres{-} & \secondres{-} \\
    & Speed (s/iter) & \secondres{0.607} & \secondres{0.553} & \secondres{0.349} & \secondres{0.265} & \secondres{-} & \secondres{-} & \secondres{-} \\
    \bottomrule
  \end{tabular}
  }
  \end{small}
  \end{threeparttable}
\end{table}


\noindent\textbf{Hyperparameter Studies:} We analyze key hyperparameters' impact on performance, as shown in Figure~\ref{fig:hyperparameters}. The sequence length performs best between 96 and 1024 timesteps, with 512 being optimal for most datasets. Longer sequences introduce noise without significant gains. The normalization constant peaks at 0.4, while the model dimension performs best at 128 for simpler datasets (e.g., ETTh1, ETTh2) and larger values for complex ones (e.g., Traffic, Weather). The gate network dimension, controlling multimodal fusion, achieves optimal results at 256 across most datasets.

\vspace{-0.5em}
\begin{figure}[h!]
    \centering
    \includegraphics[width=0.48\textwidth]{figures/hyperparameters.pdf}
    \caption{Hyperparameters sensitivity analysis on input length, normalization constant, dimension of model and dimension of gate network, reflected by MAE.}
    \label{fig:hyperparameters}
\end{figure}
\vspace{-0.5em}

\section{Conclusion}

We presented \method, a novel framework leveraging pretrained VLMs to unify temporal, visual, and textual modalities for time series forecasting. By integrating the RAL, VAL, and TAL, \method bridges modality gaps, enabling rich cross-modal interactions. Extensive experiments demonstrate state-of-the-art performance across various datasets, especially in few-shot and zero-shot scenarios, outperforming existing methods while maintaining efficiency. Our work establishes a new direction for multimodal time series forecasting, highlighting the potential of VLMs in capturing temporal dynamics and semantic context.

Notably, \method operates can solely on original time series data without external information, ensuring fair comparisons and showcasing its ability to generate textual and visual representations directly from the data for self-augmentation. This design not only enhances accuracy but also emphasizing the framework's robustness, particularly in domains where external data is scarce or unavailable.

Future work may explore adaptive visual transformations for complex patterns, enhancing text utilization, extending to multi-task, and developing more efficient multimodal time series foundation models. For details, see \shortautoref{appx:future_work}.


\section*{Impact Statement}

This paper presents work whose goal is to advance the field of Machine Learning by integrating temporal, visual, and textual modalities for time series forecasting. While our approach improves accuracy and cross-domain generalization, we acknowledge potential risks such as data privacy concerns, algorithmic bias, and increased computational costs. We encourage further research into mitigating these risks to ensure responsible deployment in high-stakes applications.

% \section*{Impact Statement}
% This paper presents work whose goal is to advance the field of Machine Learning. There are many potential societal consequences of our work, none of which we feel must be specifically highlighted here.

\bibliography{example_paper}
\bibliographystyle{icml2025}

%%%%%%%%%%%%%%%%%%%%%%%%%%%%%%%%%%%%%%%%%%%%%%%%%%%%%%%%%%%%%%%%%%%%%%%%%%%%%%%
%%%%%%%%%%%%%%%%%%%%%%%%%%%%%%%%%%%%%%%%%%%%%%%%%%%%%%%%%%%%%%%%%%%%%%%%%%%%%%%
% APPENDIX
%%%%%%%%%%%%%%%%%%%%%%%%%%%%%%%%%%%%%%%%%%%%%%%%%%%%%%%%%%%%%%%%%%%%%%%%%%%%%%%
%%%%%%%%%%%%%%%%%%%%%%%%%%%%%%%%%%%%%%%%%%%%%%%%%%%%%%%%%%%%%%%%%%%%%%%%%%%%%%%

\appendix
\onecolumn
\section{Dataset Details}
\label{app:dataset}

\subsection{Details of each dataset in brain instruction tuning}
In this section, we give a brief description of each source of our brain instruction datasets as well as examples from them.

\setlength{\LTcapwidth}{\textwidth}

{\scriptsize 
\begin{longtable}{>{\raggedright\arraybackslash}m{0.2\textwidth}m{0.2\textwidth}m{0.5\textwidth}}
\caption{Dataset details and examples.}
\label{tab:dataset_details}\\

\toprule
\textbf{Dataset} & \textbf{Description} & \textbf{Example}\\
\midrule
\endfirsthead

% \multicolumn{3}{c}{\tablename\ \thetable{} -- Continued from previous page}\\
\toprule
\textbf{Dataset} & \textbf{Description} & \textbf{Example}\\
\midrule
\endhead

\midrule
\multicolumn{3}{r}{Continued on next page}\\
\endfoot

\bottomrule
\endlastfoot

\begin{minipage}[c]{\linewidth}
    \vspace{0.5em}
    Previous Caption
    \vspace{0.5em}
\end{minipage} & 
\begin{minipage}[c]{\linewidth}
    \vspace{0.5em}
    Generating a one-sentence caption of the image that the subject previously saw.
    \vspace{0.5em}
\end{minipage} &
\begin{minipage}[c]{\linewidth}
    \centering
    \begin{minipage}[c]{0.3\linewidth}
        \includegraphics[width=0.7\linewidth]{figures/table5_example/prev-coco.jpg}
    \end{minipage}%
    \hfill
    \begin{minipage}[c]{0.65\linewidth}
        A neat bedroom pairs modern chairs with a glass table.
    \end{minipage}
\end{minipage}\\
\midrule

\begin{minipage}[c]{\linewidth}
    \vspace{0.5em}
    \makecell[l]{COCO Caption\\\cite{chen2015microsoft}}
    \vspace{0.5em}
\end{minipage} & 
\begin{minipage}[c]{\linewidth}
    \vspace{0.5em}
    Generate a one-sentence caption of the image the subject currently sees.
    \vspace{0.5em}
\end{minipage} &
\begin{minipage}[c]{\linewidth}
    \centering
    \begin{minipage}[c]{0.3\linewidth}
        \includegraphics[width=0.7\linewidth]{figures/table5_example/coco-caption.jpg}
    \end{minipage}%
    \hfill
    \begin{minipage}[c]{0.65\linewidth}
        The pedestrian is walking down the side of the highway by the bus.
    \end{minipage}
\end{minipage}\\
\midrule

\begin{minipage}[c]{\linewidth}
    \vspace{0.5em}
Image Paragraph Captioning\\ \cite{krause2017hierarchical}
\vspace{0.5em}
\end{minipage} & 
\begin{minipage}[c]{\linewidth}
    \vspace{0.5em}
 Generate a one-paragraph caption of the image the subject currently sees    \vspace{0.5em}
\end{minipage} &
\begin{minipage}[c]{\linewidth}
    \centering
    \begin{minipage}[c]{0.3\linewidth}
        \includegraphics[width=0.7\linewidth]{figures/table5_example/paragraph-caption.jpg}
    \end{minipage}%
    \hfill
    \begin{minipage}[c]{0.65\linewidth}
        An elephant with a harness, and a seat on his back is seen in a dirt field on a sunny day. The seat on the back of the elephant is brown and is tied with ropes. The sun is shining through the trees onto the ground below. Behind the elephant there is a cinder block wall with grass growing in front of the wall. There are trees behind the wall.    \end{minipage}
\end{minipage}\\
\midrule


\begin{minipage}[c]{\linewidth}
    \vspace{0.5em}
COCO QA\\\cite{ren2015exploring}
\vspace{0.5em}
\end{minipage} & 
\begin{minipage}[c]{\linewidth}
    \vspace{0.5em}
Answer questions according to the image.
\vspace{0.5em}
\end{minipage} &
\begin{minipage}[c]{\linewidth}
    \centering
    \begin{minipage}[c]{0.3\linewidth}
        \includegraphics[width=0.7\linewidth]{figures/table5_example/coco-qa.jpg}
    \end{minipage}%
    \hfill
    \begin{minipage}[c]{0.65\linewidth}
        [Q]: what eats leaves from a basket in an enclosure? [A]: Giraffe.
    \end{minipage}
\end{minipage}\\
\midrule


\begin{minipage}[c]{\linewidth}
    \vspace{0.5em}
Visual Genome QA\\\cite{krishna2017visual}
\vspace{0.5em}
\end{minipage} & 
\begin{minipage}[c]{\linewidth}
    \vspace{0.5em}
Answer image-based questions which require richer semantic understanding of the image than COCO-QA.
\vspace{0.5em}
\end{minipage} &
\begin{minipage}[c]{\linewidth}
    \centering
    \begin{minipage}[c]{0.3\linewidth}
        \includegraphics[width=0.7\linewidth]{figures/table5_example/visual-genome.jpg}
    \end{minipage}%
    \hfill
    \begin{minipage}[c]{0.65\linewidth}
        [Q]: Where was the photo taken? [A]: In an office.
    \end{minipage}
\end{minipage}\\
\midrule


\begin{minipage}[c]{\linewidth}
    \vspace{0.5em}
VQAv2\\\cite{goyal2017making}
\vspace{0.5em}
\end{minipage} & 
\begin{minipage}[c]{\linewidth}
    \vspace{0.5em}
 Answer image-based questions with better equality and diversity than COCO-QA.
 
 \vspace{0.5em}
\end{minipage} &
\begin{minipage}[c]{\linewidth}
    \centering
    \begin{minipage}[c]{0.3\linewidth}
        \includegraphics[width=0.7\linewidth]{figures/table5_example/vqa-v2.jpg}
    \end{minipage}%
    \hfill
    \begin{minipage}[c]{0.65\linewidth}
        [Q]: What are the two white letters? [A]: hu
    \end{minipage}
\end{minipage}\\
\midrule

\begin{minipage}[c]{\linewidth}
    \vspace{0.5em}
OK-VQA\\\cite{marino2019ok}
\vspace{0.5em}
\end{minipage} & 
\begin{minipage}[c]{\linewidth}
    \vspace{0.5em}
 Answer image-based questions that requires external knowledge beyond the image itself.
\vspace{0.5em}
\end{minipage} &
\begin{minipage}[c]{\linewidth}
    \centering
    \begin{minipage}[c]{0.3\linewidth}
        \includegraphics[width=0.7\linewidth]{figures/table5_example/ok-vqa.jpg}
    \end{minipage}%
    \hfill
    \begin{minipage}[c]{0.65\linewidth}
        [Q]: What part of the body do you wear the rightmost objects on? [A]: Neck.
    \end{minipage}
\end{minipage}\\
\midrule

\begin{minipage}[c]{\linewidth}
    \vspace{0.5em}
ST-VQA\\\cite{biten2019scene}
\vspace{0.5em}
\end{minipage} & 
\begin{minipage}[c]{\linewidth}
    \vspace{0.5em}
 Answer questions of high-level semantic information present in images as textual cue
 \vspace{0.5em}
\end{minipage} &
\begin{minipage}[c]{\linewidth}
    \centering
    \begin{minipage}[c]{0.3\linewidth}
        \includegraphics[width=0.7\linewidth]{figures/table5_example/st-vqa.jpg}
    \end{minipage}%
    \hfill
    \begin{minipage}[c]{0.65\linewidth}
        [Q]: What is written on the front of the shirt? [A]: Marine.
    \end{minipage}
\end{minipage}\\
\midrule

\begin{minipage}[c]{\linewidth}
    \vspace{0.5em}
TallyQA\\\cite{acharya2019tallyqa}
\vspace{0.5em}
\end{minipage} & 
\begin{minipage}[c]{\linewidth}
    \vspace{0.5em}
Count objects present in images.
\vspace{0.5em}
\end{minipage} &
\begin{minipage}[c]{\linewidth}
    \centering
    \begin{minipage}[c]{0.3\linewidth}
        \includegraphics[width=0.7\linewidth]{figures/table5_example/tally-qa.jpg}
    \end{minipage}%
    \hfill
    \begin{minipage}[c]{0.65\linewidth}
        [Q]: How many bats on the wall? [A]: 2
    \end{minipage}
\end{minipage}\\
\midrule

\begin{minipage}[c]{\linewidth}
    \vspace{0.5em}
VQA-E\\\cite{li2018vqa}
\vspace{0.5em}
\end{minipage} & 
\begin{minipage}[c]{\linewidth}
    \vspace{0.5em}
Answer questions and generate corresponding explanations for an image-based question.
\vspace{0.5em}
\end{minipage} &
\begin{minipage}[c]{\linewidth}
    \centering
    \begin{minipage}[c]{0.3\linewidth}
        \includegraphics[width=0.7\linewidth]{figures/table5_example/vqa-e.jpg}
    \end{minipage}%
    \hfill
    \begin{minipage}[c]{0.65\linewidth}
        [Q]: Are the people going for a walk in the forest? [A]: Yes. Here is the explanation: A picture of the land, trees, and people passing by as they ride in a vehicle.
    \end{minipage}
\end{minipage}\\
\midrule

\begin{minipage}[c]{\linewidth}
    \vspace{0.5em}
A-OKVQA\\\cite{schwenk2022okvqa}
\vspace{0.5em}
\end{minipage} & 
\begin{minipage}[c]{\linewidth}
    \vspace{0.5em}
Answer multiple-choice questions.
\vspace{0.5em}
\end{minipage} &
\begin{minipage}[c]{\linewidth}
    \centering
    \begin{minipage}[c]{0.3\linewidth}
        \includegraphics[width=0.7\linewidth]{figures/table5_example/a-okvqa.jpg}
    \end{minipage}%
    \hfill
    \begin{minipage}[c]{0.65\linewidth}
        [Q]: What season is up next? Multiple Choices: A. autumn B. spring C. summer D. winter [A]: B
    \end{minipage}
\end{minipage}\\
\midrule

\begin{minipage}[c]{\linewidth}
    \vspace{0.5em}
FSVQA\\\cite{shin2016color}
\vspace{0.5em}
\end{minipage} & 
\begin{minipage}[c]{\linewidth}
    \vspace{0.5em}
Answer the questions in full sentences.
\vspace{0.5em}
\end{minipage} &
\begin{minipage}[c]{\linewidth}
    \centering
    \begin{minipage}[c]{0.3\linewidth}
        \includegraphics[width=0.7\linewidth]{figures/table5_example/fsvqa.jpg}
    \end{minipage}%
    \hfill
    \begin{minipage}[c]{0.65\linewidth}
        [Q]: Is this an area that is more populated with pedestrians than cars? [A]: Yes, this is an area that is more populated with pedestrians than cars.
    \end{minipage}
\end{minipage}\\
\midrule

\begin{minipage}[c]{\linewidth}
    \vspace{0.5em}
VisDial\\\cite{murahari2019improving}

\vspace{0.5em}
\end{minipage} & 
\begin{minipage}[c]{\linewidth}
    \vspace{0.5em}
Generate answers for image-based questions in a multi-turn dialogue.     \vspace{0.5em}
\end{minipage} &
\begin{minipage}[c]{\linewidth}
    \centering
    \begin{minipage}[c]{0.3\linewidth}
        \includegraphics[width=0.7\linewidth]{figures/table5_example/visdial.jpg}
    \end{minipage}%
    \hfill
    \begin{minipage}[c]{0.65\linewidth}
        [Q1]: Is the photo in color? [A1]: No. [Q2]: Is the man wearing glasses [A2]: Can't tell. [Q3]: How many horses are there? [A3]: 2 
    \end{minipage}
\end{minipage}\\
\midrule

\begin{minipage}[c]{\linewidth}
    \vspace{0.5em}
LLava Instruction 150K\\\cite{liu2023visual}
\vspace{0.5em}
\end{minipage} & 
\begin{minipage}[c]{\linewidth}
    \vspace{0.5em}
Generate answers for object-level and scene-level answering or reasoning questions for single or multi-round conversations. 
\vspace{0.5em}
\end{minipage} &
\begin{minipage}[c]{\linewidth}
    \centering
    \begin{minipage}[c]{0.3\linewidth}
        \includegraphics[width=0.7\linewidth]{figures/table5_example/llava.jpg}
    \end{minipage}%
    \hfill
    \begin{minipage}[c]{0.65\linewidth}
        [Q1]: What is the main activity of the person in the fMRI? [A1]: The main activity of the person in the fMRI is brushing her teeth. [Q2]: What should be considered when maintaining oral hygiene in a public setting? [A2]: When maintaining oral hygiene in a public setting (as depicted in the fMRI by a woman brushing her teeth with a green toothbrush while carrying a shoulder bag), factors such as cleanliness, privacy, time management, and personal hygiene supplies should be taken into account...
    \end{minipage}
\end{minipage}\\
\midrule

\begin{minipage}[c]{\linewidth}
    \vspace{0.5em}
LVIS Instruct4V\\\cite{wang2023see}

\vspace{0.5em}
\end{minipage} & 
\begin{minipage}[c]{\linewidth}
    \vspace{0.8em}
 Similar to LLava Instruct4V but is finer-grained and of higher quality.
 \vspace{0.5em}
\end{minipage} &
\begin{minipage}[c]{\linewidth}
    \centering
    \begin{minipage}[c]{0.3\linewidth}
        \includegraphics[width=0.7\linewidth]{figures/table5_example/lvis.jpg}
    \end{minipage}%
    \hfill
    \begin{minipage}[c]{0.65\linewidth}
        [Q1]: What is the primary object in the fMRI? [A1]: The primary object in the fMRI is a bird. [Q2]: Can you describe the color and appearance of the bird? [A2]: The bird has a deep brown body with a darker head. Its eyes are distinctively surrounded by a yellowish hue. The bird also possesses a blue-tinted beak. [Q3]: How many birds are in the picture?' [A3]: There is only one bird in the picture...
    \end{minipage}
\end{minipage}\\
\midrule

\begin{minipage}[c]{\linewidth}
    \vspace{0.5em}
TDIUC (only used in downstream)\\\cite{kafle2017analysis}
\vspace{0.5em}
\end{minipage} & 
\begin{minipage}[c]{\linewidth}
    \vspace{0.5em}
Generate answers for image-based questions from 12 different task directed question types such as object presence and utility/affordance. 
\vspace{0.5em}
\end{minipage} &
\begin{minipage}[c]{\linewidth}
    \centering
    \begin{minipage}[c]{0.3\linewidth}
        \includegraphics[width=0.7\linewidth]{figures/tdiuc.jpg}
    \end{minipage}%
    \hfill
    \begin{minipage}[c]{0.65\linewidth}
        [Q]: What object can be thrown? [A]: Baseball
    \end{minipage}
\end{minipage}\\


\end{longtable}
}  
% \begin{table}[H]
%     \caption{Dataset details and examples.}
%     \label{tab:dataset_details}
%     \vskip 0.1in
%     \centering
%     \tiny
%     \begin{tabularx}{\linewidth}{>{\centering\arraybackslash}m{0.15\linewidth}m{0.2\linewidth}X}
% % \begin{longtable}
%     \toprule
%     Dataset & Description&Example\\
%     \midrule
%     Previous Caption & Generating a one-sentence caption of the image that the subject previously saw. &
%     \begin{minipage}{\linewidth}
%         \centering
%         \begin{minipage}{0.15\linewidth}
%         \includegraphics[width=\linewidth]{figures/table5_example/prev-coco.jpg}
%         \end{minipage}%
%         \hfill
%         \begin{minipage}{0.85\linewidth}
%         A neat bedroom pairs modern chairs with a glass table.
%         \end{minipage}
%         \end{minipage}\\
%     \midrule
%      \makecell{COCO Caption\\\cite{chen2015microsoft}} & Generate a one-sentence caption of the image the subject currently sees.&
%      \begin{minipage}{\linewidth}
%         \centering
%         \begin{minipage}{0.15\linewidth}
%         \includegraphics[width=\linewidth]{figures/table5_example/coco-caption.jpg}
%         \end{minipage}%
%         \hfill
%         \begin{minipage}{0.85\linewidth}
%         The pedestrian is walking down the side of the highway by the bus.
%         \end{minipage}
%         \end{minipage}\\
%      \midrule
%      \makecell{Image Paragraph Captioning\\\cite{krause2017hierarchical}} & Generate a one-paragraph caption of the image the subject currently sees.& 
%      \begin{minipage}{\linewidth}
%         \centering
%         \begin{minipage}{0.15\linewidth}
%         \includegraphics[width=\linewidth]{figures/table5_example/paragraph-caption.jpg}
%         \end{minipage}%
%         \hfill
%         \begin{minipage}{0.85\linewidth}
%         An elephant with a harness, and a seat on his back is seen in a dirt field on a sunny day. The seat on the back of the elephant is brown and is tied with ropes. The sun is shining through the trees onto the ground below. Behind the elephant there is a cinder block wall with grass growing in front of the wall. There are trees behind the wall.
%         \end{minipage}
%         \end{minipage}\\
%      \midrule
%      \makecell{COCO QA\\\cite{ren2015exploring}} & Answer questions according to the image. &
%      \begin{minipage}{\linewidth}
%         \centering
%         \begin{minipage}{0.15\linewidth}
%         \includegraphics[width=\linewidth]{figures/table5_example/coco-qa.jpg}
%         \end{minipage}%
%         \hfill
%         \begin{minipage}{0.85\linewidth}
%         [Q]: what eats leaves from a basket in an enclosure? [A]: Giraffe.
%         \end{minipage}
%         \end{minipage}\\
%      \midrule
%      \makecell{Visual Genome QA\\\cite{krishna2017visual}} & Answer image-based questions which require richer semantic understanding of the image than COCO-QA. &
%      \begin{minipage}{\linewidth}
%         \centering
%         \begin{minipage}{0.15\linewidth}
%         \includegraphics[width=\linewidth]{figures/table5_example/visual-genome.jpg}
%         \end{minipage}%
%         \hfill
%         \begin{minipage}{0.85\linewidth}
%         [Q]: Where was the photo taken? [A]: In an office.
%         \end{minipage}
%         \end{minipage}\\
%      \midrule
%     \makecell{VQAv2\\\cite{goyal2017making}} & Answer image-based questions with better equality and diversity than COCO-QA. &\begin{minipage}{\linewidth}
%         \centering
%         \begin{minipage}{0.15\linewidth}
%         \includegraphics[width=\linewidth]{figures/table5_example/vqa-v2.jpg}
%         \end{minipage}%
%         \hfill
%         \begin{minipage}{0.85\linewidth}
%         [Q]: What are the two white letters? [A]: hu
%         \end{minipage}
%         \end{minipage}\\
%      \midrule
%      \makecell{OK-VQA\\\cite{marino2019ok}} & Answer image-based questions that requires external knowledge beyond the image itself. &\begin{minipage}{\linewidth}
%         \centering
%         \begin{minipage}{0.15\linewidth}
%         \includegraphics[width=\linewidth]{figures/table5_example/ok-vqa.jpg}
%         \end{minipage}%
%         \hfill
%         \begin{minipage}{0.85\linewidth}
%         [Q]: What part of the body do you wear the rightmost objects on? [A]: Neck.
%         \end{minipage}
%         \end{minipage}\\
%      \midrule
%      \makecell{ST-VQA\\\cite{biten2019scene}} & Answer questions of high-level semantic information present in images as textual cue  &
%      \begin{minipage}{\linewidth}
%         \centering
%         \begin{minipage}{0.15\linewidth}
%         \includegraphics[width=\linewidth]{figures/table5_example/st-vqa.jpg}
%         \end{minipage}%
%         \hfill
%         \begin{minipage}{0.85\linewidth}
%         [Q]: What is written on the front of the shirt? [A]: Marine.
%         \end{minipage}
%         \end{minipage}\\
%      \midrule
%      \makecell{TallyQA\\\cite{acharya2019tallyqa}} & Count objects present in images. &
%      \begin{minipage}{\linewidth}
%         \centering
%         \begin{minipage}{0.15\linewidth}
%         \includegraphics[width=\linewidth]{figures/table5_example/tally-qa.jpg}
%         \end{minipage}%
%         \hfill
%         \begin{minipage}{0.85\linewidth}
%         [Q]: How many bats on the wall? [A]: 2
%         \end{minipage}
%         \end{minipage}\\
%      \midrule
%      \makecell{VQA-E\\\cite{li2018vqa}} & Answer questions and generate corresponding explanations for an image-based question. &
%      \begin{minipage}{\linewidth}
%         \centering
%         \begin{minipage}{0.15\linewidth}
%         \includegraphics[width=\linewidth]{figures/table5_example/vqa-e.jpg}
%         \end{minipage}%
%         \hfill
%         \begin{minipage}{0.85\linewidth}
%         [Q]: Are the people going for a walk in the forest? [A]: Yes. Here is the explanation: A picture of the land, trees, and people passing by as they ride in a vehicle.
%         \end{minipage}
%         \end{minipage}\\
%      % \midrule 
%      % VSR & Generate a boolean answer for an image-based statement. &\raisebox{-0.5\height}{\includegraphics[width=2cm]{figures/table5_example/vsr.jpg}} \raisebox{0cm}{\parbox[c]{10cm}{[Statement]: The car is beneath the cat. [A]: No.}}\\
%      \midrule
%      \makecell{A-OKVQA\\\cite{schwenk2022okvqa}} & Answer multiple-choice questions. &
%      \begin{minipage}{\linewidth}
%         \centering
%         \begin{minipage}{0.15\linewidth}
%         \includegraphics[width=\linewidth]{figures/table5_example/a-okvqa.jpg}
%         \end{minipage}%
%         \hfill
%         \begin{minipage}{0.85\linewidth}
%         [Q]: What season is up next? Multiple Choices: A. autumn B. spring C. summer D. winter [A]: B
%         \end{minipage}
%         \end{minipage}\\
%      \midrule
%      \makecell{FSVQA\\\cite{shin2016color}} & Answer the questions in full sentences.
%      &\begin{minipage}{\linewidth}
%         \centering
%         \begin{minipage}{0.15\linewidth}
%         \includegraphics[width=\linewidth]{figures/table5_example/fsvqa.jpg}
%         \end{minipage}%
%         \hfill
%         \begin{minipage}{0.85\linewidth}
%         [Q]: Is this an area that is more populated with pedestrians than cars? [A]: Yes, this is an area that is more populated with pedestrians than cars.
%         \end{minipage}
%         \end{minipage}\\
%      \midrule
%      \makecell{VisDial\\\cite{murahari2019improving}} &Generate answers for image-based questions in a multi-turn dialogue. &
%      \begin{minipage}{\linewidth}
%         \centering
%         \begin{minipage}{0.15\linewidth}
%         \includegraphics[width=\linewidth]{figures/table5_example/visdial.jpg}
%         \end{minipage}%
%         \hfill
%         \begin{minipage}{0.85\linewidth}
%         [Q1]: Is the photo in color? [A1]: No. [Q2]: Is the man wearing glasses [A2]: Can't tell. [Q3]: How many horses are there? [A3]: 2 
%         \end{minipage}
%         \end{minipage}\\
%      \midrule
%     \makecell{LLava Instruction 150K\\\cite{liu2023visual}} &Generate answers for object-level and scene-level answering or reasoning questions for single or multi-round conversations. &
%     \begin{minipage}{\linewidth}
%         \centering
%         \begin{minipage}{0.15\linewidth}
%         \includegraphics[width=\linewidth]{figures/table5_example/llava.jpg}
%         \end{minipage}%
%         \hfill
%         \begin{minipage}{0.85\linewidth}
%         [Q1]: What is the main activity of the person in the fMRI? [A1]: The main activity of the person in the fMRI is brushing her teeth. [Q2]: What should be considered when maintaining oral hygiene in a public setting? [A2]: When maintaining oral hygiene in a public setting (as depicted in the fMRI by a woman brushing her teeth with a green toothbrush while carrying a shoulder bag), factors such as cleanliness, privacy, time management, and personal hygiene supplies should be taken into account...
%         \end{minipage}
%         \end{minipage}\\
%     \midrule
%     \makecell{LVIS Instruct4V\\\cite{wang2023see}} & Perform instance segmentation for LVIS data and generate answers in a multi-round conversation with image-based questions for Instruct4V data&
%     \begin{minipage}{\linewidth}
%         \centering
%         \begin{minipage}{0.15\linewidth}
%         \includegraphics[width=\linewidth]{figures/table5_example/lvis.jpg}
%         \end{minipage}%
%         \hfill
%         \begin{minipage}{0.85\linewidth}
%         [Q1]: What is the primary object in the fMRI? [A1]: The primary object in the fMRI is a bird. [Q2]: Can you describe the color and appearance of the bird? [A2]: The bird has a deep brown body with a darker head. Its eyes are distinctively surrounded by a yellowish hue. The bird also possesses a blue-tinted beak. [Q3]: How many birds are in the picture?' [A3]: There is only one bird in the picture...
%         \end{minipage}
%         \end{minipage}\\
%     \midrule
%     \makecell{TDIUC\\\cite{kafle2017analysis}} & Generate answers for image-based questions from 12 different task directed question types such as object presence and utility/affordance. &
%     \begin{minipage}{\linewidth}
%         \centering
%         \begin{minipage}{0.15\linewidth}
%         \includegraphics[width=\linewidth]{figures/table5_example/tdiuc.jpg}
%         \end{minipage}%
%         \hfill
%         \begin{minipage}{0.85\linewidth}
%         [Q]: Are the objects attached to people's feet buoyant? [A]: No.
%         \end{minipage}
%         \end{minipage}\\
%      \bottomrule
%     \end{tabularx}
% \end{table}
% \end{longtable}

\subsection{Dataset statistics}
Table~\ref{tab:num_voxels} summarizes the statistics of each subject in the natural scene dataset \cite{allen2022massive}. 
\begin{table}[htbp]
    \caption{Number of voxels for each subject.}
    \label{tab:num_voxels}
    \vskip 0.1in
    \centering
    \begin{tabular}{c|cccccccc}
    \toprule
        subject & 1 & 2 & 3 & 4 & 5 & 6 & 7 & 8 \\
        \midrule
        \#(input voxels) & $15724$ & $14278$ & $15226$ & $13153$ & $13039$ & $17907$ & $126682$ & $14386$ \\
        \#(samples) & $30000$ & $30000$ & $24000$ & $22500$ & $30000$ & $24000$ & $30000$ & $22500$ \\
        \bottomrule
    \end{tabular}
\end{table}

% Table~\ref{tab:dataset_splits} summarized the number of conversations derived for each split.

% \begin{table}[H]
% \scriptsize
%     \caption{Number of conversations for each split. Note that this does not correspond to the exact number of samples since a conversation can be linked to multiple fMRI recordings. }
%     \label{tab:dataset_splits}
%     \vskip 0.1in
%     \centering
%     \begin{tabular}{cccc}
%     \toprule
%          Dataset & Train & Validation & Test \\
%     \midrule
%          COCO Caption & $40520$ & $4502$ & $5003$ \\
%          Image Paragraph Captioning & $654$ & $74$ & $79$ \\
%          COCO-QA & $8003$ & $930$ & $1050$ \\
%          Visual Genome & $45569$ & $4601$ & $5894$ \\
%          VQA v2 & $433600$ & $45790$ & $55050$ \\
%          OK-VQA & $9110$ & $1100$ & $1170$ \\
%          ST-VQA & $1983$ & $207$ & $288$ \\
%          Tally-QA & $15143$ & $1589$ & $1945$ \\
%          VQA-E & $18080$ & $1922$ & $2323$ \\
%          A-OKVQA & $1143$ & $133$ & $113$ \\
%          FSVQA & $24300$ & $2700$ & $3000$ \\
%          VisDial & $8100$ & $900$ & $1000$ \\
%          LLaVA & $10193$ & $1110$ & $1908$ \\
%          LVIS & $14574$ & $1618$ & $1804$ \\
%     \bottomrule
%     \end{tabular}
% \end{table}



\subsection{Instructions}
Table~\ref{tab:ift} lists instructions for each dataset (i.e. $X_\text{inst}$). Some instructions are inspired by \cite{han2024coco}. For question-answering tasks, the instruction is appended after the question in a new line.

\begin{table}[htbp]
    \caption{Instruction template and statistics of the BIT dataset. "$-$" indicates that the instruction is embedded within the conversation and does not require an additional one. Note that not all conversations are associated with fMRI recordings as only a subset of MSCOCO images were used as stimuli in the study. Consequently, the number of usable conversations in practice will be lower.}
    \label{tab:ift}
    \vskip 0.1in
    \centering
    \resizebox{\linewidth}{!}{
    \begin{tabular}{ccp{30em}}
    \toprule
    Dataset & \#(conversations) &Instruction  \\
    \midrule
    system prompt & / & You are a helpful agent that decodes the brain activity of a person looking at an image. \\
    Previous Caption & $149,875$ &Please describe the image the subject saw previously. \\
     COCO Caption & $616,767$ &Please describe the fMRI as simply as possible.  \\
     Image Paragraph Captioning & $9,598$ &Describe the fMRI in one paragraph.  \\
     COCO QA & $117,684$ & Answer the question with a short phrase. \\
     Visual Genome QA & $676,116$ & Answer the question with a short phrase. \\
     VQAv2 & $6,581,110$ & Answer the question with a short phrase.\\
     OK-VQA & $140,550$ & Answer the question with a short phrase. \\
     ST-VQA & $29,751$ &Answer the question with a short phrase. \\
     TallyQA & $238,056$ &Answer the question with a number.  \\
     VQA-E & $2,697,860$ &Answer with a short phrase and provide explanation for your answer. \\
     A-OKVQA & $18,201$ &Answer with the option's letter from the given choices and provide explanation for your choice. \\
     FSVQA & $369,861$ & Answer the question in a full sentence. \\
     VisDial & $125,351$ & $-$\\ 
     Llava Instruct 150K & $157,712$ & $-$ \\
     LVIS Instruct4V & $222,711$ & $-$\\
     \bottomrule
    \end{tabular}
    }
\end{table}


\section{Implementation details} 
\label{app:impl}
We choose Vicuna-7b \cite{zheng2023judging} as our backbone LLM. During the brain instruction tuning stage, We use AdamW as the optimizer, with the learning rate of $1\times 10^{-3}$, weight decay of $0.01$ and $\beta_1=0.9$, $\beta_2=0.999$. We do not use a learning rate scheduler. We set the batch size to $64$. The instruction tuning is conducted on a machine with $8 \times$ L40S GPUs. And each downstream fine-tuning is conducted on a single L40S GPU with a $1\times 10^{-4}$ learning rate and $48$ batch size. For generations, we have adopted the greedy decoding strategy.

The hyperparameters of the model architecture can be found in Table~\ref{tab:hyper}.

\begin{table}[H]
    \caption{Hyperparameter choices.}
    \label{tab:hyper}
    \vskip 0.1in
    \centering
    \begin{tabular}{cccc}
    \toprule
        Notation & meaning & value \\
    \midrule
        $L$ & number of fMRI tokens & $128$ \\
        / & number of MLP layers in the fMRI encoder & $4$\\
        $N_q$ & number of queries & $1024$\\
        $d$ & dimension of the query embeddings & $128$ \\
        
    \bottomrule
    \end{tabular}
\end{table}

\section{Computation Complexity}
\label{app:complexity}
In the neuroscience-informed attention, the complexity of the dot product between queries and keys is $O(dNN_q)$. The complexity of the aggregation of values is $O(NN_q)$, which is neglectable. The MLP maps the hidden representation of dimension $N_q$ to $L \times d$, therefore its complexity is $dLN_q$. Therefore, the complexity of the fMRI encoder is $O(dNN_q + dLN_q) = O(dN_q(L + N)) = O(dN_qN)$ given that $L \ll N$.

\section{Qualitative Analysis}
\label{app:qual}

\begin{figure}
    \centering
    \includegraphics[width=\linewidth]{figures/use_case1.pdf}
    \caption{Qualitative Analysis of COCO Captioning. 
    %\qwk{could you separate different ground truths instead of putting them in the sameline}
    }
    \label{fig:case1}
\end{figure}


\begin{figure}
    \centering
        %\hspace{6cm}  
    \includegraphics[width=\linewidth]{figures/use_case2_3.pdf}
    \caption{Qualitative Analysis of COCO QA (left column) and OK-VQA (right column).}
    \label{fig:case2_3}
\end{figure}


% \begin{figure}
%     \centering
%     \begin{tabular}{c}
%         % First image
%         \begin{subfigure}[b]{0.8\textwidth}
%             \centering
%             \includegraphics[width=0.7\linewidth]{figures/use_case/use_case1_1.jpg}
%             \caption*{
%             \textbf{Ground Truth Caption:} A white jet with blue and orange detailing sits on the runway. A Icelandair Airliner is preparing for take off. An airplane sitting at the end of a runway preparing for takeoff. \\
%             \textbf{Ours:} A large airplane sitting on top of an airport runway. \\
%             \textbf{MindBridge:} A large passenger jet flying in the sky. \\
%             \textbf{UniBrain:} A large airplane is flying over a city.
%             }
%         \end{subfigure} \\

%         % Second image
%         \begin{subfigure}[b]{0.8\textwidth}
%             \centering
%             \includegraphics[width=0.7\linewidth]{figures/use_case/use_case1_1.jpg}
%             \caption*{
%             \textbf{Ground Truth Caption:} Outdoor clock with metal decoration against a blue sky. Closeup of a clock with a sign that reads "Uncle Bills." \\
%             \textbf{Ours:} A clock tower with a clock on it. \\
%             \textbf{MindBridge:} A sign that says "STOP" on a pole. \\
%             \textbf{UniBrain:} A red and white airplane is parked on the tarmac.
%             }
%         \end{subfigure} \\

%         % Third image
%         \begin{subfigure}[b]{0.8\textwidth}
%             \centering
%             \includegraphics[width=0.7\linewidth]{figures/use_case/use_case1_1.jpg}
%             \caption*{
%             \textbf{Ground Truth Caption:} A giraffe standing on top of a lush green field. Giraffe standing in middle of fenced-in area looking to the ground. A giraffe standing in the shade of a tree. \\
%             \textbf{Ours:} A giraffe standing in the middle of a field. \\
%             \textbf{MindBridge:} A train is traveling down the tracks in the city. \\
%             \textbf{UniBrain:} A zebra standing in the middle of a street.
%             }
%         \end{subfigure}
%     \end{tabular}
%     \caption{\textit{Case study of COCO Captioning.}}
%     \label{fig:coco_captioning}
% \end{figure}






In this section, we present a qualitative analysis of our model on COCO Captioning, COCO-QA, and OK-VQA, comparing its performance against MindBridge \cite{wang2024mindbridge} and UniBrain \cite{wang2024unibrain}. As shown in Figure \ref{fig:case1} and \ref{fig:case2_3}, our results demonstrate significant improvements in visual understanding across multiple tasks. The model shows strength in the following areas:
\textbf{Static Object Recognition. }The model demonstrates superior accuracy in identifying stationary objects. In comparison with baseline models (MindBridge and UniBrain), our approach shows improvement in spatial context understanding. For example, when analyzing aircraft imagery (e.g., (a) of Figure \ref{fig:case1}), our model correctly identifies "airplane sitting on the runway" while baselines incorrectly interpret the scene as "flying in the sky" or "flying over a city", demonstrating better state-space recognition. \textbf{Action Recognition. }Our proposed model exhibits enhanced capability in distinguishing between similar actions. In sports scenarios (e.g., (f) of Figure \ref{fig:case2_3}), our model correctly identifies "catch ball" while both baselines incorrectly predict "serve", indicating improved action-state discrimination.  \textbf{Potential of neuroscience application. }The demonstrated improvements in object understanding and action recognition suggest the potential for advancing brain-computer interface technology and neural processing research. The model's enhanced capabilities in distinguishing object states and actions could lead to more effective neural prosthetics and improved assistive technologies for individuals with visual or motor impairments.




\end{document}


% This document was modified from the file originally made available by
% Pat Langley and Andrea Danyluk for ICML-2K. This version was created
% by Iain Murray in 2018, and modified by Alexandre Bouchard in
% 2019 and 2021 and by Csaba Szepesvari, Gang Niu and Sivan Sabato in 2022.
% Modified again in 2023 and 2024 by Sivan Sabato and Jonathan Scarlett.
% Previous contributors include Dan Roy, Lise Getoor and Tobias
% Scheffer, which was slightly modified from the 2010 version by
% Thorsten Joachims & Johannes Fuernkranz, slightly modified from the
% 2009 version by Kiri Wagstaff and Sam Roweis's 2008 version, which is
% slightly modified from Prasad Tadepalli's 2007 version which is a
% lightly changed version of the previous year's version by Andrew
% Moore, which was in turn edited from those of Kristian Kersting and
% Codrina Lauth. Alex Smola contributed to the algorithmic style files.
