% This version of CVPR template is provided by Ming-Ming Cheng.
% Please leave an issue if you found a bug:
% https://github.com/MCG-NKU/CVPR_Template.

\documentclass[review]{cvpr}
%\documentclass[final]{cvpr}
%\documentclass[review]{IEEEtran}

\usepackage{times}
\usepackage{epsfig}
\usepackage{graphicx}
\usepackage{amsmath}
\usepackage{amssymb}


\usepackage{booktabs}
\usepackage{color}
\usepackage{amsmath,bm}
\usepackage{multirow}
\usepackage{float}
\usepackage{subcaption} % 提供子标题功能
\usepackage{tabularray}
\usepackage{pifont}
\usepackage{enumitem}
\usepackage{colortbl}
\usepackage{algorithm}
\usepackage{makecell}
\usepackage{float} 
% Include other packages here, before hyperref.

% If you comment hyperref and then uncomment it, you should delete
% egpaper.aux before re-running latex.  (Or just hit 'q' on the first latex
% run, let it finish, and you should be clear).
\usepackage[pagebackref,breaklinks,colorlinks]{hyperref}


\def\cvprPaperID{****} % *** Enter the CVPR Paper ID here
\def\confName{ICCV}
\def\confYear{2025}
\graphicspath{{./figs/}}
%\setcounter{page}{4321} % For final version only


\begin{document}

%%%%%%%%% TITLE
\title{OpenFly: A Versatile Toolchain and a High-Quality Benchmark for Aerial Vision-and-Language Navigation}

\author{First Author\\
Institution1\\
Institution1 address\\
{\tt\small firstauthor@i1.org}
% For a paper whose authors are all at the same institution,
% omit the following lines up until the closing ``}''.
% Additional authors and addresses can be added with ``\and'',
% just like the second author.
% To save space, use either the email address or home page, not both
\and
Second Author\\
Institution2\\
First line of institution2 address\\
{\tt\small secondauthor@i2.org}
}

\maketitle


%%%%%%%%% ABSTRACT
\begin{abstract}
Vision-and-language navigation (VLN) seeks to guide agents through complex environments using language instructions and visual cues, playing a critical role in embodied AI. Indoor VLN has been extensively studied, whereas outdoor aerial VLN remains underexplored. The potential reason is that outdoor aerial perspectives cover vast scene areas, making data collection challenging, which in turn leads to a lack of corresponding benchmarks. To address this problem, we propose \textbf{OpenFly}, a platform comprising a versatile toolchain and a high-quality benchmark for aerial VLN. \textbf{First}, we develop a highly automated toolchain for data generation, enabling automatic scene semantic extraction, point cloud acquisition, flight trajectory creation, and instruction generation, thus significantly improving efficiency. \textbf{Second}, based on the developed toolchain, we construct a large-scale aerial VLN dataset with \textbf{100k} trajectories, covering samples of diverse heights and difficulty levels across 16 distinct scenes. The corresponding visual data are generated using various rendering engines or techniques, including Unreal Engine (UE), GTA5, Google Earth, and 3D Gaussian Splatting (3D GS). All data feature high visual quality, with 3D GS supporting real-to-sim rendering, further enhancing the realism of the dataset. \textbf{Third}, we propose a keyframe-aware vision-language-action (VLA) model to directly output flight actions based on instructions, current observations, and key histories. Extensive experiments are conducted to evaluate the proposed model and comparison methods, establishing a comprehensive benchmark for aerial VLN. The toolchain, dataset, and codes will be available upon the acceptance of the paper.
%at \url{openfly.github.io}. 
\end{abstract}

%%%%%%%%% BODY TEXT
\section{Introduction}\label{sec:intro}

In computational finance, Monte Carlo simulations are used extensively to estimate the expected value of financial payoffs based on the solution of stochastic differential equations (SDEs) which model the evolution of stock prices, interest rates, exchange rates and other quantities \cite{glasserman04}.  Monte Carlo methods are very general and flexible, but for high accuracy it requires generating a large number of costly SDE path approximations, which has motivated research into a number of variance reduction or, equivalently, cost reduction techniques. One such method is
Multilevel Monte Carlo (MLMC), which was proposed in \cite{GILES2008} and was adapted for various applications that are summarised in \cite{Giles_overview17} and successfully combined with other methods such as quasi-Monte Carlo methods. The main idea of MLMC is to approximate the payoff using different time stepping resolutions when numerically solving the underlying SDE and to generate an optimal number of samples on each level, such that the overall computational cost is minimised subject to the desired bound on the variance. %, such that the total computational cost is minimised. 
The computational savings come from the fact that most samples are computed on the coarser levels and hence are less expensive while only a few samples from the finest levels are required \cite{GILES2008}.


Among the directions in which the computational cost 
of MLMC methods could further be reduced, an important avenue is the use of lower precision calculations, especially for the first Monte Carlo levels where the targeted accuracy is relatively low. 
 An overview of the research on mixed precision for the standard Monte Carlo (MC) framework is provided in \cite{ChowMixedPrecisionStandardMC} but only a few references study the potential of low precision computation in the MLMC framework \cite{Rounding_error_oliver}. To the best of our knowledge, the only MLMC framework with customised precision in the literature is \cite{brugger2014mixed}, but they use a uniform precision for all operations on each Monte Carlo level instead of optimising 
 the precision of each intermediary variable to reduce as much as possible the cost of path generation.
 
An important motivation for an MLMC framework with variable precision would be performing the low precision computations on reconfigurable hardware devices such as Field Programmable Gate Arrays (FPGAs). FPGAs contain customizable logic blocks and connectors that make it easy to adapt the digital circuit architecture for a specific application, leading to a highly parallel and optimised implementation. Therefore they are successfully exploited in applications that require high speed and have high computational workload, such as signal processing \cite{woods2008fpga}, and real time applications like high frequency trading \cite{HFT1,HFT2}. That is why a number of previous works in hardware architecture design implemented the MLMC algorithm to price financial options using FPGAs as accelerators, which resulted in improved speed and power efficiency compared to full CPU architectures \cite{Schryver2013AMM}. The paper \cite{lindsey2016domain} also proposed 
a Domain Specific Language to automate the configuration of FPGAs for this specific application. However, only \cite{brugger2014mixed} proposed a heuristic to reduce the precision in calculations.

In addition, all aforementioned works considered that the random number generation (RNG) is performed in single or double precision. Yet in most cases an important portion of the workload in the overall MLMC simulation comes from the RNG and in \cite{brugger2014mixed} this limited the total computational savings.
To reduce the cost of MLMC simulations in particular those based on the Geometric Brownian Motion (GBM), \cite{approximateICDF_Oliver, NestedOliver} have proposed to use approximate random numbers that are generated by applying an approximation of the inverse CDF to uniform random numbers. In \cite{NestedOliver}, the authors proposed a way to integrate these lower precision random variables into a \textit{nested} MLMC framework and completed a numerical analysis to bound the resulting error at each MC level by a product of the time step and the error in the random number approximation. The same authors show in \cite{approximateICDF_Oliver} that using approximate random variables reduces the cost of path generation by a factor 7.


In this paper we propose a nested MLMC framework that combines the use of approximate random normal variables and lower precision calculations to reduce the computational cost of MLMC even further than \cite{brugger2014mixed,NestedOliver}. We illustrate the efficiency of our framework in Matlab, after making several assumptions on the cost of operations and size of the errors that we carefully justify. We focus on the case of GBM and use the approximate RNG methods presented in \cite{approximateICDF_Oliver} as well as a new slightly modified method that combines CDF inversion and the central limit theorem. To choose the precision of the variables in the low precision path generation, we introduce a novel method to optimise the bit-widths. This optimisation is performed before the main path generation loop is executed and is based on a linear model of the payoff error  
due to rounding when computing in low precision. The error model relies on algorithmic differentiation in a similar manner to \cite{unifying-bwoptim,bitwidth-AD,ADAPT}. The bit-width optimisation procedure can be performed off-line, so this stage can be excluded from the on-line time complexity of our framework. The user specified desired accuracy is then enforced by calculating on-line the number of samples that need to be generated.

In terms of hardware design, we suggest implementing the low precision path generation on FPGAs and the full-precision ones on a CPU or GPU. 
The FPGA offers enough flexibility to define a separate bit-width for every variable in the low precision path generation, and can be reconfigured periodically to update the bit-widths when the market parameters have changed considerably. 


The paper is organized as follows : \Cref{sec:MLMC} introduces MLMC and nested MLMC to make clear the estimator that is implemented in our framework. Then in \Cref{sec:RNG} we detail the methods that could be used to obtain approximate random normally distributed numbers very cheaply for the low precision path generation. In \Cref{sec:error_model} and \Cref{sec:costModel} we propose an error model and a cost model (resp.) that we then use to formulate the optimisation problem that is solved to obtain the optimal bit-widths of fixed point variables in \Cref{sec:optimisation}. Finally we summarise our results and future directions in \Cref{sec:conclusion}.



\section{Related Work}
\label{sec:related_work}

The original investigation \cite{gibson1979ecological} on the relationship between visual perception and human action defines \emph{affordance} as the opportunities for interaction with the surrounding environment. Behavioral studies on regular and cognitively impaired persons have shown evidence that perception results in both visual and motor signals in the human brain. An extended study \cite{anderson2002attentional} shows that visual attention to the spatial characteristics of the perceived objects initiates automatic motor signals for different actions. In computer vision, human affordance learning involves novel pose prediction such that the estimated pose represents a valid human action within the scene context. The task is fundamental to many problems requiring robust semantic reasoning about the environment, such as human motion synthesis \cite{wang2021scene} and scene-aware human pose generation \cite{wang2017binge, roy2016multi, zhang2022inpaint, yao2023scene}.

Earlier methods of affordance learning have explored knowledge mining \cite{zhu2014reasoning} and multimodal feature cues \cite{roy2016multi} to address the problem. In \cite{zhu2014reasoning}, the authors use a Markov Logic Network for constructing a knowledge base by extracting several object attributes from different image and metadata sources, which can perform various downstream visual inference tasks without any additional classifier, including zero-shot affordance prediction. In \cite{roy2016multi}, the authors use depth map, surface normals, and segmentation map as multimodal cues to train a multi-scale convolutional neural network (CNN) for scene-level semantic label assignment associated with specific human actions. In \cite{do2018affordancenet}, the authors design a multi-branch end-to-end CNN with two separate pathways for object detection and affordance label assignment to achieve high real-time inference throughput. Researchers \cite{chuang2018learning} have also explored socially imposed constraints for affordance learning. In \cite{chuang2018learning}, the authors propose a graph neural network (GNN) to propagate contextual scene information from egocentric views for action-object affordance reasoning.

Probabilistic modeling of scene-aware human motion generation also involves semantic reasoning of human interaction with the environment. Initial works on human motion synthesis have taken different architectural approaches, such as sequence-to-sequence models \cite{barsoum2018hp}, generative adversarial networks (GAN) \cite{barsoum2018hp, cai2018deep, yang2018pose}, graph convolutional networks (GCN) \cite{yan2019convolutional}, and variational autoencoders (VAE) \cite{guo2020action2motion}. However, these methods have mostly ignored the role of environmental semantics. Due to potential uncertainty in human motion, in a recent approach \cite{wang2021scene}, the authors address such motion synthesis with a GAN conditioned on scene attributes and motion trajectory to predict probable body pose dynamics.

One key challenge of human affordance generation in 2D scenes is the lack of large-scale datasets with rich pose annotations. In \cite{wang2017binge}, the authors compile the only public dataset of annotated human body poses in complex 2D indoor scenes by extracting frames from sitcom videos. Aiming to generate a contextually valid human affordance at a user-defined location, the authors propose sampling the scale and deformation parameters for an existing human pose template using a VAE conditioned on the localized image patches as scene context. In \cite{zhang2022inpaint}, the authors introduce a two-stage GAN architecture for achieving a similar goal by estimating the affine bounding box parameters to localize a probable human in the scene and then generating a potential body pose at that location. The method uses the input scene, corresponding depth, and segmentation maps as semantic guidance. In \cite{yao2023scene}, the authors propose a transformer-based approach with knowledge distillation for generating human affordances in 2D indoor scenes.


\section{\method Dataset}

In this section, we introduce the \method dataset, which covers authentic data of 17,966 characters from 771 renowned books. 
\method features its authentic, non-synthesized dialogues with real-world intricacies, and comprehensive data representations supporting various usages. 
In Table ~\ref{tab:dataset_stats}, we provide a comprehensive comparison with existing datasets. 
We illustrate our dataset's design principles in \S\ref{sec:data_design},  curation pipeline in \S\ref{sec:data_pipeline}, and statistical analysis in \S\ref{sec:data_statistics}.

\begin{figure*}[!t]
    \centering
    \includegraphics[width=\textwidth, center]{Figures/CoSER-main.pdf}
    \vspace{0.2cm}
    \caption{
    Overview of \method's dataset, training and evaluation. 
    Left: The \method dataset is sourced from renowned books and processed via LLM-based pipeline. 
    It contains rich data types on plots, conversations and characters.  
    Right: 
    We apply given-circumstance acting to train and evaluate role-playing LLMs using these conversations.  
    For training, each sample trains the LLM to portray  a specific character in a conversation, using their  original dialogue.  
    For evaluation, 
    we build a multi-agent system for conversation simulation given the same scenario, and assess the simulated dialogue via  penalty-based LLM critics. 
    }
    \label{fig:main}
\end{figure*}

\subsection{Design Principles}
\label{sec:data_design}

As shown in Table ~\ref{tab:dataset_stats}, \method differs from previous RPLA datasets mainly in its:  
\textit{1)} rich data types, 
\textit{2)} internal thoughts and physical actions in messages,
\textit{3)} environment as a role.


\textbf{Rich Types of Data} \quad 
The persona data $\personadata_\persona$ can represent a character $\persona$ from fictional works in diverse forms, \eg, narratives, profiles, dialogues, experiences, \etc. 
Previous work focuses primarily on profiles and dialogues, which represent limited knowledge.  
Hence, we propose a more comprehensive set of data types that are: 
\textit{1)} Comprehensive: covering extensive knowledge about characters and plots from the books; 
\textit{2)} Orthogonal: carrying distinct, complementary information with little redundancy;
\textit{3)} Contextual-rich: providing sufficient context to enable $\agent_\persona$ to faithfully reproduce $\persona$'s behaviors and responses in given scenarios.


Specifically, we organizes knowledge from books hierarchically via three interconnected elements: plots, conversations and characters. 
Each \textbf{plot} comprises its raw text, summary, conversations in this plot, and key characters' current states and experiences in this plot. 
A \textbf{conversation} contains not only the dialogue transcripts, but also rich contextual settings including scenario descriptions and characters' motivations. 
\textbf{Characters} are associated with their conversations and plots, based on which we craft their profiles. 

\textbf{Thoughts and Actions in Messages} \quad 
Previous RPLA studies typically restrict RPLAs' output space to verbal speech alone, limiting their ability to fully represent human interactions. 
In this paper, we extend the message space of RPLAs and character datasets into three distinct dimensions: speech ($\mathcal{L}$), action ($\mathcal{A}$), and thought ($\mathcal{T}$), significantly enriching the expressiveness. 
For instance, an RPLA can convey silence by generating only thoughts and actions without verbal speech. 
The three dimensions are distinguished by markup symbols and function mechanisms:  
\begin{itemize}[itemsep=-3pt, topsep=0pt, partopsep=0pt]
    \item \textbf{Speech} is for verbal communications of characters.
    \item \textbf{Action} captures physical behaviors, body language, facial expressions, \etc. Similar to  tool use in agents~\citep{weng2023agent}, actions can be programmed to trigger downstream events in multi-agent systems. 
    \item \textbf{Thought} represents internal thinking processes, which enable RPLAs to simulate sophisticated human cognition. 
    Thoughts should be invisible to others, forming  information asymmetry~\citep{zhou2024sotopia}. 
\end{itemize}

\textbf{Environment as a Role} \quad 
In RPLA applications like AI TRPG~\footnote{Tabletop Role-Playing Games}~\citep{liang2023tachikuma}, LLMs often serve as world simulators that respond to players' actions. 
To promote this ability, we consider environment as a special role $e$, which provide environmental responses such as physical changes and reactions from unspecified characters or crowds.

\subsection{Dataset Curation} 
\label{sec:data_pipeline}

We curate the \method dataset through a systematic LLM-based pipeline that transforms book content into high-quality data for RPLAs  
~\footnote{In this paper, we employs Claude-3.5-Sonnet (20240620).}. 
The details are as follows. 

\textbf{Source Selection} \quad 
Our dataset is sourced from most acclaimed literary works to ensure data quality and character depth. 
We identify the top 1,000 books on \textit{Goodreads}'s \textit{Best Books Ever} list~\footnote{https://www.goodreads.com/list/show/1.Best\_Books\_Ever}, and obtain the content for 771 books.
As shown in Table~\ref{tab:selected_books}, these books  offer characters and narratives with literary significance and widespread recognition across diverse genres, time periods, and cultural backgrounds.

\textbf{Chunking} \quad 
We segment book contents into chunks to fit in LLMs' context window. 
We employs both static, chapter-based strategy and dynamic, plot-based strategy. 
Initially, we use regular expressions to identify chapter titles as natural chunk boundaries. 
Then, we merge adjacent small chunks and split large chunks to ensure moderate chunk sizes. 
However, static chunking neglects the storyline and truncates important plots or conversations. 
To address this, we implement dynamic plot-based chunking, \ie, during data extraction, we also prompt LLMs to identify truncated plots or trailing content in the current chunk, and concatenate them with the subsequent chunk to ensure plot integrity.


\textbf{Data Extraction} \quad 
We employ LLMs to extract plot and conversation data from book chunks, including (1) contents, summaries and character experiences of plots, and (2) dialogues and background settings of conversations. 
The extracted data representations are illustrated in Fig. ~\ref{fig:front} and introduced in \S\ref{sec:data_design}. 
In the messages, speeches are always extracted from the original dialogues, while actions and thoughts can either be extracted or inferred by LLMs based on the context. 
For evaluation purposes, we hold out data from the final 10\%  plots in each book.


\textbf{Organizing Character Data} \quad 
Based on the extracted data, we form the knowledge bases for characters in three steps.  
First, we unify character references by establishing name mappings between aliases and canonical names using LLMs, \eg, mapping \textit{Lord Snow} to an unified identifier \textit{Jon Snow}. 
Second, we aggregate relevant plots and conversations for each character. 
Finally, we leverage LLMs to generate character profiles based on their extracted data, describing them from multiple perspectives including background, experiences, physical characteristics, personality traits, core motivations, relationships, character arcs, \etc. 

For technical details, including our prompts, engineering implementation, and handling mechanisms for exception caused by LLMs, please refer to ~\S\ref{sec:app_dataset}. 




%Task/Problem formulation
\subsection{Problem Definition}
In the aerial VLN task, a UAV is randomly positioned within a 3D environment with its initial pose defined as $P = [x, y, z, \phi, \theta, \psi]$. At each timestamp $t$, the UAV perceives the surrounding environment through an egocentric image as its observation. Guided by natural language instructions, the task involves predicting the next navigation action. Notably, the UAV can utilize either the current observations or the frames from all previous timestamps to make its prediction.

\subsection{Model Architecture}
As shown in Fig. \ref{fig:model}, we take OpenVLA~\cite{openvla} as the baseline and design an end-to-end model for aerial VLN. In contrast, our model takes a sequence of images to indicate the observation instead of one image in the original OpenVLA. Moreover, to mitigate visual redundancy between adjacent video frames while maintaining key information, two strategies are proposed, \emph{i.e.,} keyframe selection and visual token merging. First, a series of candidate keyframes are selected. Then, these keyframes are merged temporally before and after the vision encoder, resulting in a compact sequence of visual tokens. Finally, the action decoder discretizes the predicted tokens into uniformly distributed bins, which are subsequently mapped to the 6 action types specific to drones. 


\subsubsection{Keyframe Selection}
The length of contextual visual tokens is a major challenge for VLMs when processing videos. Many open-source VLMs use uniform frame sampling \cite{buch2022revisiting, ranasinghe2024understanding, wang2025videoagent} to reduce calculation, but this strategy is not suitable for aerial VLN, since it may miss frames containing key landmarks. 
To address this issue, we adopt a heuristic method to identify keyframes by detecting the change point of the UAV's movement. We notice that sudden changes in the UAV's trajectory are often caused by the observation of landmarks, which can serve as cues to determine keyframes. Specifically, we use the movement of the drone over time to draw turning curves, and the frames near the peaks of the wave are selected as candidate keyframes. The resulting data is interpolated and smoothed, forming a wave-like curve that represents the UAV's movement. 

To further ensure the precision of training data, scene segmentation maps collected in Sec. \ref{sec:Automatic} are used on selected frames to detect key landmarks. Frames containing landmarks are selected as keyframes, yielding reasonably accurate results. Note that each sudden change of actions, \emph{e.g.,} from `Forward' to `Turn Left', will produce a set of keyframes. Consequently, we obtain several sets of keyframes for a long trajectory. 
%For testing, we select keyframes where the action changes, as these often correspond to the observation of a critical landmark.
%Sec Parag

%This keyframe selection scheme gives model the guidance for action prediction via semantic relationship from the observation of the subgoal. Next, with the candidate frame sequences, we introduce the online visual Token merging module for the next action prediction.

\subsubsection{Visual Token Merging}
To further reduce redundant information in keyframes, we design visual token merging, where the core concept is to recognize the similarity between image tokens. It compares adjacent keyframes to merge similar regions and maintains its simplicity by token compression.

%合并阶段。
% 在获得候选帧之后,我们先逐帧过一遍vision encoder获取visual features,再利用标记相似性来合并相邻帧的视觉标记。类似 ToMe [] ,我们通过定期合并之后相邻帧中最相似的标记来进行记忆巩固。我们计算 N 个嵌入标记之间的平均余弦相似度s,在每次合并操作后保留K帧,这也嵌入了存储在长期记忆中的丰富信息。K是控制性能和效率之间权衡的超参数。因此,我们通过加权平均地合并每组相邻帧相似度最高的tokens。合并操作迭代进行,直到token计数达到每个合并操作的预定义值集K。合并阶段应用于Vision Transformer的倒数第二层特征patch token,以逐步合并相似的标记,直到相似标记的数量低于特定层的阈值 Nthreshold。合并阶段之后,剩余的唯一标记将进入压缩阶段。


\begin{table*}[t!]

\centering
\caption{Comparison results on the test-seen split.}
%\vspace{-5pt}
\label{tab:seen_results}
\begin{adjustbox}{center}
\resizebox{\textwidth}{!}{ 
%\setlength{\tabcolsep}{1.6pt}
\renewcommand{\arraystretch}{1.3}
% \scalebox{0.95}{
\begin{tabular}{lccccccccccccccccc}
\toprule
\multirow{2}{*}{Method} & \multicolumn{4}{c}{Easy} & \multicolumn{4}{c}{Moderate} & \multicolumn{4}{c}{Hard} & \multicolumn{4}{c}{Total}\\ 
\cmidrule(lr){2-5} \cmidrule(lr){6-9} \cmidrule(lr){10-13} \cmidrule(lr){14-17}
& NE$\downarrow$ & SR$\uparrow$ & OSR$\uparrow$ & SPL$\uparrow$ 
& NE$\downarrow$ & SR$\uparrow$ & OSR$\uparrow$ & SPL$\uparrow$ 
& NE$\downarrow$ & SR$\uparrow$ & OSR$\uparrow$ & SPL$\uparrow$
& NE$\downarrow$ & SR$\uparrow$ & OSR$\uparrow$ & SPL$\uparrow$ \\ \midrule 

Random & 289m & 0.9\% & 1.1\% & 0\% & 351m & 1.3\% & 1.3\% & 0\% & 374m & 0\% & 0\% & 0\% & 242m & 0.7\% & 0.8\% & 0\% \\
Seq2Seq\cite{VLN-CE}&  201m &  0.9\% & 21.2\% & 0.9\% & 190m & 8.9\% & 19.2\% & 6.5\% & 192m & 2.1\% & 10.1\% & 1.9\% & 194m & 4.0\% & 16.8\%  &  3.1\% \\
CMA\cite{VLN-CE}&  156m & 1.2\% &  35.6\% & 1.6\% & 120m & 11.2\% & 34.5\% & 8.4\% & 156m & 4.6\% & 20.1\% & 5.3\% & 144m & 5.7\% & 30.0\% & 5.1\%\\
AerialVLN\cite{aerialVLN}& \underline{148m} & 1.5\% &  \underline{40.2\%} & 2.6\% & \textbf{94m} & \underline{13.2\%} & \textbf{58.6\%} & \underline{10.7\%} & 147m & 5.4\% & \underline{23.6\%} & \underline{7.6\%} & \underline{130m} & 6.6\% &\underline{40.8\%} & \underline{7.0\%}\\
Navid\cite{navid}& 151m & \underline{11.2\%} & 28.9\% & \underline{4.5\%} & 138m & 8.0\% & 21.3\% & 2.8\% & \underline{134m} & \textbf{10.3\%} & 21.3\% & 4.6\% & 142m & \underline{9.9\%} & 24.3\% & 3.9\% \\
Ours& \textbf{111m} &  \textbf{26.5\%} & \textbf{55.6\%}  & \textbf{16.0\%} & \underline{115m} & \textbf{16.4\%} & \underline{51.2\%} & \textbf{11.2\%} & \textbf{120m} & \textbf{10.3\%} & \textbf{29.6\%} & \textbf{8.2\%}  & \textbf{115m} & \textbf{18.5\%} & \textbf{50.9\%} & \textbf{12.2\%} \\
\bottomrule
\end{tabular}
}
\end{adjustbox}
\end{table*}



For each set of candidate keyframes obtained in the previous selection process, a visual encoder maps each input image to multiple visual tokens, with each token representing the information of an image patch. Considering the potential inter-frame patch redundancy, we take a strategy that similar tokens in subsequent adjacent frames are periodically merged. Specifically, we select the first frame in a keyframe set as the reference, since it usually contains the crucial observation indicating the time for action transition. Then, we densely calculate the cosine similarities between each pair of visual tokens of the reference image and the subsequent image. Next, we merge the tokens with high similarity by averaging them. The unmerged tokens in the subsequent frame will be discarded. The merging operation is iteratively performed until the entire keyframe set has been traversed. Besides, we maintain a memory bank with a capacity of $K$ images, which follows a first-in-first-out (FIFO) policy to retain the latest keyframes.

After the above process, $M$ visual tokens $E=\{e_1, e_2, \cdots, e_M\}$ are obtained for each set of keyframes. Since aerial VLN requires UAVs to perform long-distance flights based on instructions, we continue to carry out token compress to reduce the computational burden. The compressed visual tokens $E_c$ are obtained through grid pooling~\cite{llama_vid}. Notably, we keep the visual tokens of the current frame uncompressed to capture the latest visual observation, as it contains the most important information for flight action prediction.




\subsubsection{Action Prediction}
Similar to~\cite{aerialVLN,CityNav}, 6 actions for UAVs are defined as $\{$Forward, Turn Left, Turn Right, Move Up, Move Down, Stop$\}$ in this work. The units for `Move up' and `Move down' are 3 m, the units for `Turn Left' and `Turn Right' are 30 degrees. `Forward' has three distinct units, namely 3 m, 6 m, and 9 m, respectively. For flight action prediction, each action type is discretized into multiple bins with one non-activate bin indicating that the current action is not activated. We map the model output to one of the bins for each action type, where the bin number corresponds to the amount of units in each action.


\section{Experiments}
\textbf{Setup.} We evaluate the performance of PINNMamba on three standard PDE benchmarks: convection, wave, and reaction equations, all of which are identified as being affected by failure modes~\cite{krishnapriyan2021characterizing,zhao2024pinnsformer}. The details of those PDEs can be found in Appendix~\ref{apx:setup}.
    We compare PINNMamba with four baseline models, vanilla PINN~\cite{raissi2019physics}, QRes~\cite{bu2021quadratic}, PINNsFormer~\cite{zhao2024pinnsformer}, and KAN~\cite{liu2024kan} .
For fair comparison, we sample 101$\times$101 collection points with uniformly grid sampling, following previous work~\cite{zhao2024pinnsformer,wu2024ropinn}. We also evaluate on PINNacle Benchmark~\cite{hao2023pinnacle} and Navier–Stokes equation~\cite{raissi2019physics}.

\begin{table*}
\vspace{-3mm}
  \caption{Results for solving convection, reaction, and wave equations.}
  \label{sample-table}
  
  \centering
    \small
  \begin{tabular}{l|c|ccc|ccc|ccc}

    \toprule 
  & & \multicolumn{3}{c}{Convection }&\multicolumn{3}{c}{Reaction}&\multicolumn{3}{c}{Wave}\\
    \cmidrule(lr){3-5}\cmidrule(lr){6-8}\cmidrule(lr){9-11}
   Model & \#Params &Loss & rMAE & rRMSE & Loss & rMAE & rRMSE& Loss & rMAE & rRMSE
 \\   \midrule
    PINN&527361& 0.0239 & 0.8514 & 0.8989& 0.1991 & 0.9803 & 0.9785& 0.0320 & 0.4101 & 0.4141\\
    QRes & 396545& 0.0798 & 0.9035 & 0.9245& 0.1991 & 0.9826 & 0.9830& 0.0987 & 0.5349 & 0.5265\\
    PINNsFormer &453561 & 0.0068 & 0.4527 & 0.5217& 3e-6& 0.0146 & 0.0296 & 0.0216 & 0.3559 & 0.3632\\
     KAN&891& 0.0250 & 0.6049 & 0.6587& 7e-6 & 0.0166 & 0.0343& 0.0067 & 0.1433 & 0.1458\\
   \rowcolor{mygray}   PINNMamba  & 285763&0.0001 & \textbf{0.0188} & \textbf{0.0201}&1e-6&\textbf{0.0094}&\textbf{0.0217}& 0.0002 & \textbf{0.0197} & \textbf{0.0199} \\

    \bottomrule
  \end{tabular}
  \normalsize
  \label{tab:diff}
  \vspace{-4mm}
\end{table*}

\begin{figure*}[t!]
    \centering
    \includegraphics[width=\textwidth]{_fig/wave}
    \vspace{-8mm}
    \caption{The ground truth solution, prediction (top), and absolute error (bottom) on wave equations.}
    \label{fig:wave}
    \vspace{-5mm}
  %  \vspace{-1mm}
\end{figure*}

\textbf{Training Details.} We train PINNMamba and all the baseline models 1000 epochs with L-BFGS optimizer~\cite{liu1989limited}.
We set the sub-sequence length to 7 for PINNMamba, and keep the original pseudo-sequence setup for PINNsFormers. The weights of loss terms $[\lambda_\mathcal F,\lambda_\mathcal I,\lambda_\mathcal B]$ are set to $[1,1,10]$ for all three equations, as we find that strengthening the boundary conditions can lead to better convergence. $\lambda_\text{alig}$ is set to 1000 for convection and reaction equations, and auto-adapted by $\lambda_\mathcal F$ for wave equation.
%Loss weights are also actively adapted by neural tangent kernel~\cite{wang2022and} for wave equations for test the orthogonality of PINNMamba with other methods.
All experiments are implemented in PyTorch 2.1.1 and trained on an NVIDIA H100 GPU.  More training details are in Appendix~\ref{apx:hyperparam}. Our code and weights are available at \url{https://github.com/miniHuiHui/PINNMamba}.

\textbf{Metrics.} To evaluate the performance of the models, we take relative Mean Absolute Error (rMAE, a.k.a  $\ell_1$ relative error) and relative Root Mean Square Error (rRMSE, a.k.a $\ell_2$ relative error) following common practive~\cite{zhao2024pinnsformer,wu2024ropinn}. The metrics are formulated as:
\begin{align}
\text { rMAE }(\hat u)&=\frac{\sum_{n=1}^N\left|\hat{u}\left(x_n, t_n\right)-u\left(x_n, t_n\right)\right|}{\sum_{n=1}^{N}\left|u\left(x_n, t_n\right)\right|}, \\
\text { rRMSE }(\hat u)&=\sqrt{\frac{\sum_{n=1}^N\left|\hat{u}\left(x_n, t_n\right)-u\left(x_n, t_n\right)\right|^2}{\sum_{n=1}^N\left|u\left(x_n, t_n\right)\right|^2}},
\end{align}
where N is the number of test points, $u(x,t)$ is the ground truth solution, and $\hat u(x,t)$ is the model's prediction.

\vspace{-2mm}

\subsection{Main Results}
\vspace{-1mm}
We present the rMAE and rRMSE for approximating convection, reaction and wave equation's solution in Table~\ref{tab:diff}. Our model consistently outperforms other model architectures, achieving new state-of-the-art.
Notably, as shown in Fig.~\ref{fig:conv}, for the convection equation, PINNMamba allows sufficient propagation of information about the initial conditions, whereas on all the other models there is a varying degree of distortion in the time coordinates.
    As shown in Fig.~\ref{fig:reac}, PINNMamba can further optimize at the boundary, resulting in a lower error than KAN and PINNsFormer for reaction equations. For problems as intrinsically difficult to optimize as the wave, as in Fig.~\ref{fig:wave}, PINNMamba effectively combats simplicity bias and aligns the scales of multi-order differentiation, and thus achieves significantly higher accuracy. This illustrates that PINNMamba can be effective against PINN's failure modes. It's also worth noting that, PINNMamba has the lowest number of parameters (except KAN), while achieving consistently the best performance.

\begin{table}
\vspace{-3mm}
  \caption{Integrating PINNMamba with advanced training strategies and loss auto-balancing strategy. The rMAE is reported here.}
  
  \centering
    \small
  \begin{tabular}{lccc}

    \toprule 
    Method & Convection & Reaction & Wave\\
   \midrule
   PINNMamba & 0.0188 & 0.0094 & 0.0197\\
   +gPINN & 0.0172& 0.0123 & 0.0264 \\
   +vPINN & 0.0236 & 0.0092& 0.0169\\
   +RoPINN & 0.0102& 0.0099& 0.0121\\
    \midrule
    +NTK &0.0179& 0.0079& 0.0147\\
    +NTK+RoPINN &0.0127& 0.0072& 0.0106\\
   

    \bottomrule
  \end{tabular}
  \normalsize
  \label{tab:para}
  \vspace{-6mm}
\end{table}

\begin{figure*}[t!]
    \centering
    \includegraphics[width=\textwidth]{_fig/reac}
    \vspace{-8mm}
    \caption{The ground truth solution, prediction (top), and absolute error (bottom) on reaction equations.}
    \label{fig:reac}
    \vspace{-5mm}
  %  \vspace{-1mm}
\end{figure*}


\subsection{Combination with Other Methods}
\vspace{-1mm}
Since PINNMamba mainly focuses on model architecture, it can be integrated with other methods effortlessly. 
    We explore the feasibility and their performance in combination with advanced training paradigm, as well as loss balancing.

\textbf{Training Paradigm.} We show the rMAE of PINNMamba when integrated with advanced strategies in Table~\ref{tab:para}. We observe that gPINN~\cite{yu2022gradient} and vPINN~\cite{kharazmi2019variational} erratically deliver some performance gains on some tasks. 
    This is due to the fact that the regularization provided by gPINN and vPINN in the form of a loss function through the gradient and variational residuals has little effect on PINNMamba, since SSM itself is sufficiently regularized. RoPINN~\cite{wu2024ropinn} reduces the PINNMamba's error on convection and wave equations by about 40\%, since it complements the spatial continuity dependency.

\textbf{Neural Tangent Kernel.} Dynamic tuning of losses via Neural Tangent Kernel(NTK)~\cite{wang2022and} has been shown to have the effect of smoothing out the loss landscape. 
PINNMamba also works well with the NTK-adopted loss function. As shown in Table~\ref{tab:para}, NTK can reduce PINNMamba error by 5-25\%. 
The combination of RoPINN and NTK can further improve the overall performance of PINNMamba, which demonstrates the excellent suitability of PINNMamba with other PINN optimization methods.

\begin{figure}[t!]
    \centering
    \includegraphics[width=\linewidth]{_fig/loss_error}
    \vspace{-4mm}
    \caption{Loss and $\ell_1$-Error Curve w.r.t Training Iteration.}
    \label{fig:losserror}
    \vspace{-4mm}
  %  \vspace{-1mm}
\end{figure}
\vspace{-2mm}
\subsection{Loss-Error Consistency Analysis}
\vspace{-1mm}

Our other interest is the role of PINNMamba for the elimination of simplicity bias. Models affected by simplicity bias that fall into over-smoothing solutions will show inconsistent decreasing trends in loss and error during training. 
    As shown in Fig.~\ref{fig:losserror}, in the training process for solving convection equations, the rMAE of PINN doesn't descend as $\mathcal L_\mathcal F$ and $\mathcal L_\mathcal I$. 
        This suggests that PINN is trapped in an over-smoothing solution, which is in agreement with our observation in Fig.~\ref{fig:conv}. 
As a comparison, we find that PINNMamba's losses descent processes show a high degree of consistency with its error descent process. 
    This indicates that PINNMamba does not tend to fall into a local optimum of oversimplified patterns.
        Instead, it tends to exhibit patterns that are consistent with the original PDEs.

\vspace{-2mm}
\subsection{Ablation Study}
\vspace{-1mm}
\begin{table*}
  [t]
  \centering
  \resizebox{\textwidth}{!}{%
  \begin{tabular}{cccccccccccc}
    \toprule \multicolumn{2}{c}{Components}                                                             & \multicolumn{5}{c}{Re-executability Rate (\%)} & \multicolumn{5}{c}{Readability (\#)} \\
    \cmidrule(lr){1-2} \cmidrule(lr){3-7} \cmidrule(lr){8-12}        \hspace{8pt}\labelemoji\hspace{8pt}                                                                & \hspace{8pt}\toolemoji\hspace{8pt}                                      & O0                                 & O1             & O2             & O3             & AVG            & O0             & O1             & O2             & O3             & AVG            \\
    \hline
    \rowcolor[rgb]{0.93,0.93,0.93}\multicolumn{12}{c}{\textbf{Initialize with LLM4Decompile-End-6.7B~\citep{llm4decompile}}}   \\
    \xmark                                                                                              & \xmark                                    & 69.51                              & 46.95          & 50.61          & 46.34          & 53.35          & 3.98 & 3.41 & 3.44 & 3.38 & 3.55 \\
    \cmark                                                                                              & \xmark                                    & 75.61                              & 50.61          & 50.00          & 50.00          & 56.55          & 4.01 & 3.44 & 3.39 & \textbf{3.49} & 3.58 \\
    \xmark                                                                                              & \cmark                                    & 83.54                     & \textbf{56.10}          & 51.22          & 50.61 & 60.37 & 4.05 & 3.51 & 3.51 & 3.42 & 3.62 \\
    \cmark                                                                                              & \cmark                                    & \textbf{85.37}                            & \textbf{56.10}                     & \textbf{51.83} & \textbf{52.43}          & \textbf{61.43} & \textbf{4.13} & \textbf{3.60} & \textbf{3.54} & \textbf{3.49} & \textbf{3.69} \\

    \rowcolor[rgb]{0.93,0.93,0.93}\multicolumn{12}{c}{\textbf{Initialize with Deepseek-Coder-6.7B-base~\citep{deepseekcoder}}} \\
    \xmark                                                                                              & \xmark                                    & 59.15                              & 35.98          & 39.02          & 37.80          & 42.99          & 3.71 & 3.05 & 3.16 & 3.05 & 3.24 \\
    \cmark                                                                                              & \xmark                                    & 66.46                              & 41.46          & 38.41          & 36.59          & 45.73          & 3.76 & 3.17 & \textbf{3.21} & 3.08 & 3.31 \\
    \xmark                                                                                              & \cmark                                    & 70.73                              & 39.63          & 39.02          & 40.24          & 47.41          & 3.90 & 3.17 & 3.08 & 3.11 & 3.31 \\
    \cmark                                                                                              & \cmark                                    & \textbf{79.88}                     & \textbf{45.73} & \textbf{43.90} & \textbf{42.68} & \textbf{53.05} & \textbf{3.96} & \textbf{3.21} & 3.18 & \textbf{3.19} & \textbf{3.38} \\
    \bottomrule
  \end{tabular}%
  }
  \caption{The ablation study of different methods across four optimization levels
  (O0, O1, O2, O3), as well as their average scores (AVG). The results in bold represent the optimal performance. The ~\labelemoji~ and ~\toolemoji~ means Relabedling and Function Call. \textbf{Bold} denotes the best performance.}
  \label{tab:ablation}
\end{table*}

To verify the validity of the various components of the PINNMamba, as shown in Table~\ref{tab:ablation}, we evaluate the performance of models subtracting these components from PINNMamba.

\textbf{Sub-Sequence.} We remove the sub-sequence alignment, which leads to a decrease in model performance, indicating the significance of the agreement formed through alignment in eliminating simplicity bias.
After replacing the sub-sequence with a long sequence of the entire domain, the model shows failure modes, in line with the sequence granularity analysis in Section~\ref{sec:subseq}.

\textbf{Time-Varying SSM.} We replace the selective SSM~\cite{gu2023mamba} with a linear time-invariant structure SSM~\cite{gu2022efficiently}, and there is some decrease in model performance, illustrating the role of predictive diversity in eliminating simplicity bias. 
And when we remove SSM completely and switch to MLP instead, the model has severe failure modes. 
        This demonstrates that SSM's adaptation for \textit{Continuous-Discrete Mismatch} allows the initial condition information to propagate sufficiently in time coordinates.

In addition, we also conducted a sensitivity analysis of the choice of sub-sequence length, activation. See Appendix~\ref{apx:sense}.

\vspace{-3mm}
\subsection{Experiments on Complex Problems}
\vspace{-1mm}
To further demonstrate the generalization of our method, we tested our model on partial PINNacle Benchmark~\cite{hao2023pinnacle} and Navier-Stokes equations. As shown in Fig.~\ref{fig:ns}, PINNMamba achieves the lowest error on the N-S equation. Just like PINNsFormer, PINNMamba also gets out-of-memory on some problems in PINNacle, which we identify as a major limitation of sequence-based methods. We discuss the details of PINNacle experiments in Appendix~\ref{apx:comp}.

\begin{figure}[t!]
    \centering
    \includegraphics[width=\linewidth]{_fig/NS}
    \vspace{-6mm}
    \caption{Absolute Error of pressure prediction of N-S equation}
    \label{fig:ns}
    \vspace{-3mm}
  %  \vspace{-1mm}
\end{figure}

\section{Limitations and Future Work}
The proposed OpenFly platform incorporates various rendering engines/techniques to provide high-quality scenes. Specifically, this is the first attempt to use 3D GS reconstructed scenes to support real-to-sim training and testing, while in the reconstruction of large-scale areas, a few visual artifacts are inevitably present. Future work will focus on exploring more effective reconstruction methods to enhance realism in large-scale scenes. Besides, the proposed OpenFly-Agent is built upon the large VLN model architecture, which is not practical for real-time deployment on UAVs. To address this, future research should focus on developing more efficient architectures and effective quantization techniques. 


\section{Conclusion}
In this work, we present OpenFly, a platform designed for large-scale data collection in aerial Vision-and-Language Navigation (VLN). OpenFly integrates multiple rendering engines and advanced real-to-sim techniques for data generation, enabling efficient collection of diverse, high-quality aerial VLN data. The resulting large-scale dataset comprises 100k trajectories across 18 distinct scenes, spanning a wide range of altitudes and difficulty levels, which is significantly superior than existing ones. Furthermore, we propose OpenFly-Agent, a keyframe-aware aerial navigation model capable of directly predicting flight actions based on observations and language instructions. Extensive experiments validate the effectiveness of the proposed method, and establishing a comprehensive benchmark for future advancements in aerial navigation. 
%The toolchain, dataset, and code will be publicly released, providing a valuable resource for future research in this field.


{\small
\bibliographystyle{ieee_fullname}
\bibliography{BibForOpenFly}
}

\end{document}
