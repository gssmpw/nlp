% This version of CVPR template is provided by Ming-Ming Cheng.
% Please leave an issue if you found a bug:
% https://github.com/MCG-NKU/CVPR_Template.

\documentclass[review]{cvpr}
%\documentclass[final]{cvpr}
%\documentclass[review]{IEEEtran}

\usepackage{times}
\usepackage{epsfig}
\usepackage{graphicx}
\usepackage{amsmath}
\usepackage{amssymb}


\usepackage{booktabs}
\usepackage{color}
\usepackage{amsmath,bm}
\usepackage{multirow}
\usepackage{float}
\usepackage{subcaption} % 提供子标题功能
\usepackage{tabularray}
\usepackage{pifont}
\usepackage{enumitem}
\usepackage{colortbl}
\usepackage{algorithm}
\usepackage{makecell}
\usepackage{float} 
% Include other packages here, before hyperref.

% If you comment hyperref and then uncomment it, you should delete
% egpaper.aux before re-running latex.  (Or just hit 'q' on the first latex
% run, let it finish, and you should be clear).
\usepackage[pagebackref,breaklinks,colorlinks]{hyperref}


\def\cvprPaperID{****} % *** Enter the CVPR Paper ID here
\def\confName{ICCV}
\def\confYear{2025}
\graphicspath{{./figs/}}
%\setcounter{page}{4321} % For final version only


\begin{document}

%%%%%%%%% TITLE
\title{OpenFly: A Versatile Toolchain and a High-Quality Benchmark for Aerial Vision-and-Language Navigation}

\author{First Author\\
Institution1\\
Institution1 address\\
{\tt\small firstauthor@i1.org}
% For a paper whose authors are all at the same institution,
% omit the following lines up until the closing ``}''.
% Additional authors and addresses can be added with ``\and'',
% just like the second author.
% To save space, use either the email address or home page, not both
\and
Second Author\\
Institution2\\
First line of institution2 address\\
{\tt\small secondauthor@i2.org}
}

\maketitle


%%%%%%%%% ABSTRACT
\begin{abstract}
Vision-and-language navigation (VLN) seeks to guide agents through complex environments using language instructions and visual cues, playing a critical role in embodied AI. Indoor VLN has been extensively studied, whereas outdoor aerial VLN remains underexplored. The potential reason is that outdoor aerial perspectives cover vast scene areas, making data collection challenging, which in turn leads to a lack of corresponding benchmarks. To address this problem, we propose \textbf{OpenFly}, a platform comprising a versatile toolchain and a high-quality benchmark for aerial VLN. \textbf{First}, we develop a highly automated toolchain for data generation, enabling automatic scene semantic extraction, point cloud acquisition, flight trajectory creation, and instruction generation, thus significantly improving efficiency. \textbf{Second}, based on the developed toolchain, we construct a large-scale aerial VLN dataset with \textbf{100k} trajectories, covering samples of diverse heights and difficulty levels across 16 distinct scenes. The corresponding visual data are generated using various rendering engines or techniques, including Unreal Engine (UE), GTA5, Google Earth, and 3D Gaussian Splatting (3D GS). All data feature high visual quality, with 3D GS supporting real-to-sim rendering, further enhancing the realism of the dataset. \textbf{Third}, we propose a keyframe-aware vision-language-action (VLA) model to directly output flight actions based on instructions, current observations, and key histories. Extensive experiments are conducted to evaluate the proposed model and comparison methods, establishing a comprehensive benchmark for aerial VLN. The toolchain, dataset, and codes will be available upon the acceptance of the paper.
%at \url{openfly.github.io}. 
\end{abstract}

%%%%%%%%% BODY TEXT
\section{Introduction}

Video generation has garnered significant attention owing to its transformative potential across a wide range of applications, such media content creation~\citep{polyak2024movie}, advertising~\citep{zhang2024virbo,bacher2021advert}, video games~\citep{yang2024playable,valevski2024diffusion, oasis2024}, and world model simulators~\citep{ha2018world, videoworldsimulators2024, agarwal2025cosmos}. Benefiting from advanced generative algorithms~\citep{goodfellow2014generative, ho2020denoising, liu2023flow, lipman2023flow}, scalable model architectures~\citep{vaswani2017attention, peebles2023scalable}, vast amounts of internet-sourced data~\citep{chen2024panda, nan2024openvid, ju2024miradata}, and ongoing expansion of computing capabilities~\citep{nvidia2022h100, nvidia2023dgxgh200, nvidia2024h200nvl}, remarkable advancements have been achieved in the field of video generation~\citep{ho2022video, ho2022imagen, singer2023makeavideo, blattmann2023align, videoworldsimulators2024, kuaishou2024klingai, yang2024cogvideox, jin2024pyramidal, polyak2024movie, kong2024hunyuanvideo, ji2024prompt}.


In this work, we present \textbf{\ours}, a family of rectified flow~\citep{lipman2023flow, liu2023flow} transformer models designed for joint image and video generation, establishing a pathway toward industry-grade performance. This report centers on four key components: data curation, model architecture design, flow formulation, and training infrastructure optimization—each rigorously refined to meet the demands of high-quality, large-scale video generation.


\begin{figure}[ht]
    \centering
    \begin{subfigure}[b]{0.82\linewidth}
        \centering
        \includegraphics[width=\linewidth]{figures/t2i_1024.pdf}
        \caption{Text-to-Image Samples}\label{fig:main-demo-t2i}
    \end{subfigure}
    \vfill
    \begin{subfigure}[b]{0.82\linewidth}
        \centering
        \includegraphics[width=\linewidth]{figures/t2v_samples.pdf}
        \caption{Text-to-Video Samples}\label{fig:main-demo-t2v}
    \end{subfigure}
\caption{\textbf{Generated samples from \ours.} Key components are highlighted in \textcolor{red}{\textbf{RED}}.}\label{fig:main-demo}
\end{figure}


First, we present a comprehensive data processing pipeline designed to construct large-scale, high-quality image and video-text datasets. The pipeline integrates multiple advanced techniques, including video and image filtering based on aesthetic scores, OCR-driven content analysis, and subjective evaluations, to ensure exceptional visual and contextual quality. Furthermore, we employ multimodal large language models~(MLLMs)~\citep{yuan2025tarsier2} to generate dense and contextually aligned captions, which are subsequently refined using an additional large language model~(LLM)~\citep{yang2024qwen2} to enhance their accuracy, fluency, and descriptive richness. As a result, we have curated a robust training dataset comprising approximately 36M video-text pairs and 160M image-text pairs, which are proven sufficient for training industry-level generative models.

Secondly, we take a pioneering step by applying rectified flow formulation~\citep{lipman2023flow} for joint image and video generation, implemented through the \ours model family, which comprises Transformer architectures with 2B and 8B parameters. At its core, the \ours framework employs a 3D joint image-video variational autoencoder (VAE) to compress image and video inputs into a shared latent space, facilitating unified representation. This shared latent space is coupled with a full-attention~\citep{vaswani2017attention} mechanism, enabling seamless joint training of image and video. This architecture delivers high-quality, coherent outputs across both images and videos, establishing a unified framework for visual generation tasks.


Furthermore, to support the training of \ours at scale, we have developed a robust infrastructure tailored for large-scale model training. Our approach incorporates advanced parallelism strategies~\citep{jacobs2023deepspeed, pytorch_fsdp} to manage memory efficiently during long-context training. Additionally, we employ ByteCheckpoint~\citep{wan2024bytecheckpoint} for high-performance checkpointing and integrate fault-tolerant mechanisms from MegaScale~\citep{jiang2024megascale} to ensure stability and scalability across large GPU clusters. These optimizations enable \ours to handle the computational and data challenges of generative modeling with exceptional efficiency and reliability.


We evaluate \ours on both text-to-image and text-to-video benchmarks to highlight its competitive advantages. For text-to-image generation, \ours-T2I demonstrates strong performance across multiple benchmarks, including T2I-CompBench~\citep{huang2023t2i-compbench}, GenEval~\citep{ghosh2024geneval}, and DPG-Bench~\citep{hu2024ella_dbgbench}, excelling in both visual quality and text-image alignment. In text-to-video benchmarks, \ours-T2V achieves state-of-the-art performance on the UCF-101~\citep{ucf101} zero-shot generation task. Additionally, \ours-T2V attains an impressive score of \textbf{84.85} on VBench~\citep{huang2024vbench}, securing the top position on the leaderboard (as of 2025-01-25) and surpassing several leading commercial text-to-video models. Qualitative results, illustrated in \Cref{fig:main-demo}, further demonstrate the superior quality of the generated media samples. These findings underscore \ours's effectiveness in multi-modal generation and its potential as a high-performing solution for both research and commercial applications.
\section{Related Work}

\subsection{Large 3D Reconstruction Models}
Recently, generalized feed-forward models for 3D reconstruction from sparse input views have garnered considerable attention due to their applicability in heavily under-constrained scenarios. The Large Reconstruction Model (LRM)~\cite{hong2023lrm} uses a transformer-based encoder-decoder pipeline to infer a NeRF reconstruction from just a single image. Newer iterations have shifted the focus towards generating 3D Gaussian representations from four input images~\cite{tang2025lgm, xu2024grm, zhang2025gslrm, charatan2024pixelsplat, chen2025mvsplat, liu2025mvsgaussian}, showing remarkable novel view synthesis results. The paradigm of transformer-based sparse 3D reconstruction has also successfully been applied to lifting monocular videos to 4D~\cite{ren2024l4gm}. \\
Yet, none of the existing works in the domain have studied the use-case of inferring \textit{animatable} 3D representations from sparse input images, which is the focus of our work. To this end, we build on top of the Large Gaussian Reconstruction Model (GRM)~\cite{xu2024grm}.

\subsection{3D-aware Portrait Animation}
A different line of work focuses on animating portraits in a 3D-aware manner.
MegaPortraits~\cite{drobyshev2022megaportraits} builds a 3D Volume given a source and driving image, and renders the animated source actor via orthographic projection with subsequent 2D neural rendering.
3D morphable models (3DMMs)~\cite{blanz19993dmm} are extensively used to obtain more interpretable control over the portrait animation. For example, StyleRig~\cite{tewari2020stylerig} demonstrates how a 3DMM can be used to control the data generated from a pre-trained StyleGAN~\cite{karras2019stylegan} network. ROME~\cite{khakhulin2022rome} predicts vertex offsets and texture of a FLAME~\cite{li2017flame} mesh from the input image.
A TriPlane representation is inferred and animated via FLAME~\cite{li2017flame} in multiple methods like Portrait4D~\cite{deng2024portrait4d}, Portrait4D-v2~\cite{deng2024portrait4dv2}, and GPAvatar~\cite{chu2024gpavatar}.
Others, such as VOODOO 3D~\cite{tran2024voodoo3d} and VOODOO XP~\cite{tran2024voodooxp}, learn their own expression encoder to drive the source person in a more detailed manner. \\
All of the aforementioned methods require nothing more than a single image of a person to animate it. This allows them to train on large monocular video datasets to infer a very generic motion prior that even translates to paintings or cartoon characters. However, due to their task formulation, these methods mostly focus on image synthesis from a frontal camera, often trading 3D consistency for better image quality by using 2D screen-space neural renderers. In contrast, our work aims to produce a truthful and complete 3D avatar representation from the input images that can be viewed from any angle.  

\subsection{Photo-realistic 3D Face Models}
The increasing availability of large-scale multi-view face datasets~\cite{kirschstein2023nersemble, ava256, pan2024renderme360, yang2020facescape} has enabled building photo-realistic 3D face models that learn a detailed prior over both geometry and appearance of human faces. HeadNeRF~\cite{hong2022headnerf} conditions a Neural Radiance Field (NeRF)~\cite{mildenhall2021nerf} on identity, expression, albedo, and illumination codes. VRMM~\cite{yang2024vrmm} builds a high-quality and relightable 3D face model using volumetric primitives~\cite{lombardi2021mvp}. One2Avatar~\cite{yu2024one2avatar} extends a 3DMM by anchoring a radiance field to its surface. More recently, GPHM~\cite{xu2025gphm} and HeadGAP~\cite{zheng2024headgap} have adopted 3D Gaussians to build a photo-realistic 3D face model. \\
Photo-realistic 3D face models learn a powerful prior over human facial appearance and geometry, which can be fitted to a single or multiple images of a person, effectively inferring a 3D head avatar. However, the fitting procedure itself is non-trivial and often requires expensive test-time optimization, impeding casual use-cases on consumer-grade devices. While this limitation may be circumvented by learning a generalized encoder that maps images into the 3D face model's latent space, another fundamental limitation remains. Even with more multi-view face datasets being published, the number of available training subjects rarely exceeds the thousands, making it hard to truly learn the full distibution of human facial appearance. Instead, our approach avoids generalizing over the identity axis by conditioning on some images of a person, and only generalizes over the expression axis for which plenty of data is available. 

A similar motivation has inspired recent work on codec avatars where a generalized network infers an animatable 3D representation given a registered mesh of a person~\cite{cao2022authentic, li2024uravatar}.
The resulting avatars exhibit excellent quality at the cost of several minutes of video capture per subject and expensive test-time optimization.
For example, URAvatar~\cite{li2024uravatar} finetunes their network on the given video recording for 3 hours on 8 A100 GPUs, making inference on consumer-grade devices impossible. In contrast, our approach directly regresses the final 3D head avatar from just four input images without the need for expensive test-time fine-tuning.


\section{OpenFly Data Generation Platform}

\begin{figure*}[t]
\begin{center}
   \includegraphics[width=\linewidth]{Fig/all_images.png}
\end{center}
   \caption{High-quality examples from different rendering engines and techniques, including several large cities such as Shanghai, Guangzhou, Los Angeles, Osaka, and etc., cover an area of over a hundred square kilometers in total. 3D GS provides five large campus scenes, further enhancing the diversity and realism of the data.}
\label{fig:all_dataset}
\end{figure*}


In this section, we first describe several basic simulators and data resources, and then present the developed toolchain. The framework of the whole automatic data generation platform is illustrated in Fig. \ref{fig:data_gen}.

\subsection{Basic Simulators and Data Resources}
\label{sec:Automatic}


\indent \indent To collect a wide range of high-quality and realistic simulation data, we source the dataset from multiple rendering engines integrated with various simulators. Fig. \ref{fig:all_dataset} showcases several examples obtained from these rendering engines/techniques.

%\footnote{https://www.unrealengine.com/marketplace/product/city-sample/}
\textbf{Unreal Engine + AirSim/UnrealCV.} UE is a rendering engine capable of providing highly realistic interactive virtual environments. This platform has undergone five iterations, and each version features comprehensive and high-quality digital assets. In UE5, we meticulously select an official sample project named `City Sample', which provides us with a large urban scene covering $25.3 km^2$ and a smaller one covering $2.7 km^2$. These scenes include a variety of assets such as buildings, streets, traffic lights, vehicles, and pedestrians. Besides, in UE4, we prepare six more high-quality scenes. Specifically, there are two large scenes showcasing the central urban areas of Shanghai and Guangzhou, covering areas of $30.88 km^2$ and $58.56 km^2$, respectively. The remaining four scenes are selected from AerialVLN~\cite{aerialVLN}. They have smaller areas for totally about $26.64 km^2$. These scenes encompass a wide range of architectural styles, including both Chinese and Western influences, as well as classical and modern designs. Additionally, the UE4 engine allows us to make adjustments in scene time to achieve different appearances of scenes under varying lighting conditions.

% Meanwhile, it can also offer RGB, depth, and segmentation maps with realistic physics and sensor models. 
%UnrealCV is an open-source plugin for Unreal Engine, providing a simple interface for capturing RGB, depth, and segmentation images, thus facilitating research in computer vision and robotics.
Airsim is an open-source simulator, which provides highly realistic simulated environments for UAVs and cars. We integrate the AirSim plugin into UE4 to obtain image data easily from the perspective of a UAV.
%\footnote{https://github.com/microsoft/AirSim}
Since AirSim does not support UE5 and stopped updating in 2022, we use the UnrealCV~\cite{unrealcv} plugin as an alternative for image acquisition in UE5. To realize a highly efficient data collection in simulated scenes, we modify the UE5 project to a C++ project, integrate the UnrealCV plugin, and package executables for multiple systems like Windows and Linux. 

\textbf{GTA V + Script Hook V.} 
GTA V is an open-world game that is frequently used by computer vision researchers due to its highly realistic and dynamic virtual environment. The game features a meticulously crafted cityscape modeled after Los Angeles, encompassing various buildings and locations such as skyscrapers, gas stations, parks, and plazas, along with dynamic traffic flows and changes in lighting and shadows. 

Script Hook V is a third-party library with the interface to GTA V's native script functions. With the help of Script Hook V, we build an efficient and stable interface, which receives the pose information and returns accurate RGB images and lidar data. From the interface, we can control a virtual agent to collect the required data in an arbitrary pose and angle in the game.

%Specifically, it uses various 3D modeling techniques provided by softwares such as 3D Max and SketchUp to model urban-level scene images into 3D models. 
%These models are then uploaded to Google Earth and stitched together to form continuous 3D scenes for display.
\textbf{Google Earth + Google Earth Studio.} 
Google Earth is a virtual globe software, which builds a 3D earth model by integrating satellite imagery, aerial photographs, and Geographic Information System (GIS) data. From this engine, we select four urban scenes covering a total area of $53.60 km^2 $, \emph{i.e.,} Berkeley, primarily consisting of traditional neighborhoods; Osaka, which features a mix of skyscrapers and historic buildings; and two areas with numerous landmarks: Washington, D.C., and St. Louis.

%\footnote{earth.google.com}

%\footnote{https://www.google.com/earth/studio/}
%It expands upon the Google Earth browsing interface by integrating features commonly found in video production software. These enhancements
%developed by the Google Earth team
Google Earth Studio is a web-based animation and video production tool that allows us to create keyframes and set camera target points on the 2D and 3D maps of Google Earth. Using this functionality, we can quickly generate customized tour videos by selecting specific routes and angles. In order to efficiently plan the route, we develop a function that automatically draws the flight trajectory in Google Earth Studio according to the selected area and predefined photo interval. 
%We also implement a function that stores the collected images in a normalized coordinate system based on the GPS information.
%according to the data collection area and photo interval we set,
%The tool also supports exporting the output as MP4 files or image sequences, providing flexibility for further use.


%However, under the drone's perspective, choosing the appropriate shooting altitude posed a dilemma, \emph{i.e.,} if the altitude is too low, the sparse point cloud generated during the initialization of the 3D GS reconstruction will be suboptimal due to insufficient feature point matches between photos. On the other hand, if the altitude is too high, the Gaussian reconstruction will result in overly coarse training of details. After multiple attempts, the data collection plan using the M30T was determined as follows. For large-scale block scenes, oblique photography is performed at approximately twice the average building height using the default parameters of the M30T’s wide-angle camera, with a tilt angle of -45°. For landmark buildings with heights significantly different from the average height, additional targeted data collection is conducted at twice their height. This altitude setting can, to a certain extent, ensure both higher-quality point cloud initialization and Gaussian splatting training.(放到supp)

\textbf{3D Gaussian Splatting + SIBR viewers.} As a highly realistic reconstruction method, hierarchical 3D GS~~\cite{kerbl2024hierarchical} employs a hierarchical training and display architecture, making it particularly suitable for rendering large-scale areas. Due to these features, we use this method to reconstruct and render multiple real scenes. We utilize the DJI M30T drone as the data collection device, which offers an automated oblique photography mode, enabling us to capture a large area of real-world data with minimal manpower. Practically, we gathered data from five campuses across three universities, which are East China University of Science and Technology, Northwestern Polytechnical University, and Shanghai Jiao Tong University (referred to as ECUST, NWPU, and SJTU). These campus scenes include various types and styles of landmarks, such as libraries, bell towers, waterways, lakes, playgrounds, construction sites, and lawns. The detailed information for the five campuses is presented in Table~\ref{tab:GS_information}. More details of the 3D GS data collection can be found in our supplementary material.

SIBR~\cite{sibr2020} viewers is a rendering tool designed for the 3D GS project, enabling visualization of a scene from arbitrary viewpoints. The tool supports high-frame-rate scene rendering and provides various interactive modes for navigation. Building upon SIBR viewers, we developed an HTTP RESTful API that generates RGB images from arbitrary poses, simulating a UAV's perspective.


\begin{table}[t]
\caption{\textbf{Different 3D GS Scenes}}
\label{tab:GS_information}
\centering
\begin{tabular}{ccc}
\toprule
Campus Name&Images&Area \\
\midrule
\makecell{ECUST  (Fengxian Campus)} & 12008 & $1.06km^2$ \\
\midrule
\makecell{NWPU  (Youyi Campus)} & 4648 & $0.8km^2$ \\
\makecell{NWPU  (Changan Campus)} & 23798 & $2.6km^2$ \\
\midrule
\makecell{SJTU  (Minghang-East Zone)} & 20934 & $1.72km^2$ \\
\makecell{SJTU  (Minghang-West Zone)} & 9536 & $0.95km^2$ \\
\bottomrule
\end{tabular}
\end{table}



%In order to achieve automatic trajectory generation, it is necessary to structure the scene, which involves obtaining the 3D point cloud and the semantic segmentation. Based on these, a unified interface for image collection, a trajectory generation tool, and an instruction generation tool are developed.
\subsection{Toolchain for Automatic Data Colleciton}
\indent \indent To achieve automatic data generation, we integrate the above five simulators and design three unified interfaces, \emph{i.e.,} the agent movement interface, the lidar data acquisition interface, and the image acquisition interface, allowing an agent to interact with any scene. Based on these interfaces, we further develop a toolchain, including 3D point cloud acquisition, scene semantic segmentation, automatic trajectory generation, and instruction generation. The framework of the whole data generation platform is illustrated in Fig. \ref{fig:data_gen}, with details of these interfaces and tools elaborated below.

%To facilitate the collection of trajectories and images across various simulation environments, the OpenFly platform integrates all the aforementioned simulators and provides unified interfaces that enable interaction between an agent and the environment in any scene. Specifically, there are three main interfaces. 
%$[X, Y, Z] [QW, QX, QY, QZ]$
%$[d_x, d_y, d_z]$ and $[d_{roll}, d_{pitch}, d_{yaw}]$
%using either absolute poses or relative poses in the agent's body frame
%3) Scene Information Interface: The 2D coordinates and height information of all landmarks, along with the point cloud map of the entire scene, can be accessed through this interface.
\textbf{Unified Interfaces.} 
1) Agent Movement Interface: We design a \textit{CoorTrans} module, which implements a customized pose transformation matrix and scaling function to unify all simulator coordinate systems into a meter-based FLU (Front-Left-Up) convention. This interface enables precise agent positioning among regular scenes, point clouds, and scene segmentations, ensuring consistency and facilitating automatic trajectory generation.
2) Lidar Data Acquisition: For different simulators, point cloud data is acquired  through different methods, including lidar sensor collection, depth map back-projection, and image feature matching. We develop a unified interface to integrate these methods and leverage the proposed \textit{CoorTrans} module to align all data to the same FLU coordinate system.
3) Image Acquisition Interface: We integrate HTTP RESTful and TCP/IP protocols to form a unified image request interface, allowing image data to be obtained from any location with flexible  resolutions and agent viewpoints. 

%we use three methods to obtain 3D point clouds from the scenes, \emph{i.e.,} depth map back-projection, LiDAR scan reconstruction, and sparse reconstruction. For the UE5+UnrealCV simulator, a project named MatrixCity~~\cite{li2023matrixcity} provides us with the depth maps and camera parameters of small-city and big-city scenes. 1) Through back-projecting the 2D depth information into 3D space, we generate the point clouds for the two datasets. 2) For the UE4+AirSim and GTA5 simulators, we directly utilize the LiDAR sensors provided by the simulators to traverse each scene in a grid pattern, obtaining local point clouds. These are then transformed into the global coordinate system using the LiDAR coordinate information and finally merged into complete scene point clouds. 3) In 3D GS, since the first step of Gaussian Splatting reconstruction involves using the open-source project COLMAP ~\cite{colmap} to perform sparse structure-from-motion (SFM) point cloud reconstruction based on input images, we could directly export and use the point clouds obtained from this step.
%structure-from-motion (SFM)
\textbf{3D Point Cloud Acquisition.} 
For different simulators, we provide two methods to reconstruct the point cloud map of an entire scene. 1) Rasterized Sampling Reconstruction:
For the UE5 + UnrealCV simulator, the MatrixCity~\cite{li2023matrixcity} project offers a convenient rasterized sampling solution. We use the aforementioned lidar data acquisition interface to obtain the local point cloud at the sampling points. Since these data are already aligned within the same coordinate system, the point cloud map of the entire scene can be constructed by simply stitching local point clouds. For the UE4 + AirSim and GTA V simulators, we customize rasterized sampling points at appropriate resolutions, and perform sampling and reconstruction using the agent movement and lidar data acquisition interfaces. 2) Image-based Sparse Reconstruction: In 3D GS, the scene reconstruction process begins with the open-source COLMAP~\cite{colmap} framework, which geneoverlearates a sparse point cloud from input images. We directly export and use the point clouds obtained from this step. 


\textbf{Scene Semantic Segmentation.} 
Vision-and-Language Navigation (VLN) requires meaningful landmarks as navigation targets. We perform semantic segmentation on four types of simulation scenes using the following three methods. 1) 3D Scene Understanding: A sequence of top-down views of the scene is captured in a rasterized format. We then use Octree graph~\cite{octree_graph} to extract 2D mask proposals, which are subsequently projected into the 3D point cloud space to generate semantic 3D segments. 2) Point Cloud Projection and Contour Extraction: We first acquire the point cloud of a scene, then project the voxelized point cloud onto a projection plane slightly above the ground. Using OpenCV, we perform a series of operations on the projected image, including binarization, erosion, and contour extraction, to obtain multiple instances along with their 2D coordinates. For each instance, the maximum height of the points within its neighborhood is used as the final height. This method provides a more computationally efficient yet coarser segmentation compared to the first approach, allowing users to choose based on their requirements. 3) Manual Annotation: When the point cloud quality of a scene is low or finer segmentation is required, we provide a method for annotating instances in the point cloud space by mouse clicks, based on ROS2 and RVIZ2. Users can annotate instances directly in the point cloud space using mouse clicks to define landmarks of interest for the task. This method is applicable to four simulators, \emph{i.e.,} UE + UnrealCV, UE + Airsim, GTA V, and 3D GS.

\begin{comment}
\begin{figure}
    \centering
    % 第一列子图
    \begin{subfigure}[b]{0.15\textwidth}   % 0.3\textwidth 是每个子图的宽度
        \centering
        \includegraphics[width=\textwidth]{Fig/whole_scene.pdf} % 替换为你的图片文件
        \caption{}
        \label{fig:sub1}
    \end{subfigure}
    \hfill  % 用来在子图之间增加水平间隔
    % 第二列子图
    \begin{subfigure}[b]{0.15\textwidth}
        \centering
        \includegraphics[width=\textwidth]{Fig/scene_point_cloud.pdf}
        \caption{}
        \label{fig:sub2}
    \end{subfigure}
    \hfill
    % 第三列子图
    \begin{subfigure}[b]{0.15\textwidth}
        \centering
        \includegraphics[width=\textwidth]{Fig/scene_seg.pdf}
        \caption{}
        \label{fig:sub3}
    \end{subfigure}
    
    \caption{Illustration of results obtained by our point cloud acquisition and semantic segmentation tools. (a) Overview of an urban scene. (b) The point cloud of (a). (c) The semantic segmentation of (a).}
    \label{fig:main}
\end{figure}
\end{comment}


\textbf{Automatic Trajectory Generation.}
Leveraging the aforementioned point cloud map and segmentation tools, OpenFly can generate trajectories using the following two methods. 
%1) Path search based on customized action space: First, a global hash voxel map $M_{global}$ is constructed from the scene point cloud $P$ and the voxelized point cloud is projected onto the horizontal plane to obtain the bird's eye view (BEV) occupancy map $M_{bev}$. 
1) Path search based on customized action space: First, a global hash voxel map $M_{global}$ and a bird's eye view (BEV) occupancy map $M_{bev}$ are constructed from the scene point cloud $P$.
Second, the flight altitude is randomly selected within the user-defined height range, and landmarks that are not lower than the height threshold $H_{\tau}$ are chosen as targets. A starting point is selected within the distance range $[r, R]$ from the landmark, ensuring that it is not occupied in both $M_{global}$ and $M_{bev}$. Then, a point on the line connecting the starting point and the landmark, which is close to the landmark and unoccupied in $M_{bev}$, is chosen as the endpoint. 
Third, A collision-free trajectory from the starting point to the endpoint is generated using the A*~~\cite{astar} pathfinding algorithm, where the granularity of exploration step size and direction can be adjusted according to the action space. Besides, by repeatedly selecting the endpoint as the new starting point, complex trajectories can be generated. Finally, utilizing OpenFly's interface, images corresponding to the trajectory points can be obtained. 2) Path search based on grid: Google Map data does not allow image retrieval at arbitrary poses in the space. Thus we rasterize a pre-selected area and collect images from each grid point in all possible orientations. Starting and ending points are chosen within the grid points to generate trajectories. Corresponding images for these trajectory points are then selected from the pre-collected image set.



% 视觉语言导航数据中,语言指令的质量至关重要。然而,以往的研究大多依赖人工标注,不仅成本高昂,还限制了数据集规模。为此,OpenFly 提出了基于视觉语言模型(VLM)的全自动化语言指令生成方法。

% 自然的想法是将所有图像全部提交给VLMs进行轨迹分析,生成指令。但全部图像的输入带来了巨额的开销,并带来了信息冗余。因此 OpenFly 将 完整轨迹拆分为子轨迹来进行处理,提取每一个子轨迹的关键动作和landmark特征,最终进行整合。与室内视觉语言导航不同,在空中视觉语言导航(Aerial Vision Language Navigation)中,环境中的障碍物较少,因此指令中的“Move Forward” 占据了大部分比例,关键动作集中在“左转/右转”和“上升/下降”。此外,无人机飞行轨迹中不可避免地会出现轻微角度调整,这些调整往往无明确目的地,因此被简化为“slightly turn left/right”。所以我们根据轨迹是否发生了连续的非直行动作来进行拆分。举一个例子,轨迹动作序列为[1,2,1,2,2,1,0],这里1代表move forward,2代表turn left,0代表 stop。该轨迹会被拆分为[1,2,1],[2,2],[1,0]。第一条的动作为Move forward and slightly turn left,与此同时我们将第一条子轨迹的最后一张图像提交给VLM提取对应的landmark特征,为了更加精确,我们还会提供Landmark在图像中的位置星系

% User:You are an assistant proficient in image recognition. You can accurately identify the object closest to you in the image and its different features from surrounding objects.The object is at the center of the image.

% GPT 4o:{color: bule, feature: Steel, glass, size: medium size, type: building}

% 最终我们得到了如下的子指令序列:
% 1. Action 1(Move forward and slightly turn left) , Landmark 1
% 2.Action 2( turn left), Landmark 2
% 3.Action 3 (Move forward ), Landmark 3

% 我们会将子指令序列提交给GPT 4o,并获得最终的指令。


% 基于上述方法,我们生成了一系列“动作 + Landmark”的子指令。随后,利用 VLM 的语言生成能力,将这些子指令整合为流畅、完整的自然语言导航指令

%Unlike indoor VLN, in aerial VLN, there are fewer obstacles in the environment, meaning that , with key actions focused mainly on “Turn Left/Turn Right” and “Ascend/Descend.”



\textbf{Automatic Instruction Generation.}
Previous research has predominantly relied on manual annotation to generate trajectory instruction, which is not only costly but also limits the scalability of datasets. To address this issue, we propose a highly automated language instruction generation method based on VLMs, \emph{e.g.,} GPT-4o.
A straightforward method would be to submit all images to VLMs to analyze the trajectory and generate instructions. However, using all images introduces significant computational overhead and leads to information redundancy. For example, the `Forward' action usually occupies a larger proportion of a flight trajectory, with `Turn Left/Turn Right' or `Ascend/Descend' actions taken when encountering key landmarks.




Based on the above findings, we split the complete trajectory into multiple sub-trajectories based on the occurrence of non-consistent actions, extracting key actions and images for processing and subsequent integration. Notably, slight angle adjustments often occur during flight to change the direction subtly, and a `slightly Turn Left/Right' will be merged with subsequent `Forward' actions. Specifically, suppose that the trajectory action sequence is [1, 1, 2, 1, 1, 2, 2, 1, 0], where 1, 2, and 0 denote `Forward', `Turn Left', and `Stop', respectively. This trajectory would be split into four sub-trajectories, \emph{i.e.,} [1, 1], [2, 1, 1], [2, 2], and [1, 0]. The second sub-trajectory involves `slightly turn left' and `move forward'. We submit the action sequence and the last captured image of each sub-trajectory to a VLM to generate descriptions of both action and landmarks.
%A simplified prompt to the VLM and corresponding response are probably like this. User: `You are an assistant proficient in image recognition. You can accurately identify the object closest to you in the image and its different features from surrounding objects. The object is usually at the center of the image'. GPT 4o: `{color: blue, feature: Steel, glass, size: medium size, type: building}'.
The sub-instructions are obtained similar to the following format:
\begin{itemize}[left=0pt]
    \item `Move forward' to `Landmark 1'.
    \item `Slightly turn left and move forward' to `Landmark 2'.
    \item `Turn left' towards `Landmark 3'.
    \item `Move forward' to `Landmark 4'.
\end{itemize}

These sub-instructions are then processed by a VLM/LLM again, where they are integrated into coherent and complete instructions. The detailed prompt used for the VLM, along with the complete responses, is provided in the supplementary material.

%Based on this method, we generated a series of “Action + Landmark” sub-instructions. Subsequently, leveraging the language generation capabilities of VLMs, these sub-instructions were integrated into coherent and complete natural language instructions. More details and the complete prompt to GPT-4o are shown in our supplementary material.

%User: You need to help me combine these scattered actions and landmarks into a sentence with words that are similar in meaning and more appropriate in words, so that the sentence is fluent and accurate. At the same time, merge the same landmarks accurately. {Sub-instructions}

%GPT 4o: Move forward and slightly turn left to a high-rise building with a noticeable logo at the top.Then turn left and go straight to a futuristic tower with a large spherical structure in the middle.
% 

% Specifically, we divide the instruction generation process into two main components, \emph{i.e.,} actions and the corresponding landmarks. Unlike indoor VLN, aerial VLN features fewer obstacles in the environment, resulting in a higher proportion of the “Move Forward” action, with key movements focusing on “Turn Left/Right” and “Ascending/Descending.” Additionally, slight angular adjustments are unavoidable in drone flight trajectories. However, these adjustments often lack a specific destination and are simplified as “slightly turn left/right.”

% The wide field of view from drones and the similarity in appearance among common buildings present challenges in identifying landmarks. To overcome this, in addition to image data, we provide the VLM with landmark positional information, \emph{e.g.,} "in the center of the image," or "at the bottom of the image", to extract key attributes of landmarks, such as type, color, and shape.



    
\subsection{Quality Control.}
%一些filter机制 和 人工抽查策略
\textbf{Data Filter.}
During data collection, it is inevitable that some damaged or low-quality data will be generated. We clean the data in the following situations. 1) We remove damaged images that are produced in generation or transmission. 2) We find that UAVs sometimes appear to pass through the tree models. Therefore, we exclude the trajectories where the altitude is lower than that of the trees. 3) We believe that extremely short or long trajectories are not conducive to model training. Thus, we remove these trajectories, specifically those with fewer than 2 or more than 150 actions.




\textbf{Instruction Refinement.}
A known challenge of instruction generation is VLMs' hallucinations. During the previous instruction generation process, sometimes the same landmark appears across several frames. This results in a VLM generating similar captions for the repeated observations of a landmark, increasing the complexity of the final instruction and introducing ambiguity due to duplication.

To mitigate this challenge, we utilize the NLTK library ~\cite{bird2006nltk} to simplify the instruction by detecting and merging similar descriptions. Specifically, a syntactic parse tree is first constructed to extract all landmark captions using a rule-based approach. Then, a sentence-transformer model is employed to encode the extracted landmark captions into embedding vectors. Their similarities are computed with dot product, and high-similarity captions are then identified and replaced with referential pronouns (\emph{e.g.}, ``it," ``there," \emph{etc.}). For example, a generated instruction with redundant information is ``$\cdots$ make a left turn toward \textbf{a medium-sized beige building marked by a signboard reading CHARLIE'S CHOCOLATE}. Continue heading straight, passing \textbf{a medium-sized gray building with a prominent rooftop billboard displaying Charlie’s Chocolate} $\cdots$". After simplification, a more concise sentence is obtained, \emph{i.e.,} ``$\cdots$ make a left turn toward \textbf{a medium-sized beige building marked by a signboard reading CHARLIE'S CHOCOLATE}. Continue heading straight, passing \textbf{it} $\cdots$", demonstrating the effectiveness of this post-processing technique. 

At the same time, we built a data inspection platform to provide instructions, action sequences, and corresponding images to the examiners. If the instructions and trajectories align, they are considered qualified. We randomly select 3K samples from the entire dataset  according to data distribution in Sec. \ref{data_split}. After manually inspecting these samples, we find that the qualification rate reaches 91\%.


%我们首先用nltk库进行英文文本处理,构建语法关树,通过rule-based方式提取所有的landmark caption。 接着,是用sentence transformer将所选landmark编码embeddings,通过点积计算其相似度,筛选出高相似性的短语嵌入, 将其替换为指代性名词(it, there etc.)。



\section{Dataset Analysis}

\begin{table*}[t]
\small      
\caption{Comparison of different VLN datasets. Above the middle dividing line lies the ground-based datasets, while below is the aerial VLN datasets. $N_{traj}$: the number of total trajectories. $N_{vocab}$: vocabulary size. Path Len: the average length of trajectories, measured in meters. Intr Len: the average length of instructions. $N_{act}$: the average number of actions per trajectory.}
\begin{adjustbox}{center}
%\setlength{\tabcolsep}{3pt}
\renewcommand{\arraystretch}{1.2}
\scalebox{.99}{
\begin{tabular}{lcccclcc}
\toprule
Dataset   & $N_{traj}$ & $N_{vocab}$ & Path Len. & Intr Len. & Action Space & $N_{act}$ & Environment \\ \midrule
R2R~\cite{R2R}       & 7189      & 3.1K         & 10.0      & 29        &graph-based   & 5       & Matterport3D  \\
RxR~\cite{RxR}       & 13992     & 7.0K         & 14.9      & 129       &graph-based   & 8       & Matterport3D  \\
REVERIE~\cite{REVERIE}   & 7000      & 1.6K         & 10.0      & 18        &graph-based   & 5       & Matterport3D  \\
CVDN~\cite{CVDN}      & 7415      & 4.4K         & 25.0      & 34        &graph-based   & 7       & Matterport3D  \\
TouchDown~\cite{Touchdown} & 9326      & 5.0K         & 313.9     & 90        &graph-based   & 35      & Google Street View  \\ 
VLN-CE~\cite{VLN-CE}    & 4475      & 4.3K         & 11.1      & 19        &2 DoF         & 56      & Matterport3D  \\
LANI~\cite{LANI}      & 6000      & 2.3K         & 17.3      & 57        &2 DoF         & 116     & CHALET  \\ \midrule
ANDH~\cite{ANDH}      & 6269      & 3.3K         & 144.7     & 89        &3 DoF         & 7       & xView  \\
AerialVLN~\cite{aerialVLN} & 8446      & 4.5K         & 661.8     & 83        &4 DoF         & 204     & AirSim + UE \\
CityNav~\cite{CityNav}   & 32637     & 6.6K         & 545       & 26        &4 DoF         & -       & SensatUrban  \\
OpenUAV~\cite{openuav}   &12149      &10.8K         & 255       & 104       &6 DoF         & 264     & AirSim + UE \\ \midrule
\multirow{2}{*}{Ours} &\multirow{2}{*}{100K}     &\multirow{2}{*}{15.6K}        &\multirow{2}{*}{99.1}        &\multirow{2}{*}{59}       &\multirow{2}{*}{4 DoF}    &\multirow{2}{*}{35}     & \parbox[t]{6cm}{AirSim + UE, GTA5 + Script Hook V, \\ Google Earth Studio, 3D GS + SIBR viewers} \\

\bottomrule
\end{tabular}
}
\end{adjustbox}
\label{tab:dataset_comp}
\end{table*}

\begin{figure}
    \centering
    % 第一行子图
    \begin{subfigure}[b]{0.47\columnwidth}
        \centering
        \includegraphics[width=\textwidth]{Fig/action_num.pdf}
        \caption{Difficulty level distribution.}
        \label{fig:sub1}
    \end{subfigure}
    \hfill
    \begin{subfigure}[b]{0.47\columnwidth}
        \centering
        \includegraphics[width=\textwidth]{Fig/length_height.pdf}
        \caption{Length-height distribution.}
        \label{fig:sub2}
    \end{subfigure}

    % 换行
    \vspace{0.5cm} 

    % 第二行子图
    \begin{subfigure}[b]{0.47\columnwidth}
        \centering
       
        \includegraphics[width=\textwidth]{Fig/action.pdf}
        \caption{Action distribution with 1 type of `Forward'.}
        \label{fig:sub3}
    \end{subfigure}
    \hfill
    \begin{subfigure}[b]{0.47\columnwidth}
        \centering
        \includegraphics[width=\textwidth]{Fig/action_merge.pdf}
        \caption{Action distribution with 3 types of `Forward'.}
        \label{fig:sub4}
    \end{subfigure}

    \caption{Statistical analysis of trajectories.}
    \label{fig:traj_sta}
\end{figure}

% 词云(看看是否好分名词和动词)、总的轨迹/instruction数量、Vocabulary size、平均每条instruction词量;

%平均路径长度、平均action个数;路径长度分布统计图、action个数分布统计图(easy、middle、hard各多少轨迹);action type分布饼图(见AerialVLN)

%dataset split:train、test划分,各多少轨迹;不同场景的轨迹数量分布饼图(train的不同场景的分布饼图).
\subsection{Overview}
Using our toolchain, we collect 100k trajectories from 18 scenes, along with corresponding image sequences and language instructions. During the data generation process, we set a minimum motion step size of 3 meters to produce more granular trajectories. The details of our and previous VLN datasets are listed in Table. \ref{tab:dataset_comp}, from which we can see that our dataset features a significantly larger number of trajectories and a more extensive vocabulary, as well as greater environmental diversity. In contrast, our average trajectory length and instruction length are relatively short. This is intentional, as we believe" short- and medium-range instructions are actually more in line with the usage habits of human users. This might be more beneficial for the aerial VLN field.

\subsection{Trajectory Analysis}

In addition to a rich variety of scenes, we also strive for diversity in the difficulty level, length, and height of the trajectory data. Based on the number of actions in one trajectory, we classify trajectories with fewer than 30 actions as `Easy', those with the number of actions ranging from 30 to 60 as `Moderate', and those with more than 60 actions as `Hard'. Fig. \ref{fig:sub1} shows the corresponding difficulty level distribution. Besides, compared with ground-based VLN, the aerial VLN task has more motion dimensions. Therefore, we set different trajectory lengths and flight heights to obtain rich data. Fig. \ref{fig:sub2} exhibits the distribution of these data, with their lengths ranging from 0 to 300 meters, and the heights ranging from 0 to 150 meters. 

In the aerial VLN tasks of large-scale outdoor scenes, the proportion of moving forward is naturally higher than that of making adjustments in direction and altitude, as shown in Fig. \ref{fig:sub3}. However, this highly unbalanced action distribution might cause the VLN model to overfit to the dominant action. To alleviate this problem, we divide the `Forward' action into three granularities, \emph{i.e.,} 3m, 6m, and 9m. In the ground-truth trajectories, three consecutive `Forward' actions will be combined into one `9m Forward' action. At the end of a straight-moving trajectory, if the remaining distance is less than 9m, it will be combined into a `6m Forward' action, or remain as a `3m Forward' action. Fig. \ref{fig:sub4} presents the action distribution after this action merging process.



\begin{figure}
    \centering
    % 第一行子图
    \begin{subfigure}[b]{0.47\columnwidth}
        \centering
        \includegraphics[width=\textwidth]{Fig/train_split.pdf}
        \caption{Train set distribution.}
        \label{fig:train_sp}
    \end{subfigure}
    \hfill
    \begin{subfigure}[b]{0.47\columnwidth}
        \centering
        \includegraphics[width=\textwidth]{Fig/test_split.pdf}
        \caption{Test set distribution.}
        \label{fig:test_sp}
    \end{subfigure}
    \caption{The distribution of the data volume in different scenes under the Train and Test sets.}
    \label{fig:traj_sta}
\end{figure}


\section{OpenFly-Agent}
\begin{figure*}[t]
\centering
    \includegraphics[width=0.98\linewidth]{Fig/model_arch.pdf}
    \caption{The architecture of OpenFly-Agent. Keyframes at the time of action transitions are selected to extract crucial observations as the history, with corresponding visual tokens compressed to reduce the computational burden.}
    \label{fig:model}
\end{figure*}

%\textit{Test Seen} and \textit{Test Unseen} indicate that whether the scenes have appeared in the \textit{Train} set.
\subsection{Dataset Split}
\label{data_split}
Similar to previous works, we divide the dataset into three splits, \emph{i.e.,}  \textit{Train, Test Seen, Test Unseen}. Detailed data distributions are shown in Fig. \ref{fig:traj_sta}. For the \textit{Train} split, 7 scenes under the UE rendering engine account for $75.7\%$ of the total \textbf{100K} data, since UE provides the largest number of scenes, where different amounts of trajectories are sampled according to the scenario area. The 4 scenes created by 3D GS are also the main part of the data, accounting for nearly $20\%$ of the total amount. To ensure visual quality, we only collect data from a high-altitude perspective using Google Earth, which accounts for $4.46\%$. 
The detailed information of the \textit{Test Seen} and \textit{Test Unseen} splits are as follows:
\begin{itemize}[left=0pt]
    \item \textit{Test Seen}: 1800 trajectories uniformly sampled from 11 previously seen UE and 3D GS scenarios.

    \item \textit{Test Unseen}: 1200 trajectories uniformly generated from 3 unseen scenarios, \emph{i.e.,} UE-smallcity, 3D GS-sjtu02, and a Los Angeles-like city in GTA V.
\end{itemize}


%
%Task/Problem formulation
\subsection{Problem Definition}
In the aerial VLN task, a UAV is randomly positioned within a 3D environment with its initial pose defined as $P = [x, y, z, \phi, \theta, \psi]$. At each timestamp $t$, the UAV perceives the surrounding environment through an egocentric image as its observation. Guided by natural language instructions, the task involves predicting the next navigation action. Notably, the UAV can utilize either the current observations or the frames from all previous timestamps to make its prediction.

\subsection{Model Architecture}
As shown in Fig. \ref{fig:model}, we take OpenVLA~\cite{openvla} as the baseline and design an end-to-end model for aerial VLN. In contrast, our model takes a sequence of images to indicate the observation instead of one image in the original OpenVLA. Moreover, to mitigate visual redundancy between adjacent video frames while maintaining key information, two strategies are proposed, \emph{i.e.,} keyframe selection and visual token merging. First, a series of candidate keyframes are selected. Then, these keyframes are merged temporally before and after the vision encoder, resulting in a compact sequence of visual tokens. Finally, the action decoder discretizes the predicted tokens into uniformly distributed bins, which are subsequently mapped to the 6 action types specific to drones. 


\subsubsection{Keyframe Selection}
The length of contextual visual tokens is a major challenge for VLMs when processing videos. Many open-source VLMs use uniform frame sampling \cite{buch2022revisiting, ranasinghe2024understanding, wang2025videoagent} to reduce calculation, but this strategy is not suitable for aerial VLN, since it may miss frames containing key landmarks. 
To address this issue, we adopt a heuristic method to identify keyframes by detecting the change point of the UAV's movement. We notice that sudden changes in the UAV's trajectory are often caused by the observation of landmarks, which can serve as cues to determine keyframes. Specifically, we use the movement of the drone over time to draw turning curves, and the frames near the peaks of the wave are selected as candidate keyframes. The resulting data is interpolated and smoothed, forming a wave-like curve that represents the UAV's movement. 

To further ensure the precision of training data, scene segmentation maps collected in Sec. \ref{sec:Automatic} are used on selected frames to detect key landmarks. Frames containing landmarks are selected as keyframes, yielding reasonably accurate results. Note that each sudden change of actions, \emph{e.g.,} from `Forward' to `Turn Left', will produce a set of keyframes. Consequently, we obtain several sets of keyframes for a long trajectory. 
%For testing, we select keyframes where the action changes, as these often correspond to the observation of a critical landmark.
%Sec Parag

%This keyframe selection scheme gives model the guidance for action prediction via semantic relationship from the observation of the subgoal. Next, with the candidate frame sequences, we introduce the online visual Token merging module for the next action prediction.

\subsubsection{Visual Token Merging}
To further reduce redundant information in keyframes, we design visual token merging, where the core concept is to recognize the similarity between image tokens. It compares adjacent keyframes to merge similar regions and maintains its simplicity by token compression.

%合并阶段。
% 在获得候选帧之后,我们先逐帧过一遍vision encoder获取visual features,再利用标记相似性来合并相邻帧的视觉标记。类似 ToMe [] ,我们通过定期合并之后相邻帧中最相似的标记来进行记忆巩固。我们计算 N 个嵌入标记之间的平均余弦相似度s,在每次合并操作后保留K帧,这也嵌入了存储在长期记忆中的丰富信息。K是控制性能和效率之间权衡的超参数。因此,我们通过加权平均地合并每组相邻帧相似度最高的tokens。合并操作迭代进行,直到token计数达到每个合并操作的预定义值集K。合并阶段应用于Vision Transformer的倒数第二层特征patch token,以逐步合并相似的标记,直到相似标记的数量低于特定层的阈值 Nthreshold。合并阶段之后,剩余的唯一标记将进入压缩阶段。


\begin{table*}[t!]

\centering
\caption{Comparison results on the test-seen split.}
%\vspace{-5pt}
\label{tab:seen_results}
\begin{adjustbox}{center}
\resizebox{\textwidth}{!}{ 
%\setlength{\tabcolsep}{1.6pt}
\renewcommand{\arraystretch}{1.3}
% \scalebox{0.95}{
\begin{tabular}{lccccccccccccccccc}
\toprule
\multirow{2}{*}{Method} & \multicolumn{4}{c}{Easy} & \multicolumn{4}{c}{Moderate} & \multicolumn{4}{c}{Hard} & \multicolumn{4}{c}{Total}\\ 
\cmidrule(lr){2-5} \cmidrule(lr){6-9} \cmidrule(lr){10-13} \cmidrule(lr){14-17}
& NE$\downarrow$ & SR$\uparrow$ & OSR$\uparrow$ & SPL$\uparrow$ 
& NE$\downarrow$ & SR$\uparrow$ & OSR$\uparrow$ & SPL$\uparrow$ 
& NE$\downarrow$ & SR$\uparrow$ & OSR$\uparrow$ & SPL$\uparrow$
& NE$\downarrow$ & SR$\uparrow$ & OSR$\uparrow$ & SPL$\uparrow$ \\ \midrule 

Random & 289m & 0.9\% & 1.1\% & 0\% & 351m & 1.3\% & 1.3\% & 0\% & 374m & 0\% & 0\% & 0\% & 242m & 0.7\% & 0.8\% & 0\% \\
Seq2Seq\cite{VLN-CE}&  201m &  0.9\% & 21.2\% & 0.9\% & 190m & 8.9\% & 19.2\% & 6.5\% & 192m & 2.1\% & 10.1\% & 1.9\% & 194m & 4.0\% & 16.8\%  &  3.1\% \\
CMA\cite{VLN-CE}&  156m & 1.2\% &  35.6\% & 1.6\% & 120m & 11.2\% & 34.5\% & 8.4\% & 156m & 4.6\% & 20.1\% & 5.3\% & 144m & 5.7\% & 30.0\% & 5.1\%\\
AerialVLN\cite{aerialVLN}& \underline{148m} & 1.5\% &  \underline{40.2\%} & 2.6\% & \textbf{94m} & \underline{13.2\%} & \textbf{58.6\%} & \underline{10.7\%} & 147m & 5.4\% & \underline{23.6\%} & \underline{7.6\%} & \underline{130m} & 6.6\% &\underline{40.8\%} & \underline{7.0\%}\\
Navid\cite{navid}& 151m & \underline{11.2\%} & 28.9\% & \underline{4.5\%} & 138m & 8.0\% & 21.3\% & 2.8\% & \underline{134m} & \textbf{10.3\%} & 21.3\% & 4.6\% & 142m & \underline{9.9\%} & 24.3\% & 3.9\% \\
Ours& \textbf{111m} &  \textbf{26.5\%} & \textbf{55.6\%}  & \textbf{16.0\%} & \underline{115m} & \textbf{16.4\%} & \underline{51.2\%} & \textbf{11.2\%} & \textbf{120m} & \textbf{10.3\%} & \textbf{29.6\%} & \textbf{8.2\%}  & \textbf{115m} & \textbf{18.5\%} & \textbf{50.9\%} & \textbf{12.2\%} \\
\bottomrule
\end{tabular}
}
\end{adjustbox}
\end{table*}



For each set of candidate keyframes obtained in the previous selection process, a visual encoder maps each input image to multiple visual tokens, with each token representing the information of an image patch. Considering the potential inter-frame patch redundancy, we take a strategy that similar tokens in subsequent adjacent frames are periodically merged. Specifically, we select the first frame in a keyframe set as the reference, since it usually contains the crucial observation indicating the time for action transition. Then, we densely calculate the cosine similarities between each pair of visual tokens of the reference image and the subsequent image. Next, we merge the tokens with high similarity by averaging them. The unmerged tokens in the subsequent frame will be discarded. The merging operation is iteratively performed until the entire keyframe set has been traversed. Besides, we maintain a memory bank with a capacity of $K$ images, which follows a first-in-first-out (FIFO) policy to retain the latest keyframes.

After the above process, $M$ visual tokens $E=\{e_1, e_2, \cdots, e_M\}$ are obtained for each set of keyframes. Since aerial VLN requires UAVs to perform long-distance flights based on instructions, we continue to carry out token compress to reduce the computational burden. The compressed visual tokens $E_c$ are obtained through grid pooling~\cite{llama_vid}. Notably, we keep the visual tokens of the current frame uncompressed to capture the latest visual observation, as it contains the most important information for flight action prediction.




\subsubsection{Action Prediction}
Similar to~\cite{aerialVLN,CityNav}, 6 actions for UAVs are defined as $\{$Forward, Turn Left, Turn Right, Move Up, Move Down, Stop$\}$ in this work. The units for `Move up' and `Move down' are 3 m, the units for `Turn Left' and `Turn Right' are 30 degrees. `Forward' has three distinct units, namely 3 m, 6 m, and 9 m, respectively. For flight action prediction, each action type is discretized into multiple bins with one non-activate bin indicating that the current action is not activated. We map the model output to one of the bins for each action type, where the bin number corresponds to the amount of units in each action.
\section{Experiments}
\label{sec: experiments}

\subsection{Experimental Setup}
\label{sec: experimental_setup}
\begin{figure}[t]
\centering \includegraphics[width=\linewidth]{figure_2.png} \caption{The handheld platform configuration, including the radar, IMU, and onboard computer. The experiments are conducted in a room equipped with a motion capture system to obtain accurate ground truth.}
\label{fig2}
\end{figure}

We conduct experiments using three datasets, comprising a total of 15 sequences. One is our self-collected dataset, captured with a handheld platform as shown in Fig.~\ref{fig2}, while the other two are public radar datasets: ICINS2021~\cite{9470842}, and ColoRadar~\cite{kramer2022coloradar}. The sensors on our platform include a 4D FMCW radar, specifically the Texas Instruments AWR1843BOOST, and an Xsens MTI-670-DK IMU. No additional hardware triggers are used between the sensors, and the sensor data is recorded using an Intel NUC i7 onboard computer. The experiments are conducted in an indoor area equipped with a motion capture system to obtain precise ground truth. The extrinsic calibration between the IMU and the radar is performed manually. To highlight the significance of temporal calibration in RIO, we design the dataset with two levels of difficulty. Sequences 1 to 3 feature standard motion patterns, while Sequences 4 to 7 introduce more rotational motion to induce larger errors due to the time offset, providing a clearer demonstration of its impact.

\begin{figure*}[t]
\centering
\includegraphics[width=\linewidth]{figure_3.png}
\caption{Comparison of estimated trajectories with the ground truth. The \textcolor{black}{black} trajectory is the ground truth, the \textcolor{blue}{blue} one is the EKF-RIO, which does not account for temporal calibration, and the \textcolor{red}{red} one is the proposed RIO with online temporal calibration. Results are presented for Sequence 4, ICINS 1, and ColoRadar 1, representing one sequence from each of the three datasets.}
\label{trajectory}
\end{figure*}

In~\cite{9470842}, the ICINS2021 dataset is collected using a Texas Instruments IWR6843AOP radar sensor, an Analog Devices ADIS16448 IMU sensor, and a camera. A microcontroller board is used for active hardware triggering to accurately capture the timing of the radar measurements. Data is collected using both handheld and drone platforms. The handheld sequences, ``carried\_1'' and ``carried\_2'', are referred to as ``ICINS 1'' and ``ICINS 2'', while the drone sequences, ``flight\_1'' and ``flight\_2'', are referred to as ``ICINS 3'' and ``ICINS 4'', respectively. The ground truth is provided through visual-inertial SLAM, which performs multiple loop closures, offering a pseudo-ground truth. In~\cite{kramer2022coloradar}, the ColoRadar dataset is collected using a Texas Instruments AWR1843BOOST radar sensor, a Microstrain 3DM-GX5-25 IMU sensor, and a LiDAR mounted on a handheld platform. No specific synchronization setup is used between the sensors. The sequences, ``arpg\_lab\_run0'' and ``arpg\_lab\_run1'', are referred to as ``ColoRadar 1'' and ``ColoRadar 2'', while the sequences ``ec\_hallways\_run0'' and ``ec\_hallways\_run1'' are referred to as ``ColoRadar 3'' and ``ColoRadar 4'', respectively. The ground truth is generated via LiDAR-inertial SLAM, which includes loop closures, offering a pseudo-ground truth.
\subsection{Evaluation}
\label{sec: evaluation}

\begin{table}[t]
\centering
\caption{Quantitative Results of Fixed Offset and Online Estimation}
\label{fixed_offset}
\resizebox{\linewidth}{!}{
\begin{tblr}{
  cells = {c},
  cell{1}{1} = {r=2}{},
  cell{1}{2} = {r=2}{},
  cell{1}{3} = {r=2}{},
  cell{1}{4} = {c=2}{},
  cell{1}{6} = {c=2}{},
  cell{3}{1} = {r=6}{},
  cell{3}{2} = {r=5}{},
  cell{3}{5} = {fg=red},
  cell{4}{4} = {fg=red},
  cell{5}{4} = {fg=blue},
  cell{5}{5} = {fg=blue},
  cell{5}{6} = {fg=blue},
  cell{5}{7} = {fg=red},
  cell{6}{6} = {fg=red},
  cell{6}{7} = {fg=blue},
  cell{9}{1} = {r=6}{},
  cell{9}{2} = {r=5}{},
  cell{11}{4} = {fg=red},
  cell{11}{5} = {fg=blue},
  cell{11}{6} = {fg=red},
  cell{11}{7} = {fg=red},
  cell{12}{4} = {fg=blue},
  cell{12}{5} = {fg=red},
  cell{12}{6} = {fg=blue},
  cell{12}{7} = {fg=blue},
  hline{1,3,9,15} = {-}{},
  hline{2} = {4-7}{},
}
\textbf{Sequence} & \textbf{Method} &  \textbf{Time Offset (s)}            & \textbf{APE RMSE} &                & \textbf{RPE RMSE} &                   \\
                  &                 &                                      & Trans. (m)        & Rot. (\degree) & Trans. (m)        & Rot. (\degree)    \\
                  \hline
Sequence 1        & Fixed Offset    & 0.0             & 0.985             & 1.872          & 0.264             & 1.230          \\
                  &                 & -0.05           & 0.647             & 7.561          & 0.166             & 1.549          \\
                  &                 & -0.10           & 0.661             & 2.438          & 0.138             & 0.948          \\
                  &                 & -0.15           & 0.826             & 5.151          & \textbf{0.131}    & 1.196          \\
                  &                 & -0.20           & 0.974             & 2.698          & 0.156             & 1.274          \\
                  & Online Est.     & \textbf{-0.114} & \textbf{0.646}    & \textbf{0.935} & 0.132    & \textbf{0.774} \\
Sequence 4        & Fixed Offset    & 0.0             & 1.737             & 25.885         & 0.118             & 4.074          \\
                  &                 & -0.05           & 1.028             & 15.460         & 0.091             & 2.313          \\
                  &                 & -0.10           & 0.635             & 4.655          & 0.061             & 0.994          \\
                  &                 & -0.15           & 0.649             & 4.275          & 0.068             & 1.083          \\
                  &                 & -0.20           & 0.716             & 12.461         & 0.092             & 2.526          \\
                  & Online Est.     & \textbf{-0.115} & \textbf{0.610}    & \textbf{3.099} & \textbf{0.057}    & \textbf{0.944} 
\end{tblr}
}
\vspace{0.3em}
{\raggedright
\noindent\par {\footnotesize \textsuperscript{*}The initial time offset of `Online Est.' is set to 0.0 and the converged values are shown above.}
\noindent\par {\footnotesize \textsuperscript{**}For each sequence, the lowest error values among the fixed offsets are highlighted in \textcolor{red}{red}, and the second-lowest in \textcolor{blue}{blue}.}
\par}

\end{table}
For the performance comparison, the open-source EKF-RIO \cite{9235254}, which uses the same measurement model but does not account for temporal calibration, is employed. All parameters are kept identical to ensure a fair comparison. In the proposed method, the time offset \( t_d \) is initialized to 0.0 seconds for all sequences, reflecting a typical scenario where the initial time offset is unknown. The experimental results are evaluated using the open-source tool EVO \cite{grupp2017evo}. Figure~\ref{trajectory} illustrates the estimated trajectories compared to the ground truth for visual comparison, with one representative result from each dataset. Due to the stochastic nature of the RANSAC algorithm used in radar ego-velocity estimation, the averaged results from 100 trials across all datasets are presented. We compare the root mean square error (RMSE) of both absolute pose error (APE) and relative pose error (RPE), with the RPE calculated at 10-meter intervals.

APE evaluates the overall trajectory by calculating the difference between the ground truth and the estimated poses for all frames, making it particularly useful for assessing the global accuracy of the estimated trajectory. However, APE can be sensitive to significant rotational errors that occur early or in specific sections, potentially overshadowing smaller errors later in the trajectory. In contrast, RPE focuses on local accuracy by aligning poses at regular intervals and calculating the error, allowing discrepancies over shorter segments to be highlighted. When the temporal calibration between sensors is not accounted for, errors can accumulate over time, making RPE evaluation essential. Both metrics offer valuable insights, providing a comprehensive evaluation of the trajectory.

\subsubsection{Self-Collected Dataset}
The purpose of the self-collected dataset is to identify the actual time offset between the IMU and the radar and evaluate its impact on the accuracy of RIO. Since the handheld platform does not utilize a hardware trigger to synchronize the sensors, the exact time offset is unknown and must be estimated. To address this uncertainty, we evaluate the performance of fixed time offsets over a range of values to determine the interval that provides the best accuracy and estimate the likely time offset range.

As shown in Table \ref{fixed_offset}, error values are analyzed with fixed offsets set at 0.05-second intervals for both Sequence 1 and Sequence 4, which feature different motion patterns. The results show that the time offset falls within the -0.10 to -0.15 second range, where the highest accuracy in terms of APE and RPE is observed for both sequences. The proposed method, which utilizes online temporal calibration, estimates the time offset as -0.114 seconds for Sequence 1 and -0.115 seconds for Sequence 4, closely matching the range found through fixed offset testing. In both cases, the proposed method achieves improved performance in terms of both APE and RPE, demonstrates its effectiveness in accurately estimating the time offset.

\begin{table}[t]
\centering
\caption{Quantitative Results of Comparison study on Self-collected dataset}
\label{table_self}
\resizebox{\linewidth}{!}{
\begin{tblr}{
  cells = {c},
  cell{1}{1} = {r=2}{},
  cell{1}{2} = {r=2}{},
  cell{1}{3} = {c=2}{},
  cell{1}{5} = {c=2}{},
  cell{3}{1} = {r=2}{},
  cell{5}{1} = {r=2}{},
  cell{7}{1} = {r=2}{},
  cell{9}{1} = {r=2}{},
  cell{11}{1} = {r=2}{},
  cell{13}{1} = {r=2}{},
  cell{15}{1} = {r=2}{},
  cell{17}{1} = {r=2}{},
  hline{1,3,5,7,9,11,13,15,17,19} = {-}{},
  hline{2} = {3-6}{},
}
{\textbf{Sequence }\\\textbf{(Trajectory Length)}} & {\textbf{Method } \textbf{($\hat{t}_d$)}} & \textbf{APE RMSE } &                & \textbf{RPE RMSE } &                \\
                                                   &                                         & Trans. (m)         & Rot. (\degree)        & Trans. (m)         & Rot. (\degree)        \\
                                                   \hline
{Sequence 1\\(177 m)}                              & {EKF-RIO (N/A)}                        & 0.985              & 1.872           & 0.264              & 1.230          \\
                                                   & {Ours (-0.114 s)}                      & \textbf{0.646}     & \textbf{0.935}  & \textbf{0.132}     & \textbf{0.774} \\
{Sequence 2\\(197 m)}                              & {EKF-RIO}                              & 2.269              & 2.161           & 0.136              & 1.414          \\
                                                   & {Ours (-0.114 s)}                      & \textbf{0.587}     & \textbf{1.650}  & \textbf{0.064}     & \textbf{0.774} \\
{Sequence 3\\(144 m)}                              & {EKF-RIO}                              & 1.368              & 2.331           & 0.167              & 1.347          \\
                                                   & {Ours (-0.113 s)}                      & \textbf{0.414}     & \textbf{1.140}  & \textbf{0.088}     & \textbf{0.613} \\
{Sequence 4\\(197 m)}                              & {EKF-RIO}                              & 1.737              & 25.885          & 0.118              & 4.074          \\
                                                   & {Ours (-0.115 s)}                      & \textbf{0.610}     & \textbf{3.099}  & \textbf{0.057}     & \textbf{0.944} \\
{Sequence 5\\(190 m)}                              & {EKF-RIO}                              & 2.375              & 7.702           & 0.122              & 1.600          \\
                                                   & {Ours (-0.115 s)}                      & \textbf{1.150}     & \textbf{1.304}  & \textbf{0.069}     & \textbf{0.814} \\
{Sequence 6\\(179 m)}                              & {EKF-RIO}                              & 1.267              & 17.907          & 0.117              & 2.828          \\
                                                   & {Ours (-0.111 s)}                      & \textbf{0.661}     & \textbf{2.551}  & \textbf{0.051}     & \textbf{0.809} \\
{Sequence 7\\(223 m)}                              & {EKF-RIO}                              & 2.757              & 10.092          & 0.116              & 1.863          \\
                                                   & {Ours (-0.112 s)}                      & \textbf{1.596}     & \textbf{6.039}  & \textbf{0.057}     & \textbf{1.365} \\
{Average}                                          & {EKF-RIO}                              & 1.822              & 9.707            & 0.148             & 2.051          \\
                                                   & {Ours (-0.113 s)}                      & \textbf{0.809}     & \textbf{2.388}   & \textbf{0.074}    & \textbf{0.870}   
\end{tblr}
}
\end{table}

Since the radar delay is generally larger than IMU delay, the time offset \( t_d \), representing the difference between these delays, typically takes a negative value. To evaluate the robustness of the estimation, different initial values of \( t_d \) ranging from 0.0 to -0.3 seconds are tested. Figure \ref{sq5} illustrates the estimated time offset for each initial setting, along with the 3-sigma boundaries. As \( t_d \) is estimated from radar ego-velocity, it cannot be determined while the platform is stationary. Once the platform starts moving, the filter begins estimating \( t_d \) and quickly converges to a stable value. The filter converges to a stable time offset of -0.114 ± 0.001 seconds in Sequence 1 and -0.115 ± 0.001 seconds in Sequence 4.

Table \ref{table_self} presents the performance comparison between the proposed method with online temporal calibration and EKF-RIO across seven sequences. The proposed method outperforms EKF-RIO, significantly reducing both APE and RPE across all sequences. Specifically, it reduces APE translation error by an average of 56\%, APE rotation error by 75\%, RPE translation error by 50\%, and RPE rotation error by 58\% compared with EKF-RIO. Despite using the same measurement model, the performance improvement is achieved solely by applying propagation and updates based on a common time stream through the proposed online temporal calibration.

On average, the time offset \( t_d \) is estimated to be -0.113 ± 0.002 seconds, confirming consistent temporal calibration throughout the experiments. Compared with LiDAR-inertial and visual-inertial systems, radar-inertial systems exhibit a significantly larger time offset, as shown in Table~\ref{time_offset_comparison}. Given the radar sensor rate (10 Hz), such a large time offset is significant enough to cause a misalignment spanning more than one data frame. These findings highlight the necessity of temporal calibration in RIO, which is crucial for accurate sensor fusion and reliable pose estimation in real-world applications.

\begin{figure}[t]
\centering
\includegraphics[width=\linewidth]{figure_4.png}
\caption{Time offset estimation with 3-sigma boundaries for different initial values in Sequence 1 and 4.}
\label{sq5}
\end{figure}

\begin{table}[t]
\centering
\caption{Comparison of Time Offset in Multi-Sensor Fusion Systems}
\label{time_offset_comparison}
\begin{tabular}{|c|c|c|} 
\hline
\textbf{Systems} & \textbf{Sensor} & \textbf{Time Offset} \\ 
\hline
LiDAR-Inertial~\cite{10113826} & Velodyne VLP-32 & 0.006 s\\ 
\hline
Visual-Inertial~\cite{li2014online} & PointGrey Bumblebee2 & 0.047 s\\ 
\hline
Radar-Inertial & TI AWR1843BOOST & \textbf{0.113 s} \\
\hline
\end{tabular}
\end{table}

\subsubsection{Open Datasets}
Table \ref{opendataset} presents the results from the two open datasets. The ICINS dataset incorporates a hardware trigger for the radar, which we use to validate the accuracy of the time offset estimation for the proposed method. In this setup, a microcontroller sends radar trigger signals, prompting the radar to begin scanning. The radar data is timestamped based on the actual trigger signal, providing a pseudo-ground truth for time offset estimation. Theoretically, if the sensors are time-synchronized through triggers, the time offset \( t_d \) is expected to be close to 0.0 seconds. The proposed method estimates the time offset to be an average of 0.016 ± 0.003 seconds. Despite this slight discrepancy, the proposed method demonstrates comparable or improved performance on average in both APE and RPE compared with EKF-RIO. Although the ICINS dataset includes hardware-triggered signals for the radar, there is no such trigger signal for the IMU in the dataset, which may introduce a delay in IMU measurements. As defined in Eq.~\eqref{time_offset}, we attribute the estimated positive time offset to this IMU delay, explaining the difference from the expected value.

The ColoRadar dataset, widely used for performance comparison in the RIO field, is utilized to assess if the proposed method generalizes well across different datasets. As shown in Table \ref{opendataset}, the proposed method also demonstrates performance improvements over EKF-RIO in terms of both APE and RPE on average. However, the extent of improvement is smaller compared with the self-collected dataset, which can be explained by differences in trajectory characteristics. The radar ego-velocity model utilizes not only the accelerometer but also the gyroscope measurements. As illustrated in Fig.~\ref{trajectory}, the ColoRadar dataset involves movement over a larger area with less rotation, leading to a smaller impact of the time offset on performance. Nonetheless, the proposed method achieves 33\% reduction in RPE translation error, demonstrating its effectiveness even in this less challenging trajectory. On average, the time offset \( t_d \) is estimated to be -0.111 ± 0.003 seconds, similar to the time offset found in the self-collected dataset. This consistency is likely due to the use of the same radar sensor model in both datasets, further validating the reliability of the proposed method across different environments.

\begin{table}[t]
\centering
\caption{Quantitative Results of Comparison study on Open datasets}
\label{opendataset}
\resizebox{\linewidth}{!}{
\begin{tblr}{
  cells = {c},
  cell{1}{1} = {r=2}{},
  cell{1}{2} = {r=2}{},
  cell{1}{3} = {c=2}{},
  cell{1}{5} = {c=2}{},
  cell{3}{1} = {r=2}{},
  cell{5}{1} = {r=2}{},
  cell{7}{1} = {r=2}{},
  cell{9}{1} = {r=2}{},
  cell{11}{1} = {r=2}{},
  cell{13}{1} = {r=2}{},
  cell{15}{1} = {r=2}{},
  cell{17}{1} = {r=2}{},
  cell{19}{1} = {r=2}{},
  cell{21}{1} = {r=2}{},
  hline{1,3,5,7,9,11,13,15,17,19,21,23} = {-}{},
  hline{2-3} = {3-6}{},
}
{\textbf{Sequence }\\\textbf{(Trajectory Length)}}       & \textbf{Method ($\hat{t}_d$)} & \textbf{APE RMSE}        &                                           & \textbf{RPE RMSE}       &                         \\
                        &                               & Trans. (m)               & Rot. (\degree)                                   & Trans. (m)              & Rot. (\degree)                 \\
                        \hline
{ICINS 1\\(295 m)}      & EKF-RIO (N/A)                 & 1.959                    & 10.694                                    & \textbf{0.093}          & \textbf{0.896}          \\
                        & Ours (0.016 s)                & \textbf{1.922}           & \textbf{10.135}                           & 0.098                   & 0.918          \\
{ICINS 2\\(468 m)}      & EKF-RIO                       & 3.830                    & 23.151                                    & \textbf{0.114}          & 1.289                   \\
                        & Ours (0.013 s)                & \textbf{3.198}           & \textbf{19.235}                           & 0.121                   & \textbf{1.076}          \\
{ICINS 3\\(150 m)}      & EKF-RIO                       & \textbf{1.502}           & \textbf{9.905}                            & 0.130                   & \textbf{1.512}           \\
                        & Ours (0.015 s)                & 1.530                    & 10.189                                    & \textbf{0.126}          & 1.553          \\
{ICINS 4\\(50 m)}       & EKF-RIO                       & \textbf{0.213}           & \textbf{2.091}                            & \textbf{0.076}          & \textbf{0.923}           \\
                        & Ours (0.019 s)                & 0.216                    & 2.098                                     & 0.081                   & \textbf{0.923}          \\
Average                 & EKF-RIO                       & 1.876                    & 11.460                                    & \textbf{0.103}          & 1.155                   \\
                        & Ours (0.016 s)                & \textbf{1.716}           & \textbf{10.414}                           & 0.106                   & \textbf{1.117}          \\
                        \hline
{ColoRadar 1\\(178 m) } & EKF-RIO (N/A)                 & 6.556                    & \textbf{\textbf{1.354}}                   & 0.182                   & \textbf{1.071} \\
                        & Ours (-0.110 s)               & \textbf{\textbf{6.173}}  & 1.382                                     & \textbf{\textbf{0.155}} & 1.188                   \\
{ColoRadar 2\\(197 m) } & EKF-RIO                       & \textbf{\textbf{4.747}}  & 1.238                                     & 0.372                   & 1.375                   \\
                        & Ours (-0.114 s)               & 4.826                    & \textbf{\textbf{0.960}}                   & \textbf{\textbf{0.292}} & \textbf{\textbf{1.180}} \\
{ColoRadar 3\\(197 m) } & EKF-RIO                       & \textbf{\textbf{8.307}}  & 1.969                                     & 0.259                   & 1.015                   \\
                        & Ours (-0.108 s)               & 8.550                    & \textbf{\textbf{1.852}}                   & \textbf{\textbf{0.221}} & \textbf{\textbf{0.879}} \\
{ColoRadar 4\\(144 m) } & EKF-RIO                       & 12.111                   & 2.815                                     & 0.488                   & 1.263                   \\
                        & Ours (-0.112 s)               & \textbf{11.946}          & \textbf{2.756}                            & \textbf{0.200}          & \textbf{1.116} \\
Average                 & EKF-RIO                       & 7.930                    & 1.844                                     & 0.325                   & 1.181                   \\
                        & Ours(-0.111 s)                & \textbf{7.874}           & \textbf{1.737}                            & \textbf{0.217}          & \textbf{1.091}          
\end{tblr}
}
\end{table}

In this paper, we systematically investigate the position bias problem in the multi-constraint instruction following. To quantitatively measure the disparity of constraint order, we propose a novel Difficulty Distribution Index (CDDI). Based on the CDDI, we design a probing task. First, we construct a large number of instructions consisting of different constraint orders. Then, we conduct experiments in two distinct scenarios. Extensive results reveal a clear preference of LLMs for ``hard-to-easy'' constraint orders. To further explore this, we conduct an explanation study. We visualize the importance of different constraints located in different positions and demonstrate the strong correlation between the model's attention distribution and its performance.


{\small
\bibliographystyle{ieee_fullname}
\bibliography{BibForOpenFly}
}

\end{document}
