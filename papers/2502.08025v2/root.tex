%%%%%%%%%%%%%%%%%%%%%%%%%%%%%%%%%%%%%%%%%%%%%%%%%%%%%%%%%%%%%%%%%%%%%%%%%%%%%%%%
%2345678901234567890123456789012345678901234567890123456789012345678901234567890
%        1         2         3         4         5         6         7         8

\documentclass[letterpaper, 10 pt, conference]{ieeeconf}  % Comment this line out if you need a4paper

%\documentclass[a4paper, 10pt, conference]{ieeeconf}      % Use this line for a4 paper

\IEEEoverridecommandlockouts                              % This command is only needed if 
                                                          % you want to use the \thanks command

\overrideIEEEmargins                                      % Needed to meet printer requirements.

%In case you encounter the following error:
%Error 1010 The PDF file may be corrupt (unable to open PDF file) OR
%Error 1000 An error occurred while parsing a contents stream. Unable to analyze the PDF file.
%This is a known problem with pdfLaTeX conversion filter. The file cannot be opened with acrobat reader
%Please use one of the alternatives below to circumvent this error by uncommenting one or the other
%\pdfobjcompresslevel=0
%\pdfminorversion=4

% See the \addtolength command later in the file to balance the column lengths
% on the last page of the document

% The following packages can be found on http:\\www.ctan.org
%\usepackage{graphics} % for pdf, bitmapped graphics files
%\usepackage{epsfig} % for postscript graphics files
%\usepackage{mathptmx} % assumes new font selection scheme installed
%\usepackage{times} % assumes new font selection scheme installed
%\usepackage{amsmath} % assumes amsmath package installed
%\usepackage{amssymb}  % assumes amsmath package installed

\usepackage{graphicx}
\usepackage{dirtytalk}
\usepackage{etoolbox}
\usepackage{booktabs}
\usepackage{multirow}

\usepackage[mathscr]{euscript}

\usepackage[table,xcdraw]{xcolor}

\usepackage{hyperref}

\usepackage{amsmath}
\usepackage{amssymb}

\usepackage{algorithm}
\usepackage{algpseudocode}
\usepackage{enumerate}

\usepackage{threeparttable}
\usepackage{scalerel}
\usepackage{tikz}
\usepackage{cite}

\hyphenation{op-tical net-works semi-conduc-tor}

\title{\LARGE \bf
From Brainwaves to Brain Scans: A Robust Neural Network for EEG-to-fMRI Synthesis
}


\author{Kristofer Grover Roos, Atsushi Fukuda, and Quan Huu Cap
% <-this % stops a space
\thanks{All authors are with the AI Development Department, Aillis, Inc., Tokyo, Japan. Corresponding author: Quan Huu Cap ({\tt\small quan.cap@aillis.jp})}
}


\begin{document}

\maketitle
\thispagestyle{empty}
\pagestyle{empty}


%%%%%%%%%%%%%%%%%%%%%%%%%%%%%%%%%%%%%%%%%%%%%%%%%%%%%%%%%%%%%%%%%%%%
\begin{abstract}
    \begin{abstract}\label{00_Abstract}
Research in the field of automated vehicles, or more generally cognitive cyber-physical systems that operate in the real world, is leading to increasingly complex systems. Among other things, artificial intelligence enables an ever-increasing degree of autonomy. In this context, the V-model, which has served for decades as a process reference model of the system development lifecycle is reaching its limits. To the contrary, innovative processes and frameworks have been developed that take into account the characteristics of emerging autonomous systems. To bridge the gap and merge the different methodologies, we present an extension of the V-model for iterative data-based development processes that harmonizes and formalizes the existing methods towards a generic framework. The iterative approach allows for seamless integration of continuous system refinement. While the data-based approach constitutes the consideration of data-based development processes and formalizes the use of synthetic and real world data. In this way, formalizing the process of development, verification, validation, and continuous integration contributes to ensuring the safety of emerging complex systems that incorporate AI. 
\end{abstract}


\begin{IEEEkeywords}
	Process Reference Model, V-Model, Continuous Integration, AI Systems, Autonomy Technology, Safety Assurance
\end{IEEEkeywords}

% \newline

% \indent \textit{Clinical relevance}— This is a brief additional statement on why a this might be of interest to practicing clinicians. Example: This establishes the anesthetic efficacy of 10\% intraosseous injections with epinephrine to positively influence cardiovascular function.
\end{abstract}

% KEYWORDS
\begin{keywords}
fMRI reconstruction, functional magnetic resonance imaging, neuroimaging, deep learning, neural networks.
\end{keywords}

%%%%%%%%%%%%%%%%%%%%%%%%%%%%%%%%%%%%%%%%%%%%%%%%%%%%%%%%%%%%%%%%%%%%%%%%%%%%%%%%
\section{INTRODUCTION}
    \section{Introduction}

% \textcolor{red}{Still on working}

% \textcolor{red}{add label for each section}


Robot learning relies on diverse and high-quality data to learn complex behaviors \cite{aldaco2024aloha, wang2024dexcap}.
Recent studies highlight that models trained on datasets with greater complexity and variation in the domain tend to generalize more effectively across broader scenarios \cite{mann2020language, radford2021learning, gao2024efficient}.
% However, creating such diverse datasets in the real world presents significant challenges.
% Modifying physical environments and adjusting robot hardware settings require considerable time, effort, and financial resources.
% In contrast, simulation environments offer a flexible and efficient alternative.
% Simulations allow for the creation and modification of digital environments with a wide range of object shapes, weights, materials, lighting, textures, friction coefficients, and so on to incorporate domain randomization,
% which helps improve the robustness of models when deployed in real-world conditions.
% These environments can be easily adjusted and reset, enabling faster iterations and data collection.
% Additionally, simulations provide the ability to consistently reproduce scenarios, which is essential for benchmarking and model evaluation.
% Another advantage of simulations is their flexibility in sensor integration. Sensors such as cameras, LiDARs, and tactile sensors can be added or repositioned without the physical limitations present in real-world setups. Simulations also eliminate the risk of damaging expensive hardware during edge-case experiments, making them an ideal platform for testing rare or dangerous scenarios that are impractical to explore in real life.
By leveraging immersive perspectives and interactions, Extended Reality\footnote{Extended Reality is an umbrella term to refer to Augmented Reality, Mixed Reality, and Virtual Reality \cite{wikipediaExtendedReality}}
(XR)
is a promising candidate for efficient and intuitive large scale data collection \cite{jiang2024comprehensive, arcade}
% With the demand for collecting data, XR provides a promising approach for humans to teach robots by offering users an immersive experience.
in simulation \cite{jiang2024comprehensive, arcade, dexhub-park} and real-world scenarios \cite{openteach, opentelevision}.
However, reusing and reproducing current XR approaches for robot data collection for new settings and scenarios is complicated and requires significant effort.
% are difficult to reuse and reproduce system makes it hard to reuse and reproduce in another data collection pipeline.
This bottleneck arises from three main limitations of current XR data collection and interaction frameworks: \textit{asset limitation}, \textit{simulator limitation}, and \textit{device limitation}.
% \textcolor{red}{ASSIGN THESE CITATION PROPERLY:}
% \textcolor{red}{list them by time order???}
% of collecting data by using XR have three main limitations.
Current approaches suffering from \textit{asset limitation} \cite{arclfd, jiang2024comprehensive, arcade, george2025openvr, vicarios}
% Firstly, recent works \cite{jiang2024comprehensive, arcade, dexhub-park}
can only use predefined robot models and task scenes. Configuring new tasks requires significant effort, since each new object or model must be specifically integrated into the XR application.
% and it takes too much effort to configure new tasks in their systems since they cannot spawn arbitrary models in the XR application.
The vast majority of application are developed for specific simulators or real-world scenarios. This \textit{simulator limitation} \cite{mosbach2022accelerating, lipton2017baxter, dexhub-park, arcade}
% Secondly, existing systems are limited to a single simulation platform or real-world scenarios.
significantly reduces reusability and makes adaptation to new simulation platforms challenging.
Additionally, most current XR frameworks are designed for a specific version of a single XR headset, leading to a \textit{device limitation} 
\cite{lipton2017baxter, armada, openteach, meng2023virtual}.
% and there is no work working on the extendability of transferring to a new headsets as far as we know.
To the best of our knowledge, no existing work has explored the extensibility or transferability of their framework to different headsets.
These limitations hamper reproducibility and broader contributions of XR based data collection and interaction to the research community.
% as each research group typically has its own data collection pipeline.
% In addition to these main limitations, existing XR systems are not well suited for managing multiple robot systems,
% as they are often designed for single-operator use.

In addition to these main limitations, existing XR systems are often designed for single-operator use, prohibiting collaborative data collection.
At the same time, controlling multiple robots at once can be very difficult for a single operator,
making data collection in multi-robot scenarios particularly challenging \cite{orun2019effect}.
Although there are some works using collaborative data collection in the context of tele-operation \cite{tung2021learning, Qin2023AnyTeleopAG},
there is no XR-based data collection system supporting collaborative data collection.
This limitation highlights the need for more advanced XR solutions that can better support multi-robot and multi-user scenarios.
% \textcolor{red}{more papers about collaborative data collection}

To address all of these issues, we propose \textbf{IRIS},
an \textbf{I}mmersive \textbf{R}obot \textbf{I}nteraction \textbf{S}ystem.
This general system supports various simulators, benchmarks and real-world scenarios.
It is easily extensible to new simulators and XR headsets.
IRIS achieves generalization across six dimensions:
% \begin{itemize}
%     \item \textit{Cross-scene} : diverse object models;
%     \item \textit{Cross-embodiment}: diverse robot models;
%     \item \textit{Cross-simulator}: 
%     \item \textit{Cross-reality}: fd
%     \item \textit{Cross-platform}: fd
%     \item \textit{Cross-users}: fd
% \end{itemize}
\textbf{Cross-Scene}, \textbf{Cross-Embodiment}, \textbf{Cross-Simulator}, \textbf{Cross-Reality}, \textbf{Cross-Platform}, and \textbf{Cross-User}.

\textbf{Cross-Scene} and \textbf{Cross-Embodiment} allow the system to handle arbitrary objects and robots in the simulation,
eliminating restrictions about predefined models in XR applications.
IRIS achieves these generalizations by introducing a unified scene specification, representing all objects,
including robots, as data structures with meshes, materials, and textures.
The unified scene specification is transmitted to the XR application to create and visualize an identical scene.
By treating robots as standard objects, the system simplifies XR integration,
allowing researchers to work with various robots without special robot-specific configurations.
\textbf{Cross-Simulator} ensures compatibility with various simulation engines.
IRIS simplifies adaptation by parsing simulated scenes into the unified scene specification, eliminating the need for XR application modifications when switching simulators.
New simulators can be integrated by creating a parser to convert their scenes into the unified format.
This flexibility is demonstrated by IRIS’ support for Mujoco \cite{todorov2012mujoco}, IsaacSim \cite{mittal2023orbit}, CoppeliaSim \cite{coppeliaSim}, and even the recent Genesis \cite{Genesis} simulator.
\textbf{Cross-Reality} enables the system to function seamlessly in both virtual simulations and real-world applications.
IRIS enables real-world data collection through camera-based point cloud visualization.
\textbf{Cross-Platform} allows for compatibility across various XR devices.
Since XR device APIs differ significantly, making a single codebase impractical, IRIS XR application decouples its modules to maximize code reuse.
This application, developed by Unity \cite{unity3dUnityManual}, separates scene visualization and interaction, allowing developers to integrate new headsets by reusing the visualization code and only implementing input handling for hand, head, and motion controller tracking.
IRIS provides an implementation of the XR application in the Unity framework, allowing for a straightforward deployment to any device that supports Unity. 
So far, IRIS was successfully deployed to the Meta Quest 3 and HoloLens 2.
Finally, the \textbf{Cross-User} ability allows multiple users to interact within a shared scene.
IRIS achieves this ability by introducing a protocol to establish the communication between multiple XR headsets and the simulation or real-world scenarios.
Additionally, IRIS leverages spatial anchors to support the alignment of virtual scenes from all deployed XR headsets.
% To make an seamless user experience for robot learning data collection,
% IRIS also tested in three different robot control interface
% Furthermore, to demonstrate the extensibility of our approach, we have implemented a robot-world pipeline for real robot data collection, ensuring that the system can be used in both simulated and real-world environments.
The Immersive Robot Interaction System makes the following contributions\\
\textbf{(1) A unified scene specification} that is compatible with multiple robot simulators. It enables various XR headsets to visualize and interact with simulated objects and robots, providing an immersive experience while ensuring straightforward reusability and reproducibility.\\
\textbf{(2) A collaborative data collection framework} designed for XR environments. The framework facilitates enhanced robot data acquisition.\\
\textbf{(3) A user study} demonstrating that IRIS significantly improves data collection efficiency and intuitiveness compared to the LIBERO baseline.

% \begin{table*}[t]
%     \centering
%     \begin{tabular}{lccccccc}
%         \toprule
%         & \makecell{Physical\\Interaction}
%         & \makecell{XR\\Enabled}
%         & \makecell{Free\\View}
%         & \makecell{Multiple\\Robots}
%         & \makecell{Robot\\Control}
%         % Force Feedback???
%         & \makecell{Soft Object\\Supported}
%         & \makecell{Collaborative\\Data} \\
%         \midrule
%         ARC-LfD \cite{arclfd}                              & Real        & \cmark & \xmark & \xmark & Joint              & \xmark & \xmark \\
%         DART \cite{dexhub-park}                            & Sim         & \cmark & \cmark & \cmark & Cartesian          & \xmark & \xmark \\
%         \citet{jiang2024comprehensive}                     & Sim         & \cmark & \xmark & \xmark & Joint \& Cartesian & \xmark & \xmark \\
%         \citet{mosbach2022accelerating}                    & Sim         & \cmark & \cmark & \xmark & Cartesian          & \xmark & \xmark \\
%         ARCADE \cite{arcade}                               & Real        & \cmark & \cmark & \xmark & Cartesian          & \xmark & \xmark \\
%         Holo-Dex \cite{holodex}                            & Real        & \cmark & \xmark & \cmark & Cartesian          & \cmark & \xmark \\
%         ARMADA \cite{armada}                               & Real        & \cmark & \xmark & \cmark & Cartesian          & \cmark & \xmark \\
%         Open-TeleVision \cite{opentelevision}              & Real        & \cmark & \cmark & \cmark & Cartesian          & \cmark & \xmark \\
%         OPEN TEACH \cite{openteach}                        & Real        & \cmark & \xmark & \cmark & Cartesian          & \cmark & \cmark \\
%         GELLO \cite{wu2023gello}                           & Real        & \xmark & \cmark & \cmark & Joint              & \cmark & \xmark \\
%         DexCap \cite{wang2024dexcap}                       & Real        & \xmark & \cmark & \xmark & Cartesian          & \cmark & \xmark \\
%         AnyTeleop \cite{Qin2023AnyTeleopAG}                & Real        & \xmark & \xmark & \cmark & Cartesian          & \cmark & \cmark \\
%         Vicarios \cite{vicarios}                           & Real        & \cmark & \xmark & \xmark & Cartesian          & \cmark & \xmark \\     
%         Augmented Visual Cues \cite{augmentedvisualcues}   & Real        & \cmark & \cmark & \xmark & Cartesian          & \xmark & \xmark \\ 
%         \citet{wang2024robotic}                            & Real        & \cmark & \cmark & \xmark & Cartesian          & \cmark & \xmark \\
%         Bunny-VisionPro \cite{bunnyvisionpro}              & Real        & \cmark & \cmark & \cmark & Cartesian          & \cmark & \xmark \\
%         IMMERTWIN \cite{immertwin}                         & Real        & \cmark & \cmark & \cmark & Cartesian          & \xmark & \xmark \\
%         \citet{meng2023virtual}                            & Sim \& Real & \cmark & \cmark & \xmark & Cartesian          & \xmark & \xmark \\
%         Shared Control Framework \cite{sharedctlframework} & Real        & \cmark & \cmark & \cmark & Cartesian          & \xmark & \xmark \\
%         OpenVR \cite{openvr}                               & Real        & \cmark & \cmark & \xmark & Cartesian          & \xmark & \xmark \\
%         \citet{digitaltwinmr}                              & Real        & \cmark & \cmark & \xmark & Cartesian          & \cmark & \xmark \\
        
%         \midrule
%         \textbf{Ours} & Sim \& Real & \cmark & \cmark & \cmark & Joint \& Cartesian  & \cmark & \cmark \\
%         \bottomrule
%     \end{tabular}
%     \caption{This is a cross-column table with automatic line breaking.}
%     \label{tab:cross-column}
% \end{table*}

% \begin{table*}[t]
%     \centering
%     \begin{tabular}{lccccccc}
%         \toprule
%         & \makecell{Cross-Embodiment}
%         & \makecell{Cross-Scene}
%         & \makecell{Cross-Simulator}
%         & \makecell{Cross-Reality}
%         & \makecell{Cross-Platform}
%         & \makecell{Cross-User} \\
%         \midrule
%         ARC-LfD \cite{arclfd}                              & \xmark & \xmark & \xmark & \xmark & \xmark & \xmark \\
%         DART \cite{dexhub-park}                            & \cmark & \cmark & \xmark & \xmark & \xmark & \xmark \\
%         \citet{jiang2024comprehensive}                     & \xmark & \cmark & \xmark & \xmark & \xmark & \xmark \\
%         \citet{mosbach2022accelerating}                    & \xmark & \cmark & \xmark & \xmark & \xmark & \xmark \\
%         ARCADE \cite{arcade}                               & \xmark & \xmark & \xmark & \xmark & \xmark & \xmark \\
%         Holo-Dex \cite{holodex}                            & \cmark & \xmark & \xmark & \xmark & \xmark & \xmark \\
%         ARMADA \cite{armada}                               & \cmark & \xmark & \xmark & \xmark & \xmark & \xmark \\
%         Open-TeleVision \cite{opentelevision}              & \cmark & \xmark & \xmark & \xmark & \cmark & \xmark \\
%         OPEN TEACH \cite{openteach}                        & \cmark & \xmark & \xmark & \xmark & \xmark & \cmark \\
%         GELLO \cite{wu2023gello}                           & \cmark & \xmark & \xmark & \xmark & \xmark & \xmark \\
%         DexCap \cite{wang2024dexcap}                       & \xmark & \xmark & \xmark & \xmark & \xmark & \xmark \\
%         AnyTeleop \cite{Qin2023AnyTeleopAG}                & \cmark & \cmark & \cmark & \cmark & \xmark & \cmark \\
%         Vicarios \cite{vicarios}                           & \xmark & \xmark & \xmark & \xmark & \xmark & \xmark \\     
%         Augmented Visual Cues \cite{augmentedvisualcues}   & \xmark & \xmark & \xmark & \xmark & \xmark & \xmark \\ 
%         \citet{wang2024robotic}                            & \xmark & \xmark & \xmark & \xmark & \xmark & \xmark \\
%         Bunny-VisionPro \cite{bunnyvisionpro}              & \cmark & \xmark & \xmark & \xmark & \xmark & \xmark \\
%         IMMERTWIN \cite{immertwin}                         & \cmark & \xmark & \xmark & \xmark & \xmark & \xmark \\
%         \citet{meng2023virtual}                            & \xmark & \cmark & \xmark & \cmark & \xmark & \xmark \\
%         \citet{sharedctlframework}                         & \cmark & \xmark & \xmark & \xmark & \xmark & \xmark \\
%         OpenVR \cite{george2025openvr}                               & \xmark & \xmark & \xmark & \xmark & \xmark & \xmark \\
%         \citet{digitaltwinmr}                              & \xmark & \xmark & \xmark & \xmark & \xmark & \xmark \\
        
%         \midrule
%         \textbf{Ours} & \cmark & \cmark & \cmark & \cmark & \cmark & \cmark \\
%         \bottomrule
%     \end{tabular}
%     \caption{This is a cross-column table with automatic line breaking.}
% \end{table*}

% \begin{table*}[t]
%     \centering
%     \begin{tabular}{lccccccc}
%         \toprule
%         & \makecell{Cross-Scene}
%         & \makecell{Cross-Embodiment}
%         & \makecell{Cross-Simulator}
%         & \makecell{Cross-Reality}
%         & \makecell{Cross-Platform}
%         & \makecell{Cross-User}
%         & \makecell{Control Space} \\
%         \midrule
%         % Vicarios \cite{vicarios}                           & \xmark & \xmark & \xmark & \xmark & \xmark & \xmark \\     
%         % Augmented Visual Cues \cite{augmentedvisualcues}   & \xmark & \xmark & \xmark & \xmark & \xmark & \xmark \\ 
%         % OpenVR \cite{george2025openvr}                     & \xmark & \xmark & \xmark & \xmark & \xmark & \xmark \\
%         \citet{digitaltwinmr}                              & \xmark & \xmark & \xmark & \xmark & \xmark & \xmark &  \\
%         ARC-LfD \cite{arclfd}                              & \xmark & \xmark & \xmark & \xmark & \xmark & \xmark &  \\
%         \citet{sharedctlframework}                         & \cmark & \xmark & \xmark & \xmark & \xmark & \xmark &  \\
%         \citet{jiang2024comprehensive}                     & \cmark & \xmark & \xmark & \xmark & \xmark & \xmark &  \\
%         \citet{mosbach2022accelerating}                    & \cmark & \xmark & \xmark & \xmark & \xmark & \xmark & \\
%         Holo-Dex \cite{holodex}                            & \cmark & \xmark & \xmark & \xmark & \xmark & \xmark & \\
%         ARCADE \cite{arcade}                               & \cmark & \cmark & \xmark & \xmark & \xmark & \xmark & \\
%         DART \cite{dexhub-park}                            & Limited & Limited & Mujoco & Sim & Vision Pro & \xmark &  Cartesian\\
%         ARMADA \cite{armada}                               & \cmark & \cmark & \xmark & \xmark & \xmark & \xmark & \\
%         \citet{meng2023virtual}                            & \cmark & \cmark & \xmark & \cmark & \xmark & \xmark & \\
%         % GELLO \cite{wu2023gello}                           & \cmark & \xmark & \xmark & \xmark & \xmark & \xmark \\
%         % DexCap \cite{wang2024dexcap}                       & \xmark & \xmark & \xmark & \xmark & \xmark & \xmark \\
%         % AnyTeleop \cite{Qin2023AnyTeleopAG}                & \cmark & \cmark & \cmark & \cmark & \xmark & \cmark \\
%         % \citet{wang2024robotic}                            & \xmark & \xmark & \xmark & \xmark & \xmark & \xmark \\
%         Bunny-VisionPro \cite{bunnyvisionpro}              & \cmark & \cmark & \xmark & \xmark & \xmark & \xmark & \\
%         IMMERTWIN \cite{immertwin}                         & \cmark & \cmark & \xmark & \xmark & \xmark & \xmark & \\
%         Open-TeleVision \cite{opentelevision}              & \cmark & \cmark & \xmark & \xmark & \cmark & \xmark & \\
%         \citet{szczurek2023multimodal}                     & \xmark & \xmark & \xmark & Real & \xmark & \cmark & \\
%         OPEN TEACH \cite{openteach}                        & \cmark & \cmark & \xmark & \xmark & \xmark & \cmark & \\
%         \midrule
%         \textbf{Ours} & \cmark & \cmark & \cmark & \cmark & \cmark & \cmark \\
%         \bottomrule
%     \end{tabular}
%     \caption{TODO, Bruce: this table can be further optimized.}
% \end{table*}

\definecolor{goodgreen}{HTML}{228833}
\definecolor{goodred}{HTML}{EE6677}
\definecolor{goodgray}{HTML}{BBBBBB}

\begin{table*}[t]
    \centering
    \begin{adjustbox}{max width=\textwidth}
    \renewcommand{\arraystretch}{1.2}    
    \begin{tabular}{lccccccc}
        \toprule
        & \makecell{Cross-Scene}
        & \makecell{Cross-Embodiment}
        & \makecell{Cross-Simulator}
        & \makecell{Cross-Reality}
        & \makecell{Cross-Platform}
        & \makecell{Cross-User}
        & \makecell{Control Space} \\
        \midrule
        % Vicarios \cite{vicarios}                           & \xmark & \xmark & \xmark & \xmark & \xmark & \xmark \\     
        % Augmented Visual Cues \cite{augmentedvisualcues}   & \xmark & \xmark & \xmark & \xmark & \xmark & \xmark \\ 
        % OpenVR \cite{george2025openvr}                     & \xmark & \xmark & \xmark & \xmark & \xmark & \xmark \\
        \citet{digitaltwinmr}                              & \textcolor{goodred}{Limited}     & \textcolor{goodred}{Single Robot} & \textcolor{goodred}{Unity}    & \textcolor{goodred}{Real}          & \textcolor{goodred}{Meta Quest 2} & \textcolor{goodgray}{N/A} & \textcolor{goodred}{Cartesian} \\
        ARC-LfD \cite{arclfd}                              & \textcolor{goodgray}{N/A}        & \textcolor{goodred}{Single Robot} & \textcolor{goodgray}{N/A}     & \textcolor{goodred}{Real}          & \textcolor{goodred}{HoloLens}     & \textcolor{goodgray}{N/A} & \textcolor{goodred}{Cartesian} \\
        \citet{sharedctlframework}                         & \textcolor{goodred}{Limited}     & \textcolor{goodred}{Single Robot} & \textcolor{goodgray}{N/A}     & \textcolor{goodred}{Real}          & \textcolor{goodred}{HTC Vive Pro} & \textcolor{goodgray}{N/A} & \textcolor{goodred}{Cartesian} \\
        \citet{jiang2024comprehensive}                     & \textcolor{goodred}{Limited}     & \textcolor{goodred}{Single Robot} & \textcolor{goodgray}{N/A}     & \textcolor{goodred}{Real}          & \textcolor{goodred}{HoloLens 2}   & \textcolor{goodgray}{N/A} & \textcolor{goodgreen}{Joint \& Cartesian} \\
        \citet{mosbach2022accelerating}                    & \textcolor{goodgreen}{Available} & \textcolor{goodred}{Single Robot} & \textcolor{goodred}{IsaacGym} & \textcolor{goodred}{Sim}           & \textcolor{goodred}{Vive}         & \textcolor{goodgray}{N/A} & \textcolor{goodgreen}{Joint \& Cartesian} \\
        Holo-Dex \cite{holodex}                            & \textcolor{goodgray}{N/A}        & \textcolor{goodred}{Single Robot} & \textcolor{goodgray}{N/A}     & \textcolor{goodred}{Real}          & \textcolor{goodred}{Meta Quest 2} & \textcolor{goodgray}{N/A} & \textcolor{goodred}{Joint} \\
        ARCADE \cite{arcade}                               & \textcolor{goodgray}{N/A}        & \textcolor{goodred}{Single Robot} & \textcolor{goodgray}{N/A}     & \textcolor{goodred}{Real}          & \textcolor{goodred}{HoloLens 2}   & \textcolor{goodgray}{N/A} & \textcolor{goodred}{Cartesian} \\
        DART \cite{dexhub-park}                            & \textcolor{goodred}{Limited}     & \textcolor{goodred}{Limited}      & \textcolor{goodred}{Mujoco}   & \textcolor{goodred}{Sim}           & \textcolor{goodred}{Vision Pro}   & \textcolor{goodgray}{N/A} & \textcolor{goodred}{Cartesian} \\
        ARMADA \cite{armada}                               & \textcolor{goodgray}{N/A}        & \textcolor{goodred}{Limited}      & \textcolor{goodgray}{N/A}     & \textcolor{goodred}{Real}          & \textcolor{goodred}{Vision Pro}   & \textcolor{goodgray}{N/A} & \textcolor{goodred}{Cartesian} \\
        \citet{meng2023virtual}                            & \textcolor{goodred}{Limited}     & \textcolor{goodred}{Single Robot} & \textcolor{goodred}{PhysX}   & \textcolor{goodgreen}{Sim \& Real} & \textcolor{goodred}{HoloLens 2}   & \textcolor{goodgray}{N/A} & \textcolor{goodred}{Cartesian} \\
        % GELLO \cite{wu2023gello}                           & \cmark & \xmark & \xmark & \xmark & \xmark & \xmark \\
        % DexCap \cite{wang2024dexcap}                       & \xmark & \xmark & \xmark & \xmark & \xmark & \xmark \\
        % AnyTeleop \cite{Qin2023AnyTeleopAG}                & \cmark & \cmark & \cmark & \cmark & \xmark & \cmark \\
        % \citet{wang2024robotic}                            & \xmark & \xmark & \xmark & \xmark & \xmark & \xmark \\
        Bunny-VisionPro \cite{bunnyvisionpro}              & \textcolor{goodgray}{N/A}        & \textcolor{goodred}{Single Robot} & \textcolor{goodgray}{N/A}     & \textcolor{goodred}{Real}          & \textcolor{goodred}{Vision Pro}   & \textcolor{goodgray}{N/A} & \textcolor{goodred}{Cartesian} \\
        IMMERTWIN \cite{immertwin}                         & \textcolor{goodgray}{N/A}        & \textcolor{goodred}{Limited}      & \textcolor{goodgray}{N/A}     & \textcolor{goodred}{Real}          & \textcolor{goodred}{HTC Vive}     & \textcolor{goodgray}{N/A} & \textcolor{goodred}{Cartesian} \\
        Open-TeleVision \cite{opentelevision}              & \textcolor{goodgray}{N/A}        & \textcolor{goodred}{Limited}      & \textcolor{goodgray}{N/A}     & \textcolor{goodred}{Real}          & \textcolor{goodgreen}{Meta Quest, Vision Pro} & \textcolor{goodgray}{N/A} & \textcolor{goodred}{Cartesian} \\
        \citet{szczurek2023multimodal}                     & \textcolor{goodgray}{N/A}        & \textcolor{goodred}{Limited}      & \textcolor{goodgray}{N/A}     & \textcolor{goodred}{Real}          & \textcolor{goodred}{HoloLens 2}   & \textcolor{goodgreen}{Available} & \textcolor{goodred}{Joint \& Cartesian} \\
        OPEN TEACH \cite{openteach}                        & \textcolor{goodgray}{N/A}        & \textcolor{goodgreen}{Available}  & \textcolor{goodgray}{N/A}     & \textcolor{goodred}{Real}          & \textcolor{goodred}{Meta Quest 3} & \textcolor{goodred}{N/A} & \textcolor{goodgreen}{Joint \& Cartesian} \\
        \midrule
        \textbf{Ours}                                      & \textcolor{goodgreen}{Available} & \textcolor{goodgreen}{Available}  & \textcolor{goodgreen}{Mujoco, CoppeliaSim, IsaacSim} & \textcolor{goodgreen}{Sim \& Real} & \textcolor{goodgreen}{Meta Quest 3, HoloLens 2} & \textcolor{goodgreen}{Available} & \textcolor{goodgreen}{Joint \& Cartesian} \\
        \bottomrule
        \end{tabular}
    \end{adjustbox}
    \caption{Comparison of XR-based system for robots. IRIS is compared with related works in different dimensions.}
\end{table*}



\section{METHODS}
    Our E2fNet is a fully CNN specifically designed to generate fMRI from EEG data. Let $x \in \mathbb{R}^{T \times C \times F}$ be a spectrogram of multi-channel EEG data, where $T$, $C$, and $F$ correspond to the temporal, electrode channel, and frequency dimension, respectively. 
Let $y \in \mathbb{R}^{D \times W \times H}$ be an fMRI volume representation at a given time, where $D$, $W$, and $H$ correspond to the depth, width, and height. 
Fig. \ref{fig:fig_1} illustrates the overview of the proposed E2fNet. 
Our model consists of an EEG encoder $\mathcal{M}_{\mathrm{EEG}}$, a U-Net module $\mathcal{M}_{\mathrm{UNet}}$, and an fMRI decoder $\mathcal{M}_{\mathrm{fMRI}}$. 

\subsection{The EEG Encoder}
The EEG encoder is built to capture the characteristics of EEG signals effectively. 
Given EEG data \(x \in \mathbb{R}^{T \times C \times F}\), we obtain the encoded features 
\(x_{eeg} = \mathcal{M}_{\mathrm{EEG}}(x)\), where \(x_{eeg} \in \mathbb{R}^{N \times W \times H}\). 
Here, \(N\) denotes the depth (number of feature maps), and \(W \times H\) corresponds to the 
width and height of the target fMRI volume. The design of $\mathcal{M}_{\mathrm{EEG}}$ includes several key principles. 
\textit{First}, $\mathcal{M}_{\mathrm{EEG}}$ progressively expands the temporal dimension from $T \rightarrow N$ (where $N \gg T$), enabling the capture of long-term dependencies and subtle temporal dynamics critical for modeling changes over time. 
\textit{Second}, the model retains the electrode dimension $C$ throughout the convolutional process, preserving the spatial topology and relationships between electrodes. 
This preservation ensures the integrity of spatial information for downstream analyses. 
\textit{Third}, the frequency dimension $F$ is gradually reduced toward the size of $H$. 
This emphasizes the most relevant spectral features while reducing the complexity of the encoded representation. 
All of this can be done by a series of convolutional layers with kernel size of $[1 \times k]$, where $k > 1$, to progressively shrink the $F$ dimension. 

The convolutional process produces a feature tensor of size $N \times C \times H$, which is then resized to $N \times W \times H$ using bicubic interpolation to match the spatial size of the target fMRI. In this work, we designed the $\mathcal{M}_{\mathrm{EEG}}$ to output the feature's depth $N=256$ (i.e., $x_{eeg}\in \mathbb{R}^{256 \times W \times H}$). From our observations, this feature resizing approach is significantly more effective compared to traditional encoders which often compress data into a much lower dimensionality. 

\subsection{The U-Net Module and fMRI Decoder}
The encoded features $x_{eeg} \in \mathbb{R}^{N \times W \times H}$ are then fed into $\mathcal{M}_{\mathrm{UNet}}$ for further processing. 
This U-Net module consists of two down-sample blocks followed by two up-sample blocks, enabling the extraction of multi-scale features. 
Each down-sample block reduces the spatial dimensions by half while the up-sample blocks restore them. 
The number of output channels for each block is $N$ ($N=256$ in this work) and the output of this model is the same as its input, with $x_{unet}=\mathcal{M}_{\mathrm{UNet}}(x) \in \mathbb{R}^{256 \times W \times H}$. 
Here, the extraction of multi-scale features in $\mathcal{M}_{\mathrm{UNet}}$ is crucial, as our preliminary experiments revealed that the model without the U-Net module failed to accurately reconstruct the fMRI targets. 

The fMRI decoder $\mathcal{M}_{\mathrm{fMRI}}$ then gradually reduces the depth of $x_{unet}$ from $N \rightarrow D$ while maintaining the spatial size $W \times H$. 
Finally, the generated fMRI is $\hat{y} = \mathcal{M}_{\mathrm{fMRI}}(x_{unet}) = \mathrm{E2fNet}(x) \in \mathbb{R}^{D \times W \times H}$. 

\subsection{Loss Function}
To train the E2fNet model, we employed the SSIM and mean square error (MSE) losses for balanced structural and pixel-wise accuracy. 
The objective function for training is:
%% Equation (1)
\begin{equation}
\label{eq:1}
\mathcal{L}(y, \hat{y}) = \lambda_{1} \mathrm{SSIM}(y, \hat{y}) + \lambda_{2} \mathrm{MSE}(y, \hat{y}),
\end{equation}
where $y$ and $\hat{y}$ are the ground-truth and generated fMRI, respectively. 
$\lambda_{1}$ and $\lambda_{2}$ are the loss coefficients. 
For more details on the architecture and implementation, please refer to our GitHub repository. 
% Table I
\begin{table}[t]
\centering
\caption{Performance comparison in Dice score on the $\mathrm{LQ}_{seg}$ dataset}
\label{tab:table_1}
\resizebox{0.97\linewidth}{!}{
\begin{tabular}{@{}llllll@{}}
\toprule
\multicolumn{1}{c}{\multirow{2}{*}{Model}} & \multirow{2}{*}{W/o DCIN} & \multicolumn{4}{c}{DCIN}                    \\ \cmidrule(l){3-6} 
\multicolumn{1}{c}{}                        &                           & ExRI & GRIS & LRIS          & GRIS + LRIS   \\ \midrule
$S_{\mathrm{base}}$                                    & 36.9                      & 53.7 & 52.4 & \textbf{58.9} & 56.9          \\ \midrule
$S_{\mathrm{aug}}$                                & 54.0                      & 55.9 & 53.7 & \textbf{58.1} & 57.2          \\ \midrule
$S_{\mathrm{CQG}}$                                & 64.5                      & 67.2 & 67.5 & 68.5          & \textbf{69.2} \\ \bottomrule
\end{tabular}
}
\end{table}
    
\section{EXPERIMENTAL SETTINGS}
    \subsection{Data Preprocessing}
We followed the preprocessing method described in the literature \cite{calhas2022eeg}. 
Specifically, given an EEG-fMRI dataset with an EEG sampling rate of $f_s$ (Hz) and an fMRI Time Response (TR) of $f_{TR}$ (second), the EEG signal at each electrode is divided into non-overlapping windows of length $f_s \times f_{TR}$. 
Then, the Fast Fourier Transform (FFT) is applied to extract the frequency components of size $F$ from a 1-dimensional EEG window signal. 
An upper-band filter is applied to the FFT with a cutoff at 250 Hz, and the Direct Current (DC) component is removed. 
As a result, the frequency dimension is $F = 249$. 
For $C$ electrodes in EEG data, the frequency components for a single window have a size of $C \times F$. 
The temporal dimension $T$ of EEG data is set to 20 (i.e., 20 consecutive windows) so that an EEG sample $x \in \mathbb{R}^{20 \times C \times 249}$ corresponds to an fMRI volume $y \in \mathbb{R}^{D \times W \times H}$. 
We normalized fMRI data to range [0, 1] per each volume and the EEG data to range [0, 1] across the entire dataset. For more details on the data pre-processing, please refer to the literature \cite{calhas2022eeg} and our GitHub repository. 

\subsection{Datasets}
In this study, we used three publicly available simultaneous EEG and fMRI datasets namely NODDI \cite{deligianni2014relating}, Oddball \cite{walz2015prestimulus}, and CN-EPFL \cite{pereira2020disentangling}. 
The specific of each dataset is described as follows: 
\subsubsection{NODDI}
A dataset comprising simultaneous EEG-fMRI recordings from 17 healthy adults during resting-state sessions with eyes open. 
EEG was acquired using a 64-channel cap (10-20 system) at a 250Hz sampling rate, capturing millisecond-level neural signals. 
In parallel, fMRI data were collected using a Siemens Avanto 1.5 T clinical scanner with a time response (TR) of 2.16s. 
The resolution of a volume is $30 \times 64 \times 64$. 
Following \cite{lanzino2024nt}, we only processed 15 out of the original 17 participants, resulting in 4,110 paired EEG-fMRI samples. 

\subsubsection{Oddball}
The dataset includes EEG-fMRI recordings from 17 healthy adults performing an oddball paradigm involving auditory and visual tasks. 
EEG was recorded at a 1000 Hz sampling rate across 43 channels, capturing rapid neural oscillations linked to attentional engagement. 
Concurrently, fMRI scans were acquired in a $32 \times 64 \times 64$ volume at a 2-second TR using a Philips Achieva 3T clinical scanner with a single-channel head coil, capturing hemodynamic responses to these rare events. 
This dataset consists of 14,688 paired EEG-fMRI samples. 

\subsubsection{CN-EPFL}
This dataset was collected by the Center for Neuroprosthetics, École Polytechnique Fédérale de Lausanne (EPFL) in Switzerland. 
It comprises concurrent EEG–fMRI recordings from 20 participants, each engaged in a speeded visual discrimination task during data acquisition. 
EEG was collected via a 64-channel cap (10-20 system) at a high sampling rate of 5,000Hz. 
Meanwhile, fMRI scans were obtained at a repetition time (TR) of 1.28 seconds in a $54 \times 108 \times 108$ volume. 
Following \cite{calhas2022eeg}, the discrete cosine transform (DCT II \& III) \cite{ahmed2006discrete} was applied to the original volumes, and subsequently down-sampled to $30 \times 64 \times 64$ for consistency with the other datasets. 
This dataset has a total of 6,880 paired EEG-fMRI samples.

\subsection{Comparison Models}
We compared our E2fNet model with previous methods: CNN-TC \cite{liu2019convolutional}, CNN-TAG \cite{calhas2022eeg}, and NT-ViT \cite{lanzino2024nt}. 
Additionally, we evaluated our model against a GAN-based approach. 
For this, we treated the E2fNet as the generator and employed a binary discriminator to classify an fMRI volume as real or fake. 
We refer to this model as E2fGAN. 
The E2fGAN model was trained using the loss function in Eq. \ref{eq:1} and the GAN loss \cite{goodfellow2014generative}, with the same training settings as in E2fNet. 
Details of the E2fGAN architecture are available in our GitHub repository. 

\subsection{Training and Evaluation}
We trained our E2fNet on the three datasets described above. 
The loss in Eq. \ref{eq:1} with $\lambda_{1} = \lambda_{2} = 0.5$ was employed to train our models. 
The AdamW optimizer \cite{loshchilov2017adamw} was used with the learning rate set to $10^{-3}$. 
The learning rate warm-up was applied for the first 50 training steps. 
All models were trained for 50 epochs on an NVIDIA GeForce RTX 3090 GPU with batch size set to 64. 

In this work, we used the SSIM and peak signal-to-noise ratio (PSNR) metrics to evaluate the quality of the generated results. 
To compare with \cite{lanzino2024nt}, we adopted the leave-one-subject-out cross-validation scheme for the NODDI and Oddball datasets. 
For a dataset with $K$ subjects, we perform $K$ experiments, each time using one subject for evaluation while the remaining subjects are used for training. 
The result is averaged across $K$ experiments. 
For the CN-EPFL dataset, we used the first 16 individuals for training, and the last four for evaluation as in \cite{calhas2022eeg}. 
% Fig. 2
%Figure 2
\begin{figure*}[!t]
\centering
\includegraphics[width=0.99\textwidth]{figures/Fig_results.pdf}
\caption{
    Visual comparison of the generated fMRI by E2fNet and E2fGAN models on the NODDI, Oddball, and CN-EPFL datasets. 
}
\label{fig:fig_2}
\end{figure*}

\section{RESULTS AND DISCUSSION}
    Table \ref{tab:table_1} presents the performance comparison of different models on the three datasets. 
Our proposed E2fNet achieved state-of-the-art results in terms of SSIM across all benchmarks, with SSIM scores of 0.605 on NODDI, 0.631 on Oddball, and 0.674 on CN-EPFL datasets. 
The second-best performing model was NT-ViT, followed by our E2fGAN, with the CNN-TAG and CNN-TC models ranking lower.

In comparison to the previous CNN-TC and CNN-TAG models, our models significantly improved the fMRI reconstruction performance by substantial margins. 
Most notably on the Oddball dataset with a 0.442 SSIM increment over the CNN-TC \cite{liu2019convolutional} model (from 0.189 to 0.631), marking a considerable leap in performance. 
The CNN-TAG and CNN-TC models, in contrast, were less effective in capturing meaningful features from EEG data for generating fMRI targets. 

The strong performance of E2fNet can be attributed to two primary innovations: the design of the EEG encoder $\mathcal{M}_{\mathrm{EEG}}$ and the integration of the U-Net module $\mathcal{M}_{\mathrm{UNet}}$. 
The encoder effectively captures meaningful features across EEG electrodes, while the U-Net module enables the extraction of multi-scale features. 
We observed that the model without the U-Net module failed to accurately reconstruct the fMRI targets, with relatively low SSIM scores (e.g., below 0.45 on the NODDI dataset). 

Compared to the NT-ViT model, E2fNet achieved higher SSIM scores but lower PSNR on the NODDI and Oddball datasets (see Table \ref{tab:table_1}). 
The NT-ViT’s strong PSNR results may be attributed to the use of the Vision Transformer architecture \cite{dosovitskiy21vit}. 
However, we argue that SSIM is a more relevant metric for assessing perceptual quality from a human visual perspective, which is critical in medical imaging applications. 
Furthermore, E2fNet offers a much simpler design compared to NT-ViT. 
Specifically, our E2fNet follows a simple encoder-decoder architecture design whereas the NT-ViT employed two transformer-based encoder-decoder networks. 

To provide a more comprehensive evaluation, we compared E2fNet with the GAN-based E2fGAN model, which uses E2fNet as the generator. 
Fig. \ref{fig:fig_2} presents a visual comparison of E2fNet and E2fGAN across the three datasets. 
During the experiments, we observed that training the GAN model was often unstable, leading to unexpected outputs (Fig. \ref{fig:fig_2}, first row of the NODDI and Oddball datasets, first example in the second row of the CN-EPFL dataset). 
This instability contributed to E2fGAN's lower SSIM and PSNR scores compared to E2fNet. 
Nevertheless, in some cases, E2fGAN produced visually sharper fMRI volumes. 
Future work could explore improvements by adopting more robust GAN training strategies.

It is important to note that SSIM and PSNR metrics may be sometimes less reliable when evaluating small or blurry images. 
For instance, while E2fGAN frequently generated oversmoothed and less detailed results on the CN-EPFL dataset, it still achieved seemingly reasonable SSIM and PSNR scores. 
We believe developing better similarity metrics could help address this limitation. There is still room for improvement, and we plan to investigate these aspects in future work.

\section{CONCLUSION}
    \section{Conclusion}
We introduce the Difficulty and Uncertainty-Aware Lightweight (DUAL) score, a novel scoring metric designed for cost-effective pruning. The DUAL score is the first metric to integrate both difficulty and uncertainty into a single measure, and its effectiveness in identifying the most informative samples early in training is further supported by theoretical analysis. Additionally, we propose pruning-ratio-adaptive sampling to account for sample diversity, particularly when the pruning ratio is extremely high. Our proposed pruning methods, DUAL score and DUAL score combined with Beta sampling demonstrate remarkable performance, particularly in realistic scenarios involving label noise and image corruption, by effectively distinguishing noisy samples.

Data pruning research has been evolving in a direction that contradicts its primary objective of reducing computational and storage costs while improving training efficiency. This is mainly because the computational cost of pruning often exceeds that of full training. By introducing our DUAL method, we take a crucial step toward overcoming this challenge by significantly reducing the computation cost associated with data pruning, making it feasible for practical scenarios. Ultimately, we believe this will help minimize resource waste and enhance training efficiency.

% \section*{ACKNOWLEDGMENT}
% We would like to thank all researchers at Aillis Inc., especially the AI Development team, Dr. Memori Fukuda (MD), Dr. Kei Katsuno (MD), and Dr. Takaya Hanawa (MD) for their valuable comments and feedback. 

\nocite{*}
\footnotesize{
\bibliographystyle{IEEEtran}
\bibliography{reference}
}

\end{document}
