%%%%%%%%%%%%%%%%%%%%%%%%%%%%%%%%%%%%%%%%%%%%%%%%%%%%%%%%%%%%%%%%%%%%%%%%%%%%%%%%
%2345678901234567890123456789012345678901234567890123456789012345678901234567890
%        1         2         3         4         5         6         7         8

\documentclass[letterpaper, 10 pt, conference]{ieeeconf}  % Comment this line out if you need a4paper

%\documentclass[a4paper, 10pt, conference]{ieeeconf}      % Use this line for a4 paper

\IEEEoverridecommandlockouts                              % This command is only needed if 
                                                          % you want to use the \thanks command

\overrideIEEEmargins                                      % Needed to meet printer requirements.

%In case you encounter the following error:
%Error 1010 The PDF file may be corrupt (unable to open PDF file) OR
%Error 1000 An error occurred while parsing a contents stream. Unable to analyze the PDF file.
%This is a known problem with pdfLaTeX conversion filter. The file cannot be opened with acrobat reader
%Please use one of the alternatives below to circumvent this error by uncommenting one or the other
%\pdfobjcompresslevel=0
%\pdfminorversion=4

% See the \addtolength command later in the file to balance the column lengths
% on the last page of the document

% The following packages can be found on http:\\www.ctan.org
%\usepackage{graphics} % for pdf, bitmapped graphics files
%\usepackage{epsfig} % for postscript graphics files
%\usepackage{mathptmx} % assumes new font selection scheme installed
%\usepackage{times} % assumes new font selection scheme installed
%\usepackage{amsmath} % assumes amsmath package installed
%\usepackage{amssymb}  % assumes amsmath package installed

\usepackage{graphicx}
\usepackage{dirtytalk}
\usepackage{etoolbox}
\usepackage{booktabs}
\usepackage{multirow}

\usepackage[mathscr]{euscript}

\usepackage[table,xcdraw]{xcolor}

\usepackage{hyperref}

\usepackage{amsmath}
\usepackage{amssymb}

\usepackage{algorithm}
\usepackage{algpseudocode}
\usepackage{enumerate}

\usepackage{threeparttable}
\usepackage{scalerel}
\usepackage{tikz}
\usepackage{cite}

\hyphenation{op-tical net-works semi-conduc-tor}

\title{\LARGE \bf
From Brainwaves to Brain Scans: A Robust Neural Network for EEG-to-fMRI Synthesis
}


\author{Kristofer Grover Roos, Atsushi Fukuda, and Quan Huu Cap
% <-this % stops a space
\thanks{All authors are with the AI Development Department, Aillis, Inc., Tokyo, Japan. Corresponding author: Quan Huu Cap ({\tt\small quan.cap@aillis.jp})}
}


\begin{document}

\maketitle
\thispagestyle{empty}
\pagestyle{empty}


%%%%%%%%%%%%%%%%%%%%%%%%%%%%%%%%%%%%%%%%%%%%%%%%%%%%%%%%%%%%%%%%%%%%
\begin{abstract}
    % 150words
% Replying to workplace emails that are typically long and require politeness is time-consuming and cognitively demanding.
% Replying to lengthy and polite workplace emails is often time-consuming and cognitively demanding.
% takes time to understand and reply
\red{Replying to formal emails is time-consuming and cognitively demanding, as it requires crafting polite phrasing and providing an adequate response to the sender's demands.}
% \red{Replying to formal emails, which often takes time to understand and require polite phrasing, is time-consuming and cognitively demanding.}
Although systems with Large Language Models (LLM) were designed to simplify the email replying process, users still need to provide detailed prompts to obtain the expected output.
Therefore, we proposed and evaluated an \red{LLM-powered question-and-answer (QA)-based approach} for users to reply to emails by answering a set of simple and short questions generated from the incoming email.
We developed a prototype system, \textit{ResQ}, and conducted controlled and field experiments with 12 and \red{8} participants.
Our results demonstrated that \red{the QA-based approach} improves the efficiency of replying to emails and reduces workload while maintaining email quality, compared to a conventional prompt-based approach that requires users to craft appropriate prompts to obtain email drafts.
We discuss how \red{the QA-based approach} influences the email reply process and interpersonal relationship dynamics, as well as the opportunities and challenges associated with using a QA-based approach in AI-mediated communication.

% original
% Replying to lengthy and polite workplace emails is often time-consuming and cognitively demanding.
% Although systems with Large Language Models were designed to simplify the email replying process, users still needed to provide detailed prompts to obtain the expected output.
% Therefore, we proposed and evaluated a question-and-answer-based approach for users to reply to emails by answering a set of simple and short questions generated from the incoming email.
% We developed a prototype system, \textit{ResQ}, and conducted both controlled and field experiments with 12 and 9 participants.
% Our results demonstrated that ResQ improves the efficiency of replying to emails and reduces workload while maintaining email quality compared to a conventional prompt-based approach that requires users to craft appropriate prompts to obtain email drafts.
% We discuss how ResQ influences the email reply process and interpersonal relationship dynamics, as well as the opportunities and challenges associated with using a QA-based approach in AI-mediated communication.
% \newline

% \indent \textit{Clinical relevance}— This is a brief additional statement on why a this might be of interest to practicing clinicians. Example: This establishes the anesthetic efficacy of 10\% intraosseous injections with epinephrine to positively influence cardiovascular function.
\end{abstract}

% KEYWORDS
\begin{keywords}
fMRI reconstruction, functional magnetic resonance imaging, neuroimaging, deep learning, neural networks.
\end{keywords}

%%%%%%%%%%%%%%%%%%%%%%%%%%%%%%%%%%%%%%%%%%%%%%%%%%%%%%%%%%%%%%%%%%%%%%%%%%%%%%%%
\section{INTRODUCTION}
    
\section{Introduction}

\begin{figure*}
    \centering
    \includegraphics[width=\textwidth]{figures/Introduction.pdf}
    \caption{Showing the novel problem statement applied to traffic prediction use case. Multiple unstructured observations from the past are used to reconstruct a hidden traffic state from which a full traffic state is forecast with a set of query locations. }
    \label{fig:intro}
\end{figure*}

% Was sagen denn die anderen warum Traffic Prediction gut ist? 
Forecasting the traffic in the near future is an important task for city management.
Data from the near past is used to predict future traffic states with spatio-temporal Graph Neural Networks \cite{bui22}.
Accurate prediction provides the opportunity to optimize traffic flow, reduce traffic jams and increase air quality \cite{Po19}.

% Wieso ist Sparsity in allen Dimensionen wichtig.
While traffic prediction relies on the availability of data from traffic sensors, there exists a plethora of reasons why sensors may stop working temporarily, such as simple errors, energy saving, or overloaded communication systems.
Considering small- or medium-sized cities, the coverage of sensors may be low because the sensors are too expensive or not available.
Also, the sensors are typically static and do not adapt to changes in the traffic flow (e.g. caused by a construction site), which motivates moving sensors that for example could be mounted on cars. 
However, both missing and moving sensors introduce sparsity, since measurements may not be available for all locations at all times.
This sparsity must be explicitly addressed in traffic prediction for a realistic application scenario, which is illustrated in figure \ref{fig:intro}.
From one hour of data on Sunday morning, only few observations of the traffic state are available at each timestep.
The number of observations may differ throughout the observed time and the observation itself can be distributed arbitrarily in the city. 
We assume a relatively low number of sensors to account for resource saving and sensor failure in our proposed framework SUSTeR.
The task is to predict the dense traffic state one timestep after the observations at all possible sensor locations.
We study this problem on the traffic dataset Metr-LA and PEMS-BAY to test our assumption that only a fraction of the sensor values would be enough for good predictions.
By modifying an existing traffic dataset, we are able to compare our results from very sparse observations to the bottom line with all information available.
A successful study will provide insights in how sensors in new cities can be reduced before installing them and further mobile sensors would save more resources and are able to adapt to new traffic situations.
We argue that in order to be adaptable to other cities and changes in traffic flows, prior information like the road network should be neglected and just the sparse observations considered.
This comes with the added benefit of making our solution applicable in regions where no openly available road network is maintained or pathways change frequently (e.g. flood areas, animal observations). 


The aforementioned problem is novel and more challenging than the commonly considered traffic prediction problem, since there exist very few observations in each input sample.
Current works for the traffic prediction problem do not consider any missing values. \cite{Li2021, Shao22}
A common method among state of the art approaches is the usage of Graph Neural Networks on graphs that model the sensor network \cite{bui22}.
The values of a sensor are applied to the same graph node for each timestep which prohibits any non-stationary sensors . 
With fixed sensor locations, the resulting sensor network is highly correlated with the road network.
Streets connecting two intersections with sensors should be also an interesting point for correlations in the sensor network.
However, variable observations and high temporal sparsity rates can not be modeled adequately in a static network.
We show in our experiments that the road network has only a small influence on the traffic predictions.

Besides the traffic prediction for future timesteps, some works explore the field of traffic speed imputation \cite{Cini22, Cuza22} where missing sensor values are predicted.
But the amount of missing values is assumed to be at most 80\%, which on average are still over 40 given sensors in each timestep in the Metr-LA dataset with a total of 207 sensors.
We consider up to 99.9\% missing values which are on average 2.4 observations in each timestep that are used as input.
Such high sparsity rates drastically decrease the chance that multiple values are present in one input sample from the same sensor location, which makes it challenging to recognize and learn temporal correlations for each location on its own.

High sparsity rates (>95\%) result in few sensor values, but if a reconstruction of the traffic state would be possible, we question if spatio-temporal graphs require nodes for each sensor.
In SUSTeR we utilize only a small amount of graph nodes for the encoding of information and do not relate such nodes to the sensor network.
We call this the hidden graph (see figure \ref{fig:intro}), which is still able to reconstruct the complete traffic state.
Due to the reduced number of nodes SUSTeR achieves faster runtimes, as shown in the experiments.
This hidden graph is not embedded directly in the spatial domain, which is why the assignment of observations, as well as the querying of the future traffic, is done with an encoder and a decoder, implemented as neural networks.
The decoding from the hidden graph to future values depends on a set of query locations.
Figure \ref{fig:intro} shows the query locations as given from outside and in combination with the reconstructed traffic state the future values are predicted.

To construct the hidden graph we encode observations from each timestep into from multiple graphs, one for each timestep. 
The graphs are created in a residual style and information is added to the node embeddings from the previous timesteps.
We choose this method to incorporate all timesteps equally into the hidden state because the redundant information along the past is non-existing for high sparsity rates.
From the sequence of graphs where our framework inserted the observations step by step we apply STGCN \cite{Yu18}, an algorithm for traffic prediction to find and learn the spatio-temporal correlations on our small number of graph nodes.
The first future timestep of the STGCN is our hidden graph in which the traffic state is reconstructed. 

% Recent work has an implicit embedding of the graph nodes into the spatial domain as the assignment from the sensor to graph node is fixed one by one.
% Because the graph has the same structure as the road network spatio-temporal correlations can be learned between those sensors.
% We reduce the number of nodes and use a non-linear assignment learned data-driven from the observations.

We find in the experiments that SUSTeR outperforms the plain STGCN and modern traffic prediction frameworks like D2STGNN for high sparsity rates $(\geq 99\%)$.
This is equivalent to only $0.2$ to $2.4$ observation for each timestep on average.
SUSTeR uses fewer parameters than the baselines and can train faster and with less training data.
Our main contributions can be summarized as follows:
\begin{itemize}
    \item We introduce a sparse and unstructured variant of the traffic prediction problem with sparsity in all dimensions. The sensors report only a fraction of their values and are arbitrarily distributed in the spatial domain.
    \item We propose SUSTeR, a framework around the STGCN architecture, which maps sparse observations onto a dense hidden graph to reconstruct the complete traffic state.
    Our code is available at github.\footnote{https://github.com/ywoelker/SUSTeR}
    \item We conducts experiments that show that SUSTeR outperforms the baselines in very sparse situations ($\geq 95\%$) and has a competitive performance in low sparsity rates.
    % \item SUSTeR trains a third faster than the next competitor.
\end{itemize}


\section{METHODS}
    Our E2fNet is a fully CNN specifically designed to generate fMRI from EEG data. Let $x \in \mathbb{R}^{T \times C \times F}$ be a spectrogram of multi-channel EEG data, where $T$, $C$, and $F$ correspond to the temporal, electrode channel, and frequency dimension, respectively. 
Let $y \in \mathbb{R}^{D \times W \times H}$ be an fMRI volume representation at a given time, where $D$, $W$, and $H$ correspond to the depth, width, and height. 
Fig. \ref{fig:fig_1} illustrates the overview of the proposed E2fNet. 
Our model consists of an EEG encoder $\mathcal{M}_{\mathrm{EEG}}$, a U-Net module $\mathcal{M}_{\mathrm{UNet}}$, and an fMRI decoder $\mathcal{M}_{\mathrm{fMRI}}$. 

\subsection{The EEG Encoder}
The EEG encoder is built to capture the characteristics of EEG signals effectively. 
Given EEG data \(x \in \mathbb{R}^{T \times C \times F}\), we obtain the encoded features 
\(x_{eeg} = \mathcal{M}_{\mathrm{EEG}}(x)\), where \(x_{eeg} \in \mathbb{R}^{N \times W \times H}\). 
Here, \(N\) denotes the depth (number of feature maps), and \(W \times H\) corresponds to the 
width and height of the target fMRI volume. The design of $\mathcal{M}_{\mathrm{EEG}}$ includes several key principles. 
\textit{First}, $\mathcal{M}_{\mathrm{EEG}}$ progressively expands the temporal dimension from $T \rightarrow N$ (where $N \gg T$), enabling the capture of long-term dependencies and subtle temporal dynamics critical for modeling changes over time. 
\textit{Second}, the model retains the electrode dimension $C$ throughout the convolutional process, preserving the spatial topology and relationships between electrodes. 
This preservation ensures the integrity of spatial information for downstream analyses. 
\textit{Third}, the frequency dimension $F$ is gradually reduced toward the size of $H$. 
This emphasizes the most relevant spectral features while reducing the complexity of the encoded representation. 
All of this can be done by a series of convolutional layers with kernel size of $[1 \times k]$, where $k > 1$, to progressively shrink the $F$ dimension. 

The convolutional process produces a feature tensor of size $N \times C \times H$, which is then resized to $N \times W \times H$ using bicubic interpolation to match the spatial size of the target fMRI. In this work, we designed the $\mathcal{M}_{\mathrm{EEG}}$ to output the feature's depth $N=256$ (i.e., $x_{eeg}\in \mathbb{R}^{256 \times W \times H}$). From our observations, this feature resizing approach is significantly more effective compared to traditional encoders which often compress data into a much lower dimensionality. 

\subsection{The U-Net Module and fMRI Decoder}
The encoded features $x_{eeg} \in \mathbb{R}^{N \times W \times H}$ are then fed into $\mathcal{M}_{\mathrm{UNet}}$ for further processing. 
This U-Net module consists of two down-sample blocks followed by two up-sample blocks, enabling the extraction of multi-scale features. 
Each down-sample block reduces the spatial dimensions by half while the up-sample blocks restore them. 
The number of output channels for each block is $N$ ($N=256$ in this work) and the output of this model is the same as its input, with $x_{unet}=\mathcal{M}_{\mathrm{UNet}}(x) \in \mathbb{R}^{256 \times W \times H}$. 
Here, the extraction of multi-scale features in $\mathcal{M}_{\mathrm{UNet}}$ is crucial, as our preliminary experiments revealed that the model without the U-Net module failed to accurately reconstruct the fMRI targets. 

The fMRI decoder $\mathcal{M}_{\mathrm{fMRI}}$ then gradually reduces the depth of $x_{unet}$ from $N \rightarrow D$ while maintaining the spatial size $W \times H$. 
Finally, the generated fMRI is $\hat{y} = \mathcal{M}_{\mathrm{fMRI}}(x_{unet}) = \mathrm{E2fNet}(x) \in \mathbb{R}^{D \times W \times H}$. 

\subsection{Loss Function}
To train the E2fNet model, we employed the SSIM and mean square error (MSE) losses for balanced structural and pixel-wise accuracy. 
The objective function for training is:
%% Equation (1)
\begin{equation}
\label{eq:1}
\mathcal{L}(y, \hat{y}) = \lambda_{1} \mathrm{SSIM}(y, \hat{y}) + \lambda_{2} \mathrm{MSE}(y, \hat{y}),
\end{equation}
where $y$ and $\hat{y}$ are the ground-truth and generated fMRI, respectively. 
$\lambda_{1}$ and $\lambda_{2}$ are the loss coefficients. 
For more details on the architecture and implementation, please refer to our GitHub repository. 
% Table I
\begin{table}[t]
\centering
\caption{Performance comparison in Dice score on the $\mathrm{LQ}_{seg}$ dataset}
\label{tab:table_1}
\resizebox{0.97\linewidth}{!}{
\begin{tabular}{@{}llllll@{}}
\toprule
\multicolumn{1}{c}{\multirow{2}{*}{Model}} & \multirow{2}{*}{W/o DCIN} & \multicolumn{4}{c}{DCIN}                    \\ \cmidrule(l){3-6} 
\multicolumn{1}{c}{}                        &                           & ExRI & GRIS & LRIS          & GRIS + LRIS   \\ \midrule
$S_{\mathrm{base}}$                                    & 36.9                      & 53.7 & 52.4 & \textbf{58.9} & 56.9          \\ \midrule
$S_{\mathrm{aug}}$                                & 54.0                      & 55.9 & 53.7 & \textbf{58.1} & 57.2          \\ \midrule
$S_{\mathrm{CQG}}$                                & 64.5                      & 67.2 & 67.5 & 68.5          & \textbf{69.2} \\ \bottomrule
\end{tabular}
}
\end{table}
    
\section{EXPERIMENTAL SETTINGS}
    \subsection{Data Preprocessing}
We followed the preprocessing method described in the literature \cite{calhas2022eeg}. 
Specifically, given an EEG-fMRI dataset with an EEG sampling rate of $f_s$ (Hz) and an fMRI Time Response (TR) of $f_{TR}$ (second), the EEG signal at each electrode is divided into non-overlapping windows of length $f_s \times f_{TR}$. 
Then, the Fast Fourier Transform (FFT) is applied to extract the frequency components of size $F$ from a 1-dimensional EEG window signal. 
An upper-band filter is applied to the FFT with a cutoff at 250 Hz, and the Direct Current (DC) component is removed. 
As a result, the frequency dimension is $F = 249$. 
For $C$ electrodes in EEG data, the frequency components for a single window have a size of $C \times F$. 
The temporal dimension $T$ of EEG data is set to 20 (i.e., 20 consecutive windows) so that an EEG sample $x \in \mathbb{R}^{20 \times C \times 249}$ corresponds to an fMRI volume $y \in \mathbb{R}^{D \times W \times H}$. 
We normalized fMRI data to range [0, 1] per each volume and the EEG data to range [0, 1] across the entire dataset. For more details on the data pre-processing, please refer to the literature \cite{calhas2022eeg} and our GitHub repository. 

\subsection{Datasets}
In this study, we used three publicly available simultaneous EEG and fMRI datasets namely NODDI \cite{deligianni2014relating}, Oddball \cite{walz2015prestimulus}, and CN-EPFL \cite{pereira2020disentangling}. 
The specific of each dataset is described as follows: 
\subsubsection{NODDI}
A dataset comprising simultaneous EEG-fMRI recordings from 17 healthy adults during resting-state sessions with eyes open. 
EEG was acquired using a 64-channel cap (10-20 system) at a 250Hz sampling rate, capturing millisecond-level neural signals. 
In parallel, fMRI data were collected using a Siemens Avanto 1.5 T clinical scanner with a time response (TR) of 2.16s. 
The resolution of a volume is $30 \times 64 \times 64$. 
Following \cite{lanzino2024nt}, we only processed 15 out of the original 17 participants, resulting in 4,110 paired EEG-fMRI samples. 

\subsubsection{Oddball}
The dataset includes EEG-fMRI recordings from 17 healthy adults performing an oddball paradigm involving auditory and visual tasks. 
EEG was recorded at a 1000 Hz sampling rate across 43 channels, capturing rapid neural oscillations linked to attentional engagement. 
Concurrently, fMRI scans were acquired in a $32 \times 64 \times 64$ volume at a 2-second TR using a Philips Achieva 3T clinical scanner with a single-channel head coil, capturing hemodynamic responses to these rare events. 
This dataset consists of 14,688 paired EEG-fMRI samples. 

\subsubsection{CN-EPFL}
This dataset was collected by the Center for Neuroprosthetics, École Polytechnique Fédérale de Lausanne (EPFL) in Switzerland. 
It comprises concurrent EEG–fMRI recordings from 20 participants, each engaged in a speeded visual discrimination task during data acquisition. 
EEG was collected via a 64-channel cap (10-20 system) at a high sampling rate of 5,000Hz. 
Meanwhile, fMRI scans were obtained at a repetition time (TR) of 1.28 seconds in a $54 \times 108 \times 108$ volume. 
Following \cite{calhas2022eeg}, the discrete cosine transform (DCT II \& III) \cite{ahmed2006discrete} was applied to the original volumes, and subsequently down-sampled to $30 \times 64 \times 64$ for consistency with the other datasets. 
This dataset has a total of 6,880 paired EEG-fMRI samples.

\subsection{Comparison Models}
We compared our E2fNet model with previous methods: CNN-TC \cite{liu2019convolutional}, CNN-TAG \cite{calhas2022eeg}, and NT-ViT \cite{lanzino2024nt}. 
Additionally, we evaluated our model against a GAN-based approach. 
For this, we treated the E2fNet as the generator and employed a binary discriminator to classify an fMRI volume as real or fake. 
We refer to this model as E2fGAN. 
The E2fGAN model was trained using the loss function in Eq. \ref{eq:1} and the GAN loss \cite{goodfellow2014generative}, with the same training settings as in E2fNet. 
Details of the E2fGAN architecture are available in our GitHub repository. 

\subsection{Training and Evaluation}
We trained our E2fNet on the three datasets described above. 
The loss in Eq. \ref{eq:1} with $\lambda_{1} = \lambda_{2} = 0.5$ was employed to train our models. 
The AdamW optimizer \cite{loshchilov2017adamw} was used with the learning rate set to $10^{-3}$. 
The learning rate warm-up was applied for the first 50 training steps. 
All models were trained for 50 epochs on an NVIDIA GeForce RTX 3090 GPU with batch size set to 64. 

In this work, we used the SSIM and peak signal-to-noise ratio (PSNR) metrics to evaluate the quality of the generated results. 
To compare with \cite{lanzino2024nt}, we adopted the leave-one-subject-out cross-validation scheme for the NODDI and Oddball datasets. 
For a dataset with $K$ subjects, we perform $K$ experiments, each time using one subject for evaluation while the remaining subjects are used for training. 
The result is averaged across $K$ experiments. 
For the CN-EPFL dataset, we used the first 16 individuals for training, and the last four for evaluation as in \cite{calhas2022eeg}. 
% Fig. 2
%Figure 2
\begin{figure*}[!t]
\centering
\includegraphics[width=0.8\textwidth]{figures/Figure_DCIN_CQG.pdf}
\caption{
    The overview of our proposed dynamic color image normalization (DCIN) and color-quality generalization (CQG) loss.
}
\label{fig:fig_2}
\end{figure*}

\section{RESULTS AND DISCUSSION}
    Table \ref{tab:table_1} presents the performance comparison of different models on the three datasets. 
Our proposed E2fNet achieved state-of-the-art results in terms of SSIM across all benchmarks, with SSIM scores of 0.605 on NODDI, 0.631 on Oddball, and 0.674 on CN-EPFL datasets. 
The second-best performing model was NT-ViT, followed by our E2fGAN, with the CNN-TAG and CNN-TC models ranking lower.

In comparison to the previous CNN-TC and CNN-TAG models, our models significantly improved the fMRI reconstruction performance by substantial margins. 
Most notably on the Oddball dataset with a 0.442 SSIM increment over the CNN-TC \cite{liu2019convolutional} model (from 0.189 to 0.631), marking a considerable leap in performance. 
The CNN-TAG and CNN-TC models, in contrast, were less effective in capturing meaningful features from EEG data for generating fMRI targets. 

The strong performance of E2fNet can be attributed to two primary innovations: the design of the EEG encoder $\mathcal{M}_{\mathrm{EEG}}$ and the integration of the U-Net module $\mathcal{M}_{\mathrm{UNet}}$. 
The encoder effectively captures meaningful features across EEG electrodes, while the U-Net module enables the extraction of multi-scale features. 
We observed that the model without the U-Net module failed to accurately reconstruct the fMRI targets, with relatively low SSIM scores (e.g., below 0.45 on the NODDI dataset). 

Compared to the NT-ViT model, E2fNet achieved higher SSIM scores but lower PSNR on the NODDI and Oddball datasets (see Table \ref{tab:table_1}). 
The NT-ViT’s strong PSNR results may be attributed to the use of the Vision Transformer architecture \cite{dosovitskiy21vit}. 
However, we argue that SSIM is a more relevant metric for assessing perceptual quality from a human visual perspective, which is critical in medical imaging applications. 
Furthermore, E2fNet offers a much simpler design compared to NT-ViT. 
Specifically, our E2fNet follows a simple encoder-decoder architecture design whereas the NT-ViT employed two transformer-based encoder-decoder networks. 

To provide a more comprehensive evaluation, we compared E2fNet with the GAN-based E2fGAN model, which uses E2fNet as the generator. 
Fig. \ref{fig:fig_2} presents a visual comparison of E2fNet and E2fGAN across the three datasets. 
During the experiments, we observed that training the GAN model was often unstable, leading to unexpected outputs (Fig. \ref{fig:fig_2}, first row of the NODDI and Oddball datasets, first example in the second row of the CN-EPFL dataset). 
This instability contributed to E2fGAN's lower SSIM and PSNR scores compared to E2fNet. 
Nevertheless, in some cases, E2fGAN produced visually sharper fMRI volumes. 
Future work could explore improvements by adopting more robust GAN training strategies.

It is important to note that SSIM and PSNR metrics may be sometimes less reliable when evaluating small or blurry images. 
For instance, while E2fGAN frequently generated oversmoothed and less detailed results on the CN-EPFL dataset, it still achieved seemingly reasonable SSIM and PSNR scores. 
We believe developing better similarity metrics could help address this limitation. There is still room for improvement, and we plan to investigate these aspects in future work.

\section{CONCLUSION}
    \section{Conclusion}
 In this brief, we presented a 16nm reliable, time-predictable heterogeneous RISC-V SoC for \gls{ai}-enhanced mixed-criticality applications. To the best of our knowledge, this is the first SoC that combines safety features for \glspl{mcs} with hardware IPs for time-predictable execution of \glspl{mct} and leading-edge domain-specialized programmable accelerators within the same heterogeneous SoC. With a peak performance of 304.9 GOPS at 1.6TOPS/W and 260.7 GOPS/mm$^2$, and 121.8 GFLOPS at 1.1TFLOPS/W and 107GFLOPS/mm$^2$, the proposed SoC offers a comprehensive solution for reliable and deterministic execution of \gls{ai}/\gls{dsp}-enhanced \gls{mc} edge applications, achieving \gls{soa} energy efficiency under \looseness=-1  1.2~W power envelope. %The hardware description of the design is released open-source to foster future research~\footnote{\texttt{\url{https://github.com/pulp-platform/carfield}}}.
%
%To the best of our knowledge, this is the first SoC to combine safety-critical features for MCS with HW IPs for time-predictable execution of MCTs and leading edge domain-specialized programmable processor clusters into the same heterogeneous SoC, providing a comprehensive solution for reliable and deterministic execution of ML-enhanced mixed-criticality edge applications, at state-of-the-art energy efficiency and less than 1.5W power envelope. 

%\textcolor{red}{The platform and all the non-technological IPs integrated in Carfield are released open source under a liberal licence to foster future software and hardware research on reliable, time-predictable, accelerated heterogeneous computing devices for mixed-criticality applications.}


% \section*{ACKNOWLEDGMENT}
% We would like to thank all researchers at Aillis Inc., especially the AI Development team, Dr. Memori Fukuda (MD), Dr. Kei Katsuno (MD), and Dr. Takaya Hanawa (MD) for their valuable comments and feedback. 

\nocite{*}
\footnotesize{
\bibliographystyle{IEEEtran}
\bibliography{reference}
}

\end{document}
